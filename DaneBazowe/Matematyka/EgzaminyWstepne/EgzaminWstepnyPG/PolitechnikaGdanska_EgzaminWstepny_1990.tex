\documentclass[a4paper,12pt]{article}
\usepackage{latexsym}
\usepackage{amsmath}
\usepackage{amssymb}
\usepackage{graphicx}
\usepackage{wrapfig}
\pagestyle{plain}
\usepackage{fancybox}
\usepackage{bm}

\begin{document}

POLITECHNIKA $\mathrm{G}\mathrm{D}\mathrm{A}\acute{\mathrm{N}}$ SKA

Gdańsk, lipiec 1990 r.

Tematy I części egzaminu z matematyki

dla kandydatów ubiegających się o przyjęcie na I rok studiów dziennych.

Kandydat wybierał 3 dowo1ne zadania. Rozwiązania wybranych zadań oceniane

byly w skali $0-10$ punktów. Egzamin trwa1120 minut.

l. Zbadać przebieg zmienności funkcji

{\it y}$=$ -{\it xx}22 $+$-{\it xx} $++$11'

sporządzić jej wykres i na tej podstawie ustalič ile pierwiastków posiada rów-

nanie

--{\it xx}22$+$-{\it xx}$++$11$=${\it m}

w zalezności od parametru $m.$

2. Dlajakich wartości parametru $t$, przy dowolnej wartości parametru $k$, równa-

nie

$x^{2}+x\sqrt{k^{2}+4}-k\log_{\frac{1}{2}}(t+1)=0$

posiada dwa rózne pierwiastki?

3. Rozwiązać nierówność

$\displaystyle \lim_{n\rightarrow\infty}(\sqrt{n^{2}+(2+\sin 2x)n+4}-n)<1+\frac{1}{2}\cos 2x.$

4. Dwie kule o promieniach $R\mathrm{i}x(R>x)$ są styczne zewnętrznie. Przy jakim $x$

objętość stozka opisanego na tych kulach będzie najmniejsza?

5. $\mathrm{W}$ urnie $U_{1}$ znajdują się dwie kule czarne i pewna ilość kul bialych. $\mathrm{W}$ urnie

$U_{2}$ znajduje się 5 ku1 bia1ych i 3 czarne. $\mathrm{Z}$ pierwszej urny losujemy dwie kule i

przekładamy je do urny drugiej. Następnie z urny drugiej losujemy jedna kulę.

Podać minimalną ilość bialych kul znajdujących się w urnie $U_{1}$, jeśli wiadomo,

$\dot{\mathrm{z}}\mathrm{e}$ prawdopodobieństwo wylosowania kuli bialej z urny $U_{2}$ jest większe od 0, 6.




POLITECHNIKA $\mathrm{G}\mathrm{D}\mathrm{A}\acute{\mathrm{N}}$ SKA

Gdańsk, lipiec 1990 r.

Tematy II części egzaminu z matematyki

dla kandydatów ubiegajqcych się o przyjęcie na I rok studiów dziennych.

Wszystkie zadania byfy oceniane w skali $0-2$ punkty. Egzamin trwa1120 minut.

l. Naszkicować wykres funkcji $y=x|x+1|.$

2. Obliczyć $\cos^{2}105^{\mathrm{o}}-\sin^{2}105^{\mathrm{o}}$

3. Rozwiązać nierówność

$||x|-1|<2.$

4. Obliczyć granicę $\displaystyle \lim_{n\rightarrow\infty}(1-\frac{1}{2^{1}}+\frac{1}{2^{2}}-\frac{1}{2^{3}}+\ldots+(-1)^{n}\frac{1}{2^{n}}).$

5. Wektor $\vec{a}=[3$, 7$]$

$\mathrm{i}\vec{e}_{2}=[-1,1].$

przedstawić jako kombinację liniową wektorów $\vec{e}_{1}=[2$, 3$]$

6. Obliczyć granice $\displaystyle \lim_{x\rightarrow 0}x\sin\frac{1}{x}$

i

$\displaystyle \lim_{x\rightarrow+\infty}x\sin\frac{1}{x}.$

7. Dana jest funkcja $f(x)=\log_{\frac{1}{3}}(x+1)$. Rozwiązać nierówność $f(f(x))>0.$

8. Rozwiązać równanie $2^{2x}+4^{x}=5^{x}$

9. Podać równanie jednej z prostych, na której $\mathrm{l}\mathrm{e}\dot{\mathrm{z}}\mathrm{y}$ środek okręgu opisanego na

trójkącie o wierzcholkach $A(1,3), B(2,7)\mathrm{i}C(3,10).$

10. Dlajakich wartości parametru $k$ funkcja $f(x)=x^{3}-x^{2}+kx$ będzie rosnąca

w calym zbiorze liczb rzeczywistych?

ll. Dane są zbiory

$A=\{(x,y):(x-1)^{2}+y^{2}\leq 1\}$

oraz $B=\{(x,y):y\geq x\}.$

Naszkicować zbiór $A\cap B$ i obliczyć jego pole.

12. $\mathrm{W}$ oparciu o definicję pochodnej obliczyć $f'(1)$ dla funkcji $f(x)=\sqrt{x^{2}+3}.$

13. Zdarzenia losowe $A \mathrm{i} B$ są rozlączne i $P(A) =$

$P(A\cup B)$ oraz $P(A-B).$

$\displaystyle \frac{1}{3}$, a $P(B) = \displaystyle \frac{1}{2}$. Obliczyć

14. Napisać równanie sycznej do krzywej $y=x^{3}+x^{2}+x+1$ równolegfej do prostej

{\it y}$=$-32{\it x}.

15. Sformułować twierdzenie odwrotne do twierdzenia Pitagorasa.



\end{document}