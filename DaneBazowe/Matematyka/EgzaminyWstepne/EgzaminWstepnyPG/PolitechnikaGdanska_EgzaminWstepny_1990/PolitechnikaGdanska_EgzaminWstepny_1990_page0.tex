\documentclass[a4paper,12pt]{article}
\usepackage{latexsym}
\usepackage{amsmath}
\usepackage{amssymb}
\usepackage{graphicx}
\usepackage{wrapfig}
\pagestyle{plain}
\usepackage{fancybox}
\usepackage{bm}

\begin{document}

POLITECHNIKA $\mathrm{G}\mathrm{D}\mathrm{A}\acute{\mathrm{N}}$ SKA

Gdańsk, lipiec 1990 r.

Tematy I części egzaminu z matematyki

dla kandydatów ubiegających się o przyjęcie na I rok studiów dziennych.

Kandydat wybierał 3 dowo1ne zadania. Rozwiązania wybranych zadań oceniane

byly w skali $0-10$ punktów. Egzamin trwa1120 minut.

l. Zbadać przebieg zmienności funkcji

{\it y}$=$ -{\it xx}22 $+$-{\it xx} $++$11'

sporządzić jej wykres i na tej podstawie ustalič ile pierwiastków posiada rów-

nanie

--{\it xx}22$+$-{\it xx}$++$11$=${\it m}

w zalezności od parametru $m.$

2. Dlajakich wartości parametru $t$, przy dowolnej wartości parametru $k$, równa-

nie

$x^{2}+x\sqrt{k^{2}+4}-k\log_{\frac{1}{2}}(t+1)=0$

posiada dwa rózne pierwiastki?

3. Rozwiązać nierówność

$\displaystyle \lim_{n\rightarrow\infty}(\sqrt{n^{2}+(2+\sin 2x)n+4}-n)<1+\frac{1}{2}\cos 2x.$

4. Dwie kule o promieniach $R\mathrm{i}x(R>x)$ są styczne zewnętrznie. Przy jakim $x$

objętość stozka opisanego na tych kulach będzie najmniejsza?

5. $\mathrm{W}$ urnie $U_{1}$ znajdują się dwie kule czarne i pewna ilość kul bialych. $\mathrm{W}$ urnie

$U_{2}$ znajduje się 5 ku1 bia1ych i 3 czarne. $\mathrm{Z}$ pierwszej urny losujemy dwie kule i

przekładamy je do urny drugiej. Następnie z urny drugiej losujemy jedna kulę.

Podać minimalną ilość bialych kul znajdujących się w urnie $U_{1}$, jeśli wiadomo,

$\dot{\mathrm{z}}\mathrm{e}$ prawdopodobieństwo wylosowania kuli bialej z urny $U_{2}$ jest większe od 0, 6.
\end{document}
