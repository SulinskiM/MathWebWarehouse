\documentclass[a4paper,12pt]{article}
\usepackage{latexsym}
\usepackage{amsmath}
\usepackage{amssymb}
\usepackage{graphicx}
\usepackage{wrapfig}
\pagestyle{plain}
\usepackage{fancybox}
\usepackage{bm}

\begin{document}

POLITECHNIKA $\mathrm{G}\mathrm{D}\mathrm{A}\acute{\mathrm{N}}$ SKA

Gdańsk, 2.07.1991 r.

Tematy I części egzaminu z matematyki

dla kandydatów ubiegających się o przyjęcie na I rok studiów dziennych.

Kandydat wybierał 3 dowo1ne zadania. Rozwiązania wybranych zadań oceniane

byly w skali $0-10$ punktów. Egzamin trwa1120 minut.

l. Zbadać przebieg zmienności funkcji

{\it y}$=$ -4{\it xx}2 -$+$15

i na tej podstawie ustalić liczbę pierwiastków równania

-4{\it xx}2 -$+$51 $=${\it m}

w zalezności od parametru $m.$

2. $\mathrm{W}$ trójkącie $ABC$ dany jest wierzcholek $A(1,3)$ oraz równanie środkowej $y=7$

i równanie wysokości $x+4y-51=0$. Wiedząc, $\dot{\mathrm{z}}\mathrm{e}$ środkowa i wysokość wy-

chodzą z róznych wierzchofków trójkąta podać równania boków tego trójkąta.

3. Dla jakich wartości parametru $m\in R$ równanie

$\log_{2}(x+3)-2\log_{4}x=m$

posiada rozwiązanie nalezące do przedziafu $\langle$3; 4)?

4. $\mathrm{W}$ urnie znajdują się trzy kule biale o numerach 1, 2 $\mathrm{i}3$ oraz pięć kul czarnych

o numerach 1, 2, 3, 4 $\mathrm{i}5$. Losujemy bez zwracania dwukrotnie po jednej kuli.

Jakie jest prawdopodobieństwo tego, $\dot{\mathrm{z}}\mathrm{e}$ pierwsza z wylosowanych kul będzie

biala, a druga będzie kulą o numerze l?

5. Na trójkącie prostokątnym o kącie ostrym $x$ opisano okrąg. Okrąg ten i trójkąt

obracają się dookofa przeciwprostokątnej. Przy jakim $x$ stosunek objętości

kuli powstalej z obrotu okręgu do objętości bryly powstalej z obrotu trójkąta

będzie najmniejszy?




POLITECHNIKA $\mathrm{G}\mathrm{D}\mathrm{A}\acute{\mathrm{N}}$ SKA

Gdańsk, 2.07.1991 r.

Tematy II części egzaminu z matematyki

dla kandydatów ubiegajqcych się o przyjęcie na I rok studiów dziennych.

Wszystkie zadania byfy oceniane w skali $0-2$ punkty. Egzamin trwa1120 minut.

l. Dana jest funkcja $f(x)=\sin^{2}4x$. Rozwiązać równanie $f'(x)=-2.$

2. Rozwiązać nierówność $\log_{x}5<1.$

3. Dany jest trójkąt prostokątny o przyprostokątnych dlugości 3 $\mathrm{i}4$. Obliczyć

wysokość trójkąta poprowadzoną z wierzchofka kąta prostego.

4. Rozwiązać nierówność

$\displaystyle \frac{1}{x}>2-x.$

5. Rozwiązać nierównośč tg$(2x)\geq 1.$

6. $\mathrm{W}\mathrm{p}$laszczy $\acute{\mathrm{z}}\mathrm{n}\mathrm{i}\mathrm{e}0xy$ zaznaczyć punkty nalezące do zbioru

$A=\{(x,y):|x|<y\}.$

7. Obliczyć $\displaystyle \lim_{x\rightarrow 1}\frac{\sin 2(x-1)}{3(x^{2}-1)}.$

8. Podać resztę z dzielenia wielomianu $W(x) = 5x^{4}+2x^{2}+1$ przez dwumian

$x+1.$

9. $\mathrm{W}$ trójkącie o wierzcholkach $A(3,1,1), B(2,2,1) \mathrm{i}C(2,1,2)$ wyznaczyć kąt

wewnętrzny przy wierzcholku $A.$

10. Podać liczby naturalne spelniające nierówność $\left(\begin{array}{l}
n\\
2
\end{array}\right) -n\leq 14.$

ll. Dla jakich wartości parametru $k$ funkcja $f(x) = \displaystyle \frac{1}{3}x^{3}+\frac{3}{2}x^{2}+kx+1$ będzie

rosnąca w calej swojej dziedzinie?

12. Obliczyc $\displaystyle \lim_{n\rightarrow\infty}\frac{\sqrt{n^{2}+1}}{\sqrt[3]{8n^{3}+2n+1}}.$

13. Obliczyć prawdopodobieństwo wyrzucenia w pięciu rzutach kostkq co naj-

mniej raz liczby oczek nie większej od 3.

14. Napisać równanie prostej przechodzącej przez punkt $P(1,3)$ i prostopadfej do

prostej $y=2x+5.$

15. Suma wyrazów nieskończonego ciągu geometrycznego o pierwszym wyrazie

$a_{1}=3$ wynosi 5. Podać i1oraz tego ciągu.



\end{document}