\documentclass[a4paper,12pt]{article}
\usepackage{latexsym}
\usepackage{amsmath}
\usepackage{amssymb}
\usepackage{graphicx}
\usepackage{wrapfig}
\pagestyle{plain}
\usepackage{fancybox}
\usepackage{bm}

\begin{document}

POLITECHNIKA $\mathrm{G}\mathrm{D}\mathrm{A}\acute{\mathrm{N}}$ SKA

Gdańsk, 2.07.1991 r.

Tematy I części egzaminu z matematyki

dla kandydatów ubiegających się o przyjęcie na I rok studiów dziennych.

Kandydat wybierał 3 dowo1ne zadania. Rozwiązania wybranych zadań oceniane

byly w skali $0-10$ punktów. Egzamin trwa1120 minut.

l. Zbadać przebieg zmienności funkcji

{\it y}$=$ -4{\it xx}2 -$+$15

i na tej podstawie ustalić liczbę pierwiastków równania

-4{\it xx}2 -$+$51 $=${\it m}

w zalezności od parametru $m.$

2. $\mathrm{W}$ trójkącie $ABC$ dany jest wierzcholek $A(1,3)$ oraz równanie środkowej $y=7$

i równanie wysokości $x+4y-51=0$. Wiedząc, $\dot{\mathrm{z}}\mathrm{e}$ środkowa i wysokość wy-

chodzą z róznych wierzchofków trójkąta podać równania boków tego trójkąta.

3. Dla jakich wartości parametru $m\in R$ równanie

$\log_{2}(x+3)-2\log_{4}x=m$

posiada rozwiązanie nalezące do przedziafu $\langle$3; 4)?

4. $\mathrm{W}$ urnie znajdują się trzy kule biale o numerach 1, 2 $\mathrm{i}3$ oraz pięć kul czarnych

o numerach 1, 2, 3, 4 $\mathrm{i}5$. Losujemy bez zwracania dwukrotnie po jednej kuli.

Jakie jest prawdopodobieństwo tego, $\dot{\mathrm{z}}\mathrm{e}$ pierwsza z wylosowanych kul będzie

biala, a druga będzie kulą o numerze l?

5. Na trójkącie prostokątnym o kącie ostrym $x$ opisano okrąg. Okrąg ten i trójkąt

obracają się dookofa przeciwprostokątnej. Przy jakim $x$ stosunek objętości

kuli powstalej z obrotu okręgu do objętości bryly powstalej z obrotu trójkąta

będzie najmniejszy?
\end{document}
