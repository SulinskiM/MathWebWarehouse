\documentclass[a4paper,12pt]{article}
\usepackage{latexsym}
\usepackage{amsmath}
\usepackage{amssymb}
\usepackage{graphicx}
\usepackage{wrapfig}
\pagestyle{plain}
\usepackage{fancybox}
\usepackage{bm}

\begin{document}

POLITECHNIKA $\mathrm{G}\mathrm{D}\mathrm{A}\acute{\mathrm{N}}$ SKA

Gdańsk, 30.06.1992 r.

Tematy I części egzaminu z matematyki

dla kandydatów ubiegających się o przyjęcie na I rok studiów dziennych.

Kandydat wybierał 3 dowo1ne zadania. Rozwiązania wybranych zadań oceniane

byly w skali $0-10$ punktów. Egzamin trwa1120 minut.

l. Rozwiązać ukfad nierówności

$\left\{\begin{array}{l}
\sqrt{x+6}>x\\
2+\log_{0,5}(-x)>0
\end{array}\right.$

2. Dla jakich $a$ równanie

$\cos^{4}x+(a+2)\sin^{2}x-(2a+5)=0$

ma rozwiązanie?

3. Wykazać, $\dot{\mathrm{z}}\mathrm{e}$ pole trójkąta ograniczonego osiami ukladu wspólrzędnych i do-

wolną styczną do hiperboli $y=\displaystyle \frac{a^{2}}{x}$ jest równe $2a^{2}$

4. Wysokość stozka jest $x$ razy większa od promienia jego podstawy. Wyrazič

stosunek promieni kul opisanej i wpisanej w ten stozekjako funkcję $f(x)$ oraz

obliczyć granicę $\displaystyle \lim_{x\rightarrow+\infty} \underline{f(x)}.$

$x$

5. Dane są zbiory

$A=\{1$, 2, 3, $\ldots$, 222$\}$

$\mathrm{i} B=\{1$, 2, 3, $\ldots$, 444$\}.$

Losowo wybieramy zbiór, a z niego liczbę $x$. Obliczyć prawdopodobieństwo

tego, $\dot{\mathrm{z}}\mathrm{e}$ liczba $x^{2}+1$ dzieli się przez 10.
\end{document}
