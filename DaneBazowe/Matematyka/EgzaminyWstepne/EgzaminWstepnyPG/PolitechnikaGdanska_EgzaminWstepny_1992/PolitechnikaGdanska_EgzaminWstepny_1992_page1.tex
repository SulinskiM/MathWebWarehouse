\documentclass[a4paper,12pt]{article}
\usepackage{latexsym}
\usepackage{amsmath}
\usepackage{amssymb}
\usepackage{graphicx}
\usepackage{wrapfig}
\pagestyle{plain}
\usepackage{fancybox}
\usepackage{bm}

\begin{document}

POLITECHNIKA $\mathrm{G}\mathrm{D}\mathrm{A}\acute{\mathrm{N}}$ SKA

Gdańsk, 30.06.1992 r.

Tematy II części egzaminu z matematyki

dla kandydatów ubiegajqcych się o przyjęcie na I rok studiów dziennych.

Wszystkie zadania byfy oceniane w skali $0-2$ punkty. Egzamin trwa1120 minut.

l. Wyznaczyć dziedzinę funkcji $f(x)=\sqrt{\frac{5}{x+2}-1}.$

2. Rozwiązać równanie $\displaystyle \frac{\cos x}{1-\sin x}=1+\sin x.$

3. Narysować wykres funkcji $f(x)=x\displaystyle \sqrt{x^{2}}+\frac{x}{|x|}.$

4. Na paraboli $y=48-x^{2}$ znalez$\acute{}$ć wszystkie punkty $(x,y)$ takie, $\dot{\mathrm{z}}\mathrm{e}$ liczby 3, $x,$

$y$ tworzą ciąg geometryczny.

5. Wyznaczyć dziedzinę funkcji $f(x)=\log(3^{x}-5^{x}).$

6. Rózniczkując $\mathrm{t}\mathrm{o}\dot{\mathrm{z}}$ samość $\sin 2x=2\sin x\cos x$ wykazać $\mathrm{t}\mathrm{o}\dot{\mathrm{z}}$ samość $\cos 2x=$

$\sin^{2}x.$

$\cos^{2}x-$

7. Obliczyć $\displaystyle \lim_{x\rightarrow 0}\frac{x^{2}}{1-\cos 2x}.$

8. $\mathrm{W}$ trójkącie ostrokątnym $ABC$ z wierzchofków $A\mathrm{i}C$ opuszczono wysokości

{\it AD} $\mathrm{i}$ CE na boki $BC\mathrm{i}$ AB. Wykazać, $\dot{\mathrm{z}}\mathrm{e}$ trójkqty $ABC\mathrm{i}BDE$ są podobne.

9. Suma pierwiastków trójmianu $y = ax^{2}+bx+c$ jest równa $\log_{a^{2}}c\cdot\log_{c^{2}}a.$

Znalez$\acute{}$ć odciętq wierzchofka paraboli.

10. Dane są wektory $\overline{A}\mathfrak{F}=[1$, 2, 3$]\mathrm{i}\overline{A}7=[3$, 2, 1$]$. Obliczyć pole trójkąta $ABC.$

ll. Proste $P_{1}, P_{2} \mathrm{i}P_{3}$ są równoległe i lezą w jednej płaszczyz$\acute{}$nie. Na prostej $P_{1}$

wybrano 3 punkty, na $\ell_{2}$ wybrano 4 punkty, a na $\ell_{3}$ wybrano 5 punktów. I1e

co najwyzej istnieje trójkątów o wierzchofkach w tych punktach?

12. Obliczyč $\displaystyle \lim_{n\rightarrow\infty}\frac{1+4+7+\ldots+(3n-2)}{2n^{2}+3n+4}.$

13. Wykazać, $\dot{\mathrm{z}}\mathrm{e}$ finkcja $f(x) = \sqrt{1+x+x^{2}}-\sqrt{1-x+x^{2}}$ jest nieparzysta w

swojej dziedzinie.

14. Dany jest trójkat o wierzcholkach $A(1,-1), B(3,3)\mathrm{i}C(-5,1)$. Napisać rów-

nanie symetralnej boku $\overline{BC}.$

15. Zbadać monotoniczność funkcji $f(x)=x^{4}-\displaystyle \frac{1}{x}+5$ w przedziale $(0;+\infty).$
\end{document}
