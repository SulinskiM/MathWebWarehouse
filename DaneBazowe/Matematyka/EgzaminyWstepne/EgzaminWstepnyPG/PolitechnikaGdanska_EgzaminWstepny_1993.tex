\documentclass[a4paper,12pt]{article}
\usepackage{latexsym}
\usepackage{amsmath}
\usepackage{amssymb}
\usepackage{graphicx}
\usepackage{wrapfig}
\pagestyle{plain}
\usepackage{fancybox}
\usepackage{bm}

\begin{document}

POLITECHNIKA $\mathrm{G}\mathrm{D}\mathrm{A}\acute{\mathrm{N}}$ SKA

Gdańsk, 30.06.1993 r.

EGZAMIN WSTĘPNY Z MATEMATYKI

Zestaw sklada się z 30 zadań. Zadania $1-10$ oceniane będą w skali $0-2$ punkty, zadania

$11-30$ w skali $0-4$ punkty. Czas trwania egzaminu -- 180 minut.

{\it Powodzenia}.'

l. Obliczyć

$\displaystyle \lim_{n\rightarrow\infty}\frac{n\sqrt{1+3+5++(2n-1)}}{2n^{2}+n+1}.$

2. Rozwiązać nierówność $x^{2}-4x+9\displaystyle \leq\frac{18}{x+2}.$

3. Rozwiązać nierówność

$\log_{0,3}(x+1)>-1.$

4. Rozwiązać nierówność $2-|1-2x|>1.$

5. Dla jakich wartości parametru $\alpha \in (0;2\pi)$ równanie $\sin 2x=  2\cos\alpha$ posiada roz-

wiązanie?

6. Obliczyč dlugośč wektora $\vec{a}, \mathrm{j}\mathrm{e}\dot{\mathrm{z}}$ eli ã $0\vec{b}=7,$

$\vec{a}\Vert\vec{b} \mathrm{i} \vec{b}=[3,-2,1].$

7. Rozwiązać nierówność $2^{x^{2}}<5^{x}$

8. Wykazać, $\dot{\mathrm{z}}\mathrm{e}$ funkcja $f(x)=3x^{3}+4x+\cos 2x$ jest rosnąca w calej swojej dziedzinie.

9. Wyznaczyc te wartości parametru $k$, dla których prosta $y=kx+4$ będzie równolegla

do prostej 

10. Dla jakich $a\mathrm{i}b$ wielomian $W(x)=12x^{4}-17x^{2}+ax+b$ dzieli się bez reszty przez

$2x^{2}+x-1$?

ll. Dany jest trójkat o wierzchofkach $A(1,1), B(-1,3), C(3,7)$ i polu $S$. Przez wierz-

cholek $A$ poprowadzić jedną z prostych, ktora dzieli dany trójkąt na dwa trójkąty

o polach $\displaystyle \frac{1}{4}S\mathrm{i}\frac{3}{4}S$. Podać równanie tej prostej.

12. Znalez$\acute{}$ć ekstrema funkcji $f(x) = (x+3)^{2}(x+8)^{3}$

$f(x)=108$?

Ile pierwiastków ma równanie

13. Dla jakiej wartości parametru a funkcja

$f(x)=$

dla

dla

$x\neq 0$

$x=0$

będzie funkcją ciąglą w punkcie $x=0$?

14. Który z punktów paraboli $y=x^{2}$ jest polozony najblizej prostej $y=2x-2$?




15. Wykazać, $\dot{\mathrm{z}}\mathrm{e}$ pole dowolnego wypukfego czworokątajest równe pofowie iloczynu jego

przekątnych pomnozonego przez sinus kąta między nimi, $S=\displaystyle \frac{1}{2}d_{1}d_{2}\sin\alpha.$

16. Dany jest ciąg arytmetyczny (o róznicy róznej od zera), w którym suma $n$ począt-

kowych wyrazów jest równa polowie sumy następnych $n$ wyrazów. Wyznaczyć iloraz

$\displaystyle \frac{S_{3n}}{S_{n}}$, gdzie $S_{k}$ oznacza sumę $k$ początkowych wyrazów tego ciągu.

17. Wykazać, $\dot{\mathrm{z}}\mathrm{e}$ dwie styczne do paraboli $y=x^{2}$ poprowadzone z dowolnego punktu

prostej $y=-\displaystyle \frac{1}{4}$ są do siebie prostopadle.

18. Dany jest trójkąt równoramienny o ramionach $\overline{AC}\mathrm{i}\overline{BC}$ dlugości 3 cm i podstawie

$\overline{AB}$ dlugości 4 cm. Ob1iczyć i1oczyn ska1arny AS o $\overline{B}7.$

19. Miary kątów wewnętrznych trójkąta tworzą ciąg arytmetyczny. Najmniejszy bok

jest trzy razy mniejszy od największego boku w tym trójkącie. Obliczyć cosinus

najmniejszego kąta.

20. Ze zbioru liczb \{l, 2, 3, 4, 5, 6, 7, 8, 9, l0\} losujemy dwukrotnie po jednej liczbie

bez zwracania. Obliczyć prawdopodobieństwo tego, $\dot{\mathrm{z}}\mathrm{e}$ druga z wylosowanych liczb

będzie większa od pierwszej.

21. Podać definicję asymptoty pionowej i wyznaczyć asymptoty pionowe funkcji $f(x)=$

$\displaystyle \frac{1}{x(2^{x}-4)}.$

22. Wyznaczyć najmniejszą i największą wartość funkcji $f(x)=\displaystyle \cos(\frac{\pi}{2}\cdot x)-3x$ w prze-

dziale $\langle 0;1\rangle.$

23. Dla jakiej wartości parametru $m$ okrąg $(x-m)^{2}+(y-1)^{2}=1$ będzie styczny do

prostej $3x+4y+1=0$?

24. Wykazać, $\dot{\mathrm{z}}\mathrm{e}$ równanie $x=\displaystyle \frac{1}{2}\sin x+a$, gdzie $a>0$, ma dokladnie jeden pierwiastek

w przedziale $\langle 0;a+1\rangle.$

25. $\mathrm{Z}$ definicji pochodnej obliczyć $f'(3)$, gdy $f(x)=\sqrt{2x+3}.$

26. Rozwiązač równanie

$\left(\begin{array}{l}
x+3\\
2
\end{array}\right)+\left(\begin{array}{l}
x+1\\
x-1
\end{array}\right)=31.$

27. Dlugość dluzszej podstawy trapezu równoramiennego jest równa l3 cm, a jego ob-

wódjest równy 28 cm. Wyrazić po1e trapezujako funkcję d1ugości ramienia trapezu.

Znalez$\acute{}$ć dziedzinę i zbiór wartości tej funkcji.

28. Dla jakich wartości parametru $k$ ciąg $(a_{n})$, gdzie $a_{n} = \displaystyle \frac{n^{k}}{2+4+\ldots+2n}$, będzie

rozbiezny do $+\infty$?

29. Dana jest funkcja $f(x)=\displaystyle \cos^{2}3x+\frac{3}{2}x-$ log5. Rozwiązać równanie $f'(\displaystyle \frac{1}{3}x)=0.$

30. Dane są liczby $A = \displaystyle \frac{5678901234}{6789012345} \mathrm{i} B = \displaystyle \frac{5678901235}{6789012346}$. Która z nich jest większa?

Swoją odpowied $\acute{\mathrm{z}}$ uzasadnić.



\end{document}