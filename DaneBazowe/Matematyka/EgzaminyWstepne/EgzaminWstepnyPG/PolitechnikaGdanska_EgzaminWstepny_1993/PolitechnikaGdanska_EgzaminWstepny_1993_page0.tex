\documentclass[a4paper,12pt]{article}
\usepackage{latexsym}
\usepackage{amsmath}
\usepackage{amssymb}
\usepackage{graphicx}
\usepackage{wrapfig}
\pagestyle{plain}
\usepackage{fancybox}
\usepackage{bm}

\begin{document}

POLITECHNIKA $\mathrm{G}\mathrm{D}\mathrm{A}\acute{\mathrm{N}}$ SKA

Gdańsk, 30.06.1993 r.

EGZAMIN WSTĘPNY Z MATEMATYKI

Zestaw sklada się z 30 zadań. Zadania $1-10$ oceniane będą w skali $0-2$ punkty, zadania

$11-30$ w skali $0-4$ punkty. Czas trwania egzaminu -- 180 minut.

{\it Powodzenia}.'

l. Obliczyć

$\displaystyle \lim_{n\rightarrow\infty}\frac{n\sqrt{1+3+5++(2n-1)}}{2n^{2}+n+1}.$

2. Rozwiązać nierówność $x^{2}-4x+9\displaystyle \leq\frac{18}{x+2}.$

3. Rozwiązać nierówność

$\log_{0,3}(x+1)>-1.$

4. Rozwiązać nierówność $2-|1-2x|>1.$

5. Dla jakich wartości parametru $\alpha \in (0;2\pi)$ równanie $\sin 2x=  2\cos\alpha$ posiada roz-

wiązanie?

6. Obliczyč dlugośč wektora $\vec{a}, \mathrm{j}\mathrm{e}\dot{\mathrm{z}}$ eli ã $0\vec{b}=7,$

$\vec{a}\Vert\vec{b} \mathrm{i} \vec{b}=[3,-2,1].$

7. Rozwiązać nierówność $2^{x^{2}}<5^{x}$

8. Wykazać, $\dot{\mathrm{z}}\mathrm{e}$ funkcja $f(x)=3x^{3}+4x+\cos 2x$ jest rosnąca w calej swojej dziedzinie.

9. Wyznaczyc te wartości parametru $k$, dla których prosta $y=kx+4$ będzie równolegla

do prostej 

10. Dla jakich $a\mathrm{i}b$ wielomian $W(x)=12x^{4}-17x^{2}+ax+b$ dzieli się bez reszty przez

$2x^{2}+x-1$?

ll. Dany jest trójkat o wierzchofkach $A(1,1), B(-1,3), C(3,7)$ i polu $S$. Przez wierz-

cholek $A$ poprowadzić jedną z prostych, ktora dzieli dany trójkąt na dwa trójkąty

o polach $\displaystyle \frac{1}{4}S\mathrm{i}\frac{3}{4}S$. Podać równanie tej prostej.

12. Znalez$\acute{}$ć ekstrema funkcji $f(x) = (x+3)^{2}(x+8)^{3}$

$f(x)=108$?

Ile pierwiastków ma równanie

13. Dla jakiej wartości parametru a funkcja

$f(x)=$

dla

dla

$x\neq 0$

$x=0$

będzie funkcją ciąglą w punkcie $x=0$?

14. Który z punktów paraboli $y=x^{2}$ jest polozony najblizej prostej $y=2x-2$?
\end{document}
