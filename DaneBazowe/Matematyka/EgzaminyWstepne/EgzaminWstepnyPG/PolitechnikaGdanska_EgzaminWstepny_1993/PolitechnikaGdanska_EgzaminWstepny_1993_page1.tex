\documentclass[a4paper,12pt]{article}
\usepackage{latexsym}
\usepackage{amsmath}
\usepackage{amssymb}
\usepackage{graphicx}
\usepackage{wrapfig}
\pagestyle{plain}
\usepackage{fancybox}
\usepackage{bm}

\begin{document}

15. Wykazać, $\dot{\mathrm{z}}\mathrm{e}$ pole dowolnego wypukfego czworokątajest równe pofowie iloczynu jego

przekątnych pomnozonego przez sinus kąta między nimi, $S=\displaystyle \frac{1}{2}d_{1}d_{2}\sin\alpha.$

16. Dany jest ciąg arytmetyczny (o róznicy róznej od zera), w którym suma $n$ począt-

kowych wyrazów jest równa polowie sumy następnych $n$ wyrazów. Wyznaczyć iloraz

$\displaystyle \frac{S_{3n}}{S_{n}}$, gdzie $S_{k}$ oznacza sumę $k$ początkowych wyrazów tego ciągu.

17. Wykazać, $\dot{\mathrm{z}}\mathrm{e}$ dwie styczne do paraboli $y=x^{2}$ poprowadzone z dowolnego punktu

prostej $y=-\displaystyle \frac{1}{4}$ są do siebie prostopadle.

18. Dany jest trójkąt równoramienny o ramionach $\overline{AC}\mathrm{i}\overline{BC}$ dlugości 3 cm i podstawie

$\overline{AB}$ dlugości 4 cm. Ob1iczyć i1oczyn ska1arny AS o $\overline{B}7.$

19. Miary kątów wewnętrznych trójkąta tworzą ciąg arytmetyczny. Najmniejszy bok

jest trzy razy mniejszy od największego boku w tym trójkącie. Obliczyć cosinus

najmniejszego kąta.

20. Ze zbioru liczb \{l, 2, 3, 4, 5, 6, 7, 8, 9, l0\} losujemy dwukrotnie po jednej liczbie

bez zwracania. Obliczyć prawdopodobieństwo tego, $\dot{\mathrm{z}}\mathrm{e}$ druga z wylosowanych liczb

będzie większa od pierwszej.

21. Podać definicję asymptoty pionowej i wyznaczyć asymptoty pionowe funkcji $f(x)=$

$\displaystyle \frac{1}{x(2^{x}-4)}.$

22. Wyznaczyć najmniejszą i największą wartość funkcji $f(x)=\displaystyle \cos(\frac{\pi}{2}\cdot x)-3x$ w prze-

dziale $\langle 0;1\rangle.$

23. Dla jakiej wartości parametru $m$ okrąg $(x-m)^{2}+(y-1)^{2}=1$ będzie styczny do

prostej $3x+4y+1=0$?

24. Wykazać, $\dot{\mathrm{z}}\mathrm{e}$ równanie $x=\displaystyle \frac{1}{2}\sin x+a$, gdzie $a>0$, ma dokladnie jeden pierwiastek

w przedziale $\langle 0;a+1\rangle.$

25. $\mathrm{Z}$ definicji pochodnej obliczyć $f'(3)$, gdy $f(x)=\sqrt{2x+3}.$

26. Rozwiązač równanie

$\left(\begin{array}{l}
x+3\\
2
\end{array}\right)+\left(\begin{array}{l}
x+1\\
x-1
\end{array}\right)=31.$

27. Dlugość dluzszej podstawy trapezu równoramiennego jest równa l3 cm, a jego ob-

wódjest równy 28 cm. Wyrazić po1e trapezujako funkcję d1ugości ramienia trapezu.

Znalez$\acute{}$ć dziedzinę i zbiór wartości tej funkcji.

28. Dla jakich wartości parametru $k$ ciąg $(a_{n})$, gdzie $a_{n} = \displaystyle \frac{n^{k}}{2+4+\ldots+2n}$, będzie

rozbiezny do $+\infty$?

29. Dana jest funkcja $f(x)=\displaystyle \cos^{2}3x+\frac{3}{2}x-$ log5. Rozwiązać równanie $f'(\displaystyle \frac{1}{3}x)=0.$

30. Dane są liczby $A = \displaystyle \frac{5678901234}{6789012345} \mathrm{i} B = \displaystyle \frac{5678901235}{6789012346}$. Która z nich jest większa?

Swoją odpowied $\acute{\mathrm{z}}$ uzasadnić.
\end{document}
