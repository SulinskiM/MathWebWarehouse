\documentclass[a4paper,12pt]{article}
\usepackage{latexsym}
\usepackage{amsmath}
\usepackage{amssymb}
\usepackage{graphicx}
\usepackage{wrapfig}
\pagestyle{plain}
\usepackage{fancybox}
\usepackage{bm}

\begin{document}

POLITECHNIKA $\mathrm{G}\mathrm{D}\mathrm{A}\acute{\mathrm{N}}$ SKA

Gdańsk, czerwiec 1994 r.

EGZAMIN WSTĘPNY Z MATEMATYKI

Zestaw składa się z 30 zadań. Zadania $1-10$ oceniane będą w skali $0-2$ punkty,

zadania $11-30$ w skali $0-4$ punkty. Czas trwania egzaminu -- 240 minut.

{\it Powodzenia}.'

l. Rozwiązać nierówność $x-\displaystyle \frac{2}{x}\geq 1.$

2. Dla jakich a równanie

rzeczywisty?

$x^{2}+ax+a-1=0$ posiada co najmniej jeden pierwiastek

3. Rozwiązać równanie $\sqrt{x}+2=x.$

4. Trzy liczby tworzą ciag arytmetyczny o sumie równej l8. Największa z nich jest

równa 9. Wyznaczyć pozosta1e 1iczby.

5. Rozwiązać nierówność $(\displaystyle \frac{1}{2})^{|x-3|}\geq\frac{1}{4}.$

6. Dany jest sześcian o krawędzi $a$ Obliczyć objętość kuli stycznej do wszystkich

krawędzi tego sześcianu.

7. Obliczyć $(\sqrt[3]{4})^{\frac{3}{2\log_{3}2}}$

8. Dla jakich $x\in(0;\pi)$ spefniona jest nierówność $\mathrm{c}\mathrm{t}\mathrm{g}^{2}x\geq 3$?

9. Obliczyć granicę $\displaystyle \lim_{n\rightarrow\infty}\frac{(n+2)!}{n^{2}\cdot n!}.$

10. Graficznie rozwiązać nierówność $\log_{\frac{1}{2}}|x|\geq x^{2}-1.$

ll. Wielomian $w(x)=x^{3}-3x+a$ rozfozyć na czynniki wiedząc, $\dot{\mathrm{z}}\mathrm{e}$ liczba $-1$ jest

jego pierwiastkiem.

12. Dla jakich parametrów m uklad równań

$\left\{\begin{array}{l}
mx-2y=1\\
8x-my=2
\end{array}\right.$

jest sprzeczny?

13. Trójkąt ma boki długości 6, 8 i l0. Obliczyć promień okręgu opisanego na tym

trójkącie i promień okręgu wpisanego w ten trójkąt.

14. Napisać równanie stycznej do wykresu funkcji

$x_{0}=0.$

$f(x)=4\sqrt[3]{8+\sin 3x}$ w punkcie

15. Dlajakich wartości parametru $m$ okręgi $x^{2}+y^{2}-2x=0$ oraz $x^{2}+(y-m)^{2}=9$

są styczne wewnętrznie?




16. Wyznaczyć dziedzinę funkcji

$f(x)=\sqrt{\log(3^{x}-2^{x}+1)}.$

17. Trzy razy rzucamy dwiema kostkami do gry. Jakiejest prawdopodobieństwo tego,

$\dot{\mathrm{z}}\mathrm{e}$ co najmniej raz suma oczek będzie większa od 9?

18. Obliczyć granicę $\displaystyle \lim_{x\rightarrow 0}\log_{2}(\frac{x^{2}}{1-\cos 4x}).$

19. Niech $f(m)$ oznacza liczbę pierwiastków równania $|4x^{2}-4x-3|=m$. Narysować

wykres funkcji $f(m).$

20. Na prostej $y-x-1=0$ znalez$\acute{}$ć punkt $A$ taki, $\dot{\mathrm{z}}\mathrm{e}$ pole trójkąta o wierzchofkach

w punktach $A, B(4,-1)\mathrm{i}C(4,3)$ jest równe 2.

21. Obliczyć kat między wektorami ã $\mathrm{i}\vec{b}$, jeśli wiadomo, $\dot{\mathrm{z}}\mathrm{e}$ wektory ũ $= -\vec{a}+4\vec{b}$

$\mathrm{i}\vec{v}=3\vec{a}+2\vec{b}$ są prostopadle i lãl $=|\vec{b}|=1.$

22. Uzasadnić, $\dot{\mathrm{z}}\mathrm{e}$ prosta $4x+2y-3=0$jest równolegfa do prostej 

Obliczyć odleglość między tymi prostymi.

$= -t+1$

$= 2t-3$

23. Zbadać monotoniczność funkcji $f(x)=x^{3}-3x^{2}+4x+\cos x.$

24. $\mathrm{W}$ trapez równoramienny o polu $S$ wpisano czworokąt tak, $\dot{\mathrm{z}}$ ejego wierzcholki sa

środkami boków trapezu. Jaki to czworokąt? Obliczyć jego pole.

25. Niech $A\mathrm{i}B$ będą zdarzeniami losowymi takimi, $\dot{\mathrm{z}}\mathrm{e}P(A) =0, 7\mathrm{i}P(B) =0$, 9.

Wykazać, $\displaystyle \dot{\mathrm{z}}\mathrm{e}P(A|B)\geq\frac{2}{3}.$

26. Obliczyć granice $\displaystyle \lim_{x\rightarrow+\infty}(x-\sqrt{x^{2}-x+1})$

oraz $\displaystyle \lim_{x\rightarrow-\infty}(x-\sqrt{x^{2}-x+1}).$

27. Rozwiązač równanie $1+\displaystyle \frac{\mathrm{l}}{2\sin x}+\frac{\mathrm{l}}{4\sin^{2}x}+\cdots=\frac{2}{\sin x}.$

28. Wyznaczyć największą i najmniejsza wartość funkcji

przedziale $\displaystyle \langle 0;\frac{\pi}{2}\rangle.$

$f(x) = \displaystyle \frac{1}{\sin x+\cos x}$

w

29. Podać definicję ciągu ograniczonego. Następnie wykazać, $\dot{\mathrm{z}}\mathrm{e}$ ciąg

$a_{n}=\displaystyle \frac{1}{n+1}+\frac{1}{n+2}+\cdots+\frac{1}{2n}$

jest ograniczony.

30. Podać i udowodnić warunek konieczny istnienia maksimum lokalnego funkcji róz-

niczkowalnej.





Odpowiedzi do kolejnych zadań:

1. $x\in\langle-1;0)\cup\langle 2;+\infty)$ ;

2. dla $\mathrm{k}\mathrm{a}\dot{\mathrm{z}}$ dej liczby $a\in R$;

3. $x=4$;

4. 1iczbami tymi są $a_{1}=3, a_{2}=6\mathrm{i}a_{3}=9$;

5. $ x\in\langle 1;5\rangle$;

6. $V=\displaystyle \frac{1}{3}\pi a^{3}\sqrt{2}$;

7.

$(\sqrt[3]{4})^{\frac{3}{2\log_{3}2}}=3$;

8. $x\displaystyle \in(0;\frac{\pi}{6}\rangle\cup\langle\frac{5}{6}\pi;\pi)$ ;

9. 1;

10. $ x\in\langle-1;0)\cup(0;1\rangle$;

11. $w(x)=(x+1)^{2}(x-2)$ ;

12. $m=-4$;

13. $R=5\mathrm{i}r=2$;

14. $y=x+8$;

15. $m=\pm\sqrt{3}$;

16. $x\geq 0$;

17. $P=\displaystyle \frac{91}{216}$;

18. $-3$;
\begin{center}
\includegraphics[width=72.036mm,height=58.008mm]{./PolitechnikaGdanska_EgzaminWstepny_1994_page2_images/image001.eps}
\end{center}
4  0

3

2

1

0 l  2 3 4

19.

20. A(3,4) lub A(5,6) ;

21.

$\displaystyle \frac{2}{3}\pi$;





22. $d=\displaystyle \frac{\sqrt{5}}{2}$;

23. Zauwazmy, $\dot{\mathrm{z}}\mathrm{e}$ spelnionajest nierównośč $3x^{2}-6x+4\geq 1$ (i nawet $3x^{2}-6x+4>1,$

gdy $x\neq 1)$. Stąd $\mathrm{j}\mathrm{u}\dot{\mathrm{z}}$ wynika, $\dot{\mathrm{z}}\mathrm{e}$ pochodna funkcji $f(x)$ jest dodatnia,

$f'(x)=3x^{2}-6x+4-\sin x>0$ (takze dla $x=1$),

i dlatego fnkcja $f(x)$ jest rosnąca;

24. czworokąt jest rombem o polu $P=S/2$;

25. $P(A|B)=\displaystyle \frac{P(A\cap B)}{P(B)}=\frac{P(A)+P(B)-P(A\cup B)}{P(B)}\geq\frac{0,7+0,9-1}{0,9}=\frac{2}{3}$;

26. 1/2 $\mathrm{i}-\infty$;

27. $x=\displaystyle \frac{\pi}{2}+2k\pi \mathrm{i}k$ jest liczbą cafkowitą;

28. $M=1\displaystyle \mathrm{i}m=\frac{\sqrt{2}}{2}$;

29. $(a_{n})$ jest ograniczony, gdy istnieje liczba rzeczywista $M$ taka, $\dot{\mathrm{z}}\mathrm{e} |a_{n}| \leq M$ dla

$\mathrm{k}\mathrm{a}\dot{\mathrm{z}}$ dej liczby naturalnej $n$. Dla rozwazanego ciągu i dla $\mathrm{k}\mathrm{a}\dot{\mathrm{z}}$ dej liczby naturalnej

$n$ jest

$|\displaystyle \alpha_{n}|=a_{n}=\frac{1}{n+1}+\frac{1}{n+2}+\cdots+\frac{1}{2n}\leq\frac{1}{n+1}+\frac{1}{n+1}+\cdots+\frac{1}{n+1}=\frac{n}{n+1}\leq 1,$

więc ciąg ten jest ograniczony.

30. Jeśli funkcja $f(x)$ jest rózniczkowalna w punkcie $x_{0}$ i jeśli ma ona maksimum

lokalne w tym punkcie, to $f'(x_{0})=0.$



\end{document}