\documentclass[a4paper,12pt]{article}
\usepackage{latexsym}
\usepackage{amsmath}
\usepackage{amssymb}
\usepackage{graphicx}
\usepackage{wrapfig}
\pagestyle{plain}
\usepackage{fancybox}
\usepackage{bm}

\begin{document}

POLITECHNIKA $\mathrm{G}\mathrm{D}\mathrm{A}\acute{\mathrm{N}}$ SKA

Gdańsk, czerwiec 1994 r.

EGZAMIN WSTĘPNY Z MATEMATYKI

Zestaw składa się z 30 zadań. Zadania $1-10$ oceniane będą w skali $0-2$ punkty,

zadania $11-30$ w skali $0-4$ punkty. Czas trwania egzaminu -- 240 minut.

{\it Powodzenia}.'

l. Rozwiązać nierówność $x-\displaystyle \frac{2}{x}\geq 1.$

2. Dla jakich a równanie

rzeczywisty?

$x^{2}+ax+a-1=0$ posiada co najmniej jeden pierwiastek

3. Rozwiązać równanie $\sqrt{x}+2=x.$

4. Trzy liczby tworzą ciag arytmetyczny o sumie równej l8. Największa z nich jest

równa 9. Wyznaczyć pozosta1e 1iczby.

5. Rozwiązać nierówność $(\displaystyle \frac{1}{2})^{|x-3|}\geq\frac{1}{4}.$

6. Dany jest sześcian o krawędzi $a$ Obliczyć objętość kuli stycznej do wszystkich

krawędzi tego sześcianu.

7. Obliczyć $(\sqrt[3]{4})^{\frac{3}{2\log_{3}2}}$

8. Dla jakich $x\in(0;\pi)$ spefniona jest nierówność $\mathrm{c}\mathrm{t}\mathrm{g}^{2}x\geq 3$?

9. Obliczyć granicę $\displaystyle \lim_{n\rightarrow\infty}\frac{(n+2)!}{n^{2}\cdot n!}.$

10. Graficznie rozwiązać nierówność $\log_{\frac{1}{2}}|x|\geq x^{2}-1.$

ll. Wielomian $w(x)=x^{3}-3x+a$ rozfozyć na czynniki wiedząc, $\dot{\mathrm{z}}\mathrm{e}$ liczba $-1$ jest

jego pierwiastkiem.

12. Dla jakich parametrów m uklad równań

$\left\{\begin{array}{l}
mx-2y=1\\
8x-my=2
\end{array}\right.$

jest sprzeczny?

13. Trójkąt ma boki długości 6, 8 i l0. Obliczyć promień okręgu opisanego na tym

trójkącie i promień okręgu wpisanego w ten trójkąt.

14. Napisać równanie stycznej do wykresu funkcji

$x_{0}=0.$

$f(x)=4\sqrt[3]{8+\sin 3x}$ w punkcie

15. Dlajakich wartości parametru $m$ okręgi $x^{2}+y^{2}-2x=0$ oraz $x^{2}+(y-m)^{2}=9$

są styczne wewnętrznie?
\end{document}
