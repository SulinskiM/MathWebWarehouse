\documentclass[a4paper,12pt]{article}
\usepackage{latexsym}
\usepackage{amsmath}
\usepackage{amssymb}
\usepackage{graphicx}
\usepackage{wrapfig}
\pagestyle{plain}
\usepackage{fancybox}
\usepackage{bm}

\begin{document}

16. Wyznaczyć dziedzinę funkcji

$f(x)=\sqrt{\log(3^{x}-2^{x}+1)}.$

17. Trzy razy rzucamy dwiema kostkami do gry. Jakiejest prawdopodobieństwo tego,

$\dot{\mathrm{z}}\mathrm{e}$ co najmniej raz suma oczek będzie większa od 9?

18. Obliczyć granicę $\displaystyle \lim_{x\rightarrow 0}\log_{2}(\frac{x^{2}}{1-\cos 4x}).$

19. Niech $f(m)$ oznacza liczbę pierwiastków równania $|4x^{2}-4x-3|=m$. Narysować

wykres funkcji $f(m).$

20. Na prostej $y-x-1=0$ znalez$\acute{}$ć punkt $A$ taki, $\dot{\mathrm{z}}\mathrm{e}$ pole trójkąta o wierzchofkach

w punktach $A, B(4,-1)\mathrm{i}C(4,3)$ jest równe 2.

21. Obliczyć kat między wektorami ã $\mathrm{i}\vec{b}$, jeśli wiadomo, $\dot{\mathrm{z}}\mathrm{e}$ wektory ũ $= -\vec{a}+4\vec{b}$

$\mathrm{i}\vec{v}=3\vec{a}+2\vec{b}$ są prostopadle i lãl $=|\vec{b}|=1.$

22. Uzasadnić, $\dot{\mathrm{z}}\mathrm{e}$ prosta $4x+2y-3=0$jest równolegfa do prostej 

Obliczyć odleglość między tymi prostymi.

$= -t+1$

$= 2t-3$

23. Zbadać monotoniczność funkcji $f(x)=x^{3}-3x^{2}+4x+\cos x.$

24. $\mathrm{W}$ trapez równoramienny o polu $S$ wpisano czworokąt tak, $\dot{\mathrm{z}}$ ejego wierzcholki sa

środkami boków trapezu. Jaki to czworokąt? Obliczyć jego pole.

25. Niech $A\mathrm{i}B$ będą zdarzeniami losowymi takimi, $\dot{\mathrm{z}}\mathrm{e}P(A) =0, 7\mathrm{i}P(B) =0$, 9.

Wykazać, $\displaystyle \dot{\mathrm{z}}\mathrm{e}P(A|B)\geq\frac{2}{3}.$

26. Obliczyć granice $\displaystyle \lim_{x\rightarrow+\infty}(x-\sqrt{x^{2}-x+1})$

oraz $\displaystyle \lim_{x\rightarrow-\infty}(x-\sqrt{x^{2}-x+1}).$

27. Rozwiązač równanie $1+\displaystyle \frac{\mathrm{l}}{2\sin x}+\frac{\mathrm{l}}{4\sin^{2}x}+\cdots=\frac{2}{\sin x}.$

28. Wyznaczyć największą i najmniejsza wartość funkcji

przedziale $\displaystyle \langle 0;\frac{\pi}{2}\rangle.$

$f(x) = \displaystyle \frac{1}{\sin x+\cos x}$

w

29. Podać definicję ciągu ograniczonego. Następnie wykazać, $\dot{\mathrm{z}}\mathrm{e}$ ciąg

$a_{n}=\displaystyle \frac{1}{n+1}+\frac{1}{n+2}+\cdots+\frac{1}{2n}$

jest ograniczony.

30. Podać i udowodnić warunek konieczny istnienia maksimum lokalnego funkcji róz-

niczkowalnej.
\end{document}
