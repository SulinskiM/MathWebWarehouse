\documentclass[a4paper,12pt]{article}
\usepackage{latexsym}
\usepackage{amsmath}
\usepackage{amssymb}
\usepackage{graphicx}
\usepackage{wrapfig}
\pagestyle{plain}
\usepackage{fancybox}
\usepackage{bm}

\begin{document}

Odpowiedzi do kolejnych zadań:

1. $x\in\langle-1;0)\cup\langle 2;+\infty)$ ;

2. dla $\mathrm{k}\mathrm{a}\dot{\mathrm{z}}$ dej liczby $a\in R$;

3. $x=4$;

4. 1iczbami tymi są $a_{1}=3, a_{2}=6\mathrm{i}a_{3}=9$;

5. $ x\in\langle 1;5\rangle$;

6. $V=\displaystyle \frac{1}{3}\pi a^{3}\sqrt{2}$;

7.

$(\sqrt[3]{4})^{\frac{3}{2\log_{3}2}}=3$;

8. $x\displaystyle \in(0;\frac{\pi}{6}\rangle\cup\langle\frac{5}{6}\pi;\pi)$ ;

9. 1;

10. $ x\in\langle-1;0)\cup(0;1\rangle$;

11. $w(x)=(x+1)^{2}(x-2)$ ;

12. $m=-4$;

13. $R=5\mathrm{i}r=2$;

14. $y=x+8$;

15. $m=\pm\sqrt{3}$;

16. $x\geq 0$;

17. $P=\displaystyle \frac{91}{216}$;

18. $-3$;
\begin{center}
\includegraphics[width=72.036mm,height=58.008mm]{./PolitechnikaGdanska_EgzaminWstepny_1994_page2_images/image001.eps}
\end{center}
4  0

3

2

1

0 l  2 3 4

19.

20. A(3,4) lub A(5,6) ;

21.

$\displaystyle \frac{2}{3}\pi$;
\end{document}
