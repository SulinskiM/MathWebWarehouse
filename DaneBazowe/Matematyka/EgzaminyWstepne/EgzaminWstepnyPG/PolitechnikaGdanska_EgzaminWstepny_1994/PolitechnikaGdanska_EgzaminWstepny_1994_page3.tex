\documentclass[a4paper,12pt]{article}
\usepackage{latexsym}
\usepackage{amsmath}
\usepackage{amssymb}
\usepackage{graphicx}
\usepackage{wrapfig}
\pagestyle{plain}
\usepackage{fancybox}
\usepackage{bm}

\begin{document}

22. $d=\displaystyle \frac{\sqrt{5}}{2}$;

23. Zauwazmy, $\dot{\mathrm{z}}\mathrm{e}$ spelnionajest nierównośč $3x^{2}-6x+4\geq 1$ (i nawet $3x^{2}-6x+4>1,$

gdy $x\neq 1)$. Stąd $\mathrm{j}\mathrm{u}\dot{\mathrm{z}}$ wynika, $\dot{\mathrm{z}}\mathrm{e}$ pochodna funkcji $f(x)$ jest dodatnia,

$f'(x)=3x^{2}-6x+4-\sin x>0$ (takze dla $x=1$),

i dlatego fnkcja $f(x)$ jest rosnąca;

24. czworokąt jest rombem o polu $P=S/2$;

25. $P(A|B)=\displaystyle \frac{P(A\cap B)}{P(B)}=\frac{P(A)+P(B)-P(A\cup B)}{P(B)}\geq\frac{0,7+0,9-1}{0,9}=\frac{2}{3}$;

26. 1/2 $\mathrm{i}-\infty$;

27. $x=\displaystyle \frac{\pi}{2}+2k\pi \mathrm{i}k$ jest liczbą cafkowitą;

28. $M=1\displaystyle \mathrm{i}m=\frac{\sqrt{2}}{2}$;

29. $(a_{n})$ jest ograniczony, gdy istnieje liczba rzeczywista $M$ taka, $\dot{\mathrm{z}}\mathrm{e} |a_{n}| \leq M$ dla

$\mathrm{k}\mathrm{a}\dot{\mathrm{z}}$ dej liczby naturalnej $n$. Dla rozwazanego ciągu i dla $\mathrm{k}\mathrm{a}\dot{\mathrm{z}}$ dej liczby naturalnej

$n$ jest

$|\displaystyle \alpha_{n}|=a_{n}=\frac{1}{n+1}+\frac{1}{n+2}+\cdots+\frac{1}{2n}\leq\frac{1}{n+1}+\frac{1}{n+1}+\cdots+\frac{1}{n+1}=\frac{n}{n+1}\leq 1,$

więc ciąg ten jest ograniczony.

30. Jeśli funkcja $f(x)$ jest rózniczkowalna w punkcie $x_{0}$ i jeśli ma ona maksimum

lokalne w tym punkcie, to $f'(x_{0})=0.$
\end{document}
