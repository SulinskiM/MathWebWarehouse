\documentclass[a4paper,12pt]{article}
\usepackage{latexsym}
\usepackage{amsmath}
\usepackage{amssymb}
\usepackage{graphicx}
\usepackage{wrapfig}
\pagestyle{plain}
\usepackage{fancybox}
\usepackage{bm}

\begin{document}

POLITECHNIKA $\mathrm{G}\mathrm{D}\mathrm{A}\acute{\mathrm{N}}$ SKA

Gdańsk, 28.06.1995 r.

EGZAMIN WSTĘPNY Z MATEMATYKI

Zestaw sklada się z 30 zadań. Zadania $1-10$ oceniane będą w skali $0-2$ punkty,

zadania $11-30$ w skali $0-4$ punkty. Czas trwania egzaminu -- 240 minut.

{\it Powodzenia}.$l$

l. Rozwiązać nierówność $|4+x-|3x-2||\leq 0.$

2. Rozwiązać równanie $2^{2x+1}+3\cdot 4^{x}=10.$

3. Rozwiązać nierówność $\displaystyle \frac{1}{x-1}\geq\frac{2}{x-2}.$

4. Obliczyč $1000^{\frac{1}{3}-\log\sqrt[3]{3}}.$

5. Na osi $0y$ znalez$\acute{}$ć punkt $M$ równo oddalony od punktów $A(2,-1,5)\mathrm{i}B(-3,2,4).$

6. Wielomian $w(x)=x^{4}+x^{2}+1$ rozlozyć na czynniki.

7. Wyznaczyć n

z równania

$1+5+9+\ldots+(4n-3)=120.$

8. Obliczyć granicę $\displaystyle \lim_{n\rightarrow\infty}(\log(10n^{2}+1)-2\log n).$

9. Obliczyč $y'(\displaystyle \frac{\pi}{4})$, jeśli $y(x)=\sqrt{1+\cos 2x}.$

10. Obliczyć stosunek objętości kuli opisanej na walcu do objętości kuli wpisanej

w ten walec.

ll. Znalez$\acute{}$ć składnik wymierny rozwinięcia dwumianu $(\sqrt[3]{2}+\sqrt[4]{3})^{10}$

12. Dlajakich parametrów $\alpha$ równanie $x^{2}+4x\sin\alpha+1=0$ posiada co najmniej

jeden pierwiastek rzeczywisty?

13. Dlajakich wartości $a\mathrm{i}b$ liczba $-1$ jest pierwiastkiem podwójnym wielomianu

$w(x)=x^{3}+ax^{2}+bx-3$?

14. Rozwiązać nierówność $\log_{2}x+\log_{x}2\geq 2.$

15. Wiadomo, $\dot{\mathrm{z}}\mathrm{e}$ zdarzenia losowe $A\mathrm{i}B$ są niezalezne oraz $P(A)=p_{1}$ i $P(B)= p_{2}.$

Obliczyć prawdopodobieństwa $P(A|B)$ oraz $P(A-B).$
\end{document}
