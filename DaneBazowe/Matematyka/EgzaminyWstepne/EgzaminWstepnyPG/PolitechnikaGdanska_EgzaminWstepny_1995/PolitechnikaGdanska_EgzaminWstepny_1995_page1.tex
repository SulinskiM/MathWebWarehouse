\documentclass[a4paper,12pt]{article}
\usepackage{latexsym}
\usepackage{amsmath}
\usepackage{amssymb}
\usepackage{graphicx}
\usepackage{wrapfig}
\pagestyle{plain}
\usepackage{fancybox}
\usepackage{bm}

\begin{document}

16. Dane są funkcje

$= g(f(x)).$

$f(x)=\sqrt{x}$

i

$g(x)=1-x$. Rozwiązać równanie $f(g(x))$

17. Dany jest ciąg geometryczny $(a_{n})$. Pokazać, $\dot{\mathrm{z}}\mathrm{e}$ ciąg $(b_{n})$, gdzie $b_{n}=\alpha_{n+1}-\alpha_{n},$

$\mathrm{t}\mathrm{e}\dot{\mathrm{z}}$ jest ciągiem geometrycznym.

18. Dwa punkty wyruszają jednocześnie z wierzcholka kąta o mierze $120^{\mathrm{o}}$ po je-

go ramionach z prędkościami odpowiednio 5 $\mathrm{m}/\mathrm{s}$ i 3 $\mathrm{m}/\mathrm{s}$. Po jakim czasie

odleglość między nimi będzie wynosila 49 $\mathrm{m}$?

19. Napisać równanie okręgu stycznego do obu osi ukladu wspólrzędnych i prze-

chodzącego przez punkt $P(2,1).$

20. Na podstawie definicji obliczyć pochodną funkcji $f(x)=\cos 3x.$

21. Narysować wykres funkcji $f(x)=2^{\log_{\frac{1}{2}}x}$

22. Wyznaczyć największą i najmniejszą wartość funkcji $f(x)=x+$ ctg $x$ w prze-

dziale $\displaystyle \langle\frac{1}{4}\pi;\frac{3}{4}\pi\rangle.$

23. $\mathrm{Z}$ prawdopodobieństwem 1/2 w urnie znajduje się a1bo ku1a biafa, a1bo czarna.

Do urny dokladamy kulę bialą i następnie losujemy jedną kulę. Jakie jest

prawdopodobieństwo tego, $\dot{\mathrm{z}}\mathrm{e}$ wylosujemy kulę bialą?

24. Udowodnić, $\dot{\mathrm{z}}\mathrm{e}$ wszystkie trójkaty prostokątne, których boki tworzą ciąg aryt-

metyczny, są podobne.

25. Wyznaczyć asymptoty funkcji $y=\displaystyle \frac{\sqrt{x^{2}+x+1}}{x}.$

26. Obliczyć tg $\alpha$, jeśli $\displaystyle \sin\alpha-\cos\alpha=\frac{\sqrt{2}}{2}$

$\mathrm{i} \displaystyle \alpha\in(\frac{\pi}{4};\frac{\pi}{2}).$

27. Narysować na płaszczyz$\acute{}$nie zbiór punktów, których wspólrzędne spełniają nie-

równość $y^{2}+xy-2x^{2}<0.$

28. Obliczyć dlugości przekątnych równolegfoboku zbudowanego na wektorach $\vec{a}$

$\mathrm{i} \vec{b}, \mathrm{j}\mathrm{e}\dot{\mathrm{z}}$ eli ã $= 2\vec{m}$ -{\it ñ}, $\vec{b}= 3\text{{\it ñ}}-\vec{m}$, gdzie wektory $\vec{m} \mathrm{i}$ ñ są ortogonalne

$\mathrm{i}|\vec{m}|= |${\it ñ}$| =1.$

29. Wykazać, $\dot{\mathrm{z}}\mathrm{e}$ funkcja $y=\sqrt{x^{3}-1}$ jest róznowartościowa w swojej dziedzinie.

Następnie wyznaczyć funkcją do niej odwrotną.

30. Wykazać, $\dot{\mathrm{z}}\mathrm{e}$ jeśli ciąg $(a_{n})$ jest ograniczony i $\displaystyle \lim_{n\rightarrow\infty}b_{n}=0$, to $\displaystyle \lim_{n\rightarrow\infty}a_{n}\cdot b_{n}=0.$
\end{document}
