\documentclass[a4paper,12pt]{article}
\usepackage{latexsym}
\usepackage{amsmath}
\usepackage{amssymb}
\usepackage{graphicx}
\usepackage{wrapfig}
\pagestyle{plain}
\usepackage{fancybox}
\usepackage{bm}

\begin{document}

POLITECHNIKA $\mathrm{G}\mathrm{D}\mathrm{A}\acute{\mathrm{N}}$ SKA

Gdańsk, 1.07.1996 r.

EGZAMIN WSTĘPNY Z MATEMATYKI

Zestaw sklada się z 30 zadań. Zadania $1-10$ oceniane będą w skali $0-2$ punkty, zadania

$11-30$ w skali $0-4$ punkty. Czas trwania egzaminu -- 240 minut.

{\it Powodzenia}.'

l. Funkcję kwadratową $y=(x+3)(1-x)$ przedstawić w postaci kanonicznej. Naszki-

cować jej wykres.

2. Rozwiązać równanie $5^{x}\displaystyle \cdot 5^{x^{2}}\cdot 5^{x^{3}}=\frac{1}{5}.$

3. Rozwiązać równanie $\log_{\frac{1}{3}}(|x|-1)>2.$

4. Dla jakich parametrów $a\in R$ równanie $\displaystyle \cos^{2}x=\frac{2a}{a-2}$ ma rozwiązanie?

5. Naszkicować wykres funkcji $y=x\log_{x^{2}}|x|.$

6. Wyznaczyć te wartości $x$, dla których punkty $A(5,5), B(1,3)\mathrm{i}C(x,0)$ są wspófli-

niowe.

7. Wskazač większq z liczb 0, 4(9) i $\displaystyle \sin(\frac{101}{6}\pi).$

8. Napisać równanie stycznej do wykresu funkcji $f(x)=\sqrt{2x-3}$ w punkcie o odciętej

$x_{0}=6.$

9. Dana jest funkcja $f(x)=\cos^{2}x$. Narysować wykres funkcji $y=f'(x)$ w przedziale

$\langle 0;\pi\rangle.$

10. Zbadać monotoniczność funkcji $f(x)=x+\displaystyle \frac{1}{x}.$

ll. Dany jest ciąg $(a_{n})$, gdzie $a_{n}=\displaystyle \frac{(n!)^{2}}{(2n)!}$. Obliczyć $\displaystyle \lim_{n\rightarrow\infty}\frac{a_{n+1}}{\alpha_{n}}.$

12. Rozwiązać nierówność $g(f(x))\geq 1$, jeśli $f(x)=3^{x}\mathrm{i}g(x)=\sin x.$

13. Wyznaczyć wszystkie wielokąty wypukfe, w których liczba przekątnych jest 3 razy

większa od liczby wierzchofków.

14. Rozwiązać równanie $|\cos x|=\cos x+2\sin x$ w przedziale $\langle 0;2\pi\rangle.$

15. Rozwiązać nierówność $\displaystyle \frac{x^{3}-x+6}{x^{2}}\geq 0.$




16. Rozwiązać równanie $1-\displaystyle \frac{1}{x}+\frac{1}{x^{2}}-\frac{1}{x^{3}}+\ldots=x-1.$

17. Dla jakich $x\in R$ ciąg 2 $\log_{3}x, \log_{3}(x-1)$, -log34 jest ciągiem arytmetycznym?

18. Niech $g$ bedzie granicą ciągu $(a_{n})$, gdzie $a_{n} = \displaystyle \frac{3n+1}{n+1}$. Od jakiego $n$ począwszy

wyrazy ciągu $(a_{n})$ spelniają nierówność $|a_{n}-g|<0$, 01?

19. Dla jakich $a\in R$ funkcja $f(x)=\{_{\frac{\cos x\sin|2x|}{x}}+a$

dla

dla

$x\geq 0$

$x<0$

jest ciagla?

20. Wielomian $x^{2}+px+q$ ma pierwiastki $x_{1}$ i $x_{2}$. Wskazać trójmian $x^{2}+bx+c$, którego

pierwiastkami są liczby $x_{1}+1\mathrm{i}x_{2}+1.$

21. Ze zbioru \{l, 2, $\ldots$, 1000\} 1osujemy jedną liczbę. Obliczyć prawdopodobieństwo te-

go, $\dot{\mathrm{z}}\mathrm{e}$ nie będzie to liczba podzielna ani przez 6, ani przez 8.

22. Obliczyč pole trapezu o podstawach dlugości a $\mathrm{i}b, \mathrm{j}\mathrm{e}\dot{\mathrm{z}}$ eli wiadomo, $\dot{\mathrm{z}}\mathrm{e}$ na tym trapezie

$\mathrm{m}\mathrm{o}\dot{\mathrm{z}}$ na opisać okrąg i $\mathrm{m}\mathrm{o}\dot{\mathrm{z}}$ na w niego wpisać okrag.

23. Znalez$\acute{}$ć rzut prostokątny punktu $A(1,-1)$ na prostą 

24. Dane są zbiory $A=\{(x,y):x,y\in R\mathrm{i}x^{2}+y^{2}-2y\leq 1\} \mathrm{i} B=\{(x,y)$ : $x, y$

$\in R \mathrm{i} |x|+y \leq 1\}$. Narysować na plaszczy $\acute{\mathrm{z}}\mathrm{n}\mathrm{i}\mathrm{e}$ ukladu wspólrzędnych zbiór

$A\cap B$ i obliczyć jego pole.

25. Wyznaczyć asymptoty funkcji $f(x)=\displaystyle \frac{\sqrt{x^{2}-1}-x}{x}.$

26. Obliczyć $|\vec{a}-\vec{b}|$, jeśli $|\vec{a}+\vec{b}|=5,\ |${\it ã}$| =3\mathrm{i}|\vec{b}|=2\sqrt{2}.$

27. Wyznaczyć zbiór wartości funkcji $y=x\sqrt{4-x^{2}}.$

28. Rzucamy symetryczną monetą. Obliczyć prawdopodobieństwo zdarzenia, $\dot{\mathrm{z}}\mathrm{e}$ w szó-

stym rzucie otrzymamy trzeciego orla.

29. Uzasadnić, $\dot{\mathrm{z}}\mathrm{e}$ równanie $x^{3}+x+7=0$ w zbiorze liczb rzeczywistych posiada dokladnie

jedno rozwiązanie. Wraz z uzasadnieniem wskazać przedzial o dlugości co najwyzej

1/2, do którego nalezy to rozwiązanie.

30. Ostrosfup przecięto pfaszczyznq równolegla do podstawy i dzielqcą wysokość w sto-

sunku 2: 3. Ob1iczyć stosunek objętości powstafych bry1.



\end{document}