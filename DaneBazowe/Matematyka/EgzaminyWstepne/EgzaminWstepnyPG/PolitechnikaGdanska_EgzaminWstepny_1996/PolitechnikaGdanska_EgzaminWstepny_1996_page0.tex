\documentclass[a4paper,12pt]{article}
\usepackage{latexsym}
\usepackage{amsmath}
\usepackage{amssymb}
\usepackage{graphicx}
\usepackage{wrapfig}
\pagestyle{plain}
\usepackage{fancybox}
\usepackage{bm}

\begin{document}

POLITECHNIKA $\mathrm{G}\mathrm{D}\mathrm{A}\acute{\mathrm{N}}$ SKA

Gdańsk, 1.07.1996 r.

EGZAMIN WSTĘPNY Z MATEMATYKI

Zestaw sklada się z 30 zadań. Zadania $1-10$ oceniane będą w skali $0-2$ punkty, zadania

$11-30$ w skali $0-4$ punkty. Czas trwania egzaminu -- 240 minut.

{\it Powodzenia}.'

l. Funkcję kwadratową $y=(x+3)(1-x)$ przedstawić w postaci kanonicznej. Naszki-

cować jej wykres.

2. Rozwiązać równanie $5^{x}\displaystyle \cdot 5^{x^{2}}\cdot 5^{x^{3}}=\frac{1}{5}.$

3. Rozwiązać równanie $\log_{\frac{1}{3}}(|x|-1)>2.$

4. Dla jakich parametrów $a\in R$ równanie $\displaystyle \cos^{2}x=\frac{2a}{a-2}$ ma rozwiązanie?

5. Naszkicować wykres funkcji $y=x\log_{x^{2}}|x|.$

6. Wyznaczyć te wartości $x$, dla których punkty $A(5,5), B(1,3)\mathrm{i}C(x,0)$ są wspófli-

niowe.

7. Wskazač większq z liczb 0, 4(9) i $\displaystyle \sin(\frac{101}{6}\pi).$

8. Napisać równanie stycznej do wykresu funkcji $f(x)=\sqrt{2x-3}$ w punkcie o odciętej

$x_{0}=6.$

9. Dana jest funkcja $f(x)=\cos^{2}x$. Narysować wykres funkcji $y=f'(x)$ w przedziale

$\langle 0;\pi\rangle.$

10. Zbadać monotoniczność funkcji $f(x)=x+\displaystyle \frac{1}{x}.$

ll. Dany jest ciąg $(a_{n})$, gdzie $a_{n}=\displaystyle \frac{(n!)^{2}}{(2n)!}$. Obliczyć $\displaystyle \lim_{n\rightarrow\infty}\frac{a_{n+1}}{\alpha_{n}}.$

12. Rozwiązać nierówność $g(f(x))\geq 1$, jeśli $f(x)=3^{x}\mathrm{i}g(x)=\sin x.$

13. Wyznaczyć wszystkie wielokąty wypukfe, w których liczba przekątnych jest 3 razy

większa od liczby wierzchofków.

14. Rozwiązać równanie $|\cos x|=\cos x+2\sin x$ w przedziale $\langle 0;2\pi\rangle.$

15. Rozwiązać nierówność $\displaystyle \frac{x^{3}-x+6}{x^{2}}\geq 0.$
\end{document}
