\documentclass[a4paper,12pt]{article}
\usepackage{latexsym}
\usepackage{amsmath}
\usepackage{amssymb}
\usepackage{graphicx}
\usepackage{wrapfig}
\pagestyle{plain}
\usepackage{fancybox}
\usepackage{bm}

\begin{document}

POLITECHNIKA $\mathrm{G}\mathrm{D}\mathrm{A}\acute{\mathrm{N}}$ SKA

Gdańsk, 30.06.1997 r.

EGZAMIN WSTĘPNY Z MATEMATYKI

Egzamin sklada się z 30 zadań. Zadania $1-10$ oceniane bedą w skali $0-2$ punkty, zadania

$11-30$ w skali $0-4$ punkty. Czas trwania egzaminu -- 240 minut.

{\it Powodzenia}.$\displaystyle \int$

l. Syn jest o 301at młodszy od ojca. 51at temu ojciec był 7 razy starszy od syna.

$\mathrm{W}$ którym roku urodził $\mathrm{s}\mathrm{i}\mathrm{e}$ syn?

2. Znalez/č pola kwadratów, których dwoma wierzchołkami są punkty $(-1,1)\mathrm{i}(2,1).$

3. Podač przykład ciągu niemonotonicznego, którego granicą jest liczba 2.

4. Dla jakich parametrów $a$ dziedziną funkcji $y = \sqrt{ax^{2}+x+a}$ jest zbiór wszystkich

liczb rzeczywistych?

5. Rozwiązač równanie $\log_{2}x\cdot\log_{x}4=2.$

6. Obliczyč sumę wspólczynników wielomianu $w(x)=(x^{2}+2x-1)^{10}-20x-3.$

7. Obliczyč granicę $\displaystyle \lim_{n\rightarrow\infty}\frac{(n+2)!+n!}{(n+2)!-(n+1)!}.$

8. Napisač równanie prostej zawierającej tę cieciwę okręgu $x^{2}-4x+y^{2}+2y+1 =0,$

którą punkt $A(1,-\displaystyle \frac{1}{2})$ dzieli na dwie równe części.

9. Obliczyč $f'(0)$, jeśli $f(x)=x(x-1)(x-2)(x-3)(x-4)(x-5).$

10. Obliczyč $\displaystyle \sin\frac{13}{12}\pi.$

ll. Rozwi$\Phi$zač ukfad równań 

Podač ilustrację graficzną tego ukfadu.

12. Znalez/č resztę z dzielenia wielomianu $x^{1997}-x^{1996}+2$ przez $x^{3}-x.$

13. Dla jakiego $m$ równanie $|x^{2}-2|=\log_{\frac{1}{2}}m$ ma dokładnie 4 pierwiastki?

14. Rozwiązač równanie $|x-3|^{x^{2}-4x+3}=1.$

15. Rozwiqzač nierównośč $x+1\leq\sqrt{3+x}.$

16. Rozwiązač równanie tg $x=\displaystyle \mathrm{t}\mathrm{g}\frac{1}{x}.$

17. Niech $S_{n}$ oznacza sume $n$ początkowych wyrazów ciągu $a_{n}=\displaystyle \frac{2^{n}+3^{n}}{6^{n}}.$ Obliczyč $\displaystyle \lim_{n\rightarrow\infty}S_{n}.$
\end{document}
