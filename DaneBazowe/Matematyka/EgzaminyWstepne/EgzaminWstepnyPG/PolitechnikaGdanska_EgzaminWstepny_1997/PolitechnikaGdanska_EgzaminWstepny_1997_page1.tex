\documentclass[a4paper,12pt]{article}
\usepackage{latexsym}
\usepackage{amsmath}
\usepackage{amssymb}
\usepackage{graphicx}
\usepackage{wrapfig}
\pagestyle{plain}
\usepackage{fancybox}
\usepackage{bm}

\begin{document}

18. Ile razy nalezy rzucič symetryczną monetą, aby z prawdopodobieństwem większym od

$\displaystyle \frac{1}{2}$ otrzymač przynajmniej dwa orły?

19. Zdarzenia losowe $A\mathrm{i}B$ są jednakowo prawdopodobne, zawsze zachodzi przynajmniej

jedno z nich i $P(A|B)=\displaystyle \frac{1}{2}$. Obliczyč prawdopodobieństwa zdarzeń A $\mathrm{i}B$. Czy zdarzenia

$ A\mathrm{i}B\mathrm{s}\Phi$ niezalezne?

20. Uzasadnič, $\dot{\mathrm{z}}\mathrm{e}$ nie istnieje trójkąt o wysokościach długości 1, 2 $\mathrm{i}3.$

21. Znalez/č rzut równoległy punktu $A(5,2,9)$ na płaszczyznę $Oxy$ w kierunku wektora

$\vec{v}=[1$, 2, 3$].$

22. Rys. l przedstawia szkic wykresu funkcji $f(x)=$

$ax-b$

$x-c$

dla pewnych liczb $a, b\mathrm{i}c$. Wyznaczyc wspófrzedne punk-

tow $P\mathrm{i}Q$. Wskazac liczby $a, b\mathrm{i}c$, dla których wykres

funkcji $y = f(x) \mathrm{m}\mathrm{o}\dot{\mathrm{z}}$ na otrzymac z wykresu funkcji

$y=\displaystyle \frac{1}{x}$ w wyniku translacji o wektor $\vec{u}=[1$, 3$].$
\begin{center}
\includegraphics[width=37.236mm,height=39.012mm]{./PolitechnikaGdanska_EgzaminWstepny_1997_page1_images/image001.eps}
\end{center}
$y$

$---Q--||$

$|P x$

23. Wyznaczyč liczbę $a$ tak, aby funkcja $f(x)=\{_{\frac{\sin(x-1)2+ax}{|x-1|}}^{x}$

cie $x_{0}=1.$

dla

dla

Rys. l

$x<x\geq 11$ była ciągfa w punk-

24. Napisač równanie tej stycznej do wykresu funkcji $y=\displaystyle \frac{4}{x^{2}}$, która jest nachylona do osi

$Ox$ pod kątem $45^{\mathrm{o}}$

25. Wyznaczyč przedzialy, w których funkcja $f(x)=2\cos^{2}x-x$ jest rosnąca.

26. Wyznaczyč asymptoty krzywej $f(x)=\sqrt{1+x^{2}}-2x.$

27. Przedsiębiorstwo handlowe sprzedaje opony samochodowe. Cafkowity zysk przedsię-

biorstwa liczony w tysiqcach złotych ze sprzedazy $x$ setek tysięcy opon dany jest wzo-

rem $z(x)=-x^{3}+9x^{2}+120x-400$ dla $x\geq 5$. Przyjakiej ilości sprzedanych opon zysk

przedsiębiorstwa będzie największy?

28. Punkt $E$ jest srodkiem boku kwadratu ABCD przedsta-

wionego na rys. 2, a trojk $\mathrm{t}EFG$ jest rownoboczny. Oblicz

pole trojkąta $EFG, \mathrm{j}\mathrm{e}\dot{\mathrm{z}}$ eli dfugosc $\mathrm{k}\mathrm{a}\dot{\mathrm{z}}$ dego boku kwadratu

ABCD jest równa 2.
\begin{center}
\includegraphics[width=32.508mm,height=30.528mm]{./PolitechnikaGdanska_EgzaminWstepny_1997_page1_images/image002.eps}
\end{center}
{\it D} $C$

{\it H}

{\it G F}

{\it A B}

{\it E}

Rys. 2

29. Dany jest romb ABCD o bokach dlugości$\rightarrow  1\mathrm{i}\rightarrow$ kącie o mierze $60^{\mathrm{o}}$ przy wierzcholku $A.$

Obliczyč iloczyn skalarny wektorów AM $\mathrm{i}\vec{AN}$, jeśli $M\mathrm{i}N$ są odpowiednio środkami

boków $BC\mathrm{i}$ {\it CD}.

30. Obliczyč pole powierzchni i objętośč wielościanu, którego wierzchołkami są wszystkie

środki krawędzi czworościanu foremnego o boku dfugości $a.$
\end{document}
