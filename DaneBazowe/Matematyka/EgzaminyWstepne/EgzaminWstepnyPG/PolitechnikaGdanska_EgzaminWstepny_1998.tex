\documentclass[a4paper,12pt]{article}
\usepackage{latexsym}
\usepackage{amsmath}
\usepackage{amssymb}
\usepackage{graphicx}
\usepackage{wrapfig}
\pagestyle{plain}
\usepackage{fancybox}
\usepackage{bm}

\begin{document}

POLITECHNIKA $\mathrm{G}\mathrm{D}\mathrm{A}\acute{\mathrm{N}}$ SKA

Gdańsk, 30.06.1998 r.

EGZAMIN WSTĘPNY Z MATEMATYKI

Egzamin składa się z 30 zadań. Zadania $1-10$ oceniane będą w skali $0-2$ punkty, zadania

$11-30$ w skali $0-4$ punkty. Czas trwania egzaminu -- 240 minut.

{\it Powodzenia}.$\displaystyle \int$

l. Rozwiqzač nierównośč $2^{|x+1|}\leq 0,(9).$

2. Obliczyč resztę z dzielenia wielomianu $w(x)=x^{101!}-x+1$ przez dwumian $x+1.$

3. Wyznaczyč dziedzinę funkcji $f(x)=\log_{2}\log_{\frac{1}{2}}x^{2}$

4. Rozwi$\Phi$zač nierównośč $\displaystyle \cos(\pi-x)\leq\sin(\frac{\pi}{2}+x).$

5. Obliczyč największą wartośč funkcji $f(x)=\displaystyle \frac{1}{x^{2}+6x+16}.$

6. Dany jest ciąg $(a_{n})$, gdzie $ a_{n}=\displaystyle \frac{3-n}{n}\cos n\pi$ dla $n\in N$. Zbadač monotonicznośč ciągu

$(b_{n})$, w którym $b_{n}=a_{2n-1}$ dla $\mathrm{k}\mathrm{a}\dot{\mathrm{z}}$ dego $n\in N.$

7. Trzecim wyrazem $\mathrm{c}\mathrm{i}_{\Phi \mathrm{g}}\mathrm{u}$ arytmetycznego jest liczba l. Obliczyč sumę pierwszych pięciu

wyrazów tego ciągu.

8. Wśród rozpoczynających studia $\mathrm{w}\mathrm{y}\dot{\mathrm{z}}$ sze jest tyle samo $\mathrm{m}\mathrm{e}\dot{\mathrm{z}}$ czyzn co kobiet. Co czwarta

kobieta i co drugi $\mathrm{m}\text{ę}\dot{\mathrm{z}}$ czyzna z tych, którzy rozpoczęli studia, nie kończy ich. Obliczyč

jaki procent liczby wszystkich absolwentów $\mathrm{w}\mathrm{y}\dot{\mathrm{z}}$ szych uczelni stanowi liczba absolwen-

tek tychze uczelni.

9. Rys. l przedstawia szkic wykresu funkcji $y=f(x)$ dla $x\in\langle 0;4\rangle.$

Okreslič dziedzinę i naszkicowac wykres funkcji $y=f(-x+3).$
\begin{center}
\includegraphics[width=36.012mm,height=28.452mm]{./PolitechnikaGdanska_EgzaminWstepny_1998_page0_images/image001.eps}
\end{center}
$y$

2

3 4 x

Rys. l

10. Rozwi zac rownanie $x+\displaystyle \frac{x^{2}}{2}+\frac{x^{3}}{4}+\frac{x^{4}}{8}+\ldots=x+13$

ll. Dla jakich wartości parametru $a$ układ równań 

rozwiązanie?

ma co najmniej jedno

12. Przedsiębiorstwo proponuje dziesięcioletni kontrakt swojemu pracownikowi. $\mathrm{W}$ pier-

wszym roku pracy pracownik zarobi 15000 PLN, a w $\mathrm{k}\mathrm{a}\dot{\mathrm{z}}$ dym następnym roku jego

zarobki będą wzrastafy o 8\%. I1e zarobi pracownik w dziesiątym roku pracy? I1e wyniosą

łqczne zarobki pracownika za dziesięč lat pracy w przedsiębiorstwie?

($\mathrm{W}$ obliczeniach $\mathrm{m}\mathrm{o}\dot{\mathrm{z}}$ na przyjąč, $\dot{\mathrm{z}}\mathrm{e}(1,08)^{9}=2.$)

13. Dla jakich wartości parametru $m$ pierwiastki równania $mx^{2}-2mx+1=0$ spełniają

nierównośč $x_{1}^{2}+x_{2}^{2}<3$?




14. Obliczyč pole obszaru opisanego ukfadem nierówności 

15. Punkty $A(2,1) \mathrm{i} B(8,3)$ są wierzchołkami trójkąta $ABC$. Wyznaczyč współrzędne

wierzchofka $C$, jeśli środkowe trójkąta $ABC$ przecinaja się w punkcie $M(4,5).$

16. Obliczyč pole trójkqta wyznaczonego przez punkt $A(3,2)$ i tę średnicę

$x^{2}-2x+y^{2}+4y=20$, która jest równolegfa do prostej $4y-3x=0.$

okręgu

17. Dobrač parametr $a$ tak, aby funkcja $f(x)=\{^{\frac{\sqrt{1+x^{2}}-1}{2^{a}x^{2}}}$

dla

dla

$x\neq 0$

$x=0$

była ciągfa.

18. Obliczyč $\displaystyle \lim_{x\rightarrow 0^{+}}f'(x)$, jeśli $f(x)=\sin(\pi\cos\sqrt{x}).$

19. Rozwiązač równanie $\cos 2x+\cos x+1=0$ dla $x\in\langle 0;2\pi\rangle.$

20. Rozwiązač nierównośč $x\sqrt{3-2x}+1\leq 0.$

21. Wyznaczyč liczby $a\mathrm{i}b$ takie, $\dot{\mathrm{z}}\mathrm{e} \displaystyle \frac{1}{(x-1)x}=\frac{a}{x-1}+\frac{b}{x}$

obliczyč $\displaystyle \lim_{n\rightarrow\infty}(\frac{1}{1\cdot 2}+\frac{1}{2\cdot 3}+\frac{1}{3\cdot 4}+\ldots+\frac{1}{(n-1)n}).$

dla $x\in R-\{0$, 1$\}$. Następnie

22. Rys. 2 przedstawia kratę wymiaru $4\times 4$. Chcemy przejśč po od-

cinkach tej kraty od punktu $A$ do punktu $B\mathrm{m}\mathrm{o}\dot{\mathrm{z}}$ liwie najkrótszą

drogą. Ile jest takich dróg?
\begin{center}
\includegraphics[width=34.752mm,height=33.528mm]{./PolitechnikaGdanska_EgzaminWstepny_1998_page1_images/image001.eps}
\end{center}
{\it B}

{\it A}

Rys.2

23. Zdarzenia losowe $ A\mathrm{i}B\mathrm{s}\Phi$ niezalezne i $P(A\displaystyle \cap B)=\frac{1}{3}$ oraz $P(A\displaystyle \cup B)=\frac{9}{10}$. Obliczyč

$P(A), P(B)\mathrm{i}P(A-B)$, gdy $P(A)>P(B).$

24. Rzucono raz pięcioma kostkami do gry. Jakie jest prawdopodobieństwo tego, $\dot{\mathrm{z}}\mathrm{e}$ na

wszystkich kostkach wypadła taka sama liczba oczek lub na $\mathrm{k}\mathrm{a}\dot{\mathrm{z}}$ dej z nich wypadla

inna liczba oczek?

25. Napisač równanie stycznej do wykresu funkcji $f(x) = x+\sqrt{2-x}$ w jego punkcie

przecięcia z osią $Ox.$

26. Rozwiazač równanie $\displaystyle \log_{3}(3x)+\log_{x}(3x)=\log_{9}(\frac{1}{3}).$

27. Rys. 3 przedstawia szkic wykresu wielomianu stopnia trzeciego.

Wyznaczyč ten wielomian i wyznaczyč współrzędne punktu $P,$

w którym ma on minimum lokalne.
\begin{center}
\includegraphics[width=28.548mm,height=33.120mm]{./PolitechnikaGdanska_EgzaminWstepny_1998_page1_images/image002.eps}
\end{center}
{\it y}

$-3$ 2  {\it x}

$-3$

{\it P}

Rys. 3

28. Dane są punkty $A(-1,3,3), B(0,1,5)\mathrm{i}C(3,5,-1)$. Wyznaczyč taki punkt $D, \dot{\mathrm{z}}$ ewektor

$\overline{A}\mathrm{Z}$ dzieli kąt między wektorami $\overline{A}E\mathrm{i}\overline{A}z$ na polowy i $|\overline{A}\tau|=1.$

29. $\mathrm{W}$ równoramiennym trójkącie $\mathrm{p}\mathrm{r}\mathrm{o}\mathrm{s}\mathrm{t}\mathrm{o}\mathrm{k}_{\Phi^{\mathrm{t}}}\mathrm{n}\mathrm{y}\mathrm{m}$ poprowadzono z wierzchofka $\mathrm{k}_{\Phi^{\mathrm{t}\mathrm{a}}}$ prostego

dwie proste dzielące przeciwprostokątną na trzy odcinki jednakowej dfugości. Obliczyč

cosinus kąta miedzy tymi prostymi.

30. Oblilczyč objętośč kuli stycznej do wszystkich krawędzi czworościanu foremnego o boku

dfugości $a.$



\end{document}