\documentclass[a4paper,12pt]{article}
\usepackage{latexsym}
\usepackage{amsmath}
\usepackage{amssymb}
\usepackage{graphicx}
\usepackage{wrapfig}
\pagestyle{plain}
\usepackage{fancybox}
\usepackage{bm}

\begin{document}

POLITECHNIKA $\mathrm{G}\mathrm{D}\mathrm{A}\acute{\mathrm{N}}$ SKA

Gdańsk, 30.06.1998 r.

EGZAMIN WSTĘPNY Z MATEMATYKI

Egzamin składa się z 30 zadań. Zadania $1-10$ oceniane będą w skali $0-2$ punkty, zadania

$11-30$ w skali $0-4$ punkty. Czas trwania egzaminu -- 240 minut.

{\it Powodzenia}.$\displaystyle \int$

l. Rozwiqzač nierównośč $2^{|x+1|}\leq 0,(9).$

2. Obliczyč resztę z dzielenia wielomianu $w(x)=x^{101!}-x+1$ przez dwumian $x+1.$

3. Wyznaczyč dziedzinę funkcji $f(x)=\log_{2}\log_{\frac{1}{2}}x^{2}$

4. Rozwi$\Phi$zač nierównośč $\displaystyle \cos(\pi-x)\leq\sin(\frac{\pi}{2}+x).$

5. Obliczyč największą wartośč funkcji $f(x)=\displaystyle \frac{1}{x^{2}+6x+16}.$

6. Dany jest ciąg $(a_{n})$, gdzie $ a_{n}=\displaystyle \frac{3-n}{n}\cos n\pi$ dla $n\in N$. Zbadač monotonicznośč ciągu

$(b_{n})$, w którym $b_{n}=a_{2n-1}$ dla $\mathrm{k}\mathrm{a}\dot{\mathrm{z}}$ dego $n\in N.$

7. Trzecim wyrazem $\mathrm{c}\mathrm{i}_{\Phi \mathrm{g}}\mathrm{u}$ arytmetycznego jest liczba l. Obliczyč sumę pierwszych pięciu

wyrazów tego ciągu.

8. Wśród rozpoczynających studia $\mathrm{w}\mathrm{y}\dot{\mathrm{z}}$ sze jest tyle samo $\mathrm{m}\mathrm{e}\dot{\mathrm{z}}$ czyzn co kobiet. Co czwarta

kobieta i co drugi $\mathrm{m}\text{ę}\dot{\mathrm{z}}$ czyzna z tych, którzy rozpoczęli studia, nie kończy ich. Obliczyč

jaki procent liczby wszystkich absolwentów $\mathrm{w}\mathrm{y}\dot{\mathrm{z}}$ szych uczelni stanowi liczba absolwen-

tek tychze uczelni.

9. Rys. l przedstawia szkic wykresu funkcji $y=f(x)$ dla $x\in\langle 0;4\rangle.$

Okreslič dziedzinę i naszkicowac wykres funkcji $y=f(-x+3).$
\begin{center}
\includegraphics[width=36.012mm,height=28.452mm]{./PolitechnikaGdanska_EgzaminWstepny_1998_page0_images/image001.eps}
\end{center}
$y$

2

3 4 x

Rys. l

10. Rozwi zac rownanie $x+\displaystyle \frac{x^{2}}{2}+\frac{x^{3}}{4}+\frac{x^{4}}{8}+\ldots=x+13$

ll. Dla jakich wartości parametru $a$ układ równań 

rozwiązanie?

ma co najmniej jedno

12. Przedsiębiorstwo proponuje dziesięcioletni kontrakt swojemu pracownikowi. $\mathrm{W}$ pier-

wszym roku pracy pracownik zarobi 15000 PLN, a w $\mathrm{k}\mathrm{a}\dot{\mathrm{z}}$ dym następnym roku jego

zarobki będą wzrastafy o 8\%. I1e zarobi pracownik w dziesiątym roku pracy? I1e wyniosą

łqczne zarobki pracownika za dziesięč lat pracy w przedsiębiorstwie?

($\mathrm{W}$ obliczeniach $\mathrm{m}\mathrm{o}\dot{\mathrm{z}}$ na przyjąč, $\dot{\mathrm{z}}\mathrm{e}(1,08)^{9}=2.$)

13. Dla jakich wartości parametru $m$ pierwiastki równania $mx^{2}-2mx+1=0$ spełniają

nierównośč $x_{1}^{2}+x_{2}^{2}<3$?
\end{document}
