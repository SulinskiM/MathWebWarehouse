\documentclass[a4paper,12pt]{article}
\usepackage{latexsym}
\usepackage{amsmath}
\usepackage{amssymb}
\usepackage{graphicx}
\usepackage{wrapfig}
\pagestyle{plain}
\usepackage{fancybox}
\usepackage{bm}

\begin{document}

POLITECHNIKA $\mathrm{G}\mathrm{D}\mathrm{A}\acute{\mathrm{N}}$ SKA

Gdańsk, 29.06.1999 r.

EGZAMIN WSTĘPNY Z MATEMATYKI

Egzamin sklada się z 30 zadań. Zadania $1-10$ oceniane będą w skali $0-2$ punkty, zadania

$11-30$ w skali $0-4$ punkty. Czas trwania egzaminu -- 240 minut.

{\it Powodzenia}.$\displaystyle \int$

l. Znalez/č wszystkie $\mathrm{r}\mathrm{o}\mathrm{z}\mathrm{w}\mathrm{i}_{\Phi}$zania równania $81x^{4}-72x^{2}=-16.$

2. Zbiory $A, B\mathrm{i}A\cup B$ mają odpowiednio 1999, 2049 $\mathrm{i}$ 3998 elementów. Ile elementów

mają odpowiednio zbiory $A-B\mathrm{i}A\cap B$?

3. Jeden metr ma l000000 mikronów, a l00000000 angstremów to jeden centymetr. Ile

angstremów ma jeden mikron?

4. Rozwiqzač równanie $\log_{2}(-2)^{5n}=n^{2}+4$, w którym $n$ jest liczbą naturalną.

5. Obliczyč $\left(\begin{array}{l}
n\\
5
\end{array}\right)$, jeśli wiadomo, $\dot{\mathrm{z}}\mathrm{e} \left(\begin{array}{l}
n\\
3
\end{array}\right) =\left(\begin{array}{l}
n\\
4
\end{array}\right)$.

6. Rozwi$\Phi$zač nierównośč $|x-1|\displaystyle \leq\frac{x}{3}+1.$

7. Danajest funkcja $f(x)=(x-1)^{2}$. Na osobnych rysunkach naszkicowač wykresy funkcji:

(a) $y=f(x)$ ; (b) $y=f(-x)$ ; (c) $y=f(x+1)-2.$

8. Rozwi$\Phi$zač nierównośč $x+3\displaystyle \leq\frac{10}{x}.$

9. Dla jakich wartości $x$ istnieje trójkąt o bokach dlugości 1, 2, $\log x$?

10. $\mathrm{W}$ trójkacie naprzeciw boku dlugości $3\sqrt{2}\mathrm{l}\mathrm{e}\dot{\mathrm{z}}\mathrm{y}$ kąt miary $45^{\mathrm{o}}$

okręgu opisanego na tym trójkącie.

Wyznaczyč promień

ll. Mamy dwa naczynia, z których jedno zawiera 101itrów wody, a drugie 101itrów soku.

Polowe wody przelewamy do soku, mieszamy, a następnie pofowę roztworu przelewamy

z powrotem do wody. Obliczyč procentowe stęzenia otrzymanych roztworów.

12. Punkty $A(-1,0), B(3,2) \mathrm{i} C(5,-2)$ są wierzcholkami trójkąta. Pokazač, $\dot{\mathrm{z}}\mathrm{e}$ jest to

trójkąt równoramienny. Napisač równanie osi symetrii tego trójkąta.

13. Doprowadzič do najprostszej postaci wyrazenie $\displaystyle \frac{x+2+\sqrt{x^{2}-4}}{x+2-\sqrt{x^{2}-4}}+\frac{x+2-\sqrt{x^{2}-4}}{x+2+\sqrt{x^{2}-4}}.$

14. $\mathrm{W}$ obszar między trzema wzajemnie stycznymi okręgami o promieniu $R$ wpisano $\mathrm{o}\mathrm{k}\mathrm{r}\Phi \mathrm{g}.$

Znalez/č promień $r$ tego okręgu.

15. Funkcję $f(x)=x^{5}-9x^{3}-27x^{2}+243$ zapisač w postaci iloczynowej i nastepnie rozwiązač

nierównośč $f(x)>0.$



\begin{center}
\includegraphics[width=28.908mm,height=24.132mm]{./PolitechnikaGdanska_EgzaminWstepny_1999_page1_images/image001.eps}
\end{center}
{\it y}

1

$x$

16. Pokazac, $\dot{\mathrm{z}}\mathrm{e}$ funkcja $f(x) = x^{2}$ ma minimum lokalne w punkcie

$x^{2}$ dla $x\neq 0$

$x_{0} = 0$. Uzasadnic, $\dot{\mathrm{z}}\mathrm{e}$ funkcja $g(x) =$ ma maksi-

l dla $x=0$

mum lokalne w punkcie $x_{0}=0$, zob. rys. l. $\mathrm{R}\mathrm{y}\mathrm{s}$. 1

$x^{2}$

17. Napisac rownania tych stycznych do wykresu funkcji $y=$ które są rownolegle

$x-2$'

do prostej $3x+y=0.$

18. Wyznaczyč największą i $\mathrm{n}\mathrm{a}\mathrm{j}\mathrm{m}\mathrm{n}\mathrm{i}\mathrm{e}\mathrm{j}\mathrm{s}\mathrm{z}\Phi$ wartośč funkcji $f(x)=x+\sqrt{1-x^{2}}.$

19. Znalez/č asymptoty wykresu funkcji $y=\displaystyle \frac{4x^{2}+9x}{x-4}.$

20. Rozwiązač równanie $3^{2x}-2\cdot 3^{x}+a=0$, w którym $a=\displaystyle \lim_{n\rightarrow\infty}\frac{\sqrt{n^{2}+3}-4n}{n-1}.$

21. $\mathrm{W}$ prostokątnym układzie współrzędnych zaznaczyč zbiór punktów $(x,y)$, których

współrzędne spelniaja równanie $\log_{2}(x+y)=\log_{2}x+\log_{2}y.$

22. Obliczyč średnią arytmetyczną tych spośród liczb naturalnych l, 2, 3, $\ldots$, 2000, które

nie są podzielne przez 5.

23. Wyznaczyč ciąg geometryczny $a_{1}, a_{2}, \ldots, a_{n}, \ldots, \mathrm{j}\mathrm{e}\dot{\mathrm{z}}$ eli wiadomo, $\dot{\mathrm{z}}\mathrm{e}a_{1}+a_{2}+a_{3}+a_{4}=30$

$\mathrm{i}a_{2}+a_{3}+a_{4}+a_{5}=60$. Znalez/č taką liczbę $n, \dot{\mathrm{z}}\mathrm{e}a_{n}<500000<a_{n+1}.$

24. Rozwiązač równanie 2 $\sin^{2}x+\sin 2x=2.$

25. Rozwi$\Phi$zač nierównośč $\displaystyle \sin^{2}x>\frac{3}{4}$ dla $x\in\langle 0;2\pi\rangle.$

26. Znalez/č równania prostych przechodzqcych przez punkt $A(7,3)$ i przecinajqcych prostą

$x-3y-1=0$ pod $\mathrm{k}_{\Phi}\mathrm{t}\mathrm{e}\mathrm{m}45^{\mathrm{o}}$

27. Obliczyč dfugośč najkrótszej drogi poprowadzonej po powierzchni sześcianu o krawę-

dziach długości l $\mathrm{i}1_{\text{ą}^{\mathrm{C}\mathrm{Z}}\mathrm{a}}\mathrm{c}\mathrm{e}\mathrm{j}$ dwa przeciwległe wierzchołki tego sześcianu. Ile najkrót-

szych dróg fączy dwa wybrane przeciwlegfe wierzcholki tego sześcianu?

28. Obliczyč iloczyn skalarny wektorów ã $= [-1,1+x] \mathrm{i} \vec{b}= [\sqrt{x+3}$, 1$]$. Dla jakich $x$

wektory ã $\mathrm{i}\vec{b}$ są prostopadle? Jaki kąt (ostry, prosty, czy rozwarty) tworzą te wektory

dla $x=-2$?

29. Rzucono pięč razy dwiema kostkami do gry. Obliczyč prawdopodobieństwo tego, $\dot{\mathrm{z}}\mathrm{e}$ co

najmniej dwa razy suma oczek na obu kostkach jest nie mniejsza od 10.

30. Sześcian o krawędzi długości $a$ podzielono płaszczyzną przecho-

dzacą przez przekątną jednej z jego ścian i przez środki dwóch

krawędzi $\mathrm{l}\mathrm{e}\dot{\mathrm{z}}$ qcych na przeciwległej ścianie na dwie bryły, zob.

rys. 2. Ob1iczyč objętości obu otrzymanych bryf.



\end{document}