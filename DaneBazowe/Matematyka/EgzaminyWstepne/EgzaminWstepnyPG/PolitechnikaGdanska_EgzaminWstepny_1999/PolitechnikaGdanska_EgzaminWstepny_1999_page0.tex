\documentclass[a4paper,12pt]{article}
\usepackage{latexsym}
\usepackage{amsmath}
\usepackage{amssymb}
\usepackage{graphicx}
\usepackage{wrapfig}
\pagestyle{plain}
\usepackage{fancybox}
\usepackage{bm}

\begin{document}

POLITECHNIKA $\mathrm{G}\mathrm{D}\mathrm{A}\acute{\mathrm{N}}$ SKA

Gdańsk, 29.06.1999 r.

EGZAMIN WSTĘPNY Z MATEMATYKI

Egzamin sklada się z 30 zadań. Zadania $1-10$ oceniane będą w skali $0-2$ punkty, zadania

$11-30$ w skali $0-4$ punkty. Czas trwania egzaminu -- 240 minut.

{\it Powodzenia}.$\displaystyle \int$

l. Znalez/č wszystkie $\mathrm{r}\mathrm{o}\mathrm{z}\mathrm{w}\mathrm{i}_{\Phi}$zania równania $81x^{4}-72x^{2}=-16.$

2. Zbiory $A, B\mathrm{i}A\cup B$ mają odpowiednio 1999, 2049 $\mathrm{i}$ 3998 elementów. Ile elementów

mają odpowiednio zbiory $A-B\mathrm{i}A\cap B$?

3. Jeden metr ma l000000 mikronów, a l00000000 angstremów to jeden centymetr. Ile

angstremów ma jeden mikron?

4. Rozwiqzač równanie $\log_{2}(-2)^{5n}=n^{2}+4$, w którym $n$ jest liczbą naturalną.

5. Obliczyč $\left(\begin{array}{l}
n\\
5
\end{array}\right)$, jeśli wiadomo, $\dot{\mathrm{z}}\mathrm{e} \left(\begin{array}{l}
n\\
3
\end{array}\right) =\left(\begin{array}{l}
n\\
4
\end{array}\right)$.

6. Rozwi$\Phi$zač nierównośč $|x-1|\displaystyle \leq\frac{x}{3}+1.$

7. Danajest funkcja $f(x)=(x-1)^{2}$. Na osobnych rysunkach naszkicowač wykresy funkcji:

(a) $y=f(x)$ ; (b) $y=f(-x)$ ; (c) $y=f(x+1)-2.$

8. Rozwi$\Phi$zač nierównośč $x+3\displaystyle \leq\frac{10}{x}.$

9. Dla jakich wartości $x$ istnieje trójkąt o bokach dlugości 1, 2, $\log x$?

10. $\mathrm{W}$ trójkacie naprzeciw boku dlugości $3\sqrt{2}\mathrm{l}\mathrm{e}\dot{\mathrm{z}}\mathrm{y}$ kąt miary $45^{\mathrm{o}}$

okręgu opisanego na tym trójkącie.

Wyznaczyč promień

ll. Mamy dwa naczynia, z których jedno zawiera 101itrów wody, a drugie 101itrów soku.

Polowe wody przelewamy do soku, mieszamy, a następnie pofowę roztworu przelewamy

z powrotem do wody. Obliczyč procentowe stęzenia otrzymanych roztworów.

12. Punkty $A(-1,0), B(3,2) \mathrm{i} C(5,-2)$ są wierzcholkami trójkąta. Pokazač, $\dot{\mathrm{z}}\mathrm{e}$ jest to

trójkąt równoramienny. Napisač równanie osi symetrii tego trójkąta.

13. Doprowadzič do najprostszej postaci wyrazenie $\displaystyle \frac{x+2+\sqrt{x^{2}-4}}{x+2-\sqrt{x^{2}-4}}+\frac{x+2-\sqrt{x^{2}-4}}{x+2+\sqrt{x^{2}-4}}.$

14. $\mathrm{W}$ obszar między trzema wzajemnie stycznymi okręgami o promieniu $R$ wpisano $\mathrm{o}\mathrm{k}\mathrm{r}\Phi \mathrm{g}.$

Znalez/č promień $r$ tego okręgu.

15. Funkcję $f(x)=x^{5}-9x^{3}-27x^{2}+243$ zapisač w postaci iloczynowej i nastepnie rozwiązač

nierównośč $f(x)>0.$
\end{document}
