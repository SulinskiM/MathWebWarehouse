\documentclass[a4paper,12pt]{article}
\usepackage{latexsym}
\usepackage{amsmath}
\usepackage{amssymb}
\usepackage{graphicx}
\usepackage{wrapfig}
\pagestyle{plain}
\usepackage{fancybox}
\usepackage{bm}

\begin{document}

POLITECHNIKA $\mathrm{G}\mathrm{D}\mathrm{A}\acute{\mathrm{N}}$ SKA

Gdańsk, lipiec 1990 r.

Tematy I części egzaminu z matematyki

dla kandydatów ubiegających się o przyjęcie na I rok studiów dziennych.

Kandydat wybierał 3 dowo1ne zadania. Rozwiązania wybranych zadań oceniane

byly w skali $0-10$ punktów. Egzamin trwa1120 minut.

l. Zbadać przebieg zmienności funkcji

{\it y}$=$ -{\it xx}22 $+$-{\it xx} $++$11'

sporządzić jej wykres i na tej podstawie ustalič ile pierwiastków posiada rów-

nanie

--{\it xx}22$+$-{\it xx}$++$11$=${\it m}

w zalezności od parametru $m.$

2. Dlajakich wartości parametru $t$, przy dowolnej wartości parametru $k$, równa-

nie

$x^{2}+x\sqrt{k^{2}+4}-k\log_{\frac{1}{2}}(t+1)=0$

posiada dwa rózne pierwiastki?

3. Rozwiązać nierówność

$\displaystyle \lim_{n\rightarrow\infty}(\sqrt{n^{2}+(2+\sin 2x)n+4}-n)<1+\frac{1}{2}\cos 2x.$

4. Dwie kule o promieniach $R\mathrm{i}x(R>x)$ są styczne zewnętrznie. Przy jakim $x$

objętość stozka opisanego na tych kulach będzie najmniejsza?

5. $\mathrm{W}$ urnie $U_{1}$ znajdują się dwie kule czarne i pewna ilość kul bialych. $\mathrm{W}$ urnie

$U_{2}$ znajduje się 5 ku1 bia1ych i 3 czarne. $\mathrm{Z}$ pierwszej urny losujemy dwie kule i

przekładamy je do urny drugiej. Następnie z urny drugiej losujemy jedna kulę.

Podać minimalną ilość bialych kul znajdujących się w urnie $U_{1}$, jeśli wiadomo,

$\dot{\mathrm{z}}\mathrm{e}$ prawdopodobieństwo wylosowania kuli bialej z urny $U_{2}$ jest większe od 0, 6.




POLITECHNIKA $\mathrm{G}\mathrm{D}\mathrm{A}\acute{\mathrm{N}}$ SKA

Gdańsk, lipiec 1990 r.

Tematy II części egzaminu z matematyki

dla kandydatów ubiegajqcych się o przyjęcie na I rok studiów dziennych.

Wszystkie zadania byfy oceniane w skali $0-2$ punkty. Egzamin trwa1120 minut.

l. Naszkicować wykres funkcji $y=x|x+1|.$

2. Obliczyć $\cos^{2}105^{\mathrm{o}}-\sin^{2}105^{\mathrm{o}}$

3. Rozwiązać nierówność

$||x|-1|<2.$

4. Obliczyć granicę $\displaystyle \lim_{n\rightarrow\infty}(1-\frac{1}{2^{1}}+\frac{1}{2^{2}}-\frac{1}{2^{3}}+\ldots+(-1)^{n}\frac{1}{2^{n}}).$

5. Wektor $\vec{a}=[3$, 7$]$

$\mathrm{i}\vec{e}_{2}=[-1,1].$

przedstawić jako kombinację liniową wektorów $\vec{e}_{1}=[2$, 3$]$

6. Obliczyć granice $\displaystyle \lim_{x\rightarrow 0}x\sin\frac{1}{x}$

i

$\displaystyle \lim_{x\rightarrow+\infty}x\sin\frac{1}{x}.$

7. Dana jest funkcja $f(x)=\log_{\frac{1}{3}}(x+1)$. Rozwiązać nierówność $f(f(x))>0.$

8. Rozwiązać równanie $2^{2x}+4^{x}=5^{x}$

9. Podać równanie jednej z prostych, na której $\mathrm{l}\mathrm{e}\dot{\mathrm{z}}\mathrm{y}$ środek okręgu opisanego na

trójkącie o wierzcholkach $A(1,3), B(2,7)\mathrm{i}C(3,10).$

10. Dlajakich wartości parametru $k$ funkcja $f(x)=x^{3}-x^{2}+kx$ będzie rosnąca

w calym zbiorze liczb rzeczywistych?

ll. Dane są zbiory

$A=\{(x,y):(x-1)^{2}+y^{2}\leq 1\}$

oraz $B=\{(x,y):y\geq x\}.$

Naszkicować zbiór $A\cap B$ i obliczyć jego pole.

12. $\mathrm{W}$ oparciu o definicję pochodnej obliczyć $f'(1)$ dla funkcji $f(x)=\sqrt{x^{2}+3}.$

13. Zdarzenia losowe $A \mathrm{i} B$ są rozlączne i $P(A) =$

$P(A\cup B)$ oraz $P(A-B).$

$\displaystyle \frac{1}{3}$, a $P(B) = \displaystyle \frac{1}{2}$. Obliczyć

14. Napisać równanie sycznej do krzywej $y=x^{3}+x^{2}+x+1$ równolegfej do prostej

{\it y}$=$-32{\it x}.

15. Sformułować twierdzenie odwrotne do twierdzenia Pitagorasa.






POLITECHNIKA $\mathrm{G}\mathrm{D}\mathrm{A}\acute{\mathrm{N}}$ SKA

Gdańsk, 2.07.1991 r.

Tematy I części egzaminu z matematyki

dla kandydatów ubiegających się o przyjęcie na I rok studiów dziennych.

Kandydat wybierał 3 dowo1ne zadania. Rozwiązania wybranych zadań oceniane

byly w skali $0-10$ punktów. Egzamin trwa1120 minut.

l. Zbadać przebieg zmienności funkcji

{\it y}$=$ -4{\it xx}2 -$+$15

i na tej podstawie ustalić liczbę pierwiastków równania

-4{\it xx}2 -$+$51 $=${\it m}

w zalezności od parametru $m.$

2. $\mathrm{W}$ trójkącie $ABC$ dany jest wierzcholek $A(1,3)$ oraz równanie środkowej $y=7$

i równanie wysokości $x+4y-51=0$. Wiedząc, $\dot{\mathrm{z}}\mathrm{e}$ środkowa i wysokość wy-

chodzą z róznych wierzchofków trójkąta podać równania boków tego trójkąta.

3. Dla jakich wartości parametru $m\in R$ równanie

$\log_{2}(x+3)-2\log_{4}x=m$

posiada rozwiązanie nalezące do przedziafu $\langle$3; 4)?

4. $\mathrm{W}$ urnie znajdują się trzy kule biale o numerach 1, 2 $\mathrm{i}3$ oraz pięć kul czarnych

o numerach 1, 2, 3, 4 $\mathrm{i}5$. Losujemy bez zwracania dwukrotnie po jednej kuli.

Jakie jest prawdopodobieństwo tego, $\dot{\mathrm{z}}\mathrm{e}$ pierwsza z wylosowanych kul będzie

biala, a druga będzie kulą o numerze l?

5. Na trójkącie prostokątnym o kącie ostrym $x$ opisano okrąg. Okrąg ten i trójkąt

obracają się dookofa przeciwprostokątnej. Przy jakim $x$ stosunek objętości

kuli powstalej z obrotu okręgu do objętości bryly powstalej z obrotu trójkąta

będzie najmniejszy?




POLITECHNIKA $\mathrm{G}\mathrm{D}\mathrm{A}\acute{\mathrm{N}}$ SKA

Gdańsk, 2.07.1991 r.

Tematy II części egzaminu z matematyki

dla kandydatów ubiegajqcych się o przyjęcie na I rok studiów dziennych.

Wszystkie zadania byfy oceniane w skali $0-2$ punkty. Egzamin trwa1120 minut.

l. Dana jest funkcja $f(x)=\sin^{2}4x$. Rozwiązać równanie $f'(x)=-2.$

2. Rozwiązać nierówność $\log_{x}5<1.$

3. Dany jest trójkąt prostokątny o przyprostokątnych dlugości 3 $\mathrm{i}4$. Obliczyć

wysokość trójkąta poprowadzoną z wierzchofka kąta prostego.

4. Rozwiązać nierówność

$\displaystyle \frac{1}{x}>2-x.$

5. Rozwiązać nierównośč tg$(2x)\geq 1.$

6. $\mathrm{W}\mathrm{p}$laszczy $\acute{\mathrm{z}}\mathrm{n}\mathrm{i}\mathrm{e}0xy$ zaznaczyć punkty nalezące do zbioru

$A=\{(x,y):|x|<y\}.$

7. Obliczyć $\displaystyle \lim_{x\rightarrow 1}\frac{\sin 2(x-1)}{3(x^{2}-1)}.$

8. Podać resztę z dzielenia wielomianu $W(x) = 5x^{4}+2x^{2}+1$ przez dwumian

$x+1.$

9. $\mathrm{W}$ trójkącie o wierzcholkach $A(3,1,1), B(2,2,1) \mathrm{i}C(2,1,2)$ wyznaczyć kąt

wewnętrzny przy wierzcholku $A.$

10. Podać liczby naturalne spelniające nierówność $\left(\begin{array}{l}
n\\
2
\end{array}\right) -n\leq 14.$

ll. Dla jakich wartości parametru $k$ funkcja $f(x) = \displaystyle \frac{1}{3}x^{3}+\frac{3}{2}x^{2}+kx+1$ będzie

rosnąca w calej swojej dziedzinie?

12. Obliczyc $\displaystyle \lim_{n\rightarrow\infty}\frac{\sqrt{n^{2}+1}}{\sqrt[3]{8n^{3}+2n+1}}.$

13. Obliczyć prawdopodobieństwo wyrzucenia w pięciu rzutach kostkq co naj-

mniej raz liczby oczek nie większej od 3.

14. Napisać równanie prostej przechodzącej przez punkt $P(1,3)$ i prostopadfej do

prostej $y=2x+5.$

15. Suma wyrazów nieskończonego ciągu geometrycznego o pierwszym wyrazie

$a_{1}=3$ wynosi 5. Podać i1oraz tego ciągu.







POLITECHNIKA $\mathrm{G}\mathrm{D}\mathrm{A}\acute{\mathrm{N}}$ SKA

Gdańsk, 30.06.1992 r.

Tematy I części egzaminu z matematyki

dla kandydatów ubiegających się o przyjęcie na I rok studiów dziennych.

Kandydat wybierał 3 dowo1ne zadania. Rozwiązania wybranych zadań oceniane

byly w skali $0-10$ punktów. Egzamin trwa1120 minut.

l. Rozwiązać ukfad nierówności

$\left\{\begin{array}{l}
\sqrt{x+6}>x\\
2+\log_{0,5}(-x)>0
\end{array}\right.$

2. Dla jakich $a$ równanie

$\cos^{4}x+(a+2)\sin^{2}x-(2a+5)=0$

ma rozwiązanie?

3. Wykazać, $\dot{\mathrm{z}}\mathrm{e}$ pole trójkąta ograniczonego osiami ukladu wspólrzędnych i do-

wolną styczną do hiperboli $y=\displaystyle \frac{a^{2}}{x}$ jest równe $2a^{2}$

4. Wysokość stozka jest $x$ razy większa od promienia jego podstawy. Wyrazič

stosunek promieni kul opisanej i wpisanej w ten stozekjako funkcję $f(x)$ oraz

obliczyć granicę $\displaystyle \lim_{x\rightarrow+\infty} \underline{f(x)}.$

$x$

5. Dane są zbiory

$A=\{1$, 2, 3, $\ldots$, 222$\}$

$\mathrm{i} B=\{1$, 2, 3, $\ldots$, 444$\}.$

Losowo wybieramy zbiór, a z niego liczbę $x$. Obliczyć prawdopodobieństwo

tego, $\dot{\mathrm{z}}\mathrm{e}$ liczba $x^{2}+1$ dzieli się przez 10.




POLITECHNIKA $\mathrm{G}\mathrm{D}\mathrm{A}\acute{\mathrm{N}}$ SKA

Gdańsk, 30.06.1992 r.

Tematy II części egzaminu z matematyki

dla kandydatów ubiegajqcych się o przyjęcie na I rok studiów dziennych.

Wszystkie zadania byfy oceniane w skali $0-2$ punkty. Egzamin trwa1120 minut.

l. Wyznaczyć dziedzinę funkcji $f(x)=\sqrt{\frac{5}{x+2}-1}.$

2. Rozwiązać równanie $\displaystyle \frac{\cos x}{1-\sin x}=1+\sin x.$

3. Narysować wykres funkcji $f(x)=x\displaystyle \sqrt{x^{2}}+\frac{x}{|x|}.$

4. Na paraboli $y=48-x^{2}$ znalez$\acute{}$ć wszystkie punkty $(x,y)$ takie, $\dot{\mathrm{z}}\mathrm{e}$ liczby 3, $x,$

$y$ tworzą ciąg geometryczny.

5. Wyznaczyć dziedzinę funkcji $f(x)=\log(3^{x}-5^{x}).$

6. Rózniczkując $\mathrm{t}\mathrm{o}\dot{\mathrm{z}}$ samość $\sin 2x=2\sin x\cos x$ wykazać $\mathrm{t}\mathrm{o}\dot{\mathrm{z}}$ samość $\cos 2x=$

$\sin^{2}x.$

$\cos^{2}x-$

7. Obliczyć $\displaystyle \lim_{x\rightarrow 0}\frac{x^{2}}{1-\cos 2x}.$

8. $\mathrm{W}$ trójkącie ostrokątnym $ABC$ z wierzchofków $A\mathrm{i}C$ opuszczono wysokości

{\it AD} $\mathrm{i}$ CE na boki $BC\mathrm{i}$ AB. Wykazać, $\dot{\mathrm{z}}\mathrm{e}$ trójkqty $ABC\mathrm{i}BDE$ są podobne.

9. Suma pierwiastków trójmianu $y = ax^{2}+bx+c$ jest równa $\log_{a^{2}}c\cdot\log_{c^{2}}a.$

Znalez$\acute{}$ć odciętq wierzchofka paraboli.

10. Dane są wektory $\overline{A}\mathfrak{F}=[1$, 2, 3$]\mathrm{i}\overline{A}7=[3$, 2, 1$]$. Obliczyć pole trójkąta $ABC.$

ll. Proste $P_{1}, P_{2} \mathrm{i}P_{3}$ są równoległe i lezą w jednej płaszczyz$\acute{}$nie. Na prostej $P_{1}$

wybrano 3 punkty, na $\ell_{2}$ wybrano 4 punkty, a na $\ell_{3}$ wybrano 5 punktów. I1e

co najwyzej istnieje trójkątów o wierzchofkach w tych punktach?

12. Obliczyč $\displaystyle \lim_{n\rightarrow\infty}\frac{1+4+7+\ldots+(3n-2)}{2n^{2}+3n+4}.$

13. Wykazać, $\dot{\mathrm{z}}\mathrm{e}$ finkcja $f(x) = \sqrt{1+x+x^{2}}-\sqrt{1-x+x^{2}}$ jest nieparzysta w

swojej dziedzinie.

14. Dany jest trójkat o wierzcholkach $A(1,-1), B(3,3)\mathrm{i}C(-5,1)$. Napisać rów-

nanie symetralnej boku $\overline{BC}.$

15. Zbadać monotoniczność funkcji $f(x)=x^{4}-\displaystyle \frac{1}{x}+5$ w przedziale $(0;+\infty).$







POLITECHNIKA $\mathrm{G}\mathrm{D}\mathrm{A}\acute{\mathrm{N}}$ SKA

Gdańsk, 30.06.1993 r.

EGZAMIN WSTĘPNY Z MATEMATYKI

Zestaw sklada się z 30 zadań. Zadania $1-10$ oceniane będą w skali $0-2$ punkty, zadania

$11-30$ w skali $0-4$ punkty. Czas trwania egzaminu -- 180 minut.

{\it Powodzenia}.'

l. Obliczyć

$\displaystyle \lim_{n\rightarrow\infty}\frac{n\sqrt{1+3+5++(2n-1)}}{2n^{2}+n+1}.$

2. Rozwiązać nierówność $x^{2}-4x+9\displaystyle \leq\frac{18}{x+2}.$

3. Rozwiązać nierówność

$\log_{0,3}(x+1)>-1.$

4. Rozwiązać nierówność $2-|1-2x|>1.$

5. Dla jakich wartości parametru $\alpha \in (0;2\pi)$ równanie $\sin 2x=  2\cos\alpha$ posiada roz-

wiązanie?

6. Obliczyč dlugośč wektora $\vec{a}, \mathrm{j}\mathrm{e}\dot{\mathrm{z}}$ eli ã $0\vec{b}=7,$

$\vec{a}\Vert\vec{b} \mathrm{i} \vec{b}=[3,-2,1].$

7. Rozwiązać nierówność $2^{x^{2}}<5^{x}$

8. Wykazać, $\dot{\mathrm{z}}\mathrm{e}$ funkcja $f(x)=3x^{3}+4x+\cos 2x$ jest rosnąca w calej swojej dziedzinie.

9. Wyznaczyc te wartości parametru $k$, dla których prosta $y=kx+4$ będzie równolegla

do prostej 

10. Dla jakich $a\mathrm{i}b$ wielomian $W(x)=12x^{4}-17x^{2}+ax+b$ dzieli się bez reszty przez

$2x^{2}+x-1$?

ll. Dany jest trójkat o wierzchofkach $A(1,1), B(-1,3), C(3,7)$ i polu $S$. Przez wierz-

cholek $A$ poprowadzić jedną z prostych, ktora dzieli dany trójkąt na dwa trójkąty

o polach $\displaystyle \frac{1}{4}S\mathrm{i}\frac{3}{4}S$. Podać równanie tej prostej.

12. Znalez$\acute{}$ć ekstrema funkcji $f(x) = (x+3)^{2}(x+8)^{3}$

$f(x)=108$?

Ile pierwiastków ma równanie

13. Dla jakiej wartości parametru a funkcja

$f(x)=$

dla

dla

$x\neq 0$

$x=0$

będzie funkcją ciąglą w punkcie $x=0$?

14. Który z punktów paraboli $y=x^{2}$ jest polozony najblizej prostej $y=2x-2$?




15. Wykazać, $\dot{\mathrm{z}}\mathrm{e}$ pole dowolnego wypukfego czworokątajest równe pofowie iloczynu jego

przekątnych pomnozonego przez sinus kąta między nimi, $S=\displaystyle \frac{1}{2}d_{1}d_{2}\sin\alpha.$

16. Dany jest ciąg arytmetyczny (o róznicy róznej od zera), w którym suma $n$ począt-

kowych wyrazów jest równa polowie sumy następnych $n$ wyrazów. Wyznaczyć iloraz

$\displaystyle \frac{S_{3n}}{S_{n}}$, gdzie $S_{k}$ oznacza sumę $k$ początkowych wyrazów tego ciągu.

17. Wykazać, $\dot{\mathrm{z}}\mathrm{e}$ dwie styczne do paraboli $y=x^{2}$ poprowadzone z dowolnego punktu

prostej $y=-\displaystyle \frac{1}{4}$ są do siebie prostopadle.

18. Dany jest trójkąt równoramienny o ramionach $\overline{AC}\mathrm{i}\overline{BC}$ dlugości 3 cm i podstawie

$\overline{AB}$ dlugości 4 cm. Ob1iczyć i1oczyn ska1arny AS o $\overline{B}7.$

19. Miary kątów wewnętrznych trójkąta tworzą ciąg arytmetyczny. Najmniejszy bok

jest trzy razy mniejszy od największego boku w tym trójkącie. Obliczyć cosinus

najmniejszego kąta.

20. Ze zbioru liczb \{l, 2, 3, 4, 5, 6, 7, 8, 9, l0\} losujemy dwukrotnie po jednej liczbie

bez zwracania. Obliczyć prawdopodobieństwo tego, $\dot{\mathrm{z}}\mathrm{e}$ druga z wylosowanych liczb

będzie większa od pierwszej.

21. Podać definicję asymptoty pionowej i wyznaczyć asymptoty pionowe funkcji $f(x)=$

$\displaystyle \frac{1}{x(2^{x}-4)}.$

22. Wyznaczyć najmniejszą i największą wartość funkcji $f(x)=\displaystyle \cos(\frac{\pi}{2}\cdot x)-3x$ w prze-

dziale $\langle 0;1\rangle.$

23. Dla jakiej wartości parametru $m$ okrąg $(x-m)^{2}+(y-1)^{2}=1$ będzie styczny do

prostej $3x+4y+1=0$?

24. Wykazać, $\dot{\mathrm{z}}\mathrm{e}$ równanie $x=\displaystyle \frac{1}{2}\sin x+a$, gdzie $a>0$, ma dokladnie jeden pierwiastek

w przedziale $\langle 0;a+1\rangle.$

25. $\mathrm{Z}$ definicji pochodnej obliczyć $f'(3)$, gdy $f(x)=\sqrt{2x+3}.$

26. Rozwiązač równanie

$\left(\begin{array}{l}
x+3\\
2
\end{array}\right)+\left(\begin{array}{l}
x+1\\
x-1
\end{array}\right)=31.$

27. Dlugość dluzszej podstawy trapezu równoramiennego jest równa l3 cm, a jego ob-

wódjest równy 28 cm. Wyrazić po1e trapezujako funkcję d1ugości ramienia trapezu.

Znalez$\acute{}$ć dziedzinę i zbiór wartości tej funkcji.

28. Dla jakich wartości parametru $k$ ciąg $(a_{n})$, gdzie $a_{n} = \displaystyle \frac{n^{k}}{2+4+\ldots+2n}$, będzie

rozbiezny do $+\infty$?

29. Dana jest funkcja $f(x)=\displaystyle \cos^{2}3x+\frac{3}{2}x-$ log5. Rozwiązać równanie $f'(\displaystyle \frac{1}{3}x)=0.$

30. Dane są liczby $A = \displaystyle \frac{5678901234}{6789012345} \mathrm{i} B = \displaystyle \frac{5678901235}{6789012346}$. Która z nich jest większa?

Swoją odpowied $\acute{\mathrm{z}}$ uzasadnić.







POLITECHNIKA $\mathrm{G}\mathrm{D}\mathrm{A}\acute{\mathrm{N}}$ SKA

Gdańsk, czerwiec 1994 r.

EGZAMIN WSTĘPNY Z MATEMATYKI

Zestaw składa się z 30 zadań. Zadania $1-10$ oceniane będą w skali $0-2$ punkty,

zadania $11-30$ w skali $0-4$ punkty. Czas trwania egzaminu -- 240 minut.

{\it Powodzenia}.'

l. Rozwiązać nierówność $x-\displaystyle \frac{2}{x}\geq 1.$

2. Dla jakich a równanie

rzeczywisty?

$x^{2}+ax+a-1=0$ posiada co najmniej jeden pierwiastek

3. Rozwiązać równanie $\sqrt{x}+2=x.$

4. Trzy liczby tworzą ciag arytmetyczny o sumie równej l8. Największa z nich jest

równa 9. Wyznaczyć pozosta1e 1iczby.

5. Rozwiązać nierówność $(\displaystyle \frac{1}{2})^{|x-3|}\geq\frac{1}{4}.$

6. Dany jest sześcian o krawędzi $a$ Obliczyć objętość kuli stycznej do wszystkich

krawędzi tego sześcianu.

7. Obliczyć $(\sqrt[3]{4})^{\frac{3}{2\log_{3}2}}$

8. Dla jakich $x\in(0;\pi)$ spefniona jest nierówność $\mathrm{c}\mathrm{t}\mathrm{g}^{2}x\geq 3$?

9. Obliczyć granicę $\displaystyle \lim_{n\rightarrow\infty}\frac{(n+2)!}{n^{2}\cdot n!}.$

10. Graficznie rozwiązać nierówność $\log_{\frac{1}{2}}|x|\geq x^{2}-1.$

ll. Wielomian $w(x)=x^{3}-3x+a$ rozfozyć na czynniki wiedząc, $\dot{\mathrm{z}}\mathrm{e}$ liczba $-1$ jest

jego pierwiastkiem.

12. Dla jakich parametrów m uklad równań

$\left\{\begin{array}{l}
mx-2y=1\\
8x-my=2
\end{array}\right.$

jest sprzeczny?

13. Trójkąt ma boki długości 6, 8 i l0. Obliczyć promień okręgu opisanego na tym

trójkącie i promień okręgu wpisanego w ten trójkąt.

14. Napisać równanie stycznej do wykresu funkcji

$x_{0}=0.$

$f(x)=4\sqrt[3]{8+\sin 3x}$ w punkcie

15. Dlajakich wartości parametru $m$ okręgi $x^{2}+y^{2}-2x=0$ oraz $x^{2}+(y-m)^{2}=9$

są styczne wewnętrznie?




16. Wyznaczyć dziedzinę funkcji

$f(x)=\sqrt{\log(3^{x}-2^{x}+1)}.$

17. Trzy razy rzucamy dwiema kostkami do gry. Jakiejest prawdopodobieństwo tego,

$\dot{\mathrm{z}}\mathrm{e}$ co najmniej raz suma oczek będzie większa od 9?

18. Obliczyć granicę $\displaystyle \lim_{x\rightarrow 0}\log_{2}(\frac{x^{2}}{1-\cos 4x}).$

19. Niech $f(m)$ oznacza liczbę pierwiastków równania $|4x^{2}-4x-3|=m$. Narysować

wykres funkcji $f(m).$

20. Na prostej $y-x-1=0$ znalez$\acute{}$ć punkt $A$ taki, $\dot{\mathrm{z}}\mathrm{e}$ pole trójkąta o wierzchofkach

w punktach $A, B(4,-1)\mathrm{i}C(4,3)$ jest równe 2.

21. Obliczyć kat między wektorami ã $\mathrm{i}\vec{b}$, jeśli wiadomo, $\dot{\mathrm{z}}\mathrm{e}$ wektory ũ $= -\vec{a}+4\vec{b}$

$\mathrm{i}\vec{v}=3\vec{a}+2\vec{b}$ są prostopadle i lãl $=|\vec{b}|=1.$

22. Uzasadnić, $\dot{\mathrm{z}}\mathrm{e}$ prosta $4x+2y-3=0$jest równolegfa do prostej 

Obliczyć odleglość między tymi prostymi.

$= -t+1$

$= 2t-3$

23. Zbadać monotoniczność funkcji $f(x)=x^{3}-3x^{2}+4x+\cos x.$

24. $\mathrm{W}$ trapez równoramienny o polu $S$ wpisano czworokąt tak, $\dot{\mathrm{z}}$ ejego wierzcholki sa

środkami boków trapezu. Jaki to czworokąt? Obliczyć jego pole.

25. Niech $A\mathrm{i}B$ będą zdarzeniami losowymi takimi, $\dot{\mathrm{z}}\mathrm{e}P(A) =0, 7\mathrm{i}P(B) =0$, 9.

Wykazać, $\displaystyle \dot{\mathrm{z}}\mathrm{e}P(A|B)\geq\frac{2}{3}.$

26. Obliczyć granice $\displaystyle \lim_{x\rightarrow+\infty}(x-\sqrt{x^{2}-x+1})$

oraz $\displaystyle \lim_{x\rightarrow-\infty}(x-\sqrt{x^{2}-x+1}).$

27. Rozwiązač równanie $1+\displaystyle \frac{\mathrm{l}}{2\sin x}+\frac{\mathrm{l}}{4\sin^{2}x}+\cdots=\frac{2}{\sin x}.$

28. Wyznaczyć największą i najmniejsza wartość funkcji

przedziale $\displaystyle \langle 0;\frac{\pi}{2}\rangle.$

$f(x) = \displaystyle \frac{1}{\sin x+\cos x}$

w

29. Podać definicję ciągu ograniczonego. Następnie wykazać, $\dot{\mathrm{z}}\mathrm{e}$ ciąg

$a_{n}=\displaystyle \frac{1}{n+1}+\frac{1}{n+2}+\cdots+\frac{1}{2n}$

jest ograniczony.

30. Podać i udowodnić warunek konieczny istnienia maksimum lokalnego funkcji róz-

niczkowalnej.





Odpowiedzi do kolejnych zadań:

1. $x\in\langle-1;0)\cup\langle 2;+\infty)$ ;

2. dla $\mathrm{k}\mathrm{a}\dot{\mathrm{z}}$ dej liczby $a\in R$;

3. $x=4$;

4. 1iczbami tymi są $a_{1}=3, a_{2}=6\mathrm{i}a_{3}=9$;

5. $ x\in\langle 1;5\rangle$;

6. $V=\displaystyle \frac{1}{3}\pi a^{3}\sqrt{2}$;

7.

$(\sqrt[3]{4})^{\frac{3}{2\log_{3}2}}=3$;

8. $x\displaystyle \in(0;\frac{\pi}{6}\rangle\cup\langle\frac{5}{6}\pi;\pi)$ ;

9. 1;

10. $ x\in\langle-1;0)\cup(0;1\rangle$;

11. $w(x)=(x+1)^{2}(x-2)$ ;

12. $m=-4$;

13. $R=5\mathrm{i}r=2$;

14. $y=x+8$;

15. $m=\pm\sqrt{3}$;

16. $x\geq 0$;

17. $P=\displaystyle \frac{91}{216}$;

18. $-3$;
\begin{center}
\includegraphics[width=72.036mm,height=58.008mm]{./PolitechnikaGdanska_EgzaminWstepny_1994_page2_images/image001.eps}
\end{center}
4  0

3

2

1

0 l  2 3 4

19.

20. A(3,4) lub A(5,6) ;

21.

$\displaystyle \frac{2}{3}\pi$;





22. $d=\displaystyle \frac{\sqrt{5}}{2}$;

23. Zauwazmy, $\dot{\mathrm{z}}\mathrm{e}$ spelnionajest nierównośč $3x^{2}-6x+4\geq 1$ (i nawet $3x^{2}-6x+4>1,$

gdy $x\neq 1)$. Stąd $\mathrm{j}\mathrm{u}\dot{\mathrm{z}}$ wynika, $\dot{\mathrm{z}}\mathrm{e}$ pochodna funkcji $f(x)$ jest dodatnia,

$f'(x)=3x^{2}-6x+4-\sin x>0$ (takze dla $x=1$),

i dlatego fnkcja $f(x)$ jest rosnąca;

24. czworokąt jest rombem o polu $P=S/2$;

25. $P(A|B)=\displaystyle \frac{P(A\cap B)}{P(B)}=\frac{P(A)+P(B)-P(A\cup B)}{P(B)}\geq\frac{0,7+0,9-1}{0,9}=\frac{2}{3}$;

26. 1/2 $\mathrm{i}-\infty$;

27. $x=\displaystyle \frac{\pi}{2}+2k\pi \mathrm{i}k$ jest liczbą cafkowitą;

28. $M=1\displaystyle \mathrm{i}m=\frac{\sqrt{2}}{2}$;

29. $(a_{n})$ jest ograniczony, gdy istnieje liczba rzeczywista $M$ taka, $\dot{\mathrm{z}}\mathrm{e} |a_{n}| \leq M$ dla

$\mathrm{k}\mathrm{a}\dot{\mathrm{z}}$ dej liczby naturalnej $n$. Dla rozwazanego ciągu i dla $\mathrm{k}\mathrm{a}\dot{\mathrm{z}}$ dej liczby naturalnej

$n$ jest

$|\displaystyle \alpha_{n}|=a_{n}=\frac{1}{n+1}+\frac{1}{n+2}+\cdots+\frac{1}{2n}\leq\frac{1}{n+1}+\frac{1}{n+1}+\cdots+\frac{1}{n+1}=\frac{n}{n+1}\leq 1,$

więc ciąg ten jest ograniczony.

30. Jeśli funkcja $f(x)$ jest rózniczkowalna w punkcie $x_{0}$ i jeśli ma ona maksimum

lokalne w tym punkcie, to $f'(x_{0})=0.$







POLITECHNIKA $\mathrm{G}\mathrm{D}\mathrm{A}\acute{\mathrm{N}}$ SKA

Gdańsk, 28.06.1995 r.

EGZAMIN WSTĘPNY Z MATEMATYKI

Zestaw sklada się z 30 zadań. Zadania $1-10$ oceniane będą w skali $0-2$ punkty,

zadania $11-30$ w skali $0-4$ punkty. Czas trwania egzaminu -- 240 minut.

{\it Powodzenia}.$l$

l. Rozwiązać nierówność $|4+x-|3x-2||\leq 0.$

2. Rozwiązać równanie $2^{2x+1}+3\cdot 4^{x}=10.$

3. Rozwiązać nierówność $\displaystyle \frac{1}{x-1}\geq\frac{2}{x-2}.$

4. Obliczyč $1000^{\frac{1}{3}-\log\sqrt[3]{3}}.$

5. Na osi $0y$ znalez$\acute{}$ć punkt $M$ równo oddalony od punktów $A(2,-1,5)\mathrm{i}B(-3,2,4).$

6. Wielomian $w(x)=x^{4}+x^{2}+1$ rozlozyć na czynniki.

7. Wyznaczyć n

z równania

$1+5+9+\ldots+(4n-3)=120.$

8. Obliczyć granicę $\displaystyle \lim_{n\rightarrow\infty}(\log(10n^{2}+1)-2\log n).$

9. Obliczyč $y'(\displaystyle \frac{\pi}{4})$, jeśli $y(x)=\sqrt{1+\cos 2x}.$

10. Obliczyć stosunek objętości kuli opisanej na walcu do objętości kuli wpisanej

w ten walec.

ll. Znalez$\acute{}$ć składnik wymierny rozwinięcia dwumianu $(\sqrt[3]{2}+\sqrt[4]{3})^{10}$

12. Dlajakich parametrów $\alpha$ równanie $x^{2}+4x\sin\alpha+1=0$ posiada co najmniej

jeden pierwiastek rzeczywisty?

13. Dlajakich wartości $a\mathrm{i}b$ liczba $-1$ jest pierwiastkiem podwójnym wielomianu

$w(x)=x^{3}+ax^{2}+bx-3$?

14. Rozwiązać nierówność $\log_{2}x+\log_{x}2\geq 2.$

15. Wiadomo, $\dot{\mathrm{z}}\mathrm{e}$ zdarzenia losowe $A\mathrm{i}B$ są niezalezne oraz $P(A)=p_{1}$ i $P(B)= p_{2}.$

Obliczyć prawdopodobieństwa $P(A|B)$ oraz $P(A-B).$




16. Dane są funkcje

$= g(f(x)).$

$f(x)=\sqrt{x}$

i

$g(x)=1-x$. Rozwiązać równanie $f(g(x))$

17. Dany jest ciąg geometryczny $(a_{n})$. Pokazać, $\dot{\mathrm{z}}\mathrm{e}$ ciąg $(b_{n})$, gdzie $b_{n}=\alpha_{n+1}-\alpha_{n},$

$\mathrm{t}\mathrm{e}\dot{\mathrm{z}}$ jest ciągiem geometrycznym.

18. Dwa punkty wyruszają jednocześnie z wierzcholka kąta o mierze $120^{\mathrm{o}}$ po je-

go ramionach z prędkościami odpowiednio 5 $\mathrm{m}/\mathrm{s}$ i 3 $\mathrm{m}/\mathrm{s}$. Po jakim czasie

odleglość między nimi będzie wynosila 49 $\mathrm{m}$?

19. Napisać równanie okręgu stycznego do obu osi ukladu wspólrzędnych i prze-

chodzącego przez punkt $P(2,1).$

20. Na podstawie definicji obliczyć pochodną funkcji $f(x)=\cos 3x.$

21. Narysować wykres funkcji $f(x)=2^{\log_{\frac{1}{2}}x}$

22. Wyznaczyć największą i najmniejszą wartość funkcji $f(x)=x+$ ctg $x$ w prze-

dziale $\displaystyle \langle\frac{1}{4}\pi;\frac{3}{4}\pi\rangle.$

23. $\mathrm{Z}$ prawdopodobieństwem 1/2 w urnie znajduje się a1bo ku1a biafa, a1bo czarna.

Do urny dokladamy kulę bialą i następnie losujemy jedną kulę. Jakie jest

prawdopodobieństwo tego, $\dot{\mathrm{z}}\mathrm{e}$ wylosujemy kulę bialą?

24. Udowodnić, $\dot{\mathrm{z}}\mathrm{e}$ wszystkie trójkaty prostokątne, których boki tworzą ciąg aryt-

metyczny, są podobne.

25. Wyznaczyć asymptoty funkcji $y=\displaystyle \frac{\sqrt{x^{2}+x+1}}{x}.$

26. Obliczyć tg $\alpha$, jeśli $\displaystyle \sin\alpha-\cos\alpha=\frac{\sqrt{2}}{2}$

$\mathrm{i} \displaystyle \alpha\in(\frac{\pi}{4};\frac{\pi}{2}).$

27. Narysować na płaszczyz$\acute{}$nie zbiór punktów, których wspólrzędne spełniają nie-

równość $y^{2}+xy-2x^{2}<0.$

28. Obliczyć dlugości przekątnych równolegfoboku zbudowanego na wektorach $\vec{a}$

$\mathrm{i} \vec{b}, \mathrm{j}\mathrm{e}\dot{\mathrm{z}}$ eli ã $= 2\vec{m}$ -{\it ñ}, $\vec{b}= 3\text{{\it ñ}}-\vec{m}$, gdzie wektory $\vec{m} \mathrm{i}$ ñ są ortogonalne

$\mathrm{i}|\vec{m}|= |${\it ñ}$| =1.$

29. Wykazać, $\dot{\mathrm{z}}\mathrm{e}$ funkcja $y=\sqrt{x^{3}-1}$ jest róznowartościowa w swojej dziedzinie.

Następnie wyznaczyć funkcją do niej odwrotną.

30. Wykazać, $\dot{\mathrm{z}}\mathrm{e}$ jeśli ciąg $(a_{n})$ jest ograniczony i $\displaystyle \lim_{n\rightarrow\infty}b_{n}=0$, to $\displaystyle \lim_{n\rightarrow\infty}a_{n}\cdot b_{n}=0.$







POLITECHNIKA $\mathrm{G}\mathrm{D}\mathrm{A}\acute{\mathrm{N}}$ SKA

Gdańsk, 1.07.1996 r.

EGZAMIN WSTĘPNY Z MATEMATYKI

Zestaw sklada się z 30 zadań. Zadania $1-10$ oceniane będą w skali $0-2$ punkty, zadania

$11-30$ w skali $0-4$ punkty. Czas trwania egzaminu -- 240 minut.

{\it Powodzenia}.'

l. Funkcję kwadratową $y=(x+3)(1-x)$ przedstawić w postaci kanonicznej. Naszki-

cować jej wykres.

2. Rozwiązać równanie $5^{x}\displaystyle \cdot 5^{x^{2}}\cdot 5^{x^{3}}=\frac{1}{5}.$

3. Rozwiązać równanie $\log_{\frac{1}{3}}(|x|-1)>2.$

4. Dla jakich parametrów $a\in R$ równanie $\displaystyle \cos^{2}x=\frac{2a}{a-2}$ ma rozwiązanie?

5. Naszkicować wykres funkcji $y=x\log_{x^{2}}|x|.$

6. Wyznaczyć te wartości $x$, dla których punkty $A(5,5), B(1,3)\mathrm{i}C(x,0)$ są wspófli-

niowe.

7. Wskazač większq z liczb 0, 4(9) i $\displaystyle \sin(\frac{101}{6}\pi).$

8. Napisać równanie stycznej do wykresu funkcji $f(x)=\sqrt{2x-3}$ w punkcie o odciętej

$x_{0}=6.$

9. Dana jest funkcja $f(x)=\cos^{2}x$. Narysować wykres funkcji $y=f'(x)$ w przedziale

$\langle 0;\pi\rangle.$

10. Zbadać monotoniczność funkcji $f(x)=x+\displaystyle \frac{1}{x}.$

ll. Dany jest ciąg $(a_{n})$, gdzie $a_{n}=\displaystyle \frac{(n!)^{2}}{(2n)!}$. Obliczyć $\displaystyle \lim_{n\rightarrow\infty}\frac{a_{n+1}}{\alpha_{n}}.$

12. Rozwiązać nierówność $g(f(x))\geq 1$, jeśli $f(x)=3^{x}\mathrm{i}g(x)=\sin x.$

13. Wyznaczyć wszystkie wielokąty wypukfe, w których liczba przekątnych jest 3 razy

większa od liczby wierzchofków.

14. Rozwiązać równanie $|\cos x|=\cos x+2\sin x$ w przedziale $\langle 0;2\pi\rangle.$

15. Rozwiązać nierówność $\displaystyle \frac{x^{3}-x+6}{x^{2}}\geq 0.$




16. Rozwiązać równanie $1-\displaystyle \frac{1}{x}+\frac{1}{x^{2}}-\frac{1}{x^{3}}+\ldots=x-1.$

17. Dla jakich $x\in R$ ciąg 2 $\log_{3}x, \log_{3}(x-1)$, -log34 jest ciągiem arytmetycznym?

18. Niech $g$ bedzie granicą ciągu $(a_{n})$, gdzie $a_{n} = \displaystyle \frac{3n+1}{n+1}$. Od jakiego $n$ począwszy

wyrazy ciągu $(a_{n})$ spelniają nierówność $|a_{n}-g|<0$, 01?

19. Dla jakich $a\in R$ funkcja $f(x)=\{_{\frac{\cos x\sin|2x|}{x}}+a$

dla

dla

$x\geq 0$

$x<0$

jest ciagla?

20. Wielomian $x^{2}+px+q$ ma pierwiastki $x_{1}$ i $x_{2}$. Wskazać trójmian $x^{2}+bx+c$, którego

pierwiastkami są liczby $x_{1}+1\mathrm{i}x_{2}+1.$

21. Ze zbioru \{l, 2, $\ldots$, 1000\} 1osujemy jedną liczbę. Obliczyć prawdopodobieństwo te-

go, $\dot{\mathrm{z}}\mathrm{e}$ nie będzie to liczba podzielna ani przez 6, ani przez 8.

22. Obliczyč pole trapezu o podstawach dlugości a $\mathrm{i}b, \mathrm{j}\mathrm{e}\dot{\mathrm{z}}$ eli wiadomo, $\dot{\mathrm{z}}\mathrm{e}$ na tym trapezie

$\mathrm{m}\mathrm{o}\dot{\mathrm{z}}$ na opisać okrąg i $\mathrm{m}\mathrm{o}\dot{\mathrm{z}}$ na w niego wpisać okrag.

23. Znalez$\acute{}$ć rzut prostokątny punktu $A(1,-1)$ na prostą 

24. Dane są zbiory $A=\{(x,y):x,y\in R\mathrm{i}x^{2}+y^{2}-2y\leq 1\} \mathrm{i} B=\{(x,y)$ : $x, y$

$\in R \mathrm{i} |x|+y \leq 1\}$. Narysować na plaszczy $\acute{\mathrm{z}}\mathrm{n}\mathrm{i}\mathrm{e}$ ukladu wspólrzędnych zbiór

$A\cap B$ i obliczyć jego pole.

25. Wyznaczyć asymptoty funkcji $f(x)=\displaystyle \frac{\sqrt{x^{2}-1}-x}{x}.$

26. Obliczyć $|\vec{a}-\vec{b}|$, jeśli $|\vec{a}+\vec{b}|=5,\ |${\it ã}$| =3\mathrm{i}|\vec{b}|=2\sqrt{2}.$

27. Wyznaczyć zbiór wartości funkcji $y=x\sqrt{4-x^{2}}.$

28. Rzucamy symetryczną monetą. Obliczyć prawdopodobieństwo zdarzenia, $\dot{\mathrm{z}}\mathrm{e}$ w szó-

stym rzucie otrzymamy trzeciego orla.

29. Uzasadnić, $\dot{\mathrm{z}}\mathrm{e}$ równanie $x^{3}+x+7=0$ w zbiorze liczb rzeczywistych posiada dokladnie

jedno rozwiązanie. Wraz z uzasadnieniem wskazać przedzial o dlugości co najwyzej

1/2, do którego nalezy to rozwiązanie.

30. Ostrosfup przecięto pfaszczyznq równolegla do podstawy i dzielqcą wysokość w sto-

sunku 2: 3. Ob1iczyć stosunek objętości powstafych bry1.







POLITECHNIKA $\mathrm{G}\mathrm{D}\mathrm{A}\acute{\mathrm{N}}$ SKA

Gdańsk, 30.06.1997 r.

EGZAMIN WSTĘPNY Z MATEMATYKI

Egzamin sklada się z 30 zadań. Zadania $1-10$ oceniane bedą w skali $0-2$ punkty, zadania

$11-30$ w skali $0-4$ punkty. Czas trwania egzaminu -- 240 minut.

{\it Powodzenia}.$\displaystyle \int$

l. Syn jest o 301at młodszy od ojca. 51at temu ojciec był 7 razy starszy od syna.

$\mathrm{W}$ którym roku urodził $\mathrm{s}\mathrm{i}\mathrm{e}$ syn?

2. Znalez/č pola kwadratów, których dwoma wierzchołkami są punkty $(-1,1)\mathrm{i}(2,1).$

3. Podač przykład ciągu niemonotonicznego, którego granicą jest liczba 2.

4. Dla jakich parametrów $a$ dziedziną funkcji $y = \sqrt{ax^{2}+x+a}$ jest zbiór wszystkich

liczb rzeczywistych?

5. Rozwiązač równanie $\log_{2}x\cdot\log_{x}4=2.$

6. Obliczyč sumę wspólczynników wielomianu $w(x)=(x^{2}+2x-1)^{10}-20x-3.$

7. Obliczyč granicę $\displaystyle \lim_{n\rightarrow\infty}\frac{(n+2)!+n!}{(n+2)!-(n+1)!}.$

8. Napisač równanie prostej zawierającej tę cieciwę okręgu $x^{2}-4x+y^{2}+2y+1 =0,$

którą punkt $A(1,-\displaystyle \frac{1}{2})$ dzieli na dwie równe części.

9. Obliczyč $f'(0)$, jeśli $f(x)=x(x-1)(x-2)(x-3)(x-4)(x-5).$

10. Obliczyč $\displaystyle \sin\frac{13}{12}\pi.$

ll. Rozwi$\Phi$zač ukfad równań 

Podač ilustrację graficzną tego ukfadu.

12. Znalez/č resztę z dzielenia wielomianu $x^{1997}-x^{1996}+2$ przez $x^{3}-x.$

13. Dla jakiego $m$ równanie $|x^{2}-2|=\log_{\frac{1}{2}}m$ ma dokładnie 4 pierwiastki?

14. Rozwiązač równanie $|x-3|^{x^{2}-4x+3}=1.$

15. Rozwiqzač nierównośč $x+1\leq\sqrt{3+x}.$

16. Rozwiązač równanie tg $x=\displaystyle \mathrm{t}\mathrm{g}\frac{1}{x}.$

17. Niech $S_{n}$ oznacza sume $n$ początkowych wyrazów ciągu $a_{n}=\displaystyle \frac{2^{n}+3^{n}}{6^{n}}.$ Obliczyč $\displaystyle \lim_{n\rightarrow\infty}S_{n}.$




18. Ile razy nalezy rzucič symetryczną monetą, aby z prawdopodobieństwem większym od

$\displaystyle \frac{1}{2}$ otrzymač przynajmniej dwa orły?

19. Zdarzenia losowe $A\mathrm{i}B$ są jednakowo prawdopodobne, zawsze zachodzi przynajmniej

jedno z nich i $P(A|B)=\displaystyle \frac{1}{2}$. Obliczyč prawdopodobieństwa zdarzeń A $\mathrm{i}B$. Czy zdarzenia

$ A\mathrm{i}B\mathrm{s}\Phi$ niezalezne?

20. Uzasadnič, $\dot{\mathrm{z}}\mathrm{e}$ nie istnieje trójkąt o wysokościach długości 1, 2 $\mathrm{i}3.$

21. Znalez/č rzut równoległy punktu $A(5,2,9)$ na płaszczyznę $Oxy$ w kierunku wektora

$\vec{v}=[1$, 2, 3$].$

22. Rys. l przedstawia szkic wykresu funkcji $f(x)=$

$ax-b$

$x-c$

dla pewnych liczb $a, b\mathrm{i}c$. Wyznaczyc wspófrzedne punk-

tow $P\mathrm{i}Q$. Wskazac liczby $a, b\mathrm{i}c$, dla których wykres

funkcji $y = f(x) \mathrm{m}\mathrm{o}\dot{\mathrm{z}}$ na otrzymac z wykresu funkcji

$y=\displaystyle \frac{1}{x}$ w wyniku translacji o wektor $\vec{u}=[1$, 3$].$
\begin{center}
\includegraphics[width=37.236mm,height=39.012mm]{./PolitechnikaGdanska_EgzaminWstepny_1997_page1_images/image001.eps}
\end{center}
$y$

$---Q--||$

$|P x$

23. Wyznaczyč liczbę $a$ tak, aby funkcja $f(x)=\{_{\frac{\sin(x-1)2+ax}{|x-1|}}^{x}$

cie $x_{0}=1.$

dla

dla

Rys. l

$x<x\geq 11$ była ciągfa w punk-

24. Napisač równanie tej stycznej do wykresu funkcji $y=\displaystyle \frac{4}{x^{2}}$, która jest nachylona do osi

$Ox$ pod kątem $45^{\mathrm{o}}$

25. Wyznaczyč przedzialy, w których funkcja $f(x)=2\cos^{2}x-x$ jest rosnąca.

26. Wyznaczyč asymptoty krzywej $f(x)=\sqrt{1+x^{2}}-2x.$

27. Przedsiębiorstwo handlowe sprzedaje opony samochodowe. Cafkowity zysk przedsię-

biorstwa liczony w tysiqcach złotych ze sprzedazy $x$ setek tysięcy opon dany jest wzo-

rem $z(x)=-x^{3}+9x^{2}+120x-400$ dla $x\geq 5$. Przyjakiej ilości sprzedanych opon zysk

przedsiębiorstwa będzie największy?

28. Punkt $E$ jest srodkiem boku kwadratu ABCD przedsta-

wionego na rys. 2, a trojk $\mathrm{t}EFG$ jest rownoboczny. Oblicz

pole trojkąta $EFG, \mathrm{j}\mathrm{e}\dot{\mathrm{z}}$ eli dfugosc $\mathrm{k}\mathrm{a}\dot{\mathrm{z}}$ dego boku kwadratu

ABCD jest równa 2.
\begin{center}
\includegraphics[width=32.508mm,height=30.528mm]{./PolitechnikaGdanska_EgzaminWstepny_1997_page1_images/image002.eps}
\end{center}
{\it D} $C$

{\it H}

{\it G F}

{\it A B}

{\it E}

Rys. 2

29. Dany jest romb ABCD o bokach dlugości$\rightarrow  1\mathrm{i}\rightarrow$ kącie o mierze $60^{\mathrm{o}}$ przy wierzcholku $A.$

Obliczyč iloczyn skalarny wektorów AM $\mathrm{i}\vec{AN}$, jeśli $M\mathrm{i}N$ są odpowiednio środkami

boków $BC\mathrm{i}$ {\it CD}.

30. Obliczyč pole powierzchni i objętośč wielościanu, którego wierzchołkami są wszystkie

środki krawędzi czworościanu foremnego o boku dfugości $a.$







POLITECHNIKA $\mathrm{G}\mathrm{D}\mathrm{A}\acute{\mathrm{N}}$ SKA

Gdańsk, 30.06.1998 r.

EGZAMIN WSTĘPNY Z MATEMATYKI

Egzamin składa się z 30 zadań. Zadania $1-10$ oceniane będą w skali $0-2$ punkty, zadania

$11-30$ w skali $0-4$ punkty. Czas trwania egzaminu -- 240 minut.

{\it Powodzenia}.$\displaystyle \int$

l. Rozwiqzač nierównośč $2^{|x+1|}\leq 0,(9).$

2. Obliczyč resztę z dzielenia wielomianu $w(x)=x^{101!}-x+1$ przez dwumian $x+1.$

3. Wyznaczyč dziedzinę funkcji $f(x)=\log_{2}\log_{\frac{1}{2}}x^{2}$

4. Rozwi$\Phi$zač nierównośč $\displaystyle \cos(\pi-x)\leq\sin(\frac{\pi}{2}+x).$

5. Obliczyč największą wartośč funkcji $f(x)=\displaystyle \frac{1}{x^{2}+6x+16}.$

6. Dany jest ciąg $(a_{n})$, gdzie $ a_{n}=\displaystyle \frac{3-n}{n}\cos n\pi$ dla $n\in N$. Zbadač monotonicznośč ciągu

$(b_{n})$, w którym $b_{n}=a_{2n-1}$ dla $\mathrm{k}\mathrm{a}\dot{\mathrm{z}}$ dego $n\in N.$

7. Trzecim wyrazem $\mathrm{c}\mathrm{i}_{\Phi \mathrm{g}}\mathrm{u}$ arytmetycznego jest liczba l. Obliczyč sumę pierwszych pięciu

wyrazów tego ciągu.

8. Wśród rozpoczynających studia $\mathrm{w}\mathrm{y}\dot{\mathrm{z}}$ sze jest tyle samo $\mathrm{m}\mathrm{e}\dot{\mathrm{z}}$ czyzn co kobiet. Co czwarta

kobieta i co drugi $\mathrm{m}\text{ę}\dot{\mathrm{z}}$ czyzna z tych, którzy rozpoczęli studia, nie kończy ich. Obliczyč

jaki procent liczby wszystkich absolwentów $\mathrm{w}\mathrm{y}\dot{\mathrm{z}}$ szych uczelni stanowi liczba absolwen-

tek tychze uczelni.

9. Rys. l przedstawia szkic wykresu funkcji $y=f(x)$ dla $x\in\langle 0;4\rangle.$

Okreslič dziedzinę i naszkicowac wykres funkcji $y=f(-x+3).$
\begin{center}
\includegraphics[width=36.012mm,height=28.452mm]{./PolitechnikaGdanska_EgzaminWstepny_1998_page0_images/image001.eps}
\end{center}
$y$

2

3 4 x

Rys. l

10. Rozwi zac rownanie $x+\displaystyle \frac{x^{2}}{2}+\frac{x^{3}}{4}+\frac{x^{4}}{8}+\ldots=x+13$

ll. Dla jakich wartości parametru $a$ układ równań 

rozwiązanie?

ma co najmniej jedno

12. Przedsiębiorstwo proponuje dziesięcioletni kontrakt swojemu pracownikowi. $\mathrm{W}$ pier-

wszym roku pracy pracownik zarobi 15000 PLN, a w $\mathrm{k}\mathrm{a}\dot{\mathrm{z}}$ dym następnym roku jego

zarobki będą wzrastafy o 8\%. I1e zarobi pracownik w dziesiątym roku pracy? I1e wyniosą

łqczne zarobki pracownika za dziesięč lat pracy w przedsiębiorstwie?

($\mathrm{W}$ obliczeniach $\mathrm{m}\mathrm{o}\dot{\mathrm{z}}$ na przyjąč, $\dot{\mathrm{z}}\mathrm{e}(1,08)^{9}=2.$)

13. Dla jakich wartości parametru $m$ pierwiastki równania $mx^{2}-2mx+1=0$ spełniają

nierównośč $x_{1}^{2}+x_{2}^{2}<3$?




14. Obliczyč pole obszaru opisanego ukfadem nierówności 

15. Punkty $A(2,1) \mathrm{i} B(8,3)$ są wierzchołkami trójkąta $ABC$. Wyznaczyč współrzędne

wierzchofka $C$, jeśli środkowe trójkąta $ABC$ przecinaja się w punkcie $M(4,5).$

16. Obliczyč pole trójkqta wyznaczonego przez punkt $A(3,2)$ i tę średnicę

$x^{2}-2x+y^{2}+4y=20$, która jest równolegfa do prostej $4y-3x=0.$

okręgu

17. Dobrač parametr $a$ tak, aby funkcja $f(x)=\{^{\frac{\sqrt{1+x^{2}}-1}{2^{a}x^{2}}}$

dla

dla

$x\neq 0$

$x=0$

była ciągfa.

18. Obliczyč $\displaystyle \lim_{x\rightarrow 0^{+}}f'(x)$, jeśli $f(x)=\sin(\pi\cos\sqrt{x}).$

19. Rozwiązač równanie $\cos 2x+\cos x+1=0$ dla $x\in\langle 0;2\pi\rangle.$

20. Rozwiązač nierównośč $x\sqrt{3-2x}+1\leq 0.$

21. Wyznaczyč liczby $a\mathrm{i}b$ takie, $\dot{\mathrm{z}}\mathrm{e} \displaystyle \frac{1}{(x-1)x}=\frac{a}{x-1}+\frac{b}{x}$

obliczyč $\displaystyle \lim_{n\rightarrow\infty}(\frac{1}{1\cdot 2}+\frac{1}{2\cdot 3}+\frac{1}{3\cdot 4}+\ldots+\frac{1}{(n-1)n}).$

dla $x\in R-\{0$, 1$\}$. Następnie

22. Rys. 2 przedstawia kratę wymiaru $4\times 4$. Chcemy przejśč po od-

cinkach tej kraty od punktu $A$ do punktu $B\mathrm{m}\mathrm{o}\dot{\mathrm{z}}$ liwie najkrótszą

drogą. Ile jest takich dróg?
\begin{center}
\includegraphics[width=34.752mm,height=33.528mm]{./PolitechnikaGdanska_EgzaminWstepny_1998_page1_images/image001.eps}
\end{center}
{\it B}

{\it A}

Rys.2

23. Zdarzenia losowe $ A\mathrm{i}B\mathrm{s}\Phi$ niezalezne i $P(A\displaystyle \cap B)=\frac{1}{3}$ oraz $P(A\displaystyle \cup B)=\frac{9}{10}$. Obliczyč

$P(A), P(B)\mathrm{i}P(A-B)$, gdy $P(A)>P(B).$

24. Rzucono raz pięcioma kostkami do gry. Jakie jest prawdopodobieństwo tego, $\dot{\mathrm{z}}\mathrm{e}$ na

wszystkich kostkach wypadła taka sama liczba oczek lub na $\mathrm{k}\mathrm{a}\dot{\mathrm{z}}$ dej z nich wypadla

inna liczba oczek?

25. Napisač równanie stycznej do wykresu funkcji $f(x) = x+\sqrt{2-x}$ w jego punkcie

przecięcia z osią $Ox.$

26. Rozwiazač równanie $\displaystyle \log_{3}(3x)+\log_{x}(3x)=\log_{9}(\frac{1}{3}).$

27. Rys. 3 przedstawia szkic wykresu wielomianu stopnia trzeciego.

Wyznaczyč ten wielomian i wyznaczyč współrzędne punktu $P,$

w którym ma on minimum lokalne.
\begin{center}
\includegraphics[width=28.548mm,height=33.120mm]{./PolitechnikaGdanska_EgzaminWstepny_1998_page1_images/image002.eps}
\end{center}
{\it y}

$-3$ 2  {\it x}

$-3$

{\it P}

Rys. 3

28. Dane są punkty $A(-1,3,3), B(0,1,5)\mathrm{i}C(3,5,-1)$. Wyznaczyč taki punkt $D, \dot{\mathrm{z}}$ ewektor

$\overline{A}\mathrm{Z}$ dzieli kąt między wektorami $\overline{A}E\mathrm{i}\overline{A}z$ na polowy i $|\overline{A}\tau|=1.$

29. $\mathrm{W}$ równoramiennym trójkącie $\mathrm{p}\mathrm{r}\mathrm{o}\mathrm{s}\mathrm{t}\mathrm{o}\mathrm{k}_{\Phi^{\mathrm{t}}}\mathrm{n}\mathrm{y}\mathrm{m}$ poprowadzono z wierzchofka $\mathrm{k}_{\Phi^{\mathrm{t}\mathrm{a}}}$ prostego

dwie proste dzielące przeciwprostokątną na trzy odcinki jednakowej dfugości. Obliczyč

cosinus kąta miedzy tymi prostymi.

30. Oblilczyč objętośč kuli stycznej do wszystkich krawędzi czworościanu foremnego o boku

dfugości $a.$







POLITECHNIKA $\mathrm{G}\mathrm{D}\mathrm{A}\acute{\mathrm{N}}$ SKA

Gdańsk, 29.06.1999 r.

EGZAMIN WSTĘPNY Z MATEMATYKI

Egzamin sklada się z 30 zadań. Zadania $1-10$ oceniane będą w skali $0-2$ punkty, zadania

$11-30$ w skali $0-4$ punkty. Czas trwania egzaminu -- 240 minut.

{\it Powodzenia}.$\displaystyle \int$

l. Znalez/č wszystkie $\mathrm{r}\mathrm{o}\mathrm{z}\mathrm{w}\mathrm{i}_{\Phi}$zania równania $81x^{4}-72x^{2}=-16.$

2. Zbiory $A, B\mathrm{i}A\cup B$ mają odpowiednio 1999, 2049 $\mathrm{i}$ 3998 elementów. Ile elementów

mają odpowiednio zbiory $A-B\mathrm{i}A\cap B$?

3. Jeden metr ma l000000 mikronów, a l00000000 angstremów to jeden centymetr. Ile

angstremów ma jeden mikron?

4. Rozwiqzač równanie $\log_{2}(-2)^{5n}=n^{2}+4$, w którym $n$ jest liczbą naturalną.

5. Obliczyč $\left(\begin{array}{l}
n\\
5
\end{array}\right)$, jeśli wiadomo, $\dot{\mathrm{z}}\mathrm{e} \left(\begin{array}{l}
n\\
3
\end{array}\right) =\left(\begin{array}{l}
n\\
4
\end{array}\right)$.

6. Rozwi$\Phi$zač nierównośč $|x-1|\displaystyle \leq\frac{x}{3}+1.$

7. Danajest funkcja $f(x)=(x-1)^{2}$. Na osobnych rysunkach naszkicowač wykresy funkcji:

(a) $y=f(x)$ ; (b) $y=f(-x)$ ; (c) $y=f(x+1)-2.$

8. Rozwi$\Phi$zač nierównośč $x+3\displaystyle \leq\frac{10}{x}.$

9. Dla jakich wartości $x$ istnieje trójkąt o bokach dlugości 1, 2, $\log x$?

10. $\mathrm{W}$ trójkacie naprzeciw boku dlugości $3\sqrt{2}\mathrm{l}\mathrm{e}\dot{\mathrm{z}}\mathrm{y}$ kąt miary $45^{\mathrm{o}}$

okręgu opisanego na tym trójkącie.

Wyznaczyč promień

ll. Mamy dwa naczynia, z których jedno zawiera 101itrów wody, a drugie 101itrów soku.

Polowe wody przelewamy do soku, mieszamy, a następnie pofowę roztworu przelewamy

z powrotem do wody. Obliczyč procentowe stęzenia otrzymanych roztworów.

12. Punkty $A(-1,0), B(3,2) \mathrm{i} C(5,-2)$ są wierzcholkami trójkąta. Pokazač, $\dot{\mathrm{z}}\mathrm{e}$ jest to

trójkąt równoramienny. Napisač równanie osi symetrii tego trójkąta.

13. Doprowadzič do najprostszej postaci wyrazenie $\displaystyle \frac{x+2+\sqrt{x^{2}-4}}{x+2-\sqrt{x^{2}-4}}+\frac{x+2-\sqrt{x^{2}-4}}{x+2+\sqrt{x^{2}-4}}.$

14. $\mathrm{W}$ obszar między trzema wzajemnie stycznymi okręgami o promieniu $R$ wpisano $\mathrm{o}\mathrm{k}\mathrm{r}\Phi \mathrm{g}.$

Znalez/č promień $r$ tego okręgu.

15. Funkcję $f(x)=x^{5}-9x^{3}-27x^{2}+243$ zapisač w postaci iloczynowej i nastepnie rozwiązač

nierównośč $f(x)>0.$



\begin{center}
\includegraphics[width=28.908mm,height=24.132mm]{./PolitechnikaGdanska_EgzaminWstepny_1999_page1_images/image001.eps}
\end{center}
{\it y}

1

$x$

16. Pokazac, $\dot{\mathrm{z}}\mathrm{e}$ funkcja $f(x) = x^{2}$ ma minimum lokalne w punkcie

$x^{2}$ dla $x\neq 0$

$x_{0} = 0$. Uzasadnic, $\dot{\mathrm{z}}\mathrm{e}$ funkcja $g(x) =$ ma maksi-

l dla $x=0$

mum lokalne w punkcie $x_{0}=0$, zob. rys. l. $\mathrm{R}\mathrm{y}\mathrm{s}$. 1

$x^{2}$

17. Napisac rownania tych stycznych do wykresu funkcji $y=$ które są rownolegle

$x-2$'

do prostej $3x+y=0.$

18. Wyznaczyč największą i $\mathrm{n}\mathrm{a}\mathrm{j}\mathrm{m}\mathrm{n}\mathrm{i}\mathrm{e}\mathrm{j}\mathrm{s}\mathrm{z}\Phi$ wartośč funkcji $f(x)=x+\sqrt{1-x^{2}}.$

19. Znalez/č asymptoty wykresu funkcji $y=\displaystyle \frac{4x^{2}+9x}{x-4}.$

20. Rozwiązač równanie $3^{2x}-2\cdot 3^{x}+a=0$, w którym $a=\displaystyle \lim_{n\rightarrow\infty}\frac{\sqrt{n^{2}+3}-4n}{n-1}.$

21. $\mathrm{W}$ prostokątnym układzie współrzędnych zaznaczyč zbiór punktów $(x,y)$, których

współrzędne spelniaja równanie $\log_{2}(x+y)=\log_{2}x+\log_{2}y.$

22. Obliczyč średnią arytmetyczną tych spośród liczb naturalnych l, 2, 3, $\ldots$, 2000, które

nie są podzielne przez 5.

23. Wyznaczyč ciąg geometryczny $a_{1}, a_{2}, \ldots, a_{n}, \ldots, \mathrm{j}\mathrm{e}\dot{\mathrm{z}}$ eli wiadomo, $\dot{\mathrm{z}}\mathrm{e}a_{1}+a_{2}+a_{3}+a_{4}=30$

$\mathrm{i}a_{2}+a_{3}+a_{4}+a_{5}=60$. Znalez/č taką liczbę $n, \dot{\mathrm{z}}\mathrm{e}a_{n}<500000<a_{n+1}.$

24. Rozwiązač równanie 2 $\sin^{2}x+\sin 2x=2.$

25. Rozwi$\Phi$zač nierównośč $\displaystyle \sin^{2}x>\frac{3}{4}$ dla $x\in\langle 0;2\pi\rangle.$

26. Znalez/č równania prostych przechodzqcych przez punkt $A(7,3)$ i przecinajqcych prostą

$x-3y-1=0$ pod $\mathrm{k}_{\Phi}\mathrm{t}\mathrm{e}\mathrm{m}45^{\mathrm{o}}$

27. Obliczyč dfugośč najkrótszej drogi poprowadzonej po powierzchni sześcianu o krawę-

dziach długości l $\mathrm{i}1_{\text{ą}^{\mathrm{C}\mathrm{Z}}\mathrm{a}}\mathrm{c}\mathrm{e}\mathrm{j}$ dwa przeciwległe wierzchołki tego sześcianu. Ile najkrót-

szych dróg fączy dwa wybrane przeciwlegfe wierzcholki tego sześcianu?

28. Obliczyč iloczyn skalarny wektorów ã $= [-1,1+x] \mathrm{i} \vec{b}= [\sqrt{x+3}$, 1$]$. Dla jakich $x$

wektory ã $\mathrm{i}\vec{b}$ są prostopadle? Jaki kąt (ostry, prosty, czy rozwarty) tworzą te wektory

dla $x=-2$?

29. Rzucono pięč razy dwiema kostkami do gry. Obliczyč prawdopodobieństwo tego, $\dot{\mathrm{z}}\mathrm{e}$ co

najmniej dwa razy suma oczek na obu kostkach jest nie mniejsza od 10.

30. Sześcian o krawędzi długości $a$ podzielono płaszczyzną przecho-

dzacą przez przekątną jednej z jego ścian i przez środki dwóch

krawędzi $\mathrm{l}\mathrm{e}\dot{\mathrm{z}}$ qcych na przeciwległej ścianie na dwie bryły, zob.

rys. 2. Ob1iczyč objętości obu otrzymanych bryf.





\end{document}