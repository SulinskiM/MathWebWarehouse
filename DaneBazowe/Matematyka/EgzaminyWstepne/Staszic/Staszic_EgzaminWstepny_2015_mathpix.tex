\documentclass[10pt]{article}
\usepackage[polish]{babel}
\usepackage[utf8]{inputenc}
\usepackage[T1]{fontenc}
\usepackage{amsmath}
\usepackage{amsfonts}
\usepackage{amssymb}
\usepackage[version=4]{mhchem}
\usepackage{stmaryrd}

\title{Sprawdzian predyspozycji do klas matematycznych }

\author{}
\date{}


\newcommand\Varangle{\mathop{{<\!\!\!\!\!\text{\small)}}\:}\nolimits}

\begin{document}
\maketitle
XIV LO im. S. Staszica w Warszawie\\
(25 maja 2015 r.)\\
Uwagi

\begin{itemize}
  \item Poniższe zadania można rozwiązywać w dowolnej kolejności.
  \item Wszystkie zadania są jednakowo punktowane.
  \item Podanie jedynie prawidłowej odpowiedzi liczbowej nie stanowi rozwiązania zadania. Ocenie podlegał będzie tok rozumowania oraz rachunki prowadzace do uzyskanego wyniku.
\end{itemize}

\begin{enumerate}
  \item Wykaż, że jeżeli liczba \(a\) jest większa od 1 , to
\end{enumerate}

\[
a+2 a^{2}+3 a^{3}<6 a^{6} .
\]

\begin{enumerate}
  \setcounter{enumi}{1}
  \item Dany jest trójkąt \(A B C\), w którym \(\Varangle B A C=90^{\circ}\). Punkty \(D, E, F\) leżą odpowiednio na bokach \(B C, C A, A B\), przy czym \(\Varangle E D F=90^{\circ}\). Wykaż, że długość odcinka \(E F\) jest nie mniejsza od długości wysokości trójkąta \(A B C\) poprowadzonej z wierzchołka \(A\).
  \item Ile liczb naturalnych z przedziału \(\langle 1,1000\rangle\) można przedstawić w postaci sumy sześciu kolejnych liczb całkowitych (niekoniecznie dodatnich)? Odpowiedź uzasadnij.
  \item Na przyjęciu spotkało się 19 osób. Okazało się, że każda z tych osób ma wśród pozostałych co najmniej 10 znajomych. Udowodnij, że na przyjęciu można wskazać taką trójkę osób, wśród których każde dwie się znają.\\
Uwaga: Przyjmujemy, że jeśli osoba \(A\) zna \(B\), to osoba \(B\) zna \(A\).
  \item Wszystkie kąty wewnętrzne pięciokąta \(A B C D E\) są równe. Symetralne odcinków \(A B\) i \(C D\) przecinają się w punkcie \(S\). Wykaż, że proste \(E S\) i \(B C\) są prostopadłe.
  \item Dane są dwa czworościany foremne \(A B C E\) oraz \(B C D F\) o wspólnej krawędzi \(B C\), której długość wynosi 1. Punkty \(A, B, C, D\) leżą w jednej płaszczyźnie, a punkty \(E\) i \(F\) są różne i leżą po tej samej stronie płaszczyzny \(A B C D\). Oblicz objętość czworościanu \(B C E F\).
\end{enumerate}

\end{document}