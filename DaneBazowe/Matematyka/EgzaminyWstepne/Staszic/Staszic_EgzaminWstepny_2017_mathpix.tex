\documentclass[10pt]{article}
\usepackage[polish]{babel}
\usepackage[utf8]{inputenc}
\usepackage[T1]{fontenc}
\usepackage{amsmath}
\usepackage{amsfonts}
\usepackage{amssymb}
\usepackage[version=4]{mhchem}
\usepackage{stmaryrd}

\title{Sprawdzian predyspozycji do klas matematycznych }

\author{XIV LO im. S. Staszica w Warszawie}
\date{}


\begin{document}
\maketitle
(29 maja 2017 r.)

\section*{Uwagi}
\begin{itemize}
  \item Poniższe zadania można rozwiązywać w dowolnej kolejności.
  \item Wszystkie zadania są jednakowo punktowane.
  \item Podanie jedynie prawidłowej odpowiedzi liczbowej nie stanowi rozwiązania zadania. Ocenie podlegał będzie tok rozumowania oraz obliczenia prowadzące do uzyskanego wyniku.
\end{itemize}

\begin{enumerate}
  \item Wyznacz wszystkie pary \((a, b)\) nieujemnych liczb rzeczywistych, dla których
\end{enumerate}

\[
\sqrt[4]{a \cdot b}=\sqrt{a}+\sqrt{b}
\]

\begin{enumerate}
  \setcounter{enumi}{1}
  \item Wykaż, że jeżeli liczby \(a, b\) są całkowite, to liczba
\end{enumerate}

\[
(a+b)^{4}-(a-b)^{4}
\]

jest podzielna przez 16.\\
3. Dany jest kwadrat \(A B C D\). Okrąg o leży na zewnątrz kwadratu i jest styczny do odcinka \(A B\) oraz do prostych \(B D\) i \(A D\). Wykaż, że średnica okręgu \(o\) jest równa długości przekątnej kwadratu \(A B C D\).\\
4. Dany jest ( \(2 n\) )-kąt foremny, gdzie \(n \geqslant 2\) jest pewną ustaloną liczbą naturalną. Ile jest trójkątów prostokątnych nierównoramiennych, których wierzchołki są wierzchołkami tego ( \(2 n\) )-kąta? Odpowiedź uzasadnij.\\
5. Dany jest sześcian o krawędzi 1 . Sfera \(s\) leży wewnątrz tego sześcianu, jest styczna do trzech jego ścian o wspólnym wierzchołku i przechodzi przez środek sześcianu. Oblicz promień sfery \(s\).


\end{document}