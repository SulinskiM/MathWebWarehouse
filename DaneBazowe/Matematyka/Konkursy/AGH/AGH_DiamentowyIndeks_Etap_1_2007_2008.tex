\documentclass[a4paper,12pt]{article}
\usepackage{latexsym}
\usepackage{amsmath}
\usepackage{amssymb}
\usepackage{graphicx}
\usepackage{wrapfig}
\pagestyle{plain}
\usepackage{fancybox}
\usepackage{bm}

\begin{document}

AKADEMIA GÓRNICZO-HUTNICZA im. Stanisfawa Staszica

w Krakowie

OLIMPIADA,, O DIAMENTOWY INDEKS AGH'' 2007/8

MATEMATYKA- ETAP I

ZADANIA PO 10 PUNKTÓW

1. $\mathrm{W}$ trójkącie równoramiennym dane są dlugości podstawy $a$ i ramienia

$b$. Oblicz dlugośč wysokości tego trójkąta opuszczonej na jego ramię.

2. Rozwia $\dot{\mathrm{Z}}$ nierównośč

$|2x^{4}-17|<15.$

3. Oblicz granicę ciągu, którego n-ty wyraz jest równy

$a_{n}=n^{3}-\sqrt{n^{6}-5n^{3}}.$

4. Na ile sposobów $\mathrm{m}\mathrm{o}\dot{\mathrm{z}}$ na rozmieścič $k$ kul $(k \geq 4, \mathrm{k}\mathrm{a}\dot{\mathrm{z}}$ da kula innego

koloru) $\mathrm{w}k$ ponumerowanych pudetkach tak, aby

a) $\dot{\mathrm{z}}$ adne pudelko nie byIo puste?

b) dokladnie jedno pudelko bylo puste?

c) doktadnie $k-2$ pudeIka byly puste?

ZADANIA PO 20 PUNKTÓW

5. DIugośč wysokości ostroslupa prawidtowego trójkatnego jest równa dlu-

gości krawędzi podstawy. Oblicz stosunek objętości kuli wpisanej w ten

ostroslup do objętości kuli opisanej na nim.

6. Wyznacz liczbę rozwiązań równania

$(m-3)x^{4}-3(m-3)x^{2}+m+2=0$

w zalezności od parametru $m.$

7. RozIóz na czynniki wielomian

$W(x)=x^{4}+6x^{3}+11x^{2}+6x.$

Udowodnij, $\dot{\mathrm{z}}\mathrm{e}$ wartośč $W(n)$ tego wielomianu dla dowolnej liczby natu-

ralnej $n$ jest podzielna przez 12. D1ajakich natura1nych $n$ liczba $W(n)$

nie jest podzielna przez 60?


\end{document}