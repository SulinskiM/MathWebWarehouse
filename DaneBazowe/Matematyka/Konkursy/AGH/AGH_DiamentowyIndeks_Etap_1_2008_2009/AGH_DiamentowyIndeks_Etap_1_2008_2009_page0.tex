\documentclass[a4paper,12pt]{article}
\usepackage{latexsym}
\usepackage{amsmath}
\usepackage{amssymb}
\usepackage{graphicx}
\usepackage{wrapfig}
\pagestyle{plain}
\usepackage{fancybox}
\usepackage{bm}

\begin{document}

AKADEMIA GÓRNICZO-HUTNICZA im. Stanisfawa Staszica

w Krakowie

OLIMPIADA,, O DIAMENTOWY INDEKS AGH'' 2008/9

MATEMATYKA- ETAP I

ZADANIA PO 10 PUNKTÓW

l. Ile jest czwórek $(x,y,z,t)$ liczb calkowitych dodatnich spelniajacych

równanie $xy+yz+zt+tx=2008$?

2. Rozwia $\dot{\mathrm{Z}}$ równanie $\sin 4x+\sqrt{3}\sin 2x=0.$

3. Znajd $\acute{\mathrm{z}}$ liczbę $c$, dla której granica ciagu o wyrazie ogólnym

{\it an}$=$--$\sqrt{}$35{\it nn}$++${\it c}-9{\it n}2-{\it n}2{\it c}

jest równa 2.

4. Ile jest czterocyfrowych liczb naturalnych, które nie sa podzielne ani

przez 9, ani przez 12?

ZADANIA PO 20 PUNKTÓW

5. Punkty $A= (2,-2) \mathrm{i}B= (8,4)$ sa końcami podstawy trójkata rów-

noramiennego $ABC$. WierzchoIek $C\mathrm{l}\mathrm{e}\dot{\mathrm{z}}\mathrm{y}$ na prostej $x-3y+34=0.$

Znajd $\acute{\mathrm{z}}$ równanie okręgu wpisanego w trójkqt $ABC.$

6. Dla jakich wartości parametru $p\in R$ jeden z pierwiastków równania

$(12p+6)x^{2}+16px+9p=0$

jest sinusem, a drugi kosinusem tego samego kata rozwartego?

7. Cztery kule, z których trzy mają promień $r$, a czwarta $R$, ulozono na

stole w piramidę w taki sposób, $\dot{\mathrm{z}}\mathrm{e}\mathrm{k}\mathrm{a}\dot{\mathrm{z}}$ da kula jest styczna do trzech

pozostaIych, przy czym kule przystajqce tworza podstawę piramidy.

Oblicz największa odleglośč punktu kuli o promieniu $R$ od stotu. Podaj

warunek, jaki musza spelniač promienie, aby ustawienie piramidy bylo

$\mathrm{m}\mathrm{o}\dot{\mathrm{z}}$ liwe.
\end{document}
