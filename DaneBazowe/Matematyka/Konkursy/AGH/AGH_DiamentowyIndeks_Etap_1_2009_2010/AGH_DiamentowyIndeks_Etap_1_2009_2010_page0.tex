\documentclass[a4paper,12pt]{article}
\usepackage{latexsym}
\usepackage{amsmath}
\usepackage{amssymb}
\usepackage{graphicx}
\usepackage{wrapfig}
\pagestyle{plain}
\usepackage{fancybox}
\usepackage{bm}

\begin{document}

AKADEMIA GÓRNICZO-HUTNICZA

im. StanisIawa Staszica w Krakowie

OLIMPIADA,, O DIAMENTOWY INDEKS AGH'' 2009/10

MATEMATYKA- ETAP I

ZADANIA PO 10 PUNKTÓW

l. Na pólsferze o promieniu $R\mathrm{l}\mathrm{e}\dot{\mathrm{z}}$ a dwa styczne do siebie okręgi o promie-

niu $r$. Wyznacz największa odlegtośč między dwoma punktami naleza-

cymi do tych okręgów.

2. Rozwia $\dot{\mathrm{Z}}$ nierównośč

$\sqrt{x^{2}+2x+1}-2x>0.$

3. Kran $A$ napeInia basen woda w ciagu 10 godzin, a kran $B$ w ciagu 15

godzin. $\mathrm{W}$ ciagu ilu godzin napeIniony zostanie basen, $\mathrm{j}\mathrm{e}\dot{\mathrm{z}}$ eli oba krany

bedą dzialač jednocześnie?

4. Znajd $\acute{\mathrm{z}}$ wszystkie rozwiqzania równania

4 $\cos 2x\sin 2x+1=0$

nalezące do przedziaIu $(-\pi;\pi).$

ZADANIA PO 20 PUNKTÓW

5. Wyznacz zbiory $A\cap B$ oraz $A\backslash B$, gdzie

$A=\{x\in lR:x^{4}+x^{3}-3x^{2}-x+2\geq 0\},$

$B=\{x\in lR:\log_{0,5}(x+3)\geq\log_{0,5}(6-2x)\}.$

6. Oblicz pole trójkqta, mając dane dwie proste $4x+5y+17 = 0$

$\mathrm{i} x-3y=0$, zawierajace środkowe trójkqta, oraz jeden jego wierz-

cholek $A=(-1,-6).$

7. Ile jest równań postaci

$x^{2}-px+q=0,$

które maja dwa pierwiastki mniejsze od 7, przy czym 1iczby $p\mathrm{i}q$ są

calkowite i dodatnie.
\end{document}
