\documentclass[a4paper,12pt]{article}
\usepackage{latexsym}
\usepackage{amsmath}
\usepackage{amssymb}
\usepackage{graphicx}
\usepackage{wrapfig}
\pagestyle{plain}
\usepackage{fancybox}
\usepackage{bm}

\begin{document}

AKADEMIA GÓRNICZO-HUTNICZA

im. StanisIawa Staszica w Krakowie

OLIMPIADA,, O DIAMENTOWY INDEKS AGH'' 2010/11

MATEMATYKA- ETAP I

ZADANIA PO 10 PUNKTÓW

l. Kula $K$ jest wpisana w sześcian. Kula $K'$ jest styczna do trzech ścian

tego sześcianu i do kuli $K$. Oblicz stosunek promienia kuli $K$ do

promienia kuli $K'.$

2. Suma kwadratów trzech dodatnich liczb caIkowitych $a, b, c$ jest równa

2010. Ile jest wśród nich liczb parzystych?

3. Znajd $\acute{\mathrm{z}}$ liczbę $p$, dla której granica ciagu o wyrazie ogólnym

$a_{n}=\sqrt[3]{n^{3}+n^{2}+9pn}-\sqrt[3]{n^{3}-5pn^{2}}$

jest równa 2.

4. Punkty $A= (-2,3) \mathrm{i}B= (1,2)$ sa wierzcholkami trójkata $T$. Wyz-

nacz wspótrzędne trzeciego wierzchoIka wiedzac, $\dot{\mathrm{z}}\mathrm{e}$ pole trójkata $T$ jest

równe 3, a środek jego cięzkości $\mathrm{l}\mathrm{e}\dot{\mathrm{z}}\mathrm{y}$ na osi $OY.$

ZADANIA PO 20 PUNKTÓW

5. Liczba naturalna $a$ ma $2n$ cyfr, z których pierwsze $n$ cyfr to same

czwórki, a pozostale cyfry to ósemki. Udowodnij, $\dot{\mathrm{z}}\mathrm{e}\sqrt{a+1}$ jest liczba

naturalnq dla $\mathrm{k}\mathrm{a}\dot{\mathrm{z}}$ dego $n.$

6. $\mathrm{W}$ ukladzie wspólrzędnych na plaszczy $\acute{\mathrm{z}}\mathrm{n}\mathrm{i}\mathrm{e}$ narysuj zbiór

$A=\{(x,y):\log_{y}(8x+y-2-x^{2})\geq\log_{y}(8-x^{2}+8x-2y-y^{2})\}.$

7. Naszkicuj wykres funkcji $g:m\rightarrow g(m)$, która $\mathrm{k}\mathrm{a}\dot{\mathrm{z}}$ dej liczbie rzeczywis-

tej $m$ przyporzqdkowuje liczbę pierwiastków równania

$2^{2x+2}+4^{x}+4^{x-1}+\ldots=m+16^{x}$


\end{document}