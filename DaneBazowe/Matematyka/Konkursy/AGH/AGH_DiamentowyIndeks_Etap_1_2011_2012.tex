\documentclass[a4paper,12pt]{article}
\usepackage{latexsym}
\usepackage{amsmath}
\usepackage{amssymb}
\usepackage{graphicx}
\usepackage{wrapfig}
\pagestyle{plain}
\usepackage{fancybox}
\usepackage{bm}

\begin{document}

AKADEMIA GÓRNICZO-HUTNICZA

im. StanisIawa Staszica w Krakowie

OLIMPIADA,, O DIAMENTOWY INDEKS AGH'' 2011/12

MATEMATYKA- ETAP I

ZADANIA PO 10 PUNKTÓW

l. Pary $(x,y)$ liczb calkowitych spelniające równanie

$xy^{2}-y^{3}-xy+x^{2}+5=0$

sq wspótrzędnymi wierzchoIków pewnego wielokata. Oblicz jego pole.

2. Oblicz sume $n$ poczatkowych wyrazów ciagu $(a_{n})$, w którym

$a_{1}=3, a_{2}=33, a_{3}=333, a_{4}=3333,$

3. $\mathrm{W}$ pólokrag o promieniu $R$ wpisano trapez, w którym ramie jest nachylone pod

katem $\alpha$ do podstawy będqcej średnica okręgu. Oblicz pole trapezu.

4. Rozwiąz równanie

$\displaystyle \lim_{n\rightarrow+\infty}(x+x^{3}+x^{5}+\ldots+x^{2n-1})=\frac{2}{3}.$

ZADANIA PO 20 PUNKTÓW

5. Wyznacz równanie krzywej będqcej zbiorem środków wszystkich cięciw paraboli

$y = x^{2}-2$ przechodzacych przez początek ukladu wspólrzędnych. Naszkicuj tę

krzywa.

6. Narysuj w ukIadzie wspóIrzędnych zbiór

$S=\{(x,y)$

$\log_{x}|y-2|>\log_{|y-2|}x\}.$

7. a) Zbadaj w zalezności od parametru $k$, ile rozwiązań ma uklad równań

$\left\{\begin{array}{l}
kx+(k+1)y=k-1\\
4x+(k+4)y=k.
\end{array}\right.$

b) Dla jakich wartości parametru $k$ ten uktad ma dokladnie jedno rozwiazanie

nalezace do wnętrza trójkata o wierzchoikach

$A=(0,0),$

$B=(\displaystyle \frac{2}{3},0), C=(0,2).$


\end{document}