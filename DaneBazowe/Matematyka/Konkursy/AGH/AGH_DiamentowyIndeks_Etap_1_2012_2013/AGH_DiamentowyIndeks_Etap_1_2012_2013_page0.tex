\documentclass[a4paper,12pt]{article}
\usepackage{latexsym}
\usepackage{amsmath}
\usepackage{amssymb}
\usepackage{graphicx}
\usepackage{wrapfig}
\pagestyle{plain}
\usepackage{fancybox}
\usepackage{bm}

\begin{document}

AKADEMIA GÓRNICZO-HUTNICZA

im. StanisIawa Staszica w Krakowie

OLIMPIADA,, O DIAMENTOWY INDEKS AGH'' 2012/13

MATEMATYKA- ETAP I

ZADANIA PO 10 PUNKTÓW

l. Ile jest ciągów $(x_{1},x_{2},x_{3},x_{4})$ liczb calkowitych dodatnich spetniających

równanie $x_{1}+x_{2}+x_{3}+x_{4}=12$ ?

2. Dana jest funkcja

$f(x)=\displaystyle \frac{5-x}{2x+1}.$

Rozwia $\dot{\mathrm{z}}$ nierównośč $f(x+5)\geq f(x-3).$

3. Wyznacz dziedzinę i zbadaj parzystośč funkcji

$f(x)=(x^{2}+1)\displaystyle \frac{3^{2x}+3^{-2x}}{\sin^{2}2x+2}-x^{3}\log\frac{3x^{2}+5x+8}{3x^{2}-5x+8}.$

4. Znajd $\acute{\mathrm{z}}$ rzut równolegly punktu $A(1,-2)$ na prostą $x-y+3 = 0$

w kierunku wektora $\vec{v}=[1$, 2$].$

ZADANIA PO 20 PUNKTÓW

5. $\mathrm{W}$ prawidlowym ostrosIupie trójkqtnym miary katów nachylenia ściany

bocznej i krawędzi bocznej do podstawy ostrostupa wynoszą odpowied-

nio $\alpha \mathrm{i}\beta$. Oblicz stosunek objętości ostroslupa do objętości kuli wpisanej

w niego.

6. Naszkicuj wykres funkcji, która $\mathrm{k}\mathrm{a}\dot{\mathrm{z}}$ dej liczbie rzeczywistej $m$ przypo-

rządkowuje liczbQ $f(m)$ pierwiastków równania

$4^{|x|}+(m+1)2^{|x|+1}=5-m^{2}$

z niewiadoma $x.$

7. Ciag trzech liczb calkowitych $(a,b,c)$ jest ciagiem geometrycznym, któ-

rego iloraz jest liczbą calkowita. $\mathrm{J}\mathrm{e}\dot{\mathrm{z}}$ eli do najmniejszej z nich dodamy 9,

to otrzymamy trzy liczby, które odpowiednio uporządkowane utworzą

ciag arytmetyczny. Znajd $\acute{\mathrm{z}}$ wszystkie takie ciqgi $(a,b,c).$
\end{document}
