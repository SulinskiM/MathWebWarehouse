\documentclass[a4paper,12pt]{article}
\usepackage{latexsym}
\usepackage{amsmath}
\usepackage{amssymb}
\usepackage{graphicx}
\usepackage{wrapfig}
\pagestyle{plain}
\usepackage{fancybox}
\usepackage{bm}

\begin{document}

AKADEMIA GÓRNICZO-HUTNICZA

im. StanisIawa Staszica w Krakowie

OLIMPIADA,, O DIAMENTOWY INDEKS AGH'' 2013/14

MATEMATYKA- ETAP I

ZADANIA PO 10 PUNKTÓW

l. Udowodnij, $\dot{\mathrm{z}}\mathrm{e}\dot{\mathrm{z}}$ aden element zbioru $S=\{6n+2:n\in \mathbb{N}\}$ nie jest kwadratem liczby

caIkowitej.

2. Rozwia $\dot{\mathrm{Z}}$ równanie

$ 5+\displaystyle \frac{x^{2}}{5}-\frac{x^{4}}{25}+\frac{x^{6}}{125}-\frac{x^{8}}{625}+\ldots = x^{2}+1$, (4),

w którym drugi skladnik prawej strony jest ulamkiem dziesiętnym okresowym.

3. Na ile sposobów $\mathrm{m}\mathrm{o}\dot{\mathrm{z}}$ na $n$ kul rozmieścič w $n$ pudelkach tak, $\dot{\mathrm{z}}$ eby dokladnie dwa

pudelka zostaly puste? Zalóz, $\dot{\mathrm{z}}\mathrm{e}n\geq 3$ oraz zarówno kule jak i pudelka sa między

sobą rozróznialne.

4. Sporzad $\acute{\mathrm{z}}$ wykres funkcji danej wzorem

$f(x)=5^{|\log_{0,2}x|}.$

ZADANIA PO 20 PUNKTÓW

5. Dany jest prawidlowy ostroslup czworokatny. Pole przekroju pIaszczyzna przecho-

dząc4 przez przekatna podstawy i równo1eg1q do krawędzi bocznej skośnej wzg1ędem

tej przekatnej jest równe $P$. Pole przekroju plaszczyzna przechodzaca przez środki

dwóch sąsiednich boków podstawy i środek wysokości ostroslupa wynosi $S$. Oblicz

iloraz $\displaystyle \frac{P}{S}.$

6. Dla jakich $ x\in (-\displaystyle \frac{\pi}{2};\frac{\pi}{2})$ liczby

tgx, l,

$\displaystyle \frac{\cos x}{1+\sin x}$

w podanej kolejności sa trzema poczatkowymi wyrazami rosnącego ciagu arytme-

tycznego $(a_{n})$ ? Dla dowolnego $n\in \mathbb{N}$ oblicz sumę $a_{n}+a_{n+1}+\ldots+a_{2n}.$

7. Rozwiąz w zalezności od parametru $p\in \mathbb{R}$ równanie

$(1-p)(|x+2|+|x|)=4-3p.$
\end{document}
