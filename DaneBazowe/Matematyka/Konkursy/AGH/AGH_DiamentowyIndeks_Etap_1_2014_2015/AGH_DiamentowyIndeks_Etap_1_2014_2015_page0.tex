\documentclass[a4paper,12pt]{article}
\usepackage{latexsym}
\usepackage{amsmath}
\usepackage{amssymb}
\usepackage{graphicx}
\usepackage{wrapfig}
\pagestyle{plain}
\usepackage{fancybox}
\usepackage{bm}

\begin{document}

AKADEMIA GÓRNICZO-HUTNICZA

im. StanisIawa Staszica w Krakowie

OLIMPIADA,, O DIAMENTOWY INDEKS AGH'' 2014/15

MATEMATYKA- ETAP I

ZADANIA PO 10 PUNKTÓW

l. Niech p będzie dowolna liczba pierwsza.

przez 30 nie jest 1iczba ztozona.

Udowodnij, $\dot{\mathrm{z}}\mathrm{e}$ reszta z dzielenia liczby $p$

2. Rozwia $\dot{\mathrm{Z}}$ równanie

$(\sqrt{5+2\sqrt{6}})^{x}+(\sqrt{5-2\sqrt{6}})^{x}=10.$

3. Oblicz granicę ciągu o wyrazie ogólnym

$a_{n}=\displaystyle \frac{3^{n+1}+2^{3+2n}}{2^{2n+1}+3^{n}}.$

4. Na ile sposobów $\mathrm{m}\mathrm{o}\dot{\mathrm{z}}$ na zbiór $\{$1, 2, $\ldots, n\}$, gdzie $n\geq 3$, podzielič na trzy niepuste

podzbiory?

ZADANIA PO 20 PUNKTÓW

5. Dla jakich wartości parametru $m$ nierównośč

$(m^{2}-1)\cdot 25^{x}-2(m-1)\cdot 5^{x}+2>0$

jest spelniona przez $\mathrm{k}\mathrm{a}\dot{\mathrm{z}}$ da liczbę rzeczywista $x$?

6. $\mathrm{W}$ sześcianie o krawędzi dlugości $a$ zawarte sa dwie sfery zewnętrznie styczne, przy

czym ich środki $\mathrm{l}\mathrm{e}\dot{\mathrm{z}}$ ą na $\mathrm{P}^{\mathrm{r}\mathrm{z}\mathrm{e}\mathrm{k}}4^{\mathrm{t}\mathrm{n}\mathrm{e}\mathrm{j}}$ sześcianu i $\mathrm{k}\mathrm{a}\dot{\mathrm{z}}$ da z nich jest styczna przynajmniej

do trzech ścian sześcianu. Oblicz promienie tych sfer, dla których suma ich pól

powierzchni jest a) największa, b) najmniejsza.

7. Wyznacz równania stycznych do okręgu

$x^{2}+y^{2}+4x+3=0$

poprowadzonych z punktu $M = (1,0)$. Jaka krzywa stanowi zbiór wszystkich

środków cięciw tego okregu wyznaczonych przez proste przechodzące przez punkt

$M$? Napisz jej równanie.
\end{document}
