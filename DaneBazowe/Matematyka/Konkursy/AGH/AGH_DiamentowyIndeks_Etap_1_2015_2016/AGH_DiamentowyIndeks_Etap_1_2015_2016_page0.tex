\documentclass[a4paper,12pt]{article}
\usepackage{latexsym}
\usepackage{amsmath}
\usepackage{amssymb}
\usepackage{graphicx}
\usepackage{wrapfig}
\pagestyle{plain}
\usepackage{fancybox}
\usepackage{bm}

\begin{document}

AKADEMIA GÓRNICZO-HUTNICZA

im. StanisIawa Staszica w Krakowie

OLIMPIADA,, O DIAMENTOWY INDEKS AGH'' 2015/16

MATEMATYKA - ETAP I

ZADANIA PO 10 PUNKTÓW

1. Znajd $\acute{\mathrm{z}}$ wszystkie rosnace ciągi $(a_{n})$ o wyrazach calkowitych takie, $\dot{\mathrm{z}}\mathrm{e}$

$\alpha_{2}=2$ oraz $a_{mn}=a_{m}a_{n}$ dla wszystkich liczb naturalnych $m, n.$

2. Na ile sposobów $\mathrm{m}\mathrm{o}\dot{\mathrm{z}}$ na grupę $3k$ osób posadzič przy dwóch okrąglych

stoIach, $\mathrm{j}\mathrm{e}\dot{\mathrm{z}}$ eli przyjednym stolejest $2k$ ponumerowanych krzeseI, a przy

drugim $k$? A na ile sposobów $\mathrm{m}\mathrm{o}\dot{\mathrm{z}}$ na to zrobič tak, by ustalone dwie

osoby siedzialy obok siebie, $\mathrm{j}\mathrm{e}\dot{\mathrm{z}}$ eli $k\geq 2?.$

3. Rozwiqz nierównośč $3^{x}-2^{x}>3^{x-2}$

4. Oblicz granicę ciągu

$\displaystyle \lim_{n\rightarrow\infty}(2n-\sqrt[3]{8n^{3}-2n^{2}}).$

ZADANIA PO 20 PUNKTÓW

5. Dla jakich wartości parametru p równanie

$\cos^{3}x+p\cos x+p+1=0$

ma $\mathrm{d}\mathrm{o}\mathrm{k}\ddagger$adnie trzy rozwiązania w przedziale $\langle 0;2\pi\rangle$ ?

6. Napisz równanie okręgu opisanego na trójkacie o wierzchotkach $A =$

$(5,-4), B = (6,-1), C= (-2,3)$. Zbadaj wzajemne potozenie tego

okręgu oraz jego obrazu w symetrii osiowej względem prostej

$3x+4y+26=0.$

7. Krawędz$\acute{}$ podstawy ostroslupa prawidlowego trójkqtnego ma dlugośč

$a= 12$ cm, a sinus kqta między ścianami bocznymi wynosi $\displaystyle \frac{2\sqrt{2}}{3}$. Os-

trostup przecięto plaszczyzna przechodzacą przez jeden z wierzcholków

podstawy i dzielaca przeciwlegla ścianę boczna na dwie figury o równych

polach. Oblicz objętości bryl, które powstaly w wyniku przecięcia os-

troslupa ta pIaszczyzna.
\end{document}
