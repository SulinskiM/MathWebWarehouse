\documentclass[a4paper,12pt]{article}
\usepackage{latexsym}
\usepackage{amsmath}
\usepackage{amssymb}
\usepackage{graphicx}
\usepackage{wrapfig}
\pagestyle{plain}
\usepackage{fancybox}
\usepackage{bm}

\begin{document}

AKADEMIA GÓRNICZO-HUTNICZA

im. Stanislawa Staszica w Krakowie

OLIMPIADA O DIAMENTOWY INDEKS AGH'' 2016/17

MATEMATYKA- ETAP I

ZADANIA PO 10 PUNKTÓW

l. Udowodnij, $\dot{\mathrm{z}}\mathrm{e}$ jedyną liczbą $\mathrm{p}\mathrm{i}\mathrm{e}\mathrm{r}\mathrm{w}\mathrm{s}\mathrm{z}\Phi p$, taką $\dot{\mathrm{z}}\mathrm{e}$ liczba $p^{2}+2\mathrm{t}\mathrm{e}\dot{\mathrm{z}}$ jest

pierwsza, jest $p=3.$

2. Dane są punkty $A = (-1,-2), B = (3,1), C = (1,4)$. Prosta $l$ jest

równoległa do prostej $AC$ i dzieli trójkąt $ABC$ na dwie figury o równych

polach. Znajd $\acute{\mathrm{z}}$ równanie prostej $l.$

3. $\mathrm{R}\mathrm{o}\mathrm{z}\mathrm{w}\mathrm{i}_{\Phi}\dot{\mathrm{z}}$ równanie

4 $\sin^{2}x+8\sin^{2}x\cos x=2\cos x+1.$

4. Niech $a\mathrm{i}b$ będą dowolnymi liczbami rzeczywistymi. Wykaz$\cdot, \dot{\mathrm{z}}\mathrm{e}\mathrm{j}\mathrm{e}\dot{\mathrm{z}}$ eli

$\alpha<b$, to

$a^{3}-b^{3}<a^{2}b-ab^{2}$

ZADANIA PO 20 PUNKTÓW

5. Wykaz$\cdot, \dot{\mathrm{z}}\mathrm{e}\mathrm{j}\mathrm{e}\dot{\mathrm{z}}$ eli liczba $m$ spelnia nierównośč

$(1+\displaystyle \frac{1}{2m})\log_{0,5}3-\log_{0,5}(27+3^{\frac{1}{m}})\leq 2,$

to $x^{2}+mx+1>0$ dla $\mathrm{k}\mathrm{a}\dot{\mathrm{z}}$ dej liczby rzeczywistej $x.$

6. Nieskończony ciąg $(a_{n})$ dany jest wzorem $a_{n}=1+2+\ldots+n.$

a) Znajd $\acute{\mathrm{z}}$ wszystkie cyfry jedności wyrazów tego ciągu w zapisie dzie-

siętnym. Udowodnij, $\dot{\mathrm{z}}\mathrm{e}$ znalezione rozwiązanie jest poprawne.

b) Wyznacz granice ciagu $(b_{n})$, gdzie

$b_{n}=\displaystyle \frac{a_{(n-1)^{2}}}{(a_{3n}-a_{2n})^{2}}.$

7. Cztery kule ojednakowym promieniu $a$ sa parami zewnętrznie styczne.

Znajd $\acute{\mathrm{z}}$ promienie dwóch kul, z których $\mathrm{k}\mathrm{a}\dot{\mathrm{z}}$ dajest styczna do wszystkich

czterech kul.
\end{document}
