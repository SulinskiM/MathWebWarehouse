\documentclass[a4paper,12pt]{article}
\usepackage{latexsym}
\usepackage{amsmath}
\usepackage{amssymb}
\usepackage{graphicx}
\usepackage{wrapfig}
\pagestyle{plain}
\usepackage{fancybox}
\usepackage{bm}

\begin{document}

AKADEMIA GÓRNICZO-HUTNICZA

im. Stanislawa Staszica w Krakowie

OLIMPIADA O DIAMENTOWY INDEKS AGH'' 2017/18

MATEMATYKA- ETAP I

ZADANIA PO 10 PUNKTÓW

l. Udowodnij, $\dot{\mathrm{z}}\mathrm{e}$ spośród dowolnych pięciu punktów na pfaszczy $\acute{\mathrm{z}}\mathrm{n}\mathrm{i}\mathrm{e},$

z których $\dot{\mathrm{z}}$ adne trzy nie lezą na jednej prostej, $\mathrm{m}\mathrm{o}\dot{\mathrm{z}}$ na wybrač trzy

punkty, które $\mathrm{s}\Phi$ wierzcholkami trójkąta $\mathrm{r}\mathrm{o}\mathrm{z}\mathrm{w}\mathrm{a}\mathrm{r}\mathrm{t}\mathrm{o}\mathrm{k}_{\Phi}$tnego.

2. Ilejest trójek $(x_{1},x_{2},x_{3})$ liczb całkowitych niedodatnich spelniajqcych

równanie $x_{1}+x_{2}+x_{3}+37=0$?

3. Do zbiornika, w którym znajdowało $\mathrm{s}\mathrm{i}\mathrm{e}p_{0}$ hl wody, pierwszego dnia

dolano 70 h1 wody, po czym $\mathrm{k}\mathrm{a}\dot{\mathrm{z}}$ dego dnia dolewano o 7 h1 wody więcej

$\mathrm{n}\mathrm{i}\dot{\mathrm{z}}$ dnia poprzedniego. Jednocześnie codziennie ze zbiornika ubywafo

170 hl wody. Jaka powinna byč początkowa ilośč $p_{0}$ wody w zbiorniku,

aby nigdy nie brakfo w nim wody? Którego dnia w zbiorniku byfo

najmniej wody?

4. Stopień wielomianu $W(x)$ jest równy 2015. $\mathrm{W}\mathrm{i}\mathrm{e}\mathrm{d}\mathrm{z}\Phi^{\mathrm{C}}, \displaystyle \dot{\mathrm{z}}\mathrm{e}W(n)=\frac{1}{n}$ dla

$n=1$, 2, $\ldots$, 2016, oblicz $W$ (2017).

ZADANIA PO 20 PUNKTÓW

5. Dane jest równanie $(m+1)x^{2}-2(m-3)x+m+1=0$. Dla jakich

wartości parametru $m$

a) liczba llez $\mathrm{y}$ między sumą róznych pierwiastków równania a sumą

ich kwadratów?

b) wartośč bezwzględna przynajmniej jednego pierwiastka równania

jest mniejsza od 0,9?

6. Znajd $\acute{\mathrm{z}}$ sumę dlugości wszystkich przedziafów zawartych w $\langle 0;2\pi\rangle,$

w których spełniona jest nierównośč

$|\displaystyle \mathrm{c}\mathrm{t}\mathrm{g}2x-\mathrm{t}\mathrm{g}2x|\geq\frac{2}{\sqrt{3}}.$

7. $\mathrm{Z}$ wierzchołka $O$ paraboli $y^{2}=3x$ poprowadzono dwie proste wzajem-

nie prostopadłe, przecinajqce parabolę w punktach $M \mathrm{i} N.$ Znajd $\acute{\mathrm{z}}$

równanie (we wspófrzędnych kartezjańskich) zbioru środków cięzkości

wszystkich trójkątów $OMN.$
\end{document}
