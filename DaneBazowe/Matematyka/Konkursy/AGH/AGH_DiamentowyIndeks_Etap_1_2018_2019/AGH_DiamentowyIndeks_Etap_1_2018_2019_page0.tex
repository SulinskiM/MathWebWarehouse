\documentclass[a4paper,12pt]{article}
\usepackage{latexsym}
\usepackage{amsmath}
\usepackage{amssymb}
\usepackage{graphicx}
\usepackage{wrapfig}
\pagestyle{plain}
\usepackage{fancybox}
\usepackage{bm}

\begin{document}

AKADEMIA GÓRNICZO-HUTNICZA

im. Stanislawa Staszica w Krakowie

OLIMPIADA O DIAMENTOWY INDEKS AGH'' 2018/19

MATEMATYKA- ETAP I

ZADANIA PO 10 PUNKTÓW

l. Dwa okręgi o promieniach $r\mathrm{i}R$, gdzie $r<R$, sq styczne zewnętrznie.

Wyznacz pole trójk$\Phi$ta ograniczonego ich wspólnymi stycznymi.

2. Udowodnij, $\dot{\mathrm{z}}\mathrm{e}$ suma $S$ nieskończonego ciągu geometrycznego $(a_{n}),$

w którym $a_{1}<0$, spefnia nierównośč

$S\leq 4\alpha_{2}.$

Kiedy spefniona jest równośč?

3. Znajd $\acute{\mathrm{z}}$ wszystkie liczby naturalne $n$, dla których liczba

$S_{n}=1!+2!+\ldots+n!$

jest kwadratem liczby cafkowitej.

4. Rozwiqz nierównośč

$2^{1+\log_{2}x}\geq x^{\frac{1}{4}(7+\log_{2}x)}.$

ZADANIA PO 20 PUNKTÓW

5. Dane są równania

$x^{2}-px+q=0$

oraz

$x^{2}-px-q=0,$

gdzie $p\mathrm{i}q$ sq liczbami naturalnymi. Wykaz$\cdot, \dot{\mathrm{z}}$ ejezeli obydwa równania

mają pierwiastki calkowite, to istnieją liczby naturalne $a, b$, takie $\dot{\mathrm{z}}\mathrm{e}$

$p^{2}=a^{2}+b^{2}$ Czy implikacja odwrotna jest prawdziwa?

6. Okrąg $0_{1}$ ma równanie $x^{2}+y^{2}+4x-8y+16=0$, a okrqg 02 równanie

$x^{2}+y^{2}-12x+8y+16 = 0$. Oblicz skalę jednokfadności i wspóf-

rzędne środka jednokładności, w której obrazem okręgu $0_{1}$ jest okrąg

02. Napisz równania prostych, które $\mathrm{s}\Phi$ jednocześnie styczne do obu

okręgów.

7. Oblicz objętośč i pole powierzchni bryly obrotowej powstalej z ob-

rotu sześciokqta foremnego o boku $a$ wokół prostej zawierajqcej bok

sześciokąta.
\end{document}
