\documentclass[a4paper,12pt]{article}
\usepackage{latexsym}
\usepackage{amsmath}
\usepackage{amssymb}
\usepackage{graphicx}
\usepackage{wrapfig}
\pagestyle{plain}
\usepackage{fancybox}
\usepackage{bm}

\begin{document}

AKADEMIA GÓRNICZO-HUTNICZA

im. Stanisława Staszica w Krakowie

OLIMPIADA O DIAMENTOWY INDEKS AGH'' 2019/20

MATEMATYKA- ETAP I

ZADANIA PO 10 PUNKTÓW

l. Udowodnij, $\dot{\mathrm{z}}\mathrm{e}\mathrm{j}\mathrm{e}\dot{\mathrm{z}}$ eli czworokąt wypukły ma oś symetrii, to $\mathrm{m}\mathrm{o}\dot{\mathrm{z}}$ na na nim

opisač okrąg lub $\mathrm{m}\mathrm{o}\dot{\mathrm{z}}$ na weń okrąg wpisač.

2. Wyznacz dziedzinę funkcji danej wzorem

$f(x)=(32x^{2}+28x^{5}+4x^{8}-x^{11})^{-\frac{3}{4}}.$

3. $\mathrm{W}$ worku znajduje się 50 skarpet czarnych, 40 brązowych, 30 zie1onych i 20

niebieskich. Jaka jest najmniejsza liczba skarpet, które musimy wyjqč na

chybil trafif, aby mieč pewnośč, $\dot{\mathrm{z}}\mathrm{e}$ wśród nich znajdziemy jednokolorowe

pary skarpet dla 20 osób? Odpowied $\acute{\mathrm{z}}$ uzasadnij.

4. Rozwiąz nierównośč $|1-\displaystyle \log_{x}(x-\frac{1}{4})|\leq 1.$

ZADANIA PO 20 PUNKTÓW

5. Ile jest par $(a,b)$ liczb rzeczywistych, dla których układ równań

$\left\{\begin{array}{l}
ax+by+1=0\\
x^{2}+y^{2}=50
\end{array}\right.$

ma co najmniej jedno rozwiązanie, przy czym $\mathrm{k}\mathrm{a}\dot{\mathrm{z}}$ de jego rozwiązanie jest

parą $(x,y)$ liczb cafkowitych? Podaj przykfad pary $(a,b)$, dla której ukfad

ten ma dwa rozwiązania w liczbach calkowitych oraz przykład pary $(a,b),$

dla której ten uklad ma dokładnie jedno rozwiązanie i to rozwiązanie jest

parq liczb calkowitych.

6. Długości dwóch boków trójkąta wpisanego w okrąg o średnicy $D$ są odpo-

wiednio równe $\displaystyle \frac{3}{4}D$ oraz $\displaystyle \frac{\sqrt{3}}{2}D$. Oblicz długośč trzeciego boku.

7. Zbiór $S$ jest zbiorem wszystkich dodatnich liczb całkowitych $n$, dla których

istnieje permutacja $(a_{1},a_{2},\ldots,a_{n})$ liczb 1, 2, $\ldots, n$, taka $\dot{\mathrm{z}}\mathrm{e}a_{1}+a_{2}+\ldots+a_{k}$

jest wielokrotnością liczby $k$ dla $\mathrm{k}\mathrm{a}\dot{\mathrm{z}}$ dego $k=1$, 2, $\ldots, n$. Wykaz, $\dot{\mathrm{z}}\mathrm{e}\mathrm{k}\mathrm{a}\dot{\mathrm{z}}$ da

liczba naleząca do zbioru $S$ jest nieparzysta. Znajd $\acute{\mathrm{z}}$ dwie liczby tego zbioru.

Zbadaj, czy liczba 2019 na1ez $\mathrm{y}$ do zbioru $S.$
\end{document}
