\documentclass[a4paper,12pt]{article}
\usepackage{latexsym}
\usepackage{amsmath}
\usepackage{amssymb}
\usepackage{graphicx}
\usepackage{wrapfig}
\pagestyle{plain}
\usepackage{fancybox}
\usepackage{bm}

\begin{document}

AKADEMIA GÓRNICZO-HUTNICZA

im. Stanisława Staszica w Krakowie

OLIMPIADA O DIAMENTOWY INDEKS AGH'' 2020/21

MATEMATYKA- ETAP I

ZADANIA PO 10 PUNKTÓW

l. W trapezie ABCD dfuzsza podstawa AB ma długośč 48. Odcinek fączący

środki E, F przekątnych ma dlugośč 4. Ob1icz d1ugośč krótszej podstawy.

2. Znajd $\acute{\mathrm{z}}$ wszystkie elementy zbioru $\displaystyle \{\cos\frac{(n^{7}-n)\pi}{12}$

czerpująco uzasadnij.

: $n\in \mathbb{N}\}.$ Odpowied $\acute{\mathrm{z}}$ wy-

3. Oblicz długośč najdłuzszej krawędzi prostopadłościanu o objętości 2l6 i prze-

katnej dlugości $2\sqrt{91}, \mathrm{j}\mathrm{e}\dot{\mathrm{z}}$ eli długości krawędzi wychodzqcych z jednego

wierzcholka tworzą $\mathrm{c}\mathrm{i}_{\Phi \mathrm{g}}$ geometryczny.

4. Naszkicuj wykres funkcji danej wzorem

$f(x)=x-|x|-2^{|x|+x}$

Na podstawie tego wykresu podaj liczbę rozwiązań równania 3 $f(x+5)=m$

w zalezności od parametru $m.$

ZADANIA PO 20 PUNKTÓW

5. Niech $H$ będzie zbiorem wszystkich tych punktów hiperboli o równaniu

$x^{2}-y^{2} = 25$, których obie wspólrzędne są liczbami cafkowitymi. Napisz

równania okręgów, $\mathrm{z}\mathrm{a}\mathrm{w}\mathrm{i}\mathrm{e}\mathrm{r}\mathrm{a}\mathrm{j}_{\Phi}$cych co najmniej po cztery punkty zbioru $H.$

6. Dla jakich wartości parametru $p$ uklad równań

$\left\{\begin{array}{l}
4x+(p+3)y=p-1\\
(p-1)x+py=p-2.
\end{array}\right.$

ma dokładnie jedno rozwiazanie spełniające nierównośč $|x|+|y|\leq 4$ ?

7. Dla danej liczby naturalnej $n\geq 3$ rozwazmy zbiór $\Omega$ wszystkich permutacji

$(a_{1},\ldots,a_{n})$ liczb l, $\ldots, n.$

A. Ile jest permutacji niebędących ciągami monotonicznymi?

B. Ile jest permutacji, takich $\dot{\mathrm{z}}\mathrm{e}a_{i}+a_{n-i+1} =a_{j}+a_{n-j+1}$ dla wszystkich

$i, j=1, \ldots, n$?

C. Dane są liczby naturalne $k, m$, przy czym $1 <k<m\leq n$. Dla $\mathrm{k}\mathrm{a}\dot{\mathrm{z}}$ dej

permutacji $a=(a_{1},a_{2},\ldots,a_{n})$ ze zbioru $\Omega$ oznaczmy przez $g(a)$ największq

liczbę $j\leq n, \mathrm{t}\mathrm{a}\mathrm{k}_{\Phi}\dot{\mathrm{z}}\mathrm{e}a_{i}<a_{i+1}$ dla wszystkich $i<j$. Ile jest permutacji $a,$

dla których $g(a)=k$ i jednocześnie $a_{k}=m$?


\end{document}