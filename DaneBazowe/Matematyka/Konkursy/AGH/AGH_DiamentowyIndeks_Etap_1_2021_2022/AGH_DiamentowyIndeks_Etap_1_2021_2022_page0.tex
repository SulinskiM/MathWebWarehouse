\documentclass[a4paper,12pt]{article}
\usepackage{latexsym}
\usepackage{amsmath}
\usepackage{amssymb}
\usepackage{graphicx}
\usepackage{wrapfig}
\pagestyle{plain}
\usepackage{fancybox}
\usepackage{bm}

\begin{document}

AKADEMIA GÓRNICZO-HUTNICZA

im. Stanisława Staszica w Krakowie

OLIMPIADA O DIAMENTOWY INDEKS AGH'' 2021/22

MATEMATYKA- ETAP I

ZADANIA PO 10 PUNKTÓW

l. Udowodnij, $\dot{\mathrm{z}}\mathrm{e}$ dla dowolnych dodatnich liczb rzeczywistych $a, b$ prawdziwa

jest nierównośč

$a^{a-b}\geq b^{a-b}$

2. Długości boków trójkąta prostokqtnego tworzą rosnący ciag arytmetyczny.

Wykaz$\cdot, \dot{\mathrm{z}}\mathrm{e}$ róznic$\Phi$ ciągu jest dlugośč promienia okręgu wpisanego w ten

trójkąt.

3. Cztery kolejne liczby parzyste są pierwiastkami wielomianu o wspófczyn-

nikach całkowitych. Udowodnij, $\dot{\mathrm{z}}\mathrm{e}$ wartośč tego wielomianu dla dowolnej

liczby parzystej jest podzielna przez 384.

4. Udowodnij, $\dot{\mathrm{z}}\mathrm{e}$ dla dowolnego trójk$\Phi$ta o dfugościach boków $a, b, c$

$2\sqrt{a^{2}+b^{2}+c^{2}}<\sqrt{3}(a+b+c).$

ZADANIA PO 20 PUNKTÓW

5. Wyznacz zbiór wartości funkcji $g$ danej wzorem

$g(x)=\cos 4x+5\cos^{2}x+\sin^{2}x.$

Dla jakich argumentów $x$ funkcja $g$ przyjmuje najmniejszq wartośč?

6. Dana jest liczba naturalna $k \geq 4$. Na ile sposobów $\mathrm{m}\mathrm{o}\dot{\mathrm{z}}$ na $k+1$ zadań

przydzielič $k$ komputerom, tak by dokfadnie jeden komputer byl wolny,

$\mathrm{j}\mathrm{e}\dot{\mathrm{z}}$ eli

a) zadania i komputery są rozróznialne,

b) komputery $\mathrm{s}\Phi$ rozróznialne, a zadania nie,

c) zadania są rozróznialne, a komputery nie,

d) ani zadania, ani komputery nie $\mathrm{s}\varpi$ rozróznialne?

7. Na p{\it l}aszczy $\acute{\mathrm{z}}\mathrm{n}\mathrm{i}\mathrm{e}$ dane są zbiory

$S=\{(x,y):\log_{|y+1|}x\leq 1\},$

$A_{m}=\{(x,y):x^{2}+y^{2}-2mx-4y+4\leq 0\},$

gdzie $m \in$ lR. Narysuj zbiór $S$. Dla jakich liczb $m$ zbiór $A_{m}$ zawiera się

w zbiorze $S$?
\end{document}
