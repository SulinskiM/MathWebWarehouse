\documentclass[a4paper,12pt]{article}
\usepackage{latexsym}
\usepackage{amsmath}
\usepackage{amssymb}
\usepackage{graphicx}
\usepackage{wrapfig}
\pagestyle{plain}
\usepackage{fancybox}
\usepackage{bm}

\begin{document}

AKADEMIA GÓRNICZO-HUTNICZA

im. Stanislawa Staszica w Krakowie

OLIMPIADA O DIAMENTOWY INDEKS AGH'' 2022/23

MATEMATYKA- ETAP I

ZADANIA PO 10 PUNKTÓW

1. $\mathrm{W}$ prostokqcie ABCD wierzchołek $A$ połączono odcinkami ze środkami boków

$BC \mathrm{i}$ CD. Udowodnij, $\dot{\mathrm{z}}\mathrm{e}$ te odcinki dzielą $\mathrm{P}^{\mathrm{r}\mathrm{z}\mathrm{e}\mathrm{k}}\Phi^{\mathrm{t}\mathrm{n}}\Phi BD$ na trzy odcinki

równej długości.

2. Oblicz sume stu największych ujemnych rozwiązań równania

4 $\cos 2x-\sin 4x=4\cos^{3}2x.$

3. Rozwiąz równanie

$\sqrt[6]{-x^{2}+5x-6}=\sqrt[4]{x^{3}-4x^{2}+x+6}.$

4. $\mathrm{W}$ wypukłym pieciokącie ABCDE $\mathrm{k}\mathrm{a}\dot{\mathrm{z}}$ da przekątna odcina trójk$\Phi$t o polu rów-

nym l. Oblicz pole tego pięciokąta.

ZADANIA PO 20 PUNKTÓW

5. Znajd $\acute{\mathrm{z}}$ równanie stycznej do paraboli $y=2-x^{2}$, która ogranicza wraz z do-

datnimi półosiami układu współrzędnych trójkąt o najmniejszym polu.

6. Niech $S$ będzie zbiorem wszystkich ciągów $(a,b,c,d,e)$ o wyrazach nalezących

do zbioru liczb $\{0$, 1, $\ldots$, 9$\}$. Ile jest w zbiorze $S$ ci$\Phi$gów

a) malejących?

b) których iloczyn abcde jest liczbą parzystą?

c) w których suma cyfr iloczynu abcde w zapisie dziesiętnym jest podzielna

przez 9?

7. Znajd $\acute{\mathrm{z}}$ równania prostych stycznych do okręgu $x^{2}+y^{2}+4x-12=0$ i jedno-

cześnie do jego obrazu w symetrii osiowej względem prostej $2x-3y-22=0.$


\end{document}