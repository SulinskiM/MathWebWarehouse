\documentclass[a4paper,12pt]{article}
\usepackage{latexsym}
\usepackage{amsmath}
\usepackage{amssymb}
\usepackage{graphicx}
\usepackage{wrapfig}
\pagestyle{plain}
\usepackage{fancybox}
\usepackage{bm}

\begin{document}

AKADEMIA GÓRNICZO-HUTNICZA im. Stanislawa Staszica

w Krakowie

OLIMPIADA,, O DIAMENTOWY INDEKS AGH'' 2007/8

MATEMATYKA- ETAP II

ZADANIA PO 10 PUNKTÓW

1. $\mathrm{Z}$ ustalonego zbioru $n$ liczb rzeczywistych losujemy kolejno $k$ liczb, otrzymujac ciag

róznowartościowy $(a_{1},\ldots,a_{k})$. ZakIadajqc, $\dot{\mathrm{z}}\mathrm{e}2\leq k\leq n$, oblicz prawdopodobieństwo,

$\dot{\mathrm{z}}\mathrm{e}$ ten ciag nie jest ciagiem rosnqcym.

2. Sprowad $\acute{\mathrm{z}}$ do najprostszej postaci (niezawierającej ujemnych wykladników, ani ulam-

ków piętrowych) wyrazenie

$(1-x^{-1})^{-2}-(1+x^{-1})^{-2}$

3. Cena akcji pewnej firmy spadIa o 60\%. $\mathrm{O}$ ile procent musi teraz wzrosnač cena tych

akcji, aby wrócila do poprzedniego poziomu?

4. Niech $P$ będzie izometrycznym przeksztatceniem ptaszczyzny, w którym obrazem

wykresu funkcji $f(x) = x^{2}$ jest wykres funkcji $g(x) = x^{2}+x+1.$ Znajd $\acute{\mathrm{z}}$ to

przeksztalcenie i podaj wzór funkcji, której wykres jest obrazem wykresu funkcji

$h(x)=\log_{2}x$ poprzez przeksztaIcenie $P.$

ZADANIA PO 20 PUNKTÓW

5. Oblicz sumę wszystkich pierwiastków równania

$4\cos^{2}x=3$

nalezacych do przedziaIu $(-8\pi;10\pi).$

6. Znajd $\acute{\mathrm{z}}$ równanie stycznej $l$ do okręgu $C$ o równaniu

$x^{2}+y^{2}-4x+6y-12=0$

w punkcie $A(6,0)$. Napisz równanie okręgu symetrycznego do okręgu $C$ względem

prostej $l.$

7. Dla jakich wartości parametru $p$ równanie

$(p-2)\cdot 9^{x}+(p+1)\cdot 3^{x}-p=0$

ma dwa rózne pierwiastki rzeczywiste?


\end{document}