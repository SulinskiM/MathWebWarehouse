\documentclass[a4paper,12pt]{article}
\usepackage{latexsym}
\usepackage{amsmath}
\usepackage{amssymb}
\usepackage{graphicx}
\usepackage{wrapfig}
\pagestyle{plain}
\usepackage{fancybox}
\usepackage{bm}

\begin{document}

AKADEMIA GÓRNICZO-HUTNICZA im. Stanislawa Staszica

w Krakowie

OLIMPIADA,, O DIAMENTOWY INDEKS AGH'' 2008/9

MATEMATYKA- ETAP II

ZADANIA PO 10 PUNKTÓW

l. Suma dwóch liczb rzeczywistych wynosi 6. Jaką największa wartośč $\mathrm{m}\mathrm{o}\dot{\mathrm{z}}\mathrm{e}$ mieč ich

iloczyn?

2. Sprowad $\acute{\mathrm{z}}$ do najprostszej postaci wyrazenie

$\displaystyle \frac{(a^{3}+b^{3})(a^{-1}-b^{-1})}{(a^{-1}+b^{-1})[(a-b)^{2}+ab]}.$

3. Odleglośč środka okręgu opisanego na trójkacie prostokqtnym od przyprostokatnych

wynosi odpowiednio piq. Oblicz obwód tego trójkata.

4. Oblicz $\displaystyle \log_{9}\cos\frac{11\pi}{6}$ -log9 $\displaystyle \sin\frac{29\pi}{6}.$

ZADANIA PO 20 PUNKTÓW

5. Dla jakich $m$ proste $mx+(m+1)y = 2 \mathrm{i} 4x+(m+4)y = 1$ przecinajq się

w punkcie lezącym wewnatrz II lub IV čwiartki ukIadu wspóIrzędnych?

6. $k$ pasazerów wsiada do pociqgu zlozonego z 3 wagonów, przy czym $\mathrm{k}\mathrm{a}\dot{\mathrm{z}}\mathrm{d}\mathrm{y}$ wybiera

wagon niezaleznie i z jednakowym prawdopodobieństwem $\displaystyle \frac{1}{3}$. ZakIadajac, $\dot{\mathrm{z}}\mathrm{e}k\geq 3,$

oblicz prawdopodobieństwo zdarzeń:

A- wszyscy wsiądą do jednego wagonu,

B- dokladnie jeden wagon będzie pusty,

{\it C}- $\dot{\mathrm{z}}$ aden wagon nie bedzie pusty.

7. Podstawa ostroslupa jest trójkat $ABC$, w którym bok $AB$ ma dIugośč $a$, a katy

wewnętrzne do niego przylegte maja miary $\beta \mathrm{i} \gamma.$ {\it K}rawęd $\acute{\mathrm{z}}$ boczna ostroslupa

wychodząca z wierzcholka $C$ jest prostopadla do podstawy i ma dlugośč $d$. Oblicz

objętości bryl, na które ten ostroslup dzieli ptaszczyzna równolegIa do podstawy

i odlegIa od niej o $\displaystyle \frac{d}{3}.$
\end{document}
