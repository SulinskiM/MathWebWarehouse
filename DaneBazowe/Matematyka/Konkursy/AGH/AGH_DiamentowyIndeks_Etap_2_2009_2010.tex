\documentclass[a4paper,12pt]{article}
\usepackage{latexsym}
\usepackage{amsmath}
\usepackage{amssymb}
\usepackage{graphicx}
\usepackage{wrapfig}
\pagestyle{plain}
\usepackage{fancybox}
\usepackage{bm}

\begin{document}

AKADEMIA GÓRNICZO-HUTNICZA

im. StanisIawa Staszica w Krakowie

OLIMPIADA,, O DIAMENTOWY INDEKS AGH'' 2009/10

MATEMATYKA- ETAP II

ZADANIA PO 10 PUNKTÓW

l. Pole powierzchni bocznej stozka jest trzy razy większe od pola jego podstawy.

Ile razy objętośč stozka jest wieksza od objętości kuli wpisanej w ten stozek?

2. Dane sq funkcje $f(x) =2^{x+1}+5^{x-5}$

$g(\displaystyle \frac{x}{2})=f(x+3).$

$\mathrm{i} g(x) =25^{x}+4^{x}$

Rozwiąz równanie

3. Oblicz $\sin 2\alpha, \mathrm{j}\mathrm{e}\dot{\mathrm{z}}$ eli $\sin\alpha=0$, 75

$\mathrm{i} \displaystyle \alpha\in(\frac{\pi}{2};\pi).$

4. Wyznacz granice ciągu

$\displaystyle \lim_{n\rightarrow+\infty}(\sqrt[3]{n^{6}+5n^{4}}-n^{2}).$

ZADANIA PO 20 PUNKTÓW

5. Znajd $\acute{\mathrm{z}}$ równania stycznych do okręgu $x^{2}+y^{2}-8y+12=0$ przechodzacych

przez początek uktadu wspólrzednych. Znajd $\acute{\mathrm{z}}$ równania obrazów tego okręgu

i jednej z wyznaczonych stycznych w jednoktadności o środku w punkcie $S=$

$(1,2)$ i skali $k=-3.$

6. Funkcja $f$ spelnia dla $\mathrm{k}\mathrm{a}\dot{\mathrm{z}}$ dego $x$ nalezącego do jej dziedziny równanie

$ 1+f(x)+(f(x))^{2}+(f(x))^{3}+\ldots$

$= \displaystyle \frac{x}{2}+1,$

gdzie lewa strona jest sumą nieskończonego ciqgu geometrycznego. Wyznacz

dziedzinę i wzór funkcji $f$. Naszkicuj jej wykres.

7. Liczby l, 2, 3, $\ldots, n$, gdzie $n\geq 3$, losowo ustawiamy w ciag. Oblicz prawdo-

podobieństwa zdarzeń

$A$: liczba $n$ nie bQdzie ostatnim wyrazem tego ciagu;

$B$: liczby 1, 2, 3 wystqpią obok siebie w ko1ejności wzrastania;

$C$: iloczyn $\mathrm{k}\mathrm{a}\dot{\mathrm{z}}$ dej pary sasiednich wyrazów tego ciagu jest liczbą parzysta.

Wyniki zapisz w najprostszej postaci.


\end{document}