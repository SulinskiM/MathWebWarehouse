\documentclass[a4paper,12pt]{article}
\usepackage{latexsym}
\usepackage{amsmath}
\usepackage{amssymb}
\usepackage{graphicx}
\usepackage{wrapfig}
\pagestyle{plain}
\usepackage{fancybox}
\usepackage{bm}

\begin{document}

AKADEMIA GÓRNICZO-HUTNICZA

im. StanisIawa Staszica w Krakowie

OLIMPIADA,, O DIAMENTOWY INDEKS AGH'' 2010/11

MATEMATYKA- ETAP II

ZADANIA PO 10 PUNKTÓW

l. Dany jest ostrosIup prawidlowy trójkątny o krawedzi podstawy dlugości $a=1$

cm i wysokości opuszczonej na podstawę $H=2$ cm. Oblicz odlegIośč wierz-

choIka podstawy od przeciwleglej ściany.

2. Sprawd $\acute{\mathrm{z}}$, czy ciag

$\displaystyle \frac{1}{4}, \displaystyle \frac{2+\sqrt{3}}{2},$

$2+\sqrt{3}$

$2-\sqrt{3}$

jest $\mathrm{c}\mathrm{i}_{4\mathrm{g}}\mathrm{i}\mathrm{e}\mathrm{m}$ geometrycznym.

3. Dane sa punkty $A=(-1,-8)$ oraz $B=(5,4).$ Znajd $\acute{\mathrm{z}}$ taki punkt $C, \dot{\mathrm{z}}\mathrm{e}$

$A^{\rightarrow}C=5C^{\rightarrow}B.$

4. Rozwiąz równanie

$\log_{x-2}(x^{3}-x^{2}-7x+10)=2.$

ZADANIA PO 20 PUNKTÓW

5. Liczby l, 2, $\ldots, n$, gdzie $n>2$, przestawiamy w dowolny sposób. Oblicz praw-

dopodobieństwo nastepujacych zdarzeń:

$A-$ pierwszy wyraz otrzymanego ciągu będzie większy od ostatniego,

$B-$ liczby l $\mathrm{i}2$ nie będą ustawione obok siebie,

$C-$ liczby 1, 2 $\mathrm{i}3$ będa ustawione obok siebie w kolejności wzrastania.

6. Oblicz sumę trzydziestu największych ujemnych rozwiązań równania

$\cos 2x+\sin x=0.$

7. Zbadaj w zalezności od parametru k wzajemne potozenie prostych

$l_{1}$ : $kx+y=2,$

oraz

$l_{2}$:

$x+ky=k+1.$

Dla jakich $k$ te proste przecinają się $\mathrm{w}\mathrm{e}\mathrm{w}\mathrm{n}4^{\mathrm{t}\mathrm{r}\mathrm{Z}}$ kwadratu, w którym punkty

$A=(2,-2)\mathrm{i}C=(-2,2)$ sa końcami przekatnej?
\end{document}
