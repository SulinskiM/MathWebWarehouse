\documentclass[a4paper,12pt]{article}
\usepackage{latexsym}
\usepackage{amsmath}
\usepackage{amssymb}
\usepackage{graphicx}
\usepackage{wrapfig}
\pagestyle{plain}
\usepackage{fancybox}
\usepackage{bm}

\begin{document}

AKADEMIA GÓRNICZO-HUTNICZA

im. StanisIawa Staszica w Krakowie

OLIMPIADA,, O DIAMENTOWY INDEKS AGH'' 2011/12

MATEMATYKA- ETAP II

ZADANIA PO 10 PUNKTÓW

l. Wykaz, $\dot{\mathrm{z}}\mathrm{e}$ liczba $a=\sqrt{9-4\sqrt{5}}-\sqrt{9+4\sqrt{5}}$ jest catkowita.

2. Wyznacz dziedzinę funkcji danej wzorem

$f(x)=\sqrt{x^{4}+x^{3}-8x^{2}-12x}.$

3. Oblicz miarę kata między wektorami ã $\mathrm{i} \vec{b}$ wiedząc, $\dot{\mathrm{z}}\mathrm{e}$ wektory

$\vec{u}=3\text{{\it ã}}+2\vec{b} \mathrm{i} \vec{v}=-\text{{\it ã}}+4\vec{b}$ sa prostopadIe oraz $|\vec{a}|=|\vec{b}|=1.$

4. Dwa rózne automaty wykonują razem dana pracę w ciagu 6 godzin.

Gdyby pierwszy automat pracowal sam przez 2 godziny, a następnie

drugi pracowat sam przez 6 godzin, to wykonaIyby poIowę ca1ej pracy.

Wjakim czasie $\mathrm{k}\mathrm{a}\dot{\mathrm{z}}\mathrm{d}\mathrm{y}$ automat $\mathrm{m}\mathrm{o}\dot{\mathrm{z}}\mathrm{e}$ samodzielnie wykonač calq pracę?

ZADANIA PO 20 PUNKTÓW

5. Ze zbioru $S = \{1$, 2, $\ldots$, 2012$\}$ losujemy trzy liczby i ustawiamy je

w ciąg rosnacy $(a,b,c)$. Oblicz prawdopodobieństwo zdarzeń

$A$: iloczyn $abc$ jest liczba parzysta,

$B_{k}$: $b=k$, gdzie $k$ jest ustaloną liczbq ze zbioru $S.$

Dla jakich $k$ prawdopodobieństwo zdarzenia $B_{k}$ jest największe?

6. Dane są dwa punkty $A = (7,5), B = (1,-1)$ oraz punkt $P = (3,3)$

przecięcia wysokości trójkata $ABC$. Oblicz pole trójkata $ABC$ i napisz

równanie okregu opisanego na nim.

7. Stozek i walec mają równe $\mathrm{t}\mathrm{w}\mathrm{o}\mathrm{r}\mathrm{z}4^{\mathrm{C}\mathrm{e}}$, równe objętości i równe pola

powierzchni bocznej. Oblicz

a) sinus kąta nachylenia tworzacej stozka do jego podstawy,

b) stosunek pola przekroju osiowego walca do pola przekroju osiowego

stozka.
\end{document}
