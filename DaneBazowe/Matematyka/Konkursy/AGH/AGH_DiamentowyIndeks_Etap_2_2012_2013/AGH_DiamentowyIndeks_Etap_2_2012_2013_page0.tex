\documentclass[a4paper,12pt]{article}
\usepackage{latexsym}
\usepackage{amsmath}
\usepackage{amssymb}
\usepackage{graphicx}
\usepackage{wrapfig}
\pagestyle{plain}
\usepackage{fancybox}
\usepackage{bm}

\begin{document}

AKADEMIA GÓRNICZO-HUTNICZA

im. StanisIawa Staszica w Krakowie

OLIMPIADA,, O DIAMENTOWY INDEKS AGH'' 2012/13

MATEMATYKA- ETAP II

ZADANIA PO 10 PUNKTÓW

l. Rozwiąz równanie $(5\sqrt{2}-7)^{x-1}=(5\sqrt{2}+7)^{3x}$

2. Jedna z krawędzi bocznych ostrosIupa jest prostopadla do jego pod-

stawy, będacej prostokatem o bokach dIugości 5 cm i 12 cm. Najd1uzsza

krawęd $\acute{\mathrm{z}}$ boczna jest nachylona do ptaszczyzny podstawy pod kątem

$60^{o}$ Oblicz pole powierzchni bocznej ostroslupa.

3. Wyznacz dziedzinę funkcji określonej wzorem

$f(x)=\sqrt{\log_{\pi-1}(2x-1)-\log_{\pi-1}(5-3x)}.$

4. Oblicz granicę ciagu $(a_{n})$, gdzie

$a_{n}=\displaystyle \frac{3+6+9+\ldots+3n}{(2n+1)^{2}}.$

ZADANIA PO 20 PUNKTÓW

5. Wykaz$\cdot, \dot{\mathrm{z}}\mathrm{e}(2n+2)$-cyfrowa liczba ll$\ldots$122$\ldots$25

$\overline{n}\tilde{n+1}$

liczby naturalnej (dla dowolnego $n$).

jest kwadratem

6. Dane są punkty $A = (-5,0), B = (-3,-4), C = (3,4), M = (7,1).$

$\mathrm{Z}$ punktu $M$ poprowadzono styczne $k\mathrm{i}l$ do okręgu opisanego na trójka-

cie $ABC$. Oblicz pole trójkąta $KLM$, gdzie $K\mathrm{i}L$ sq punktami stycz-

ności prostych $k\mathrm{i}l$ z tym okręgiem.

7. Rzucamy moneta $n$ razy $(n\geq 2)$. Oblicz prawdopodobieństwa zdarzeń:

$A$: reszka wypadIa dokIadnie $k$ razy;

$B$: reszka wypadla więcej razy $\mathrm{n}\mathrm{i}\dot{\mathrm{z}}$ orzel;

$C$: przynajmniej dwa razy pod rząd moneta upadla tą samą strona.
\end{document}
