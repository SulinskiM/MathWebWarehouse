\documentclass[a4paper,12pt]{article}
\usepackage{latexsym}
\usepackage{amsmath}
\usepackage{amssymb}
\usepackage{graphicx}
\usepackage{wrapfig}
\pagestyle{plain}
\usepackage{fancybox}
\usepackage{bm}

\begin{document}

AKADEMIA GÓRNICZO-HUTNICZA

im. StanisIawa Staszica w Krakowie

OLIMPIADA,, O DIAMENTOWY INDEKS AGH'' 2013/14

MATEMATYKA- ETAP II

ZADANIA PO 10 PUNKTÓW

l. Urządzenie I wykonuje pewną pracę w ciagu 20 godzin, a urzadzenie II w ciagu 30

godzin. $\mathrm{W}$ jakim czasie wykonają tę pracę oba urzqdzenia pracujac jednocześnie?

2. Kotangens kata rozwartego $\alpha$ jest równy $-3$. Oblicz wartości funkcji trygonomet-

rycznych kąta $2\alpha.$

3. Rozwiąz nierównośč $|3\log_{x}2-2|>1.$

4. Zbadaj monotonicznośč ciagu $(a_{n})$, którego n-ty wyraz jest równy

$a_{n}=\displaystyle \frac{3^{n+2}}{3^{n}+2^{2n+1}}.$

Wyznacz granicę ciągu $(a_{n}).$

ZADANIA PO 20 PUNKTÓW

5. Okrag $O$ ma równanie $x^{2}+y^{2}+6x+4y-12=0$. Okrag $O'$ jest obrazem okręgu

$O$ przez translację o wektor $\vec{v}= [7$, 1$].$ Znajd $\acute{\mathrm{z}}$ równania osi symetrii sumy $O\cup O'$

tych okręgów. Wyznacz punkty wspólne obu okręgów. Znajd $\acute{\mathrm{z}}$ równania prostych

stycznych jednocześnie do $O\mathrm{i}O'.$

6. Podstaw4 ostroslupa o wysokości $H$ jest trójkqt prostokątny $ABC$ o przyprosto-

kątnych $|AB| =a\mathrm{i} |AC| =b$. Krawędz$\acute{}$ boczna wychodząca z wierzcholka $A$ jest

prostopadla do podstawy. Ostrostup ten podzielono pIaszczyzna równolegla do pod-

stawy na dwie bryIy o równych objętościach. Oblicz pole powierzchni catkowitej tej

bryly, która nie jest ostroslupem.

7. Do windy na parterze budynku czteropiętrowego wsiada osiem osób. Oblicz prawdo-

podobieństwa zdarzeń:

$A$: wszyscy wysiada na tym samym piętrze,

$B$: na czwartym piętrze wysiqdq co najmniej dwie osoby,

$C$: na $\mathrm{k}\mathrm{a}\dot{\mathrm{z}}$ dym piętrze wysiada po dwie osoby.


\end{document}