\documentclass[a4paper,12pt]{article}
\usepackage{latexsym}
\usepackage{amsmath}
\usepackage{amssymb}
\usepackage{graphicx}
\usepackage{wrapfig}
\pagestyle{plain}
\usepackage{fancybox}
\usepackage{bm}

\begin{document}

AKADEMIA GÓRNICZO-HUTNICZA

im. StanisIawa Staszica w Krakowie

OLIMPIADA,, O DIAMENTOWY INDEKS AGH'' 2014/15

MATEMATYKA- ETAP II

ZADANIA PO 10 PUNKTÓW

l. Udowodnij, $\dot{\mathrm{z}}\mathrm{e}$ dla dowolnych dodatnich liczb rzeczywistych $a, b$ spetniona jest nie-

równośč

-{\it ab}$+$-{\it ab} $\geq$2.

2. Wyznacz najmniejszq i najwiQkszq wartośč funkcji danej wzorem $f(x)=|x^{2}-8x+7|$

w przedziale $\langle 0;5\rangle.$

3. Znajd $\acute{\mathrm{z}}$ punkty nieciqgtości funkcji danej wzorem

$f(x)=\displaystyle \frac{x^{2}-4}{x^{4}+x^{3}+8x+8}.$

$\mathrm{W}$ których z tych punktów $\mathrm{m}\mathrm{o}\dot{\mathrm{z}}$ na określič wartośč funkcji tak, $\dot{\mathrm{z}}$ eby byla ciqgta?

4. $\mathrm{W}\mathrm{k}\mathrm{a}\dot{\mathrm{z}}$ dym z ostatnich dwóch notowań cena ropy spadata o k\%, gdzie $k\in(0;100).$

$\mathrm{O}$ ile procent musiataby cena wzrosnqč w najblizszym notowaniu, $\dot{\mathrm{z}}$ eby wrócita do

poczqtkowego poziomu?

ZADANIA PO 20 PUNKTÓW

5. Figura $B$ jest obrazem figury

$A=\{(x,y)$ : $x^{2}+y^{2}-6x-8y+21\leq 0$

$\wedge x-7y+25\geq 0\}.$

przez symetrię względem prostej $x-2y=0.$ Znajd $\acute{\mathrm{z}}$ nierówności opisujqce figurę $B$

i oblicz jej obwód.

6. Rozwiqz nierównośč

$\log_{2x}(x^{4}+3)\geq 2.$

7. $\mathrm{W}$ trójkqt prostokqtny o przyprostokqtnych $a = 15$ cm, $b = 20$ cm wpisany jest

okrqg. Oblicz odlegtości od $\mathrm{k}\mathrm{a}\dot{\mathrm{z}}$ dego wierzchotka trójkqta do punktu styczności

okręgu z przeciwlegtym bokiem.
\end{document}
