\documentclass[a4paper,12pt]{article}
\usepackage{latexsym}
\usepackage{amsmath}
\usepackage{amssymb}
\usepackage{graphicx}
\usepackage{wrapfig}
\pagestyle{plain}
\usepackage{fancybox}
\usepackage{bm}

\begin{document}

AKADEMIA GÓRNICZO-HUTNICZA

im. StanisIawa Staszica w Krakowie

OLIMPIADA,, O DIAMENTOWY INDEKS AGH'' 2015/16

MATEMATYKA- ETAP II

ZADANIA PO 10 PUNKTÓW

l. Wyznacz największa liczbę naturalna $k$ taka, $\dot{\mathrm{z}}\mathrm{e}$ liczba 2016! jest wie-

lokrotnością liczby $10^{k}$

2. Rozwiqz nierównośč $\displaystyle \log_{x}(x^{2}-\frac{5}{2}x+1)-2<0.$

3. Wyznacz dziedzinę $D$ funkcji określonej wzorem

$f(x)=\displaystyle \frac{\sqrt{x^{2}+6x+9}}{x^{2}-x-12}$

i zbadaj jej granice w punktach nalezacych do zbioru $lR\backslash D.$

4. Zespolowi pracowników zlecono pewną pracę. Gdyby bylo ich o 3 mniej,

to pracowaliby o 5 dni $\mathrm{d}\mathrm{I}\mathrm{u}\dot{\mathrm{z}}$ ej, a gdyby bylo ich o 4 więcej, to pracowa1iby

$02$ dni krócej. Ilu bylo pracowników i jak dlugo pracowali?

ZADANIA PO 20 PUNKTÓW

5. {\it K}rawęd $\acute{\mathrm{z}}$ boczna ostroslupa prawidlowego sześciokątnegojest nachylona

do podstawy pod katem $60^{o}$ Oblicz stosunek dIugości promienia kuli

wpisanej w ten ostroslup do jego wysokości.

6. Okrag $O'$ jest obrazem okregu $O$ o równaniu

$x^{2}+y^{2}-4x-6y-12=0$

w symetrii środkowej względem punktu $M= (6,6)$. Napisz równanie

okregu $O'$ i równania wszystkich prostych, które sajednocześnie styczne

do obu okręgów.

7. Losowo wybieramy liczbę $k$ ze zbioru \{1, 2, 3, 4\}, a następnie rzucamy

$k$ razy sześcienna kostka. Oblicz prawdopodobieństwa zdarzeń:

$A$: wypadną same szóstki,

$B$: iloczyn wyrzuconych oczek będzie liczbq parzysta,

$C$: suma wyrzuconych oczek będzie mniejsza $\mathrm{n}\mathrm{i}\dot{\mathrm{z}}22.$
\end{document}
