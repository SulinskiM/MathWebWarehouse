\documentclass[a4paper,12pt]{article}
\usepackage{latexsym}
\usepackage{amsmath}
\usepackage{amssymb}
\usepackage{graphicx}
\usepackage{wrapfig}
\pagestyle{plain}
\usepackage{fancybox}
\usepackage{bm}

\begin{document}

AKADEMIA GÓRNICZO-HUTNICZA

im. Stanislawa Staszica w Krakowie

OLIMPIADA O DIAMENTOWY INDEKS AGH'' 2016/17

MATEMATYKA- ETAP II

ZADANIA PO 10 PUNKTÓW

l. Udowodnij, $\dot{\mathrm{z}}\mathrm{e}$ spośród dowolnych pięciu liczb naturalnych $\mathrm{m}\mathrm{o}\dot{\mathrm{z}}$ na wybrač trzy,

których suma jest podzielna przez 3.

2. Rozwiqz równanie

$\displaystyle \frac{\log_{x}(x^{3}+3)}{\log_{x}(x+1)}=2.$

3. Ile jest sześciocyfrowych liczb naturalnych, w których liczba cyfr parzystych jest

równa liczbie cyfr nieparzystych?

4. Oblicz promień okręgu opisanego na trójkącie $ABC$, w którym $|AB|=10$ cm,

$|AC|=8$ cm i miara kąta przy wierzchofku $A$ jest równa $60^{\mathrm{o}}$

ZADANIA PO 20 PUNKTÓW

5. Wykres funkcji kwadratowej $f(x)$ przechodzi przez punkty $(-2,16), (1,-2)$, (3, 6).

Po przesunięciu go o wektor $\vec{v}=[2,-6]$ i przeksztalceniu przez symetrię wzglę-

dem prostej $x=0$ otrzymano wykres funkcji $g(x)$. Wykres funkcji $g(x)$ prze-

ksztalcono przez symetrię względem prostej $y=3$, otrzymując wykres funkcji

$h(x)$. Napisz wzory funkcji $f(x), g(x)\mathrm{i}h(x).$

6. $\mathrm{W}$ prawidfowym ostrosfupie czworokątnym krawędzie boczne $\mathrm{s}\Phi$ nachylone do

podstawy pod kątem $\alpha. \mathrm{W}$ ostroslup wpisano półkulę o promieniu $R$ tak, $\dot{\mathrm{z}}\mathrm{e}$

jest ona styczna do ścian bocznych, a kofo wielkie zawiera się w podstawie

ostrosfupa. Oblicz objętośč ostrosfupa.

7. Suma wszystkich wspólczynników wielomianu $W(x)$ jest równa

{\it n}l$\rightarrow$im$\infty$ -552--2{\it n-n}$++$2 -211 --22{\it nn}.

Suma wspólczynników przy parzystych potęgach zmiennej $x$ jest 3 razy większa

$\mathrm{n}\mathrm{i}\dot{\mathrm{z}}$ suma wspólczynników przy potęgach nieparzystych. Znajd $\acute{\mathrm{z}}$ reszty z dzielenia

$W(x)$ przez dwumiany: a) $x-1$, b) $x+1$, c) $x^{2}-1.$


\end{document}