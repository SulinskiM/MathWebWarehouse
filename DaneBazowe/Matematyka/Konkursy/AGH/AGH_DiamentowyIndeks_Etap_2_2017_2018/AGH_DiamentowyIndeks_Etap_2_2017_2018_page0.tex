\documentclass[a4paper,12pt]{article}
\usepackage{latexsym}
\usepackage{amsmath}
\usepackage{amssymb}
\usepackage{graphicx}
\usepackage{wrapfig}
\pagestyle{plain}
\usepackage{fancybox}
\usepackage{bm}

\begin{document}

AKADEMIA GÓRNICZO-HUTNICZA

im. Stanislawa Staszica w Krakowie

OLIMPIADA O DIAMENTOWY INDEKS AGH'' 2017/18

MATEMATYKA - ETAP II

ZADANIA PO 10 PUNKTÓW

l. Ile jest sześciocyfrowych liczb naturalnych, w których występuje $\mathrm{k}\mathrm{a}\dot{\mathrm{z}}$-

da z cyfr 0,1,2,3,4,5? I1e jest wśród nich 1iczb parzystych, a i1e 1iczb

pierwszych?

2. Odlegfości punktu $P$, lezącego wewnątrz kwadratu, od trzech jego

wierzchofków wynoszq odpowiednio 35 cm, 35 cm i 49 cm. Ob1icz

odległośč punktu $P$ od czwartego wierzchołka kwadratu.

3. Udowodnij, $\dot{\mathrm{z}}\mathrm{e}$ dla dowolnych liczb rzeczywistych $a, b, c$ spefniona jest

nierównośč

$\displaystyle \sqrt{\frac{a^{2}+b^{2}+c^{2}}{3}}\geq\frac{a+b+c}{3}.$

4. Rozwiąz równanie

$\log_{x}10+\log_{x}10^{2}+\cdots+\log_{x}10^{100}=10100.$

ZADANIA PO 20 PUNKTÓW

5. Prosta $x+2y-13=0$ zawiera bok $AB$, prosta $x-y+5=0$ zawiera

bok $BC$ trójkąta $ABC$, a prosta $x-3y+7=0$ zawiera dwusieczną

kata $BCA.$ Znajd $\acute{\mathrm{z}}$ wierzchołki tego trójkąta.

6. $\mathrm{W}$ ostrosfupie prawidfowym czworokątnym o krawędzi podstawy dfu-

gości $a=2$ dm $\mathrm{k}\mathrm{a}\mathrm{t}$ między ścianami bocznymi ma miarę $135^{o}$ Ostro-

sfup ten przecięto dwiema plaszczyznami równoległymi do postawy

na trzy bryly o równych objętościach. Oblicz odległośč między tymi

płaszczyznami.

7. Wyznacz przedziafy monotoniczności funkcji określonej wzorem

$ f(x)=x+\displaystyle \frac{3}{x}+\frac{9}{x^{3}}+\frac{27}{x^{5}}+\cdots$
\end{document}
