\documentclass[a4paper,12pt]{article}
\usepackage{latexsym}
\usepackage{amsmath}
\usepackage{amssymb}
\usepackage{graphicx}
\usepackage{wrapfig}
\pagestyle{plain}
\usepackage{fancybox}
\usepackage{bm}

\begin{document}

AKADEMIA GÓRNICZO-HUTNICZA

im. Stanislawa Staszica w Krakowie

OLIMPIADA O DIAMENTOWY INDEKS AGH'' 2018/19

MATEMATYKA - ETAP II

ZADANIA PO 10 PUNKTÓW

l. Kierowca przejechal połowę drogi autostradą, a drugą połowę lokal-

nymi drogami z prędkością dwa razy mniejszą. Jaki procent drogi

przejechal kierowca po uplywie połowy czasu podróz $\mathrm{y}$?

2. Liczba $\alpha\in (\displaystyle \frac{\pi}{4};\frac{\pi}{2})$ spefnia równanie

$\mathrm{t}\mathrm{g}^{2}\alpha=5\mathrm{t}\mathrm{g}\alpha-1.$

Oblicz $\cos 2\alpha.$

3. Na ile sposobów $\mathrm{m}\mathrm{o}\dot{\mathrm{z}}$ na tak ustawič w $\mathrm{c}\mathrm{i}\otimes \mathrm{g} k$ czarnych kul i $k+1$

bialych, by $\dot{\mathrm{z}}$ adne dwie czarne kule nie znalazfy się obok siebie? Za-

kładamy, $\dot{\mathrm{z}}\mathrm{e}$ kule tego samego koloru są nierozróznialne.

4. Wyznacz dziedzinę funkcji danej wzorem

$f(x)=\displaystyle \frac{x^{5}+8x^{2}}{x^{3}-4x^{2}-3x+18}$

$\mathrm{W}$ których punktach nienalezących do dziedziny $\mathrm{m}\mathrm{o}\dot{\mathrm{z}}$ na określič war-

tośč funkcji $f$, aby otrzymač funkcję ciqgłą w danym punkcie?

ZADANIA PO 20 PUNKTÓW

5. $\mathrm{W}$ prawidfowy ostroslup $\mathrm{c}\mathrm{z}\mathrm{w}\mathrm{o}\mathrm{r}\mathrm{o}\mathrm{k}_{\Phi}\mathrm{t}\mathrm{n}\mathrm{y}$ o krawędzi podstawy dlugości $a$

i krawędzi bocznej długości $b$ tak wpisany jest walec, $\dot{\mathrm{z}}\mathrm{e}$ jedna z pod-

staw walca zawiera się w podstawie ostrosfupa. Wyznacz wymiary

walca o $\mathrm{m}\mathrm{o}\dot{\mathrm{z}}$ liwie najwiekszej objętości.

6. Dla jakich liczb rzeczywistych $m$ równanie

$(m-3)x^{2}+2mx+m-2=0$

ma dwa rózne pierwiastki rzeczywiste $x_{1}, x_{2}$, spelniające nierównośč

$\log_{0,1}x_{1}+\log_{0,1}x_{2}\geq 0$?

7. $\mathrm{W}$ trójkąt równoboczny o boku dlugości $a$ tak wpisane są trzy przy-

stające okręgi, $\dot{\mathrm{z}}\mathrm{e}\mathrm{k}\mathrm{a}\dot{\mathrm{z}}\mathrm{d}\mathrm{y}$ z nich jest styczny do dwóch pozostałych i do

dwóch boków trójk$\Phi$ta. Oblicz promień okręgu zewnętrznie stycznego

do tych trzech okręgów.


\end{document}