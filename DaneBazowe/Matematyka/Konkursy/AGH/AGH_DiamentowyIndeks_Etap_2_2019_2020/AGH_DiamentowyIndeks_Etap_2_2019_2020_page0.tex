\documentclass[a4paper,12pt]{article}
\usepackage{latexsym}
\usepackage{amsmath}
\usepackage{amssymb}
\usepackage{graphicx}
\usepackage{wrapfig}
\pagestyle{plain}
\usepackage{fancybox}
\usepackage{bm}

\begin{document}

AKADEMIA GÓRNICZO-HUTNICZA

im. Stanisława Staszica w Krakowie

OGÓLNOPOLSKA OLIMPIADA

O DIAMENTOWY INDEKS AGH'' 2019/20

MATEMATYKA - ETAP II

ZADANIA PO 10 PUNKTÓW

l. Niech $n$ będzie dowolną $\mathrm{n}\mathrm{i}\mathrm{e}\mathrm{P}^{\mathrm{a}\mathrm{r}\mathrm{z}\mathrm{y}\mathrm{s}\mathrm{t}_{\Phi}1\mathrm{i}\mathrm{c}\mathrm{z}\mathrm{b}_{\Phi^{\mathrm{n}\mathrm{a}\mathrm{t}\mathrm{u}\mathrm{r}\mathrm{a}\ln}\Phi}}$. Udowodnij, $\dot{\mathrm{z}}\mathrm{e}$ suma

$n$ kolejnych liczb całkowitych jest podzielna przez $n.$

2. Dla jakich liczb $k$ trójmian kwadratowy

$2(1-k^{2})x^{2}+k(1+k^{2})x+2k$

jest podzielny przez dwumian $x+k$?

3. Rozwiqz równanie $\cos^{2}3x-\sin^{2}x=0.$

4. Do klasy, w której co czwarty uczeń jest jedynakiem, przyfączono $\mathrm{d}\mathrm{r}\mathrm{u}\mathrm{g}\Phi$

klasę o dwukrotnie mniejszej liczbie uczniów, wśród których jest 40\% jedy-

naków. Jaki procent uczniów w nowo utworzonej klasie ma rodzeństwo?

ZADANIA PO 20 PUNKTÓW

5. Ze zbioru \{l, 2, $\ldots$, 9\} 1osujemy jednocześnie dwie liczby. Czynnośč tę pow-

tarzamy (zwróciwszy wylosowane liczby) dotqd, $\mathrm{a}\dot{\mathrm{z}}$ wylosujmy dwie liczby

dające tę samą resztę z dzielenia przez 3. Jakie jest prawdopodobieństwo,

$\dot{\mathrm{z}}\mathrm{e}$ liczba losowań będzie

$A$: mniejsza $\mathrm{n}\mathrm{i}\dot{\mathrm{z}}10,$

$B$: równa 6,

$C$: nieparzysta.

6. Funkcja $f$ dla $\mathrm{k}\mathrm{a}\dot{\mathrm{z}}$ dego jej argumentu $x$ spefnia równośč

$f(x)+(f(x))^{2}+\ldots=x^{3},$

której lewa strona jest sumą nieskończonego ciqgu geometrycznego. Wyz-

nacz dziedzinę funkcji $f$ oraz jej ekstrema lokalne.

7. $\mathrm{W}$ równoległobok ABCD, w którym kolejnośč wierzchołków ABCD jest

przeciwna do ruchu wskazówek zegara, $\mathrm{m}\mathrm{o}\dot{\mathrm{z}}$ na wpisač okrąg. Majqc dane

wspófrzędne wierzchofków $A= (0,1) \mathrm{i}B= (\sqrt{3},0)$ oraz miarę $120^{o}$ kąta

wewnętrznego przy wierzcholku $D$, oblicz pole powierzchni równolegloboku

i napisz równanie okręgu weń wpisanego.
\end{document}
