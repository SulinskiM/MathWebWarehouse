\documentclass[a4paper,12pt]{article}
\usepackage{latexsym}
\usepackage{amsmath}
\usepackage{amssymb}
\usepackage{graphicx}
\usepackage{wrapfig}
\pagestyle{plain}
\usepackage{fancybox}
\usepackage{bm}

\begin{document}

AKADEMIA GÓRNICZO-HUTNICZA

im. Stanisława Staszica w Krakowie

OLIMPIADA O DIAMENTOWY INDEKS AGH'' 2020/21

MATEMATYKA - ETAP II

ZADANIA PO 10 PUNKTÓW

l. Udowodnij, $\dot{\mathrm{z}}\mathrm{e}\mathrm{k}\mathrm{a}\dot{\mathrm{z}}$ da liczba rzeczywista $a\neq 0$ spełnia nierównośč

$a^{2}+\displaystyle \frac{4}{a^{4}}\geq 3.$

Podaj liczby, dla których prawdziwa jest równośč.

2. $\mathrm{W}$ kwadracie ABCD punkt $K$ jest środkiem boku $AB$. Przez punkt $K$

poprowadzona jest prosta prostopadfa do prostej $KC$, która przecina bok

$AD$ w punkcie $R$. Wykaz, $\dot{\mathrm{z}}\mathrm{e}\mathrm{k}\mathrm{a}\mathrm{t}\mathrm{y}\triangleleft KCB\mathrm{i}\triangleleft KCR$ mają równe miary.

3. $\mathrm{W}$ ciągu geometrycznym $(\alpha_{n})$ dane są $a_{3}=\displaystyle \frac{1}{4}$ oraz

$a_{10}=\displaystyle \log_{2}\cos\frac{47}{12}\pi+\log_{2}\sin(-\frac{37}{12}\pi)$

Oblicz $a_{17}.$

4. $\mathrm{Z}$ pnia drzewa w kształcie walca o średnicy podstawy $D$ i długości $H$ wy-

cieto cztery przystajqce bale w kształcie walca o długości $H$ i najwiekszej

$\mathrm{m}\mathrm{o}\dot{\mathrm{z}}$ liwej objętości. Oblicz objętośč pozostalej części pnia.

ZADANIA PO 20 PUNKTÓW

5. Napisz równania asymptot wykresu funkcji f danej wzorem

$f(x)=\displaystyle \frac{x^{2}+4x}{x^{3}+4x^{2}+4x+16}.$

Wyznacz najmniejszą i największą wartośč funkcji $f$ w przedziale $\langle$1; $5\rangle.$

6. Znajd $\acute{\mathrm{z}}$ równanie okręgu, na którym $\mathrm{l}\mathrm{e}\dot{\mathrm{z}}\mathrm{q}$ punkty $A=(8,8), B=(-8,-4)$

$\mathrm{i}C= (6,-6)$. Napisz równania stycznych do tego okręgu, prostopadfych

do prostej $4x+3y-6=0.$

7. Rozwazmy zbiór $S$ wszystkich funkcji danych wzorem $f(x)=\alpha x^{2}+bx+c,$

gdzie $\alpha, b, c$ sq liczbami calkowitymi spełniającymi nierównośč

$4^{x+1}-33\cdot 2^{x}+8\leq 0.$

Wyznacz liczby elementów podzbiorów $P, Q, R$ zbioru $S$, gdzie $P$ jest zbio-

rem funkcji parzystych, $Q$ jest zbiorem funkcji, których wykres przechodzi

przez punkt $(0,3)$, a $R$ jest zbiorem funkcji rosnących w $\mathbb{R}.$


\end{document}