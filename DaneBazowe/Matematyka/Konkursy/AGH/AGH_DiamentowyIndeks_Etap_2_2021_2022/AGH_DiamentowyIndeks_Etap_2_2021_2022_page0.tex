\documentclass[a4paper,12pt]{article}
\usepackage{latexsym}
\usepackage{amsmath}
\usepackage{amssymb}
\usepackage{graphicx}
\usepackage{wrapfig}
\pagestyle{plain}
\usepackage{fancybox}
\usepackage{bm}

\begin{document}

AKADEMIA GÓRNICZO-HUTNICZA

im. Stanisława Staszica w Krakowie

OLIMPIADA O DIAMENTOWY INDEKS AGH'' 2021/22

MATEMATYKA - ETAP II

ZADANIA PO 10 PUNKTÓW

l. Jadąc z prędkościq 30 $\mathrm{k}\mathrm{m}/$godz. spóz/nimy się na spotkanie 10 minut, ajadąc

z prędkością 60 $\mathrm{k}\mathrm{m}/$godz. będziemy 10 minut za wcześnie. Zjaką prędkości$\Phi$

powinniśmy jechač, aby przybyč punktualnie?

2. Bok kwadratu jest przeciwprostokątną AB trójkąta prostokątnego, którego

trzeci wierzchofek $C\mathrm{l}\mathrm{e}\dot{\mathrm{z}}\mathrm{y}$ na zewnątrz kwadratu. Niech $S$ będzie środkiem

kwadratu. Uzasadnij, $\dot{\mathrm{z}}\mathrm{e}$ kąty $ACS\mathrm{i}BCS$ sq przystające.

3. Dane są trzy kolejne liczby cafkowite. Udowodnij, $\dot{\mathrm{z}}\mathrm{e}$ kwadraty dokladnie

dwóch z nich dają resztę l z dzielenia przez 3.

4. Liczby 2 $\log_{2}x, \log_{2}2x, \log_{2}(x-4)$ są trzema początkowymi wyrazami

ciągu arytmetycznego. Znajd $\acute{\mathrm{z}}$ setny wyraz tego ciągu.

ZADANIA PO 20 PUNKTÓW

5. Wyznacz dziedzinę i zbiór wartości funkcji $f, \mathrm{j}\mathrm{e}\dot{\mathrm{z}}$ eli dla $\mathrm{k}\mathrm{a}\dot{\mathrm{z}}$ dego $x$ nalezącego

do jej dziedziny spelniona jest równośč

$f(x)+(f(x))^{2}+(f(x))^{3}+\displaystyle \ldots=-\frac{1}{5}(x^{2}+1)$

6. Dane $\mathrm{s}\Phi$ dodatnie liczby cafkowite $n$ oraz $k$, przy czym $k \leq n$. Ze zbioru

liczb $\{$1, 2, $\ldots, n\}$ losujemy kolejno bez zwracania $k$ liczb, otrzymując w ten

sposób ciag $k$-wyrazowy. Oblicz prawdopodobieństwa zdarzeń

$A$: liczba $k$ nie występuje w tym ciągu,

$B$: $k$ jest ostatnim wyrazem ciągu,

$C$: $\mathrm{c}\mathrm{i}_{\Phi \mathrm{g}}$ jest monotoniczny i $k$ jest jego wyrazem.

7. Punkty $A = (0,7), B = (1,0), C = (-3,-2)$ są wierzcholkami trójkąta.

Znajd $\acute{\mathrm{z}}$ równanie okręgu opisanego na tym trójkącie i równanie jego obra-

zu w symetrii środkowej względem punktu $A$. Napisz równania wszystkich

prostych stycznych jednocześnie do obu tych okręgów.
\end{document}
