\documentclass[a4paper,12pt]{article}
\usepackage{latexsym}
\usepackage{amsmath}
\usepackage{amssymb}
\usepackage{graphicx}
\usepackage{wrapfig}
\pagestyle{plain}
\usepackage{fancybox}
\usepackage{bm}

\begin{document}

AKADEMIA GÓRNICZO-HUTNICZA

im. Stanislawa Staszica w Krakowie

OLIMPIADA O DIAMENTOWY INDEKS AGH'' 2022/23

MATEMATYKA - ETAP II

ZADANIA PO 10 PUNKTÓW

l. Udowodnij, $\dot{\mathrm{z}}\mathrm{e}$ istnieje tylko jedna trójka liczb pierwszych, które sa trzema

kolejnymi wyrazami $\mathrm{c}\mathrm{i}_{\Phi \mathrm{g}}\mathrm{u}$ arytmetycznego o róznicy 2.

2. Najkrótszy bok trapezu prostokątnego opisanego na okręgu o promieniu $r$ ma

długośč $\displaystyle \frac{5}{3}r$. Oblicz pole trapezu.

3. Dwa miasta $A \mathrm{i} B$ są odlegfe od siebie o 960 km. $\mathrm{Z}$ tych miast wyjechafy

naprzeciw siebie dwa pociągi, przy czym pociąg z miasta $B$ wyjecha12 godziny

póz$\acute{}$niej i jechal z predkością o 20 $\mathrm{k}\mathrm{m}/$godz. wiekszą $\mathrm{n}\mathrm{i}\dot{\mathrm{z}}$ pociąg z miasta $A.$

$\mathrm{p}_{\mathrm{o}\mathrm{c}\mathrm{i}_{\Phi \mathrm{g}\mathrm{i}}}$ te minęly się dokfadnie w pofowie drogi. Podaj prędkośč pociągu,

który wyruszył z miasta $A.$

4. Spośród wierzcholów $n$-kąta foremnego losujemy trzy. Oblicz prawdopodobień-

stwo $p_{n}$ wylosowania wierzchofków trójkąta prostokątnego. Zbadaj, czy ciąg

$(p_{n})$ ma granicę.

ZADANIA PO 20 PUNKTÓW

5. Oblicz sumę długości wszystkich krawedzi ostrosłupa prawidłowego czworo-

katnego wpisanego w sfere o promieniu $R$, który ma największą objętośč.

6. Znajd $\acute{\mathrm{z}}$ wszystkie pierwiastki równania

które spelniają nierównośč

$3^{\cos^{2}2x}+3^{\sin^{2}2x}=4,$

$\log_{x}(x+2)<2.$

7. Wielomian $P$ dany wzorem

$P(x)=x^{3}+mx+162$

ma pierwiastek wielokrotny. Uzasadnij, $\dot{\mathrm{z}}\mathrm{e}$ nie jest to jedyny pierwiastek tego

wielomianu. Wykaz, $\dot{\mathrm{z}}\mathrm{e}P(-\sqrt[6]{9})$ jest liczbą cafkowitą.


\end{document}