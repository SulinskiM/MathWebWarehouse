\documentclass[a4paper,12pt]{article}
\usepackage{latexsym}
\usepackage{amsmath}
\usepackage{amssymb}
\usepackage{graphicx}
\usepackage{wrapfig}
\pagestyle{plain}
\usepackage{fancybox}
\usepackage{bm}

\begin{document}

AKADEMIA GÓRNICZO-HUTNICZA im. Stanislawa Staszica

w Krakowie

OLIMPIADA,, O DIAMENTOWY INDEKS AGH'' 2007/8

MATEMATYKA- ETAP III

ZADANIA PO 10 PUNKTÓW

1. $\mathrm{W}$ trapezie o polu $P$ stosunek dlugości podstaw jest równy $k>1$. Oblicz pola dwóch

trójkatów, na które ten trapez dzieli jego przekatna.

2. Rozwia $\dot{\mathrm{Z}}$ nierównośč

0, $1^{\mathrm{x}}\cdot 0, 1^{x^{3}}\cdot 0, 1^{x^{5}}$

$>$ -$\sqrt{}$3 11000000.

3. PoIowę drogi kierowca jechaI autostrad4 z prędkościa l20 $\mathrm{k}\mathrm{m}/\mathrm{h}$, a drugą poIowę

na drogach lokalnych ze średni4 predkością 60 $\mathrm{k}\mathrm{m}/\mathrm{h}$. Oblicz średnia predkośč calej

podróz $\mathrm{y}.$

4. Znajd $\acute{\mathrm{z}}$ równania okrQgów o promieniu 3 stycznych jednocześnie do osi $OX$ i do

prostej $12x+5y=0.$

ZADANIA PO 20 PUNKTÓW

5. Na czworościanie foremnym opisano walec w ten sposób, $\dot{\mathrm{z}}\mathrm{e}$ dwie krawędzie czwo-

rościanu $1\mathrm{e}\dot{\mathrm{z}}4^{\mathrm{C}\mathrm{e}}$ na prostych skośnych sa średnicami podstaw walca. Oblicz stosunek

pola powierzchni sfery opisanej na walcu do pola powierzchni sfery wpisanej w czwo-

rościan.

6. Dla jakich wartości parametru $m$ dokladnie jeden pierwiastek równania

$(m-2)9^{x}+(m+1)3^{x}-m=0$

jest mniejszy od 2?

7. Ze zbioru \{l, 2, $\ldots$, 1000\} 1osujemy trójelementowy podzbiór $T = \{p,q,r\}$, przy

czym prawdopodobieństwo wylosowania $\mathrm{k}\mathrm{a}\dot{\mathrm{z}}$ dego podzbioru jest jednakowe.

a) Oblicz prawdopodobieństwo, $\dot{\mathrm{z}}\mathrm{e}$ iloczyn $pqr$ jest podzielny przez 3.

b) Niech $\varphi$ będzie funkcjq przyporządkowującq $\mathrm{k}\mathrm{a}\dot{\mathrm{z}}$ demu wylosowanemu podzbiorowi

$T,$,element pośredni'' (tzn. jeśli $p<q<r$, to $\varphi(T)=q$). Jaka wartośč funkcji $\varphi$

jest najbardziej prawdopodobna?


\end{document}