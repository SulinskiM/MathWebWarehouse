\documentclass[a4paper,12pt]{article}
\usepackage{latexsym}
\usepackage{amsmath}
\usepackage{amssymb}
\usepackage{graphicx}
\usepackage{wrapfig}
\pagestyle{plain}
\usepackage{fancybox}
\usepackage{bm}

\begin{document}

AKADEMIA GÓRNICZO-HUTNICZA im. StanisIawa Staszica

w Krakowie

OLIMPIADA,, O DIAMENTOWY INDEKS AGH'' 2008/9

MATEMATYKA- ETAP III

ZADANIA PO 10 PUNKTÓW

1. Znajd $\acute{\mathrm{z}}$ wspólrzędne obrazu punktu $C = (20,25)$ w symetrii osiowej względem

prostej przechodzqcej przez punkty $A=(6,2)\mathrm{i}B=(3,-4).$

2. Wyznacz dziedzinę funkcji danej wzorem

$f(x)=\log_{2}(x^{3}-4x^{2}-3x+18).$

3. Oblicz granicę ciagu

$\displaystyle \lim_{n\rightarrow\infty}(n-\sqrt{n^{2}+5n}).$

4. Znajd $\acute{\mathrm{z}}1\mathrm{i}\mathrm{c}\mathrm{z}\mathrm{b}_{Q}$, której 59\% stanowi okresowy ufamek dziesiętny 2, 6(81).

ZADANIA PO 20 PUNKTÓW

5. Ze zbioru $\{1,2,3, ,2n-1,2n\}$, gdzie $n$ jest ustalona liczba naturalna, losujemy

ze zwracaniem dwie liczby $x\mathrm{i}y$. Oblicz prawdopodobieństwa zdarzeń

$A$ : $x=y$; $B$ : iloczyn $xy$ jest liczba parzystą; $C$ : $\displaystyle \frac{x}{y}\in(0;1).$

6. $\mathrm{W}$ ostrosIupie prawidtowym trójkatnym o wysokości $h$ krawędz/ boczna jest nachy-

lona do krawędzi podstawy pod kątem $\alpha$. Oblicz promień kuli wpisanej w ten

ostrosIup. Jakie wartości $\mathrm{m}\mathrm{o}\dot{\mathrm{z}}\mathrm{e}$ przyjmowač miara kata $\alpha$?

7. Dla jakich wartości parametru $m$ nierównośč

$(m^{2}-1)x^{2}+2(m-1)x+2>0$

jest spelniona dla $\mathrm{k}\mathrm{a}\dot{\mathrm{z}}$ dego $x \in 1R$? Czy istnieje takie $x$, aby dla $\mathrm{k}\mathrm{a}\dot{\mathrm{z}}$ dego $m \in 1R$

powyzsza nierównośč byla prawdziwa?
\end{document}
