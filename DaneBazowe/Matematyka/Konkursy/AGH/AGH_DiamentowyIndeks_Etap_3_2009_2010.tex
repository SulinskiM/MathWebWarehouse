\documentclass[a4paper,12pt]{article}
\usepackage{latexsym}
\usepackage{amsmath}
\usepackage{amssymb}
\usepackage{graphicx}
\usepackage{wrapfig}
\pagestyle{plain}
\usepackage{fancybox}
\usepackage{bm}

\begin{document}

AKADEMIA GÓRNICZO-HUTNICZA

im. StanisIawa Staszica w Krakowie

OLIMPIADA,, O DIAMENTOWY INDEKS AGH'' 2009/10

MATEMATYKA- ETAP III

ZADANIA PO 10 PUNKTÓW

1. Rozwia $\dot{\mathrm{Z}}$ równanie

$(x^{2}+1)^{\sin 2x+\cos 2x}=1.$

2. Jakie największe pole powierzchni bocznej $\mathrm{m}\mathrm{o}\dot{\mathrm{z}}\mathrm{e}$ mieč stozek obrotowy, w którym

obwód przekroju osiowego ma dlugośč $C$ ?

3. Zbadaj wzajemne polozenie okręgów:

01 : $x^{2}+y^{2}-4x-2y-45=0,$

02 : $x^{2}+y^{2}+2y-97=0.$

4. Oblicz granicę ciagu, którego n-ty wyraz jest równy

$a_{n}=\displaystyle \frac{1}{3^{n}+2^{n}}+\frac{3}{3^{n}+2^{n}}+\frac{9}{3^{n}+2^{n}}+\ldots+\frac{3^{n}}{3^{n}+2^{n}}.$

ZADANIA PO 20 PUNKTÓW

5. $\mathrm{W}$ ostroslupie prawidIowym ośmiokatnym krawęd $\acute{\mathrm{z}}$ podstawy ma dIugośč $a$, a $\mathrm{k}\mathrm{a}\mathrm{t}$

nachylenia ściany bocznej do podstawy ma miarę $\alpha$. Wysokośč ostroslupa podzielono

na $n$ odcinków równej dlugości i przez punkty podziaIu poprowadzono plaszczyzny

równolegIe do podstawy, dzieląc w ten sposób ostrosIup na $n,$, warstw'' ZakIadając,

$\dot{\mathrm{z}}\mathrm{e}n\geq 3$, oblicz objętośč drugiej warstwy (liczac od podstawy).

6. Dany jest $n$-elementowy zbiór $S$. Ze zbioru wszystkich podzbiorów zbioru $S$ losu-

jemy kolejno ze zwracaniem dwa zbiory (prawdopodobieństwo wylosowania $\mathrm{k}\mathrm{a}\dot{\mathrm{z}}$ dego

zbioru jest jednakowe). Oblicz prawdopodobieństwa zdarzeń

$A$: przynajmniej jeden z wylosowanych zbiorów jest zbiorem pustym,

$B$: $\mathrm{k}\mathrm{a}\dot{\mathrm{z}}\mathrm{d}\mathrm{y}$ z wylosowanych zbiorów ma dokIadnie $n-1$ elementów,

$C$: wylosowane zbiory sa rozlaczne.

Wyniki zapisz w najprostszej postaci.

7. Dla jakich wartości parametru $p$ równanie

$(p-3)(9-4\sqrt{5})^{x}-(2p+6)(\sqrt{5}-2)^{x}+p+2=0$

ma dokladnie jeden pierwiastek?


\end{document}