\documentclass[a4paper,12pt]{article}
\usepackage{latexsym}
\usepackage{amsmath}
\usepackage{amssymb}
\usepackage{graphicx}
\usepackage{wrapfig}
\pagestyle{plain}
\usepackage{fancybox}
\usepackage{bm}

\begin{document}

AKADEMIA GÓRNICZO-HUTNICZA

im. StanisIawa Staszica w Krakowie

OLIMPIADA,, O DIAMENTOWY INDEKS AGH'' 2010/11

MATEMATYKA - ETAP III

ZADANIA PO 10 PUNKTÓW

l. Dany jest $n$-elementowy zbiór $X$ oraz jego $k$-elementowy podzbiór $S$. Ze zbioru $X$

wybieramy losowo $m$ elementów, tworzac zbiór $B$. Zakladając, $\dot{\mathrm{z}}\mathrm{e}k>0, m>0$ oraz

$m+k\leq n+1$, oblicz prawdopodobieństwo, $\dot{\mathrm{z}}\mathrm{e}$ zbiory $B\mathrm{i}S$ będa miaIy dokIadnie

jeden element wspólny.

2. Oblicz sumę wszystkich dwucyfrowych liczb naturalnych niepodzielnych przez 7.

3. Wyznacz dziedzinę funkcji $f$ danej wzorem

$f(x)=\displaystyle \frac{x^{3}+8}{x^{4}+2x^{3}+2x^{2}+4x}.$

Zbadaj granice funkcji $f$ w punktach nienalezacych do dziedziny.

4. Suma dwóch nieujemnych liczb rzeczywistych $x, y$ jest równa dodatniej liczbie $\alpha.$

Jaka najmniejsza wartośč $\mathrm{m}\mathrm{o}\dot{\mathrm{z}}\mathrm{e}$ mieč suma kwadratów liczb $x\mathrm{i}y$?

ZADANIA PO 20 PUNKTÓW

5. W prawidlowy graniastostup sześciokatny wpisano sferę (styczna do wszystkich ścian

bocznych i do obu podstaw). Oblicz stosunek pola powierzchni tej sfery do pola

powierzchni sfery opisanej na graniastostupie.

6. Dla jakich wartości parametru p równanie

$\displaystyle \frac{\log(px^{2})}{\log(x+1)}=2$

ma dokladnie jedno rozwiązanie?

7. Znajd $\acute{\mathrm{z}}$ równania stycznych do okręgu $C$ o równaniu

$x^{2}+y^{2}+6x-4y-12=0$

przechodzących przez punkt $P=(\displaystyle \frac{16}{3},2)$. Oblicz dlugośč promienia okręgu stycznego

do obydwu prostych i do okregu $C.$


\end{document}