\documentclass[a4paper,12pt]{article}
\usepackage{latexsym}
\usepackage{amsmath}
\usepackage{amssymb}
\usepackage{graphicx}
\usepackage{wrapfig}
\pagestyle{plain}
\usepackage{fancybox}
\usepackage{bm}

\begin{document}

AKADEMIA GÓRNICZO-HUTNICZA

im. StanisIawa Staszica w Krakowie

OLIMPIADA,, O DIAMENTOWY INDEKS AGH'' 2011/12

MATEMATYKA - ETAP III

ZADANIA PO 10 PUNKTÓW

l. Niech $a\mathrm{i}b$ będą dwiema liczbami rzeczywistymi, przy czym $a>b$. Udowodnij, $\dot{\mathrm{z}}\mathrm{e}$

$a^{3}-b^{3}\geq\alpha b^{2}-a^{2}b.$

2. Ile dzielników w zbiorze liczb naturalnych ma liczba 4$\cdot 5\cdot 6\cdot 7\cdot 8$ ?

3. Suma czterech początkowych wyrazów rosnącego ciagu arytmetycznego $(a_{n})$ jest

równa 0, a suma ich kwadratów wynosi 80. Znajd $\acute{\mathrm{z}}$ wzór na n-ty wyraz tego ciqgu.

4. Rozwiąz nierównośč

$1+\sqrt{x+5}>x.$

ZADANIA PO 20 PUNKTÓW

5. Ze zbioru $L=\{-2,-1,0,1,2\}$ losujemy ze zwracaniem dwie liczby $x, y$. Następnie

powtarzamy to losowanie dotad, $\mathrm{a}\dot{\mathrm{z}}$ otrzymamy punkt $(x,y)$ nalezacy do zbioru

$S=\{(x,y):|x|+|y|\leq 2\}.$

Oblicz prawdopodobieństwa zdarzeń:

A- bedziemy losowač doktadnie cztery razy,

B- liczba losowań będzie parzysta.

6. Dla jakich $m$ równanie

$\log_{3}(x-m)+\log_{3}x=\log_{3}(3x-4)$

ma dokladnie jedno rozwiazanie w zbiorze liczb rzeczywistych?

7. Prosta $2x+y-13=0$ zawiera bok $AB$ trójkąta $ABC$, prosta $x-y-5=0$ zawiera

bok $BC$, a prosta $3x-y-7=0$ zawiera dwusieczna kąta $ACB.$ Znajd $\acute{\mathrm{z}}$ wierzcholki

tego trójkata i oblicz jego pole.
\end{document}
