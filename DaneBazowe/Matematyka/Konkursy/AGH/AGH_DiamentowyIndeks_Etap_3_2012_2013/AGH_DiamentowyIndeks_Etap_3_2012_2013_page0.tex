\documentclass[a4paper,12pt]{article}
\usepackage{latexsym}
\usepackage{amsmath}
\usepackage{amssymb}
\usepackage{graphicx}
\usepackage{wrapfig}
\pagestyle{plain}
\usepackage{fancybox}
\usepackage{bm}

\begin{document}

AKADEMIA GÓRNICZO-HUTNICZA

im. Stanislawa Staszica w Krakowie

OLIMPIADA,, O DIAMENTOWY INDEKS AGH'' 2012/13

MATEMATYKA- ETAP III

ZADANIA PO 10 PUNKTÓW

l. Udowodnij, $\dot{\mathrm{z}}\mathrm{e}$ zbiór $S= \{6n+3:n\in 1N\}$, gdzie $mT$ jest zbiorem

wszystkich liczb naturalnych, zawiera nieskończenie wiele kwadratów

liczb calkowitych.

2. Rozwiąz równanie 4 $\cos^{2}2x=3.$

3. Sfera $S_{1}$ jest wpisana w sześcian, sfera $S_{2}$ jest styczna do wszystkich

krawędzi tego sześcianu, a sfera $S_{3}$ jest opisana na tym sześcianie.

Sprawd $\acute{\mathrm{z}}$, czy pola tych sfer tworza ciag geometryczny lub arytmety-

czny.

4. Rozwiqz nierównośč $\sqrt{x^{2}-16x+64}+x\leq 7+\sqrt{x^{2}+6x+9}.$

ZADANIA PO 20 PUNKTÓW

5. Wykaz, $\dot{\mathrm{z}}\mathrm{e}$ niezaleznie od wartości parametru $m$ równanie

$x^{3}-(m+1)x^{2}+(m+3)x-3=0$

ma pierwiastek calkowity. Dla jakich $m$ wszystkie pierwiastki rzeczy-

wiste tego równania sa calkowite?

6. Rzucamy $n$ razy sześcienną kością do gry. Oblicz prawdopodobieństwa

zdarzeń:

$A$: ani razu nie wypadta szóstka,

$B$: parzysta liczba oczek wypadla więcej razy $\mathrm{n}\mathrm{i}\dot{\mathrm{z}}$ nieparzysta,

$C$: suma wyrzuconych oczek jest równa $6n-2.$

7. Rozwiąz nierównośč

$3-\log_{0,5}x-\log_{0,5}^{2}x-\log_{0,5}^{3}x-\ldots\geq 4\log_{0,5}x.$
\end{document}
