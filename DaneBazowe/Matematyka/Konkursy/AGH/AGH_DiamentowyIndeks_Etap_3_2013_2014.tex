\documentclass[a4paper,12pt]{article}
\usepackage{latexsym}
\usepackage{amsmath}
\usepackage{amssymb}
\usepackage{graphicx}
\usepackage{wrapfig}
\pagestyle{plain}
\usepackage{fancybox}
\usepackage{bm}

\begin{document}

AKADEMIA GÓRNICZO-HUTNICZA

im. StanisIawa Staszica w Krakowie

OLIMPIADA,, O DIAMENTOWY INDEKS AGH'' 2013/14

MATEMATYKA- ETAP III

ZADANIA PO 10 PUNKTÓW

1. Rozwia $\dot{\mathrm{Z}}$ równanie

$(x^{2}+\displaystyle \frac{1}{2})^{\cos 2x}(x^{2}+\frac{1}{2})^{\sin 2x}=1.$

2. Rzucono trzy razy sześciennq kostkq do gry. Oblicz prawdopodobieństwo, $\dot{\mathrm{z}}\mathrm{e}$ suma

wyrzuconych oczek jest mniejsza $\mathrm{n}\mathrm{i}\dot{\mathrm{z}}$ sześč.

3. Po zmieszaniu roztworów soli o stęzeniach 8\% oraz 20\% otrzymano l2 litrów roz-

tworu o stęzeniu 16\%. Ob1icz objętości zmieszanych roztworów.

4. Rozwia $\dot{\mathrm{Z}}$ nierównośč

$3x^{2}+6x^{3}+12x^{4}+\ldots\leq 1.$

ZADANIA PO 20 PUNKTÓW

5. Wyznacz zbiór wszystkich liczb rzeczywistych $p$, dla których pierwiastki $x_{1}$ i $x_{2}$

równania

speIniaja nierównośč

$x+1=\displaystyle \frac{px}{p-1}+\frac{p+1}{x}$

$\displaystyle \frac{1}{x_{1}}+\frac{1}{x_{2}}\leq 2p+1.$

6. Dwie ściany ostroslupa trójkatnego są trójkątami równobocznymi o boku dlugości $a$

i dwie są trójkatami prostokatnymi. Oblicz pole powierzchni i objętośč ostroslupa.

7. Oblicz promień mniejszego z dwóch okręgów stycznych w punkcie $M(2,1)$ do prostej

$x-7y+5=0$ i jednocześnie stycznych do prostej $x+y+13=0$. Napisz równania

wszystkich okręgów o tym promieniu stycznych jednocześnie do obydwu prostych.


\end{document}