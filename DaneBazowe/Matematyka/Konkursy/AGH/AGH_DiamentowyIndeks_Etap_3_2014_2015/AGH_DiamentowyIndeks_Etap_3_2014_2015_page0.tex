\documentclass[a4paper,12pt]{article}
\usepackage{latexsym}
\usepackage{amsmath}
\usepackage{amssymb}
\usepackage{graphicx}
\usepackage{wrapfig}
\pagestyle{plain}
\usepackage{fancybox}
\usepackage{bm}

\begin{document}

AKADEMIA GÓRNICZO-HUTNICZA

im. StanisIawa Staszica w Krakowie

OLIMPIADA,, O DIAMENTOWY INDEKS AGH'' 2014/15

MATEMATYKA- ETAP III

ZADANIA PO 10 PUNKTÓW

1. Znajd $\acute{\mathrm{z}}$ wszystkie liczby naturalne mniejsze $\mathrm{n}\mathrm{i}\dot{\mathrm{z}}7$, przez które podzielna jest liczba

$L=3^{2016}+4.$

2. Rozwiqz równanie

2 $\cos^{3}x+5\sin^{2}x-11\cos x-9=0.$

3. Oblicz pole równolegtoboku zbudowanego na wektorach $\vec{u}=[3,-4]\mathrm{i}\vec{v}=[4$, 4$].$

4. Rozwiqz nierównośč

$25\cdot 0,04^{x}-0,2^{x^{2}-2}\leq 0.$

ZADANIA PO 20 PUNKTÓW

5. Wartośč funkcji $g$ w punkcie $m$ jest równa sumie pierwiastków równania

$|mx^{2}-2x|=m,$

przy czym $\mathrm{k}\mathrm{a}\dot{\mathrm{z}}\mathrm{d}\mathrm{y}$ pierwiastek jest w tej sumie uwzględniany tylko raz niezaleznie od

jego krotności. Znajd $\acute{\mathrm{z}}$ funkcję $g:m\rightarrow g(m)$ i naszkicuj jej wykres.

6. Ze zbioru $\{$1, 2, $\ldots, n\}$ losujemy kolejno bez zwracania $k$ liczb, otrzymujqc ciqg

$(a_{1},a_{2},\ldots,a_{k})$. Wiedzqc, $\dot{\mathrm{z}}\mathrm{e}3\leq k\leq n$, oblicz prawdopodobieństwa zdarzeń:

A- $a_{k}$ jest najwiQkszq liczbq wśród wylosowanych;

B- $a_{k}$ jest podzielna przez 3;

$\displaystyle \mathrm{C}-a_{1}+a_{2}+\ldots+a_{k}>\frac{1}{2}k(k+1).$

7. Wyznacz wysokośč stozka o najmniejszej objętości opisanego na kuli o promieniu

$R=2$ cm.
\end{document}
