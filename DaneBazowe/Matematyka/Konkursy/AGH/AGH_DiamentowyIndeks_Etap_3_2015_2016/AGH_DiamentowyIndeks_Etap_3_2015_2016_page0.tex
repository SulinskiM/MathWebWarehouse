\documentclass[a4paper,12pt]{article}
\usepackage{latexsym}
\usepackage{amsmath}
\usepackage{amssymb}
\usepackage{graphicx}
\usepackage{wrapfig}
\pagestyle{plain}
\usepackage{fancybox}
\usepackage{bm}

\begin{document}

AKADEMIA GÓRNICZO-HUTNICZA

im. StanisIawa Staszica w Krakowie

OLIMPIADA,, O DIAMENTOWY INDEKS AGH'' 2015/16

MATEMATYKA- ETAP III

ZADANIA PO 10 PUNKTÓW

1. Znajd $\acute{\mathrm{z}}$ wszystkie pary liczb calkowitych $(x,y)$ spelniajqcych równanie

$(x-2y-1)(x+2y+1)=3.$

2. Przy okragIym stole z l0 ponumerowanymi krzesIami siada 5 kobiet i 5 męzczyzn,

wybierajac miejsca w sposób przypadkowy. Jakie jest prawdopodobieństwo, $\dot{\mathrm{z}}\mathrm{e}$ choč

jedna osoba usiadzie obok osoby tej samej plci?

3. Rozwia $\dot{\mathrm{Z}}$ równanie

$|\cos x|^{2\cos x+1}=1.$

4. Dla jakich $\alpha$ liczby

$\log_{0,5}a^{2},$

$3+\log_{0,5}a, -1-\log_{0,5}2a^{3}$

sa kolejnymi wyrazami ciągu arytmetycznego?

ZADANIA PO 20 PUNKTÓW

5. Na p{\it l}aszczy $\acute{\mathrm{z}}\mathrm{n}\mathrm{i}\mathrm{e}$ dane sa punkty $A=(2,1), B=(-2,7), C=(-6,5).$

a) Znajd $\acute{\mathrm{z}}$ wspólrzędne punktu $D$, dla którego czworokat ABCD (w tej kolejności

wierzchoIków) jest równoleglobokiem.

b) Figura $F$ jest suma prostej $AB$ i prostej $CD$. Napisz równania wszystkich osi

symetrii figury $F.$

c) Znajd $\acute{\mathrm{z}}$ obraz figury $F$ w jednokIadności o środku w punkcie $A$ i skali równej 3.

6. Funkcja $f$ przyporządkowuje $\mathrm{k}\mathrm{a}\dot{\mathrm{z}}$ dej liczbie rzeczywistej $m$ liczbe pierwiastków rów-

nania

$|$--4{\it xx}2$++$22$|=${\it m}.

Naszkicuj wykres funkcji $f.$

7. Trójkąt równoramienny o obwodzie 36 cm obraca się wokól prostej zawierajacej

podstawę trójkąta. Jakie powinny byč wymiary tego trójkata, aby objQtośč powstalej

bryly byla największa?
\end{document}
