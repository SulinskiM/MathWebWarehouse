\documentclass[a4paper,12pt]{article}
\usepackage{latexsym}
\usepackage{amsmath}
\usepackage{amssymb}
\usepackage{graphicx}
\usepackage{wrapfig}
\pagestyle{plain}
\usepackage{fancybox}
\usepackage{bm}

\begin{document}

AKADEMIA GÓRNICZO-HUTNICZA

im. Stanislawa Staszica w Krakowie

OLIMPIADA O DIAMENTOWY INDEKS AGH'' 2016/17

MATEMATYKA- ETAP III

ZADANIA PO 10 PUNKTÓW

l. Udowodnij, $\dot{\mathrm{z}}\mathrm{e}$ dla dowolnych dwóch dodatnich liczb rzeczywistych $a, b$ spefniona

jest nierównośč

$\sqrt{}ab\geq$ ---{\it a}1$+$2-{\it b}$1^{\cdot}$

2. Oblicz $\displaystyle \log_{8}\cos\frac{11}{6}\pi$ -log8 tg $(-\displaystyle \frac{17}{3}\pi).$

3. Funkcja $f$ dana wzorem

$f(x)=\displaystyle \{\frac{x^{m}-1}{x-1,a_{m}}$

dla

dla

$x\neq 1$

$x=1$

jest ciągfa w punkcie $x=1$. Wyznacz $a_{2}, a_{6}$ oraz $a_{m}$ dla dowolnej dodatniej liczby

całkowitej $m.$

4. Zbadaj, czy trójkąt o wierzcholkach $A = (-2,0), B = (1,-1), C = (0,7)$ jest

ostrokątny, prostokątny, czy rozwartokątny.

ZADANIA PO 20 PUNKTÓW

5. Liczba $a$ jest losowo wybrana spośród wszystkich siedmiocyfrowych liczb natu-

ralnych. Oblicz prawdopodobieństwa zdarzeń:

$A$: przynajmniej jedna z cyfr 0, 11ub 2 występuje w zapisie 1iczby $a$;

$B$: kolejne cyfry liczby $a$ opisują siedmiowyrazowy ciąg arytmetyczny;

$C$: kolejne cyfry liczby $a$ opisuja siedmiowyrazowy ciąg malejący.

6. $\mathrm{W}$ trapez prostokqtny o najkrótszym boku długości $a$ wpisany jest okrąg o pro-

mieniu $\displaystyle \frac{2}{3}a$. Oblicz pole trapezu i stosunek dfugości jego przekątnych.

7. Dany jest uklad równań

$\left\{\begin{array}{l}
(p+2)x+4y\\
3x+2y
\end{array}\right.$

$2p+4$

$4$

a) Dla jakich $p$ układ ma dokfadnie jedno rozwiqzanie $(x,y)$ ?

b) Jaką największ$\Phi$ wartośč, ajaką najmniejszq, $\mathrm{o}\mathrm{s}\mathrm{i}_{\Phi \mathrm{g}}\mathrm{a}$ iloczyn $xy$ dla $ p\in\langle 0;3\rangle$?


\end{document}