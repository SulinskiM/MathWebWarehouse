\documentclass[a4paper,12pt]{article}
\usepackage{latexsym}
\usepackage{amsmath}
\usepackage{amssymb}
\usepackage{graphicx}
\usepackage{wrapfig}
\pagestyle{plain}
\usepackage{fancybox}
\usepackage{bm}

\begin{document}

AKADEMIA GÓRNICZO-HUTNICZA

im. Stanislawa Staszica w Krakowie

OLIMPIADA O DIAMENTOWY INDEKS AGH'' 2017/18

MATEMATYKA- ETAP III

ZADANIA PO 10 PUNKTÓW

l. W układzie współrzędnych narysuj zbiór

\{({\it x, y})

$: x^{3}-y^{3}\geq xy^{2}-x^{2}y\}.$

2. Na ile sposobów $\mathrm{m}\mathrm{o}\dot{\mathrm{z}}$ emy $n$ początkowych liczb naturalnych 1, 2, $\ldots, n$

ustawič w ciąg, tak by choč jedna liczba parzysta nie miała dwóch

sąsiednich wyrazów nieparzystych?

3. Napisz równanie obrazu okręgu $x^{2}+y^{2}+4x-6y+8=0$ przez translację

o wektor $\vec{v}=[2,-4]$. Czy te dwa okręgi mają punkty wspólne?

4. $\mathrm{Z}$ punktu $P$ na okręgu o promieniu $r = 4$ cm poprowadzono cięci-

wę $PQ$ nachylonq do średnicy $PR$ pod kątem $\alpha = 15^{o}$ Oblicz pole

trójkąta $PQR.$

ZADANIA PO 20 PUNKTÓW

5. Znajd $\acute{\mathrm{z}}$ sumę wszystkich pierwiastków równania

$\sqrt{3}|\mathrm{c}\mathrm{t}\mathrm{g}x+\mathrm{t}\mathrm{g}x|=4$

spełniających nierównośč

$(\sqrt{2-\sqrt{3}})^{x}+(\sqrt{2+\sqrt{3}})^{x}\leq 4.$

6. Jaką największą objętośč $\mathrm{m}\mathrm{o}\dot{\mathrm{z}}\mathrm{e}$ mieč stozek wpisany w kulę o promie-

niu $R$?

7. Rzucamy sześcienną kostką do momentu uzyskania,,szóstki'' Niech $k$

będzie dowolną, dodatnią liczbą calkowitą. Oblicz prawdopodobień-

stwo, $\dot{\mathrm{z}}\mathrm{e}$ liczba rzutów będzie

$A$: równa $k, B$: mniejsza $\mathrm{n}\mathrm{i}\dot{\mathrm{z}}k, C$: parzysta.
\end{document}
