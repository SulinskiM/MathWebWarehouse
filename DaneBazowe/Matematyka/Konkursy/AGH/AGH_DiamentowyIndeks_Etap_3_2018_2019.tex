\documentclass[a4paper,12pt]{article}
\usepackage{latexsym}
\usepackage{amsmath}
\usepackage{amssymb}
\usepackage{graphicx}
\usepackage{wrapfig}
\pagestyle{plain}
\usepackage{fancybox}
\usepackage{bm}

\begin{document}

AKADEMIA GÓRNICZO-HUTNICZA

im. Stanislawa Staszica w Krakowie

OLIMPIADA O DIAMENTOWY INDEKS AGH'' 2018/19

MATEMATYKA- ETAP III

ZADANIA PO 10 PUNKTÓW

l. Ze zbioru dziesięciu kolejnych liczb naturalnych usunięto jedną z nich.

Suma pozostałych liczb wynosi 2019. Znajd $\acute{\mathrm{z}}$ sumę wszystkich dziesię-

ciu liczb.

2. Ostrosłup podzielono na dwie bryły płaszczyzną równoległą do pod-

stawy i dzielącą jego wysokośč na dwa przystające odcinki. Jaki pro-

cent objętości ostrosłupa stanowi objętośč większej z tych brył?

3. Wyznacz liczbę $p$, dla której

$\displaystyle \lim_{n\rightarrow\infty}(n-\sqrt[3]{n^{3}+pn^{2}})=-2.$

4. Oblicz dlugości przekątnych równoległoboku o bokach dfugości 3 $\mathrm{i}5,$

przy czym sinus kąta wewnętrznego jest równy 0,8.

ZADANIA PO 20 PUNKTÓW

5. Wyznacz zbiór $(A\backslash B)\cap C$, gdzie

$A=\{x\in \mathbb{R}:\log_{\frac{1}{4}}(2^{x}+10)\leq 0,5+2\log_{\frac{1}{4}}(2^{x}-2)\},$

$B=\{x\in \mathbb{R}:x+1\leq\sqrt{x+3}\},$

$C=\{n\in \mathbb{N}:\sqrt{n}\left(\begin{array}{lll}
n & + & 2\\
 & 2 & 
\end{array}\right)>3^{n-1}\}.$

6. Losowo dzielimy $n$-elementowy zbiór $X$ na dwa zbiory $S\mathrm{i}X\backslash S$, przy

czym dla dowolnego $a\in X$ prawdopodobieństwo, $\dot{\mathrm{z}}\mathrm{e}a$ zostanie wylo-

sowany do zbioru $S$ wynosi $\displaystyle \frac{1}{2}$. Oblicz prawdopodobieństwa zdarzeń

$A$ : zbiór $S$ ma dokładnie $k$ elementów;

{\it B} : $\dot{\mathrm{z}}$ aden ze zbiorów $S\mathrm{i}X\backslash S$ nie jest pusty;

$C$ : zbiór $S$ zawiera więcej elementów $\mathrm{n}\mathrm{i}\dot{\mathrm{z}}$ zbiór $X\backslash S.$

7. Spośród wszystkich trójkątów prostokątnych o przeciwprostokatnej

długości $c$ wskazač ten, dla którego największa jest objętośč bryly

obrotowej, powstałej z obrotu tego trójkąta wokóf przyprostokątnej,

a) która jest krótsza.

b) która nie jest krótsza.


\end{document}