\documentclass[a4paper,12pt]{article}
\usepackage{latexsym}
\usepackage{amsmath}
\usepackage{amssymb}
\usepackage{graphicx}
\usepackage{wrapfig}
\pagestyle{plain}
\usepackage{fancybox}
\usepackage{bm}

\begin{document}

AKADEMIA GÓRNICZO-HUTNICZA

im. Stanisława Staszica w Krakowie

OLIMPIADA O DIAMENTOWY INDEKS AGH'' 2020/21

MATEMATYKA- ETAP III

ZADANIA PO 10 PUNKTÓW

1. $\mathrm{W}$ czworokącie ABCD kąt wewnętrzny przy wierzcholku $A$ jest kątem pros-

tym. Dfugości boków są równe $|AB| = |AD| = 15, |BC| =3, |CD| =21.$

Oblicz pole tego czworokąta.

2. Dany jest ciąg $(a_{n})$, którego n-ty wyraz jest równy

$a_{n}=40-6n.$

Oblicz sumę tych 20 początkowych wyrazów ciqgu, które są ujemne i po-

dzielne przez 8.

3. Rozwiqz równanie

$2^{\cos^{2}x}+2^{\sin^{2}x}=3.$

4. $\mathrm{W}$ wycieczce bierze udzial $2n$ osób. $K\mathrm{a}\dot{\mathrm{z}}$ da z nich ma wśród uczestników

wycieczki co najmniej $n-1$ innych osób pochodzących z tego samego miasta

co ona. $\mathrm{Z}$ ilu miast pochodzą uczestnicy wycieczki? $\mathrm{W}\mathrm{k}\mathrm{a}\dot{\mathrm{z}}$ dym $\mathrm{m}\mathrm{o}\dot{\mathrm{z}}$ liwym

przypadku podaj liczby uczestników z poszczególnych miast.

ZADANIA PO 20 PUNKTÓW

5. $\mathrm{W}$ urnie znajdują się 2 ku1e białe i 10 czerwonych.

a) Losujemy ze zwracaniem 2 ku1e. Ob1icz prawdopodobieństwo, $\dot{\mathrm{z}}\mathrm{e}$ wylo-

sujemy kule o róznych kolorach.

b) Losujemy bez zwracania $k$ kul. Wyznacz najmniejszą wartośč $k$, dla

której prawdopodobieństwo wylosowania co najmniej jednej kuli biafej jest

większe od 0, 5.

6. Dla jakich wartości parametru $p$ równanie

$\displaystyle \frac{\log_{3}(px+p)}{\log_{3}(3+x)}=2$

ma dokladnie jedno rozwiązanie?

7. $\mathrm{W}$ stozku o promieniu podstawy $R$ i wysokości $H$ zawartyjest graniastoslup

prawidlowy czworokątny tak, $\dot{\mathrm{z}}\mathrm{e}$ jego podstawa zawiera się w podstawie

stozka. Jaką największą objętośč $\mathrm{m}\mathrm{o}\dot{\mathrm{z}}\mathrm{e}$ mieč ten graniastosłup?
\end{document}
