\documentclass[a4paper,12pt]{article}
\usepackage{latexsym}
\usepackage{amsmath}
\usepackage{amssymb}
\usepackage{graphicx}
\usepackage{wrapfig}
\pagestyle{plain}
\usepackage{fancybox}
\usepackage{bm}

\begin{document}

AKADEMIA GÓRNICZO-HUTNICZA

im. Stanislawa Staszica w Krakowie

OLIMPIADA O DIAMENTOWY INDEKS AGH'' 2021/22

MATEMATYKA- ETAP III

ZADANIA PO 10 PUNKTÓW

1. Rozwia $\dot{\mathrm{z}}$ nierównośč

$|\log_{0,5}(2-x)|\geq 1.$

2. Oblicz sume wszystkich liczb dwucyfrowych podzielnych przez 6 lub przez 8.

3. Znajd $\acute{\mathrm{z}}$ równania prostych stycznych do krzywej $y=x^{2}+\displaystyle \frac{1}{x}$ i prostopadfych

do prostej $4x+15y-3=0.$

4. Punkt $S$ jest środkiem wysokości czworościanu foremnego ABCD opuszczonej

z wierzchołka $D$. Wyznacz miarę kąta $ASB.$

ZADANIA PO 20 PUNKTÓW

5. Niech $n$ będzie dowolną dodatnią liczbą cafkowitą. Ze zbioru dodatnich liczb

całkowitych mniejszych od $3n$ losujemy ze zwracaniem trzy liczby. Oblicz praw-

dopodobieństwo, $\dot{\mathrm{z}}\mathrm{e}$ dokładnie jedna z tych liczb jest równa $n.$

Niech $p_{n}$ oznacza prawdopodobieństwo, $\dot{\mathrm{z}}\mathrm{e}$ iloczyn tych trzech liczb jest po-

dzielny przez 3. Ob1icz

$\displaystyle \lim_{n\rightarrow\infty}p_{n}.$

6. Dla jakich wartości parametru $m$ równanie

$m\cdot 2^{x}+(m+3)2^{-x}=4$

ma dokładnie jedno rozwiqzanie?

7. Prosta przechodząca przez punkt $M=(3,1)$ ogranicza wraz z dodatnimi pól-

osiami układu współrzędnych $XOY$ trójkąt o najmniejszym polu. Wokół któ-

rego boku nalezy obracač ten trójkąt, aby otrzymana bryła obrotowa miała

$\mathrm{n}\mathrm{a}\mathrm{j}\mathrm{m}\mathrm{n}\mathrm{i}\mathrm{e}\mathrm{j}\mathrm{s}\mathrm{z}\Phi$ objętośč? Podaj tę objętośč.
\end{document}
