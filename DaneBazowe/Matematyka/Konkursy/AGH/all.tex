\documentclass[a4paper,12pt]{article}
\usepackage{latexsym}
\usepackage{amsmath}
\usepackage{amssymb}
\usepackage{graphicx}
\usepackage{wrapfig}
\pagestyle{plain}
\usepackage{fancybox}
\usepackage{bm}

\begin{document}

AKADEMIA GÓRNICZO-HUTNICZA im. Stanisfawa Staszica

w Krakowie

OLIMPIADA,, O DIAMENTOWY INDEKS AGH'' 2007/8

MATEMATYKA- ETAP I

ZADANIA PO 10 PUNKTÓW

1. $\mathrm{W}$ trójkącie równoramiennym dane są dlugości podstawy $a$ i ramienia

$b$. Oblicz dlugośč wysokości tego trójkąta opuszczonej na jego ramię.

2. Rozwia $\dot{\mathrm{Z}}$ nierównośč

$|2x^{4}-17|<15.$

3. Oblicz granicę ciągu, którego n-ty wyraz jest równy

$a_{n}=n^{3}-\sqrt{n^{6}-5n^{3}}.$

4. Na ile sposobów $\mathrm{m}\mathrm{o}\dot{\mathrm{z}}$ na rozmieścič $k$ kul $(k \geq 4, \mathrm{k}\mathrm{a}\dot{\mathrm{z}}$ da kula innego

koloru) $\mathrm{w}k$ ponumerowanych pudetkach tak, aby

a) $\dot{\mathrm{z}}$ adne pudelko nie byIo puste?

b) dokladnie jedno pudelko bylo puste?

c) doktadnie $k-2$ pudeIka byly puste?

ZADANIA PO 20 PUNKTÓW

5. DIugośč wysokości ostroslupa prawidtowego trójkatnego jest równa dlu-

gości krawędzi podstawy. Oblicz stosunek objętości kuli wpisanej w ten

ostroslup do objętości kuli opisanej na nim.

6. Wyznacz liczbę rozwiązań równania

$(m-3)x^{4}-3(m-3)x^{2}+m+2=0$

w zalezności od parametru $m.$

7. RozIóz na czynniki wielomian

$W(x)=x^{4}+6x^{3}+11x^{2}+6x.$

Udowodnij, $\dot{\mathrm{z}}\mathrm{e}$ wartośč $W(n)$ tego wielomianu dla dowolnej liczby natu-

ralnej $n$ jest podzielna przez 12. D1ajakich natura1nych $n$ liczba $W(n)$

nie jest podzielna przez 60?





AKADEMIA GÓRNICZO-HUTNICZA im. Stanisfawa Staszica

w Krakowie

OLIMPIADA,, O DIAMENTOWY INDEKS AGH'' 2008/9

MATEMATYKA- ETAP I

ZADANIA PO 10 PUNKTÓW

l. Ile jest czwórek $(x,y,z,t)$ liczb calkowitych dodatnich spelniajacych

równanie $xy+yz+zt+tx=2008$?

2. Rozwia $\dot{\mathrm{Z}}$ równanie $\sin 4x+\sqrt{3}\sin 2x=0.$

3. Znajd $\acute{\mathrm{z}}$ liczbę $c$, dla której granica ciagu o wyrazie ogólnym

{\it an}$=$--$\sqrt{}$35{\it nn}$++${\it c}-9{\it n}2-{\it n}2{\it c}

jest równa 2.

4. Ile jest czterocyfrowych liczb naturalnych, które nie sa podzielne ani

przez 9, ani przez 12?

ZADANIA PO 20 PUNKTÓW

5. Punkty $A= (2,-2) \mathrm{i}B= (8,4)$ sa końcami podstawy trójkata rów-

noramiennego $ABC$. WierzchoIek $C\mathrm{l}\mathrm{e}\dot{\mathrm{z}}\mathrm{y}$ na prostej $x-3y+34=0.$

Znajd $\acute{\mathrm{z}}$ równanie okręgu wpisanego w trójkqt $ABC.$

6. Dla jakich wartości parametru $p\in R$ jeden z pierwiastków równania

$(12p+6)x^{2}+16px+9p=0$

jest sinusem, a drugi kosinusem tego samego kata rozwartego?

7. Cztery kule, z których trzy mają promień $r$, a czwarta $R$, ulozono na

stole w piramidę w taki sposób, $\dot{\mathrm{z}}\mathrm{e}\mathrm{k}\mathrm{a}\dot{\mathrm{z}}$ da kula jest styczna do trzech

pozostaIych, przy czym kule przystajqce tworza podstawę piramidy.

Oblicz największa odleglośč punktu kuli o promieniu $R$ od stotu. Podaj

warunek, jaki musza spelniač promienie, aby ustawienie piramidy bylo

$\mathrm{m}\mathrm{o}\dot{\mathrm{z}}$ liwe.






AKADEMIA GÓRNICZO-HUTNICZA

im. StanisIawa Staszica w Krakowie

OLIMPIADA,, O DIAMENTOWY INDEKS AGH'' 2009/10

MATEMATYKA- ETAP I

ZADANIA PO 10 PUNKTÓW

l. Na pólsferze o promieniu $R\mathrm{l}\mathrm{e}\dot{\mathrm{z}}$ a dwa styczne do siebie okręgi o promie-

niu $r$. Wyznacz największa odlegtośč między dwoma punktami naleza-

cymi do tych okręgów.

2. Rozwia $\dot{\mathrm{Z}}$ nierównośč

$\sqrt{x^{2}+2x+1}-2x>0.$

3. Kran $A$ napeInia basen woda w ciagu 10 godzin, a kran $B$ w ciagu 15

godzin. $\mathrm{W}$ ciagu ilu godzin napeIniony zostanie basen, $\mathrm{j}\mathrm{e}\dot{\mathrm{z}}$ eli oba krany

bedą dzialač jednocześnie?

4. Znajd $\acute{\mathrm{z}}$ wszystkie rozwiqzania równania

4 $\cos 2x\sin 2x+1=0$

nalezące do przedziaIu $(-\pi;\pi).$

ZADANIA PO 20 PUNKTÓW

5. Wyznacz zbiory $A\cap B$ oraz $A\backslash B$, gdzie

$A=\{x\in lR:x^{4}+x^{3}-3x^{2}-x+2\geq 0\},$

$B=\{x\in lR:\log_{0,5}(x+3)\geq\log_{0,5}(6-2x)\}.$

6. Oblicz pole trójkqta, mając dane dwie proste $4x+5y+17 = 0$

$\mathrm{i} x-3y=0$, zawierajace środkowe trójkqta, oraz jeden jego wierz-

cholek $A=(-1,-6).$

7. Ile jest równań postaci

$x^{2}-px+q=0,$

które maja dwa pierwiastki mniejsze od 7, przy czym 1iczby $p\mathrm{i}q$ są

calkowite i dodatnie.






AKADEMIA GÓRNICZO-HUTNICZA

im. StanisIawa Staszica w Krakowie

OLIMPIADA,, O DIAMENTOWY INDEKS AGH'' 2010/11

MATEMATYKA- ETAP I

ZADANIA PO 10 PUNKTÓW

l. Kula $K$ jest wpisana w sześcian. Kula $K'$ jest styczna do trzech ścian

tego sześcianu i do kuli $K$. Oblicz stosunek promienia kuli $K$ do

promienia kuli $K'.$

2. Suma kwadratów trzech dodatnich liczb caIkowitych $a, b, c$ jest równa

2010. Ile jest wśród nich liczb parzystych?

3. Znajd $\acute{\mathrm{z}}$ liczbę $p$, dla której granica ciagu o wyrazie ogólnym

$a_{n}=\sqrt[3]{n^{3}+n^{2}+9pn}-\sqrt[3]{n^{3}-5pn^{2}}$

jest równa 2.

4. Punkty $A= (-2,3) \mathrm{i}B= (1,2)$ sa wierzcholkami trójkata $T$. Wyz-

nacz wspótrzędne trzeciego wierzchoIka wiedzac, $\dot{\mathrm{z}}\mathrm{e}$ pole trójkata $T$ jest

równe 3, a środek jego cięzkości $\mathrm{l}\mathrm{e}\dot{\mathrm{z}}\mathrm{y}$ na osi $OY.$

ZADANIA PO 20 PUNKTÓW

5. Liczba naturalna $a$ ma $2n$ cyfr, z których pierwsze $n$ cyfr to same

czwórki, a pozostale cyfry to ósemki. Udowodnij, $\dot{\mathrm{z}}\mathrm{e}\sqrt{a+1}$ jest liczba

naturalnq dla $\mathrm{k}\mathrm{a}\dot{\mathrm{z}}$ dego $n.$

6. $\mathrm{W}$ ukladzie wspólrzędnych na plaszczy $\acute{\mathrm{z}}\mathrm{n}\mathrm{i}\mathrm{e}$ narysuj zbiór

$A=\{(x,y):\log_{y}(8x+y-2-x^{2})\geq\log_{y}(8-x^{2}+8x-2y-y^{2})\}.$

7. Naszkicuj wykres funkcji $g:m\rightarrow g(m)$, która $\mathrm{k}\mathrm{a}\dot{\mathrm{z}}$ dej liczbie rzeczywis-

tej $m$ przyporzqdkowuje liczbę pierwiastków równania

$2^{2x+2}+4^{x}+4^{x-1}+\ldots=m+16^{x}$






AKADEMIA GÓRNICZO-HUTNICZA

im. StanisIawa Staszica w Krakowie

OLIMPIADA,, O DIAMENTOWY INDEKS AGH'' 2011/12

MATEMATYKA- ETAP I

ZADANIA PO 10 PUNKTÓW

l. Pary $(x,y)$ liczb calkowitych spelniające równanie

$xy^{2}-y^{3}-xy+x^{2}+5=0$

sq wspótrzędnymi wierzchoIków pewnego wielokata. Oblicz jego pole.

2. Oblicz sume $n$ poczatkowych wyrazów ciagu $(a_{n})$, w którym

$a_{1}=3, a_{2}=33, a_{3}=333, a_{4}=3333,$

3. $\mathrm{W}$ pólokrag o promieniu $R$ wpisano trapez, w którym ramie jest nachylone pod

katem $\alpha$ do podstawy będqcej średnica okręgu. Oblicz pole trapezu.

4. Rozwiąz równanie

$\displaystyle \lim_{n\rightarrow+\infty}(x+x^{3}+x^{5}+\ldots+x^{2n-1})=\frac{2}{3}.$

ZADANIA PO 20 PUNKTÓW

5. Wyznacz równanie krzywej będqcej zbiorem środków wszystkich cięciw paraboli

$y = x^{2}-2$ przechodzacych przez początek ukladu wspólrzędnych. Naszkicuj tę

krzywa.

6. Narysuj w ukIadzie wspóIrzędnych zbiór

$S=\{(x,y)$

$\log_{x}|y-2|>\log_{|y-2|}x\}.$

7. a) Zbadaj w zalezności od parametru $k$, ile rozwiązań ma uklad równań

$\left\{\begin{array}{l}
kx+(k+1)y=k-1\\
4x+(k+4)y=k.
\end{array}\right.$

b) Dla jakich wartości parametru $k$ ten uktad ma dokladnie jedno rozwiazanie

nalezace do wnętrza trójkata o wierzchoikach

$A=(0,0),$

$B=(\displaystyle \frac{2}{3},0), C=(0,2).$






AKADEMIA GÓRNICZO-HUTNICZA

im. StanisIawa Staszica w Krakowie

OLIMPIADA,, O DIAMENTOWY INDEKS AGH'' 2012/13

MATEMATYKA- ETAP I

ZADANIA PO 10 PUNKTÓW

l. Ile jest ciągów $(x_{1},x_{2},x_{3},x_{4})$ liczb calkowitych dodatnich spetniających

równanie $x_{1}+x_{2}+x_{3}+x_{4}=12$ ?

2. Dana jest funkcja

$f(x)=\displaystyle \frac{5-x}{2x+1}.$

Rozwia $\dot{\mathrm{z}}$ nierównośč $f(x+5)\geq f(x-3).$

3. Wyznacz dziedzinę i zbadaj parzystośč funkcji

$f(x)=(x^{2}+1)\displaystyle \frac{3^{2x}+3^{-2x}}{\sin^{2}2x+2}-x^{3}\log\frac{3x^{2}+5x+8}{3x^{2}-5x+8}.$

4. Znajd $\acute{\mathrm{z}}$ rzut równolegly punktu $A(1,-2)$ na prostą $x-y+3 = 0$

w kierunku wektora $\vec{v}=[1$, 2$].$

ZADANIA PO 20 PUNKTÓW

5. $\mathrm{W}$ prawidlowym ostrosIupie trójkqtnym miary katów nachylenia ściany

bocznej i krawędzi bocznej do podstawy ostrostupa wynoszą odpowied-

nio $\alpha \mathrm{i}\beta$. Oblicz stosunek objętości ostroslupa do objętości kuli wpisanej

w niego.

6. Naszkicuj wykres funkcji, która $\mathrm{k}\mathrm{a}\dot{\mathrm{z}}$ dej liczbie rzeczywistej $m$ przypo-

rządkowuje liczbQ $f(m)$ pierwiastków równania

$4^{|x|}+(m+1)2^{|x|+1}=5-m^{2}$

z niewiadoma $x.$

7. Ciag trzech liczb calkowitych $(a,b,c)$ jest ciagiem geometrycznym, któ-

rego iloraz jest liczbą calkowita. $\mathrm{J}\mathrm{e}\dot{\mathrm{z}}$ eli do najmniejszej z nich dodamy 9,

to otrzymamy trzy liczby, które odpowiednio uporządkowane utworzą

ciag arytmetyczny. Znajd $\acute{\mathrm{z}}$ wszystkie takie ciqgi $(a,b,c).$






AKADEMIA GÓRNICZO-HUTNICZA

im. StanisIawa Staszica w Krakowie

OLIMPIADA,, O DIAMENTOWY INDEKS AGH'' 2013/14

MATEMATYKA- ETAP I

ZADANIA PO 10 PUNKTÓW

l. Udowodnij, $\dot{\mathrm{z}}\mathrm{e}\dot{\mathrm{z}}$ aden element zbioru $S=\{6n+2:n\in \mathbb{N}\}$ nie jest kwadratem liczby

caIkowitej.

2. Rozwia $\dot{\mathrm{Z}}$ równanie

$ 5+\displaystyle \frac{x^{2}}{5}-\frac{x^{4}}{25}+\frac{x^{6}}{125}-\frac{x^{8}}{625}+\ldots = x^{2}+1$, (4),

w którym drugi skladnik prawej strony jest ulamkiem dziesiętnym okresowym.

3. Na ile sposobów $\mathrm{m}\mathrm{o}\dot{\mathrm{z}}$ na $n$ kul rozmieścič w $n$ pudelkach tak, $\dot{\mathrm{z}}$ eby dokladnie dwa

pudelka zostaly puste? Zalóz, $\dot{\mathrm{z}}\mathrm{e}n\geq 3$ oraz zarówno kule jak i pudelka sa między

sobą rozróznialne.

4. Sporzad $\acute{\mathrm{z}}$ wykres funkcji danej wzorem

$f(x)=5^{|\log_{0,2}x|}.$

ZADANIA PO 20 PUNKTÓW

5. Dany jest prawidlowy ostroslup czworokatny. Pole przekroju pIaszczyzna przecho-

dząc4 przez przekatna podstawy i równo1eg1q do krawędzi bocznej skośnej wzg1ędem

tej przekatnej jest równe $P$. Pole przekroju plaszczyzna przechodzaca przez środki

dwóch sąsiednich boków podstawy i środek wysokości ostroslupa wynosi $S$. Oblicz

iloraz $\displaystyle \frac{P}{S}.$

6. Dla jakich $ x\in (-\displaystyle \frac{\pi}{2};\frac{\pi}{2})$ liczby

tgx, l,

$\displaystyle \frac{\cos x}{1+\sin x}$

w podanej kolejności sa trzema poczatkowymi wyrazami rosnącego ciagu arytme-

tycznego $(a_{n})$ ? Dla dowolnego $n\in \mathbb{N}$ oblicz sumę $a_{n}+a_{n+1}+\ldots+a_{2n}.$

7. Rozwiąz w zalezności od parametru $p\in \mathbb{R}$ równanie

$(1-p)(|x+2|+|x|)=4-3p.$






AKADEMIA GÓRNICZO-HUTNICZA

im. StanisIawa Staszica w Krakowie

OLIMPIADA,, O DIAMENTOWY INDEKS AGH'' 2014/15

MATEMATYKA- ETAP I

ZADANIA PO 10 PUNKTÓW

l. Niech p będzie dowolna liczba pierwsza.

przez 30 nie jest 1iczba ztozona.

Udowodnij, $\dot{\mathrm{z}}\mathrm{e}$ reszta z dzielenia liczby $p$

2. Rozwia $\dot{\mathrm{Z}}$ równanie

$(\sqrt{5+2\sqrt{6}})^{x}+(\sqrt{5-2\sqrt{6}})^{x}=10.$

3. Oblicz granicę ciągu o wyrazie ogólnym

$a_{n}=\displaystyle \frac{3^{n+1}+2^{3+2n}}{2^{2n+1}+3^{n}}.$

4. Na ile sposobów $\mathrm{m}\mathrm{o}\dot{\mathrm{z}}$ na zbiór $\{$1, 2, $\ldots, n\}$, gdzie $n\geq 3$, podzielič na trzy niepuste

podzbiory?

ZADANIA PO 20 PUNKTÓW

5. Dla jakich wartości parametru $m$ nierównośč

$(m^{2}-1)\cdot 25^{x}-2(m-1)\cdot 5^{x}+2>0$

jest spelniona przez $\mathrm{k}\mathrm{a}\dot{\mathrm{z}}$ da liczbę rzeczywista $x$?

6. $\mathrm{W}$ sześcianie o krawędzi dlugości $a$ zawarte sa dwie sfery zewnętrznie styczne, przy

czym ich środki $\mathrm{l}\mathrm{e}\dot{\mathrm{z}}$ ą na $\mathrm{P}^{\mathrm{r}\mathrm{z}\mathrm{e}\mathrm{k}}4^{\mathrm{t}\mathrm{n}\mathrm{e}\mathrm{j}}$ sześcianu i $\mathrm{k}\mathrm{a}\dot{\mathrm{z}}$ da z nich jest styczna przynajmniej

do trzech ścian sześcianu. Oblicz promienie tych sfer, dla których suma ich pól

powierzchni jest a) największa, b) najmniejsza.

7. Wyznacz równania stycznych do okręgu

$x^{2}+y^{2}+4x+3=0$

poprowadzonych z punktu $M = (1,0)$. Jaka krzywa stanowi zbiór wszystkich

środków cięciw tego okregu wyznaczonych przez proste przechodzące przez punkt

$M$? Napisz jej równanie.






AKADEMIA GÓRNICZO-HUTNICZA

im. StanisIawa Staszica w Krakowie

OLIMPIADA,, O DIAMENTOWY INDEKS AGH'' 2015/16

MATEMATYKA - ETAP I

ZADANIA PO 10 PUNKTÓW

1. Znajd $\acute{\mathrm{z}}$ wszystkie rosnace ciągi $(a_{n})$ o wyrazach calkowitych takie, $\dot{\mathrm{z}}\mathrm{e}$

$\alpha_{2}=2$ oraz $a_{mn}=a_{m}a_{n}$ dla wszystkich liczb naturalnych $m, n.$

2. Na ile sposobów $\mathrm{m}\mathrm{o}\dot{\mathrm{z}}$ na grupę $3k$ osób posadzič przy dwóch okrąglych

stoIach, $\mathrm{j}\mathrm{e}\dot{\mathrm{z}}$ eli przyjednym stolejest $2k$ ponumerowanych krzeseI, a przy

drugim $k$? A na ile sposobów $\mathrm{m}\mathrm{o}\dot{\mathrm{z}}$ na to zrobič tak, by ustalone dwie

osoby siedzialy obok siebie, $\mathrm{j}\mathrm{e}\dot{\mathrm{z}}$ eli $k\geq 2?.$

3. Rozwiqz nierównośč $3^{x}-2^{x}>3^{x-2}$

4. Oblicz granicę ciągu

$\displaystyle \lim_{n\rightarrow\infty}(2n-\sqrt[3]{8n^{3}-2n^{2}}).$

ZADANIA PO 20 PUNKTÓW

5. Dla jakich wartości parametru p równanie

$\cos^{3}x+p\cos x+p+1=0$

ma $\mathrm{d}\mathrm{o}\mathrm{k}\ddagger$adnie trzy rozwiązania w przedziale $\langle 0;2\pi\rangle$ ?

6. Napisz równanie okręgu opisanego na trójkacie o wierzchotkach $A =$

$(5,-4), B = (6,-1), C= (-2,3)$. Zbadaj wzajemne potozenie tego

okręgu oraz jego obrazu w symetrii osiowej względem prostej

$3x+4y+26=0.$

7. Krawędz$\acute{}$ podstawy ostroslupa prawidlowego trójkqtnego ma dlugośč

$a= 12$ cm, a sinus kqta między ścianami bocznymi wynosi $\displaystyle \frac{2\sqrt{2}}{3}$. Os-

trostup przecięto plaszczyzna przechodzacą przez jeden z wierzcholków

podstawy i dzielaca przeciwlegla ścianę boczna na dwie figury o równych

polach. Oblicz objętości bryl, które powstaly w wyniku przecięcia os-

troslupa ta pIaszczyzna.






AKADEMIA GÓRNICZO-HUTNICZA

im. Stanislawa Staszica w Krakowie

OLIMPIADA O DIAMENTOWY INDEKS AGH'' 2016/17

MATEMATYKA- ETAP I

ZADANIA PO 10 PUNKTÓW

l. Udowodnij, $\dot{\mathrm{z}}\mathrm{e}$ jedyną liczbą $\mathrm{p}\mathrm{i}\mathrm{e}\mathrm{r}\mathrm{w}\mathrm{s}\mathrm{z}\Phi p$, taką $\dot{\mathrm{z}}\mathrm{e}$ liczba $p^{2}+2\mathrm{t}\mathrm{e}\dot{\mathrm{z}}$ jest

pierwsza, jest $p=3.$

2. Dane są punkty $A = (-1,-2), B = (3,1), C = (1,4)$. Prosta $l$ jest

równoległa do prostej $AC$ i dzieli trójkąt $ABC$ na dwie figury o równych

polach. Znajd $\acute{\mathrm{z}}$ równanie prostej $l.$

3. $\mathrm{R}\mathrm{o}\mathrm{z}\mathrm{w}\mathrm{i}_{\Phi}\dot{\mathrm{z}}$ równanie

4 $\sin^{2}x+8\sin^{2}x\cos x=2\cos x+1.$

4. Niech $a\mathrm{i}b$ będą dowolnymi liczbami rzeczywistymi. Wykaz$\cdot, \dot{\mathrm{z}}\mathrm{e}\mathrm{j}\mathrm{e}\dot{\mathrm{z}}$ eli

$\alpha<b$, to

$a^{3}-b^{3}<a^{2}b-ab^{2}$

ZADANIA PO 20 PUNKTÓW

5. Wykaz$\cdot, \dot{\mathrm{z}}\mathrm{e}\mathrm{j}\mathrm{e}\dot{\mathrm{z}}$ eli liczba $m$ spelnia nierównośč

$(1+\displaystyle \frac{1}{2m})\log_{0,5}3-\log_{0,5}(27+3^{\frac{1}{m}})\leq 2,$

to $x^{2}+mx+1>0$ dla $\mathrm{k}\mathrm{a}\dot{\mathrm{z}}$ dej liczby rzeczywistej $x.$

6. Nieskończony ciąg $(a_{n})$ dany jest wzorem $a_{n}=1+2+\ldots+n.$

a) Znajd $\acute{\mathrm{z}}$ wszystkie cyfry jedności wyrazów tego ciągu w zapisie dzie-

siętnym. Udowodnij, $\dot{\mathrm{z}}\mathrm{e}$ znalezione rozwiązanie jest poprawne.

b) Wyznacz granice ciagu $(b_{n})$, gdzie

$b_{n}=\displaystyle \frac{a_{(n-1)^{2}}}{(a_{3n}-a_{2n})^{2}}.$

7. Cztery kule ojednakowym promieniu $a$ sa parami zewnętrznie styczne.

Znajd $\acute{\mathrm{z}}$ promienie dwóch kul, z których $\mathrm{k}\mathrm{a}\dot{\mathrm{z}}$ dajest styczna do wszystkich

czterech kul.






AKADEMIA GÓRNICZO-HUTNICZA

im. Stanislawa Staszica w Krakowie

OLIMPIADA O DIAMENTOWY INDEKS AGH'' 2017/18

MATEMATYKA- ETAP I

ZADANIA PO 10 PUNKTÓW

l. Udowodnij, $\dot{\mathrm{z}}\mathrm{e}$ spośród dowolnych pięciu punktów na pfaszczy $\acute{\mathrm{z}}\mathrm{n}\mathrm{i}\mathrm{e},$

z których $\dot{\mathrm{z}}$ adne trzy nie lezą na jednej prostej, $\mathrm{m}\mathrm{o}\dot{\mathrm{z}}$ na wybrač trzy

punkty, które $\mathrm{s}\Phi$ wierzcholkami trójkąta $\mathrm{r}\mathrm{o}\mathrm{z}\mathrm{w}\mathrm{a}\mathrm{r}\mathrm{t}\mathrm{o}\mathrm{k}_{\Phi}$tnego.

2. Ilejest trójek $(x_{1},x_{2},x_{3})$ liczb całkowitych niedodatnich spelniajqcych

równanie $x_{1}+x_{2}+x_{3}+37=0$?

3. Do zbiornika, w którym znajdowało $\mathrm{s}\mathrm{i}\mathrm{e}p_{0}$ hl wody, pierwszego dnia

dolano 70 h1 wody, po czym $\mathrm{k}\mathrm{a}\dot{\mathrm{z}}$ dego dnia dolewano o 7 h1 wody więcej

$\mathrm{n}\mathrm{i}\dot{\mathrm{z}}$ dnia poprzedniego. Jednocześnie codziennie ze zbiornika ubywafo

170 hl wody. Jaka powinna byč początkowa ilośč $p_{0}$ wody w zbiorniku,

aby nigdy nie brakfo w nim wody? Którego dnia w zbiorniku byfo

najmniej wody?

4. Stopień wielomianu $W(x)$ jest równy 2015. $\mathrm{W}\mathrm{i}\mathrm{e}\mathrm{d}\mathrm{z}\Phi^{\mathrm{C}}, \displaystyle \dot{\mathrm{z}}\mathrm{e}W(n)=\frac{1}{n}$ dla

$n=1$, 2, $\ldots$, 2016, oblicz $W$ (2017).

ZADANIA PO 20 PUNKTÓW

5. Dane jest równanie $(m+1)x^{2}-2(m-3)x+m+1=0$. Dla jakich

wartości parametru $m$

a) liczba llez $\mathrm{y}$ między sumą róznych pierwiastków równania a sumą

ich kwadratów?

b) wartośč bezwzględna przynajmniej jednego pierwiastka równania

jest mniejsza od 0,9?

6. Znajd $\acute{\mathrm{z}}$ sumę dlugości wszystkich przedziafów zawartych w $\langle 0;2\pi\rangle,$

w których spełniona jest nierównośč

$|\displaystyle \mathrm{c}\mathrm{t}\mathrm{g}2x-\mathrm{t}\mathrm{g}2x|\geq\frac{2}{\sqrt{3}}.$

7. $\mathrm{Z}$ wierzchołka $O$ paraboli $y^{2}=3x$ poprowadzono dwie proste wzajem-

nie prostopadłe, przecinajqce parabolę w punktach $M \mathrm{i} N.$ Znajd $\acute{\mathrm{z}}$

równanie (we wspófrzędnych kartezjańskich) zbioru środków cięzkości

wszystkich trójkątów $OMN.$






AKADEMIA GÓRNICZO-HUTNICZA

im. Stanislawa Staszica w Krakowie

OLIMPIADA O DIAMENTOWY INDEKS AGH'' 2018/19

MATEMATYKA- ETAP I

ZADANIA PO 10 PUNKTÓW

l. Dwa okręgi o promieniach $r\mathrm{i}R$, gdzie $r<R$, sq styczne zewnętrznie.

Wyznacz pole trójk$\Phi$ta ograniczonego ich wspólnymi stycznymi.

2. Udowodnij, $\dot{\mathrm{z}}\mathrm{e}$ suma $S$ nieskończonego ciągu geometrycznego $(a_{n}),$

w którym $a_{1}<0$, spefnia nierównośč

$S\leq 4\alpha_{2}.$

Kiedy spefniona jest równośč?

3. Znajd $\acute{\mathrm{z}}$ wszystkie liczby naturalne $n$, dla których liczba

$S_{n}=1!+2!+\ldots+n!$

jest kwadratem liczby cafkowitej.

4. Rozwiqz nierównośč

$2^{1+\log_{2}x}\geq x^{\frac{1}{4}(7+\log_{2}x)}.$

ZADANIA PO 20 PUNKTÓW

5. Dane są równania

$x^{2}-px+q=0$

oraz

$x^{2}-px-q=0,$

gdzie $p\mathrm{i}q$ sq liczbami naturalnymi. Wykaz$\cdot, \dot{\mathrm{z}}$ ejezeli obydwa równania

mają pierwiastki calkowite, to istnieją liczby naturalne $a, b$, takie $\dot{\mathrm{z}}\mathrm{e}$

$p^{2}=a^{2}+b^{2}$ Czy implikacja odwrotna jest prawdziwa?

6. Okrąg $0_{1}$ ma równanie $x^{2}+y^{2}+4x-8y+16=0$, a okrqg 02 równanie

$x^{2}+y^{2}-12x+8y+16 = 0$. Oblicz skalę jednokfadności i wspóf-

rzędne środka jednokładności, w której obrazem okręgu $0_{1}$ jest okrąg

02. Napisz równania prostych, które $\mathrm{s}\Phi$ jednocześnie styczne do obu

okręgów.

7. Oblicz objętośč i pole powierzchni bryly obrotowej powstalej z ob-

rotu sześciokqta foremnego o boku $a$ wokół prostej zawierajqcej bok

sześciokąta.






AKADEMIA GÓRNICZO-HUTNICZA

im. Stanisława Staszica w Krakowie

OLIMPIADA O DIAMENTOWY INDEKS AGH'' 2019/20

MATEMATYKA- ETAP I

ZADANIA PO 10 PUNKTÓW

l. Udowodnij, $\dot{\mathrm{z}}\mathrm{e}\mathrm{j}\mathrm{e}\dot{\mathrm{z}}$ eli czworokąt wypukły ma oś symetrii, to $\mathrm{m}\mathrm{o}\dot{\mathrm{z}}$ na na nim

opisač okrąg lub $\mathrm{m}\mathrm{o}\dot{\mathrm{z}}$ na weń okrąg wpisač.

2. Wyznacz dziedzinę funkcji danej wzorem

$f(x)=(32x^{2}+28x^{5}+4x^{8}-x^{11})^{-\frac{3}{4}}.$

3. $\mathrm{W}$ worku znajduje się 50 skarpet czarnych, 40 brązowych, 30 zie1onych i 20

niebieskich. Jaka jest najmniejsza liczba skarpet, które musimy wyjqč na

chybil trafif, aby mieč pewnośč, $\dot{\mathrm{z}}\mathrm{e}$ wśród nich znajdziemy jednokolorowe

pary skarpet dla 20 osób? Odpowied $\acute{\mathrm{z}}$ uzasadnij.

4. Rozwiąz nierównośč $|1-\displaystyle \log_{x}(x-\frac{1}{4})|\leq 1.$

ZADANIA PO 20 PUNKTÓW

5. Ile jest par $(a,b)$ liczb rzeczywistych, dla których układ równań

$\left\{\begin{array}{l}
ax+by+1=0\\
x^{2}+y^{2}=50
\end{array}\right.$

ma co najmniej jedno rozwiązanie, przy czym $\mathrm{k}\mathrm{a}\dot{\mathrm{z}}$ de jego rozwiązanie jest

parą $(x,y)$ liczb cafkowitych? Podaj przykfad pary $(a,b)$, dla której ukfad

ten ma dwa rozwiązania w liczbach calkowitych oraz przykład pary $(a,b),$

dla której ten uklad ma dokładnie jedno rozwiązanie i to rozwiązanie jest

parq liczb calkowitych.

6. Długości dwóch boków trójkąta wpisanego w okrąg o średnicy $D$ są odpo-

wiednio równe $\displaystyle \frac{3}{4}D$ oraz $\displaystyle \frac{\sqrt{3}}{2}D$. Oblicz długośč trzeciego boku.

7. Zbiór $S$ jest zbiorem wszystkich dodatnich liczb całkowitych $n$, dla których

istnieje permutacja $(a_{1},a_{2},\ldots,a_{n})$ liczb 1, 2, $\ldots, n$, taka $\dot{\mathrm{z}}\mathrm{e}a_{1}+a_{2}+\ldots+a_{k}$

jest wielokrotnością liczby $k$ dla $\mathrm{k}\mathrm{a}\dot{\mathrm{z}}$ dego $k=1$, 2, $\ldots, n$. Wykaz, $\dot{\mathrm{z}}\mathrm{e}\mathrm{k}\mathrm{a}\dot{\mathrm{z}}$ da

liczba naleząca do zbioru $S$ jest nieparzysta. Znajd $\acute{\mathrm{z}}$ dwie liczby tego zbioru.

Zbadaj, czy liczba 2019 na1ez $\mathrm{y}$ do zbioru $S.$






AKADEMIA GÓRNICZO-HUTNICZA im. Stanislawa Staszica

w Krakowie

OLIMPIADA,, O DIAMENTOWY INDEKS AGH'' 2007/8

MATEMATYKA- ETAP II

ZADANIA PO 10 PUNKTÓW

1. $\mathrm{Z}$ ustalonego zbioru $n$ liczb rzeczywistych losujemy kolejno $k$ liczb, otrzymujac ciag

róznowartościowy $(a_{1},\ldots,a_{k})$. ZakIadajqc, $\dot{\mathrm{z}}\mathrm{e}2\leq k\leq n$, oblicz prawdopodobieństwo,

$\dot{\mathrm{z}}\mathrm{e}$ ten ciag nie jest ciagiem rosnqcym.

2. Sprowad $\acute{\mathrm{z}}$ do najprostszej postaci (niezawierającej ujemnych wykladników, ani ulam-

ków piętrowych) wyrazenie

$(1-x^{-1})^{-2}-(1+x^{-1})^{-2}$

3. Cena akcji pewnej firmy spadIa o 60\%. $\mathrm{O}$ ile procent musi teraz wzrosnač cena tych

akcji, aby wrócila do poprzedniego poziomu?

4. Niech $P$ będzie izometrycznym przeksztatceniem ptaszczyzny, w którym obrazem

wykresu funkcji $f(x) = x^{2}$ jest wykres funkcji $g(x) = x^{2}+x+1.$ Znajd $\acute{\mathrm{z}}$ to

przeksztalcenie i podaj wzór funkcji, której wykres jest obrazem wykresu funkcji

$h(x)=\log_{2}x$ poprzez przeksztaIcenie $P.$

ZADANIA PO 20 PUNKTÓW

5. Oblicz sumę wszystkich pierwiastków równania

$4\cos^{2}x=3$

nalezacych do przedziaIu $(-8\pi;10\pi).$

6. Znajd $\acute{\mathrm{z}}$ równanie stycznej $l$ do okręgu $C$ o równaniu

$x^{2}+y^{2}-4x+6y-12=0$

w punkcie $A(6,0)$. Napisz równanie okręgu symetrycznego do okręgu $C$ względem

prostej $l.$

7. Dla jakich wartości parametru $p$ równanie

$(p-2)\cdot 9^{x}+(p+1)\cdot 3^{x}-p=0$

ma dwa rózne pierwiastki rzeczywiste?






AKADEMIA GÓRNICZO-HUTNICZA im. Stanislawa Staszica

w Krakowie

OLIMPIADA,, O DIAMENTOWY INDEKS AGH'' 2008/9

MATEMATYKA- ETAP II

ZADANIA PO 10 PUNKTÓW

l. Suma dwóch liczb rzeczywistych wynosi 6. Jaką największa wartośč $\mathrm{m}\mathrm{o}\dot{\mathrm{z}}\mathrm{e}$ mieč ich

iloczyn?

2. Sprowad $\acute{\mathrm{z}}$ do najprostszej postaci wyrazenie

$\displaystyle \frac{(a^{3}+b^{3})(a^{-1}-b^{-1})}{(a^{-1}+b^{-1})[(a-b)^{2}+ab]}.$

3. Odleglośč środka okręgu opisanego na trójkacie prostokqtnym od przyprostokatnych

wynosi odpowiednio piq. Oblicz obwód tego trójkata.

4. Oblicz $\displaystyle \log_{9}\cos\frac{11\pi}{6}$ -log9 $\displaystyle \sin\frac{29\pi}{6}.$

ZADANIA PO 20 PUNKTÓW

5. Dla jakich $m$ proste $mx+(m+1)y = 2 \mathrm{i} 4x+(m+4)y = 1$ przecinajq się

w punkcie lezącym wewnatrz II lub IV čwiartki ukIadu wspóIrzędnych?

6. $k$ pasazerów wsiada do pociqgu zlozonego z 3 wagonów, przy czym $\mathrm{k}\mathrm{a}\dot{\mathrm{z}}\mathrm{d}\mathrm{y}$ wybiera

wagon niezaleznie i z jednakowym prawdopodobieństwem $\displaystyle \frac{1}{3}$. ZakIadajac, $\dot{\mathrm{z}}\mathrm{e}k\geq 3,$

oblicz prawdopodobieństwo zdarzeń:

A- wszyscy wsiądą do jednego wagonu,

B- dokladnie jeden wagon będzie pusty,

{\it C}- $\dot{\mathrm{z}}$ aden wagon nie bedzie pusty.

7. Podstawa ostroslupa jest trójkat $ABC$, w którym bok $AB$ ma dIugośč $a$, a katy

wewnętrzne do niego przylegte maja miary $\beta \mathrm{i} \gamma.$ {\it K}rawęd $\acute{\mathrm{z}}$ boczna ostroslupa

wychodząca z wierzcholka $C$ jest prostopadla do podstawy i ma dlugośč $d$. Oblicz

objętości bryl, na które ten ostroslup dzieli ptaszczyzna równolegIa do podstawy

i odlegIa od niej o $\displaystyle \frac{d}{3}.$






AKADEMIA GÓRNICZO-HUTNICZA

im. StanisIawa Staszica w Krakowie

OLIMPIADA,, O DIAMENTOWY INDEKS AGH'' 2009/10

MATEMATYKA- ETAP II

ZADANIA PO 10 PUNKTÓW

l. Pole powierzchni bocznej stozka jest trzy razy większe od pola jego podstawy.

Ile razy objętośč stozka jest wieksza od objętości kuli wpisanej w ten stozek?

2. Dane sq funkcje $f(x) =2^{x+1}+5^{x-5}$

$g(\displaystyle \frac{x}{2})=f(x+3).$

$\mathrm{i} g(x) =25^{x}+4^{x}$

Rozwiąz równanie

3. Oblicz $\sin 2\alpha, \mathrm{j}\mathrm{e}\dot{\mathrm{z}}$ eli $\sin\alpha=0$, 75

$\mathrm{i} \displaystyle \alpha\in(\frac{\pi}{2};\pi).$

4. Wyznacz granice ciągu

$\displaystyle \lim_{n\rightarrow+\infty}(\sqrt[3]{n^{6}+5n^{4}}-n^{2}).$

ZADANIA PO 20 PUNKTÓW

5. Znajd $\acute{\mathrm{z}}$ równania stycznych do okręgu $x^{2}+y^{2}-8y+12=0$ przechodzacych

przez początek uktadu wspólrzednych. Znajd $\acute{\mathrm{z}}$ równania obrazów tego okręgu

i jednej z wyznaczonych stycznych w jednoktadności o środku w punkcie $S=$

$(1,2)$ i skali $k=-3.$

6. Funkcja $f$ spelnia dla $\mathrm{k}\mathrm{a}\dot{\mathrm{z}}$ dego $x$ nalezącego do jej dziedziny równanie

$ 1+f(x)+(f(x))^{2}+(f(x))^{3}+\ldots$

$= \displaystyle \frac{x}{2}+1,$

gdzie lewa strona jest sumą nieskończonego ciqgu geometrycznego. Wyznacz

dziedzinę i wzór funkcji $f$. Naszkicuj jej wykres.

7. Liczby l, 2, 3, $\ldots, n$, gdzie $n\geq 3$, losowo ustawiamy w ciag. Oblicz prawdo-

podobieństwa zdarzeń

$A$: liczba $n$ nie bQdzie ostatnim wyrazem tego ciagu;

$B$: liczby 1, 2, 3 wystqpią obok siebie w ko1ejności wzrastania;

$C$: iloczyn $\mathrm{k}\mathrm{a}\dot{\mathrm{z}}$ dej pary sasiednich wyrazów tego ciagu jest liczbą parzysta.

Wyniki zapisz w najprostszej postaci.






AKADEMIA GÓRNICZO-HUTNICZA

im. StanisIawa Staszica w Krakowie

OLIMPIADA,, O DIAMENTOWY INDEKS AGH'' 2010/11

MATEMATYKA- ETAP II

ZADANIA PO 10 PUNKTÓW

l. Dany jest ostrosIup prawidlowy trójkątny o krawedzi podstawy dlugości $a=1$

cm i wysokości opuszczonej na podstawę $H=2$ cm. Oblicz odlegIośč wierz-

choIka podstawy od przeciwleglej ściany.

2. Sprawd $\acute{\mathrm{z}}$, czy ciag

$\displaystyle \frac{1}{4}, \displaystyle \frac{2+\sqrt{3}}{2},$

$2+\sqrt{3}$

$2-\sqrt{3}$

jest $\mathrm{c}\mathrm{i}_{4\mathrm{g}}\mathrm{i}\mathrm{e}\mathrm{m}$ geometrycznym.

3. Dane sa punkty $A=(-1,-8)$ oraz $B=(5,4).$ Znajd $\acute{\mathrm{z}}$ taki punkt $C, \dot{\mathrm{z}}\mathrm{e}$

$A^{\rightarrow}C=5C^{\rightarrow}B.$

4. Rozwiąz równanie

$\log_{x-2}(x^{3}-x^{2}-7x+10)=2.$

ZADANIA PO 20 PUNKTÓW

5. Liczby l, 2, $\ldots, n$, gdzie $n>2$, przestawiamy w dowolny sposób. Oblicz praw-

dopodobieństwo nastepujacych zdarzeń:

$A-$ pierwszy wyraz otrzymanego ciągu będzie większy od ostatniego,

$B-$ liczby l $\mathrm{i}2$ nie będą ustawione obok siebie,

$C-$ liczby 1, 2 $\mathrm{i}3$ będa ustawione obok siebie w kolejności wzrastania.

6. Oblicz sumę trzydziestu największych ujemnych rozwiązań równania

$\cos 2x+\sin x=0.$

7. Zbadaj w zalezności od parametru k wzajemne potozenie prostych

$l_{1}$ : $kx+y=2,$

oraz

$l_{2}$:

$x+ky=k+1.$

Dla jakich $k$ te proste przecinają się $\mathrm{w}\mathrm{e}\mathrm{w}\mathrm{n}4^{\mathrm{t}\mathrm{r}\mathrm{Z}}$ kwadratu, w którym punkty

$A=(2,-2)\mathrm{i}C=(-2,2)$ sa końcami przekatnej?






AKADEMIA GÓRNICZO-HUTNICZA

im. StanisIawa Staszica w Krakowie

OLIMPIADA,, O DIAMENTOWY INDEKS AGH'' 2011/12

MATEMATYKA- ETAP II

ZADANIA PO 10 PUNKTÓW

l. Wykaz, $\dot{\mathrm{z}}\mathrm{e}$ liczba $a=\sqrt{9-4\sqrt{5}}-\sqrt{9+4\sqrt{5}}$ jest catkowita.

2. Wyznacz dziedzinę funkcji danej wzorem

$f(x)=\sqrt{x^{4}+x^{3}-8x^{2}-12x}.$

3. Oblicz miarę kata między wektorami ã $\mathrm{i} \vec{b}$ wiedząc, $\dot{\mathrm{z}}\mathrm{e}$ wektory

$\vec{u}=3\text{{\it ã}}+2\vec{b} \mathrm{i} \vec{v}=-\text{{\it ã}}+4\vec{b}$ sa prostopadIe oraz $|\vec{a}|=|\vec{b}|=1.$

4. Dwa rózne automaty wykonują razem dana pracę w ciagu 6 godzin.

Gdyby pierwszy automat pracowal sam przez 2 godziny, a następnie

drugi pracowat sam przez 6 godzin, to wykonaIyby poIowę ca1ej pracy.

Wjakim czasie $\mathrm{k}\mathrm{a}\dot{\mathrm{z}}\mathrm{d}\mathrm{y}$ automat $\mathrm{m}\mathrm{o}\dot{\mathrm{z}}\mathrm{e}$ samodzielnie wykonač calq pracę?

ZADANIA PO 20 PUNKTÓW

5. Ze zbioru $S = \{1$, 2, $\ldots$, 2012$\}$ losujemy trzy liczby i ustawiamy je

w ciąg rosnacy $(a,b,c)$. Oblicz prawdopodobieństwo zdarzeń

$A$: iloczyn $abc$ jest liczba parzysta,

$B_{k}$: $b=k$, gdzie $k$ jest ustaloną liczbq ze zbioru $S.$

Dla jakich $k$ prawdopodobieństwo zdarzenia $B_{k}$ jest największe?

6. Dane są dwa punkty $A = (7,5), B = (1,-1)$ oraz punkt $P = (3,3)$

przecięcia wysokości trójkata $ABC$. Oblicz pole trójkata $ABC$ i napisz

równanie okregu opisanego na nim.

7. Stozek i walec mają równe $\mathrm{t}\mathrm{w}\mathrm{o}\mathrm{r}\mathrm{z}4^{\mathrm{C}\mathrm{e}}$, równe objętości i równe pola

powierzchni bocznej. Oblicz

a) sinus kąta nachylenia tworzacej stozka do jego podstawy,

b) stosunek pola przekroju osiowego walca do pola przekroju osiowego

stozka.






AKADEMIA GÓRNICZO-HUTNICZA

im. StanisIawa Staszica w Krakowie

OLIMPIADA,, O DIAMENTOWY INDEKS AGH'' 2012/13

MATEMATYKA- ETAP II

ZADANIA PO 10 PUNKTÓW

l. Rozwiąz równanie $(5\sqrt{2}-7)^{x-1}=(5\sqrt{2}+7)^{3x}$

2. Jedna z krawędzi bocznych ostrosIupa jest prostopadla do jego pod-

stawy, będacej prostokatem o bokach dIugości 5 cm i 12 cm. Najd1uzsza

krawęd $\acute{\mathrm{z}}$ boczna jest nachylona do ptaszczyzny podstawy pod kątem

$60^{o}$ Oblicz pole powierzchni bocznej ostroslupa.

3. Wyznacz dziedzinę funkcji określonej wzorem

$f(x)=\sqrt{\log_{\pi-1}(2x-1)-\log_{\pi-1}(5-3x)}.$

4. Oblicz granicę ciagu $(a_{n})$, gdzie

$a_{n}=\displaystyle \frac{3+6+9+\ldots+3n}{(2n+1)^{2}}.$

ZADANIA PO 20 PUNKTÓW

5. Wykaz$\cdot, \dot{\mathrm{z}}\mathrm{e}(2n+2)$-cyfrowa liczba ll$\ldots$122$\ldots$25

$\overline{n}\tilde{n+1}$

liczby naturalnej (dla dowolnego $n$).

jest kwadratem

6. Dane są punkty $A = (-5,0), B = (-3,-4), C = (3,4), M = (7,1).$

$\mathrm{Z}$ punktu $M$ poprowadzono styczne $k\mathrm{i}l$ do okręgu opisanego na trójka-

cie $ABC$. Oblicz pole trójkąta $KLM$, gdzie $K\mathrm{i}L$ sq punktami stycz-

ności prostych $k\mathrm{i}l$ z tym okręgiem.

7. Rzucamy moneta $n$ razy $(n\geq 2)$. Oblicz prawdopodobieństwa zdarzeń:

$A$: reszka wypadIa dokIadnie $k$ razy;

$B$: reszka wypadla więcej razy $\mathrm{n}\mathrm{i}\dot{\mathrm{z}}$ orzel;

$C$: przynajmniej dwa razy pod rząd moneta upadla tą samą strona.






AKADEMIA GÓRNICZO-HUTNICZA

im. StanisIawa Staszica w Krakowie

OLIMPIADA,, O DIAMENTOWY INDEKS AGH'' 2013/14

MATEMATYKA- ETAP II

ZADANIA PO 10 PUNKTÓW

l. Urządzenie I wykonuje pewną pracę w ciagu 20 godzin, a urzadzenie II w ciagu 30

godzin. $\mathrm{W}$ jakim czasie wykonają tę pracę oba urzqdzenia pracujac jednocześnie?

2. Kotangens kata rozwartego $\alpha$ jest równy $-3$. Oblicz wartości funkcji trygonomet-

rycznych kąta $2\alpha.$

3. Rozwiąz nierównośč $|3\log_{x}2-2|>1.$

4. Zbadaj monotonicznośč ciagu $(a_{n})$, którego n-ty wyraz jest równy

$a_{n}=\displaystyle \frac{3^{n+2}}{3^{n}+2^{2n+1}}.$

Wyznacz granicę ciągu $(a_{n}).$

ZADANIA PO 20 PUNKTÓW

5. Okrag $O$ ma równanie $x^{2}+y^{2}+6x+4y-12=0$. Okrag $O'$ jest obrazem okręgu

$O$ przez translację o wektor $\vec{v}= [7$, 1$].$ Znajd $\acute{\mathrm{z}}$ równania osi symetrii sumy $O\cup O'$

tych okręgów. Wyznacz punkty wspólne obu okręgów. Znajd $\acute{\mathrm{z}}$ równania prostych

stycznych jednocześnie do $O\mathrm{i}O'.$

6. Podstaw4 ostroslupa o wysokości $H$ jest trójkqt prostokątny $ABC$ o przyprosto-

kątnych $|AB| =a\mathrm{i} |AC| =b$. Krawędz$\acute{}$ boczna wychodząca z wierzcholka $A$ jest

prostopadla do podstawy. Ostrostup ten podzielono pIaszczyzna równolegla do pod-

stawy na dwie bryIy o równych objętościach. Oblicz pole powierzchni catkowitej tej

bryly, która nie jest ostroslupem.

7. Do windy na parterze budynku czteropiętrowego wsiada osiem osób. Oblicz prawdo-

podobieństwa zdarzeń:

$A$: wszyscy wysiada na tym samym piętrze,

$B$: na czwartym piętrze wysiqdq co najmniej dwie osoby,

$C$: na $\mathrm{k}\mathrm{a}\dot{\mathrm{z}}$ dym piętrze wysiada po dwie osoby.






AKADEMIA GÓRNICZO-HUTNICZA

im. StanisIawa Staszica w Krakowie

OLIMPIADA,, O DIAMENTOWY INDEKS AGH'' 2014/15

MATEMATYKA- ETAP II

ZADANIA PO 10 PUNKTÓW

l. Udowodnij, $\dot{\mathrm{z}}\mathrm{e}$ dla dowolnych dodatnich liczb rzeczywistych $a, b$ spetniona jest nie-

równośč

-{\it ab}$+$-{\it ab} $\geq$2.

2. Wyznacz najmniejszq i najwiQkszq wartośč funkcji danej wzorem $f(x)=|x^{2}-8x+7|$

w przedziale $\langle 0;5\rangle.$

3. Znajd $\acute{\mathrm{z}}$ punkty nieciqgtości funkcji danej wzorem

$f(x)=\displaystyle \frac{x^{2}-4}{x^{4}+x^{3}+8x+8}.$

$\mathrm{W}$ których z tych punktów $\mathrm{m}\mathrm{o}\dot{\mathrm{z}}$ na określič wartośč funkcji tak, $\dot{\mathrm{z}}$ eby byla ciqgta?

4. $\mathrm{W}\mathrm{k}\mathrm{a}\dot{\mathrm{z}}$ dym z ostatnich dwóch notowań cena ropy spadata o k\%, gdzie $k\in(0;100).$

$\mathrm{O}$ ile procent musiataby cena wzrosnqč w najblizszym notowaniu, $\dot{\mathrm{z}}$ eby wrócita do

poczqtkowego poziomu?

ZADANIA PO 20 PUNKTÓW

5. Figura $B$ jest obrazem figury

$A=\{(x,y)$ : $x^{2}+y^{2}-6x-8y+21\leq 0$

$\wedge x-7y+25\geq 0\}.$

przez symetrię względem prostej $x-2y=0.$ Znajd $\acute{\mathrm{z}}$ nierówności opisujqce figurę $B$

i oblicz jej obwód.

6. Rozwiqz nierównośč

$\log_{2x}(x^{4}+3)\geq 2.$

7. $\mathrm{W}$ trójkqt prostokqtny o przyprostokqtnych $a = 15$ cm, $b = 20$ cm wpisany jest

okrqg. Oblicz odlegtości od $\mathrm{k}\mathrm{a}\dot{\mathrm{z}}$ dego wierzchotka trójkqta do punktu styczności

okręgu z przeciwlegtym bokiem.






AKADEMIA GÓRNICZO-HUTNICZA

im. StanisIawa Staszica w Krakowie

OLIMPIADA,, O DIAMENTOWY INDEKS AGH'' 2015/16

MATEMATYKA- ETAP II

ZADANIA PO 10 PUNKTÓW

l. Wyznacz największa liczbę naturalna $k$ taka, $\dot{\mathrm{z}}\mathrm{e}$ liczba 2016! jest wie-

lokrotnością liczby $10^{k}$

2. Rozwiqz nierównośč $\displaystyle \log_{x}(x^{2}-\frac{5}{2}x+1)-2<0.$

3. Wyznacz dziedzinę $D$ funkcji określonej wzorem

$f(x)=\displaystyle \frac{\sqrt{x^{2}+6x+9}}{x^{2}-x-12}$

i zbadaj jej granice w punktach nalezacych do zbioru $lR\backslash D.$

4. Zespolowi pracowników zlecono pewną pracę. Gdyby bylo ich o 3 mniej,

to pracowaliby o 5 dni $\mathrm{d}\mathrm{I}\mathrm{u}\dot{\mathrm{z}}$ ej, a gdyby bylo ich o 4 więcej, to pracowa1iby

$02$ dni krócej. Ilu bylo pracowników i jak dlugo pracowali?

ZADANIA PO 20 PUNKTÓW

5. {\it K}rawęd $\acute{\mathrm{z}}$ boczna ostroslupa prawidlowego sześciokątnegojest nachylona

do podstawy pod katem $60^{o}$ Oblicz stosunek dIugości promienia kuli

wpisanej w ten ostroslup do jego wysokości.

6. Okrag $O'$ jest obrazem okregu $O$ o równaniu

$x^{2}+y^{2}-4x-6y-12=0$

w symetrii środkowej względem punktu $M= (6,6)$. Napisz równanie

okregu $O'$ i równania wszystkich prostych, które sajednocześnie styczne

do obu okręgów.

7. Losowo wybieramy liczbę $k$ ze zbioru \{1, 2, 3, 4\}, a następnie rzucamy

$k$ razy sześcienna kostka. Oblicz prawdopodobieństwa zdarzeń:

$A$: wypadną same szóstki,

$B$: iloczyn wyrzuconych oczek będzie liczbq parzysta,

$C$: suma wyrzuconych oczek będzie mniejsza $\mathrm{n}\mathrm{i}\dot{\mathrm{z}}22.$






AKADEMIA GÓRNICZO-HUTNICZA

im. Stanislawa Staszica w Krakowie

OLIMPIADA O DIAMENTOWY INDEKS AGH'' 2016/17

MATEMATYKA- ETAP II

ZADANIA PO 10 PUNKTÓW

l. Udowodnij, $\dot{\mathrm{z}}\mathrm{e}$ spośród dowolnych pięciu liczb naturalnych $\mathrm{m}\mathrm{o}\dot{\mathrm{z}}$ na wybrač trzy,

których suma jest podzielna przez 3.

2. Rozwiqz równanie

$\displaystyle \frac{\log_{x}(x^{3}+3)}{\log_{x}(x+1)}=2.$

3. Ile jest sześciocyfrowych liczb naturalnych, w których liczba cyfr parzystych jest

równa liczbie cyfr nieparzystych?

4. Oblicz promień okręgu opisanego na trójkącie $ABC$, w którym $|AB|=10$ cm,

$|AC|=8$ cm i miara kąta przy wierzchofku $A$ jest równa $60^{\mathrm{o}}$

ZADANIA PO 20 PUNKTÓW

5. Wykres funkcji kwadratowej $f(x)$ przechodzi przez punkty $(-2,16), (1,-2)$, (3, 6).

Po przesunięciu go o wektor $\vec{v}=[2,-6]$ i przeksztalceniu przez symetrię wzglę-

dem prostej $x=0$ otrzymano wykres funkcji $g(x)$. Wykres funkcji $g(x)$ prze-

ksztalcono przez symetrię względem prostej $y=3$, otrzymując wykres funkcji

$h(x)$. Napisz wzory funkcji $f(x), g(x)\mathrm{i}h(x).$

6. $\mathrm{W}$ prawidfowym ostrosfupie czworokątnym krawędzie boczne $\mathrm{s}\Phi$ nachylone do

podstawy pod kątem $\alpha. \mathrm{W}$ ostroslup wpisano półkulę o promieniu $R$ tak, $\dot{\mathrm{z}}\mathrm{e}$

jest ona styczna do ścian bocznych, a kofo wielkie zawiera się w podstawie

ostrosfupa. Oblicz objętośč ostrosfupa.

7. Suma wszystkich wspólczynników wielomianu $W(x)$ jest równa

{\it n}l$\rightarrow$im$\infty$ -552--2{\it n-n}$++$2 -211 --22{\it nn}.

Suma wspólczynników przy parzystych potęgach zmiennej $x$ jest 3 razy większa

$\mathrm{n}\mathrm{i}\dot{\mathrm{z}}$ suma wspólczynników przy potęgach nieparzystych. Znajd $\acute{\mathrm{z}}$ reszty z dzielenia

$W(x)$ przez dwumiany: a) $x-1$, b) $x+1$, c) $x^{2}-1.$






AKADEMIA GÓRNICZO-HUTNICZA

im. Stanislawa Staszica w Krakowie

OLIMPIADA O DIAMENTOWY INDEKS AGH'' 2017/18

MATEMATYKA - ETAP II

ZADANIA PO 10 PUNKTÓW

l. Ile jest sześciocyfrowych liczb naturalnych, w których występuje $\mathrm{k}\mathrm{a}\dot{\mathrm{z}}$-

da z cyfr 0,1,2,3,4,5? I1e jest wśród nich 1iczb parzystych, a i1e 1iczb

pierwszych?

2. Odlegfości punktu $P$, lezącego wewnątrz kwadratu, od trzech jego

wierzchofków wynoszq odpowiednio 35 cm, 35 cm i 49 cm. Ob1icz

odległośč punktu $P$ od czwartego wierzchołka kwadratu.

3. Udowodnij, $\dot{\mathrm{z}}\mathrm{e}$ dla dowolnych liczb rzeczywistych $a, b, c$ spefniona jest

nierównośč

$\displaystyle \sqrt{\frac{a^{2}+b^{2}+c^{2}}{3}}\geq\frac{a+b+c}{3}.$

4. Rozwiąz równanie

$\log_{x}10+\log_{x}10^{2}+\cdots+\log_{x}10^{100}=10100.$

ZADANIA PO 20 PUNKTÓW

5. Prosta $x+2y-13=0$ zawiera bok $AB$, prosta $x-y+5=0$ zawiera

bok $BC$ trójkąta $ABC$, a prosta $x-3y+7=0$ zawiera dwusieczną

kata $BCA.$ Znajd $\acute{\mathrm{z}}$ wierzchołki tego trójkąta.

6. $\mathrm{W}$ ostrosfupie prawidfowym czworokątnym o krawędzi podstawy dfu-

gości $a=2$ dm $\mathrm{k}\mathrm{a}\mathrm{t}$ między ścianami bocznymi ma miarę $135^{o}$ Ostro-

sfup ten przecięto dwiema plaszczyznami równoległymi do postawy

na trzy bryly o równych objętościach. Oblicz odległośč między tymi

płaszczyznami.

7. Wyznacz przedziafy monotoniczności funkcji określonej wzorem

$ f(x)=x+\displaystyle \frac{3}{x}+\frac{9}{x^{3}}+\frac{27}{x^{5}}+\cdots$






AKADEMIA GÓRNICZO-HUTNICZA

im. Stanislawa Staszica w Krakowie

OLIMPIADA O DIAMENTOWY INDEKS AGH'' 2018/19

MATEMATYKA - ETAP II

ZADANIA PO 10 PUNKTÓW

l. Kierowca przejechal połowę drogi autostradą, a drugą połowę lokal-

nymi drogami z prędkością dwa razy mniejszą. Jaki procent drogi

przejechal kierowca po uplywie połowy czasu podróz $\mathrm{y}$?

2. Liczba $\alpha\in (\displaystyle \frac{\pi}{4};\frac{\pi}{2})$ spefnia równanie

$\mathrm{t}\mathrm{g}^{2}\alpha=5\mathrm{t}\mathrm{g}\alpha-1.$

Oblicz $\cos 2\alpha.$

3. Na ile sposobów $\mathrm{m}\mathrm{o}\dot{\mathrm{z}}$ na tak ustawič w $\mathrm{c}\mathrm{i}\otimes \mathrm{g} k$ czarnych kul i $k+1$

bialych, by $\dot{\mathrm{z}}$ adne dwie czarne kule nie znalazfy się obok siebie? Za-

kładamy, $\dot{\mathrm{z}}\mathrm{e}$ kule tego samego koloru są nierozróznialne.

4. Wyznacz dziedzinę funkcji danej wzorem

$f(x)=\displaystyle \frac{x^{5}+8x^{2}}{x^{3}-4x^{2}-3x+18}$

$\mathrm{W}$ których punktach nienalezących do dziedziny $\mathrm{m}\mathrm{o}\dot{\mathrm{z}}$ na określič war-

tośč funkcji $f$, aby otrzymač funkcję ciqgłą w danym punkcie?

ZADANIA PO 20 PUNKTÓW

5. $\mathrm{W}$ prawidfowy ostroslup $\mathrm{c}\mathrm{z}\mathrm{w}\mathrm{o}\mathrm{r}\mathrm{o}\mathrm{k}_{\Phi}\mathrm{t}\mathrm{n}\mathrm{y}$ o krawędzi podstawy dlugości $a$

i krawędzi bocznej długości $b$ tak wpisany jest walec, $\dot{\mathrm{z}}\mathrm{e}$ jedna z pod-

staw walca zawiera się w podstawie ostrosfupa. Wyznacz wymiary

walca o $\mathrm{m}\mathrm{o}\dot{\mathrm{z}}$ liwie najwiekszej objętości.

6. Dla jakich liczb rzeczywistych $m$ równanie

$(m-3)x^{2}+2mx+m-2=0$

ma dwa rózne pierwiastki rzeczywiste $x_{1}, x_{2}$, spelniające nierównośč

$\log_{0,1}x_{1}+\log_{0,1}x_{2}\geq 0$?

7. $\mathrm{W}$ trójkąt równoboczny o boku dlugości $a$ tak wpisane są trzy przy-

stające okręgi, $\dot{\mathrm{z}}\mathrm{e}\mathrm{k}\mathrm{a}\dot{\mathrm{z}}\mathrm{d}\mathrm{y}$ z nich jest styczny do dwóch pozostałych i do

dwóch boków trójk$\Phi$ta. Oblicz promień okręgu zewnętrznie stycznego

do tych trzech okręgów.






AKADEMIA GÓRNICZO-HUTNICZA

im. Stanisława Staszica w Krakowie

OGÓLNOPOLSKA OLIMPIADA

O DIAMENTOWY INDEKS AGH'' 2019/20

MATEMATYKA - ETAP II

ZADANIA PO 10 PUNKTÓW

l. Niech $n$ będzie dowolną $\mathrm{n}\mathrm{i}\mathrm{e}\mathrm{P}^{\mathrm{a}\mathrm{r}\mathrm{z}\mathrm{y}\mathrm{s}\mathrm{t}_{\Phi}1\mathrm{i}\mathrm{c}\mathrm{z}\mathrm{b}_{\Phi^{\mathrm{n}\mathrm{a}\mathrm{t}\mathrm{u}\mathrm{r}\mathrm{a}\ln}\Phi}}$. Udowodnij, $\dot{\mathrm{z}}\mathrm{e}$ suma

$n$ kolejnych liczb całkowitych jest podzielna przez $n.$

2. Dla jakich liczb $k$ trójmian kwadratowy

$2(1-k^{2})x^{2}+k(1+k^{2})x+2k$

jest podzielny przez dwumian $x+k$?

3. Rozwiqz równanie $\cos^{2}3x-\sin^{2}x=0.$

4. Do klasy, w której co czwarty uczeń jest jedynakiem, przyfączono $\mathrm{d}\mathrm{r}\mathrm{u}\mathrm{g}\Phi$

klasę o dwukrotnie mniejszej liczbie uczniów, wśród których jest 40\% jedy-

naków. Jaki procent uczniów w nowo utworzonej klasie ma rodzeństwo?

ZADANIA PO 20 PUNKTÓW

5. Ze zbioru \{l, 2, $\ldots$, 9\} 1osujemy jednocześnie dwie liczby. Czynnośč tę pow-

tarzamy (zwróciwszy wylosowane liczby) dotqd, $\mathrm{a}\dot{\mathrm{z}}$ wylosujmy dwie liczby

dające tę samą resztę z dzielenia przez 3. Jakie jest prawdopodobieństwo,

$\dot{\mathrm{z}}\mathrm{e}$ liczba losowań będzie

$A$: mniejsza $\mathrm{n}\mathrm{i}\dot{\mathrm{z}}10,$

$B$: równa 6,

$C$: nieparzysta.

6. Funkcja $f$ dla $\mathrm{k}\mathrm{a}\dot{\mathrm{z}}$ dego jej argumentu $x$ spefnia równośč

$f(x)+(f(x))^{2}+\ldots=x^{3},$

której lewa strona jest sumą nieskończonego ciqgu geometrycznego. Wyz-

nacz dziedzinę funkcji $f$ oraz jej ekstrema lokalne.

7. $\mathrm{W}$ równoległobok ABCD, w którym kolejnośč wierzchołków ABCD jest

przeciwna do ruchu wskazówek zegara, $\mathrm{m}\mathrm{o}\dot{\mathrm{z}}$ na wpisač okrąg. Majqc dane

wspófrzędne wierzchofków $A= (0,1) \mathrm{i}B= (\sqrt{3},0)$ oraz miarę $120^{o}$ kąta

wewnętrznego przy wierzcholku $D$, oblicz pole powierzchni równolegloboku

i napisz równanie okręgu weń wpisanego.






AKADEMIA GÓRNICZO-HUTNICZA

im. Stanisława Staszica w Krakowie

OLIMPIADA O DIAMENTOWY INDEKS AGH'' 2020/21

MATEMATYKA - ETAP II

ZADANIA PO 10 PUNKTÓW

l. Udowodnij, $\dot{\mathrm{z}}\mathrm{e}\mathrm{k}\mathrm{a}\dot{\mathrm{z}}$ da liczba rzeczywista $a\neq 0$ spełnia nierównośč

$a^{2}+\displaystyle \frac{4}{a^{4}}\geq 3.$

Podaj liczby, dla których prawdziwa jest równośč.

2. $\mathrm{W}$ kwadracie ABCD punkt $K$ jest środkiem boku $AB$. Przez punkt $K$

poprowadzona jest prosta prostopadfa do prostej $KC$, która przecina bok

$AD$ w punkcie $R$. Wykaz, $\dot{\mathrm{z}}\mathrm{e}\mathrm{k}\mathrm{a}\mathrm{t}\mathrm{y}\triangleleft KCB\mathrm{i}\triangleleft KCR$ mają równe miary.

3. $\mathrm{W}$ ciągu geometrycznym $(\alpha_{n})$ dane są $a_{3}=\displaystyle \frac{1}{4}$ oraz

$a_{10}=\displaystyle \log_{2}\cos\frac{47}{12}\pi+\log_{2}\sin(-\frac{37}{12}\pi)$

Oblicz $a_{17}.$

4. $\mathrm{Z}$ pnia drzewa w kształcie walca o średnicy podstawy $D$ i długości $H$ wy-

cieto cztery przystajqce bale w kształcie walca o długości $H$ i najwiekszej

$\mathrm{m}\mathrm{o}\dot{\mathrm{z}}$ liwej objętości. Oblicz objętośč pozostalej części pnia.

ZADANIA PO 20 PUNKTÓW

5. Napisz równania asymptot wykresu funkcji f danej wzorem

$f(x)=\displaystyle \frac{x^{2}+4x}{x^{3}+4x^{2}+4x+16}.$

Wyznacz najmniejszą i największą wartośč funkcji $f$ w przedziale $\langle$1; $5\rangle.$

6. Znajd $\acute{\mathrm{z}}$ równanie okręgu, na którym $\mathrm{l}\mathrm{e}\dot{\mathrm{z}}\mathrm{q}$ punkty $A=(8,8), B=(-8,-4)$

$\mathrm{i}C= (6,-6)$. Napisz równania stycznych do tego okręgu, prostopadfych

do prostej $4x+3y-6=0.$

7. Rozwazmy zbiór $S$ wszystkich funkcji danych wzorem $f(x)=\alpha x^{2}+bx+c,$

gdzie $\alpha, b, c$ sq liczbami calkowitymi spełniającymi nierównośč

$4^{x+1}-33\cdot 2^{x}+8\leq 0.$

Wyznacz liczby elementów podzbiorów $P, Q, R$ zbioru $S$, gdzie $P$ jest zbio-

rem funkcji parzystych, $Q$ jest zbiorem funkcji, których wykres przechodzi

przez punkt $(0,3)$, a $R$ jest zbiorem funkcji rosnących w $\mathbb{R}.$






AKADEMIA GÓRNICZO-HUTNICZA

im. Stanisława Staszica w Krakowie

OLIMPIADA O DIAMENTOWY INDEKS AGH'' 2021/22

MATEMATYKA - ETAP II

ZADANIA PO 10 PUNKTÓW

l. Jadąc z prędkościq 30 $\mathrm{k}\mathrm{m}/$godz. spóz/nimy się na spotkanie 10 minut, ajadąc

z prędkością 60 $\mathrm{k}\mathrm{m}/$godz. będziemy 10 minut za wcześnie. Zjaką prędkości$\Phi$

powinniśmy jechač, aby przybyč punktualnie?

2. Bok kwadratu jest przeciwprostokątną AB trójkąta prostokątnego, którego

trzeci wierzchofek $C\mathrm{l}\mathrm{e}\dot{\mathrm{z}}\mathrm{y}$ na zewnątrz kwadratu. Niech $S$ będzie środkiem

kwadratu. Uzasadnij, $\dot{\mathrm{z}}\mathrm{e}$ kąty $ACS\mathrm{i}BCS$ sq przystające.

3. Dane są trzy kolejne liczby cafkowite. Udowodnij, $\dot{\mathrm{z}}\mathrm{e}$ kwadraty dokladnie

dwóch z nich dają resztę l z dzielenia przez 3.

4. Liczby 2 $\log_{2}x, \log_{2}2x, \log_{2}(x-4)$ są trzema początkowymi wyrazami

ciągu arytmetycznego. Znajd $\acute{\mathrm{z}}$ setny wyraz tego ciągu.

ZADANIA PO 20 PUNKTÓW

5. Wyznacz dziedzinę i zbiór wartości funkcji $f, \mathrm{j}\mathrm{e}\dot{\mathrm{z}}$ eli dla $\mathrm{k}\mathrm{a}\dot{\mathrm{z}}$ dego $x$ nalezącego

do jej dziedziny spelniona jest równośč

$f(x)+(f(x))^{2}+(f(x))^{3}+\displaystyle \ldots=-\frac{1}{5}(x^{2}+1)$

6. Dane $\mathrm{s}\Phi$ dodatnie liczby cafkowite $n$ oraz $k$, przy czym $k \leq n$. Ze zbioru

liczb $\{$1, 2, $\ldots, n\}$ losujemy kolejno bez zwracania $k$ liczb, otrzymując w ten

sposób ciag $k$-wyrazowy. Oblicz prawdopodobieństwa zdarzeń

$A$: liczba $k$ nie występuje w tym ciągu,

$B$: $k$ jest ostatnim wyrazem ciągu,

$C$: $\mathrm{c}\mathrm{i}_{\Phi \mathrm{g}}$ jest monotoniczny i $k$ jest jego wyrazem.

7. Punkty $A = (0,7), B = (1,0), C = (-3,-2)$ są wierzcholkami trójkąta.

Znajd $\acute{\mathrm{z}}$ równanie okręgu opisanego na tym trójkącie i równanie jego obra-

zu w symetrii środkowej względem punktu $A$. Napisz równania wszystkich

prostych stycznych jednocześnie do obu tych okręgów.






AKADEMIA GÓRNICZO-HUTNICZA

im. Stanislawa Staszica w Krakowie

OLIMPIADA O DIAMENTOWY INDEKS AGH'' 2022/23

MATEMATYKA - ETAP II

ZADANIA PO 10 PUNKTÓW

l. Udowodnij, $\dot{\mathrm{z}}\mathrm{e}$ istnieje tylko jedna trójka liczb pierwszych, które sa trzema

kolejnymi wyrazami $\mathrm{c}\mathrm{i}_{\Phi \mathrm{g}}\mathrm{u}$ arytmetycznego o róznicy 2.

2. Najkrótszy bok trapezu prostokątnego opisanego na okręgu o promieniu $r$ ma

długośč $\displaystyle \frac{5}{3}r$. Oblicz pole trapezu.

3. Dwa miasta $A \mathrm{i} B$ są odlegfe od siebie o 960 km. $\mathrm{Z}$ tych miast wyjechafy

naprzeciw siebie dwa pociągi, przy czym pociąg z miasta $B$ wyjecha12 godziny

póz$\acute{}$niej i jechal z predkością o 20 $\mathrm{k}\mathrm{m}/$godz. wiekszą $\mathrm{n}\mathrm{i}\dot{\mathrm{z}}$ pociąg z miasta $A.$

$\mathrm{p}_{\mathrm{o}\mathrm{c}\mathrm{i}_{\Phi \mathrm{g}\mathrm{i}}}$ te minęly się dokfadnie w pofowie drogi. Podaj prędkośč pociągu,

który wyruszył z miasta $A.$

4. Spośród wierzcholów $n$-kąta foremnego losujemy trzy. Oblicz prawdopodobień-

stwo $p_{n}$ wylosowania wierzchofków trójkąta prostokątnego. Zbadaj, czy ciąg

$(p_{n})$ ma granicę.

ZADANIA PO 20 PUNKTÓW

5. Oblicz sumę długości wszystkich krawedzi ostrosłupa prawidłowego czworo-

katnego wpisanego w sfere o promieniu $R$, który ma największą objętośč.

6. Znajd $\acute{\mathrm{z}}$ wszystkie pierwiastki równania

które spelniają nierównośč

$3^{\cos^{2}2x}+3^{\sin^{2}2x}=4,$

$\log_{x}(x+2)<2.$

7. Wielomian $P$ dany wzorem

$P(x)=x^{3}+mx+162$

ma pierwiastek wielokrotny. Uzasadnij, $\dot{\mathrm{z}}\mathrm{e}$ nie jest to jedyny pierwiastek tego

wielomianu. Wykaz, $\dot{\mathrm{z}}\mathrm{e}P(-\sqrt[6]{9})$ jest liczbą cafkowitą.






AKADEMIA GÓRNICZO-HUTNICZA im. Stanislawa Staszica

w Krakowie

OLIMPIADA,, O DIAMENTOWY INDEKS AGH'' 2007/8

MATEMATYKA- ETAP III

ZADANIA PO 10 PUNKTÓW

1. $\mathrm{W}$ trapezie o polu $P$ stosunek dlugości podstaw jest równy $k>1$. Oblicz pola dwóch

trójkatów, na które ten trapez dzieli jego przekatna.

2. Rozwia $\dot{\mathrm{Z}}$ nierównośč

0, $1^{\mathrm{x}}\cdot 0, 1^{x^{3}}\cdot 0, 1^{x^{5}}$

$>$ -$\sqrt{}$3 11000000.

3. PoIowę drogi kierowca jechaI autostrad4 z prędkościa l20 $\mathrm{k}\mathrm{m}/\mathrm{h}$, a drugą poIowę

na drogach lokalnych ze średni4 predkością 60 $\mathrm{k}\mathrm{m}/\mathrm{h}$. Oblicz średnia predkośč calej

podróz $\mathrm{y}.$

4. Znajd $\acute{\mathrm{z}}$ równania okrQgów o promieniu 3 stycznych jednocześnie do osi $OX$ i do

prostej $12x+5y=0.$

ZADANIA PO 20 PUNKTÓW

5. Na czworościanie foremnym opisano walec w ten sposób, $\dot{\mathrm{z}}\mathrm{e}$ dwie krawędzie czwo-

rościanu $1\mathrm{e}\dot{\mathrm{z}}4^{\mathrm{C}\mathrm{e}}$ na prostych skośnych sa średnicami podstaw walca. Oblicz stosunek

pola powierzchni sfery opisanej na walcu do pola powierzchni sfery wpisanej w czwo-

rościan.

6. Dla jakich wartości parametru $m$ dokladnie jeden pierwiastek równania

$(m-2)9^{x}+(m+1)3^{x}-m=0$

jest mniejszy od 2?

7. Ze zbioru \{l, 2, $\ldots$, 1000\} 1osujemy trójelementowy podzbiór $T = \{p,q,r\}$, przy

czym prawdopodobieństwo wylosowania $\mathrm{k}\mathrm{a}\dot{\mathrm{z}}$ dego podzbioru jest jednakowe.

a) Oblicz prawdopodobieństwo, $\dot{\mathrm{z}}\mathrm{e}$ iloczyn $pqr$ jest podzielny przez 3.

b) Niech $\varphi$ będzie funkcjq przyporządkowującq $\mathrm{k}\mathrm{a}\dot{\mathrm{z}}$ demu wylosowanemu podzbiorowi

$T,$,element pośredni'' (tzn. jeśli $p<q<r$, to $\varphi(T)=q$). Jaka wartośč funkcji $\varphi$

jest najbardziej prawdopodobna?






AKADEMIA GÓRNICZO-HUTNICZA im. StanisIawa Staszica

w Krakowie

OLIMPIADA,, O DIAMENTOWY INDEKS AGH'' 2008/9

MATEMATYKA- ETAP III

ZADANIA PO 10 PUNKTÓW

1. Znajd $\acute{\mathrm{z}}$ wspólrzędne obrazu punktu $C = (20,25)$ w symetrii osiowej względem

prostej przechodzqcej przez punkty $A=(6,2)\mathrm{i}B=(3,-4).$

2. Wyznacz dziedzinę funkcji danej wzorem

$f(x)=\log_{2}(x^{3}-4x^{2}-3x+18).$

3. Oblicz granicę ciagu

$\displaystyle \lim_{n\rightarrow\infty}(n-\sqrt{n^{2}+5n}).$

4. Znajd $\acute{\mathrm{z}}1\mathrm{i}\mathrm{c}\mathrm{z}\mathrm{b}_{Q}$, której 59\% stanowi okresowy ufamek dziesiętny 2, 6(81).

ZADANIA PO 20 PUNKTÓW

5. Ze zbioru $\{1,2,3, ,2n-1,2n\}$, gdzie $n$ jest ustalona liczba naturalna, losujemy

ze zwracaniem dwie liczby $x\mathrm{i}y$. Oblicz prawdopodobieństwa zdarzeń

$A$ : $x=y$; $B$ : iloczyn $xy$ jest liczba parzystą; $C$ : $\displaystyle \frac{x}{y}\in(0;1).$

6. $\mathrm{W}$ ostrosIupie prawidtowym trójkatnym o wysokości $h$ krawędz/ boczna jest nachy-

lona do krawędzi podstawy pod kątem $\alpha$. Oblicz promień kuli wpisanej w ten

ostrosIup. Jakie wartości $\mathrm{m}\mathrm{o}\dot{\mathrm{z}}\mathrm{e}$ przyjmowač miara kata $\alpha$?

7. Dla jakich wartości parametru $m$ nierównośč

$(m^{2}-1)x^{2}+2(m-1)x+2>0$

jest spelniona dla $\mathrm{k}\mathrm{a}\dot{\mathrm{z}}$ dego $x \in 1R$? Czy istnieje takie $x$, aby dla $\mathrm{k}\mathrm{a}\dot{\mathrm{z}}$ dego $m \in 1R$

powyzsza nierównośč byla prawdziwa?






AKADEMIA GÓRNICZO-HUTNICZA

im. StanisIawa Staszica w Krakowie

OLIMPIADA,, O DIAMENTOWY INDEKS AGH'' 2009/10

MATEMATYKA- ETAP III

ZADANIA PO 10 PUNKTÓW

1. Rozwia $\dot{\mathrm{Z}}$ równanie

$(x^{2}+1)^{\sin 2x+\cos 2x}=1.$

2. Jakie największe pole powierzchni bocznej $\mathrm{m}\mathrm{o}\dot{\mathrm{z}}\mathrm{e}$ mieč stozek obrotowy, w którym

obwód przekroju osiowego ma dlugośč $C$ ?

3. Zbadaj wzajemne polozenie okręgów:

01 : $x^{2}+y^{2}-4x-2y-45=0,$

02 : $x^{2}+y^{2}+2y-97=0.$

4. Oblicz granicę ciagu, którego n-ty wyraz jest równy

$a_{n}=\displaystyle \frac{1}{3^{n}+2^{n}}+\frac{3}{3^{n}+2^{n}}+\frac{9}{3^{n}+2^{n}}+\ldots+\frac{3^{n}}{3^{n}+2^{n}}.$

ZADANIA PO 20 PUNKTÓW

5. $\mathrm{W}$ ostroslupie prawidIowym ośmiokatnym krawęd $\acute{\mathrm{z}}$ podstawy ma dIugośč $a$, a $\mathrm{k}\mathrm{a}\mathrm{t}$

nachylenia ściany bocznej do podstawy ma miarę $\alpha$. Wysokośč ostroslupa podzielono

na $n$ odcinków równej dlugości i przez punkty podziaIu poprowadzono plaszczyzny

równolegIe do podstawy, dzieląc w ten sposób ostrosIup na $n,$, warstw'' ZakIadając,

$\dot{\mathrm{z}}\mathrm{e}n\geq 3$, oblicz objętośč drugiej warstwy (liczac od podstawy).

6. Dany jest $n$-elementowy zbiór $S$. Ze zbioru wszystkich podzbiorów zbioru $S$ losu-

jemy kolejno ze zwracaniem dwa zbiory (prawdopodobieństwo wylosowania $\mathrm{k}\mathrm{a}\dot{\mathrm{z}}$ dego

zbioru jest jednakowe). Oblicz prawdopodobieństwa zdarzeń

$A$: przynajmniej jeden z wylosowanych zbiorów jest zbiorem pustym,

$B$: $\mathrm{k}\mathrm{a}\dot{\mathrm{z}}\mathrm{d}\mathrm{y}$ z wylosowanych zbiorów ma dokIadnie $n-1$ elementów,

$C$: wylosowane zbiory sa rozlaczne.

Wyniki zapisz w najprostszej postaci.

7. Dla jakich wartości parametru $p$ równanie

$(p-3)(9-4\sqrt{5})^{x}-(2p+6)(\sqrt{5}-2)^{x}+p+2=0$

ma dokladnie jeden pierwiastek?






AKADEMIA GÓRNICZO-HUTNICZA

im. StanisIawa Staszica w Krakowie

OLIMPIADA,, O DIAMENTOWY INDEKS AGH'' 2010/11

MATEMATYKA - ETAP III

ZADANIA PO 10 PUNKTÓW

l. Dany jest $n$-elementowy zbiór $X$ oraz jego $k$-elementowy podzbiór $S$. Ze zbioru $X$

wybieramy losowo $m$ elementów, tworzac zbiór $B$. Zakladając, $\dot{\mathrm{z}}\mathrm{e}k>0, m>0$ oraz

$m+k\leq n+1$, oblicz prawdopodobieństwo, $\dot{\mathrm{z}}\mathrm{e}$ zbiory $B\mathrm{i}S$ będa miaIy dokIadnie

jeden element wspólny.

2. Oblicz sumę wszystkich dwucyfrowych liczb naturalnych niepodzielnych przez 7.

3. Wyznacz dziedzinę funkcji $f$ danej wzorem

$f(x)=\displaystyle \frac{x^{3}+8}{x^{4}+2x^{3}+2x^{2}+4x}.$

Zbadaj granice funkcji $f$ w punktach nienalezacych do dziedziny.

4. Suma dwóch nieujemnych liczb rzeczywistych $x, y$ jest równa dodatniej liczbie $\alpha.$

Jaka najmniejsza wartośč $\mathrm{m}\mathrm{o}\dot{\mathrm{z}}\mathrm{e}$ mieč suma kwadratów liczb $x\mathrm{i}y$?

ZADANIA PO 20 PUNKTÓW

5. W prawidlowy graniastostup sześciokatny wpisano sferę (styczna do wszystkich ścian

bocznych i do obu podstaw). Oblicz stosunek pola powierzchni tej sfery do pola

powierzchni sfery opisanej na graniastostupie.

6. Dla jakich wartości parametru p równanie

$\displaystyle \frac{\log(px^{2})}{\log(x+1)}=2$

ma dokladnie jedno rozwiązanie?

7. Znajd $\acute{\mathrm{z}}$ równania stycznych do okręgu $C$ o równaniu

$x^{2}+y^{2}+6x-4y-12=0$

przechodzących przez punkt $P=(\displaystyle \frac{16}{3},2)$. Oblicz dlugośč promienia okręgu stycznego

do obydwu prostych i do okregu $C.$






AKADEMIA GÓRNICZO-HUTNICZA

im. StanisIawa Staszica w Krakowie

OLIMPIADA,, O DIAMENTOWY INDEKS AGH'' 2011/12

MATEMATYKA - ETAP III

ZADANIA PO 10 PUNKTÓW

l. Niech $a\mathrm{i}b$ będą dwiema liczbami rzeczywistymi, przy czym $a>b$. Udowodnij, $\dot{\mathrm{z}}\mathrm{e}$

$a^{3}-b^{3}\geq\alpha b^{2}-a^{2}b.$

2. Ile dzielników w zbiorze liczb naturalnych ma liczba 4$\cdot 5\cdot 6\cdot 7\cdot 8$ ?

3. Suma czterech początkowych wyrazów rosnącego ciagu arytmetycznego $(a_{n})$ jest

równa 0, a suma ich kwadratów wynosi 80. Znajd $\acute{\mathrm{z}}$ wzór na n-ty wyraz tego ciqgu.

4. Rozwiąz nierównośč

$1+\sqrt{x+5}>x.$

ZADANIA PO 20 PUNKTÓW

5. Ze zbioru $L=\{-2,-1,0,1,2\}$ losujemy ze zwracaniem dwie liczby $x, y$. Następnie

powtarzamy to losowanie dotad, $\mathrm{a}\dot{\mathrm{z}}$ otrzymamy punkt $(x,y)$ nalezacy do zbioru

$S=\{(x,y):|x|+|y|\leq 2\}.$

Oblicz prawdopodobieństwa zdarzeń:

A- bedziemy losowač doktadnie cztery razy,

B- liczba losowań będzie parzysta.

6. Dla jakich $m$ równanie

$\log_{3}(x-m)+\log_{3}x=\log_{3}(3x-4)$

ma dokladnie jedno rozwiazanie w zbiorze liczb rzeczywistych?

7. Prosta $2x+y-13=0$ zawiera bok $AB$ trójkąta $ABC$, prosta $x-y-5=0$ zawiera

bok $BC$, a prosta $3x-y-7=0$ zawiera dwusieczna kąta $ACB.$ Znajd $\acute{\mathrm{z}}$ wierzcholki

tego trójkata i oblicz jego pole.






AKADEMIA GÓRNICZO-HUTNICZA

im. Stanislawa Staszica w Krakowie

OLIMPIADA,, O DIAMENTOWY INDEKS AGH'' 2012/13

MATEMATYKA- ETAP III

ZADANIA PO 10 PUNKTÓW

l. Udowodnij, $\dot{\mathrm{z}}\mathrm{e}$ zbiór $S= \{6n+3:n\in 1N\}$, gdzie $mT$ jest zbiorem

wszystkich liczb naturalnych, zawiera nieskończenie wiele kwadratów

liczb calkowitych.

2. Rozwiąz równanie 4 $\cos^{2}2x=3.$

3. Sfera $S_{1}$ jest wpisana w sześcian, sfera $S_{2}$ jest styczna do wszystkich

krawędzi tego sześcianu, a sfera $S_{3}$ jest opisana na tym sześcianie.

Sprawd $\acute{\mathrm{z}}$, czy pola tych sfer tworza ciag geometryczny lub arytmety-

czny.

4. Rozwiqz nierównośč $\sqrt{x^{2}-16x+64}+x\leq 7+\sqrt{x^{2}+6x+9}.$

ZADANIA PO 20 PUNKTÓW

5. Wykaz, $\dot{\mathrm{z}}\mathrm{e}$ niezaleznie od wartości parametru $m$ równanie

$x^{3}-(m+1)x^{2}+(m+3)x-3=0$

ma pierwiastek calkowity. Dla jakich $m$ wszystkie pierwiastki rzeczy-

wiste tego równania sa calkowite?

6. Rzucamy $n$ razy sześcienną kością do gry. Oblicz prawdopodobieństwa

zdarzeń:

$A$: ani razu nie wypadta szóstka,

$B$: parzysta liczba oczek wypadla więcej razy $\mathrm{n}\mathrm{i}\dot{\mathrm{z}}$ nieparzysta,

$C$: suma wyrzuconych oczek jest równa $6n-2.$

7. Rozwiąz nierównośč

$3-\log_{0,5}x-\log_{0,5}^{2}x-\log_{0,5}^{3}x-\ldots\geq 4\log_{0,5}x.$






AKADEMIA GÓRNICZO-HUTNICZA

im. StanisIawa Staszica w Krakowie

OLIMPIADA,, O DIAMENTOWY INDEKS AGH'' 2013/14

MATEMATYKA- ETAP III

ZADANIA PO 10 PUNKTÓW

1. Rozwia $\dot{\mathrm{Z}}$ równanie

$(x^{2}+\displaystyle \frac{1}{2})^{\cos 2x}(x^{2}+\frac{1}{2})^{\sin 2x}=1.$

2. Rzucono trzy razy sześciennq kostkq do gry. Oblicz prawdopodobieństwo, $\dot{\mathrm{z}}\mathrm{e}$ suma

wyrzuconych oczek jest mniejsza $\mathrm{n}\mathrm{i}\dot{\mathrm{z}}$ sześč.

3. Po zmieszaniu roztworów soli o stęzeniach 8\% oraz 20\% otrzymano l2 litrów roz-

tworu o stęzeniu 16\%. Ob1icz objętości zmieszanych roztworów.

4. Rozwia $\dot{\mathrm{Z}}$ nierównośč

$3x^{2}+6x^{3}+12x^{4}+\ldots\leq 1.$

ZADANIA PO 20 PUNKTÓW

5. Wyznacz zbiór wszystkich liczb rzeczywistych $p$, dla których pierwiastki $x_{1}$ i $x_{2}$

równania

speIniaja nierównośč

$x+1=\displaystyle \frac{px}{p-1}+\frac{p+1}{x}$

$\displaystyle \frac{1}{x_{1}}+\frac{1}{x_{2}}\leq 2p+1.$

6. Dwie ściany ostroslupa trójkatnego są trójkątami równobocznymi o boku dlugości $a$

i dwie są trójkatami prostokatnymi. Oblicz pole powierzchni i objętośč ostroslupa.

7. Oblicz promień mniejszego z dwóch okręgów stycznych w punkcie $M(2,1)$ do prostej

$x-7y+5=0$ i jednocześnie stycznych do prostej $x+y+13=0$. Napisz równania

wszystkich okręgów o tym promieniu stycznych jednocześnie do obydwu prostych.






AKADEMIA GÓRNICZO-HUTNICZA

im. StanisIawa Staszica w Krakowie

OLIMPIADA,, O DIAMENTOWY INDEKS AGH'' 2014/15

MATEMATYKA- ETAP III

ZADANIA PO 10 PUNKTÓW

1. Znajd $\acute{\mathrm{z}}$ wszystkie liczby naturalne mniejsze $\mathrm{n}\mathrm{i}\dot{\mathrm{z}}7$, przez które podzielna jest liczba

$L=3^{2016}+4.$

2. Rozwiqz równanie

2 $\cos^{3}x+5\sin^{2}x-11\cos x-9=0.$

3. Oblicz pole równolegtoboku zbudowanego na wektorach $\vec{u}=[3,-4]\mathrm{i}\vec{v}=[4$, 4$].$

4. Rozwiqz nierównośč

$25\cdot 0,04^{x}-0,2^{x^{2}-2}\leq 0.$

ZADANIA PO 20 PUNKTÓW

5. Wartośč funkcji $g$ w punkcie $m$ jest równa sumie pierwiastków równania

$|mx^{2}-2x|=m,$

przy czym $\mathrm{k}\mathrm{a}\dot{\mathrm{z}}\mathrm{d}\mathrm{y}$ pierwiastek jest w tej sumie uwzględniany tylko raz niezaleznie od

jego krotności. Znajd $\acute{\mathrm{z}}$ funkcję $g:m\rightarrow g(m)$ i naszkicuj jej wykres.

6. Ze zbioru $\{$1, 2, $\ldots, n\}$ losujemy kolejno bez zwracania $k$ liczb, otrzymujqc ciqg

$(a_{1},a_{2},\ldots,a_{k})$. Wiedzqc, $\dot{\mathrm{z}}\mathrm{e}3\leq k\leq n$, oblicz prawdopodobieństwa zdarzeń:

A- $a_{k}$ jest najwiQkszq liczbq wśród wylosowanych;

B- $a_{k}$ jest podzielna przez 3;

$\displaystyle \mathrm{C}-a_{1}+a_{2}+\ldots+a_{k}>\frac{1}{2}k(k+1).$

7. Wyznacz wysokośč stozka o najmniejszej objętości opisanego na kuli o promieniu

$R=2$ cm.






AKADEMIA GÓRNICZO-HUTNICZA

im. StanisIawa Staszica w Krakowie

OLIMPIADA,, O DIAMENTOWY INDEKS AGH'' 2015/16

MATEMATYKA- ETAP III

ZADANIA PO 10 PUNKTÓW

1. Znajd $\acute{\mathrm{z}}$ wszystkie pary liczb calkowitych $(x,y)$ spelniajqcych równanie

$(x-2y-1)(x+2y+1)=3.$

2. Przy okragIym stole z l0 ponumerowanymi krzesIami siada 5 kobiet i 5 męzczyzn,

wybierajac miejsca w sposób przypadkowy. Jakie jest prawdopodobieństwo, $\dot{\mathrm{z}}\mathrm{e}$ choč

jedna osoba usiadzie obok osoby tej samej plci?

3. Rozwia $\dot{\mathrm{Z}}$ równanie

$|\cos x|^{2\cos x+1}=1.$

4. Dla jakich $\alpha$ liczby

$\log_{0,5}a^{2},$

$3+\log_{0,5}a, -1-\log_{0,5}2a^{3}$

sa kolejnymi wyrazami ciągu arytmetycznego?

ZADANIA PO 20 PUNKTÓW

5. Na p{\it l}aszczy $\acute{\mathrm{z}}\mathrm{n}\mathrm{i}\mathrm{e}$ dane sa punkty $A=(2,1), B=(-2,7), C=(-6,5).$

a) Znajd $\acute{\mathrm{z}}$ wspólrzędne punktu $D$, dla którego czworokat ABCD (w tej kolejności

wierzchoIków) jest równoleglobokiem.

b) Figura $F$ jest suma prostej $AB$ i prostej $CD$. Napisz równania wszystkich osi

symetrii figury $F.$

c) Znajd $\acute{\mathrm{z}}$ obraz figury $F$ w jednokIadności o środku w punkcie $A$ i skali równej 3.

6. Funkcja $f$ przyporządkowuje $\mathrm{k}\mathrm{a}\dot{\mathrm{z}}$ dej liczbie rzeczywistej $m$ liczbe pierwiastków rów-

nania

$|$--4{\it xx}2$++$22$|=${\it m}.

Naszkicuj wykres funkcji $f.$

7. Trójkąt równoramienny o obwodzie 36 cm obraca się wokól prostej zawierajacej

podstawę trójkąta. Jakie powinny byč wymiary tego trójkata, aby objQtośč powstalej

bryly byla największa?






AKADEMIA GÓRNICZO-HUTNICZA

im. Stanislawa Staszica w Krakowie

OLIMPIADA O DIAMENTOWY INDEKS AGH'' 2016/17

MATEMATYKA- ETAP III

ZADANIA PO 10 PUNKTÓW

l. Udowodnij, $\dot{\mathrm{z}}\mathrm{e}$ dla dowolnych dwóch dodatnich liczb rzeczywistych $a, b$ spefniona

jest nierównośč

$\sqrt{}ab\geq$ ---{\it a}1$+$2-{\it b}$1^{\cdot}$

2. Oblicz $\displaystyle \log_{8}\cos\frac{11}{6}\pi$ -log8 tg $(-\displaystyle \frac{17}{3}\pi).$

3. Funkcja $f$ dana wzorem

$f(x)=\displaystyle \{\frac{x^{m}-1}{x-1,a_{m}}$

dla

dla

$x\neq 1$

$x=1$

jest ciągfa w punkcie $x=1$. Wyznacz $a_{2}, a_{6}$ oraz $a_{m}$ dla dowolnej dodatniej liczby

całkowitej $m.$

4. Zbadaj, czy trójkąt o wierzcholkach $A = (-2,0), B = (1,-1), C = (0,7)$ jest

ostrokątny, prostokątny, czy rozwartokątny.

ZADANIA PO 20 PUNKTÓW

5. Liczba $a$ jest losowo wybrana spośród wszystkich siedmiocyfrowych liczb natu-

ralnych. Oblicz prawdopodobieństwa zdarzeń:

$A$: przynajmniej jedna z cyfr 0, 11ub 2 występuje w zapisie 1iczby $a$;

$B$: kolejne cyfry liczby $a$ opisują siedmiowyrazowy ciąg arytmetyczny;

$C$: kolejne cyfry liczby $a$ opisuja siedmiowyrazowy ciąg malejący.

6. $\mathrm{W}$ trapez prostokqtny o najkrótszym boku długości $a$ wpisany jest okrąg o pro-

mieniu $\displaystyle \frac{2}{3}a$. Oblicz pole trapezu i stosunek dfugości jego przekątnych.

7. Dany jest uklad równań

$\left\{\begin{array}{l}
(p+2)x+4y\\
3x+2y
\end{array}\right.$

$2p+4$

$4$

a) Dla jakich $p$ układ ma dokfadnie jedno rozwiqzanie $(x,y)$ ?

b) Jaką największ$\Phi$ wartośč, ajaką najmniejszq, $\mathrm{o}\mathrm{s}\mathrm{i}_{\Phi \mathrm{g}}\mathrm{a}$ iloczyn $xy$ dla $ p\in\langle 0;3\rangle$?






AKADEMIA GÓRNICZO-HUTNICZA

im. Stanislawa Staszica w Krakowie

OLIMPIADA O DIAMENTOWY INDEKS AGH'' 2017/18

MATEMATYKA- ETAP III

ZADANIA PO 10 PUNKTÓW

l. W układzie współrzędnych narysuj zbiór

\{({\it x, y})

$: x^{3}-y^{3}\geq xy^{2}-x^{2}y\}.$

2. Na ile sposobów $\mathrm{m}\mathrm{o}\dot{\mathrm{z}}$ emy $n$ początkowych liczb naturalnych 1, 2, $\ldots, n$

ustawič w ciąg, tak by choč jedna liczba parzysta nie miała dwóch

sąsiednich wyrazów nieparzystych?

3. Napisz równanie obrazu okręgu $x^{2}+y^{2}+4x-6y+8=0$ przez translację

o wektor $\vec{v}=[2,-4]$. Czy te dwa okręgi mają punkty wspólne?

4. $\mathrm{Z}$ punktu $P$ na okręgu o promieniu $r = 4$ cm poprowadzono cięci-

wę $PQ$ nachylonq do średnicy $PR$ pod kątem $\alpha = 15^{o}$ Oblicz pole

trójkąta $PQR.$

ZADANIA PO 20 PUNKTÓW

5. Znajd $\acute{\mathrm{z}}$ sumę wszystkich pierwiastków równania

$\sqrt{3}|\mathrm{c}\mathrm{t}\mathrm{g}x+\mathrm{t}\mathrm{g}x|=4$

spełniających nierównośč

$(\sqrt{2-\sqrt{3}})^{x}+(\sqrt{2+\sqrt{3}})^{x}\leq 4.$

6. Jaką największą objętośč $\mathrm{m}\mathrm{o}\dot{\mathrm{z}}\mathrm{e}$ mieč stozek wpisany w kulę o promie-

niu $R$?

7. Rzucamy sześcienną kostką do momentu uzyskania,,szóstki'' Niech $k$

będzie dowolną, dodatnią liczbą calkowitą. Oblicz prawdopodobień-

stwo, $\dot{\mathrm{z}}\mathrm{e}$ liczba rzutów będzie

$A$: równa $k, B$: mniejsza $\mathrm{n}\mathrm{i}\dot{\mathrm{z}}k, C$: parzysta.






AKADEMIA GÓRNICZO-HUTNICZA

im. Stanislawa Staszica w Krakowie

OLIMPIADA O DIAMENTOWY INDEKS AGH'' 2018/19

MATEMATYKA- ETAP III

ZADANIA PO 10 PUNKTÓW

l. Ze zbioru dziesięciu kolejnych liczb naturalnych usunięto jedną z nich.

Suma pozostałych liczb wynosi 2019. Znajd $\acute{\mathrm{z}}$ sumę wszystkich dziesię-

ciu liczb.

2. Ostrosłup podzielono na dwie bryły płaszczyzną równoległą do pod-

stawy i dzielącą jego wysokośč na dwa przystające odcinki. Jaki pro-

cent objętości ostrosłupa stanowi objętośč większej z tych brył?

3. Wyznacz liczbę $p$, dla której

$\displaystyle \lim_{n\rightarrow\infty}(n-\sqrt[3]{n^{3}+pn^{2}})=-2.$

4. Oblicz dlugości przekątnych równoległoboku o bokach dfugości 3 $\mathrm{i}5,$

przy czym sinus kąta wewnętrznego jest równy 0,8.

ZADANIA PO 20 PUNKTÓW

5. Wyznacz zbiór $(A\backslash B)\cap C$, gdzie

$A=\{x\in \mathbb{R}:\log_{\frac{1}{4}}(2^{x}+10)\leq 0,5+2\log_{\frac{1}{4}}(2^{x}-2)\},$

$B=\{x\in \mathbb{R}:x+1\leq\sqrt{x+3}\},$

$C=\{n\in \mathbb{N}:\sqrt{n}\left(\begin{array}{lll}
n & + & 2\\
 & 2 & 
\end{array}\right)>3^{n-1}\}.$

6. Losowo dzielimy $n$-elementowy zbiór $X$ na dwa zbiory $S\mathrm{i}X\backslash S$, przy

czym dla dowolnego $a\in X$ prawdopodobieństwo, $\dot{\mathrm{z}}\mathrm{e}a$ zostanie wylo-

sowany do zbioru $S$ wynosi $\displaystyle \frac{1}{2}$. Oblicz prawdopodobieństwa zdarzeń

$A$ : zbiór $S$ ma dokładnie $k$ elementów;

{\it B} : $\dot{\mathrm{z}}$ aden ze zbiorów $S\mathrm{i}X\backslash S$ nie jest pusty;

$C$ : zbiór $S$ zawiera więcej elementów $\mathrm{n}\mathrm{i}\dot{\mathrm{z}}$ zbiór $X\backslash S.$

7. Spośród wszystkich trójkątów prostokątnych o przeciwprostokatnej

długości $c$ wskazač ten, dla którego największa jest objętośč bryly

obrotowej, powstałej z obrotu tego trójkąta wokóf przyprostokątnej,

a) która jest krótsza.

b) która nie jest krótsza.






AKADEMIA GÓRNICZO-HUTNICZA

im. Stanisława Staszica w Krakowie

OLIMPIADA O DIAMENTOWY INDEKS AGH'' 2020/21

MATEMATYKA- ETAP III

ZADANIA PO 10 PUNKTÓW

1. $\mathrm{W}$ czworokącie ABCD kąt wewnętrzny przy wierzcholku $A$ jest kątem pros-

tym. Dfugości boków są równe $|AB| = |AD| = 15, |BC| =3, |CD| =21.$

Oblicz pole tego czworokąta.

2. Dany jest ciąg $(a_{n})$, którego n-ty wyraz jest równy

$a_{n}=40-6n.$

Oblicz sumę tych 20 początkowych wyrazów ciqgu, które są ujemne i po-

dzielne przez 8.

3. Rozwiqz równanie

$2^{\cos^{2}x}+2^{\sin^{2}x}=3.$

4. $\mathrm{W}$ wycieczce bierze udzial $2n$ osób. $K\mathrm{a}\dot{\mathrm{z}}$ da z nich ma wśród uczestników

wycieczki co najmniej $n-1$ innych osób pochodzących z tego samego miasta

co ona. $\mathrm{Z}$ ilu miast pochodzą uczestnicy wycieczki? $\mathrm{W}\mathrm{k}\mathrm{a}\dot{\mathrm{z}}$ dym $\mathrm{m}\mathrm{o}\dot{\mathrm{z}}$ liwym

przypadku podaj liczby uczestników z poszczególnych miast.

ZADANIA PO 20 PUNKTÓW

5. $\mathrm{W}$ urnie znajdują się 2 ku1e białe i 10 czerwonych.

a) Losujemy ze zwracaniem 2 ku1e. Ob1icz prawdopodobieństwo, $\dot{\mathrm{z}}\mathrm{e}$ wylo-

sujemy kule o róznych kolorach.

b) Losujemy bez zwracania $k$ kul. Wyznacz najmniejszą wartośč $k$, dla

której prawdopodobieństwo wylosowania co najmniej jednej kuli biafej jest

większe od 0, 5.

6. Dla jakich wartości parametru $p$ równanie

$\displaystyle \frac{\log_{3}(px+p)}{\log_{3}(3+x)}=2$

ma dokladnie jedno rozwiązanie?

7. $\mathrm{W}$ stozku o promieniu podstawy $R$ i wysokości $H$ zawartyjest graniastoslup

prawidlowy czworokątny tak, $\dot{\mathrm{z}}\mathrm{e}$ jego podstawa zawiera się w podstawie

stozka. Jaką największą objętośč $\mathrm{m}\mathrm{o}\dot{\mathrm{z}}\mathrm{e}$ mieč ten graniastosłup?






AKADEMIA GÓRNICZO-HUTNICZA

im. Stanislawa Staszica w Krakowie

OLIMPIADA O DIAMENTOWY INDEKS AGH'' 2021/22

MATEMATYKA- ETAP III

ZADANIA PO 10 PUNKTÓW

1. Rozwia $\dot{\mathrm{z}}$ nierównośč

$|\log_{0,5}(2-x)|\geq 1.$

2. Oblicz sume wszystkich liczb dwucyfrowych podzielnych przez 6 lub przez 8.

3. Znajd $\acute{\mathrm{z}}$ równania prostych stycznych do krzywej $y=x^{2}+\displaystyle \frac{1}{x}$ i prostopadfych

do prostej $4x+15y-3=0.$

4. Punkt $S$ jest środkiem wysokości czworościanu foremnego ABCD opuszczonej

z wierzchołka $D$. Wyznacz miarę kąta $ASB.$

ZADANIA PO 20 PUNKTÓW

5. Niech $n$ będzie dowolną dodatnią liczbą cafkowitą. Ze zbioru dodatnich liczb

całkowitych mniejszych od $3n$ losujemy ze zwracaniem trzy liczby. Oblicz praw-

dopodobieństwo, $\dot{\mathrm{z}}\mathrm{e}$ dokładnie jedna z tych liczb jest równa $n.$

Niech $p_{n}$ oznacza prawdopodobieństwo, $\dot{\mathrm{z}}\mathrm{e}$ iloczyn tych trzech liczb jest po-

dzielny przez 3. Ob1icz

$\displaystyle \lim_{n\rightarrow\infty}p_{n}.$

6. Dla jakich wartości parametru $m$ równanie

$m\cdot 2^{x}+(m+3)2^{-x}=4$

ma dokładnie jedno rozwiqzanie?

7. Prosta przechodząca przez punkt $M=(3,1)$ ogranicza wraz z dodatnimi pól-

osiami układu współrzędnych $XOY$ trójkąt o najmniejszym polu. Wokół któ-

rego boku nalezy obracač ten trójkąt, aby otrzymana bryła obrotowa miała

$\mathrm{n}\mathrm{a}\mathrm{j}\mathrm{m}\mathrm{n}\mathrm{i}\mathrm{e}\mathrm{j}\mathrm{s}\mathrm{z}\Phi$ objętośč? Podaj tę objętośč.






AKADEMIA GÓRNICZO-HUTNICZA

im. Stanisława Staszica w Krakowie

OLIMPIADA O DIAMENTOWY INDEKS AGH'' 2020/21

MATEMATYKA- ETAP I

ZADANIA PO 10 PUNKTÓW

l. W trapezie ABCD dfuzsza podstawa AB ma długośč 48. Odcinek fączący

środki E, F przekątnych ma dlugośč 4. Ob1icz d1ugośč krótszej podstawy.

2. Znajd $\acute{\mathrm{z}}$ wszystkie elementy zbioru $\displaystyle \{\cos\frac{(n^{7}-n)\pi}{12}$

czerpująco uzasadnij.

: $n\in \mathbb{N}\}.$ Odpowied $\acute{\mathrm{z}}$ wy-

3. Oblicz długośč najdłuzszej krawędzi prostopadłościanu o objętości 2l6 i prze-

katnej dlugości $2\sqrt{91}, \mathrm{j}\mathrm{e}\dot{\mathrm{z}}$ eli długości krawędzi wychodzqcych z jednego

wierzcholka tworzą $\mathrm{c}\mathrm{i}_{\Phi \mathrm{g}}$ geometryczny.

4. Naszkicuj wykres funkcji danej wzorem

$f(x)=x-|x|-2^{|x|+x}$

Na podstawie tego wykresu podaj liczbę rozwiązań równania 3 $f(x+5)=m$

w zalezności od parametru $m.$

ZADANIA PO 20 PUNKTÓW

5. Niech $H$ będzie zbiorem wszystkich tych punktów hiperboli o równaniu

$x^{2}-y^{2} = 25$, których obie wspólrzędne są liczbami cafkowitymi. Napisz

równania okręgów, $\mathrm{z}\mathrm{a}\mathrm{w}\mathrm{i}\mathrm{e}\mathrm{r}\mathrm{a}\mathrm{j}_{\Phi}$cych co najmniej po cztery punkty zbioru $H.$

6. Dla jakich wartości parametru $p$ uklad równań

$\left\{\begin{array}{l}
4x+(p+3)y=p-1\\
(p-1)x+py=p-2.
\end{array}\right.$

ma dokładnie jedno rozwiazanie spełniające nierównośč $|x|+|y|\leq 4$ ?

7. Dla danej liczby naturalnej $n\geq 3$ rozwazmy zbiór $\Omega$ wszystkich permutacji

$(a_{1},\ldots,a_{n})$ liczb l, $\ldots, n.$

A. Ile jest permutacji niebędących ciągami monotonicznymi?

B. Ile jest permutacji, takich $\dot{\mathrm{z}}\mathrm{e}a_{i}+a_{n-i+1} =a_{j}+a_{n-j+1}$ dla wszystkich

$i, j=1, \ldots, n$?

C. Dane są liczby naturalne $k, m$, przy czym $1 <k<m\leq n$. Dla $\mathrm{k}\mathrm{a}\dot{\mathrm{z}}$ dej

permutacji $a=(a_{1},a_{2},\ldots,a_{n})$ ze zbioru $\Omega$ oznaczmy przez $g(a)$ największq

liczbę $j\leq n, \mathrm{t}\mathrm{a}\mathrm{k}_{\Phi}\dot{\mathrm{z}}\mathrm{e}a_{i}<a_{i+1}$ dla wszystkich $i<j$. Ile jest permutacji $a,$

dla których $g(a)=k$ i jednocześnie $a_{k}=m$?






AKADEMIA GÓRNICZO-HUTNICZA

im. Stanisława Staszica w Krakowie

OLIMPIADA O DIAMENTOWY INDEKS AGH'' 2021/22

MATEMATYKA- ETAP I

ZADANIA PO 10 PUNKTÓW

l. Udowodnij, $\dot{\mathrm{z}}\mathrm{e}$ dla dowolnych dodatnich liczb rzeczywistych $a, b$ prawdziwa

jest nierównośč

$a^{a-b}\geq b^{a-b}$

2. Długości boków trójkąta prostokqtnego tworzą rosnący ciag arytmetyczny.

Wykaz$\cdot, \dot{\mathrm{z}}\mathrm{e}$ róznic$\Phi$ ciągu jest dlugośč promienia okręgu wpisanego w ten

trójkąt.

3. Cztery kolejne liczby parzyste są pierwiastkami wielomianu o wspófczyn-

nikach całkowitych. Udowodnij, $\dot{\mathrm{z}}\mathrm{e}$ wartośč tego wielomianu dla dowolnej

liczby parzystej jest podzielna przez 384.

4. Udowodnij, $\dot{\mathrm{z}}\mathrm{e}$ dla dowolnego trójk$\Phi$ta o dfugościach boków $a, b, c$

$2\sqrt{a^{2}+b^{2}+c^{2}}<\sqrt{3}(a+b+c).$

ZADANIA PO 20 PUNKTÓW

5. Wyznacz zbiór wartości funkcji $g$ danej wzorem

$g(x)=\cos 4x+5\cos^{2}x+\sin^{2}x.$

Dla jakich argumentów $x$ funkcja $g$ przyjmuje najmniejszq wartośč?

6. Dana jest liczba naturalna $k \geq 4$. Na ile sposobów $\mathrm{m}\mathrm{o}\dot{\mathrm{z}}$ na $k+1$ zadań

przydzielič $k$ komputerom, tak by dokfadnie jeden komputer byl wolny,

$\mathrm{j}\mathrm{e}\dot{\mathrm{z}}$ eli

a) zadania i komputery są rozróznialne,

b) komputery $\mathrm{s}\Phi$ rozróznialne, a zadania nie,

c) zadania są rozróznialne, a komputery nie,

d) ani zadania, ani komputery nie $\mathrm{s}\varpi$ rozróznialne?

7. Na p{\it l}aszczy $\acute{\mathrm{z}}\mathrm{n}\mathrm{i}\mathrm{e}$ dane są zbiory

$S=\{(x,y):\log_{|y+1|}x\leq 1\},$

$A_{m}=\{(x,y):x^{2}+y^{2}-2mx-4y+4\leq 0\},$

gdzie $m \in$ lR. Narysuj zbiór $S$. Dla jakich liczb $m$ zbiór $A_{m}$ zawiera się

w zbiorze $S$?






AKADEMIA GÓRNICZO-HUTNICZA

im. Stanislawa Staszica w Krakowie

OLIMPIADA O DIAMENTOWY INDEKS AGH'' 2022/23

MATEMATYKA- ETAP I

ZADANIA PO 10 PUNKTÓW

1. $\mathrm{W}$ prostokqcie ABCD wierzchołek $A$ połączono odcinkami ze środkami boków

$BC \mathrm{i}$ CD. Udowodnij, $\dot{\mathrm{z}}\mathrm{e}$ te odcinki dzielą $\mathrm{P}^{\mathrm{r}\mathrm{z}\mathrm{e}\mathrm{k}}\Phi^{\mathrm{t}\mathrm{n}}\Phi BD$ na trzy odcinki

równej długości.

2. Oblicz sume stu największych ujemnych rozwiązań równania

4 $\cos 2x-\sin 4x=4\cos^{3}2x.$

3. Rozwiąz równanie

$\sqrt[6]{-x^{2}+5x-6}=\sqrt[4]{x^{3}-4x^{2}+x+6}.$

4. $\mathrm{W}$ wypukłym pieciokącie ABCDE $\mathrm{k}\mathrm{a}\dot{\mathrm{z}}$ da przekątna odcina trójk$\Phi$t o polu rów-

nym l. Oblicz pole tego pięciokąta.

ZADANIA PO 20 PUNKTÓW

5. Znajd $\acute{\mathrm{z}}$ równanie stycznej do paraboli $y=2-x^{2}$, która ogranicza wraz z do-

datnimi półosiami układu współrzędnych trójkąt o najmniejszym polu.

6. Niech $S$ będzie zbiorem wszystkich ciągów $(a,b,c,d,e)$ o wyrazach nalezących

do zbioru liczb $\{0$, 1, $\ldots$, 9$\}$. Ile jest w zbiorze $S$ ci$\Phi$gów

a) malejących?

b) których iloczyn abcde jest liczbą parzystą?

c) w których suma cyfr iloczynu abcde w zapisie dziesiętnym jest podzielna

przez 9?

7. Znajd $\acute{\mathrm{z}}$ równania prostych stycznych do okręgu $x^{2}+y^{2}+4x-12=0$ i jedno-

cześnie do jego obrazu w symetrii osiowej względem prostej $2x-3y-22=0.$




\end{document}