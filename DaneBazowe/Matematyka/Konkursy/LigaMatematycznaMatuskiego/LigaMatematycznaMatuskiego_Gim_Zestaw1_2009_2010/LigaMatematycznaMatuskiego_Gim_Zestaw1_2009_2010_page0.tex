\documentclass[a4paper,12pt]{article}
\usepackage{latexsym}
\usepackage{amsmath}
\usepackage{amssymb}
\usepackage{graphicx}
\usepackage{wrapfig}
\pagestyle{plain}
\usepackage{fancybox}
\usepackage{bm}

\begin{document}

LIGA MATEMATYCZNA

$\mathrm{P}\mathrm{A}\acute{\mathrm{Z}}$ DZIERNIK 2009

GIMNAZJUM

ZADANIE I.

Przez wierzchołek kwadratu poprowadzono prostą, która dzieli kwadrat na trójkąt o polu 24 $\mathrm{c}\mathrm{m}^{2}$

i trapez o polu 40 $\mathrm{c}\mathrm{m}^{2}$ Oblicz długości podstaw trapezu.

ZADANIE 2.

Rozwiqz uklad równań

$\left\{\begin{array}{l}
x(y+z)=8\\
y(z+x)=18\\
z(x+y)=20.
\end{array}\right.$

ZADANIE 3.

Uzasadnij, $\dot{\mathrm{z}}\mathrm{e}$ liczba

2007. 2009. 2011$+$8036

jest sześcianem liczby naturalnej.

ZADANIE 4.

Wyznacz wszystkie liczby siedmiocyfrowe podzielne przez 3 i przez 4, w zapisie których wystę-

pują tylko cyfry 2 $\mathrm{i}3$, przy czym dwójek jest więcej $\mathrm{n}\mathrm{i}\dot{\mathrm{z}}$ trójek.

ZADANIE 5.

$\mathrm{W}$ konkursie matematycznym uczeń ma rozwiązać 20 zadań. Za $\mathrm{k}\mathrm{a}\dot{\mathrm{z}}$ de zadanie poprawnie

rozwiązane otrzymuje 12 punktów, za z'1e rozwiązane $(-5)$ punktów, a za brak rozwiązania

0 punktów. $\mathrm{W}$ podsumowaniu otrzyma117 punktów. I1e zadań rozwiąza1 b1ędnie?
\end{document}
