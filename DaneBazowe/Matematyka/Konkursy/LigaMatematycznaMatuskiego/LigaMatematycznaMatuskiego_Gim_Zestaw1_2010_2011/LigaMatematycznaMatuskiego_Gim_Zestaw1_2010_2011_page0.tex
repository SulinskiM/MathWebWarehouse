\documentclass[a4paper,12pt]{article}
\usepackage{latexsym}
\usepackage{amsmath}
\usepackage{amssymb}
\usepackage{graphicx}
\usepackage{wrapfig}
\pagestyle{plain}
\usepackage{fancybox}
\usepackage{bm}

\begin{document}

LIGA MATEMATYCZNA

$\mathrm{P}\mathrm{A}\acute{\mathrm{Z}}$ DZIERNIK 2010

GIMNAZJUM

ZADANIE I.

Suma i iloczyn pewnych dziesięciu liczb calkowitych są parzyste. Ile najwięcej $\mathrm{m}\mathrm{o}\dot{\mathrm{z}}\mathrm{e}$ być wśród

nich liczb nieparzystych?

ZADANIE 2.

$\mathrm{W}$ turnieju tenisa stolowego wzięło udzial pięćdziesięciu zawodników. $\mathrm{K}\mathrm{a}\dot{\mathrm{z}}\mathrm{d}\mathrm{y}$ zawodnik rozegrał

jeden mecz z $\mathrm{k}\mathrm{a}\dot{\mathrm{z}}$ dym innym zawodnikiem. Nie bylo remisów. Czy $\mathrm{m}\mathrm{o}\dot{\mathrm{z}}$ liwe jest, aby $\mathrm{k}\mathrm{a}\dot{\mathrm{z}}\mathrm{d}\mathrm{y}$

z zawodników wygrał taką samą liczbę meczów?

ZADANIE 3.

Wykaz, $\dot{\mathrm{z}}\mathrm{e}$ wartość wyrazenia $\displaystyle \frac{1}{\sqrt{2}+\sqrt{3}}+\frac{1}{\sqrt{3}+\sqrt{4}}+\frac{1}{\sqrt{4}+\sqrt{5}}+\ldots+\frac{1}{\sqrt{127}+\sqrt{128}}$ jest mniejsza od 10.

ZADANIE 4.

Dany jest czworokąt wypukły ABCD. Niech $E$ będzie środkiem odcinka $AB$ oraz $F-$ środkiem

odcinka $CD$. Oblicz pole czworokąta EBFD wiedząc, $\dot{\mathrm{z}}\mathrm{e}$ pole czworokąta ABCD jest równe 77.

ZADANIE 5.

Jan, Henryk, Stanisfaw i Paweł to znajomi rzemiešlnicy. $K\mathrm{a}\dot{\mathrm{z}}\mathrm{d}\mathrm{y}$ z nich wykonuje inny zawód

i mieszka przy innej ulicy.

Na podstawie podanych informacji okrešl, przy jakiej ulicy $\mathrm{k}\mathrm{a}\dot{\mathrm{z}}\mathrm{d}\mathrm{y}$

z nich mieszka i jaki wykonuje zawód.

$\bullet$ Jan nie jest jubilerem.

$\bullet$ Zegarmistrz nie mieszka przy ulicy Złotej.

$\bullet$ Henryk jest krawcem, ale nie mieszka przy ulicy Glównej.

$\bullet$ Jubiler mieszka przy ulicy Mokrej.

$\bullet$ Stanislaw mieszka przy ulicy Cichej, ale nie jest szewcem.
\end{document}
