\documentclass[a4paper,12pt]{article}
\usepackage{latexsym}
\usepackage{amsmath}
\usepackage{amssymb}
\usepackage{graphicx}
\usepackage{wrapfig}
\pagestyle{plain}
\usepackage{fancybox}
\usepackage{bm}

\begin{document}

LIGA MATEMATYCZNA

$\mathrm{P}\mathrm{A}\acute{\mathrm{Z}}$ DZIERNIK 2011

GIMNAZJUM

ZADANIE I.

Pola $P$ niektórych figur plaskich $\mathrm{m}\mathrm{o}\dot{\mathrm{z}}$ emy obliczyć ze wzoru Simpsona

$P=\displaystyle \frac{d_{1}+4d+d_{2}}{6}\cdot h,$

w którym przyjęto następujące oznaczenia:

$d_{1}-$ dlugość dolnej podstawy;

$d-$ dlugość środkowego odcinka, równoleglego do podstawy dolnej w polowie wysokości;

$d_{2}-$ dlugość górnej podstawy;

$h-$ wysokość figury.

$\bullet$ Wykonaj rysunek, wprowad $\acute{\mathrm{z}}$ oznaczenia i $\mathrm{s}\mathrm{p}\mathrm{r}\mathrm{a}\mathrm{w}\mathrm{d}\acute{\mathrm{z}}$, czy ze wzoru Simpsona $\mathrm{m}\mathrm{o}\dot{\mathrm{z}}$ na otrzy-

mać wzór na pole trapezu. Odpowiedz' uzasadnij.

$\bullet$ Sprawd $\acute{\mathrm{z}}$, czy ze wzoru Simpsona $\mathrm{m}\mathrm{o}\dot{\mathrm{z}}$ na wyprowadzić wzór na pole sześciokata foremnego

o boku dlugości $a.$

ZADANIE 2.

$\mathrm{W}\mathrm{k}\mathrm{a}\dot{\mathrm{z}}$ dym kroku wykonujemy na liczbie jedną z operacji:

(a) podwajamy liczbę;

(b) skreślamy jej ostatnią cyfrę.

Czy w taki sposób po skończonej ilości operacji $\mathrm{m}\mathrm{o}\dot{\mathrm{z}}$ na z liczby 458 uzyskać 14?

ZADANIE 3.

Znajd $\acute{\mathrm{z}}$ wszystkie liczby dwucyfrowe $n$ spelniające warunek

$n-p=3\cdot f(n),$

gdzie $p$ oznacza liczbę dwucyfrową powstalą z przestawienia cyfr liczby $n$, a $f(n)-$ sumę cyfr

liczby $n$ oraz iloczynu jej cyfr.

ZADANIE 4.

Wykaz, $\dot{\mathrm{z}}\mathrm{e}$ liczba $(2\sqrt{2}+3)\sqrt{5-12\sqrt{3-2\sqrt{2}}}$ jest calkowita.

ZADANIE 5.

$\mathrm{W}$ liczbie czterystucyfrowej 84198419$\ldots$ 8419 skreśl pewną ilość cyfr z początku i końca tak,

aby suma pozostałych cyfr była równa 1984.
\end{document}
