\documentclass[a4paper,12pt]{article}
\usepackage{latexsym}
\usepackage{amsmath}
\usepackage{amssymb}
\usepackage{graphicx}
\usepackage{wrapfig}
\pagestyle{plain}
\usepackage{fancybox}
\usepackage{bm}

\begin{document}

LIGA MATEMATYCZNA

im. Zdzisława Matuskiego

$\mathrm{P}\mathrm{A}\acute{\mathrm{Z}}$ DZIERNIK 2012

GIMNAZJUM

ZADANIE I.

Zapisano wszystkie liczby naturalne od $n$ do $n^{2}$ Jest ich 601. Ob1icz $n.$

ZADANIE 2.

Udowodnij, $\dot{\mathrm{z}}\mathrm{e}$ suma kwadratów trzech kolejnych liczb całkowitych przy dzieleniu przez 3 daje

resztę 2.

ZADANIE 3.

$\mathrm{W}$ duzym kwadracie umieszczony jest maly kwadrat w taki sposób, $\dot{\mathrm{z}}\mathrm{e}$ jeden jego bok lezy

na przekątnej, a dwa wierzcholki na bokach $\mathrm{d}\mathrm{u}\dot{\mathrm{z}}$ ego kwadratu. Oblicz stosunek pól tych kwa-

dratów.

ZADANIE 4.

Dany jest trójkąt równoramienny $ABC$, w którym $AC = BC$. Na odcinku $AC$ wybrano

punkty $X$ oraz $Y$ w taki sposób, $\dot{\mathrm{z}}\mathrm{e}AB=BX=XY$ oraz $BY=YC$. Wyznacz miary kątów

wewnętrznych trójkąta $ABC.$

ZADANIE 5.

Wykaz$\cdot, \dot{\mathrm{z}}\mathrm{e}$ dla $\mathrm{k}\mathrm{a}\dot{\mathrm{z}}$ dej liczby naturalnej $n$ liczba $n^{3}-19n$ jest podzielna przez 6.


\end{document}