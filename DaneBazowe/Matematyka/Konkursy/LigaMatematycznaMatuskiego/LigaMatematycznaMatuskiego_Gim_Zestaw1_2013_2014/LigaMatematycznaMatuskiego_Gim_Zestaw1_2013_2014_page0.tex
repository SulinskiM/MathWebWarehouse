\documentclass[a4paper,12pt]{article}
\usepackage{latexsym}
\usepackage{amsmath}
\usepackage{amssymb}
\usepackage{graphicx}
\usepackage{wrapfig}
\pagestyle{plain}
\usepackage{fancybox}
\usepackage{bm}

\begin{document}

LIGA MATEMATYCZNA

im. Zdzisława Matuskiego

$\mathrm{P}\mathrm{A}\overline{\mathrm{Z}}$ DZIERNIK 2013

GIMNAZJUM

ZADANIE I.

Dany jest ulamek $\displaystyle \frac{a}{b}$. Do licznika tego ufamka dodano liczbę l.

do mianownika, aby otrzymać ufamek równy danemu?

Jaką liczbę nalezy dodać

ZADANIE 2.

Na przyjęcie przybyla pewna liczba gości. $K\mathrm{a}\dot{\mathrm{z}}\mathrm{d}\mathrm{y}$ z $\mathrm{k}\mathrm{a}\dot{\mathrm{z}}$ dym wymienif uścisk dloni, z wyjątkiem

pana Jana, który dwunastu gościom nie chcial podać ręki. $\mathrm{W}$ sumie wymieniono 2004 uściski

dłoni. Ile osób było na przyjęciu?

ZADANIE 3.

$\mathrm{W}$ finale Ligi Matematycznej uczestniczyfo stu uczniów. Uzasadnij, $\dot{\mathrm{z}}\mathrm{e}$ wśród nich byfo piętna-

stu (lub więcej) uczniów, którzy urodzili się w tym samym dniu tygodnia.

ZADANIE 4.

Długości boków kwadratów ABCD $\mathrm{i}$ KLMN sa równe 4 cm. Kwadraty te są tak połozone,

$\dot{\mathrm{z}}\mathrm{e}$ wierzchofek $K$ nalezy do boku $AD$, wierzcholek $L-$ do boku $AB$, a przekątne kwadratu

KLMN są prostopadfe do odpowiednich boków kwadratu ABCD. Oblicz pole figury będącej

częścią wspólną obu kwadratów. Oblicz odleglość wierzchofka $C$ od prostej $MN.$

ZADANIE 5.

Na Międzynarodową Olimpiadę Matematyczną przyjechalo 1000 osób. $\mathrm{W}$ sprawozdaniu po-

dano, $\dot{\mathrm{z}}\mathrm{e}$ wśród nich 811 wfada językiem angie1skim, 752- językiem rosyjskim, 418- językiem

francuskim, $356 -$ językiem rosyjskim i francuskim, $570 -$ językiem rosyjskim i angielskim,

348- językiem angielskim i francuskim, 297 osób mówi wszystkimi trzema językami. Wykaz,

$\dot{\mathrm{z}}\mathrm{e}$ w sprawozdaniu popełniono bfąd.
\end{document}
