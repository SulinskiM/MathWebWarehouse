\documentclass[a4paper,12pt]{article}
\usepackage{latexsym}
\usepackage{amsmath}
\usepackage{amssymb}
\usepackage{graphicx}
\usepackage{wrapfig}
\pagestyle{plain}
\usepackage{fancybox}
\usepackage{bm}

\begin{document}

LIGA MATEMATYCZNA

im. Zdzisława Matuskiego

$\mathrm{P}\mathrm{A}\overline{\mathrm{Z}}$ DZIERNIK 2014

GIMNAZJUM

ZADANIE I.

Piszemy liczbę l, potem 0. Trzecią 1iczbą jest róznica 1iczby drugiej i pierwszej, czwartą -

róznica trzeciej i drugiej, piatą- róznica czwartej i trzeciej, i tak dalej. Wyznacz liczbę stojącą

na 2014 miejscu.

ZADANIE 2.

Wyznacz wszystkie pary liczb naturalnych $a, b$ spefniające warunek $a^{2}-4b^{2}=45.$

ZADANIE 3.

$\mathrm{W}$ prostokącie o bokach dlugości 9 cm i 7 cm umieszczono prostokąt tak, $\dot{\mathrm{z}}\mathrm{e}$ jedna z jego

przekątnych fączy środki krótszych boków większego prostokąta, a dwa pozostale wierzchołki

mniejszego prostokąta lezą na dluzszych bokach większego prostokąta. Oblicz obwód mniejszego

prostokąta.

ZADANIE 4.

Między dwiema dodatnimi liczbami cafkowitymi $a\mathrm{i}b$ jest dziesięć liczb cafkowitych większych

od $a$ i mniejszych od $b$, zaś między $a^{2}\mathrm{i}b^{2}$ jest tysiąc liczb cafkowitych większych od $a^{2}$ i mniej-

szych od $b^{2}$ Wyznacz a $\mathrm{i}b.$

ZADANIE 5.

Dany jest kwadrat ABCD o boku l. Punkt $M$ jest środkiem boku $BC, L$ jest środkiem boku

$CD$. Odcinki AM $\mathrm{i}BL$ podzieliły kwadrat na cztery obszary. Oblicz pole $\mathrm{k}\mathrm{a}\dot{\mathrm{z}}$ dego z nich.
\end{document}
