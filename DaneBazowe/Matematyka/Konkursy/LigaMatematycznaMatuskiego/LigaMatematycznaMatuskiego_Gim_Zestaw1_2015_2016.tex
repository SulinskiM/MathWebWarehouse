\documentclass[a4paper,12pt]{article}
\usepackage{latexsym}
\usepackage{amsmath}
\usepackage{amssymb}
\usepackage{graphicx}
\usepackage{wrapfig}
\pagestyle{plain}
\usepackage{fancybox}
\usepackage{bm}

\begin{document}

LIGA MATEMATYCZNA

im. Zdzisława Matuskiego

$\mathrm{P}\mathrm{A}\dot{\mathrm{Z}}$ DZIERNIK 2015

GIMNAZJUM

ZADANIE I.

Wykaz$\cdot, \dot{\mathrm{z}}\mathrm{e}$ liczba $6^{100}-2\cdot 6^{99}+10\cdot 6^{98}$ jest podzielna przez 17.

ZADANIE 2.

$\mathrm{W}$ pólokrąg o promieniu 5 wpisano dwa kwadraty, jak na rysunku.

kwadratów.

Oblicz sumę pól tych

ZADANIE 3.

Czy istnieją takie liczby naturalne $m, n$, aby w wyniku mnozenia ich sumy przez ich iloczyn

otrzymač liczbę 20162015?

ZADANIE 4.

Rozwazmy $1001\mathrm{l}\mathrm{i}\mathrm{c}\mathrm{z}\mathrm{b}$: 1, $1+2, 1+2+3$, 1$+$2$+$3$+$4, $\ldots, 1+2+3+\ldots+1000, 1+2+3+\ldots+1001.$

Ile jest wśród nich liczb parzystych?

ZADANIE 5.

Przygotowując prezent dla Ani, Bartek wlozyl go do malego pudefka, to pudelko wlozył do

większego, a to do jeszcze większego, przy czym $\mathrm{k}\mathrm{a}\dot{\mathrm{z}}$ de następne pudelko cafkowicie mieścifo

poprzednie. Ustal, w jakiej kolejności bral pudelka, $\mathrm{j}\mathrm{e}\dot{\mathrm{z}}$ eli wiadomo, $\dot{\mathrm{z}}\mathrm{e}$:

$\bullet$ pudelko zólte jest prostopadfościanem o objętości 12144 $\mathrm{c}\mathrm{m}^{\mathrm{S}}$ i jego jedna ściana ma wy-

miary 23 cm i 24 cm;

$\bullet$ pudelko zielone jest sześcianem o objętości 8000 $\mathrm{c}\mathrm{m}^{3}$;

$\bullet$ pudefko rózowe jest sześcianem o sumie dfugości wszystkich krawędzi równej 312 cm.


\end{document}