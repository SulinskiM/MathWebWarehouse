\documentclass[a4paper,12pt]{article}
\usepackage{latexsym}
\usepackage{amsmath}
\usepackage{amssymb}
\usepackage{graphicx}
\usepackage{wrapfig}
\pagestyle{plain}
\usepackage{fancybox}
\usepackage{bm}

\begin{document}

LIGA MATEMATYCZNA

im. Zdzisława Matuskiego

$\mathrm{P}\mathrm{A}\dot{\mathrm{Z}}$ DZIERNIK 2016

GIMNAZJUM

ZADANIE I.

$\mathrm{W}$ okrąg o promieniu o długości 10 wpisano prostokat ABCD. Następnie na tym okręgu

wybrano dowolny punkt $E$. Oblicz sumę kwadratów odleglości punktu $E$ od wierzchołków

prostokąta, czyli $|EA|^{2}+|EB|^{2}+|EC|^{2}+|ED|^{2}$

ZADANIE 2.

Ile jest dodatnich liczb calkowitych, których największy dzielnik wlaściwy (to znaczy dzielnik

rózny od l i od danej liczby) jest równy 91?

ZADANIE 3.

Ania ma 36 karteczek. Poma1owafaje $\mathrm{u}\dot{\mathrm{z}}$ ywając trzech kolorów: zielonego, czerwonego i niebie-

skiego. Niektóre karteczki są pomalowane tylko jednym kolorem, inne dwoma, a pozostale pięć

karteczek wszystkimi trzema kolorami. Zielonej kredki $\mathrm{u}\dot{\mathrm{z}}$ yla do pokolorowania 25 karteczek,

czerwonej do 28, a niebieskiej do 20 karteczek. I1e karteczek Ania poma1owa1ajednym ko1orem?

ZADANIE 4.

Operacją nazywamy przyporządkowanie trójce liczb $(a,b,c)$ nowej trójki $(b+c,a+c,a+b).$

Początkową trójka jest (1, 3, 5). Po wykonaniu 2016 takich operacji na otrzymywanych trójkach

liczb uzyskano $(x,y,z)$. Oblicz róznicę $x-y.$

ZADANIE 5.

Dany jest prostokąt ABCD, w którym $|AB|=20, |BC|=10$. Punkty $W\mathrm{i}K$ lezą na zewnątrz

tego prostokąta oraz $|WA| = |KC| = 12$ oraz $|WB| = |KD| = 16$. Oblicz dfugość odcinka

$WK.$
\begin{center}
\includegraphics[width=40.584mm,height=58.728mm]{./LigaMatematycznaMatuskiego_Gim_Zestaw1_2016_2017_page0_images/image001.eps}
\end{center}
K

D c

A B

w
\end{document}
