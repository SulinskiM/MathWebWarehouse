\documentclass[a4paper,12pt]{article}
\usepackage{latexsym}
\usepackage{amsmath}
\usepackage{amssymb}
\usepackage{graphicx}
\usepackage{wrapfig}
\pagestyle{plain}
\usepackage{fancybox}
\usepackage{bm}

\begin{document}

LIGA MATEMATYCZNA

im. Zdzisława Matuskiego

$\mathrm{P}\mathrm{A}\dot{\mathrm{Z}}$ DZIERNIK 2017

GIMNAZJUM

ZADANIE I.

Wiadomo, $\dot{\mathrm{z}}\mathrm{e}$

$x-y+2017,y-z+2017,z-t+2017,t-w+2017,w-x+2017$

są kolejnymi liczbami calkowitymi. Znajdz' je.

ZADANIE 2.

Trzy pary malzeńskie: Ania i Adam, Beata i Bartek, Celina i Czarek mają w sumie 1371at.

$K\mathrm{a}\dot{\mathrm{z}}\mathrm{d}\mathrm{y}$ z panów jest o 51at starszy od swojej $\dot{\mathrm{z}}$ ony. Suma liczby lat Bartka i Beaty wynosi

471at. Ania jest najstarsza wśród pań i ma o 4 lata więcej $\mathrm{n}\mathrm{i}\dot{\mathrm{z}}$ najmlodsza z kobiet. Ile lat ma

$\mathrm{k}\mathrm{a}\dot{\mathrm{z}}$ da osoba?

ZADANIE 3.

$\mathrm{W}$ okrąg o promieniu $r$ wpisano kwadrat i na tym okręgu opisano trójkąt równoboczny. Suma

dlugości boku kwadratu i boku trójkąta równobocznego jest równa 10. Wyznacz promień

okręgu.

ZADANIE 4.

Znajd $\acute{\mathrm{z}}$ dwie liczby naturalne, których sumajest równa 432 i których największy wspó1ny dzie1-

nik to 36.

ZADANIE 5.

$\mathrm{W}$ liczbie trzycyfrowej $x$ skreślono cyfrę setek i otrzymano dwucyfrową liczbę $k$. Gdy w liczbie

$x$ skreślono cyfrę dziesiątek, to otrzymano liczbę dwucyfrową $l$, a po skreśleniu w liczbie $x$

cyfry jedności powstala dwucyfrowa liczba $m$. Okazalo się, $\dot{\mathrm{z}}\mathrm{e}$ suma $k+l+m$ jest trzykrotnie

mniejsza od liczby $x.$ Znajd $\acute{\mathrm{z}}x.$


\end{document}