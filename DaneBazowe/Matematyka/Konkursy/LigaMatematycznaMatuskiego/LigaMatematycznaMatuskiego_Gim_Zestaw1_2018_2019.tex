\documentclass[a4paper,12pt]{article}
\usepackage{latexsym}
\usepackage{amsmath}
\usepackage{amssymb}
\usepackage{graphicx}
\usepackage{wrapfig}
\pagestyle{plain}
\usepackage{fancybox}
\usepackage{bm}

\begin{document}

LIGA MATEMATYCZNA

im. Zdzisława Matuskiego

$\mathrm{P}\mathrm{A}\dot{\mathrm{Z}}$ DZIERNIK 2018

GIMNAZJUM

(klasa VII i VIII szkoły podstawowej, klasa III gimnazjum)

ZADANIE I.

Wewnatrz kwadratu ABCD $\mathrm{l}\mathrm{e}\dot{\mathrm{z}}\mathrm{y}$ punkt $P$. Odleglość tego punktu od wierzchofków $A, B, C$

jest równa odpowiednio 2, 7, 9. Ob1icz od1eg1ość punktu $P$ od wierzchofka $D.$

ZADANIE 2.

Na ile sposobów $\mathrm{m}\mathrm{o}\dot{\mathrm{z}}$ na przedstawić liczbę 3 jako sumę sześcianów pięciu 1iczb ca1kowitych?

Przedstawień rózniących się jedynie kolejnością skfadników nie uznajemy za rózne.

ZADANIE 3.

$\mathrm{W}$ klasie VII, liczącej 24 uczniów, średnia 1iczba punktów uzyskanych na k1asówce wynios1a

75 na l00 $\mathrm{m}\mathrm{o}\dot{\mathrm{z}}$ liwych do zdobycia. Maksymalną liczbę 100 punktów otrzymafo 4 uczniów.

Wyznacz średnią liczbę punktów zdobytych przez pozostafych 20 uczniów.

ZADANIE 4.

Znajd $\acute{\mathrm{z}}$ wszystkie takie liczby trzycyfrowe, $\dot{\mathrm{z}}\mathrm{e}$ po skreśleniu cyfry setek otrzymamy liczbę dwu-

krotnie mniejszą $\mathrm{n}\mathrm{i}\dot{\mathrm{z}}$ po skreśleniu cyfry jedności.

ZADANIE 5.

Dany jest kwadrat ABCD. Oblicz miarę kąta $\alpha$ (patrz rysunek).
\begin{center}
\includegraphics[width=59.280mm,height=43.080mm]{./LigaMatematycznaMatuskiego_Gim_Zestaw1_2018_2019_page0_images/image001.eps}
\end{center}
{\it D  C}

{\it a}

$\alpha$


\end{document}