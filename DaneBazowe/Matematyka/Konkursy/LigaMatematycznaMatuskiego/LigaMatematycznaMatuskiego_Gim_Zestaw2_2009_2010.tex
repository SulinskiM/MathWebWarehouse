\documentclass[a4paper,12pt]{article}
\usepackage{latexsym}
\usepackage{amsmath}
\usepackage{amssymb}
\usepackage{graphicx}
\usepackage{wrapfig}
\pagestyle{plain}
\usepackage{fancybox}
\usepackage{bm}

\begin{document}

LIGA MATEMATYCZNA

LISTOPAD 2009

GIMNAZJUM

ZADANIE I.

Wykaz, $\dot{\mathrm{z}}\mathrm{e}$ liczba

2006. 2008. 2010. 2012$+$16

jest kwadratem liczby naturalnej.

ZADANIE 2.

Oblicz pole wielokąta przedstawionego na rysunku wiedząc, $\dot{\mathrm{z}}\mathrm{e}0<x<1.$
\begin{center}
\includegraphics[width=34.140mm,height=26.412mm]{./LigaMatematycznaMatuskiego_Gim_Zestaw2_2009_2010_page0_images/image001.eps}
\end{center}
1  1

x  $\rceil$-x

1

ZADANIE 3.

$\acute{\mathrm{S}}$ rodki kolejnych boków trapezu nierównoramiennego połqczono odcinkami. Wykaz$\cdot, \dot{\mathrm{z}}\mathrm{e}$ suma

pól powstałych czterech trójkątów jest równa polu otrzymanego czworokąta.

ZADANIE 4.

Rozwiąz równanie

$1-(2-(3-\ldots-(2009-x)\ldots))=1000.$

ZADANIE 5.

$\mathrm{W}$ pięciu skarbonkach była jednakowa ilość monet. Po pewnym czasie okazało się, $\dot{\mathrm{z}}\mathrm{e}$ wyjęto

ze skarbonek połowę wszystkich posiadanych monet. Z pierwszej skarbonki wyjęto 2 monety,

z drugiej $-5$ monet, z trzeciej $-9$, z czwartej $-24$. Nie wiadomo, ile monet wyjęto z piątej

skarbonki, ale w $\mathrm{k}\mathrm{a}\dot{\mathrm{z}}$ dej skarbonce zostala co najmniej jedna moneta. Ile było monet na początku

i ile monet pozostało w $\mathrm{k}\mathrm{a}\dot{\mathrm{z}}$ dej skarbonce?


\end{document}