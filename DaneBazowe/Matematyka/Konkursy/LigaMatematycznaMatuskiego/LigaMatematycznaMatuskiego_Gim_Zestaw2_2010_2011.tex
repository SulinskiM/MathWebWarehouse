\documentclass[a4paper,12pt]{article}
\usepackage{latexsym}
\usepackage{amsmath}
\usepackage{amssymb}
\usepackage{graphicx}
\usepackage{wrapfig}
\pagestyle{plain}
\usepackage{fancybox}
\usepackage{bm}

\begin{document}

LIGA MATEMATYCZNA

LISTOPAD 2010

GIMNAZJUM

ZADANIE I.

$\mathrm{W}$ pewnej liczbie trzycyfrowej $x$ skreślono cyfrę setek i otrzymano dwucyfrową liczbę $k$. Gdy

w liczbie $x$ skrešlono cyfrę dziesiątek, otrzymano liczbę dwucyfrową $l$, a po skreśleniu w liczbie

$x$ cyfry jednošci powstała liczba dwucyfrowa $m$. Okazalo się, $\dot{\mathrm{z}}\mathrm{e}$ suma $k+l+m$ jest trzykrotnie

mniejsza od liczby $x$. Znajdz' $x.$

ZADANIE 2.

Wykaz$\cdot, \dot{\mathrm{z}}\mathrm{e}$ suma $2^{1}+2^{2}+2^{3}+\ldots+2^{2009}$ jest podzielna przez 127.

ZADANIE 3.

$\mathrm{W}$ kwadracie ABCD punkty $E\mathrm{i}F$ są środkami, odpowiednio, boków AD $\mathrm{i}BC$. Obrano punkty

$G\mathrm{i}H$ w taki sposób, $\dot{\mathrm{z}}\mathrm{e}E$ jest punktem odcinka $GB\mathrm{i}F$ jest punktem odcinka $AH$. Wiedząc,

$\dot{\mathrm{z}}\mathrm{e}|GA|=|AB|=|BH|=1$, oblicz długość odcinka $GH.$

ZADANIE 4.

Rozwazmy liczby całkowite dodatnie $m \mathrm{i} n$, które spełniają warunek $75m = n^{3}$ Jaka jest

najmniejsza $\mathrm{m}\mathrm{o}\dot{\mathrm{z}}$ liwa suma liczb $m\mathrm{i}n$?

ZADANIE 5.

Dwóch uczonych napisało na siedmiu kartkach liczby 5, 6, 7, 8, 9, 10, $11$ - na $\mathrm{k}\mathrm{a}\dot{\mathrm{z}}$ dej kartce

jedną liczbę.

Następnie pierwszy wziąl losowo trzy kartki, drugi dwie inne kartki, a ostatnie

dwie, bez oglądania ich, wyrzucili. Pierwszy uczony, zaglądając do swoich kartek, powiedział

do drugiego:,,Wiem, $\dot{\mathrm{z}}\mathrm{e}$ suma liczb na twoich kartkach jest parzysta'' Jakie liczby wylosował

pierwszy z uczonych?


\end{document}