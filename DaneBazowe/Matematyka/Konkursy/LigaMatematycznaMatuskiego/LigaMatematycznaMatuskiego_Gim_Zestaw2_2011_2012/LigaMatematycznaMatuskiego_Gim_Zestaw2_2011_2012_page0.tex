\documentclass[a4paper,12pt]{article}
\usepackage{latexsym}
\usepackage{amsmath}
\usepackage{amssymb}
\usepackage{graphicx}
\usepackage{wrapfig}
\pagestyle{plain}
\usepackage{fancybox}
\usepackage{bm}

\begin{document}

LIGA MATEMATYCZNA

LISTOPAD 2011

GIMNAZJUM

ZADANIE I.

Rozwiąz równanie

$(\displaystyle \frac{x}{\sqrt{144}+\sqrt{145}}-\frac{1}{\sqrt{36}+\sqrt{37}})+(\frac{x}{\sqrt{145}+\sqrt{146}}-\frac{1}{\sqrt{37}+\sqrt{38}})+\ldots$

. . . $+(\displaystyle \frac{x}{\sqrt{168}+\sqrt{169}}-\frac{1}{\sqrt{60}+\sqrt{61}}) =0.$

ZADANIE 2.

Przecinając prostokątny arkusz papieru, otrzymano kwadrat oraz mniejszy prostokąt. $\mathrm{Z}$ tego

prostokąta równiez odcięto kwadrat i znów otrzymano mniejszy prostokat. Sytuacja powtórzyła

się jeszcze kilkakrotnie, $\mathrm{a}\dot{\mathrm{z}}$ do momentu otrzymania dziewięciu róznych kwadratów i jednego

prostokata o wymiarach lcm $\times 2\mathrm{c}\mathrm{m}$. Jakie pole mial arkusz papieru?

ZADANIE 3.

Liczby nieparzyste od l do 49 wypisano w ponizszej tab1icy:

(432111111

3 5

13

23

33

43

15

25

35

45

7

17

27

37

47

432199999)

Wybieramy z tej tablicy pięć liczb tak, aby $\dot{\mathrm{z}}$ adne dwie nie $\mathrm{l}\mathrm{e}\dot{\mathrm{z}}$ afy ani w jednej kolumnie, ani

wjednym wierszu. Wyznacz wszystkie wartości, jakie $\mathrm{m}\mathrm{o}\dot{\mathrm{z}}\mathrm{e}$ przyjmować suma wybranych liczb.

ZADANIE 4.

Ania napisafa trzy liczby pięciocyfrowe $\mathrm{u}\dot{\mathrm{z}}$ ywając do zapisu $\mathrm{k}\mathrm{a}\dot{\mathrm{z}}$ dej z tych liczb wszystkich cyfr

spośród 1, 2, 3, 4, 5. Czy suma tych 1iczb jest podzie1na przez 3? Czy jest podzie1na przez 9?

ZADANIE 5.

Dany jest kwadrat o boku dlugości $\alpha$. Na jego bokach, na zewnątrz, zbudowano trójkąty

równoboczne. Wierzcholki kolejnych trójkątów, nie będące wierzcholkami danego kwadratu,

polączono odcinkami. Oblicz pole otrzymanego czworokata.
\end{document}
