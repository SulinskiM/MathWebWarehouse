\documentclass[a4paper,12pt]{article}
\usepackage{latexsym}
\usepackage{amsmath}
\usepackage{amssymb}
\usepackage{graphicx}
\usepackage{wrapfig}
\pagestyle{plain}
\usepackage{fancybox}
\usepackage{bm}

\begin{document}

LIGA MATEMATYCZNA

im. Zdzisława Matuskiego

LISTOPAD 2012

GIMNAZJUM

ZADANIE I.

$\mathrm{O}$ liczbach $a, b$ wiemy, $\dot{\mathrm{z}}\mathrm{e}a<b<0$. Która z liczb $\displaystyle \frac{1}{2}a-b, \displaystyle \frac{1}{2}b-a$ jest większa?

ZADANIE 2.

$\mathrm{W}$ prostokącie ABCD punkt $E$ jest šrodkiem boku $BC, F$ jest šrodkiem boku $CD.$

trójkąta $AEF$ jest równe 15 $\mathrm{c}\mathrm{m}^{2}$ Oblicz pole prostokąta ABCD.

Pole

ZADANIE 3.

Wykaz, $\dot{\mathrm{z}}\mathrm{e}$ liczba $n^{3}+5n$ jest podzielna przez 6 d1a $\mathrm{k}\mathrm{a}\dot{\mathrm{z}}$ dej liczby naturalnej $n.$

ZADANIE 4.

$\mathrm{W}$ kwadrat ABCD wpisano koło. $\mathrm{W}$ to koło wpisano kwadrat tak, $\dot{\mathrm{z}}\mathrm{e}$ jego boki sq równoległe

do boków kwadratu ABCD. Róznica pól tych kwadratów jest równa 2 $\mathrm{c}\mathrm{m}^{2}$ Oblicz pole kofa.

ZADANIE 5.

Piotr znalazf wszystkie dzielniki pewnej liczby naturalnej n, uporządkował je rosnąco, a na-

stępnie wykreślił co drugi otrzymujqc liczby: 1, 3, 6, 12, 21, 42. Wyznacz 1iczbę n i pozostałe

jej dzielniki.
\end{document}
