\documentclass[a4paper,12pt]{article}
\usepackage{latexsym}
\usepackage{amsmath}
\usepackage{amssymb}
\usepackage{graphicx}
\usepackage{wrapfig}
\pagestyle{plain}
\usepackage{fancybox}
\usepackage{bm}

\begin{document}

LIGA MATEMATYCZNA

im. Zdzisława Matuskiego

LISTOPAD 2013

GIMNAZJUM

ZADANIE I.

Pewna liczba ma cztery dzielniki, których średnia arytmetycznajest równa 10. Znajd $\acute{\mathrm{z}}$ tę liczbę.

ZADANIE 2.

W szkole uczy się 600 uczniów w 21 k1asach.

co najmniej 29 osób.

Uzasadnij, $\dot{\mathrm{z}}\mathrm{e}$ istnieje klasa, w której uczy się

ZADANIE 3.

Jaką część pola trójkąta ABC stanowi pole czworokąta ABFD, jeśli odcinki AB oraz AD mają

dlugość 2, natomiast odcinki BE i CD mają dfugość 5.
\begin{center}
\includegraphics[width=70.152mm,height=36.012mm]{./LigaMatematycznaMatuskiego_Gim_Zestaw2_2013_2014_page0_images/image001.eps}
\end{center}
c

D F

A  B  E

ZADANIE 4.

Wykaz$\cdot, \dot{\mathrm{z}}\mathrm{e}$ jeśli $a>1$ oraz $b<1$, to $ab+1<a+b.$

ZADANIE 5.

Liczba sześciocyfrowa w zapisie dziesiątkowym kończy się cyfrą 4. $\mathrm{J}\mathrm{e}\dot{\mathrm{z}}$ eli cyfrę 4 przeniesiemy

na początek zapisu, pozostawiając pozostale cyfry bez zmian, to otrzymamy nową liczbę, która

będzie cztery razy większa od początkowej. Znajd $\acute{\mathrm{z}}$ początkową liczbę.
\end{document}
