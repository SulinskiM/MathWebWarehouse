\documentclass[a4paper,12pt]{article}
\usepackage{latexsym}
\usepackage{amsmath}
\usepackage{amssymb}
\usepackage{graphicx}
\usepackage{wrapfig}
\pagestyle{plain}
\usepackage{fancybox}
\usepackage{bm}

\begin{document}

LIGA MATEMATYCZNA

im. Zdzisława Matuskiego

LISTOPAD 2014

GIMNAZJUM

ZADANIE I.

Zeszyt Bartka do matematyki ma ponumerowane strony od l do 60. Chfopiec wyrwa1 z niego

dziesięć kartek i dodal liczby numerujące ich strony. Sprawd $\acute{\mathrm{z}}$, czy mógl otrzymać liczbę 101.

ZADANIE 2.

Znajd $\acute{\mathrm{z}}$ taką liczbę trzycyfrowa, $\dot{\mathrm{z}}$ ejeśli z prawej strony dopiszemy cyfrę 8, to otrzymamy 1iczbę

czterocyfrową dwa razy większą $\mathrm{n}\mathrm{i}\dot{\mathrm{z}}$ gdybyśmy z lewej strony dopisali cyfrę 3 i uzyska1i inna

liczbę czterocyfrowa.

ZADANIE 3.

Rozwiąz uklad równań

$\left\{\begin{array}{l}
5(y+z)-x=-1\\
4(x+z)-2y=2\\
3(x+y)-3z=-1.
\end{array}\right.$

ZADANIE 4.

Wykaz, $\dot{\mathrm{z}}\mathrm{e}\mathrm{j}\mathrm{e}\dot{\mathrm{z}}$ eli $a\mathrm{i}b$ sq dowolnymi dodatnimi liczbami rzeczywistymi, to

$(a+b)(\displaystyle \frac{1}{a}+\frac{1}{b})\geq 4.$

ZADANIE 5.

Prosta k dzieli boki prostokąta na odcinki, których dlugości pozostają w stosunku 1 : 4 oraz

l : l tak, jak na ponizszym rysunku. Oblicz stosunek pól powstałych w ten sposób figur.
\begin{center}
\includegraphics[width=64.104mm,height=38.808mm]{./LigaMatematycznaMatuskiego_Gim_Zestaw2_2014_2015_page0_images/image001.eps}
\end{center}
k
\end{document}
