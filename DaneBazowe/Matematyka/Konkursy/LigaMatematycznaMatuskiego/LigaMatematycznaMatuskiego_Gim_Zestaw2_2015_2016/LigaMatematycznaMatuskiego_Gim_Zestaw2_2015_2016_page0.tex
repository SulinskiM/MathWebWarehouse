\documentclass[a4paper,12pt]{article}
\usepackage{latexsym}
\usepackage{amsmath}
\usepackage{amssymb}
\usepackage{graphicx}
\usepackage{wrapfig}
\pagestyle{plain}
\usepackage{fancybox}
\usepackage{bm}

\begin{document}

LIGA MATEMATYCZNA

im. Zdzisława Matuskiego

LISTOPAD 2015

GIMNAZJUM

ZADANIE I.

$\mathrm{W}$ prostokącie ABCD punkty $Y, L, K, X$ są środkami boków odpowiednio AB, $BC$, {\it CD}, $DA,$

zaś punkt $M$ jest środkiem odcinka $XY$. Pole prostokąta ABCD jest równe 2015 $\mathrm{c}\mathrm{m}^{2}$ Oblicz

pole trójkąta $KLM.$

ZADANIE 2.

Dwie trzycyfrowe liczby zapisane są przy pomocy takich samych cyfr, z których jedna jest

równa 4. Pierwsza 1iczba ma czwórkę w rzędzie jedności, a druga w rzędzie setek, zaš pozostałe

jej cyfry zapisane sq w takiej samej kolejności, jak w pierwszej. Druga liczba jest o 400 większa

od róznicy liczby 400 i pierwszej 1iczby. Jakie to 1iczby?

ZADANIE 3.

Wyznacz setną cyfrę od końca liczby 2015!. Liczbę $n!$ (czytamy $n$ silnia) definiujemy jako

iloczyn kolejnych liczb naturalnych od l do $n.$

ZADANIE 4.

Wykaz$\cdot, \dot{\mathrm{z}}\mathrm{e}7^{n+2}+7^{n+1}-2\cdot 7^{n}$ jest liczbq parzystą dla dowolnej liczby naturalnej $n.$

ZADANIE 5.

$\mathrm{W}$ zbiorze liczb rzeczywistych rozwiąz uklad równań

$\left\{\begin{array}{l}
x+y+z+t=36\\
x+y-z-t=24\\
x-y+z-t=12\\
x-y-z+t=0.
\end{array}\right.$
\end{document}
