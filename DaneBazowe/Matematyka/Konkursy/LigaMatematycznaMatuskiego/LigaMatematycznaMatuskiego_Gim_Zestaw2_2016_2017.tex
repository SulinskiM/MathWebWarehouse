\documentclass[a4paper,12pt]{article}
\usepackage{latexsym}
\usepackage{amsmath}
\usepackage{amssymb}
\usepackage{graphicx}
\usepackage{wrapfig}
\pagestyle{plain}
\usepackage{fancybox}
\usepackage{bm}

\begin{document}

LIGA MATEMATYCZNA

im. Zdzisława Matuskiego

LISTOPAD 2016

GIMNAZJUM

ZADANIE I.

Na przyprostokatnych $BC\mathrm{i}CA$ trójkąta prostokatnego $ABC$ zbudowano na zewnątrz kwadraty

{\it ECBD oraz CFGA. Prosta} $AD$ przecina bok $BC$ w punkcie $P$, prosta $BG$ przecina bok $CA$

w punkcie $R$. Udowodnij, $\dot{\mathrm{z}}\mathrm{e}$ odcinki $CP\mathrm{i}CR$ mają równe dlugości.

ZADANIE 2.

Adam mial pomnozyć dwie liczby naturalne. Jeden z czynników byl liczbą dwucyfrową, w której

cyfra jedności byla dwukrotnie mniejsza od cyfry dziesiątek. Chlopiec pomylił się, przestawił

cyfry tej liczby i otrzymal iloczyn o 1539 mniejszy od poprawnego. Podaj poprawny wynik

tego mnozenia i liczby, które miał pomnozyć Adam.

ZADANIE 3.

Na stole $\mathrm{l}\mathrm{e}\dot{\mathrm{z}}\mathrm{y}$ 2017 monet. $\mathrm{W}$ jednym ruchu Bartek $\mathrm{m}\mathrm{o}\dot{\mathrm{z}}\mathrm{e}$ wziąć dokladnie 3, 361ub 69 monet.

Czy wykonując wiele takich ruchów Bartek $\mathrm{m}\mathrm{o}\dot{\mathrm{z}}\mathrm{e}$ wziąć wszystkie monety ze stołu?

ZADANIE 4.

Wykaz$\cdot, \dot{\mathrm{z}}\mathrm{e}$ liczba $256^{4}+8^{9}$ jest podzielna przez ll.

ZADANIE 5.

Róznica między czwartymi potęgami pewnych dwóch liczb naturalnych jest równa 34481, a róz-

nica między drugimi potęgami tych liczb wynosi 41. Wyznacz róznicę tych 1iczb.


\end{document}