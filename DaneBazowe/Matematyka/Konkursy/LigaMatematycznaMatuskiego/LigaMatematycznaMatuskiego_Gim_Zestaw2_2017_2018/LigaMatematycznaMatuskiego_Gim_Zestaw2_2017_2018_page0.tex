\documentclass[a4paper,12pt]{article}
\usepackage{latexsym}
\usepackage{amsmath}
\usepackage{amssymb}
\usepackage{graphicx}
\usepackage{wrapfig}
\pagestyle{plain}
\usepackage{fancybox}
\usepackage{bm}

\begin{document}

LIGA MATEMATYCZNA

im. Zdzisława Matuskiego

LISTOPAD 2017

GIMNAZJUM

ZADANIE I.

$\mathrm{O}$ liczbach $a, b, c, d$ wiadomo, $\dot{\mathrm{z}}\mathrm{e}$

$\left\{\begin{array}{l}
a=bcd\\
a+b=cd\\
a+b+c=d\\
a+b+c+d=1.
\end{array}\right.$

Wyznacz te liczby.

ZADANIE 2.

Niech $p$ będzie liczbą pierwszą taką, $\dot{\mathrm{z}}\mathrm{e}$ liczba dzielników liczby $p^{6}$ jest dzielnikiem tej liczby.

Ile dzielników ma liczba $(p+1)^{6}$?

ZADANIE 3.

Dziadek Ani urodzil się przed II wojną światową, ale ma mniej $\mathrm{n}\mathrm{i}\dot{\mathrm{z}}90$ lat. Gdy w 2007 roku

obchodzil urodziny, Ania zauwazyla, $\dot{\mathrm{z}}\mathrm{e}$ numer roku byl równy numerowi roku urodzenia dziadka

powiększonemu o pięciokrotną sumę cyfr roku urodzenia. $\mathrm{W}$ którym roku urodzil się dziadek

Ani?

ZADANIE 4.

Dwa jednakowe kofa mniejsze i jedno kolo większe wpisano w prostokąt w taki sposób, $\dot{\mathrm{z}}\mathrm{e}$ kofa

są styczne do boków prostokąta i wzajemnie styczne zewnętrznie. Mniejszy z boków prostokąta

ma dlugość 4. Ob1icz obwód prostokąta oraz róznicę między po1em prostokąta a sumą pó1 kó1.

ZADANIE 5.

Wykaz, $\dot{\mathrm{z}}\mathrm{e}$ liczba $4^{202}+2\cdot 4^{101}\cdot 6^{101}+6^{202}$ jest podzielna przez 100.
\end{document}
