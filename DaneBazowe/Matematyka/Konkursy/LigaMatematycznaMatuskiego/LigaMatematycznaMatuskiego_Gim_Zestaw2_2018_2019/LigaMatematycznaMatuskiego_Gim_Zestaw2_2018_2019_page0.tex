\documentclass[a4paper,12pt]{article}
\usepackage{latexsym}
\usepackage{amsmath}
\usepackage{amssymb}
\usepackage{graphicx}
\usepackage{wrapfig}
\pagestyle{plain}
\usepackage{fancybox}
\usepackage{bm}

\begin{document}

LIGA MATEMATYCZNA

im. Zdzisława Matuskiego

LISTOPAD 2018

GIMNAZJUM

(klasa VII i VIII szkoły podstawowej, klasa III gimnazjum)

ZADANIE I.

Czy istnieje trójkąt prostokątny mający boki o dfugościach cafkowitych i obwód równy 2019?

ZADANIE 2.

Wykaz, $\dot{\mathrm{z}}\mathrm{e}$ liczba

$\displaystyle \frac{n^{5}}{120}-\frac{n^{3}}{24}+\frac{n}{30}$

jest całkowita dla $\mathrm{k}\mathrm{a}\dot{\mathrm{z}}$ dej liczby całkowitej $n.$

ZADANIE 3.

Wyznacz wszystkie pary liczb cafkowitych dodatnich $(x,y)$ o następujących wlasnościach:

$x+y=240$ oraz

$\mathrm{N}\mathrm{W}\mathrm{D}\{x,y\}=30.$

ZADANIE 4.

Ile jest liczb trzycyfrowych podzielnych przez 9 mających następującą wfasność:

ilorazu tej liczby przez 9 jest o 9 mniejsza od sumy jej cyfr?

suma cyfr

ZADANIE 5.

Sześciokąt mający wszystkie kąty wewnętrzne tej samej miary wpisano w trójkąt równoboczny

o boku o dlugości 7. Wyznacz d1ugości boków x, y, z.
\begin{center}
\includegraphics[width=49.332mm,height=45.360mm]{./LigaMatematycznaMatuskiego_Gim_Zestaw2_2018_2019_page0_images/image001.eps}
\end{center}
3

{\it 4}

5
\end{document}
