\documentclass[a4paper,12pt]{article}
\usepackage{latexsym}
\usepackage{amsmath}
\usepackage{amssymb}
\usepackage{graphicx}
\usepackage{wrapfig}
\pagestyle{plain}
\usepackage{fancybox}
\usepackage{bm}

\begin{document}

LIGA MATEMATYCZNA

GRUD Z$\mathrm{I}\mathrm{E}\acute{\mathrm{N}}$ 2009

GIMNAZJUM

ZADANIE I.

Trójkqt $ABC$ podzielono na 5 trójkątów. Liczba wewnątrz $\mathrm{k}\mathrm{a}\dot{\mathrm{z}}$ dego trójkąta oznacza jego pole.

Oblicz $x.$
\begin{center}
\includegraphics[width=70.812mm,height=33.528mm]{./LigaMatematycznaMatuskiego_Gim_Zestaw3_2008_2009_page0_images/image001.eps}
\end{center}
c

14

7

21  x

A  B

ZADANIE 2.

Oblicz $\sqrt{37-20\sqrt{3}}+\sqrt{13-4\sqrt{3}}.$

ZADANIE 3.

Rozwiąz układ równań

$\left\{\begin{array}{l}
5x(x+y+z)=4\\
2y(x+y+z)=6\\
4z(x+y+z)=4.
\end{array}\right.$

ZADANIE 4.

$\mathrm{W}$ grupie 300 studentów $\mathrm{k}\mathrm{a}\dot{\mathrm{z}}\mathrm{d}\mathrm{y}$ jest matematykiem, chemikiem lub fizykiem. Polowa fizyków

zajmuje się chemią, połowa chemików zajmuje się matematyką, a polowa matematyków to fi-

zycy. Wiedząc, $\dot{\mathrm{z}}\mathrm{e}\dot{\mathrm{z}}$ aden fizyk nie zajmuje się chemią i matematyką, odpowiedz, z ilu osób

składają się te grupy.

ZADANIE 5.

Liczbę czterocyfrową pomnozono przez 9 i otrzymano 1iczbę czterocyfrową zapisaną za pomocą

tych samych cyfr w odwrotnej kolejności. Jaka to liczba?
\end{document}
