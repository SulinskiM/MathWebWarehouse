\documentclass[a4paper,12pt]{article}
\usepackage{latexsym}
\usepackage{amsmath}
\usepackage{amssymb}
\usepackage{graphicx}
\usepackage{wrapfig}
\pagestyle{plain}
\usepackage{fancybox}
\usepackage{bm}

\begin{document}

LIGA MATEMATYCZNA

GRUD Z$\mathrm{I}\mathrm{E}\acute{\mathrm{N}}$ 2010

GIMNAZJUM

ZADANIE I.

Wiadomo, $\dot{\mathrm{z}}\mathrm{e}a-b+2010, b-c+2010, c-a+2010$ są trzema kolejnymi liczbami calkowitymi.

Jakimi?

ZADANIE 2.

Znajd $\acute{\mathrm{z}}$ wszystkie rozwiązania równania $x^{4}-y^{4}=65$ będqce liczbami naturalnymi.

ZADANIE 3.

Z przeciwległych wierzcholków prostokąta poprowadzono odcinki prostopadłe do przekątnej.

Odcinki te podzieliły przekątną na trzy równe częšci. Znajdz' stosunek długošci boków tego

prostokąta.

ZADANIE 4.

W ciągu tygodnia waga małej foki wzrosla o 4\%, a słoniątka o 4 kg. Skutkiem tego średnia

waga obu zwierząt wzrosfa o 3 kg, czy1i o 2\%. I1e obecnie wazy sfoniątko?

ZADANIE 5.

Znajd $\acute{\mathrm{z}}$ ostatnią cyfrę liczby $1^{2010}+2^{2010}+3^{2010}+\ldots+10^{2010}$


\end{document}