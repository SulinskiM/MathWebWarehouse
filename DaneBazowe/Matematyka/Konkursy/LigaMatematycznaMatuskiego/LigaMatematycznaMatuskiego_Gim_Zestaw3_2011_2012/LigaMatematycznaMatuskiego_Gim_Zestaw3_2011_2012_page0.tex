\documentclass[a4paper,12pt]{article}
\usepackage{latexsym}
\usepackage{amsmath}
\usepackage{amssymb}
\usepackage{graphicx}
\usepackage{wrapfig}
\pagestyle{plain}
\usepackage{fancybox}
\usepackage{bm}

\begin{document}

LIGA MATEMATYCZNA

GRUD Z$\mathrm{I}\mathrm{E}\acute{\mathrm{N}}$ 2011

GIMNAZJUM

ZADANIE I.

Na Wigilii $\mathrm{u}$ babci spotkala się liczna rodzina. Przy stole zasiadlo 20 osób. Babcia przygoto-

wala sto ciasteczek. $K\mathrm{a}\dot{\mathrm{z}}\mathrm{d}\mathrm{y}$ męzczyzna zjadł siedem ciasteczek, $\mathrm{k}\mathrm{a}\dot{\mathrm{z}}$ da kobieta- pięć, a $\mathrm{k}\mathrm{a}\dot{\mathrm{z}}$ de

z wnucząt-jedno. Oblicz, ilu bylo doroslych, a ile dzieci.

ZADANIE 2.

Wykaz, $\dot{\mathrm{z}}\mathrm{e}\sqrt{18+8\sqrt{2}}+\sqrt{6-4\sqrt{2}}$ jest liczbą calkowita.

ZADANIE 3.

Wykaz, $\dot{\mathrm{z}}\mathrm{e}$ jeśli do iloczynu dwóch kolejnych liczb naturalnych dodamy sumę kwadratów tych

liczb powiększoną o 5, to otrzymamy 1iczbę podzie1na przez 6.

ZADANIE 4.

Piszemy liczby 1, 1, 2, 3, 5, 8, $\ldots$ w taki sposób, $\dot{\mathrm{z}}\mathrm{e}$ począwszy od trzeciej, $\mathrm{k}\mathrm{a}\dot{\mathrm{z}}$ da następna liczba

jest sumą dwóch poprzednich. Jaką liczbą (parzystą czy nieparzystą) jest liczba znajdująca się

na 2011 miejscu?

ZADANIE 5.

$\mathrm{Z}$ dwóch jednakowych plytek w ksztalcie trójkata prostokątnego o obwodzie 40 $\mathrm{m}\mathrm{o}\dot{\mathrm{z}}$ na ufozyć

trójkąt o obwodzie 50 a1bo trójkąt o obwodzie 64, a1bo de1toid. Ob1icz d1ugość przekątnych

tego deltoidu.
\end{document}
