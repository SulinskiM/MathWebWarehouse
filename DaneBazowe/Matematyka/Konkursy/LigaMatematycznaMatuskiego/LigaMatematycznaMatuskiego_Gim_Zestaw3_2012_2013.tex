\documentclass[a4paper,12pt]{article}
\usepackage{latexsym}
\usepackage{amsmath}
\usepackage{amssymb}
\usepackage{graphicx}
\usepackage{wrapfig}
\pagestyle{plain}
\usepackage{fancybox}
\usepackage{bm}

\begin{document}

LIGA MATEMATYCZNA

im. Zdzisława Matuskiego

GRUD Z$\mathrm{I}\mathrm{E}\acute{\mathrm{N}}$ 2012

GIMNAZJUM

ZADANIE I.

Liczba 390 jest sumą kwadratów trzech róznych 1iczb pierwszych. Znajd $\acute{\mathrm{z}}$ te liczby.

ZADANIE 2.

$\mathrm{W}$ trójkącie prostokątnym na dłuzszej przyprostokątnej jako na średnicy opisano okrąg. Wy-

znacz długość okręgu, $\mathrm{j}\mathrm{e}\dot{\mathrm{z}}$ eli krótsza przyprostokątna jest równa 30, a cięciwa 1ącząca wierzcho-

lek kąta prostego z punktem przecięcia przeciwprostokątnej z okręgiem (róznym od wierzchol-

ków trójkąta) jest równa 24.

ZADANIE 3.

Pewna liczba dwucyfrowa ma trzy dzielniki jednocyfrowe i trzy dzielniki dwucyfrowe. Suma

wszystkich dzielników jednocyfrowych jest równa 8. Ob1icz sumę wszystkich dzie1ników dwu-

cyfrowych tej liczby.

ZADANIE 4.

Pole prostokąta ABCD jest równe 24 $\mathrm{c}\mathrm{m}^{2}$ Na boku $AB$ zaznaczono punkt $E$ rózny od punktów

{\it A} $\mathrm{i}B$, na odcinku $DC$ zaznaczono punkt $F$ rózny od punktów $C\mathrm{i}D$. Pole trójkąta $ADF$ jest

równe 5 $\mathrm{c}\mathrm{m}^{2}$ Oblicz pole trójkąta $CFE.$

ZADANIE 5.

Wykaz$\cdot, \dot{\mathrm{z}}\mathrm{e}$ liczba $n^{3}+11n$ jest podzielna przez 6 d1a $\mathrm{k}\mathrm{a}\dot{\mathrm{z}}$ dej liczby naturalnej $n.$


\end{document}