\documentclass[a4paper,12pt]{article}
\usepackage{latexsym}
\usepackage{amsmath}
\usepackage{amssymb}
\usepackage{graphicx}
\usepackage{wrapfig}
\pagestyle{plain}
\usepackage{fancybox}
\usepackage{bm}

\begin{document}

LIGA MATEMATYCZNA

im. Zdzisława Matuskiego

GRUD Z$\mathrm{I}\mathrm{E}\acute{\mathrm{N}}$ 2013

GIMNAZJUM

ZADANIE I.

Wyznacz najmniejszą $\mathrm{m}\mathrm{o}\dot{\mathrm{z}}$ liwą wartość wyrazenia

$x_{1}x_{2}+x_{2}x_{3}+x_{3}x_{4}+\ldots+x_{100}x_{101}+x_{101}x_{1},$

gdy $\mathrm{k}\mathrm{a}\dot{\mathrm{z}}$ da z liczb $x_{1}, x_{2}, x_{3}, \ldots, x_{101}$ jest równa llub $-1.$

ZADANIE 2.

Podstawa trójkąta równoramiennego ABC ma długość 2 cm, a ramię - 4 cm.

trójkąta, którego wierzcholkami są spodki wysokości trójkąta ABC.

Oblicz obwód

ZADANIE 3.

Wykaz, $\dot{\mathrm{z}}\mathrm{e}\mathrm{j}\mathrm{e}\dot{\mathrm{z}}$ eli liczby $a$ oraz $b$ są dodatnie, to

-{\it a}1 $+$ -{\it b}1 $\geq$ -{\it a} $+$4 {\it b}.

ZADANIE 4.

Wyznacz dwie kolejne liczby naturalne, z których większa dzieli się przez 2009, a mniejsza

przez 45.

ZADANIE 5.

$\mathrm{W}$ kasynie stoją automaty do gry. Pracownik opróznif je i przyniósf do biura 2013 $\dot{\mathrm{z}}$ etonów.

Oświadczyl, $\dot{\mathrm{z}}\mathrm{e}$ wyjąl $\dot{\mathrm{z}}$ etony ze wszystkich 31 automatów, w $\mathrm{k}\mathrm{a}\dot{\mathrm{z}}$ dym bylo co najmniej 50

$\dot{\mathrm{z}}$ etonów, ale w $\dot{\mathrm{z}}$ adnych dwóch maszynach nie bylo tej samej liczby $\dot{\mathrm{z}}$ etonów. Kierownik kasyna

oskarzyf go o oszustwo. Dlaczego?


\end{document}