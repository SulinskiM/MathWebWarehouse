\documentclass[a4paper,12pt]{article}
\usepackage{latexsym}
\usepackage{amsmath}
\usepackage{amssymb}
\usepackage{graphicx}
\usepackage{wrapfig}
\pagestyle{plain}
\usepackage{fancybox}
\usepackage{bm}

\begin{document}

LIGA MATEMATYCZNA

im. Zdzisława Matuskiego

GRUD Z$\mathrm{I}\mathrm{E}\acute{\mathrm{N}}$ 2014

GIMNAZJUM

ZADANIE I.

Trzy okręgi o jednakowym promieniu r przecinają się w jednym punkcie S i w punktach M,

N, P, przy czym S lezy wewnątrz trójkąta MNP. Oblicz długość promienia okręgu opisanego

na trójkącie MNP.

ZADANIE 2.

Na jednej z pófek biblioteki Bartek umieścil słowniki i encyklopedie. Jedną trzecią tej pólki

zajmują slowniki, a pozostałą część - encyklopedie. $K\mathrm{a}\dot{\mathrm{z}}\mathrm{d}\mathrm{y}$ ze słowników ma grubość 5 cm,

a $\mathrm{k}\mathrm{a}\dot{\mathrm{z}}$ da encyklopedia- 7 cm. Wyznacz najmniejszą $\mathrm{m}\mathrm{o}\dot{\mathrm{z}}$ liwą liczbę woluminów na półce.

ZADANIE 3.

Oblicz sumę cyfr liczby $2^{2010}\cdot 5^{2014}$

ZADANIE 4.

Danych jest 20141iczb natura1nych, o których wiadomo, $\dot{\mathrm{z}}\mathrm{e}$ ich suma jest liczbą nieparzystą.

Jaką liczbq, parzystą czy nieparzystą, jest ich iloczyn?

ZADANIE 5.

Wykaz$\cdot, \dot{\mathrm{z}}\mathrm{e}$ dla dowolnych nieujemnych liczb rzeczywistych $a, b$ spelniona jest nierównošć

$a^{3}+b^{3}\geq a^{2}b+ab^{2}$


\end{document}