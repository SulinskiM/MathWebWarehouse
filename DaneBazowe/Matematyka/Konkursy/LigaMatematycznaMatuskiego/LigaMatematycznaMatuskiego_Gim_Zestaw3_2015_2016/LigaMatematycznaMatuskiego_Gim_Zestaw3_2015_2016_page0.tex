\documentclass[a4paper,12pt]{article}
\usepackage{latexsym}
\usepackage{amsmath}
\usepackage{amssymb}
\usepackage{graphicx}
\usepackage{wrapfig}
\pagestyle{plain}
\usepackage{fancybox}
\usepackage{bm}

\begin{document}

LIGA MATEMATYCZNA

im. Zdzisława Matuskiego

GRUD Z$\mathrm{I}\mathrm{E}\acute{\mathrm{N}}$ 2015

GIMNAZJUM

ZADANIE I.

Punkty $D, E, F, G, H, I$ dzielą $\mathrm{k}\mathrm{a}\dot{\mathrm{z}}\mathrm{d}\mathrm{y}$ bok trójkąta $ABC$ na trzy równe części. Oblicz stosunek

pola czworokąta DEGI do pola trójkąta $ABC.$
\begin{center}
\includegraphics[width=57.252mm,height=31.704mm]{./LigaMatematycznaMatuskiego_Gim_Zestaw3_2015_2016_page0_images/image001.eps}
\end{center}
c

H G

1  F

A  D E  B

ZADANIE 2.

$\mathrm{D}\mathrm{u}\dot{\mathrm{z}}$ a bombka na choinkę kosztuje 5 monet, średnia 3 monety, a za trzy mafe bombki w kszta1cie

aniofka trzeba zapfacić jedna monetę. Za sto monet kupiono sto bombek na choinkę. Ile wśród

nich bylo $\mathrm{d}\mathrm{u}\dot{\mathrm{z}}$ ych, średnich i malych bombek? Rozwaz wszystkie $\mathrm{m}\mathrm{o}\dot{\mathrm{z}}$ liwości.

ZADANIE 3.

$\mathrm{W}$ zbiorze liczb rzeczywistych rozwiąz uklad równań

$\left\{\begin{array}{l}
ab=a+b+1\\
bc=b+c+2\\
ac=a+c+5.
\end{array}\right.$

ZADANIE 4.

Znajdujemy ostateczną sumę cyfr liczby naturalnej - sumujemy jej cyfry i $\mathrm{j}\mathrm{e}\dot{\mathrm{z}}$ eli wynik nie

jest jednocyfrowy, to operację powtarzamy do skutku. Na przykład ostateczną sumq cyfr

liczby 78987 jest 3, gdyz $7+8+9+8+7= 39, 3+9 = 12, 1+2 = 3$ i do jej obliczenia

potrzeba trzykrotnego sumowania cyfr. Podaj najmniejszą liczbę, która wymaga czterokrotnego

sumowania, aby wyznaczyć ostateczną sumę jej cyfr.

ZADANIE 5.

Bartek rzucil sto razy kostką do gry i zsumowal liczby wyrzuconych oczek. Czy jest $\mathrm{m}\mathrm{o}\dot{\mathrm{z}}$ liwe,

aby suma ta byfa równa 211, $\mathrm{j}\mathrm{e}\dot{\mathrm{z}}$ eli ani razu nie wypadla liczba parzysta?
\end{document}
