\documentclass[a4paper,12pt]{article}
\usepackage{latexsym}
\usepackage{amsmath}
\usepackage{amssymb}
\usepackage{graphicx}
\usepackage{wrapfig}
\pagestyle{plain}
\usepackage{fancybox}
\usepackage{bm}

\begin{document}

LIGA MATEMATYCZNA

im. Zdzisława Matuskiego

GRUD Z$\mathrm{I}\mathrm{E}\acute{\mathrm{N}}$ 2016

GIMNAZJUM

ZADANIE I.

Obwód trójkąta prostokątnego jest równy 132, a suma kwadratów długości boków trójkątajest

równa 6050. Wyznacz dfugości boków trójkąta.

ZADANIE 2.

$\mathrm{W}$ liczbie $\overline{aabb}$ suma cyfr $\alpha \mathrm{i}b$ jest równa ll. Wykaz, $\dot{\mathrm{z}}\mathrm{e}$ ta liczba jest podzielna przez 121.

ZADANIE 3.

Trapez podzielono przekątnymi na cztery trójkąty. Pole trapezu jest równe 20, a stosunek

dlugości jego podstaw jest równy 4. Ob1icz po1e $\mathrm{k}\mathrm{a}\dot{\mathrm{z}}$ dego z otrzymanych trójkątów.

ZADANIE 4.

Dzielna i dzielnik są liczbami dwucyfrowymi, a iloraz i reszta są równymi liczbami jednocyfro-

wymi. Dzielnik jest iloczynem ilorazu i reszty. Wyznacz dzielną.

ZADANIE 5.

Klasa licząca 25 uczniów kupi1a na 1oterii 1osy z numerami od 1 do 25. $K\mathrm{a}\dot{\mathrm{z}}\mathrm{d}\mathrm{y}$ uczeń wyloso-

wał jeden los, a następnie dodał numer losu do swojego numeru w dzienniku. Uzasadnij, $\dot{\mathrm{z}}\mathrm{e}$

przynajmniej jeden uczeń otrzymal w wyniku tego dodawania liczbę parzystą.


\end{document}