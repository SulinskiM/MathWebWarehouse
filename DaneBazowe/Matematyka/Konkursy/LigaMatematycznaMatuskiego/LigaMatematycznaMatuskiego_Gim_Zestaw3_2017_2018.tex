\documentclass[a4paper,12pt]{article}
\usepackage{latexsym}
\usepackage{amsmath}
\usepackage{amssymb}
\usepackage{graphicx}
\usepackage{wrapfig}
\pagestyle{plain}
\usepackage{fancybox}
\usepackage{bm}

\begin{document}

LIGA MATEMATYCZNA

im. Zdzisława Matuskiego

GRUD Z$\mathrm{I}\mathrm{E}\acute{\mathrm{N}}$ 2017

GIMNAZJUM

(klasa VII szkoły podstawowej, klasa II i III gimnazjum)

ZADANIE I.

$\mathrm{W}$ trójkąt prostokątny o przyprostokatnych 5 $\mathrm{i}12$ wpisano okrąg. Oblicz najmniejszą z odle-

gfości wierzchołka kąta prostego od punktów tego okręgu.

ZADANIE 2.

$\mathrm{W}$ 2001 roku Adam miaf dwa razy tyle lat, ile wynosi suma cyfr roku jego urodzenia. Ostatniq

cyfrą roku urodzenia Adama jest 7. I1e 1at będzie mia1 Adam w 2018 roku?

ZADANIE 3.

Wykaz, $\dot{\mathrm{z}}\mathrm{e}$ dla $\mathrm{k}\mathrm{a}\dot{\mathrm{z}}$ dej liczby naturalnej $n$ wartość wyrazenia

$\displaystyle \frac{1}{9}(100^{n+1}+4\cdot 10^{n+1}+4)$

jest kwadratem liczby naturalnej.

ZADANIE 4.

Uzasadnij, $\dot{\mathrm{z}}\mathrm{e}\mathrm{j}\mathrm{e}\dot{\mathrm{z}}$ eli do licznika i mianownika wfaściwego dodatniego ulamka dodamy l, to

otrzymamy ulamek większy od wyjściowego.

ZADANIE 5.

Na tablicy napisano pięć liczb, niekoniecznie róznych.

policzyl ich sumę i zapisal wyniki:

Dla $\mathrm{k}\mathrm{a}\dot{\mathrm{z}}$ dej pary tych liczb Mikolaj

1, 2, 3, 5, 5, 6, 7, 8, 9, 10

wymazując początkowe liczby. Wyznacz wszystkie $\mathrm{m}\mathrm{o}\dot{\mathrm{z}}$ liwe wartości iloczynu wymazanych liczb.


\end{document}