\documentclass[a4paper,12pt]{article}
\usepackage{latexsym}
\usepackage{amsmath}
\usepackage{amssymb}
\usepackage{graphicx}
\usepackage{wrapfig}
\pagestyle{plain}
\usepackage{fancybox}
\usepackage{bm}

\begin{document}

LIGA MATEMATYCZNA

im. Zdzisława Matuskiego

GRUD Z$\mathrm{I}\mathrm{E}\acute{\mathrm{N}}$ 2018

GIMNAZJUM

(klasa VII i VIII szkoły podstawowej, klasa III gimnazjum)

ZADANIE I.

Liczba $a$ przy dzieleniu przez 9 daje resztę 2, a mniejsza od niej 1iczba $b$ przy dzieleniu przez 9

daje resztę 7. Znajd $\acute{\mathrm{z}}$ resztę z dzielenia róznicy $a-b$ przez 9.

ZADANIE 2.

Proste $p\mathrm{i}q$ są równolegle, a miary katów przy wierzchofkach $A\mathrm{i}C$ sa równe $135^{\mathrm{o}}$ i $115^{\mathrm{o}}$ tak,

jak na rysunku. Oblicz miarę kąta $ABC.$
\begin{center}
\includegraphics[width=65.892mm,height=37.800mm]{./LigaMatematycznaMatuskiego_Gim_Zestaw3_2018_2019_page0_images/image001.eps}
\end{center}
{\it q}

c

$\rho$

{\it A}

ZADANIE 3.

$\mathrm{W}$ wierszu zapisano kolejno 20171iczb. Pierwsza zapisana 1iczba jest równa 8, a suma $\mathrm{k}\mathrm{a}\dot{\mathrm{z}}$ dych

kolejnych siedmiu liczb jest równa 70. Ob1icz ostatnia z zapisanych 1iczb.

ZADANIE 4.

Znajd $\acute{\mathrm{z}}$ takie liczby naturalne $x, y, \dot{\mathrm{z}}\mathrm{e}xy=8400$ oraz $\mathrm{N}\mathrm{W}\mathrm{D}\{x,y\}=20.$

ZADANIE 5.

Kwadrat $A$ ma dwa boki pokrywające się z promieniami okręgu, a kwadrat $B$ ma dwa wierz-

chofki lezące na tym samym okręgu oraz częściowo wspófdzieli bok z $A$. Wyznacz stosunek

pola kwadratu $A$ do pola kwadratu $B.$
\begin{center}
\includegraphics[width=43.332mm,height=43.332mm]{./LigaMatematycznaMatuskiego_Gim_Zestaw3_2018_2019_page0_images/image002.eps}
\end{center}
{\it A}

{\it B}


\end{document}