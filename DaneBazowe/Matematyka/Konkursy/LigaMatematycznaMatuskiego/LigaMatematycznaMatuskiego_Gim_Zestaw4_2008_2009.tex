\documentclass[a4paper,12pt]{article}
\usepackage{latexsym}
\usepackage{amsmath}
\usepackage{amssymb}
\usepackage{graphicx}
\usepackage{wrapfig}
\pagestyle{plain}
\usepackage{fancybox}
\usepackage{bm}

\begin{document}

LIGA MATEMATYCZNA

Gimnazjum

Półfinał

201utego 2009

ZADANIE I.

Pan Jan produkuje reklamowe chusty w ksztafcie trójkąta prostokątnego równoramiennego.

Poniewaz klienci skarzyli się, $\dot{\mathrm{z}}\mathrm{e}$ są za male, więc postanowił powiększyć je wydłuzając oba

krótsze boki po 10 cm. Skutkiem tego powierzchnia chusty wzrosła o 550 $\mathrm{c}\mathrm{m}^{2}$ Ile jest teraz

równa powierzchnia chusty?

ZADANIE 2.

Mamy 5 kawałków papieru. Niektóre z nich rozcinamy na 5 kawa1ków.

Następnie niektóre

kawałki znów dzielimy na 5 kawa1ków, itd. Czy w ten sposób $\mathrm{m}\mathrm{o}\dot{\mathrm{z}}$ na otrzymać 1000 kawałków

papieru?

ZADANIE 3.

Pawef ma 10 kieszeni i 54 monety jednozfotowe. Chce umieścić swoje pieniądze w kieszeniach

w taki sposób, aby w $\mathrm{k}\mathrm{a}\dot{\mathrm{z}}$ dej kieszeni byla inna ilość monet. Czy jest to $\mathrm{m}\mathrm{o}\dot{\mathrm{z}}$ liwe?

ZADANIE 4.

Czterech przyjaciół wędkarzy, wśród nich Adam i Piotr, wybrało się na ryby. Po zakończonym

wędkowaniu okazało się, $\dot{\mathrm{z}}\mathrm{e}$ trzej z nich- bez Adama- złowili średnio po 14 ryb, a trzej - bez

Piotra- średnio po 10 ryb. Kto złowił więcej ryb: Adam czy Piotr i o i1e?

ZADANIE 5.

Fabryka produkująca cukierki pakuje je do sześciennych pudelek o krawędzi dlugošci 10 cm.

Pudełka te mają być pakowane po 12 sztuk w prostopadłościenne paczki. Jak na1ezy u1ozyć

pudefka, aby pole powierzchni paczki było najmniejsze?


\end{document}