\documentclass[a4paper,12pt]{article}
\usepackage{latexsym}
\usepackage{amsmath}
\usepackage{amssymb}
\usepackage{graphicx}
\usepackage{wrapfig}
\pagestyle{plain}
\usepackage{fancybox}
\usepackage{bm}

\begin{document}

LIGA MATEMATYCZNA

PÓLFINAL

51utego 20l0

GIMNAZJUM

ZADANIE I.

Czy liczba $\sqrt{16+8\sqrt{3}}-\sqrt{16-8\sqrt{3}}$ jest całkowita? Odpowiedz' uzasadnij.

ZADANIE 2.

$\mathrm{W}$ auli odbylo się zebranie uczniów klas pierwszych dotyczące wyboru języków obcych. $\mathrm{K}\mathrm{a}\dot{\mathrm{z}}\mathrm{d}\mathrm{y}$

uczeń wybral co najmniej jeden język i nie więcej $\mathrm{n}\mathrm{i}\dot{\mathrm{z}}$ dwa. 50 uczniów chce uczyć się ję-

zyka angielskiego, 25 języka niemieckiego, 13 języka francuskiego i 5 języka włoskiego. $\dot{\mathrm{Z}}$ aden

z uczniów chcqcych uczyć się języka wfoskiego nie chce uczyć się innego języka. 15 uczniów

spośród chcących uczyć sięjęzyka angielskiego chce uczyć się $\mathrm{t}\mathrm{e}\dot{\mathrm{z}}$ języka niemieckiego, a 3języka

francuskiego. Tylko jeden uczeń zamierza uczyć się języka niemieckiego i języka francuskiego.

Ilu uczniów bylo na tym spotkaniu?

ZADANIE 3.

Rozwiąz układ równań

({\it yx  z}((({\it xxx} $+++${\it yyy} $+++${\it zzz})))$==$4120.

ZADANIE 4.

Wyznacz liczbę naturalną n, która jest podzielna przez 16 i ma 9 mniejszych od siebie dzie1ni-

ków, których suma równa się n.

ZADANIE 5.

Punkty E, F, G, H dzielą boki prostokąta ABCD w stosunku 1 : 2. Jaki jest stosunek po1a

czworokąta EFGH do pola prostokąta ABCD?
\begin{center}
\includegraphics[width=68.376mm,height=33.324mm]{./LigaMatematycznaMatuskiego_Gim_Zestaw4_2009_2010_page0_images/image001.eps}
\end{center}
D  G  c

H

F

A  E  B
\end{document}
