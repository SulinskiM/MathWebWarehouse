\documentclass[a4paper,12pt]{article}
\usepackage{latexsym}
\usepackage{amsmath}
\usepackage{amssymb}
\usepackage{graphicx}
\usepackage{wrapfig}
\pagestyle{plain}
\usepackage{fancybox}
\usepackage{bm}

\begin{document}

LIGA MATEMATYCZNA

PÓLFINAL

181utego 20ll

GIMNAZJUM

ZADANIE I.

Dane są liczby 1, 2, 3, 4, 5, 6. Wykonujemy operację po1egającą na dodaniu do dwóch spošród

nich liczby l. Na sześciu nowych liczbach wykonujemy tę samą operację. Czy powtarzając

wielokrotnie tę czynnošć $\mathrm{m}\mathrm{o}\dot{\mathrm{z}}$ emy uzyskać wszystkie liczby równe?

ZADANIE 2.

$\mathrm{W}$ trapezie ABCD odcinki AB $\mathrm{i}DC$ są równolegle oraz punkt $E$ jest środkiem boku $AD$. Pole

trójkąta $EBC$ jest równe $16\sqrt{7}$. Oblicz pole trapezu ABCD.

ZADANIE 3.

Wykaz, $\dot{\mathrm{z}}\mathrm{e}$

$(\displaystyle \sqrt{2011}+1)(\frac{1}{\sqrt{1}+\sqrt{2}}+\frac{1}{\sqrt{2}+\sqrt{3}}+\frac{1}{\sqrt{3}+\sqrt{4}}+\ldots+\frac{1}{\sqrt{2010}+\sqrt{2011}})$

jest liczbą całkowitą.

ZADANIE 4.

Panowie Pawel, Andrzej i Jarek uczą matematyki, fizyki i chemii w szkołach w Toruniu, Zako-

panem i Warszawie. Wiadomo, $\dot{\mathrm{z}}\mathrm{e}$

$\bullet$ Pan Paweł nie pracuje w Toruniu;

$\bullet$ Pan Andrzej nie pracuje w Warszawie;

$\bullet$ Torunianin nie uczy chemii;

$\bullet$ Warszawiak jest nauczycielem matematyki;

$\bullet$ Pan Andrzej nie uczy fizyki.

Jakiego przedmiotu i w którym miešcie uczy $\mathrm{k}\mathrm{a}\dot{\mathrm{z}}\mathrm{d}\mathrm{y}$ z nich?

ZADANIE 5.

Czy liczba $10^{11}+10^{12}+10^{13}+10^{14}$ jest podzielna przez 101?


\end{document}