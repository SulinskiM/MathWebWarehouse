\documentclass[a4paper,12pt]{article}
\usepackage{latexsym}
\usepackage{amsmath}
\usepackage{amssymb}
\usepackage{graphicx}
\usepackage{wrapfig}
\pagestyle{plain}
\usepackage{fancybox}
\usepackage{bm}

\begin{document}

LIGA MATEMATYCZNA

PÓLFINAL

161utego 20l2

GIMNAZJUM

ZADANIE I.

Uzasadnij, $\dot{\mathrm{z}}\mathrm{e}$ suma czterech kolejnych liczb naturalnych nieparzystych nie $\mathrm{m}\mathrm{o}\dot{\mathrm{z}}\mathrm{e}$ być liczbą

pierwszą.

ZADANIE 2.

Wykaz, $\dot{\mathrm{z}}\mathrm{e}\sqrt{17-12\sqrt{2}}+\sqrt{17+12\sqrt{2}}$ jest liczbą całkowitą.

ZADANIE 3.

$\mathrm{W}\mathrm{k}\mathrm{a}\dot{\mathrm{z}}$ dym kroku wykonujemy na liczbie jedną z operacji (w dowolnej kolejności):

$\bullet$ podwajamy liczbę;

$\bullet$ skreślamy jej ostatnią cyfrę.

Czy w taki sposób po skończonej ilošci operacji $\mathrm{m}\mathrm{o}\dot{\mathrm{z}}$ na z liczby 378 uzyskać 16?

ZADANIE 4.

Oblicz $1+2-3-4+5+6-7-8+9+10-\ldots-2011-2012+2013+2014.$

ZADANIE 5.

Na prostokątnej tacy Asia ukfadała dwie kwadratowe serwetki o polu 900 $\mathrm{c}\mathrm{m}^{2} \mathrm{k}\mathrm{a}\dot{\mathrm{z}}$ da. Gdy

ułozyła je tak, jak na pierwszym rysunku, to zachodzily na siebie na obszarze o polu 300 $\mathrm{c}\mathrm{m}^{2},$

gdy tak, jak na drugim rysunku, to wspólny obszar miaf 750 $\mathrm{c}\mathrm{m}^{2}$ Jakie pole będzie mial

wspólny obszar obu serwetek, gdy Asia ułozy je w sposób przedstawiony na trzecim rysunku?


\end{document}