\documentclass[a4paper,12pt]{article}
\usepackage{latexsym}
\usepackage{amsmath}
\usepackage{amssymb}
\usepackage{graphicx}
\usepackage{wrapfig}
\pagestyle{plain}
\usepackage{fancybox}
\usepackage{bm}

\begin{document}

LIGA MATEMATYCZNA

im. Zdzisława Matuskiego

PÓLFINAL

71utego 20l3

GIMNAZJUM

ZADANIE I.

Kwadraty ACEG, BCDI, JFGH mają pola równe odpowiednio 3600 $\mathrm{c}\mathrm{m}^{2}, 900 \mathrm{c}\mathrm{m}^{2}$ oraz

400 $\mathrm{c}\mathrm{m}^{2}$ Oblicz pole trójkąta $AIJ.$
\begin{center}
\includegraphics[width=59.340mm,height=58.776mm]{./LigaMatematycznaMatuskiego_Gim_Zestaw4_2012_2013_page0_images/image001.eps}
\end{center}
F  E

H  J

1  D

A  c

ZADANIE 2.

W ulamku

$\mathrm{V}\mathrm{V}$ uIdIIlKu

$\displaystyle \frac{1\cdot 2\cdot 3\cdot\ldots\cdot 23}{1-2+3-4+5-6+\ldots+203}$

licznik jest iloczynem kolejnych liczb naturalnych od l do 23, natomiast w mianowniku ko1ejne

liczby naturalne od l do 203 są naprzemian odejmowane i dodawane. Czy wartość tego u1amka

jest liczbą calkowitą?

ZADANIE 3.

Na boku $CD$ prostokąta ABCD wybrano punkt $E$ w taki sposób, $\dot{\mathrm{z}}\mathrm{e}$ trapez ABCE ma pole

równe 57, 5 $\mathrm{c}\mathrm{m}^{2}$, a pole trapezu ABED jest równe 70 $\mathrm{c}\mathrm{m}^{2}$ Oblicz pole prostokąta ABCD.

ZADANIE 4.

Porównaj liczby $\displaystyle \frac{a}{\alpha-1}$ oraz $\displaystyle \frac{b}{b-1}$, gdy $a\mathrm{i}b$ spefniają warunek $1<a<b.$

ZADANIE 5.

Wykaz, $\dot{\mathrm{z}}\mathrm{e}$ liczba $n^{3}+23n$ jest podzielna przez 6 d1a $\mathrm{k}\mathrm{a}\dot{\mathrm{z}}$ dej liczby naturalnej $n.$


\end{document}