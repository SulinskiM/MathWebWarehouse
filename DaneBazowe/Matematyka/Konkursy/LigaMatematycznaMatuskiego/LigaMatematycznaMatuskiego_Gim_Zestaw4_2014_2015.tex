\documentclass[a4paper,12pt]{article}
\usepackage{latexsym}
\usepackage{amsmath}
\usepackage{amssymb}
\usepackage{graphicx}
\usepackage{wrapfig}
\pagestyle{plain}
\usepackage{fancybox}
\usepackage{bm}

\begin{document}

LIGA MATEMATYCZNA

im. Zdzisława Matuskiego

PÓLFINAL

ll lutego 2014

GIMNAZJUM

ZADANIE I.

Wewnątrz kwadratu ABCD wybrano punkt M w równej odległości od boku CD i od wierz-

chołków A oraz B. Jaką częšć pola kwadratu stanowi pole trójkąta ABM?

ZADANIE 2.

Czy 2014 orzechów $\mathrm{m}\mathrm{o}\dot{\mathrm{z}}$ na wlozyć do 50 woreczków w taki sposób, aby w $\mathrm{k}\mathrm{a}\dot{\mathrm{z}}$ dym bylo więcej $\mathrm{n}\mathrm{i}\dot{\mathrm{z}}$

$20$ orzechów, ale w $\mathrm{k}\mathrm{a}\dot{\mathrm{z}}$ dym inna ich liczba? Czy $\mathrm{m}\mathrm{o}\dot{\mathrm{z}}$ na rozłozyć te orzechy tak, aby w $\mathrm{k}\mathrm{a}\dot{\mathrm{z}}$ dym

woreczku było co najmniej 10 orzechów i w $\mathrm{k}\mathrm{a}\dot{\mathrm{z}}$ dym inna ich liczba?

ZADANIE 3.

Spośród trzystu uczniów klas drugich i trzecich gimnazjum 100 wzię1o udzia1 w o1impiadzie

matematycznej, 80 w fizycznej, 60 w informatycznej, w tym 23 w o1impiadzie matematycznej

i fizycznej, 16 w o1impiadzie matematycznej i informatycznej, 14 w o1impiadzie fizycznej i infor-

matycznej, 5 we wszystkich trzech o1impiadach. I1u uczniów wzię1o udzia1 ty1ko w o1impiadzie

matematycznej? Ilu uczniów wzięło udział tylko wjednej olimpiadzie, a ilu dokladnie w dwóch?

Ilu uczniów nie wzięło udzialu w $\dot{\mathrm{z}}$ adnej olimpiadzie?

ZADANIE 4.

$\acute{\mathrm{S}}$ rednia arytmetyczna liczb $a, b, c$ równa się 12, a średnia arytmetyczna 1iczb $2a+1, 2b, c$ równa

się 17. Ob1icz średnią arytmetyczną 1iczb a $\mathrm{i}b.$

ZADANIE 5.

Wykaz$\cdot, \dot{\mathrm{z}}\mathrm{e}$ dla dowolnych liczb rzeczywistych $a, b, c$ spełniona jest nierówność

$a^{2}+b^{2}+c^{2}+3\geq 2(a+b+c).$


\end{document}