\documentclass[a4paper,12pt]{article}
\usepackage{latexsym}
\usepackage{amsmath}
\usepackage{amssymb}
\usepackage{graphicx}
\usepackage{wrapfig}
\pagestyle{plain}
\usepackage{fancybox}
\usepackage{bm}

\begin{document}

AHADEMIA POMORSHA

III SLUPSHU
\begin{center}
\includegraphics[width=40.740mm,height=4.476mm]{./LigaMatematycznaMatuskiego_Gim_Zestaw4_2015_2016_page0_images/image001.eps}
\end{center}
LIGA MATEMATYCZNA

im. Zdzisława Matuskiego

PÓLFINAL
\begin{center}
\includegraphics[width=34.548mm,height=42.576mm]{./LigaMatematycznaMatuskiego_Gim_Zestaw4_2015_2016_page0_images/image002.eps}
\end{center}
2 marca 20l5

GIMNAZJUM

ZADANIE I.

Ania napisała dziesięć liczb calkowitych. Najpierw napisala dwie liczby, a kolejne uzyskiwała

dodając dwie poprzednie. Wyznacz sumę tych liczb, $\mathrm{j}\mathrm{e}\dot{\mathrm{z}}$ eli wiadomo, $\dot{\mathrm{z}}\mathrm{e}$ pierwszą liczbą jest 34,

a ostatnią 0.

ZADANIE 2.

Równoległobok ABCD zbudowany jest z czterech trójkątów równobocznych o boku o dlugo-

ści l. Wyznacz dlugości przekątnych tego równolegloboku.

ZADANIE 3.

Czy z 1000 kwadratów o boku o długości 1 cm $\mathrm{m}\mathrm{o}\dot{\mathrm{z}}$ na ufozyć prostokąt o obwodzie 1005 cm?

ZADANIE 4.

Wykaz$\cdot, \dot{\mathrm{z}}\mathrm{e}$

-{\it aa}2 $++$11 $\geq$ -{\it a} $+$2 1

dla $\mathrm{k}\mathrm{a}\dot{\mathrm{z}}$ dej liczby dodatniej $a.$

ZADANIE 5.

Trzecią część półki w biblioteczce Bartka zajmują ksiązki o grubości 15 mm, ko1ejną trzecią

część tej półki - ksiązki o grubości 12 mm, a pozostałą część - ksiqzki o grubości 18 mm.

Czytając jedną ksiązkę dziennie w czasie wakacji, Bartek przeczytał wszystkie ksiązki z tej

pófki. Zajęło mu to niecałe dwa miesiące. Ile ksiązek bylo na tej pólce?
\end{document}
