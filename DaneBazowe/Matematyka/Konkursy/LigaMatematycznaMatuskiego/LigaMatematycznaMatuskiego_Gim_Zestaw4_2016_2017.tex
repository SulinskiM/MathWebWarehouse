\documentclass[a4paper,12pt]{article}
\usepackage{latexsym}
\usepackage{amsmath}
\usepackage{amssymb}
\usepackage{graphicx}
\usepackage{wrapfig}
\pagestyle{plain}
\usepackage{fancybox}
\usepackage{bm}

\begin{document}

LIGA MATEMATYCZNA

im. Zdzisława Matuskiego

PÓLFINAL

291utego 20l6

GIMNAZJUM

ZADANIE I.

Dany jest czworokqt wypukly ABCD, gdzie $\triangleleft CBA = 60^{\mathrm{o}}, \triangleleft DBA = 50^{\mathrm{o}}, \triangleleft BAC = 60^{\mathrm{o}},$

$\triangleleft CAD=20^{\mathrm{o}}$ Wyznacz miarę kąta$\triangleleft$ACD.
\begin{center}
\includegraphics[width=37.344mm,height=33.780mm]{./LigaMatematycznaMatuskiego_Gim_Zestaw4_2016_2017_page0_images/image001.eps}
\end{center}
D

c

A  B

ZADANIE 2.

Uzasadnij, $\dot{\mathrm{z}}\mathrm{e}$ dla dowolnej liczby naturalnej $n$ liczba

$\displaystyle \frac{(n+2015)(n+2016)}{2}$

jest naturalna.

ZADANIE 3.

Suma cyfr pewnej liczby trzycyfrowej jest równa ll. $\mathrm{J}\mathrm{e}\dot{\mathrm{z}}$ eli przestawimy cyfry jedności i setek,

nie zmieniając cyfry dziesiątek, to otrzymamy liczbę o 99 mniejszą. Wyznacz wszystkie takie

liczby trzycyfrowe.

ZADANIE 4.

Wykaz, $\dot{\mathrm{z}}\mathrm{e}$ dla dowolnej liczby naturalnej $n$ liczba

11$\ldots$ 122$\ldots$ 233$\ldots$ 344$\ldots$ 4

jest podzielna przez 12, gdy jedynek jest $n$, dwójek jest $2n$, trójek jest $3n$, czwórek jest $4n.$

ZADANIE 5.

Odcinek $BC$ jest średnicą okręgu oraz $|BC|=\sqrt{901}, |BD|=1, |DA|=16$. Niech $|EC|=x.$

Oblicz $x.$
\begin{center}
\includegraphics[width=36.324mm,height=45.156mm]{./LigaMatematycznaMatuskiego_Gim_Zestaw4_2016_2017_page0_images/image002.eps}
\end{center}
A

E

D

B c


\end{document}