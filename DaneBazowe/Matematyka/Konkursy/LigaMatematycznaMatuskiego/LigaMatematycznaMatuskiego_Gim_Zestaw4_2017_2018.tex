\documentclass[a4paper,12pt]{article}
\usepackage{latexsym}
\usepackage{amsmath}
\usepackage{amssymb}
\usepackage{graphicx}
\usepackage{wrapfig}
\pagestyle{plain}
\usepackage{fancybox}
\usepackage{bm}

\begin{document}

LIGA MATEMATYCZNA

im. Zdzisława Matuskiego

PÓLFINAL

161utego 20l7

GIMNAZJUM

ZADANIE I.

$\mathrm{W}$ kwadracie ABCD punkt $E$ jest środkiem boku AD, $F$ jest środkiem boku $DC$ oraz $G$ jest

środkiem odcinka $EF$. Odcinki $EF$ oraz $BG$ podzielily kwadrat na trzy części, z których jedna

- czworokąt- ma pole równe 28. Ob1icz po1e kwadratu.

ZADANIE 2.

$\mathrm{J}\mathrm{e}\dot{\mathrm{z}}$ eli pewną liczbę dwucyfrową pomnozymy przez sumę jej cyfr, to otrzymamy 90. $\mathrm{J}\mathrm{e}\dot{\mathrm{z}}$ eli

przestawimy cyfry tej liczby i $\mathrm{t}\mathrm{e}\dot{\mathrm{z}}$ pomnozymy przez ich sumę, to uzyskamy 306. Znajd $\acute{\mathrm{z}}$ tę liczbę.

ZADANIE 3.

Rozwiąz ukfad równań

$\left\{\begin{array}{l}
a+b=1\\
\frac{1}{2\sqrt{a}}+\frac{1}{2\sqrt{b}}=\frac{2}{\sqrt{a}+\sqrt{b}}.
\end{array}\right.$

ZADANIE 4.

Prostokąt o wymiarach calkowitych zostal rozcięty na dwanaście kwadratów o bokach o dlugości

2, 2, 3, 3, 5, 5, 7, 7, 8, 8, 9, 9. Oblicz obwód tego prostokąta.

ZADANIE 5.

Do zapisania liczby trzydziestocyfrowej wykorzystano dziesięć cyfr 0, dziesięć cyfr 1 i dzie-

sięč cyfr 2. Czy $\mathrm{m}\mathrm{o}\dot{\mathrm{z}}$ na w tej liczbie dokonać takiego przestawienia cyfr, aby otrzymać liczbę

podzielną przez 9?


\end{document}