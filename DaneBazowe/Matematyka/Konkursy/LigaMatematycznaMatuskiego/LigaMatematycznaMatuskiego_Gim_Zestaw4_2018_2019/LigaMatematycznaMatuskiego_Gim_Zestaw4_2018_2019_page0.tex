\documentclass[a4paper,12pt]{article}
\usepackage{latexsym}
\usepackage{amsmath}
\usepackage{amssymb}
\usepackage{graphicx}
\usepackage{wrapfig}
\pagestyle{plain}
\usepackage{fancybox}
\usepackage{bm}

\begin{document}

LIGA MATEMATYCZNA

im. Zdzisława Matuskiego

PÓLFINAL

211utego 20l8

GIMNAZJUM

(klasa VII szkoły podstawowej, klasa II i III gimnazjum)

ZADANIE I.

$\mathrm{W}$ styczniu 1993 roku pani Ania ukończy1a ty1e 1at, i1e wynosi suma cyfr jej roku urodzenia.

$\mathrm{W}$ którym roku urodzila się pani Ania?

ZADANIE 2.

$\mathrm{J}\mathrm{e}\dot{\mathrm{z}}$ eli liczbę dwucyfrową $A$ podzielimy przez sumę jej cyfr, to otrzymamy 4 i resztę 6. $\mathrm{J}\mathrm{e}\dot{\mathrm{z}}$ eli

podzielimy liczbę $A$ przez sumę jej cyfr pomniejszoną o 2, to uzyskamy 5 i resztę 3. Znajd $\acute{\mathrm{z}}$

liczbę $A.$

ZADANIE 3.

Dane sa dwa okręgi o środkach $S_{1}, S_{2}$ i $\mathrm{k}\mathrm{a}\dot{\mathrm{z}}\mathrm{d}\mathrm{y}$ o promieniu 24. Okrąg o środku $S$ jest styczny

zewnętrznie do danych dwóch okręgów oraz do prostej przechodzącej przez punkty $S_{1}$ i $S_{2}$. Wia-

domo, $\dot{\mathrm{z}}\mathrm{e}$ odległość między punktami $S_{1}$ i $S_{2}$ jest równa 72. Ob1icz promień okręgu o środku $S.$

ZADANIE 4.

Wyznacz wszystkie liczby cafkowite $k$, dla których liczba

$k+2016$

$k+2018$

jest calkowita.

ZADANIE 5.

Konik polny skacze wzdfuz prostej. Pierwszy skok ma dlugość l cm, drugi 3 cm (w tę samą

lub w przeciwnq stronę), następny 5 cm, i tak da1ej. Czy $\mathrm{m}\mathrm{o}\dot{\mathrm{z}}\mathrm{e}$ się zdarzyć, $\dot{\mathrm{z}}\mathrm{e}$ po 99 skokach

konik polny znajdzie się w punkcie wyjścia?
\end{document}
