\documentclass[a4paper,12pt]{article}
\usepackage{latexsym}
\usepackage{amsmath}
\usepackage{amssymb}
\usepackage{graphicx}
\usepackage{wrapfig}
\pagestyle{plain}
\usepackage{fancybox}
\usepackage{bm}

\begin{document}

50

$\rightarrow\not\subset \mathrm{D}\vdash$

flkademia

P omorskawStupsku

LIGA MATEMATYCZNA

im. Zdzisława Matuskiego

PÓLFINAL 261utego 2019

GIMNAZJUM

(klasa VII i VIII szkoły podstawowej, klasa III gimnazjum)

ZADANIE I.

$\mathrm{W}$ rombie ABCD punkty $M\mathrm{i}N$, rózne od punktów $A, B\mathrm{i}C$, lezą na odcinkach odpowiednio

{\it AB}, $BC$ tak, $\dot{\mathrm{z}}\mathrm{e}$ trójkąt $DMN$ jest równoboczny oraz $|AD| = |MD|$. Wyznacz miarę kąta

$ABC.$

ZADANIE 2.

Znajd $\acute{\mathrm{z}}$ wszystkie liczby trzycyfrowe, które są ll razy większe od sumy swoich cyfr.

ZADANIE 3.

Wyznacz wszystkie pary $(x,y)$ liczb calkowitych dodatnich spełniające warunki $x+y=320$

oraz $\mathrm{N}\mathrm{W}\mathrm{D}\{x,y\}=40.$

ZADANIE 4.

Bartek wybral pewną liczbę (niekoniecznie róznych) liczb ze zbioru $\{-1,0,1,2\}$ w taki sposób,

$\dot{\mathrm{z}}\mathrm{e}$ ich suma jest równa 19, a suma ich kwadratów jest równa 99. Jaka jest największa $\mathrm{m}\mathrm{o}\dot{\mathrm{z}}$ liwa

wartość sumy sześcianów liczb wybranych przez Bartka?

ZADANIE 5.

Sześciokąt, którego wszystkie kąty mają miarę $120^{\mathrm{o}}$ wpisano w trójkąt równoboczny o boku

o dlugości 9. D1ugości trzech boków sześciokąta są równe 5, 5, 3. Ob1icz obwód sześciokąta.
\begin{center}
\includegraphics[width=34.848mm,height=32.508mm]{./LigaMatematycznaMatuskiego_Gim_Zestaw4_2019_2020_page0_images/image001.eps}
\end{center}
3

5
\end{document}
