\documentclass[a4paper,12pt]{article}
\usepackage{latexsym}
\usepackage{amsmath}
\usepackage{amssymb}
\usepackage{graphicx}
\usepackage{wrapfig}
\pagestyle{plain}
\usepackage{fancybox}
\usepackage{bm}

\begin{document}

LIGA MATEMATYCZNA

FINAL

25 kwietnia 2009

GIMNAZJUM

ZADANIE I.

Długošci boków trzech kwadratów, które widzisz na rysunku, są liczbami naturalnymi. Wia-

domo, $\dot{\mathrm{z}}\mathrm{e}BC=CD$ oraz zamalowana figura ma pole równe 31 $\mathrm{c}\mathrm{m}^{2}$ Oblicz pole największego

z tych kwadratów.

ZADANIE 2.

Tablicę $3\mathrm{x}3$ podzielono na 9 jednakowych kwadratów, w których umieszczono 1iczby $-1, 0$, 1.

Uzasadnij, $\dot{\mathrm{z}}\mathrm{e}$ wśród ošmiu sum (liczb z $\mathrm{k}\mathrm{a}\dot{\mathrm{z}}$ dego wiersza, $\mathrm{k}\mathrm{a}\dot{\mathrm{z}}$ dej kolumny i glównych przeką-

tnych) co najmniej dwie są równe.

ZADANIE 3.

Na boku $BC$ prostokąta ABCD wybrano punkt $E$ tak, $\dot{\mathrm{z}}\mathrm{e}$ stosunek pól trójkąta $CDE$ i trapezu

ABED jest równy $\displaystyle \frac{1}{4}$. Oblicz $\displaystyle \frac{CE}{EB}.$

ZADANIE 4.

$\mathrm{W}$ pewnej szkole zorganizowano kólko plastyczne, informatyczne i sportowe. $\mathrm{W}$ zajęciach pla-

stycznych uczestniczy 73 uczniów, w informatycznych- 128, w sportowych- 103, przy czym

w plastycznych i informatycznych-28, w p1astycznych i sportowych-20, w informatycznych

i sportowych-43 oraz w p1astycznych, informatycznych i sportowych-7. I1u uczniów uczestni-

czy w zajęciach informatycznych i sportowych, ale nie uczestniczy w plastycznych? Ilu uczniów

uczestniczy dokładnie w dwóch rodzajach zajęć? Ilu uczniów bierze udziaf tylko w zajęciach

sportowych?

ZADANIE 5.

Janek z Olą zebrali trzy razy więcej grzybów $\mathrm{n}\mathrm{i}\dot{\mathrm{z}}$ Franek, a Ola z Frankiem- pięć razy więcej

grzybów $\mathrm{n}\mathrm{i}\dot{\mathrm{z}}$ Janek. Kto uzbierał więcej grzybów: Janek razem z Frankiem czy Ola?
\end{document}
