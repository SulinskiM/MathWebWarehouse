\documentclass[a4paper,12pt]{article}
\usepackage{latexsym}
\usepackage{amsmath}
\usepackage{amssymb}
\usepackage{graphicx}
\usepackage{wrapfig}
\pagestyle{plain}
\usepackage{fancybox}
\usepackage{bm}

\begin{document}

LIGA MATEMATYCZNA

FINAL

26 marca 20l0

GIMNAZJUM

ZADANIE I.

Wykaz, $\dot{\mathrm{z}}\mathrm{e}$ liczba $2009^{2010}-2. 2009^{2009}+2009^{2008}$ jest podzielna przez 2008.

ZADANIE 2.

Rozwiąz układ równań

({\it yx  z}((({\it xxy} $+++${\it yzz})))$==$51130.

ZADANIE 3.

Na zajęcia do Mlodziezowego Centrum Kultury uczęszcza 100 osób: 38 osób na zajęcia te-

atralne, $49-$ na zajęcia muzyczne, $34-$ na zajęcia plastyczne, $9-$ na teatralne i muzyczne,

$8-\mathrm{n}\mathrm{a}$ teatralne i plastyczne, $6-\mathrm{n}\mathrm{a}$ muzyczne i plastyczne. Ile osób bierze udziaf we wszyst-

kich trzech rodzajach zajęć?

ZADANIE 4.

Na okręgu zaznaczono sześć punktów. $K\mathrm{a}\dot{\mathrm{z}}\mathrm{d}\mathrm{y}$ z odcinków fączących te punkty pomalowano

na czerwono lub niebiesko. Wykaz, $\dot{\mathrm{z}}\mathrm{e}$ otrzymano przynajmniej jeden jednokolorowy trójkąt.

ZADANIE 5.

Przekątne czworokąta wypukfego dzielą go na cztery trójkąty. Pola trzech z nich są równe l,

2, 3. Znajd $\acute{\mathrm{z}}$ pole czworokąta.
\end{document}
