\documentclass[a4paper,12pt]{article}
\usepackage{latexsym}
\usepackage{amsmath}
\usepackage{amssymb}
\usepackage{graphicx}
\usepackage{wrapfig}
\pagestyle{plain}
\usepackage{fancybox}
\usepackage{bm}

\begin{document}

LIGA MATEMATYCZNA

FINAL

30 marca 20ll

GIMNAZJUM

ZADANIE I.

Na Dzień Kobiet Stefek, Tomek i Romek podarowali swoim dziewczynom: Sabinie, Teresie

i Renacie bukiety kwiatów: róze, storczyki i tulipany. Imiona narzeczonych w $\mathrm{k}\mathrm{a}\dot{\mathrm{z}}$ dej parze

zaczynają się na rózne litery. $\dot{\mathrm{Z}}$ aden chlopiec nie dał dziewczynie kwiatów, których nazwa

zaczyna się na tę samą literę, co jego imię. Wiadomo, $\dot{\mathrm{z}}\mathrm{e}$ ten, który dał storczyki Sabinie ma

imię zaczynające się na tę samą literę, co imię narzeczonej Romka i inną $\mathrm{n}\mathrm{i}\dot{\mathrm{z}}$ nazwa kwiatów,

które Stefek dal narzeczonej. Ktojest parą ijakie kwiaty dal $\mathrm{k}\mathrm{a}\dot{\mathrm{z}}\mathrm{d}\mathrm{y}$ chlopak swojej dziewczynie?

ZADANIE 2.

Liczbę pierwszą 2011 zapisano jako $\alpha^{2}-b^{2}$, gdzie $a\mathrm{i}b$ są liczbami naturalnymi. Oblicz a $\mathrm{i}b.$

ZADANIE 3.

W państwie Cyfry zbudowano dziewięć miast, które nazwano 1, 2, 3, 4, 5, 6, 7, 8, 9. Ki1ka z nich

polączono liniami lotniczymi. Podrózny zauwazył, $\dot{\mathrm{z}}\mathrm{e}$ dwa miasta mają połączenia lotnicze

wtedy i tylko wtedy, gdy dwucyfrowa liczba utworzona z cyfr- nazw tych miast jest podzielna

przez 3. Czy $\mathrm{m}\mathrm{o}\dot{\mathrm{z}}$ na z miasta l dolecieć do miasta 9?

ZADANIE 4.

W trapezie prostokątnym róznica kwadratów dlugošci przekątnych wynosi 21, wysokość jest

równa 4, a dfuzsze ramię ma długość 5. Ob1icz po1e trapezu.

ZADANIE 5.

Wykaz$\cdot, \dot{\mathrm{z}}\mathrm{e}$ równolegloboki ABCD $\mathrm{i}$ DEFG mają równe pola.
\begin{center}
\includegraphics[width=76.356mm,height=33.024mm]{./LigaMatematycznaMatuskiego_Gim_Zestaw5_2010_2011_page0_images/image001.eps}
\end{center}
D  c

F

A  B
\end{document}
