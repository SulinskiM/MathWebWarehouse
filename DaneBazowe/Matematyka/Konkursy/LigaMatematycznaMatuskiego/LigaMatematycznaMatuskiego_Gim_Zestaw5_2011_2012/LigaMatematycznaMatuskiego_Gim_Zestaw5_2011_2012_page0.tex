\documentclass[a4paper,12pt]{article}
\usepackage{latexsym}
\usepackage{amsmath}
\usepackage{amssymb}
\usepackage{graphicx}
\usepackage{wrapfig}
\pagestyle{plain}
\usepackage{fancybox}
\usepackage{bm}

\begin{document}

1 Liceum O$\mathrm{g}\text{ó} 1\mathrm{o}\mathrm{k}\mathrm{s}\mathrm{z}\mathrm{t}\mathrm{a}1_{\mathrm{C}}\mathrm{a}^{\mathrm{c}\mathrm{e}}\mathrm{w}\mathrm{S}1$psku

AkadmiPomorskawSiupsku

LIGA MATEMATYCZNA

FINAL

ll kwietnia 2012

GIMNAZJUM

ZADANIE I.

Wykaz$\cdot, \dot{\mathrm{z}}\mathrm{e}$ liczba $\sqrt{13-4\sqrt{3}}+\sqrt{37-20\sqrt{3}}$ jest calkowita.

ZADANIE 2.

Jedna z przekątnych wielokąta wypuklego, którego obwód jest równy 31 cm, dzie1i go na dwa

wielokąty o obwodach 21 cm i 30 cm. Wyznacz d1ugość przekątnej.

ZADANIE 3.

$\mathrm{W}$ jednym domu mieszkają bracia Pawel i Gawel. Paweł ma więcej $\mathrm{n}\mathrm{i}\dot{\mathrm{z}}30$, a mniej $\mathrm{n}\mathrm{i}\dot{\mathrm{z}}40$ lat.

Gaweł ma więcej $\mathrm{n}\mathrm{i}\dot{\mathrm{z}}40$, ale mniej $\mathrm{n}\mathrm{i}\dot{\mathrm{z}}50$ lat. Ile lat ma $\mathrm{k}\mathrm{a}\dot{\mathrm{z}}\mathrm{d}\mathrm{y}$ z braci, $\mathrm{j}\mathrm{e}\dot{\mathrm{z}}$ eli wiadomo, $\dot{\mathrm{z}}\mathrm{e}$ iloczyn

ich lat jest równy trzeciej potędze liczby naturalnej?

ZADANIE 4.

Wykaz, $\dot{\mathrm{z}}\mathrm{e}$ suma kwadratów trzech kolejnych liczb calkowitych nieparzystych powiększona

$01$ jest podzielna przez 12.

ZADANIE 5.

Liczby naturalne od l do 1000 pomnozono ko1ejno $\mathrm{k}\mathrm{a}\dot{\mathrm{z}}$ da przez $\mathrm{k}\mathrm{a}\dot{\mathrm{z}}$ dą. Uzasadnij, $\dot{\mathrm{z}}\mathrm{e}$ wśród tych

iloczynów więcej jest liczb parzystych $\mathrm{n}\mathrm{i}\dot{\mathrm{z}}$ nieparzystych.
\end{document}
