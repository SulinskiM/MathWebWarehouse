\documentclass[a4paper,12pt]{article}
\usepackage{latexsym}
\usepackage{amsmath}
\usepackage{amssymb}
\usepackage{graphicx}
\usepackage{wrapfig}
\pagestyle{plain}
\usepackage{fancybox}
\usepackage{bm}

\begin{document}

LIGA MATEMATYCZNA

im. Zdzisława Matuskiego

FINAL

15 kwietnia 20l4

GIMNAZJUM

ZADANIE I.

$\mathrm{W}$ pewnej klasiejest 31 uczniów. Jeden z nich zrobi1 w dyktandzie 13 b1ędów, wszyscy pozosta1i

mniej. Wykaz, $\dot{\mathrm{z}}\mathrm{e}$ przynajmniej trzech uczniów zrobilo po tyle samo blędów.

ZADANIE 2.

Kwadrat podzielono na dwa prostokąty, których stosunek obwodów jest równy 5: 4. Wyznacz

stosunek pól tych prostokątów.

ZADANIE 3.

$\mathrm{J}\mathrm{e}\dot{\mathrm{z}}$ eli w pewnej liczbie skreślimy ostatnią cyfrę, która jest równa 7, to 1iczba zmniejszy się

$0$ 31156. Jaka to liczba?

ZADANIE 4.

Uzasadnij, $\dot{\mathrm{z}}\mathrm{e}$ dla dowolnych liczb rzeczywistych $a, b, c$ spefniona jest nierówność

$2a^{2}+b^{2}+c^{2}\geq 2a(b+c).$

ZADANIE 5.

$\acute{\mathrm{S}}$ redni wiek babci, dziadka i siedmiu wnucząt jest równy 281at, natomiast średni wiek siedmiu

wnucząt jest równy 151at. I1e 1at ma dziadek, $\mathrm{j}\mathrm{e}\dot{\mathrm{z}}$ eli jest starszy od babci o trzy lata?


\end{document}