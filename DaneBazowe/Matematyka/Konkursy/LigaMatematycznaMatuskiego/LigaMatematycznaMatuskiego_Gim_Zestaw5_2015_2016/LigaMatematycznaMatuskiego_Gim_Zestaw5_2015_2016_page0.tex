\documentclass[a4paper,12pt]{article}
\usepackage{latexsym}
\usepackage{amsmath}
\usepackage{amssymb}
\usepackage{graphicx}
\usepackage{wrapfig}
\pagestyle{plain}
\usepackage{fancybox}
\usepackage{bm}

\begin{document}

AHADEMIA POMORSHA

III SLUPSHU
\begin{center}
\includegraphics[width=40.740mm,height=4.476mm]{./LigaMatematycznaMatuskiego_Gim_Zestaw5_2015_2016_page0_images/image001.eps}
\end{center}
LIGA MATEMATYCZNA

im. Zdzislawa Matuskiego

FINAL
\begin{center}
\includegraphics[width=34.548mm,height=42.576mm]{./LigaMatematycznaMatuskiego_Gim_Zestaw5_2015_2016_page0_images/image002.eps}
\end{center}
16 kwietnia 20l5

GIMNAZJUM

ZADANIE I.

Wykaz$\cdot, \dot{\mathrm{z}}\mathrm{e}$ dla dowolnych liczb rzeczywistych $a, b, c$ spelniona jest nierówność

$a^{2}+2b^{2}+3c^{2}-2a-8b-18c>-37.$

ZADANIE 2.

Czy 59 miast $\mathrm{m}\mathrm{o}\dot{\mathrm{z}}$ na pofączyć drogami tak, aby $\mathrm{k}\mathrm{a}\dot{\mathrm{z}}$ de miasto bylo polaczone drogą z trzema

innymi miastami?

ZADANIE 3.

Długości boków AB i AD prostokąta ABCD są równe, odpowiednio, 8 i 4. Punkty E, F, G, H

są środkami boków AB, BC, CD, AD, a punkty MiN są, odpowiednio, środkami odcinków

EFiGH. Oblicz pole trójkąta AMN.

ZADANIE 4.

Niech

$\displaystyle \frac{x}{a-b}=\frac{y}{b-c}=\frac{z}{c-a}=2015,$

gdzie $a, b, c, x, y, z$ są liczbami rzeczywistymi. Oblicz sumę $x+y+z.$

ZADANIE 5.

Do pewnej liczby dwucyfrowej dopisujemy cyfrę 2 raz z 1ewej, raz z prawej strony. Róznica

otrzymanych liczb trzycyfrowych jest dwa razy większa od szukanej liczby. Jaka to liczba?
\end{document}
