\documentclass[a4paper,12pt]{article}
\usepackage{latexsym}
\usepackage{amsmath}
\usepackage{amssymb}
\usepackage{graphicx}
\usepackage{wrapfig}
\pagestyle{plain}
\usepackage{fancybox}
\usepackage{bm}

\begin{document}

LIGA MATEMATYCZNA

im. Zdzisława Matuskiego

FINAL

25 kwietnia 20l6

GIMNAZJUM

ZADANIE I.

Wykaz, $\dot{\mathrm{z}}\mathrm{e}$ liczba
\begin{center}
\includegraphics[width=79.812mm,height=10.416mm]{./LigaMatematycznaMatuskiego_Gim_Zestaw5_2016_2017_page0_images/image001.eps}
\end{center}
2222$\ldots$ 23333$\ldots$ 34444$\ldots$ 45555$\ldots$ 5

2n cyfr 2 3n cyfr 3  4n cyfr 4 5n cyfr 5

jest podzielna przez 45 d1a $\mathrm{k}\mathrm{a}\dot{\mathrm{z}}$ dej liczby naturalnej $n.$

ZADANIE 2.

Rozwiąz ukfad równań

$\left\{\begin{array}{l}
(x+y)(x+y+z)=72\\
(y+z)(x+y+z)=120\\
(x+z)(x+y+z)=96.
\end{array}\right.$

ZADANIE 3.

$\mathrm{W}$ trapezie ABCD punkt $S\mathrm{l}\mathrm{e}\dot{\mathrm{z}}\mathrm{y}$ na podstawie $AB$, punkt $R\mathrm{l}\mathrm{e}\dot{\mathrm{z}}\mathrm{y}$ na podstawie $CD$. Odcinki

$DS\mathrm{i}AR$ przecinają się w punkcie $K$, a odcinki $CS\mathrm{i}BR$ przecinają się w punkcie $L$. Wykaz,

$\dot{\mathrm{z}}\mathrm{e}$ suma pól trójkątów $AKD\mathrm{i}LBC$ jest równa polu czworokąta KSLR.

ZADANIE 4.

Na final Ligi Matematycznej w dniu 25 kwietnia przysz1o 149 fina1istów ze szkofy podstawowej

i gimnazjum. $K\mathrm{a}\dot{\mathrm{z}}\mathrm{d}\mathrm{y}$ z nich uściskiem dloni przywital $\mathrm{k}\mathrm{a}\dot{\mathrm{z}}$ dego swego znajomego wśród fina-

listów. Uzasadnij, $\dot{\mathrm{z}}\mathrm{e}$ istnieje finalista, który ma parzystą liczbę znajomych wśród finalistów.

ZADANIE 5.

Wykaz, $\dot{\mathrm{z}}\mathrm{e}$ liczba czterocyfrowa, której cyfra tysięcy jest równa cyfrze dziesiątek, a cyfra setek

jest równa cyfrze jedności, nie $\mathrm{m}\mathrm{o}\dot{\mathrm{z}}\mathrm{e}$ być kwadratem liczby naturalnej.


\end{document}