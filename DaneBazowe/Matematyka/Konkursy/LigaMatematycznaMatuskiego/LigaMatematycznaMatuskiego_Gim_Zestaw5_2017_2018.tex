\documentclass[a4paper,12pt]{article}
\usepackage{latexsym}
\usepackage{amsmath}
\usepackage{amssymb}
\usepackage{graphicx}
\usepackage{wrapfig}
\pagestyle{plain}
\usepackage{fancybox}
\usepackage{bm}

\begin{document}

LIGA MATEMATYCZNA

im. Zdzisława Matuskiego

FINAL

24 kwietnia 20l7

GIMNAZJUM

ZADANIE I.

Na stole pofozono po jednym patyczku o dlugości 2, 4, 6, 8, 9, 10, 30, 40, 50 $\mathrm{i}60$. Adam

zbudowal ramkę wybierając trzy patyczki tak, aby obwód trójkąta byl jak najmniejszy. $\mathrm{Z}$ po-

zostalych patyków Bartek wybral trzy, z których $\mathrm{m}\mathrm{o}\dot{\mathrm{z}}$ na zbudować trójkąt o największym ob-

wodzie. Ostatnimi czterema patykami zainteresowal się Czarek, wybraf trzy i zbudowal z nich

trójkatną ramkę. Który patyk pozostal na stole?

ZADANIE 2.

$\mathrm{W}$ rombie ABCD punkty $M\mathrm{i}N$, rózne od punktów $A, B\mathrm{i}C$, lezą na odcinkach odpowiednio

{\it AB}, $BC$ tak, $\dot{\mathrm{z}}\mathrm{e}$ trójkat $DMN$ jest równoboczny oraz $|AD| = |MD|$. Wyznacz miarę kata

$\triangleleft ABC.$

ZADANIE 3.

Suma liczby trzycyfrowej i liczby otrzymanej z napisania cyfr poprzedniej liczby w odwrotnej

kolejności jest równa 444. Róznicą tych 1iczb jest 198. Wyznacz 1iczbę trzycyfrową wiedząc, $\dot{\mathrm{z}}\mathrm{e}$

suma jej cyfr jest równa 6.

ZADANIE 4.

Czy suma 2017 róznych 1iczb pierwszych $\mathrm{m}\mathrm{o}\dot{\mathrm{z}}\mathrm{e}$ być liczba parzystą? Czy iloczyn 2017 róznych

liczb pierwszych $\mathrm{m}\mathrm{o}\dot{\mathrm{z}}\mathrm{e}$ być liczbą parzystą? Odpowied $\acute{\mathrm{z}}$ uzasadnij.

ZADANIE 5.

Rozwazmy liczby trzycyfrowe zaczynające i kończące się tą samą cyfrą. Wykaz, $\dot{\mathrm{z}}\mathrm{e}\mathrm{j}\mathrm{e}\dot{\mathrm{z}}$ eli suma

pierwszej i drugiej cyfry takiej liczby jest podzielna przez 7, to sama 1iczba $\mathrm{t}\mathrm{e}\dot{\mathrm{z}}$ dzieli się przez 7.


\end{document}