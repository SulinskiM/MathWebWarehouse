\documentclass[a4paper,12pt]{article}
\usepackage{latexsym}
\usepackage{amsmath}
\usepackage{amssymb}
\usepackage{graphicx}
\usepackage{wrapfig}
\pagestyle{plain}
\usepackage{fancybox}
\usepackage{bm}

\begin{document}

LIGA MATEMATYCZNA

im. Zdzisława Matuskiego

FINAL

16 kwietnia 20l8

GIMNAZJUM

(klasa VII szkoły podstawowej, klasa II i III gimnazjum)

ZADANIE I.

Suma 20181iczb ca1kowitych jest 1iczbą nieparzystą. Jaką 1iczbą, parzystą czy nieparzystą, jest

iloczyn tych liczb? Odpowied $\acute{\mathrm{z}}$ uzasadnij.

ZADANIE 2.

Obecnie ($\mathrm{w}$ 2018 roku) ojciec i syn mają razem $131\mathrm{l}\mathrm{a}\mathrm{t}$. Obaj urodzili się w $XX$ wieku. Ostat-

nie dwie cyfry roku urodzenia ojca stanowią liczbę będacą polową liczby utworzonej z dwóch

ostatnich cyfr roku urodzenia syna. Podaj lata urodzenia ojca i syna.

ZADANIE 3.

Wykaz, $\dot{\mathrm{z}}\mathrm{e}$ róznica kwadratów dowolnej liczby pierwszej większej od 2 i 1iczby o 2 od niej

mniejszej jest podzielna przez 8.

ZADANIE 4.

Dane są dwa kwadraty o bokach $a\mathrm{i}b$ (jak na rysunku). Oblicz stosunek pól czworokąta ABCH

i kwadratu ABCD.
\begin{center}
\includegraphics[width=58.776mm,height=37.440mm]{./LigaMatematycznaMatuskiego_Gim_Zestaw5_2018_2019_page0_images/image001.eps}
\end{center}
{\it F E}

{\it H}

{\it G  b A}

{\it a}

ZADANIE 5.

Wiadomo, $\dot{\mathrm{z}}\mathrm{e}$ punkty $B, C1\mathrm{e}\dot{\mathrm{Z}}$ a na bokach trójkąta $AED, |AB| = |BC|=|CD|= |DE|$ oraz

$\triangleleft ADE=140^{\mathrm{o}}$ Wyznacz miarę kata $EAD.$
\begin{center}
\includegraphics[width=107.436mm,height=20.724mm]{./LigaMatematycznaMatuskiego_Gim_Zestaw5_2018_2019_page0_images/image002.eps}
\end{center}
{\it D}

c
\end{document}
