\documentclass[a4paper,12pt]{article}
\usepackage{latexsym}
\usepackage{amsmath}
\usepackage{amssymb}
\usepackage{graphicx}
\usepackage{wrapfig}
\pagestyle{plain}
\usepackage{fancybox}
\usepackage{bm}

\begin{document}
\begin{center}
\includegraphics[width=20.628mm,height=30.024mm]{./LigaMatematycznaMatuskiego_Gim_Zestaw5_2019_2020_page0_images/image001.eps}
\end{center}
0

flkademia

P omorskawStupsku

LIGA MATEMATYCZNA

im. Zdzisława Matuskiego

FINAL 26 marca 2019

GIMNAZJUM

(klasa VII i VIII szkoły podstawowej, klasa III gimnazjum)

ZADANIE I.

Trójkąt podzielono na dwa trójkąty równoramienne i czworokąt o kącie o mierze $80^{\mathrm{o}}$ tak, jak

na rysunku. Wyznacz miarę kąta $\alpha.$
\begin{center}
\includegraphics[width=34.488mm,height=55.020mm]{./LigaMatematycznaMatuskiego_Gim_Zestaw5_2019_2020_page0_images/image002.eps}
\end{center}
$80^{\circ}$

ZADANIE 2.

Ania chce zbudować sześcienną kostkę o wymiarach $4\times 4\times 4$ mając 32 biafe i 32 czarne sześciany

jednostkowe. Planuje uzyskać na powierzchni kostki jak najwięcej bialych ścian sześcianów

jednostkowych. Jaka część powierzchni kostki będzie biafa przy takim ustawieniu?

ZADANIE 3.

Bartek napisal kilka dwucyfrowych liczb naturalnych majqcych tę wlasność, $\dot{\mathrm{z}}\mathrm{e} \mathrm{k}\mathrm{a}\dot{\mathrm{z}}$ de dwie

z nich są względnie pierwsze, ale $\dot{\mathrm{z}}$ adna z nich nie jest liczbą pierwszą. Ile najwięcej liczb mógl

napisać?

ZADANIE 4.

Wyznacz wszystkie liczby całkowite dodatnie $n\mathrm{i}d$ o tej własności, $\dot{\mathrm{z}}\mathrm{e}$ dzieląc liczbę 164 przez

$d$ otrzymamy iloraz $n$ oraz resztę 10.

ZADANIE 5.

Znajd $\acute{\mathrm{z}}$ wszystkie liczby trzycyfrowe, które są 50 razy większe od sumy swoich cyfr.


\end{document}