\documentclass[a4paper,12pt]{article}
\usepackage{latexsym}
\usepackage{amsmath}
\usepackage{amssymb}
\usepackage{graphicx}
\usepackage{wrapfig}
\pagestyle{plain}
\usepackage{fancybox}
\usepackage{bm}

\begin{document}

LIGA MATEMATYCZNA

im. Zdzisława Matuskiego

$\mathrm{P}\mathrm{A}\dot{\mathrm{Z}}$ DZIERNIK 2019

SZKOLA PODSTAWOWA

klasy IV - VI

ZADANIE I.

W pola diagramu wpisano siedem kolejnych liczb naturalnych (w kolejności od najmniejszej

do największej). Suma trzech pierwszych liczb jest równa 69. I1e 1iczb podzie1nych przez 3

znajduje się wśród tych liczb?
\begin{center}
\includegraphics[width=35.556mm,height=6.456mm]{./LigaMatematycznaMatuskiego_Klasa5i6_Zestaw1_2019_2020_page0_images/image001.eps}
\end{center}
ZADANIE 2.

$\mathrm{W}$ regatach $\dot{\mathrm{z}}$ eglarskich wzięło udzia148 ch1opców. Sześciu z nich przyby1o z jednym bra-

tem, dziewięciu z dwoma braćmi i czterech z trzema braćmi. Pozostali chlopcy przybyli bez

rodzeństwa. $\mathrm{Z}$ ilu rodzin było tych 48 ch1opców?

ZADANIE 3.

Prostokątną kartkę papieru podzielono na kwadraty i prostokąt A. Dlugość boku kwadratu $\mathrm{B}$

jest równa 4. Ob1icz obwód prostokąta A.

ZADANIE 4.

Dziewczynki zbierafy koniczynę na lące. Niektóre galązki mialy po trzy listki, a inne po cztery.

Razem zebraly 39 ga1ązek, na których byfo 1ącznie 1281istków. I1e gafązek cztero1istnej koni-

czyny zebraly dziewczynki?

ZADANIE 5.

Ile jest liczb czterocyfrowych o sumie cyfr równej 3? Czy suma tych 1iczb jest 1iczbą podzie1ną

przez 9? A przez 3?


\end{document}