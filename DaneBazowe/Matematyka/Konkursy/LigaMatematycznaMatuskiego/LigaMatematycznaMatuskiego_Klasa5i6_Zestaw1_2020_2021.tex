\documentclass[a4paper,12pt]{article}
\usepackage{latexsym}
\usepackage{amsmath}
\usepackage{amssymb}
\usepackage{graphicx}
\usepackage{wrapfig}
\pagestyle{plain}
\usepackage{fancybox}
\usepackage{bm}

\begin{document}

LIGA MATEMATYCZNA

im. Zdzisława Matuskiego

$\mathrm{P}\mathrm{A}\dot{\mathrm{Z}}$ DZIERNIK 2020

SZKOLA PODSTAWOWA

klasy IV - VI

ZADANIE I.

Uczniowie klasy IV zjadają worek prazonej kukurydzy w ciągu 6 minut, uczniowie k1asy V taki

worek zjadają w ciągu 3 minut. $\mathrm{W}$ ciągu ilu minut zostanie zjedzony taki worek kukurydzy

wspólnie przez uczniów obu klas?

ZADANIE 2.

Ania bardzo lubi jabfka, marchewki i ciastka. $K\mathrm{a}\dot{\mathrm{z}}$ dego dnia zjada albo 9 marchewek, a1bo

2 jabfka, albo l jablko i 4 marchewki, albo l ciastko. Przez l0 kolejnych dni Ania zjadla 30

marchewek i 9 jab1ek. I1e ciastek zjad1a dziewczynka w czasie tych 10 dni?

ZADANIE 3.

Znajd $\acute{\mathrm{z}}$ najmniejszą liczbę naturalną podzielnq przez 15, która zapisano za pomocą samych zer

i jedynek.

ZADANIE 4.

Spotkafo się trzech artystów: Adam Bialy-aktor, Bartek Czarny- muzyk i Czarek Rudy-

malarz.

- Zauwazcie, $\dot{\mathrm{z}}\mathrm{e}$ kolor naszych wfosów nie pokrywa się z nazwiskiem. - powiedzial ten z czarnymi

włosami.

- Masz rację. - odpowiedział Adam.

Jaki kolor wlosów mial $\mathrm{k}\mathrm{a}\dot{\mathrm{z}}\mathrm{d}\mathrm{y}$ z nich?

ZADANIE 5.

$\mathrm{W} \mathrm{k}\mathrm{a}\dot{\mathrm{z}}$ dym wierzcholku kwadratu Adam wpisaf pewna liczbę, a na $\mathrm{k}\mathrm{a}\dot{\mathrm{z}}$ dym boku kwadratu

zapisaf sumę liczb z obu końców. Liczby wpisane na bokach kwadratu to cztery z następujących:

15, 11, 19, 21, 23. Która z liczb nie znalazla się na $\dot{\mathrm{z}}$ adnym boku?


\end{document}