\documentclass[a4paper,12pt]{article}
\usepackage{latexsym}
\usepackage{amsmath}
\usepackage{amssymb}
\usepackage{graphicx}
\usepackage{wrapfig}
\pagestyle{plain}
\usepackage{fancybox}
\usepackage{bm}

\begin{document}

LIGA MATEMATYCZNA

im. Zdzislawa Matuskiego

$\mathrm{P}\mathrm{A}\dot{\mathrm{Z}}$ DZIERNIK 2021

SZKOLA PODSTAWOWA

klasy IV - VI

ZADANIE I.

W Polsce popularne są dwa rodzaje biedronek: dwukropki i siedmiokropki. Na planszach w pra-

cowni przyrodniczej uczniowie narysowali 35 biedronek bez kropek i nak1ei1i na nie 175 czarnych

kropek. Ile biedronek ma siedem kropek, a ile dwie kropki?

ZADANIE 2.

W figurze zlozonej z czterech kwadratów, kwadrat B ma bok o dfugości 8 cm. D1ugości dwóch

odcinków zaznaczono na rysunku. Oblicz obwód tej figury.

ZADANIE 3.

$\mathrm{W}$ skarbonce Adama sa tylko monety pięćdziesięciogroszowe i dziesięciogroszowe. Wszystkich

monet jest 40. Pewnego dnia ch1opiec rozmieni1 po1owę posiadanych pięćdziesięciogroszówek na

dziesięciogroszówki i teraz ma w skarbonce 60 monet. I1e dziesięciogroszówek jest wśród nich?

ZADANIE 4.

$\mathrm{W}$ niektóre pola tablicy wpisano 0. $\mathrm{W}$ pozostale puste pola wpisz dwie jedynki, dwie dwójki,

dwie trójki i dwie czwórki tak, aby suma liczb w $\mathrm{k}\mathrm{a}\dot{\mathrm{z}}$ dym wierszu i w $\mathrm{k}\mathrm{a}\dot{\mathrm{z}}$ dej kolumnie była taka

sama. Czy jest tylko jeden sposób uzupelnienia tablicy? Odpowied $\acute{\mathrm{z}}$ uzasadnij.
\begin{center}
\begin{tabular}{|l|l|l|l|}
\hline
\multicolumn{1}{|l|}{}&	\multicolumn{1}{|l|}{$0$}&	\multicolumn{1}{|l|}{ $0$}&	\multicolumn{1}{|l|}{}	\\
\hline
\multicolumn{1}{|l|}{ $0$}&	\multicolumn{1}{|l|}{}&	\multicolumn{1}{|l|}{}&	\multicolumn{1}{|l|}{ $0$}	\\
\hline
\multicolumn{1}{|l|}{}&	\multicolumn{1}{|l|}{ $0$}&	\multicolumn{1}{|l|}{ $0$}&	\multicolumn{1}{|l|}{}	\\
\hline
\multicolumn{1}{|l|}{ $0$}&	\multicolumn{1}{|l|}{}&	\multicolumn{1}{|l|}{}&	\multicolumn{1}{|l|}{ $0$}	\\
\hline
\end{tabular}

\end{center}
ZADANIE 5.

Ile cyfr ma najdluzszy $\mathrm{m}\mathrm{o}\dot{\mathrm{z}}$ liwy ciąg cyfr nie zawierający cyfry 0, w którym $\mathrm{k}\mathrm{a}\dot{\mathrm{z}}$ de dwie kolejne

cyfry tworzą liczbę będącą kwadratem liczby naturalnej?
\end{document}
