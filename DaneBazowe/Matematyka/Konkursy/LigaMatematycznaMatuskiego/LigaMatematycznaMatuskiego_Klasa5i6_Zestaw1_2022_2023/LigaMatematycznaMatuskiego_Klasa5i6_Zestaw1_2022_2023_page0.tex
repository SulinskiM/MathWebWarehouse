\documentclass[a4paper,12pt]{article}
\usepackage{latexsym}
\usepackage{amsmath}
\usepackage{amssymb}
\usepackage{graphicx}
\usepackage{wrapfig}
\pagestyle{plain}
\usepackage{fancybox}
\usepackage{bm}

\begin{document}

LIGA MATEMATYCZNA

im. Zdzislawa Matuskiego

$\mathrm{P}\mathrm{A}\dot{\mathrm{Z}}$ DZIERNIK 2022

SZKOLA PODSTAWOWA

klasy IV - VI

ZADANIE I.

$\mathrm{W}$ pewnym sześciokącie $\mathrm{k}\mathrm{a}\dot{\mathrm{z}}$ de dwa kolejne boki są prostopadfe. Dlugości pięciu boków tego

sześciokąta są równe 5, 6, 8, 10, 16. Jaką dfugość $\mathrm{m}\mathrm{o}\dot{\mathrm{z}}\mathrm{e}$ mieć szósty bok?

ZADANIE 2.

Prostokat ABCD podzielono na trzy mniejsze prostokąty tak, jak na rysunku. Wyznacz pole

środkowego prostokąta i jego wymiary wiedząc, $\dot{\mathrm{z}}\mathrm{e}$ pola dwóch prostokątów i dlugości dwóch

odcinków podane są na rysunku.

D

c

8
\begin{center}
\begin{tabular}{|l|}
\hline
\multicolumn{1}{|l|}{$20$}	\\
\hline
\multicolumn{1}{|l|}{}	\\
\hline
\multicolumn{1}{|l|}{ $28$}	\\
\hline
\end{tabular}

\end{center}
A

110

$\mathrm{B}\downarrow$

ZADANIE 3.

Ile jest liczb stucyfrowych, których iloczyn cyfr jest równy 6?

ZADANIE 4.

Adam zbiera modele samochodów. $\mathrm{W}11$ ponumerowanych pudelkach ułozyf 350 mode1i. $\mathrm{W}\mathrm{k}\mathrm{a}\dot{\mathrm{z}}$-

dych trzech kolejnych pudefkach liczba modelijest równa 99. I1e mode1ijest w szóstym pude1ku?

ZADANIE 5.

Ania i Bartek dostali od mamy 35 cukierków. Rozdzie1i1i je na ki1ka ta1erzy, z których $\mathrm{k}\mathrm{a}\dot{\mathrm{z}}\mathrm{d}\mathrm{y}$

zawieral więcej $\mathrm{n}\mathrm{i}\dot{\mathrm{z}}$ jeden cukierek. Następnie z $\mathrm{k}\mathrm{a}\dot{\mathrm{z}}$ dego talerza zabrali po jednym cukierku

i dolozyli je do pierwszego talerza. Wtedy okazalo się, $\dot{\mathrm{z}}\mathrm{e}$ na $\mathrm{k}\mathrm{a}\dot{\mathrm{z}}$ dym talerzu jest tyle samo

cukierków. Ile cukierków bylo początkowo na $\mathrm{k}\mathrm{a}\dot{\mathrm{z}}$ dym talerzu?
\end{document}
