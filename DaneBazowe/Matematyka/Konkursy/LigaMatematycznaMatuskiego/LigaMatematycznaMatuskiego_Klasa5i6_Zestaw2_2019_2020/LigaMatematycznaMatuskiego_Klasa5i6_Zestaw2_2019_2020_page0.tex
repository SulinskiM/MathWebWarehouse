\documentclass[a4paper,12pt]{article}
\usepackage{latexsym}
\usepackage{amsmath}
\usepackage{amssymb}
\usepackage{graphicx}
\usepackage{wrapfig}
\pagestyle{plain}
\usepackage{fancybox}
\usepackage{bm}

\begin{document}

LIGA MATEMATYCZNA

im. Zdzisława Matuskiego

LISTOPAD 2019

SZKOLA PODSTAWOWA

klasy IV - VI

ZADANIE I.

Kawa ze śmietanką kosztuje 4, 50 zf.

czarna kawa?

Kawa jest drozsza od śmietanki o 3, 90 zf. I1e kosztuje

ZADANIE 2.

$\mathrm{W}$ pudelku jest 30 ku1: biafe, niebieskie i czarne. Ku1 niebieskich jest 8 razy więcej $\mathrm{n}\mathrm{i}\dot{\mathrm{z}}$ bialych.

Ile jest kul $\mathrm{k}\mathrm{a}\dot{\mathrm{z}}$ dego koloru? Rozwaz wszystkie przypadki.

ZADANIE 3.

Prostokątną dziafkę o obwodzie 100 $\mathrm{m}$ podzielono na dwa prostokąty o obwodach 76 $\mathrm{m}$ i 58 $\mathrm{m}.$

Oblicz wymiary mniejszych dzialek.

ZADANIE 4.

Na zlot smoków przybyły 333 potwory. By1y to smoki trzyg1owe i siedmiog1owe. Razem miafy

1399 głów. Których smoków bylo więcej i o ile?

ZADANIE 5.

Czy przestawiając cyfry liczby 111111222222 $\mathrm{m}\mathrm{o}\dot{\mathrm{z}}$ na uzyskać liczbę pierwszą?
\end{document}
