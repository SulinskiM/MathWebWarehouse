\documentclass[a4paper,12pt]{article}
\usepackage{latexsym}
\usepackage{amsmath}
\usepackage{amssymb}
\usepackage{graphicx}
\usepackage{wrapfig}
\pagestyle{plain}
\usepackage{fancybox}
\usepackage{bm}

\begin{document}

LIGA MATEMATYCZNA

im. Zdzisława Matuskiego

LISTOPAD 2020

SZKOLA PODSTAWOWA

klasy IV - VI

ZADANIE I.

Znajd $\acute{\mathrm{z}}$ najmniejszą liczbę naturalną zfozoną tylko z ósemek i zer podzielną przez 72.

ZADANIE 2.

Bartek ma pięć sześciennych klocków. Gdy są ufozone od najmniejszego do największego, to

wysokości $\mathrm{k}\mathrm{a}\dot{\mathrm{z}}$ dych dwóch sqsiednich klocków róznią się o 2 cm. Wysokość wiezy zbudowa-

nej z dwóch najmniejszych sześcianów jest równa wysokości największego sześcianu. Oblicz

wysokośč wiez $\mathrm{y}$ zbudowanej ze wszystkich pięciu sześciennych klocków.

ZADANIE 3.

Ania i Bartek stoją na sąsiednich stopniach schodów. Gdy Bartek stoi na $\mathrm{n}\mathrm{i}\dot{\mathrm{z}}$ szym stopniu,

a Ania na $\mathrm{w}\mathrm{y}\dot{\mathrm{z}}$ szym, to Ania jest o 5 cm $\mathrm{w}\mathrm{y}\dot{\mathrm{z}}$ sza od Bartka. Gdy zamienią się miejscami,

to Bartek jest $\mathrm{w}\mathrm{y}\dot{\mathrm{z}}$ szy od Ani o 25 cm. Jaką wysokość ma jeden stopień schodów?

ZADANIE 4.

Wiadomo, $\dot{\mathrm{z}}\mathrm{e}$ koty zjadly 999919 myszy, $\mathrm{k}\mathrm{a}\dot{\mathrm{z}}\mathrm{d}\mathrm{y}$ kot zjadl tyle samo myszy i $\mathrm{k}\mathrm{a}\dot{\mathrm{z}}\mathrm{d}\mathrm{y}$ kot zjadl

więcej myszy $\mathrm{n}\mathrm{i}\dot{\mathrm{z}}$ bylo kotów. Ile byfo kotów?

ZADANIE 5.

Ania ma kilkanaście dwuzlotówek, zaś Basia ma tyle samo pieniędzy, ale w monetach pięcio-

zlotowych. Ile monet lącznie mają obie dziewczynki?
\end{document}
