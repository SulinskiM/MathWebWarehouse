\documentclass[a4paper,12pt]{article}
\usepackage{latexsym}
\usepackage{amsmath}
\usepackage{amssymb}
\usepackage{graphicx}
\usepackage{wrapfig}
\pagestyle{plain}
\usepackage{fancybox}
\usepackage{bm}

\begin{document}

LIGA MATEMATYCZNA

im. Zdzislawa Matuskiego

LISTOPAD 2021

SZKOLA PODSTAWOWA

klasy IV - VI

ZADANIE I.

Liczba trzycyfrowa $n$ ma następujące wfasności:

$\bullet$ suma cyfr jest równa 16;

$\bullet$ iloczyn cyfr jest rózny od zera, ale cyfrą jedności tego iloczynu jest zero;

$\bullet$ suma cyfr iloczynu cyfr liczby $n$ jest równa 3.

Znajd $\acute{\mathrm{z}}$ największą liczbę $n$ o tych wlasnościach.

ZADANIE 2.

Uzupelnij brakujące mianowniki

-21$+$-31$+$-\fbox{fbox}1$+$-\fbox{fbox}1$+$--712$+$--1018$+$--2116$=$1.

Wskaz wszystkie rozwiazania.

ZADANIE 3.

Boki czworokąta mają dfugość 8, 6, 5 $\mathrm{i} 7$ (kolejność zapisu tych liczb nie musi być zgodna

z dlugościami kolejnych boków). Przekątna o dlugości 12 dzie1i ten czworokąt na dwa trójkąty.

Podaj obwód $\mathrm{k}\mathrm{a}\dot{\mathrm{z}}$ dego z nich.

ZADANIE 4.

$\mathrm{W}$ pewnej kamienicy jest 9 mieszkań. $K\mathrm{a}\dot{\mathrm{z}}$ de z nich ma dwa lub trzy pokoje. Ile jest mieszkań

trzypokojowych, $\mathrm{j}\mathrm{e}\dot{\mathrm{z}}$ eli fącznie wszystkie mieszkania mają 24 pokoje?

ZADANIE 5.

$\mathrm{W}$ ośmiujednakowych skrzynkach znajduje się 140 bute1ek soku, przy czym wjednej ze skrzynek

brakuje kilku butelek, zaś pozostałe skrzynki są pelne. Ile butelek mieści się w dwunastu pelnych

skrzynkach?
\end{document}
