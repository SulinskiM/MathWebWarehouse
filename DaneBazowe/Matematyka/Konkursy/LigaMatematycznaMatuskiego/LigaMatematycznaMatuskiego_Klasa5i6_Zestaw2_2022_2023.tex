\documentclass[a4paper,12pt]{article}
\usepackage{latexsym}
\usepackage{amsmath}
\usepackage{amssymb}
\usepackage{graphicx}
\usepackage{wrapfig}
\pagestyle{plain}
\usepackage{fancybox}
\usepackage{bm}

\begin{document}

LIGA MATEMATYCZNA

im. Zdzislawa Matuskiego

LISTOPAD 2022

SZKOLA PODSTAWOWA

klasy IV - VI

ZADANIE I.

Pewien bogaty logik zostawif swoim dzieciom następujący testament:

$\mathrm{W}$ ogrodzie rosną kolejno posadzone cztery drzewa owocowe: l- czereśnia, 2- grusza, 3-jab1oń,

4- śliwa. Pod jednym z nich zakopalem skarb. Aby go znalez$\acute{}$ć musicie zrywać po jednym liściu

z tych drzew w następujący sposób:

12343211234321$\ldots.$

Pod drzewem, z którego zerwiecie 20221iść znajduje się skarb

Które to drzewo?

ZADANIE 2.

Figura przedstawiona na rysunku sklada się z czterech przystających prostokątów. $K\mathrm{a}\dot{\mathrm{z}}\mathrm{d}\mathrm{y}$ pro-

stokąt ma boki o dlugości a $\mathrm{i}2a$. Oblicz pole figury, $\mathrm{j}\mathrm{e}\dot{\mathrm{z}}$ eli jej obwód jest równy 80.

ZADANIE 3.

Zagadkowy kalkulator ma tylko dwa klawisze $\fbox{$+1$} \mathrm{i} \fbox{$\times 2$}$. Naciśnięcie klawisza $\fbox{$+1$}$ powoduje

dodanie l do liczby na wyświetlaczu, naciśnięcie klawisza $\fbox{$\times 2$}$ powoduje pomnozenie tej liczby

przez 2. Na wyświet1aczu jest teraz 0. Czy uzyskamy 1iczbę 22, $\mathrm{j}\mathrm{e}\dot{\mathrm{z}}$ eli klawisze $\mathrm{m}\mathrm{o}\dot{\mathrm{z}}$ na nacisnąć

tylko siedem razy?

ZADANIE 4.

Znajd $\acute{\mathrm{z}}$ wszystkie liczby trzycyfrowe, których suma cyfr jest równa 6.

ZADANIE 5.

Bartek wykonaf dwadzieścia rzutów sześcienną kostką do gry i otrzymal w sumie 100 oczek. Co

najwyzej ile razy mógl wyrzucić jedno oczko?


\end{document}