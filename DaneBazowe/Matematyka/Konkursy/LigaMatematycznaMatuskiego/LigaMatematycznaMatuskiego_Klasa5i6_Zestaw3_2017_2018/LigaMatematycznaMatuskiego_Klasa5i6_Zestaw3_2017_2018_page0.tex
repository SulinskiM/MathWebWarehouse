\documentclass[a4paper,12pt]{article}
\usepackage{latexsym}
\usepackage{amsmath}
\usepackage{amssymb}
\usepackage{graphicx}
\usepackage{wrapfig}
\pagestyle{plain}
\usepackage{fancybox}
\usepackage{bm}

\begin{document}

LIGA MATEMATYCZNA

im. Zdzisława Matuskiego

GRUD Z$\mathrm{I}\mathrm{E}\acute{\mathrm{N}}$ 2017

SZKOLA PODSTAWOWA

(klasy IV - VI)

ZADANIE I.

Czy wśród liczb

$\bullet$ 66, 666, 6666, 66666, $\ldots$

$\bullet 55$, 555, 5555, 55555, $\ldots$

znajduje się kwadrat liczby naturalnej?

ZADANIE 2.

Drwal Mikołaj ciął drewno na opał do kominka. Wykonując cięcie, rozcinał zawsze jeden kawa-

lek drewna na dwie części. Po wykonaniu 53 cięć Mikofaj mia172 kawa1ki drewna. I1e kawałków

drewna byfo na początku?

ZADANIE 3.

Na Wigilii spotkalo się dwóch kuzynów- matematyków: Adam i Bartek. Adam zapytal kuzyna,

ile lat ma trójka jego dzieci. Ten odparł, $\dot{\mathrm{z}}\mathrm{e}$ iloczyn ich wieku jest równy 72. D1a Adama ta

informacja nie byla wystarczająca. Wtedy Bartek dodal, $\dot{\mathrm{z}}\mathrm{e}$ suma ich wieku to 14. Jednak

i ta wskazówka nie pozwolila Adamowi ustalić wieku dzieci. Dopiero, gdy Bartek powiedzial,

$\dot{\mathrm{z}}\mathrm{e}$ najmlodsze dziecko ma na imię Ewa, Adam poprawnie ustalif wiek dzieci kuzyna. Ile lat

mają dzieci Bartka?

ZADANIE 4.

Obwód prostokąta jest równy 67. Dwusieczna jednego z kątów dzie1i obwód na dwie części

rózniące się o 20. Ob1icz d1ugości boków prostokąta.

ZADANIE 5.

Dany jest trójkąt prostokątny $ABC$ o kącie prostym przy wierzcholku $C$. Dwusieczne popro-

wadzone z wierzchołków $A\mathrm{i}B$ przecinają się w punkcie $D.$ Znajd $\acute{\mathrm{z}}$ miarę kąta $ADB.$

{\it Dwusieczna kąta jest to pótprosta o początku w wierzchotku kąta dzieląca ten kąt na dwa kąty}

{\it przystajqce}.
\end{document}
