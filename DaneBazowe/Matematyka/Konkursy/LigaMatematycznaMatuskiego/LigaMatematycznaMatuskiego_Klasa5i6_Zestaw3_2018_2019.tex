\documentclass[a4paper,12pt]{article}
\usepackage{latexsym}
\usepackage{amsmath}
\usepackage{amssymb}
\usepackage{graphicx}
\usepackage{wrapfig}
\pagestyle{plain}
\usepackage{fancybox}
\usepackage{bm}

\begin{document}

LIGA MATEMATYCZNA

im. Zdzisława Matuskiego

GRUD Z$\mathrm{I}\mathrm{E}\acute{\mathrm{N}}$ 2018

SZKOLA PODSTAWOWA

(klasy IV - VI)

ZADANIE I.

Na $\mathrm{k}\mathrm{a}\dot{\mathrm{z}}$ dej ściance sześciennej kostki napisano dodatnią liczbę calkowitą. Iloczyn liczb na prze-

ciwleglych ściankach jest taki sam dla $\mathrm{k}\mathrm{a}\dot{\mathrm{z}}$ dej pary takich ścianek. Napisane liczby nie muszą

być rózne. Jaka jest najmniejsza $\mathrm{m}\mathrm{o}\dot{\mathrm{z}}$ liwa suma wszystkich liczb znajdujących się na kostce?
\begin{center}
\includegraphics[width=29.976mm,height=31.704mm]{./LigaMatematycznaMatuskiego_Klasa5i6_Zestaw3_2018_2019_page0_images/image001.eps}
\end{center}
9 {\it 6}

ZADANIE 2.

Babcia upiekfa na święta $\mathrm{B}\mathrm{o}\dot{\mathrm{z}}$ ego Narodzenia 1000 pierników. Wnuki Ania i Bartek bawią się

w następujacą grę: w $\mathrm{k}\mathrm{a}\dot{\mathrm{z}}$ dym ruchu zabierają z koszyka pofowę pierników, $\mathrm{j}\mathrm{e}\dot{\mathrm{z}}$ eli ich liczba

jest parzysta lub jedno ciastko, $\mathrm{j}\mathrm{e}\dot{\mathrm{z}}$ eli liczba pierników jest nieparzysta. Po ilu ruchach wyjmą

wszystkie pierniki z koszyka?

ZADANIE 3.

W Wigilię bracia Adam, Bartek i Czarek zjedli pólmisek pierogów. Adam zjadl o 6 pierogów

mniej $\mathrm{n}\mathrm{i}\dot{\mathrm{z}}$ Bartek, Bartek zjadl dwa razy więcej pierogów $\mathrm{n}\mathrm{i}\dot{\mathrm{z}}$ Czarek, a Czarek o 2 więcej $\mathrm{n}\mathrm{i}\dot{\mathrm{z}}$

Adam. Ile pierogów zjedli chfopcy?

ZADANIE 4.

Dodając 9 jednakowych 1iczb dwucyfrowych oraz jednq 1iczbę jednocyfrową Miko1aj otrzyma1

257. Znajd $\acute{\mathrm{z}}$ liczbę jednocyfrową.

ZADANIE 5.

Dane są dwa okręgi o środkach $O_{1}, O_{2}$ styczne zewnętrznie, $\mathrm{k}\mathrm{a}\dot{\mathrm{z}}\mathrm{d}\mathrm{y}$ o promieniu 3. Prosta

$p$ równolegfa do prostej $pr(O_{1},O_{2})$ przecina te okręgi w punktach $A, B, C, D$ tak, jak na

rysunku. Odcinek $BC$ ma dfugość 2. Ob1icz d1ugość odcinka $AD.$
\begin{center}
\includegraphics[width=89.868mm,height=30.780mm]{./LigaMatematycznaMatuskiego_Klasa5i6_Zestaw3_2018_2019_page0_images/image002.eps}
\end{center}
$o_{1}  \dot{o}_{2}$


\end{document}