\documentclass[a4paper,12pt]{article}
\usepackage{latexsym}
\usepackage{amsmath}
\usepackage{amssymb}
\usepackage{graphicx}
\usepackage{wrapfig}
\pagestyle{plain}
\usepackage{fancybox}
\usepackage{bm}

\begin{document}

LIGA MATEMATYCZNA

im. Zdzisława Matuskiego

GRUD Z$\mathrm{I}\mathrm{E}\acute{\mathrm{N}}$ 2019

SZKOLA PODSTAWOWA

klasy IV - VI

ZADANIE I.

Suma cyfr pewnej liczby dwucyfrowej jest równa 7. Ob1icz sumę tej 1iczby i 1iczby o przesta-

wionych cyfrach.

ZADANIE 2.

Numery pokoi w pewnym hotelu są liczbami trzycyfrowymi, przy czym cyfra setek oznacza

numer piętra, na którym znajduje się pokój. Na którym piętrze znajduje się pokój, którego

numer jest sześcianem sumy swoich cyfr?

ZADANIE 3.

$\mathrm{W}$ pudełku jest sześć karteczek, $\mathrm{k}\mathrm{a}\dot{\mathrm{z}}$ da biafa lub czarna. Na pięciu z nich zapisane są liczby 5, 7,

8, 9, 13 (na $\mathrm{k}\mathrm{a}\dot{\mathrm{z}}$ dej kartce jedna liczba). Jaka najmniejsza liczba calkowita dodatnia $\mathrm{m}\mathrm{o}\dot{\mathrm{z}}\mathrm{e}$ być

zapisana na szóstej kartce, $\mathrm{j}\mathrm{e}\dot{\mathrm{z}}$ eli suma liczb z bialych kartek jest równa sumie liczb z czarnych

kartek?

ZADANIE 4.

Herb jednego z matematycznych rodów sklada się z nalozonych na siebie dwóch figur: trójkąta

równobocznego i pięciokąta foremnego (jak na rysunku). Oblicz miary katów $\alpha \mathrm{i}\beta.$
\begin{center}
\includegraphics[width=49.524mm,height=43.992mm]{./LigaMatematycznaMatuskiego_Klasa5i6_Zestaw3_2019_2020_page0_images/image001.eps}
\end{center}
$\beta$

0

ZADANIE 5.

Prostokąt o obwodzie 42 podzie1ono na dwa prostokąty i sześć kwadratów tak, jak na rysunku.

Oblicz obwód prostokąta A.
\begin{center}
\includegraphics[width=48.924mm,height=19.764mm]{./LigaMatematycznaMatuskiego_Klasa5i6_Zestaw3_2019_2020_page0_images/image002.eps}
\end{center}

\end{document}