\documentclass[a4paper,12pt]{article}
\usepackage{latexsym}
\usepackage{amsmath}
\usepackage{amssymb}
\usepackage{graphicx}
\usepackage{wrapfig}
\pagestyle{plain}
\usepackage{fancybox}
\usepackage{bm}

\begin{document}

LIGA MATEMATYCZNA

im. Zdzisława Matuskiego

GRUD Z$\mathrm{I}\mathrm{E}\acute{\mathrm{N}}$ 2020

SZKOLA PODSTAWOWA

klasy IV - VI

ZADANIE I.

$\mathrm{W}$ sadzie rośnie więcej $\mathrm{n}\mathrm{i}\dot{\mathrm{z}}90$, ale mniej $\mathrm{n}\mathrm{i}\dot{\mathrm{z}}100$ drzewek owocowych. Trzecią ich częśč stanowią

jabfonie, czwartą część grusze, a resztę wiśnie. Ile drzew jest w sadzie? Podaj liczbę drzew

$\mathrm{k}\mathrm{a}\dot{\mathrm{z}}$ dego gatunku.

ZADANIE 2.

Skacząc z trampoliny na basenie Adam odbija się od niej na wysokość l $\mathrm{m}$, następnie spada

w dó15 $\mathrm{m}$, wreszczie- wyplywając w górę 2 m- osiaga powierzchnię wody. Najakiej wysokości

nad powierzchnią wody znajduje się trampolina?

ZADANIE 3.

Rozwazmy liczby naturalne od l do 90. I1e jest wśród nich 1iczb podzie1nych dokfadnie przez

dwie spośród liczb 2, 3, 5?

ZADANIE 4.

$\mathrm{W}$ szkole uczy się 100 uczniów. Języka angie1skiego uczy się 85 uczniów, języka niemieckiego

75, języka francuskiego 48, języka hiszpańskiego 93. Uzasadnij, $\dot{\mathrm{z}}\mathrm{e}$ co najmniej jeden uczeń

poznaje wszystkie cztery języki obce.

ZADANIE 5.

Na prostokątnej kartce papieru o wymiarach $11\times 24$ Ania nakleila kwadrat $A$ i trzy identyczne

prostokąty $B$ tak, $\dot{\mathrm{z}}\mathrm{e}$ figury nie nachodzify na siebie. Oblicz obwód wyklejonej figury.
\begin{center}
\begin{tabular}{|l|l|l|}
\hline
\multicolumn{1}{|l|}{}&	\multicolumn{1}{|l|}{$8$}&	\multicolumn{1}{|l|}{}	\\
\hline
\multicolumn{1}{|l|}{{\it B}}&	\multicolumn{1}{|l|}{{\it A}}&	\multicolumn{1}{|l|}{$8$}	\\
\hline
\end{tabular}

\end{center}

\end{document}