\documentclass[a4paper,12pt]{article}
\usepackage{latexsym}
\usepackage{amsmath}
\usepackage{amssymb}
\usepackage{graphicx}
\usepackage{wrapfig}
\pagestyle{plain}
\usepackage{fancybox}
\usepackage{bm}

\begin{document}

LIGA MATEMATYCZNA

im. Zdzislawa Matuskiego

GRUD Z$\mathrm{I}\mathrm{E}\acute{\mathrm{N}}$ 2021

SZKOLA PODSTAWOWA

klasy IV - VI

ZADANIE I.

Numer mieszkania Mikolajajest liczbą dwucyfrowa podzielną przez 13, a po dodaniu do numeru

mieszkania liczby l dostajemy wielokrotność liczby ll. Wyznacz sumę cyfr numeru mieszkania

Mikolaja.

ZADANIE 2.

Obwód pięciokąta wypukfego ABCDE jest równy 82. Obwód czworokąta ABCD wynosi 64,

a obwód czworokąta ACDE równa się 45. Ob1icz obwód trójkąta $ACD.$

ZADANIE 3.

Znajd $\acute{\mathrm{z}}$ cyfrę dziesiqtek najmniejszej liczby zlozonej, która niejest podzielna przez $\dot{\mathrm{z}}$ adną z liczb

pierwszych 2, 3, 5, 7.

ZADANIE 4.

$\mathrm{W}$ pracowni plastycznej jest 19 pędz1i, 13 tubek z niebieską farbą, 12 tubek z czerwoną farbą

$\mathrm{i}8$ z zólta. $K\mathrm{a}\dot{\mathrm{z}}\mathrm{d}\mathrm{y}$ uczeń powinien dostać po dwie tubki farb róznych kolorów i jeden pędzel.

Dla ilu mlodych artystów $\mathrm{m}\mathrm{o}\dot{\mathrm{z}}$ na przygotować taki zestaw?

ZADANIE 5.

Uzupelnij diagram liczbami w taki sposób, aby $\mathrm{k}\mathrm{a}\dot{\mathrm{z}}$ de pole sąsiadowalo z dwoma, w których są

wpisane liczby o sumie równej 10. Dwie z 1iczb zosta1y $\mathrm{j}\mathrm{u}\dot{\mathrm{z}}$ wpisane. Oblicz sumę wszystkich

liczb w diagramie.
\begin{center}
\includegraphics[width=67.968mm,height=18.996mm]{./LigaMatematycznaMatuskiego_Klasa5i6_Zestaw3_2021_2022_page0_images/image001.eps}
\end{center}
3

2
\end{document}
