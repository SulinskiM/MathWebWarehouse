\documentclass[a4paper,12pt]{article}
\usepackage{latexsym}
\usepackage{amsmath}
\usepackage{amssymb}
\usepackage{graphicx}
\usepackage{wrapfig}
\pagestyle{plain}
\usepackage{fancybox}
\usepackage{bm}

\begin{document}

LIGA MATEMATYCZNA

im. Zdzislawa Matuskiego

GRUD Z$\mathrm{I}\mathrm{E}\acute{\mathrm{N}}$ 2022

SZKOLA PODSTAWOWA

klasy IV - VI

ZADANIE I.

Znajd $\acute{\mathrm{z}}$ takie trzy cyfry, z których $\mathrm{m}\mathrm{o}\dot{\mathrm{z}}$ na utworzyč dwa rózne trzycyfrowe sześciany liczb natu-

ralnych. Podaj sumę znalezionych cyfr.

ZADANIE 2.

Róznica cyfr liczby dwucyfrowej jest równa 5. Róznica tej 1iczby i 1iczby utworzonej z niej przez

przestawienie cyfr jest równa 45. Znajd $\acute{\mathrm{z}}$ te liczby.

ZADANIE 3.

Do przygotowania paczek świątecznych Mikolaj $\mathrm{u}\dot{\mathrm{z}}$ y1120 mandarynek i 180 pierników. $\mathrm{W}\mathrm{k}\mathrm{a}\dot{\mathrm{z}}$ dej

paczcejest jednakowa liczba mandarynek ijednakowa liczba pierników. Ile maksymalnie paczek

udalo się przygotować?

ZADANIE 4.

Pan Kowalski ma pięciu synów: Adama, Bartka, Czarka, Darka i Edka. Przygotowal dla nich

pod choinkę pięć prezentów: album ze zdjęciami wozów policyjnych, pilkę, wóz strazacki, klocki

i misia. $K\mathrm{a}\dot{\mathrm{z}}\mathrm{d}\mathrm{y}$ z chłopców otrzymal jeden prezent. Wiadomo, $\dot{\mathrm{z}}\mathrm{e}$

$\bullet$ najstarszy syn dostaf album, a najmfodszy misia;

$\bullet$ Czarek jest starszy od Bartka, ale mlodszy od Edka i nie dostal klocków;

$\bullet$ Adam ma czterech starszych braci;

$\bullet$ Bartek dostal pifkę;

$\bullet$ Darek nie jest najstarszy.

Jaki prezent dostal $\mathrm{k}\mathrm{a}\dot{\mathrm{z}}\mathrm{d}\mathrm{y}$ z chłopców?

ZADANIE 5.

Mikolaj połozyf dziewięć orzechów na planszy $9 \times 9$ w sposób przedstawiony na rysunku.

Po chwili Ania przefozyla trzy orzechy na sąsiednie pola (to znaczy na pola mające wspól-

ny wierzchofek lub bok), a sześć zostafo na swoich miejscach. Mimo $\dot{\mathrm{z}}\mathrm{e}$ trzy orzechy zmienily

miejsce, to nadal w $\mathrm{k}\mathrm{a}\dot{\mathrm{z}}$ dym rzędzie poziomym i pionowym znajduje się jeden orzech. Pokaz

w jaki sposób Ania przelozyfa orzechy.
\begin{center}
\begin{tabular}{|l|l|l|l|l|l|l|l|l|}
\hline
\multicolumn{1}{|l|}{}&	\multicolumn{1}{|l|}{}&	\multicolumn{1}{|l|}{}&	\multicolumn{1}{|l|}{}&	\multicolumn{1}{|l|}{}&	\multicolumn{1}{|l|}{}&	\multicolumn{1}{|l|}{}&	\multicolumn{1}{|l|}{}&	\multicolumn{1}{|l|}{}	\\
\hline
\multicolumn{1}{|l|}{}&	\multicolumn{1}{|l|}{}&	\multicolumn{1}{|l|}{}&	\multicolumn{1}{|l|}{}&	\multicolumn{1}{|l|}{}&	\multicolumn{1}{|l|}{}&	\multicolumn{1}{|l|}{}&	\multicolumn{1}{|l|}{}&	\multicolumn{1}{|l|}{}	\\
\hline
\multicolumn{1}{|l|}{}&	\multicolumn{1}{|l|}{}&	\multicolumn{1}{|l|}{}&	\multicolumn{1}{|l|}{}&	\multicolumn{1}{|l|}{}&	\multicolumn{1}{|l|}{}&	\multicolumn{1}{|l|}{}&	\multicolumn{1}{|l|}{}&	\multicolumn{1}{|l|}{}	\\
\hline
\multicolumn{1}{|l|}{}&	\multicolumn{1}{|l|}{}&	\multicolumn{1}{|l|}{}&	\multicolumn{1}{|l|}{}&	\multicolumn{1}{|l|}{}&	\multicolumn{1}{|l|}{}&	\multicolumn{1}{|l|}{}&	\multicolumn{1}{|l|}{}&	\multicolumn{1}{|l|}{}	\\
\hline
\multicolumn{1}{|l|}{}&	\multicolumn{1}{|l|}{}&	\multicolumn{1}{|l|}{}&	\multicolumn{1}{|l|}{}&	\multicolumn{1}{|l|}{}&	\multicolumn{1}{|l|}{}&	\multicolumn{1}{|l|}{}&	\multicolumn{1}{|l|}{}&	\multicolumn{1}{|l|}{}	\\
\hline
\multicolumn{1}{|l|}{}&	\multicolumn{1}{|l|}{}&	\multicolumn{1}{|l|}{}&	\multicolumn{1}{|l|}{}&	\multicolumn{1}{|l|}{}&	\multicolumn{1}{|l|}{}&	\multicolumn{1}{|l|}{}&	\multicolumn{1}{|l|}{}&	\multicolumn{1}{|l|}{}	\\
\hline
\multicolumn{1}{|l|}{}&	\multicolumn{1}{|l|}{}&	\multicolumn{1}{|l|}{}&	\multicolumn{1}{|l|}{}&	\multicolumn{1}{|l|}{}&	\multicolumn{1}{|l|}{}&	\multicolumn{1}{|l|}{}&	\multicolumn{1}{|l|}{}&	\multicolumn{1}{|l|}{}	\\
\hline
\multicolumn{1}{|l|}{}&	\multicolumn{1}{|l|}{}&	\multicolumn{1}{|l|}{}&	\multicolumn{1}{|l|}{}&	\multicolumn{1}{|l|}{}&	\multicolumn{1}{|l|}{}&	\multicolumn{1}{|l|}{}&	\multicolumn{1}{|l|}{}&	\multicolumn{1}{|l|}{}	\\
\hline
\multicolumn{1}{|l|}{}&	\multicolumn{1}{|l|}{}&	\multicolumn{1}{|l|}{}&	\multicolumn{1}{|l|}{}&	\multicolumn{1}{|l|}{}&	\multicolumn{1}{|l|}{}&	\multicolumn{1}{|l|}{}&	\multicolumn{1}{|l|}{}&	\multicolumn{1}{|l|}{}	\\
\hline
\end{tabular}
\end{center}\end{document}
