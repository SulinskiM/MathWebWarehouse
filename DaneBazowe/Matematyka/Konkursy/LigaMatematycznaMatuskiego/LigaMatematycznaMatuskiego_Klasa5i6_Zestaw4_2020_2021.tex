\documentclass[a4paper,12pt]{article}
\usepackage{latexsym}
\usepackage{amsmath}
\usepackage{amssymb}
\usepackage{graphicx}
\usepackage{wrapfig}
\pagestyle{plain}
\usepackage{fancybox}
\usepackage{bm}

\begin{document}

flkademia

P omorskamStupsku

LIGA MATEMATYCZNA

im. Zdzisława Matuskiego

PÓLFINAL 27 kwietnia 2021

SZKOLA PODSTAWOWA

klasy IV - VI

ZADANIE I.

Ania ma siedem monet dwuzlotowych, a Bartek ma osiem monet pięciozłotowych. Jaką naj-

mniejszą liczbę monet muszą wymienić między sobą, aby mieć równe kwoty?

ZADANIE 2.

Znajd $\acute{\mathrm{z}}$ najmniejszą liczbę naturalną, która w zapisie dziesiętnym ma tylko 0 $\mathrm{i} 1$ oraz jest

podzielna przez 45. Odpowied $\acute{\mathrm{z}}$ uzasadnij.

ZADANIE 3.

Za pięč lat dwie siostry i dwaj bracia będa mieli razem 601at.

piętnaście lat.

Wyznacz lączny ich wiek za

ZADANIE 4.

Tort trzywarstwowy fącznie z paterą, na której stoi, ma 70 cm wysokości. Tort jednowarstwowy

lacznie z taką samą paterą, ma wysokość 36 cm. $\mathrm{W}$ obu tortach $\mathrm{k}\mathrm{a}\dot{\mathrm{z}}$ da warstwa ma tę samą

wysokość. Wyznacz wysokość patery.

ZADANIE 5.

Prostokąt podzielono na dziewięć mniejszych prostokątów.

rysunku. Oblicz obwód $\mathrm{d}\mathrm{u}\dot{\mathrm{z}}$ ego prostokata.

Obwody trzech z nich podano na
\begin{center}
\begin{tabular}{|l|l|l|}
\hline
\multicolumn{1}{|l|}{$20$}&	\multicolumn{1}{|l|}{}&	\multicolumn{1}{|l|}{}	\\
\hline
\multicolumn{1}{|l|}{}&	\multicolumn{1}{|l|}{}&	\multicolumn{1}{|l|}{ $15$}	\\
\hline
\multicolumn{1}{|l|}{}&	\multicolumn{1}{|l|}{ $9$}&	\multicolumn{1}{|l|}{}	\\
\hline
\end{tabular}

\end{center}

\end{document}