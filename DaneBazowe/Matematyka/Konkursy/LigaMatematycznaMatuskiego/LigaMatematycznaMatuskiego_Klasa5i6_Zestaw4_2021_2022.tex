\documentclass[a4paper,12pt]{article}
\usepackage{latexsym}
\usepackage{amsmath}
\usepackage{amssymb}
\usepackage{graphicx}
\usepackage{wrapfig}
\pagestyle{plain}
\usepackage{fancybox}
\usepackage{bm}

\begin{document}

flkademia

P omorskamStupsku

LIGA MATEMATYCZNA

im. Zdzisława Matuskiego

PÓLFINAL 15 marca 2022

SZKOLA PODSTAWOWA

klasy IV - VI

ZADANIE I.

Zestaw $A$ zawiera dziesięć róznych liczb jednocyfrowych. Adam wykreślił jedną liczbę, otrzy-

mując zestaw $B$. Suma wszystkich liczb w tym zestawie to 39. Spośród nich Adam wykreś1i1

dwie liczby, uzyskując zestaw $C$ o sumie liczb równej 37. Następnie usuną1 jeszcze trzy 1iczby,

których suma jest równa 22. Zestaw $D$ sklada się z liczb, które pozostaly. Podaj największą

liczbę zestawu $D.$

ZADANIE 2.

Flamastry w sklepie papierniczym zapakowane są w pudelka po 31ub po 5 sztuk. Wszystkich

pudelek jest 30, a w nich 1ącznie 110 f1amastrów. I1e jest pude1ek z trzema f1amastrami?

ZADANIE 3.

Jaki jest najmniejszy $\mathrm{m}\mathrm{o}\dot{\mathrm{z}}$ liwy obwód trójkąta, którego $\mathrm{k}\mathrm{a}\dot{\mathrm{z}}\mathrm{d}\mathrm{y}$ bok ma inną dlugość, a dlugość

$\mathrm{k}\mathrm{a}\dot{\mathrm{z}}$ dego boku jest liczbą pierwszą?

ZADANIE 4.

$\mathrm{Z}$ cyfr 1, 2, 3, 4, 5, 6, 7, 8, 9 Basia wybrafa cztery ko1ejne, z których ufozy1a 1iczby czterocyfrowe

podzielne przez 36. I1e by1o tych 1iczb?

ZADANIE 5.

$\mathrm{Z}$ czterech jednakowych prostokątów i czterech jednakowych kwadratów Ania ulozyfa kwadrat

o obwodzie 32. Potem z tych samych czworokątów powsta1a figura przedstawiona na drugim

rysunku. Oblicz obwód drugiej figury.


\end{document}