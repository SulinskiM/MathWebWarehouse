\documentclass[a4paper,12pt]{article}
\usepackage{latexsym}
\usepackage{amsmath}
\usepackage{amssymb}
\usepackage{graphicx}
\usepackage{wrapfig}
\pagestyle{plain}
\usepackage{fancybox}
\usepackage{bm}

\begin{document}

LIGA MATEMATYCZNA

im. Zdzisława Matuskiego

PÓLFINAL 281utego 2020

SZKOLAPODSTAklasyIV-VIWOWA

ZADANIE I.

Łączna dfugośč pociagu i mostu jest równa 190 $\mathrm{m}. \mathrm{W}$ pewnym momencie pociąg cafkowicie

wjechal na most i teraz dfugość części mostu bez pociągu jest równa 110 $\mathrm{m}$. Oblicz długość

mostu i dlugość pociągu.

ZADANIE 2.

Kopciuszek mia1100 ziaren maku. Wszystkie wfozy1 do pięciu miseczek w taki sposób, $\dot{\mathrm{z}}\mathrm{e}$

w dwóch pierwszych jest fącznie 30 ziaren, w drugiej i trzeciej 33 ziarna, a w trzeciej i czwartej

41 ziaren. $\mathrm{W}$ piątej jest o ll ziaren więcej $\mathrm{n}\mathrm{i}\dot{\mathrm{z}}$ w pierwszej. Do której miseczki Kopciuszek

wfozyl najmniej ziaren maku?

ZADANIE 3.

Wszystkie smoki i smoczyce mają po sześč łap. $K\mathrm{a}\dot{\mathrm{z}}\mathrm{d}\mathrm{y}$ samiec ma cztery glowy, a $\mathrm{k}\mathrm{a}\dot{\mathrm{z}}$ da samica

trzy glowy. Pewna smocza rodzina kupifa na jesienne pluchy 51 czapek i 84 sztuki ka1oszy.

$K\mathrm{a}\dot{\mathrm{z}}\mathrm{d}\mathrm{y}$ czlonek rodziny na $\mathrm{k}\mathrm{a}\dot{\mathrm{z}}$ dą gfowę zalozyf jedną czapkę, a na $\mathrm{k}\mathrm{a}\dot{\mathrm{z}}$ dą lapę jeden kalosz. Ile

smoków i smoczyc liczy ta rodzina?

ZADANIE 4.

$\mathrm{Z}$ cyfr 1, 2, 3, 4 ukfadamy 1iczby dwucyfrowe o róznych cyfrach. I1e jest takich 1iczb? I1e wśród

nich jest liczb pierwszych?

ZADANIE 5.

Ania wycięła z brystolu jednakowe kartoniki. $K\mathrm{a}\dot{\mathrm{z}}\mathrm{d}\mathrm{y}$ jest prostokątem o bokach o dfugości

16 cm i 7 cm. $\mathrm{Z}$ pięciu takich kartoników dziewczynka ufozyla figurę, jak na rysunku. Oblicz

obwód tej figury.
\end{document}
