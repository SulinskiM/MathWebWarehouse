\documentclass[a4paper,12pt]{article}
\usepackage{latexsym}
\usepackage{amsmath}
\usepackage{amssymb}
\usepackage{graphicx}
\usepackage{wrapfig}
\pagestyle{plain}
\usepackage{fancybox}
\usepackage{bm}

\begin{document}

flkademia

P omorskamStupsku

LIGA MATEMATYCZNA

im. Zdzisława Matuskiego

FINAL 18 maja 2021

SZKOLAPODSTAklasyIV-VIWOWA

ZADANIE I.

Bartek miał osiem karteczek z cyframi 1, 1, 2, 2, 3, 3, 4 $\mathrm{i}4$. Próbował ułozyć z nich liczbę

parzystą podzielną przez 9. $\mathrm{W}$ końcu usunąl jedną karteczkę. $\mathrm{Z}$ siedmiu pozostalych ufozyl

liczbę parzystą podzielną przez 9. Wyznacz największą 1iczbę, którą móg1 utworzyć Bartek.

Odpowied $\acute{\mathrm{z}}$ uzasadnij.

ZADANIE 2.

$\mathrm{W}$ biegu na 100 metrów startuje 625 zawodników. Bieznia stadionu ma 5 torów i ty1ko zwycięzca

$\mathrm{k}\mathrm{a}\dot{\mathrm{z}}$ dego biegu przechodzi do kolejnej rundy, a wszyscy pozostali odpadają z dalszej rywalizacji.

Oblicz najmniejszą liczbę biegów konieczną do wylonienia zwycięzcy zawodów.

ZADANIE 3.

Znajd $\acute{\mathrm{z}}$ wszystkie liczby trzycyfrowe, których iloczyn cyfr jest równy 6.

ZADANIE 4.

Ania ma 183 z1, a Bartek 75 z1. I1e pieniędzy Ania powinna dać Bartkowi, aby zostafo jej dwa

razy więcej $\mathrm{n}\mathrm{i}\dot{\mathrm{z}}$ miafby wtedy chfopiec?

ZADANIE 5.

Pięć kolezanek z grupy kolonijnej ulozyfo kwadrat ze swoich ręczników tak, jak na rysunku.

Ręczniki Ani i Basi mają ksztaft kwadratów, $\mathrm{k}\mathrm{a}\dot{\mathrm{z}}\mathrm{d}\mathrm{y}$ o obwodzie 720 cm. Ręczniki Ce1iny,

Darii i Eli są prostokątami ojednakowych wymiarach. Oblicz obwody prostokątnych ręczników

i $\mathrm{d}\mathrm{u}\dot{\mathrm{z}}$ ego kwadratu utworzonego ze wszystkich ręczników.


\end{document}