\documentclass[a4paper,12pt]{article}
\usepackage{latexsym}
\usepackage{amsmath}
\usepackage{amssymb}
\usepackage{graphicx}
\usepackage{wrapfig}
\pagestyle{plain}
\usepackage{fancybox}
\usepackage{bm}

\begin{document}

flkademia

P omorskamStupsku

LIGA MATEMATYCZNA

im. Zdzisława Matuskiego

FINAL 21 kwietnia 2022

SZKOLA PODSTAWOWA

klasy IV - VI

ZADANIE I.

Ania chce ulozyć kod do swojej szafki w szatni składajacy się z czterech cyfr dopisując do

liczby 77jedną cyfrę z 1ewej i jedną cyfrę z prawej strony w taki sposób, aby otrzymana 1iczba

czterocyfrowa dzielila się przez 18. I1e jest takich kodów? Podaj wszystkie $\mathrm{m}\mathrm{o}\dot{\mathrm{z}}$ liwości.

ZADANIE 2.

Adam zapisal liczbę za pomocą pięciu dziewiątek. Następnie dodal dwie pierwsze cyfry (licząc

od lewej), zmazaf je i w ich miejsce wpisaf otrzymaną sumę. Potem to samo zrobił z nową

liczbą i powtarzal tę czynność z $\mathrm{k}\mathrm{a}\dot{\mathrm{z}}$ dą kolejną uzyskaną liczbą tak dlugo, $\mathrm{a}\dot{\mathrm{z}}$ pojawifa się liczba

jednocyfrowa. Ile razy Adam wykonal opisaną operację zamiany pierwszych dwóch cyfr liczby

na ich sumę?

ZADANIE 3.

Wyznacz cztery kolejne liczby parzyste mające tę wlasność, $\dot{\mathrm{z}}\mathrm{e}$ suma dwóch mniejszych jest

mniejsza od 1000, a suma dwóch większych jest wieksza $\mathrm{n}\mathrm{i}\dot{\mathrm{z}}$ 1000. Podaj wszystkie $\mathrm{m}\mathrm{o}\dot{\mathrm{z}}$ liwości.

ZADANIE 4.

Bartek rzucal dwadzieścia razy sześcienną kostką do gry. Jedno oczko wypadlo dwukrotnie

rzadziej $\mathrm{n}\mathrm{i}\dot{\mathrm{z}}$ dwójka, ale trzy razy częściej $\mathrm{n}\mathrm{i}\dot{\mathrm{z}}$ trójka. Cztery oczka wypadly tyle samo razy

co dwójka, a pięć oczek tyle samo razy co trzy oczka. Ile razy wypadlo sześć oczek?

ZADANIE 5.

Bok kwadratu $A$ ma długość 10, a bok kwadratu $B$ ma dlugość 20. Pomiędzy te kwadraty

wstawiono kwadrat $C$ tak, jak na rysunku. Oblicz obwód otrzymanej figury.
\begin{center}
\includegraphics[width=62.076mm,height=37.284mm]{./LigaMatematycznaMatuskiego_Klasa5i6_Zestaw5_2022_2023_page0_images/image001.eps}
\end{center}
{\it A  C}
\end{document}
