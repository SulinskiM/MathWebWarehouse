\documentclass[a4paper,12pt]{article}
\usepackage{latexsym}
\usepackage{amsmath}
\usepackage{amssymb}
\usepackage{graphicx}
\usepackage{wrapfig}
\pagestyle{plain}
\usepackage{fancybox}
\usepackage{bm}

\begin{document}

LIGA MATEMATYCZNA

im. Zdzisława Matuskiego

$\mathrm{P}\mathrm{A}\dot{\mathrm{Z}}$ DZIERNIK 2019

SZKOLA PODSTAWOWA

klasy VII- VIII

ZADANIE I.

$\mathrm{Z}$ sześciu jednakowych trójkątów prostokątnych o kącie ostrym $60^{\mathrm{o}}$ i najkrótszym boku 10 cm

zbudowano równoleglobok ABCD. Oblicz dfugości obu przekątnych tego równolegfoboku.
\begin{center}
\includegraphics[width=139.596mm,height=26.928mm]{./LigaMatematycznaMatuskiego_Klasa7i8_Zestaw1_2019_2020_page0_images/image001.eps}
\end{center}
ZADANIE 2.

Adam napisal cztery razy z rzędu liczbę dwucyfrową. Wykaz, $\dot{\mathrm{z}}\mathrm{e}$ otrzymana liczba ośmiocyfrowa

jest podzielna przez 101.

ZADANIE 3.

Jedna przekatna pewnego czworokąta dzieli go na dwa trójkąty o obwodach 20 $\mathrm{i}40$, a druga

na trójkąty o obwodach 30 $\mathrm{i}50$. Wiedząc, $\dot{\mathrm{z}}\mathrm{e}$ suma dlugości przekątnych jest równa 26, ob1icz

obwód czworokąta.

ZADANIE 4.

Rozwiąz uklad równań

$\left\{\begin{array}{l}
xy=6\\
yz=12\\
xz=8.
\end{array}\right.$

ZADANIE 5.

Dwa tysiące dziewiętnaście liczb zapisano jedna za drugą. Wiadomo, $\dot{\mathrm{z}}\mathrm{e}$ suma $\mathrm{k}\mathrm{a}\dot{\mathrm{z}}$ dych trzech

kolejnych z nichjest równa 200. Pierwsza z nichjest równa 19, a ostatnia 99. Wyznacz pozosta1e

20171iczb.
\end{document}
