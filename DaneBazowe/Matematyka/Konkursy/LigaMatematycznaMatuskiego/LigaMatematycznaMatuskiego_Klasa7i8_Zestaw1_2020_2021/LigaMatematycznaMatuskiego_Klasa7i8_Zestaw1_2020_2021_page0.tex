\documentclass[a4paper,12pt]{article}
\usepackage{latexsym}
\usepackage{amsmath}
\usepackage{amssymb}
\usepackage{graphicx}
\usepackage{wrapfig}
\pagestyle{plain}
\usepackage{fancybox}
\usepackage{bm}

\begin{document}

LIGA MATEMATYCZNA

im. Zdzisława Matuskiego

$\mathrm{P}\mathrm{A}\dot{\mathrm{Z}}$ DZIERNIK 2020

SZKOLA PODSTAWOWA

klasy VII- VIII

ZADANIE I.

Na promenadzie w Helu koloniści kupowali pamiątki: bursztynowe bransoletki, korale z musze-

lek i pluszowe foczki. $K\mathrm{a}\dot{\mathrm{z}}\mathrm{d}\mathrm{y}$ wybral dwie rózne pamiątki. Foczek kupili dwa razy więcej $\mathrm{n}\mathrm{i}\dot{\mathrm{z}}$

bransoletek, a korali trzy razy więcej $\mathrm{n}\mathrm{i}\dot{\mathrm{z}}$ foczek. Uzasadnij, $\dot{\mathrm{z}}\mathrm{e}$ liczba kolonistów byla podzielna

przez 9, a 1iczba kupionych branso1etek by1a parzysta.

ZADANIE 2.

Pole prostokąta ABCD jest równe l. $K\mathrm{a}\dot{\mathrm{z}}\mathrm{d}\mathrm{y}$ bok tego prostokąta przedfuzono o odcinek równy

temu bokowi i otrzymano punkty $P, Q, R, S$ w taki sposób, $\dot{\mathrm{z}}\mathrm{e}$ punkt $A$ jest środkiem odcinka

$PB, B$ jest środkiem $CQ, C$ jest środkiem $DR, D$ jest środkiem $AS$. Oblicz pole czworokąta

{\it PQRS}.

ZADANIE 3.

Punkt $E\mathrm{l}\mathrm{e}\dot{\mathrm{z}}\mathrm{y}$ wewnątrz kwadratu ABCD tak, $\dot{\mathrm{z}}\mathrm{e}$ trójkąt $ABE$ jest równoboczny. Oblicz miarę

kąta $DCE.$

ZADANIE 4.

$\mathrm{W}$ kolekcji firmy jubilerskiej sa trzy rodzaje naszyjników: z dwiema perlami, z jedną perlą

i takie, które nie mają perel. Naszyjników bez peref jest dwa razy mniej $\mathrm{n}\mathrm{i}\dot{\mathrm{z}}$ wszystkich pozo-

stałych. $\mathrm{W}99$ naszyjnikach jest 100 pereł. I1e jest naszyjników z jedną per1a?

ZADANIE 5.

Liczba trzycyfrowa ma cyfrę jedności równą 5. $\mathrm{J}\mathrm{e}\dot{\mathrm{z}}$ eli do tej liczby dodamy l i otrzymaną sumę

podzielimy przez 3, to otrzymamy 1iczbą trzycyfrową, której cyfra setek jest 1, a następne jej

cyfry są pierwszą i drugą cyfrą liczby wyjściowej. Wyznacz tę liczbę.
\end{document}
