\documentclass[a4paper,12pt]{article}
\usepackage{latexsym}
\usepackage{amsmath}
\usepackage{amssymb}
\usepackage{graphicx}
\usepackage{wrapfig}
\pagestyle{plain}
\usepackage{fancybox}
\usepackage{bm}

\begin{document}

LIGA MATEMATYCZNA

im. Zdzislawa Matuskiego

$\mathrm{P}\mathrm{A}\dot{\mathrm{Z}}$ DZIERNIK 2021

SZKOLA PODSTAWOWA

klasy VII- VIII

ZADANIE I.

$\mathrm{W}$ klasach sportowych VIIa i VIIb $\mathrm{k}\mathrm{a}\dot{\mathrm{z}}\mathrm{d}\mathrm{y}$ uczeń gra w siatkówkę lub w koszykówkę. Jedna

piąta wszystkich uczniów uprawia obie dyscypliny sportu, 24 uczniów gra w siatkówkę, $42$ -

w koszykówkę. Ilu uczniów jest w klasach siódmych, ilu uprawia tylko siatkówkę, ilu tylko

koszykówkę, ilu obie te dyscypliny?

ZADANIE 2.

Sto skfadników zmieniono następująco: pierwszą liczbę zmniejszono o l, drugą zwiększono o 2,

trzecią zmniejszono o 3, czwartq zwiększono o 4, i tak da1ej, setną zwiększono o 100. Jak

zmienila się suma tych stu skladników?

ZADANIE 3.

Rozetnij kwadrat na sześć kwadratów. Oblicz stosunek pól największego i najmniejszego z otrzy-

manych kwadratów.

ZADANIE 4.

Ile jest dwunastocyfrowych liczb podzielnych przez 36, które sk1adają się ty1ko z zer ijedynek?

Odpowied $\acute{\mathrm{z}}$ uzasadnij.

ZADANIE 5.

Na rysunku podane są pola czterech prostokątów. Oblicz $x.$
\begin{center}
\begin{tabular}{|l|ll|}
\hline
\multicolumn{1}{|l|}{$23$}&	\multicolumn{1}{|l|}{ $x$}&	\multicolumn{1}{|l|}{ $19$}	\\
\hline
\multicolumn{1}{|l|}{ $17$}&	\multicolumn{1}{|l}{ $51$}&	\multicolumn{1}{l|}{}	\\
\hline
\end{tabular}

\end{center}

\end{document}