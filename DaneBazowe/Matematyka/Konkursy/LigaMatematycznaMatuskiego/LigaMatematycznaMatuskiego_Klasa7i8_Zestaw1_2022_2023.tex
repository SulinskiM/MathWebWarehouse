\documentclass[a4paper,12pt]{article}
\usepackage{latexsym}
\usepackage{amsmath}
\usepackage{amssymb}
\usepackage{graphicx}
\usepackage{wrapfig}
\pagestyle{plain}
\usepackage{fancybox}
\usepackage{bm}

\begin{document}

LIGA MATEMATYCZNA

im. Zdzislawa Matuskiego

$\mathrm{P}\mathrm{A}\dot{\mathrm{Z}}$ DZIERNIK 2022

SZKOLA PODSTAWOWA

klasy VII- VIII

ZADANIE I.

Wyjez $\mathrm{d}\dot{\mathrm{z}}$ ając na wakacje dwudziestu trzech uczniów klasy VII postanowilo pisać do siebie wia-

domości tekstowe. Pewnego dnia $\mathrm{k}\mathrm{a}\dot{\mathrm{z}}\mathrm{d}\mathrm{y}$ z nich wysfal dwie lub cztery wiadomości. Czy $\mathrm{k}\mathrm{a}\dot{\mathrm{z}}\mathrm{d}\mathrm{y}$

uczeń mógl tego dnia otrzymać dokladnie trzy wiadomości?

ZADANIE 2.

Znajd $\acute{\mathrm{z}}$ taką liczbę trzycyfrową, $\dot{\mathrm{z}}$ eby po dodaniu do niej 500 otrzymać 1iczbę czterocyfrową

podzielną przez 12, a po odjęciu od niej 500 mieč 1iczbę dwucyfrową podzie1ną przez 23.

ZADANIE 3.

Pewien szyfr do sejfu sklada się z 8 róznych cyfr ułozonych ma1ejąco (od 1ewej do prawej).

Liczba ośmiocyfrowa tworzaca szyfr dzieli się przez 180. Jaki to szyfr?

ZADANIE 4.

Suma pięciu liczb trzycyfrowych $\overline{abc}, \overline{bcd}, \overline{cde}, \overline{dea}, \overline{eab}$ jest równa 3996. Ob1icz $a+b+c+d+e.$

ZADANIE 5.

Kwadrat ABCD o polu 400 podzie1ono na kwadrat $K_{1}$ o polu 49, kwadrat $K_{2}$ i figurę $F$. Oblicz

obwód figury $F.$
\begin{center}
\includegraphics[width=46.632mm,height=42.672mm]{./LigaMatematycznaMatuskiego_Klasa7i8_Zestaw1_2022_2023_page0_images/image001.eps}
\end{center}
D  c

$K_{1}$

{\it F}

$K_{2}$

A  B


\end{document}