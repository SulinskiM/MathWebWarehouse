\documentclass[a4paper,12pt]{article}
\usepackage{latexsym}
\usepackage{amsmath}
\usepackage{amssymb}
\usepackage{graphicx}
\usepackage{wrapfig}
\pagestyle{plain}
\usepackage{fancybox}
\usepackage{bm}

\begin{document}

LIGA MATEMATYCZNA

im. Zdzisława Matuskiego

LISTOPAD 2019

SZKOLA PODSTAWOWA

klasy VII- VIII

ZADANIE I.

Wyznacz takie cyfry $a, b, c, \dot{\mathrm{z}}\mathrm{e}\overline{aaa}+b=\overline{bccc}$. Symbol $\overline{xyz}$ oznacza liczbę trzycyfrowa zapisana

w dziesiętnym systemie pozycyjnym.

ZADANIE 2.

Cztery kolezanki z wakacji: Ania, Basia, Daria i Ela kupily sukienki. $K\mathrm{a}\dot{\mathrm{z}}$ da mieszka w innym

mieście i $\mathrm{k}\mathrm{a}\dot{\mathrm{z}}$ da kupila sukienkę w innym kolorze. Odgadnij, w jakim mieście mieszka $\mathrm{k}\mathrm{a}\dot{\mathrm{z}}$ da

z kolezanek oraz jaki jest kolor jej sukienki, $\mathrm{j}\mathrm{e}\dot{\mathrm{z}}$ eli:

$\bullet$ sukienka Ani nie jest czerwona;

$\bullet$ dziewczyna w zielonej sukience nie mieszka w Slupsku;

$\bullet$ Basia ma białą sukienkę, ale nie mieszka w Gdańsku;

$\bullet$ dziewczynka w czerwonej sukience mieszka w Lęborku;

$\bullet$ Daria mieszka w Bytowie, ajej sukienka nie jest niebieska.

ZADANIE 3.

Wyznacz wszystkie liczby naturalne co najmniej dwucyfrowe, które malejąjedenastokrotnie po

skreśleniu cyfry jedności.

ZADANIE 4.

Przy dzieleniu liczb a, b, c przez 5 otrzymujemy odpowiednio reszty 1, 2, 3.

z dzielenia sumy kwadratów tych liczb przez 5.

Podaj resztę

ZADANIE 5.

Czworokąt podzielono przekątnymi na cztery trójkąty. Pola trzech z nich podane są na rysunku.

Oblicz pole szarego trójkąta.
\begin{center}
\includegraphics[width=34.644mm,height=26.064mm]{./LigaMatematycznaMatuskiego_Klasa7i8_Zestaw2_2019_2020_page0_images/image001.eps}
\end{center}
{\it 8}

{\it 6}

3


\end{document}