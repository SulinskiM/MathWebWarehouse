\documentclass[a4paper,12pt]{article}
\usepackage{latexsym}
\usepackage{amsmath}
\usepackage{amssymb}
\usepackage{graphicx}
\usepackage{wrapfig}
\pagestyle{plain}
\usepackage{fancybox}
\usepackage{bm}

\begin{document}

LIGA MATEMATYCZNA

im. Zdzisława Matuskiego

LISTOPAD 2020

SZKOLA PODSTAWOWA

klasy VII- VIII

ZADANIE I.

Znajd $\acute{\mathrm{z}}$ dwie takie liczby naturalne $a, b, \dot{\mathrm{z}}\mathrm{e}$ róznica ich iloczynu i ich sumy jest równa 1000 oraz

$a$ jest kwadratem pewnej liczby naturalnej.

ZADANIE 2.

Punkt $E\mathrm{l}\mathrm{e}\dot{\mathrm{z}}\mathrm{y}$ wewnątrz czworokata ABCD oraz $|AE| = 1, |BE| =4, |CE| =3, |DE| =2.$

Czy obwód tego czworokąta $\mathrm{m}\mathrm{o}\dot{\mathrm{z}}\mathrm{e}$ być równy 20?

ZADANIE 3.

Dane są takie liczby calkowite dodatnie $a, b, c, d, \dot{\mathrm{z}}\mathrm{e}\mathrm{k}\mathrm{a}\dot{\mathrm{z}}$ da z sum $a+b, c+d$ jest nieparzysta.

Uzasadnij, $\dot{\mathrm{z}}\mathrm{e}$ iloczyn abcd jest podzielny przez 4.

ZADANIE 4.

Wykaz, $\dot{\mathrm{z}}$ ejezeli między cyfry liczby dwucyfrowej wstawimy 3 i od otrzymanej 1iczby odejmiemy

daną liczbę dwucyfrową, to otrzymamy liczbę podzielną przez 6.

ZADANIE 5.

Dwa kwadraty lezą wewnatrz $\mathrm{d}\mathrm{u}\dot{\mathrm{z}}$ ego kwadratu tak, jak na rysunku.

równe 48. Ob1icz po1e kwadratu $A.$

Pole kwadratu B jest

{\it B}

{\it A}
\end{document}
