\documentclass[a4paper,12pt]{article}
\usepackage{latexsym}
\usepackage{amsmath}
\usepackage{amssymb}
\usepackage{graphicx}
\usepackage{wrapfig}
\pagestyle{plain}
\usepackage{fancybox}
\usepackage{bm}

\begin{document}

LIGA MATEMATYCZNA

im. Zdzislawa Matuskiego

LISTOPAD 2021

SZKOLA PODSTAWOWA

klasy VII- VIII

ZADANIE I.

Znajd $\acute{\mathrm{z}}$ wszystkie liczby czterocyfrowe, które mają takie dwa dzielniki, $\dot{\mathrm{z}}\mathrm{e}$ ich suma jest równa

110, a róznica 36.

ZADANIE 2.

Pole prostokąta ABCD jest równe 24. Na boku $AB$ zaznaczono punkt $E$ rózny od punktów $A$

$\mathrm{i}B$, na $DC$ zaznaczono punkt $F$ rózny od punktów $C\mathrm{i}D$. Pole trójkąta $AFD$ jest równe 5.

Oblicz pole trójkąta $ECF.$

ZADANIE 3.

Adam dodał zerową, pierwszą, drugą i trzecią potęgę pewnej liczby naturalnej i otrzyma1400.

Jaka to liczba?

ZADANIE 4.

Na okręgu zaznaczono 55 punktów. Trzy z nich oznaczono $A, B, C$. Ania policzyła punkty od

$A$ do $C$, przechodzac raz przez $B$, i otrzymala 31. Gdy 1iczyfa od $A$ do $B$ przechodząc raz przez

$C$, to uzyskala 39.

$\bullet$ Wyznacz najmniejszą liczbę punktów od $C$ do $B$ przy jednokrotnym przejściu przez

punkt $A.$

$\bullet$ Wyznacz najmniejszą liczbę punktów od $B$ do $A$ przy przejściu przez punkt $C.$

ZADANIE 5.

Wfadca pewnego królestwa nagrodzif swoich dwóch dzielnych rycerzy: starszego 110 dukatami,

mlodszego 100 dukatami. Monety znajdowafy się w dwóch rodzajach sakiewek: w ma1ych by1o

po 7 dukatów, w $\mathrm{d}\mathrm{u}\dot{\mathrm{z}}$ ych po 17 dukatów. $K\mathrm{a}\dot{\mathrm{z}}\mathrm{d}\mathrm{y}$ rycerz otrzyma110 sakiewek.

$\bullet$ Ile $\mathrm{d}\mathrm{u}\dot{\mathrm{z}}$ ych sakiewek otrzymal starszy rycerz?

$\bullet$ Ile malych sakiewek dostal mlodszy rycerz?


\end{document}