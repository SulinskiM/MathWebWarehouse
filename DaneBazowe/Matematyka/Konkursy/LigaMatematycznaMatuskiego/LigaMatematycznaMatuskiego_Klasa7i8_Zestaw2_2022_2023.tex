\documentclass[a4paper,12pt]{article}
\usepackage{latexsym}
\usepackage{amsmath}
\usepackage{amssymb}
\usepackage{graphicx}
\usepackage{wrapfig}
\pagestyle{plain}
\usepackage{fancybox}
\usepackage{bm}

\begin{document}

LIGA MATEMATYCZNA

im. Zdzislawa Matuskiego

LISTOPAD 2022

SZKOLA PODSTAWOWA

klasy VII- VIII

ZADANIE I.

Ania pomnozyfa pewną liczbę naturalna przez $\mathrm{k}\mathrm{a}\dot{\mathrm{z}}$ dą zjej cyfr i otrzymafa 1995. Jaka to 1iczba?

ZADANIE 2.

Wiadomo, $\dot{\mathrm{z}}\mathrm{e}x+y+z=0$ oraz $xyz=78$. Oblicz $(x+y)(y+z)(x+z).$

ZADANIE 3.

$\mathrm{W}$ kratkach tablicy o wymiarach $9 \times 17$ rozmieszczono liczby naturalne tak, $\dot{\mathrm{z}}\mathrm{e}$ w $\mathrm{k}\mathrm{a}\dot{\mathrm{z}}$ dym

prostokącie o wymiarach $3\rangle\langle 1$ suma liczbjest nieparzysta. Czy suma wszystkich liczb zapisanych

na tablicy jest parzysta?

ZADANIE 4.

Wykaz, $\dot{\mathrm{z}}\mathrm{e}$ liczba $3^{n+3}+3^{n+4}+3^{n+5}+3^{n+6}$ nie jest podzielna przez 7, a1e jest podzie1na przez

$\mathrm{k}\mathrm{a}\dot{\mathrm{z}}$ dą liczbę naturalną mniejszą $\mathrm{n}\mathrm{i}\dot{\mathrm{z}}11$ i rózną od 7.

ZADANIE 5.

Przekątne pewnego czworokąta są prostopadfe i rozcinają go na cztery trójkąty. Pola dwóch

z nich są równe 9 $\mathrm{i}16$, a pola pozostalych dwóch są równe. Oblicz pole czworokqta.


\end{document}