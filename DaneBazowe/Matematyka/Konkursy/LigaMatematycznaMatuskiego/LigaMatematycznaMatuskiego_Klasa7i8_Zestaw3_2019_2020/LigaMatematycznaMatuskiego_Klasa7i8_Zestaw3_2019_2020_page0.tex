\documentclass[a4paper,12pt]{article}
\usepackage{latexsym}
\usepackage{amsmath}
\usepackage{amssymb}
\usepackage{graphicx}
\usepackage{wrapfig}
\pagestyle{plain}
\usepackage{fancybox}
\usepackage{bm}

\begin{document}

LIGA MATEMATYCZNA

im. Zdzisława Matuskiego

GRUD Z$\mathrm{I}\mathrm{E}\acute{\mathrm{N}}$ 2019

SZKOLA PODSTAWOWA

klasy VII- VIII

ZADANIE I.

Przekatne $AC\mathrm{i}BD$ trapezu ABCD przecinają się w punkcie $O$. Pola trójkątów $AOB\mathrm{i}$ {\it COD}

są równe 9 $\mathrm{i}4$. Oblicz pole trapezu.

ZADANIE 2.

Spośród liczb 30, 31, 32, 33jednajest dzie1nikiem, inna i1orazem, ajeszcze inna resztą w pewnym

dzieleniu liczby trzycyfrowej. Znajd $\acute{\mathrm{z}}$ tę liczbę.

ZADANIE 3.

Liczby $a, b, c$ są calkowite. Wykaz, $\dot{\mathrm{z}}\mathrm{e}$ liczba $(a-b)(b-c)(c-a)$ jest parzysta.

ZADANIE 4.

Na dlugim pasku Adam zapisal liczby calkowite dodatnie w taki sposób, $\dot{\mathrm{z}}\mathrm{e}$ suma $\mathrm{k}\mathrm{a}\dot{\mathrm{z}}$ dych

trzech kolejnych jest równa 10. Liczby 1, 4, 5 są widoczne. Sprawd $\acute{\mathrm{z}}$, czy liczba stojąca na

setnym miejscu jest większa od liczby stojącej na dwusetnym miejscu.
\begin{center}
\includegraphics[width=68.280mm,height=6.396mm]{./LigaMatematycznaMatuskiego_Klasa7i8_Zestaw3_2019_2020_page0_images/image001.eps}
\end{center}
4 1  5

ZADANIE 5.

Grupa 41 studentów za1iczy1a sesję sk1adającą się z trzech egzaminów, w których $\mathrm{m}\mathrm{o}\dot{\mathrm{z}}$ liwymi

ocenami były: bdb, db, dst. Wykaz, $\dot{\mathrm{z}}\mathrm{e}$ co najmniej pięciu studentów zaliczylo sesję zjednako-

wym zbiorem ocen.
\end{document}
