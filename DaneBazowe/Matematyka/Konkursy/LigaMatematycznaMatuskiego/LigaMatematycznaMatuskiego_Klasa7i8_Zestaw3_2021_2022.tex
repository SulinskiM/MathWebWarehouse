\documentclass[a4paper,12pt]{article}
\usepackage{latexsym}
\usepackage{amsmath}
\usepackage{amssymb}
\usepackage{graphicx}
\usepackage{wrapfig}
\pagestyle{plain}
\usepackage{fancybox}
\usepackage{bm}

\begin{document}

LIGA MATEMATYCZNA

im. Zdzislawa Matuskiego

GRUD Z$\mathrm{I}\mathrm{E}\acute{\mathrm{N}}$ 2021

SZKOLA PODSTAWOWA

klasy VII- VIII

ZADANIE I.

Suma dwóch liczb jest równa 57460. Jeś1i do mniejszej 1iczby dopiszemy z prawej strony 92, to

otrzymamy równe liczby. Znajd $\acute{\mathrm{z}}$ je.

ZADANIE 2.

Bartek napisa1151iczb natura1nych i ob1iczyf ich sumę otrzymując wynik 2022. Basia dopisała

znak,,minus'' przed kilkoma z tych liczb i obliczyfa sumę wszystkich swoich liczb. Czy mogla

uzyskać wynik llll?

ZADANIE 3.

Pewna liczba ma cztery dzielniki, z których dwa sa liczbami pierwszymi. Mikolaj wypisal je

w kolejności od najmniejszego do największego. Drugi dzielnik jest o 10 większy od pierwszego,

a czwarty o 130 większy od trzeciego. Która 1iczba ma takie dzie1niki?

ZADANIE 4.

Ania dzielila kolejne liczby parzyste (zaczynając od 0) przez pewną 1iczbę natura1ną i wypisy-

wala reszty z ich dzielenia. Początek zapisu byl następujący: 0, 2, 4, 6, 1, 3, 5, 0, 2, 4, $\ldots$. Bartek

$\mathrm{t}\mathrm{e}\dot{\mathrm{z}}$ wypisywal reszty z dzielenia przez tę samą liczbę co Ania. Dzielil jednak kolejne liczby

nieparzyste zaczynając od l. Zakończyl pracę, gdy po raz trzeci uzyska10. I1e 1iczb wypisa1

Bartek?

ZADANIE 5.

Rozetnij kwadrat na siedem kwadratów. Znajd $\acute{\mathrm{z}}$ stosunek obwodów największego i najmniej-

szego z otrzymanych kwadratów.


\end{document}