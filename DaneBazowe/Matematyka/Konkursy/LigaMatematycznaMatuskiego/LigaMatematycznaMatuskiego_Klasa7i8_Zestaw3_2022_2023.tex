\documentclass[a4paper,12pt]{article}
\usepackage{latexsym}
\usepackage{amsmath}
\usepackage{amssymb}
\usepackage{graphicx}
\usepackage{wrapfig}
\pagestyle{plain}
\usepackage{fancybox}
\usepackage{bm}

\begin{document}

LIGA MATEMATYCZNA

im. Zdzislawa Matuskiego

GRUD Z$\mathrm{I}\mathrm{E}\acute{\mathrm{N}}$ 2022

SZKOLA PODSTAWOWA

klasy VII- VIII

ZADANIE I.

Ile dzielników ma liczba $2^{2}\cdot 3^{5}+2\cdot 3^{6}+2^{3}\cdot 3^{7}$?

ZADANIE 2.

Na prostej zawierającej wysokość $BD$ trójkąta równobocznego $ABC$ wybrano punkt $K$ tak,

aby $|BK|=|AC|$. Punkt $K$ polaczono z punktami A $\mathrm{i}C$. Oblicz miarę kąta $AKC.$

ZADANIE 3.

Czy liczbę 55555553 $\mathrm{m}\mathrm{o}\dot{\mathrm{z}}$ na przedstawić w postaci sumy dwóch liczb pierwszych?

ZADANIE 4.

Suma cyfr pewnej nieparzystej liczby trzycyfrowej podzielnej przez pięć jest trzy razy większa

$\mathrm{n}\mathrm{i}\dot{\mathrm{z}}$ cyfra jedności. Suma cyfr jedności i setek jest cztery razy większa $\mathrm{n}\mathrm{i}\dot{\mathrm{z}}$ cyfra dziesiątek.

Znajd $\acute{\mathrm{z}}$ tę liczbę.

ZADANIE 5.

$\mathrm{D}\mathrm{u}\dot{\mathrm{z}}\mathrm{y}$ trójkąt podzielono na mniejsze trójkąty tak, jak na rysunku. Liczby wewnątrz malych

trójkatów oznaczają ich obwody. Oblicz obwód $\mathrm{d}\mathrm{u}\dot{\mathrm{z}}$ ego trójkąta.
\begin{center}
\includegraphics[width=84.180mm,height=48.204mm]{./LigaMatematycznaMatuskiego_Klasa7i8_Zestaw3_2022_2023_page0_images/image001.eps}
\end{center}
11

9

14

10  12  20


\end{document}