\documentclass[a4paper,12pt]{article}
\usepackage{latexsym}
\usepackage{amsmath}
\usepackage{amssymb}
\usepackage{graphicx}
\usepackage{wrapfig}
\pagestyle{plain}
\usepackage{fancybox}
\usepackage{bm}

\begin{document}

flkademia

P omorskamStupsku

LIGA MATEMATYCZNA

im. Zdzisława Matuskiego

PÓLFINAL 27 kwietnia 2021

SZKOLA PODSTAWOWA

klasy VII- VIII

ZADANIE I.

$\mathrm{J}\mathrm{e}\dot{\mathrm{z}}$ eli do liczby dwucyfrowej $a$ dopiszemy na początku cyfrę 5, to otrzymamy 1iczbę o 234

mniejszą od liczby, którą otrzymamy po dopisaniu cyfry 5 na końcu 1iczby $a$. Wyznacz liczbę $a.$

ZADANIE 2.

Dany jest trapez ABCD o podstawach AB $\mathrm{i}$ CD, w którym $|AD|=|CD|=|BC|$. Przekątna

$AC$ jest prostopadla do boku $BC$. Oblicz miary kątów tego trapezu.

ZADANIE 3.

Trzy rózne jednocyfrowe liczby pierwsze zapisane w pewnej kolejności utworzyły liczbę trzycy-

frową, która jest podzielna przez $\mathrm{k}\mathrm{a}\dot{\mathrm{z}}$ dą z tych liczb pierwszych. Jaka to liczba?

ZADANIE 4.

Dane są liczby naturalne $a, b$ takie, $\dot{\mathrm{z}}\mathrm{e}3a+5b=ab$. Uzasadnij, $\dot{\mathrm{z}}\mathrm{e}a\mathrm{i}b$ są liczbami parzystymi.

ZADANIE 5.

Danyjest równoleglobok, któregojeden bok ma dlugość 18. Czy przekątne tego równo1eg1oboku

mogą mieć dfugości 16 $\mathrm{i}12$?


\end{document}