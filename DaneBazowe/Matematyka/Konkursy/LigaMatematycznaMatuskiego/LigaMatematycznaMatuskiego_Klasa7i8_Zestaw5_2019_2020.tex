\documentclass[a4paper,12pt]{article}
\usepackage{latexsym}
\usepackage{amsmath}
\usepackage{amssymb}
\usepackage{graphicx}
\usepackage{wrapfig}
\pagestyle{plain}
\usepackage{fancybox}
\usepackage{bm}

\begin{document}

LIGA MATEMATYCZNA

im. Zdzisława Matuskiego

PÓLFINAL 281utego 2020

SZKOLA PODSTAWOWA

klasy VII- VIII

ZADANIE I.

Dwa tysiące dwadzieścia liczb zapisano jedna za druga. Druga z nich jest równa 15, a ostatnia

46. Wiadomo, $\dot{\mathrm{z}}\mathrm{e}$ suma $\mathrm{k}\mathrm{a}\dot{\mathrm{z}}$ dych trzech kolejnych liczb jest równa 100. Wyznacz pozosta1e 2018

liczb.

ZADANIE 2.

Kawalek czworokątnego materialu o obwodzie 3 $\mathrm{m}$ przecięto wzdluz jednej przekątnej i powstafy

dwie chusty w ksztalcie trójkątów równoramiennych, pierwszy o obwodzie 1, 8 $\mathrm{m}$, a drugi 2, 8 $\mathrm{m}.$

Linia rozcięcia stanowi podstawę pierwszego trójkata, a dla drugiego trójkąta jest ramieniem.

Wyznacz wymiary obu chust.

ZADANIE 3.

Dane są liczby rzeczywiste $x, y$ spelniające równanie

$(x-y)^{2}+(x+y-4)^{2}=0.$

Oblicz iloczyn tych liczb.

ZADANIE 4.

Na stole $\mathrm{l}\mathrm{e}\dot{\mathrm{z}}\mathrm{y}$ 2020 kapsli. $\mathrm{W}$ jednym ruchu Bartek $\mathrm{m}\mathrm{o}\dot{\mathrm{z}}\mathrm{e}$ zdjąć dokladnie 3, 241ub 51 kaps1i.

Wolno mu wykonać wiele takich ruchów. Czy w pewnej chwili wszystkie kapsle zostaną zdjęte

ze stołu?

ZADANIE 5.

Trzy liczby naturalne dwucyfrowe ustawione w kolejności malejącej stanowia szyfr do sejfu.

Iloczyn pewnych dwóch spośród nich jest równy 888, a i1oczyn innych dwóch jest równy 999.

Znajd $\acute{\mathrm{z}}$ szyfr do sejfu.


\end{document}