\documentclass[a4paper,12pt]{article}
\usepackage{latexsym}
\usepackage{amsmath}
\usepackage{amssymb}
\usepackage{graphicx}
\usepackage{wrapfig}
\pagestyle{plain}
\usepackage{fancybox}
\usepackage{bm}

\begin{document}

flkademia

P omorskamStupsku

LIGA MATEMATYCZNA

im. Zdzisława Matuskiego

FINAL 18 maja 2021

SZKOLA PODSTAWOWA

klasy VII- VIII

ZADANIE I.

Cyfra dziesiątek pewnej liczby dwucyfrowej jest o 4 większa od cyfry jedności. $\mathrm{J}\mathrm{e}\dot{\mathrm{z}}$ eli między

cyfry tej liczby wstawimy 0, to otrzymamy 1iczbę o 630 większą od pierwotnej. Wyznacz

początkową liczbę.

ZADANIE 2.

Oblicz sumę cyfr liczby $4^{1009}\cdot 5^{2021}$

ZADANIE 3.

Dany jest trójkąt równoramienny $ABC$, gdzie $|AC| = |BC|$. Na boku $AB$ wybrano punkt $D$

taki, $\dot{\mathrm{z}}\mathrm{e}|AD|=|CD|$. Miara kąta$\triangleleft$DAC jest równa $27^{\mathrm{o}}$ Oblicz miarę kąta$\triangleleft$DCB.

ZADANIE 4.

Czy istniejq takie liczby naturalne $x, y, z, \dot{\mathrm{z}}\mathrm{e}x+y+z=444$ oraz $xyz=121275$? Odpowiedz'

uzasadnij.

ZADANIE 5.

$\mathrm{W}$ trapezie prostokątnym ABCD, gdzie $AB\Vert DC$, krótsza podstawa jest równa wysokości tra-

pezu, a krótsza przekątna ma dlugość równa dlugości dluzszego ramienia trapezu. Pole trapezu

jest równe 96 $\mathrm{c}\mathrm{m}^{2}$ Oblicz dlugości jego boków.
\end{document}
