\documentclass[a4paper,12pt]{article}
\usepackage{latexsym}
\usepackage{amsmath}
\usepackage{amssymb}
\usepackage{graphicx}
\usepackage{wrapfig}
\pagestyle{plain}
\usepackage{fancybox}
\usepackage{bm}

\begin{document}

flkademia

P omorskamStupsku

LIGA MATEMATYCZNA

im. Zdzisława Matuskiego

FINAL 21 kwietnia 2022

SZKOLA PODSTAWOWA

klasy VII- VIII

ZADANIE I.

Cyfrą dziesiatek liczby trzycyfrowej $A$ jest 8. $\mathrm{J}\mathrm{e}\dot{\mathrm{z}}$ eli tę cyfrę przestawimy na miejsce cyfry

jedności, to otrzymamy liczbę o 9 mniejszą od 1iczby $A. \mathrm{J}\mathrm{e}\dot{\mathrm{z}}$ eli przestawimy 8 na miejsce cyfry

setek, to otrzymamy liczbę o 630 wiekszą od 1iczby $A$. Wyznacz liczbę $A.$

ZADANIE 2.

Znajd $\acute{\mathrm{z}}$ najmniejszą liczbę naturalną zapisaną tylko za pomocą zer i jedynek, podzielną przez

45.

ZADANIE 3.

Wokól okrągfego stofu siedzi 13 osób. $K\mathrm{a}\dot{\mathrm{z}}$ da z nich ma na talerzu inną liczbę pierogów. Czy

$\mathrm{m}\mathrm{o}\dot{\mathrm{z}}$ na znalez$\acute{}$ć dwie sąsiednie osoby, które w sumie mają parzystą liczbę pierogów? Odpowied $\acute{\mathrm{z}}$

uzasadnij.

ZADANIE 4.

Adam pomnozyf sześć kolejnych liczb calkowitych dodatnich i uzyskaf iloczyn $A$. Bartek $\mathrm{t}\mathrm{e}\dot{\mathrm{z}}$

pomnozyl sześć kolejnych liczb calkowitych dodatnich, ale zacząl od liczby o l większej $\mathrm{n}\mathrm{i}\dot{\mathrm{z}}$

Adam. Otrzymaf liczbę $B$. Wyznacz najmniejszy czynnik iloczynu Bartka, $\mathrm{j}\mathrm{e}\dot{\mathrm{z}}$ eli $\displaystyle \frac{A}{B}=\frac{5}{6}.$

ZADANIE 5.

$\mathrm{W}$ kwadracie o boku o dlugości 4 cm umieszczono prostokqt tak, jak na rysunku. Ob1icz obwód

tego prostokąta.


\end{document}