\documentclass[a4paper,12pt]{article}
\usepackage{latexsym}
\usepackage{amsmath}
\usepackage{amssymb}
\usepackage{graphicx}
\usepackage{wrapfig}
\pagestyle{plain}
\usepackage{fancybox}
\usepackage{bm}

\begin{document}

LIGA MATEMATYCZNA

$\mathrm{P}\mathrm{A}\acute{\mathrm{Z}}$ DZIERNIK 2009

SZKOLA PONADGIMNAZJALNA

ZADANIE I.

Mamy $n+1$ róznych liczb naturalnych mniejszych od $2n$. Uzasadnij, $\dot{\mathrm{z}}\mathrm{e}\mathrm{m}\mathrm{o}\dot{\mathrm{z}}$ na wybrać z nich

trzy takie, aby jedna była równa sumie pozostałych.

ZADANIE 2.

Wykaz$\cdot, \dot{\mathrm{z}}\mathrm{e}$ okrąg wpisany w trójkqt prostokqtny jest styczny do przeciwprostokątnej w punkcie

dzielącym przeciwprostokątną na dwa odcinki, których iloczyn dlugości jest równy polu tego

trójkąta.

ZADANIE 3.

Znajd $\acute{\mathrm{z}}$ wartość $f(2)$, jeśli dla dowolnego $x$ róznego od zera spełniona jest równość

$f(x)+3f(\displaystyle \frac{1}{x})=x^{2}$

ZADANIE 4.

Wyznacz wszystkie liczby pierwsze $p, q$ takie, $\dot{\mathrm{z}}\mathrm{e}$ liczba $4pq+1$ jest kwadratem liczby naturalnej.

ZADANIE 5.

Od liczby naturalnej odjęto sumę jej cyfr. Następnie z otrzymaną liczbą postąpiono podobnie.

Po wykonaniu ll takich operacji po raz pierwszy otrzymano 0. Jaka była początkowa 1iczba?
\end{document}
