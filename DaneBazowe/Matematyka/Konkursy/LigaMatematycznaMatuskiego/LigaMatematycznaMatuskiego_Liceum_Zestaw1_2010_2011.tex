\documentclass[a4paper,12pt]{article}
\usepackage{latexsym}
\usepackage{amsmath}
\usepackage{amssymb}
\usepackage{graphicx}
\usepackage{wrapfig}
\pagestyle{plain}
\usepackage{fancybox}
\usepackage{bm}

\begin{document}

LIGA MATEMATYCZNA

$\mathrm{P}\mathrm{A}\acute{\mathrm{Z}}$ DZIERNIK 2010

SZKOLA PONADGIMNAZJALNA

ZADANIE I.

Dwa okręgi są styczne w punkcie S. Przez ten punkt poprowadzono proste KL i MN, od-

powiednio, przecinające pierwszy okrąg w punktach K i M, a drugi w L i N. Udowodnij,

$\dot{\mathrm{z}}\mathrm{e}KM\Vert LN.$

ZADANIE 2.

$\mathrm{W}$ olimpiadzie matematycznej startowało 100 uczniów, w fizycznej 50, w informatycznej 48.

$\mathrm{W}$ co najmniej dwóch olimpiadach startowało dwa razy mniej uczniów $\mathrm{n}\mathrm{i}\dot{\mathrm{z}}$ w co najmniej jednej.

$\mathrm{W}$ trzech olimpiadach bralo udział trzy razy mniej osób $\mathrm{n}\mathrm{i}\dot{\mathrm{z}}$ w co najmniej jednej. Ilu było

wszystkich uczestników tych olimpiad?

ZADANIE 3.

Ile jest funkcji liniowych $f(x) = ax+b$ takich, $\dot{\mathrm{z}}\mathrm{e} f(b) = 2009a$, gdzie $a \mathrm{i} b$ są liczbami

calkowitymi?

ZADANIE 4.

Pierwszym wyrazem ciągu jest l, drugim 3, a $\mathrm{k}\mathrm{a}\dot{\mathrm{z}}\mathrm{d}\mathrm{y}$ następny wyraz jest sumą dwóch poprzed-

nich. Jaka jest cyfra jedności tysięcznego wyrazu?

ZADANIE 5.

Uzasadnij, $\dot{\mathrm{z}}\mathrm{e}$ liczba $2^{2010}+3^{2012}$ jest złozona.


\end{document}