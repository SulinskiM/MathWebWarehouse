\documentclass[a4paper,12pt]{article}
\usepackage{latexsym}
\usepackage{amsmath}
\usepackage{amssymb}
\usepackage{graphicx}
\usepackage{wrapfig}
\pagestyle{plain}
\usepackage{fancybox}
\usepackage{bm}

\begin{document}

LIGA MATEMATYCZNA

$\mathrm{P}\mathrm{A}\acute{\mathrm{Z}}$ DZIERNIK 2011

SZKOLA PONADGIMNAZJALNA

ZADANIE I.

Znajd $\acute{\mathrm{z}}$ wszystkie funkcje $f:\mathbb{R}\rightarrow \mathbb{R}$ takie, $\dot{\mathrm{z}}\mathrm{e}$

$xf(x)-f(1-x)=2$

dla $\mathrm{k}\mathrm{a}\dot{\mathrm{z}}$ dej liczby rzeczywistej $x.$

ZADANIE 2.

$\mathrm{W}$ ostrokątnym trójkącie $ABC$ poprowadzono wysokości AD $\mathrm{i}$ {\it CE}. Znajd $\acute{\mathrm{z}}$ miarę kąta przy

wierzchofku $B, \mathrm{j}\mathrm{e}\dot{\mathrm{z}}$ eli wiadomo, $\dot{\mathrm{z}}\mathrm{e}|AC|=2|DE|.$

ZADANIE 3.

Znajd $\acute{\mathrm{z}}$ wszystkie liczby trzycyfrowe $n$ takie, $\dot{\mathrm{z}}\mathrm{e}$

$\displaystyle \frac{f(n)}{n}=1,$

gdzie f(n) oznacza sumę cyfr liczby n, iloczynu jej cyfr oraz trzech iloczynów róznych par cyfr

liczby n.

ZADANIE 4.

Dana jest liczba rzeczywista $b$, gdzie $b \not\in \{-1,0,1\}$. Definiujemy liczby $a_{n}$ w następujący

sposób:

$\left\{\begin{array}{l}
\alpha_{1}=\frac{b-1}{b+1}\\
a_{n+1}=\frac{a_{n}-1}{a_{n}+1},n\in \mathbb{N},n\geq 1.
\end{array}\right.$

Oblicz $b$ wiedząc, $\dot{\mathrm{z}}\mathrm{e}$ a2011$=2011.$

ZADANIE 5.

Oblicz

$\displaystyle \frac{1}{2\sqrt{1}+\sqrt{2}}+\frac{1}{3\sqrt{2}+2\sqrt{3}}+\ldots+\frac{1}{100\sqrt{99}+99\sqrt{100}}.$
\end{document}
