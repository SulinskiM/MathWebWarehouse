\documentclass[a4paper,12pt]{article}
\usepackage{latexsym}
\usepackage{amsmath}
\usepackage{amssymb}
\usepackage{graphicx}
\usepackage{wrapfig}
\pagestyle{plain}
\usepackage{fancybox}
\usepackage{bm}

\begin{document}

LIGA MATEMATYCZNA

im. Zdzisława Matuskiego

$\mathrm{P}\mathrm{A}\overline{\mathrm{Z}}$ DZIERNIK 2013

SZKOLA PONADGIMNAZJALNA

ZADANIE I.

Wyznacz wszystkie funkcje $f$: $\mathbb{R}\backslash \{0\}\rightarrow \mathbb{R}$ spelniające warunek $f(x)+2f(\displaystyle \frac{1}{x})=x$ dla $\mathrm{k}\mathrm{a}\dot{\mathrm{z}}$ dej

liczby rzeczywistej $x$ róznej od zera.

ZADANIE 2.

Na okręgu dane są punkty w kolejności $A, B, C, D$. Niech $M$ będzie środkiem luku $AB.$

Oznaczmy punkty przecięcia cięciw $MC\mathrm{i}MD$ z cięciwą AB, odpowiednio, $E$ oraz $K$. Wykaz,

$\dot{\mathrm{z}}\mathrm{e}$ na czworokącie EKDC $\mathrm{m}\mathrm{o}\dot{\mathrm{z}}$ na opisać okrąg.

ZADANIE 3.

Rozwia $\dot{\mathrm{z}}$ uklad równań

$\left\{\begin{array}{l}
x^{2}-(y-z)^{2}=1\\
y^{2}-(z-x)^{2}=4\\
z^{2}-(x-y)^{2}=9.
\end{array}\right.$

ZADANIE 4.

Uzasadnij, $\dot{\mathrm{z}}\mathrm{e}$ liczba

$3^{1}+3^{2}+3^{3}+\ldots+3^{998}+3^{999}$

jest podzielna przez 13.

ZADANIE 5.

Niech $a, b, c$ będą liczbami nieparzystymi. Wykaz, $\dot{\mathrm{z}}\mathrm{e}$ nie istnieje liczba cafkowita $x$ spefniająca

równość

$ax^{2}+bx+c=0.$
\end{document}
