\documentclass[a4paper,12pt]{article}
\usepackage{latexsym}
\usepackage{amsmath}
\usepackage{amssymb}
\usepackage{graphicx}
\usepackage{wrapfig}
\pagestyle{plain}
\usepackage{fancybox}
\usepackage{bm}

\begin{document}

LIGA MATEMATYCZNA

im. Zdzisława Matuskiego

$\mathrm{P}\mathrm{A}\overline{\mathrm{Z}}$ DZIERNIK 2014

SZKOLA PONADGIMNAZJALNA

ZADANIE I.

Na boku $AC$ trójkąta $ABC$ wybrano punkty $D\mathrm{i}E$ w taki sposób, $\dot{\mathrm{z}}\mathrm{e}|AB|=|AD|, |BE|=|EC|$

oraz punkt $E \mathrm{l}\mathrm{e}\dot{\mathrm{z}}\mathrm{y}$ pomiędzy punktami A $\mathrm{i} D$. Niech $F$ będzie środkiem luku $BC$ okręgu

opisanego na trójkącie $ABC$. Wykaz, $\dot{\mathrm{z}}\mathrm{e}$ punkty $B, E, D, F$ lezą na jednym okręgu.

ZADANIE 2.

Wewnatrz sześcianu o krawędzi 13 cm wybrano w dowo1ny sposób 2014 punktów. Czy w tym

sześcianie zawartyjest sześcian o krawędzi l cm, w którego wnętrzu nie ma $\dot{\mathrm{z}}$ adnego z wybranych

punktów?

ZADANIE 3.

$\mathrm{W}$ zbiorze liczb naturalnych rozwiąz równanie

$a+b+2014=ab.$

ZADANIE 4.

Wykaz$\cdot, \dot{\mathrm{z}}\mathrm{e}\mathrm{j}\mathrm{e}\dot{\mathrm{z}}$ eli $n\mathrm{i}6$ sa liczbami względnie pierwszymi, to $n^{2}-1$ dzieli się przez 24.

ZADANIE 5.

Rozwiąz uklad równań

( ---{\it xxyxyx}$+++${\it zyzzyz}$==$ -4--81517300.
\end{document}
