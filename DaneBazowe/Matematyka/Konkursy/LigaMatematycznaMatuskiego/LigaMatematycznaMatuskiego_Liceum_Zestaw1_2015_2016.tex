\documentclass[a4paper,12pt]{article}
\usepackage{latexsym}
\usepackage{amsmath}
\usepackage{amssymb}
\usepackage{graphicx}
\usepackage{wrapfig}
\pagestyle{plain}
\usepackage{fancybox}
\usepackage{bm}

\begin{document}

LIGA MATEMATYCZNA

im. Zdzisława Matuskiego

$\mathrm{P}\mathrm{A}\dot{\mathrm{Z}}$ DZIERNIK 2015

SZKOLA PONADGIMNAZJALNA

ZADANIE I.

Na bokach $BC\mathrm{i}$ CD kwadratu ABCD wybrano takie punkty $E\mathrm{i}F, \dot{\mathrm{z}}\mathrm{e}$ miara kąta $EAF$ jest

równa $45^{\mathrm{o}}$ Odcinki $AE$ oraz $AF$ przecinają przekątną $BD$ kwadratu odpowiednio w punktach

$G\mathrm{i}H$. Wykaz, $\dot{\mathrm{z}}\mathrm{e}$ pole trójkąta $AGH$ jest równe polu czworokata GEFH.

ZADANIE 2.

Rozwiąz uklad równań

$\left\{\begin{array}{l}
2y+3z=2yz\\
5z+2x=4xz\\
3x+5y=8xy.
\end{array}\right.$

ZADANIE 3.

Znajd $\acute{\mathrm{z}}$ wszystkie funkcje $f$: $\mathbb{R}\rightarrow \mathbb{R}$ spefniające warunek

$f(x)f(y)-xy=f(x)+f(y)-1$

dla dowolnych liczb rzeczywistych $x, y.$

ZADANIE 4.

Liczba $A$ ma 2015 cyfr i jest podzie1na przez 9. Liczba $B$ jest sumą cyfr liczby $A.$

jest sumą cyfr liczby $B$. Wyznacz sumę cyfr liczby $C.$

Liczba C

ZADANIE 5.

Piła ma długość 60 cm i zęby będące trójkątami równoramiennymi (niekoniecznie jednako-

wymi). Wysokość $\mathrm{k}\mathrm{a}\dot{\mathrm{z}}$ dego z zębów jest równa $\displaystyle \frac{2}{3}$ jego podstawy. Po zębach pify wędruje pająk.

Jaką drogę przebędzie, pokonując wszystkie zęby tej pily?


\end{document}