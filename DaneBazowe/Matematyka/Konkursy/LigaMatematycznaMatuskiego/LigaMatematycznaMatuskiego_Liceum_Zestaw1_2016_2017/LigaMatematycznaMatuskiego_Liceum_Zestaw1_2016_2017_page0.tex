\documentclass[a4paper,12pt]{article}
\usepackage{latexsym}
\usepackage{amsmath}
\usepackage{amssymb}
\usepackage{graphicx}
\usepackage{wrapfig}
\pagestyle{plain}
\usepackage{fancybox}
\usepackage{bm}

\begin{document}

LIGA MATEMATYCZNA

im. Zdzisława Matuskiego

$\mathrm{P}\mathrm{A}\dot{\mathrm{Z}}$ DZIERNIK 2016

SZKOLA PONADGIMNAZJALNA

ZADANIE I.

Punkt $P\mathrm{l}\mathrm{e}\dot{\mathrm{z}}\mathrm{y}$ na zewnątrz równoległoboku ABCD, przy czym $\triangleleft PAB=\triangleleft PCB$. Udowodnij,

$\dot{\mathrm{z}}\mathrm{e}\triangleleft APB=\triangleleft CPD.$

ZADANIE 2.

Liczby dodatnie $\alpha, b$ spelniaja warunek

$\displaystyle \frac{a+b}{2}=\sqrt{ab+3}.$

Wykaz, $\dot{\mathrm{z}}\mathrm{e}$ co najmniej jedna z liczb $a, b$ jest niewymierna.

ZADANIE 3.

Wyznacz wszystkie liczby naturalne $n$, dla których $n^{4}+33$ jest kwadratem liczby naturalnej.

ZADANIE 4.

Liczby całkowite $a\mathrm{i}b$ są tak dobrane, $\dot{\mathrm{z}}\mathrm{e}a^{2}+119ab+b^{2}$ jest podzielna przez ll.

$a^{3}-b^{3}\mathrm{t}\mathrm{e}\dot{\mathrm{z}}$ dzieli się przez ll.

Wykaz$\cdot, \dot{\mathrm{z}}\mathrm{e}$

ZADANIE 5.

Rozwia $\dot{\mathrm{z}}$ ukfad równań

\{{\it yx  z}222 $+++$222654$==$999{\it xzy} $+++$ -{\it x}--{\it yx}$+++$222{\it yzz}.
\end{document}
