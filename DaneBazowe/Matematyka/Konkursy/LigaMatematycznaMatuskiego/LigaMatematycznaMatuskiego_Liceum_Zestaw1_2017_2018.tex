\documentclass[a4paper,12pt]{article}
\usepackage{latexsym}
\usepackage{amsmath}
\usepackage{amssymb}
\usepackage{graphicx}
\usepackage{wrapfig}
\pagestyle{plain}
\usepackage{fancybox}
\usepackage{bm}

\begin{document}

LIGA MATEMATYCZNA

im. Zdzisława Matuskiego

$\mathrm{P}\mathrm{A}\dot{\mathrm{Z}}$ DZIERNIK 2017

SZKOLA PONADGIMNAZJALNA

ZADANIE I.

Oblicz sumę

$[\sqrt{1}]+[\sqrt{2}]+[\sqrt{3}]+[\sqrt{4}]+\ldots+[\sqrt{n^{2}-1}],$

gdzie $n$ jest dowolna liczbą naturalną większą od l, a symbol $[x]$ oznacza największą liczbę

calkowitą nie przekraczającą liczby $x.$

ZADANIE 2.

Rozwia $\dot{\mathrm{z}}$ równanie

$20a^{2}+10b^{2}=2010$

w zbiorze liczb naturalnych.

ZADANIE 3.

Wykaz$\cdot, \dot{\mathrm{z}}\mathrm{e}$ liczba $201^{8}+3\cdot 201^{4}-4$ jest podzielna przez 4000.

ZADANIE 4.

Wykaz, $\dot{\mathrm{z}}\mathrm{e}\mathrm{k}\mathrm{a}\dot{\mathrm{z}}$ da liczba naturalna większa od 10 jest sumą trzech 1iczb: dwóch róznych 1iczb

pierwszych i jednej zlozonej.

ZADANIE 5.

Dany jest trapez ABCD o podstawach a $\mathrm{i} b$. Odcinek $EF$ o dlugości $x$ jest równolegly do

podstaw trapezu i podzielil go na dwa trapezy o równych polach. Wyznacz $x.$
\begin{center}
\includegraphics[width=72.384mm,height=34.284mm]{./LigaMatematycznaMatuskiego_Liceum_Zestaw1_2017_2018_page0_images/image001.eps}
\end{center}
{\it D  b  c}

{\it E  x  F}

{\it A  a  B}


\end{document}