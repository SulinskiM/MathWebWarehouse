\documentclass[a4paper,12pt]{article}
\usepackage{latexsym}
\usepackage{amsmath}
\usepackage{amssymb}
\usepackage{graphicx}
\usepackage{wrapfig}
\pagestyle{plain}
\usepackage{fancybox}
\usepackage{bm}

\begin{document}

LIGA MATEMATYCZNA

im. Zdzisława Matuskiego

$\mathrm{P}\mathrm{A}\dot{\mathrm{Z}}$ DZIERNIK 2018

SZKOLA PONADPODSTAWOWA

ZADANIE I.

Dany jest odcinek $AB$ o dlugości 4. Punkty $A\mathrm{i}B$ sa środkami okręgów o promieniu 4. Znajd $\acute{\mathrm{z}}$

promień okręgu stycznego do prostej $AB$, stycznego zewnętrznie do okręgu o środku $A$ oraz

stycznego wewnętrznie do okręgu o środku $B.$

ZADANIE 2.

Czy istnieje taka liczba pierwsza $p, \dot{\mathrm{z}}\mathrm{e}p+16$ jest kwadratem liczby pierwszej?

uzasadnij.

Odpowiedz'

ZADANIE 3.

Funkcja rzeczywista $f$: $\mathbb{R} \rightarrow \mathbb{R}$ spelnia równanie $f(x)+xf(1-x) = x$ dla $\mathrm{k}\mathrm{a}\dot{\mathrm{z}}$ dej liczby

rzeczywistej $x$. Wyznacz $f(-2).$

ZADANIE 4.

Wysokości pewnego trójkata mają dlugości 156, 65, 60. Ob1icz po1e tego trójkąta.

ZADANIE 5.

Wyznacz liczbę czwórek $(a,b,c,d)$ liczb calkowitych dodatnich spelniających warunek

$ab+bc+cd+da=2018+a+b+c+d.$
\end{document}
