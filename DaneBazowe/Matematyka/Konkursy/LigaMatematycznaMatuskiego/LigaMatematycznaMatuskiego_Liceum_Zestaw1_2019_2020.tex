\documentclass[a4paper,12pt]{article}
\usepackage{latexsym}
\usepackage{amsmath}
\usepackage{amssymb}
\usepackage{graphicx}
\usepackage{wrapfig}
\pagestyle{plain}
\usepackage{fancybox}
\usepackage{bm}

\begin{document}

LIGA MATEMATYCZNA

im. Zdzisława Matuskiego

$\mathrm{P}\mathrm{A}\dot{\mathrm{Z}}$ DZIERNIK 2019

SZKOLA PONADPODSTAWOWA

ZADANIE I.

Która z liczb jest większa $7^{31}$ czy $19^{21}$?

ZADANIE 2.

Niech $n$ będzie dowolną liczbą calkowitą dodatnia. Wewnatrz prostokąta o bokach o dlugości l

$\mathrm{i}2$ znajduje się $8n^{2}+1$ punktów. Wykaz, $\dot{\mathrm{z}}\mathrm{e}$ istnieje kolo o promieniu $\displaystyle \frac{1}{n}$ zawierajace co najmniej

trzy spośród danych punktów.

ZADANIE 3.

Przez punkt $W$ lezący wewnątrz trójkąta $ABC$ poprowadzono trzy proste równolegfe do boków

trójkąta. Proste te podzielify trójkąt na sześć części, z których trzy są trójkatami o polach l,

4 $\mathrm{i}9$. Wyznacz pole trójkąta $ABC.$

ZADANIE 4.

Ile jest liczb trzycyfrowych $\overline{xyz}$ podzielnych przez 3 i takich, $\dot{\mathrm{z}}\mathrm{e}(\overline{xy})^{2}+(\overline{yz})^{2}=(\overline{yx})^{2}+(\overline{zy})^{2}$?

Symbol $\overline{xyz}$ oznacza liczbę trzycyfrową zapisaną w dziesiętnym systemie pozycyjnym.

ZADANIE 5.

$\mathrm{W}$ zbiorze liczb rzeczywistych rozwiąz uklad równań

$\left\{\begin{array}{l}
x-y^{2}+2y=2\\
y-z^{2}+2z=2\\
z-x^{2}+2x=2.
\end{array}\right.$


\end{document}