\documentclass[a4paper,12pt]{article}
\usepackage{latexsym}
\usepackage{amsmath}
\usepackage{amssymb}
\usepackage{graphicx}
\usepackage{wrapfig}
\pagestyle{plain}
\usepackage{fancybox}
\usepackage{bm}

\begin{document}

LIGA MATEMATYCZNA

im. Zdzisława Matuskiego

$\mathrm{P}\mathrm{A}\dot{\mathrm{Z}}$ DZIERNIK 2020

SZKOLA PONADPODSTAWOWA

ZADANIE I.

Czy istnieją liczby naturalne $a, b, c, d$ takie, $\dot{\mathrm{z}}\mathrm{e}a+b+c+d=478$ {\it oraz abcd}$=132706$?

ZADANIE 2.

$\mathrm{W}$ kwadracie o boku o dfugości l danych jest $2n+1$ punktów, z których $\dot{\mathrm{z}}$ adne trzy nie są

wspólliniowe. Udowodnij, $\dot{\mathrm{z}}\mathrm{e}$ trzy spośród nich są wierzchofkami trójkata o polu nie większym

$\displaystyle \mathrm{n}\mathrm{i}\dot{\mathrm{z}}\frac{1}{2n}.$

ZADANIE 3.

Dwa trójkąty równoboczne mają boki równolegle i wspólne ortocentrum. Pole jednego z nich

jest dwa razy większe $\mathrm{n}\mathrm{i}\dot{\mathrm{z}}$ pole drugiego, a bok mniejszego trójkąta ma dlugość l. Oblicz

odleglośč między równoleglymi bokami.

ZADANIE 4.

$\mathrm{W}$ zbiorze liczb rzeczywistych rozwiąz układ równań, gdy $n>3,$

$\left\{\begin{array}{l}
x_{1}+x_{2}=x_{3}\\
x_{2}+x_{3}=x_{4}\\
x_{3}+x_{4}=x_{5}\\
x_{n-2}+x_{n-1}=x_{n}\\
x_{n-1}+x_{n}=x_{1}\\
x_{n}+x_{1}=x_{2}.
\end{array}\right.$

ZADANIE 5.

Niech $f:\mathbb{R}\rightarrow \mathbb{R}$ będzie funkcją spelniającą warunki

$\bullet f(0)=2020$;

$\bullet f(x+2)=\displaystyle \frac{f(x)}{5f(x)-1}.$

Oblicz $f$ (2020).
\end{document}
