\documentclass[a4paper,12pt]{article}
\usepackage{latexsym}
\usepackage{amsmath}
\usepackage{amssymb}
\usepackage{graphicx}
\usepackage{wrapfig}
\pagestyle{plain}
\usepackage{fancybox}
\usepackage{bm}

\begin{document}

LIGA MATEMATYCZNA

im. Zdzislawa Matuskiego

$\mathrm{P}\mathrm{A}\dot{\mathrm{Z}}$ DZIERNIK 2021

SZKOLA PONADPODSTAWOWA

ZADANIE I.

Na stole w koszyku $\mathrm{l}\mathrm{e}\dot{\mathrm{z}}\mathrm{y}$ sto kapsli. Adam i Bartek zabieraja na zmianę po kilka kapsli. Wjednym

ruchu $\mathrm{m}\mathrm{o}\dot{\mathrm{z}}$ na zabraćjeden, dwa lub trzy kapsle. Wygrywa ten, kto $\mathrm{w}\mathrm{e}\acute{\mathrm{z}}\mathrm{m}\mathrm{i}\mathrm{e}$ ostatni kapsel. Adam

rozpocząl grę biorąc jeden kapsel. Ile kapsli powinien teraz wziąč Bartek, aby być pewnym

wygranej?

ZADANIE 2.

Pole prostokąta jest trzy razy większe od jego obwodu, a dlugości boków są liczbami natural-

nymi. Wyznacz dlugości boków prostokąta.

ZADANIE 3.

$\mathrm{W}$ trójkąt równoboczny $ABC$ wpisano okrąg. Dlugość luku laczącego dwa punkty styczności

tego okręgu z bokami trójkąta jest równa l. Oblicz obwód trójkąta.

ZADANIE 4.

Wykaz, $\dot{\mathrm{z}}\mathrm{e}\mathrm{j}\mathrm{e}\dot{\mathrm{z}}$ eli wysokości $h_{1}, h_{2}, h_{3}$ trójkąta spelniają warunek

$(h_{1}h_{3})^{2}+(h_{2}h_{3})^{2}=(h_{1}h_{2})^{2},$

to trójkąt jest prostokątny.

ZADANIE 5.

$\mathrm{W}$ zbiorze liczb rzeczywistych rozwiąz uklad równań

$\left\{\begin{array}{l}
2x+2y+z=6\\
8xy-z^{2}=36.
\end{array}\right.$


\end{document}