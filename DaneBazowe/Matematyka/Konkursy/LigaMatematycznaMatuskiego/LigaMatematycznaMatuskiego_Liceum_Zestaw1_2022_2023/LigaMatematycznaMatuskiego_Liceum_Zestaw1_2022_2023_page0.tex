\documentclass[a4paper,12pt]{article}
\usepackage{latexsym}
\usepackage{amsmath}
\usepackage{amssymb}
\usepackage{graphicx}
\usepackage{wrapfig}
\pagestyle{plain}
\usepackage{fancybox}
\usepackage{bm}

\begin{document}

LIGA MATEMATYCZNA

im. Zdzislawa Matuskiego

$\mathrm{P}\mathrm{A}\dot{\mathrm{Z}}$ DZIERNIK 2022

SZKOLA PONADPODSTAWOWA

ZADANIE I.

Czy istnieją liczby całkowite $x, y, z$ takie, $\dot{\mathrm{z}}\mathrm{e}(3x-5y)(7y-3z)(3z-x)=20222023$?

ZADANIE 2.

Pola szachownicy $9\times 9$ pokolorowano w tradycyjny sposób, ale jej narozne pola są biafe. Ruch

polega na wybraniu dwóch sąsiednich pól i przemalowaniu ich na przeciwny kolor (to znaczy:

$\mathrm{j}\mathrm{e}\dot{\mathrm{z}}$ eli wybrane polejest biale, to zmieniamy jego kolor na czarny, a gdy polejest czarne, to zmie-

niamy jego kolor na biały). Czy $\mathrm{m}\mathrm{o}\dot{\mathrm{z}}$ na dobierać ruchy tak, aby w pewnym momencie wszystkie

pola byly czarne?

ZADANIE 3.

Wierzcholki $B, C$ trójkata $ABC$ polączono odcinkami z przeciwleglymi bokami otrzymując

male trójkąty o polach 3, 3 $\mathrm{i}1$ (jak na rysunku). Oblicz pole trójkąta $ABC.$
\begin{center}
\includegraphics[width=64.764mm,height=56.592mm]{./LigaMatematycznaMatuskiego_Liceum_Zestaw1_2022_2023_page0_images/image001.eps}
\end{center}
3 1

3

{\it C}

ZADANIE 4.

Przedstaw liczbę 2023 jako róznicę kwadratów dwóch 1iczb natura1nych.

ZADANIE 5.

W zbiorze liczb rzeczywistych rozwiąz uklad równań

({\it x}((({\it xxx}1111({\it x}$+++$1 {\it xxx}$+$222{\it x})$++$(2{\it xx}$+${\it x}233$+${\it x})$+$(3{\it xx}$+${\it x}334$++${\it x}) {\it x}4{\it xx}4)44$=$)) $==$11. 11
\end{document}
