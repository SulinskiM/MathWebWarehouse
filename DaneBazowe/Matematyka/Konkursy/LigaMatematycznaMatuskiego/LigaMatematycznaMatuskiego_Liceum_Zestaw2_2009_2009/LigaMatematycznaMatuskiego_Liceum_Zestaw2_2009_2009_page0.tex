\documentclass[a4paper,12pt]{article}
\usepackage{latexsym}
\usepackage{amsmath}
\usepackage{amssymb}
\usepackage{graphicx}
\usepackage{wrapfig}
\pagestyle{plain}
\usepackage{fancybox}
\usepackage{bm}

\begin{document}

LIGA MATEMATYCZNA

LISTOPAD 2009

SZKOLA PONADGIMNAZJALNA

ZADANIE I.

Znajd $\acute{\mathrm{z}}$ największą liczbę naturalną $n$ taką, $\dot{\mathrm{z}}\mathrm{e}$ 1000! $(1000!=1\cdot 2\cdot 3\cdot\ldots\cdot 1000)$ jest podzielne

przez $2^{n}$

ZADANIE 2.

Udowodnij, $\dot{\mathrm{z}}\mathrm{e}\mathrm{j}\mathrm{e}\dot{\mathrm{z}}$ eli ramiona trapezu zawierają się w dwóch prostych prostopadlych, to suma

kwadratów dlugości podstaw równa się sumie kwadratów dlugości przekątnych.

ZADANIE 3.

Wykaz$\cdot, \dot{\mathrm{z}}\mathrm{e}$

$\displaystyle \frac{1}{2^{2}}+\frac{1}{3^{2}}+\frac{1}{4^{2}}+\ldots+\frac{1}{100^{2}}<\frac{99}{100}.$

ZADANIE 4.

Dla liczby naturalnej $n$ przez $p(n)$ oznaczmy iloczyn cyfr liczby $n$, np. $p(23) = 2$

$3 = 6,$

$p(100)=1\cdot 0\cdot 0=0$. Oblicz

$p(1)+p(2)+\ldots+p(100).$

ZADANIE 5.

W zbiorze liczb naturalnych trzycyfrowych znajdz' liczbę, której stosunek do sumy jej cyfr jest

najmniejszy.
\end{document}
