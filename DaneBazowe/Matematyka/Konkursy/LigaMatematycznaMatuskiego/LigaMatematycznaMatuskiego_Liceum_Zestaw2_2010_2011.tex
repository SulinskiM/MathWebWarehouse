\documentclass[a4paper,12pt]{article}
\usepackage{latexsym}
\usepackage{amsmath}
\usepackage{amssymb}
\usepackage{graphicx}
\usepackage{wrapfig}
\pagestyle{plain}
\usepackage{fancybox}
\usepackage{bm}

\begin{document}

LIGA MATEMATYCZNA

LISTOPAD 2010

SZKOLA PONADGIMNAZJALNA

ZADANIE I.

Rozwiąz równanie $x^{2}+y^{2}=x+y+2$ w zbiorze liczb całkowitych.

ZADANIE 2.

Wypisujemy kolejno liczby wedlug następującej reguły: dowolna liczba, oprócz pierwszej, jest

ostatnią cyfrą zwiększonego o l kwadratu poprzedniej liczby. Jaka liczba znajduje się na pierw-

szym miejscu, $\mathrm{j}\mathrm{e}\dot{\mathrm{z}}$ eli na 2010 pozycji znajduje się zero?

ZADANIE 3.

Na stole $\mathrm{l}\mathrm{e}\dot{\mathrm{z}}\mathrm{y}$ 2009 $\dot{\mathrm{z}}$ etonów czerwonych oraz 2009 $\dot{\mathrm{z}}$ etonów zielonych. Dwaj gracze na przemian

wykonują ruchy. Ruch polega na zdjęciu ze stołu dwóch $\dot{\mathrm{z}}$ etonów, przy czymjeśli były to $\dot{\mathrm{z}}$ etony

tego samego koloru, gracz kładzie na stóf $\dot{\mathrm{z}}$ eton czerwony, ajeśli $\dot{\mathrm{z}}$ etony byly rózne, kladzie $\dot{\mathrm{z}}$ eton

zielony. Zatem po $\mathrm{k}\mathrm{a}\dot{\mathrm{z}}$ dym ruchu liczba $\dot{\mathrm{z}}$ etonów na stole zmniejsza się o l. Gracz zaczynający

wygrywa, jeśli ostatni $\dot{\mathrm{z}}$ eton, jaki pozostanie na stole, będzie koloru czerwonego. Jaki powinien

wykonać pierwszy ruch, by wygrać?

ZADANIE 4.

Udowodnij, $\dot{\mathrm{z}}\mathrm{e}$ liczba $3^{2010}-5\cdot 15^{1005}+5^{2012}$ jest złozona.

ZADANIE 5.

$\mathrm{W}$ czworokącie wypuklym dwa przeciwlegle boki podzielono na trzy równe częšci. Wykaz,

$\dot{\mathrm{z}}\mathrm{e}$ pole czworokąta jest równe $3S, \mathrm{j}\mathrm{e}\dot{\mathrm{z}}$ eli pole zamalowanej części jest równe $S.$


\end{document}