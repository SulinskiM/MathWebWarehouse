\documentclass[a4paper,12pt]{article}
\usepackage{latexsym}
\usepackage{amsmath}
\usepackage{amssymb}
\usepackage{graphicx}
\usepackage{wrapfig}
\pagestyle{plain}
\usepackage{fancybox}
\usepackage{bm}

\begin{document}

LIGA MATEMATYCZNA

LISTOPAD 2011

SZKOLA PONADGIMNAZJALNA

ZADANIE I.

Przekątne trapezu ABCD, gdzie AB i CD są równolegle, przecinają się w punkcie E.

trójkąta ABE jest równe P, a pole trójkąta DEC jest równe S. Oblicz pole trapezu.

Pole

ZADANIE 2.

Oblicz $2010^{2}+2010^{2}\cdot 2011^{2}+2011^{2}-2010^{2}$

ZADANIE 3.

Znajd $\acute{\mathrm{z}}$ wszystkie róznowartościowe funkcje $f$: $\mathbb{R}\rightarrow \mathbb{R}$ spełniające równość

$f(f(x)+y)=f(x+y)+1$

dla dowolnych liczb rzeczywistych $x, y.$

ZADANIE 4.

Wykaz, $\dot{\mathrm{z}}\mathrm{e}$ liczba naturalna i jej piąta potęga mają tę samą cyfrę jedności.

ZADANIE 5.

$\mathrm{W}$ klasie jest 31 uczniów, wpisanych do dziennika pod numerami od 1 do 31. Przed 6 grudnia

przygotowali losy z numerami od l do 31, by usta1ić, kto komu będzie kupować prezent miko-

lajkowy. Udowodnij, $\dot{\mathrm{z}}\mathrm{e}$ iloczyn liczb będących sumami numeru ucznia w dzienniku i numeru

z karteczki przez niego wylosowanej jest liczbą parzystą.
\end{document}
