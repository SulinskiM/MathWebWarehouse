\documentclass[a4paper,12pt]{article}
\usepackage{latexsym}
\usepackage{amsmath}
\usepackage{amssymb}
\usepackage{graphicx}
\usepackage{wrapfig}
\pagestyle{plain}
\usepackage{fancybox}
\usepackage{bm}

\begin{document}

LIGA MATEMATYCZNA

im. Zdzisława Matuskiego

LISTOPAD 2012

SZKOLA PONADGIMNAZJALNA

ZADANIE I.

Wykaz$\cdot, \dot{\mathrm{z}}\mathrm{e}$ trójkąt prostokątny o bokach będących liczbami calkowitymi ma obwód, który jest

liczbą parzystą.

ZADANIE 2.

Wyznacz wszystkie funkcje $f:\mathbb{R}\rightarrow \mathbb{R}$ spefniajace warunek

$f(x+y)-f(x-y)=f(x)f(y)$

dla $\mathrm{k}\mathrm{a}\dot{\mathrm{z}}$ dych liczb rzeczywistych $x, y.$

ZADANIE 3.

$\mathrm{W}$ prostokącie o bokach 10 $\mathrm{i}20$ wybrano 401 punktów. Wykaz, $\dot{\mathrm{z}}\mathrm{e}$ istnieje kwadrat o boku l,

do którego nalezą co najmniej trzy spośród danych punktów.

ZADANIE 4.

Kwadrat o polu 144 $\mathrm{c}\mathrm{m}^{2}$ ma wspólną przekątną z prostokątem. Część wspólna kwadratu i pro-

stokata ma pole 96 $\mathrm{c}\mathrm{m}^{2}$ Oblicz pole prostokąta.

ZADANIE 5.

Suma dzielników pewnej liczby naturalnej $n$, bez liczby l i bez dzielnika będącego liczba $n$, jest

równa 41. Znajd $\acute{\mathrm{z}}$ liczbę $n$ wiedząc, $\dot{\mathrm{z}}\mathrm{e}$ rozklada się na trzy czynniki pierwsze, ajednym z nich

jest liczba 5.
\end{document}
