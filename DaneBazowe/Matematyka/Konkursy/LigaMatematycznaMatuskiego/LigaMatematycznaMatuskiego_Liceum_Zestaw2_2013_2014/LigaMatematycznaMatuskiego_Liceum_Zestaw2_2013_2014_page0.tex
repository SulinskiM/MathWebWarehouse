\documentclass[a4paper,12pt]{article}
\usepackage{latexsym}
\usepackage{amsmath}
\usepackage{amssymb}
\usepackage{graphicx}
\usepackage{wrapfig}
\pagestyle{plain}
\usepackage{fancybox}
\usepackage{bm}

\begin{document}

LIGA MATEMATYCZNA

im. Zdzisława Matuskiego

LISTOPAD 2013

SZKOLA PONADGIMNAZJALNA

ZADANIE I.

Dany jest kwadrat ABCD o boku dlugości a. Punkt K jest środkiem boku AB, punkt L

jest środkiem boku CD. Prosta AL przecina odcinek DK w punkcie M oraz przekątną BD

w punkcie S. Oblicz pole trójkąta DMS.

ZADANIE 2.

Wykaz, $\dot{\mathrm{z}}\mathrm{e}$ liczby 5050505 nie $\mathrm{m}\mathrm{o}\dot{\mathrm{z}}$ na przedstawić w postaci sumy dwóch liczb pierwszych.

ZADANIE 3.

Rozwiąz uklad równań

$\left\{\begin{array}{l}
2x^{2}+y^{2}=2\\
xy+2x=-3.
\end{array}\right.$

ZADANIE 4.

Liczby $a_{1}, a_{2}, a_{3}, \ldots$, a2013 są róznymi e1ementami zbioru \{1, 2, 3, $\ldots$, 2013\}. Czy liczba

$(a_{1}-1)(a_{2}-2)(a_{3}-3)\ldots$ (a2013 - 2013)

jest parzysta, czy nieparzysta?

ZADANIE 5.

Funkcja $f$: $\mathbb{R}\rightarrow \mathbb{R}$ spelnia następujące warunki:

$\bullet f(x+y)=f(x)+f(y)$ ;

$\bullet f(1)=1.$

Oblicz $f(\displaystyle \frac{1}{4}).$
\end{document}
