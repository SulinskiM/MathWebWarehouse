\documentclass[a4paper,12pt]{article}
\usepackage{latexsym}
\usepackage{amsmath}
\usepackage{amssymb}
\usepackage{graphicx}
\usepackage{wrapfig}
\pagestyle{plain}
\usepackage{fancybox}
\usepackage{bm}

\begin{document}

LIGA MATEMATYCZNA

im. Zdzisława Matuskiego

LISTOPAD 2014

SZKOLA PONADGIMNAZJALNA

ZADANIE I.

$\mathrm{Z}$ punktu $A\mathrm{l}\mathrm{e}\dot{\mathrm{z}}$ ącego na zewnątrz okręgu o środku $O$ i promieniu $r$ poprowadzono sieczna, która

przecina dany okrąg w punktach $B\mathrm{i}C$ w taki sposób, $\dot{\mathrm{z}}\mathrm{e}$ okrąg zbudowany na odcinku $BC$

jako na średnicy okręgu jest styczny do prostej $AO$ w punkcie $D$. Wyznacz dlugość odcinka

$AD$, gdy $|AO|=a.$

ZADANIE 2.

Na okręgu umieszczono $101$liczb naturalnych. Wykaz, $\dot{\mathrm{z}}\mathrm{e}$ znajdziemy dwie sąsiadujące ze sobą

liczby, których suma jest liczbą parzystą.

ZADANIE 3.

Dwie niezerowe rózne liczby rzeczywiste $a, b$ spelniają warunek $\displaystyle \frac{a}{b}+a=\frac{b}{a}+b$. Oblicz $\displaystyle \frac{1}{a}+\frac{1}{b}.$

ZADANIE 4.

Wyznacz wszystkie liczby pierwsze $p, q$ takie, $\dot{\mathrm{z}}\mathrm{e}2pq=3(p+q).$

ZADANIE 5.

Rozwiąz ukfad równań

$\left\{\begin{array}{l}
3x+5y=8xy\\
2y+3z=2yz\\
5z+2x=4xz.
\end{array}\right.$
\end{document}
