\documentclass[a4paper,12pt]{article}
\usepackage{latexsym}
\usepackage{amsmath}
\usepackage{amssymb}
\usepackage{graphicx}
\usepackage{wrapfig}
\pagestyle{plain}
\usepackage{fancybox}
\usepackage{bm}

\begin{document}

LIGA MATEMATYCZNA

im. Zdzisława Matuskiego

LISTOPAD 2015

SZKOLA PONADGIMNAZJALNA

ZADANIE I.

Czworokąt wypukły ABCD jest wpisany w okrąg $0$. Dwusieczne kątów $\triangleleft BAD, \triangleleft CBA,$

$\triangleleft DCB, \triangleleft ADC$ przecinają okrąg $0$ odpowiednio w punktach $M, N, P\mathrm{i}Q$. Wykaz, $\dot{\mathrm{z}}\mathrm{e}$ punkty

$M, N, P, Q$ są wierzchofkami prostokąta.

ZADANIE 2.

$\mathrm{W}$ zbiorze liczb rzeczywistych rozwiąz uklad równań

$\left\{\begin{array}{l}
x^{2}y=150\\
x^{3}y^{2}=4500.
\end{array}\right.$

ZADANIE 3.

Wyznacz najmniejszą liczbę naturalną $n$ taką, $\dot{\mathrm{z}}\mathrm{e}$ liczby $n+3, n-100$ sq kwadratami liczb

naturalnych.

ZADANIE 4.

Funkcja liniowa $f$ określona dla wszystkich liczb rzeczywistych spelnia warunek

$f(2016)+f(1)=2.$

Oblicz wartość wyrazenia $f(0)+f(1)+f(2)+\ldots+f(2016)+f(2017).$

ZADANIE 5.

Zbiór $A$ zawiera wszystkie liczby siedmiocyfrowe o róznych cyfrach nalezących do zbioru

\{1, 2, 3, 4, 5, 6, 7\}.

Czy w zbiorze $A$ istnieje 77 takich 1iczb, $\dot{\mathrm{z}}\mathrm{e}$ suma 33 z nich jest równa sumie 44 pozostałych?
\end{document}
