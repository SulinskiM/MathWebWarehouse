\documentclass[a4paper,12pt]{article}
\usepackage{latexsym}
\usepackage{amsmath}
\usepackage{amssymb}
\usepackage{graphicx}
\usepackage{wrapfig}
\pagestyle{plain}
\usepackage{fancybox}
\usepackage{bm}

\begin{document}

LIGA MATEMATYCZNA

im. Zdzisława Matuskiego

LISTOPAD 2016

SZKOLA PONADGIMNAZJALNA

ZADANIE I.

Wewnatrz trójkąta równobocznego $ABC$ znajduje się punkt $O$. Prosta przechodząca przez

punkt $O$ i środek cięzkości $G$ tego trójkąta (punkt przecięcia się środkowych) przecina jego

boki lub ich przedluzenia odpowiednio w punktach $D, E\mathrm{i}F$. Wykaz, $\dot{\mathrm{z}}\mathrm{e}$

-$||${\it DDOG}$|| +$ -$||${\it EEOG}$|| +$ -$||${\it FFOG}$|| =$3.

ZADANIE 2.

Rozwiąz równanie

$x(x+1)+(x+1)(x+2)+(x+2)(x+3)+\ldots+(x+14)(x+15)=2016x+2017$

w zbiorze liczb calkowitych.

ZADANIE 3.

Czy wierzchofki ośmiokąta foremnego $\mathrm{m}\mathrm{o}\dot{\mathrm{z}}$ na tak ponumerować liczbami 1, 2, 3, 4, 5, 6, 7, 8,

aby dla dowolnych trzech kolejnych wierzcholków suma ich numerów byla większa od 13?

ZADANIE 4.

$\mathrm{W}$ liczbie naturalnej, która byla co najmniej dwucyfrowa, wykreślono ostatnią cyfrę. Otrzy-

mana liczba jest $n$ razy mniejsza od poprzedniej. Wyznacz najmniejszą i największa $\mathrm{m}\mathrm{o}\dot{\mathrm{z}}$ liwą

wartość liczby $n.$

ZADANIE 5.

Rozwiąz ukfad równań

({\it tuyxz}22222 $+++++$22222$=====$22222{\it tzyxu}$+++++${\it utzyx}.


\end{document}