\documentclass[a4paper,12pt]{article}
\usepackage{latexsym}
\usepackage{amsmath}
\usepackage{amssymb}
\usepackage{graphicx}
\usepackage{wrapfig}
\pagestyle{plain}
\usepackage{fancybox}
\usepackage{bm}

\begin{document}

LIGA MATEMATYCZNA

im. Zdzisława Matuskiego

LISTOPAD 2017

SZKOLA PONADGIMNAZJALNA

ZADANIE I.

Trzy okręgi o promieniu r, okrąg o promieniu l i prosta są ułozone tak, jak na rysunku. Oblicz

dlugość promienia r.
\begin{center}
\includegraphics[width=116.784mm,height=42.972mm]{./LigaMatematycznaMatuskiego_Liceum_Zestaw2_2017_2018_page0_images/image001.eps}
\end{center}
{\it r}

{\it r  r}

1

ZADANIE 2.

Wykaz$\cdot, \dot{\mathrm{z}}\mathrm{e}$ liczba

$1^{3}+2^{3}+3^{3}+\ldots+2015^{3}+2016^{3}$

jest podzielna przez 2017.

ZADANIE 3.

Znajd $\acute{\mathrm{z}}$ wszystkie nieujemne liczby rzeczywiste $x$ spelniające równanie

$[x]=\sqrt{x\{x\}},$

gdzie $\{x\}=x-[x]$ oraz $[x]$ oznacza największą liczbę cafkowita nie przekraczajacą liczby $x.$

ZADANIE 4.

Danych jest 26 ko1ejnych 1iczb natura1nych. Okaza1o się, $\dot{\mathrm{z}}\mathrm{e}$ suma pewnych dziesięciu z nich

jest liczbą pierwszą. Wykaz, $\dot{\mathrm{z}}\mathrm{e}$ suma pozostalych 161iczb jest 1iczbą z1ozoną.

ZADANIE 5.

Wykaz, $\dot{\mathrm{z}}\mathrm{e}$

$(\displaystyle \frac{3}{2})^{2016}+(\frac{3}{2})^{2017}>(\frac{3}{2})^{2018}$
\end{document}
