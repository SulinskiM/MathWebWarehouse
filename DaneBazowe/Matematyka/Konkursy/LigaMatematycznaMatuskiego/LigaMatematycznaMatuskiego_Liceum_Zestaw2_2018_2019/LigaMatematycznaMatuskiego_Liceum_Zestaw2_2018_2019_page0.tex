\documentclass[a4paper,12pt]{article}
\usepackage{latexsym}
\usepackage{amsmath}
\usepackage{amssymb}
\usepackage{graphicx}
\usepackage{wrapfig}
\pagestyle{plain}
\usepackage{fancybox}
\usepackage{bm}

\begin{document}

LIGA MATEMATYCZNA

im. Zdzisława Matuskiego

LISTOPAD 2018

SZKOLA PONADPODSTAWOWA

ZADANIE I.

Wykaz$\cdot, \dot{\mathrm{z}}\mathrm{e}$ istnieje nieskończenie wiele liczb naturalnych, dla których iloczyn cyfr oraz suma

cyfr są liczbami pierwszymi.

ZADANIE 2.

Trójkąt $ABC$ podzielono dwiema prostymi, przechodzqcymi przez punkty $A\mathrm{i}B$ odpowiednio,

na cztery części. Pola trzech z nich są równe 3, 4, 6. Ob1icz po1e czwartej części.

ZADANIE 3.

Danych jest 301iczb rzeczywistych, których suma jest równa 300. Wykaz, $\dot{\mathrm{z}}\mathrm{e}$ wśród tych liczb

istnieje takich 51iczb, których suma jest równa co najmniej 50.

ZADANIE 4.

Funkcja $f$: $\mathbb{R}\rightarrow \mathbb{R}$ spelnia warunek

$2f(x)+3f(\displaystyle \frac{2010}{x})=5x$

dla $\mathrm{k}\mathrm{a}\dot{\mathrm{z}}$ dej liczby rzeczywistej dodatniej $x$. Wyznacz $f(6).$

ZADANIE 5.

Znajd $\acute{\mathrm{z}}$ wszystkie pary liczb calkowitych dodatnich $(x,y)$, które spefniają równanie

$4^{x}+260=y^{2}$

1
\end{document}
