\documentclass[a4paper,12pt]{article}
\usepackage{latexsym}
\usepackage{amsmath}
\usepackage{amssymb}
\usepackage{graphicx}
\usepackage{wrapfig}
\pagestyle{plain}
\usepackage{fancybox}
\usepackage{bm}

\begin{document}

LIGA MATEMATYCZNA

im. Zdzisława Matuskiego

LISTOPAD 2019

SZKOLA PONADPODSTAWOWA

ZADANIE I.

Znajd $\acute{\mathrm{z}}$ takie cyfry $x, y$, aby $(\overline{xy})^{2}+\overline{xy}=(\overline{yx})^{2}+\overline{yx}.$

ZADANIE 2.

Dany jest 2020-kąt foremny $A_{1}A_{2}A_{3}\ldots A_{2019}A_{2020}$. Punkt $P$ jest dowolnym punktem okręgu

o promieniu $R$ opisanego na wielokacie $A_{1}A_{2}A_{3}\ldots A_{2019}A_{2020}$. Oblicz

$|PA_{1}|^{2}+|PA_{2}|^{2}+\ldots+|PA_{2020}|^{2}$

ZADANIE 3.

Czy z odcinków o dlugościach $2018^{2018}, 2019^{2019}, 2020^{2020}\mathrm{m}\mathrm{o}\dot{\mathrm{z}}$ na zbudować trójkąt?

ZADANIE 4.

Zbiór $A$ sklada się z 2019 róznych 1iczb natura1nych. Wykaz, $\dot{\mathrm{z}}\mathrm{e}$ ze zbioru $A\mathrm{m}\mathrm{o}\dot{\mathrm{z}}$ na wybrać trzy

takie liczby $a, b, c, \dot{\mathrm{z}}\mathrm{e}$ iloczyn $a(b-c)$ jest podzielny przez 2019.

ZADANIE 5.

$\mathrm{W}$ zbiorze liczb rzeczywistych rozwia $\dot{\mathrm{z}}$ uklad równań

$\left\{\begin{array}{l}
xy+x+y=8\\
yz+y+z=8\\
xz+x+z=8.
\end{array}\right.$


\end{document}