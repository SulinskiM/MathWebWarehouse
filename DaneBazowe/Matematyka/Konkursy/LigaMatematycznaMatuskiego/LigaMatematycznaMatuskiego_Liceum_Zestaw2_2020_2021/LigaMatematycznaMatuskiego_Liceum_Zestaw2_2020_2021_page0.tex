\documentclass[a4paper,12pt]{article}
\usepackage{latexsym}
\usepackage{amsmath}
\usepackage{amssymb}
\usepackage{graphicx}
\usepackage{wrapfig}
\pagestyle{plain}
\usepackage{fancybox}
\usepackage{bm}

\begin{document}

LIGA MATEMATYCZNA

im. Zdzisława Matuskiego

LISTOPAD 2020

SZKOLA PONADPODSTAWOWA

ZADANIE I.

Dany jest trójkąt równoboczny $T$ o boku o dlugości $a$ i środku cięzkości $S$. Zakreślono okrąg

o środku $S$ i promieniu $\displaystyle \frac{a}{3}$ ograniczający kofo $K$. Oblicz pole figury $K-T.$

ZADANIE 2.

Na tablicy napisano kilka róznych liczb cafkowitych dodatnich (co najmniej cztery). Okazalo

się, $\dot{\mathrm{z}}\mathrm{e}$ suma $\mathrm{k}\mathrm{a}\dot{\mathrm{z}}$ dych trzech spośród nich jest liczbą pierwsza. Ile liczb napisano na tablicy?

ZADANIE 3.

Dany jest następujący ciąg liczb: pierwsza liczba to 2020, $\mathrm{k}\mathrm{a}\dot{\mathrm{z}}$ dą następną oblicza się wedlug

wzoru $\displaystyle \frac{1-a}{1+a}$, gdzie $a$ oznacza poprzednią liczbę. Znajdz' dwa tysiące dwudziesty pierwszy wyraz

tego ciągu.

ZADANIE 4.

$\mathrm{W}$ zbiorze liczb rzeczywistych rozwiąz układ równań

$\left\{\begin{array}{l}
x^{2}+x(y-4)=-2\\
y^{2}+y(x-4)=-2.
\end{array}\right.$

ZADANIE 5.

Wyznacz wszystkie funkcje $f:\mathbb{R}\rightarrow \mathbb{R}$ spefniające warunek

$f(x)\cdot f(y)=f(xy)+x^{2}+y^{2}$

dla dowolnych liczb rzeczywistych $x, y.$
\end{document}
