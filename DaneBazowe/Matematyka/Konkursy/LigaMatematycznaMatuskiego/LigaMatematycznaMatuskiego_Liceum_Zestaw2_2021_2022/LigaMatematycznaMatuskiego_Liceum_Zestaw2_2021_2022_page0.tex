\documentclass[a4paper,12pt]{article}
\usepackage{latexsym}
\usepackage{amsmath}
\usepackage{amssymb}
\usepackage{graphicx}
\usepackage{wrapfig}
\pagestyle{plain}
\usepackage{fancybox}
\usepackage{bm}

\begin{document}

LIGA MATEMATYCZNA

im. Zdzislawa Matuskiego

LISTOPAD 2021

SZKOLA PONADPODSTAWOWA

ZADANIE I.

$\mathrm{W}\mathrm{k}\mathrm{a}\dot{\mathrm{z}}$ dym wierzchołku dziesięciokąta napisano jedna z liczb: 1, 2, 3, 4, 5. $K\mathrm{a}\dot{\mathrm{z}}\mathrm{d}\mathrm{y}$ bok dziesię-

ciokąta ma dlugość równą sumie liczb napisanych na końcach tego boku. Uzasadnij, $\dot{\mathrm{z}}\mathrm{e}$ przy-

najmniej dwa boki mają równe dfugości.

ZADANIE 2.

Wyznacz dlugości boków trójkąta prostokątnego, $\mathrm{j}\mathrm{e}\dot{\mathrm{z}}$ eli są one liczbami naturalnymi, a liczby

oznaczajace pole i obwód spełniają warunek: pole jest równe podwojonemu obwodowi.

ZADANIE 3.

Punkty $M\mathrm{i}N$ są środkami boków $BC\mathrm{i}$ CD równolegfoboku ABCD. Niech $K\mathrm{i}L$ będą punk-

tami przecięcia przekątnej $BD$ odpowiednio przez proste AM $\mathrm{i}$ AN. Wykaz, $\dot{\mathrm{z}}\mathrm{e}$ punkty $K\mathrm{i}L$

dzielą przekątną $BD$ na trzy równe części. Jaką częścią pola równolegloboku ABCD jest pole

pięciokąta LKMCN?

ZADANIE 4.

Czy istnieją takie liczby calkowite $a, b, c, d, e, f, \dot{\mathrm{z}}\mathrm{e}a-b, b-c, c-d, d-e, e-f, f-a$

wypisane w pewnym porządku są kolejnymi liczbami calkowitymi? Odpowied $\acute{\mathrm{z}}$ uzasadnij.

ZADANIE 5.

$\mathrm{W}$ zbiorze liczb rzeczywistych rozwiąz uklad równań

$\left\{\begin{array}{l}
x^{2}+x+y=y^{3}\\
y^{2}+y+x=x^{3}
\end{array}\right.$
\end{document}
