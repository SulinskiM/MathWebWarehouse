\documentclass[a4paper,12pt]{article}
\usepackage{latexsym}
\usepackage{amsmath}
\usepackage{amssymb}
\usepackage{graphicx}
\usepackage{wrapfig}
\pagestyle{plain}
\usepackage{fancybox}
\usepackage{bm}

\begin{document}

LIGA MATEMATYCZNA

im. Zdzislawa Matuskiego

LISTOPAD 2022

SZKOLA PONADPODSTAWOWA

ZADANIE I.

Uzasadnij, $\dot{\mathrm{z}}\mathrm{e}$ czworokąt wypukły ABCD, w którym obwody trójkatów $ABC, BCD, CDA$

$\mathrm{i}DAB$ są równe, jest prostokatem.

ZADANIE 2.

Znajd $\acute{\mathrm{z}}$ wszystkie liczby pierwsze $p, q$ takie, $\dot{\mathrm{z}}\mathrm{e}p^{2}-6q^{2}=1.$

ZADANIE 3.

Na tablicy wypisano liczby 1, 2, 3, $\ldots$, 10. Ruch polega na wybraniu trzech liczb $a, b, c$ i zastą-

pieniu ich liczbami $2a+b, 2b+c, 2c+a$. Czy po wykonaniu pewnej liczby takich operacji na

tablicy otrzymamy dziesięć równych liczb?

ZADANIE 4.

Dodatnia liczba calkowita $a$ ma dwa dzielniki naturalne, liczba $a+1$ ma trzy dzielniki naturalne.

Ile dzielników naturalnych ma liczba $a+2$?

ZADANIE 5.

$\mathrm{W}$ zbiorze liczb rzeczywistych rozwia $\dot{\mathrm{z}}$ układ równań

(222{\it xyz}222 $+++$ --{\it yx-z}111222 $==$ {\it xyz}222 $+++$222.


\end{document}