\documentclass[a4paper,12pt]{article}
\usepackage{latexsym}
\usepackage{amsmath}
\usepackage{amssymb}
\usepackage{graphicx}
\usepackage{wrapfig}
\pagestyle{plain}
\usepackage{fancybox}
\usepackage{bm}

\begin{document}

LIGA MATEMATYCZNA

GRUD Z$\mathrm{I}\mathrm{E}\acute{\mathrm{N}}$ 2009

SZKOLA PONADGIMNAZJALNA

ZADANIE I.

Wykaz, $\dot{\mathrm{z}}\mathrm{e}$ liczba $3^{32}-1$ jest podzielna przez 8.

ZADANIE 2.

Dany jest uklad równań

$\left\{\begin{array}{l}
x+y+z=28\\
2x-y=32.
\end{array}\right.$

Okrešl, która z liczb jest większa $x$ czy $y, \mathrm{j}\mathrm{e}\dot{\mathrm{z}}$ eli $x>0, y>0\mathrm{i}z>0.$

ZADANIE 3.

Znajd $\acute{\mathrm{z}}$ sumę ufamków okresowych 0, $(ABC)+0, (BCA)+0$, ({\it CAB}).

ZADANIE 4.

Podstawą trójkąta równobocznego jest średnica koła o promieniu r. Oblicz stosunek pola części

trójkąta lezącej na zewnątrz koła do pola części trójkąta lezącej wewnątrz kola.

ZADANIE 5.

$\mathrm{W}$ kwadracie o boku l $\mathrm{m}$ obrano 51 punktów. Uzasadnij, $\dot{\mathrm{z}}\mathrm{e}$ przy dowolnym wyborze tych

punktów znajdą się trzy, które mieszczą się w kwadracie o boku 20 cm.
\end{document}
