\documentclass[a4paper,12pt]{article}
\usepackage{latexsym}
\usepackage{amsmath}
\usepackage{amssymb}
\usepackage{graphicx}
\usepackage{wrapfig}
\pagestyle{plain}
\usepackage{fancybox}
\usepackage{bm}

\begin{document}

LIGA MATEMATYCZNA

GRUD Z$\mathrm{I}\mathrm{E}\acute{\mathrm{N}}$ 2010

SZKOLA PONADGIMNAZJALNA

ZADANIE I.

Częšcią calkowitą liczby rzeczywistej $x$ nazywamy największą liczbę całkowitą nie większą $\mathrm{n}\mathrm{i}\dot{\mathrm{z}}$

$x$ i oznaczamy $[x]$. Rozwiąz układ równań

$\left\{\begin{array}{l}
[x]+y-2[z]=1\\
x+y-[z]=2\\
3[x]-4[y]+z=3.
\end{array}\right.$

ZADANIE 2.

Punkt $S$ lezy wewnątrz sześciokąta foremnego ABCDEF. Udowodnij, $\dot{\mathrm{z}}\mathrm{e}$ suma pól trójkątów

$ABS, CDS, EFS$ jest równa połowie pola szešciokąta ABCDEF.

ZADANIE 3.

Jan napisal na tablicy dwie liczby naturalne. Potem starł je i w ich miejsce wpisal iloczyn

zmniejszony o l oraz ich sumę. Nie zadowolifo go tojednak i powtórzył tę czynność. Znowu starl

wszystko i zapisal sumę otrzymanych liczb: 1309. Ob1icz sumę 1iczb zapisanych na poczqtku.

ZADANIE 4.

Wykaz, $\dot{\mathrm{z}}\mathrm{e}$ z grupy 2010 osób $\mathrm{m}\mathrm{o}\dot{\mathrm{z}}$ na wybrać 45 osób mających tak samo na imię 1ub 45 osób,

z których $\mathrm{k}\mathrm{a}\dot{\mathrm{z}}$ da nosi inne imię.

ZADANIE 5.

Wykaz$\cdot, \dot{\mathrm{z}}\mathrm{e}$ liczba $3^{2012}+15^{1006}+5^{2012}$ jest zlozona.


\end{document}