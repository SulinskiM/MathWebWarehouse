\documentclass[a4paper,12pt]{article}
\usepackage{latexsym}
\usepackage{amsmath}
\usepackage{amssymb}
\usepackage{graphicx}
\usepackage{wrapfig}
\pagestyle{plain}
\usepackage{fancybox}
\usepackage{bm}

\begin{document}

LIGA MATEMATYCZNA

GRUD Z$\mathrm{I}\mathrm{E}\acute{\mathrm{N}}$ 2011

SZKOLA PONADGIMNAZJALNA

ZADANIE I.

Łąka w ksztalcie kwadratu ma powierzchnię l hektara. Swoje norki wykopalo tam 2011 zajęcy

(kazdy ma jedną norkę). Po pewnym czasie pojawił się jeszcze jeden zając- samotnik, który

nie chce mieć sąsiada blizej $\mathrm{n}\mathrm{i}\dot{\mathrm{z}}1$ metr od swojej norki. Udowodnij, $\dot{\mathrm{z}}\mathrm{e}\mathrm{m}\mathrm{o}\dot{\mathrm{z}}\mathrm{e}$ tam zamieszkać.

ZADANIE 2.

Znajd $\acute{\mathrm{z}}$ wszystkie funkcje $f$: $\mathbb{R}\backslash \{0\}\rightarrow \mathbb{R}$ spelniające warunek

$f(x)+3f(\displaystyle \frac{1}{x})=\frac{2}{x}$

dla $\mathrm{k}\mathrm{a}\dot{\mathrm{z}}$ dej liczby rzeczywistej $x$ róznej od zera.

ZADANIE 3.

Kwadrat podzielono na mniejszy kwadrat i trzy prostokąty, jak na rysunku. Czy trzy spośród

tych czterech części mogą mieć taki sam obwód?

ZADANIE 4.

Wykaz, $\dot{\mathrm{z}}\mathrm{e}$ dla liczby całkowitej $k$ liczba $k^{6}-2k^{4}+k^{2}$ jest podzielna przez 36.

ZADANIE 5.

Wykaz, $\dot{\mathrm{z}}\mathrm{e}$ wśród pięciu dowolnie wybranych liczb naturalnych zawsze znajdą się trzy takie,

których suma jest podzielna przez 3.
\end{document}
