\documentclass[a4paper,12pt]{article}
\usepackage{latexsym}
\usepackage{amsmath}
\usepackage{amssymb}
\usepackage{graphicx}
\usepackage{wrapfig}
\pagestyle{plain}
\usepackage{fancybox}
\usepackage{bm}

\begin{document}

LIGA MATEMATYCZNA

im. Zdzisława Matuskiego

GRUD Z$\mathrm{I}\mathrm{E}\acute{\mathrm{N}}$ 2012

SZKOLA PONADGIMNAZJALNA

ZADANIE I.

$\mathrm{W}$ trójkącie $ABC$ o polu $S$ poprowadzono dwusieczną CE i środkową $BD$, które przecinają się

w punkcie $F$. Oblicz pole czworokąta AEFD, mając dane $BC=a$ oraz $AC=b.$

ZADANIE 2.

Znajd $\acute{\mathrm{z}}$ wszystkie trójki liczb calkowitych nieujemnych $a, b, c$ spelniające uklad równań

$\left\{\begin{array}{l}
a+bc=3b\\
b+ac=3c\\
c+ab=3a.
\end{array}\right.$

ZADANIE 3.

$\mathrm{W}$ kwadracie o boku l $\mathrm{m}$ wybrano w dowolny sposób 100 punktów. Wykaz$\cdot, \dot{\mathrm{z}}\mathrm{e}$ istnieje kwadrat

o boku 25 cm, który zawiera co najmniej 7 punktów spośród wybranych.

ZADANIE 4.

Na stu kartkach trzeba wpisać liczby od l do 200 umieszczajqc jedną 1iczbę na $\mathrm{k}\mathrm{a}\dot{\mathrm{z}}$ dej stronie.

Czy $\mathrm{m}\mathrm{o}\dot{\mathrm{z}}$ liwe jest takie wpisanie liczb, by na $\mathrm{k}\mathrm{a}\dot{\mathrm{z}}$ dej kartce suma liczb z obu jej stron byla

podzielna przez 6?

ZADANIE 5.

Wyznacz wszystkie funkcje $f$: $\mathbb{R}\rightarrow \mathbb{R}$ spełniające równanie $2f(x)-f(-x)=3x^{2}+x+3$ dla

$\mathrm{k}\mathrm{a}\dot{\mathrm{z}}$ dej liczby rzeczywistej $x.$


\end{document}