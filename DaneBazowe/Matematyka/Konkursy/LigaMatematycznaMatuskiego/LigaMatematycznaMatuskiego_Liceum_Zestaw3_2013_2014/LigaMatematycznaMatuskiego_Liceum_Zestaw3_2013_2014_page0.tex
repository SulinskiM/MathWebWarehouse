\documentclass[a4paper,12pt]{article}
\usepackage{latexsym}
\usepackage{amsmath}
\usepackage{amssymb}
\usepackage{graphicx}
\usepackage{wrapfig}
\pagestyle{plain}
\usepackage{fancybox}
\usepackage{bm}

\begin{document}

LIGA MATEMATYCZNA

im. Zdzisława Matuskiego

GRUD Z$\mathrm{I}\mathrm{E}\acute{\mathrm{N}}$ 2013

SZKOLA PONADGIMNAZJALNA

ZADANIE I.

Na plaszczy $\acute{\mathrm{z}}\mathrm{n}\mathrm{i}\mathrm{e}$ danych jest siedem prostych. Wykaz$\cdot, \dot{\mathrm{z}}\mathrm{e}$ kąt pomiędzy pewnymi dwiema pro-

stymi, spośród danych, jest mniejszy $\mathrm{n}\mathrm{i}\dot{\mathrm{z}}26^{\mathrm{o}}$

ZADANIE 2.

$\mathrm{W}$ kola wpisano liczby w taki sposób, $\dot{\mathrm{z}}\mathrm{e}$ suma liczb w $\mathrm{k}\mathrm{a}\dot{\mathrm{z}}$ dych trzech stycznych kolach jest

równa 2013. Ob1icz sumę 1iczb w ko1ach po1ozonych w wierzcho1kach trójkąta.

ZADANIE 3.

Znajd $\acute{\mathrm{z}}$ liczbę sześciocyfrową $\overline{abcdef}$ wiedząc, $\dot{\mathrm{z}}\mathrm{e}$ liczby trzycyfrowe $\overline{abc}, \overline{cde}$ są sześcianami,

a liczby $\overline{bcd}\mathrm{i}\overline{def}$ sa kwadratami pewnych liczb naturalnych.

ZADANIE 4.

Wyznacz wszystkie funkcje $f:\mathbb{R}\rightarrow \mathbb{R}$ spefniające równość

$f(x-|x|)+f(x+|x|)=x$

dla $\mathrm{k}\mathrm{a}\dot{\mathrm{z}}$ dej liczby rzeczywistej $x.$

ZADANIE 5.

Rozwia $\dot{\mathrm{Z}}$ równanie

-{\it a}1 $+$ -{\it a}2{\it b}$+$ -{\it a}3{\it bc} $=$1

w zbiorze liczb calkowitych dodatnich.
\end{document}
