\documentclass[a4paper,12pt]{article}
\usepackage{latexsym}
\usepackage{amsmath}
\usepackage{amssymb}
\usepackage{graphicx}
\usepackage{wrapfig}
\pagestyle{plain}
\usepackage{fancybox}
\usepackage{bm}

\begin{document}

LIGA MATEMATYCZNA

im. Zdzisława Matuskiego

GRUD Z$\mathrm{I}\mathrm{E}\acute{\mathrm{N}}$ 2014

SZKOLA PONADGIMNAZJALNA

ZADANIE I.

$\mathrm{W}$ trójkącie równoramiennym $ABC(|AC|=|BC|)$ na boku $AC$ obrano punkt $D$. Na trójkątach

$ABD\mathrm{i}DBC$ opisano okręgi $0_{1}$ oraz 02. Styczna do okręgu $0_{1}$ w punkcie $D$ przecina okrąg 02

w punkcie $M$. Wykaz$\cdot, \dot{\mathrm{z}}\mathrm{e}$ prosta $CM$ jest równoległa do prostej $AB.$

ZADANIE 2.

Znajd $\acute{\mathrm{z}}$ liczby naturalne $a, b$, których najmniejsza wspólna wielokrotność jest równa 630, a naj-

większy wspólny dzielnik 18, wiedząc, $\dot{\mathrm{z}}\mathrm{e}$ te liczby nie dzielq się przez siebie.

ZADANIE 3.

Oblicz sumę

$\sqrt{3-2\sqrt{2}}+\sqrt{5-2\sqrt{6}}+\sqrt{7-2\sqrt{12}}+\ldots+\sqrt{4029-2\sqrt{2014}}$2015.

ZADANIE 4.

Dodatnie liczby $a, b, c$ spelniają warunki $a+b+c=9$ oraz $\displaystyle \frac{1}{b+c}+\frac{1}{c+a}+\frac{1}{a+b}=\frac{10}{9}$. Oblicz

$\displaystyle \frac{a}{b+c}+\frac{b}{c+a}+\frac{c}{a+b}.$

ZADANIE 5.

$\mathrm{W}$ kwadracie o boku 2 wybrano w sposób dowo1ny 9 punktów. Wykaz, $\dot{\mathrm{z}}\mathrm{e}$ istnieje taka trójka

punktów wśród nich, $\dot{\mathrm{z}}\mathrm{e}$ pole figury, której wierzcholkami są te trzy punkty nie przekracza $\displaystyle \frac{1}{2}.$


\end{document}