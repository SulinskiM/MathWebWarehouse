\documentclass[a4paper,12pt]{article}
\usepackage{latexsym}
\usepackage{amsmath}
\usepackage{amssymb}
\usepackage{graphicx}
\usepackage{wrapfig}
\pagestyle{plain}
\usepackage{fancybox}
\usepackage{bm}

\begin{document}

LIGA MATEMATYCZNA

im. Zdzisława Matuskiego

GRUD Z$\mathrm{I}\mathrm{E}\acute{\mathrm{N}}$ 2015

SZKOLA PONADGIMNAZJALNA

ZADANIE I.

Na przyprostokątnych $BC\mathrm{i}CA$ trójkata prostokątnego $ABC$ zbudowano, po zewnętrznej stro-

nie, kwadraty BEFC oraz CGHA. Odcinek $CD$ jest wysokościa trójkąta $ABC$. Wykaz, $\dot{\mathrm{z}}\mathrm{e}$

proste $AE, BH$ oraz $CD$ przecinaja się w jednym punkcie.

ZADANIE 2.

$\mathrm{W}$ zbiorze liczb rzeczywistych rozwiąz uklad równań

$\left\{\begin{array}{l}
x^{2}+2y^{2}=2yz+100\\
z^{2}=2xy-100.
\end{array}\right.$

ZADANIE 3.

Rózne dodatnie liczby rzeczywiste a, b spelniają równość

$\displaystyle \frac{5a}{a+b}+\frac{5b}{a-b}=6.$

Wykaz, $\dot{\mathrm{z}}\mathrm{e}$ co najmniej jedna z nich jest niewymierna.

ZADANIE 4.

Czy istnieje taka dodatnia liczba calkowita $n$, aby zapis dziesiętny liczby $2^{n}$ zawiera110jedynek,

10 dwójek, l0 trójek, $\ldots$, 10 ósemek, l0 dziewiątek i pewną ilość zer?

ZADANIE 5.

Funkcja $f$: $\mathbb{R}\rightarrow \mathbb{R}$ spelnia warunek

2 $f(x)+f(1-x)=3x$

dla wszystkich liczb rzeczywistych $x$. Wyznacz $f(2015).$


\end{document}