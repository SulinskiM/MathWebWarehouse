\documentclass[a4paper,12pt]{article}
\usepackage{latexsym}
\usepackage{amsmath}
\usepackage{amssymb}
\usepackage{graphicx}
\usepackage{wrapfig}
\pagestyle{plain}
\usepackage{fancybox}
\usepackage{bm}

\begin{document}

LIGA MATEMATYCZNA

im. Zdzisława Matuskiego

GRUD Z$\mathrm{I}\mathrm{E}\acute{\mathrm{N}}$ 2016

SZKOLA PONADGIMNAZJALNA

ZADANIE I.

Na bokach $AC\mathrm{i}BC$ trójkąta $ABC$ zbudowano równoległoboki ACDE oraz BFGC tak, jak

na rysunku. Punkty $K\mathrm{i}L$ sq odpowiednio środkami odcinków $EG\mathrm{i}DF$. Oblicz $\displaystyle \frac{|KL|}{|AB|}.$
\begin{center}
\includegraphics[width=76.656mm,height=43.332mm]{./LigaMatematycznaMatuskiego_Liceum_Zestaw3_2016_2017_page0_images/image001.eps}
\end{center}
{\it G}

c  {\it F}

{\it E}

ZADANIE 2.

Znajd $\acute{\mathrm{z}}$ wszystkie pary $(x,y)$ liczb cafkowitych spelniające równanie

$x^{4}=y^{4}+1223334444.$

ZADANIE 3.

Na $\mathrm{k}\mathrm{a}\dot{\mathrm{z}}$ dej ścianie sześcianu napisano pewną dodatnią liczbę calkowitą. Następnie w $\mathrm{k}\mathrm{a}\dot{\mathrm{z}}$ dym

wierzchofku sześcianu umieszczono liczbę, którajest równa iloczynowi liczb znajdujących się na

ściankach, do których ten wierzcholek nalezy. Oblicz sumę liczb znajdujących się na wszystkich

ścianach wiedząc, $\dot{\mathrm{z}}\mathrm{e}$ suma liczb umieszczonych w wierzchołkach jest równa 70.

ZADANIE 4.

Suma pewnych dziewięciu liczb jest równa 90.

suma jest równa co najmniej 40.

Wykaz$\cdot, \dot{\mathrm{z}}\mathrm{e}$ wśród nich są takie cztery, których

ZADANIE 5.

Rozwia $\dot{\mathrm{z}}$ uklad równań

$\left\{\begin{array}{l}
x^{3}-y^{3}=26\\
x^{2}y-xy^{2}=6.
\end{array}\right.$
\end{document}
