\documentclass[a4paper,12pt]{article}
\usepackage{latexsym}
\usepackage{amsmath}
\usepackage{amssymb}
\usepackage{graphicx}
\usepackage{wrapfig}
\pagestyle{plain}
\usepackage{fancybox}
\usepackage{bm}

\begin{document}

LIGA MATEMATYCZNA

im. Zdzisława Matuskiego

GRUD Z$\mathrm{I}\mathrm{E}\acute{\mathrm{N}}$ 2017

SZKOLA PONADGIMNAZJALNA

ZADANIE I.

Trapez prostokątny o podstawach $a, b$ opisanyjest na okręgu o średnicy $d$. Wykaz$\cdot, \dot{\mathrm{z}}\mathrm{e}$ prawdziwa

jest nierównośč

$d\leq\sqrt{\frac{a^{2}+b^{2}}{2}}.$

ZADANIE 2.

$\mathrm{W}$ zbiorze liczb rzeczywistych rozwiąz równanie

$x^{2}-8[x]+7=0,$

gdzie $[x]$ oznacza największą liczbę całkowitą nie przekraczającą liczby $x.$

ZADANIE 3.

Czy istnieją liczby $x_{1}, x_{2}, x_{3}, \ldots, x_{1001}$ równe $(-1)$ lub l takie, $\dot{\mathrm{z}}\mathrm{e}$

$x_{1}x_{2}+x_{2}x_{3}+x_{3}x_{4}+\ldots+x_{1000}x_{1001}+x_{1001}x_{1}=499$?

ZADANIE 4.

$\mathrm{W}$ zbiorze liczb rzeczywistych rozwiąz ukfad równań

$\left\{\begin{array}{l}
x(y+z)=6-x^{2}\\
y(z+x)=12-y^{2}\\
z(x+y)=18-z^{2}
\end{array}\right.$

ZADANIE 5.

Mikolaj wybral trzy liczby rzeczywiste $a, b, c$ i określił dzialanie $\star$ wzorem

$x\star y=ax+by+cxy$

dla dowolnych liczb rzeczywistych $x, y$. Obliczyll $\star 2=3\mathrm{i}2\star 3=4$ oraz zauwazyf, $\dot{\mathrm{z}}\mathrm{e}$ istnieje

niezerowa liczba rzeczywista $t$ taka, $\dot{\mathrm{z}}\mathrm{e}x\star t=x$ dla $\mathrm{k}\mathrm{a}\dot{\mathrm{z}}$ dej liczby rzeczywistej $x$. Wyznacz $t.$
\end{document}
