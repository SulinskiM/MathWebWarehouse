\documentclass[a4paper,12pt]{article}
\usepackage{latexsym}
\usepackage{amsmath}
\usepackage{amssymb}
\usepackage{graphicx}
\usepackage{wrapfig}
\pagestyle{plain}
\usepackage{fancybox}
\usepackage{bm}

\begin{document}

LIGA MATEMATYCZNA

im. Zdzisława Matuskiego

GRUD Z$\mathrm{I}\mathrm{E}\acute{\mathrm{N}}$ 2018

SZKOLA PONADPODSTAWOWA

ZADANIE I.

Dany jest trapez ABCD o polu 15 i podstawach $AB$ oraz $CD$. Dwusieczna kąta $CBA$ jest

prostopadla do ramienia $AD$ i przecina je w takim punkcie $E, \dot{\mathrm{z}}\mathrm{e} \displaystyle \frac{|AE|}{|ED|}=2$. Oblicz pola figur

$ABE\mathrm{i}$ EBCD, na które zostal podzielony trapez.

ZADANIE 2.

Wykaz, $\dot{\mathrm{z}}\mathrm{e}$ liczba

$[\displaystyle \frac{n+4}{2}]+3n-2\cdot(-1)^{n}$

jest podzielna przez 7 d1a $\mathrm{k}\mathrm{a}\dot{\mathrm{z}}$ dej liczby naturalnej $n$, gdzie $[x]$ oznacza największą liczbę cal-

kowitą nie większą $\mathrm{n}\mathrm{i}\dot{\mathrm{z}}x.$

ZADANIE 3.

Wyznacz wszystkie liczby pierwsze, które są równocześnie sumami i róznicami dwóch liczb

pierwszych.

ZADANIE 4.

Czy istnieje liczba sześciocyfrowa podzielna przez ll o sumie cyfr równej ll, której dwie ostatnie

cyfry tworzą liczbę ll?

ZADANIE 5.

Znajd $\acute{\mathrm{z}}$ wszystkie trójki liczb rzeczywistych $(x,y,z)$ spelniające uklad równań

$\left\{\begin{array}{l}
x^{2}+y^{2}+z^{2}=33\\
x+3y+5z=34.
\end{array}\right.$
\end{document}
