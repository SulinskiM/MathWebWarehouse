\documentclass[a4paper,12pt]{article}
\usepackage{latexsym}
\usepackage{amsmath}
\usepackage{amssymb}
\usepackage{graphicx}
\usepackage{wrapfig}
\pagestyle{plain}
\usepackage{fancybox}
\usepackage{bm}

\begin{document}

LIGA MATEMATYCZNA

im. Zdzisława Matuskiego

GRUD Z$\mathrm{I}\mathrm{E}\acute{\mathrm{N}}$ 2019

SZKOLA PONADPODSTAWOWA

ZADANIE I.

Długości $x, y, z$ boków trójkąta sa liczbami naturalnymi oraz $z=xy$. Wykaz$\cdot, \dot{\mathrm{z}}\mathrm{e}$ ten trójkat

jest równoramienny.

ZADANIE 2.

Znajd $\acute{\mathrm{z}}$ takie liczby calkowite dodatnie $n, \dot{\mathrm{z}}\mathrm{e}5^{n}-2\mathrm{i}5^{n}+2$ są liczbami pierwszymi.

ZADANIE 3.

Dane są liczby calkowite $a_{1}, a_{2}, \ldots$, a2019. Liczby $b_{1}, b_{2}, \ldots$, b2019 to 1iczby $a_{1}, a_{2}, \ldots$, a2019, a1e

ustawione w innej, przypadkowej, kolejności. Wykaz, $\dot{\mathrm{z}}\mathrm{e}$ iloczyn

$(a_{1}-b_{1})(a_{2}-b_{2})\ldots$ (a2019 - $b_{2019}$)

jest liczbą parzystą.

ZADANIE 4.

Wykaz, $\dot{\mathrm{z}}\mathrm{e}$ dla $\mathrm{k}\mathrm{a}\dot{\mathrm{z}}$ dej liczby naturalnej $n$ liczba $n^{4}-n^{2}$ jest podzielna przez 12.

ZADANIE 5.

$\mathrm{W}$ zbiorze liczb rzeczywistych rozwiąz układ równań

$\left\{\begin{array}{l}
x^{2}=y+z+2\\
y^{2}=z+x+2\\
z^{2}=x+y+2.
\end{array}\right.$


\end{document}