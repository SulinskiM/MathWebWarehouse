\documentclass[a4paper,12pt]{article}
\usepackage{latexsym}
\usepackage{amsmath}
\usepackage{amssymb}
\usepackage{graphicx}
\usepackage{wrapfig}
\pagestyle{plain}
\usepackage{fancybox}
\usepackage{bm}

\begin{document}

LIGA MATEMATYCZNA

im. Zdzisława Matuskiego

GRUD Z$\mathrm{I}\mathrm{E}\acute{\mathrm{N}}$ 2020

SZKOLA PONADPODSTAWOWA

ZADANIE I.

$\mathrm{W}$ trójkącie prostokątnym na dluzszej przyprostokątnej jako na średnicy opisano pólokrąg

tak, $\dot{\mathrm{z}}\mathrm{e}$ przecina on przeciwprostokątną w punkcie $K$. Krótsza przyprostokątna ma dlugość $a.$

Stosunek dlugości cięciwy lączącej wierzcholek kata prostego z punktem $K$ do dlugości krótszej

przyprostokątnej jest równy $\displaystyle \frac{4}{5}$. Wyznacz dlugość pófokręgu.

ZADANIE 2.

Uzasadnij, $\dot{\mathrm{z}}\mathrm{e}$ wśród dwunastu róznych liczb naturalnych dwucyfrowych $\mathrm{m}\mathrm{o}\dot{\mathrm{z}}$ na znalez$\acute{}$ć dwie,

których róznica jest liczbą dwucyfrową o jednakowych cyfrach dziesiqtek i jedności.

ZADANIE 3.

$\mathrm{W}$ zbiorze liczb calkowitych rozwiąz równanie

$ x(x+1)(x+2)+(x+1)(x+2)(x+3)+(x+2)(x+3)(x+4)+\ldots$

. . . $+(x+98)(x+99)(x+100)=2019x+2020.$

ZADANIE 4.

Wyznacz wszystkie funkcje $f:\mathbb{R}\rightarrow \mathbb{R}$ spelniające warunek

$f(x)\cdot f(y)-f(xy)=x+y$

dla dowolnych liczb rzeczywistych $x, y.$

ZADANIE 5.

$\mathrm{W}$ zbiorze liczb rzeczywistych rozwiąz uklad równań

$\left\{\begin{array}{l}
(a+b)^{2}=4c\\
(b+c)^{2}=4a\\
(c+a)^{2}=4b.
\end{array}\right.$


\end{document}