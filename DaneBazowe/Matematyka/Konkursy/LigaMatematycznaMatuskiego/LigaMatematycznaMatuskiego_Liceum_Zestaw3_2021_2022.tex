\documentclass[a4paper,12pt]{article}
\usepackage{latexsym}
\usepackage{amsmath}
\usepackage{amssymb}
\usepackage{graphicx}
\usepackage{wrapfig}
\pagestyle{plain}
\usepackage{fancybox}
\usepackage{bm}

\begin{document}

LIGA MATEMATYCZNA

im. Zdzislawa Matuskiego

GRUD Z$\mathrm{I}\mathrm{E}\acute{\mathrm{N}}$ 2021

SZKOLA PONADPODSTAWOWA

ZADANIE I.

$\mathrm{W}$ zbiorze liczb całkowitych rozwiąz równanie $x^{2}+y^{2}+3=xy+2x+2y.$

ZADANIE 2.

Czy liczbę 123456789 $\mathrm{m}\mathrm{o}\dot{\mathrm{z}}$ na przedstawič w postaci sumy dwóch skladników, z których jeden

jest zapisany tylko cyframi parzystymi, a drugi nieparzystymi?

ZADANIE 3.

Wykaz, $\dot{\mathrm{z}}\mathrm{e}$ w dowolnym ciqgu siedmiu liczb całkowitych zawsze $\mathrm{m}\mathrm{o}\acute{\mathrm{z}}\mathrm{n}\mathrm{a}$ wskazać pewną liczbę

kolejnych wyrazów, których suma jest podzielna przez 7.

ZADANIE 4.

Niech $p\mathrm{i}q$ będą takimi dodatnimi liczbami rzeczywistymi, $\dot{\mathrm{z}}\mathrm{e}q>p$. Wykaz, $\dot{\mathrm{z}}\mathrm{e}\mathrm{j}\mathrm{e}\dot{\mathrm{z}}$ eli $p\mathrm{i}q$ są

dlugościami przekątnych rombu o kącie o mierze $\displaystyle \frac{\pi}{6}$, to $\displaystyle \frac{p}{q}=2-\sqrt{3}.$

ZADANIE 5.

Pewna liczba dwucyfrowa ma trzy dzielniki jednocyfrowe i trzy dzielniki dwucyfrowe. Suma

wszystkich dzielników jednocyfrowych jest równa 8. Ob1icz sumę wszystkich dzie1ników dwucy-

frowych tej liczby.


\end{document}