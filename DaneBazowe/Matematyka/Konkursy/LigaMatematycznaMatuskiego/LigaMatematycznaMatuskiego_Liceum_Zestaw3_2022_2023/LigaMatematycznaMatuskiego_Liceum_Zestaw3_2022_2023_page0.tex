\documentclass[a4paper,12pt]{article}
\usepackage{latexsym}
\usepackage{amsmath}
\usepackage{amssymb}
\usepackage{graphicx}
\usepackage{wrapfig}
\pagestyle{plain}
\usepackage{fancybox}
\usepackage{bm}

\begin{document}

LIGA MATEMATYCZNA

im. Zdzislawa Matuskiego

GRUD Z$\mathrm{I}\mathrm{E}\acute{\mathrm{N}}$ 2022

SZKOLA PONADPODSTAWOWA

ZADANIE I.

Znajd $\acute{\mathrm{z}}$ wszystkie pary $(p,q)$ liczb pierwszych takich, $\dot{\mathrm{z}}\mathrm{e}p+q=(p-q)^{3}$

ZADANIE 2.

$\mathrm{W}$ trójkąt $ABC$ wspisano okrąg, przy czym $|AC| = 5, |AB| = 6, |BC| = 3$. Na boku $AC$

wybrano punkt $D$, na boku $AB$ wybrano punkt $E$ w taki sposób, $\dot{\mathrm{z}}\mathrm{e}$ odcinek $ED$ jest styczny

do okręgu. Oblicz obwód trójkąta $AED.$

ZADANIE 3.

Dodatnie liczby rzeczywiste $a, b, c$ spelniają warunek

$\displaystyle \frac{(a+b+c)^{2}}{ab+bc+ac}=3.$

Wykaz, $\dot{\mathrm{z}}\mathrm{e}$ liczby $\alpha, b, c$ są równe.

ZADANIE 4.

Czy istnieją takie liczby calkowite $x, y, z, t, \dot{\mathrm{z}}\mathrm{e}x^{2}+y^{2}=z^{2}+t^{2}$ oraz $x+y+z+t=2023$?

ZADANIE 5.

Na tablicy napisano liczby naturalne od l do 10. Czy $\mathrm{m}\mathrm{o}\dot{\mathrm{z}}$ na umieścić między nimi znaki plus

oraz minus w taki sposób, aby otrzymać 0?
\end{document}
