\documentclass[a4paper,12pt]{article}
\usepackage{latexsym}
\usepackage{amsmath}
\usepackage{amssymb}
\usepackage{graphicx}
\usepackage{wrapfig}
\pagestyle{plain}
\usepackage{fancybox}
\usepackage{bm}

\begin{document}

LIGA MATEMATYCZNA

STYC Z$\mathrm{E}\acute{\mathrm{N}}$ 2010

SZKOLA PONADGIMNAZJALNA

ZADANIE I.

Znajd $\acute{\mathrm{z}}$ wszystkie funkcje $f$: $\mathbb{R}\rightarrow \mathbb{R}$ spelniające następujące warunki

$\bullet f(xy)=x^{2}f(y)+yf(x)$ dla dowolnych liczb rzeczywistych $x, y$;

$\bullet f(2)=2.$

ZADANIE 2.

Wewnątrz danego czworokąta wypukłego znajdz' taki punkt, $\dot{\mathrm{z}}$ eby odcinki łączące ten punkt

ze środkami boków czworokqta dzielily czworokąt na cztery części o równych polach.

ZADANIE 3.

Uzasadnij, $\dot{\mathrm{z}}\mathrm{e}$ wśród 651iczb natura1nych znajduje się 91iczb takich, $\dot{\mathrm{z}}\mathrm{e}$ ich suma jest podzielna

przez 9.

ZADANIE 4.

Liczba naturalna $n$ jest większa od 2000. Wykaz$\cdot, \dot{\mathrm{z}}\mathrm{e}$ liczba $n+1$ jest podzielna przez 6, $\mathrm{j}\mathrm{e}\dot{\mathrm{z}}$ eli

wiadomo, $\dot{\mathrm{z}}\mathrm{e}n\mathrm{i}n+2$ są liczbami pierwszymi.

ZADANIE 5.

$\mathrm{W}$ prostokącie ABCD punkt $M$ jest środkiem boku AD, a N- środkiem boku $BC$. Na prze-

dluzeniu odcinka $CD$ poza punktem $D$ wybrano punkt $P$. Niech $S$ będzie punktem przecięcia

prostych $PM\mathrm{i}AC$. Udowodnij, $\dot{\mathrm{z}}\mathrm{e}$ kąty $SNM\mathrm{i}MNP$ są równe.
\end{document}
