\documentclass[a4paper,12pt]{article}
\usepackage{latexsym}
\usepackage{amsmath}
\usepackage{amssymb}
\usepackage{graphicx}
\usepackage{wrapfig}
\pagestyle{plain}
\usepackage{fancybox}
\usepackage{bm}

\begin{document}

LIGA MATEMATYCZNA

STYC Z$\mathrm{E}\acute{\mathrm{N}}$ 2011

SZKOLA PONADGIMNAZJALNA

ZADANIE I.

Wewnątrz trójkąta równobocznego $ABC$ wybrano dowolny punkt $P$. Punkty $D, E, F$ są

rzutami prostokątnymi punktu $P$ na boki odpowiednio AB, $BC, CA$. Wyznacz wartości, jakie

$\mathrm{m}\mathrm{o}\dot{\mathrm{z}}\mathrm{e}$ przyjmować wyrazenie

$PD+PE+PF$

$AD+BE+CF$

ZADANIE 2.

Wyznacz wszystkie trójki $(a,b,c)$ liczb całkowitych, dla których $a^{2}-b^{2}-c^{2}=1\mathrm{i}a-b-c=-3.$

ZADANIE 3.

$\mathrm{W}$ polach tablicy $4\times 4$ umieszczono liczbę $-1$ oraz piętnaście liczb l. $\mathrm{M}\mathrm{o}\dot{\mathrm{z}}$ na jednoczešnie

zmienić znaki wszystkich liczb wjednym wierszu lub wjednej kolumnie. Wykaz$\cdot, \dot{\mathrm{z}}\mathrm{e}$ po dowolnej

liczbie takich zmian nie $\mathrm{m}\mathrm{o}\dot{\mathrm{z}}$ na uzyskać tablicy wypelnionej samymi jedynkami.

ZADANIE 4.

Załózmy, $\dot{\mathrm{z}}\mathrm{e}$ liczby $a\mathrm{i}b$ sq utworzone z tych samych cyfr, lecz ułozonych w innej kolejności.

Czy róznica tych liczb jest podzielna przez 9?

ZADANIE 5.

Czworokąty ABCD $\mathrm{i}$ EFGD są kwadratami. Oblicz dlugošć odcinka $BF$ wiedząc, $\dot{\mathrm{z}}\mathrm{e}$ dlugość

odcinka $AE$ jest równa $a.$
\begin{center}
\includegraphics[width=40.236mm,height=46.632mm]{./LigaMatematycznaMatuskiego_Liceum_Zestaw4_2010_2011_page0_images/image001.eps}
\end{center}
D c

F

A  B


\end{document}