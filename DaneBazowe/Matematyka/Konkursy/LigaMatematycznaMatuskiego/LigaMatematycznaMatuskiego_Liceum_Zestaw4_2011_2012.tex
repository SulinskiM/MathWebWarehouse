\documentclass[a4paper,12pt]{article}
\usepackage{latexsym}
\usepackage{amsmath}
\usepackage{amssymb}
\usepackage{graphicx}
\usepackage{wrapfig}
\pagestyle{plain}
\usepackage{fancybox}
\usepackage{bm}

\begin{document}

LIGA MATEMATYCZNA

STYC Z$\mathrm{E}\acute{\mathrm{N}}$ 2012

SZKOLA PONADGIMNAZJALNA

ZADANIE I.

Rozwiąz równanie

-{\it x}1 $+$ -{\it y}1 $=$ 1- -{\it x}1{\it y}

w zbiorze liczb calkowitych.

ZADANIE 2.

Wykaz, $\dot{\mathrm{z}}\mathrm{e}\mathrm{j}\mathrm{e}\dot{\mathrm{z}}$ eli $a-1, a+1$ są liczbami pierwszymi większymi od 10, to 1iczba $a^{3}-4a$ jest

podzielna przez 240.

ZADANIE 3.

Wyznacz wszystkie funkcje $f:\mathbb{R}\rightarrow \mathbb{R}$ spelniające warunek

2 $f(x)+3f(1-x)=4x-1$

dla $\mathrm{k}\mathrm{a}\dot{\mathrm{z}}$ dej liczby rzeczywistej $x.$

ZADANIE 4.

Dwusieczne kątów zewnętrznych wypukfego czworokąta ABCD utworzyly nowy czworokąt.

Udowodnij, $\dot{\mathrm{z}}\mathrm{e}$ suma dlugości przekątnych nowego czworokątajest nie mniejsza $\mathrm{n}\mathrm{i}\dot{\mathrm{z}}$ obwód czwo-

rokąta ABCD.

ZADANIE 5.

$\mathrm{W}$ klasiejest 20 uczniów wpisanych do dziennika pod numerami od 1 do 20. Czy uda się ustawić

uczniów w pary tak, aby suma numerów uczniów $\mathrm{k}\mathrm{a}\dot{\mathrm{z}}$ dej pary byla podzielna przez 6?


\end{document}