\documentclass[a4paper,12pt]{article}
\usepackage{latexsym}
\usepackage{amsmath}
\usepackage{amssymb}
\usepackage{graphicx}
\usepackage{wrapfig}
\pagestyle{plain}
\usepackage{fancybox}
\usepackage{bm}

\begin{document}

LIGA MATEMATYCZNA

im. Zdzisława Matuskiego

STYC Z$\mathrm{E}\acute{\mathrm{N}}$ 2013

SZKOLA PONADGIMNAZJALNA

ZADANIE I.

Dwa okręgi są styczne zewnętrznie. Punkt $A\mathrm{l}\mathrm{e}\dot{\mathrm{z}}\mathrm{y}$ na jednym z okręgów i nalez $\mathrm{y}$ do wspólnej

stycznej, natomiast $AB$ jest średnicą okręgu. $\mathrm{Z}$ punktu $B$ prowadzimy styczną do drugiego

okręgu w punkcie $M$. Wykaz$\cdot, \dot{\mathrm{z}}\mathrm{e}AB=BM.$

ZADANIE 2.

Na dfugim pasku papieru wypisano kolejno obok siebie 2010 wybranych 1iczb natura1nych.

Liczby są dobrane w taki sposób, $\dot{\mathrm{z}}\mathrm{e}$ iloczyn $\mathrm{k}\mathrm{a}\dot{\mathrm{z}}$ dych siedmiu sąsiednich jest równy 2010. Jaka

jest najmniejsza $\mathrm{m}\mathrm{o}\dot{\mathrm{z}}$ liwa wartość sumy tych 20101iczb? Jaka jest największa $\mathrm{m}\mathrm{o}\dot{\mathrm{z}}$ liwa wartość

tej sumy?

ZADANIE 3.

Czy istnieją takie liczby calkowite $a, b, \dot{\mathrm{z}}\mathrm{e}a^{2}+b$ oraz $a+b^{2}$ są kolejnymi liczbami calkowitymi?

ZADANIE 4.

Danych jest lll dodatnich liczb cafkowitych. Wykaz$\cdot, \dot{\mathrm{z}}\mathrm{e}$ spośród nich $\mathrm{m}\mathrm{o}\dot{\mathrm{z}}$ na wybrać ll liczb,

których suma jest podzielna przez ll.

ZADANIE 5.

Rozwia $\dot{\mathrm{z}}$ ukfad równań

$\left\{\begin{array}{l}
(x+y)(x+y+z)=72\\
(y+z)(x+y+z)=120\\
(z+x)(x+y+z)=96.
\end{array}\right.$


\end{document}