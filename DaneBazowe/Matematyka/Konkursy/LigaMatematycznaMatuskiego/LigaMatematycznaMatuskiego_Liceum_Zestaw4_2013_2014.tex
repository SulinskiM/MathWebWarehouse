\documentclass[a4paper,12pt]{article}
\usepackage{latexsym}
\usepackage{amsmath}
\usepackage{amssymb}
\usepackage{graphicx}
\usepackage{wrapfig}
\pagestyle{plain}
\usepackage{fancybox}
\usepackage{bm}

\begin{document}

LIGA MATEMATYCZNA

im. Zdzisława Matuskiego

STYC Z$\mathrm{E}\acute{\mathrm{N}}$ 2014

SZKOLA PONADGIMNAZJALNA

ZADANIE I.

Boki trójkąta $ABC$ są podzielone punktami $M, N\mathrm{i}P$ tak, $\dot{\mathrm{z}}\mathrm{e}$

{\it AM BN CP l}

{\it MB NC PA 4}

Wyznacz stosunek pola trójkąta ograniczonego prostymi AN, BP, CM do pola trójkąta ABC.

ZADANIE 2.

Ciąg liczbowy $(a_{n})_{n\in \mathbb{N}}$ jest określony następująco:

$a_{1}=1,$

$a_{2}=1,$

$a_{3}=-1,$

$\alpha_{n}=a_{n-1}a_{n-3}$, gdy $n\geq 4.$

Oblicz a2014.

ZADANIE 3.

$K\mathrm{a}\dot{\mathrm{z}}\mathrm{d}\mathrm{y}$ punkt plaszczyzny pomalowano na jeden z czterech kolorów: zólty, czerwony, zielony

oraz niebieski. $\mathrm{K}\mathrm{a}\dot{\mathrm{z}}\mathrm{d}\mathrm{y}$ kolor został wykorzystany. Wykaz$\cdot, \dot{\mathrm{z}}\mathrm{e}$ istnieje prosta, której punkty

sa co najmniej trzech kolorów.

ZADANIE 4.

Rozwiąz ukfad równań

$\left\{\begin{array}{l}
2x^{2014}+2y^{2014}-\mathrm{z}^{2014}=4\\
2y^{2014}+2z^{2014}-\mathrm{x}^{2014}=22\\
2z^{2014}+2x^{2014}-\mathrm{y}^{2014}=16.
\end{array}\right.$

ZADANIE 5.

Porównując wyniki w lowieniu ryb Adam, Bartek i Czarek stwierdzili, $\dot{\mathrm{z}}\mathrm{e}$ jeden z nich zlowil

tylko okonie, jeden tylko pstragi i jeden tylko lososie. Liczba ryb Adama jest o 7 większa od $\displaystyle \frac{3}{5}$

liczby okoni. Liczba ryb Bartka jest o 3 większa od $\displaystyle \frac{5}{7}$ liczby lososi. Natomiast liczba wszystkich

ryb jest trzycyfrowa liczbą pierwszą. Ile ryb złowil $\mathrm{k}\mathrm{a}\dot{\mathrm{z}}\mathrm{d}\mathrm{y}$ z chłopców?


\end{document}