\documentclass[a4paper,12pt]{article}
\usepackage{latexsym}
\usepackage{amsmath}
\usepackage{amssymb}
\usepackage{graphicx}
\usepackage{wrapfig}
\pagestyle{plain}
\usepackage{fancybox}
\usepackage{bm}

\begin{document}

LIGA MATEMATYCZNA

im. Zdzisława Matuskiego

STYC Z$\mathrm{E}\acute{\mathrm{N}}$ 2015

SZKOLA PONADGIMNAZJALNA

ZADANIE I.

Okręgi o promieniach riR przecinają się w punkcie K. Niech MiN będą punktami styczności

z okręgami wspólnej stycznej. Oblicz dfugość promienia okręgu opisanego na trójkącie KMN.

ZADANIE 2.

Wykaz, $\dot{\mathrm{z}}\mathrm{e}$ uklad równań

$\left\{\begin{array}{l}
x^{2}+2y=19\\
y^{2}+2z=9\\
z^{2}+2x=8
\end{array}\right.$

nie ma rozwiązań w zbiorze liczb cafkowitych.

ZADANIE 3.

$\mathrm{W}$ pola nieskończonej szachownicy wpisano liczby naturalne w taki sposób, $\dot{\mathrm{z}}\mathrm{e} \mathrm{k}\mathrm{a}\dot{\mathrm{z}}$ da liczba

w polujest średnią arytmetyczną ośmiu liczb z nią sąsiadujących. Wykaz, $\dot{\mathrm{z}}\mathrm{e}$ liczba 100 pojawi1a

się na szachownicy wiele razy lub nie pojawila się wcale.

ZADANIE 4.

Oblicz wartość wyrazenia

-{\it aa} $+$-{\it bb}'

jeśli $2a^{2}+4ab=ab+2b^{2}$ oraz $a\neq b.$

ZADANIE 5.

Wyznacz wszystkie funkcje $f:\mathbb{R}\rightarrow \mathbb{R}$ spelniające warunek

$f(f(x)-y)=f(x)+f(f(y)-f(-x))+x$

dla dowolnych liczb rzeczywistych $x, y.$
\end{document}
