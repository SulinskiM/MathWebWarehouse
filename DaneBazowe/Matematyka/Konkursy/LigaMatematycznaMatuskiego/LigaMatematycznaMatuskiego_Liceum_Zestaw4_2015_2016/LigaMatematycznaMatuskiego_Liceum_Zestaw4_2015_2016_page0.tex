\documentclass[a4paper,12pt]{article}
\usepackage{latexsym}
\usepackage{amsmath}
\usepackage{amssymb}
\usepackage{graphicx}
\usepackage{wrapfig}
\pagestyle{plain}
\usepackage{fancybox}
\usepackage{bm}

\begin{document}

LIGA MATEMATYCZNA

im. Zdzisława Matuskiego

STYCZEN 2016

SZKOLA PONADGIMNAZJALNA

ZADANIE I.

Dany jest trapez ABCD o podstawach AB $\mathrm{i}$ CD oraz taki punkt $E\mathrm{l}\mathrm{e}\dot{\mathrm{z}}\mathrm{a}\mathrm{c}\mathrm{y}$ wewnątrz trapezu,

$\dot{\mathrm{z}}\mathrm{e}$ kąty $\triangleleft AED \mathrm{i} \triangleleft BEC$ sq proste. Punkt $S$ jest punktem przecięcia przekątnych trapezu.

Wykaz, $\dot{\mathrm{z}}\mathrm{e}\mathrm{j}\mathrm{e}\dot{\mathrm{z}}$ eli $E\neq S$, to prosta $ES$ jest prostopadla do podstaw trapezu.

ZADANIE 2.

Wykaz, $\dot{\mathrm{z}}\mathrm{e}$

$\sqrt[3]{120+\sqrt[3]{120+\sqrt[3]{120+}}}$

jest liczbą naturalna.

ZADANIE 3.

Rozstrzygnij, czy istnieje czworościan, w którym środki okręgów opisanych na ścianach $\mathrm{l}\mathrm{e}\dot{\mathrm{z}}\mathrm{a}$

na jednej plaszczy $\acute{\mathrm{z}}\mathrm{n}\mathrm{i}\mathrm{e}.$

ZADANIE 4.

Liczby 1, 2, 3, 4, $\ldots$, 32, 33 umieszczono w wierzcholkach 33-kąta foremnego, a następnie na

środku $\mathrm{k}\mathrm{a}\dot{\mathrm{z}}$ dego jego boku zapisano sumę liczb stojących na jego końcach. Czy istnieje takie

rozstawienie tych liczb w wierzchołkach wielokąta, aby wszystkie liczby zapisane na środkach

jego boków byfy liczbami podzielnymi przez 4?

ZADANIE 5.

Wyznacz wszystkie funkcje $f:\mathbb{R}\rightarrow \mathbb{R}$ spefniajqce warunek

$f(x+y)-f(x-y)=4xy$

dla dowolnych liczb rzeczywistych $x, y.$
\end{document}
