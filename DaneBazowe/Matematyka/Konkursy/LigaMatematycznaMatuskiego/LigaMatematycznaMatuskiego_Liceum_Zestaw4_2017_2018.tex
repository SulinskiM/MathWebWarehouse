\documentclass[a4paper,12pt]{article}
\usepackage{latexsym}
\usepackage{amsmath}
\usepackage{amssymb}
\usepackage{graphicx}
\usepackage{wrapfig}
\pagestyle{plain}
\usepackage{fancybox}
\usepackage{bm}

\begin{document}

LIGA MATEMATYCZNA

im. Zdzisława Matuskiego

STYCZEN 2018

SZKOLA PONADGIMNAZJALNA

ZADANIE I.

Czy istnieje czworościan, który ma siatkę będącą trójkatem prostokątnym?

sadnij.

Odpowied $\acute{\mathrm{z}}$ uza-

ZADANIE 2.

Wyznacz $T(2018)$ w ciągu o podanym wzorze rekurencyjnym

$\left\{\begin{array}{l}
T(1)=1\\
T(n)=2\cdot T([\frac{n}{2}]),\mathrm{g}\mathrm{d}\mathrm{y}n\geq 2.
\end{array}\right.$

gdzie $[x]$ oznacza największą liczbę całkowitą nie przekraczającą liczby $x.$

ZADANIE 3.

Czy istnieją liczby calkowite a $\mathrm{i}b$, które spelniają równanie

$|a^{2}+b|+|a^{2}-b|+|a+b^{2}|+|a-b^{2}|=1234567$?

ZADANIE 4.

$\mathrm{W}$ zbiorze liczb rzeczywistych rozwiąz ukfad równań

$\left\{\begin{array}{l}
\sqrt{2x-y+11}-\sqrt{3x+y-9}=3\\
\sqrt[4]{2x-y+11}+\sqrt[4]{3x+y-9}=3.
\end{array}\right.$

ZADANIE 5.

W okrąg wpisano dwa wielokąty równokątne: 2016-kąt i 2018-kąt.

wspólnych boków mogą mieć te dwa wielokąty?

Jaka największa liczbę


\end{document}