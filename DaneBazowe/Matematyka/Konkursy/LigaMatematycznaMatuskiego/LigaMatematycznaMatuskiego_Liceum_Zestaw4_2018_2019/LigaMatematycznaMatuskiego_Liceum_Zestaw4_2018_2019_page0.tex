\documentclass[a4paper,12pt]{article}
\usepackage{latexsym}
\usepackage{amsmath}
\usepackage{amssymb}
\usepackage{graphicx}
\usepackage{wrapfig}
\pagestyle{plain}
\usepackage{fancybox}
\usepackage{bm}

\begin{document}

LIGA MATEMATYCZNA

im. Zdzisława Matuskiego

STYCZEN 2019

SZKOLA PONADPODSTAWOWA

ZADANIE I.

Boki trójkąta $ABC$ podzielono takimi punktami $D, E, F, \displaystyle \dot{\mathrm{z}}\mathrm{e}\frac{|AD|}{|DB|}=\frac{|BE|}{|EC|}=\frac{|CF|}{|FA|}=6$. Wyznacz

stosunek pola trójkąta $DEF$ do pola trójkąta $ABC.$
\begin{center}
\includegraphics[width=63.096mm,height=38.964mm]{./LigaMatematycznaMatuskiego_Liceum_Zestaw4_2018_2019_page0_images/image001.eps}
\end{center}
C

E

F

A  D  B

ZADANIE 2. Liczba dodatnia $x$ jest $p$ razy większa od liczby $y$. Suma liczb $x\mathrm{i}y$ jest $q$ razy

większa od ich róznicy. Znajd $\acute{\mathrm{z}}$ sumę $p+q$ wiedząc, $\dot{\mathrm{z}}\mathrm{e}p\mathrm{i}q$ są liczbami cafkowitymi dodatnimi.

ZADANIE 3.

Wyznacz największą liczbę pięciocyfrową spelniającą warunki:

$\bullet \dot{\mathrm{z}}$ adna cyfra nie jest zerem;

$\bullet$ pierwsze trzy cyfry tworza liczbę, która jest 9 razy większa od 1iczby utworzonej przez

dwie ostanie cyfry;

$\bullet$ trzy ostatnie cyfry tworzą liczbę, która jest 7 razy większa od 1iczby utworzonej przez

pierwsze dwie cyfry.

(Uwaga. Przyjmujemy, $\dot{\mathrm{z}}\mathrm{e}$ ostatnią cyfrą liczby jest cyfra jedności.)

ZADANIE 4.

Wyznacz wszystkie liczby pierwsze $p$ takie, $\dot{\mathrm{z}}\mathrm{e}p+6, p+12, p+18, p+24$ są równiez liczbami

pierwszymi.

ZADANIE 5.

Znajd $\acute{\mathrm{z}}$ wszystkie funkcje $f$: $\mathbb{R}\backslash \{0,1\}\rightarrow \mathbb{R}$ spełniające warunek

$(1-x)f(x)-2xf(1-x)=1$

dla $\mathrm{k}\mathrm{a}\dot{\mathrm{z}}$ dej liczby rzeczywistej $x$ róznej od 0 $\mathrm{i}1.$
\end{document}
