\documentclass[a4paper,12pt]{article}
\usepackage{latexsym}
\usepackage{amsmath}
\usepackage{amssymb}
\usepackage{graphicx}
\usepackage{wrapfig}
\pagestyle{plain}
\usepackage{fancybox}
\usepackage{bm}

\begin{document}

LIGA MATEMATYCZNA

im. Zdzisława Matuskiego

STYC Z$\mathrm{E}\acute{\mathrm{N}}$ 2020

SZKOLA PONADPODSTAWOWA

ZADANIE I.

Ile jest liczb trzycyfrowych $\overline{xyz}$ podzielnych przez 21ub 5 takich, $\dot{\mathrm{z}}\mathrm{e}$

$\overline{xyz}+\overline{xzy}+\overline{yxz}=\overline{yzx}+\overline{zxy}+\overline{zyx}$?

ZADANIE 2.

Pole trapezu ABCD jest równe $s$, a stosunek dlugości podstaw AB $\mathrm{i}$ CD jest równy $k$. Prze-

kątne $AC\mathrm{i}BD$ przecinają się w punkcie $O$. Oblicz pole trójkąta $ABO.$

ZADANIE 3.

Uzasadnij, $\dot{\mathrm{z}}\mathrm{e}$ wśród pięciu liczb calkowitych $\mathrm{m}\mathrm{o}\dot{\mathrm{z}}$ na wybrać kilka tak, aby suma wybranych

liczb była podzielna przez 5.

ZADANIE 4.

Znajd $\acute{\mathrm{z}}$ wszystkie liczby pierwsze $p, q$ takie, $\dot{\mathrm{z}}\mathrm{e}7p+q$ oraz $pq+11\mathrm{t}\mathrm{e}\dot{\mathrm{z}}$ są liczbami pierwszymi.

ZADANIE 5.

$\mathrm{W}$ zbiorze liczb rzeczywistych rozwiąz uklad równań

$\left\{\begin{array}{l}
x^{2}+y^{2}+z=2\\
y^{2}+z^{2}+x=2\\
z^{2}+x^{2}+y=2.
\end{array}\right.$
\end{document}
