\documentclass[a4paper,12pt]{article}
\usepackage{latexsym}
\usepackage{amsmath}
\usepackage{amssymb}
\usepackage{graphicx}
\usepackage{wrapfig}
\pagestyle{plain}
\usepackage{fancybox}
\usepackage{bm}

\begin{document}

LIGA MATEMATYCZNA

im. Zdzisława Matuskiego

STYCZEN 2021

SZKOLA PONADPODSTAWOWA

ZADANIE I.

Czy istnieją funkcje rzeczywiste $f$: $\mathbb{R}\rightarrow \mathbb{R}, g:\mathbb{R}\rightarrow \mathbb{R}$ takie, $\dot{\mathrm{z}}\mathrm{e}$

$f(x)g(y)=x+y+1$

dla dowolnych liczb rzeczywistych $x, y$?

ZADANIE 2.

Rozwiąz uklad równań

({\it xxxx}..32110.{\it xxxx}4321{\it xxxx}4532$===$--1-1-11

w zbiorze liczb rzeczywistych.

ZADANIE 3.

Czy liczbę 100 $\mathrm{m}\mathrm{o}\dot{\mathrm{z}}$ na przedstawič w postaci sumy liczb jednocyfrowych lub dwucyfrowych tak,

aby $\mathrm{u}\dot{\mathrm{z}}$ yć $\mathrm{k}\mathrm{a}\dot{\mathrm{z}}$ dą z cyfr dokładnie jeden raz?

ZADANIE 4.

$\mathrm{W}$ zbiorze liczb calkowitych dodatnich rozwiąz równanie

$9^{x}-2^{y}=1.$

ZADANIE 5.

$\mathrm{W}$ prostokącie ABCD po jego wewnętrznej stronie budujemy trójkqty równoboczne $ABE$ oraz

$BCF$. Wykaz, $\dot{\mathrm{z}}\mathrm{e}$ trójkąt $DFE$ jest równoboczny.
\end{document}
