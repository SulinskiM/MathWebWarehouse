\documentclass[a4paper,12pt]{article}
\usepackage{latexsym}
\usepackage{amsmath}
\usepackage{amssymb}
\usepackage{graphicx}
\usepackage{wrapfig}
\pagestyle{plain}
\usepackage{fancybox}
\usepackage{bm}

\begin{document}

LIGA MATEMATYCZNA

im. Zdzislawa Matuskiego

STYCZEN 2022

SZKOLA PONADPODSTAWOWA

ZADANIE I.

Znajd $\acute{\mathrm{z}}$ wszystkie liczby pierwsze $p$ takie, $\dot{\mathrm{z}}\mathrm{e}p^{2}+2\mathrm{i}p^{3}+2$ są liczbami pierwszymi.

ZADANIE 2.

Na 101 kartkach Adam zapisaf 1iczby natura1ne od 1 do 101, po jednej na $\mathrm{k}\mathrm{a}\dot{\mathrm{z}}$ dej kartce. Potem

kartki odwrócil, pomieszaf i zapisal znowu liczby od l do 101, po jednej na $\mathrm{k}\mathrm{a}\dot{\mathrm{z}}$ dej kartce.

Następnie dodal liczby z obu stron kartki. Wykaz, $\dot{\mathrm{z}}\mathrm{e}$ iloczyn otrzymanych wyników jest liczbą

parzysta.

ZADANIE 3.

$\mathrm{W}$ trójkącie prostokątnym $ABC$ dwusieczna kąta ostrego dzieli przeciwlegfy bok w stosunku

3 : 5. Oblicz stosunek promienia $r$ okręgu wpisanego w ten trójkąt do promienia $R$ okręgu

opisanego na tym trójkącie.

ZADANIE 4.

Wyznacz wszystkie pary róznych liczb calkowitych $(x,y)$ spelniające równanie

$x+\displaystyle \frac{1}{y-2021}=y+\frac{1}{x-2021}.$

ZADANIE 5.

$\mathrm{W}$ zbiorze liczb rzeczywistych rozwiąz uklad równań

$\left\{\begin{array}{l}
25x^{2}+9y^{2}=12yz\\
9y^{2}+4z^{2}=20xz\\
4z^{2}+25x^{2}=30xy.
\end{array}\right.$


\end{document}