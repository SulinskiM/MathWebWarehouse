\documentclass[a4paper,12pt]{article}
\usepackage{latexsym}
\usepackage{amsmath}
\usepackage{amssymb}
\usepackage{graphicx}
\usepackage{wrapfig}
\pagestyle{plain}
\usepackage{fancybox}
\usepackage{bm}

\begin{document}

LIGA MATEMATYCZNA

im. Zdzislawa Matuskiego

STYCZEN 2023

SZKOLA PONADPODSTAWOWA

ZADANIE I.

Liczby 1, 2, 3, $\ldots$, 9 umieszczono na okręgu. Przez operację rozumiemy dodanie pewnej (tej

samej) liczby cafkowitej do dwóch wybranych sąsiednich liczb i umieszczenie tych sum na

okręgu w miejsce poprzednich liczb. Czy po wykonaniu skończonej liczby takich operacji $\mathrm{m}\mathrm{o}\dot{\mathrm{z}}$ na

otrzymać na okręgu dziewięć zer?

ZADANIE 2.

Czy $\mathrm{u}\dot{\mathrm{z}}$ ywając wszystkich dziesięciu cyfr $\mathrm{m}\mathrm{o}\dot{\mathrm{z}}$ na ułozyć liczbę podzielną przez ll?

ZADANIE 3.

Na boku $AB$ kwadratu ABCD wybrano punkt $E$, a na boku $BC$ wybrano punkt $F$ i pofaczo-

no je z wierzcholkami kwadratu. Odcinki te podzielify kwadrat na osiem części. Na rysunku

zapisano pola trzech z nich. Oblicz pole zaznaczonego czworokąta.
\begin{center}
\includegraphics[width=55.680mm,height=54.408mm]{./LigaMatematycznaMatuskiego_Liceum_Zestaw4_2022_2023_page0_images/image001.eps}
\end{center}
D  c

2

F

9

3

A  B  E

ZADANIE 4.

$\mathrm{W}$ zbiorze liczb rzeczywistych rozwiąz uklad równań

$\left\{\begin{array}{l}
2x+3y=5y^{2}\\
2y+3x=5x^{2}
\end{array}\right.$

ZADANIE 5.

Wyznacz wszystkie liczby pierwsze $p$ o tej wlasności, $\dot{\mathrm{z}}\mathrm{e}p+11$ jest dzielnikiem liczby

$p(p+1)(p+2).$
\end{document}
