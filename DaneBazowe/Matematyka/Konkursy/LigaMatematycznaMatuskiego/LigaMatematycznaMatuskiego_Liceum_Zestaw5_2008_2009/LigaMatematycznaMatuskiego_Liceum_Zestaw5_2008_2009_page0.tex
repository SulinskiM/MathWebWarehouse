\documentclass[a4paper,12pt]{article}
\usepackage{latexsym}
\usepackage{amsmath}
\usepackage{amssymb}
\usepackage{graphicx}
\usepackage{wrapfig}
\pagestyle{plain}
\usepackage{fancybox}
\usepackage{bm}

\begin{document}

LIGA MATEMATYCZNA

FINAL

25 kwietnia 2009

SZKOLA PONADGIMNAZJALNA

ZADANIE I.

Znajd $\acute{\mathrm{z}}$ wszystkie funkcje $f$: $\mathbb{R}\rightarrow \mathbb{R}$ spełniające warunek $2f(x)+f(1-x)=x$ dla wszystkich

liczb rzeczywistych $x.$

ZADANIE 2.

Wypisujemy kolejne liczby naturalne od l do 2009. $K\mathrm{a}\dot{\mathrm{z}}$ dą z tych liczb zastępujemy sumą jej

cyfr i powtarzamy to $\mathrm{a}\dot{\mathrm{z}}$ do momentu uzyskania liczb jednocyfrowych. Jakich liczb w tym ciągu

jest więcej: jedynek czy ósemek?

ZADANIE 3.

Wyznacz wszystkie wartości naturalne $n$, dla których $3^{n}-1$ jest liczbą podzielną przez 13.

Wykaz, $\dot{\mathrm{z}}\mathrm{e}$ dla $\dot{\mathrm{z}}$ adnej wartości naturalnej $n$ liczba $3^{n}+1$ nie jest podzielna przez 13.

ZADANIE 4.

Liczby $n+2$ oraz $n-10$ są kwadratami liczb naturalnych. Znajdz' $n.$

ZADANIE 5.

$\mathrm{W}$ czworokącie wypuklym ABCD trójkąty $ABC, BCD, CDA, DAB$ mają równe obwody.

Udowodnij, $\dot{\mathrm{z}}\mathrm{e}$ ten czworokąt jest prostokątem.

ZADANIE 6.

Wykaz, $\dot{\mathrm{z}}\mathrm{e}$ wśród 401iczb natura1nych $\mathrm{m}\mathrm{o}\dot{\mathrm{z}}$ na wybrać 4, z których $\mathrm{k}\mathrm{a}\dot{\mathrm{z}}$ de dwie dają róznicę

podzielną przez 13.

ZADANIE 7.

Trapez prostokątny opisano na okręgu. Oblicz długości boków nierównoległych, $\mathrm{j}\mathrm{e}\dot{\mathrm{z}}$ eli podstawy

są równe a $\mathrm{i}b.$
\end{document}
