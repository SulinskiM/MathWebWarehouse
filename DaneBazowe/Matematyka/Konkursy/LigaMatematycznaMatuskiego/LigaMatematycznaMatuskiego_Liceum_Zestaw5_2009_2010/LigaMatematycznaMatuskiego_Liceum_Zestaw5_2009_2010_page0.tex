\documentclass[a4paper,12pt]{article}
\usepackage{latexsym}
\usepackage{amsmath}
\usepackage{amssymb}
\usepackage{graphicx}
\usepackage{wrapfig}
\pagestyle{plain}
\usepackage{fancybox}
\usepackage{bm}

\begin{document}

LIGA MATEMATYCZNA

FINAL

26 marca 20l0

SZKOLA PONADGIMNAZJALNA

ZADANIE I.

Wykaz, $\dot{\mathrm{z}}\mathrm{e}\mathrm{j}\mathrm{e}\dot{\mathrm{z}}$ eli w trapez $\mathrm{m}\mathrm{o}\dot{\mathrm{z}}$ na wpisać okrqg, to okręgi, których średnicami są ramiona tego

trapezu, są styczne.

ZADANIE 2.

Udowodnij, $\dot{\mathrm{z}}\mathrm{e}$ jeśli $p\mathrm{i}q$ sq liczbami pierwszymi takimi, $\dot{\mathrm{z}}\mathrm{e}p\geq 5$ oraz $q-p=2$, to liczba $p+q$

jest podzielna przez 12.

ZADANIE 3.

Pole trójkąta $EFG$ jest równe l. Oblicz pole trójkąta $ABC$, wiedzqc, $\dot{\mathrm{z}}\mathrm{e}$

$|AE|=|EG|,|EF|=|FB|,|FG|=|GC|.$
\begin{center}
\includegraphics[width=78.840mm,height=33.372mm]{./LigaMatematycznaMatuskiego_Liceum_Zestaw5_2009_2010_page0_images/image001.eps}
\end{center}
C

E  F

A  H

ZADANIE 4.

Znajd $\acute{\mathrm{z}}$ wszystkie funkcje $f$: $\mathbb{R}\rightarrow \mathbb{R}$ spełniające warunek

$f(x)f(y)-f(xy)=x+y$ dla wszystkich liczb rzeczywistych $x, y.$

ZADANIE 5.

Oblicz wartość wyrazenia $q^{4}-6q^{3}+9q^{2}-7$ wiedząc, $\dot{\mathrm{z}}\mathrm{e}q^{2}-3q+1=0.$

ZADANIE 6.

Przedstaw liczbę l jako sumę kwadratów: (a) dwóch;

(b) trzech;

(c) czterech,

parami róznych dodatnich liczb wymiernych.

ZADANIE 7.

Na okręgu zaznaczono sześć punktów. $K\mathrm{a}\dot{\mathrm{z}}\mathrm{d}\mathrm{y}$ z odcinków lączących te punkty pomalowano

na czerwono lub niebiesko. Wykaz, $\dot{\mathrm{z}}\mathrm{e}$ otrzymano przynajmniej jeden jednokolorowy trójkąt.
\end{document}
