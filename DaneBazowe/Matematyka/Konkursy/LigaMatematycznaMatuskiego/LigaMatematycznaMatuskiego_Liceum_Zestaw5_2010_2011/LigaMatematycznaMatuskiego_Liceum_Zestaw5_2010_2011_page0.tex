\documentclass[a4paper,12pt]{article}
\usepackage{latexsym}
\usepackage{amsmath}
\usepackage{amssymb}
\usepackage{graphicx}
\usepackage{wrapfig}
\pagestyle{plain}
\usepackage{fancybox}
\usepackage{bm}

\begin{document}

LIGA MATEMATYCZNA

FINAL

30 marca 20ll

SZKOLA PONADGIMNAZJALNA

ZADANIE I.

Na okręgu napisano jedenaście liczb. Suma $\mathrm{k}\mathrm{a}\dot{\mathrm{z}}$ dych trzech kolejnych jest taka sama.

z liczb jest 9. Wyznacz pozostałe 1iczby.

Jedną

ZADANIE 2.

Rozwiąz uklad równań

$\left\{\begin{array}{l}
ab=a+b+1\\
bc=b+c+2\\
ac=\alpha+c+5.
\end{array}\right.$

ZADANIE 3.

Wykaz$\cdot, \dot{\mathrm{z}}\mathrm{e}$ dla dowolnych liczb cafkowitych $a\mathrm{i}b$ liczba $a^{3}b-ab^{3}$ jest podzielna przez 3.

ZADANIE 4.

Dany jest pięciokąt wypukly ABCDE, w którym przekątna $BD$ jest równolegla do boku $AE,$

a przekątna $CE$ jest równoległa do boku $AB$. Wykazać, $\dot{\mathrm{z}}\mathrm{e}$ pola trójkątów $ABC \mathrm{i} ADE$

są równe.

ZADANIE 5.

Wykaz$\cdot, \dot{\mathrm{z}}\mathrm{e}$ kwadrat liczby pierwszej większej od 3 z dzie1enia przez 12 daje resztę 1.

ZADANIE 6.

Rozwiąz równanie $x^{2}+4x-y^{2}-2y-8=0$ w zbiorze liczb naturalnych.

ZADANIE 7.

Czworokąt ABCD jest kwadratem. Punkty $E\mathrm{i}F$ lezą na bokach $BC\mathrm{i}$ CD tego kwadratu tak,

$\dot{\mathrm{z}}\mathrm{e}\angle EAF=45^{\mathrm{o}}$ Wykaz, $\dot{\mathrm{z}}\mathrm{e}|BE|+|DF|=|EF|.$
\begin{center}
\includegraphics[width=33.432mm,height=32.916mm]{./LigaMatematycznaMatuskiego_Liceum_Zestaw5_2010_2011_page0_images/image001.eps}
\end{center}
F

D c

E

A  B
\end{document}
