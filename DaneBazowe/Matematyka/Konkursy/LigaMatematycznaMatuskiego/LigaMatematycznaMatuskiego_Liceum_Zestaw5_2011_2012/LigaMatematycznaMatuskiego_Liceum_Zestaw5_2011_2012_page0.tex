\documentclass[a4paper,12pt]{article}
\usepackage{latexsym}
\usepackage{amsmath}
\usepackage{amssymb}
\usepackage{graphicx}
\usepackage{wrapfig}
\pagestyle{plain}
\usepackage{fancybox}
\usepackage{bm}

\begin{document}

l LiceumOgóloksztalcacewSlpsku

AkadmiPomorskawSiupsku

LIGA MATEMATYCZNA

FINAL

ll kwietnia 2012

SZKOLA PONADGIMNAZJALNA

ZADANIE I.

Liczby naturalne od l do 1000 pomnozono ko1ejno $\mathrm{k}\mathrm{a}\dot{\mathrm{z}}$ da przez $\mathrm{k}\mathrm{a}\dot{\mathrm{z}}$ dą. Wykaz$\cdot, \dot{\mathrm{z}}\mathrm{e}$ wšród tych

iloczynów więcej jest liczb parzystych $\mathrm{n}\mathrm{i}\dot{\mathrm{z}}$ nieparzystych.

ZADANIE 2.

Uzasadnij, $\dot{\mathrm{z}}\mathrm{e}$ dla $\mathrm{k}\mathrm{a}\dot{\mathrm{z}}$ dej liczby naturalnej $n$ liczba $n^{3}+5n$ jest podzielna przez 6.

ZADANIE 3.

Wyznacz wszystkie funkcje $f:\mathbb{R}\rightarrow \mathbb{R}$ spełniające równanie

2 $f(x)+f(-x)=3x^{2}+x+3$

dla $\mathrm{k}\mathrm{a}\dot{\mathrm{z}}$ dej liczby rzeczywistej $x.$

ZADANIE 4.

Na bokach AB, $BC\mathrm{i}AC$ trójkąta wybrano odpowiednio punkty $P, Q\mathrm{i}R$ tak, $\dot{\mathrm{z}}\mathrm{e}AP=CQ$

oraz na czworokącie RPBQ $\mathrm{m}\mathrm{o}\dot{\mathrm{z}}$ na opisać okrąg. Styczne do okręgu opisanego na trójkącie

$ABC$ w punktach $A\mathrm{i}C$ przecinają proste RP $\mathrm{i}RQ$ odpowiednio w punktach $X\mathrm{i}Y$. Wykaz,

$\dot{\mathrm{z}}\mathrm{e}RX=RY.$

ZADANIE 5.

$\mathrm{W}$ wierzchołkach siedmiokąta foremnego ustawiono pionki czerwone lub niebieskie- po jednym

w $\mathrm{k}\mathrm{a}\dot{\mathrm{z}}$ dym wierzcholku. Uzasadnij, $\dot{\mathrm{z}}\mathrm{e}$ znajdą się trzy wierzchołki z pionkami tego samego koloru

takie, $\dot{\mathrm{z}}\mathrm{e}$ będą wierzcholkami trójkąta równoramiennego.

ZADANIE 6.

Znajd $\acute{\mathrm{z}}$ wszystkie liczby pierwsze $p$ takie, $\dot{\mathrm{z}}\mathrm{e}2p-1, 2p+1$ są równiez liczbami pierwszymi.

ZADANIE 7.

Tarczę podzielono na szešć sektorów i w $\mathrm{k}\mathrm{a}\dot{\mathrm{z}}\mathrm{d}\mathrm{y}$ wpisano inną liczbę naturalną od l do 6. Zmie-

niamy te liczby przez dodanie do dwóch z nich tej samej liczby. Operację tę powtarzamy

wielokrotnie. Czy w którymś momencie we wszystkich sektorach będzie ta sama liczba?
\end{document}
