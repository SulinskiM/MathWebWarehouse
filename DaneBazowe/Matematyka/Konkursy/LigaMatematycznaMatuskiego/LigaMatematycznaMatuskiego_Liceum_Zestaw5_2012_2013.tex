\documentclass[a4paper,12pt]{article}
\usepackage{latexsym}
\usepackage{amsmath}
\usepackage{amssymb}
\usepackage{graphicx}
\usepackage{wrapfig}
\pagestyle{plain}
\usepackage{fancybox}
\usepackage{bm}

\begin{document}

LIGA MATEMATYCZNA

im. Zdzisława Matuskiego

FINAL

10 kwietnia 20l3

SZKOLA PONADGIMNAZJALNA

ZADANIE I.

Iloczyn 221iczb ca1kowitych jest równy 1. Czy suma tych 1iczb $\mathrm{m}\mathrm{o}\dot{\mathrm{z}}\mathrm{e}$ być równa 0?

ZADANIE 2.

Rozwia $\dot{\mathrm{z}}$ uklad równań 

ZADANIE 3.

Liczby rzeczywiste $\alpha, b$ spelniają równośč $\displaystyle \frac{2a}{a+b}+\frac{b}{a-b}=2$. Wyznacz wszystkie wartościjakie

$\mathrm{m}\mathrm{o}\dot{\mathrm{z}}\mathrm{e}$ przyjmowač ufamek $\displaystyle \frac{3a-b}{a+5b}.$

ZADANIE 4.

Wyznacz wszystkie takie pary liczb naturalnych $x, y, \dot{\mathrm{z}}\mathrm{e}$ wyrazenie $(x-y)-(\sqrt{x}-\sqrt{y})$ jest

liczbą pierwszą.

ZADANIE 5.

Na plaszczy $\acute{\mathrm{z}}\mathrm{n}\mathrm{i}\mathrm{e}$ danych jest pięć punktów kratowych (są to punkty o wspólrzędnych będących

liczbami calkowitymi). Uzasadnij, $\dot{\mathrm{z}}\mathrm{e}$ środek jednego z odcinków lączących te punkty $\mathrm{t}\mathrm{e}\dot{\mathrm{z}}$ jest

punktem kratowym.

ZADANIE 6.

Dane są dwa okręgi styczne zewnętrznie w punkcie K. Odleglości K od punktów styczności

okręgów ze wspólną styczną są równe 6 i8. Wyznacz promienie okręgów.

ZADANIE 7.

Wyznacz wszystkie funkcje $f:\mathbb{R}\rightarrow \mathbb{R}$ spelniające warunki

$\bullet f(2)=2$

$\bullet f(xy)=x^{2}f(y)+yf(x)$

dla $\mathrm{k}\mathrm{a}\dot{\mathrm{z}}$ dych liczb rzeczywistych $x, y.$


\end{document}