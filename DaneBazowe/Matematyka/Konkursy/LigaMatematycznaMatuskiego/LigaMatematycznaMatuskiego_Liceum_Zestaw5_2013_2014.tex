\documentclass[a4paper,12pt]{article}
\usepackage{latexsym}
\usepackage{amsmath}
\usepackage{amssymb}
\usepackage{graphicx}
\usepackage{wrapfig}
\pagestyle{plain}
\usepackage{fancybox}
\usepackage{bm}

\begin{document}

LIGA MATEMATYCZNA

im. Zdzisława Matuskiego

FINAL

15 kwietnia 20l4

SZKOLA PONADGIMNAZJALNA

ZADANIE I.

Wykaz, $\dot{\mathrm{z}}$ ejezeli $a, b, c, d$ są liczbami nieparzystymi, to nie istnieje taka liczba calkowita $x$, aby

spelniona byla równość

$x^{4}+ax^{3}+bx^{2}+cx+d=0.$

ZADANIE 2.

Rozwiąz uklad równań

$\left\{\begin{array}{l}
x^{2}+2y^{2}-2yz=100\\
2xy-z^{2}=100.
\end{array}\right.$

ZADANIE 3.

Wyznacz wszystkie funkcje $f:\mathbb{R}\backslash \{0\}\rightarrow \mathbb{R}$ spelniające równanie

$2f(x)+3f(\displaystyle \frac{1}{x})=x^{2}$

dla $\mathrm{k}\mathrm{a}\dot{\mathrm{z}}$ dej liczby rzeczywistej $x$ róznej od 0.

ZADANIE 4.

Wysokość i środkowa poprowadzone z jednego wierzchofka trójkąta tworzą z bokami tego trój-

kąta jednakowe kąty. $\acute{\mathrm{S}}$ rodkowa ma dlugość $a$. Oblicz promień okręgu opisanego na tym trój-

kącie.

ZADANIE 5.

Na okręgu wybrano 2015 punktów, z których 2014 poko1orowano na biafo oraz jeden na czer-

wono. Których wielokątów o wierzchofkach w tych punktach jest więcej: wielokątów o białych

wierzcholkach czy wielokątów z jednym wierzchołkiem czerwonym?

ZADANIE 6.

Czy istnieje liczba naturalna $n$ taka, $\dot{\mathrm{z}}\mathrm{e}$ w zapisie dziesiętnym liczby $2^{n}\mathrm{k}\mathrm{a}\dot{\mathrm{z}}$ da z cyfr 0, 1, 2, $\ldots$, 9

występuje 1000 razy?

ZADANIE 7.

Prosta przechodzaca przez środki przekątnych $AC\mathrm{i}BD$ czworokąta ABCD przecina boki $AD$

$\mathrm{i}BC$ w punktach, odpowiednio, $M\mathrm{i}N$. Wykaz, $\dot{\mathrm{z}}\mathrm{e}$ trójkqty AND $\mathrm{i}BCM$ mają równe pola.


\end{document}