\documentclass[a4paper,12pt]{article}
\usepackage{latexsym}
\usepackage{amsmath}
\usepackage{amssymb}
\usepackage{graphicx}
\usepackage{wrapfig}
\pagestyle{plain}
\usepackage{fancybox}
\usepackage{bm}

\begin{document}

AHADEMIA POMORSHA

III SLUPSHU
\begin{center}
\includegraphics[width=40.740mm,height=4.476mm]{./LigaMatematycznaMatuskiego_Liceum_Zestaw5_2015_2016_page0_images/image001.eps}
\end{center}
LIGA MATEMATYCZNA

im. Zdzisława Matuskiego

FINAL
\begin{center}
\includegraphics[width=34.548mm,height=42.576mm]{./LigaMatematycznaMatuskiego_Liceum_Zestaw5_2015_2016_page0_images/image002.eps}
\end{center}
16 kwietnia 20l5

SZKOLA PONADGIMNAZJALNA

ZADANIE I.

Uzasadnij, $\dot{\mathrm{z}}\mathrm{e}$ liczba $S$ jest podzielna przez 45, gdy

$ S=111\ldots$ 1$+$222$\ldots$ 2$+$333$\ldots 3+\ldots+999\ldots 9.$
\begin{center}
\includegraphics[width=64.716mm,height=6.852mm]{./LigaMatematycznaMatuskiego_Liceum_Zestaw5_2015_2016_page0_images/image003.eps}
\end{center}
2015 cyfr 20l5 cyfr  2015 cyfr
\begin{center}
\includegraphics[width=17.628mm,height=6.852mm]{./LigaMatematycznaMatuskiego_Liceum_Zestaw5_2015_2016_page0_images/image004.eps}
\end{center}
2015 cyfr

ZADANIE 2.

Dany jest okrąg $0_{1}$ o środku $S$ oraz okrąg 02 przechodzący przez $S$, przecinający okrąg 01

w punktach A $\mathrm{i}B. \mathrm{Z}$ punktu $A$ poprowadzono prostą, przecinająca okrąg $0_{1}$ w punkcie $C$, zaś

okrąg 02 w punkcie $D$. Udowodnij, $\dot{\mathrm{z}}\mathrm{e}$ trójkąt $BCD$ jest równoramienny.

ZADANIE 3.

$\mathrm{W}$ kwadracie o boku o dlugości 3 wybrano dowo1nie dziesięć punktów. Wykaz, $\dot{\mathrm{z}}\mathrm{e}$ wśród tych

punktów zawsze znajdą się dwa, których odleglość jest nie większa $\mathrm{n}\mathrm{i}\dot{\mathrm{z}}\sqrt{2}.$

ZADANIE 4.

Wykaz, $\dot{\mathrm{z}}\mathrm{e}$ dla $\mathrm{k}\mathrm{a}\dot{\mathrm{z}}$ dej liczby cafkowitej $n$ liczba $\displaystyle \frac{1}{6}(n^{3}-7n+2016)$ jest calkowita.

ZADANIE 5.

$\mathrm{W}$ klasie jest 30 uczniów. Siedzą oni w piętnastu dwuosobowych fawkach tak, $\dot{\mathrm{z}}\mathrm{e}$ polowa

dziewcząt siedzi z chłopcami. Rozstrzygnij, czy $\mathrm{m}\mathrm{o}\dot{\mathrm{z}}$ na uczniów tej klasy tak posadzić, aby

pofowa chlopców siedziala z dziewczętami.

ZADANIE 6.

$\mathrm{W}$ okrąg $0$ wpisany jest taki pięciokąt ABCDE, $\dot{\mathrm{z}}\mathrm{e} |AE| = |BC| = |CD|$. Proste AB $\mathrm{i}$ {\it DE}

przecinają się w punkcie $F$. Udowodnij, $\dot{\mathrm{z}}\mathrm{e}$ środek okręgu opisanego na trójkącie $BDF\mathrm{l}\mathrm{e}\dot{\mathrm{z}}\mathrm{y}$ na

okręgu $0.$

ZADANIE 7.

Rozwiąz uklad równań

$\left\{\begin{array}{l}
x-\frac{1}{xyz}=0\\
yzxx2y3y7zz==00.
\end{array}\right.$
\end{document}
