\documentclass[a4paper,12pt]{article}
\usepackage{latexsym}
\usepackage{amsmath}
\usepackage{amssymb}
\usepackage{graphicx}
\usepackage{wrapfig}
\pagestyle{plain}
\usepackage{fancybox}
\usepackage{bm}

\begin{document}

LIGA MATEMATYCZNA

im. Zdzisława Matuskiego

FINAL

25 kwietnia 20l6

SZKOLA PONADGIMNAZJALNA

ZADANIE I.

Wykaz, $\dot{\mathrm{z}}\mathrm{e}$ suma kwadratów trzech kolejnych liczb naturalnych nie $\mathrm{m}\mathrm{o}\dot{\mathrm{z}}\mathrm{e}$ być kwadratem liczby

naturalnej.

ZADANIE 2.

Funkcja $f$: $\mathbb{R}\rightarrow \mathbb{R}$ spelnia warunki:

{\it a}) $f(x+y)=f(x)+f(y)$ dla dowolnych liczb rzeczywistych $x, y$;

{\it b}) $f(1)=1.$

Wyznacz $f(\displaystyle \frac{9}{32}).$

ZADANIE 3.

Dany jest czworokąt wypukły ABCD. Punkty $K \mathrm{i} L 1\mathrm{e}\dot{\mathrm{Z}}$ a odpowiednio na odcinkach $AB$

$\mathrm{i}$ AD, przy czym czworokąt AKCL jest równoleglobokiem. Odcinki $KD\mathrm{i}BL$ przecinają się

w punkcie $M$. Wykaz, $\dot{\mathrm{z}}\mathrm{e}$ pola czworokątów AKML $\mathrm{i}$ BCDM są równe.

ZADANIE 4.

Znajd $\acute{\mathrm{z}}$ wszystkie liczby pierwsze $p$ o tej wfasności, $\dot{\mathrm{z}}\mathrm{e}$ liczba $19p+1$ jest sześcianem pewnej

liczby całkowitej.

ZADANIE 5.

Wykaz, $\dot{\mathrm{z}}\mathrm{e}$ dla $\mathrm{k}\mathrm{a}\dot{\mathrm{z}}$ dej liczby naturalnej $n$ liczba

4444$\ldots$ 47777$\ldots$ 74444$\ldots$ 4$+$2016
\begin{center}
\includegraphics[width=59.940mm,height=6.852mm]{./LigaMatematycznaMatuskiego_Liceum_Zestaw5_2016_2017_page0_images/image001.eps}
\end{center}
7n cyfr 4 n cyfr 7  7n cyfr 4

jest zfozona.

ZADANIE 6.

Dany jest trójkqt ostrokątny $ABC$ oraz jego wysokości AD $\mathrm{i}$ BE. Punkty $P\mathrm{i}Q$ są rzutami

prostokątnymi odpowiednio punktów $A\mathrm{i}B$ na prostą DE. Wykaz, $\dot{\mathrm{z}}\mathrm{e}|PE|=|QD|.$

ZADANIE 7.

Rozwiąz uklad równań

$\left\{\begin{array}{l}
x_{1}^{2}-3x_{1}+4=x_{2}\\
x_{2}^{2}-3x_{2}+4=x_{3}\\
x_{3}^{2}-3x_{3}+4=x_{4}\\
x_{n-1}^{2}-3x_{n-1}+4=x_{n}\\
x_{n}^{2}-3x_{n}+4=x_{1}.
\end{array}\right.$


\end{document}