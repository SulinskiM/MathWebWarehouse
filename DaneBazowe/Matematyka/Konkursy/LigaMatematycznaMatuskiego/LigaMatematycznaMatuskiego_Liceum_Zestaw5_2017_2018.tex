\documentclass[a4paper,12pt]{article}
\usepackage{latexsym}
\usepackage{amsmath}
\usepackage{amssymb}
\usepackage{graphicx}
\usepackage{wrapfig}
\pagestyle{plain}
\usepackage{fancybox}
\usepackage{bm}

\begin{document}

LIGA MATEMATYCZNA

im. Zdzisława Matuskiego

FINAL

24 kwietnia 20l7

SZKOLA PONADGIMNAZJALNA

ZADANIE I.

Dane są dwie $\mathrm{b}\mathrm{l}\mathrm{i}\acute{\mathrm{z}}$niacze (to znaczy rózniące się o 2) 1iczby pierwsze. Udowodnij, $\dot{\mathrm{z}}\mathrm{e}$ nie mogą

one być przyprostokątnymi trójkąta prostokątnego o wszystkich bokach o dlugości calkowitej.

ZADANIE 2.

Na tablicy napisano dwie liczby: pierwszq i drugą. Następnie napisano liczbę trzecią, która

jest sumą pierwszej i drugiej, potem zapisano czwartą liczbę, która jest sumą drugiej i trzeciej.

I tak dalej $\mathrm{a}\dot{\mathrm{z}}$ do dziesiątej liczby. Suma wszystkich dziesięciu liczb napisanych na tablicy jest

równa 5005. Wyznacz siódmą 1iczbę.

ZADANIE 3.

Czy liczbę l $\mathrm{m}\mathrm{o}\dot{\mathrm{z}}$ na przedstawićjako sumę ufamków $\displaystyle \frac{1}{a}, \displaystyle \frac{1}{b}, \displaystyle \frac{1}{c}, \displaystyle \frac{1}{d}$, gdzie $a, b, c, d$ są nieparzystymi

liczbami naturalnymi?

ZADANIE 4.

Danych jest $21$liczb rzeczywistych. Wiadomo, $\dot{\mathrm{z}}\mathrm{e}$ suma $\mathrm{k}\mathrm{a}\dot{\mathrm{z}}$ dych jedenastu spośród tych liczb

jest większa od sumy pozostalych dziesięciu. Wykaz, $\dot{\mathrm{z}}\mathrm{e}$ wszystkie liczby są dodatnie.

ZADANIE 5.

Oblicz pole trapezu prostokątnego wiedząc, $\dot{\mathrm{z}}\mathrm{e}$ odleglości środka okręgu wpisanego w ten trapez

od końców ramienia nieprostopadlego do podstaw, są równe $\alpha$ oraz $2a.$

ZADANIE 6.

$\mathrm{W}$ zbiorze liczb rzeczywistych rozwiąz uklad równań

$\left\{\begin{array}{l}
x^{2}+9=4y\\
y^{2}+1=6z\\
z^{2}+4=2x.
\end{array}\right.$

ZADANIE 7.

Czworokąt ABCD jest wpisany w okrąg. Proste AB $\mathrm{i}$ CD przecinają się w punkcie $E$, a proste

{\it AD} $\mathrm{i}BC$ przecinają się w punkcie $F$. Udowodnij, $\dot{\mathrm{z}}\mathrm{e}$ jeśli $|BE|=|DF|$, to $|CE|=|CF|.$
\begin{center}
\includegraphics[width=57.816mm,height=55.164mm]{./LigaMatematycznaMatuskiego_Liceum_Zestaw5_2017_2018_page0_images/image001.eps}
\end{center}
{\it F}

{\it D}

C

{\it E}


\end{document}