\documentclass[a4paper,12pt]{article}
\usepackage{latexsym}
\usepackage{amsmath}
\usepackage{amssymb}
\usepackage{graphicx}
\usepackage{wrapfig}
\pagestyle{plain}
\usepackage{fancybox}
\usepackage{bm}

\begin{document}

LIGA MATEMATYCZNA

im. Zdzisława Matuskiego

FINAL

16 kwietnia 20l8

SZKOLA PONADGIMNAZJALNA

ZADANIE I.

$\mathrm{W}$ zbiorze liczb rzeczywistych rozwiąz równanie

$x^{2}-7[x]+6=0.$

ZADANIE 2.

Wykaz$\cdot, \dot{\mathrm{z}}\mathrm{e}$ kwadrat iloczynu dwóch kolejnych liczb calkowitych podzielnych przez 5 dzie1i się

przez 2500.

ZADANIE 3.

$\mathrm{W}$ zbiorze liczb rzeczywistych rozwiąz uklad równań

$\left\{\begin{array}{l}
a^{2}+b^{2}+c^{2}=ab+bc+ca\\
abc=8.
\end{array}\right.$

ZADANIE 4.

Wyznacz wszystkie trójki liczb pierwszych $p, q, r$ takie, $\dot{\mathrm{z}}\mathrm{e}$

$\displaystyle \frac{pqr}{p+q+r}=11.$

ZADANIE 5.

Symetralne ramion równoramiennego trójkąta rozwartokątnego dzielą podstawę na trzy równe

części. Oblicz miary kątów danego trójkąta.

ZADANIE 6.

Sprawd $\acute{\mathrm{z}}$, czy istnieją liczby calkowite $a, b, c$ spelniające równanie

$(9a-5b)(7b-3c)(5c-a)=20182019.$

Odpowied $\acute{\mathrm{z}}$ uzasadnij.

ZADANIE 7.

Cyfrą jedności pewnej liczby czterocyfrowej jest 5. $\mathrm{J}\mathrm{e}\dot{\mathrm{z}}$ eli tę cyfrę przeniesiemy z ostatniego

miejsca na pierwsze, to otrzymamy liczbę o 2277 większą od danej. Jaka to 1iczba?
\end{document}
