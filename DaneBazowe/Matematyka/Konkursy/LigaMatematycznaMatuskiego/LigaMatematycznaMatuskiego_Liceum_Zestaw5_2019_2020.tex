\documentclass[a4paper,12pt]{article}
\usepackage{latexsym}
\usepackage{amsmath}
\usepackage{amssymb}
\usepackage{graphicx}
\usepackage{wrapfig}
\pagestyle{plain}
\usepackage{fancybox}
\usepackage{bm}

\begin{document}
\begin{center}
\includegraphics[width=20.628mm,height=30.024mm]{./LigaMatematycznaMatuskiego_Liceum_Zestaw5_2019_2020_page0_images/image001.eps}
\end{center}
0

flkademia

P omorskawStupsku

LIGA MATEMATYCZNA

im. Zdzisława Matuskiego

FINAL 26 marca 2019

SZKOLA PONADPODSTAWOWA

ZADANIE I.

Wyznacz wszystkie funkcje rzeczywiste $f$: $\mathbb{R}\rightarrow \mathbb{R}$ spelniające równanie

$2f(x)+f(1-x)=x+7$

dla $\mathrm{k}\mathrm{a}\dot{\mathrm{z}}$ dej liczby rzeczywistej $x.$

ZADANIE 2.

Ania napisala kilka kolejnych liczb naturalnych. Wśród nich są trzy liczby pierwsze, trzy liczby

podzielne przez 3 i trzy 1iczby parzyste. I1e co najwyzej $\mathrm{m}\mathrm{o}\dot{\mathrm{z}}\mathrm{e}$ być równa suma liczb napisanych

przez Anię?

ZADANIE 3.

Znajd $\acute{\mathrm{z}}$ wszystkie trójki liczb rzeczywistych $(x,y,z)$ spelniające uklad równań

$\left\{\begin{array}{l}
x^{2}+y^{2}+z^{2}=14\\
x+2y+3z=14.
\end{array}\right.$

ZADANIE 4.

Czy liczba sześciocyfrowa, której cyframi są liczby 1, 2, 3, 4, 5, 6 (kazda $\mathrm{u}\dot{\mathrm{z}}$ yta jeden raz) $\mathrm{m}\mathrm{o}\dot{\mathrm{z}}\mathrm{e}$

być podzielna przez ll?

ZADANIE 5.

Pole trójkąta $ABC$ jest równe $p$. Odcinek $DE$ równolegfy do $AB$ odcina trójkąt o polu $q$. Niech

$F$ będzie dowolnym punktem $1\mathrm{e}\dot{\mathrm{Z}}$ acym na podstawie $AB$. Oblicz pole czworokąta DFEC.

ZADANIE 6.

Wykaz, $\dot{\mathrm{z}}\mathrm{e}\mathrm{j}\mathrm{e}\dot{\mathrm{z}}$ eli liczby dodatnie $a, b, c$ spelniają warunek $abc=1$, to

$\displaystyle \frac{1}{1+a^{2}b}+\frac{1}{1+bc^{2}}=1.$

ZADANIE 7.

Wyznacz wszystkie trójki liczb pierwszych $(p,q,r)$ spefniające uklad równań

$\left\{\begin{array}{l}
q=p^{2}+6\\
r=q^{2}+6.
\end{array}\right.$


\end{document}