\documentclass[a4paper,12pt]{article}
\usepackage{latexsym}
\usepackage{amsmath}
\usepackage{amssymb}
\usepackage{graphicx}
\usepackage{wrapfig}
\pagestyle{plain}
\usepackage{fancybox}
\usepackage{bm}

\begin{document}

flkademia

P omorskamStupsku

LIGA MATEMATYCZNA

im. Zdzisława Matuskiego

FINAL 20 kwietnia 2021

SZKOLA PONADPODSTAWOWA

ZADANIE I.

Wykaz, $\dot{\mathrm{z}}\mathrm{e}$ spośród dowolnych siedmiu liczb naturalnych $\mathrm{m}\mathrm{o}\dot{\mathrm{z}}$ na wybrać dwie liczby $a, b$ takie,

$\dot{\mathrm{z}}\mathrm{e}$ róznica $a^{2}-b^{2}$ jest podzielna przez 10.

ZADANIE 2.

$\mathrm{W}$ zbiorze liczb rzeczywistych rozwiąz uklad równań

$\left\{\begin{array}{l}
x^{2}+x(y-4)=-2\\
y^{2}+y(x-4)=-2.
\end{array}\right.$

ZADANIE 3.

$a, b, c, d, e$ są to liczby 7, 8, 9, 10, 11, a1e ustawione w innej, przypadkowej ko1ejności. Wykaz,

$\dot{\mathrm{z}}\mathrm{e}$ iloczyn $(a-1)(b-2)(c-3)(d-4)(e-5)$ jest liczba parzysta.

ZADANIE 4.

$\mathrm{W}$ wycinek kola o promieniu $R$ wpisano okrąg o promieniu $r$. Cięciwa lącząca końce promieni

wycinka kola ma dlugość $ 2\alpha$. Wykaz, $\displaystyle \dot{\mathrm{z}}\mathrm{e}\frac{1}{r}=\frac{1}{R}+\frac{1}{a}.$

ZADANIE 5.

Przyprostokątne trójkąta prostokatnego mają dlugości $a, b$. Wyznacz dlugość odcinka wyciętego

z dwusiecznej kąta prostego przez przeciwprostokątną.

ZADANIE 6.

Adam $\mathrm{u}\dot{\mathrm{z}}$ yf dwukrotnie $\mathrm{k}\mathrm{a}\dot{\mathrm{z}}$ dej z cyfr 1, 2, 3, $\ldots$, 9 i utworzyl kilka parami róznych liczb pierw-

szych w taki sposób, $\dot{\mathrm{z}}\mathrm{e}$ suma tych liczb jest $\mathrm{m}\mathrm{o}\dot{\mathrm{z}}$ liwie najmniejsza. Oblicz tę sumę.

ZADANIE 7.

Funkcja $f$: $\mathbb{R}\rightarrow \mathbb{R}$ spelnia warunek

$2f(x)+3f(\displaystyle \frac{2022}{x})=5x$

dla dowolnej dodatniej liczby rzeczywistej $x$. Oblicz $f(6).$


\end{document}