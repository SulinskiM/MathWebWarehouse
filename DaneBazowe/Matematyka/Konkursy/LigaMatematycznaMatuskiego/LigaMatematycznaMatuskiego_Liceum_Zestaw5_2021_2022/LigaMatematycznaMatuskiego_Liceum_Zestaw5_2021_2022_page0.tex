\documentclass[a4paper,12pt]{article}
\usepackage{latexsym}
\usepackage{amsmath}
\usepackage{amssymb}
\usepackage{graphicx}
\usepackage{wrapfig}
\pagestyle{plain}
\usepackage{fancybox}
\usepackage{bm}

\begin{document}

flkademia

P omorskamStupsku

LIGA MATEMATYCZNA

im. Zdzisława Matuskiego

FINAL 12 kwietnia 2022

SZKOLA PONADPODSTAWOWA

ZADANIE I.

Dane są dodatnie liczby calkowite $a, b, c, d, e, f$ takie, $\dot{\mathrm{z}}\mathrm{e}\mathrm{k}\mathrm{a}\dot{\mathrm{z}}$ da z sum $\alpha+b+c, b+c+d+e,$

$d+e+f$ jest liczbą nieparzystą. Uzasadnij, $\dot{\mathrm{z}}\mathrm{e}$ iloczyn abcdef jest liczba podzielną przez 4.

ZADANIE 2.

Kwadrat $K$ i prostokąt $P$, który nie jest kwadratem, mają równe pola. Która z tych figur

ma większy obwód? Odpowied $\acute{\mathrm{z}}$ uzasadnij.

ZADANIE 3.

Na okręgu umieszczono sześć liczb, których sumajest równa l. $K\mathrm{a}\dot{\mathrm{z}}$ da z nichjest równa wartości

bezwzględnej róznicy dwóch liczb następujących po niej, gdy poruszamy się po okręgu zgodnie

z ruchem wskazówek zegara. Wyznacz te liczby.

ZADANIE 4.

$\mathrm{W}$ trójkąt $ABC$ wpisano okrag i poprowadzono styczną do tego okręgu równoleglą do boku $AB,$

nie zawierającą tego boku. Oblicz długość odcinka stycznej zawartego w trójkącie w zalezności

od dlugości boków trójkąta.

ZADANIE 5.

Wykaz$\cdot, \dot{\mathrm{z}}\mathrm{e}$ liczba

111$\ldots$ 1777$\ldots$ 7111$\ldots$ 1$+$2022
\begin{center}
\includegraphics[width=52.428mm,height=6.396mm]{./LigaMatematycznaMatuskiego_Liceum_Zestaw5_2021_2022_page0_images/image001.eps}
\end{center}
n cyfr n cyfr  n cyfr

jest zfozona dla $\mathrm{k}\mathrm{a}\dot{\mathrm{z}}$ dej liczby naturalnej $n.$

ZADANIE 6.

Liczby 14, 20, $n$ spelniają warunek: iloczyn $\mathrm{k}\mathrm{a}\dot{\mathrm{z}}$ dych dwóch z nich jest podzielny przez trzecią

liczbę. Wyznacz wszystkie liczby całkowite $n$ spełniające tę własność.

ZADANIE 7.

$\mathrm{W}$ zbiorze liczb rzeczywistych rozwia $\dot{\mathrm{z}}$ uklad równań

$\left\{\begin{array}{l}
x^{2}+2y^{2}=2yz+100\\
z^{2}=2xy-100.
\end{array}\right.$
\end{document}
