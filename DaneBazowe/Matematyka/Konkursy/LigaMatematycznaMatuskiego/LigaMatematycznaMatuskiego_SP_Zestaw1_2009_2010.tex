\documentclass[a4paper,12pt]{article}
\usepackage{latexsym}
\usepackage{amsmath}
\usepackage{amssymb}
\usepackage{graphicx}
\usepackage{wrapfig}
\pagestyle{plain}
\usepackage{fancybox}
\usepackage{bm}

\begin{document}

LIGA MATEMATYCZNA

$\mathrm{P}\mathrm{A}\acute{\mathrm{Z}}$ DZIERNIK 2009

SZKOLA PODSTAWOWA

ZADANIE I.

Ania zgubila sześcienną kostkę do gry i samodzielnie wykonała inną kostkę w taki sposób,

$\dot{\mathrm{z}}\mathrm{e}$ sumy oczek na parach šcianek przeciwległych tworzą trzy kolejne liczby naturalne (w typowej

kostce do gry sumy oczek na ściankach przeciwleglych są równe). Okazalo się, $\dot{\mathrm{z}}\mathrm{e}$ suma oczek

na pewnych trzech ściankach majqcych wspólny wierzchołek jest równa 14. I1e oczek jest

na ściance przeciwleglej do ścianki z trzema oczkami?

ZADANIE 2.

Czy z jedenastu kwadratów o bokach l cm, l cm, 2 cm, 2 cm, 2 cm, 3 cm, 3 cm, 4 cm, 6 cm,

6 cm, 7 cm $\mathrm{m}\mathrm{o}\dot{\mathrm{z}}$ na zbudować kwadrat?

ZADANIE 3.

Dwóch uczniów rozwiązuje dwa rebusy w ciągu dwóch minut. Ile rebusów rozwiąze 10 uczniów

w ciągu 10 minut?

ZADANIE 4.

Piszemy liczbę 0, następnie 1iczbę 1 i znowu 1, potem piszemy najmniejszą z tych 1iczb ca1ko-

witych nieujemnych, która nie wystąpiła na trzech poprzednich miejscach. Dalej postępujemy

podobnie. Jaka liczba będzie na 99 miejscu?

ZADANIE 5.

Obwody zamalowanych prostokątów są równe 24 cm i 12 cm. Ob1icz obwód prostokąta ABCD.
\begin{center}
\includegraphics[width=48.816mm,height=33.168mm]{./LigaMatematycznaMatuskiego_SP_Zestaw1_2009_2010_page0_images/image001.eps}
\end{center}
D  C

A  B


\end{document}