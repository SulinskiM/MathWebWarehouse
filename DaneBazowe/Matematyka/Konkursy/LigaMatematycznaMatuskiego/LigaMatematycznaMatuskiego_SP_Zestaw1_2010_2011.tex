\documentclass[a4paper,12pt]{article}
\usepackage{latexsym}
\usepackage{amsmath}
\usepackage{amssymb}
\usepackage{graphicx}
\usepackage{wrapfig}
\pagestyle{plain}
\usepackage{fancybox}
\usepackage{bm}

\begin{document}

LIGA MATEMATYCZNA

$\mathrm{P}\mathrm{A}\acute{\mathrm{Z}}$ DZIERNIK 2010

SZKOLA PODSTAWOWA

ZADANIE I.

Ile jest liczb naturalnych o sumie cyfr w zapisie dziesiętnym równej 100 i i1oczynie tych cyfr

równym 5?

ZADANIE 2.

Dane są trzy figury: koło, kwadrat i trójkąt, róznej wielkości i w róznych kolorach: czerwonym,

zielonym, niebieskim. Kolo nie jest małe ani czerwone.

Trójkąt nie jest średni ani zielony.

Kwadrat nie jest duzy ani niebieski. Określ wielkość i kolor $\mathrm{k}\mathrm{a}\dot{\mathrm{z}}$ dej figury, jeśli wiadomo, $\dot{\mathrm{z}}\mathrm{e}$

mala figura jest niebieska.

ZADANIE 3.

$\mathrm{W}$ kole stanęfo 15 dziewcząt i 15 chfopców. Zaczynając od usta1onego miejsca nastąpi od1iczanie

do dziewięciu zgodnie z ruchem wskazówek zegara. Dziewiąta osoba odpada z gry, a odliczanie

będzie odbywać się dalej. Następnie kolejna dziewiąta osoba zostanie wykluczona z koła, i tak

dalej $\mathrm{a}\dot{\mathrm{z}}$ do momentu, gdy w kole zostanie 15 osób. Jak na1ezy ustawić ch1opców i dziewczęta,

aby wszystkie dziewczęta zostafy w kole?

ZADANIE 4.

Z siedmiu patyczków o dlugościach 3, 4, 6, 7, 9, 10, 11 u1óz prostokąt. Patyczków nie wo1no

łamać ani nakładać na siebie.

ZADANIE 5.

$\mathrm{Z}$ miejscowości $\mathrm{A}$, w której mieści się firma kurierska, wyjezdza kurier do miejscowości $\mathrm{B}, \mathrm{C},$

$\mathrm{D}$, aby dostarczyć przesyfki. $\mathrm{W}$ jakiej kolejności powinien objechać te miejscowości, aby trasa

objazdu z A przez pozostale miejscowości i z powrotem do A była $\mathrm{m}\mathrm{o}\dot{\mathrm{z}}$ liwie najkrótsza, jeśli

długošć drogi od A do $\mathrm{B}$ wynosi 50 km, od A do C-70 km, od A do D-70 km, od $\mathrm{B}$ do $\mathrm{D}-$

$80$ km, od $\mathrm{B}$ do C-100 km, od $\mathrm{C}$ do D-60 km?


\end{document}