\documentclass[a4paper,12pt]{article}
\usepackage{latexsym}
\usepackage{amsmath}
\usepackage{amssymb}
\usepackage{graphicx}
\usepackage{wrapfig}
\pagestyle{plain}
\usepackage{fancybox}
\usepackage{bm}

\begin{document}

LIGA MATEMATYCZNA

$\mathrm{P}\mathrm{A}\acute{\mathrm{Z}}$ DZIERNIK 2011

SZKOLA PODSTAWOWA

ZADANIE I.

Ile jest liczb dziesięciocyfrowych, które $\mathrm{m}\mathrm{o}\dot{\mathrm{z}}$ na napisač przy $\mathrm{u}\dot{\mathrm{z}}$ yciu cyfr 1, 2, 3 (nie trzeba wy-

korzystać wszystkich cyfr jednocześnie) tak, aby $\mathrm{k}\mathrm{a}\dot{\mathrm{z}}$ de dwie sąsiednie cyfry rózniły się ojeden?

ZADANIE 2.

$\mathrm{W}$ klasie jest 27 uczniów. $K\mathrm{a}\dot{\mathrm{z}}\mathrm{d}\mathrm{y}$ z nich uprawia przynajmniej jedną z trzech dyscyplin spor-

towych:

pilkę $\mathrm{n}\mathrm{o}\dot{\mathrm{z}}$ ną, plywanie lub tenis. Najwięcej uczniów uprawia pfywanie, a najmniej

tenis. $\mathrm{W}$ piłkę $\mathrm{n}\mathrm{o}\dot{\mathrm{z}}$ ną gra 15 uczniów. Ty1ko jeden uczeń uprawia jednocześnie trzy dyscyp1iny

sportu. Dwoje uprawia tenis i pifkę $\mathrm{n}\mathrm{o}\dot{\mathrm{z}}$ ną. Czworo uprawia plywanie i pifkę $\mathrm{n}\mathrm{o}\dot{\mathrm{z}}$ ną, troje- tenis

i pfywanie. Ilu uczniów uprawia pfywanie? Ilu uczniów uprawia tylko tenis?

ZADANIE 3.

Rozlozono sto cukierków na pięciu talerzach:

$\bullet$ na pierwszym i drugim talerzu znalazfy się 52 cukierki;

$\bullet$ na drugim i trzecim talerzu 43 cukierki;

$\bullet$ na trzecim i czwartym- 34 cukierki;

$\bullet$ na czwartym i piatym- 30 cukierków.

Ile cukierków znajdowafo się na $\mathrm{k}\mathrm{a}\dot{\mathrm{z}}$ dym talerzu?

ZADANIE 4.

$\mathrm{W}$ schronisku dla zwierząt byfa taka sama liczba psów i kotów. Trzecia część psów i pofowa

kotów znalazla opiekunów. Po sześć psów ijednego kota zgloszą się wfaściciele i wtedy w schro-

nisku będzie więcej kotów $\mathrm{n}\mathrm{i}\dot{\mathrm{z}}$ psów. Ile psów mogfo znajdować się w schronisku na początku?

Podaj wszystkie $\mathrm{m}\mathrm{o}\dot{\mathrm{z}}$ liwości.

ZADANIE 5.

Prostokąt podzielono na siedem kwadratów. Bok $\mathrm{k}\mathrm{a}\dot{\mathrm{z}}$ dego z zaciemnionych kwadratów ma dlu-

gość 8. Jaka d1ugość ma bok największego kwadratu?
\end{document}
