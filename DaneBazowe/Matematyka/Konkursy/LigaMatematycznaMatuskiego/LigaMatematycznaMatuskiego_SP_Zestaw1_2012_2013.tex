\documentclass[a4paper,12pt]{article}
\usepackage{latexsym}
\usepackage{amsmath}
\usepackage{amssymb}
\usepackage{graphicx}
\usepackage{wrapfig}
\pagestyle{plain}
\usepackage{fancybox}
\usepackage{bm}

\begin{document}

LIGA MATEMATYCZNA

im. Zdzisława Matuskiego

$\mathrm{P}\mathrm{A}\acute{\mathrm{Z}}$ DZIERNIK 2012

SZKOLA PODSTAWOWA

ZADANIE I.

Sloń wazy tyle, ile dwa nosorozce, nosorozec tyle, ile dwa nied $\acute{\mathrm{z}}\mathrm{w}\mathrm{i}\mathrm{e}\mathrm{d}\mathrm{z}\mathrm{i}\mathrm{e}, \mathrm{n}\mathrm{i}\mathrm{e}\mathrm{d}\acute{\mathrm{z}}\mathrm{w}\mathrm{i}\mathrm{e}\mathrm{d}\acute{\mathrm{z}}$ tyle, ile dwa

sumy, sum wazy tyle, ile dwa tygrysy, tygrys tyle, ile dwa strusie, struś tyle, ile dwie sarny,

sarna tyle, ile dwa borsuki, borsuk tyle, ile dwa lisy, lis tyle, ile dwa zające. Sloń wazy o 6, 25 kg

więcej $\mathrm{n}\mathrm{i}\dot{\mathrm{z}}$ w sumie nosorozec, nied $\acute{\mathrm{z}}\mathrm{w}\mathrm{i}\mathrm{e}\mathrm{d}\acute{\mathrm{z}}$, sum, tygrys, struś, sarna, borsuk, lis i zając. Ile wazy

słoń?

ZADANIE 2.

Kwadrat o boku długości 9 cm rozetnij na trzy prostokąty o obwodach 20 cm, 24 cm i 28 cm.

ZADANIE 3.

Do zapisania pewnej liczby dziesięciocyfrowej uzytojednej jedynki, dwóch dwójek, trzech trójek

i czterech czwórek. Rozmieszczenie cyfr jest nieznane. Czy $\mathrm{m}\mathrm{o}\dot{\mathrm{z}}\mathrm{e}$ to być liczba pierwsza?

ZADANIE 4.

Pierwszy ślimak potrafi przejšć 3 metry w ciągu czterech minut, a drugi-4 metry w trzy minuty.

W tym samym momencie wyszli z tego samego miejsca odleglego o ll metrów od stacji leśnej

kolejki. Czy obaj zdązą, jeśli do odjazdu pociągu zostalo 13 minut?

ZADANIE 5.

Na odcinku $AB$ zaznaczono punkty $C, D$. Odleglość punktu $C$ od jednego z końców danego

odcinka stanowi $\displaystyle \frac{5}{6}$ jego długości, a odleglość punktu $D$ od jednego z końców- $\displaystyle \frac{3}{4}$ długości tego

odcinka. Wiedząc, $\dot{\mathrm{z}}\mathrm{e}$ dlugość odcinka $CD$ jest równa 35 cm, ob1icz d1ugość odcinka $AB.$

Rozwaz wszystkie $\mathrm{m}\mathrm{o}\dot{\mathrm{z}}$ liwości.


\end{document}