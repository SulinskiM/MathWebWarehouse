\documentclass[a4paper,12pt]{article}
\usepackage{latexsym}
\usepackage{amsmath}
\usepackage{amssymb}
\usepackage{graphicx}
\usepackage{wrapfig}
\pagestyle{plain}
\usepackage{fancybox}
\usepackage{bm}

\begin{document}

LIGA MATEMATYCZNA

im. Zdzisława Matuskiego

$\mathrm{P}\mathrm{A}\overline{\mathrm{Z}}$ DZIERNIK 2013

SZKOLA PODSTAWOWA

ZADANIE I.

Jaka największa, a jaką najmniejszą liczbę trzycyfrową $\mathrm{m}\mathrm{o}\dot{\mathrm{z}}$ emy otrzymać z liczby 18094015

przez wykreślenie pięciu cyfr bez zmiany ich porządku?

ZADANIE 2.

Jaką liczbę nalezy wpisać w górne pole, $\mathrm{j}\mathrm{e}\dot{\mathrm{z}}$ eli liczba w $\mathrm{k}\mathrm{a}\dot{\mathrm{z}}$ dym polu w rzędzie $\mathrm{w}\mathrm{y}\dot{\mathrm{z}}$ szym jest

iloczynem dwóch liczb z pól $\mathrm{n}\mathrm{i}\dot{\mathrm{z}}$ szego rzędu sąsiadujących z nim?
\begin{center}
\includegraphics[width=31.140mm,height=28.296mm]{./LigaMatematycznaMatuskiego_SP_Zestaw1_2013_2014_page0_images/image001.eps}
\end{center}
5 4

5 3

ZADANIE 3.

Bartek, Maciek i Tomek łowili ryby. Zapytani o to, ile ryb zlowili, odpowiedzieli:

$\bullet$ Bartek zlowif 22 ryby, Maciek 21;

$\bullet$ Tomek zlowif 19 ryb, a Bartek 21;

$\bullet$ Tomek zlowi121 ryb, Maciek 18.

Wiadomo, $\dot{\mathrm{z}}\mathrm{e}$ w $\mathrm{k}\mathrm{a}\dot{\mathrm{z}}$ dej odpowiedzi tylko jedna część jest prawdziwa oraz $\dot{\mathrm{z}}$ adni dwaj nie zlowili

tej samej ilości ryb. Ile ryb zfowil $\mathrm{k}\mathrm{a}\dot{\mathrm{z}}\mathrm{d}\mathrm{y}$ chlopiec?

ZADANIE 4.

Z prostokąta ABCD o obwodzie 30 cm wycięto trójkat równoboczny AOD o obwodzie 15 cm.

Oblicz obwód figury ABCDO.
\begin{center}
\includegraphics[width=60.708mm,height=30.024mm]{./LigaMatematycznaMatuskiego_SP_Zestaw1_2013_2014_page0_images/image002.eps}
\end{center}
D  c

0

A  B

ZADANIE 5.

Na lawce siedzi Ania, jej mama, babcia i dziadek. Babcia siedzi obok Ani, ale nie siedzi obok

dziadka. Dziadek nie siedzi obok mamy Ani. Kto siedzi obok mamy?


\end{document}