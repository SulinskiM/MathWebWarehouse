\documentclass[a4paper,12pt]{article}
\usepackage{latexsym}
\usepackage{amsmath}
\usepackage{amssymb}
\usepackage{graphicx}
\usepackage{wrapfig}
\pagestyle{plain}
\usepackage{fancybox}
\usepackage{bm}

\begin{document}

LIGA MATEMATYCZNA

im. Zdzisława Matuskiego

$\mathrm{P}\mathrm{A}\overline{\mathrm{Z}}$ DZIERNIK 2014

SZKOLA PODSTAWOWA

ZADANIE I.

Podczas kolejnych lotów trzej piloci spotkali się w Paryz $\mathrm{u}6$ września 2014 roku. Wiadomo,

$\dot{\mathrm{z}}\mathrm{e}$ pierwszy pilot lata do Paryza co 7 dni, drugi co 14, a trzeci co 5 dni. Wyznacz datę ich

kolejnego spotkania w Paryzu.

ZADANIE 2.

Pifka $\mathrm{n}\mathrm{o}\dot{\mathrm{z}}$ na i golfowa wazą razem tyle, ile fącznie wazą pilka bejsbolowa i tenisowa. Trzy pifki

golfowe wazą tyle, ile pilka bejsbolowa i tenisowa. Pilka bejsbolowa $\mathrm{w}\mathrm{a}\dot{\mathrm{z}}\mathrm{y}$ tyle, ile osiem piłek

tenisowych. Ile pifek tenisowych $\mathrm{w}\mathrm{a}\dot{\mathrm{z}}\mathrm{y}$ tyle, ile jedna pilka $\mathrm{n}\mathrm{o}\dot{\mathrm{z}}$ na?

ZADANIE 3.

Wyznacz cyfry a, b tak, aby liczba sześciocyfrowa a2479b by1a podzie1na przez 72.

ZADANIE 4.

Na $\mathrm{k}\mathrm{a}\dot{\mathrm{z}}$ dej ze ścian sześcianu zapisano jedną z liczb 1, 2, 3, 4, 5, 6 w taki sposób, $\dot{\mathrm{z}}\mathrm{e}$ suma liczb

na $\mathrm{k}\mathrm{a}\dot{\mathrm{z}}$ dej parze przeciwleglych ścian jest równa 7. $\mathrm{Z}\mathrm{k}\mathrm{a}\dot{\mathrm{z}}$ dym wierzchofkiem sześcianu związane

sa trzy ściany. Liczby zapisane na tych ścianach mnozymy. Oblicz sumę tych iloczynów.

ZADANIE 5.

Prostokąt podzielono na dziewięć mniejszych prostokątów. Obwody pięciu z nich podane sa na

rysunku. Oblicz obwód $\mathrm{d}\mathrm{u}\dot{\mathrm{z}}$ ego prostokąta.
\begin{center}
\begin{tabular}{|l|l|l|}
\hline
\multicolumn{1}{|l|}{}&	\multicolumn{1}{|l|}{$6$}&	\multicolumn{1}{|l|}{}	\\
\hline
\multicolumn{1}{|l|}{ $12$}&	\multicolumn{1}{|l|}{ $4$}&	\multicolumn{1}{|l|}{ $6$}	\\
\hline
\multicolumn{1}{|l|}{}&	\multicolumn{1}{|l|}{ $8$}&	\multicolumn{1}{|l|}{}	\\
\hline
\end{tabular}

\end{center}

\end{document}