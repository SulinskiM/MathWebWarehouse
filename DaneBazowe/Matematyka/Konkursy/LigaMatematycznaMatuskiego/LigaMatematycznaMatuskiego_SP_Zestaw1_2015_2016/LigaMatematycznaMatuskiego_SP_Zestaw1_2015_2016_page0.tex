\documentclass[a4paper,12pt]{article}
\usepackage{latexsym}
\usepackage{amsmath}
\usepackage{amssymb}
\usepackage{graphicx}
\usepackage{wrapfig}
\pagestyle{plain}
\usepackage{fancybox}
\usepackage{bm}

\begin{document}

LIGA MATEMATYCZNA

im. Zdzisława Matuskiego

$\mathrm{P}\mathrm{A}\dot{\mathrm{Z}}$ DZIERNIK 2015

SZKOLA PODSTAWOWA

ZADANIE I.

Pole trapezu, którego jedna podstawa jest dwa razy dluzsza od drugiej, jest równe 840 $\mathrm{c}\mathrm{m}^{2}$

Oblicz pola trójkątów, na jakie podzielila ten trapez jedna z przekątnych.

ZADANIE 2.

Wykaz, $\dot{\mathrm{z}}\mathrm{e}$ liczba 4777$\ldots$ 75 jest podzielna przez 45.
\begin{center}
\includegraphics[width=17.676mm,height=6.852mm]{./LigaMatematycznaMatuskiego_SP_Zestaw1_2015_2016_page0_images/image001.eps}
\end{center}
90 cyfr 7

ZADANIE 3.

$K\mathrm{a}\dot{\mathrm{z}}\mathrm{d}\mathrm{y}$ ufoludek ma trzy ręce. Czy 19 ufo1udków $\mathrm{m}\mathrm{o}\dot{\mathrm{z}}\mathrm{e}$ się wziąć za ręce tak, aby $\dot{\mathrm{z}}$ adna ręka

nie pozostala wolna?

ZADANIE 4.

Wszyskie figury skfadające się na $\mathrm{d}\mathrm{u}\dot{\mathrm{z}}\mathrm{y}$ prostokat są kwadratami. Pole czarnego kwadratu jest

równe 9. Ob1icz d1ugości boków $\mathrm{k}\mathrm{a}\dot{\mathrm{z}}$ dego kwadratu i dlugości boków $\mathrm{d}\mathrm{u}\dot{\mathrm{z}}$ ego prostokąta.

ZADANIE 5.

Na terenie wokól jeziora pan Piotr urządzif pole namiotowe dzieląc teren na sześć sektorów, jak

na rysunku. Pewnej niedzieli przyjechalo nadjezioro sześciu wędkarzy, którzy chcieli zamieszkać

na polu namiotowym, ale mieli następujące wymagania:

$\bullet$ Andrzej nie chce sąsiadować ani z Frankiem, ani z Bartkiem;

$\bullet$ Bartek nie chce sąsiadować ani z Andrzejem, ani z Czarkiem;

$\bullet$ Czarek nie chce sąsiadować ani z Bartkiem, ani z Damianem;

$\bullet$ Damian nie chce sąsiadować ani z Czarkiem, ani z Emilem;

$\bullet$ Emil nie chce sasiadować ani z Damianem, ani z Frankiem;

$\bullet$ Franek nie chce sąsiadować ani z Emilem, ani z Andrzejem.

Jak rozmieścić wędkarzy, aby spelnić ich $\dot{\mathrm{z}}$ yczenia?
\end{document}
