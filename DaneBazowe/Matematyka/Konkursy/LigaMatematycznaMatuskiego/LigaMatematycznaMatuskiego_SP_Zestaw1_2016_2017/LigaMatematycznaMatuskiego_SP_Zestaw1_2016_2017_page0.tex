\documentclass[a4paper,12pt]{article}
\usepackage{latexsym}
\usepackage{amsmath}
\usepackage{amssymb}
\usepackage{graphicx}
\usepackage{wrapfig}
\pagestyle{plain}
\usepackage{fancybox}
\usepackage{bm}

\begin{document}

LIGA MATEMATYCZNA

im. Zdzisława Matuskiego

$\mathrm{P}\mathrm{A}\dot{\mathrm{Z}}$ DZIERNIK 2016

SZKOLA PODSTAWOWA

ZADANIE I.

Ile jest dziesięciocyfrowych liczb nieparzystych o sumie cyfr równej 3?

ZADANIE 2.

Chlopcy i dziewczynki z klasy Ani i Bartka ustawili się wjednej linii. Na prawo od Ani jest 16

uczniów, w tym Bartek. Na lewo od Bartka jest 14 uczniów, wśród nich Ania. Pomiędzy Anią

i Bartkiem stoi 7 uczniów. I1u uczniów 1iczy ta k1asa?

ZADANIE 3.

Spośród boków czworokąta wybrano trzy i obliczono sumę ich dlugości. Taką operację prze-

prowadzono czterokrotnie, wybierając za $\mathrm{k}\mathrm{a}\dot{\mathrm{z}}$ dym razem trzy inne boki. Otrzymano sumy: 10,

13, 15, 16. Oblicz dlugości boków tego czworokata.

ZADANIE 4.

Ania, Beata, Celina i Dorota wybraly się na grzyby. Ania zebrafa trzy razy więcej grzybów

$\mathrm{n}\mathrm{i}\dot{\mathrm{z}}$ Beata, Beata trzy razy więcej $\mathrm{n}\mathrm{i}\dot{\mathrm{z}}$ Celina, Celina trzy razy więcej $\mathrm{n}\mathrm{i}\dot{\mathrm{z}}$ Dorota. Wiadomo, $\dot{\mathrm{z}}\mathrm{e}$

razem mają więcej $\mathrm{n}\mathrm{i}\dot{\mathrm{z}}50$, ale mniej $\mathrm{n}\mathrm{i}\dot{\mathrm{z}}100$ grzybów. Ile grzybów zebrafa $\mathrm{k}\mathrm{a}\dot{\mathrm{z}}$ da z dziewczynek?

ZADANIE 5.

Uzupelnij krzyzówkę tak, aby otrzymane liczby trzycyfrowe dzielily się przez podane liczby.
\begin{center}
\begin{tabular}{|l|l|l|l|}
\hline
\multicolumn{1}{|l|}{}&	\multicolumn{1}{|l|}{przez 11}&	\multicolumn{1}{|l|}{przez 7}&	\multicolumn{1}{|l|}{przez 2}	\\
\hline
\multicolumn{1}{|l|}{przez 7}&	\multicolumn{1}{|l|}{$7$}&	\multicolumn{1}{|l|}{ $1$}&	\multicolumn{1}{|l|}{}	\\
\hline
\multicolumn{1}{|l|}{przez 9}&	\multicolumn{1}{|l|}{$8$}&	\multicolumn{1}{|l|}{ $5$}&	\multicolumn{1}{|l|}{}	\\
\hline
\multicolumn{1}{|l|}{przez 5}&	\multicolumn{1}{|l|}{}&	\multicolumn{1}{|l|}{}&	\multicolumn{1}{|l|}{}	\\
\hline
\end{tabular}

\end{center}\end{document}
