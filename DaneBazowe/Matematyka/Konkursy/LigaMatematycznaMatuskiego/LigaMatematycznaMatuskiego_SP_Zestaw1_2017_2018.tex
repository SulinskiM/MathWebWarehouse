\documentclass[a4paper,12pt]{article}
\usepackage{latexsym}
\usepackage{amsmath}
\usepackage{amssymb}
\usepackage{graphicx}
\usepackage{wrapfig}
\pagestyle{plain}
\usepackage{fancybox}
\usepackage{bm}

\begin{document}

LIGA MATEMATYCZNA

im. Zdzisława Matuskiego

$\mathrm{P}\mathrm{A}\dot{\mathrm{Z}}$ DZIERNIK 2017

SZKOLA PODSTAWOWA

ZADANIE I.

Czy liczbę 7777777 $\mathrm{m}\mathrm{o}\dot{\mathrm{z}}$ na przedstawić jako sumę dwóch liczb pierwszych?

ZADANIE 2.

Oblicz pole figury przedstawionej na rysunku, gdzie odpowiednie kąty sa proste.
\begin{center}
\includegraphics[width=138.936mm,height=44.964mm]{./LigaMatematycznaMatuskiego_SP_Zestaw1_2017_2018_page0_images/image001.eps}
\end{center}
5  7  9

4  5

6

8  2 3

1  7

5  4

3

2  6  8  2

ZADANIE 3.

Pan Adamjest staruszkiem, ale nie majeszcze stu lat. Dla treningu lubi układač i rozwiazywać

lamiglówki liczbowe. Wfaśnie zauwazyl, $\dot{\mathrm{z}}\mathrm{e}$ w ubieglym roku jego wiek byl liczbą calkowitą

podzielną przez 8, a w przyszłym roku będzie 1iczbą podzie1ną przez 7. I1e 1at ma pan Adam

obecnie?

ZADANIE 4.

Liczbą palindromiczną nazywamy liczbę, która czytana od lewej do prawej oraz od prawej do

lewej jest taka sama. Ile jest liczb palindromicznych trzycyfrowych podzielnych przez 4?

ZADANIE 5.

Wewnątrz pięciokąta foremnego ABCDE obrano punkt $F$ w taki sposób, $\dot{\mathrm{z}}\mathrm{e}$ trójkąt $ABF$ jest

równoboczny. Oblicz miarę kata $CFE.$
\begin{center}
\includegraphics[width=48.468mm,height=46.884mm]{./LigaMatematycznaMatuskiego_SP_Zestaw1_2017_2018_page0_images/image002.eps}
\end{center}
{\it D}

{\it E}

{\it F c}

{\it A}

{\it B}


\end{document}