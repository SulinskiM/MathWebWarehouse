\documentclass[a4paper,12pt]{article}
\usepackage{latexsym}
\usepackage{amsmath}
\usepackage{amssymb}
\usepackage{graphicx}
\usepackage{wrapfig}
\pagestyle{plain}
\usepackage{fancybox}
\usepackage{bm}

\begin{document}

LIGA MATEMATYCZNA

im. Zdzisława Matuskiego

$\mathrm{P}\mathrm{A}\dot{\mathrm{Z}}$ DZIERNIK 2018

SZKOLA PODSTAWOWA

(klasy IV - VI)

ZADANIE I.

$\mathrm{W}\mathrm{k}\mathrm{a}\dot{\mathrm{z}}$ dym wierzcholku trójkąta umieszczono pewną liczbę, a na $\mathrm{k}\mathrm{a}\dot{\mathrm{z}}$ dym boku - sumę liczb

z obu jego końców. Znajdz' liczby zapisane w wierzchołkach, $\mathrm{j}\mathrm{e}\dot{\mathrm{z}}$ eli na bokach znajdowały się

liczby 1256, 1820, 2018.

ZADANIE 2.

Trzej bracia (kazdy $\mathrm{w}\mathrm{a}\dot{\mathrm{z}}\mathrm{y}120$ kg), $\mathrm{k}\mathrm{a}\dot{\mathrm{z}}\mathrm{d}\mathrm{y}$ z $\dot{\mathrm{z}}$ oną (wazącą 60 kg) i dzieckiem (o wadze 30 kg) chcą

przeprawić się przez rzekę. Na brzegu znale $\acute{\mathrm{z}}\mathrm{l}\mathrm{i}$ lódkę o ladowności 120 kg. I1e co najmniej razy

lódka będzie musiala pokonać drogę od jednego brzegu do drugiego, aby cafa rodzina (dziewięć

osób) znalazla się na drugim brzegu rzeki? Podczas $\mathrm{k}\mathrm{a}\dot{\mathrm{z}}$ dej przeprawy w lódce musi znajdować

się przynajmniej jedna osoba dorosfa.

ZADANIE 3.

Na okręgu umieszczono cztery liczby: 2, 5, 7, 8 (w tej ko1ejności). Ruch po1ega na wstawieniu

między $\mathrm{k}\mathrm{a}\dot{\mathrm{z}}$ dą parę sasiednich liczb ich dodatniej róznicy, a następnie wymazaniu wszystkich

starych liczb. Po ilu ruchach po raz pierwszy otrzymamy same zera?

ZADANIE 4.

Znajd $\acute{\mathrm{z}}$ wszystkie liczby dwucyfrowe, których iloczyn cyfr jest liczbą pierwszą.

ZADANIE 5.

Na ponizszym rysunku przedstawiono dziesięciokąt, w którym $\mathrm{k}\mathrm{a}\dot{\mathrm{z}}$ de dwa sąsiednie boki są

prostopadfe. Dfugości niektórych boków zostafy podane. Oblicz obwód dziesięciokąta.
\begin{center}
\includegraphics[width=100.740mm,height=55.116mm]{./LigaMatematycznaMatuskiego_SP_Zestaw1_2018_2019_page0_images/image001.eps}
\end{center}
1000

52

83

45

2018
\end{document}
