\documentclass[a4paper,12pt]{article}
\usepackage{latexsym}
\usepackage{amsmath}
\usepackage{amssymb}
\usepackage{graphicx}
\usepackage{wrapfig}
\pagestyle{plain}
\usepackage{fancybox}
\usepackage{bm}

\begin{document}

LIGA MATEMATYCZNA

LISTOPAD 2009

SZKOLA PODSTAWOWA

ZADANIE I.

Agnieszka, Michal, Romek, Adam i Jacek poszli do lasu na grzyby. Ale grzyby zbierala tylko

Agnieszka, chłopcom nie chciało się trudzić. Wracając do domu, dziewczynka wszystkie zna-

lezione grzyby - miala ich $42 -$ rozdzielila między chłopców. $\mathrm{W}$ drodze powrotnej Michał

znalazł jeszcze dwa grzyby, Romek zgubil dwa grzyby, Adam znalazł $\mathrm{a}\dot{\mathrm{z}}$ połowę tej ilošci grzy-

bów, którą mial w koszyku, za to Jacek zgubil polowę swoich grzybów. Po powrocie do domu

chłopcy policzyli grzyby i okazało się, $\dot{\mathrm{z}}\mathrm{e}\mathrm{k}\mathrm{a}\dot{\mathrm{z}}\mathrm{d}\mathrm{y}$ z nich miał ich jednakowq ilość. Ile grzybów

$\mathrm{k}\mathrm{a}\dot{\mathrm{z}}\mathrm{d}\mathrm{y}$ z chłopców dostal od Agnieszki?

ZADANIE 2.

Mamy 24 beczki ojednakowej objętości. Pięć z nich jest pe1nych wody, jedenaście napełnionych

do połowy, a osiem pustych. Jaką maksymalną ilość osób $\mathrm{m}\mathrm{o}\dot{\mathrm{z}}$ na obdzielić, nie przelewając wody,

tak, aby $\mathrm{k}\mathrm{a}\dot{\mathrm{z}}$ da dostafa jednakową ilość wody i tę samą ilość beczek?

ZADANIE 3.

W szkolnych zawodach sportowych Adam startowaf w skoku w dal i w końcowej klasyfikacji

zajqł siódme miejsce. Jego kolega Marcin miał dalsze skoki i ostatecznie zająl miejsce dokladnie

w środku tabeli wyników (tzn. wyprzedzało go tylu zawodników, ilu on poprzedzal).

Inny

kolega Adama mial gorsze wyniki i zajqł dziesiqte miejsce w tabeli. Ilu chlopców startowalo

w skoku w dal?

ZADANIE 4.

Wpisz liczby w miejsce liter tak, aby zachodzify wszystkie równości. Róznym literom odpowia-

dają rózne liczby.

$M\cdot A=T-E=M$: $A=T$: $Y=K-A.$

ZADANIE 5.

Trzy rózne cyfry $\mathrm{m}\mathrm{o}\dot{\mathrm{z}}$ na wpisać do pustych pól diagramu na sześć róznych sposobów. Jakie

to powinny być cyfry, aby w $\mathrm{k}\mathrm{a}\dot{\mathrm{z}}$ dym z tych sześciu przypadków otrzymać pięciocyfrową liczbę

podzielną przez 12?

1 2


\end{document}