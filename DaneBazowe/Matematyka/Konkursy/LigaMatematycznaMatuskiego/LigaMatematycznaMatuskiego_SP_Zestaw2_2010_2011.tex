\documentclass[a4paper,12pt]{article}
\usepackage{latexsym}
\usepackage{amsmath}
\usepackage{amssymb}
\usepackage{graphicx}
\usepackage{wrapfig}
\pagestyle{plain}
\usepackage{fancybox}
\usepackage{bm}

\begin{document}

LIGA MATEMATYCZNA

LISTOPAD 2010

SZKOLA PODSTAWOWA

ZADANIE I.

Jakie wymiary, będące liczbami calkowitymi, powinna mieć prostokątna kartka papieru o polu

powierzchni 112 $\mathrm{c}\mathrm{m}^{2}$, aby $\mathrm{m}\mathrm{o}\dot{\mathrm{z}}$ na bylo z niej wyciąć jak najwięcej kwadratów o wymiarach

całkowitych i róznych polach?

ZADANIE 2.

Podziel kwadrat $4\times 4$ przedstawiony na rysunku na cztery jednakowe części tak, aby $\mathrm{k}\mathrm{a}\dot{\mathrm{z}}$ da

litera była w innej części.
\begin{center}
\includegraphics[width=33.276mm,height=33.228mm]{./LigaMatematycznaMatuskiego_SP_Zestaw2_2010_2011_page0_images/image001.eps}
\end{center}
A

C D

ZADANIE 3.

$\mathrm{W}$ klasie IV bjest 29 uczniów. 18 uczniów ma brata, 17 uczniów ma siostrę. Ty1ko Zosia, Micha1

i Tomek nie mają $\dot{\mathrm{z}}$ adnego rodzeństwa. Ilu uczniów ma brata i siostrę? $(\dot{\mathrm{Z}}$ aden z uczniów nie

ma więcej $\mathrm{n}\mathrm{i}\dot{\mathrm{z}}$ dwie osoby rodzeństwa.)

ZADANIE 4.

Ala, Ela, Jola, Ola, Tola i Ula mieszkają w czteropiętrowym bloku. Ala mieszka $\mathrm{w}\mathrm{y}\dot{\mathrm{z}}$ ej $\mathrm{n}\mathrm{i}\dot{\mathrm{z}}$ Ela,

ale $\mathrm{n}\mathrm{i}\dot{\mathrm{z}}$ ej $\mathrm{n}\mathrm{i}\dot{\mathrm{z}}$ Jola. Ola i Tola mieszkają $\mathrm{n}\mathrm{i}\dot{\mathrm{z}}$ ej $\mathrm{n}\mathrm{i}\dot{\mathrm{z}}$ Ula. Ola mieszka $\mathrm{w}\mathrm{y}\dot{\mathrm{z}}$ ej $\mathrm{n}\mathrm{i}\dot{\mathrm{z}}$ Ala, a Tola $\mathrm{w}\mathrm{y}\dot{\mathrm{z}}$ ej

$\mathrm{n}\mathrm{i}\dot{\mathrm{z}}$ Jola. Która z dziewczynek mieszka na pierwszym piętrze?

ZADANIE 5.

Uzupelnij puste kratki liczbami w taki sposób, aby suma $\mathrm{k}\mathrm{a}\dot{\mathrm{z}}$ dej czwórki kolejnych liczb była

równa 20.


\end{document}