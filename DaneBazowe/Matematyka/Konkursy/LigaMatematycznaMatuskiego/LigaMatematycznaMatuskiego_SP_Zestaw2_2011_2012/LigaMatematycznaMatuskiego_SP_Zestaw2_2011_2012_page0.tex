\documentclass[a4paper,12pt]{article}
\usepackage{latexsym}
\usepackage{amsmath}
\usepackage{amssymb}
\usepackage{graphicx}
\usepackage{wrapfig}
\pagestyle{plain}
\usepackage{fancybox}
\usepackage{bm}

\begin{document}

LIGA MATEMATYCZNA

LISTOPAD 2011

SZKOLA PODSTAWOWA

ZADANIE I.

Automat matematyczny dziala na następującej zasadzie: do danej liczby dodaje jeden lub

ją podwaja. Do automatu wprowadzono liczbę 0. Ten po wykonaniu pewnej 1iczby operacji

otrzymał liczbę 100. Jaka jest najmniejsza i1ość operacji, które musi wykonać automat, aby

otrzymač taki wynik?

ZADANIE 2.

$\mathrm{W}$ smoczej jamie $\dot{\mathrm{z}}$ yfy smoki czerwone i smoki zielone. $K\mathrm{a}\dot{\mathrm{z}}\mathrm{d}\mathrm{y}$ czerwony smok mial sześć gfów,

osiem nóg i dwa ogony, natomiast zielony smok mial osiem gfów, sześć nóg i cztery ogony.

Wszystkich ogonów byfo 44, a zie1onych nóg o 6 mniej $\mathrm{n}\mathrm{i}\dot{\mathrm{z}}$ czerwonych glów. Ile czerwonych

smoków $\dot{\mathrm{z}}$ ylo w tej jamie?

ZADANIE 3.

Jaś pomyślal pewną liczbę naturalną, pomnozyl ją przez 13, odrzuci1 ostatnią cyfrę wyniku,

otrzymaną liczbę pomnozyf przez 7, znów odrzuci1 ostatnią cyfrę wyniku i otrzyma121. Jaką

liczbę pomyślaf Jasiu?

ZADANIE 4.

Ile róznych prostokątów $\mathrm{m}\mathrm{o}\dot{\mathrm{z}}$ na zbudować z patyczków o dlugościach 3 cm, 5 cm, 8 cm, 10 cm,

ll cm, 13 cm, 14 cm? Patyczki tworzace boki prostokąta nie mogą zachodzić na siebie i nie

$\mathrm{m}\mathrm{o}\dot{\mathrm{z}}$ na ich łamać. Za $\mathrm{k}\mathrm{a}\dot{\mathrm{z}}$ dym razem nalezy wykorzystać wszystkie patyczki.

ZADANIE 5.

Do sklepu przywieziono 223 kg cukierków w pojemnikach 10 kg i 17 kg. I1e by1o pojemników?
\end{document}
