\documentclass[a4paper,12pt]{article}
\usepackage{latexsym}
\usepackage{amsmath}
\usepackage{amssymb}
\usepackage{graphicx}
\usepackage{wrapfig}
\pagestyle{plain}
\usepackage{fancybox}
\usepackage{bm}

\begin{document}

LIGA MATEMATYCZNA

im. Zdzisława Matuskiego

LISTOPAD 2012

SZKOLA PODSTAWOWA

ZADANIE I.

Pasterz Matmek wędrując ze stadem 10 owiec trafił na most strzezony przez straznika Kwa-

dratko. Przepuszczał on ludzi przez most za darmo, a za owce pobierał opłatę w dukatach,

których ilošć musiafa być równa liczbie owiec podniesionej do kwadratu. Ubogi pasterz nie

mial stu dukatów, ale wytargowal $\mathrm{u}$ straznika, $\dot{\mathrm{z}}\mathrm{e}$ będzie przeprowadzał po kilka owiec, płacąc

za $\mathrm{k}\mathrm{a}\dot{\mathrm{z}}$ dą część stada oddzielnie. Straznik zgodzil się na to pod warunkiem, $\dot{\mathrm{z}}\mathrm{e}$ stado będzie

podzielone nie więcej $\mathrm{n}\mathrm{i}\dot{\mathrm{z}}$ na trzy części. $\mathrm{W}$ jaki sposób Matmek ma przeprowadzić owce przez

most, aby zapłacić jak najmniej?

ZADANIE 2.

Za pomocq trzech róznych cyfr parzystych zapisz wszystkie liczby czterocyfrowe niepodzielne

przez 4 i mające sumę cyfr równą 26.

ZADANIE 3.

Kwadrat o obwodzie 24 cm rozetnij na trzy prostokąty, z których $\mathrm{m}\mathrm{o}\dot{\mathrm{z}}$ na złozyć prostokąt

o obwodzie 26 cm.

ZADANIE 4.

Wojtek napisał na kartce pewną liczbę naturalną. Następnie dopisal do niej dwa zera. Do tak

zmienionej liczby doda115, a następnie podzie1ił ją przez 5. Od wyniku dzie1enia odjąf 3

i w otrzymanej liczbie skreślił cyfrę jedności. Uzyskany wynik podzielił przez 2 i z dumą

napisal rezultat: 2012. Jaka 1iczbę Wojtek zapisał na początku?

ZADANIE 5.

Wpisz do diagramu wszystkie cyfry od l do 9 tak, aby trzy 1iczby powsta1e w ko1umnach

(czytane z góry na dól) byfy podzielne przez 3, a1e $\dot{\mathrm{z}}$ adna liczba trzycyfrowa czytana w wierszach

nie była podzielna przez 3.
\begin{center}
\includegraphics[width=20.268mm,height=20.832mm]{./LigaMatematycznaMatuskiego_SP_Zestaw2_2012_2013_page0_images/image001.eps}
\end{center}\end{document}
