\documentclass[a4paper,12pt]{article}
\usepackage{latexsym}
\usepackage{amsmath}
\usepackage{amssymb}
\usepackage{graphicx}
\usepackage{wrapfig}
\pagestyle{plain}
\usepackage{fancybox}
\usepackage{bm}

\begin{document}

LIGA MATEMATYCZNA

im. Zdzisława Matuskiego

LISTOPAD 2013

SZKOLA PODSTAWOWA

ZADANIE I.

Oblicz sumę liczb w pustych polach diagramu, $\mathrm{j}\mathrm{e}\dot{\mathrm{z}}$ eli liczba w $\mathrm{k}\mathrm{a}\dot{\mathrm{z}}$ dym polu w rzędzie $\mathrm{w}\mathrm{y}\dot{\mathrm{z}}$ szym

jest sumą dwóch liczb z $\mathrm{n}\mathrm{i}\dot{\mathrm{z}}$ szego rzędu sąsiadujacych z nim.
\begin{center}
\includegraphics[width=29.616mm,height=26.868mm]{./LigaMatematycznaMatuskiego_SP_Zestaw2_2013_2014_page0_images/image001.eps}
\end{center}
96

52

20

ZADANIE 2.

Pięciu chlopców $\mathrm{w}\mathrm{a}\dot{\mathrm{z}}$ ylo się parami $\mathrm{k}\mathrm{a}\dot{\mathrm{z}}\mathrm{d}\mathrm{y}$ z $\mathrm{k}\mathrm{a}\dot{\mathrm{z}}$ dym. Otrzymano następujqce rezultaty: 90 kg,

92 kg, 93 kg, 94 kg, 95 kg, 96 kg, 97 kg, 98 kg, l00 kg, l0l kg. Podaj lączna wagę tych

chlopców.

ZADANIE 3.

Obwód prostokąta zbudowanego z dwudziestu jednakowych kwadratów jest równy 126. Ob1icz

pole prostokąta. Rozwaz wszystkie $\mathrm{m}\mathrm{o}\dot{\mathrm{z}}$ liwości.

ZADANIE 4.

Ogrodnik wlozy180 gruszek do 12 koszyków w taki sposób, $\dot{\mathrm{z}}\mathrm{e}$ w $\mathrm{k}\mathrm{a}\dot{\mathrm{z}}$ dym koszyku znalazla się

co najmniej jedna gruszka. Czy jest $\mathrm{m}\mathrm{o}\dot{\mathrm{z}}$ liwe, $\dot{\mathrm{z}}\mathrm{e}$ w $\mathrm{k}\mathrm{a}\dot{\mathrm{z}}$ dym koszyku znajduje się inna liczba

gruszek?

ZADANIE 5.

$\mathrm{W}$ upalny dzień w kawiarni usiedli Ania, Antek, Bartek, Czarek i Darek. Wszyscy zamówili

zimne napoje i $\mathrm{k}\mathrm{a}\dot{\mathrm{z}}\mathrm{d}\mathrm{y}$ zamówif coś innego. Ustal kto zamówif jaki napój, $\mathrm{j}\mathrm{e}\dot{\mathrm{z}}$ eli

$\bullet$ Antek jako jedyny lubi oranzadę;

$\bullet$ Bartek nie lubi napojów gazowanych;

$\bullet$ Czarek nie lubi coca-coli ani wody gazowanej;

$\bullet$ Ania pije tylko wodę gazowaną;

$\bullet$ zamówiono $\mathrm{t}\mathrm{e}\dot{\mathrm{z}}$ fantę i wodę niegazowaną.
\end{document}
