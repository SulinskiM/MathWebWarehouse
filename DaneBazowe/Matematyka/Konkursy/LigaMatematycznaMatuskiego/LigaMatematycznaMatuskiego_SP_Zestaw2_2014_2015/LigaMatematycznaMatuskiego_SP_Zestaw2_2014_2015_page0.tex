\documentclass[a4paper,12pt]{article}
\usepackage{latexsym}
\usepackage{amsmath}
\usepackage{amssymb}
\usepackage{graphicx}
\usepackage{wrapfig}
\pagestyle{plain}
\usepackage{fancybox}
\usepackage{bm}

\begin{document}

LIGA MATEMATYCZNA

im. Zdzisława Matuskiego

LISTOPAD 2014

SZKOLA PODSTAWOWA

ZADANIE I.

Oblicz sumę cyfr liczby $10^{50}-2014.$

ZADANIE 2.

Do trzech samochodów nalez $\mathrm{y}$ zapakować pięć pojemników wypefnionych do pefna farbą, pięć

takich pojemników wypelnionych farbą do polowy i pięć pustych pojemników. Nalezy je zapa-

kować tak, aby $\mathrm{k}\mathrm{a}\dot{\mathrm{z}}\mathrm{d}\mathrm{y}$ samochód zostal jednakowo obciązony. Jak to zrobič?

ZADANIE 3.

$\mathrm{W}$ pudelku są drewniane klocki w trzech róznych kolorach. Klocków niebieskich i zielonych

razem jest 50, zie1onych i czerwonych fącznie jest 40, niebieskich i czerwonych- 68. Ob1icz i1e

jest klocków $\mathrm{k}\mathrm{a}\dot{\mathrm{z}}$ dego koloru.

ZADANIE 4.

$\mathrm{Z}500$ kwadratów o obwodzie 20 cm $\mathrm{k}\mathrm{a}\dot{\mathrm{z}}$ dy, ulozono prostokąt, którego dlugość jest pięć razy

większa od szerokości. Wyznacz pole i obwód otrzymanego prostokąta.

ZADANIE 5.

$K\mathrm{a}\dot{\mathrm{z}}$ dą z liczb 1, 2, 3, 4, 5 na1ezy wpisać w wo1ne po1a figury przedstawionej na rysunku tak,

aby sumy liczb w wierszu i kolumnach byly takie same. Na ile sposobów $\mathrm{m}\mathrm{o}\dot{\mathrm{z}}$ na to zrobić?
\begin{center}
\includegraphics[width=30.684mm,height=30.684mm]{./LigaMatematycznaMatuskiego_SP_Zestaw2_2014_2015_page0_images/image001.eps}
\end{center}
7

6
\end{document}
