\documentclass[a4paper,12pt]{article}
\usepackage{latexsym}
\usepackage{amsmath}
\usepackage{amssymb}
\usepackage{graphicx}
\usepackage{wrapfig}
\pagestyle{plain}
\usepackage{fancybox}
\usepackage{bm}

\begin{document}

LIGA MATEMATYCZNA

im. Zdzisława Matuskiego

LISTOPAD 2015

SZKOLA PODSTAWOWA

ZADANIE I.

Liczby 49, 29, 9, 40, 22, 15, 53, 33, 13, 47 połączono w pary tak, $\dot{\mathrm{z}}\mathrm{e}$ suma liczb w $\mathrm{k}\mathrm{a}\dot{\mathrm{z}}$ dej parze

jest taka sama. Która z liczb stanowi parę z liczbą 15?

ZADANIE 2.

Wykaz, $\dot{\mathrm{z}}\mathrm{e}$ liczba 2555$\ldots$ 52 jest podzielna przez l2.
\begin{center}
\includegraphics[width=17.676mm,height=6.804mm]{./LigaMatematycznaMatuskiego_SP_Zestaw2_2015_2016_page0_images/image001.eps}
\end{center}
100 cyfr 5

ZADANIE 3.

$\mathrm{W}$ skarbcu odkrytym przez Ali-Babę bylo 15 worków z monetami. Wiadomo, $\dot{\mathrm{z}}\mathrm{e}$ w jednym

worku wszystkie monety są fałszywe. Prawdziwa moneta wazy 20 gramów, a fałszywa 19 gra-

mów. Ali-Baba ma bardzo dokladną wagę, dzięki której $\mathrm{m}\mathrm{o}\dot{\mathrm{z}}\mathrm{e}$ stwierdzić, ile wazy konkretny

obiekt. Jak za pomocą jednego $\mathrm{w}\mathrm{a}\dot{\mathrm{z}}$ enia odkryć, w którym worku są fałszywe monety?

ZADANIE 4.

$\mathrm{Z}$ kwadratu wycinamy w rogach cztery kwadratowe kawałki. Ich boki mają dlugošć odpowiednio

l cm, 2 cm, 3 cm, 6 cm. Po ich wycięciu po1e figury zmniejszy1o się dwukrotnie. Wyznacz

obwód powstałej figury.

ZADANIE 5.

Za pomocą czterech czwórek, wpisując między nie znaki matematyczne (dozwolone są: doda-

wanie, odejmowanie, mnozenie, dzielenie, pierwiastkowanie i nawiasy) zapisz liczby od 0 do 10.

Powinno powstać jedenaście róznych zapisów z czterema czwórkami.
\end{document}
