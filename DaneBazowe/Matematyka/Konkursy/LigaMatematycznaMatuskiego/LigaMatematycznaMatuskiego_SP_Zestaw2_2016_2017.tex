\documentclass[a4paper,12pt]{article}
\usepackage{latexsym}
\usepackage{amsmath}
\usepackage{amssymb}
\usepackage{graphicx}
\usepackage{wrapfig}
\pagestyle{plain}
\usepackage{fancybox}
\usepackage{bm}

\begin{document}

LIGA MATEMATYCZNA

im. Zdzisława Matuskiego

LISTOPAD 2016

SZKOLA PODSTAWOWA

ZADANIE I.

Adam i Bartek załozyli się o jedną czekoladę. $\mathrm{J}\mathrm{e}\dot{\mathrm{z}}$ eli Adam wygra zakład, to będzie miał trzy

razy tyle czekolad, co Bartek. $\mathrm{J}\mathrm{e}\dot{\mathrm{z}}$ eli Adam przegra, to będzie mial tylko dwa razy więcej

czekolad $\mathrm{n}\mathrm{i}\dot{\mathrm{z}}$ Bartek. Ile czekolad mial $\mathrm{k}\mathrm{a}\dot{\mathrm{z}}\mathrm{d}\mathrm{y}$ z nich na początku?

ZADANIE 2.

$\mathrm{W}$ kwadracie o polu 64 wybrano punkt $M$ i połączono go ze wszystkimi wierzcholkami. Po-

wstaly w ten sposób cztery trójkaty, z których jeden ma pole 12, a inny ma po1e 24. Podaj

odleglości punktu $M$ od wszystkich boków kwadratu.

ZADANIE 3.

Iloczyn dwóch liczb dwucyfrowych jest równy 525. Zaokrąg1ono te 1iczby do pe1nych dziesiątek.

Iloczyn tych zaokrągleń jest równy 600. Znajd $\acute{\mathrm{z}}$ początkowe liczby.

ZADANIE 4.

Ile jest liczb dziesięciocyfrowych o sumie cyfr równej 3?

ZADANIE 5.

Obwód największego z narysowanych kwadratów jest równy 80, a po1e najmniejszego - 16.

Oblicz obwód całej figury oraz pola kwadratów $B\mathrm{i}C.$
\begin{center}
\includegraphics[width=34.236mm,height=54.300mm]{./LigaMatematycznaMatuskiego_SP_Zestaw2_2016_2017_page0_images/image001.eps}
\end{center}
{\it C}


\end{document}