\documentclass[a4paper,12pt]{article}
\usepackage{latexsym}
\usepackage{amsmath}
\usepackage{amssymb}
\usepackage{graphicx}
\usepackage{wrapfig}
\pagestyle{plain}
\usepackage{fancybox}
\usepackage{bm}

\begin{document}

LIGA MATEMATYCZNA

im. Zdzisława Matuskiego

LISTOPAD 2017

SZKOLA PODSTAWOWA

ZADANIE I.

Mikołaj zrobil sok malinowy na zimę i wlał go do trzylitrowych butelek. Potem jednak posta-

nowil przelać sok do pięciolitrowych slojów. Okazalo się, $\dot{\mathrm{z}}\mathrm{e}$ jedenaście slojów to za malo, więc

wlaf sok po równo do dwunastu takich slojów, chociaz teraz nie są one pelne. Ile soku jest

w $\mathrm{k}\mathrm{a}\dot{\mathrm{z}}$ dym sloju?

ZADANIE 2.

$\mathrm{W}$ trójkącie równoramiennym wysokości poprowadzone do ramion przecinaja się pod kątem

o mierze $100^{\mathrm{o}}$ Oblicz miary kątów wewnętrznych trójkąta.

ZADANIE 3.

$\mathrm{W}$ pewnym bloku w Slupsku jest sto mieszkań ponumerowanych liczbami od l do 100. $\mathrm{W}\mathrm{k}\mathrm{a}\dot{\mathrm{z}}$-

dym z nich mieszka jedna, dwie lub trzy osoby. Lączna liczba osób zamieszkujących lokale

od l do 55 jest równa 105, a fączna 1iczba mieszkańców 1oka1i od 51 do 100 to 150. I1e osób

mieszka w tym budynku?

ZADANIE 4.

$\mathrm{W}$ pewnej klasie szkoly podstawowej wszyscy uczniowie mają tyle samo lat, z wyjątkiem dwóch,

którzy są o rok starsi, ijednego, który ma o rok mniej. $\mathrm{J}\mathrm{e}\dot{\mathrm{z}}$ eli dodamy lata wszystkich uczniów,

to otrzymamy 208. I1u uczniów jest w tej k1asie?

ZADANIE 5.

$\mathrm{U}\dot{\mathrm{z}}$ ywając $\mathrm{k}\mathrm{a}\dot{\mathrm{z}}$ dej z cyfr 0, 1, 2, 3, $\ldots$, 9 tylko raz, uzupefnij diagram tak, aby dwie liczby czte-

rocyfrowe czytane poziomo byly podzielne przez 3, a dwie 1iczby trzycyfrowe czytane pionowo

byly podzielne przez 4. Uzasadnij podzie1ność.
\end{document}
