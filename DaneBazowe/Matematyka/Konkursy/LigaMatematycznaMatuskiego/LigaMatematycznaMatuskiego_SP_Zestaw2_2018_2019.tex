\documentclass[a4paper,12pt]{article}
\usepackage{latexsym}
\usepackage{amsmath}
\usepackage{amssymb}
\usepackage{graphicx}
\usepackage{wrapfig}
\pagestyle{plain}
\usepackage{fancybox}
\usepackage{bm}

\begin{document}

LIGA MATEMATYCZNA

im. Zdzisława Matuskiego

LISTOPAD 2018

SZKOLA PODSTAWOWA

(klasy IV - VI)

ZADANIE I.

Ile jest róznych prostokątów, których dlugości boków wyraz $\mathrm{a}\mathrm{j}\mathrm{a}$ się cafkowitą liczbą centyme-

trów, a pole jest równe 2002 $\mathrm{c}\mathrm{m}^{2}$?

ZADANIE 2.

Znajd $\acute{\mathrm{z}}$ najmniejsza liczbę calkowitą dodatnią, która w zapisie dziesiętnym ma tylko 0 $\mathrm{i}1$ oraz

jest podzielna przez 225.

ZADANIE 3.

$\mathrm{W}$ pewnym dziewięciopiętrowym bloku w Slupsku na $\mathrm{k}\mathrm{a}\dot{\mathrm{z}}$ dym poziomie znajdują się trzy miesz-

kania. $\mathrm{W}\dot{\mathrm{z}}$ adnym mieszkaniu nie mieszka więcej $\mathrm{n}\mathrm{i}\dot{\mathrm{z}}$ troje dzieci. Na $\mathrm{k}\mathrm{a}\dot{\mathrm{z}}$ dym piętrze mieszka

inna liczba dzieci. Ile dzieci mieszka w tym bloku?

ZADANIE 4.

Ze 123 czerwonych i 123 białych sześcianików o krawędzi o długości 1 cm budujemy sześciany

o krawędzi dłuzszej $\mathrm{n}\mathrm{i}\dot{\mathrm{z}}1$ cm tak, aby $\dot{\mathrm{z}}$ adne dwa nie byly tego samego rozmiaru i by powierzch-

nia $\mathrm{k}\mathrm{a}\dot{\mathrm{z}}$ dego sześcianu byfa jednokolorowa. Ile najwięcej sześcianów $\mathrm{m}\mathrm{o}\dot{\mathrm{z}}\mathrm{e}$ powstać? Nie trzeba

wykorzystać wszystkich klocków.

ZADANIE 5.

Sześciokąt, w którym wszystkie kąty mają miarę $120^{\mathrm{o}}$, wpisano w trójkat tak, jak na rysunku.

Wyznacz dlugości boków tego trójkąta.


\end{document}