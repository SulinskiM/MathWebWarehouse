\documentclass[a4paper,12pt]{article}
\usepackage{latexsym}
\usepackage{amsmath}
\usepackage{amssymb}
\usepackage{graphicx}
\usepackage{wrapfig}
\pagestyle{plain}
\usepackage{fancybox}
\usepackage{bm}

\begin{document}

LIGA MATEMATYCZNA

GRUD Z$\mathrm{I}\mathrm{E}\acute{\mathrm{N}}$ 2009

SZKOLA PODSTAWOWA

ZADANIE I.

Monia, Tonia, Ponia i Sonia są kolezankami. Dwie z nich są rówieśniczkami. Monia bylaby

starsza od Toni, gdyby nie była młodsza od Soni. Ponia byłaby młodsza od Toni, gdyby nie

byla starsza od Soni. Kto jest rówieśniczką Toni: Ponia, Monia czy Sonia?

ZADANIE 2.

Przez stację kolejową przejechały trzy pociągi wojskowe. $\mathrm{W}$ pierwszym było 462 $\dot{\mathrm{z}}$ ofnierzy,

w drugim 546, w trzecim 630. Czy $\mathrm{m}\mathrm{o}\dot{\mathrm{z}}$ na obliczyć z ilu wagonów składał się $\mathrm{k}\mathrm{a}\dot{\mathrm{z}}\mathrm{d}\mathrm{y}$ z pocią-

gów, $\mathrm{j}\mathrm{e}\dot{\mathrm{z}}$ eli wiadomo, $\dot{\mathrm{z}}\mathrm{e}$ w $\mathrm{k}\mathrm{a}\dot{\mathrm{z}}$ dym wagonie była jednakowa ilość $\dot{\mathrm{z}}$ ofnierzy i $\dot{\mathrm{z}}\mathrm{e}$ ta liczba była

największa ze wszystkich $\mathrm{m}\mathrm{o}\dot{\mathrm{z}}$ liwych?

ZADANIE 3.

Figura przedstawiona na rysunku sklada się z siedmiu kwadratów. Dlugości boków dwóch spo-

šród tych kwadratów zostały podane. Iloma kwadratami typu $B\mathrm{m}\mathrm{o}\dot{\mathrm{z}}$ na wypelnić kwadrat $A$?
\begin{center}
\includegraphics[width=42.168mm,height=34.140mm]{./LigaMatematycznaMatuskiego_SP_Zestaw3_2009_2010_page0_images/image001.eps}
\end{center}
A

B

2

3

ZADANIE 4.

Prostokqt podzielono na cztery mniejsze prostokąty. Pola trzech z nich są równe odpowiednio

3, 4, 5. Jakie jest pole czwartego prostokąta?
\begin{center}
\begin{tabular}{|l|l|}
\hline
\multicolumn{1}{|l|}{$3$}&	\multicolumn{1}{|l|}{ $4$}	\\
\hline
\multicolumn{1}{|l|}{ $\mathrm{x}$}&	\multicolumn{1}{|l|}{ $5$}	\\
\hline
\end{tabular}

\end{center}
ZADANIE 5.

Mamy trzy rodzaje patyczków: 16 patyczków o d1ugości 1 cm, 15 patyczków o długości 2 cm

$\mathrm{i} 15$ patyczków o długošci 3 cm. Czy $\mathrm{m}\mathrm{o}\dot{\mathrm{z}}$ na zbudować ze wszystkich patyczków prostokąt?

Patyczki tworzące boki prostokąta nie mogą zachodzić na siebie i nie $\mathrm{m}\mathrm{o}\dot{\mathrm{z}}$ na ich lamać.
\end{document}
