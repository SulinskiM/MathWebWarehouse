\documentclass[a4paper,12pt]{article}
\usepackage{latexsym}
\usepackage{amsmath}
\usepackage{amssymb}
\usepackage{graphicx}
\usepackage{wrapfig}
\pagestyle{plain}
\usepackage{fancybox}
\usepackage{bm}

\begin{document}

LIGA MATEMATYCZNA

GRUD Z$\mathrm{I}\mathrm{E}\acute{\mathrm{N}}$ 2010

SZKOLA PODSTAWOWA

ZADANIE I.

Gwarno będzie w šwięta $\mathrm{u}$ dziadków, gdy zjadą się dzieci z wlasnym potomstwem. Które dzieci

sq czyje, ješli:

$\bullet$ Barbara ma więcej dzieci $\mathrm{n}\mathrm{i}\dot{\mathrm{z}}$ brat;

$\bullet$ Piotrek i Oleńka mówią do Jerzego- wujku, a Magda do Haliny- ciociu;

$\bullet$ Misia nie jest siostrą ani Grzesia, ani Arka, który ma troje rodzeństwa:

i dwie siostry;

brata Mačka

$\bullet$ Halina ma córkę i syna.

ZADANIE 2.

Zepsuty kalkulator nie wyświetla cyfry 5. Na przyk1ad, jeś1i napiszemy 1iczbę 3535, to pokazuje

on liczbę 33 bez $\dot{\mathrm{z}}$ adnych odstępów między cyframi. Michał napisaf na tym kalkulatorze pewną

liczbę sześciocyfrową i na wyświetlaczu kalkulatora pojawiła się liczba 2010. D1a i1u 1iczb mog1o

się tak zdarzyć?

ZADANIE 3.

Lączna pojemność butelki i szklanki jest równa pojemnošci dzbanka. Pojemność butelki jest

równa lącznej pojemności szklanki i kufla. Lączna pojemność trzech kufii jest równa łącznej

pojemności dwóch dzbanków. Ile szklanek ma łączną pojemność jednego kufla?

ZADANIE 4.

Osiem zer i osiemjedynek ustaw w tablicy $4\times 4$ tak, aby sumy liczb w $\mathrm{k}\mathrm{a}\dot{\mathrm{z}}$ dym wierszu i w $\mathrm{k}\mathrm{a}\dot{\mathrm{z}}$ dej

kolumnie były nieparzyste.

ZADANIE 5.

$\mathrm{W}$ pomieszczeniu, które ma ksztalt prostopadlościanu, osiem pająków mieszka w ośmiu na-

$\mathrm{r}\mathrm{o}\dot{\mathrm{z}}$ ach. Jeden z nich postanowił odwiedzić wszystkich swoich kolegów, a następnie wrócić

do swojego naroznika wybierajqc drogę wzdluz krawędzi. Odleglości do najblizszych sąsiadów

są równe odpowiednio 4 $\mathrm{m}, 6\mathrm{m}, 8\mathrm{m}. \mathrm{W}$ jakiej kolejności pająk powinien odwiedzić swoich

s$\mathfrak{B}$iadów, aby jego spacer był najkrótszy? Jak długi będzie ten spacer?


\end{document}