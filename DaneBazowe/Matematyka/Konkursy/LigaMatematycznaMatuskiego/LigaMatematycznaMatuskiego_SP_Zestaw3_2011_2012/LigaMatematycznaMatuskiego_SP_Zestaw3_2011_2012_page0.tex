\documentclass[a4paper,12pt]{article}
\usepackage{latexsym}
\usepackage{amsmath}
\usepackage{amssymb}
\usepackage{graphicx}
\usepackage{wrapfig}
\pagestyle{plain}
\usepackage{fancybox}
\usepackage{bm}

\begin{document}

LIGA MATEMATYCZNA

GRUD Z$\mathrm{I}\mathrm{E}\acute{\mathrm{N}}$ 2011

SZKOLA PODSTAWOWA

ZADANIE I.

Dziewięciu Mikofajów w 30 minut rozdaje 60 prezentów.

w ciągu trzech godzin?

Ile prezentów rozda 36 Mikofajów

ZADANIE 2.

$K\mathrm{a}\dot{\mathrm{z}}\mathrm{d}\mathrm{y}$ uczeń pewnej klasy interesuje się matematyka, historią lub geografią. Tylko jednego

ucznia pasjonują wszystkie te dziedziny nauki. Matematykę i geografię zglębia troje uczniów.

Tych, którzy nie lubią matematyki, ale poszerzają swoją wiedzę historyczną jest dziesięcioro.

Geografią i historia interesuje się pięcioro. Zapalonych matematyków jest 19. Historia i ma-

tematyka to ulubione przedmioty ośmiorga. Ilu jest milośników geografii, skoro wszystkich

uczniów jest 36?

ZADANIE 3.

Ulóz kwadrat z trzech kwadratów o boku l, trzech kwadratów o boku 2, dwóch kwadratów

o boku 3 i jednego o boku 4.

ZADANIE 4.

$\mathrm{W}$ trzech jednakowych puszkach znajduje się mleko, cukier i sól. Niestety, pomylono nalepki

i $\dot{\mathrm{z}}$ adna nie opisuje poprawnie zawartości puszki. Potrząsajac tylko jedną z nich ustal, co zawiera

$\mathrm{k}\mathrm{a}\dot{\mathrm{z}}$ da puszka.

ZADANIE 5.

Róznica liczby sześciocyfrowej i liczby pięciocyfrowej jest równa 6.

wszystkie rozwiązania.

Wyznacz te liczby. Podaj
\end{document}
