\documentclass[a4paper,12pt]{article}
\usepackage{latexsym}
\usepackage{amsmath}
\usepackage{amssymb}
\usepackage{graphicx}
\usepackage{wrapfig}
\pagestyle{plain}
\usepackage{fancybox}
\usepackage{bm}

\begin{document}

LIGA MATEMATYCZNA

im. Zdzisława Matuskiego

GRUD Z$\mathrm{I}\mathrm{E}\acute{\mathrm{N}}$ 2012

SZKOLA PODSTAWOWA

ZADANIE I.

Mikołaj złowil karpie. Wypuścił jednego do rzeki, a połowę pozostalych dal Adamowi. Potem

znów wypuścił jednego karpia i połowę pozostalych ofiarował Ewie. Zostalo mu jeszcze 6 karpi.

Ile karpi zlowil Mikofaj?

ZADANIE 2.

Obwód trójkąta równoramiennego jest równy 56. $\acute{\mathrm{S}}$ rodek jednego z ramion polączono z wierz-

chołkiem przeciwleglego kąta. Powstaly w ten sposób dwa nowe trójkąty, z których jeden (ten,

który zawiera podstawę trójkąta równoramiennego) ma obwód o 10 krótszy $\mathrm{n}\mathrm{i}\dot{\mathrm{z}}$ drugi. Oblicz

długošci boków trójkqta równoramiennego.

ZADANIE 3.

Za siedmioma górami, za siedmioma lasami, za siedmioma morzami rosła czarodziejska wierzba.

$\mathrm{Z}$ jej grubego pnia wyrastala pewna liczba konarów. $\mathrm{Z}\mathrm{k}\mathrm{a}\dot{\mathrm{z}}$ dego konara wyrastalo tyle galęzi,

ile było wszystkich konarów. Na $\mathrm{k}\mathrm{a}\dot{\mathrm{z}}$ dej galęzi rosło dwa razy więcej magicznych gruszek $\mathrm{n}\mathrm{i}\dot{\mathrm{z}}$

było wszystkich galęzi na tym drzewie. Magicznych gruszek bylo 1250. I1e konarów wyrasta1o

z pnia czarodziejskiej wierzby?

ZADANIE 4.

$\mathrm{W}$ prawej i lewej kieszeni Karol mial lącznie 38 monet. $\mathrm{J}\mathrm{e}\dot{\mathrm{z}}$ eli przefozy z prawej do lewej

kieszeni tyle monet, ile jest w lewej, a następnie z lewej do prawej tyle monet, ile będzie

w prawej kieszeni po pierwszym przełozeniu, to w prawej będzie miał o 2 monety więcej $\mathrm{n}\mathrm{i}\dot{\mathrm{z}}$

w lewej kieszeni. Ile monet miał Karol na początku w $\mathrm{k}\mathrm{a}\dot{\mathrm{z}}$ dej kieszeni?

ZADANIE 5.

Liczbę naturalną nazywamy palindromiczną, $\mathrm{j}\mathrm{e}\dot{\mathrm{z}}$ eli jej zapis dziesiętny czytany od lewej strony

do prawej jest taki sam, jak czytany od prawej strony do lewej. Podaj wszystkie pary liczb

pięciocyfrowych palindromicznych, których róznica jest równa ll.


\end{document}