\documentclass[a4paper,12pt]{article}
\usepackage{latexsym}
\usepackage{amsmath}
\usepackage{amssymb}
\usepackage{graphicx}
\usepackage{wrapfig}
\pagestyle{plain}
\usepackage{fancybox}
\usepackage{bm}

\begin{document}

LIGA MATEMATYCZNA

im. Zdzisława Matuskiego

GRUD Z$\mathrm{I}\mathrm{E}\acute{\mathrm{N}}$ 2013

SZKOLA PODSTAWOWA

ZADANIE I.

Kwadrat o boku dfugości 10 cm podzie1ono na mniejszy kwadrat i cztery jednakowe prostokąty.

$K\mathrm{a}\dot{\mathrm{z}}$ da z pięciu części ma taki sam obwód. Oblicz pole malego kwadratu.

ZADANIE 2.

Liczbę naturalną nazywamy dobrą, $\mathrm{j}\mathrm{e}\dot{\mathrm{z}}$ eli $\mathrm{m}\mathrm{o}\dot{\mathrm{z}}$ na ją zapisač przy pomocy róznych cyfr, których

iloczyn jest równy 360. Podaj co najmniej dwie takie 1iczby natura1ne. Wyznacz największą

dobrą liczbę naturalną.

ZADANIE 3.

W pięciu rzutach kostką do gry otrzymano 27 punktów. I1e razy maksyma1nie mog1o wypaść 6

oczek? Ile razy mogfo wypaść 5 oczek?

ZADANIE 4.

Do sklepu dostarczono $\mathrm{g}\mathrm{w}\mathrm{o}\acute{\mathrm{z}}$dzie w sześciu skrzynkach. $\mathrm{G}\mathrm{w}\mathrm{o}\acute{\mathrm{z}}$dzie $\mathrm{w}\mathrm{a}\dot{\mathrm{z}}$ yly 2 kg, 3 kg, 5 kg, 8 kg,

9 kg i l0 kg. Którą skrzynkę sprzedawca powinien sprzedać panu Nowakowi, aby pozostale 5

skrzynek $\mathrm{m}\mathrm{o}\dot{\mathrm{z}}$ na byfo rozdzielić między dwóch klientów tak, aby $\mathrm{g}\mathrm{w}\mathrm{o}\acute{\mathrm{z}}$dzie w zakupionych przez

nich skrzynkach $\mathrm{w}\mathrm{a}\dot{\mathrm{z}}$ yfy tyle samo?

ZADANIE 5.

Joanna, Paweł, Michal, Piotr i Jarek mieszkają w jednym bloku. $\mathrm{K}\mathrm{a}\dot{\mathrm{z}}\mathrm{d}\mathrm{y}$ z nich prowadzi inny

samochód: peugeot, renault (samochody francuskie), fiat (samochód wloski), ligier (francuski

samochód wyścigowy) i ferrari (wloski samochód wyścigowy). Ustal jaki samochód posiada

$\mathrm{k}\mathrm{a}\dot{\mathrm{z}}\mathrm{d}\mathrm{y}$ z nich, $\mathrm{j}\mathrm{e}\dot{\mathrm{z}}$ eli:

$\bullet$ Pawel i Michaf mają samochody francuskie;

$\bullet$ Joanna i Jarek lubią bezpiecznąjazdę i nie mają samochodów wyścigowych;

$\bullet$ Michal, Joanna i Jarek wsiadają często do renaulta, ale go nie prowadzą;

$\bullet$ Joanna, Paweł i właściciel peugeota sa przyjaciółmi.
\end{document}
