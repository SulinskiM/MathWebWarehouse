\documentclass[a4paper,12pt]{article}
\usepackage{latexsym}
\usepackage{amsmath}
\usepackage{amssymb}
\usepackage{graphicx}
\usepackage{wrapfig}
\pagestyle{plain}
\usepackage{fancybox}
\usepackage{bm}

\begin{document}

LIGA MATEMATYCZNA

im. Zdzisława Matuskiego

GRUD Z$\mathrm{I}\mathrm{E}\acute{\mathrm{N}}$ 2014

SZKOLA PODSTAWOWA

ZADANIE I.

Mikołaj przydzielal siedmiu elfom prezenty do rozdania. Kolejno $\mathrm{k}\mathrm{a}\dot{\mathrm{z}}\mathrm{d}\mathrm{y}$ otrzymywal po jednej

paczce. Kiedy $\mathrm{k}\mathrm{a}\dot{\mathrm{z}}\mathrm{d}\mathrm{y}$ elf miał $\mathrm{j}\mathrm{u}\dot{\mathrm{z}} 18$ prezentów pozostała reszta, która nie wystarczyla, by

$\mathrm{k}\mathrm{a}\dot{\mathrm{z}}\mathrm{d}\mathrm{y}$ otrzymał jeszcze po jednym. Resztę oddali Mikołajowi. Ile prezentów było do rozdania?

ZADANIE 2.

Na kiermaszu przedświątecznym zaplanowano sprzedaz 400 bombek choinkowych po 16 zł za

sztukę. Po sprzedaniu 30\% bombek okaza1o się, $\dot{\mathrm{z}}\mathrm{e}$ część popękała w czasie transportu. Odło-

$\dot{\mathrm{z}}$ ono więc te bombki. Aby uzyskać zaplanowany przychód, pozostale bombki zostały sprzedane

po 20 zł za sztukę. I1e bombek popęka1o podczas transportu?

ZADANIE 3.

Uzywając tylko cyfr 2 $\mathrm{i}8$ (być $\mathrm{m}\mathrm{o}\dot{\mathrm{z}}\mathrm{e}$ kilkukrotnie) zapisz najmniejszą liczbę naturalną podzielną

przez 9 oraz najmniejszą 1iczbę natura1ną podzie1ną przez 12.

ZADANIE 4.

Rozetnij kwadrat na cztery jednakowe części tak, aby w $\mathrm{k}\mathrm{a}\dot{\mathrm{z}}$ dej znalazly się trzy kólka.
\begin{center}
\begin{tabular}{|l|l|l|l|l|l|}
\hline
\multicolumn{1}{|l|}{}&	\multicolumn{1}{|l|}{}&	\multicolumn{1}{|l|}{}&	\multicolumn{1}{|l|}{}&	\multicolumn{1}{|l|}{}&	\multicolumn{1}{|l|}{}	\\
\hline
\multicolumn{1}{|l|}{}&	\multicolumn{1}{|l|}{}&	\multicolumn{1}{|l|}{}&	\multicolumn{1}{|l|}{}&	\multicolumn{1}{|l|}{}&	\multicolumn{1}{|l|}{}	\\
\hline
\multicolumn{1}{|l|}{}&	\multicolumn{1}{|l|}{}&	\multicolumn{1}{|l|}{}&	\multicolumn{1}{|l|}{}&	\multicolumn{1}{|l|}{}&	\multicolumn{1}{|l|}{}	\\
\hline
\multicolumn{1}{|l|}{}&	\multicolumn{1}{|l|}{}&	\multicolumn{1}{|l|}{}&	\multicolumn{1}{|l|}{}&	\multicolumn{1}{|l|}{}&	\multicolumn{1}{|l|}{}	\\
\hline
\multicolumn{1}{|l|}{}&	\multicolumn{1}{|l|}{}&	\multicolumn{1}{|l|}{}&	\multicolumn{1}{|l|}{}&	\multicolumn{1}{|l|}{}&	\multicolumn{1}{|l|}{}	\\
\hline
\multicolumn{1}{|l|}{}&	\multicolumn{1}{|l|}{}&	\multicolumn{1}{|l|}{}&	\multicolumn{1}{|l|}{}&	\multicolumn{1}{|l|}{}&	\multicolumn{1}{|l|}{}	\\
\hline
\end{tabular}
\end{center}
ZADANIE 5.

$\mathrm{Z}$ trzech trójkątów prostokątnych równoramiennych zbudowano choinkę, jak na rysunku. Pod-

stawa największego trójkąta ma długość 20 cm. Podstawa następnego jest umieszczona w po-

łowie wysokości poprzedniego trójkąta. $\mathrm{K}\mathrm{a}\dot{\mathrm{z}}\mathrm{d}\mathrm{y}$ kolejny trójkąt jest o 2 cm $\mathrm{n}\mathrm{i}\dot{\mathrm{z}}$ szy. Oblicz pole

powierzchni największego i najmniejszego trójkąta.


\end{document}