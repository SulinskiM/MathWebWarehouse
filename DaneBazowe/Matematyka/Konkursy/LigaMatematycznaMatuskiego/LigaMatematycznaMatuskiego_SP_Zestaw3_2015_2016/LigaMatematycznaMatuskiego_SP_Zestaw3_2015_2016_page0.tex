\documentclass[a4paper,12pt]{article}
\usepackage{latexsym}
\usepackage{amsmath}
\usepackage{amssymb}
\usepackage{graphicx}
\usepackage{wrapfig}
\pagestyle{plain}
\usepackage{fancybox}
\usepackage{bm}

\begin{document}

LIGA MATEMATYCZNA

im. Zdzisława Matuskiego

GRUD Z$\mathrm{I}\mathrm{E}\acute{\mathrm{N}}$ 2015

SZKOLA PODSTAWOWA

ZADANIE I.

Do zapakowania jest więcej $\mathrm{n}\mathrm{i}\dot{\mathrm{z}}150$ bombek, ale mniej $\mathrm{n}\mathrm{i}\dot{\mathrm{z}}200$. Mamy dwa rodzaje opakowań

do wyboru. Gdy wlozymy do pudefek po 10 sztuk, to zostana 4 bombki, a gdy zapakujemy po

8 sztuk, to $\mathrm{t}\mathrm{e}\dot{\mathrm{z}}$ zostaną 4. I1e bombek jest do zapakowania? I1e na1ezy wziąć pude1ek $\mathrm{k}\mathrm{a}\dot{\mathrm{z}}$ dego

rodzaju, abyje zapelnić i aby wszystkie bombki byly zapakowane? Podaj wszystkie $\mathrm{m}\mathrm{o}\dot{\mathrm{z}}$ liwości.

ZADANIE 2.

Wykaz$\cdot, \dot{\mathrm{z}}\mathrm{e}$ liczba $10^{49}+20$ jest podzielna przez 12.

ZADANIE 3.

Pan Jan hoduje koty. Ma ich tyle, $\dot{\mathrm{z}}\mathrm{e}$ gdy dodaf liczbę kocich ogonów, uszu i fapek, to otrzymaf

ponad 100. Gdy zsumowa1 ty1ko 1iczbę ogonów i 1ap, otrzyma1 mniej $\mathrm{n}\mathrm{i}\dot{\mathrm{z}}80$. Ile kotów ma pan

Jan?

ZADANIE 4.

Czarodziej podarowaf Ani zaczarowaną szkatufkę i podal dwa zaklęcia. Szkatulka ta na jedno

z zaklęć powiększa swoją zawartość ojednego denara, a na drugie podwaja liczbę denarów znaj-

dujących się w niej. Podaj najmniejszą liczbę zaklęć, jakie trzeba wypowiedzieć, aby w szkatulce

(na początku pustej) znalazlo się 40 denarów (nie wyjmujemy monet ze szkatu1ki).

ZADANIE 5.

$\mathrm{Z}$ ilu najmniejszych kwadracików (dwa z nich zaznaczono na czarno) składa się $\mathrm{d}\mathrm{u}\dot{\mathrm{z}}\mathrm{y}$ kwadrat

o czarnym obwodzie?
\end{document}
