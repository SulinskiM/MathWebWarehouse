\documentclass[a4paper,12pt]{article}
\usepackage{latexsym}
\usepackage{amsmath}
\usepackage{amssymb}
\usepackage{graphicx}
\usepackage{wrapfig}
\pagestyle{plain}
\usepackage{fancybox}
\usepackage{bm}

\begin{document}

LIGA MATEMATYCZNA

im. Zdzisława Matuskiego

GRUD Z$\mathrm{I}\mathrm{E}\acute{\mathrm{N}}$ 2016

SZKOLA PODSTAWOWA

ZADANIE I.

Mikołaj napisal kolejne liczby naturalne $\mathrm{u}\dot{\mathrm{z}}$ ywajac łącznie siedmiu cyfr. Znajdz' te liczby wie-

dząc, $\dot{\mathrm{z}}\mathrm{e}$ ponad polowa spośród $\mathrm{u}\dot{\mathrm{z}}$ ytych cyfr byla taka sama.

ZADANIE 2.

$K\mathrm{a}\dot{\mathrm{z}}\mathrm{d}\mathrm{y}$ uczestnik mikolajkowego turnieju dostaje dziesięć punktów na starcie i musi odpowie-

dzieć na 10 pytań. Za dobrą odpowied $\acute{\mathrm{z}}$ dostaje l punkt, za zlą $\mathrm{o}\mathrm{d}\mathrm{p}\mathrm{o}\mathrm{w}\mathrm{i}\mathrm{e}\mathrm{d}\acute{\mathrm{z}}$ lub jej brak traci

l punkt. Mikolaj ukończył turniej z 14 punktami. I1u dobrych odpowiedzi udzie1ił?

ZADANIE 3.

Mianownik pewnego ulamka jest o 3 większy od 1icznika. $\mathrm{J}\mathrm{e}\dot{\mathrm{z}}$ eli jego licznik zwiększymy o 10,

a mianownik zwiększymy o l, to otrzymany ulamek będzie odwrotnością poszukiwanego. Jaki

to ulamek?

ZADANIE 4.

Ania rozpoczęla czytanie ksiązki w sobotę. Przez pierwsze cztery dni czytala $\mathrm{k}\mathrm{a}\dot{\mathrm{z}}$ dego dnia śred-

nio po 12 stron. Przez następne dni czytafa dziennie po 20 stron. Ostatniego dnia przeczytafa

ostatnie 10 stron ksiązki. Okaza1o się, $\dot{\mathrm{z}}\mathrm{e}$ gdyby czytala po 14 stron dziennie, to cafą ksiązkę

przeczytalaby w tym samym czasie. Ile dni zajęfo Ani przeczytanie tej ksiązki? $\mathrm{W}$ którym

dniu tygodnia skończyła czytać?

ZADANIE 5.

Prostokąt przedstawiony na rysunku podzielono na sześć figur. Czworokąty $A, B, C, D$ sa

kwadratami. Pole kwadratu $A$ jest równe 9 $\mathrm{c}\mathrm{m}^{2}$, pole B- $4\mathrm{c}\mathrm{m}^{2}$, a pole $D$ - $49\mathrm{c}\mathrm{m}^{2}$ Oblicz

pole tego prostokąta.

{\it A}

{\it B}

{\it D}

{\it C}
\end{document}
