\documentclass[a4paper,12pt]{article}
\usepackage{latexsym}
\usepackage{amsmath}
\usepackage{amssymb}
\usepackage{graphicx}
\usepackage{wrapfig}
\pagestyle{plain}
\usepackage{fancybox}
\usepackage{bm}

\begin{document}

LIGA MATEMATYCZNA

Szkoła Podstawowa

Półfinał

201utego 2009

ZADANIE I.

Kasię, Ewę i Anię poczęstowano trzema czekoladkami z orzechami, z rodzynkami i truskawkami.

Kasia nie lubi orzechów, Ewa-rodzynek, Ania jest uczulona na truskawki. Na ile sposobów

$\mathrm{m}\mathrm{o}\dot{\mathrm{z}}$ na podzielić te czekoladki między dziewczynki tak, aby były zadowolone?

ZADANIE 2.

$\mathrm{W}$ trójkącie równoramiennym $ABC$, w którym $|AC| = |BC|$, poprowadzono wysokość CD.

Oblicz dlugość tej wysokości, $\mathrm{j}\mathrm{e}\dot{\mathrm{z}}$ eli obwód trójkąta $ABC$ jest równy 32 cm, a obwód trójkąta

$ADC$ jest o 6 cm krótszy od obwodu trójkąta $ABC.$

ZADANIE 3.

Ania i Basia wazą łącznie 44 kg, Basia i Ce1ina-47 kg, Ce1ina i Dorota-46 kg, Dorota i Ewa

$-49$ kg, Ewa i Ania-48 kg. I1e wazy Ania?

ZADANIE 4.

Przy uzyciu cyfr: 1, 2, 3, 4, 5, 6, Tomek napisał dwie 1iczby całkowite dodatnie takie, $\dot{\mathrm{z}}\mathrm{e}\mathrm{k}\mathrm{a}\dot{\mathrm{z}}$ da

z cyfr występowała tylko w jednej z dwóch liczb, i to dokladnie raz. Gdy liczby te dodał,

otrzymał 750. Jakie 1iczby napisał Tomek? Podaj wszystkie pary tych 1iczb.

ZADANIE 5.

Ala pomaga cioci w prowadzeniu sklepu cukierniczego. Po zamknięciu sklepu dziewczynka po-

liczyla, ile tabliczek czekolady pozostalo na pólkach, ale przez roztargnienie wpisała do zeszytu

otrzymaną liczbę bez ostatniej cyfry. Na drugi dzień stwierdzila ze zdziwieniem, $\dot{\mathrm{z}}\mathrm{e}$ liczba tabli-

czek czekolady na pólkach jest o 89 większa od 1iczby wpisanej do zeszytu. Jaką 1iczbę powinna

byla wpisać Ala?
\end{document}
