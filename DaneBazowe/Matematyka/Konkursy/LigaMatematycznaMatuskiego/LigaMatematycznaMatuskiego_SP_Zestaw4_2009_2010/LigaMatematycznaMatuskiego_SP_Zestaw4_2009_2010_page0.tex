\documentclass[a4paper,12pt]{article}
\usepackage{latexsym}
\usepackage{amsmath}
\usepackage{amssymb}
\usepackage{graphicx}
\usepackage{wrapfig}
\pagestyle{plain}
\usepackage{fancybox}
\usepackage{bm}

\begin{document}

LIGA MATEMATYCZNA

PÓLFINAL

51utego 20l0

SZKOLA PODSTAWOWA

ZADANIE I.

Arbuz jest o $\displaystyle \frac{4}{5}$ kg cięzszy od $\displaystyle \frac{4}{5}$ tego arbuza. Ile wazy arbuz?

ZADANIE 2.

$\mathrm{W}$ pewnej rodzinie jest czworo dzieci w wieku 5, 8, 13 $\mathrm{i} 15$ lat. Imiona tych dzieci to Ania,

Bartek, Czesia i Daria. Ile lat ma $\mathrm{k}\mathrm{a}\dot{\mathrm{z}}$ de z nich, $\mathrm{j}\mathrm{e}\dot{\mathrm{z}}$ eli jedna dziewczynka chodzi do przedszkola,

Ania jest starsza od Bartka, a suma lat Ani i Czesi dzieli się przez 3?

ZADANIE 3.

Do hurtowni nadszedł transport trzech gatunków herbaty. $K\mathrm{a}\dot{\mathrm{z}}$ dego gatunku było tyle samo,

a caly transport wazył mniej $\mathrm{n}\mathrm{i}\dot{\mathrm{z}}4$ tony. Pierwszy gatunek herbaty przysłano w 76jednakowych

paczkach, drugi w 57 jednakowych paczkach, a trzeci w 60 jednakowych paczkach. $\mathrm{W}\mathrm{k}\mathrm{a}\dot{\mathrm{z}}$-

dej paczce byla całkowita liczba kilogramów herbaty. Ile nadeszło herbaty $\mathrm{k}\mathrm{a}\dot{\mathrm{z}}$ dego gatunku?

Ile herbaty było w $\mathrm{k}\mathrm{a}\dot{\mathrm{z}}$ dej z trzech rodzajów paczek?

ZADANIE 4.

Tablicę $3\times 3$ podzielono na dziewięć jednakowych kwadratów. $\mathrm{W}\mathrm{k}\mathrm{a}\dot{\mathrm{z}}\mathrm{d}\mathrm{y}$ z nich wpisano liczbę

-llub 0, 1ub 1. Wykaz, $\dot{\mathrm{z}}\mathrm{e}$ wśród wszystkich sum z trzech wierszy, trzech kolumn i dwóch

gfównych przekątnych co najmniej dwie są równe.

ZADANIE 5.

Cztery kwadratowe plytki ulozono jak na rysunku. Dfugości boków dwóch z tych plytek za-

znaczono na rysunku. Jaka jest długość boku największej płytki?
\begin{center}
\includegraphics[width=56.184mm,height=32.616mm]{./LigaMatematycznaMatuskiego_SP_Zestaw4_2009_2010_page0_images/image001.eps}
\end{center}
1

40
\end{document}
