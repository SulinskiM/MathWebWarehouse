\documentclass[a4paper,12pt]{article}
\usepackage{latexsym}
\usepackage{amsmath}
\usepackage{amssymb}
\usepackage{graphicx}
\usepackage{wrapfig}
\pagestyle{plain}
\usepackage{fancybox}
\usepackage{bm}

\begin{document}

LIGA MATEMATYCZNA

PÓLFINAL

181utego 20ll

SZKOLA PODSTAWOWA

ZADANIE I.

$\mathrm{W}$ maratonie startowalo 2011 zawodników. Które miejsce zajqf Michaf, $\mathrm{j}\mathrm{e}\dot{\mathrm{z}}$ eli wiadomo, $\dot{\mathrm{z}}\mathrm{e}$ liczba

uczestników, którzy przybiegli na metę przed nim jest cztery razy mniejsza od liczby uczestni-

ków, którzy przybiegli po nim?

ZADANIE 2.

Uzupelniamy tablicę wpisując w $\mathrm{k}\mathrm{a}\dot{\mathrm{z}}$ de jej pole 01ub 1 tak, aby sumy 1iczb w $\mathrm{k}\mathrm{a}\dot{\mathrm{z}}$ dym wierszu

i w $\mathrm{k}\mathrm{a}\dot{\mathrm{z}}$ dej kolumnie były równe 2. Jakie są wartości a $\mathrm{i}b$?
\begin{center}
\begin{tabular}{|l|l|l|l|}
\hline
\multicolumn{1}{|l|}{$a$}&	\multicolumn{1}{|l|}{}&	\multicolumn{1}{|l|}{}&	\multicolumn{1}{|l|}{}	\\
\hline
\multicolumn{1}{|l|}{}&	\multicolumn{1}{|l|}{ $b$}&	\multicolumn{1}{|l|}{}&	\multicolumn{1}{|l|}{ $1$}	\\
\hline
\multicolumn{1}{|l|}{}&	\multicolumn{1}{|l|}{}&	\multicolumn{1}{|l|}{ $0$}&	\multicolumn{1}{|l|}{}	\\
\hline
\multicolumn{1}{|l|}{ $0$}&	\multicolumn{1}{|l|}{}&	\multicolumn{1}{|l|}{ $0$}&	\multicolumn{1}{|l|}{}	\\
\hline
\end{tabular}

\end{center}
ZADANIE 3.

Na skraju lasu stoi siedem domków. $K\mathrm{a}\dot{\mathrm{z}}\mathrm{d}\mathrm{y}$ domek zamieszkiwany jest przez inną liczbę miesz-

kańców oraz $\dot{\mathrm{z}}$ aden domek nie jest pusty. Ile osób mieszka w poszczególnych domkach, $\mathrm{j}\mathrm{e}\dot{\mathrm{z}}$ eli

wszystkich mieszkańców jest 29?

ZADANIE 4.

$K\mathrm{a}\dot{\mathrm{z}}$ dą z dwóch identycznych prostokątnych kartek rozcięto na dwie części. $\mathrm{Z}$ pierwszej kartki

otrzymano dwa prostokąty o obwodach 40 cm $\mathrm{k}\mathrm{a}\dot{\mathrm{z}}$ dy, z drugiej - dwa prostokąty o obwodach

50 cm $\mathrm{k}\mathrm{a}\dot{\mathrm{z}}$ dy. Oblicz obwód $\mathrm{k}\mathrm{a}\dot{\mathrm{z}}$ dej z wyjściowych kartek.

ZADANIE 5.

Adaś dostał pod choinkę modele trzech samochodów: mercedesa, opla i fiata. Wszystkie są róz-

nej wielkości i w róznych kolorach: bialym, czerwonym i czarnym. Mercedes nie jest biały ani

czarny, opel nie jest šredni, a fiat nie jest $\mathrm{d}\mathrm{u}\dot{\mathrm{z}}\mathrm{y}$ ani czarny. Okrešl wielkošć i kolor $\mathrm{k}\mathrm{a}\dot{\mathrm{z}}$ dego

samochodu, jeśli wiadomo, $\dot{\mathrm{z}}\mathrm{e}$ mały samochód jest czarny.
\end{document}
