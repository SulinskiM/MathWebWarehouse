\documentclass[a4paper,12pt]{article}
\usepackage{latexsym}
\usepackage{amsmath}
\usepackage{amssymb}
\usepackage{graphicx}
\usepackage{wrapfig}
\pagestyle{plain}
\usepackage{fancybox}
\usepackage{bm}

\begin{document}

LIGA MATEMATYCZNA

PÓLFINAL

161utego 20l2

SZKOLA PODSTAWOWA

ZADANIE I.

Trzy kolejne liczby trzycyfrowe zapisano obok siebie, bez odstępów, otrzymując liczbę dzie-

więciocyfrową podzielną przez 4 $\mathrm{i}25.$ Znajd $\acute{\mathrm{z}}$ te liczby wiedząc, $\dot{\mathrm{z}}\mathrm{e}$ w ich zapisie dziesiętnym

występują jedynie trzy rózne cyfry.

ZADANIE 2.

$K\mathrm{a}\dot{\mathrm{z}}\mathrm{d}\mathrm{y}$ uczeń pewnej klasy sportowej uprawia $\dot{\mathrm{z}}$ eglarstwo, plywanie lub judo. Tylko dwóch

uczniów uprawia wszystkie te dyscypliny sportu. Judo i $\dot{\mathrm{z}}$ eglarstwem zajmuje się czworo

uczniów. Dziesięciu uprawiających pfywanie nie zajmuje się $\dot{\mathrm{z}}$ eglarstwem. Pięciu uprawia ply-

wanie i judo. $\dot{\mathrm{Z}}$ eglarstwem pasjonuje się 19 uczniów. Pływanie i $\dot{\mathrm{z}}$ eglarstwo to dyscypliny

uprawiane przez 8 uczniów. I1u uczniów uprawia judo w tej k1asie, skoro wszystkich uczniów

jest 36? I1u uczniów uprawia ty1ko jedną dyscyp1inę sportu?

ZADANIE 3.

Ola, Basia, Ewa i Kasia wybrały się na grzyby. Ola i Basia zebraly razem 40 grzybów, Ewa

i Kasia 42, a O1a i Kasia 30 grzybów. I1e grzybów zebra1y 1ącznie Basia i Ewa?

ZADANIE 4.

Rozpoczynając od pewnej liczby naturalnej, wypisano pięć kolejnych jej wielokrotnošci. Suma

trzech najmniejszych jest równa 33330. Ob1icz sumę trzech największych.

ZADANIE 5.

Prostokąt został podzielony na kwadratyjak na rysunku. Pole kwadratu zaznaczonego ciemnym

kolorem jest równe l. Wyznacz dlugošci boków wszystkich kwadratów oraz oblicz pole tego

prostokąta.
\end{document}
