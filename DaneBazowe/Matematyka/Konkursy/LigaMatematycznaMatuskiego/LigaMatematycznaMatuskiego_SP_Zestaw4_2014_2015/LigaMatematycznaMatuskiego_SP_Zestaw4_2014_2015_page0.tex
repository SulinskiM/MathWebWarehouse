\documentclass[a4paper,12pt]{article}
\usepackage{latexsym}
\usepackage{amsmath}
\usepackage{amssymb}
\usepackage{graphicx}
\usepackage{wrapfig}
\pagestyle{plain}
\usepackage{fancybox}
\usepackage{bm}

\begin{document}

LIGA MATEMATYCZNA

im. Zdzisława Matuskiego

PÓLFINAL

ll lutego 2014

SZKOLA PODSTAWOWA

ZADANIE I.

Uzupelnij diagram w taki sposób, aby liczba w $\mathrm{k}\mathrm{a}\dot{\mathrm{z}}$ dym polu w rzędzie $\mathrm{w}\mathrm{y}\dot{\mathrm{z}}$ szym była sumą

dwóch liczb z pól $\mathrm{n}\mathrm{i}\dot{\mathrm{z}}$ szego rzędu sąsiadujących z nim. Oblicz sumę liczb w podstawie diagramu.
\begin{center}
\includegraphics[width=33.276mm,height=29.712mm]{./LigaMatematycznaMatuskiego_SP_Zestaw4_2014_2015_page0_images/image001.eps}
\end{center}
32

23  10

14  2

ZADANIE 2.

Krzyś pomnozył trzy liczby naturalne i otrzymał 5400. Liczby pierwsza i druga nie dzie1ą się

przez 2, druga i trzecia- nie dzie1ą się przez 3, a pierwsza i trzecia- nie dzie1ą się przez 5. Jakie

to liczby?

ZADANIE 3.

Suma cyfr pewnej liczby dziewięciocyfrowej jest równa 9. Cyfra 2 występuje w niej ty1ko raz.

Oblicz iloczyn cyfr tej liczby.

ZADANIE 4.

$\mathrm{O}$ czterech kolegach wiadomo, $\dot{\mathrm{z}}\mathrm{e}$:

$\bullet$ Mirek i kierowca są starsi od Pawfa;

$\bullet$ Leszek i policjant trenujq boks;

$\bullet$ elektryk jest najmłodszy z całej czwórki;

$\bullet$ w soboty Zbyszek i piekarz grają w brydza przeciw Pawłowi i elektrykowi.

Jaki zawód wykonuje $\mathrm{k}\mathrm{a}\dot{\mathrm{z}}\mathrm{d}\mathrm{y}$ z przyjaciól?

ZADANIE 5.

$\mathrm{W}$ kredensie stoją 24 dzbany, w tym 8 z nich jest pustych, 11 wype1nionych miodem do połowy,

5 pełnych miodu. $\mathrm{W}$ jaki sposób $\mathrm{m}\mathrm{o}\dot{\mathrm{z}}$ na podzielić miód i dzbany między trzech braci tak, aby

$\mathrm{k}\mathrm{a}\dot{\mathrm{z}}\mathrm{d}\mathrm{y}$ z nich otrzyma18 dzbanów zjednakową zawartością w nich miodu? (Nie wo1no prze1ewać

miodu z dzbana do dzbana.)
\end{document}
