\documentclass[a4paper,12pt]{article}
\usepackage{latexsym}
\usepackage{amsmath}
\usepackage{amssymb}
\usepackage{graphicx}
\usepackage{wrapfig}
\pagestyle{plain}
\usepackage{fancybox}
\usepackage{bm}

\begin{document}

AHADEMIA POMORSHA

III SLUPSHU
\begin{center}
\includegraphics[width=40.740mm,height=4.476mm]{./LigaMatematycznaMatuskiego_SP_Zestaw4_2015_2016_page0_images/image001.eps}
\end{center}
LIGA MATEMATYCZNA

im. Zdzisława Matuskiego

PÓLFINAL
\begin{center}
\includegraphics[width=34.548mm,height=42.576mm]{./LigaMatematycznaMatuskiego_SP_Zestaw4_2015_2016_page0_images/image002.eps}
\end{center}
2 marca 20l5

SZKOLA PODSTAWOWA

ZADANIE I.

Do liczby 18 dopisz jedną cyfrę na końcu 1ub na początku, 1ub w środku tak, aby otrzymana

liczba trzycyfrowa była podzielna przez 6. Wyznacz wszystkie takie 1iczby.

ZADANIE 2.

W trójkącie równoramiennym ABC, o ramionach AC i BC, połqczono środek E boku AC

z wierzcholkiem B oraz środek D boku BC z wierzcholkiem A. Obwód trójkąta ABC jest

równy 50, a obwód trójkąta ABE jest o 8 większy od obwodu trójkąta ADC. Ob1icz d1ugości

boków trójkąta ABC.

ZADANIE 3.

Przy ognisku na kocach siedziały elfy i skrzaty. Wszystkich duszków leśnych bylo mniej $\mathrm{n}\mathrm{i}\dot{\mathrm{z}}$

400. Dla $\mathrm{k}\mathrm{a}\dot{\mathrm{z}}$ dego elfa przygotowano jedną porcję nektaru, a dla $\mathrm{k}\mathrm{a}\dot{\mathrm{z}}$ dego skrzata dwie porcje.

Wszyscy siedzieli na 51 kocach, na $\mathrm{k}\mathrm{a}\dot{\mathrm{z}}$ dym taka sama liczba duszków leśnych, przy czym elfy

stanowiły $\displaystyle \frac{7}{12}$ wszystkich. Ile porcji nektaru przygotowano?

ZADANIE 4.

Uczeń klasy VI kupił cztery podręczniki: do języka polskiego, do języka angielskiego, do ma-

tematyki i przyrody. Wszystkie ksiązki bez podręcznika do języka polskiego kosztowafy 42 z1,

wszystkie bez języka angielskiego 40 zł, wszystkie bez matematyki 38 $\mathrm{z}l$, a wszystkie bez przy-

rody 36 zf. I1e kosztowa1 $\mathrm{k}\mathrm{a}\dot{\mathrm{z}}\mathrm{d}\mathrm{y}$ podręcznik?

ZADANIE 5.

Na spacerze Ania robiła zdjęcia Bartkowi i jego psu. Lącznie zrobiła 24 zdjęcia. Bartek jest

na 18 zdjęciach, a pies na 14. Jaką częšć wszystkich zdjęć stanowią te, na których jest Bartek

razem z psem?


\end{document}