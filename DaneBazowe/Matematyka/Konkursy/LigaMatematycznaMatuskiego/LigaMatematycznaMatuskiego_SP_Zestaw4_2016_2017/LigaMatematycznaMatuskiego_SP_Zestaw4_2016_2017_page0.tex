\documentclass[a4paper,12pt]{article}
\usepackage{latexsym}
\usepackage{amsmath}
\usepackage{amssymb}
\usepackage{graphicx}
\usepackage{wrapfig}
\pagestyle{plain}
\usepackage{fancybox}
\usepackage{bm}

\begin{document}

LIGA MATEMATYCZNA

im. Zdzisława Matuskiego

PÓLFINAL

291utego 20l6

SZKOLA PODSTAWOWA

ZADANIE I.

Kwadrat, którego dlugość boku jest równa 12, podzie1ono na mniejsze kwadraty (najmniejszy

ma bok o dfugości l) i prostokąty, które nie są kwadratami. Na rysunku podano dfugości

boków niektórych figur. Które figury- kwadraty czy prostokąty- zajmuja większa część $\mathrm{d}\mathrm{u}\dot{\mathrm{z}}$ ego

kwadratu?
\begin{center}
\includegraphics[width=44.556mm,height=45.816mm]{./LigaMatematycznaMatuskiego_SP_Zestaw4_2016_2017_page0_images/image001.eps}
\end{center}
3

2

2

1

3

6

2 2 4

ZADANIE 2.

Do liczby 36 dopisz po jednej cyfrze na końcu i na początku tak, aby otrzymana 1iczba cztero-

cyfrowa byla podzielna przez 36. Podaj wszystkie $\mathrm{m}\mathrm{o}\dot{\mathrm{z}}$ liwości.

ZADANIE 3.

Mama postawila na stole tacę z cukierkami dla Ani, Bartka i Czarka. Ania wzięla trzecią część

wszystkich cukierków, Bartek- trzecia część tego, co zostalo na tacy. Na końcu Czarek wzial

trzecią część reszty cukierków. Na tacy pozostaly 24 cukierki. $\mathrm{W}$ jaki sposób mama powinna

podzielić pozostałe cukierki, aby $\mathrm{k}\mathrm{a}\dot{\mathrm{z}}$ de dziecko otrzymalo jedną trzecią wszystkich cukierków?

ZADANIE 4.

Wojtek wypisa112 ko1ejnych 1iczb natura1nych w porządku rosnącym. Gdy dodaf co drugą

z nich, zaczynajac od drugiej, otrzyma13330. Jaką sumę uzyska, gdy doda co trzecią 1iczbę,

zaczynając od trzeciej?

ZADANIE 5.

Wykaz, $\dot{\mathrm{z}}\mathrm{e}$ liczba 13333$\ldots$ 35 jest podzielna przez l5.
\begin{center}
\includegraphics[width=20.172mm,height=6.852mm]{./LigaMatematycznaMatuskiego_SP_Zestaw4_2016_2017_page0_images/image002.eps}
\end{center}
$2016$ cyfr 3
\end{document}
