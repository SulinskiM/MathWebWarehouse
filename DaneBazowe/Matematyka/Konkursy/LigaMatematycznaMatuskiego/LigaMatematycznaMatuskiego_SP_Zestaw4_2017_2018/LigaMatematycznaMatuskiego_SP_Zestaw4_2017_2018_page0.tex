\documentclass[a4paper,12pt]{article}
\usepackage{latexsym}
\usepackage{amsmath}
\usepackage{amssymb}
\usepackage{graphicx}
\usepackage{wrapfig}
\pagestyle{plain}
\usepackage{fancybox}
\usepackage{bm}

\begin{document}

LIGA MATEMATYCZNA

im. Zdzisława Matuskiego

PÓLFINAL

161utego 20l7

SZKOLA PODSTAWOWA

ZADANIE I.

Monika wykonala pięćdziesiąt rzutów sześcienną kostką i otrzymala w sumie 100 oczek. I1e co naj-

$\mathrm{w}\mathrm{y}\dot{\mathrm{z}}$ ej razy mogla wypaść,,piątka''?

ZADANIE 2.

$\mathrm{W}$ pewnej grze komputerowej Bartek zdobyl najpierw 157 punktów, potem ki1ka razy straci1

po 19 punktów, a następnie odrobi1 pofowę strat i skończy1 grę z rezu1tatem 100 punktów.

Ile razy poniósl stratę?

ZADANIE 3.

Piotrek wypisaf wszystkie rózne liczby zapisane za pomocą trzech trójek i trzech zer.

sumę tych liczb.

Oblicz

ZADANIE 4.

Prostokąt o polu 100 podzie1ono na trzy prostokąty, z których jeden ma obwód 21 i dfugość 8,

a drugi ma obwód 23 i szerokość 1, 5. Ob1icz po1e trzeciego prostokąta.

ZADANIE 5.

Liczba A jest podzielna przez 42 i 45. Czy 1iczba A dzie1i się przez 210?
\end{document}
