\documentclass[a4paper,12pt]{article}
\usepackage{latexsym}
\usepackage{amsmath}
\usepackage{amssymb}
\usepackage{graphicx}
\usepackage{wrapfig}
\pagestyle{plain}
\usepackage{fancybox}
\usepackage{bm}

\begin{document}

LIGA MATEMATYCZNA

im. Zdzisława Matuskiego

PÓLFINAL

211utego 20l8

SZKOLA PODSTAWOWA

(klasy IV - VI)

ZADANIE I.

$\mathrm{Z}$ czterocyfrowej liczby pierwszej Adam wymazal jedną cyfrę i otrzyma1630. Wyznacz tę 1iczbę

czterocyfrową.

ZADANIE 2.

$K\mathrm{a}\dot{\mathrm{z}}\mathrm{d}\mathrm{y}$ z trzech synów państwa Malinowskich ma calkowitą liczbę lat. Iloczyn ich lat jest równy

18, a za rok wyniesie 60. Podaj wiek $\mathrm{k}\mathrm{a}\dot{\mathrm{z}}$ dego z nich.

ZADANIE 3.

$\mathrm{W}$ trapezie równoramiennym przekątna dzieli kąt ostry na pofowy. Dluzsza podstawa trapezu

ma dfugość 24, a obwód jest równy 54. Ob1icz d1ugości pozosta1ych boków trapezu.

ZADANIE 4.

Adam, Bartek i Czarek mają razem 30 pi1ek. Gdy Bartek da15 pifek Czarkowi, Czarek daf

4 pifki Adamowi, Adam dal 2 pifki Bartkowi, to okazalo się, $\dot{\mathrm{z}}\mathrm{e}$ wszyscy mająjednakową liczbę

pifek. Ile pifek mieli na początku?

ZADANIE 5.

$\mathrm{W}$ pewnym trójkącie równoramiennym kąt między dwusiecznymi jednakowych kątów jest trzy

razy większy $\mathrm{n}\mathrm{i}\dot{\mathrm{z}}\mathrm{k}\mathrm{a}\mathrm{t}$ między ramionami trójkąta. Wyznacz miary katów trójkąta.
\end{document}
