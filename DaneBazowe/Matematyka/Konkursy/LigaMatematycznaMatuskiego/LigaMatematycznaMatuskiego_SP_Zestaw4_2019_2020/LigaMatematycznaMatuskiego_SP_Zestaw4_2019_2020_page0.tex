\documentclass[a4paper,12pt]{article}
\usepackage{latexsym}
\usepackage{amsmath}
\usepackage{amssymb}
\usepackage{graphicx}
\usepackage{wrapfig}
\pagestyle{plain}
\usepackage{fancybox}
\usepackage{bm}

\begin{document}

50

$\rightarrow\not\subset \mathrm{D}\vdash$

flkademia

P omorskawStupsku

LIGA MATEMATYCZNA

im. Zdzisława Matuskiego

PÓLFINAL 261utego 2019

SZKOLA PODSTAWOWA

(klasy IV- VI)

ZADANIE I.

$\mathrm{W}$ pojedynczym ruchu $\mathrm{m}\mathrm{o}\dot{\mathrm{z}}$ na albo wrzucić do urny jedną kulkę, albo podwoić liczbę kulek

znajdujących się w urnie. Wyznacz najmniejszą liczbę ruchów pozwalającą zamienić pustą

urnę w urnę zawierającą 200 ku1ek.

ZADANIE 2.

Bartek, jego ojciec i $\mathrm{k}\mathrm{a}\dot{\mathrm{z}}\mathrm{d}\mathrm{y}$ z dwóch braci: Darek i Czarek obchodzą urodziny w lutym. Wia-

domo, $\dot{\mathrm{z}}\mathrm{e}$ mnoząc liczby lat obu braci Bartka otrzymujemy wiek Bartka, a mnoząc liczby lat

całej trójki rodzeństwa dostajemy wiek taty. Ile lat ma tata Bartka, $\mathrm{j}\mathrm{e}\dot{\mathrm{z}}$ eli wiadomo, $\dot{\mathrm{z}}\mathrm{e}$ ma

mniej $\mathrm{n}\mathrm{i}\dot{\mathrm{z}}50$ lat i $\mathrm{k}\mathrm{a}\dot{\mathrm{z}}$ de z jego dzieci jest w innym wieku?

ZADANIE 3.

Ile jest liczb trzycyfrowych podzielnych przez 9, które $\mathrm{m}\mathrm{o}\dot{\mathrm{z}}$ na ulozyć z cyfr 2, 7, 9, 0 wykorzy-

stując w jednej liczbie $\mathrm{k}\mathrm{a}\dot{\mathrm{z}}$ dą z cyfr co najwyzej raz? Podaj wszystkie $\mathrm{m}\mathrm{o}\dot{\mathrm{z}}$ liwości.

ZADANIE 4.

$\mathrm{W}$ równolegfoboku ABCD bok $AB$ jest dwa razy dluzszy od boku $BC$. Punkt $M$, dzielący

bok $AB$ na polowy, polączono z punktami $C\mathrm{i}D$. Oblicz miarę kąta $CMD.$

ZADANIE 5.

Obwód pięciokąta wypuklego ABCDE jest równy 74, obwód czworokąta ABCD jest równy

56, a czworokata ACDE- 37. Oblicz obwód trójkąta $ACD.$
\begin{center}
\includegraphics[width=74.268mm,height=54.912mm]{./LigaMatematycznaMatuskiego_SP_Zestaw4_2019_2020_page0_images/image001.eps}
\end{center}
D

E

c

A

B
\end{document}
