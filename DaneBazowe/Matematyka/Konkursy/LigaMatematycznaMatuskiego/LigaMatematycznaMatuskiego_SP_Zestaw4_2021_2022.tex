\documentclass[a4paper,12pt]{article}
\usepackage{latexsym}
\usepackage{amsmath}
\usepackage{amssymb}
\usepackage{graphicx}
\usepackage{wrapfig}
\pagestyle{plain}
\usepackage{fancybox}
\usepackage{bm}

\begin{document}

flkademia

P omorskamStupsku

LIGA MATEMATYCZNA

im. Zdzisława Matuskiego

PÓLFINAL 16 marca 2022

SZKOLA PODSTAWOWA

klasy VII- VIII

ZADANIE I.

Adam narysowaf dwa jednakowe kwadraty. Następnie jeden podzielif na osiem, a drugi na

trzynaście mniejszych kwadratów. Oblicz stosunek pola największego kwadratu z podzialu na

osiem części do pola najmniejszego kwadratu z podzialu na trzynaście części.

ZADANIE 2.

Suma 20221iczb ca1kowitych jest 1iczbą nieparzystą. Uzasadnij, $\dot{\mathrm{z}}\mathrm{e}$ iloczyn tych liczb jest liczbą

parzysta.

ZADANIE 3.

Ania podzielila liczbę 13 przez 10 róznych 1iczb natura1nych nie większych $\mathrm{n}\mathrm{i}\dot{\mathrm{z}}13$ i otrzymafa

reszty, których suma jest równa 13. Przez które 1iczby dzie1ifa?

ZADANIE 4.

Cyfrą jedności liczby trzycyfrowej $A$ jest 3. $\mathrm{J}\mathrm{e}\dot{\mathrm{z}}$ eli do liczby $A$ dodamy 3 i uzyskana sumę

podzielimy przez 3, to powstanie 1iczba trzycyfrowa, która na miejscu setek ma 1, a ko1ejne jej

cyfry są pierwszą i drugą cyfrą liczby $A$ (licząc od lewej strony). Wyznacz $A.$

ZADANIE 5.

Szyfr otwierający zamek kuferka Basi składa się z czterech róznych cyfr. Liczba czterocyfrowa

tworząca szyfr dzieli się przez 17 $\mathrm{i}137$, a suma jej cyfr jest $\mathrm{m}\mathrm{o}\dot{\mathrm{z}}$ liwie najmniejsza. Znajd $\acute{\mathrm{z}}$ ten

szyfr.


\end{document}