\documentclass[a4paper,12pt]{article}
\usepackage{latexsym}
\usepackage{amsmath}
\usepackage{amssymb}
\usepackage{graphicx}
\usepackage{wrapfig}
\pagestyle{plain}
\usepackage{fancybox}
\usepackage{bm}

\begin{document}

LIGA MATEMATYCZNA

FINAL

25 kwietnia 2009

SZKOLA PODSTAWOWA

ZADANIE I.

Dziesięć śliwek $\mathrm{w}\mathrm{a}\dot{\mathrm{z}}\mathrm{y}$ tyle, co trzy jablka i gruszka. Jabfko i gruszka wazą tyle, co sześć śliwek.

Waga ilu śliwek jest równa wadze jednej gruszki?

ZADANIE 2.

Na prostej obrano kolejno pięć punktów $A, B, C, D, E$. Wiadomo, $\dot{\mathrm{z}}\mathrm{e}AB=19$ cm, $CE=97$ cm,

$AC=BD$. Znajdz' długošć odcinka $DE.$

ZADANIE 3.

$\mathrm{W}$ prostokącie jeden z boków stanowi $\displaystyle \frac{2}{3}$ drugiego. $\mathrm{Z}$ wierzcholka prostokąta do środka dluzszego

boku poprowadzono odcinek. Dzieli on prostokąt na dwie figury: trójkąt o obwodzie równym

12 cm i trapez o obwodzie l8 cm. Oblicz obwód prostokąta.

ZADANIE 4.

Adam, Bartek i Witek uczą się w tej samej klasie. Jeden z nich dojezdza do szkoły autobusem,

drugi tramwajem, a trzeci rowerem. Pewnego dnia Adam odprowadzal kolegę na przystanek

autobusowy. $\mathrm{W}$ tym samym czasie obok nich przejechał rowerem trzeci kolega i zawolal:

,,Bartek, zostawifeś zeszyt w szkole'' Jakim środkiem lokomocji dojez $\mathrm{d}\dot{\mathrm{z}}$ a $\mathrm{k}\mathrm{a}\dot{\mathrm{z}}\mathrm{d}\mathrm{y}$ z nich?

ZADANIE 5.

Państwo Kowalscy i Wiśniewscy mają po dwóch synów, z których $\mathrm{k}\mathrm{a}\dot{\mathrm{z}}\mathrm{d}\mathrm{y}$ ma mniej $\mathrm{n}\mathrm{i}\dot{\mathrm{z}}9$ lat.

$\mathrm{W}\mathrm{k}\mathrm{a}\dot{\mathrm{z}}$ dej rodzinie jeden z synów ma więcej $\mathrm{n}\mathrm{i}\dot{\mathrm{z}} 5$ lat, a drugi mniej. Andrzej jest o 31ata

mlodszy od swojego brata. Wojtek jest najstarszy ze wszystkich chlopców. Krzyś jest o 21ata

młodszy od młodszego syna państwa Kowalskich, a Robert jest 51at starszy od młodszego syna

państwa Wiśniewskich. Podaj imiona i nazwiska chlopców oraz ich wiek.
\end{document}
