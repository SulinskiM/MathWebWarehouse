\documentclass[a4paper,12pt]{article}
\usepackage{latexsym}
\usepackage{amsmath}
\usepackage{amssymb}
\usepackage{graphicx}
\usepackage{wrapfig}
\pagestyle{plain}
\usepackage{fancybox}
\usepackage{bm}

\begin{document}

LIGA MATEMATYCZNA

FINAL

26 marca 20l0

SZKOLA PODSTAWOWA

ZADANIE I.

Dziesięć pająków zjada dziesięć much w ciągu dwudziestu sekund. Ile czasu potrzeba stu

pająkom na zjedzenie stu much?

ZADANIE 2.

Suma trzynastu róznych liczb całkowitych dodatnich jest równa 92. Wyznacz te 1iczby.

ZADANIE 3.

Kamila, Ania i Marek twierdzą, $\dot{\mathrm{z}}\mathrm{e}$ suma ich lat jest równa 35. Jednak $\dot{\mathrm{z}}$ adne z nich nie podało

prawdziwego wieku. Kamila zanizyła swój wiek o 31ata, Ania zawyzyła o 21ata, a Marek

postarzył się o 41ata. Kiedy naprawdę suma ich 1at będzie równa 35?

ZADANIE 4.

$\acute{\mathrm{S}}$limak wspina się na drzewo o wysokości 10 $\mathrm{m}. \mathrm{W}$ ciągu dnia podnosi się o 4 metry, a w nocy

obsuwa się o 3 metry. Po i1u dniach ś1imak dostanie się na wierzcho1ek drzewa?

ZADANIE 5.

Prostokqt podzielono na dziewięć róznych kwadratów (rysunek na odwrocie). Długość boku

najmniejszego kwadratu jest równa l.

(a) Znajd $\acute{\mathrm{z}}$ długość boku zamalowanego kwadratu.

(b) Znajdz'dlugošci boków prostokąta.
\end{document}
