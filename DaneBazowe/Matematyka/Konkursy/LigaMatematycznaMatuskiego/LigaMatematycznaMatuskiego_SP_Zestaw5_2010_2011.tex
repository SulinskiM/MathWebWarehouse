\documentclass[a4paper,12pt]{article}
\usepackage{latexsym}
\usepackage{amsmath}
\usepackage{amssymb}
\usepackage{graphicx}
\usepackage{wrapfig}
\pagestyle{plain}
\usepackage{fancybox}
\usepackage{bm}

\begin{document}

LIGA MATEMATYCZNA

FINAL

30 marca 20ll

SZKOLA PODSTAWOWA

ZADANIE I.

Trapez o obwodzie 72 cm podzie1ono wysokošciami na dwa trójkąty i prostokąt. Suma obwodów

tych trzech figur jest równa 142 cm. Ob1icz wysokość tego trapezu.

ZADANIE 2.

$1=1^{2}-0^{2}$

$3=2^{2}-1^{2}$

$5=3^{2}-2^{2}$

Napisz trzy kolejne wiersze. Przedstaw liczbę 2011 w postaci róznicy kwadratów dwóch 1iczb

naturalnych.

ZADANIE 3.

Skrzynia, kufry i pudelka mają zamki. $\mathrm{W}$ skrzyni jest sześć kufrów, w $\mathrm{k}\mathrm{a}\dot{\mathrm{z}}$ dym kufrze są po trzy

pudełka, a w $\mathrm{k}\mathrm{a}\dot{\mathrm{z}}$ dym pudełku po trzy zlote monety. Jaka jest najmniejsza liczba zamków, które

trzeba otworzyć, aby wyjqć 22 z1ote monety?

ZADANIE 4.

$\mathrm{W}$ pewnej klasie jest 30 uczniów. Wšród nich pięciu ma brata i siostrę, a siedmiu nie ma brata

ani siostry. Ilu uczniów tej klasy ma brata, $\mathrm{j}\mathrm{e}\dot{\mathrm{z}}$ eli wiadomo, $\dot{\mathrm{z}}\mathrm{e}$ trzynastu ma siostrę?

ZADANIE 5.

$\mathrm{W}$ dane kólka wpisz liczby tak, aby suma liczb w $\mathrm{k}\mathrm{a}\dot{\mathrm{z}}$ dych trzech kolejnych kólkach była

równa 15.
\begin{center}
\includegraphics[width=168.804mm,height=10.872mm]{./LigaMatematycznaMatuskiego_SP_Zestaw5_2010_2011_page0_images/image001.eps}
\end{center}
4


\end{document}