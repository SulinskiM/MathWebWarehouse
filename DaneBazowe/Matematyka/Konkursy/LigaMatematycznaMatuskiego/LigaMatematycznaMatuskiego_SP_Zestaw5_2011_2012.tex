\documentclass[a4paper,12pt]{article}
\usepackage{latexsym}
\usepackage{amsmath}
\usepackage{amssymb}
\usepackage{graphicx}
\usepackage{wrapfig}
\pagestyle{plain}
\usepackage{fancybox}
\usepackage{bm}

\begin{document}

l LiceumOgóloksztalcacewSlpsku

AkadmiPomorskawSiupsku

LIGA MATEMATYCZNA

FINAL

ll kwietnia 2012

SZKOLA PODSTAWOWA

ZADANIE I.

Piszemy liczbę 9, następnie 8 i znowu 8. Potem piszemy największą jednocyfrową 1iczbę ca1ko-

witą dodatniq, która nie wystąpiła na trzech poprzednich miejscach. Znowu piszemy największą

jednocyfrową liczbę calkowitą dodatniq, która nie wystąpila na trzech poprzednich miejscach.

Wypisywanie liczb o tej wlasności kontynuujemy. Jaka liczba będzie na 2012 miejscu?

ZADANIE 2.

$\mathrm{W}$ domach przy ulicy Owocowej mieszkają Jabłońscy, S$\acute{}$liwińscy i Wiśniewscy. Jablońscy miesz-

kają w 12 domach, S$\acute{}$1iwińscy w 16, Wiśniewscy w 14. $\mathrm{W}8$ domach mieszkają S$\acute{}$liwińscy i Wi-

śniewscy, $\mathrm{w}7$- Jablońscy i Wiśniewscy, $\mathrm{w}5$- Jabfońscy i S$\acute{}$liwińscy, a w 4- Jabłońscy, S$\acute{}$1iwińscy

i Wiśniewscy. Ile jest domów przy tej ulicy? $\mathrm{W}$ ilu domach mieszkają rodziny o tylko jednym

nazwisku?

ZADANIE 3.

$\mathrm{W}\mathrm{k}\mathrm{a}\dot{\mathrm{z}}$ dym z siedmiu kolejnych lat, zawsze ll kwietnia, urodzif się jeden krasnoludek.

najmłodsze krasnoludki mają razem 421ata. I1e 1at mają razem trzy najstarsze?

Trzy

ZADANIE 4.

Ania i Jarek stoją w kolejce po bilety na koncert. Jarek jest blizej kasy $\mathrm{n}\mathrm{i}\dot{\mathrm{z}}$ Ania. Między nimi

stoją trzy osoby. Za Jarkiem ustawifo się 10 osób, a przed Anią 8 osób. I1e osób stoi w ko1ejce?

Które miejsce w kolejce zajmuje Ania, a które Jarek?

ZADANIE 5.

Przedstawiony na rysunku prostokąt sklada się z szešciu kwadratów.

bok dfugości 2 cm. Ob1icz po1e prostokąta.

Najmniejszy z nich ma


\end{document}