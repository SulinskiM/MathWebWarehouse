\documentclass[a4paper,12pt]{article}
\usepackage{latexsym}
\usepackage{amsmath}
\usepackage{amssymb}
\usepackage{graphicx}
\usepackage{wrapfig}
\pagestyle{plain}
\usepackage{fancybox}
\usepackage{bm}

\begin{document}

LIGA MATEMATYCZNA

im. Zdzisława Matuskiego

FINAL

10 kwietnia 20l3

SZKOLA PODSTAWOWA

ZADANIE I.

Kwadrat ABCD podzielono na mniejsze kwadraty tak, jak na rysunku. Ile ijakich kwadratów

trzeba zamalować, aby powierzchnia zamalowana stanowila piątą część powierzchni kwadratu

ABCD? Kwadratów nie $\mathrm{m}\mathrm{o}\dot{\mathrm{z}}$ na dzielić na mniejsze części.
\begin{center}
\includegraphics[width=47.604mm,height=45.420mm]{./LigaMatematycznaMatuskiego_SP_Zestaw5_2012_2013_page0_images/image001.eps}
\end{center}
D  c

A  B

ZADANIE 2.

Wyznacz cztery rózne liczby naturalne parzyste, których iloczyn jest równy 8880.

ZADANIE 3.

Dany jest kwadrat ABCD oraz trójkąt równoboczny ABE, gdzie bok AB jest wspólny dla obu

figur. Wyznacz miarę kąta DEC. Rozwaz wszystkie przypadki pofozenia punktu E.

ZADANIE 4.

Liczba uczniów pewnej szkoly podstawowej jest zawarta między 500 a 1000. Gdy grupujemy

ich po 181ub po 20, 1ub po 24, to za $\mathrm{k}\mathrm{a}\dot{\mathrm{z}}$ dym razem pozostaje 9 uczniów. I1u uczniów uczęszcza

do tej szkoly?

ZADANIE 5.

$\mathrm{W}$ dwóch workach bylo 216 kg mąki. Jeś1i z pierwszego worka przesypiemy do drugiego 93 kg,

a następnie z drugiego worka przesypiemy do pierwszego tyle, aby jego zawartość podwoiła

się, to w obu workach będzie tyle samo mąki. Ile kilogramów mąki bylo w $\mathrm{k}\mathrm{a}\dot{\mathrm{z}}$ dym worku

na początku?


\end{document}