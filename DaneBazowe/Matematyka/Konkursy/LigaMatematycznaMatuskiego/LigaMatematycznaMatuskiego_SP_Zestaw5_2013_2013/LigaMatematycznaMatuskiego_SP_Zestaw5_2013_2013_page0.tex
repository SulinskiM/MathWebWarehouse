\documentclass[a4paper,12pt]{article}
\usepackage{latexsym}
\usepackage{amsmath}
\usepackage{amssymb}
\usepackage{graphicx}
\usepackage{wrapfig}
\pagestyle{plain}
\usepackage{fancybox}
\usepackage{bm}

\begin{document}

LIGA MATEMATYCZNA

im. Zdzisława Matuskiego

PÓLFINAL

71utego 20l3

SZKOLA PODSTAWOWA

ZADANIE I.

Kwadrat ABCD podzielono na czteryjednakowe większe prostokaty, czteryjednakowe mniejsze

prostokąty oraz kwadrat. Kwadraty ABCD, EFGH, IJKL mają obwody równe odpowiednio

360 cm, l20 cm oraz 40 cm. Wyznacz obwody mniejszych i większych prostokatów.

$\displaystyle \bigcap_{d}$

H

K

G

J

E

F

{\it 3}

ZADANIE 2.

$\mathrm{W}$ trójkącie równobocznym odcięto od jednego naroza trójkąt równoramienny, a od drugiego

- trójkąt prostokatny tak, $\dot{\mathrm{z}}\mathrm{e}$ pozostala część jest pięciokatem. Wyznacz miary kątów tego

pięciokąta.

ZADANIE 3.

Wojtek ma trzy patyczki: czerwony, zielony i niebieski. Ich dlugości to 2 cm, 3 cm oraz 5 cm

(kolejność tych liczb nie musi odpowiadać kolejności kolorów). Za pomocą $\mathrm{k}\mathrm{a}\dot{\mathrm{z}}$ dego patyczka

Wojtek zmierzyl dlugość krawędzi stolu. Zielony patyk zmieścif się 75 razy, a niebieski 50 razy.

Czerwony takze zmieścil się calkowitą liczbę razy- ile?

ZADANIE 4.

Wyznacz trzy kolejne liczby naturalne, których iloczyn jest sto razy większy od największej

liczby czterocyfrowej.

ZADANIE 5.

W Tfusty Czwartek mama kupifa mini-pączki dla swojej licznej rodzinki. Wojtek z Darkiem

otrzymali trzecią część wszystkich i jeszcze trzy mini-pączki. Agnieszka z Basią wzięfy trze-

cią częśč pozostalych i jeszcze dwa. Pofowę pozostalych mini-pączków mama dala Jarkowi,

a ostatnie sześć zjadla z tata do kawy. Ile mini-paczków kupila mama?
\end{document}
