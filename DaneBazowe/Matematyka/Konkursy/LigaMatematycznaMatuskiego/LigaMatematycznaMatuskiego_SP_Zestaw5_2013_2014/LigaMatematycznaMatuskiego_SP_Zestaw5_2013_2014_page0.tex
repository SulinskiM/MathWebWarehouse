\documentclass[a4paper,12pt]{article}
\usepackage{latexsym}
\usepackage{amsmath}
\usepackage{amssymb}
\usepackage{graphicx}
\usepackage{wrapfig}
\pagestyle{plain}
\usepackage{fancybox}
\usepackage{bm}

\begin{document}

LIGA MATEMATYCZNA

im. Zdzisława Matuskiego

FINAL

15 kwietnia 20l4

SZKOLA PODSTAWOWA

ZADANIE I.

Znajd $\acute{\mathrm{z}}$ najmniejsza liczbę naturalną większą od 2014 i majacą taką samą sumę cyfr jak 2014.

ZADANIE 2.

Sześć dziewczynek $\mathrm{w}\mathrm{a}\dot{\mathrm{z}}$ acych 18 kg, 19 kg, 20 kg, 23 kg, 38 kg oraz 42 kg ma podzie1ić się na

dwie grupy w taki sposób, aby lączna waga dziewczynek w $\mathrm{k}\mathrm{a}\dot{\mathrm{z}}$ dej grupie byfa taka sama. Na ile

sposobów $\mathrm{m}\mathrm{o}\dot{\mathrm{z}}$ na dokonač takiego podziafu?

ZADANIE 3.

W rodzinie Wojtka są cztery osoby. Suma ich lat jest równa 100. Wojtek jest o cztery 1ata

starszy od Asi, a tata jest o sześć lat starszy od mamy. Asia poprosifa zlotą rybkę, aby cof-

nęla czas o calkowitą liczbę lat do takiego momentu, w którym Asia byla sześć razy młodsza

od mamy. Zfota rybka zastanowila się i cofnęla czas o pięć lat. Ile lat mają czlonkowie rodziny

po cofnięciu czasu?

ZADANIE 4.

Obwód kwadratu jest równy 32 cm. $\acute{\mathrm{S}}$ rodki dwóch kolejnych boków tego kwadratu polączono

ze soba i z wierzchołkiem nie nalezącym do tych boków. Oblicz pole otrzymanego w ten sposób

trójkąta.

ZADANIE 5.

Adam, Bartek, Czarek i Darek lubią milo spędzać czas wolny. $K\mathrm{a}\dot{\mathrm{z}}\mathrm{d}\mathrm{y}$ wybiera swoje ulubione

miejsce i dociera tam w inny sposób. Odkryj, kto dokad wychodzi i jak się tam dostaje, $\mathrm{j}\mathrm{e}\dot{\mathrm{z}}$ eli:

$\bullet$ Bartek nigdy nie chodzi do kina i zawsze $\mathrm{j}\mathrm{e}\acute{\mathrm{z}}\mathrm{d}\mathrm{z}\mathrm{i}$ pociągiem;

$\bullet$ Adam nie opuszcza $\dot{\mathrm{z}}$ adnego koncertu w filharmonii, ale nie ma roweru;

$\bullet$ Czarek jest bywalcem muzeów. Udaje się tam na piechotę;

$\bullet$ Jeden z kolegów spędza wieczory w teatrze, a inny wszędzie $\mathrm{j}\mathrm{e}\acute{\mathrm{z}}\mathrm{d}\mathrm{z}\mathrm{i}$ samochodem.
\end{document}
