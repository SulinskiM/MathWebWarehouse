\documentclass[a4paper,12pt]{article}
\usepackage{latexsym}
\usepackage{amsmath}
\usepackage{amssymb}
\usepackage{graphicx}
\usepackage{wrapfig}
\pagestyle{plain}
\usepackage{fancybox}
\usepackage{bm}

\begin{document}

AHADEMIA POMORSHA

III SLUPSHU
\begin{center}
\includegraphics[width=40.740mm,height=4.476mm]{./LigaMatematycznaMatuskiego_SP_Zestaw5_2015_2016_page0_images/image001.eps}
\end{center}
LIGA MATEMATYCZNA

im. Zdzislawa Matuskiego

FINAL
\begin{center}
\includegraphics[width=34.548mm,height=42.576mm]{./LigaMatematycznaMatuskiego_SP_Zestaw5_2015_2016_page0_images/image002.eps}
\end{center}
16 kwietnia 20l5

SZKOLA PODSTAWOWA

ZADANIE I.

W pewnym biurowcu w Słupsku jest 200 okien. Rano otwartych było 60 okien. Po połu-

dniu zamknięto co drugie okno, a następnie otwarto co drugie okno zamknięte. Ile okien jest

otwartych?

ZADANIE 2.

Dwie kostki, piramida i walec $\mathrm{w}\mathrm{a}\dot{\mathrm{z}}$ a 17 kg. Kostka, dwie piramidy i wa1ec wazą 14 kg. Kostka,

piramida i dwa walce $\mathrm{w}\mathrm{a}\dot{\mathrm{z}}$ a 13 kg. Usta1, i1e $\mathrm{w}\mathrm{a}\dot{\mathrm{z}}\mathrm{y}\mathrm{k}\mathrm{a}\dot{\mathrm{z}}\mathrm{d}\mathrm{y}$ przedmiot.

ZADANIE 3.

$\mathrm{W}$ pięciokącie jedna przekątna ma 7 cm d1ugości, a druga- wychodząca z tego samego wierz-

chofka- ma 8 cm d1ugości. Przekątne te podzie1i1y pięciokąt na trzy trójkąty, $\mathrm{k}\mathrm{a}\dot{\mathrm{z}}\mathrm{d}\mathrm{y}$ o obwodzie

równym 20 cm. Ob1icz obwód pięciokąta.

ZADANIE 4.

W trójkącie równoramiennymjeden z kątów jest cztery razy większy od drugiego. Oblicz miary

kątów tego trójkąta. Rozwaz wszystkie przypadki.

ZADANIE 5.

Ania zbiera pocztówki z kwiatami. Ma ich więcej $\mathrm{n}\mathrm{i}\dot{\mathrm{z}}500$, ale mniej $\mathrm{n}\mathrm{i}\dot{\mathrm{z}}900$. Chce je umieścić

w kopertach, w $\mathrm{k}\mathrm{a}\dot{\mathrm{z}}$ dej tę samą ilość. Gdy wklada po 16, zostają 2 pocztówki. Tak samo, gdy

wkfada po 24 i po 30. I1e pocztówek ma Ania? Zaproponuj takie rozłozenie kartek, aby $\dot{\mathrm{z}}$ adna

nie zostafa i aby $\mathrm{u}\dot{\mathrm{z}}$ yć jak najmniej kopert. $\mathrm{W}$ jednej kopercie nie zmieści się więcej $\mathrm{n}\mathrm{i}\dot{\mathrm{z}} 30$

pocztówek.


\end{document}