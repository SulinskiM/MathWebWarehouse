\documentclass[a4paper,12pt]{article}
\usepackage{latexsym}
\usepackage{amsmath}
\usepackage{amssymb}
\usepackage{graphicx}
\usepackage{wrapfig}
\pagestyle{plain}
\usepackage{fancybox}
\usepackage{bm}

\begin{document}

LIGA MATEMATYCZNA

im. Zdzisława Matuskiego

FINAL

25 kwietnia 20l6

SZKOLA PODSTAWOWA

ZADANIE I.

Znajd $\acute{\mathrm{z}}$ wszystkie liczby czterocyfrowe podzielne przez 4 o sumie cyfr równej 4.

ZADANIE 2.

$\mathrm{D}\mathrm{u}\dot{\mathrm{z}}\mathrm{y}$ prostokat o obwodzie 136 cm podzie1ono na siedem przystających prostokątów tak, jak

na rysunku. Oblicz pole $\mathrm{d}\mathrm{u}\dot{\mathrm{z}}$ ego prostokata.

ZADANIE 3.

Wykaz$\cdot, \dot{\mathrm{z}}\mathrm{e}$ liczba $10^{45}+2$ jest podzielna przez 6.

ZADANIE 4.

Tysiąc punktów umieszczono równomiernie na okręgu i ponumerowano kolejno od l do 1000.

Jaki numer ma punkt lezący naprzeciw punktu o numerze 657?

ZADANIE 5.

Ramię trapezu równoramiennego ma dlugość 5 cm. Obwód trapezu jest równy 28 cm. Prosta

przechodząca przez środki podstaw podzielila ten trapez na dwie figury o obwodach po 18 cm.

Oblicz pole trapezu.
\end{document}
