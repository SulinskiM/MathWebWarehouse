\documentclass[a4paper,12pt]{article}
\usepackage{latexsym}
\usepackage{amsmath}
\usepackage{amssymb}
\usepackage{graphicx}
\usepackage{wrapfig}
\pagestyle{plain}
\usepackage{fancybox}
\usepackage{bm}

\begin{document}

LIGA MATEMATYCZNA

im. Zdzisława Matuskiego

FINAL

24 kwietnia 20l7

SZKOLA PODSTAWOWA

ZADANIE I.

Ania miafa 96 jednakowych patyczków i zbudowa1a z nich kwadraty i trójkąty. Boki wszyst-

kich figur mialy dfugość jednego patyczka. Powstafo 27 roz1ącznych figur przy wykorzystaniu

wszystkich patyczków. Ile kwadratów i ile trójkątów zbudowala Ania?

ZADANIE 2.

Wszystkie figury znajdujące się wewnątrz prostokąta są kwadratami. Czarny kwadrat ma pole l,

kwadrat $A$ ma pole 81. Wyznacz po1e kwadratu $X$ oraz oblicz obwód $\mathrm{d}\mathrm{u}\dot{\mathrm{z}}$ ego prostokąta.

{\it A}

{\it X}

ZADANIE 3.

Ania i Bartek odrabiają pracę domową. Ania ma znalez$\acute{}$ć najmniejszą liczbę naturalną podzielną

przez sześć kolejnych liczb nieparzystych, a Bartek - najmniejszą liczbę naturalna podzielną

przez osiem kolejnych liczb parzystych. Czyja liczba będzie mniejsza?

ZADANIE 4.

W pewnej rodzinie jest pięć córek: Ania, Basia, Czesia, Daria i Ela. Rodzily się one w podanej

kolejności co trzy lata. Najstarsza Aniajest siedem razy starsza od najmlodszej Eli. Ile lat ma

Czesia?

ZADANIE 5.

Przez jaką liczbę nalezy podzielić liczby 331 i 459, aby w obu przypadkach otrzymać resztę

z dzielenia równą ll? Podaj wszystkie rozwiązania.
\end{document}
