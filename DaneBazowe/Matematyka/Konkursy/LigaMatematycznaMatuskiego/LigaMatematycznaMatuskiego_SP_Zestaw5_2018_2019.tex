\documentclass[a4paper,12pt]{article}
\usepackage{latexsym}
\usepackage{amsmath}
\usepackage{amssymb}
\usepackage{graphicx}
\usepackage{wrapfig}
\pagestyle{plain}
\usepackage{fancybox}
\usepackage{bm}

\begin{document}

LIGA MATEMATYCZNA

im. Zdzisława Matuskiego

FINAL

16 kwietnia 20l8

SZKOLA PODSTAWOWA

(klasy IV - VI)

ZADANIE I.

W rodzinie Bartka są cztery osoby. Suma ich lat jest równa 100. Bartek jest o 41ata starszy

od Ani, a tata jest o 61at starszy od mamy. Ania poprosi1a z1otą rybkę, aby cofnęfa czas

o calkowitą liczbę lat do momentu, w którym Ania byfa sześč razy młodsza od mamy. Zlota

rybka zastanowifa się i cofnęfa czas o pięč lat. Ile lat mają czlonkowie rodziny po cofnięciu

czasu?

ZADANIE 2.

Podaj wszystkie liczby trzycyfrowe o sumie cyfr równej 9 i cyfrze setek podzie1nej przez 4.

ZADANIE 3.

Sierzant przygotowywal oddzial $\dot{\mathrm{z}}$ ołnierzy do defilady. Próbowal ustawiać ich trójkami, ale

jeden $\dot{\mathrm{z}}$ olnierz pozostawał. Takz $\mathrm{e}$ po ustawieniu czwórkami, piatkami i szóstkamijeden $\dot{\mathrm{z}}$ ołnierz

zostawal. $\mathrm{W}$ końcu ustawil ich siódemkami i wtedy siódemki byly kompletne. Jaka mogla być

najmniejsza liczba $\dot{\mathrm{z}}$ ofnierzy w tym oddziale?

ZADANIE 4.

Wykaz, $\dot{\mathrm{z}}\mathrm{e}$ liczba $10^{20}+19^{3}-2$ jest podzielna przez 9.

ZADANIE 5.

Punkt $S$ jest środkiem okręgu opisanego na trójkącie ostrokątnym $ABC$. Miara kąta $ACS$ jest

trzy razy większa od miary kąta BAS, a miara kąta $CBS$ jest dwa razy większa od miary kąta

BAS. Wyznacz miary kątów trójkąta $ABC.$
\begin{center}
\includegraphics[width=37.344mm,height=36.576mm]{./LigaMatematycznaMatuskiego_SP_Zestaw5_2018_2019_page0_images/image001.eps}
\end{center}
C


\end{document}