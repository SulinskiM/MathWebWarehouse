\documentclass[a4paper,12pt]{article}
\usepackage{latexsym}
\usepackage{amsmath}
\usepackage{amssymb}
\usepackage{graphicx}
\usepackage{wrapfig}
\pagestyle{plain}
\usepackage{fancybox}
\usepackage{bm}

\begin{document}
\begin{center}
\includegraphics[width=20.628mm,height=30.024mm]{./LigaMatematycznaMatuskiego_SP_Zestaw5_2019_2020_page0_images/image001.eps}
\end{center}
0

flkademia

P omorskawStupsku

LIGA MATEMATYCZNA

im. Zdzisława Matuskiego

FINAL 26 marca 2019

SZKOLA PODSTAWOWA

(klasy IV - VI)

ZADANIE I.

Sakiewka Adama zawiera monety srebrne i zlote. Wszystkie monety srebrne są jednakowe

i wszystkie monety zlote są jednakowe. Dwie monety srebrne i cztery złote wazą fącznie 72 g,

a dwie zlote i cztery srebrne wazą 66 g. Ob1icz 1ączną wagę trzech zfotych i trzech srebrnych

monet.

ZADANIE 2.

Znajd $\acute{\mathrm{z}}$ wszystkie liczby trzycyfrowe, których iloczyn cyfr jest liczbą pierwszą.

ZADANIE 3.

Jedna z przekątnych dzieli pewien czworokąt na dwa trójkąty o obwodach 16 $\mathrm{i}18$, druga prze-

kątna dzieli ten czworokąt na trójkaty o obwodach 12 $\mathrm{i}20$. Oblicz róznicę dlugości przekątnych

tego czworokąta.

ZADANIE 4.

Iloczyn trzech liczb naturalnych jest równy 36. Nawet gdy podamy ich sumę, nie będzie wia-

domo, co to za liczby. Oblicz ich sumę.

ZADANIE 5.

Numer szyfru do szkatulki Basi sklada się z dziewięciu róznych cyfr i róznych od 0. $K\mathrm{a}\dot{\mathrm{z}}$ de dwie

kolejne cyfry szyfru róznią się o 31ub 5. Pierwszą cyfrą kodu jest 3. Wyznacz przedostatnią

cyfrę szyfru.


\end{document}