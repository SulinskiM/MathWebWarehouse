\documentclass[a4paper,12pt]{article}
\usepackage{latexsym}
\usepackage{amsmath}
\usepackage{amssymb}
\usepackage{graphicx}
\usepackage{wrapfig}
\pagestyle{plain}
\usepackage{fancybox}
\usepackage{bm}

\begin{document}

LIGA MATEMATYCZNA

$\mathrm{P}\mathrm{A}\acute{\mathrm{Z}}$ DZIERNIK 2009

GIMNAZJUM

ZADANIE I.

Przez wierzchołek kwadratu poprowadzono prostą, która dzieli kwadrat na trójkąt o polu 24 $\mathrm{c}\mathrm{m}^{2}$

i trapez o polu 40 $\mathrm{c}\mathrm{m}^{2}$ Oblicz długości podstaw trapezu.

ZADANIE 2.

Rozwiqz uklad równań

$\left\{\begin{array}{l}
x(y+z)=8\\
y(z+x)=18\\
z(x+y)=20.
\end{array}\right.$

ZADANIE 3.

Uzasadnij, $\dot{\mathrm{z}}\mathrm{e}$ liczba

2007. 2009. 2011$+$8036

jest sześcianem liczby naturalnej.

ZADANIE 4.

Wyznacz wszystkie liczby siedmiocyfrowe podzielne przez 3 i przez 4, w zapisie których wystę-

pują tylko cyfry 2 $\mathrm{i}3$, przy czym dwójek jest więcej $\mathrm{n}\mathrm{i}\dot{\mathrm{z}}$ trójek.

ZADANIE 5.

$\mathrm{W}$ konkursie matematycznym uczeń ma rozwiązać 20 zadań. Za $\mathrm{k}\mathrm{a}\dot{\mathrm{z}}$ de zadanie poprawnie

rozwiązane otrzymuje 12 punktów, za z'1e rozwiązane $(-5)$ punktów, a za brak rozwiązania

0 punktów. $\mathrm{W}$ podsumowaniu otrzyma117 punktów. I1e zadań rozwiąza1 b1ędnie?





LIGA MATEMATYCZNA

$\mathrm{P}\mathrm{A}\acute{\mathrm{Z}}$ DZIERNIK 2010

GIMNAZJUM

ZADANIE I.

Suma i iloczyn pewnych dziesięciu liczb calkowitych są parzyste. Ile najwięcej $\mathrm{m}\mathrm{o}\dot{\mathrm{z}}\mathrm{e}$ być wśród

nich liczb nieparzystych?

ZADANIE 2.

$\mathrm{W}$ turnieju tenisa stolowego wzięło udzial pięćdziesięciu zawodników. $\mathrm{K}\mathrm{a}\dot{\mathrm{z}}\mathrm{d}\mathrm{y}$ zawodnik rozegrał

jeden mecz z $\mathrm{k}\mathrm{a}\dot{\mathrm{z}}$ dym innym zawodnikiem. Nie bylo remisów. Czy $\mathrm{m}\mathrm{o}\dot{\mathrm{z}}$ liwe jest, aby $\mathrm{k}\mathrm{a}\dot{\mathrm{z}}\mathrm{d}\mathrm{y}$

z zawodników wygrał taką samą liczbę meczów?

ZADANIE 3.

Wykaz, $\dot{\mathrm{z}}\mathrm{e}$ wartość wyrazenia $\displaystyle \frac{1}{\sqrt{2}+\sqrt{3}}+\frac{1}{\sqrt{3}+\sqrt{4}}+\frac{1}{\sqrt{4}+\sqrt{5}}+\ldots+\frac{1}{\sqrt{127}+\sqrt{128}}$ jest mniejsza od 10.

ZADANIE 4.

Dany jest czworokąt wypukły ABCD. Niech $E$ będzie środkiem odcinka $AB$ oraz $F-$ środkiem

odcinka $CD$. Oblicz pole czworokąta EBFD wiedząc, $\dot{\mathrm{z}}\mathrm{e}$ pole czworokąta ABCD jest równe 77.

ZADANIE 5.

Jan, Henryk, Stanisfaw i Paweł to znajomi rzemiešlnicy. $K\mathrm{a}\dot{\mathrm{z}}\mathrm{d}\mathrm{y}$ z nich wykonuje inny zawód

i mieszka przy innej ulicy.

Na podstawie podanych informacji okrešl, przy jakiej ulicy $\mathrm{k}\mathrm{a}\dot{\mathrm{z}}\mathrm{d}\mathrm{y}$

z nich mieszka i jaki wykonuje zawód.

$\bullet$ Jan nie jest jubilerem.

$\bullet$ Zegarmistrz nie mieszka przy ulicy Złotej.

$\bullet$ Henryk jest krawcem, ale nie mieszka przy ulicy Glównej.

$\bullet$ Jubiler mieszka przy ulicy Mokrej.

$\bullet$ Stanislaw mieszka przy ulicy Cichej, ale nie jest szewcem.






LIGA MATEMATYCZNA

$\mathrm{P}\mathrm{A}\acute{\mathrm{Z}}$ DZIERNIK 2011

GIMNAZJUM

ZADANIE I.

Pola $P$ niektórych figur plaskich $\mathrm{m}\mathrm{o}\dot{\mathrm{z}}$ emy obliczyć ze wzoru Simpsona

$P=\displaystyle \frac{d_{1}+4d+d_{2}}{6}\cdot h,$

w którym przyjęto następujące oznaczenia:

$d_{1}-$ dlugość dolnej podstawy;

$d-$ dlugość środkowego odcinka, równoleglego do podstawy dolnej w polowie wysokości;

$d_{2}-$ dlugość górnej podstawy;

$h-$ wysokość figury.

$\bullet$ Wykonaj rysunek, wprowad $\acute{\mathrm{z}}$ oznaczenia i $\mathrm{s}\mathrm{p}\mathrm{r}\mathrm{a}\mathrm{w}\mathrm{d}\acute{\mathrm{z}}$, czy ze wzoru Simpsona $\mathrm{m}\mathrm{o}\dot{\mathrm{z}}$ na otrzy-

mać wzór na pole trapezu. Odpowiedz' uzasadnij.

$\bullet$ Sprawd $\acute{\mathrm{z}}$, czy ze wzoru Simpsona $\mathrm{m}\mathrm{o}\dot{\mathrm{z}}$ na wyprowadzić wzór na pole sześciokata foremnego

o boku dlugości $a.$

ZADANIE 2.

$\mathrm{W}\mathrm{k}\mathrm{a}\dot{\mathrm{z}}$ dym kroku wykonujemy na liczbie jedną z operacji:

(a) podwajamy liczbę;

(b) skreślamy jej ostatnią cyfrę.

Czy w taki sposób po skończonej ilości operacji $\mathrm{m}\mathrm{o}\dot{\mathrm{z}}$ na z liczby 458 uzyskać 14?

ZADANIE 3.

Znajd $\acute{\mathrm{z}}$ wszystkie liczby dwucyfrowe $n$ spelniające warunek

$n-p=3\cdot f(n),$

gdzie $p$ oznacza liczbę dwucyfrową powstalą z przestawienia cyfr liczby $n$, a $f(n)-$ sumę cyfr

liczby $n$ oraz iloczynu jej cyfr.

ZADANIE 4.

Wykaz, $\dot{\mathrm{z}}\mathrm{e}$ liczba $(2\sqrt{2}+3)\sqrt{5-12\sqrt{3-2\sqrt{2}}}$ jest calkowita.

ZADANIE 5.

$\mathrm{W}$ liczbie czterystucyfrowej 84198419$\ldots$ 8419 skreśl pewną ilość cyfr z początku i końca tak,

aby suma pozostałych cyfr była równa 1984.






LIGA MATEMATYCZNA

im. Zdzisława Matuskiego

$\mathrm{P}\mathrm{A}\acute{\mathrm{Z}}$ DZIERNIK 2012

GIMNAZJUM

ZADANIE I.

Zapisano wszystkie liczby naturalne od $n$ do $n^{2}$ Jest ich 601. Ob1icz $n.$

ZADANIE 2.

Udowodnij, $\dot{\mathrm{z}}\mathrm{e}$ suma kwadratów trzech kolejnych liczb całkowitych przy dzieleniu przez 3 daje

resztę 2.

ZADANIE 3.

$\mathrm{W}$ duzym kwadracie umieszczony jest maly kwadrat w taki sposób, $\dot{\mathrm{z}}\mathrm{e}$ jeden jego bok lezy

na przekątnej, a dwa wierzcholki na bokach $\mathrm{d}\mathrm{u}\dot{\mathrm{z}}$ ego kwadratu. Oblicz stosunek pól tych kwa-

dratów.

ZADANIE 4.

Dany jest trójkąt równoramienny $ABC$, w którym $AC = BC$. Na odcinku $AC$ wybrano

punkty $X$ oraz $Y$ w taki sposób, $\dot{\mathrm{z}}\mathrm{e}AB=BX=XY$ oraz $BY=YC$. Wyznacz miary kątów

wewnętrznych trójkąta $ABC.$

ZADANIE 5.

Wykaz$\cdot, \dot{\mathrm{z}}\mathrm{e}$ dla $\mathrm{k}\mathrm{a}\dot{\mathrm{z}}$ dej liczby naturalnej $n$ liczba $n^{3}-19n$ jest podzielna przez 6.






LIGA MATEMATYCZNA

im. Zdzisława Matuskiego

$\mathrm{P}\mathrm{A}\overline{\mathrm{Z}}$ DZIERNIK 2013

GIMNAZJUM

ZADANIE I.

Dany jest ulamek $\displaystyle \frac{a}{b}$. Do licznika tego ufamka dodano liczbę l.

do mianownika, aby otrzymać ufamek równy danemu?

Jaką liczbę nalezy dodać

ZADANIE 2.

Na przyjęcie przybyla pewna liczba gości. $K\mathrm{a}\dot{\mathrm{z}}\mathrm{d}\mathrm{y}$ z $\mathrm{k}\mathrm{a}\dot{\mathrm{z}}$ dym wymienif uścisk dloni, z wyjątkiem

pana Jana, który dwunastu gościom nie chcial podać ręki. $\mathrm{W}$ sumie wymieniono 2004 uściski

dłoni. Ile osób było na przyjęciu?

ZADANIE 3.

$\mathrm{W}$ finale Ligi Matematycznej uczestniczyfo stu uczniów. Uzasadnij, $\dot{\mathrm{z}}\mathrm{e}$ wśród nich byfo piętna-

stu (lub więcej) uczniów, którzy urodzili się w tym samym dniu tygodnia.

ZADANIE 4.

Długości boków kwadratów ABCD $\mathrm{i}$ KLMN sa równe 4 cm. Kwadraty te są tak połozone,

$\dot{\mathrm{z}}\mathrm{e}$ wierzchofek $K$ nalezy do boku $AD$, wierzcholek $L-$ do boku $AB$, a przekątne kwadratu

KLMN są prostopadfe do odpowiednich boków kwadratu ABCD. Oblicz pole figury będącej

częścią wspólną obu kwadratów. Oblicz odleglość wierzchofka $C$ od prostej $MN.$

ZADANIE 5.

Na Międzynarodową Olimpiadę Matematyczną przyjechalo 1000 osób. $\mathrm{W}$ sprawozdaniu po-

dano, $\dot{\mathrm{z}}\mathrm{e}$ wśród nich 811 wfada językiem angie1skim, 752- językiem rosyjskim, 418- językiem

francuskim, $356 -$ językiem rosyjskim i francuskim, $570 -$ językiem rosyjskim i angielskim,

348- językiem angielskim i francuskim, 297 osób mówi wszystkimi trzema językami. Wykaz,

$\dot{\mathrm{z}}\mathrm{e}$ w sprawozdaniu popełniono bfąd.






LIGA MATEMATYCZNA

im. Zdzisława Matuskiego

$\mathrm{P}\mathrm{A}\overline{\mathrm{Z}}$ DZIERNIK 2014

GIMNAZJUM

ZADANIE I.

Piszemy liczbę l, potem 0. Trzecią 1iczbą jest róznica 1iczby drugiej i pierwszej, czwartą -

róznica trzeciej i drugiej, piatą- róznica czwartej i trzeciej, i tak dalej. Wyznacz liczbę stojącą

na 2014 miejscu.

ZADANIE 2.

Wyznacz wszystkie pary liczb naturalnych $a, b$ spefniające warunek $a^{2}-4b^{2}=45.$

ZADANIE 3.

$\mathrm{W}$ prostokącie o bokach dlugości 9 cm i 7 cm umieszczono prostokąt tak, $\dot{\mathrm{z}}\mathrm{e}$ jedna z jego

przekątnych fączy środki krótszych boków większego prostokąta, a dwa pozostale wierzchołki

mniejszego prostokąta lezą na dluzszych bokach większego prostokąta. Oblicz obwód mniejszego

prostokąta.

ZADANIE 4.

Między dwiema dodatnimi liczbami cafkowitymi $a\mathrm{i}b$ jest dziesięć liczb cafkowitych większych

od $a$ i mniejszych od $b$, zaś między $a^{2}\mathrm{i}b^{2}$ jest tysiąc liczb cafkowitych większych od $a^{2}$ i mniej-

szych od $b^{2}$ Wyznacz a $\mathrm{i}b.$

ZADANIE 5.

Dany jest kwadrat ABCD o boku l. Punkt $M$ jest środkiem boku $BC, L$ jest środkiem boku

$CD$. Odcinki AM $\mathrm{i}BL$ podzieliły kwadrat na cztery obszary. Oblicz pole $\mathrm{k}\mathrm{a}\dot{\mathrm{z}}$ dego z nich.






LIGA MATEMATYCZNA

im. Zdzisława Matuskiego

$\mathrm{P}\mathrm{A}\dot{\mathrm{Z}}$ DZIERNIK 2015

GIMNAZJUM

ZADANIE I.

Wykaz$\cdot, \dot{\mathrm{z}}\mathrm{e}$ liczba $6^{100}-2\cdot 6^{99}+10\cdot 6^{98}$ jest podzielna przez 17.

ZADANIE 2.

$\mathrm{W}$ pólokrąg o promieniu 5 wpisano dwa kwadraty, jak na rysunku.

kwadratów.

Oblicz sumę pól tych

ZADANIE 3.

Czy istnieją takie liczby naturalne $m, n$, aby w wyniku mnozenia ich sumy przez ich iloczyn

otrzymač liczbę 20162015?

ZADANIE 4.

Rozwazmy $1001\mathrm{l}\mathrm{i}\mathrm{c}\mathrm{z}\mathrm{b}$: 1, $1+2, 1+2+3$, 1$+$2$+$3$+$4, $\ldots, 1+2+3+\ldots+1000, 1+2+3+\ldots+1001.$

Ile jest wśród nich liczb parzystych?

ZADANIE 5.

Przygotowując prezent dla Ani, Bartek wlozyl go do malego pudefka, to pudelko wlozył do

większego, a to do jeszcze większego, przy czym $\mathrm{k}\mathrm{a}\dot{\mathrm{z}}$ de następne pudelko cafkowicie mieścifo

poprzednie. Ustal, w jakiej kolejności bral pudelka, $\mathrm{j}\mathrm{e}\dot{\mathrm{z}}$ eli wiadomo, $\dot{\mathrm{z}}\mathrm{e}$:

$\bullet$ pudelko zólte jest prostopadfościanem o objętości 12144 $\mathrm{c}\mathrm{m}^{\mathrm{S}}$ i jego jedna ściana ma wy-

miary 23 cm i 24 cm;

$\bullet$ pudelko zielone jest sześcianem o objętości 8000 $\mathrm{c}\mathrm{m}^{3}$;

$\bullet$ pudefko rózowe jest sześcianem o sumie dfugości wszystkich krawędzi równej 312 cm.






LIGA MATEMATYCZNA

im. Zdzisława Matuskiego

$\mathrm{P}\mathrm{A}\dot{\mathrm{Z}}$ DZIERNIK 2016

GIMNAZJUM

ZADANIE I.

$\mathrm{W}$ okrąg o promieniu o długości 10 wpisano prostokat ABCD. Następnie na tym okręgu

wybrano dowolny punkt $E$. Oblicz sumę kwadratów odleglości punktu $E$ od wierzchołków

prostokąta, czyli $|EA|^{2}+|EB|^{2}+|EC|^{2}+|ED|^{2}$

ZADANIE 2.

Ile jest dodatnich liczb calkowitych, których największy dzielnik wlaściwy (to znaczy dzielnik

rózny od l i od danej liczby) jest równy 91?

ZADANIE 3.

Ania ma 36 karteczek. Poma1owafaje $\mathrm{u}\dot{\mathrm{z}}$ ywając trzech kolorów: zielonego, czerwonego i niebie-

skiego. Niektóre karteczki są pomalowane tylko jednym kolorem, inne dwoma, a pozostale pięć

karteczek wszystkimi trzema kolorami. Zielonej kredki $\mathrm{u}\dot{\mathrm{z}}$ yla do pokolorowania 25 karteczek,

czerwonej do 28, a niebieskiej do 20 karteczek. I1e karteczek Ania poma1owa1ajednym ko1orem?

ZADANIE 4.

Operacją nazywamy przyporządkowanie trójce liczb $(a,b,c)$ nowej trójki $(b+c,a+c,a+b).$

Początkową trójka jest (1, 3, 5). Po wykonaniu 2016 takich operacji na otrzymywanych trójkach

liczb uzyskano $(x,y,z)$. Oblicz róznicę $x-y.$

ZADANIE 5.

Dany jest prostokąt ABCD, w którym $|AB|=20, |BC|=10$. Punkty $W\mathrm{i}K$ lezą na zewnątrz

tego prostokąta oraz $|WA| = |KC| = 12$ oraz $|WB| = |KD| = 16$. Oblicz dfugość odcinka

$WK.$
\begin{center}
\includegraphics[width=40.584mm,height=58.728mm]{./LigaMatematycznaMatuskiego_Gim_Zestaw1_2016_2017_page0_images/image001.eps}
\end{center}
K

D c

A B

w






LIGA MATEMATYCZNA

im. Zdzisława Matuskiego

$\mathrm{P}\mathrm{A}\dot{\mathrm{Z}}$ DZIERNIK 2017

GIMNAZJUM

ZADANIE I.

Wiadomo, $\dot{\mathrm{z}}\mathrm{e}$

$x-y+2017,y-z+2017,z-t+2017,t-w+2017,w-x+2017$

są kolejnymi liczbami calkowitymi. Znajdz' je.

ZADANIE 2.

Trzy pary malzeńskie: Ania i Adam, Beata i Bartek, Celina i Czarek mają w sumie 1371at.

$K\mathrm{a}\dot{\mathrm{z}}\mathrm{d}\mathrm{y}$ z panów jest o 51at starszy od swojej $\dot{\mathrm{z}}$ ony. Suma liczby lat Bartka i Beaty wynosi

471at. Ania jest najstarsza wśród pań i ma o 4 lata więcej $\mathrm{n}\mathrm{i}\dot{\mathrm{z}}$ najmlodsza z kobiet. Ile lat ma

$\mathrm{k}\mathrm{a}\dot{\mathrm{z}}$ da osoba?

ZADANIE 3.

$\mathrm{W}$ okrąg o promieniu $r$ wpisano kwadrat i na tym okręgu opisano trójkąt równoboczny. Suma

dlugości boku kwadratu i boku trójkąta równobocznego jest równa 10. Wyznacz promień

okręgu.

ZADANIE 4.

Znajd $\acute{\mathrm{z}}$ dwie liczby naturalne, których sumajest równa 432 i których największy wspó1ny dzie1-

nik to 36.

ZADANIE 5.

$\mathrm{W}$ liczbie trzycyfrowej $x$ skreślono cyfrę setek i otrzymano dwucyfrową liczbę $k$. Gdy w liczbie

$x$ skreślono cyfrę dziesiątek, to otrzymano liczbę dwucyfrową $l$, a po skreśleniu w liczbie $x$

cyfry jedności powstala dwucyfrowa liczba $m$. Okazalo się, $\dot{\mathrm{z}}\mathrm{e}$ suma $k+l+m$ jest trzykrotnie

mniejsza od liczby $x.$ Znajd $\acute{\mathrm{z}}x.$






LIGA MATEMATYCZNA

im. Zdzisława Matuskiego

$\mathrm{P}\mathrm{A}\dot{\mathrm{Z}}$ DZIERNIK 2018

GIMNAZJUM

(klasa VII i VIII szkoły podstawowej, klasa III gimnazjum)

ZADANIE I.

Wewnatrz kwadratu ABCD $\mathrm{l}\mathrm{e}\dot{\mathrm{z}}\mathrm{y}$ punkt $P$. Odleglość tego punktu od wierzchofków $A, B, C$

jest równa odpowiednio 2, 7, 9. Ob1icz od1eg1ość punktu $P$ od wierzchofka $D.$

ZADANIE 2.

Na ile sposobów $\mathrm{m}\mathrm{o}\dot{\mathrm{z}}$ na przedstawić liczbę 3 jako sumę sześcianów pięciu 1iczb ca1kowitych?

Przedstawień rózniących się jedynie kolejnością skfadników nie uznajemy za rózne.

ZADANIE 3.

$\mathrm{W}$ klasie VII, liczącej 24 uczniów, średnia 1iczba punktów uzyskanych na k1asówce wynios1a

75 na l00 $\mathrm{m}\mathrm{o}\dot{\mathrm{z}}$ liwych do zdobycia. Maksymalną liczbę 100 punktów otrzymafo 4 uczniów.

Wyznacz średnią liczbę punktów zdobytych przez pozostafych 20 uczniów.

ZADANIE 4.

Znajd $\acute{\mathrm{z}}$ wszystkie takie liczby trzycyfrowe, $\dot{\mathrm{z}}\mathrm{e}$ po skreśleniu cyfry setek otrzymamy liczbę dwu-

krotnie mniejszą $\mathrm{n}\mathrm{i}\dot{\mathrm{z}}$ po skreśleniu cyfry jedności.

ZADANIE 5.

Dany jest kwadrat ABCD. Oblicz miarę kąta $\alpha$ (patrz rysunek).
\begin{center}
\includegraphics[width=59.280mm,height=43.080mm]{./LigaMatematycznaMatuskiego_Gim_Zestaw1_2018_2019_page0_images/image001.eps}
\end{center}
{\it D  C}

{\it a}

$\alpha$






LIGA MATEMATYCZNA

LISTOPAD 2009

GIMNAZJUM

ZADANIE I.

Wykaz, $\dot{\mathrm{z}}\mathrm{e}$ liczba

2006. 2008. 2010. 2012$+$16

jest kwadratem liczby naturalnej.

ZADANIE 2.

Oblicz pole wielokąta przedstawionego na rysunku wiedząc, $\dot{\mathrm{z}}\mathrm{e}0<x<1.$
\begin{center}
\includegraphics[width=34.140mm,height=26.412mm]{./LigaMatematycznaMatuskiego_Gim_Zestaw2_2009_2010_page0_images/image001.eps}
\end{center}
1  1

x  $\rceil$-x

1

ZADANIE 3.

$\acute{\mathrm{S}}$ rodki kolejnych boków trapezu nierównoramiennego połqczono odcinkami. Wykaz$\cdot, \dot{\mathrm{z}}\mathrm{e}$ suma

pól powstałych czterech trójkątów jest równa polu otrzymanego czworokąta.

ZADANIE 4.

Rozwiąz równanie

$1-(2-(3-\ldots-(2009-x)\ldots))=1000.$

ZADANIE 5.

$\mathrm{W}$ pięciu skarbonkach była jednakowa ilość monet. Po pewnym czasie okazało się, $\dot{\mathrm{z}}\mathrm{e}$ wyjęto

ze skarbonek połowę wszystkich posiadanych monet. Z pierwszej skarbonki wyjęto 2 monety,

z drugiej $-5$ monet, z trzeciej $-9$, z czwartej $-24$. Nie wiadomo, ile monet wyjęto z piątej

skarbonki, ale w $\mathrm{k}\mathrm{a}\dot{\mathrm{z}}$ dej skarbonce zostala co najmniej jedna moneta. Ile było monet na początku

i ile monet pozostało w $\mathrm{k}\mathrm{a}\dot{\mathrm{z}}$ dej skarbonce?






LIGA MATEMATYCZNA

LISTOPAD 2010

GIMNAZJUM

ZADANIE I.

$\mathrm{W}$ pewnej liczbie trzycyfrowej $x$ skreślono cyfrę setek i otrzymano dwucyfrową liczbę $k$. Gdy

w liczbie $x$ skrešlono cyfrę dziesiątek, otrzymano liczbę dwucyfrową $l$, a po skreśleniu w liczbie

$x$ cyfry jednošci powstała liczba dwucyfrowa $m$. Okazalo się, $\dot{\mathrm{z}}\mathrm{e}$ suma $k+l+m$ jest trzykrotnie

mniejsza od liczby $x$. Znajdz' $x.$

ZADANIE 2.

Wykaz$\cdot, \dot{\mathrm{z}}\mathrm{e}$ suma $2^{1}+2^{2}+2^{3}+\ldots+2^{2009}$ jest podzielna przez 127.

ZADANIE 3.

$\mathrm{W}$ kwadracie ABCD punkty $E\mathrm{i}F$ są środkami, odpowiednio, boków AD $\mathrm{i}BC$. Obrano punkty

$G\mathrm{i}H$ w taki sposób, $\dot{\mathrm{z}}\mathrm{e}E$ jest punktem odcinka $GB\mathrm{i}F$ jest punktem odcinka $AH$. Wiedząc,

$\dot{\mathrm{z}}\mathrm{e}|GA|=|AB|=|BH|=1$, oblicz długość odcinka $GH.$

ZADANIE 4.

Rozwazmy liczby całkowite dodatnie $m \mathrm{i} n$, które spełniają warunek $75m = n^{3}$ Jaka jest

najmniejsza $\mathrm{m}\mathrm{o}\dot{\mathrm{z}}$ liwa suma liczb $m\mathrm{i}n$?

ZADANIE 5.

Dwóch uczonych napisało na siedmiu kartkach liczby 5, 6, 7, 8, 9, 10, $11$ - na $\mathrm{k}\mathrm{a}\dot{\mathrm{z}}$ dej kartce

jedną liczbę.

Następnie pierwszy wziąl losowo trzy kartki, drugi dwie inne kartki, a ostatnie

dwie, bez oglądania ich, wyrzucili. Pierwszy uczony, zaglądając do swoich kartek, powiedział

do drugiego:,,Wiem, $\dot{\mathrm{z}}\mathrm{e}$ suma liczb na twoich kartkach jest parzysta'' Jakie liczby wylosował

pierwszy z uczonych?






LIGA MATEMATYCZNA

LISTOPAD 2011

GIMNAZJUM

ZADANIE I.

Rozwiąz równanie

$(\displaystyle \frac{x}{\sqrt{144}+\sqrt{145}}-\frac{1}{\sqrt{36}+\sqrt{37}})+(\frac{x}{\sqrt{145}+\sqrt{146}}-\frac{1}{\sqrt{37}+\sqrt{38}})+\ldots$

. . . $+(\displaystyle \frac{x}{\sqrt{168}+\sqrt{169}}-\frac{1}{\sqrt{60}+\sqrt{61}}) =0.$

ZADANIE 2.

Przecinając prostokątny arkusz papieru, otrzymano kwadrat oraz mniejszy prostokąt. $\mathrm{Z}$ tego

prostokąta równiez odcięto kwadrat i znów otrzymano mniejszy prostokat. Sytuacja powtórzyła

się jeszcze kilkakrotnie, $\mathrm{a}\dot{\mathrm{z}}$ do momentu otrzymania dziewięciu róznych kwadratów i jednego

prostokata o wymiarach lcm $\times 2\mathrm{c}\mathrm{m}$. Jakie pole mial arkusz papieru?

ZADANIE 3.

Liczby nieparzyste od l do 49 wypisano w ponizszej tab1icy:

(432111111

3 5

13

23

33

43

15

25

35

45

7

17

27

37

47

432199999)

Wybieramy z tej tablicy pięć liczb tak, aby $\dot{\mathrm{z}}$ adne dwie nie $\mathrm{l}\mathrm{e}\dot{\mathrm{z}}$ afy ani w jednej kolumnie, ani

wjednym wierszu. Wyznacz wszystkie wartości, jakie $\mathrm{m}\mathrm{o}\dot{\mathrm{z}}\mathrm{e}$ przyjmować suma wybranych liczb.

ZADANIE 4.

Ania napisafa trzy liczby pięciocyfrowe $\mathrm{u}\dot{\mathrm{z}}$ ywając do zapisu $\mathrm{k}\mathrm{a}\dot{\mathrm{z}}$ dej z tych liczb wszystkich cyfr

spośród 1, 2, 3, 4, 5. Czy suma tych 1iczb jest podzie1na przez 3? Czy jest podzie1na przez 9?

ZADANIE 5.

Dany jest kwadrat o boku dlugości $\alpha$. Na jego bokach, na zewnątrz, zbudowano trójkąty

równoboczne. Wierzcholki kolejnych trójkątów, nie będące wierzcholkami danego kwadratu,

polączono odcinkami. Oblicz pole otrzymanego czworokata.






LIGA MATEMATYCZNA

im. Zdzisława Matuskiego

LISTOPAD 2012

GIMNAZJUM

ZADANIE I.

$\mathrm{O}$ liczbach $a, b$ wiemy, $\dot{\mathrm{z}}\mathrm{e}a<b<0$. Która z liczb $\displaystyle \frac{1}{2}a-b, \displaystyle \frac{1}{2}b-a$ jest większa?

ZADANIE 2.

$\mathrm{W}$ prostokącie ABCD punkt $E$ jest šrodkiem boku $BC, F$ jest šrodkiem boku $CD.$

trójkąta $AEF$ jest równe 15 $\mathrm{c}\mathrm{m}^{2}$ Oblicz pole prostokąta ABCD.

Pole

ZADANIE 3.

Wykaz, $\dot{\mathrm{z}}\mathrm{e}$ liczba $n^{3}+5n$ jest podzielna przez 6 d1a $\mathrm{k}\mathrm{a}\dot{\mathrm{z}}$ dej liczby naturalnej $n.$

ZADANIE 4.

$\mathrm{W}$ kwadrat ABCD wpisano koło. $\mathrm{W}$ to koło wpisano kwadrat tak, $\dot{\mathrm{z}}\mathrm{e}$ jego boki sq równoległe

do boków kwadratu ABCD. Róznica pól tych kwadratów jest równa 2 $\mathrm{c}\mathrm{m}^{2}$ Oblicz pole kofa.

ZADANIE 5.

Piotr znalazf wszystkie dzielniki pewnej liczby naturalnej n, uporządkował je rosnąco, a na-

stępnie wykreślił co drugi otrzymujqc liczby: 1, 3, 6, 12, 21, 42. Wyznacz 1iczbę n i pozostałe

jej dzielniki.






LIGA MATEMATYCZNA

im. Zdzisława Matuskiego

LISTOPAD 2013

GIMNAZJUM

ZADANIE I.

Pewna liczba ma cztery dzielniki, których średnia arytmetycznajest równa 10. Znajd $\acute{\mathrm{z}}$ tę liczbę.

ZADANIE 2.

W szkole uczy się 600 uczniów w 21 k1asach.

co najmniej 29 osób.

Uzasadnij, $\dot{\mathrm{z}}\mathrm{e}$ istnieje klasa, w której uczy się

ZADANIE 3.

Jaką część pola trójkąta ABC stanowi pole czworokąta ABFD, jeśli odcinki AB oraz AD mają

dlugość 2, natomiast odcinki BE i CD mają dfugość 5.
\begin{center}
\includegraphics[width=70.152mm,height=36.012mm]{./LigaMatematycznaMatuskiego_Gim_Zestaw2_2013_2014_page0_images/image001.eps}
\end{center}
c

D F

A  B  E

ZADANIE 4.

Wykaz$\cdot, \dot{\mathrm{z}}\mathrm{e}$ jeśli $a>1$ oraz $b<1$, to $ab+1<a+b.$

ZADANIE 5.

Liczba sześciocyfrowa w zapisie dziesiątkowym kończy się cyfrą 4. $\mathrm{J}\mathrm{e}\dot{\mathrm{z}}$ eli cyfrę 4 przeniesiemy

na początek zapisu, pozostawiając pozostale cyfry bez zmian, to otrzymamy nową liczbę, która

będzie cztery razy większa od początkowej. Znajd $\acute{\mathrm{z}}$ początkową liczbę.






LIGA MATEMATYCZNA

im. Zdzisława Matuskiego

LISTOPAD 2014

GIMNAZJUM

ZADANIE I.

Zeszyt Bartka do matematyki ma ponumerowane strony od l do 60. Chfopiec wyrwa1 z niego

dziesięć kartek i dodal liczby numerujące ich strony. Sprawd $\acute{\mathrm{z}}$, czy mógl otrzymać liczbę 101.

ZADANIE 2.

Znajd $\acute{\mathrm{z}}$ taką liczbę trzycyfrowa, $\dot{\mathrm{z}}$ ejeśli z prawej strony dopiszemy cyfrę 8, to otrzymamy 1iczbę

czterocyfrową dwa razy większą $\mathrm{n}\mathrm{i}\dot{\mathrm{z}}$ gdybyśmy z lewej strony dopisali cyfrę 3 i uzyska1i inna

liczbę czterocyfrowa.

ZADANIE 3.

Rozwiąz uklad równań

$\left\{\begin{array}{l}
5(y+z)-x=-1\\
4(x+z)-2y=2\\
3(x+y)-3z=-1.
\end{array}\right.$

ZADANIE 4.

Wykaz, $\dot{\mathrm{z}}\mathrm{e}\mathrm{j}\mathrm{e}\dot{\mathrm{z}}$ eli $a\mathrm{i}b$ sq dowolnymi dodatnimi liczbami rzeczywistymi, to

$(a+b)(\displaystyle \frac{1}{a}+\frac{1}{b})\geq 4.$

ZADANIE 5.

Prosta k dzieli boki prostokąta na odcinki, których dlugości pozostają w stosunku 1 : 4 oraz

l : l tak, jak na ponizszym rysunku. Oblicz stosunek pól powstałych w ten sposób figur.
\begin{center}
\includegraphics[width=64.104mm,height=38.808mm]{./LigaMatematycznaMatuskiego_Gim_Zestaw2_2014_2015_page0_images/image001.eps}
\end{center}
k






LIGA MATEMATYCZNA

im. Zdzisława Matuskiego

LISTOPAD 2015

GIMNAZJUM

ZADANIE I.

$\mathrm{W}$ prostokącie ABCD punkty $Y, L, K, X$ są środkami boków odpowiednio AB, $BC$, {\it CD}, $DA,$

zaś punkt $M$ jest środkiem odcinka $XY$. Pole prostokąta ABCD jest równe 2015 $\mathrm{c}\mathrm{m}^{2}$ Oblicz

pole trójkąta $KLM.$

ZADANIE 2.

Dwie trzycyfrowe liczby zapisane są przy pomocy takich samych cyfr, z których jedna jest

równa 4. Pierwsza 1iczba ma czwórkę w rzędzie jedności, a druga w rzędzie setek, zaš pozostałe

jej cyfry zapisane sq w takiej samej kolejności, jak w pierwszej. Druga liczba jest o 400 większa

od róznicy liczby 400 i pierwszej 1iczby. Jakie to 1iczby?

ZADANIE 3.

Wyznacz setną cyfrę od końca liczby 2015!. Liczbę $n!$ (czytamy $n$ silnia) definiujemy jako

iloczyn kolejnych liczb naturalnych od l do $n.$

ZADANIE 4.

Wykaz$\cdot, \dot{\mathrm{z}}\mathrm{e}7^{n+2}+7^{n+1}-2\cdot 7^{n}$ jest liczbq parzystą dla dowolnej liczby naturalnej $n.$

ZADANIE 5.

$\mathrm{W}$ zbiorze liczb rzeczywistych rozwiąz uklad równań

$\left\{\begin{array}{l}
x+y+z+t=36\\
x+y-z-t=24\\
x-y+z-t=12\\
x-y-z+t=0.
\end{array}\right.$






LIGA MATEMATYCZNA

im. Zdzisława Matuskiego

LISTOPAD 2016

GIMNAZJUM

ZADANIE I.

Na przyprostokatnych $BC\mathrm{i}CA$ trójkąta prostokatnego $ABC$ zbudowano na zewnątrz kwadraty

{\it ECBD oraz CFGA. Prosta} $AD$ przecina bok $BC$ w punkcie $P$, prosta $BG$ przecina bok $CA$

w punkcie $R$. Udowodnij, $\dot{\mathrm{z}}\mathrm{e}$ odcinki $CP\mathrm{i}CR$ mają równe dlugości.

ZADANIE 2.

Adam mial pomnozyć dwie liczby naturalne. Jeden z czynników byl liczbą dwucyfrową, w której

cyfra jedności byla dwukrotnie mniejsza od cyfry dziesiątek. Chlopiec pomylił się, przestawił

cyfry tej liczby i otrzymal iloczyn o 1539 mniejszy od poprawnego. Podaj poprawny wynik

tego mnozenia i liczby, które miał pomnozyć Adam.

ZADANIE 3.

Na stole $\mathrm{l}\mathrm{e}\dot{\mathrm{z}}\mathrm{y}$ 2017 monet. $\mathrm{W}$ jednym ruchu Bartek $\mathrm{m}\mathrm{o}\dot{\mathrm{z}}\mathrm{e}$ wziąć dokladnie 3, 361ub 69 monet.

Czy wykonując wiele takich ruchów Bartek $\mathrm{m}\mathrm{o}\dot{\mathrm{z}}\mathrm{e}$ wziąć wszystkie monety ze stołu?

ZADANIE 4.

Wykaz$\cdot, \dot{\mathrm{z}}\mathrm{e}$ liczba $256^{4}+8^{9}$ jest podzielna przez ll.

ZADANIE 5.

Róznica między czwartymi potęgami pewnych dwóch liczb naturalnych jest równa 34481, a róz-

nica między drugimi potęgami tych liczb wynosi 41. Wyznacz róznicę tych 1iczb.






LIGA MATEMATYCZNA

im. Zdzisława Matuskiego

LISTOPAD 2017

GIMNAZJUM

ZADANIE I.

$\mathrm{O}$ liczbach $a, b, c, d$ wiadomo, $\dot{\mathrm{z}}\mathrm{e}$

$\left\{\begin{array}{l}
a=bcd\\
a+b=cd\\
a+b+c=d\\
a+b+c+d=1.
\end{array}\right.$

Wyznacz te liczby.

ZADANIE 2.

Niech $p$ będzie liczbą pierwszą taką, $\dot{\mathrm{z}}\mathrm{e}$ liczba dzielników liczby $p^{6}$ jest dzielnikiem tej liczby.

Ile dzielników ma liczba $(p+1)^{6}$?

ZADANIE 3.

Dziadek Ani urodzil się przed II wojną światową, ale ma mniej $\mathrm{n}\mathrm{i}\dot{\mathrm{z}}90$ lat. Gdy w 2007 roku

obchodzil urodziny, Ania zauwazyla, $\dot{\mathrm{z}}\mathrm{e}$ numer roku byl równy numerowi roku urodzenia dziadka

powiększonemu o pięciokrotną sumę cyfr roku urodzenia. $\mathrm{W}$ którym roku urodzil się dziadek

Ani?

ZADANIE 4.

Dwa jednakowe kofa mniejsze i jedno kolo większe wpisano w prostokąt w taki sposób, $\dot{\mathrm{z}}\mathrm{e}$ kofa

są styczne do boków prostokąta i wzajemnie styczne zewnętrznie. Mniejszy z boków prostokąta

ma dlugość 4. Ob1icz obwód prostokąta oraz róznicę między po1em prostokąta a sumą pó1 kó1.

ZADANIE 5.

Wykaz, $\dot{\mathrm{z}}\mathrm{e}$ liczba $4^{202}+2\cdot 4^{101}\cdot 6^{101}+6^{202}$ jest podzielna przez 100.






LIGA MATEMATYCZNA

im. Zdzisława Matuskiego

LISTOPAD 2018

GIMNAZJUM

(klasa VII i VIII szkoły podstawowej, klasa III gimnazjum)

ZADANIE I.

Czy istnieje trójkąt prostokątny mający boki o dfugościach cafkowitych i obwód równy 2019?

ZADANIE 2.

Wykaz, $\dot{\mathrm{z}}\mathrm{e}$ liczba

$\displaystyle \frac{n^{5}}{120}-\frac{n^{3}}{24}+\frac{n}{30}$

jest całkowita dla $\mathrm{k}\mathrm{a}\dot{\mathrm{z}}$ dej liczby całkowitej $n.$

ZADANIE 3.

Wyznacz wszystkie pary liczb cafkowitych dodatnich $(x,y)$ o następujących wlasnościach:

$x+y=240$ oraz

$\mathrm{N}\mathrm{W}\mathrm{D}\{x,y\}=30.$

ZADANIE 4.

Ile jest liczb trzycyfrowych podzielnych przez 9 mających następującą wfasność:

ilorazu tej liczby przez 9 jest o 9 mniejsza od sumy jej cyfr?

suma cyfr

ZADANIE 5.

Sześciokąt mający wszystkie kąty wewnętrzne tej samej miary wpisano w trójkąt równoboczny

o boku o dlugości 7. Wyznacz d1ugości boków x, y, z.
\begin{center}
\includegraphics[width=49.332mm,height=45.360mm]{./LigaMatematycznaMatuskiego_Gim_Zestaw2_2018_2019_page0_images/image001.eps}
\end{center}
3

{\it 4}

5






LIGA MATEMATYCZNA

GRUD Z$\mathrm{I}\mathrm{E}\acute{\mathrm{N}}$ 2009

GIMNAZJUM

ZADANIE I.

Trójkqt $ABC$ podzielono na 5 trójkątów. Liczba wewnątrz $\mathrm{k}\mathrm{a}\dot{\mathrm{z}}$ dego trójkąta oznacza jego pole.

Oblicz $x.$
\begin{center}
\includegraphics[width=70.812mm,height=33.528mm]{./LigaMatematycznaMatuskiego_Gim_Zestaw3_2008_2009_page0_images/image001.eps}
\end{center}
c

14

7

21  x

A  B

ZADANIE 2.

Oblicz $\sqrt{37-20\sqrt{3}}+\sqrt{13-4\sqrt{3}}.$

ZADANIE 3.

Rozwiąz układ równań

$\left\{\begin{array}{l}
5x(x+y+z)=4\\
2y(x+y+z)=6\\
4z(x+y+z)=4.
\end{array}\right.$

ZADANIE 4.

$\mathrm{W}$ grupie 300 studentów $\mathrm{k}\mathrm{a}\dot{\mathrm{z}}\mathrm{d}\mathrm{y}$ jest matematykiem, chemikiem lub fizykiem. Polowa fizyków

zajmuje się chemią, połowa chemików zajmuje się matematyką, a polowa matematyków to fi-

zycy. Wiedząc, $\dot{\mathrm{z}}\mathrm{e}\dot{\mathrm{z}}$ aden fizyk nie zajmuje się chemią i matematyką, odpowiedz, z ilu osób

składają się te grupy.

ZADANIE 5.

Liczbę czterocyfrową pomnozono przez 9 i otrzymano 1iczbę czterocyfrową zapisaną za pomocą

tych samych cyfr w odwrotnej kolejności. Jaka to liczba?






LIGA MATEMATYCZNA

GRUD Z$\mathrm{I}\mathrm{E}\acute{\mathrm{N}}$ 2010

GIMNAZJUM

ZADANIE I.

Wiadomo, $\dot{\mathrm{z}}\mathrm{e}a-b+2010, b-c+2010, c-a+2010$ są trzema kolejnymi liczbami calkowitymi.

Jakimi?

ZADANIE 2.

Znajd $\acute{\mathrm{z}}$ wszystkie rozwiązania równania $x^{4}-y^{4}=65$ będqce liczbami naturalnymi.

ZADANIE 3.

Z przeciwległych wierzcholków prostokąta poprowadzono odcinki prostopadłe do przekątnej.

Odcinki te podzieliły przekątną na trzy równe częšci. Znajdz' stosunek długošci boków tego

prostokąta.

ZADANIE 4.

W ciągu tygodnia waga małej foki wzrosla o 4\%, a słoniątka o 4 kg. Skutkiem tego średnia

waga obu zwierząt wzrosfa o 3 kg, czy1i o 2\%. I1e obecnie wazy sfoniątko?

ZADANIE 5.

Znajd $\acute{\mathrm{z}}$ ostatnią cyfrę liczby $1^{2010}+2^{2010}+3^{2010}+\ldots+10^{2010}$






LIGA MATEMATYCZNA

GRUD Z$\mathrm{I}\mathrm{E}\acute{\mathrm{N}}$ 2011

GIMNAZJUM

ZADANIE I.

Na Wigilii $\mathrm{u}$ babci spotkala się liczna rodzina. Przy stole zasiadlo 20 osób. Babcia przygoto-

wala sto ciasteczek. $K\mathrm{a}\dot{\mathrm{z}}\mathrm{d}\mathrm{y}$ męzczyzna zjadł siedem ciasteczek, $\mathrm{k}\mathrm{a}\dot{\mathrm{z}}$ da kobieta- pięć, a $\mathrm{k}\mathrm{a}\dot{\mathrm{z}}$ de

z wnucząt-jedno. Oblicz, ilu bylo doroslych, a ile dzieci.

ZADANIE 2.

Wykaz, $\dot{\mathrm{z}}\mathrm{e}\sqrt{18+8\sqrt{2}}+\sqrt{6-4\sqrt{2}}$ jest liczbą calkowita.

ZADANIE 3.

Wykaz, $\dot{\mathrm{z}}\mathrm{e}$ jeśli do iloczynu dwóch kolejnych liczb naturalnych dodamy sumę kwadratów tych

liczb powiększoną o 5, to otrzymamy 1iczbę podzie1na przez 6.

ZADANIE 4.

Piszemy liczby 1, 1, 2, 3, 5, 8, $\ldots$ w taki sposób, $\dot{\mathrm{z}}\mathrm{e}$ począwszy od trzeciej, $\mathrm{k}\mathrm{a}\dot{\mathrm{z}}$ da następna liczba

jest sumą dwóch poprzednich. Jaką liczbą (parzystą czy nieparzystą) jest liczba znajdująca się

na 2011 miejscu?

ZADANIE 5.

$\mathrm{Z}$ dwóch jednakowych plytek w ksztalcie trójkata prostokątnego o obwodzie 40 $\mathrm{m}\mathrm{o}\dot{\mathrm{z}}$ na ufozyć

trójkąt o obwodzie 50 a1bo trójkąt o obwodzie 64, a1bo de1toid. Ob1icz d1ugość przekątnych

tego deltoidu.






LIGA MATEMATYCZNA

im. Zdzisława Matuskiego

GRUD Z$\mathrm{I}\mathrm{E}\acute{\mathrm{N}}$ 2012

GIMNAZJUM

ZADANIE I.

Liczba 390 jest sumą kwadratów trzech róznych 1iczb pierwszych. Znajd $\acute{\mathrm{z}}$ te liczby.

ZADANIE 2.

$\mathrm{W}$ trójkącie prostokątnym na dłuzszej przyprostokątnej jako na średnicy opisano okrąg. Wy-

znacz długość okręgu, $\mathrm{j}\mathrm{e}\dot{\mathrm{z}}$ eli krótsza przyprostokątna jest równa 30, a cięciwa 1ącząca wierzcho-

lek kąta prostego z punktem przecięcia przeciwprostokątnej z okręgiem (róznym od wierzchol-

ków trójkąta) jest równa 24.

ZADANIE 3.

Pewna liczba dwucyfrowa ma trzy dzielniki jednocyfrowe i trzy dzielniki dwucyfrowe. Suma

wszystkich dzielników jednocyfrowych jest równa 8. Ob1icz sumę wszystkich dzie1ników dwu-

cyfrowych tej liczby.

ZADANIE 4.

Pole prostokąta ABCD jest równe 24 $\mathrm{c}\mathrm{m}^{2}$ Na boku $AB$ zaznaczono punkt $E$ rózny od punktów

{\it A} $\mathrm{i}B$, na odcinku $DC$ zaznaczono punkt $F$ rózny od punktów $C\mathrm{i}D$. Pole trójkąta $ADF$ jest

równe 5 $\mathrm{c}\mathrm{m}^{2}$ Oblicz pole trójkąta $CFE.$

ZADANIE 5.

Wykaz$\cdot, \dot{\mathrm{z}}\mathrm{e}$ liczba $n^{3}+11n$ jest podzielna przez 6 d1a $\mathrm{k}\mathrm{a}\dot{\mathrm{z}}$ dej liczby naturalnej $n.$






LIGA MATEMATYCZNA

im. Zdzisława Matuskiego

GRUD Z$\mathrm{I}\mathrm{E}\acute{\mathrm{N}}$ 2013

GIMNAZJUM

ZADANIE I.

Wyznacz najmniejszą $\mathrm{m}\mathrm{o}\dot{\mathrm{z}}$ liwą wartość wyrazenia

$x_{1}x_{2}+x_{2}x_{3}+x_{3}x_{4}+\ldots+x_{100}x_{101}+x_{101}x_{1},$

gdy $\mathrm{k}\mathrm{a}\dot{\mathrm{z}}$ da z liczb $x_{1}, x_{2}, x_{3}, \ldots, x_{101}$ jest równa llub $-1.$

ZADANIE 2.

Podstawa trójkąta równoramiennego ABC ma długość 2 cm, a ramię - 4 cm.

trójkąta, którego wierzcholkami są spodki wysokości trójkąta ABC.

Oblicz obwód

ZADANIE 3.

Wykaz, $\dot{\mathrm{z}}\mathrm{e}\mathrm{j}\mathrm{e}\dot{\mathrm{z}}$ eli liczby $a$ oraz $b$ są dodatnie, to

-{\it a}1 $+$ -{\it b}1 $\geq$ -{\it a} $+$4 {\it b}.

ZADANIE 4.

Wyznacz dwie kolejne liczby naturalne, z których większa dzieli się przez 2009, a mniejsza

przez 45.

ZADANIE 5.

$\mathrm{W}$ kasynie stoją automaty do gry. Pracownik opróznif je i przyniósf do biura 2013 $\dot{\mathrm{z}}$ etonów.

Oświadczyl, $\dot{\mathrm{z}}\mathrm{e}$ wyjąl $\dot{\mathrm{z}}$ etony ze wszystkich 31 automatów, w $\mathrm{k}\mathrm{a}\dot{\mathrm{z}}$ dym bylo co najmniej 50

$\dot{\mathrm{z}}$ etonów, ale w $\dot{\mathrm{z}}$ adnych dwóch maszynach nie bylo tej samej liczby $\dot{\mathrm{z}}$ etonów. Kierownik kasyna

oskarzyf go o oszustwo. Dlaczego?






LIGA MATEMATYCZNA

im. Zdzisława Matuskiego

GRUD Z$\mathrm{I}\mathrm{E}\acute{\mathrm{N}}$ 2014

GIMNAZJUM

ZADANIE I.

Trzy okręgi o jednakowym promieniu r przecinają się w jednym punkcie S i w punktach M,

N, P, przy czym S lezy wewnątrz trójkąta MNP. Oblicz długość promienia okręgu opisanego

na trójkącie MNP.

ZADANIE 2.

Na jednej z pófek biblioteki Bartek umieścil słowniki i encyklopedie. Jedną trzecią tej pólki

zajmują slowniki, a pozostałą część - encyklopedie. $K\mathrm{a}\dot{\mathrm{z}}\mathrm{d}\mathrm{y}$ ze słowników ma grubość 5 cm,

a $\mathrm{k}\mathrm{a}\dot{\mathrm{z}}$ da encyklopedia- 7 cm. Wyznacz najmniejszą $\mathrm{m}\mathrm{o}\dot{\mathrm{z}}$ liwą liczbę woluminów na półce.

ZADANIE 3.

Oblicz sumę cyfr liczby $2^{2010}\cdot 5^{2014}$

ZADANIE 4.

Danych jest 20141iczb natura1nych, o których wiadomo, $\dot{\mathrm{z}}\mathrm{e}$ ich suma jest liczbą nieparzystą.

Jaką liczbq, parzystą czy nieparzystą, jest ich iloczyn?

ZADANIE 5.

Wykaz$\cdot, \dot{\mathrm{z}}\mathrm{e}$ dla dowolnych nieujemnych liczb rzeczywistych $a, b$ spelniona jest nierównošć

$a^{3}+b^{3}\geq a^{2}b+ab^{2}$






LIGA MATEMATYCZNA

im. Zdzisława Matuskiego

GRUD Z$\mathrm{I}\mathrm{E}\acute{\mathrm{N}}$ 2015

GIMNAZJUM

ZADANIE I.

Punkty $D, E, F, G, H, I$ dzielą $\mathrm{k}\mathrm{a}\dot{\mathrm{z}}\mathrm{d}\mathrm{y}$ bok trójkąta $ABC$ na trzy równe części. Oblicz stosunek

pola czworokąta DEGI do pola trójkąta $ABC.$
\begin{center}
\includegraphics[width=57.252mm,height=31.704mm]{./LigaMatematycznaMatuskiego_Gim_Zestaw3_2015_2016_page0_images/image001.eps}
\end{center}
c

H G

1  F

A  D E  B

ZADANIE 2.

$\mathrm{D}\mathrm{u}\dot{\mathrm{z}}$ a bombka na choinkę kosztuje 5 monet, średnia 3 monety, a za trzy mafe bombki w kszta1cie

aniofka trzeba zapfacić jedna monetę. Za sto monet kupiono sto bombek na choinkę. Ile wśród

nich bylo $\mathrm{d}\mathrm{u}\dot{\mathrm{z}}$ ych, średnich i malych bombek? Rozwaz wszystkie $\mathrm{m}\mathrm{o}\dot{\mathrm{z}}$ liwości.

ZADANIE 3.

$\mathrm{W}$ zbiorze liczb rzeczywistych rozwiąz uklad równań

$\left\{\begin{array}{l}
ab=a+b+1\\
bc=b+c+2\\
ac=a+c+5.
\end{array}\right.$

ZADANIE 4.

Znajdujemy ostateczną sumę cyfr liczby naturalnej - sumujemy jej cyfry i $\mathrm{j}\mathrm{e}\dot{\mathrm{z}}$ eli wynik nie

jest jednocyfrowy, to operację powtarzamy do skutku. Na przykład ostateczną sumq cyfr

liczby 78987 jest 3, gdyz $7+8+9+8+7= 39, 3+9 = 12, 1+2 = 3$ i do jej obliczenia

potrzeba trzykrotnego sumowania cyfr. Podaj najmniejszą liczbę, która wymaga czterokrotnego

sumowania, aby wyznaczyć ostateczną sumę jej cyfr.

ZADANIE 5.

Bartek rzucil sto razy kostką do gry i zsumowal liczby wyrzuconych oczek. Czy jest $\mathrm{m}\mathrm{o}\dot{\mathrm{z}}$ liwe,

aby suma ta byfa równa 211, $\mathrm{j}\mathrm{e}\dot{\mathrm{z}}$ eli ani razu nie wypadla liczba parzysta?






LIGA MATEMATYCZNA

im. Zdzisława Matuskiego

GRUD Z$\mathrm{I}\mathrm{E}\acute{\mathrm{N}}$ 2016

GIMNAZJUM

ZADANIE I.

Obwód trójkąta prostokątnego jest równy 132, a suma kwadratów długości boków trójkątajest

równa 6050. Wyznacz dfugości boków trójkąta.

ZADANIE 2.

$\mathrm{W}$ liczbie $\overline{aabb}$ suma cyfr $\alpha \mathrm{i}b$ jest równa ll. Wykaz, $\dot{\mathrm{z}}\mathrm{e}$ ta liczba jest podzielna przez 121.

ZADANIE 3.

Trapez podzielono przekątnymi na cztery trójkąty. Pole trapezu jest równe 20, a stosunek

dlugości jego podstaw jest równy 4. Ob1icz po1e $\mathrm{k}\mathrm{a}\dot{\mathrm{z}}$ dego z otrzymanych trójkątów.

ZADANIE 4.

Dzielna i dzielnik są liczbami dwucyfrowymi, a iloraz i reszta są równymi liczbami jednocyfro-

wymi. Dzielnik jest iloczynem ilorazu i reszty. Wyznacz dzielną.

ZADANIE 5.

Klasa licząca 25 uczniów kupi1a na 1oterii 1osy z numerami od 1 do 25. $K\mathrm{a}\dot{\mathrm{z}}\mathrm{d}\mathrm{y}$ uczeń wyloso-

wał jeden los, a następnie dodał numer losu do swojego numeru w dzienniku. Uzasadnij, $\dot{\mathrm{z}}\mathrm{e}$

przynajmniej jeden uczeń otrzymal w wyniku tego dodawania liczbę parzystą.






LIGA MATEMATYCZNA

im. Zdzisława Matuskiego

GRUD Z$\mathrm{I}\mathrm{E}\acute{\mathrm{N}}$ 2017

GIMNAZJUM

(klasa VII szkoły podstawowej, klasa II i III gimnazjum)

ZADANIE I.

$\mathrm{W}$ trójkąt prostokątny o przyprostokatnych 5 $\mathrm{i}12$ wpisano okrąg. Oblicz najmniejszą z odle-

gfości wierzchołka kąta prostego od punktów tego okręgu.

ZADANIE 2.

$\mathrm{W}$ 2001 roku Adam miaf dwa razy tyle lat, ile wynosi suma cyfr roku jego urodzenia. Ostatniq

cyfrą roku urodzenia Adama jest 7. I1e 1at będzie mia1 Adam w 2018 roku?

ZADANIE 3.

Wykaz, $\dot{\mathrm{z}}\mathrm{e}$ dla $\mathrm{k}\mathrm{a}\dot{\mathrm{z}}$ dej liczby naturalnej $n$ wartość wyrazenia

$\displaystyle \frac{1}{9}(100^{n+1}+4\cdot 10^{n+1}+4)$

jest kwadratem liczby naturalnej.

ZADANIE 4.

Uzasadnij, $\dot{\mathrm{z}}\mathrm{e}\mathrm{j}\mathrm{e}\dot{\mathrm{z}}$ eli do licznika i mianownika wfaściwego dodatniego ulamka dodamy l, to

otrzymamy ulamek większy od wyjściowego.

ZADANIE 5.

Na tablicy napisano pięć liczb, niekoniecznie róznych.

policzyl ich sumę i zapisal wyniki:

Dla $\mathrm{k}\mathrm{a}\dot{\mathrm{z}}$ dej pary tych liczb Mikolaj

1, 2, 3, 5, 5, 6, 7, 8, 9, 10

wymazując początkowe liczby. Wyznacz wszystkie $\mathrm{m}\mathrm{o}\dot{\mathrm{z}}$ liwe wartości iloczynu wymazanych liczb.






LIGA MATEMATYCZNA

im. Zdzisława Matuskiego

GRUD Z$\mathrm{I}\mathrm{E}\acute{\mathrm{N}}$ 2018

GIMNAZJUM

(klasa VII i VIII szkoły podstawowej, klasa III gimnazjum)

ZADANIE I.

Liczba $a$ przy dzieleniu przez 9 daje resztę 2, a mniejsza od niej 1iczba $b$ przy dzieleniu przez 9

daje resztę 7. Znajd $\acute{\mathrm{z}}$ resztę z dzielenia róznicy $a-b$ przez 9.

ZADANIE 2.

Proste $p\mathrm{i}q$ są równolegle, a miary katów przy wierzchofkach $A\mathrm{i}C$ sa równe $135^{\mathrm{o}}$ i $115^{\mathrm{o}}$ tak,

jak na rysunku. Oblicz miarę kąta $ABC.$
\begin{center}
\includegraphics[width=65.892mm,height=37.800mm]{./LigaMatematycznaMatuskiego_Gim_Zestaw3_2018_2019_page0_images/image001.eps}
\end{center}
{\it q}

c

$\rho$

{\it A}

ZADANIE 3.

$\mathrm{W}$ wierszu zapisano kolejno 20171iczb. Pierwsza zapisana 1iczba jest równa 8, a suma $\mathrm{k}\mathrm{a}\dot{\mathrm{z}}$ dych

kolejnych siedmiu liczb jest równa 70. Ob1icz ostatnia z zapisanych 1iczb.

ZADANIE 4.

Znajd $\acute{\mathrm{z}}$ takie liczby naturalne $x, y, \dot{\mathrm{z}}\mathrm{e}xy=8400$ oraz $\mathrm{N}\mathrm{W}\mathrm{D}\{x,y\}=20.$

ZADANIE 5.

Kwadrat $A$ ma dwa boki pokrywające się z promieniami okręgu, a kwadrat $B$ ma dwa wierz-

chofki lezące na tym samym okręgu oraz częściowo wspófdzieli bok z $A$. Wyznacz stosunek

pola kwadratu $A$ do pola kwadratu $B.$
\begin{center}
\includegraphics[width=43.332mm,height=43.332mm]{./LigaMatematycznaMatuskiego_Gim_Zestaw3_2018_2019_page0_images/image002.eps}
\end{center}
{\it A}

{\it B}






LIGA MATEMATYCZNA

Gimnazjum

Półfinał

201utego 2009

ZADANIE I.

Pan Jan produkuje reklamowe chusty w ksztafcie trójkąta prostokątnego równoramiennego.

Poniewaz klienci skarzyli się, $\dot{\mathrm{z}}\mathrm{e}$ są za male, więc postanowił powiększyć je wydłuzając oba

krótsze boki po 10 cm. Skutkiem tego powierzchnia chusty wzrosła o 550 $\mathrm{c}\mathrm{m}^{2}$ Ile jest teraz

równa powierzchnia chusty?

ZADANIE 2.

Mamy 5 kawałków papieru. Niektóre z nich rozcinamy na 5 kawa1ków.

Następnie niektóre

kawałki znów dzielimy na 5 kawa1ków, itd. Czy w ten sposób $\mathrm{m}\mathrm{o}\dot{\mathrm{z}}$ na otrzymać 1000 kawałków

papieru?

ZADANIE 3.

Pawef ma 10 kieszeni i 54 monety jednozfotowe. Chce umieścić swoje pieniądze w kieszeniach

w taki sposób, aby w $\mathrm{k}\mathrm{a}\dot{\mathrm{z}}$ dej kieszeni byla inna ilość monet. Czy jest to $\mathrm{m}\mathrm{o}\dot{\mathrm{z}}$ liwe?

ZADANIE 4.

Czterech przyjaciół wędkarzy, wśród nich Adam i Piotr, wybrało się na ryby. Po zakończonym

wędkowaniu okazało się, $\dot{\mathrm{z}}\mathrm{e}$ trzej z nich- bez Adama- złowili średnio po 14 ryb, a trzej - bez

Piotra- średnio po 10 ryb. Kto złowił więcej ryb: Adam czy Piotr i o i1e?

ZADANIE 5.

Fabryka produkująca cukierki pakuje je do sześciennych pudelek o krawędzi dlugošci 10 cm.

Pudełka te mają być pakowane po 12 sztuk w prostopadłościenne paczki. Jak na1ezy u1ozyć

pudefka, aby pole powierzchni paczki było najmniejsze?






LIGA MATEMATYCZNA

PÓLFINAL

51utego 20l0

GIMNAZJUM

ZADANIE I.

Czy liczba $\sqrt{16+8\sqrt{3}}-\sqrt{16-8\sqrt{3}}$ jest całkowita? Odpowiedz' uzasadnij.

ZADANIE 2.

$\mathrm{W}$ auli odbylo się zebranie uczniów klas pierwszych dotyczące wyboru języków obcych. $\mathrm{K}\mathrm{a}\dot{\mathrm{z}}\mathrm{d}\mathrm{y}$

uczeń wybral co najmniej jeden język i nie więcej $\mathrm{n}\mathrm{i}\dot{\mathrm{z}}$ dwa. 50 uczniów chce uczyć się ję-

zyka angielskiego, 25 języka niemieckiego, 13 języka francuskiego i 5 języka włoskiego. $\dot{\mathrm{Z}}$ aden

z uczniów chcqcych uczyć się języka wfoskiego nie chce uczyć się innego języka. 15 uczniów

spośród chcących uczyć sięjęzyka angielskiego chce uczyć się $\mathrm{t}\mathrm{e}\dot{\mathrm{z}}$ języka niemieckiego, a 3języka

francuskiego. Tylko jeden uczeń zamierza uczyć się języka niemieckiego i języka francuskiego.

Ilu uczniów bylo na tym spotkaniu?

ZADANIE 3.

Rozwiąz układ równań

({\it yx  z}((({\it xxx} $+++${\it yyy} $+++${\it zzz})))$==$4120.

ZADANIE 4.

Wyznacz liczbę naturalną n, która jest podzielna przez 16 i ma 9 mniejszych od siebie dzie1ni-

ków, których suma równa się n.

ZADANIE 5.

Punkty E, F, G, H dzielą boki prostokąta ABCD w stosunku 1 : 2. Jaki jest stosunek po1a

czworokąta EFGH do pola prostokąta ABCD?
\begin{center}
\includegraphics[width=68.376mm,height=33.324mm]{./LigaMatematycznaMatuskiego_Gim_Zestaw4_2009_2010_page0_images/image001.eps}
\end{center}
D  G  c

H

F

A  E  B






LIGA MATEMATYCZNA

PÓLFINAL

181utego 20ll

GIMNAZJUM

ZADANIE I.

Dane są liczby 1, 2, 3, 4, 5, 6. Wykonujemy operację po1egającą na dodaniu do dwóch spošród

nich liczby l. Na sześciu nowych liczbach wykonujemy tę samą operację. Czy powtarzając

wielokrotnie tę czynnošć $\mathrm{m}\mathrm{o}\dot{\mathrm{z}}$ emy uzyskać wszystkie liczby równe?

ZADANIE 2.

$\mathrm{W}$ trapezie ABCD odcinki AB $\mathrm{i}DC$ są równolegle oraz punkt $E$ jest środkiem boku $AD$. Pole

trójkąta $EBC$ jest równe $16\sqrt{7}$. Oblicz pole trapezu ABCD.

ZADANIE 3.

Wykaz, $\dot{\mathrm{z}}\mathrm{e}$

$(\displaystyle \sqrt{2011}+1)(\frac{1}{\sqrt{1}+\sqrt{2}}+\frac{1}{\sqrt{2}+\sqrt{3}}+\frac{1}{\sqrt{3}+\sqrt{4}}+\ldots+\frac{1}{\sqrt{2010}+\sqrt{2011}})$

jest liczbą całkowitą.

ZADANIE 4.

Panowie Pawel, Andrzej i Jarek uczą matematyki, fizyki i chemii w szkołach w Toruniu, Zako-

panem i Warszawie. Wiadomo, $\dot{\mathrm{z}}\mathrm{e}$

$\bullet$ Pan Paweł nie pracuje w Toruniu;

$\bullet$ Pan Andrzej nie pracuje w Warszawie;

$\bullet$ Torunianin nie uczy chemii;

$\bullet$ Warszawiak jest nauczycielem matematyki;

$\bullet$ Pan Andrzej nie uczy fizyki.

Jakiego przedmiotu i w którym miešcie uczy $\mathrm{k}\mathrm{a}\dot{\mathrm{z}}\mathrm{d}\mathrm{y}$ z nich?

ZADANIE 5.

Czy liczba $10^{11}+10^{12}+10^{13}+10^{14}$ jest podzielna przez 101?






LIGA MATEMATYCZNA

PÓLFINAL

161utego 20l2

GIMNAZJUM

ZADANIE I.

Uzasadnij, $\dot{\mathrm{z}}\mathrm{e}$ suma czterech kolejnych liczb naturalnych nieparzystych nie $\mathrm{m}\mathrm{o}\dot{\mathrm{z}}\mathrm{e}$ być liczbą

pierwszą.

ZADANIE 2.

Wykaz, $\dot{\mathrm{z}}\mathrm{e}\sqrt{17-12\sqrt{2}}+\sqrt{17+12\sqrt{2}}$ jest liczbą całkowitą.

ZADANIE 3.

$\mathrm{W}\mathrm{k}\mathrm{a}\dot{\mathrm{z}}$ dym kroku wykonujemy na liczbie jedną z operacji (w dowolnej kolejności):

$\bullet$ podwajamy liczbę;

$\bullet$ skreślamy jej ostatnią cyfrę.

Czy w taki sposób po skończonej ilošci operacji $\mathrm{m}\mathrm{o}\dot{\mathrm{z}}$ na z liczby 378 uzyskać 16?

ZADANIE 4.

Oblicz $1+2-3-4+5+6-7-8+9+10-\ldots-2011-2012+2013+2014.$

ZADANIE 5.

Na prostokątnej tacy Asia ukfadała dwie kwadratowe serwetki o polu 900 $\mathrm{c}\mathrm{m}^{2} \mathrm{k}\mathrm{a}\dot{\mathrm{z}}$ da. Gdy

ułozyła je tak, jak na pierwszym rysunku, to zachodzily na siebie na obszarze o polu 300 $\mathrm{c}\mathrm{m}^{2},$

gdy tak, jak na drugim rysunku, to wspólny obszar miaf 750 $\mathrm{c}\mathrm{m}^{2}$ Jakie pole będzie mial

wspólny obszar obu serwetek, gdy Asia ułozy je w sposób przedstawiony na trzecim rysunku?






LIGA MATEMATYCZNA

im. Zdzisława Matuskiego

PÓLFINAL

71utego 20l3

GIMNAZJUM

ZADANIE I.

Kwadraty ACEG, BCDI, JFGH mają pola równe odpowiednio 3600 $\mathrm{c}\mathrm{m}^{2}, 900 \mathrm{c}\mathrm{m}^{2}$ oraz

400 $\mathrm{c}\mathrm{m}^{2}$ Oblicz pole trójkąta $AIJ.$
\begin{center}
\includegraphics[width=59.340mm,height=58.776mm]{./LigaMatematycznaMatuskiego_Gim_Zestaw4_2012_2013_page0_images/image001.eps}
\end{center}
F  E

H  J

1  D

A  c

ZADANIE 2.

W ulamku

$\mathrm{V}\mathrm{V}$ uIdIIlKu

$\displaystyle \frac{1\cdot 2\cdot 3\cdot\ldots\cdot 23}{1-2+3-4+5-6+\ldots+203}$

licznik jest iloczynem kolejnych liczb naturalnych od l do 23, natomiast w mianowniku ko1ejne

liczby naturalne od l do 203 są naprzemian odejmowane i dodawane. Czy wartość tego u1amka

jest liczbą calkowitą?

ZADANIE 3.

Na boku $CD$ prostokąta ABCD wybrano punkt $E$ w taki sposób, $\dot{\mathrm{z}}\mathrm{e}$ trapez ABCE ma pole

równe 57, 5 $\mathrm{c}\mathrm{m}^{2}$, a pole trapezu ABED jest równe 70 $\mathrm{c}\mathrm{m}^{2}$ Oblicz pole prostokąta ABCD.

ZADANIE 4.

Porównaj liczby $\displaystyle \frac{a}{\alpha-1}$ oraz $\displaystyle \frac{b}{b-1}$, gdy $a\mathrm{i}b$ spefniają warunek $1<a<b.$

ZADANIE 5.

Wykaz, $\dot{\mathrm{z}}\mathrm{e}$ liczba $n^{3}+23n$ jest podzielna przez 6 d1a $\mathrm{k}\mathrm{a}\dot{\mathrm{z}}$ dej liczby naturalnej $n.$






LIGA MATEMATYCZNA

im. Zdzisława Matuskiego

PÓLFINAL

ll lutego 2014

GIMNAZJUM

ZADANIE I.

Wewnątrz kwadratu ABCD wybrano punkt M w równej odległości od boku CD i od wierz-

chołków A oraz B. Jaką częšć pola kwadratu stanowi pole trójkąta ABM?

ZADANIE 2.

Czy 2014 orzechów $\mathrm{m}\mathrm{o}\dot{\mathrm{z}}$ na wlozyć do 50 woreczków w taki sposób, aby w $\mathrm{k}\mathrm{a}\dot{\mathrm{z}}$ dym bylo więcej $\mathrm{n}\mathrm{i}\dot{\mathrm{z}}$

$20$ orzechów, ale w $\mathrm{k}\mathrm{a}\dot{\mathrm{z}}$ dym inna ich liczba? Czy $\mathrm{m}\mathrm{o}\dot{\mathrm{z}}$ na rozłozyć te orzechy tak, aby w $\mathrm{k}\mathrm{a}\dot{\mathrm{z}}$ dym

woreczku było co najmniej 10 orzechów i w $\mathrm{k}\mathrm{a}\dot{\mathrm{z}}$ dym inna ich liczba?

ZADANIE 3.

Spośród trzystu uczniów klas drugich i trzecich gimnazjum 100 wzię1o udzia1 w o1impiadzie

matematycznej, 80 w fizycznej, 60 w informatycznej, w tym 23 w o1impiadzie matematycznej

i fizycznej, 16 w o1impiadzie matematycznej i informatycznej, 14 w o1impiadzie fizycznej i infor-

matycznej, 5 we wszystkich trzech o1impiadach. I1u uczniów wzię1o udzia1 ty1ko w o1impiadzie

matematycznej? Ilu uczniów wzięło udział tylko wjednej olimpiadzie, a ilu dokladnie w dwóch?

Ilu uczniów nie wzięło udzialu w $\dot{\mathrm{z}}$ adnej olimpiadzie?

ZADANIE 4.

$\acute{\mathrm{S}}$ rednia arytmetyczna liczb $a, b, c$ równa się 12, a średnia arytmetyczna 1iczb $2a+1, 2b, c$ równa

się 17. Ob1icz średnią arytmetyczną 1iczb a $\mathrm{i}b.$

ZADANIE 5.

Wykaz$\cdot, \dot{\mathrm{z}}\mathrm{e}$ dla dowolnych liczb rzeczywistych $a, b, c$ spełniona jest nierówność

$a^{2}+b^{2}+c^{2}+3\geq 2(a+b+c).$






AHADEMIA POMORSHA

III SLUPSHU
\begin{center}
\includegraphics[width=40.740mm,height=4.476mm]{./LigaMatematycznaMatuskiego_Gim_Zestaw4_2015_2016_page0_images/image001.eps}
\end{center}
LIGA MATEMATYCZNA

im. Zdzisława Matuskiego

PÓLFINAL
\begin{center}
\includegraphics[width=34.548mm,height=42.576mm]{./LigaMatematycznaMatuskiego_Gim_Zestaw4_2015_2016_page0_images/image002.eps}
\end{center}
2 marca 20l5

GIMNAZJUM

ZADANIE I.

Ania napisała dziesięć liczb calkowitych. Najpierw napisala dwie liczby, a kolejne uzyskiwała

dodając dwie poprzednie. Wyznacz sumę tych liczb, $\mathrm{j}\mathrm{e}\dot{\mathrm{z}}$ eli wiadomo, $\dot{\mathrm{z}}\mathrm{e}$ pierwszą liczbą jest 34,

a ostatnią 0.

ZADANIE 2.

Równoległobok ABCD zbudowany jest z czterech trójkątów równobocznych o boku o dlugo-

ści l. Wyznacz dlugości przekątnych tego równolegloboku.

ZADANIE 3.

Czy z 1000 kwadratów o boku o długości 1 cm $\mathrm{m}\mathrm{o}\dot{\mathrm{z}}$ na ufozyć prostokąt o obwodzie 1005 cm?

ZADANIE 4.

Wykaz$\cdot, \dot{\mathrm{z}}\mathrm{e}$

-{\it aa}2 $++$11 $\geq$ -{\it a} $+$2 1

dla $\mathrm{k}\mathrm{a}\dot{\mathrm{z}}$ dej liczby dodatniej $a.$

ZADANIE 5.

Trzecią część półki w biblioteczce Bartka zajmują ksiązki o grubości 15 mm, ko1ejną trzecią

część tej półki - ksiązki o grubości 12 mm, a pozostałą część - ksiqzki o grubości 18 mm.

Czytając jedną ksiązkę dziennie w czasie wakacji, Bartek przeczytał wszystkie ksiązki z tej

pófki. Zajęło mu to niecałe dwa miesiące. Ile ksiązek bylo na tej pólce?






LIGA MATEMATYCZNA

im. Zdzisława Matuskiego

PÓLFINAL

291utego 20l6

GIMNAZJUM

ZADANIE I.

Dany jest czworokqt wypukly ABCD, gdzie $\triangleleft CBA = 60^{\mathrm{o}}, \triangleleft DBA = 50^{\mathrm{o}}, \triangleleft BAC = 60^{\mathrm{o}},$

$\triangleleft CAD=20^{\mathrm{o}}$ Wyznacz miarę kąta$\triangleleft$ACD.
\begin{center}
\includegraphics[width=37.344mm,height=33.780mm]{./LigaMatematycznaMatuskiego_Gim_Zestaw4_2016_2017_page0_images/image001.eps}
\end{center}
D

c

A  B

ZADANIE 2.

Uzasadnij, $\dot{\mathrm{z}}\mathrm{e}$ dla dowolnej liczby naturalnej $n$ liczba

$\displaystyle \frac{(n+2015)(n+2016)}{2}$

jest naturalna.

ZADANIE 3.

Suma cyfr pewnej liczby trzycyfrowej jest równa ll. $\mathrm{J}\mathrm{e}\dot{\mathrm{z}}$ eli przestawimy cyfry jedności i setek,

nie zmieniając cyfry dziesiątek, to otrzymamy liczbę o 99 mniejszą. Wyznacz wszystkie takie

liczby trzycyfrowe.

ZADANIE 4.

Wykaz, $\dot{\mathrm{z}}\mathrm{e}$ dla dowolnej liczby naturalnej $n$ liczba

11$\ldots$ 122$\ldots$ 233$\ldots$ 344$\ldots$ 4

jest podzielna przez 12, gdy jedynek jest $n$, dwójek jest $2n$, trójek jest $3n$, czwórek jest $4n.$

ZADANIE 5.

Odcinek $BC$ jest średnicą okręgu oraz $|BC|=\sqrt{901}, |BD|=1, |DA|=16$. Niech $|EC|=x.$

Oblicz $x.$
\begin{center}
\includegraphics[width=36.324mm,height=45.156mm]{./LigaMatematycznaMatuskiego_Gim_Zestaw4_2016_2017_page0_images/image002.eps}
\end{center}
A

E

D

B c






LIGA MATEMATYCZNA

im. Zdzisława Matuskiego

PÓLFINAL

161utego 20l7

GIMNAZJUM

ZADANIE I.

$\mathrm{W}$ kwadracie ABCD punkt $E$ jest środkiem boku AD, $F$ jest środkiem boku $DC$ oraz $G$ jest

środkiem odcinka $EF$. Odcinki $EF$ oraz $BG$ podzielily kwadrat na trzy części, z których jedna

- czworokąt- ma pole równe 28. Ob1icz po1e kwadratu.

ZADANIE 2.

$\mathrm{J}\mathrm{e}\dot{\mathrm{z}}$ eli pewną liczbę dwucyfrową pomnozymy przez sumę jej cyfr, to otrzymamy 90. $\mathrm{J}\mathrm{e}\dot{\mathrm{z}}$ eli

przestawimy cyfry tej liczby i $\mathrm{t}\mathrm{e}\dot{\mathrm{z}}$ pomnozymy przez ich sumę, to uzyskamy 306. Znajd $\acute{\mathrm{z}}$ tę liczbę.

ZADANIE 3.

Rozwiąz ukfad równań

$\left\{\begin{array}{l}
a+b=1\\
\frac{1}{2\sqrt{a}}+\frac{1}{2\sqrt{b}}=\frac{2}{\sqrt{a}+\sqrt{b}}.
\end{array}\right.$

ZADANIE 4.

Prostokąt o wymiarach calkowitych zostal rozcięty na dwanaście kwadratów o bokach o dlugości

2, 2, 3, 3, 5, 5, 7, 7, 8, 8, 9, 9. Oblicz obwód tego prostokąta.

ZADANIE 5.

Do zapisania liczby trzydziestocyfrowej wykorzystano dziesięć cyfr 0, dziesięć cyfr 1 i dzie-

sięč cyfr 2. Czy $\mathrm{m}\mathrm{o}\dot{\mathrm{z}}$ na w tej liczbie dokonać takiego przestawienia cyfr, aby otrzymać liczbę

podzielną przez 9?






LIGA MATEMATYCZNA

im. Zdzisława Matuskiego

PÓLFINAL

211utego 20l8

GIMNAZJUM

(klasa VII szkoły podstawowej, klasa II i III gimnazjum)

ZADANIE I.

$\mathrm{W}$ styczniu 1993 roku pani Ania ukończy1a ty1e 1at, i1e wynosi suma cyfr jej roku urodzenia.

$\mathrm{W}$ którym roku urodzila się pani Ania?

ZADANIE 2.

$\mathrm{J}\mathrm{e}\dot{\mathrm{z}}$ eli liczbę dwucyfrową $A$ podzielimy przez sumę jej cyfr, to otrzymamy 4 i resztę 6. $\mathrm{J}\mathrm{e}\dot{\mathrm{z}}$ eli

podzielimy liczbę $A$ przez sumę jej cyfr pomniejszoną o 2, to uzyskamy 5 i resztę 3. Znajd $\acute{\mathrm{z}}$

liczbę $A.$

ZADANIE 3.

Dane sa dwa okręgi o środkach $S_{1}, S_{2}$ i $\mathrm{k}\mathrm{a}\dot{\mathrm{z}}\mathrm{d}\mathrm{y}$ o promieniu 24. Okrąg o środku $S$ jest styczny

zewnętrznie do danych dwóch okręgów oraz do prostej przechodzącej przez punkty $S_{1}$ i $S_{2}$. Wia-

domo, $\dot{\mathrm{z}}\mathrm{e}$ odległość między punktami $S_{1}$ i $S_{2}$ jest równa 72. Ob1icz promień okręgu o środku $S.$

ZADANIE 4.

Wyznacz wszystkie liczby cafkowite $k$, dla których liczba

$k+2016$

$k+2018$

jest calkowita.

ZADANIE 5.

Konik polny skacze wzdfuz prostej. Pierwszy skok ma dlugość l cm, drugi 3 cm (w tę samą

lub w przeciwnq stronę), następny 5 cm, i tak da1ej. Czy $\mathrm{m}\mathrm{o}\dot{\mathrm{z}}\mathrm{e}$ się zdarzyć, $\dot{\mathrm{z}}\mathrm{e}$ po 99 skokach

konik polny znajdzie się w punkcie wyjścia?






50

$\rightarrow\not\subset \mathrm{D}\vdash$

flkademia

P omorskawStupsku

LIGA MATEMATYCZNA

im. Zdzisława Matuskiego

PÓLFINAL 261utego 2019

GIMNAZJUM

(klasa VII i VIII szkoły podstawowej, klasa III gimnazjum)

ZADANIE I.

$\mathrm{W}$ rombie ABCD punkty $M\mathrm{i}N$, rózne od punktów $A, B\mathrm{i}C$, lezą na odcinkach odpowiednio

{\it AB}, $BC$ tak, $\dot{\mathrm{z}}\mathrm{e}$ trójkąt $DMN$ jest równoboczny oraz $|AD| = |MD|$. Wyznacz miarę kąta

$ABC.$

ZADANIE 2.

Znajd $\acute{\mathrm{z}}$ wszystkie liczby trzycyfrowe, które są ll razy większe od sumy swoich cyfr.

ZADANIE 3.

Wyznacz wszystkie pary $(x,y)$ liczb calkowitych dodatnich spełniające warunki $x+y=320$

oraz $\mathrm{N}\mathrm{W}\mathrm{D}\{x,y\}=40.$

ZADANIE 4.

Bartek wybral pewną liczbę (niekoniecznie róznych) liczb ze zbioru $\{-1,0,1,2\}$ w taki sposób,

$\dot{\mathrm{z}}\mathrm{e}$ ich suma jest równa 19, a suma ich kwadratów jest równa 99. Jaka jest największa $\mathrm{m}\mathrm{o}\dot{\mathrm{z}}$ liwa

wartość sumy sześcianów liczb wybranych przez Bartka?

ZADANIE 5.

Sześciokąt, którego wszystkie kąty mają miarę $120^{\mathrm{o}}$ wpisano w trójkąt równoboczny o boku

o dlugości 9. D1ugości trzech boków sześciokąta są równe 5, 5, 3. Ob1icz obwód sześciokąta.
\begin{center}
\includegraphics[width=34.848mm,height=32.508mm]{./LigaMatematycznaMatuskiego_Gim_Zestaw4_2019_2020_page0_images/image001.eps}
\end{center}
3

5






LIGA MATEMATYCZNA

FINAL

25 kwietnia 2009

GIMNAZJUM

ZADANIE I.

Długošci boków trzech kwadratów, które widzisz na rysunku, są liczbami naturalnymi. Wia-

domo, $\dot{\mathrm{z}}\mathrm{e}BC=CD$ oraz zamalowana figura ma pole równe 31 $\mathrm{c}\mathrm{m}^{2}$ Oblicz pole największego

z tych kwadratów.

ZADANIE 2.

Tablicę $3\mathrm{x}3$ podzielono na 9 jednakowych kwadratów, w których umieszczono 1iczby $-1, 0$, 1.

Uzasadnij, $\dot{\mathrm{z}}\mathrm{e}$ wśród ošmiu sum (liczb z $\mathrm{k}\mathrm{a}\dot{\mathrm{z}}$ dego wiersza, $\mathrm{k}\mathrm{a}\dot{\mathrm{z}}$ dej kolumny i glównych przeką-

tnych) co najmniej dwie są równe.

ZADANIE 3.

Na boku $BC$ prostokąta ABCD wybrano punkt $E$ tak, $\dot{\mathrm{z}}\mathrm{e}$ stosunek pól trójkąta $CDE$ i trapezu

ABED jest równy $\displaystyle \frac{1}{4}$. Oblicz $\displaystyle \frac{CE}{EB}.$

ZADANIE 4.

$\mathrm{W}$ pewnej szkole zorganizowano kólko plastyczne, informatyczne i sportowe. $\mathrm{W}$ zajęciach pla-

stycznych uczestniczy 73 uczniów, w informatycznych- 128, w sportowych- 103, przy czym

w plastycznych i informatycznych-28, w p1astycznych i sportowych-20, w informatycznych

i sportowych-43 oraz w p1astycznych, informatycznych i sportowych-7. I1u uczniów uczestni-

czy w zajęciach informatycznych i sportowych, ale nie uczestniczy w plastycznych? Ilu uczniów

uczestniczy dokładnie w dwóch rodzajach zajęć? Ilu uczniów bierze udziaf tylko w zajęciach

sportowych?

ZADANIE 5.

Janek z Olą zebrali trzy razy więcej grzybów $\mathrm{n}\mathrm{i}\dot{\mathrm{z}}$ Franek, a Ola z Frankiem- pięć razy więcej

grzybów $\mathrm{n}\mathrm{i}\dot{\mathrm{z}}$ Janek. Kto uzbierał więcej grzybów: Janek razem z Frankiem czy Ola?






LIGA MATEMATYCZNA

FINAL

26 marca 20l0

GIMNAZJUM

ZADANIE I.

Wykaz, $\dot{\mathrm{z}}\mathrm{e}$ liczba $2009^{2010}-2. 2009^{2009}+2009^{2008}$ jest podzielna przez 2008.

ZADANIE 2.

Rozwiąz układ równań

({\it yx  z}((({\it xxy} $+++${\it yzz})))$==$51130.

ZADANIE 3.

Na zajęcia do Mlodziezowego Centrum Kultury uczęszcza 100 osób: 38 osób na zajęcia te-

atralne, $49-$ na zajęcia muzyczne, $34-$ na zajęcia plastyczne, $9-$ na teatralne i muzyczne,

$8-\mathrm{n}\mathrm{a}$ teatralne i plastyczne, $6-\mathrm{n}\mathrm{a}$ muzyczne i plastyczne. Ile osób bierze udziaf we wszyst-

kich trzech rodzajach zajęć?

ZADANIE 4.

Na okręgu zaznaczono sześć punktów. $K\mathrm{a}\dot{\mathrm{z}}\mathrm{d}\mathrm{y}$ z odcinków fączących te punkty pomalowano

na czerwono lub niebiesko. Wykaz, $\dot{\mathrm{z}}\mathrm{e}$ otrzymano przynajmniej jeden jednokolorowy trójkąt.

ZADANIE 5.

Przekątne czworokąta wypukfego dzielą go na cztery trójkąty. Pola trzech z nich są równe l,

2, 3. Znajd $\acute{\mathrm{z}}$ pole czworokąta.






LIGA MATEMATYCZNA

FINAL

30 marca 20ll

GIMNAZJUM

ZADANIE I.

Na Dzień Kobiet Stefek, Tomek i Romek podarowali swoim dziewczynom: Sabinie, Teresie

i Renacie bukiety kwiatów: róze, storczyki i tulipany. Imiona narzeczonych w $\mathrm{k}\mathrm{a}\dot{\mathrm{z}}$ dej parze

zaczynają się na rózne litery. $\dot{\mathrm{Z}}$ aden chlopiec nie dał dziewczynie kwiatów, których nazwa

zaczyna się na tę samą literę, co jego imię. Wiadomo, $\dot{\mathrm{z}}\mathrm{e}$ ten, który dał storczyki Sabinie ma

imię zaczynające się na tę samą literę, co imię narzeczonej Romka i inną $\mathrm{n}\mathrm{i}\dot{\mathrm{z}}$ nazwa kwiatów,

które Stefek dal narzeczonej. Ktojest parą ijakie kwiaty dal $\mathrm{k}\mathrm{a}\dot{\mathrm{z}}\mathrm{d}\mathrm{y}$ chlopak swojej dziewczynie?

ZADANIE 2.

Liczbę pierwszą 2011 zapisano jako $\alpha^{2}-b^{2}$, gdzie $a\mathrm{i}b$ są liczbami naturalnymi. Oblicz a $\mathrm{i}b.$

ZADANIE 3.

W państwie Cyfry zbudowano dziewięć miast, które nazwano 1, 2, 3, 4, 5, 6, 7, 8, 9. Ki1ka z nich

polączono liniami lotniczymi. Podrózny zauwazył, $\dot{\mathrm{z}}\mathrm{e}$ dwa miasta mają połączenia lotnicze

wtedy i tylko wtedy, gdy dwucyfrowa liczba utworzona z cyfr- nazw tych miast jest podzielna

przez 3. Czy $\mathrm{m}\mathrm{o}\dot{\mathrm{z}}$ na z miasta l dolecieć do miasta 9?

ZADANIE 4.

W trapezie prostokątnym róznica kwadratów dlugošci przekątnych wynosi 21, wysokość jest

równa 4, a dfuzsze ramię ma długość 5. Ob1icz po1e trapezu.

ZADANIE 5.

Wykaz$\cdot, \dot{\mathrm{z}}\mathrm{e}$ równolegloboki ABCD $\mathrm{i}$ DEFG mają równe pola.
\begin{center}
\includegraphics[width=76.356mm,height=33.024mm]{./LigaMatematycznaMatuskiego_Gim_Zestaw5_2010_2011_page0_images/image001.eps}
\end{center}
D  c

F

A  B






1 Liceum O$\mathrm{g}\text{ó} 1\mathrm{o}\mathrm{k}\mathrm{s}\mathrm{z}\mathrm{t}\mathrm{a}1_{\mathrm{C}}\mathrm{a}^{\mathrm{c}\mathrm{e}}\mathrm{w}\mathrm{S}1$psku

AkadmiPomorskawSiupsku

LIGA MATEMATYCZNA

FINAL

ll kwietnia 2012

GIMNAZJUM

ZADANIE I.

Wykaz$\cdot, \dot{\mathrm{z}}\mathrm{e}$ liczba $\sqrt{13-4\sqrt{3}}+\sqrt{37-20\sqrt{3}}$ jest calkowita.

ZADANIE 2.

Jedna z przekątnych wielokąta wypuklego, którego obwód jest równy 31 cm, dzie1i go na dwa

wielokąty o obwodach 21 cm i 30 cm. Wyznacz d1ugość przekątnej.

ZADANIE 3.

$\mathrm{W}$ jednym domu mieszkają bracia Pawel i Gawel. Paweł ma więcej $\mathrm{n}\mathrm{i}\dot{\mathrm{z}}30$, a mniej $\mathrm{n}\mathrm{i}\dot{\mathrm{z}}40$ lat.

Gaweł ma więcej $\mathrm{n}\mathrm{i}\dot{\mathrm{z}}40$, ale mniej $\mathrm{n}\mathrm{i}\dot{\mathrm{z}}50$ lat. Ile lat ma $\mathrm{k}\mathrm{a}\dot{\mathrm{z}}\mathrm{d}\mathrm{y}$ z braci, $\mathrm{j}\mathrm{e}\dot{\mathrm{z}}$ eli wiadomo, $\dot{\mathrm{z}}\mathrm{e}$ iloczyn

ich lat jest równy trzeciej potędze liczby naturalnej?

ZADANIE 4.

Wykaz, $\dot{\mathrm{z}}\mathrm{e}$ suma kwadratów trzech kolejnych liczb calkowitych nieparzystych powiększona

$01$ jest podzielna przez 12.

ZADANIE 5.

Liczby naturalne od l do 1000 pomnozono ko1ejno $\mathrm{k}\mathrm{a}\dot{\mathrm{z}}$ da przez $\mathrm{k}\mathrm{a}\dot{\mathrm{z}}$ dą. Uzasadnij, $\dot{\mathrm{z}}\mathrm{e}$ wśród tych

iloczynów więcej jest liczb parzystych $\mathrm{n}\mathrm{i}\dot{\mathrm{z}}$ nieparzystych.






LIGA MATEMATYCZNA

im. Zdzisława Matuskiego

FINAL

15 kwietnia 20l4

GIMNAZJUM

ZADANIE I.

$\mathrm{W}$ pewnej klasiejest 31 uczniów. Jeden z nich zrobi1 w dyktandzie 13 b1ędów, wszyscy pozosta1i

mniej. Wykaz, $\dot{\mathrm{z}}\mathrm{e}$ przynajmniej trzech uczniów zrobilo po tyle samo blędów.

ZADANIE 2.

Kwadrat podzielono na dwa prostokąty, których stosunek obwodów jest równy 5: 4. Wyznacz

stosunek pól tych prostokątów.

ZADANIE 3.

$\mathrm{J}\mathrm{e}\dot{\mathrm{z}}$ eli w pewnej liczbie skreślimy ostatnią cyfrę, która jest równa 7, to 1iczba zmniejszy się

$0$ 31156. Jaka to liczba?

ZADANIE 4.

Uzasadnij, $\dot{\mathrm{z}}\mathrm{e}$ dla dowolnych liczb rzeczywistych $a, b, c$ spefniona jest nierówność

$2a^{2}+b^{2}+c^{2}\geq 2a(b+c).$

ZADANIE 5.

$\acute{\mathrm{S}}$ redni wiek babci, dziadka i siedmiu wnucząt jest równy 281at, natomiast średni wiek siedmiu

wnucząt jest równy 151at. I1e 1at ma dziadek, $\mathrm{j}\mathrm{e}\dot{\mathrm{z}}$ eli jest starszy od babci o trzy lata?






AHADEMIA POMORSHA

III SLUPSHU
\begin{center}
\includegraphics[width=40.740mm,height=4.476mm]{./LigaMatematycznaMatuskiego_Gim_Zestaw5_2015_2016_page0_images/image001.eps}
\end{center}
LIGA MATEMATYCZNA

im. Zdzislawa Matuskiego

FINAL
\begin{center}
\includegraphics[width=34.548mm,height=42.576mm]{./LigaMatematycznaMatuskiego_Gim_Zestaw5_2015_2016_page0_images/image002.eps}
\end{center}
16 kwietnia 20l5

GIMNAZJUM

ZADANIE I.

Wykaz$\cdot, \dot{\mathrm{z}}\mathrm{e}$ dla dowolnych liczb rzeczywistych $a, b, c$ spelniona jest nierówność

$a^{2}+2b^{2}+3c^{2}-2a-8b-18c>-37.$

ZADANIE 2.

Czy 59 miast $\mathrm{m}\mathrm{o}\dot{\mathrm{z}}$ na pofączyć drogami tak, aby $\mathrm{k}\mathrm{a}\dot{\mathrm{z}}$ de miasto bylo polaczone drogą z trzema

innymi miastami?

ZADANIE 3.

Długości boków AB i AD prostokąta ABCD są równe, odpowiednio, 8 i 4. Punkty E, F, G, H

są środkami boków AB, BC, CD, AD, a punkty MiN są, odpowiednio, środkami odcinków

EFiGH. Oblicz pole trójkąta AMN.

ZADANIE 4.

Niech

$\displaystyle \frac{x}{a-b}=\frac{y}{b-c}=\frac{z}{c-a}=2015,$

gdzie $a, b, c, x, y, z$ są liczbami rzeczywistymi. Oblicz sumę $x+y+z.$

ZADANIE 5.

Do pewnej liczby dwucyfrowej dopisujemy cyfrę 2 raz z 1ewej, raz z prawej strony. Róznica

otrzymanych liczb trzycyfrowych jest dwa razy większa od szukanej liczby. Jaka to liczba?






LIGA MATEMATYCZNA

im. Zdzisława Matuskiego

FINAL

25 kwietnia 20l6

GIMNAZJUM

ZADANIE I.

Wykaz, $\dot{\mathrm{z}}\mathrm{e}$ liczba
\begin{center}
\includegraphics[width=79.812mm,height=10.416mm]{./LigaMatematycznaMatuskiego_Gim_Zestaw5_2016_2017_page0_images/image001.eps}
\end{center}
2222$\ldots$ 23333$\ldots$ 34444$\ldots$ 45555$\ldots$ 5

2n cyfr 2 3n cyfr 3  4n cyfr 4 5n cyfr 5

jest podzielna przez 45 d1a $\mathrm{k}\mathrm{a}\dot{\mathrm{z}}$ dej liczby naturalnej $n.$

ZADANIE 2.

Rozwiąz ukfad równań

$\left\{\begin{array}{l}
(x+y)(x+y+z)=72\\
(y+z)(x+y+z)=120\\
(x+z)(x+y+z)=96.
\end{array}\right.$

ZADANIE 3.

$\mathrm{W}$ trapezie ABCD punkt $S\mathrm{l}\mathrm{e}\dot{\mathrm{z}}\mathrm{y}$ na podstawie $AB$, punkt $R\mathrm{l}\mathrm{e}\dot{\mathrm{z}}\mathrm{y}$ na podstawie $CD$. Odcinki

$DS\mathrm{i}AR$ przecinają się w punkcie $K$, a odcinki $CS\mathrm{i}BR$ przecinają się w punkcie $L$. Wykaz,

$\dot{\mathrm{z}}\mathrm{e}$ suma pól trójkątów $AKD\mathrm{i}LBC$ jest równa polu czworokąta KSLR.

ZADANIE 4.

Na final Ligi Matematycznej w dniu 25 kwietnia przysz1o 149 fina1istów ze szkofy podstawowej

i gimnazjum. $K\mathrm{a}\dot{\mathrm{z}}\mathrm{d}\mathrm{y}$ z nich uściskiem dloni przywital $\mathrm{k}\mathrm{a}\dot{\mathrm{z}}$ dego swego znajomego wśród fina-

listów. Uzasadnij, $\dot{\mathrm{z}}\mathrm{e}$ istnieje finalista, który ma parzystą liczbę znajomych wśród finalistów.

ZADANIE 5.

Wykaz, $\dot{\mathrm{z}}\mathrm{e}$ liczba czterocyfrowa, której cyfra tysięcy jest równa cyfrze dziesiątek, a cyfra setek

jest równa cyfrze jedności, nie $\mathrm{m}\mathrm{o}\dot{\mathrm{z}}\mathrm{e}$ być kwadratem liczby naturalnej.






LIGA MATEMATYCZNA

im. Zdzisława Matuskiego

FINAL

24 kwietnia 20l7

GIMNAZJUM

ZADANIE I.

Na stole pofozono po jednym patyczku o dlugości 2, 4, 6, 8, 9, 10, 30, 40, 50 $\mathrm{i}60$. Adam

zbudowal ramkę wybierając trzy patyczki tak, aby obwód trójkąta byl jak najmniejszy. $\mathrm{Z}$ po-

zostalych patyków Bartek wybral trzy, z których $\mathrm{m}\mathrm{o}\dot{\mathrm{z}}$ na zbudować trójkąt o największym ob-

wodzie. Ostatnimi czterema patykami zainteresowal się Czarek, wybraf trzy i zbudowal z nich

trójkatną ramkę. Który patyk pozostal na stole?

ZADANIE 2.

$\mathrm{W}$ rombie ABCD punkty $M\mathrm{i}N$, rózne od punktów $A, B\mathrm{i}C$, lezą na odcinkach odpowiednio

{\it AB}, $BC$ tak, $\dot{\mathrm{z}}\mathrm{e}$ trójkat $DMN$ jest równoboczny oraz $|AD| = |MD|$. Wyznacz miarę kata

$\triangleleft ABC.$

ZADANIE 3.

Suma liczby trzycyfrowej i liczby otrzymanej z napisania cyfr poprzedniej liczby w odwrotnej

kolejności jest równa 444. Róznicą tych 1iczb jest 198. Wyznacz 1iczbę trzycyfrową wiedząc, $\dot{\mathrm{z}}\mathrm{e}$

suma jej cyfr jest równa 6.

ZADANIE 4.

Czy suma 2017 róznych 1iczb pierwszych $\mathrm{m}\mathrm{o}\dot{\mathrm{z}}\mathrm{e}$ być liczba parzystą? Czy iloczyn 2017 róznych

liczb pierwszych $\mathrm{m}\mathrm{o}\dot{\mathrm{z}}\mathrm{e}$ być liczbą parzystą? Odpowied $\acute{\mathrm{z}}$ uzasadnij.

ZADANIE 5.

Rozwazmy liczby trzycyfrowe zaczynające i kończące się tą samą cyfrą. Wykaz, $\dot{\mathrm{z}}\mathrm{e}\mathrm{j}\mathrm{e}\dot{\mathrm{z}}$ eli suma

pierwszej i drugiej cyfry takiej liczby jest podzielna przez 7, to sama 1iczba $\mathrm{t}\mathrm{e}\dot{\mathrm{z}}$ dzieli się przez 7.






LIGA MATEMATYCZNA

im. Zdzisława Matuskiego

FINAL

16 kwietnia 20l8

GIMNAZJUM

(klasa VII szkoły podstawowej, klasa II i III gimnazjum)

ZADANIE I.

Suma 20181iczb ca1kowitych jest 1iczbą nieparzystą. Jaką 1iczbą, parzystą czy nieparzystą, jest

iloczyn tych liczb? Odpowied $\acute{\mathrm{z}}$ uzasadnij.

ZADANIE 2.

Obecnie ($\mathrm{w}$ 2018 roku) ojciec i syn mają razem $131\mathrm{l}\mathrm{a}\mathrm{t}$. Obaj urodzili się w $XX$ wieku. Ostat-

nie dwie cyfry roku urodzenia ojca stanowią liczbę będacą polową liczby utworzonej z dwóch

ostatnich cyfr roku urodzenia syna. Podaj lata urodzenia ojca i syna.

ZADANIE 3.

Wykaz, $\dot{\mathrm{z}}\mathrm{e}$ róznica kwadratów dowolnej liczby pierwszej większej od 2 i 1iczby o 2 od niej

mniejszej jest podzielna przez 8.

ZADANIE 4.

Dane są dwa kwadraty o bokach $a\mathrm{i}b$ (jak na rysunku). Oblicz stosunek pól czworokąta ABCH

i kwadratu ABCD.
\begin{center}
\includegraphics[width=58.776mm,height=37.440mm]{./LigaMatematycznaMatuskiego_Gim_Zestaw5_2018_2019_page0_images/image001.eps}
\end{center}
{\it F E}

{\it H}

{\it G  b A}

{\it a}

ZADANIE 5.

Wiadomo, $\dot{\mathrm{z}}\mathrm{e}$ punkty $B, C1\mathrm{e}\dot{\mathrm{Z}}$ a na bokach trójkąta $AED, |AB| = |BC|=|CD|= |DE|$ oraz

$\triangleleft ADE=140^{\mathrm{o}}$ Wyznacz miarę kata $EAD.$
\begin{center}
\includegraphics[width=107.436mm,height=20.724mm]{./LigaMatematycznaMatuskiego_Gim_Zestaw5_2018_2019_page0_images/image002.eps}
\end{center}
{\it D}

c





\begin{center}
\includegraphics[width=20.628mm,height=30.024mm]{./LigaMatematycznaMatuskiego_Gim_Zestaw5_2019_2020_page0_images/image001.eps}
\end{center}
0

flkademia

P omorskawStupsku

LIGA MATEMATYCZNA

im. Zdzisława Matuskiego

FINAL 26 marca 2019

GIMNAZJUM

(klasa VII i VIII szkoły podstawowej, klasa III gimnazjum)

ZADANIE I.

Trójkąt podzielono na dwa trójkąty równoramienne i czworokąt o kącie o mierze $80^{\mathrm{o}}$ tak, jak

na rysunku. Wyznacz miarę kąta $\alpha.$
\begin{center}
\includegraphics[width=34.488mm,height=55.020mm]{./LigaMatematycznaMatuskiego_Gim_Zestaw5_2019_2020_page0_images/image002.eps}
\end{center}
$80^{\circ}$

ZADANIE 2.

Ania chce zbudować sześcienną kostkę o wymiarach $4\times 4\times 4$ mając 32 biafe i 32 czarne sześciany

jednostkowe. Planuje uzyskać na powierzchni kostki jak najwięcej bialych ścian sześcianów

jednostkowych. Jaka część powierzchni kostki będzie biafa przy takim ustawieniu?

ZADANIE 3.

Bartek napisal kilka dwucyfrowych liczb naturalnych majqcych tę wlasność, $\dot{\mathrm{z}}\mathrm{e} \mathrm{k}\mathrm{a}\dot{\mathrm{z}}$ de dwie

z nich są względnie pierwsze, ale $\dot{\mathrm{z}}$ adna z nich nie jest liczbą pierwszą. Ile najwięcej liczb mógl

napisać?

ZADANIE 4.

Wyznacz wszystkie liczby całkowite dodatnie $n\mathrm{i}d$ o tej własności, $\dot{\mathrm{z}}\mathrm{e}$ dzieląc liczbę 164 przez

$d$ otrzymamy iloraz $n$ oraz resztę 10.

ZADANIE 5.

Znajd $\acute{\mathrm{z}}$ wszystkie liczby trzycyfrowe, które są 50 razy większe od sumy swoich cyfr.






LIGA MATEMATYCZNA

im. Zdzisława Matuskiego

$\mathrm{P}\mathrm{A}\dot{\mathrm{Z}}$ DZIERNIK 2019

SZKOLA PODSTAWOWA

klasy IV - VI

ZADANIE I.

W pola diagramu wpisano siedem kolejnych liczb naturalnych (w kolejności od najmniejszej

do największej). Suma trzech pierwszych liczb jest równa 69. I1e 1iczb podzie1nych przez 3

znajduje się wśród tych liczb?
\begin{center}
\includegraphics[width=35.556mm,height=6.456mm]{./LigaMatematycznaMatuskiego_Klasa5i6_Zestaw1_2019_2020_page0_images/image001.eps}
\end{center}
ZADANIE 2.

$\mathrm{W}$ regatach $\dot{\mathrm{z}}$ eglarskich wzięło udzia148 ch1opców. Sześciu z nich przyby1o z jednym bra-

tem, dziewięciu z dwoma braćmi i czterech z trzema braćmi. Pozostali chlopcy przybyli bez

rodzeństwa. $\mathrm{Z}$ ilu rodzin było tych 48 ch1opców?

ZADANIE 3.

Prostokątną kartkę papieru podzielono na kwadraty i prostokąt A. Dlugość boku kwadratu $\mathrm{B}$

jest równa 4. Ob1icz obwód prostokąta A.

ZADANIE 4.

Dziewczynki zbierafy koniczynę na lące. Niektóre galązki mialy po trzy listki, a inne po cztery.

Razem zebraly 39 ga1ązek, na których byfo 1ącznie 1281istków. I1e gafązek cztero1istnej koni-

czyny zebraly dziewczynki?

ZADANIE 5.

Ile jest liczb czterocyfrowych o sumie cyfr równej 3? Czy suma tych 1iczb jest 1iczbą podzie1ną

przez 9? A przez 3?






LIGA MATEMATYCZNA

im. Zdzisława Matuskiego

$\mathrm{P}\mathrm{A}\dot{\mathrm{Z}}$ DZIERNIK 2020

SZKOLA PODSTAWOWA

klasy IV - VI

ZADANIE I.

Uczniowie klasy IV zjadają worek prazonej kukurydzy w ciągu 6 minut, uczniowie k1asy V taki

worek zjadają w ciągu 3 minut. $\mathrm{W}$ ciągu ilu minut zostanie zjedzony taki worek kukurydzy

wspólnie przez uczniów obu klas?

ZADANIE 2.

Ania bardzo lubi jabfka, marchewki i ciastka. $K\mathrm{a}\dot{\mathrm{z}}$ dego dnia zjada albo 9 marchewek, a1bo

2 jabfka, albo l jablko i 4 marchewki, albo l ciastko. Przez l0 kolejnych dni Ania zjadla 30

marchewek i 9 jab1ek. I1e ciastek zjad1a dziewczynka w czasie tych 10 dni?

ZADANIE 3.

Znajd $\acute{\mathrm{z}}$ najmniejszą liczbę naturalną podzielnq przez 15, która zapisano za pomocą samych zer

i jedynek.

ZADANIE 4.

Spotkafo się trzech artystów: Adam Bialy-aktor, Bartek Czarny- muzyk i Czarek Rudy-

malarz.

- Zauwazcie, $\dot{\mathrm{z}}\mathrm{e}$ kolor naszych wfosów nie pokrywa się z nazwiskiem. - powiedzial ten z czarnymi

włosami.

- Masz rację. - odpowiedział Adam.

Jaki kolor wlosów mial $\mathrm{k}\mathrm{a}\dot{\mathrm{z}}\mathrm{d}\mathrm{y}$ z nich?

ZADANIE 5.

$\mathrm{W} \mathrm{k}\mathrm{a}\dot{\mathrm{z}}$ dym wierzcholku kwadratu Adam wpisaf pewna liczbę, a na $\mathrm{k}\mathrm{a}\dot{\mathrm{z}}$ dym boku kwadratu

zapisaf sumę liczb z obu końców. Liczby wpisane na bokach kwadratu to cztery z następujących:

15, 11, 19, 21, 23. Która z liczb nie znalazla się na $\dot{\mathrm{z}}$ adnym boku?






LIGA MATEMATYCZNA

im. Zdzislawa Matuskiego

$\mathrm{P}\mathrm{A}\dot{\mathrm{Z}}$ DZIERNIK 2021

SZKOLA PODSTAWOWA

klasy IV - VI

ZADANIE I.

W Polsce popularne są dwa rodzaje biedronek: dwukropki i siedmiokropki. Na planszach w pra-

cowni przyrodniczej uczniowie narysowali 35 biedronek bez kropek i nak1ei1i na nie 175 czarnych

kropek. Ile biedronek ma siedem kropek, a ile dwie kropki?

ZADANIE 2.

W figurze zlozonej z czterech kwadratów, kwadrat B ma bok o dfugości 8 cm. D1ugości dwóch

odcinków zaznaczono na rysunku. Oblicz obwód tej figury.

ZADANIE 3.

$\mathrm{W}$ skarbonce Adama sa tylko monety pięćdziesięciogroszowe i dziesięciogroszowe. Wszystkich

monet jest 40. Pewnego dnia ch1opiec rozmieni1 po1owę posiadanych pięćdziesięciogroszówek na

dziesięciogroszówki i teraz ma w skarbonce 60 monet. I1e dziesięciogroszówek jest wśród nich?

ZADANIE 4.

$\mathrm{W}$ niektóre pola tablicy wpisano 0. $\mathrm{W}$ pozostale puste pola wpisz dwie jedynki, dwie dwójki,

dwie trójki i dwie czwórki tak, aby suma liczb w $\mathrm{k}\mathrm{a}\dot{\mathrm{z}}$ dym wierszu i w $\mathrm{k}\mathrm{a}\dot{\mathrm{z}}$ dej kolumnie była taka

sama. Czy jest tylko jeden sposób uzupelnienia tablicy? Odpowied $\acute{\mathrm{z}}$ uzasadnij.
\begin{center}
\begin{tabular}{|l|l|l|l|}
\hline
\multicolumn{1}{|l|}{}&	\multicolumn{1}{|l|}{$0$}&	\multicolumn{1}{|l|}{ $0$}&	\multicolumn{1}{|l|}{}	\\
\hline
\multicolumn{1}{|l|}{ $0$}&	\multicolumn{1}{|l|}{}&	\multicolumn{1}{|l|}{}&	\multicolumn{1}{|l|}{ $0$}	\\
\hline
\multicolumn{1}{|l|}{}&	\multicolumn{1}{|l|}{ $0$}&	\multicolumn{1}{|l|}{ $0$}&	\multicolumn{1}{|l|}{}	\\
\hline
\multicolumn{1}{|l|}{ $0$}&	\multicolumn{1}{|l|}{}&	\multicolumn{1}{|l|}{}&	\multicolumn{1}{|l|}{ $0$}	\\
\hline
\end{tabular}

\end{center}
ZADANIE 5.

Ile cyfr ma najdluzszy $\mathrm{m}\mathrm{o}\dot{\mathrm{z}}$ liwy ciąg cyfr nie zawierający cyfry 0, w którym $\mathrm{k}\mathrm{a}\dot{\mathrm{z}}$ de dwie kolejne

cyfry tworzą liczbę będącą kwadratem liczby naturalnej?






LIGA MATEMATYCZNA

im. Zdzislawa Matuskiego

$\mathrm{P}\mathrm{A}\dot{\mathrm{Z}}$ DZIERNIK 2022

SZKOLA PODSTAWOWA

klasy IV - VI

ZADANIE I.

$\mathrm{W}$ pewnym sześciokącie $\mathrm{k}\mathrm{a}\dot{\mathrm{z}}$ de dwa kolejne boki są prostopadfe. Dlugości pięciu boków tego

sześciokąta są równe 5, 6, 8, 10, 16. Jaką dfugość $\mathrm{m}\mathrm{o}\dot{\mathrm{z}}\mathrm{e}$ mieć szósty bok?

ZADANIE 2.

Prostokat ABCD podzielono na trzy mniejsze prostokąty tak, jak na rysunku. Wyznacz pole

środkowego prostokąta i jego wymiary wiedząc, $\dot{\mathrm{z}}\mathrm{e}$ pola dwóch prostokątów i dlugości dwóch

odcinków podane są na rysunku.

D

c

8
\begin{center}
\begin{tabular}{|l|}
\hline
\multicolumn{1}{|l|}{$20$}	\\
\hline
\multicolumn{1}{|l|}{}	\\
\hline
\multicolumn{1}{|l|}{ $28$}	\\
\hline
\end{tabular}

\end{center}
A

110

$\mathrm{B}\downarrow$

ZADANIE 3.

Ile jest liczb stucyfrowych, których iloczyn cyfr jest równy 6?

ZADANIE 4.

Adam zbiera modele samochodów. $\mathrm{W}11$ ponumerowanych pudelkach ułozyf 350 mode1i. $\mathrm{W}\mathrm{k}\mathrm{a}\dot{\mathrm{z}}$-

dych trzech kolejnych pudefkach liczba modelijest równa 99. I1e mode1ijest w szóstym pude1ku?

ZADANIE 5.

Ania i Bartek dostali od mamy 35 cukierków. Rozdzie1i1i je na ki1ka ta1erzy, z których $\mathrm{k}\mathrm{a}\dot{\mathrm{z}}\mathrm{d}\mathrm{y}$

zawieral więcej $\mathrm{n}\mathrm{i}\dot{\mathrm{z}}$ jeden cukierek. Następnie z $\mathrm{k}\mathrm{a}\dot{\mathrm{z}}$ dego talerza zabrali po jednym cukierku

i dolozyli je do pierwszego talerza. Wtedy okazalo się, $\dot{\mathrm{z}}\mathrm{e}$ na $\mathrm{k}\mathrm{a}\dot{\mathrm{z}}$ dym talerzu jest tyle samo

cukierków. Ile cukierków bylo początkowo na $\mathrm{k}\mathrm{a}\dot{\mathrm{z}}$ dym talerzu?






LIGA MATEMATYCZNA

im. Zdzisława Matuskiego

LISTOPAD 2019

SZKOLA PODSTAWOWA

klasy IV - VI

ZADANIE I.

Kawa ze śmietanką kosztuje 4, 50 zf.

czarna kawa?

Kawa jest drozsza od śmietanki o 3, 90 zf. I1e kosztuje

ZADANIE 2.

$\mathrm{W}$ pudelku jest 30 ku1: biafe, niebieskie i czarne. Ku1 niebieskich jest 8 razy więcej $\mathrm{n}\mathrm{i}\dot{\mathrm{z}}$ bialych.

Ile jest kul $\mathrm{k}\mathrm{a}\dot{\mathrm{z}}$ dego koloru? Rozwaz wszystkie przypadki.

ZADANIE 3.

Prostokątną dziafkę o obwodzie 100 $\mathrm{m}$ podzielono na dwa prostokąty o obwodach 76 $\mathrm{m}$ i 58 $\mathrm{m}.$

Oblicz wymiary mniejszych dzialek.

ZADANIE 4.

Na zlot smoków przybyły 333 potwory. By1y to smoki trzyg1owe i siedmiog1owe. Razem miafy

1399 głów. Których smoków bylo więcej i o ile?

ZADANIE 5.

Czy przestawiając cyfry liczby 111111222222 $\mathrm{m}\mathrm{o}\dot{\mathrm{z}}$ na uzyskać liczbę pierwszą?






LIGA MATEMATYCZNA

im. Zdzisława Matuskiego

LISTOPAD 2020

SZKOLA PODSTAWOWA

klasy IV - VI

ZADANIE I.

Znajd $\acute{\mathrm{z}}$ najmniejszą liczbę naturalną zfozoną tylko z ósemek i zer podzielną przez 72.

ZADANIE 2.

Bartek ma pięć sześciennych klocków. Gdy są ufozone od najmniejszego do największego, to

wysokości $\mathrm{k}\mathrm{a}\dot{\mathrm{z}}$ dych dwóch sqsiednich klocków róznią się o 2 cm. Wysokość wiezy zbudowa-

nej z dwóch najmniejszych sześcianów jest równa wysokości największego sześcianu. Oblicz

wysokośč wiez $\mathrm{y}$ zbudowanej ze wszystkich pięciu sześciennych klocków.

ZADANIE 3.

Ania i Bartek stoją na sąsiednich stopniach schodów. Gdy Bartek stoi na $\mathrm{n}\mathrm{i}\dot{\mathrm{z}}$ szym stopniu,

a Ania na $\mathrm{w}\mathrm{y}\dot{\mathrm{z}}$ szym, to Ania jest o 5 cm $\mathrm{w}\mathrm{y}\dot{\mathrm{z}}$ sza od Bartka. Gdy zamienią się miejscami,

to Bartek jest $\mathrm{w}\mathrm{y}\dot{\mathrm{z}}$ szy od Ani o 25 cm. Jaką wysokość ma jeden stopień schodów?

ZADANIE 4.

Wiadomo, $\dot{\mathrm{z}}\mathrm{e}$ koty zjadly 999919 myszy, $\mathrm{k}\mathrm{a}\dot{\mathrm{z}}\mathrm{d}\mathrm{y}$ kot zjadl tyle samo myszy i $\mathrm{k}\mathrm{a}\dot{\mathrm{z}}\mathrm{d}\mathrm{y}$ kot zjadl

więcej myszy $\mathrm{n}\mathrm{i}\dot{\mathrm{z}}$ bylo kotów. Ile byfo kotów?

ZADANIE 5.

Ania ma kilkanaście dwuzlotówek, zaś Basia ma tyle samo pieniędzy, ale w monetach pięcio-

zlotowych. Ile monet lącznie mają obie dziewczynki?






LIGA MATEMATYCZNA

im. Zdzislawa Matuskiego

LISTOPAD 2021

SZKOLA PODSTAWOWA

klasy IV - VI

ZADANIE I.

Liczba trzycyfrowa $n$ ma następujące wfasności:

$\bullet$ suma cyfr jest równa 16;

$\bullet$ iloczyn cyfr jest rózny od zera, ale cyfrą jedności tego iloczynu jest zero;

$\bullet$ suma cyfr iloczynu cyfr liczby $n$ jest równa 3.

Znajd $\acute{\mathrm{z}}$ największą liczbę $n$ o tych wlasnościach.

ZADANIE 2.

Uzupelnij brakujące mianowniki

-21$+$-31$+$-\fbox{fbox}1$+$-\fbox{fbox}1$+$--712$+$--1018$+$--2116$=$1.

Wskaz wszystkie rozwiazania.

ZADANIE 3.

Boki czworokąta mają dfugość 8, 6, 5 $\mathrm{i} 7$ (kolejność zapisu tych liczb nie musi być zgodna

z dlugościami kolejnych boków). Przekątna o dlugości 12 dzie1i ten czworokąt na dwa trójkąty.

Podaj obwód $\mathrm{k}\mathrm{a}\dot{\mathrm{z}}$ dego z nich.

ZADANIE 4.

$\mathrm{W}$ pewnej kamienicy jest 9 mieszkań. $K\mathrm{a}\dot{\mathrm{z}}$ de z nich ma dwa lub trzy pokoje. Ile jest mieszkań

trzypokojowych, $\mathrm{j}\mathrm{e}\dot{\mathrm{z}}$ eli fącznie wszystkie mieszkania mają 24 pokoje?

ZADANIE 5.

$\mathrm{W}$ ośmiujednakowych skrzynkach znajduje się 140 bute1ek soku, przy czym wjednej ze skrzynek

brakuje kilku butelek, zaś pozostałe skrzynki są pelne. Ile butelek mieści się w dwunastu pelnych

skrzynkach?






LIGA MATEMATYCZNA

im. Zdzislawa Matuskiego

LISTOPAD 2022

SZKOLA PODSTAWOWA

klasy IV - VI

ZADANIE I.

Pewien bogaty logik zostawif swoim dzieciom następujący testament:

$\mathrm{W}$ ogrodzie rosną kolejno posadzone cztery drzewa owocowe: l- czereśnia, 2- grusza, 3-jab1oń,

4- śliwa. Pod jednym z nich zakopalem skarb. Aby go znalez$\acute{}$ć musicie zrywać po jednym liściu

z tych drzew w następujący sposób:

12343211234321$\ldots.$

Pod drzewem, z którego zerwiecie 20221iść znajduje się skarb

Które to drzewo?

ZADANIE 2.

Figura przedstawiona na rysunku sklada się z czterech przystających prostokątów. $K\mathrm{a}\dot{\mathrm{z}}\mathrm{d}\mathrm{y}$ pro-

stokąt ma boki o dlugości a $\mathrm{i}2a$. Oblicz pole figury, $\mathrm{j}\mathrm{e}\dot{\mathrm{z}}$ eli jej obwód jest równy 80.

ZADANIE 3.

Zagadkowy kalkulator ma tylko dwa klawisze $\fbox{$+1$} \mathrm{i} \fbox{$\times 2$}$. Naciśnięcie klawisza $\fbox{$+1$}$ powoduje

dodanie l do liczby na wyświetlaczu, naciśnięcie klawisza $\fbox{$\times 2$}$ powoduje pomnozenie tej liczby

przez 2. Na wyświet1aczu jest teraz 0. Czy uzyskamy 1iczbę 22, $\mathrm{j}\mathrm{e}\dot{\mathrm{z}}$ eli klawisze $\mathrm{m}\mathrm{o}\dot{\mathrm{z}}$ na nacisnąć

tylko siedem razy?

ZADANIE 4.

Znajd $\acute{\mathrm{z}}$ wszystkie liczby trzycyfrowe, których suma cyfr jest równa 6.

ZADANIE 5.

Bartek wykonaf dwadzieścia rzutów sześcienną kostką do gry i otrzymal w sumie 100 oczek. Co

najwyzej ile razy mógl wyrzucić jedno oczko?






LIGA MATEMATYCZNA

im. Zdzisława Matuskiego

GRUD Z$\mathrm{I}\mathrm{E}\acute{\mathrm{N}}$ 2017

SZKOLA PODSTAWOWA

(klasy IV - VI)

ZADANIE I.

Czy wśród liczb

$\bullet$ 66, 666, 6666, 66666, $\ldots$

$\bullet 55$, 555, 5555, 55555, $\ldots$

znajduje się kwadrat liczby naturalnej?

ZADANIE 2.

Drwal Mikołaj ciął drewno na opał do kominka. Wykonując cięcie, rozcinał zawsze jeden kawa-

lek drewna na dwie części. Po wykonaniu 53 cięć Mikofaj mia172 kawa1ki drewna. I1e kawałków

drewna byfo na początku?

ZADANIE 3.

Na Wigilii spotkalo się dwóch kuzynów- matematyków: Adam i Bartek. Adam zapytal kuzyna,

ile lat ma trójka jego dzieci. Ten odparł, $\dot{\mathrm{z}}\mathrm{e}$ iloczyn ich wieku jest równy 72. D1a Adama ta

informacja nie byla wystarczająca. Wtedy Bartek dodal, $\dot{\mathrm{z}}\mathrm{e}$ suma ich wieku to 14. Jednak

i ta wskazówka nie pozwolila Adamowi ustalić wieku dzieci. Dopiero, gdy Bartek powiedzial,

$\dot{\mathrm{z}}\mathrm{e}$ najmlodsze dziecko ma na imię Ewa, Adam poprawnie ustalif wiek dzieci kuzyna. Ile lat

mają dzieci Bartka?

ZADANIE 4.

Obwód prostokąta jest równy 67. Dwusieczna jednego z kątów dzie1i obwód na dwie części

rózniące się o 20. Ob1icz d1ugości boków prostokąta.

ZADANIE 5.

Dany jest trójkąt prostokątny $ABC$ o kącie prostym przy wierzcholku $C$. Dwusieczne popro-

wadzone z wierzchołków $A\mathrm{i}B$ przecinają się w punkcie $D.$ Znajd $\acute{\mathrm{z}}$ miarę kąta $ADB.$

{\it Dwusieczna kąta jest to pótprosta o początku w wierzchotku kąta dzieląca ten kąt na dwa kąty}

{\it przystajqce}.






LIGA MATEMATYCZNA

im. Zdzisława Matuskiego

GRUD Z$\mathrm{I}\mathrm{E}\acute{\mathrm{N}}$ 2018

SZKOLA PODSTAWOWA

(klasy IV - VI)

ZADANIE I.

Na $\mathrm{k}\mathrm{a}\dot{\mathrm{z}}$ dej ściance sześciennej kostki napisano dodatnią liczbę calkowitą. Iloczyn liczb na prze-

ciwleglych ściankach jest taki sam dla $\mathrm{k}\mathrm{a}\dot{\mathrm{z}}$ dej pary takich ścianek. Napisane liczby nie muszą

być rózne. Jaka jest najmniejsza $\mathrm{m}\mathrm{o}\dot{\mathrm{z}}$ liwa suma wszystkich liczb znajdujących się na kostce?
\begin{center}
\includegraphics[width=29.976mm,height=31.704mm]{./LigaMatematycznaMatuskiego_Klasa5i6_Zestaw3_2018_2019_page0_images/image001.eps}
\end{center}
9 {\it 6}

ZADANIE 2.

Babcia upiekfa na święta $\mathrm{B}\mathrm{o}\dot{\mathrm{z}}$ ego Narodzenia 1000 pierników. Wnuki Ania i Bartek bawią się

w następujacą grę: w $\mathrm{k}\mathrm{a}\dot{\mathrm{z}}$ dym ruchu zabierają z koszyka pofowę pierników, $\mathrm{j}\mathrm{e}\dot{\mathrm{z}}$ eli ich liczba

jest parzysta lub jedno ciastko, $\mathrm{j}\mathrm{e}\dot{\mathrm{z}}$ eli liczba pierników jest nieparzysta. Po ilu ruchach wyjmą

wszystkie pierniki z koszyka?

ZADANIE 3.

W Wigilię bracia Adam, Bartek i Czarek zjedli pólmisek pierogów. Adam zjadl o 6 pierogów

mniej $\mathrm{n}\mathrm{i}\dot{\mathrm{z}}$ Bartek, Bartek zjadl dwa razy więcej pierogów $\mathrm{n}\mathrm{i}\dot{\mathrm{z}}$ Czarek, a Czarek o 2 więcej $\mathrm{n}\mathrm{i}\dot{\mathrm{z}}$

Adam. Ile pierogów zjedli chfopcy?

ZADANIE 4.

Dodając 9 jednakowych 1iczb dwucyfrowych oraz jednq 1iczbę jednocyfrową Miko1aj otrzyma1

257. Znajd $\acute{\mathrm{z}}$ liczbę jednocyfrową.

ZADANIE 5.

Dane są dwa okręgi o środkach $O_{1}, O_{2}$ styczne zewnętrznie, $\mathrm{k}\mathrm{a}\dot{\mathrm{z}}\mathrm{d}\mathrm{y}$ o promieniu 3. Prosta

$p$ równolegfa do prostej $pr(O_{1},O_{2})$ przecina te okręgi w punktach $A, B, C, D$ tak, jak na

rysunku. Odcinek $BC$ ma dfugość 2. Ob1icz d1ugość odcinka $AD.$
\begin{center}
\includegraphics[width=89.868mm,height=30.780mm]{./LigaMatematycznaMatuskiego_Klasa5i6_Zestaw3_2018_2019_page0_images/image002.eps}
\end{center}
$o_{1}  \dot{o}_{2}$






LIGA MATEMATYCZNA

im. Zdzisława Matuskiego

GRUD Z$\mathrm{I}\mathrm{E}\acute{\mathrm{N}}$ 2019

SZKOLA PODSTAWOWA

klasy IV - VI

ZADANIE I.

Suma cyfr pewnej liczby dwucyfrowej jest równa 7. Ob1icz sumę tej 1iczby i 1iczby o przesta-

wionych cyfrach.

ZADANIE 2.

Numery pokoi w pewnym hotelu są liczbami trzycyfrowymi, przy czym cyfra setek oznacza

numer piętra, na którym znajduje się pokój. Na którym piętrze znajduje się pokój, którego

numer jest sześcianem sumy swoich cyfr?

ZADANIE 3.

$\mathrm{W}$ pudełku jest sześć karteczek, $\mathrm{k}\mathrm{a}\dot{\mathrm{z}}$ da biafa lub czarna. Na pięciu z nich zapisane są liczby 5, 7,

8, 9, 13 (na $\mathrm{k}\mathrm{a}\dot{\mathrm{z}}$ dej kartce jedna liczba). Jaka najmniejsza liczba calkowita dodatnia $\mathrm{m}\mathrm{o}\dot{\mathrm{z}}\mathrm{e}$ być

zapisana na szóstej kartce, $\mathrm{j}\mathrm{e}\dot{\mathrm{z}}$ eli suma liczb z bialych kartek jest równa sumie liczb z czarnych

kartek?

ZADANIE 4.

Herb jednego z matematycznych rodów sklada się z nalozonych na siebie dwóch figur: trójkąta

równobocznego i pięciokąta foremnego (jak na rysunku). Oblicz miary katów $\alpha \mathrm{i}\beta.$
\begin{center}
\includegraphics[width=49.524mm,height=43.992mm]{./LigaMatematycznaMatuskiego_Klasa5i6_Zestaw3_2019_2020_page0_images/image001.eps}
\end{center}
$\beta$

0

ZADANIE 5.

Prostokąt o obwodzie 42 podzie1ono na dwa prostokąty i sześć kwadratów tak, jak na rysunku.

Oblicz obwód prostokąta A.
\begin{center}
\includegraphics[width=48.924mm,height=19.764mm]{./LigaMatematycznaMatuskiego_Klasa5i6_Zestaw3_2019_2020_page0_images/image002.eps}
\end{center}





LIGA MATEMATYCZNA

im. Zdzisława Matuskiego

GRUD Z$\mathrm{I}\mathrm{E}\acute{\mathrm{N}}$ 2020

SZKOLA PODSTAWOWA

klasy IV - VI

ZADANIE I.

$\mathrm{W}$ sadzie rośnie więcej $\mathrm{n}\mathrm{i}\dot{\mathrm{z}}90$, ale mniej $\mathrm{n}\mathrm{i}\dot{\mathrm{z}}100$ drzewek owocowych. Trzecią ich częśč stanowią

jabfonie, czwartą część grusze, a resztę wiśnie. Ile drzew jest w sadzie? Podaj liczbę drzew

$\mathrm{k}\mathrm{a}\dot{\mathrm{z}}$ dego gatunku.

ZADANIE 2.

Skacząc z trampoliny na basenie Adam odbija się od niej na wysokość l $\mathrm{m}$, następnie spada

w dó15 $\mathrm{m}$, wreszczie- wyplywając w górę 2 m- osiaga powierzchnię wody. Najakiej wysokości

nad powierzchnią wody znajduje się trampolina?

ZADANIE 3.

Rozwazmy liczby naturalne od l do 90. I1e jest wśród nich 1iczb podzie1nych dokfadnie przez

dwie spośród liczb 2, 3, 5?

ZADANIE 4.

$\mathrm{W}$ szkole uczy się 100 uczniów. Języka angie1skiego uczy się 85 uczniów, języka niemieckiego

75, języka francuskiego 48, języka hiszpańskiego 93. Uzasadnij, $\dot{\mathrm{z}}\mathrm{e}$ co najmniej jeden uczeń

poznaje wszystkie cztery języki obce.

ZADANIE 5.

Na prostokątnej kartce papieru o wymiarach $11\times 24$ Ania nakleila kwadrat $A$ i trzy identyczne

prostokąty $B$ tak, $\dot{\mathrm{z}}\mathrm{e}$ figury nie nachodzify na siebie. Oblicz obwód wyklejonej figury.
\begin{center}
\begin{tabular}{|l|l|l|}
\hline
\multicolumn{1}{|l|}{}&	\multicolumn{1}{|l|}{$8$}&	\multicolumn{1}{|l|}{}	\\
\hline
\multicolumn{1}{|l|}{{\it B}}&	\multicolumn{1}{|l|}{{\it A}}&	\multicolumn{1}{|l|}{$8$}	\\
\hline
\end{tabular}

\end{center}





LIGA MATEMATYCZNA

im. Zdzislawa Matuskiego

GRUD Z$\mathrm{I}\mathrm{E}\acute{\mathrm{N}}$ 2021

SZKOLA PODSTAWOWA

klasy IV - VI

ZADANIE I.

Numer mieszkania Mikolajajest liczbą dwucyfrowa podzielną przez 13, a po dodaniu do numeru

mieszkania liczby l dostajemy wielokrotność liczby ll. Wyznacz sumę cyfr numeru mieszkania

Mikolaja.

ZADANIE 2.

Obwód pięciokąta wypukfego ABCDE jest równy 82. Obwód czworokąta ABCD wynosi 64,

a obwód czworokąta ACDE równa się 45. Ob1icz obwód trójkąta $ACD.$

ZADANIE 3.

Znajd $\acute{\mathrm{z}}$ cyfrę dziesiqtek najmniejszej liczby zlozonej, która niejest podzielna przez $\dot{\mathrm{z}}$ adną z liczb

pierwszych 2, 3, 5, 7.

ZADANIE 4.

$\mathrm{W}$ pracowni plastycznej jest 19 pędz1i, 13 tubek z niebieską farbą, 12 tubek z czerwoną farbą

$\mathrm{i}8$ z zólta. $K\mathrm{a}\dot{\mathrm{z}}\mathrm{d}\mathrm{y}$ uczeń powinien dostać po dwie tubki farb róznych kolorów i jeden pędzel.

Dla ilu mlodych artystów $\mathrm{m}\mathrm{o}\dot{\mathrm{z}}$ na przygotować taki zestaw?

ZADANIE 5.

Uzupelnij diagram liczbami w taki sposób, aby $\mathrm{k}\mathrm{a}\dot{\mathrm{z}}$ de pole sąsiadowalo z dwoma, w których są

wpisane liczby o sumie równej 10. Dwie z 1iczb zosta1y $\mathrm{j}\mathrm{u}\dot{\mathrm{z}}$ wpisane. Oblicz sumę wszystkich

liczb w diagramie.
\begin{center}
\includegraphics[width=67.968mm,height=18.996mm]{./LigaMatematycznaMatuskiego_Klasa5i6_Zestaw3_2021_2022_page0_images/image001.eps}
\end{center}
3

2






LIGA MATEMATYCZNA

im. Zdzislawa Matuskiego

GRUD Z$\mathrm{I}\mathrm{E}\acute{\mathrm{N}}$ 2022

SZKOLA PODSTAWOWA

klasy IV - VI

ZADANIE I.

Znajd $\acute{\mathrm{z}}$ takie trzy cyfry, z których $\mathrm{m}\mathrm{o}\dot{\mathrm{z}}$ na utworzyč dwa rózne trzycyfrowe sześciany liczb natu-

ralnych. Podaj sumę znalezionych cyfr.

ZADANIE 2.

Róznica cyfr liczby dwucyfrowej jest równa 5. Róznica tej 1iczby i 1iczby utworzonej z niej przez

przestawienie cyfr jest równa 45. Znajd $\acute{\mathrm{z}}$ te liczby.

ZADANIE 3.

Do przygotowania paczek świątecznych Mikolaj $\mathrm{u}\dot{\mathrm{z}}$ y1120 mandarynek i 180 pierników. $\mathrm{W}\mathrm{k}\mathrm{a}\dot{\mathrm{z}}$ dej

paczcejest jednakowa liczba mandarynek ijednakowa liczba pierników. Ile maksymalnie paczek

udalo się przygotować?

ZADANIE 4.

Pan Kowalski ma pięciu synów: Adama, Bartka, Czarka, Darka i Edka. Przygotowal dla nich

pod choinkę pięć prezentów: album ze zdjęciami wozów policyjnych, pilkę, wóz strazacki, klocki

i misia. $K\mathrm{a}\dot{\mathrm{z}}\mathrm{d}\mathrm{y}$ z chłopców otrzymal jeden prezent. Wiadomo, $\dot{\mathrm{z}}\mathrm{e}$

$\bullet$ najstarszy syn dostaf album, a najmfodszy misia;

$\bullet$ Czarek jest starszy od Bartka, ale mlodszy od Edka i nie dostal klocków;

$\bullet$ Adam ma czterech starszych braci;

$\bullet$ Bartek dostal pifkę;

$\bullet$ Darek nie jest najstarszy.

Jaki prezent dostal $\mathrm{k}\mathrm{a}\dot{\mathrm{z}}\mathrm{d}\mathrm{y}$ z chłopców?

ZADANIE 5.

Mikolaj połozyf dziewięć orzechów na planszy $9 \times 9$ w sposób przedstawiony na rysunku.

Po chwili Ania przefozyla trzy orzechy na sąsiednie pola (to znaczy na pola mające wspól-

ny wierzchofek lub bok), a sześć zostafo na swoich miejscach. Mimo $\dot{\mathrm{z}}\mathrm{e}$ trzy orzechy zmienily

miejsce, to nadal w $\mathrm{k}\mathrm{a}\dot{\mathrm{z}}$ dym rzędzie poziomym i pionowym znajduje się jeden orzech. Pokaz

w jaki sposób Ania przelozyfa orzechy.
\begin{center}
\begin{tabular}{|l|l|l|l|l|l|l|l|l|}
\hline
\multicolumn{1}{|l|}{}&	\multicolumn{1}{|l|}{}&	\multicolumn{1}{|l|}{}&	\multicolumn{1}{|l|}{}&	\multicolumn{1}{|l|}{}&	\multicolumn{1}{|l|}{}&	\multicolumn{1}{|l|}{}&	\multicolumn{1}{|l|}{}&	\multicolumn{1}{|l|}{}	\\
\hline
\multicolumn{1}{|l|}{}&	\multicolumn{1}{|l|}{}&	\multicolumn{1}{|l|}{}&	\multicolumn{1}{|l|}{}&	\multicolumn{1}{|l|}{}&	\multicolumn{1}{|l|}{}&	\multicolumn{1}{|l|}{}&	\multicolumn{1}{|l|}{}&	\multicolumn{1}{|l|}{}	\\
\hline
\multicolumn{1}{|l|}{}&	\multicolumn{1}{|l|}{}&	\multicolumn{1}{|l|}{}&	\multicolumn{1}{|l|}{}&	\multicolumn{1}{|l|}{}&	\multicolumn{1}{|l|}{}&	\multicolumn{1}{|l|}{}&	\multicolumn{1}{|l|}{}&	\multicolumn{1}{|l|}{}	\\
\hline
\multicolumn{1}{|l|}{}&	\multicolumn{1}{|l|}{}&	\multicolumn{1}{|l|}{}&	\multicolumn{1}{|l|}{}&	\multicolumn{1}{|l|}{}&	\multicolumn{1}{|l|}{}&	\multicolumn{1}{|l|}{}&	\multicolumn{1}{|l|}{}&	\multicolumn{1}{|l|}{}	\\
\hline
\multicolumn{1}{|l|}{}&	\multicolumn{1}{|l|}{}&	\multicolumn{1}{|l|}{}&	\multicolumn{1}{|l|}{}&	\multicolumn{1}{|l|}{}&	\multicolumn{1}{|l|}{}&	\multicolumn{1}{|l|}{}&	\multicolumn{1}{|l|}{}&	\multicolumn{1}{|l|}{}	\\
\hline
\multicolumn{1}{|l|}{}&	\multicolumn{1}{|l|}{}&	\multicolumn{1}{|l|}{}&	\multicolumn{1}{|l|}{}&	\multicolumn{1}{|l|}{}&	\multicolumn{1}{|l|}{}&	\multicolumn{1}{|l|}{}&	\multicolumn{1}{|l|}{}&	\multicolumn{1}{|l|}{}	\\
\hline
\multicolumn{1}{|l|}{}&	\multicolumn{1}{|l|}{}&	\multicolumn{1}{|l|}{}&	\multicolumn{1}{|l|}{}&	\multicolumn{1}{|l|}{}&	\multicolumn{1}{|l|}{}&	\multicolumn{1}{|l|}{}&	\multicolumn{1}{|l|}{}&	\multicolumn{1}{|l|}{}	\\
\hline
\multicolumn{1}{|l|}{}&	\multicolumn{1}{|l|}{}&	\multicolumn{1}{|l|}{}&	\multicolumn{1}{|l|}{}&	\multicolumn{1}{|l|}{}&	\multicolumn{1}{|l|}{}&	\multicolumn{1}{|l|}{}&	\multicolumn{1}{|l|}{}&	\multicolumn{1}{|l|}{}	\\
\hline
\multicolumn{1}{|l|}{}&	\multicolumn{1}{|l|}{}&	\multicolumn{1}{|l|}{}&	\multicolumn{1}{|l|}{}&	\multicolumn{1}{|l|}{}&	\multicolumn{1}{|l|}{}&	\multicolumn{1}{|l|}{}&	\multicolumn{1}{|l|}{}&	\multicolumn{1}{|l|}{}	\\
\hline
\end{tabular}
\end{center}





flkademia

P omorskamStupsku

LIGA MATEMATYCZNA

im. Zdzisława Matuskiego

PÓLFINAL 27 kwietnia 2021

SZKOLA PODSTAWOWA

klasy IV - VI

ZADANIE I.

Ania ma siedem monet dwuzlotowych, a Bartek ma osiem monet pięciozłotowych. Jaką naj-

mniejszą liczbę monet muszą wymienić między sobą, aby mieć równe kwoty?

ZADANIE 2.

Znajd $\acute{\mathrm{z}}$ najmniejszą liczbę naturalną, która w zapisie dziesiętnym ma tylko 0 $\mathrm{i} 1$ oraz jest

podzielna przez 45. Odpowied $\acute{\mathrm{z}}$ uzasadnij.

ZADANIE 3.

Za pięč lat dwie siostry i dwaj bracia będa mieli razem 601at.

piętnaście lat.

Wyznacz lączny ich wiek za

ZADANIE 4.

Tort trzywarstwowy fącznie z paterą, na której stoi, ma 70 cm wysokości. Tort jednowarstwowy

lacznie z taką samą paterą, ma wysokość 36 cm. $\mathrm{W}$ obu tortach $\mathrm{k}\mathrm{a}\dot{\mathrm{z}}$ da warstwa ma tę samą

wysokość. Wyznacz wysokość patery.

ZADANIE 5.

Prostokąt podzielono na dziewięć mniejszych prostokątów.

rysunku. Oblicz obwód $\mathrm{d}\mathrm{u}\dot{\mathrm{z}}$ ego prostokata.

Obwody trzech z nich podano na
\begin{center}
\begin{tabular}{|l|l|l|}
\hline
\multicolumn{1}{|l|}{$20$}&	\multicolumn{1}{|l|}{}&	\multicolumn{1}{|l|}{}	\\
\hline
\multicolumn{1}{|l|}{}&	\multicolumn{1}{|l|}{}&	\multicolumn{1}{|l|}{ $15$}	\\
\hline
\multicolumn{1}{|l|}{}&	\multicolumn{1}{|l|}{ $9$}&	\multicolumn{1}{|l|}{}	\\
\hline
\end{tabular}

\end{center}





flkademia

P omorskamStupsku

LIGA MATEMATYCZNA

im. Zdzisława Matuskiego

PÓLFINAL 15 marca 2022

SZKOLA PODSTAWOWA

klasy IV - VI

ZADANIE I.

Zestaw $A$ zawiera dziesięć róznych liczb jednocyfrowych. Adam wykreślił jedną liczbę, otrzy-

mując zestaw $B$. Suma wszystkich liczb w tym zestawie to 39. Spośród nich Adam wykreś1i1

dwie liczby, uzyskując zestaw $C$ o sumie liczb równej 37. Następnie usuną1 jeszcze trzy 1iczby,

których suma jest równa 22. Zestaw $D$ sklada się z liczb, które pozostaly. Podaj największą

liczbę zestawu $D.$

ZADANIE 2.

Flamastry w sklepie papierniczym zapakowane są w pudelka po 31ub po 5 sztuk. Wszystkich

pudelek jest 30, a w nich 1ącznie 110 f1amastrów. I1e jest pude1ek z trzema f1amastrami?

ZADANIE 3.

Jaki jest najmniejszy $\mathrm{m}\mathrm{o}\dot{\mathrm{z}}$ liwy obwód trójkąta, którego $\mathrm{k}\mathrm{a}\dot{\mathrm{z}}\mathrm{d}\mathrm{y}$ bok ma inną dlugość, a dlugość

$\mathrm{k}\mathrm{a}\dot{\mathrm{z}}$ dego boku jest liczbą pierwszą?

ZADANIE 4.

$\mathrm{Z}$ cyfr 1, 2, 3, 4, 5, 6, 7, 8, 9 Basia wybrafa cztery ko1ejne, z których ufozy1a 1iczby czterocyfrowe

podzielne przez 36. I1e by1o tych 1iczb?

ZADANIE 5.

$\mathrm{Z}$ czterech jednakowych prostokątów i czterech jednakowych kwadratów Ania ulozyfa kwadrat

o obwodzie 32. Potem z tych samych czworokątów powsta1a figura przedstawiona na drugim

rysunku. Oblicz obwód drugiej figury.






LIGA MATEMATYCZNA

im. Zdzisława Matuskiego

PÓLFINAL 281utego 2020

SZKOLAPODSTAklasyIV-VIWOWA

ZADANIE I.

Łączna dfugośč pociagu i mostu jest równa 190 $\mathrm{m}. \mathrm{W}$ pewnym momencie pociąg cafkowicie

wjechal na most i teraz dfugość części mostu bez pociągu jest równa 110 $\mathrm{m}$. Oblicz długość

mostu i dlugość pociągu.

ZADANIE 2.

Kopciuszek mia1100 ziaren maku. Wszystkie wfozy1 do pięciu miseczek w taki sposób, $\dot{\mathrm{z}}\mathrm{e}$

w dwóch pierwszych jest fącznie 30 ziaren, w drugiej i trzeciej 33 ziarna, a w trzeciej i czwartej

41 ziaren. $\mathrm{W}$ piątej jest o ll ziaren więcej $\mathrm{n}\mathrm{i}\dot{\mathrm{z}}$ w pierwszej. Do której miseczki Kopciuszek

wfozyl najmniej ziaren maku?

ZADANIE 3.

Wszystkie smoki i smoczyce mają po sześč łap. $K\mathrm{a}\dot{\mathrm{z}}\mathrm{d}\mathrm{y}$ samiec ma cztery glowy, a $\mathrm{k}\mathrm{a}\dot{\mathrm{z}}$ da samica

trzy glowy. Pewna smocza rodzina kupifa na jesienne pluchy 51 czapek i 84 sztuki ka1oszy.

$K\mathrm{a}\dot{\mathrm{z}}\mathrm{d}\mathrm{y}$ czlonek rodziny na $\mathrm{k}\mathrm{a}\dot{\mathrm{z}}$ dą gfowę zalozyf jedną czapkę, a na $\mathrm{k}\mathrm{a}\dot{\mathrm{z}}$ dą lapę jeden kalosz. Ile

smoków i smoczyc liczy ta rodzina?

ZADANIE 4.

$\mathrm{Z}$ cyfr 1, 2, 3, 4 ukfadamy 1iczby dwucyfrowe o róznych cyfrach. I1e jest takich 1iczb? I1e wśród

nich jest liczb pierwszych?

ZADANIE 5.

Ania wycięła z brystolu jednakowe kartoniki. $K\mathrm{a}\dot{\mathrm{z}}\mathrm{d}\mathrm{y}$ jest prostokątem o bokach o dfugości

16 cm i 7 cm. $\mathrm{Z}$ pięciu takich kartoników dziewczynka ufozyla figurę, jak na rysunku. Oblicz

obwód tej figury.






flkademia

P omorskamStupsku

LIGA MATEMATYCZNA

im. Zdzisława Matuskiego

FINAL 18 maja 2021

SZKOLAPODSTAklasyIV-VIWOWA

ZADANIE I.

Bartek miał osiem karteczek z cyframi 1, 1, 2, 2, 3, 3, 4 $\mathrm{i}4$. Próbował ułozyć z nich liczbę

parzystą podzielną przez 9. $\mathrm{W}$ końcu usunąl jedną karteczkę. $\mathrm{Z}$ siedmiu pozostalych ufozyl

liczbę parzystą podzielną przez 9. Wyznacz największą 1iczbę, którą móg1 utworzyć Bartek.

Odpowied $\acute{\mathrm{z}}$ uzasadnij.

ZADANIE 2.

$\mathrm{W}$ biegu na 100 metrów startuje 625 zawodników. Bieznia stadionu ma 5 torów i ty1ko zwycięzca

$\mathrm{k}\mathrm{a}\dot{\mathrm{z}}$ dego biegu przechodzi do kolejnej rundy, a wszyscy pozostali odpadają z dalszej rywalizacji.

Oblicz najmniejszą liczbę biegów konieczną do wylonienia zwycięzcy zawodów.

ZADANIE 3.

Znajd $\acute{\mathrm{z}}$ wszystkie liczby trzycyfrowe, których iloczyn cyfr jest równy 6.

ZADANIE 4.

Ania ma 183 z1, a Bartek 75 z1. I1e pieniędzy Ania powinna dać Bartkowi, aby zostafo jej dwa

razy więcej $\mathrm{n}\mathrm{i}\dot{\mathrm{z}}$ miafby wtedy chfopiec?

ZADANIE 5.

Pięć kolezanek z grupy kolonijnej ulozyfo kwadrat ze swoich ręczników tak, jak na rysunku.

Ręczniki Ani i Basi mają ksztaft kwadratów, $\mathrm{k}\mathrm{a}\dot{\mathrm{z}}\mathrm{d}\mathrm{y}$ o obwodzie 720 cm. Ręczniki Ce1iny,

Darii i Eli są prostokątami ojednakowych wymiarach. Oblicz obwody prostokątnych ręczników

i $\mathrm{d}\mathrm{u}\dot{\mathrm{z}}$ ego kwadratu utworzonego ze wszystkich ręczników.






flkademia

P omorskamStupsku

LIGA MATEMATYCZNA

im. Zdzisława Matuskiego

FINAL 21 kwietnia 2022

SZKOLA PODSTAWOWA

klasy IV - VI

ZADANIE I.

Ania chce ulozyć kod do swojej szafki w szatni składajacy się z czterech cyfr dopisując do

liczby 77jedną cyfrę z 1ewej i jedną cyfrę z prawej strony w taki sposób, aby otrzymana 1iczba

czterocyfrowa dzielila się przez 18. I1e jest takich kodów? Podaj wszystkie $\mathrm{m}\mathrm{o}\dot{\mathrm{z}}$ liwości.

ZADANIE 2.

Adam zapisal liczbę za pomocą pięciu dziewiątek. Następnie dodal dwie pierwsze cyfry (licząc

od lewej), zmazaf je i w ich miejsce wpisaf otrzymaną sumę. Potem to samo zrobił z nową

liczbą i powtarzal tę czynność z $\mathrm{k}\mathrm{a}\dot{\mathrm{z}}$ dą kolejną uzyskaną liczbą tak dlugo, $\mathrm{a}\dot{\mathrm{z}}$ pojawifa się liczba

jednocyfrowa. Ile razy Adam wykonal opisaną operację zamiany pierwszych dwóch cyfr liczby

na ich sumę?

ZADANIE 3.

Wyznacz cztery kolejne liczby parzyste mające tę wlasność, $\dot{\mathrm{z}}\mathrm{e}$ suma dwóch mniejszych jest

mniejsza od 1000, a suma dwóch większych jest wieksza $\mathrm{n}\mathrm{i}\dot{\mathrm{z}}$ 1000. Podaj wszystkie $\mathrm{m}\mathrm{o}\dot{\mathrm{z}}$ liwości.

ZADANIE 4.

Bartek rzucal dwadzieścia razy sześcienną kostką do gry. Jedno oczko wypadlo dwukrotnie

rzadziej $\mathrm{n}\mathrm{i}\dot{\mathrm{z}}$ dwójka, ale trzy razy częściej $\mathrm{n}\mathrm{i}\dot{\mathrm{z}}$ trójka. Cztery oczka wypadly tyle samo razy

co dwójka, a pięć oczek tyle samo razy co trzy oczka. Ile razy wypadlo sześć oczek?

ZADANIE 5.

Bok kwadratu $A$ ma długość 10, a bok kwadratu $B$ ma dlugość 20. Pomiędzy te kwadraty

wstawiono kwadrat $C$ tak, jak na rysunku. Oblicz obwód otrzymanej figury.
\begin{center}
\includegraphics[width=62.076mm,height=37.284mm]{./LigaMatematycznaMatuskiego_Klasa5i6_Zestaw5_2022_2023_page0_images/image001.eps}
\end{center}
{\it A  C}






LIGA MATEMATYCZNA

im. Zdzisława Matuskiego

$\mathrm{P}\mathrm{A}\dot{\mathrm{Z}}$ DZIERNIK 2019

SZKOLA PODSTAWOWA

klasy VII- VIII

ZADANIE I.

$\mathrm{Z}$ sześciu jednakowych trójkątów prostokątnych o kącie ostrym $60^{\mathrm{o}}$ i najkrótszym boku 10 cm

zbudowano równoleglobok ABCD. Oblicz dfugości obu przekątnych tego równolegfoboku.
\begin{center}
\includegraphics[width=139.596mm,height=26.928mm]{./LigaMatematycznaMatuskiego_Klasa7i8_Zestaw1_2019_2020_page0_images/image001.eps}
\end{center}
ZADANIE 2.

Adam napisal cztery razy z rzędu liczbę dwucyfrową. Wykaz, $\dot{\mathrm{z}}\mathrm{e}$ otrzymana liczba ośmiocyfrowa

jest podzielna przez 101.

ZADANIE 3.

Jedna przekatna pewnego czworokąta dzieli go na dwa trójkąty o obwodach 20 $\mathrm{i}40$, a druga

na trójkąty o obwodach 30 $\mathrm{i}50$. Wiedząc, $\dot{\mathrm{z}}\mathrm{e}$ suma dlugości przekątnych jest równa 26, ob1icz

obwód czworokąta.

ZADANIE 4.

Rozwiąz uklad równań

$\left\{\begin{array}{l}
xy=6\\
yz=12\\
xz=8.
\end{array}\right.$

ZADANIE 5.

Dwa tysiące dziewiętnaście liczb zapisano jedna za drugą. Wiadomo, $\dot{\mathrm{z}}\mathrm{e}$ suma $\mathrm{k}\mathrm{a}\dot{\mathrm{z}}$ dych trzech

kolejnych z nichjest równa 200. Pierwsza z nichjest równa 19, a ostatnia 99. Wyznacz pozosta1e

20171iczb.






LIGA MATEMATYCZNA

im. Zdzisława Matuskiego

$\mathrm{P}\mathrm{A}\dot{\mathrm{Z}}$ DZIERNIK 2020

SZKOLA PODSTAWOWA

klasy VII- VIII

ZADANIE I.

Na promenadzie w Helu koloniści kupowali pamiątki: bursztynowe bransoletki, korale z musze-

lek i pluszowe foczki. $K\mathrm{a}\dot{\mathrm{z}}\mathrm{d}\mathrm{y}$ wybral dwie rózne pamiątki. Foczek kupili dwa razy więcej $\mathrm{n}\mathrm{i}\dot{\mathrm{z}}$

bransoletek, a korali trzy razy więcej $\mathrm{n}\mathrm{i}\dot{\mathrm{z}}$ foczek. Uzasadnij, $\dot{\mathrm{z}}\mathrm{e}$ liczba kolonistów byla podzielna

przez 9, a 1iczba kupionych branso1etek by1a parzysta.

ZADANIE 2.

Pole prostokąta ABCD jest równe l. $K\mathrm{a}\dot{\mathrm{z}}\mathrm{d}\mathrm{y}$ bok tego prostokąta przedfuzono o odcinek równy

temu bokowi i otrzymano punkty $P, Q, R, S$ w taki sposób, $\dot{\mathrm{z}}\mathrm{e}$ punkt $A$ jest środkiem odcinka

$PB, B$ jest środkiem $CQ, C$ jest środkiem $DR, D$ jest środkiem $AS$. Oblicz pole czworokąta

{\it PQRS}.

ZADANIE 3.

Punkt $E\mathrm{l}\mathrm{e}\dot{\mathrm{z}}\mathrm{y}$ wewnątrz kwadratu ABCD tak, $\dot{\mathrm{z}}\mathrm{e}$ trójkąt $ABE$ jest równoboczny. Oblicz miarę

kąta $DCE.$

ZADANIE 4.

$\mathrm{W}$ kolekcji firmy jubilerskiej sa trzy rodzaje naszyjników: z dwiema perlami, z jedną perlą

i takie, które nie mają perel. Naszyjników bez peref jest dwa razy mniej $\mathrm{n}\mathrm{i}\dot{\mathrm{z}}$ wszystkich pozo-

stałych. $\mathrm{W}99$ naszyjnikach jest 100 pereł. I1e jest naszyjników z jedną per1a?

ZADANIE 5.

Liczba trzycyfrowa ma cyfrę jedności równą 5. $\mathrm{J}\mathrm{e}\dot{\mathrm{z}}$ eli do tej liczby dodamy l i otrzymaną sumę

podzielimy przez 3, to otrzymamy 1iczbą trzycyfrową, której cyfra setek jest 1, a następne jej

cyfry są pierwszą i drugą cyfrą liczby wyjściowej. Wyznacz tę liczbę.






LIGA MATEMATYCZNA

im. Zdzislawa Matuskiego

$\mathrm{P}\mathrm{A}\dot{\mathrm{Z}}$ DZIERNIK 2021

SZKOLA PODSTAWOWA

klasy VII- VIII

ZADANIE I.

$\mathrm{W}$ klasach sportowych VIIa i VIIb $\mathrm{k}\mathrm{a}\dot{\mathrm{z}}\mathrm{d}\mathrm{y}$ uczeń gra w siatkówkę lub w koszykówkę. Jedna

piąta wszystkich uczniów uprawia obie dyscypliny sportu, 24 uczniów gra w siatkówkę, $42$ -

w koszykówkę. Ilu uczniów jest w klasach siódmych, ilu uprawia tylko siatkówkę, ilu tylko

koszykówkę, ilu obie te dyscypliny?

ZADANIE 2.

Sto skfadników zmieniono następująco: pierwszą liczbę zmniejszono o l, drugą zwiększono o 2,

trzecią zmniejszono o 3, czwartq zwiększono o 4, i tak da1ej, setną zwiększono o 100. Jak

zmienila się suma tych stu skladników?

ZADANIE 3.

Rozetnij kwadrat na sześć kwadratów. Oblicz stosunek pól największego i najmniejszego z otrzy-

manych kwadratów.

ZADANIE 4.

Ile jest dwunastocyfrowych liczb podzielnych przez 36, które sk1adają się ty1ko z zer ijedynek?

Odpowied $\acute{\mathrm{z}}$ uzasadnij.

ZADANIE 5.

Na rysunku podane są pola czterech prostokątów. Oblicz $x.$
\begin{center}
\begin{tabular}{|l|ll|}
\hline
\multicolumn{1}{|l|}{$23$}&	\multicolumn{1}{|l|}{ $x$}&	\multicolumn{1}{|l|}{ $19$}	\\
\hline
\multicolumn{1}{|l|}{ $17$}&	\multicolumn{1}{|l}{ $51$}&	\multicolumn{1}{l|}{}	\\
\hline
\end{tabular}

\end{center}





LIGA MATEMATYCZNA

im. Zdzislawa Matuskiego

$\mathrm{P}\mathrm{A}\dot{\mathrm{Z}}$ DZIERNIK 2022

SZKOLA PODSTAWOWA

klasy VII- VIII

ZADANIE I.

Wyjez $\mathrm{d}\dot{\mathrm{z}}$ ając na wakacje dwudziestu trzech uczniów klasy VII postanowilo pisać do siebie wia-

domości tekstowe. Pewnego dnia $\mathrm{k}\mathrm{a}\dot{\mathrm{z}}\mathrm{d}\mathrm{y}$ z nich wysfal dwie lub cztery wiadomości. Czy $\mathrm{k}\mathrm{a}\dot{\mathrm{z}}\mathrm{d}\mathrm{y}$

uczeń mógl tego dnia otrzymać dokladnie trzy wiadomości?

ZADANIE 2.

Znajd $\acute{\mathrm{z}}$ taką liczbę trzycyfrową, $\dot{\mathrm{z}}$ eby po dodaniu do niej 500 otrzymać 1iczbę czterocyfrową

podzielną przez 12, a po odjęciu od niej 500 mieč 1iczbę dwucyfrową podzie1ną przez 23.

ZADANIE 3.

Pewien szyfr do sejfu sklada się z 8 róznych cyfr ułozonych ma1ejąco (od 1ewej do prawej).

Liczba ośmiocyfrowa tworzaca szyfr dzieli się przez 180. Jaki to szyfr?

ZADANIE 4.

Suma pięciu liczb trzycyfrowych $\overline{abc}, \overline{bcd}, \overline{cde}, \overline{dea}, \overline{eab}$ jest równa 3996. Ob1icz $a+b+c+d+e.$

ZADANIE 5.

Kwadrat ABCD o polu 400 podzie1ono na kwadrat $K_{1}$ o polu 49, kwadrat $K_{2}$ i figurę $F$. Oblicz

obwód figury $F.$
\begin{center}
\includegraphics[width=46.632mm,height=42.672mm]{./LigaMatematycznaMatuskiego_Klasa7i8_Zestaw1_2022_2023_page0_images/image001.eps}
\end{center}
D  c

$K_{1}$

{\it F}

$K_{2}$

A  B






LIGA MATEMATYCZNA

im. Zdzisława Matuskiego

LISTOPAD 2019

SZKOLA PODSTAWOWA

klasy VII- VIII

ZADANIE I.

Wyznacz takie cyfry $a, b, c, \dot{\mathrm{z}}\mathrm{e}\overline{aaa}+b=\overline{bccc}$. Symbol $\overline{xyz}$ oznacza liczbę trzycyfrowa zapisana

w dziesiętnym systemie pozycyjnym.

ZADANIE 2.

Cztery kolezanki z wakacji: Ania, Basia, Daria i Ela kupily sukienki. $K\mathrm{a}\dot{\mathrm{z}}$ da mieszka w innym

mieście i $\mathrm{k}\mathrm{a}\dot{\mathrm{z}}$ da kupila sukienkę w innym kolorze. Odgadnij, w jakim mieście mieszka $\mathrm{k}\mathrm{a}\dot{\mathrm{z}}$ da

z kolezanek oraz jaki jest kolor jej sukienki, $\mathrm{j}\mathrm{e}\dot{\mathrm{z}}$ eli:

$\bullet$ sukienka Ani nie jest czerwona;

$\bullet$ dziewczyna w zielonej sukience nie mieszka w Slupsku;

$\bullet$ Basia ma białą sukienkę, ale nie mieszka w Gdańsku;

$\bullet$ dziewczynka w czerwonej sukience mieszka w Lęborku;

$\bullet$ Daria mieszka w Bytowie, ajej sukienka nie jest niebieska.

ZADANIE 3.

Wyznacz wszystkie liczby naturalne co najmniej dwucyfrowe, które malejąjedenastokrotnie po

skreśleniu cyfry jedności.

ZADANIE 4.

Przy dzieleniu liczb a, b, c przez 5 otrzymujemy odpowiednio reszty 1, 2, 3.

z dzielenia sumy kwadratów tych liczb przez 5.

Podaj resztę

ZADANIE 5.

Czworokąt podzielono przekątnymi na cztery trójkąty. Pola trzech z nich podane są na rysunku.

Oblicz pole szarego trójkąta.
\begin{center}
\includegraphics[width=34.644mm,height=26.064mm]{./LigaMatematycznaMatuskiego_Klasa7i8_Zestaw2_2019_2020_page0_images/image001.eps}
\end{center}
{\it 8}

{\it 6}

3






LIGA MATEMATYCZNA

im. Zdzisława Matuskiego

LISTOPAD 2020

SZKOLA PODSTAWOWA

klasy VII- VIII

ZADANIE I.

Znajd $\acute{\mathrm{z}}$ dwie takie liczby naturalne $a, b, \dot{\mathrm{z}}\mathrm{e}$ róznica ich iloczynu i ich sumy jest równa 1000 oraz

$a$ jest kwadratem pewnej liczby naturalnej.

ZADANIE 2.

Punkt $E\mathrm{l}\mathrm{e}\dot{\mathrm{z}}\mathrm{y}$ wewnątrz czworokata ABCD oraz $|AE| = 1, |BE| =4, |CE| =3, |DE| =2.$

Czy obwód tego czworokąta $\mathrm{m}\mathrm{o}\dot{\mathrm{z}}\mathrm{e}$ być równy 20?

ZADANIE 3.

Dane są takie liczby calkowite dodatnie $a, b, c, d, \dot{\mathrm{z}}\mathrm{e}\mathrm{k}\mathrm{a}\dot{\mathrm{z}}$ da z sum $a+b, c+d$ jest nieparzysta.

Uzasadnij, $\dot{\mathrm{z}}\mathrm{e}$ iloczyn abcd jest podzielny przez 4.

ZADANIE 4.

Wykaz, $\dot{\mathrm{z}}$ ejezeli między cyfry liczby dwucyfrowej wstawimy 3 i od otrzymanej 1iczby odejmiemy

daną liczbę dwucyfrową, to otrzymamy liczbę podzielną przez 6.

ZADANIE 5.

Dwa kwadraty lezą wewnatrz $\mathrm{d}\mathrm{u}\dot{\mathrm{z}}$ ego kwadratu tak, jak na rysunku.

równe 48. Ob1icz po1e kwadratu $A.$

Pole kwadratu B jest

{\it B}

{\it A}






LIGA MATEMATYCZNA

im. Zdzislawa Matuskiego

LISTOPAD 2021

SZKOLA PODSTAWOWA

klasy VII- VIII

ZADANIE I.

Znajd $\acute{\mathrm{z}}$ wszystkie liczby czterocyfrowe, które mają takie dwa dzielniki, $\dot{\mathrm{z}}\mathrm{e}$ ich suma jest równa

110, a róznica 36.

ZADANIE 2.

Pole prostokąta ABCD jest równe 24. Na boku $AB$ zaznaczono punkt $E$ rózny od punktów $A$

$\mathrm{i}B$, na $DC$ zaznaczono punkt $F$ rózny od punktów $C\mathrm{i}D$. Pole trójkąta $AFD$ jest równe 5.

Oblicz pole trójkąta $ECF.$

ZADANIE 3.

Adam dodał zerową, pierwszą, drugą i trzecią potęgę pewnej liczby naturalnej i otrzyma1400.

Jaka to liczba?

ZADANIE 4.

Na okręgu zaznaczono 55 punktów. Trzy z nich oznaczono $A, B, C$. Ania policzyła punkty od

$A$ do $C$, przechodzac raz przez $B$, i otrzymala 31. Gdy 1iczyfa od $A$ do $B$ przechodząc raz przez

$C$, to uzyskala 39.

$\bullet$ Wyznacz najmniejszą liczbę punktów od $C$ do $B$ przy jednokrotnym przejściu przez

punkt $A.$

$\bullet$ Wyznacz najmniejszą liczbę punktów od $B$ do $A$ przy przejściu przez punkt $C.$

ZADANIE 5.

Wfadca pewnego królestwa nagrodzif swoich dwóch dzielnych rycerzy: starszego 110 dukatami,

mlodszego 100 dukatami. Monety znajdowafy się w dwóch rodzajach sakiewek: w ma1ych by1o

po 7 dukatów, w $\mathrm{d}\mathrm{u}\dot{\mathrm{z}}$ ych po 17 dukatów. $K\mathrm{a}\dot{\mathrm{z}}\mathrm{d}\mathrm{y}$ rycerz otrzyma110 sakiewek.

$\bullet$ Ile $\mathrm{d}\mathrm{u}\dot{\mathrm{z}}$ ych sakiewek otrzymal starszy rycerz?

$\bullet$ Ile malych sakiewek dostal mlodszy rycerz?






LIGA MATEMATYCZNA

im. Zdzislawa Matuskiego

LISTOPAD 2022

SZKOLA PODSTAWOWA

klasy VII- VIII

ZADANIE I.

Ania pomnozyfa pewną liczbę naturalna przez $\mathrm{k}\mathrm{a}\dot{\mathrm{z}}$ dą zjej cyfr i otrzymafa 1995. Jaka to 1iczba?

ZADANIE 2.

Wiadomo, $\dot{\mathrm{z}}\mathrm{e}x+y+z=0$ oraz $xyz=78$. Oblicz $(x+y)(y+z)(x+z).$

ZADANIE 3.

$\mathrm{W}$ kratkach tablicy o wymiarach $9 \times 17$ rozmieszczono liczby naturalne tak, $\dot{\mathrm{z}}\mathrm{e}$ w $\mathrm{k}\mathrm{a}\dot{\mathrm{z}}$ dym

prostokącie o wymiarach $3\rangle\langle 1$ suma liczbjest nieparzysta. Czy suma wszystkich liczb zapisanych

na tablicy jest parzysta?

ZADANIE 4.

Wykaz, $\dot{\mathrm{z}}\mathrm{e}$ liczba $3^{n+3}+3^{n+4}+3^{n+5}+3^{n+6}$ nie jest podzielna przez 7, a1e jest podzie1na przez

$\mathrm{k}\mathrm{a}\dot{\mathrm{z}}$ dą liczbę naturalną mniejszą $\mathrm{n}\mathrm{i}\dot{\mathrm{z}}11$ i rózną od 7.

ZADANIE 5.

Przekątne pewnego czworokąta są prostopadfe i rozcinają go na cztery trójkąty. Pola dwóch

z nich są równe 9 $\mathrm{i}16$, a pola pozostalych dwóch są równe. Oblicz pole czworokqta.






LIGA MATEMATYCZNA

im. Zdzisława Matuskiego

GRUD Z$\mathrm{I}\mathrm{E}\acute{\mathrm{N}}$ 2019

SZKOLA PODSTAWOWA

klasy VII- VIII

ZADANIE I.

Przekatne $AC\mathrm{i}BD$ trapezu ABCD przecinają się w punkcie $O$. Pola trójkątów $AOB\mathrm{i}$ {\it COD}

są równe 9 $\mathrm{i}4$. Oblicz pole trapezu.

ZADANIE 2.

Spośród liczb 30, 31, 32, 33jednajest dzie1nikiem, inna i1orazem, ajeszcze inna resztą w pewnym

dzieleniu liczby trzycyfrowej. Znajd $\acute{\mathrm{z}}$ tę liczbę.

ZADANIE 3.

Liczby $a, b, c$ są calkowite. Wykaz, $\dot{\mathrm{z}}\mathrm{e}$ liczba $(a-b)(b-c)(c-a)$ jest parzysta.

ZADANIE 4.

Na dlugim pasku Adam zapisal liczby calkowite dodatnie w taki sposób, $\dot{\mathrm{z}}\mathrm{e}$ suma $\mathrm{k}\mathrm{a}\dot{\mathrm{z}}$ dych

trzech kolejnych jest równa 10. Liczby 1, 4, 5 są widoczne. Sprawd $\acute{\mathrm{z}}$, czy liczba stojąca na

setnym miejscu jest większa od liczby stojącej na dwusetnym miejscu.
\begin{center}
\includegraphics[width=68.280mm,height=6.396mm]{./LigaMatematycznaMatuskiego_Klasa7i8_Zestaw3_2019_2020_page0_images/image001.eps}
\end{center}
4 1  5

ZADANIE 5.

Grupa 41 studentów za1iczy1a sesję sk1adającą się z trzech egzaminów, w których $\mathrm{m}\mathrm{o}\dot{\mathrm{z}}$ liwymi

ocenami były: bdb, db, dst. Wykaz, $\dot{\mathrm{z}}\mathrm{e}$ co najmniej pięciu studentów zaliczylo sesję zjednako-

wym zbiorem ocen.






LIGA MATEMATYCZNA

im. Zdzisława Matuskiego

GRUD Z$\mathrm{I}\mathrm{E}\acute{\mathrm{N}}$ 2020

SZKOLA PODSTAWOWA

klasy VII- VIII

ZADANIE I.

Cyfrąjedności pewnej liczby czterocyfrowej jest 5. $\mathrm{J}\mathrm{e}\dot{\mathrm{z}}$ eli tę cyfrę przestawimy z ostatniego miej-

sca na pierwsze, to otrzymamy liczbę o 2277 większa od 1iczby pierwotnej. Znajdz' początkową

liczbę.

ZADANIE 2.

$\mathrm{W}$ prostokącie ABCD punkt $E$ jest środkiem boku $AB$, punkt $F$ jest środkiem boku $BC$. Pole

czworokąta ABFD jest równe 19. Ob1icz po1e czworokąta CDEB.

ZADANIE 3.

$\acute{\mathrm{S}}$ rednia arytmetyczna wieku trzech kolegów jest równa 141at. Gdyby najm1odszy by1 dwa razy

starszy, to średnia ich wieku wynosiłaby 181at. Podaj wiek najmfodszego ch1opca.

ZADANIE 4.

Znajd $\acute{\mathrm{z}}$ wszystkie liczby podzielne przez 8, których suma cyfr w ukfadzie dziesiętnym wynosi 7

i iloczyn cyfr jest równy 6.

ZADANIE 5.

Na zewnatrz kwadratu ABCD zbudowano trójkąt równoboczny $AEB.$

$\triangleleft CEA.$

Wyznacz miarę kąta






LIGA MATEMATYCZNA

im. Zdzislawa Matuskiego

GRUD Z$\mathrm{I}\mathrm{E}\acute{\mathrm{N}}$ 2021

SZKOLA PODSTAWOWA

klasy VII- VIII

ZADANIE I.

Suma dwóch liczb jest równa 57460. Jeś1i do mniejszej 1iczby dopiszemy z prawej strony 92, to

otrzymamy równe liczby. Znajd $\acute{\mathrm{z}}$ je.

ZADANIE 2.

Bartek napisa1151iczb natura1nych i ob1iczyf ich sumę otrzymując wynik 2022. Basia dopisała

znak,,minus'' przed kilkoma z tych liczb i obliczyfa sumę wszystkich swoich liczb. Czy mogla

uzyskać wynik llll?

ZADANIE 3.

Pewna liczba ma cztery dzielniki, z których dwa sa liczbami pierwszymi. Mikolaj wypisal je

w kolejności od najmniejszego do największego. Drugi dzielnik jest o 10 większy od pierwszego,

a czwarty o 130 większy od trzeciego. Która 1iczba ma takie dzie1niki?

ZADANIE 4.

Ania dzielila kolejne liczby parzyste (zaczynając od 0) przez pewną 1iczbę natura1ną i wypisy-

wala reszty z ich dzielenia. Początek zapisu byl następujący: 0, 2, 4, 6, 1, 3, 5, 0, 2, 4, $\ldots$. Bartek

$\mathrm{t}\mathrm{e}\dot{\mathrm{z}}$ wypisywal reszty z dzielenia przez tę samą liczbę co Ania. Dzielil jednak kolejne liczby

nieparzyste zaczynając od l. Zakończyl pracę, gdy po raz trzeci uzyska10. I1e 1iczb wypisa1

Bartek?

ZADANIE 5.

Rozetnij kwadrat na siedem kwadratów. Znajd $\acute{\mathrm{z}}$ stosunek obwodów największego i najmniej-

szego z otrzymanych kwadratów.






LIGA MATEMATYCZNA

im. Zdzislawa Matuskiego

GRUD Z$\mathrm{I}\mathrm{E}\acute{\mathrm{N}}$ 2022

SZKOLA PODSTAWOWA

klasy VII- VIII

ZADANIE I.

Ile dzielników ma liczba $2^{2}\cdot 3^{5}+2\cdot 3^{6}+2^{3}\cdot 3^{7}$?

ZADANIE 2.

Na prostej zawierającej wysokość $BD$ trójkąta równobocznego $ABC$ wybrano punkt $K$ tak,

aby $|BK|=|AC|$. Punkt $K$ polaczono z punktami A $\mathrm{i}C$. Oblicz miarę kąta $AKC.$

ZADANIE 3.

Czy liczbę 55555553 $\mathrm{m}\mathrm{o}\dot{\mathrm{z}}$ na przedstawić w postaci sumy dwóch liczb pierwszych?

ZADANIE 4.

Suma cyfr pewnej nieparzystej liczby trzycyfrowej podzielnej przez pięć jest trzy razy większa

$\mathrm{n}\mathrm{i}\dot{\mathrm{z}}$ cyfra jedności. Suma cyfr jedności i setek jest cztery razy większa $\mathrm{n}\mathrm{i}\dot{\mathrm{z}}$ cyfra dziesiątek.

Znajd $\acute{\mathrm{z}}$ tę liczbę.

ZADANIE 5.

$\mathrm{D}\mathrm{u}\dot{\mathrm{z}}\mathrm{y}$ trójkąt podzielono na mniejsze trójkąty tak, jak na rysunku. Liczby wewnątrz malych

trójkatów oznaczają ich obwody. Oblicz obwód $\mathrm{d}\mathrm{u}\dot{\mathrm{z}}$ ego trójkąta.
\begin{center}
\includegraphics[width=84.180mm,height=48.204mm]{./LigaMatematycznaMatuskiego_Klasa7i8_Zestaw3_2022_2023_page0_images/image001.eps}
\end{center}
11

9

14

10  12  20






flkademia

P omorskamStupsku

LIGA MATEMATYCZNA

im. Zdzisława Matuskiego

PÓLFINAL 27 kwietnia 2021

SZKOLA PODSTAWOWA

klasy VII- VIII

ZADANIE I.

$\mathrm{J}\mathrm{e}\dot{\mathrm{z}}$ eli do liczby dwucyfrowej $a$ dopiszemy na początku cyfrę 5, to otrzymamy 1iczbę o 234

mniejszą od liczby, którą otrzymamy po dopisaniu cyfry 5 na końcu 1iczby $a$. Wyznacz liczbę $a.$

ZADANIE 2.

Dany jest trapez ABCD o podstawach AB $\mathrm{i}$ CD, w którym $|AD|=|CD|=|BC|$. Przekątna

$AC$ jest prostopadla do boku $BC$. Oblicz miary kątów tego trapezu.

ZADANIE 3.

Trzy rózne jednocyfrowe liczby pierwsze zapisane w pewnej kolejności utworzyły liczbę trzycy-

frową, która jest podzielna przez $\mathrm{k}\mathrm{a}\dot{\mathrm{z}}$ dą z tych liczb pierwszych. Jaka to liczba?

ZADANIE 4.

Dane są liczby naturalne $a, b$ takie, $\dot{\mathrm{z}}\mathrm{e}3a+5b=ab$. Uzasadnij, $\dot{\mathrm{z}}\mathrm{e}a\mathrm{i}b$ są liczbami parzystymi.

ZADANIE 5.

Danyjest równoleglobok, któregojeden bok ma dlugość 18. Czy przekątne tego równo1eg1oboku

mogą mieć dfugości 16 $\mathrm{i}12$?






LIGA MATEMATYCZNA

im. Zdzisława Matuskiego

PÓLFINAL 281utego 2020

SZKOLA PODSTAWOWA

klasy VII- VIII

ZADANIE I.

Dwa tysiące dwadzieścia liczb zapisano jedna za druga. Druga z nich jest równa 15, a ostatnia

46. Wiadomo, $\dot{\mathrm{z}}\mathrm{e}$ suma $\mathrm{k}\mathrm{a}\dot{\mathrm{z}}$ dych trzech kolejnych liczb jest równa 100. Wyznacz pozosta1e 2018

liczb.

ZADANIE 2.

Kawalek czworokątnego materialu o obwodzie 3 $\mathrm{m}$ przecięto wzdluz jednej przekątnej i powstafy

dwie chusty w ksztalcie trójkątów równoramiennych, pierwszy o obwodzie 1, 8 $\mathrm{m}$, a drugi 2, 8 $\mathrm{m}.$

Linia rozcięcia stanowi podstawę pierwszego trójkata, a dla drugiego trójkąta jest ramieniem.

Wyznacz wymiary obu chust.

ZADANIE 3.

Dane są liczby rzeczywiste $x, y$ spelniające równanie

$(x-y)^{2}+(x+y-4)^{2}=0.$

Oblicz iloczyn tych liczb.

ZADANIE 4.

Na stole $\mathrm{l}\mathrm{e}\dot{\mathrm{z}}\mathrm{y}$ 2020 kapsli. $\mathrm{W}$ jednym ruchu Bartek $\mathrm{m}\mathrm{o}\dot{\mathrm{z}}\mathrm{e}$ zdjąć dokladnie 3, 241ub 51 kaps1i.

Wolno mu wykonać wiele takich ruchów. Czy w pewnej chwili wszystkie kapsle zostaną zdjęte

ze stołu?

ZADANIE 5.

Trzy liczby naturalne dwucyfrowe ustawione w kolejności malejącej stanowia szyfr do sejfu.

Iloczyn pewnych dwóch spośród nich jest równy 888, a i1oczyn innych dwóch jest równy 999.

Znajd $\acute{\mathrm{z}}$ szyfr do sejfu.






flkademia

P omorskamStupsku

LIGA MATEMATYCZNA

im. Zdzisława Matuskiego

FINAL 18 maja 2021

SZKOLA PODSTAWOWA

klasy VII- VIII

ZADANIE I.

Cyfra dziesiątek pewnej liczby dwucyfrowej jest o 4 większa od cyfry jedności. $\mathrm{J}\mathrm{e}\dot{\mathrm{z}}$ eli między

cyfry tej liczby wstawimy 0, to otrzymamy 1iczbę o 630 większą od pierwotnej. Wyznacz

początkową liczbę.

ZADANIE 2.

Oblicz sumę cyfr liczby $4^{1009}\cdot 5^{2021}$

ZADANIE 3.

Dany jest trójkąt równoramienny $ABC$, gdzie $|AC| = |BC|$. Na boku $AB$ wybrano punkt $D$

taki, $\dot{\mathrm{z}}\mathrm{e}|AD|=|CD|$. Miara kąta$\triangleleft$DAC jest równa $27^{\mathrm{o}}$ Oblicz miarę kąta$\triangleleft$DCB.

ZADANIE 4.

Czy istniejq takie liczby naturalne $x, y, z, \dot{\mathrm{z}}\mathrm{e}x+y+z=444$ oraz $xyz=121275$? Odpowiedz'

uzasadnij.

ZADANIE 5.

$\mathrm{W}$ trapezie prostokątnym ABCD, gdzie $AB\Vert DC$, krótsza podstawa jest równa wysokości tra-

pezu, a krótsza przekątna ma dlugość równa dlugości dluzszego ramienia trapezu. Pole trapezu

jest równe 96 $\mathrm{c}\mathrm{m}^{2}$ Oblicz dlugości jego boków.






flkademia

P omorskamStupsku

LIGA MATEMATYCZNA

im. Zdzisława Matuskiego

FINAL 21 kwietnia 2022

SZKOLA PODSTAWOWA

klasy VII- VIII

ZADANIE I.

Cyfrą dziesiatek liczby trzycyfrowej $A$ jest 8. $\mathrm{J}\mathrm{e}\dot{\mathrm{z}}$ eli tę cyfrę przestawimy na miejsce cyfry

jedności, to otrzymamy liczbę o 9 mniejszą od 1iczby $A. \mathrm{J}\mathrm{e}\dot{\mathrm{z}}$ eli przestawimy 8 na miejsce cyfry

setek, to otrzymamy liczbę o 630 wiekszą od 1iczby $A$. Wyznacz liczbę $A.$

ZADANIE 2.

Znajd $\acute{\mathrm{z}}$ najmniejszą liczbę naturalną zapisaną tylko za pomocą zer i jedynek, podzielną przez

45.

ZADANIE 3.

Wokól okrągfego stofu siedzi 13 osób. $K\mathrm{a}\dot{\mathrm{z}}$ da z nich ma na talerzu inną liczbę pierogów. Czy

$\mathrm{m}\mathrm{o}\dot{\mathrm{z}}$ na znalez$\acute{}$ć dwie sąsiednie osoby, które w sumie mają parzystą liczbę pierogów? Odpowied $\acute{\mathrm{z}}$

uzasadnij.

ZADANIE 4.

Adam pomnozyf sześć kolejnych liczb calkowitych dodatnich i uzyskaf iloczyn $A$. Bartek $\mathrm{t}\mathrm{e}\dot{\mathrm{z}}$

pomnozyl sześć kolejnych liczb calkowitych dodatnich, ale zacząl od liczby o l większej $\mathrm{n}\mathrm{i}\dot{\mathrm{z}}$

Adam. Otrzymaf liczbę $B$. Wyznacz najmniejszy czynnik iloczynu Bartka, $\mathrm{j}\mathrm{e}\dot{\mathrm{z}}$ eli $\displaystyle \frac{A}{B}=\frac{5}{6}.$

ZADANIE 5.

$\mathrm{W}$ kwadracie o boku o dlugości 4 cm umieszczono prostokqt tak, jak na rysunku. Ob1icz obwód

tego prostokąta.






LIGA MATEMATYCZNA

$\mathrm{P}\mathrm{A}\acute{\mathrm{Z}}$ DZIERNIK 2009

SZKOLA PONADGIMNAZJALNA

ZADANIE I.

Mamy $n+1$ róznych liczb naturalnych mniejszych od $2n$. Uzasadnij, $\dot{\mathrm{z}}\mathrm{e}\mathrm{m}\mathrm{o}\dot{\mathrm{z}}$ na wybrać z nich

trzy takie, aby jedna była równa sumie pozostałych.

ZADANIE 2.

Wykaz$\cdot, \dot{\mathrm{z}}\mathrm{e}$ okrąg wpisany w trójkqt prostokqtny jest styczny do przeciwprostokątnej w punkcie

dzielącym przeciwprostokątną na dwa odcinki, których iloczyn dlugości jest równy polu tego

trójkąta.

ZADANIE 3.

Znajd $\acute{\mathrm{z}}$ wartość $f(2)$, jeśli dla dowolnego $x$ róznego od zera spełniona jest równość

$f(x)+3f(\displaystyle \frac{1}{x})=x^{2}$

ZADANIE 4.

Wyznacz wszystkie liczby pierwsze $p, q$ takie, $\dot{\mathrm{z}}\mathrm{e}$ liczba $4pq+1$ jest kwadratem liczby naturalnej.

ZADANIE 5.

Od liczby naturalnej odjęto sumę jej cyfr. Następnie z otrzymaną liczbą postąpiono podobnie.

Po wykonaniu ll takich operacji po raz pierwszy otrzymano 0. Jaka była początkowa 1iczba?






LIGA MATEMATYCZNA

$\mathrm{P}\mathrm{A}\acute{\mathrm{Z}}$ DZIERNIK 2010

SZKOLA PONADGIMNAZJALNA

ZADANIE I.

Dwa okręgi są styczne w punkcie S. Przez ten punkt poprowadzono proste KL i MN, od-

powiednio, przecinające pierwszy okrąg w punktach K i M, a drugi w L i N. Udowodnij,

$\dot{\mathrm{z}}\mathrm{e}KM\Vert LN.$

ZADANIE 2.

$\mathrm{W}$ olimpiadzie matematycznej startowało 100 uczniów, w fizycznej 50, w informatycznej 48.

$\mathrm{W}$ co najmniej dwóch olimpiadach startowało dwa razy mniej uczniów $\mathrm{n}\mathrm{i}\dot{\mathrm{z}}$ w co najmniej jednej.

$\mathrm{W}$ trzech olimpiadach bralo udział trzy razy mniej osób $\mathrm{n}\mathrm{i}\dot{\mathrm{z}}$ w co najmniej jednej. Ilu było

wszystkich uczestników tych olimpiad?

ZADANIE 3.

Ile jest funkcji liniowych $f(x) = ax+b$ takich, $\dot{\mathrm{z}}\mathrm{e} f(b) = 2009a$, gdzie $a \mathrm{i} b$ są liczbami

calkowitymi?

ZADANIE 4.

Pierwszym wyrazem ciągu jest l, drugim 3, a $\mathrm{k}\mathrm{a}\dot{\mathrm{z}}\mathrm{d}\mathrm{y}$ następny wyraz jest sumą dwóch poprzed-

nich. Jaka jest cyfra jedności tysięcznego wyrazu?

ZADANIE 5.

Uzasadnij, $\dot{\mathrm{z}}\mathrm{e}$ liczba $2^{2010}+3^{2012}$ jest złozona.






LIGA MATEMATYCZNA

$\mathrm{P}\mathrm{A}\acute{\mathrm{Z}}$ DZIERNIK 2011

SZKOLA PONADGIMNAZJALNA

ZADANIE I.

Znajd $\acute{\mathrm{z}}$ wszystkie funkcje $f:\mathbb{R}\rightarrow \mathbb{R}$ takie, $\dot{\mathrm{z}}\mathrm{e}$

$xf(x)-f(1-x)=2$

dla $\mathrm{k}\mathrm{a}\dot{\mathrm{z}}$ dej liczby rzeczywistej $x.$

ZADANIE 2.

$\mathrm{W}$ ostrokątnym trójkącie $ABC$ poprowadzono wysokości AD $\mathrm{i}$ {\it CE}. Znajd $\acute{\mathrm{z}}$ miarę kąta przy

wierzchofku $B, \mathrm{j}\mathrm{e}\dot{\mathrm{z}}$ eli wiadomo, $\dot{\mathrm{z}}\mathrm{e}|AC|=2|DE|.$

ZADANIE 3.

Znajd $\acute{\mathrm{z}}$ wszystkie liczby trzycyfrowe $n$ takie, $\dot{\mathrm{z}}\mathrm{e}$

$\displaystyle \frac{f(n)}{n}=1,$

gdzie f(n) oznacza sumę cyfr liczby n, iloczynu jej cyfr oraz trzech iloczynów róznych par cyfr

liczby n.

ZADANIE 4.

Dana jest liczba rzeczywista $b$, gdzie $b \not\in \{-1,0,1\}$. Definiujemy liczby $a_{n}$ w następujący

sposób:

$\left\{\begin{array}{l}
\alpha_{1}=\frac{b-1}{b+1}\\
a_{n+1}=\frac{a_{n}-1}{a_{n}+1},n\in \mathbb{N},n\geq 1.
\end{array}\right.$

Oblicz $b$ wiedząc, $\dot{\mathrm{z}}\mathrm{e}$ a2011$=2011.$

ZADANIE 5.

Oblicz

$\displaystyle \frac{1}{2\sqrt{1}+\sqrt{2}}+\frac{1}{3\sqrt{2}+2\sqrt{3}}+\ldots+\frac{1}{100\sqrt{99}+99\sqrt{100}}.$






LIGA MATEMATYCZNA

im. Zdzisława Matuskiego

$\mathrm{P}\mathrm{A}\acute{\mathrm{Z}}$ DZIERNIK 2012

SZKOLA PONADGIMNAZJALNA

ZADANIE I.

Wyznacz wszystkie funkcje $f$: $\mathbb{R}\rightarrow \mathbb{R}$ spelniające warunek $2f(x)+f(1-x) =x^{2}$ dla $\mathrm{k}\mathrm{a}\dot{\mathrm{z}}$ dej

liczby rzeczywistej $x.$

ZADANIE 2.

$K\mathrm{a}\dot{\mathrm{z}}\mathrm{d}\mathrm{y}$ punkt płaszczyzny pokolorowano jednym z dwóch kolorów.

punkty tego samego koloru odlegle od siebie o l.

Wykaz, $\dot{\mathrm{z}}\mathrm{e}$ istnieją dwa

ZADANIE 3.

Podziel kwadrat o boku dfugości 6 na osiem trójkątów o po1ach równych odpowiednio 1, 2, 3,

4, 5, 6, 7, 8.

ZADANIE 4.

W trójkąt prostokątny o przyprostokątnych a, b, wpisano kwadrat, którego wszystkie wierz-

cholki nalezą do boków trójkąta. Oblicz pole tego kwadratu.

ZADANIE 5.

Wyznacz wszystkie trójki liczb pierwszych $p, q, r$, które spelniają warunek $\displaystyle \frac{pqr}{p+q+r}=11.$






LIGA MATEMATYCZNA

im. Zdzisława Matuskiego

$\mathrm{P}\mathrm{A}\overline{\mathrm{Z}}$ DZIERNIK 2013

SZKOLA PONADGIMNAZJALNA

ZADANIE I.

Wyznacz wszystkie funkcje $f$: $\mathbb{R}\backslash \{0\}\rightarrow \mathbb{R}$ spelniające warunek $f(x)+2f(\displaystyle \frac{1}{x})=x$ dla $\mathrm{k}\mathrm{a}\dot{\mathrm{z}}$ dej

liczby rzeczywistej $x$ róznej od zera.

ZADANIE 2.

Na okręgu dane są punkty w kolejności $A, B, C, D$. Niech $M$ będzie środkiem luku $AB.$

Oznaczmy punkty przecięcia cięciw $MC\mathrm{i}MD$ z cięciwą AB, odpowiednio, $E$ oraz $K$. Wykaz,

$\dot{\mathrm{z}}\mathrm{e}$ na czworokącie EKDC $\mathrm{m}\mathrm{o}\dot{\mathrm{z}}$ na opisać okrąg.

ZADANIE 3.

Rozwia $\dot{\mathrm{z}}$ uklad równań

$\left\{\begin{array}{l}
x^{2}-(y-z)^{2}=1\\
y^{2}-(z-x)^{2}=4\\
z^{2}-(x-y)^{2}=9.
\end{array}\right.$

ZADANIE 4.

Uzasadnij, $\dot{\mathrm{z}}\mathrm{e}$ liczba

$3^{1}+3^{2}+3^{3}+\ldots+3^{998}+3^{999}$

jest podzielna przez 13.

ZADANIE 5.

Niech $a, b, c$ będą liczbami nieparzystymi. Wykaz, $\dot{\mathrm{z}}\mathrm{e}$ nie istnieje liczba cafkowita $x$ spefniająca

równość

$ax^{2}+bx+c=0.$






LIGA MATEMATYCZNA

im. Zdzisława Matuskiego

$\mathrm{P}\mathrm{A}\overline{\mathrm{Z}}$ DZIERNIK 2014

SZKOLA PONADGIMNAZJALNA

ZADANIE I.

Na boku $AC$ trójkąta $ABC$ wybrano punkty $D\mathrm{i}E$ w taki sposób, $\dot{\mathrm{z}}\mathrm{e}|AB|=|AD|, |BE|=|EC|$

oraz punkt $E \mathrm{l}\mathrm{e}\dot{\mathrm{z}}\mathrm{y}$ pomiędzy punktami A $\mathrm{i} D$. Niech $F$ będzie środkiem luku $BC$ okręgu

opisanego na trójkącie $ABC$. Wykaz, $\dot{\mathrm{z}}\mathrm{e}$ punkty $B, E, D, F$ lezą na jednym okręgu.

ZADANIE 2.

Wewnatrz sześcianu o krawędzi 13 cm wybrano w dowo1ny sposób 2014 punktów. Czy w tym

sześcianie zawartyjest sześcian o krawędzi l cm, w którego wnętrzu nie ma $\dot{\mathrm{z}}$ adnego z wybranych

punktów?

ZADANIE 3.

$\mathrm{W}$ zbiorze liczb naturalnych rozwiąz równanie

$a+b+2014=ab.$

ZADANIE 4.

Wykaz$\cdot, \dot{\mathrm{z}}\mathrm{e}\mathrm{j}\mathrm{e}\dot{\mathrm{z}}$ eli $n\mathrm{i}6$ sa liczbami względnie pierwszymi, to $n^{2}-1$ dzieli się przez 24.

ZADANIE 5.

Rozwiąz uklad równań

( ---{\it xxyxyx}$+++${\it zyzzyz}$==$ -4--81517300.






LIGA MATEMATYCZNA

im. Zdzisława Matuskiego

$\mathrm{P}\mathrm{A}\dot{\mathrm{Z}}$ DZIERNIK 2015

SZKOLA PONADGIMNAZJALNA

ZADANIE I.

Na bokach $BC\mathrm{i}$ CD kwadratu ABCD wybrano takie punkty $E\mathrm{i}F, \dot{\mathrm{z}}\mathrm{e}$ miara kąta $EAF$ jest

równa $45^{\mathrm{o}}$ Odcinki $AE$ oraz $AF$ przecinają przekątną $BD$ kwadratu odpowiednio w punktach

$G\mathrm{i}H$. Wykaz, $\dot{\mathrm{z}}\mathrm{e}$ pole trójkąta $AGH$ jest równe polu czworokata GEFH.

ZADANIE 2.

Rozwiąz uklad równań

$\left\{\begin{array}{l}
2y+3z=2yz\\
5z+2x=4xz\\
3x+5y=8xy.
\end{array}\right.$

ZADANIE 3.

Znajd $\acute{\mathrm{z}}$ wszystkie funkcje $f$: $\mathbb{R}\rightarrow \mathbb{R}$ spefniające warunek

$f(x)f(y)-xy=f(x)+f(y)-1$

dla dowolnych liczb rzeczywistych $x, y.$

ZADANIE 4.

Liczba $A$ ma 2015 cyfr i jest podzie1na przez 9. Liczba $B$ jest sumą cyfr liczby $A.$

jest sumą cyfr liczby $B$. Wyznacz sumę cyfr liczby $C.$

Liczba C

ZADANIE 5.

Piła ma długość 60 cm i zęby będące trójkątami równoramiennymi (niekoniecznie jednako-

wymi). Wysokość $\mathrm{k}\mathrm{a}\dot{\mathrm{z}}$ dego z zębów jest równa $\displaystyle \frac{2}{3}$ jego podstawy. Po zębach pify wędruje pająk.

Jaką drogę przebędzie, pokonując wszystkie zęby tej pily?






LIGA MATEMATYCZNA

im. Zdzisława Matuskiego

$\mathrm{P}\mathrm{A}\dot{\mathrm{Z}}$ DZIERNIK 2016

SZKOLA PONADGIMNAZJALNA

ZADANIE I.

Punkt $P\mathrm{l}\mathrm{e}\dot{\mathrm{z}}\mathrm{y}$ na zewnątrz równoległoboku ABCD, przy czym $\triangleleft PAB=\triangleleft PCB$. Udowodnij,

$\dot{\mathrm{z}}\mathrm{e}\triangleleft APB=\triangleleft CPD.$

ZADANIE 2.

Liczby dodatnie $\alpha, b$ spelniaja warunek

$\displaystyle \frac{a+b}{2}=\sqrt{ab+3}.$

Wykaz, $\dot{\mathrm{z}}\mathrm{e}$ co najmniej jedna z liczb $a, b$ jest niewymierna.

ZADANIE 3.

Wyznacz wszystkie liczby naturalne $n$, dla których $n^{4}+33$ jest kwadratem liczby naturalnej.

ZADANIE 4.

Liczby całkowite $a\mathrm{i}b$ są tak dobrane, $\dot{\mathrm{z}}\mathrm{e}a^{2}+119ab+b^{2}$ jest podzielna przez ll.

$a^{3}-b^{3}\mathrm{t}\mathrm{e}\dot{\mathrm{z}}$ dzieli się przez ll.

Wykaz$\cdot, \dot{\mathrm{z}}\mathrm{e}$

ZADANIE 5.

Rozwia $\dot{\mathrm{z}}$ ukfad równań

\{{\it yx  z}222 $+++$222654$==$999{\it xzy} $+++$ -{\it x}--{\it yx}$+++$222{\it yzz}.






LIGA MATEMATYCZNA

im. Zdzisława Matuskiego

$\mathrm{P}\mathrm{A}\dot{\mathrm{Z}}$ DZIERNIK 2017

SZKOLA PONADGIMNAZJALNA

ZADANIE I.

Oblicz sumę

$[\sqrt{1}]+[\sqrt{2}]+[\sqrt{3}]+[\sqrt{4}]+\ldots+[\sqrt{n^{2}-1}],$

gdzie $n$ jest dowolna liczbą naturalną większą od l, a symbol $[x]$ oznacza największą liczbę

calkowitą nie przekraczającą liczby $x.$

ZADANIE 2.

Rozwia $\dot{\mathrm{z}}$ równanie

$20a^{2}+10b^{2}=2010$

w zbiorze liczb naturalnych.

ZADANIE 3.

Wykaz$\cdot, \dot{\mathrm{z}}\mathrm{e}$ liczba $201^{8}+3\cdot 201^{4}-4$ jest podzielna przez 4000.

ZADANIE 4.

Wykaz, $\dot{\mathrm{z}}\mathrm{e}\mathrm{k}\mathrm{a}\dot{\mathrm{z}}$ da liczba naturalna większa od 10 jest sumą trzech 1iczb: dwóch róznych 1iczb

pierwszych i jednej zlozonej.

ZADANIE 5.

Dany jest trapez ABCD o podstawach a $\mathrm{i} b$. Odcinek $EF$ o dlugości $x$ jest równolegly do

podstaw trapezu i podzielil go na dwa trapezy o równych polach. Wyznacz $x.$
\begin{center}
\includegraphics[width=72.384mm,height=34.284mm]{./LigaMatematycznaMatuskiego_Liceum_Zestaw1_2017_2018_page0_images/image001.eps}
\end{center}
{\it D  b  c}

{\it E  x  F}

{\it A  a  B}






LIGA MATEMATYCZNA

im. Zdzisława Matuskiego

$\mathrm{P}\mathrm{A}\dot{\mathrm{Z}}$ DZIERNIK 2018

SZKOLA PONADPODSTAWOWA

ZADANIE I.

Dany jest odcinek $AB$ o dlugości 4. Punkty $A\mathrm{i}B$ sa środkami okręgów o promieniu 4. Znajd $\acute{\mathrm{z}}$

promień okręgu stycznego do prostej $AB$, stycznego zewnętrznie do okręgu o środku $A$ oraz

stycznego wewnętrznie do okręgu o środku $B.$

ZADANIE 2.

Czy istnieje taka liczba pierwsza $p, \dot{\mathrm{z}}\mathrm{e}p+16$ jest kwadratem liczby pierwszej?

uzasadnij.

Odpowiedz'

ZADANIE 3.

Funkcja rzeczywista $f$: $\mathbb{R} \rightarrow \mathbb{R}$ spelnia równanie $f(x)+xf(1-x) = x$ dla $\mathrm{k}\mathrm{a}\dot{\mathrm{z}}$ dej liczby

rzeczywistej $x$. Wyznacz $f(-2).$

ZADANIE 4.

Wysokości pewnego trójkata mają dlugości 156, 65, 60. Ob1icz po1e tego trójkąta.

ZADANIE 5.

Wyznacz liczbę czwórek $(a,b,c,d)$ liczb calkowitych dodatnich spelniających warunek

$ab+bc+cd+da=2018+a+b+c+d.$






LIGA MATEMATYCZNA

im. Zdzisława Matuskiego

$\mathrm{P}\mathrm{A}\dot{\mathrm{Z}}$ DZIERNIK 2019

SZKOLA PONADPODSTAWOWA

ZADANIE I.

Która z liczb jest większa $7^{31}$ czy $19^{21}$?

ZADANIE 2.

Niech $n$ będzie dowolną liczbą calkowitą dodatnia. Wewnatrz prostokąta o bokach o dlugości l

$\mathrm{i}2$ znajduje się $8n^{2}+1$ punktów. Wykaz, $\dot{\mathrm{z}}\mathrm{e}$ istnieje kolo o promieniu $\displaystyle \frac{1}{n}$ zawierajace co najmniej

trzy spośród danych punktów.

ZADANIE 3.

Przez punkt $W$ lezący wewnątrz trójkąta $ABC$ poprowadzono trzy proste równolegfe do boków

trójkąta. Proste te podzielify trójkąt na sześć części, z których trzy są trójkatami o polach l,

4 $\mathrm{i}9$. Wyznacz pole trójkąta $ABC.$

ZADANIE 4.

Ile jest liczb trzycyfrowych $\overline{xyz}$ podzielnych przez 3 i takich, $\dot{\mathrm{z}}\mathrm{e}(\overline{xy})^{2}+(\overline{yz})^{2}=(\overline{yx})^{2}+(\overline{zy})^{2}$?

Symbol $\overline{xyz}$ oznacza liczbę trzycyfrową zapisaną w dziesiętnym systemie pozycyjnym.

ZADANIE 5.

$\mathrm{W}$ zbiorze liczb rzeczywistych rozwiąz uklad równań

$\left\{\begin{array}{l}
x-y^{2}+2y=2\\
y-z^{2}+2z=2\\
z-x^{2}+2x=2.
\end{array}\right.$






LIGA MATEMATYCZNA

im. Zdzisława Matuskiego

$\mathrm{P}\mathrm{A}\dot{\mathrm{Z}}$ DZIERNIK 2020

SZKOLA PONADPODSTAWOWA

ZADANIE I.

Czy istnieją liczby naturalne $a, b, c, d$ takie, $\dot{\mathrm{z}}\mathrm{e}a+b+c+d=478$ {\it oraz abcd}$=132706$?

ZADANIE 2.

$\mathrm{W}$ kwadracie o boku o dfugości l danych jest $2n+1$ punktów, z których $\dot{\mathrm{z}}$ adne trzy nie są

wspólliniowe. Udowodnij, $\dot{\mathrm{z}}\mathrm{e}$ trzy spośród nich są wierzchofkami trójkata o polu nie większym

$\displaystyle \mathrm{n}\mathrm{i}\dot{\mathrm{z}}\frac{1}{2n}.$

ZADANIE 3.

Dwa trójkąty równoboczne mają boki równolegle i wspólne ortocentrum. Pole jednego z nich

jest dwa razy większe $\mathrm{n}\mathrm{i}\dot{\mathrm{z}}$ pole drugiego, a bok mniejszego trójkąta ma dlugość l. Oblicz

odleglośč między równoleglymi bokami.

ZADANIE 4.

$\mathrm{W}$ zbiorze liczb rzeczywistych rozwiąz układ równań, gdy $n>3,$

$\left\{\begin{array}{l}
x_{1}+x_{2}=x_{3}\\
x_{2}+x_{3}=x_{4}\\
x_{3}+x_{4}=x_{5}\\
x_{n-2}+x_{n-1}=x_{n}\\
x_{n-1}+x_{n}=x_{1}\\
x_{n}+x_{1}=x_{2}.
\end{array}\right.$

ZADANIE 5.

Niech $f:\mathbb{R}\rightarrow \mathbb{R}$ będzie funkcją spelniającą warunki

$\bullet f(0)=2020$;

$\bullet f(x+2)=\displaystyle \frac{f(x)}{5f(x)-1}.$

Oblicz $f$ (2020).






LIGA MATEMATYCZNA

im. Zdzislawa Matuskiego

$\mathrm{P}\mathrm{A}\dot{\mathrm{Z}}$ DZIERNIK 2021

SZKOLA PONADPODSTAWOWA

ZADANIE I.

Na stole w koszyku $\mathrm{l}\mathrm{e}\dot{\mathrm{z}}\mathrm{y}$ sto kapsli. Adam i Bartek zabieraja na zmianę po kilka kapsli. Wjednym

ruchu $\mathrm{m}\mathrm{o}\dot{\mathrm{z}}$ na zabraćjeden, dwa lub trzy kapsle. Wygrywa ten, kto $\mathrm{w}\mathrm{e}\acute{\mathrm{z}}\mathrm{m}\mathrm{i}\mathrm{e}$ ostatni kapsel. Adam

rozpocząl grę biorąc jeden kapsel. Ile kapsli powinien teraz wziąč Bartek, aby być pewnym

wygranej?

ZADANIE 2.

Pole prostokąta jest trzy razy większe od jego obwodu, a dlugości boków są liczbami natural-

nymi. Wyznacz dlugości boków prostokąta.

ZADANIE 3.

$\mathrm{W}$ trójkąt równoboczny $ABC$ wpisano okrąg. Dlugość luku laczącego dwa punkty styczności

tego okręgu z bokami trójkąta jest równa l. Oblicz obwód trójkąta.

ZADANIE 4.

Wykaz, $\dot{\mathrm{z}}\mathrm{e}\mathrm{j}\mathrm{e}\dot{\mathrm{z}}$ eli wysokości $h_{1}, h_{2}, h_{3}$ trójkąta spelniają warunek

$(h_{1}h_{3})^{2}+(h_{2}h_{3})^{2}=(h_{1}h_{2})^{2},$

to trójkąt jest prostokątny.

ZADANIE 5.

$\mathrm{W}$ zbiorze liczb rzeczywistych rozwiąz uklad równań

$\left\{\begin{array}{l}
2x+2y+z=6\\
8xy-z^{2}=36.
\end{array}\right.$






LIGA MATEMATYCZNA

im. Zdzislawa Matuskiego

$\mathrm{P}\mathrm{A}\dot{\mathrm{Z}}$ DZIERNIK 2022

SZKOLA PONADPODSTAWOWA

ZADANIE I.

Czy istnieją liczby całkowite $x, y, z$ takie, $\dot{\mathrm{z}}\mathrm{e}(3x-5y)(7y-3z)(3z-x)=20222023$?

ZADANIE 2.

Pola szachownicy $9\times 9$ pokolorowano w tradycyjny sposób, ale jej narozne pola są biafe. Ruch

polega na wybraniu dwóch sąsiednich pól i przemalowaniu ich na przeciwny kolor (to znaczy:

$\mathrm{j}\mathrm{e}\dot{\mathrm{z}}$ eli wybrane polejest biale, to zmieniamy jego kolor na czarny, a gdy polejest czarne, to zmie-

niamy jego kolor na biały). Czy $\mathrm{m}\mathrm{o}\dot{\mathrm{z}}$ na dobierać ruchy tak, aby w pewnym momencie wszystkie

pola byly czarne?

ZADANIE 3.

Wierzcholki $B, C$ trójkata $ABC$ polączono odcinkami z przeciwleglymi bokami otrzymując

male trójkąty o polach 3, 3 $\mathrm{i}1$ (jak na rysunku). Oblicz pole trójkąta $ABC.$
\begin{center}
\includegraphics[width=64.764mm,height=56.592mm]{./LigaMatematycznaMatuskiego_Liceum_Zestaw1_2022_2023_page0_images/image001.eps}
\end{center}
3 1

3

{\it C}

ZADANIE 4.

Przedstaw liczbę 2023 jako róznicę kwadratów dwóch 1iczb natura1nych.

ZADANIE 5.

W zbiorze liczb rzeczywistych rozwiąz uklad równań

({\it x}((({\it xxx}1111({\it x}$+++$1 {\it xxx}$+$222{\it x})$++$(2{\it xx}$+${\it x}233$+${\it x})$+$(3{\it xx}$+${\it x}334$++${\it x}) {\it x}4{\it xx}4)44$=$)) $==$11. 11






LIGA MATEMATYCZNA

LISTOPAD 2009

SZKOLA PONADGIMNAZJALNA

ZADANIE I.

Znajd $\acute{\mathrm{z}}$ największą liczbę naturalną $n$ taką, $\dot{\mathrm{z}}\mathrm{e}$ 1000! $(1000!=1\cdot 2\cdot 3\cdot\ldots\cdot 1000)$ jest podzielne

przez $2^{n}$

ZADANIE 2.

Udowodnij, $\dot{\mathrm{z}}\mathrm{e}\mathrm{j}\mathrm{e}\dot{\mathrm{z}}$ eli ramiona trapezu zawierają się w dwóch prostych prostopadlych, to suma

kwadratów dlugości podstaw równa się sumie kwadratów dlugości przekątnych.

ZADANIE 3.

Wykaz$\cdot, \dot{\mathrm{z}}\mathrm{e}$

$\displaystyle \frac{1}{2^{2}}+\frac{1}{3^{2}}+\frac{1}{4^{2}}+\ldots+\frac{1}{100^{2}}<\frac{99}{100}.$

ZADANIE 4.

Dla liczby naturalnej $n$ przez $p(n)$ oznaczmy iloczyn cyfr liczby $n$, np. $p(23) = 2$

$3 = 6,$

$p(100)=1\cdot 0\cdot 0=0$. Oblicz

$p(1)+p(2)+\ldots+p(100).$

ZADANIE 5.

W zbiorze liczb naturalnych trzycyfrowych znajdz' liczbę, której stosunek do sumy jej cyfr jest

najmniejszy.






LIGA MATEMATYCZNA

LISTOPAD 2010

SZKOLA PONADGIMNAZJALNA

ZADANIE I.

Rozwiąz równanie $x^{2}+y^{2}=x+y+2$ w zbiorze liczb całkowitych.

ZADANIE 2.

Wypisujemy kolejno liczby wedlug następującej reguły: dowolna liczba, oprócz pierwszej, jest

ostatnią cyfrą zwiększonego o l kwadratu poprzedniej liczby. Jaka liczba znajduje się na pierw-

szym miejscu, $\mathrm{j}\mathrm{e}\dot{\mathrm{z}}$ eli na 2010 pozycji znajduje się zero?

ZADANIE 3.

Na stole $\mathrm{l}\mathrm{e}\dot{\mathrm{z}}\mathrm{y}$ 2009 $\dot{\mathrm{z}}$ etonów czerwonych oraz 2009 $\dot{\mathrm{z}}$ etonów zielonych. Dwaj gracze na przemian

wykonują ruchy. Ruch polega na zdjęciu ze stołu dwóch $\dot{\mathrm{z}}$ etonów, przy czymjeśli były to $\dot{\mathrm{z}}$ etony

tego samego koloru, gracz kładzie na stóf $\dot{\mathrm{z}}$ eton czerwony, ajeśli $\dot{\mathrm{z}}$ etony byly rózne, kladzie $\dot{\mathrm{z}}$ eton

zielony. Zatem po $\mathrm{k}\mathrm{a}\dot{\mathrm{z}}$ dym ruchu liczba $\dot{\mathrm{z}}$ etonów na stole zmniejsza się o l. Gracz zaczynający

wygrywa, jeśli ostatni $\dot{\mathrm{z}}$ eton, jaki pozostanie na stole, będzie koloru czerwonego. Jaki powinien

wykonać pierwszy ruch, by wygrać?

ZADANIE 4.

Udowodnij, $\dot{\mathrm{z}}\mathrm{e}$ liczba $3^{2010}-5\cdot 15^{1005}+5^{2012}$ jest złozona.

ZADANIE 5.

$\mathrm{W}$ czworokącie wypuklym dwa przeciwlegle boki podzielono na trzy równe częšci. Wykaz,

$\dot{\mathrm{z}}\mathrm{e}$ pole czworokąta jest równe $3S, \mathrm{j}\mathrm{e}\dot{\mathrm{z}}$ eli pole zamalowanej części jest równe $S.$






LIGA MATEMATYCZNA

LISTOPAD 2011

SZKOLA PONADGIMNAZJALNA

ZADANIE I.

Przekątne trapezu ABCD, gdzie AB i CD są równolegle, przecinają się w punkcie E.

trójkąta ABE jest równe P, a pole trójkąta DEC jest równe S. Oblicz pole trapezu.

Pole

ZADANIE 2.

Oblicz $2010^{2}+2010^{2}\cdot 2011^{2}+2011^{2}-2010^{2}$

ZADANIE 3.

Znajd $\acute{\mathrm{z}}$ wszystkie róznowartościowe funkcje $f$: $\mathbb{R}\rightarrow \mathbb{R}$ spełniające równość

$f(f(x)+y)=f(x+y)+1$

dla dowolnych liczb rzeczywistych $x, y.$

ZADANIE 4.

Wykaz, $\dot{\mathrm{z}}\mathrm{e}$ liczba naturalna i jej piąta potęga mają tę samą cyfrę jedności.

ZADANIE 5.

$\mathrm{W}$ klasie jest 31 uczniów, wpisanych do dziennika pod numerami od 1 do 31. Przed 6 grudnia

przygotowali losy z numerami od l do 31, by usta1ić, kto komu będzie kupować prezent miko-

lajkowy. Udowodnij, $\dot{\mathrm{z}}\mathrm{e}$ iloczyn liczb będących sumami numeru ucznia w dzienniku i numeru

z karteczki przez niego wylosowanej jest liczbą parzystą.






LIGA MATEMATYCZNA

im. Zdzisława Matuskiego

LISTOPAD 2012

SZKOLA PONADGIMNAZJALNA

ZADANIE I.

Wykaz$\cdot, \dot{\mathrm{z}}\mathrm{e}$ trójkąt prostokątny o bokach będących liczbami calkowitymi ma obwód, który jest

liczbą parzystą.

ZADANIE 2.

Wyznacz wszystkie funkcje $f:\mathbb{R}\rightarrow \mathbb{R}$ spefniajace warunek

$f(x+y)-f(x-y)=f(x)f(y)$

dla $\mathrm{k}\mathrm{a}\dot{\mathrm{z}}$ dych liczb rzeczywistych $x, y.$

ZADANIE 3.

$\mathrm{W}$ prostokącie o bokach 10 $\mathrm{i}20$ wybrano 401 punktów. Wykaz, $\dot{\mathrm{z}}\mathrm{e}$ istnieje kwadrat o boku l,

do którego nalezą co najmniej trzy spośród danych punktów.

ZADANIE 4.

Kwadrat o polu 144 $\mathrm{c}\mathrm{m}^{2}$ ma wspólną przekątną z prostokątem. Część wspólna kwadratu i pro-

stokata ma pole 96 $\mathrm{c}\mathrm{m}^{2}$ Oblicz pole prostokąta.

ZADANIE 5.

Suma dzielników pewnej liczby naturalnej $n$, bez liczby l i bez dzielnika będącego liczba $n$, jest

równa 41. Znajd $\acute{\mathrm{z}}$ liczbę $n$ wiedząc, $\dot{\mathrm{z}}\mathrm{e}$ rozklada się na trzy czynniki pierwsze, ajednym z nich

jest liczba 5.






LIGA MATEMATYCZNA

im. Zdzisława Matuskiego

LISTOPAD 2013

SZKOLA PONADGIMNAZJALNA

ZADANIE I.

Dany jest kwadrat ABCD o boku dlugości a. Punkt K jest środkiem boku AB, punkt L

jest środkiem boku CD. Prosta AL przecina odcinek DK w punkcie M oraz przekątną BD

w punkcie S. Oblicz pole trójkąta DMS.

ZADANIE 2.

Wykaz, $\dot{\mathrm{z}}\mathrm{e}$ liczby 5050505 nie $\mathrm{m}\mathrm{o}\dot{\mathrm{z}}$ na przedstawić w postaci sumy dwóch liczb pierwszych.

ZADANIE 3.

Rozwiąz uklad równań

$\left\{\begin{array}{l}
2x^{2}+y^{2}=2\\
xy+2x=-3.
\end{array}\right.$

ZADANIE 4.

Liczby $a_{1}, a_{2}, a_{3}, \ldots$, a2013 są róznymi e1ementami zbioru \{1, 2, 3, $\ldots$, 2013\}. Czy liczba

$(a_{1}-1)(a_{2}-2)(a_{3}-3)\ldots$ (a2013 - 2013)

jest parzysta, czy nieparzysta?

ZADANIE 5.

Funkcja $f$: $\mathbb{R}\rightarrow \mathbb{R}$ spelnia następujące warunki:

$\bullet f(x+y)=f(x)+f(y)$ ;

$\bullet f(1)=1.$

Oblicz $f(\displaystyle \frac{1}{4}).$






LIGA MATEMATYCZNA

im. Zdzisława Matuskiego

LISTOPAD 2014

SZKOLA PONADGIMNAZJALNA

ZADANIE I.

$\mathrm{Z}$ punktu $A\mathrm{l}\mathrm{e}\dot{\mathrm{z}}$ ącego na zewnątrz okręgu o środku $O$ i promieniu $r$ poprowadzono sieczna, która

przecina dany okrąg w punktach $B\mathrm{i}C$ w taki sposób, $\dot{\mathrm{z}}\mathrm{e}$ okrąg zbudowany na odcinku $BC$

jako na średnicy okręgu jest styczny do prostej $AO$ w punkcie $D$. Wyznacz dlugość odcinka

$AD$, gdy $|AO|=a.$

ZADANIE 2.

Na okręgu umieszczono $101$liczb naturalnych. Wykaz, $\dot{\mathrm{z}}\mathrm{e}$ znajdziemy dwie sąsiadujące ze sobą

liczby, których suma jest liczbą parzystą.

ZADANIE 3.

Dwie niezerowe rózne liczby rzeczywiste $a, b$ spelniają warunek $\displaystyle \frac{a}{b}+a=\frac{b}{a}+b$. Oblicz $\displaystyle \frac{1}{a}+\frac{1}{b}.$

ZADANIE 4.

Wyznacz wszystkie liczby pierwsze $p, q$ takie, $\dot{\mathrm{z}}\mathrm{e}2pq=3(p+q).$

ZADANIE 5.

Rozwiąz ukfad równań

$\left\{\begin{array}{l}
3x+5y=8xy\\
2y+3z=2yz\\
5z+2x=4xz.
\end{array}\right.$






LIGA MATEMATYCZNA

im. Zdzisława Matuskiego

LISTOPAD 2015

SZKOLA PONADGIMNAZJALNA

ZADANIE I.

Czworokąt wypukły ABCD jest wpisany w okrąg $0$. Dwusieczne kątów $\triangleleft BAD, \triangleleft CBA,$

$\triangleleft DCB, \triangleleft ADC$ przecinają okrąg $0$ odpowiednio w punktach $M, N, P\mathrm{i}Q$. Wykaz, $\dot{\mathrm{z}}\mathrm{e}$ punkty

$M, N, P, Q$ są wierzchofkami prostokąta.

ZADANIE 2.

$\mathrm{W}$ zbiorze liczb rzeczywistych rozwiąz uklad równań

$\left\{\begin{array}{l}
x^{2}y=150\\
x^{3}y^{2}=4500.
\end{array}\right.$

ZADANIE 3.

Wyznacz najmniejszą liczbę naturalną $n$ taką, $\dot{\mathrm{z}}\mathrm{e}$ liczby $n+3, n-100$ sq kwadratami liczb

naturalnych.

ZADANIE 4.

Funkcja liniowa $f$ określona dla wszystkich liczb rzeczywistych spelnia warunek

$f(2016)+f(1)=2.$

Oblicz wartość wyrazenia $f(0)+f(1)+f(2)+\ldots+f(2016)+f(2017).$

ZADANIE 5.

Zbiór $A$ zawiera wszystkie liczby siedmiocyfrowe o róznych cyfrach nalezących do zbioru

\{1, 2, 3, 4, 5, 6, 7\}.

Czy w zbiorze $A$ istnieje 77 takich 1iczb, $\dot{\mathrm{z}}\mathrm{e}$ suma 33 z nich jest równa sumie 44 pozostałych?






LIGA MATEMATYCZNA

im. Zdzisława Matuskiego

LISTOPAD 2016

SZKOLA PONADGIMNAZJALNA

ZADANIE I.

Wewnatrz trójkąta równobocznego $ABC$ znajduje się punkt $O$. Prosta przechodząca przez

punkt $O$ i środek cięzkości $G$ tego trójkąta (punkt przecięcia się środkowych) przecina jego

boki lub ich przedluzenia odpowiednio w punktach $D, E\mathrm{i}F$. Wykaz, $\dot{\mathrm{z}}\mathrm{e}$

-$||${\it DDOG}$|| +$ -$||${\it EEOG}$|| +$ -$||${\it FFOG}$|| =$3.

ZADANIE 2.

Rozwiąz równanie

$x(x+1)+(x+1)(x+2)+(x+2)(x+3)+\ldots+(x+14)(x+15)=2016x+2017$

w zbiorze liczb calkowitych.

ZADANIE 3.

Czy wierzchofki ośmiokąta foremnego $\mathrm{m}\mathrm{o}\dot{\mathrm{z}}$ na tak ponumerować liczbami 1, 2, 3, 4, 5, 6, 7, 8,

aby dla dowolnych trzech kolejnych wierzcholków suma ich numerów byla większa od 13?

ZADANIE 4.

$\mathrm{W}$ liczbie naturalnej, która byla co najmniej dwucyfrowa, wykreślono ostatnią cyfrę. Otrzy-

mana liczba jest $n$ razy mniejsza od poprzedniej. Wyznacz najmniejszą i największa $\mathrm{m}\mathrm{o}\dot{\mathrm{z}}$ liwą

wartość liczby $n.$

ZADANIE 5.

Rozwiąz ukfad równań

({\it tuyxz}22222 $+++++$22222$=====$22222{\it tzyxu}$+++++${\it utzyx}.






LIGA MATEMATYCZNA

im. Zdzisława Matuskiego

LISTOPAD 2017

SZKOLA PONADGIMNAZJALNA

ZADANIE I.

Trzy okręgi o promieniu r, okrąg o promieniu l i prosta są ułozone tak, jak na rysunku. Oblicz

dlugość promienia r.
\begin{center}
\includegraphics[width=116.784mm,height=42.972mm]{./LigaMatematycznaMatuskiego_Liceum_Zestaw2_2017_2018_page0_images/image001.eps}
\end{center}
{\it r}

{\it r  r}

1

ZADANIE 2.

Wykaz$\cdot, \dot{\mathrm{z}}\mathrm{e}$ liczba

$1^{3}+2^{3}+3^{3}+\ldots+2015^{3}+2016^{3}$

jest podzielna przez 2017.

ZADANIE 3.

Znajd $\acute{\mathrm{z}}$ wszystkie nieujemne liczby rzeczywiste $x$ spelniające równanie

$[x]=\sqrt{x\{x\}},$

gdzie $\{x\}=x-[x]$ oraz $[x]$ oznacza największą liczbę cafkowita nie przekraczajacą liczby $x.$

ZADANIE 4.

Danych jest 26 ko1ejnych 1iczb natura1nych. Okaza1o się, $\dot{\mathrm{z}}\mathrm{e}$ suma pewnych dziesięciu z nich

jest liczbą pierwszą. Wykaz, $\dot{\mathrm{z}}\mathrm{e}$ suma pozostalych 161iczb jest 1iczbą z1ozoną.

ZADANIE 5.

Wykaz, $\dot{\mathrm{z}}\mathrm{e}$

$(\displaystyle \frac{3}{2})^{2016}+(\frac{3}{2})^{2017}>(\frac{3}{2})^{2018}$






LIGA MATEMATYCZNA

im. Zdzisława Matuskiego

LISTOPAD 2018

SZKOLA PONADPODSTAWOWA

ZADANIE I.

Wykaz$\cdot, \dot{\mathrm{z}}\mathrm{e}$ istnieje nieskończenie wiele liczb naturalnych, dla których iloczyn cyfr oraz suma

cyfr są liczbami pierwszymi.

ZADANIE 2.

Trójkąt $ABC$ podzielono dwiema prostymi, przechodzqcymi przez punkty $A\mathrm{i}B$ odpowiednio,

na cztery części. Pola trzech z nich są równe 3, 4, 6. Ob1icz po1e czwartej części.

ZADANIE 3.

Danych jest 301iczb rzeczywistych, których suma jest równa 300. Wykaz, $\dot{\mathrm{z}}\mathrm{e}$ wśród tych liczb

istnieje takich 51iczb, których suma jest równa co najmniej 50.

ZADANIE 4.

Funkcja $f$: $\mathbb{R}\rightarrow \mathbb{R}$ spelnia warunek

$2f(x)+3f(\displaystyle \frac{2010}{x})=5x$

dla $\mathrm{k}\mathrm{a}\dot{\mathrm{z}}$ dej liczby rzeczywistej dodatniej $x$. Wyznacz $f(6).$

ZADANIE 5.

Znajd $\acute{\mathrm{z}}$ wszystkie pary liczb calkowitych dodatnich $(x,y)$, które spefniają równanie

$4^{x}+260=y^{2}$

1






LIGA MATEMATYCZNA

im. Zdzisława Matuskiego

LISTOPAD 2019

SZKOLA PONADPODSTAWOWA

ZADANIE I.

Znajd $\acute{\mathrm{z}}$ takie cyfry $x, y$, aby $(\overline{xy})^{2}+\overline{xy}=(\overline{yx})^{2}+\overline{yx}.$

ZADANIE 2.

Dany jest 2020-kąt foremny $A_{1}A_{2}A_{3}\ldots A_{2019}A_{2020}$. Punkt $P$ jest dowolnym punktem okręgu

o promieniu $R$ opisanego na wielokacie $A_{1}A_{2}A_{3}\ldots A_{2019}A_{2020}$. Oblicz

$|PA_{1}|^{2}+|PA_{2}|^{2}+\ldots+|PA_{2020}|^{2}$

ZADANIE 3.

Czy z odcinków o dlugościach $2018^{2018}, 2019^{2019}, 2020^{2020}\mathrm{m}\mathrm{o}\dot{\mathrm{z}}$ na zbudować trójkąt?

ZADANIE 4.

Zbiór $A$ sklada się z 2019 róznych 1iczb natura1nych. Wykaz, $\dot{\mathrm{z}}\mathrm{e}$ ze zbioru $A\mathrm{m}\mathrm{o}\dot{\mathrm{z}}$ na wybrać trzy

takie liczby $a, b, c, \dot{\mathrm{z}}\mathrm{e}$ iloczyn $a(b-c)$ jest podzielny przez 2019.

ZADANIE 5.

$\mathrm{W}$ zbiorze liczb rzeczywistych rozwia $\dot{\mathrm{z}}$ uklad równań

$\left\{\begin{array}{l}
xy+x+y=8\\
yz+y+z=8\\
xz+x+z=8.
\end{array}\right.$






LIGA MATEMATYCZNA

im. Zdzisława Matuskiego

LISTOPAD 2020

SZKOLA PONADPODSTAWOWA

ZADANIE I.

Dany jest trójkąt równoboczny $T$ o boku o dlugości $a$ i środku cięzkości $S$. Zakreślono okrąg

o środku $S$ i promieniu $\displaystyle \frac{a}{3}$ ograniczający kofo $K$. Oblicz pole figury $K-T.$

ZADANIE 2.

Na tablicy napisano kilka róznych liczb cafkowitych dodatnich (co najmniej cztery). Okazalo

się, $\dot{\mathrm{z}}\mathrm{e}$ suma $\mathrm{k}\mathrm{a}\dot{\mathrm{z}}$ dych trzech spośród nich jest liczbą pierwsza. Ile liczb napisano na tablicy?

ZADANIE 3.

Dany jest następujący ciąg liczb: pierwsza liczba to 2020, $\mathrm{k}\mathrm{a}\dot{\mathrm{z}}$ dą następną oblicza się wedlug

wzoru $\displaystyle \frac{1-a}{1+a}$, gdzie $a$ oznacza poprzednią liczbę. Znajdz' dwa tysiące dwudziesty pierwszy wyraz

tego ciągu.

ZADANIE 4.

$\mathrm{W}$ zbiorze liczb rzeczywistych rozwiąz układ równań

$\left\{\begin{array}{l}
x^{2}+x(y-4)=-2\\
y^{2}+y(x-4)=-2.
\end{array}\right.$

ZADANIE 5.

Wyznacz wszystkie funkcje $f:\mathbb{R}\rightarrow \mathbb{R}$ spefniające warunek

$f(x)\cdot f(y)=f(xy)+x^{2}+y^{2}$

dla dowolnych liczb rzeczywistych $x, y.$






LIGA MATEMATYCZNA

im. Zdzislawa Matuskiego

LISTOPAD 2021

SZKOLA PONADPODSTAWOWA

ZADANIE I.

$\mathrm{W}\mathrm{k}\mathrm{a}\dot{\mathrm{z}}$ dym wierzchołku dziesięciokąta napisano jedna z liczb: 1, 2, 3, 4, 5. $K\mathrm{a}\dot{\mathrm{z}}\mathrm{d}\mathrm{y}$ bok dziesię-

ciokąta ma dlugość równą sumie liczb napisanych na końcach tego boku. Uzasadnij, $\dot{\mathrm{z}}\mathrm{e}$ przy-

najmniej dwa boki mają równe dfugości.

ZADANIE 2.

Wyznacz dlugości boków trójkąta prostokątnego, $\mathrm{j}\mathrm{e}\dot{\mathrm{z}}$ eli są one liczbami naturalnymi, a liczby

oznaczajace pole i obwód spełniają warunek: pole jest równe podwojonemu obwodowi.

ZADANIE 3.

Punkty $M\mathrm{i}N$ są środkami boków $BC\mathrm{i}$ CD równolegfoboku ABCD. Niech $K\mathrm{i}L$ będą punk-

tami przecięcia przekątnej $BD$ odpowiednio przez proste AM $\mathrm{i}$ AN. Wykaz, $\dot{\mathrm{z}}\mathrm{e}$ punkty $K\mathrm{i}L$

dzielą przekątną $BD$ na trzy równe części. Jaką częścią pola równolegloboku ABCD jest pole

pięciokąta LKMCN?

ZADANIE 4.

Czy istnieją takie liczby calkowite $a, b, c, d, e, f, \dot{\mathrm{z}}\mathrm{e}a-b, b-c, c-d, d-e, e-f, f-a$

wypisane w pewnym porządku są kolejnymi liczbami calkowitymi? Odpowied $\acute{\mathrm{z}}$ uzasadnij.

ZADANIE 5.

$\mathrm{W}$ zbiorze liczb rzeczywistych rozwiąz uklad równań

$\left\{\begin{array}{l}
x^{2}+x+y=y^{3}\\
y^{2}+y+x=x^{3}
\end{array}\right.$






LIGA MATEMATYCZNA

im. Zdzislawa Matuskiego

LISTOPAD 2022

SZKOLA PONADPODSTAWOWA

ZADANIE I.

Uzasadnij, $\dot{\mathrm{z}}\mathrm{e}$ czworokąt wypukły ABCD, w którym obwody trójkatów $ABC, BCD, CDA$

$\mathrm{i}DAB$ są równe, jest prostokatem.

ZADANIE 2.

Znajd $\acute{\mathrm{z}}$ wszystkie liczby pierwsze $p, q$ takie, $\dot{\mathrm{z}}\mathrm{e}p^{2}-6q^{2}=1.$

ZADANIE 3.

Na tablicy wypisano liczby 1, 2, 3, $\ldots$, 10. Ruch polega na wybraniu trzech liczb $a, b, c$ i zastą-

pieniu ich liczbami $2a+b, 2b+c, 2c+a$. Czy po wykonaniu pewnej liczby takich operacji na

tablicy otrzymamy dziesięć równych liczb?

ZADANIE 4.

Dodatnia liczba calkowita $a$ ma dwa dzielniki naturalne, liczba $a+1$ ma trzy dzielniki naturalne.

Ile dzielników naturalnych ma liczba $a+2$?

ZADANIE 5.

$\mathrm{W}$ zbiorze liczb rzeczywistych rozwia $\dot{\mathrm{z}}$ układ równań

(222{\it xyz}222 $+++$ --{\it yx-z}111222 $==$ {\it xyz}222 $+++$222.






LIGA MATEMATYCZNA

GRUD Z$\mathrm{I}\mathrm{E}\acute{\mathrm{N}}$ 2009

SZKOLA PONADGIMNAZJALNA

ZADANIE I.

Wykaz, $\dot{\mathrm{z}}\mathrm{e}$ liczba $3^{32}-1$ jest podzielna przez 8.

ZADANIE 2.

Dany jest uklad równań

$\left\{\begin{array}{l}
x+y+z=28\\
2x-y=32.
\end{array}\right.$

Okrešl, która z liczb jest większa $x$ czy $y, \mathrm{j}\mathrm{e}\dot{\mathrm{z}}$ eli $x>0, y>0\mathrm{i}z>0.$

ZADANIE 3.

Znajd $\acute{\mathrm{z}}$ sumę ufamków okresowych 0, $(ABC)+0, (BCA)+0$, ({\it CAB}).

ZADANIE 4.

Podstawą trójkąta równobocznego jest średnica koła o promieniu r. Oblicz stosunek pola części

trójkąta lezącej na zewnątrz koła do pola części trójkąta lezącej wewnątrz kola.

ZADANIE 5.

$\mathrm{W}$ kwadracie o boku l $\mathrm{m}$ obrano 51 punktów. Uzasadnij, $\dot{\mathrm{z}}\mathrm{e}$ przy dowolnym wyborze tych

punktów znajdą się trzy, które mieszczą się w kwadracie o boku 20 cm.






LIGA MATEMATYCZNA

GRUD Z$\mathrm{I}\mathrm{E}\acute{\mathrm{N}}$ 2010

SZKOLA PONADGIMNAZJALNA

ZADANIE I.

Częšcią calkowitą liczby rzeczywistej $x$ nazywamy największą liczbę całkowitą nie większą $\mathrm{n}\mathrm{i}\dot{\mathrm{z}}$

$x$ i oznaczamy $[x]$. Rozwiąz układ równań

$\left\{\begin{array}{l}
[x]+y-2[z]=1\\
x+y-[z]=2\\
3[x]-4[y]+z=3.
\end{array}\right.$

ZADANIE 2.

Punkt $S$ lezy wewnątrz sześciokąta foremnego ABCDEF. Udowodnij, $\dot{\mathrm{z}}\mathrm{e}$ suma pól trójkątów

$ABS, CDS, EFS$ jest równa połowie pola szešciokąta ABCDEF.

ZADANIE 3.

Jan napisal na tablicy dwie liczby naturalne. Potem starł je i w ich miejsce wpisal iloczyn

zmniejszony o l oraz ich sumę. Nie zadowolifo go tojednak i powtórzył tę czynność. Znowu starl

wszystko i zapisal sumę otrzymanych liczb: 1309. Ob1icz sumę 1iczb zapisanych na poczqtku.

ZADANIE 4.

Wykaz, $\dot{\mathrm{z}}\mathrm{e}$ z grupy 2010 osób $\mathrm{m}\mathrm{o}\dot{\mathrm{z}}$ na wybrać 45 osób mających tak samo na imię 1ub 45 osób,

z których $\mathrm{k}\mathrm{a}\dot{\mathrm{z}}$ da nosi inne imię.

ZADANIE 5.

Wykaz$\cdot, \dot{\mathrm{z}}\mathrm{e}$ liczba $3^{2012}+15^{1006}+5^{2012}$ jest zlozona.






LIGA MATEMATYCZNA

GRUD Z$\mathrm{I}\mathrm{E}\acute{\mathrm{N}}$ 2011

SZKOLA PONADGIMNAZJALNA

ZADANIE I.

Łąka w ksztalcie kwadratu ma powierzchnię l hektara. Swoje norki wykopalo tam 2011 zajęcy

(kazdy ma jedną norkę). Po pewnym czasie pojawił się jeszcze jeden zając- samotnik, który

nie chce mieć sąsiada blizej $\mathrm{n}\mathrm{i}\dot{\mathrm{z}}1$ metr od swojej norki. Udowodnij, $\dot{\mathrm{z}}\mathrm{e}\mathrm{m}\mathrm{o}\dot{\mathrm{z}}\mathrm{e}$ tam zamieszkać.

ZADANIE 2.

Znajd $\acute{\mathrm{z}}$ wszystkie funkcje $f$: $\mathbb{R}\backslash \{0\}\rightarrow \mathbb{R}$ spelniające warunek

$f(x)+3f(\displaystyle \frac{1}{x})=\frac{2}{x}$

dla $\mathrm{k}\mathrm{a}\dot{\mathrm{z}}$ dej liczby rzeczywistej $x$ róznej od zera.

ZADANIE 3.

Kwadrat podzielono na mniejszy kwadrat i trzy prostokąty, jak na rysunku. Czy trzy spośród

tych czterech części mogą mieć taki sam obwód?

ZADANIE 4.

Wykaz, $\dot{\mathrm{z}}\mathrm{e}$ dla liczby całkowitej $k$ liczba $k^{6}-2k^{4}+k^{2}$ jest podzielna przez 36.

ZADANIE 5.

Wykaz, $\dot{\mathrm{z}}\mathrm{e}$ wśród pięciu dowolnie wybranych liczb naturalnych zawsze znajdą się trzy takie,

których suma jest podzielna przez 3.






LIGA MATEMATYCZNA

im. Zdzisława Matuskiego

GRUD Z$\mathrm{I}\mathrm{E}\acute{\mathrm{N}}$ 2012

SZKOLA PONADGIMNAZJALNA

ZADANIE I.

$\mathrm{W}$ trójkącie $ABC$ o polu $S$ poprowadzono dwusieczną CE i środkową $BD$, które przecinają się

w punkcie $F$. Oblicz pole czworokąta AEFD, mając dane $BC=a$ oraz $AC=b.$

ZADANIE 2.

Znajd $\acute{\mathrm{z}}$ wszystkie trójki liczb calkowitych nieujemnych $a, b, c$ spelniające uklad równań

$\left\{\begin{array}{l}
a+bc=3b\\
b+ac=3c\\
c+ab=3a.
\end{array}\right.$

ZADANIE 3.

$\mathrm{W}$ kwadracie o boku l $\mathrm{m}$ wybrano w dowolny sposób 100 punktów. Wykaz$\cdot, \dot{\mathrm{z}}\mathrm{e}$ istnieje kwadrat

o boku 25 cm, który zawiera co najmniej 7 punktów spośród wybranych.

ZADANIE 4.

Na stu kartkach trzeba wpisać liczby od l do 200 umieszczajqc jedną 1iczbę na $\mathrm{k}\mathrm{a}\dot{\mathrm{z}}$ dej stronie.

Czy $\mathrm{m}\mathrm{o}\dot{\mathrm{z}}$ liwe jest takie wpisanie liczb, by na $\mathrm{k}\mathrm{a}\dot{\mathrm{z}}$ dej kartce suma liczb z obu jej stron byla

podzielna przez 6?

ZADANIE 5.

Wyznacz wszystkie funkcje $f$: $\mathbb{R}\rightarrow \mathbb{R}$ spełniające równanie $2f(x)-f(-x)=3x^{2}+x+3$ dla

$\mathrm{k}\mathrm{a}\dot{\mathrm{z}}$ dej liczby rzeczywistej $x.$






LIGA MATEMATYCZNA

im. Zdzisława Matuskiego

GRUD Z$\mathrm{I}\mathrm{E}\acute{\mathrm{N}}$ 2013

SZKOLA PONADGIMNAZJALNA

ZADANIE I.

Na plaszczy $\acute{\mathrm{z}}\mathrm{n}\mathrm{i}\mathrm{e}$ danych jest siedem prostych. Wykaz$\cdot, \dot{\mathrm{z}}\mathrm{e}$ kąt pomiędzy pewnymi dwiema pro-

stymi, spośród danych, jest mniejszy $\mathrm{n}\mathrm{i}\dot{\mathrm{z}}26^{\mathrm{o}}$

ZADANIE 2.

$\mathrm{W}$ kola wpisano liczby w taki sposób, $\dot{\mathrm{z}}\mathrm{e}$ suma liczb w $\mathrm{k}\mathrm{a}\dot{\mathrm{z}}$ dych trzech stycznych kolach jest

równa 2013. Ob1icz sumę 1iczb w ko1ach po1ozonych w wierzcho1kach trójkąta.

ZADANIE 3.

Znajd $\acute{\mathrm{z}}$ liczbę sześciocyfrową $\overline{abcdef}$ wiedząc, $\dot{\mathrm{z}}\mathrm{e}$ liczby trzycyfrowe $\overline{abc}, \overline{cde}$ są sześcianami,

a liczby $\overline{bcd}\mathrm{i}\overline{def}$ sa kwadratami pewnych liczb naturalnych.

ZADANIE 4.

Wyznacz wszystkie funkcje $f:\mathbb{R}\rightarrow \mathbb{R}$ spefniające równość

$f(x-|x|)+f(x+|x|)=x$

dla $\mathrm{k}\mathrm{a}\dot{\mathrm{z}}$ dej liczby rzeczywistej $x.$

ZADANIE 5.

Rozwia $\dot{\mathrm{Z}}$ równanie

-{\it a}1 $+$ -{\it a}2{\it b}$+$ -{\it a}3{\it bc} $=$1

w zbiorze liczb calkowitych dodatnich.






LIGA MATEMATYCZNA

im. Zdzisława Matuskiego

GRUD Z$\mathrm{I}\mathrm{E}\acute{\mathrm{N}}$ 2014

SZKOLA PONADGIMNAZJALNA

ZADANIE I.

$\mathrm{W}$ trójkącie równoramiennym $ABC(|AC|=|BC|)$ na boku $AC$ obrano punkt $D$. Na trójkątach

$ABD\mathrm{i}DBC$ opisano okręgi $0_{1}$ oraz 02. Styczna do okręgu $0_{1}$ w punkcie $D$ przecina okrąg 02

w punkcie $M$. Wykaz$\cdot, \dot{\mathrm{z}}\mathrm{e}$ prosta $CM$ jest równoległa do prostej $AB.$

ZADANIE 2.

Znajd $\acute{\mathrm{z}}$ liczby naturalne $a, b$, których najmniejsza wspólna wielokrotność jest równa 630, a naj-

większy wspólny dzielnik 18, wiedząc, $\dot{\mathrm{z}}\mathrm{e}$ te liczby nie dzielq się przez siebie.

ZADANIE 3.

Oblicz sumę

$\sqrt{3-2\sqrt{2}}+\sqrt{5-2\sqrt{6}}+\sqrt{7-2\sqrt{12}}+\ldots+\sqrt{4029-2\sqrt{2014}}$2015.

ZADANIE 4.

Dodatnie liczby $a, b, c$ spelniają warunki $a+b+c=9$ oraz $\displaystyle \frac{1}{b+c}+\frac{1}{c+a}+\frac{1}{a+b}=\frac{10}{9}$. Oblicz

$\displaystyle \frac{a}{b+c}+\frac{b}{c+a}+\frac{c}{a+b}.$

ZADANIE 5.

$\mathrm{W}$ kwadracie o boku 2 wybrano w sposób dowo1ny 9 punktów. Wykaz, $\dot{\mathrm{z}}\mathrm{e}$ istnieje taka trójka

punktów wśród nich, $\dot{\mathrm{z}}\mathrm{e}$ pole figury, której wierzcholkami są te trzy punkty nie przekracza $\displaystyle \frac{1}{2}.$






LIGA MATEMATYCZNA

im. Zdzisława Matuskiego

GRUD Z$\mathrm{I}\mathrm{E}\acute{\mathrm{N}}$ 2015

SZKOLA PONADGIMNAZJALNA

ZADANIE I.

Na przyprostokątnych $BC\mathrm{i}CA$ trójkata prostokątnego $ABC$ zbudowano, po zewnętrznej stro-

nie, kwadraty BEFC oraz CGHA. Odcinek $CD$ jest wysokościa trójkąta $ABC$. Wykaz, $\dot{\mathrm{z}}\mathrm{e}$

proste $AE, BH$ oraz $CD$ przecinaja się w jednym punkcie.

ZADANIE 2.

$\mathrm{W}$ zbiorze liczb rzeczywistych rozwiąz uklad równań

$\left\{\begin{array}{l}
x^{2}+2y^{2}=2yz+100\\
z^{2}=2xy-100.
\end{array}\right.$

ZADANIE 3.

Rózne dodatnie liczby rzeczywiste a, b spelniają równość

$\displaystyle \frac{5a}{a+b}+\frac{5b}{a-b}=6.$

Wykaz, $\dot{\mathrm{z}}\mathrm{e}$ co najmniej jedna z nich jest niewymierna.

ZADANIE 4.

Czy istnieje taka dodatnia liczba calkowita $n$, aby zapis dziesiętny liczby $2^{n}$ zawiera110jedynek,

10 dwójek, l0 trójek, $\ldots$, 10 ósemek, l0 dziewiątek i pewną ilość zer?

ZADANIE 5.

Funkcja $f$: $\mathbb{R}\rightarrow \mathbb{R}$ spelnia warunek

2 $f(x)+f(1-x)=3x$

dla wszystkich liczb rzeczywistych $x$. Wyznacz $f(2015).$






LIGA MATEMATYCZNA

im. Zdzisława Matuskiego

GRUD Z$\mathrm{I}\mathrm{E}\acute{\mathrm{N}}$ 2016

SZKOLA PONADGIMNAZJALNA

ZADANIE I.

Na bokach $AC\mathrm{i}BC$ trójkąta $ABC$ zbudowano równoległoboki ACDE oraz BFGC tak, jak

na rysunku. Punkty $K\mathrm{i}L$ sq odpowiednio środkami odcinków $EG\mathrm{i}DF$. Oblicz $\displaystyle \frac{|KL|}{|AB|}.$
\begin{center}
\includegraphics[width=76.656mm,height=43.332mm]{./LigaMatematycznaMatuskiego_Liceum_Zestaw3_2016_2017_page0_images/image001.eps}
\end{center}
{\it G}

c  {\it F}

{\it E}

ZADANIE 2.

Znajd $\acute{\mathrm{z}}$ wszystkie pary $(x,y)$ liczb cafkowitych spelniające równanie

$x^{4}=y^{4}+1223334444.$

ZADANIE 3.

Na $\mathrm{k}\mathrm{a}\dot{\mathrm{z}}$ dej ścianie sześcianu napisano pewną dodatnią liczbę calkowitą. Następnie w $\mathrm{k}\mathrm{a}\dot{\mathrm{z}}$ dym

wierzchofku sześcianu umieszczono liczbę, którajest równa iloczynowi liczb znajdujących się na

ściankach, do których ten wierzcholek nalezy. Oblicz sumę liczb znajdujących się na wszystkich

ścianach wiedząc, $\dot{\mathrm{z}}\mathrm{e}$ suma liczb umieszczonych w wierzchołkach jest równa 70.

ZADANIE 4.

Suma pewnych dziewięciu liczb jest równa 90.

suma jest równa co najmniej 40.

Wykaz$\cdot, \dot{\mathrm{z}}\mathrm{e}$ wśród nich są takie cztery, których

ZADANIE 5.

Rozwia $\dot{\mathrm{z}}$ uklad równań

$\left\{\begin{array}{l}
x^{3}-y^{3}=26\\
x^{2}y-xy^{2}=6.
\end{array}\right.$






LIGA MATEMATYCZNA

im. Zdzisława Matuskiego

GRUD Z$\mathrm{I}\mathrm{E}\acute{\mathrm{N}}$ 2017

SZKOLA PONADGIMNAZJALNA

ZADANIE I.

Trapez prostokątny o podstawach $a, b$ opisanyjest na okręgu o średnicy $d$. Wykaz$\cdot, \dot{\mathrm{z}}\mathrm{e}$ prawdziwa

jest nierównośč

$d\leq\sqrt{\frac{a^{2}+b^{2}}{2}}.$

ZADANIE 2.

$\mathrm{W}$ zbiorze liczb rzeczywistych rozwiąz równanie

$x^{2}-8[x]+7=0,$

gdzie $[x]$ oznacza największą liczbę całkowitą nie przekraczającą liczby $x.$

ZADANIE 3.

Czy istnieją liczby $x_{1}, x_{2}, x_{3}, \ldots, x_{1001}$ równe $(-1)$ lub l takie, $\dot{\mathrm{z}}\mathrm{e}$

$x_{1}x_{2}+x_{2}x_{3}+x_{3}x_{4}+\ldots+x_{1000}x_{1001}+x_{1001}x_{1}=499$?

ZADANIE 4.

$\mathrm{W}$ zbiorze liczb rzeczywistych rozwiąz ukfad równań

$\left\{\begin{array}{l}
x(y+z)=6-x^{2}\\
y(z+x)=12-y^{2}\\
z(x+y)=18-z^{2}
\end{array}\right.$

ZADANIE 5.

Mikolaj wybral trzy liczby rzeczywiste $a, b, c$ i określił dzialanie $\star$ wzorem

$x\star y=ax+by+cxy$

dla dowolnych liczb rzeczywistych $x, y$. Obliczyll $\star 2=3\mathrm{i}2\star 3=4$ oraz zauwazyf, $\dot{\mathrm{z}}\mathrm{e}$ istnieje

niezerowa liczba rzeczywista $t$ taka, $\dot{\mathrm{z}}\mathrm{e}x\star t=x$ dla $\mathrm{k}\mathrm{a}\dot{\mathrm{z}}$ dej liczby rzeczywistej $x$. Wyznacz $t.$






LIGA MATEMATYCZNA

im. Zdzisława Matuskiego

GRUD Z$\mathrm{I}\mathrm{E}\acute{\mathrm{N}}$ 2018

SZKOLA PONADPODSTAWOWA

ZADANIE I.

Dany jest trapez ABCD o polu 15 i podstawach $AB$ oraz $CD$. Dwusieczna kąta $CBA$ jest

prostopadla do ramienia $AD$ i przecina je w takim punkcie $E, \dot{\mathrm{z}}\mathrm{e} \displaystyle \frac{|AE|}{|ED|}=2$. Oblicz pola figur

$ABE\mathrm{i}$ EBCD, na które zostal podzielony trapez.

ZADANIE 2.

Wykaz, $\dot{\mathrm{z}}\mathrm{e}$ liczba

$[\displaystyle \frac{n+4}{2}]+3n-2\cdot(-1)^{n}$

jest podzielna przez 7 d1a $\mathrm{k}\mathrm{a}\dot{\mathrm{z}}$ dej liczby naturalnej $n$, gdzie $[x]$ oznacza największą liczbę cal-

kowitą nie większą $\mathrm{n}\mathrm{i}\dot{\mathrm{z}}x.$

ZADANIE 3.

Wyznacz wszystkie liczby pierwsze, które są równocześnie sumami i róznicami dwóch liczb

pierwszych.

ZADANIE 4.

Czy istnieje liczba sześciocyfrowa podzielna przez ll o sumie cyfr równej ll, której dwie ostatnie

cyfry tworzą liczbę ll?

ZADANIE 5.

Znajd $\acute{\mathrm{z}}$ wszystkie trójki liczb rzeczywistych $(x,y,z)$ spelniające uklad równań

$\left\{\begin{array}{l}
x^{2}+y^{2}+z^{2}=33\\
x+3y+5z=34.
\end{array}\right.$






LIGA MATEMATYCZNA

im. Zdzisława Matuskiego

GRUD Z$\mathrm{I}\mathrm{E}\acute{\mathrm{N}}$ 2019

SZKOLA PONADPODSTAWOWA

ZADANIE I.

Długości $x, y, z$ boków trójkąta sa liczbami naturalnymi oraz $z=xy$. Wykaz$\cdot, \dot{\mathrm{z}}\mathrm{e}$ ten trójkat

jest równoramienny.

ZADANIE 2.

Znajd $\acute{\mathrm{z}}$ takie liczby calkowite dodatnie $n, \dot{\mathrm{z}}\mathrm{e}5^{n}-2\mathrm{i}5^{n}+2$ są liczbami pierwszymi.

ZADANIE 3.

Dane są liczby calkowite $a_{1}, a_{2}, \ldots$, a2019. Liczby $b_{1}, b_{2}, \ldots$, b2019 to 1iczby $a_{1}, a_{2}, \ldots$, a2019, a1e

ustawione w innej, przypadkowej, kolejności. Wykaz, $\dot{\mathrm{z}}\mathrm{e}$ iloczyn

$(a_{1}-b_{1})(a_{2}-b_{2})\ldots$ (a2019 - $b_{2019}$)

jest liczbą parzystą.

ZADANIE 4.

Wykaz, $\dot{\mathrm{z}}\mathrm{e}$ dla $\mathrm{k}\mathrm{a}\dot{\mathrm{z}}$ dej liczby naturalnej $n$ liczba $n^{4}-n^{2}$ jest podzielna przez 12.

ZADANIE 5.

$\mathrm{W}$ zbiorze liczb rzeczywistych rozwiąz układ równań

$\left\{\begin{array}{l}
x^{2}=y+z+2\\
y^{2}=z+x+2\\
z^{2}=x+y+2.
\end{array}\right.$






LIGA MATEMATYCZNA

im. Zdzisława Matuskiego

GRUD Z$\mathrm{I}\mathrm{E}\acute{\mathrm{N}}$ 2020

SZKOLA PONADPODSTAWOWA

ZADANIE I.

$\mathrm{W}$ trójkącie prostokątnym na dluzszej przyprostokątnej jako na średnicy opisano pólokrąg

tak, $\dot{\mathrm{z}}\mathrm{e}$ przecina on przeciwprostokątną w punkcie $K$. Krótsza przyprostokątna ma dlugość $a.$

Stosunek dlugości cięciwy lączącej wierzcholek kata prostego z punktem $K$ do dlugości krótszej

przyprostokątnej jest równy $\displaystyle \frac{4}{5}$. Wyznacz dlugość pófokręgu.

ZADANIE 2.

Uzasadnij, $\dot{\mathrm{z}}\mathrm{e}$ wśród dwunastu róznych liczb naturalnych dwucyfrowych $\mathrm{m}\mathrm{o}\dot{\mathrm{z}}$ na znalez$\acute{}$ć dwie,

których róznica jest liczbą dwucyfrową o jednakowych cyfrach dziesiqtek i jedności.

ZADANIE 3.

$\mathrm{W}$ zbiorze liczb calkowitych rozwiąz równanie

$ x(x+1)(x+2)+(x+1)(x+2)(x+3)+(x+2)(x+3)(x+4)+\ldots$

. . . $+(x+98)(x+99)(x+100)=2019x+2020.$

ZADANIE 4.

Wyznacz wszystkie funkcje $f:\mathbb{R}\rightarrow \mathbb{R}$ spelniające warunek

$f(x)\cdot f(y)-f(xy)=x+y$

dla dowolnych liczb rzeczywistych $x, y.$

ZADANIE 5.

$\mathrm{W}$ zbiorze liczb rzeczywistych rozwiąz uklad równań

$\left\{\begin{array}{l}
(a+b)^{2}=4c\\
(b+c)^{2}=4a\\
(c+a)^{2}=4b.
\end{array}\right.$






LIGA MATEMATYCZNA

im. Zdzislawa Matuskiego

GRUD Z$\mathrm{I}\mathrm{E}\acute{\mathrm{N}}$ 2021

SZKOLA PONADPODSTAWOWA

ZADANIE I.

$\mathrm{W}$ zbiorze liczb całkowitych rozwiąz równanie $x^{2}+y^{2}+3=xy+2x+2y.$

ZADANIE 2.

Czy liczbę 123456789 $\mathrm{m}\mathrm{o}\dot{\mathrm{z}}$ na przedstawič w postaci sumy dwóch skladników, z których jeden

jest zapisany tylko cyframi parzystymi, a drugi nieparzystymi?

ZADANIE 3.

Wykaz, $\dot{\mathrm{z}}\mathrm{e}$ w dowolnym ciqgu siedmiu liczb całkowitych zawsze $\mathrm{m}\mathrm{o}\acute{\mathrm{z}}\mathrm{n}\mathrm{a}$ wskazać pewną liczbę

kolejnych wyrazów, których suma jest podzielna przez 7.

ZADANIE 4.

Niech $p\mathrm{i}q$ będą takimi dodatnimi liczbami rzeczywistymi, $\dot{\mathrm{z}}\mathrm{e}q>p$. Wykaz, $\dot{\mathrm{z}}\mathrm{e}\mathrm{j}\mathrm{e}\dot{\mathrm{z}}$ eli $p\mathrm{i}q$ są

dlugościami przekątnych rombu o kącie o mierze $\displaystyle \frac{\pi}{6}$, to $\displaystyle \frac{p}{q}=2-\sqrt{3}.$

ZADANIE 5.

Pewna liczba dwucyfrowa ma trzy dzielniki jednocyfrowe i trzy dzielniki dwucyfrowe. Suma

wszystkich dzielników jednocyfrowych jest równa 8. Ob1icz sumę wszystkich dzie1ników dwucy-

frowych tej liczby.






LIGA MATEMATYCZNA

im. Zdzislawa Matuskiego

GRUD Z$\mathrm{I}\mathrm{E}\acute{\mathrm{N}}$ 2022

SZKOLA PONADPODSTAWOWA

ZADANIE I.

Znajd $\acute{\mathrm{z}}$ wszystkie pary $(p,q)$ liczb pierwszych takich, $\dot{\mathrm{z}}\mathrm{e}p+q=(p-q)^{3}$

ZADANIE 2.

$\mathrm{W}$ trójkąt $ABC$ wspisano okrąg, przy czym $|AC| = 5, |AB| = 6, |BC| = 3$. Na boku $AC$

wybrano punkt $D$, na boku $AB$ wybrano punkt $E$ w taki sposób, $\dot{\mathrm{z}}\mathrm{e}$ odcinek $ED$ jest styczny

do okręgu. Oblicz obwód trójkąta $AED.$

ZADANIE 3.

Dodatnie liczby rzeczywiste $a, b, c$ spelniają warunek

$\displaystyle \frac{(a+b+c)^{2}}{ab+bc+ac}=3.$

Wykaz, $\dot{\mathrm{z}}\mathrm{e}$ liczby $\alpha, b, c$ są równe.

ZADANIE 4.

Czy istnieją takie liczby calkowite $x, y, z, t, \dot{\mathrm{z}}\mathrm{e}x^{2}+y^{2}=z^{2}+t^{2}$ oraz $x+y+z+t=2023$?

ZADANIE 5.

Na tablicy napisano liczby naturalne od l do 10. Czy $\mathrm{m}\mathrm{o}\dot{\mathrm{z}}$ na umieścić między nimi znaki plus

oraz minus w taki sposób, aby otrzymać 0?






LIGA MATEMATYCZNA

STYC Z$\mathrm{E}\acute{\mathrm{N}}$ 2010

SZKOLA PONADGIMNAZJALNA

ZADANIE I.

Znajd $\acute{\mathrm{z}}$ wszystkie funkcje $f$: $\mathbb{R}\rightarrow \mathbb{R}$ spelniające następujące warunki

$\bullet f(xy)=x^{2}f(y)+yf(x)$ dla dowolnych liczb rzeczywistych $x, y$;

$\bullet f(2)=2.$

ZADANIE 2.

Wewnątrz danego czworokąta wypukłego znajdz' taki punkt, $\dot{\mathrm{z}}$ eby odcinki łączące ten punkt

ze środkami boków czworokqta dzielily czworokąt na cztery części o równych polach.

ZADANIE 3.

Uzasadnij, $\dot{\mathrm{z}}\mathrm{e}$ wśród 651iczb natura1nych znajduje się 91iczb takich, $\dot{\mathrm{z}}\mathrm{e}$ ich suma jest podzielna

przez 9.

ZADANIE 4.

Liczba naturalna $n$ jest większa od 2000. Wykaz$\cdot, \dot{\mathrm{z}}\mathrm{e}$ liczba $n+1$ jest podzielna przez 6, $\mathrm{j}\mathrm{e}\dot{\mathrm{z}}$ eli

wiadomo, $\dot{\mathrm{z}}\mathrm{e}n\mathrm{i}n+2$ są liczbami pierwszymi.

ZADANIE 5.

$\mathrm{W}$ prostokącie ABCD punkt $M$ jest środkiem boku AD, a N- środkiem boku $BC$. Na prze-

dluzeniu odcinka $CD$ poza punktem $D$ wybrano punkt $P$. Niech $S$ będzie punktem przecięcia

prostych $PM\mathrm{i}AC$. Udowodnij, $\dot{\mathrm{z}}\mathrm{e}$ kąty $SNM\mathrm{i}MNP$ są równe.






LIGA MATEMATYCZNA

STYC Z$\mathrm{E}\acute{\mathrm{N}}$ 2011

SZKOLA PONADGIMNAZJALNA

ZADANIE I.

Wewnątrz trójkąta równobocznego $ABC$ wybrano dowolny punkt $P$. Punkty $D, E, F$ są

rzutami prostokątnymi punktu $P$ na boki odpowiednio AB, $BC, CA$. Wyznacz wartości, jakie

$\mathrm{m}\mathrm{o}\dot{\mathrm{z}}\mathrm{e}$ przyjmować wyrazenie

$PD+PE+PF$

$AD+BE+CF$

ZADANIE 2.

Wyznacz wszystkie trójki $(a,b,c)$ liczb całkowitych, dla których $a^{2}-b^{2}-c^{2}=1\mathrm{i}a-b-c=-3.$

ZADANIE 3.

$\mathrm{W}$ polach tablicy $4\times 4$ umieszczono liczbę $-1$ oraz piętnaście liczb l. $\mathrm{M}\mathrm{o}\dot{\mathrm{z}}$ na jednoczešnie

zmienić znaki wszystkich liczb wjednym wierszu lub wjednej kolumnie. Wykaz$\cdot, \dot{\mathrm{z}}\mathrm{e}$ po dowolnej

liczbie takich zmian nie $\mathrm{m}\mathrm{o}\dot{\mathrm{z}}$ na uzyskać tablicy wypelnionej samymi jedynkami.

ZADANIE 4.

Załózmy, $\dot{\mathrm{z}}\mathrm{e}$ liczby $a\mathrm{i}b$ sq utworzone z tych samych cyfr, lecz ułozonych w innej kolejności.

Czy róznica tych liczb jest podzielna przez 9?

ZADANIE 5.

Czworokąty ABCD $\mathrm{i}$ EFGD są kwadratami. Oblicz dlugošć odcinka $BF$ wiedząc, $\dot{\mathrm{z}}\mathrm{e}$ dlugość

odcinka $AE$ jest równa $a.$
\begin{center}
\includegraphics[width=40.236mm,height=46.632mm]{./LigaMatematycznaMatuskiego_Liceum_Zestaw4_2010_2011_page0_images/image001.eps}
\end{center}
D c

F

A  B






LIGA MATEMATYCZNA

STYC Z$\mathrm{E}\acute{\mathrm{N}}$ 2012

SZKOLA PONADGIMNAZJALNA

ZADANIE I.

Rozwiąz równanie

-{\it x}1 $+$ -{\it y}1 $=$ 1- -{\it x}1{\it y}

w zbiorze liczb calkowitych.

ZADANIE 2.

Wykaz, $\dot{\mathrm{z}}\mathrm{e}\mathrm{j}\mathrm{e}\dot{\mathrm{z}}$ eli $a-1, a+1$ są liczbami pierwszymi większymi od 10, to 1iczba $a^{3}-4a$ jest

podzielna przez 240.

ZADANIE 3.

Wyznacz wszystkie funkcje $f:\mathbb{R}\rightarrow \mathbb{R}$ spelniające warunek

2 $f(x)+3f(1-x)=4x-1$

dla $\mathrm{k}\mathrm{a}\dot{\mathrm{z}}$ dej liczby rzeczywistej $x.$

ZADANIE 4.

Dwusieczne kątów zewnętrznych wypukfego czworokąta ABCD utworzyly nowy czworokąt.

Udowodnij, $\dot{\mathrm{z}}\mathrm{e}$ suma dlugości przekątnych nowego czworokątajest nie mniejsza $\mathrm{n}\mathrm{i}\dot{\mathrm{z}}$ obwód czwo-

rokąta ABCD.

ZADANIE 5.

$\mathrm{W}$ klasiejest 20 uczniów wpisanych do dziennika pod numerami od 1 do 20. Czy uda się ustawić

uczniów w pary tak, aby suma numerów uczniów $\mathrm{k}\mathrm{a}\dot{\mathrm{z}}$ dej pary byla podzielna przez 6?






LIGA MATEMATYCZNA

im. Zdzisława Matuskiego

STYC Z$\mathrm{E}\acute{\mathrm{N}}$ 2013

SZKOLA PONADGIMNAZJALNA

ZADANIE I.

Dwa okręgi są styczne zewnętrznie. Punkt $A\mathrm{l}\mathrm{e}\dot{\mathrm{z}}\mathrm{y}$ na jednym z okręgów i nalez $\mathrm{y}$ do wspólnej

stycznej, natomiast $AB$ jest średnicą okręgu. $\mathrm{Z}$ punktu $B$ prowadzimy styczną do drugiego

okręgu w punkcie $M$. Wykaz$\cdot, \dot{\mathrm{z}}\mathrm{e}AB=BM.$

ZADANIE 2.

Na dfugim pasku papieru wypisano kolejno obok siebie 2010 wybranych 1iczb natura1nych.

Liczby są dobrane w taki sposób, $\dot{\mathrm{z}}\mathrm{e}$ iloczyn $\mathrm{k}\mathrm{a}\dot{\mathrm{z}}$ dych siedmiu sąsiednich jest równy 2010. Jaka

jest najmniejsza $\mathrm{m}\mathrm{o}\dot{\mathrm{z}}$ liwa wartość sumy tych 20101iczb? Jaka jest największa $\mathrm{m}\mathrm{o}\dot{\mathrm{z}}$ liwa wartość

tej sumy?

ZADANIE 3.

Czy istnieją takie liczby calkowite $a, b, \dot{\mathrm{z}}\mathrm{e}a^{2}+b$ oraz $a+b^{2}$ są kolejnymi liczbami calkowitymi?

ZADANIE 4.

Danych jest lll dodatnich liczb cafkowitych. Wykaz$\cdot, \dot{\mathrm{z}}\mathrm{e}$ spośród nich $\mathrm{m}\mathrm{o}\dot{\mathrm{z}}$ na wybrać ll liczb,

których suma jest podzielna przez ll.

ZADANIE 5.

Rozwia $\dot{\mathrm{z}}$ ukfad równań

$\left\{\begin{array}{l}
(x+y)(x+y+z)=72\\
(y+z)(x+y+z)=120\\
(z+x)(x+y+z)=96.
\end{array}\right.$






LIGA MATEMATYCZNA

im. Zdzisława Matuskiego

STYC Z$\mathrm{E}\acute{\mathrm{N}}$ 2014

SZKOLA PONADGIMNAZJALNA

ZADANIE I.

Boki trójkąta $ABC$ są podzielone punktami $M, N\mathrm{i}P$ tak, $\dot{\mathrm{z}}\mathrm{e}$

{\it AM BN CP l}

{\it MB NC PA 4}

Wyznacz stosunek pola trójkąta ograniczonego prostymi AN, BP, CM do pola trójkąta ABC.

ZADANIE 2.

Ciąg liczbowy $(a_{n})_{n\in \mathbb{N}}$ jest określony następująco:

$a_{1}=1,$

$a_{2}=1,$

$a_{3}=-1,$

$\alpha_{n}=a_{n-1}a_{n-3}$, gdy $n\geq 4.$

Oblicz a2014.

ZADANIE 3.

$K\mathrm{a}\dot{\mathrm{z}}\mathrm{d}\mathrm{y}$ punkt plaszczyzny pomalowano na jeden z czterech kolorów: zólty, czerwony, zielony

oraz niebieski. $\mathrm{K}\mathrm{a}\dot{\mathrm{z}}\mathrm{d}\mathrm{y}$ kolor został wykorzystany. Wykaz$\cdot, \dot{\mathrm{z}}\mathrm{e}$ istnieje prosta, której punkty

sa co najmniej trzech kolorów.

ZADANIE 4.

Rozwiąz ukfad równań

$\left\{\begin{array}{l}
2x^{2014}+2y^{2014}-\mathrm{z}^{2014}=4\\
2y^{2014}+2z^{2014}-\mathrm{x}^{2014}=22\\
2z^{2014}+2x^{2014}-\mathrm{y}^{2014}=16.
\end{array}\right.$

ZADANIE 5.

Porównując wyniki w lowieniu ryb Adam, Bartek i Czarek stwierdzili, $\dot{\mathrm{z}}\mathrm{e}$ jeden z nich zlowil

tylko okonie, jeden tylko pstragi i jeden tylko lososie. Liczba ryb Adama jest o 7 większa od $\displaystyle \frac{3}{5}$

liczby okoni. Liczba ryb Bartka jest o 3 większa od $\displaystyle \frac{5}{7}$ liczby lososi. Natomiast liczba wszystkich

ryb jest trzycyfrowa liczbą pierwszą. Ile ryb złowil $\mathrm{k}\mathrm{a}\dot{\mathrm{z}}\mathrm{d}\mathrm{y}$ z chłopców?






LIGA MATEMATYCZNA

im. Zdzisława Matuskiego

STYC Z$\mathrm{E}\acute{\mathrm{N}}$ 2015

SZKOLA PONADGIMNAZJALNA

ZADANIE I.

Okręgi o promieniach riR przecinają się w punkcie K. Niech MiN będą punktami styczności

z okręgami wspólnej stycznej. Oblicz dfugość promienia okręgu opisanego na trójkącie KMN.

ZADANIE 2.

Wykaz, $\dot{\mathrm{z}}\mathrm{e}$ uklad równań

$\left\{\begin{array}{l}
x^{2}+2y=19\\
y^{2}+2z=9\\
z^{2}+2x=8
\end{array}\right.$

nie ma rozwiązań w zbiorze liczb cafkowitych.

ZADANIE 3.

$\mathrm{W}$ pola nieskończonej szachownicy wpisano liczby naturalne w taki sposób, $\dot{\mathrm{z}}\mathrm{e} \mathrm{k}\mathrm{a}\dot{\mathrm{z}}$ da liczba

w polujest średnią arytmetyczną ośmiu liczb z nią sąsiadujących. Wykaz, $\dot{\mathrm{z}}\mathrm{e}$ liczba 100 pojawi1a

się na szachownicy wiele razy lub nie pojawila się wcale.

ZADANIE 4.

Oblicz wartość wyrazenia

-{\it aa} $+$-{\it bb}'

jeśli $2a^{2}+4ab=ab+2b^{2}$ oraz $a\neq b.$

ZADANIE 5.

Wyznacz wszystkie funkcje $f:\mathbb{R}\rightarrow \mathbb{R}$ spelniające warunek

$f(f(x)-y)=f(x)+f(f(y)-f(-x))+x$

dla dowolnych liczb rzeczywistych $x, y.$






LIGA MATEMATYCZNA

im. Zdzisława Matuskiego

STYCZEN 2016

SZKOLA PONADGIMNAZJALNA

ZADANIE I.

Dany jest trapez ABCD o podstawach AB $\mathrm{i}$ CD oraz taki punkt $E\mathrm{l}\mathrm{e}\dot{\mathrm{z}}\mathrm{a}\mathrm{c}\mathrm{y}$ wewnątrz trapezu,

$\dot{\mathrm{z}}\mathrm{e}$ kąty $\triangleleft AED \mathrm{i} \triangleleft BEC$ sq proste. Punkt $S$ jest punktem przecięcia przekątnych trapezu.

Wykaz, $\dot{\mathrm{z}}\mathrm{e}\mathrm{j}\mathrm{e}\dot{\mathrm{z}}$ eli $E\neq S$, to prosta $ES$ jest prostopadla do podstaw trapezu.

ZADANIE 2.

Wykaz, $\dot{\mathrm{z}}\mathrm{e}$

$\sqrt[3]{120+\sqrt[3]{120+\sqrt[3]{120+}}}$

jest liczbą naturalna.

ZADANIE 3.

Rozstrzygnij, czy istnieje czworościan, w którym środki okręgów opisanych na ścianach $\mathrm{l}\mathrm{e}\dot{\mathrm{z}}\mathrm{a}$

na jednej plaszczy $\acute{\mathrm{z}}\mathrm{n}\mathrm{i}\mathrm{e}.$

ZADANIE 4.

Liczby 1, 2, 3, 4, $\ldots$, 32, 33 umieszczono w wierzcholkach 33-kąta foremnego, a następnie na

środku $\mathrm{k}\mathrm{a}\dot{\mathrm{z}}$ dego jego boku zapisano sumę liczb stojących na jego końcach. Czy istnieje takie

rozstawienie tych liczb w wierzchołkach wielokąta, aby wszystkie liczby zapisane na środkach

jego boków byfy liczbami podzielnymi przez 4?

ZADANIE 5.

Wyznacz wszystkie funkcje $f:\mathbb{R}\rightarrow \mathbb{R}$ spefniajqce warunek

$f(x+y)-f(x-y)=4xy$

dla dowolnych liczb rzeczywistych $x, y.$






LIGA MATEMATYCZNA

im. Zdzisława Matuskiego

STYCZEN 2018

SZKOLA PONADGIMNAZJALNA

ZADANIE I.

Czy istnieje czworościan, który ma siatkę będącą trójkatem prostokątnym?

sadnij.

Odpowied $\acute{\mathrm{z}}$ uza-

ZADANIE 2.

Wyznacz $T(2018)$ w ciągu o podanym wzorze rekurencyjnym

$\left\{\begin{array}{l}
T(1)=1\\
T(n)=2\cdot T([\frac{n}{2}]),\mathrm{g}\mathrm{d}\mathrm{y}n\geq 2.
\end{array}\right.$

gdzie $[x]$ oznacza największą liczbę całkowitą nie przekraczającą liczby $x.$

ZADANIE 3.

Czy istnieją liczby calkowite a $\mathrm{i}b$, które spelniają równanie

$|a^{2}+b|+|a^{2}-b|+|a+b^{2}|+|a-b^{2}|=1234567$?

ZADANIE 4.

$\mathrm{W}$ zbiorze liczb rzeczywistych rozwiąz ukfad równań

$\left\{\begin{array}{l}
\sqrt{2x-y+11}-\sqrt{3x+y-9}=3\\
\sqrt[4]{2x-y+11}+\sqrt[4]{3x+y-9}=3.
\end{array}\right.$

ZADANIE 5.

W okrąg wpisano dwa wielokąty równokątne: 2016-kąt i 2018-kąt.

wspólnych boków mogą mieć te dwa wielokąty?

Jaka największa liczbę






LIGA MATEMATYCZNA

im. Zdzisława Matuskiego

STYCZEN 2019

SZKOLA PONADPODSTAWOWA

ZADANIE I.

Boki trójkąta $ABC$ podzielono takimi punktami $D, E, F, \displaystyle \dot{\mathrm{z}}\mathrm{e}\frac{|AD|}{|DB|}=\frac{|BE|}{|EC|}=\frac{|CF|}{|FA|}=6$. Wyznacz

stosunek pola trójkąta $DEF$ do pola trójkąta $ABC.$
\begin{center}
\includegraphics[width=63.096mm,height=38.964mm]{./LigaMatematycznaMatuskiego_Liceum_Zestaw4_2018_2019_page0_images/image001.eps}
\end{center}
C

E

F

A  D  B

ZADANIE 2. Liczba dodatnia $x$ jest $p$ razy większa od liczby $y$. Suma liczb $x\mathrm{i}y$ jest $q$ razy

większa od ich róznicy. Znajd $\acute{\mathrm{z}}$ sumę $p+q$ wiedząc, $\dot{\mathrm{z}}\mathrm{e}p\mathrm{i}q$ są liczbami cafkowitymi dodatnimi.

ZADANIE 3.

Wyznacz największą liczbę pięciocyfrową spelniającą warunki:

$\bullet \dot{\mathrm{z}}$ adna cyfra nie jest zerem;

$\bullet$ pierwsze trzy cyfry tworza liczbę, która jest 9 razy większa od 1iczby utworzonej przez

dwie ostanie cyfry;

$\bullet$ trzy ostatnie cyfry tworzą liczbę, która jest 7 razy większa od 1iczby utworzonej przez

pierwsze dwie cyfry.

(Uwaga. Przyjmujemy, $\dot{\mathrm{z}}\mathrm{e}$ ostatnią cyfrą liczby jest cyfra jedności.)

ZADANIE 4.

Wyznacz wszystkie liczby pierwsze $p$ takie, $\dot{\mathrm{z}}\mathrm{e}p+6, p+12, p+18, p+24$ są równiez liczbami

pierwszymi.

ZADANIE 5.

Znajd $\acute{\mathrm{z}}$ wszystkie funkcje $f$: $\mathbb{R}\backslash \{0,1\}\rightarrow \mathbb{R}$ spełniające warunek

$(1-x)f(x)-2xf(1-x)=1$

dla $\mathrm{k}\mathrm{a}\dot{\mathrm{z}}$ dej liczby rzeczywistej $x$ róznej od 0 $\mathrm{i}1.$






LIGA MATEMATYCZNA

im. Zdzisława Matuskiego

STYC Z$\mathrm{E}\acute{\mathrm{N}}$ 2020

SZKOLA PONADPODSTAWOWA

ZADANIE I.

Ile jest liczb trzycyfrowych $\overline{xyz}$ podzielnych przez 21ub 5 takich, $\dot{\mathrm{z}}\mathrm{e}$

$\overline{xyz}+\overline{xzy}+\overline{yxz}=\overline{yzx}+\overline{zxy}+\overline{zyx}$?

ZADANIE 2.

Pole trapezu ABCD jest równe $s$, a stosunek dlugości podstaw AB $\mathrm{i}$ CD jest równy $k$. Prze-

kątne $AC\mathrm{i}BD$ przecinają się w punkcie $O$. Oblicz pole trójkąta $ABO.$

ZADANIE 3.

Uzasadnij, $\dot{\mathrm{z}}\mathrm{e}$ wśród pięciu liczb calkowitych $\mathrm{m}\mathrm{o}\dot{\mathrm{z}}$ na wybrać kilka tak, aby suma wybranych

liczb była podzielna przez 5.

ZADANIE 4.

Znajd $\acute{\mathrm{z}}$ wszystkie liczby pierwsze $p, q$ takie, $\dot{\mathrm{z}}\mathrm{e}7p+q$ oraz $pq+11\mathrm{t}\mathrm{e}\dot{\mathrm{z}}$ są liczbami pierwszymi.

ZADANIE 5.

$\mathrm{W}$ zbiorze liczb rzeczywistych rozwiąz uklad równań

$\left\{\begin{array}{l}
x^{2}+y^{2}+z=2\\
y^{2}+z^{2}+x=2\\
z^{2}+x^{2}+y=2.
\end{array}\right.$






LIGA MATEMATYCZNA

im. Zdzisława Matuskiego

STYCZEN 2021

SZKOLA PONADPODSTAWOWA

ZADANIE I.

Czy istnieją funkcje rzeczywiste $f$: $\mathbb{R}\rightarrow \mathbb{R}, g:\mathbb{R}\rightarrow \mathbb{R}$ takie, $\dot{\mathrm{z}}\mathrm{e}$

$f(x)g(y)=x+y+1$

dla dowolnych liczb rzeczywistych $x, y$?

ZADANIE 2.

Rozwiąz uklad równań

({\it xxxx}..32110.{\it xxxx}4321{\it xxxx}4532$===$--1-1-11

w zbiorze liczb rzeczywistych.

ZADANIE 3.

Czy liczbę 100 $\mathrm{m}\mathrm{o}\dot{\mathrm{z}}$ na przedstawič w postaci sumy liczb jednocyfrowych lub dwucyfrowych tak,

aby $\mathrm{u}\dot{\mathrm{z}}$ yć $\mathrm{k}\mathrm{a}\dot{\mathrm{z}}$ dą z cyfr dokładnie jeden raz?

ZADANIE 4.

$\mathrm{W}$ zbiorze liczb calkowitych dodatnich rozwiąz równanie

$9^{x}-2^{y}=1.$

ZADANIE 5.

$\mathrm{W}$ prostokącie ABCD po jego wewnętrznej stronie budujemy trójkqty równoboczne $ABE$ oraz

$BCF$. Wykaz, $\dot{\mathrm{z}}\mathrm{e}$ trójkąt $DFE$ jest równoboczny.






LIGA MATEMATYCZNA

im. Zdzislawa Matuskiego

STYCZEN 2022

SZKOLA PONADPODSTAWOWA

ZADANIE I.

Znajd $\acute{\mathrm{z}}$ wszystkie liczby pierwsze $p$ takie, $\dot{\mathrm{z}}\mathrm{e}p^{2}+2\mathrm{i}p^{3}+2$ są liczbami pierwszymi.

ZADANIE 2.

Na 101 kartkach Adam zapisaf 1iczby natura1ne od 1 do 101, po jednej na $\mathrm{k}\mathrm{a}\dot{\mathrm{z}}$ dej kartce. Potem

kartki odwrócil, pomieszaf i zapisal znowu liczby od l do 101, po jednej na $\mathrm{k}\mathrm{a}\dot{\mathrm{z}}$ dej kartce.

Następnie dodal liczby z obu stron kartki. Wykaz, $\dot{\mathrm{z}}\mathrm{e}$ iloczyn otrzymanych wyników jest liczbą

parzysta.

ZADANIE 3.

$\mathrm{W}$ trójkącie prostokątnym $ABC$ dwusieczna kąta ostrego dzieli przeciwlegfy bok w stosunku

3 : 5. Oblicz stosunek promienia $r$ okręgu wpisanego w ten trójkąt do promienia $R$ okręgu

opisanego na tym trójkącie.

ZADANIE 4.

Wyznacz wszystkie pary róznych liczb calkowitych $(x,y)$ spelniające równanie

$x+\displaystyle \frac{1}{y-2021}=y+\frac{1}{x-2021}.$

ZADANIE 5.

$\mathrm{W}$ zbiorze liczb rzeczywistych rozwiąz uklad równań

$\left\{\begin{array}{l}
25x^{2}+9y^{2}=12yz\\
9y^{2}+4z^{2}=20xz\\
4z^{2}+25x^{2}=30xy.
\end{array}\right.$






LIGA MATEMATYCZNA

im. Zdzislawa Matuskiego

STYCZEN 2023

SZKOLA PONADPODSTAWOWA

ZADANIE I.

Liczby 1, 2, 3, $\ldots$, 9 umieszczono na okręgu. Przez operację rozumiemy dodanie pewnej (tej

samej) liczby cafkowitej do dwóch wybranych sąsiednich liczb i umieszczenie tych sum na

okręgu w miejsce poprzednich liczb. Czy po wykonaniu skończonej liczby takich operacji $\mathrm{m}\mathrm{o}\dot{\mathrm{z}}$ na

otrzymać na okręgu dziewięć zer?

ZADANIE 2.

Czy $\mathrm{u}\dot{\mathrm{z}}$ ywając wszystkich dziesięciu cyfr $\mathrm{m}\mathrm{o}\dot{\mathrm{z}}$ na ułozyć liczbę podzielną przez ll?

ZADANIE 3.

Na boku $AB$ kwadratu ABCD wybrano punkt $E$, a na boku $BC$ wybrano punkt $F$ i pofaczo-

no je z wierzcholkami kwadratu. Odcinki te podzielify kwadrat na osiem części. Na rysunku

zapisano pola trzech z nich. Oblicz pole zaznaczonego czworokąta.
\begin{center}
\includegraphics[width=55.680mm,height=54.408mm]{./LigaMatematycznaMatuskiego_Liceum_Zestaw4_2022_2023_page0_images/image001.eps}
\end{center}
D  c

2

F

9

3

A  B  E

ZADANIE 4.

$\mathrm{W}$ zbiorze liczb rzeczywistych rozwiąz uklad równań

$\left\{\begin{array}{l}
2x+3y=5y^{2}\\
2y+3x=5x^{2}
\end{array}\right.$

ZADANIE 5.

Wyznacz wszystkie liczby pierwsze $p$ o tej wlasności, $\dot{\mathrm{z}}\mathrm{e}p+11$ jest dzielnikiem liczby

$p(p+1)(p+2).$






LIGA MATEMATYCZNA

FINAL

25 kwietnia 2009

SZKOLA PONADGIMNAZJALNA

ZADANIE I.

Znajd $\acute{\mathrm{z}}$ wszystkie funkcje $f$: $\mathbb{R}\rightarrow \mathbb{R}$ spełniające warunek $2f(x)+f(1-x)=x$ dla wszystkich

liczb rzeczywistych $x.$

ZADANIE 2.

Wypisujemy kolejne liczby naturalne od l do 2009. $K\mathrm{a}\dot{\mathrm{z}}$ dą z tych liczb zastępujemy sumą jej

cyfr i powtarzamy to $\mathrm{a}\dot{\mathrm{z}}$ do momentu uzyskania liczb jednocyfrowych. Jakich liczb w tym ciągu

jest więcej: jedynek czy ósemek?

ZADANIE 3.

Wyznacz wszystkie wartości naturalne $n$, dla których $3^{n}-1$ jest liczbą podzielną przez 13.

Wykaz, $\dot{\mathrm{z}}\mathrm{e}$ dla $\dot{\mathrm{z}}$ adnej wartości naturalnej $n$ liczba $3^{n}+1$ nie jest podzielna przez 13.

ZADANIE 4.

Liczby $n+2$ oraz $n-10$ są kwadratami liczb naturalnych. Znajdz' $n.$

ZADANIE 5.

$\mathrm{W}$ czworokącie wypuklym ABCD trójkąty $ABC, BCD, CDA, DAB$ mają równe obwody.

Udowodnij, $\dot{\mathrm{z}}\mathrm{e}$ ten czworokąt jest prostokątem.

ZADANIE 6.

Wykaz, $\dot{\mathrm{z}}\mathrm{e}$ wśród 401iczb natura1nych $\mathrm{m}\mathrm{o}\dot{\mathrm{z}}$ na wybrać 4, z których $\mathrm{k}\mathrm{a}\dot{\mathrm{z}}$ de dwie dają róznicę

podzielną przez 13.

ZADANIE 7.

Trapez prostokątny opisano na okręgu. Oblicz długości boków nierównoległych, $\mathrm{j}\mathrm{e}\dot{\mathrm{z}}$ eli podstawy

są równe a $\mathrm{i}b.$






LIGA MATEMATYCZNA

FINAL

26 marca 20l0

SZKOLA PONADGIMNAZJALNA

ZADANIE I.

Wykaz, $\dot{\mathrm{z}}\mathrm{e}\mathrm{j}\mathrm{e}\dot{\mathrm{z}}$ eli w trapez $\mathrm{m}\mathrm{o}\dot{\mathrm{z}}$ na wpisać okrqg, to okręgi, których średnicami są ramiona tego

trapezu, są styczne.

ZADANIE 2.

Udowodnij, $\dot{\mathrm{z}}\mathrm{e}$ jeśli $p\mathrm{i}q$ sq liczbami pierwszymi takimi, $\dot{\mathrm{z}}\mathrm{e}p\geq 5$ oraz $q-p=2$, to liczba $p+q$

jest podzielna przez 12.

ZADANIE 3.

Pole trójkąta $EFG$ jest równe l. Oblicz pole trójkąta $ABC$, wiedzqc, $\dot{\mathrm{z}}\mathrm{e}$

$|AE|=|EG|,|EF|=|FB|,|FG|=|GC|.$
\begin{center}
\includegraphics[width=78.840mm,height=33.372mm]{./LigaMatematycznaMatuskiego_Liceum_Zestaw5_2009_2010_page0_images/image001.eps}
\end{center}
C

E  F

A  H

ZADANIE 4.

Znajd $\acute{\mathrm{z}}$ wszystkie funkcje $f$: $\mathbb{R}\rightarrow \mathbb{R}$ spełniające warunek

$f(x)f(y)-f(xy)=x+y$ dla wszystkich liczb rzeczywistych $x, y.$

ZADANIE 5.

Oblicz wartość wyrazenia $q^{4}-6q^{3}+9q^{2}-7$ wiedząc, $\dot{\mathrm{z}}\mathrm{e}q^{2}-3q+1=0.$

ZADANIE 6.

Przedstaw liczbę l jako sumę kwadratów: (a) dwóch;

(b) trzech;

(c) czterech,

parami róznych dodatnich liczb wymiernych.

ZADANIE 7.

Na okręgu zaznaczono sześć punktów. $K\mathrm{a}\dot{\mathrm{z}}\mathrm{d}\mathrm{y}$ z odcinków lączących te punkty pomalowano

na czerwono lub niebiesko. Wykaz, $\dot{\mathrm{z}}\mathrm{e}$ otrzymano przynajmniej jeden jednokolorowy trójkąt.






LIGA MATEMATYCZNA

FINAL

30 marca 20ll

SZKOLA PONADGIMNAZJALNA

ZADANIE I.

Na okręgu napisano jedenaście liczb. Suma $\mathrm{k}\mathrm{a}\dot{\mathrm{z}}$ dych trzech kolejnych jest taka sama.

z liczb jest 9. Wyznacz pozostałe 1iczby.

Jedną

ZADANIE 2.

Rozwiąz uklad równań

$\left\{\begin{array}{l}
ab=a+b+1\\
bc=b+c+2\\
ac=\alpha+c+5.
\end{array}\right.$

ZADANIE 3.

Wykaz$\cdot, \dot{\mathrm{z}}\mathrm{e}$ dla dowolnych liczb cafkowitych $a\mathrm{i}b$ liczba $a^{3}b-ab^{3}$ jest podzielna przez 3.

ZADANIE 4.

Dany jest pięciokąt wypukly ABCDE, w którym przekątna $BD$ jest równolegla do boku $AE,$

a przekątna $CE$ jest równoległa do boku $AB$. Wykazać, $\dot{\mathrm{z}}\mathrm{e}$ pola trójkątów $ABC \mathrm{i} ADE$

są równe.

ZADANIE 5.

Wykaz$\cdot, \dot{\mathrm{z}}\mathrm{e}$ kwadrat liczby pierwszej większej od 3 z dzie1enia przez 12 daje resztę 1.

ZADANIE 6.

Rozwiąz równanie $x^{2}+4x-y^{2}-2y-8=0$ w zbiorze liczb naturalnych.

ZADANIE 7.

Czworokąt ABCD jest kwadratem. Punkty $E\mathrm{i}F$ lezą na bokach $BC\mathrm{i}$ CD tego kwadratu tak,

$\dot{\mathrm{z}}\mathrm{e}\angle EAF=45^{\mathrm{o}}$ Wykaz, $\dot{\mathrm{z}}\mathrm{e}|BE|+|DF|=|EF|.$
\begin{center}
\includegraphics[width=33.432mm,height=32.916mm]{./LigaMatematycznaMatuskiego_Liceum_Zestaw5_2010_2011_page0_images/image001.eps}
\end{center}
F

D c

E

A  B






l LiceumOgóloksztalcacewSlpsku

AkadmiPomorskawSiupsku

LIGA MATEMATYCZNA

FINAL

ll kwietnia 2012

SZKOLA PONADGIMNAZJALNA

ZADANIE I.

Liczby naturalne od l do 1000 pomnozono ko1ejno $\mathrm{k}\mathrm{a}\dot{\mathrm{z}}$ da przez $\mathrm{k}\mathrm{a}\dot{\mathrm{z}}$ dą. Wykaz$\cdot, \dot{\mathrm{z}}\mathrm{e}$ wšród tych

iloczynów więcej jest liczb parzystych $\mathrm{n}\mathrm{i}\dot{\mathrm{z}}$ nieparzystych.

ZADANIE 2.

Uzasadnij, $\dot{\mathrm{z}}\mathrm{e}$ dla $\mathrm{k}\mathrm{a}\dot{\mathrm{z}}$ dej liczby naturalnej $n$ liczba $n^{3}+5n$ jest podzielna przez 6.

ZADANIE 3.

Wyznacz wszystkie funkcje $f:\mathbb{R}\rightarrow \mathbb{R}$ spełniające równanie

2 $f(x)+f(-x)=3x^{2}+x+3$

dla $\mathrm{k}\mathrm{a}\dot{\mathrm{z}}$ dej liczby rzeczywistej $x.$

ZADANIE 4.

Na bokach AB, $BC\mathrm{i}AC$ trójkąta wybrano odpowiednio punkty $P, Q\mathrm{i}R$ tak, $\dot{\mathrm{z}}\mathrm{e}AP=CQ$

oraz na czworokącie RPBQ $\mathrm{m}\mathrm{o}\dot{\mathrm{z}}$ na opisać okrąg. Styczne do okręgu opisanego na trójkącie

$ABC$ w punktach $A\mathrm{i}C$ przecinają proste RP $\mathrm{i}RQ$ odpowiednio w punktach $X\mathrm{i}Y$. Wykaz,

$\dot{\mathrm{z}}\mathrm{e}RX=RY.$

ZADANIE 5.

$\mathrm{W}$ wierzchołkach siedmiokąta foremnego ustawiono pionki czerwone lub niebieskie- po jednym

w $\mathrm{k}\mathrm{a}\dot{\mathrm{z}}$ dym wierzcholku. Uzasadnij, $\dot{\mathrm{z}}\mathrm{e}$ znajdą się trzy wierzchołki z pionkami tego samego koloru

takie, $\dot{\mathrm{z}}\mathrm{e}$ będą wierzcholkami trójkąta równoramiennego.

ZADANIE 6.

Znajd $\acute{\mathrm{z}}$ wszystkie liczby pierwsze $p$ takie, $\dot{\mathrm{z}}\mathrm{e}2p-1, 2p+1$ są równiez liczbami pierwszymi.

ZADANIE 7.

Tarczę podzielono na szešć sektorów i w $\mathrm{k}\mathrm{a}\dot{\mathrm{z}}\mathrm{d}\mathrm{y}$ wpisano inną liczbę naturalną od l do 6. Zmie-

niamy te liczby przez dodanie do dwóch z nich tej samej liczby. Operację tę powtarzamy

wielokrotnie. Czy w którymś momencie we wszystkich sektorach będzie ta sama liczba?






LIGA MATEMATYCZNA

im. Zdzisława Matuskiego

FINAL

10 kwietnia 20l3

SZKOLA PONADGIMNAZJALNA

ZADANIE I.

Iloczyn 221iczb ca1kowitych jest równy 1. Czy suma tych 1iczb $\mathrm{m}\mathrm{o}\dot{\mathrm{z}}\mathrm{e}$ być równa 0?

ZADANIE 2.

Rozwia $\dot{\mathrm{z}}$ uklad równań 

ZADANIE 3.

Liczby rzeczywiste $\alpha, b$ spelniają równośč $\displaystyle \frac{2a}{a+b}+\frac{b}{a-b}=2$. Wyznacz wszystkie wartościjakie

$\mathrm{m}\mathrm{o}\dot{\mathrm{z}}\mathrm{e}$ przyjmowač ufamek $\displaystyle \frac{3a-b}{a+5b}.$

ZADANIE 4.

Wyznacz wszystkie takie pary liczb naturalnych $x, y, \dot{\mathrm{z}}\mathrm{e}$ wyrazenie $(x-y)-(\sqrt{x}-\sqrt{y})$ jest

liczbą pierwszą.

ZADANIE 5.

Na plaszczy $\acute{\mathrm{z}}\mathrm{n}\mathrm{i}\mathrm{e}$ danych jest pięć punktów kratowych (są to punkty o wspólrzędnych będących

liczbami calkowitymi). Uzasadnij, $\dot{\mathrm{z}}\mathrm{e}$ środek jednego z odcinków lączących te punkty $\mathrm{t}\mathrm{e}\dot{\mathrm{z}}$ jest

punktem kratowym.

ZADANIE 6.

Dane są dwa okręgi styczne zewnętrznie w punkcie K. Odleglości K od punktów styczności

okręgów ze wspólną styczną są równe 6 i8. Wyznacz promienie okręgów.

ZADANIE 7.

Wyznacz wszystkie funkcje $f:\mathbb{R}\rightarrow \mathbb{R}$ spelniające warunki

$\bullet f(2)=2$

$\bullet f(xy)=x^{2}f(y)+yf(x)$

dla $\mathrm{k}\mathrm{a}\dot{\mathrm{z}}$ dych liczb rzeczywistych $x, y.$






LIGA MATEMATYCZNA

im. Zdzisława Matuskiego

FINAL

15 kwietnia 20l4

SZKOLA PONADGIMNAZJALNA

ZADANIE I.

Wykaz, $\dot{\mathrm{z}}$ ejezeli $a, b, c, d$ są liczbami nieparzystymi, to nie istnieje taka liczba calkowita $x$, aby

spelniona byla równość

$x^{4}+ax^{3}+bx^{2}+cx+d=0.$

ZADANIE 2.

Rozwiąz uklad równań

$\left\{\begin{array}{l}
x^{2}+2y^{2}-2yz=100\\
2xy-z^{2}=100.
\end{array}\right.$

ZADANIE 3.

Wyznacz wszystkie funkcje $f:\mathbb{R}\backslash \{0\}\rightarrow \mathbb{R}$ spelniające równanie

$2f(x)+3f(\displaystyle \frac{1}{x})=x^{2}$

dla $\mathrm{k}\mathrm{a}\dot{\mathrm{z}}$ dej liczby rzeczywistej $x$ róznej od 0.

ZADANIE 4.

Wysokość i środkowa poprowadzone z jednego wierzchofka trójkąta tworzą z bokami tego trój-

kąta jednakowe kąty. $\acute{\mathrm{S}}$ rodkowa ma dlugość $a$. Oblicz promień okręgu opisanego na tym trój-

kącie.

ZADANIE 5.

Na okręgu wybrano 2015 punktów, z których 2014 poko1orowano na biafo oraz jeden na czer-

wono. Których wielokątów o wierzchofkach w tych punktach jest więcej: wielokątów o białych

wierzcholkach czy wielokątów z jednym wierzchołkiem czerwonym?

ZADANIE 6.

Czy istnieje liczba naturalna $n$ taka, $\dot{\mathrm{z}}\mathrm{e}$ w zapisie dziesiętnym liczby $2^{n}\mathrm{k}\mathrm{a}\dot{\mathrm{z}}$ da z cyfr 0, 1, 2, $\ldots$, 9

występuje 1000 razy?

ZADANIE 7.

Prosta przechodzaca przez środki przekątnych $AC\mathrm{i}BD$ czworokąta ABCD przecina boki $AD$

$\mathrm{i}BC$ w punktach, odpowiednio, $M\mathrm{i}N$. Wykaz, $\dot{\mathrm{z}}\mathrm{e}$ trójkqty AND $\mathrm{i}BCM$ mają równe pola.






AHADEMIA POMORSHA

III SLUPSHU
\begin{center}
\includegraphics[width=40.740mm,height=4.476mm]{./LigaMatematycznaMatuskiego_Liceum_Zestaw5_2015_2016_page0_images/image001.eps}
\end{center}
LIGA MATEMATYCZNA

im. Zdzisława Matuskiego

FINAL
\begin{center}
\includegraphics[width=34.548mm,height=42.576mm]{./LigaMatematycznaMatuskiego_Liceum_Zestaw5_2015_2016_page0_images/image002.eps}
\end{center}
16 kwietnia 20l5

SZKOLA PONADGIMNAZJALNA

ZADANIE I.

Uzasadnij, $\dot{\mathrm{z}}\mathrm{e}$ liczba $S$ jest podzielna przez 45, gdy

$ S=111\ldots$ 1$+$222$\ldots$ 2$+$333$\ldots 3+\ldots+999\ldots 9.$
\begin{center}
\includegraphics[width=64.716mm,height=6.852mm]{./LigaMatematycznaMatuskiego_Liceum_Zestaw5_2015_2016_page0_images/image003.eps}
\end{center}
2015 cyfr 20l5 cyfr  2015 cyfr
\begin{center}
\includegraphics[width=17.628mm,height=6.852mm]{./LigaMatematycznaMatuskiego_Liceum_Zestaw5_2015_2016_page0_images/image004.eps}
\end{center}
2015 cyfr

ZADANIE 2.

Dany jest okrąg $0_{1}$ o środku $S$ oraz okrąg 02 przechodzący przez $S$, przecinający okrąg 01

w punktach A $\mathrm{i}B. \mathrm{Z}$ punktu $A$ poprowadzono prostą, przecinająca okrąg $0_{1}$ w punkcie $C$, zaś

okrąg 02 w punkcie $D$. Udowodnij, $\dot{\mathrm{z}}\mathrm{e}$ trójkąt $BCD$ jest równoramienny.

ZADANIE 3.

$\mathrm{W}$ kwadracie o boku o dlugości 3 wybrano dowo1nie dziesięć punktów. Wykaz, $\dot{\mathrm{z}}\mathrm{e}$ wśród tych

punktów zawsze znajdą się dwa, których odleglość jest nie większa $\mathrm{n}\mathrm{i}\dot{\mathrm{z}}\sqrt{2}.$

ZADANIE 4.

Wykaz, $\dot{\mathrm{z}}\mathrm{e}$ dla $\mathrm{k}\mathrm{a}\dot{\mathrm{z}}$ dej liczby cafkowitej $n$ liczba $\displaystyle \frac{1}{6}(n^{3}-7n+2016)$ jest calkowita.

ZADANIE 5.

$\mathrm{W}$ klasie jest 30 uczniów. Siedzą oni w piętnastu dwuosobowych fawkach tak, $\dot{\mathrm{z}}\mathrm{e}$ polowa

dziewcząt siedzi z chłopcami. Rozstrzygnij, czy $\mathrm{m}\mathrm{o}\dot{\mathrm{z}}$ na uczniów tej klasy tak posadzić, aby

pofowa chlopców siedziala z dziewczętami.

ZADANIE 6.

$\mathrm{W}$ okrąg $0$ wpisany jest taki pięciokąt ABCDE, $\dot{\mathrm{z}}\mathrm{e} |AE| = |BC| = |CD|$. Proste AB $\mathrm{i}$ {\it DE}

przecinają się w punkcie $F$. Udowodnij, $\dot{\mathrm{z}}\mathrm{e}$ środek okręgu opisanego na trójkącie $BDF\mathrm{l}\mathrm{e}\dot{\mathrm{z}}\mathrm{y}$ na

okręgu $0.$

ZADANIE 7.

Rozwiąz uklad równań

$\left\{\begin{array}{l}
x-\frac{1}{xyz}=0\\
yzxx2y3y7zz==00.
\end{array}\right.$






LIGA MATEMATYCZNA

im. Zdzisława Matuskiego

FINAL

25 kwietnia 20l6

SZKOLA PONADGIMNAZJALNA

ZADANIE I.

Wykaz, $\dot{\mathrm{z}}\mathrm{e}$ suma kwadratów trzech kolejnych liczb naturalnych nie $\mathrm{m}\mathrm{o}\dot{\mathrm{z}}\mathrm{e}$ być kwadratem liczby

naturalnej.

ZADANIE 2.

Funkcja $f$: $\mathbb{R}\rightarrow \mathbb{R}$ spelnia warunki:

{\it a}) $f(x+y)=f(x)+f(y)$ dla dowolnych liczb rzeczywistych $x, y$;

{\it b}) $f(1)=1.$

Wyznacz $f(\displaystyle \frac{9}{32}).$

ZADANIE 3.

Dany jest czworokąt wypukły ABCD. Punkty $K \mathrm{i} L 1\mathrm{e}\dot{\mathrm{Z}}$ a odpowiednio na odcinkach $AB$

$\mathrm{i}$ AD, przy czym czworokąt AKCL jest równoleglobokiem. Odcinki $KD\mathrm{i}BL$ przecinają się

w punkcie $M$. Wykaz, $\dot{\mathrm{z}}\mathrm{e}$ pola czworokątów AKML $\mathrm{i}$ BCDM są równe.

ZADANIE 4.

Znajd $\acute{\mathrm{z}}$ wszystkie liczby pierwsze $p$ o tej wfasności, $\dot{\mathrm{z}}\mathrm{e}$ liczba $19p+1$ jest sześcianem pewnej

liczby całkowitej.

ZADANIE 5.

Wykaz, $\dot{\mathrm{z}}\mathrm{e}$ dla $\mathrm{k}\mathrm{a}\dot{\mathrm{z}}$ dej liczby naturalnej $n$ liczba

4444$\ldots$ 47777$\ldots$ 74444$\ldots$ 4$+$2016
\begin{center}
\includegraphics[width=59.940mm,height=6.852mm]{./LigaMatematycznaMatuskiego_Liceum_Zestaw5_2016_2017_page0_images/image001.eps}
\end{center}
7n cyfr 4 n cyfr 7  7n cyfr 4

jest zfozona.

ZADANIE 6.

Dany jest trójkqt ostrokątny $ABC$ oraz jego wysokości AD $\mathrm{i}$ BE. Punkty $P\mathrm{i}Q$ są rzutami

prostokątnymi odpowiednio punktów $A\mathrm{i}B$ na prostą DE. Wykaz, $\dot{\mathrm{z}}\mathrm{e}|PE|=|QD|.$

ZADANIE 7.

Rozwiąz uklad równań

$\left\{\begin{array}{l}
x_{1}^{2}-3x_{1}+4=x_{2}\\
x_{2}^{2}-3x_{2}+4=x_{3}\\
x_{3}^{2}-3x_{3}+4=x_{4}\\
x_{n-1}^{2}-3x_{n-1}+4=x_{n}\\
x_{n}^{2}-3x_{n}+4=x_{1}.
\end{array}\right.$






LIGA MATEMATYCZNA

im. Zdzisława Matuskiego

FINAL

24 kwietnia 20l7

SZKOLA PONADGIMNAZJALNA

ZADANIE I.

Dane są dwie $\mathrm{b}\mathrm{l}\mathrm{i}\acute{\mathrm{z}}$niacze (to znaczy rózniące się o 2) 1iczby pierwsze. Udowodnij, $\dot{\mathrm{z}}\mathrm{e}$ nie mogą

one być przyprostokątnymi trójkąta prostokątnego o wszystkich bokach o dlugości calkowitej.

ZADANIE 2.

Na tablicy napisano dwie liczby: pierwszq i drugą. Następnie napisano liczbę trzecią, która

jest sumą pierwszej i drugiej, potem zapisano czwartą liczbę, która jest sumą drugiej i trzeciej.

I tak dalej $\mathrm{a}\dot{\mathrm{z}}$ do dziesiątej liczby. Suma wszystkich dziesięciu liczb napisanych na tablicy jest

równa 5005. Wyznacz siódmą 1iczbę.

ZADANIE 3.

Czy liczbę l $\mathrm{m}\mathrm{o}\dot{\mathrm{z}}$ na przedstawićjako sumę ufamków $\displaystyle \frac{1}{a}, \displaystyle \frac{1}{b}, \displaystyle \frac{1}{c}, \displaystyle \frac{1}{d}$, gdzie $a, b, c, d$ są nieparzystymi

liczbami naturalnymi?

ZADANIE 4.

Danych jest $21$liczb rzeczywistych. Wiadomo, $\dot{\mathrm{z}}\mathrm{e}$ suma $\mathrm{k}\mathrm{a}\dot{\mathrm{z}}$ dych jedenastu spośród tych liczb

jest większa od sumy pozostalych dziesięciu. Wykaz, $\dot{\mathrm{z}}\mathrm{e}$ wszystkie liczby są dodatnie.

ZADANIE 5.

Oblicz pole trapezu prostokątnego wiedząc, $\dot{\mathrm{z}}\mathrm{e}$ odleglości środka okręgu wpisanego w ten trapez

od końców ramienia nieprostopadlego do podstaw, są równe $\alpha$ oraz $2a.$

ZADANIE 6.

$\mathrm{W}$ zbiorze liczb rzeczywistych rozwiąz uklad równań

$\left\{\begin{array}{l}
x^{2}+9=4y\\
y^{2}+1=6z\\
z^{2}+4=2x.
\end{array}\right.$

ZADANIE 7.

Czworokąt ABCD jest wpisany w okrąg. Proste AB $\mathrm{i}$ CD przecinają się w punkcie $E$, a proste

{\it AD} $\mathrm{i}BC$ przecinają się w punkcie $F$. Udowodnij, $\dot{\mathrm{z}}\mathrm{e}$ jeśli $|BE|=|DF|$, to $|CE|=|CF|.$
\begin{center}
\includegraphics[width=57.816mm,height=55.164mm]{./LigaMatematycznaMatuskiego_Liceum_Zestaw5_2017_2018_page0_images/image001.eps}
\end{center}
{\it F}

{\it D}

C

{\it E}






LIGA MATEMATYCZNA

im. Zdzisława Matuskiego

FINAL

16 kwietnia 20l8

SZKOLA PONADGIMNAZJALNA

ZADANIE I.

$\mathrm{W}$ zbiorze liczb rzeczywistych rozwiąz równanie

$x^{2}-7[x]+6=0.$

ZADANIE 2.

Wykaz$\cdot, \dot{\mathrm{z}}\mathrm{e}$ kwadrat iloczynu dwóch kolejnych liczb calkowitych podzielnych przez 5 dzie1i się

przez 2500.

ZADANIE 3.

$\mathrm{W}$ zbiorze liczb rzeczywistych rozwiąz uklad równań

$\left\{\begin{array}{l}
a^{2}+b^{2}+c^{2}=ab+bc+ca\\
abc=8.
\end{array}\right.$

ZADANIE 4.

Wyznacz wszystkie trójki liczb pierwszych $p, q, r$ takie, $\dot{\mathrm{z}}\mathrm{e}$

$\displaystyle \frac{pqr}{p+q+r}=11.$

ZADANIE 5.

Symetralne ramion równoramiennego trójkąta rozwartokątnego dzielą podstawę na trzy równe

części. Oblicz miary kątów danego trójkąta.

ZADANIE 6.

Sprawd $\acute{\mathrm{z}}$, czy istnieją liczby calkowite $a, b, c$ spelniające równanie

$(9a-5b)(7b-3c)(5c-a)=20182019.$

Odpowied $\acute{\mathrm{z}}$ uzasadnij.

ZADANIE 7.

Cyfrą jedności pewnej liczby czterocyfrowej jest 5. $\mathrm{J}\mathrm{e}\dot{\mathrm{z}}$ eli tę cyfrę przeniesiemy z ostatniego

miejsca na pierwsze, to otrzymamy liczbę o 2277 większą od danej. Jaka to 1iczba?





\begin{center}
\includegraphics[width=20.628mm,height=30.024mm]{./LigaMatematycznaMatuskiego_Liceum_Zestaw5_2019_2020_page0_images/image001.eps}
\end{center}
0

flkademia

P omorskawStupsku

LIGA MATEMATYCZNA

im. Zdzisława Matuskiego

FINAL 26 marca 2019

SZKOLA PONADPODSTAWOWA

ZADANIE I.

Wyznacz wszystkie funkcje rzeczywiste $f$: $\mathbb{R}\rightarrow \mathbb{R}$ spelniające równanie

$2f(x)+f(1-x)=x+7$

dla $\mathrm{k}\mathrm{a}\dot{\mathrm{z}}$ dej liczby rzeczywistej $x.$

ZADANIE 2.

Ania napisala kilka kolejnych liczb naturalnych. Wśród nich są trzy liczby pierwsze, trzy liczby

podzielne przez 3 i trzy 1iczby parzyste. I1e co najwyzej $\mathrm{m}\mathrm{o}\dot{\mathrm{z}}\mathrm{e}$ być równa suma liczb napisanych

przez Anię?

ZADANIE 3.

Znajd $\acute{\mathrm{z}}$ wszystkie trójki liczb rzeczywistych $(x,y,z)$ spelniające uklad równań

$\left\{\begin{array}{l}
x^{2}+y^{2}+z^{2}=14\\
x+2y+3z=14.
\end{array}\right.$

ZADANIE 4.

Czy liczba sześciocyfrowa, której cyframi są liczby 1, 2, 3, 4, 5, 6 (kazda $\mathrm{u}\dot{\mathrm{z}}$ yta jeden raz) $\mathrm{m}\mathrm{o}\dot{\mathrm{z}}\mathrm{e}$

być podzielna przez ll?

ZADANIE 5.

Pole trójkąta $ABC$ jest równe $p$. Odcinek $DE$ równolegfy do $AB$ odcina trójkąt o polu $q$. Niech

$F$ będzie dowolnym punktem $1\mathrm{e}\dot{\mathrm{Z}}$ acym na podstawie $AB$. Oblicz pole czworokąta DFEC.

ZADANIE 6.

Wykaz, $\dot{\mathrm{z}}\mathrm{e}\mathrm{j}\mathrm{e}\dot{\mathrm{z}}$ eli liczby dodatnie $a, b, c$ spelniają warunek $abc=1$, to

$\displaystyle \frac{1}{1+a^{2}b}+\frac{1}{1+bc^{2}}=1.$

ZADANIE 7.

Wyznacz wszystkie trójki liczb pierwszych $(p,q,r)$ spefniające uklad równań

$\left\{\begin{array}{l}
q=p^{2}+6\\
r=q^{2}+6.
\end{array}\right.$






flkademia

P omorskamStupsku

LIGA MATEMATYCZNA

im. Zdzisława Matuskiego

FINAL 20 kwietnia 2021

SZKOLA PONADPODSTAWOWA

ZADANIE I.

Wykaz, $\dot{\mathrm{z}}\mathrm{e}$ spośród dowolnych siedmiu liczb naturalnych $\mathrm{m}\mathrm{o}\dot{\mathrm{z}}$ na wybrać dwie liczby $a, b$ takie,

$\dot{\mathrm{z}}\mathrm{e}$ róznica $a^{2}-b^{2}$ jest podzielna przez 10.

ZADANIE 2.

$\mathrm{W}$ zbiorze liczb rzeczywistych rozwiąz uklad równań

$\left\{\begin{array}{l}
x^{2}+x(y-4)=-2\\
y^{2}+y(x-4)=-2.
\end{array}\right.$

ZADANIE 3.

$a, b, c, d, e$ są to liczby 7, 8, 9, 10, 11, a1e ustawione w innej, przypadkowej ko1ejności. Wykaz,

$\dot{\mathrm{z}}\mathrm{e}$ iloczyn $(a-1)(b-2)(c-3)(d-4)(e-5)$ jest liczba parzysta.

ZADANIE 4.

$\mathrm{W}$ wycinek kola o promieniu $R$ wpisano okrąg o promieniu $r$. Cięciwa lącząca końce promieni

wycinka kola ma dlugość $ 2\alpha$. Wykaz, $\displaystyle \dot{\mathrm{z}}\mathrm{e}\frac{1}{r}=\frac{1}{R}+\frac{1}{a}.$

ZADANIE 5.

Przyprostokątne trójkąta prostokatnego mają dlugości $a, b$. Wyznacz dlugość odcinka wyciętego

z dwusiecznej kąta prostego przez przeciwprostokątną.

ZADANIE 6.

Adam $\mathrm{u}\dot{\mathrm{z}}$ yf dwukrotnie $\mathrm{k}\mathrm{a}\dot{\mathrm{z}}$ dej z cyfr 1, 2, 3, $\ldots$, 9 i utworzyl kilka parami róznych liczb pierw-

szych w taki sposób, $\dot{\mathrm{z}}\mathrm{e}$ suma tych liczb jest $\mathrm{m}\mathrm{o}\dot{\mathrm{z}}$ liwie najmniejsza. Oblicz tę sumę.

ZADANIE 7.

Funkcja $f$: $\mathbb{R}\rightarrow \mathbb{R}$ spelnia warunek

$2f(x)+3f(\displaystyle \frac{2022}{x})=5x$

dla dowolnej dodatniej liczby rzeczywistej $x$. Oblicz $f(6).$






flkademia

P omorskamStupsku

LIGA MATEMATYCZNA

im. Zdzisława Matuskiego

FINAL 12 kwietnia 2022

SZKOLA PONADPODSTAWOWA

ZADANIE I.

Dane są dodatnie liczby calkowite $a, b, c, d, e, f$ takie, $\dot{\mathrm{z}}\mathrm{e}\mathrm{k}\mathrm{a}\dot{\mathrm{z}}$ da z sum $\alpha+b+c, b+c+d+e,$

$d+e+f$ jest liczbą nieparzystą. Uzasadnij, $\dot{\mathrm{z}}\mathrm{e}$ iloczyn abcdef jest liczba podzielną przez 4.

ZADANIE 2.

Kwadrat $K$ i prostokąt $P$, który nie jest kwadratem, mają równe pola. Która z tych figur

ma większy obwód? Odpowied $\acute{\mathrm{z}}$ uzasadnij.

ZADANIE 3.

Na okręgu umieszczono sześć liczb, których sumajest równa l. $K\mathrm{a}\dot{\mathrm{z}}$ da z nichjest równa wartości

bezwzględnej róznicy dwóch liczb następujących po niej, gdy poruszamy się po okręgu zgodnie

z ruchem wskazówek zegara. Wyznacz te liczby.

ZADANIE 4.

$\mathrm{W}$ trójkąt $ABC$ wpisano okrag i poprowadzono styczną do tego okręgu równoleglą do boku $AB,$

nie zawierającą tego boku. Oblicz długość odcinka stycznej zawartego w trójkącie w zalezności

od dlugości boków trójkąta.

ZADANIE 5.

Wykaz$\cdot, \dot{\mathrm{z}}\mathrm{e}$ liczba

111$\ldots$ 1777$\ldots$ 7111$\ldots$ 1$+$2022
\begin{center}
\includegraphics[width=52.428mm,height=6.396mm]{./LigaMatematycznaMatuskiego_Liceum_Zestaw5_2021_2022_page0_images/image001.eps}
\end{center}
n cyfr n cyfr  n cyfr

jest zfozona dla $\mathrm{k}\mathrm{a}\dot{\mathrm{z}}$ dej liczby naturalnej $n.$

ZADANIE 6.

Liczby 14, 20, $n$ spelniają warunek: iloczyn $\mathrm{k}\mathrm{a}\dot{\mathrm{z}}$ dych dwóch z nich jest podzielny przez trzecią

liczbę. Wyznacz wszystkie liczby całkowite $n$ spełniające tę własność.

ZADANIE 7.

$\mathrm{W}$ zbiorze liczb rzeczywistych rozwia $\dot{\mathrm{z}}$ uklad równań

$\left\{\begin{array}{l}
x^{2}+2y^{2}=2yz+100\\
z^{2}=2xy-100.
\end{array}\right.$






LIGA MATEMATYCZNA

$\mathrm{P}\mathrm{A}\acute{\mathrm{Z}}$ DZIERNIK 2009

SZKOLA PODSTAWOWA

ZADANIE I.

Ania zgubila sześcienną kostkę do gry i samodzielnie wykonała inną kostkę w taki sposób,

$\dot{\mathrm{z}}\mathrm{e}$ sumy oczek na parach šcianek przeciwległych tworzą trzy kolejne liczby naturalne (w typowej

kostce do gry sumy oczek na ściankach przeciwleglych są równe). Okazalo się, $\dot{\mathrm{z}}\mathrm{e}$ suma oczek

na pewnych trzech ściankach majqcych wspólny wierzchołek jest równa 14. I1e oczek jest

na ściance przeciwleglej do ścianki z trzema oczkami?

ZADANIE 2.

Czy z jedenastu kwadratów o bokach l cm, l cm, 2 cm, 2 cm, 2 cm, 3 cm, 3 cm, 4 cm, 6 cm,

6 cm, 7 cm $\mathrm{m}\mathrm{o}\dot{\mathrm{z}}$ na zbudować kwadrat?

ZADANIE 3.

Dwóch uczniów rozwiązuje dwa rebusy w ciągu dwóch minut. Ile rebusów rozwiąze 10 uczniów

w ciągu 10 minut?

ZADANIE 4.

Piszemy liczbę 0, następnie 1iczbę 1 i znowu 1, potem piszemy najmniejszą z tych 1iczb ca1ko-

witych nieujemnych, która nie wystąpiła na trzech poprzednich miejscach. Dalej postępujemy

podobnie. Jaka liczba będzie na 99 miejscu?

ZADANIE 5.

Obwody zamalowanych prostokątów są równe 24 cm i 12 cm. Ob1icz obwód prostokąta ABCD.
\begin{center}
\includegraphics[width=48.816mm,height=33.168mm]{./LigaMatematycznaMatuskiego_SP_Zestaw1_2009_2010_page0_images/image001.eps}
\end{center}
D  C

A  B






LIGA MATEMATYCZNA

$\mathrm{P}\mathrm{A}\acute{\mathrm{Z}}$ DZIERNIK 2010

SZKOLA PODSTAWOWA

ZADANIE I.

Ile jest liczb naturalnych o sumie cyfr w zapisie dziesiętnym równej 100 i i1oczynie tych cyfr

równym 5?

ZADANIE 2.

Dane są trzy figury: koło, kwadrat i trójkąt, róznej wielkości i w róznych kolorach: czerwonym,

zielonym, niebieskim. Kolo nie jest małe ani czerwone.

Trójkąt nie jest średni ani zielony.

Kwadrat nie jest duzy ani niebieski. Określ wielkość i kolor $\mathrm{k}\mathrm{a}\dot{\mathrm{z}}$ dej figury, jeśli wiadomo, $\dot{\mathrm{z}}\mathrm{e}$

mala figura jest niebieska.

ZADANIE 3.

$\mathrm{W}$ kole stanęfo 15 dziewcząt i 15 chfopców. Zaczynając od usta1onego miejsca nastąpi od1iczanie

do dziewięciu zgodnie z ruchem wskazówek zegara. Dziewiąta osoba odpada z gry, a odliczanie

będzie odbywać się dalej. Następnie kolejna dziewiąta osoba zostanie wykluczona z koła, i tak

dalej $\mathrm{a}\dot{\mathrm{z}}$ do momentu, gdy w kole zostanie 15 osób. Jak na1ezy ustawić ch1opców i dziewczęta,

aby wszystkie dziewczęta zostafy w kole?

ZADANIE 4.

Z siedmiu patyczków o dlugościach 3, 4, 6, 7, 9, 10, 11 u1óz prostokąt. Patyczków nie wo1no

łamać ani nakładać na siebie.

ZADANIE 5.

$\mathrm{Z}$ miejscowości $\mathrm{A}$, w której mieści się firma kurierska, wyjezdza kurier do miejscowości $\mathrm{B}, \mathrm{C},$

$\mathrm{D}$, aby dostarczyć przesyfki. $\mathrm{W}$ jakiej kolejności powinien objechać te miejscowości, aby trasa

objazdu z A przez pozostale miejscowości i z powrotem do A była $\mathrm{m}\mathrm{o}\dot{\mathrm{z}}$ liwie najkrótsza, jeśli

długošć drogi od A do $\mathrm{B}$ wynosi 50 km, od A do C-70 km, od A do D-70 km, od $\mathrm{B}$ do $\mathrm{D}-$

$80$ km, od $\mathrm{B}$ do C-100 km, od $\mathrm{C}$ do D-60 km?






LIGA MATEMATYCZNA

$\mathrm{P}\mathrm{A}\acute{\mathrm{Z}}$ DZIERNIK 2011

SZKOLA PODSTAWOWA

ZADANIE I.

Ile jest liczb dziesięciocyfrowych, które $\mathrm{m}\mathrm{o}\dot{\mathrm{z}}$ na napisač przy $\mathrm{u}\dot{\mathrm{z}}$ yciu cyfr 1, 2, 3 (nie trzeba wy-

korzystać wszystkich cyfr jednocześnie) tak, aby $\mathrm{k}\mathrm{a}\dot{\mathrm{z}}$ de dwie sąsiednie cyfry rózniły się ojeden?

ZADANIE 2.

$\mathrm{W}$ klasie jest 27 uczniów. $K\mathrm{a}\dot{\mathrm{z}}\mathrm{d}\mathrm{y}$ z nich uprawia przynajmniej jedną z trzech dyscyplin spor-

towych:

pilkę $\mathrm{n}\mathrm{o}\dot{\mathrm{z}}$ ną, plywanie lub tenis. Najwięcej uczniów uprawia pfywanie, a najmniej

tenis. $\mathrm{W}$ piłkę $\mathrm{n}\mathrm{o}\dot{\mathrm{z}}$ ną gra 15 uczniów. Ty1ko jeden uczeń uprawia jednocześnie trzy dyscyp1iny

sportu. Dwoje uprawia tenis i pifkę $\mathrm{n}\mathrm{o}\dot{\mathrm{z}}$ ną. Czworo uprawia plywanie i pifkę $\mathrm{n}\mathrm{o}\dot{\mathrm{z}}$ ną, troje- tenis

i pfywanie. Ilu uczniów uprawia pfywanie? Ilu uczniów uprawia tylko tenis?

ZADANIE 3.

Rozlozono sto cukierków na pięciu talerzach:

$\bullet$ na pierwszym i drugim talerzu znalazfy się 52 cukierki;

$\bullet$ na drugim i trzecim talerzu 43 cukierki;

$\bullet$ na trzecim i czwartym- 34 cukierki;

$\bullet$ na czwartym i piatym- 30 cukierków.

Ile cukierków znajdowafo się na $\mathrm{k}\mathrm{a}\dot{\mathrm{z}}$ dym talerzu?

ZADANIE 4.

$\mathrm{W}$ schronisku dla zwierząt byfa taka sama liczba psów i kotów. Trzecia część psów i pofowa

kotów znalazla opiekunów. Po sześć psów ijednego kota zgloszą się wfaściciele i wtedy w schro-

nisku będzie więcej kotów $\mathrm{n}\mathrm{i}\dot{\mathrm{z}}$ psów. Ile psów mogfo znajdować się w schronisku na początku?

Podaj wszystkie $\mathrm{m}\mathrm{o}\dot{\mathrm{z}}$ liwości.

ZADANIE 5.

Prostokąt podzielono na siedem kwadratów. Bok $\mathrm{k}\mathrm{a}\dot{\mathrm{z}}$ dego z zaciemnionych kwadratów ma dlu-

gość 8. Jaka d1ugość ma bok największego kwadratu?






LIGA MATEMATYCZNA

im. Zdzisława Matuskiego

$\mathrm{P}\mathrm{A}\acute{\mathrm{Z}}$ DZIERNIK 2012

SZKOLA PODSTAWOWA

ZADANIE I.

Sloń wazy tyle, ile dwa nosorozce, nosorozec tyle, ile dwa nied $\acute{\mathrm{z}}\mathrm{w}\mathrm{i}\mathrm{e}\mathrm{d}\mathrm{z}\mathrm{i}\mathrm{e}, \mathrm{n}\mathrm{i}\mathrm{e}\mathrm{d}\acute{\mathrm{z}}\mathrm{w}\mathrm{i}\mathrm{e}\mathrm{d}\acute{\mathrm{z}}$ tyle, ile dwa

sumy, sum wazy tyle, ile dwa tygrysy, tygrys tyle, ile dwa strusie, struś tyle, ile dwie sarny,

sarna tyle, ile dwa borsuki, borsuk tyle, ile dwa lisy, lis tyle, ile dwa zające. Sloń wazy o 6, 25 kg

więcej $\mathrm{n}\mathrm{i}\dot{\mathrm{z}}$ w sumie nosorozec, nied $\acute{\mathrm{z}}\mathrm{w}\mathrm{i}\mathrm{e}\mathrm{d}\acute{\mathrm{z}}$, sum, tygrys, struś, sarna, borsuk, lis i zając. Ile wazy

słoń?

ZADANIE 2.

Kwadrat o boku długości 9 cm rozetnij na trzy prostokąty o obwodach 20 cm, 24 cm i 28 cm.

ZADANIE 3.

Do zapisania pewnej liczby dziesięciocyfrowej uzytojednej jedynki, dwóch dwójek, trzech trójek

i czterech czwórek. Rozmieszczenie cyfr jest nieznane. Czy $\mathrm{m}\mathrm{o}\dot{\mathrm{z}}\mathrm{e}$ to być liczba pierwsza?

ZADANIE 4.

Pierwszy ślimak potrafi przejšć 3 metry w ciągu czterech minut, a drugi-4 metry w trzy minuty.

W tym samym momencie wyszli z tego samego miejsca odleglego o ll metrów od stacji leśnej

kolejki. Czy obaj zdązą, jeśli do odjazdu pociągu zostalo 13 minut?

ZADANIE 5.

Na odcinku $AB$ zaznaczono punkty $C, D$. Odleglość punktu $C$ od jednego z końców danego

odcinka stanowi $\displaystyle \frac{5}{6}$ jego długości, a odleglość punktu $D$ od jednego z końców- $\displaystyle \frac{3}{4}$ długości tego

odcinka. Wiedząc, $\dot{\mathrm{z}}\mathrm{e}$ dlugość odcinka $CD$ jest równa 35 cm, ob1icz d1ugość odcinka $AB.$

Rozwaz wszystkie $\mathrm{m}\mathrm{o}\dot{\mathrm{z}}$ liwości.






LIGA MATEMATYCZNA

im. Zdzisława Matuskiego

$\mathrm{P}\mathrm{A}\overline{\mathrm{Z}}$ DZIERNIK 2013

SZKOLA PODSTAWOWA

ZADANIE I.

Jaka największa, a jaką najmniejszą liczbę trzycyfrową $\mathrm{m}\mathrm{o}\dot{\mathrm{z}}$ emy otrzymać z liczby 18094015

przez wykreślenie pięciu cyfr bez zmiany ich porządku?

ZADANIE 2.

Jaką liczbę nalezy wpisać w górne pole, $\mathrm{j}\mathrm{e}\dot{\mathrm{z}}$ eli liczba w $\mathrm{k}\mathrm{a}\dot{\mathrm{z}}$ dym polu w rzędzie $\mathrm{w}\mathrm{y}\dot{\mathrm{z}}$ szym jest

iloczynem dwóch liczb z pól $\mathrm{n}\mathrm{i}\dot{\mathrm{z}}$ szego rzędu sąsiadujących z nim?
\begin{center}
\includegraphics[width=31.140mm,height=28.296mm]{./LigaMatematycznaMatuskiego_SP_Zestaw1_2013_2014_page0_images/image001.eps}
\end{center}
5 4

5 3

ZADANIE 3.

Bartek, Maciek i Tomek łowili ryby. Zapytani o to, ile ryb zlowili, odpowiedzieli:

$\bullet$ Bartek zlowif 22 ryby, Maciek 21;

$\bullet$ Tomek zlowif 19 ryb, a Bartek 21;

$\bullet$ Tomek zlowi121 ryb, Maciek 18.

Wiadomo, $\dot{\mathrm{z}}\mathrm{e}$ w $\mathrm{k}\mathrm{a}\dot{\mathrm{z}}$ dej odpowiedzi tylko jedna część jest prawdziwa oraz $\dot{\mathrm{z}}$ adni dwaj nie zlowili

tej samej ilości ryb. Ile ryb zfowil $\mathrm{k}\mathrm{a}\dot{\mathrm{z}}\mathrm{d}\mathrm{y}$ chlopiec?

ZADANIE 4.

Z prostokąta ABCD o obwodzie 30 cm wycięto trójkat równoboczny AOD o obwodzie 15 cm.

Oblicz obwód figury ABCDO.
\begin{center}
\includegraphics[width=60.708mm,height=30.024mm]{./LigaMatematycznaMatuskiego_SP_Zestaw1_2013_2014_page0_images/image002.eps}
\end{center}
D  c

0

A  B

ZADANIE 5.

Na lawce siedzi Ania, jej mama, babcia i dziadek. Babcia siedzi obok Ani, ale nie siedzi obok

dziadka. Dziadek nie siedzi obok mamy Ani. Kto siedzi obok mamy?






LIGA MATEMATYCZNA

im. Zdzisława Matuskiego

$\mathrm{P}\mathrm{A}\overline{\mathrm{Z}}$ DZIERNIK 2014

SZKOLA PODSTAWOWA

ZADANIE I.

Podczas kolejnych lotów trzej piloci spotkali się w Paryz $\mathrm{u}6$ września 2014 roku. Wiadomo,

$\dot{\mathrm{z}}\mathrm{e}$ pierwszy pilot lata do Paryza co 7 dni, drugi co 14, a trzeci co 5 dni. Wyznacz datę ich

kolejnego spotkania w Paryzu.

ZADANIE 2.

Pifka $\mathrm{n}\mathrm{o}\dot{\mathrm{z}}$ na i golfowa wazą razem tyle, ile fącznie wazą pilka bejsbolowa i tenisowa. Trzy pifki

golfowe wazą tyle, ile pilka bejsbolowa i tenisowa. Pilka bejsbolowa $\mathrm{w}\mathrm{a}\dot{\mathrm{z}}\mathrm{y}$ tyle, ile osiem piłek

tenisowych. Ile pifek tenisowych $\mathrm{w}\mathrm{a}\dot{\mathrm{z}}\mathrm{y}$ tyle, ile jedna pilka $\mathrm{n}\mathrm{o}\dot{\mathrm{z}}$ na?

ZADANIE 3.

Wyznacz cyfry a, b tak, aby liczba sześciocyfrowa a2479b by1a podzie1na przez 72.

ZADANIE 4.

Na $\mathrm{k}\mathrm{a}\dot{\mathrm{z}}$ dej ze ścian sześcianu zapisano jedną z liczb 1, 2, 3, 4, 5, 6 w taki sposób, $\dot{\mathrm{z}}\mathrm{e}$ suma liczb

na $\mathrm{k}\mathrm{a}\dot{\mathrm{z}}$ dej parze przeciwleglych ścian jest równa 7. $\mathrm{Z}\mathrm{k}\mathrm{a}\dot{\mathrm{z}}$ dym wierzchofkiem sześcianu związane

sa trzy ściany. Liczby zapisane na tych ścianach mnozymy. Oblicz sumę tych iloczynów.

ZADANIE 5.

Prostokąt podzielono na dziewięć mniejszych prostokątów. Obwody pięciu z nich podane sa na

rysunku. Oblicz obwód $\mathrm{d}\mathrm{u}\dot{\mathrm{z}}$ ego prostokąta.
\begin{center}
\begin{tabular}{|l|l|l|}
\hline
\multicolumn{1}{|l|}{}&	\multicolumn{1}{|l|}{$6$}&	\multicolumn{1}{|l|}{}	\\
\hline
\multicolumn{1}{|l|}{ $12$}&	\multicolumn{1}{|l|}{ $4$}&	\multicolumn{1}{|l|}{ $6$}	\\
\hline
\multicolumn{1}{|l|}{}&	\multicolumn{1}{|l|}{ $8$}&	\multicolumn{1}{|l|}{}	\\
\hline
\end{tabular}

\end{center}





LIGA MATEMATYCZNA

im. Zdzisława Matuskiego

$\mathrm{P}\mathrm{A}\dot{\mathrm{Z}}$ DZIERNIK 2015

SZKOLA PODSTAWOWA

ZADANIE I.

Pole trapezu, którego jedna podstawa jest dwa razy dluzsza od drugiej, jest równe 840 $\mathrm{c}\mathrm{m}^{2}$

Oblicz pola trójkątów, na jakie podzielila ten trapez jedna z przekątnych.

ZADANIE 2.

Wykaz, $\dot{\mathrm{z}}\mathrm{e}$ liczba 4777$\ldots$ 75 jest podzielna przez 45.
\begin{center}
\includegraphics[width=17.676mm,height=6.852mm]{./LigaMatematycznaMatuskiego_SP_Zestaw1_2015_2016_page0_images/image001.eps}
\end{center}
90 cyfr 7

ZADANIE 3.

$K\mathrm{a}\dot{\mathrm{z}}\mathrm{d}\mathrm{y}$ ufoludek ma trzy ręce. Czy 19 ufo1udków $\mathrm{m}\mathrm{o}\dot{\mathrm{z}}\mathrm{e}$ się wziąć za ręce tak, aby $\dot{\mathrm{z}}$ adna ręka

nie pozostala wolna?

ZADANIE 4.

Wszyskie figury skfadające się na $\mathrm{d}\mathrm{u}\dot{\mathrm{z}}\mathrm{y}$ prostokat są kwadratami. Pole czarnego kwadratu jest

równe 9. Ob1icz d1ugości boków $\mathrm{k}\mathrm{a}\dot{\mathrm{z}}$ dego kwadratu i dlugości boków $\mathrm{d}\mathrm{u}\dot{\mathrm{z}}$ ego prostokąta.

ZADANIE 5.

Na terenie wokól jeziora pan Piotr urządzif pole namiotowe dzieląc teren na sześć sektorów, jak

na rysunku. Pewnej niedzieli przyjechalo nadjezioro sześciu wędkarzy, którzy chcieli zamieszkać

na polu namiotowym, ale mieli następujące wymagania:

$\bullet$ Andrzej nie chce sąsiadować ani z Frankiem, ani z Bartkiem;

$\bullet$ Bartek nie chce sąsiadować ani z Andrzejem, ani z Czarkiem;

$\bullet$ Czarek nie chce sąsiadować ani z Bartkiem, ani z Damianem;

$\bullet$ Damian nie chce sąsiadować ani z Czarkiem, ani z Emilem;

$\bullet$ Emil nie chce sasiadować ani z Damianem, ani z Frankiem;

$\bullet$ Franek nie chce sąsiadować ani z Emilem, ani z Andrzejem.

Jak rozmieścić wędkarzy, aby spelnić ich $\dot{\mathrm{z}}$ yczenia?






LIGA MATEMATYCZNA

im. Zdzisława Matuskiego

$\mathrm{P}\mathrm{A}\dot{\mathrm{Z}}$ DZIERNIK 2016

SZKOLA PODSTAWOWA

ZADANIE I.

Ile jest dziesięciocyfrowych liczb nieparzystych o sumie cyfr równej 3?

ZADANIE 2.

Chlopcy i dziewczynki z klasy Ani i Bartka ustawili się wjednej linii. Na prawo od Ani jest 16

uczniów, w tym Bartek. Na lewo od Bartka jest 14 uczniów, wśród nich Ania. Pomiędzy Anią

i Bartkiem stoi 7 uczniów. I1u uczniów 1iczy ta k1asa?

ZADANIE 3.

Spośród boków czworokąta wybrano trzy i obliczono sumę ich dlugości. Taką operację prze-

prowadzono czterokrotnie, wybierając za $\mathrm{k}\mathrm{a}\dot{\mathrm{z}}$ dym razem trzy inne boki. Otrzymano sumy: 10,

13, 15, 16. Oblicz dlugości boków tego czworokata.

ZADANIE 4.

Ania, Beata, Celina i Dorota wybraly się na grzyby. Ania zebrafa trzy razy więcej grzybów

$\mathrm{n}\mathrm{i}\dot{\mathrm{z}}$ Beata, Beata trzy razy więcej $\mathrm{n}\mathrm{i}\dot{\mathrm{z}}$ Celina, Celina trzy razy więcej $\mathrm{n}\mathrm{i}\dot{\mathrm{z}}$ Dorota. Wiadomo, $\dot{\mathrm{z}}\mathrm{e}$

razem mają więcej $\mathrm{n}\mathrm{i}\dot{\mathrm{z}}50$, ale mniej $\mathrm{n}\mathrm{i}\dot{\mathrm{z}}100$ grzybów. Ile grzybów zebrafa $\mathrm{k}\mathrm{a}\dot{\mathrm{z}}$ da z dziewczynek?

ZADANIE 5.

Uzupelnij krzyzówkę tak, aby otrzymane liczby trzycyfrowe dzielily się przez podane liczby.
\begin{center}
\begin{tabular}{|l|l|l|l|}
\hline
\multicolumn{1}{|l|}{}&	\multicolumn{1}{|l|}{przez 11}&	\multicolumn{1}{|l|}{przez 7}&	\multicolumn{1}{|l|}{przez 2}	\\
\hline
\multicolumn{1}{|l|}{przez 7}&	\multicolumn{1}{|l|}{$7$}&	\multicolumn{1}{|l|}{ $1$}&	\multicolumn{1}{|l|}{}	\\
\hline
\multicolumn{1}{|l|}{przez 9}&	\multicolumn{1}{|l|}{$8$}&	\multicolumn{1}{|l|}{ $5$}&	\multicolumn{1}{|l|}{}	\\
\hline
\multicolumn{1}{|l|}{przez 5}&	\multicolumn{1}{|l|}{}&	\multicolumn{1}{|l|}{}&	\multicolumn{1}{|l|}{}	\\
\hline
\end{tabular}

\end{center}





LIGA MATEMATYCZNA

im. Zdzisława Matuskiego

$\mathrm{P}\mathrm{A}\dot{\mathrm{Z}}$ DZIERNIK 2017

SZKOLA PODSTAWOWA

ZADANIE I.

Czy liczbę 7777777 $\mathrm{m}\mathrm{o}\dot{\mathrm{z}}$ na przedstawić jako sumę dwóch liczb pierwszych?

ZADANIE 2.

Oblicz pole figury przedstawionej na rysunku, gdzie odpowiednie kąty sa proste.
\begin{center}
\includegraphics[width=138.936mm,height=44.964mm]{./LigaMatematycznaMatuskiego_SP_Zestaw1_2017_2018_page0_images/image001.eps}
\end{center}
5  7  9

4  5

6

8  2 3

1  7

5  4

3

2  6  8  2

ZADANIE 3.

Pan Adamjest staruszkiem, ale nie majeszcze stu lat. Dla treningu lubi układač i rozwiazywać

lamiglówki liczbowe. Wfaśnie zauwazyl, $\dot{\mathrm{z}}\mathrm{e}$ w ubieglym roku jego wiek byl liczbą calkowitą

podzielną przez 8, a w przyszłym roku będzie 1iczbą podzie1ną przez 7. I1e 1at ma pan Adam

obecnie?

ZADANIE 4.

Liczbą palindromiczną nazywamy liczbę, która czytana od lewej do prawej oraz od prawej do

lewej jest taka sama. Ile jest liczb palindromicznych trzycyfrowych podzielnych przez 4?

ZADANIE 5.

Wewnątrz pięciokąta foremnego ABCDE obrano punkt $F$ w taki sposób, $\dot{\mathrm{z}}\mathrm{e}$ trójkąt $ABF$ jest

równoboczny. Oblicz miarę kata $CFE.$
\begin{center}
\includegraphics[width=48.468mm,height=46.884mm]{./LigaMatematycznaMatuskiego_SP_Zestaw1_2017_2018_page0_images/image002.eps}
\end{center}
{\it D}

{\it E}

{\it F c}

{\it A}

{\it B}






LIGA MATEMATYCZNA

im. Zdzisława Matuskiego

$\mathrm{P}\mathrm{A}\dot{\mathrm{Z}}$ DZIERNIK 2018

SZKOLA PODSTAWOWA

(klasy IV - VI)

ZADANIE I.

$\mathrm{W}\mathrm{k}\mathrm{a}\dot{\mathrm{z}}$ dym wierzcholku trójkąta umieszczono pewną liczbę, a na $\mathrm{k}\mathrm{a}\dot{\mathrm{z}}$ dym boku - sumę liczb

z obu jego końców. Znajdz' liczby zapisane w wierzchołkach, $\mathrm{j}\mathrm{e}\dot{\mathrm{z}}$ eli na bokach znajdowały się

liczby 1256, 1820, 2018.

ZADANIE 2.

Trzej bracia (kazdy $\mathrm{w}\mathrm{a}\dot{\mathrm{z}}\mathrm{y}120$ kg), $\mathrm{k}\mathrm{a}\dot{\mathrm{z}}\mathrm{d}\mathrm{y}$ z $\dot{\mathrm{z}}$ oną (wazącą 60 kg) i dzieckiem (o wadze 30 kg) chcą

przeprawić się przez rzekę. Na brzegu znale $\acute{\mathrm{z}}\mathrm{l}\mathrm{i}$ lódkę o ladowności 120 kg. I1e co najmniej razy

lódka będzie musiala pokonać drogę od jednego brzegu do drugiego, aby cafa rodzina (dziewięć

osób) znalazla się na drugim brzegu rzeki? Podczas $\mathrm{k}\mathrm{a}\dot{\mathrm{z}}$ dej przeprawy w lódce musi znajdować

się przynajmniej jedna osoba dorosfa.

ZADANIE 3.

Na okręgu umieszczono cztery liczby: 2, 5, 7, 8 (w tej ko1ejności). Ruch po1ega na wstawieniu

między $\mathrm{k}\mathrm{a}\dot{\mathrm{z}}$ dą parę sasiednich liczb ich dodatniej róznicy, a następnie wymazaniu wszystkich

starych liczb. Po ilu ruchach po raz pierwszy otrzymamy same zera?

ZADANIE 4.

Znajd $\acute{\mathrm{z}}$ wszystkie liczby dwucyfrowe, których iloczyn cyfr jest liczbą pierwszą.

ZADANIE 5.

Na ponizszym rysunku przedstawiono dziesięciokąt, w którym $\mathrm{k}\mathrm{a}\dot{\mathrm{z}}$ de dwa sąsiednie boki są

prostopadfe. Dfugości niektórych boków zostafy podane. Oblicz obwód dziesięciokąta.
\begin{center}
\includegraphics[width=100.740mm,height=55.116mm]{./LigaMatematycznaMatuskiego_SP_Zestaw1_2018_2019_page0_images/image001.eps}
\end{center}
1000

52

83

45

2018






LIGA MATEMATYCZNA

LISTOPAD 2009

SZKOLA PODSTAWOWA

ZADANIE I.

Agnieszka, Michal, Romek, Adam i Jacek poszli do lasu na grzyby. Ale grzyby zbierala tylko

Agnieszka, chłopcom nie chciało się trudzić. Wracając do domu, dziewczynka wszystkie zna-

lezione grzyby - miala ich $42 -$ rozdzielila między chłopców. $\mathrm{W}$ drodze powrotnej Michał

znalazł jeszcze dwa grzyby, Romek zgubil dwa grzyby, Adam znalazł $\mathrm{a}\dot{\mathrm{z}}$ połowę tej ilošci grzy-

bów, którą mial w koszyku, za to Jacek zgubil polowę swoich grzybów. Po powrocie do domu

chłopcy policzyli grzyby i okazało się, $\dot{\mathrm{z}}\mathrm{e}\mathrm{k}\mathrm{a}\dot{\mathrm{z}}\mathrm{d}\mathrm{y}$ z nich miał ich jednakowq ilość. Ile grzybów

$\mathrm{k}\mathrm{a}\dot{\mathrm{z}}\mathrm{d}\mathrm{y}$ z chłopców dostal od Agnieszki?

ZADANIE 2.

Mamy 24 beczki ojednakowej objętości. Pięć z nich jest pe1nych wody, jedenaście napełnionych

do połowy, a osiem pustych. Jaką maksymalną ilość osób $\mathrm{m}\mathrm{o}\dot{\mathrm{z}}$ na obdzielić, nie przelewając wody,

tak, aby $\mathrm{k}\mathrm{a}\dot{\mathrm{z}}$ da dostafa jednakową ilość wody i tę samą ilość beczek?

ZADANIE 3.

W szkolnych zawodach sportowych Adam startowaf w skoku w dal i w końcowej klasyfikacji

zajqł siódme miejsce. Jego kolega Marcin miał dalsze skoki i ostatecznie zająl miejsce dokladnie

w środku tabeli wyników (tzn. wyprzedzało go tylu zawodników, ilu on poprzedzal).

Inny

kolega Adama mial gorsze wyniki i zajqł dziesiqte miejsce w tabeli. Ilu chlopców startowalo

w skoku w dal?

ZADANIE 4.

Wpisz liczby w miejsce liter tak, aby zachodzify wszystkie równości. Róznym literom odpowia-

dają rózne liczby.

$M\cdot A=T-E=M$: $A=T$: $Y=K-A.$

ZADANIE 5.

Trzy rózne cyfry $\mathrm{m}\mathrm{o}\dot{\mathrm{z}}$ na wpisać do pustych pól diagramu na sześć róznych sposobów. Jakie

to powinny być cyfry, aby w $\mathrm{k}\mathrm{a}\dot{\mathrm{z}}$ dym z tych sześciu przypadków otrzymać pięciocyfrową liczbę

podzielną przez 12?

1 2






LIGA MATEMATYCZNA

LISTOPAD 2010

SZKOLA PODSTAWOWA

ZADANIE I.

Jakie wymiary, będące liczbami calkowitymi, powinna mieć prostokątna kartka papieru o polu

powierzchni 112 $\mathrm{c}\mathrm{m}^{2}$, aby $\mathrm{m}\mathrm{o}\dot{\mathrm{z}}$ na bylo z niej wyciąć jak najwięcej kwadratów o wymiarach

całkowitych i róznych polach?

ZADANIE 2.

Podziel kwadrat $4\times 4$ przedstawiony na rysunku na cztery jednakowe części tak, aby $\mathrm{k}\mathrm{a}\dot{\mathrm{z}}$ da

litera była w innej części.
\begin{center}
\includegraphics[width=33.276mm,height=33.228mm]{./LigaMatematycznaMatuskiego_SP_Zestaw2_2010_2011_page0_images/image001.eps}
\end{center}
A

C D

ZADANIE 3.

$\mathrm{W}$ klasie IV bjest 29 uczniów. 18 uczniów ma brata, 17 uczniów ma siostrę. Ty1ko Zosia, Micha1

i Tomek nie mają $\dot{\mathrm{z}}$ adnego rodzeństwa. Ilu uczniów ma brata i siostrę? $(\dot{\mathrm{Z}}$ aden z uczniów nie

ma więcej $\mathrm{n}\mathrm{i}\dot{\mathrm{z}}$ dwie osoby rodzeństwa.)

ZADANIE 4.

Ala, Ela, Jola, Ola, Tola i Ula mieszkają w czteropiętrowym bloku. Ala mieszka $\mathrm{w}\mathrm{y}\dot{\mathrm{z}}$ ej $\mathrm{n}\mathrm{i}\dot{\mathrm{z}}$ Ela,

ale $\mathrm{n}\mathrm{i}\dot{\mathrm{z}}$ ej $\mathrm{n}\mathrm{i}\dot{\mathrm{z}}$ Jola. Ola i Tola mieszkają $\mathrm{n}\mathrm{i}\dot{\mathrm{z}}$ ej $\mathrm{n}\mathrm{i}\dot{\mathrm{z}}$ Ula. Ola mieszka $\mathrm{w}\mathrm{y}\dot{\mathrm{z}}$ ej $\mathrm{n}\mathrm{i}\dot{\mathrm{z}}$ Ala, a Tola $\mathrm{w}\mathrm{y}\dot{\mathrm{z}}$ ej

$\mathrm{n}\mathrm{i}\dot{\mathrm{z}}$ Jola. Która z dziewczynek mieszka na pierwszym piętrze?

ZADANIE 5.

Uzupelnij puste kratki liczbami w taki sposób, aby suma $\mathrm{k}\mathrm{a}\dot{\mathrm{z}}$ dej czwórki kolejnych liczb była

równa 20.






LIGA MATEMATYCZNA

LISTOPAD 2011

SZKOLA PODSTAWOWA

ZADANIE I.

Automat matematyczny dziala na następującej zasadzie: do danej liczby dodaje jeden lub

ją podwaja. Do automatu wprowadzono liczbę 0. Ten po wykonaniu pewnej 1iczby operacji

otrzymał liczbę 100. Jaka jest najmniejsza i1ość operacji, które musi wykonać automat, aby

otrzymač taki wynik?

ZADANIE 2.

$\mathrm{W}$ smoczej jamie $\dot{\mathrm{z}}$ yfy smoki czerwone i smoki zielone. $K\mathrm{a}\dot{\mathrm{z}}\mathrm{d}\mathrm{y}$ czerwony smok mial sześć gfów,

osiem nóg i dwa ogony, natomiast zielony smok mial osiem gfów, sześć nóg i cztery ogony.

Wszystkich ogonów byfo 44, a zie1onych nóg o 6 mniej $\mathrm{n}\mathrm{i}\dot{\mathrm{z}}$ czerwonych glów. Ile czerwonych

smoków $\dot{\mathrm{z}}$ ylo w tej jamie?

ZADANIE 3.

Jaś pomyślal pewną liczbę naturalną, pomnozyl ją przez 13, odrzuci1 ostatnią cyfrę wyniku,

otrzymaną liczbę pomnozyf przez 7, znów odrzuci1 ostatnią cyfrę wyniku i otrzyma121. Jaką

liczbę pomyślaf Jasiu?

ZADANIE 4.

Ile róznych prostokątów $\mathrm{m}\mathrm{o}\dot{\mathrm{z}}$ na zbudować z patyczków o dlugościach 3 cm, 5 cm, 8 cm, 10 cm,

ll cm, 13 cm, 14 cm? Patyczki tworzace boki prostokąta nie mogą zachodzić na siebie i nie

$\mathrm{m}\mathrm{o}\dot{\mathrm{z}}$ na ich łamać. Za $\mathrm{k}\mathrm{a}\dot{\mathrm{z}}$ dym razem nalezy wykorzystać wszystkie patyczki.

ZADANIE 5.

Do sklepu przywieziono 223 kg cukierków w pojemnikach 10 kg i 17 kg. I1e by1o pojemników?






LIGA MATEMATYCZNA

im. Zdzisława Matuskiego

LISTOPAD 2012

SZKOLA PODSTAWOWA

ZADANIE I.

Pasterz Matmek wędrując ze stadem 10 owiec trafił na most strzezony przez straznika Kwa-

dratko. Przepuszczał on ludzi przez most za darmo, a za owce pobierał opłatę w dukatach,

których ilošć musiafa być równa liczbie owiec podniesionej do kwadratu. Ubogi pasterz nie

mial stu dukatów, ale wytargowal $\mathrm{u}$ straznika, $\dot{\mathrm{z}}\mathrm{e}$ będzie przeprowadzał po kilka owiec, płacąc

za $\mathrm{k}\mathrm{a}\dot{\mathrm{z}}$ dą część stada oddzielnie. Straznik zgodzil się na to pod warunkiem, $\dot{\mathrm{z}}\mathrm{e}$ stado będzie

podzielone nie więcej $\mathrm{n}\mathrm{i}\dot{\mathrm{z}}$ na trzy części. $\mathrm{W}$ jaki sposób Matmek ma przeprowadzić owce przez

most, aby zapłacić jak najmniej?

ZADANIE 2.

Za pomocq trzech róznych cyfr parzystych zapisz wszystkie liczby czterocyfrowe niepodzielne

przez 4 i mające sumę cyfr równą 26.

ZADANIE 3.

Kwadrat o obwodzie 24 cm rozetnij na trzy prostokąty, z których $\mathrm{m}\mathrm{o}\dot{\mathrm{z}}$ na złozyć prostokąt

o obwodzie 26 cm.

ZADANIE 4.

Wojtek napisał na kartce pewną liczbę naturalną. Następnie dopisal do niej dwa zera. Do tak

zmienionej liczby doda115, a następnie podzie1ił ją przez 5. Od wyniku dzie1enia odjąf 3

i w otrzymanej liczbie skreślił cyfrę jedności. Uzyskany wynik podzielił przez 2 i z dumą

napisal rezultat: 2012. Jaka 1iczbę Wojtek zapisał na początku?

ZADANIE 5.

Wpisz do diagramu wszystkie cyfry od l do 9 tak, aby trzy 1iczby powsta1e w ko1umnach

(czytane z góry na dól) byfy podzielne przez 3, a1e $\dot{\mathrm{z}}$ adna liczba trzycyfrowa czytana w wierszach

nie była podzielna przez 3.
\begin{center}
\includegraphics[width=20.268mm,height=20.832mm]{./LigaMatematycznaMatuskiego_SP_Zestaw2_2012_2013_page0_images/image001.eps}
\end{center}





LIGA MATEMATYCZNA

im. Zdzisława Matuskiego

LISTOPAD 2013

SZKOLA PODSTAWOWA

ZADANIE I.

Oblicz sumę liczb w pustych polach diagramu, $\mathrm{j}\mathrm{e}\dot{\mathrm{z}}$ eli liczba w $\mathrm{k}\mathrm{a}\dot{\mathrm{z}}$ dym polu w rzędzie $\mathrm{w}\mathrm{y}\dot{\mathrm{z}}$ szym

jest sumą dwóch liczb z $\mathrm{n}\mathrm{i}\dot{\mathrm{z}}$ szego rzędu sąsiadujacych z nim.
\begin{center}
\includegraphics[width=29.616mm,height=26.868mm]{./LigaMatematycznaMatuskiego_SP_Zestaw2_2013_2014_page0_images/image001.eps}
\end{center}
96

52

20

ZADANIE 2.

Pięciu chlopców $\mathrm{w}\mathrm{a}\dot{\mathrm{z}}$ ylo się parami $\mathrm{k}\mathrm{a}\dot{\mathrm{z}}\mathrm{d}\mathrm{y}$ z $\mathrm{k}\mathrm{a}\dot{\mathrm{z}}$ dym. Otrzymano następujqce rezultaty: 90 kg,

92 kg, 93 kg, 94 kg, 95 kg, 96 kg, 97 kg, 98 kg, l00 kg, l0l kg. Podaj lączna wagę tych

chlopców.

ZADANIE 3.

Obwód prostokąta zbudowanego z dwudziestu jednakowych kwadratów jest równy 126. Ob1icz

pole prostokąta. Rozwaz wszystkie $\mathrm{m}\mathrm{o}\dot{\mathrm{z}}$ liwości.

ZADANIE 4.

Ogrodnik wlozy180 gruszek do 12 koszyków w taki sposób, $\dot{\mathrm{z}}\mathrm{e}$ w $\mathrm{k}\mathrm{a}\dot{\mathrm{z}}$ dym koszyku znalazla się

co najmniej jedna gruszka. Czy jest $\mathrm{m}\mathrm{o}\dot{\mathrm{z}}$ liwe, $\dot{\mathrm{z}}\mathrm{e}$ w $\mathrm{k}\mathrm{a}\dot{\mathrm{z}}$ dym koszyku znajduje się inna liczba

gruszek?

ZADANIE 5.

$\mathrm{W}$ upalny dzień w kawiarni usiedli Ania, Antek, Bartek, Czarek i Darek. Wszyscy zamówili

zimne napoje i $\mathrm{k}\mathrm{a}\dot{\mathrm{z}}\mathrm{d}\mathrm{y}$ zamówif coś innego. Ustal kto zamówif jaki napój, $\mathrm{j}\mathrm{e}\dot{\mathrm{z}}$ eli

$\bullet$ Antek jako jedyny lubi oranzadę;

$\bullet$ Bartek nie lubi napojów gazowanych;

$\bullet$ Czarek nie lubi coca-coli ani wody gazowanej;

$\bullet$ Ania pije tylko wodę gazowaną;

$\bullet$ zamówiono $\mathrm{t}\mathrm{e}\dot{\mathrm{z}}$ fantę i wodę niegazowaną.






LIGA MATEMATYCZNA

im. Zdzisława Matuskiego

LISTOPAD 2014

SZKOLA PODSTAWOWA

ZADANIE I.

Oblicz sumę cyfr liczby $10^{50}-2014.$

ZADANIE 2.

Do trzech samochodów nalez $\mathrm{y}$ zapakować pięć pojemników wypefnionych do pefna farbą, pięć

takich pojemników wypelnionych farbą do polowy i pięć pustych pojemników. Nalezy je zapa-

kować tak, aby $\mathrm{k}\mathrm{a}\dot{\mathrm{z}}\mathrm{d}\mathrm{y}$ samochód zostal jednakowo obciązony. Jak to zrobič?

ZADANIE 3.

$\mathrm{W}$ pudelku są drewniane klocki w trzech róznych kolorach. Klocków niebieskich i zielonych

razem jest 50, zie1onych i czerwonych fącznie jest 40, niebieskich i czerwonych- 68. Ob1icz i1e

jest klocków $\mathrm{k}\mathrm{a}\dot{\mathrm{z}}$ dego koloru.

ZADANIE 4.

$\mathrm{Z}500$ kwadratów o obwodzie 20 cm $\mathrm{k}\mathrm{a}\dot{\mathrm{z}}$ dy, ulozono prostokąt, którego dlugość jest pięć razy

większa od szerokości. Wyznacz pole i obwód otrzymanego prostokąta.

ZADANIE 5.

$K\mathrm{a}\dot{\mathrm{z}}$ dą z liczb 1, 2, 3, 4, 5 na1ezy wpisać w wo1ne po1a figury przedstawionej na rysunku tak,

aby sumy liczb w wierszu i kolumnach byly takie same. Na ile sposobów $\mathrm{m}\mathrm{o}\dot{\mathrm{z}}$ na to zrobić?
\begin{center}
\includegraphics[width=30.684mm,height=30.684mm]{./LigaMatematycznaMatuskiego_SP_Zestaw2_2014_2015_page0_images/image001.eps}
\end{center}
7

6






LIGA MATEMATYCZNA

im. Zdzisława Matuskiego

LISTOPAD 2015

SZKOLA PODSTAWOWA

ZADANIE I.

Liczby 49, 29, 9, 40, 22, 15, 53, 33, 13, 47 połączono w pary tak, $\dot{\mathrm{z}}\mathrm{e}$ suma liczb w $\mathrm{k}\mathrm{a}\dot{\mathrm{z}}$ dej parze

jest taka sama. Która z liczb stanowi parę z liczbą 15?

ZADANIE 2.

Wykaz, $\dot{\mathrm{z}}\mathrm{e}$ liczba 2555$\ldots$ 52 jest podzielna przez l2.
\begin{center}
\includegraphics[width=17.676mm,height=6.804mm]{./LigaMatematycznaMatuskiego_SP_Zestaw2_2015_2016_page0_images/image001.eps}
\end{center}
100 cyfr 5

ZADANIE 3.

$\mathrm{W}$ skarbcu odkrytym przez Ali-Babę bylo 15 worków z monetami. Wiadomo, $\dot{\mathrm{z}}\mathrm{e}$ w jednym

worku wszystkie monety są fałszywe. Prawdziwa moneta wazy 20 gramów, a fałszywa 19 gra-

mów. Ali-Baba ma bardzo dokladną wagę, dzięki której $\mathrm{m}\mathrm{o}\dot{\mathrm{z}}\mathrm{e}$ stwierdzić, ile wazy konkretny

obiekt. Jak za pomocą jednego $\mathrm{w}\mathrm{a}\dot{\mathrm{z}}$ enia odkryć, w którym worku są fałszywe monety?

ZADANIE 4.

$\mathrm{Z}$ kwadratu wycinamy w rogach cztery kwadratowe kawałki. Ich boki mają dlugošć odpowiednio

l cm, 2 cm, 3 cm, 6 cm. Po ich wycięciu po1e figury zmniejszy1o się dwukrotnie. Wyznacz

obwód powstałej figury.

ZADANIE 5.

Za pomocą czterech czwórek, wpisując między nie znaki matematyczne (dozwolone są: doda-

wanie, odejmowanie, mnozenie, dzielenie, pierwiastkowanie i nawiasy) zapisz liczby od 0 do 10.

Powinno powstać jedenaście róznych zapisów z czterema czwórkami.






LIGA MATEMATYCZNA

im. Zdzisława Matuskiego

LISTOPAD 2016

SZKOLA PODSTAWOWA

ZADANIE I.

Adam i Bartek załozyli się o jedną czekoladę. $\mathrm{J}\mathrm{e}\dot{\mathrm{z}}$ eli Adam wygra zakład, to będzie miał trzy

razy tyle czekolad, co Bartek. $\mathrm{J}\mathrm{e}\dot{\mathrm{z}}$ eli Adam przegra, to będzie mial tylko dwa razy więcej

czekolad $\mathrm{n}\mathrm{i}\dot{\mathrm{z}}$ Bartek. Ile czekolad mial $\mathrm{k}\mathrm{a}\dot{\mathrm{z}}\mathrm{d}\mathrm{y}$ z nich na początku?

ZADANIE 2.

$\mathrm{W}$ kwadracie o polu 64 wybrano punkt $M$ i połączono go ze wszystkimi wierzcholkami. Po-

wstaly w ten sposób cztery trójkaty, z których jeden ma pole 12, a inny ma po1e 24. Podaj

odleglości punktu $M$ od wszystkich boków kwadratu.

ZADANIE 3.

Iloczyn dwóch liczb dwucyfrowych jest równy 525. Zaokrąg1ono te 1iczby do pe1nych dziesiątek.

Iloczyn tych zaokrągleń jest równy 600. Znajd $\acute{\mathrm{z}}$ początkowe liczby.

ZADANIE 4.

Ile jest liczb dziesięciocyfrowych o sumie cyfr równej 3?

ZADANIE 5.

Obwód największego z narysowanych kwadratów jest równy 80, a po1e najmniejszego - 16.

Oblicz obwód całej figury oraz pola kwadratów $B\mathrm{i}C.$
\begin{center}
\includegraphics[width=34.236mm,height=54.300mm]{./LigaMatematycznaMatuskiego_SP_Zestaw2_2016_2017_page0_images/image001.eps}
\end{center}
{\it C}






LIGA MATEMATYCZNA

im. Zdzisława Matuskiego

LISTOPAD 2017

SZKOLA PODSTAWOWA

ZADANIE I.

Mikołaj zrobil sok malinowy na zimę i wlał go do trzylitrowych butelek. Potem jednak posta-

nowil przelać sok do pięciolitrowych slojów. Okazalo się, $\dot{\mathrm{z}}\mathrm{e}$ jedenaście slojów to za malo, więc

wlaf sok po równo do dwunastu takich slojów, chociaz teraz nie są one pelne. Ile soku jest

w $\mathrm{k}\mathrm{a}\dot{\mathrm{z}}$ dym sloju?

ZADANIE 2.

$\mathrm{W}$ trójkącie równoramiennym wysokości poprowadzone do ramion przecinaja się pod kątem

o mierze $100^{\mathrm{o}}$ Oblicz miary kątów wewnętrznych trójkąta.

ZADANIE 3.

$\mathrm{W}$ pewnym bloku w Slupsku jest sto mieszkań ponumerowanych liczbami od l do 100. $\mathrm{W}\mathrm{k}\mathrm{a}\dot{\mathrm{z}}$-

dym z nich mieszka jedna, dwie lub trzy osoby. Lączna liczba osób zamieszkujących lokale

od l do 55 jest równa 105, a fączna 1iczba mieszkańców 1oka1i od 51 do 100 to 150. I1e osób

mieszka w tym budynku?

ZADANIE 4.

$\mathrm{W}$ pewnej klasie szkoly podstawowej wszyscy uczniowie mają tyle samo lat, z wyjątkiem dwóch,

którzy są o rok starsi, ijednego, który ma o rok mniej. $\mathrm{J}\mathrm{e}\dot{\mathrm{z}}$ eli dodamy lata wszystkich uczniów,

to otrzymamy 208. I1u uczniów jest w tej k1asie?

ZADANIE 5.

$\mathrm{U}\dot{\mathrm{z}}$ ywając $\mathrm{k}\mathrm{a}\dot{\mathrm{z}}$ dej z cyfr 0, 1, 2, 3, $\ldots$, 9 tylko raz, uzupefnij diagram tak, aby dwie liczby czte-

rocyfrowe czytane poziomo byly podzielne przez 3, a dwie 1iczby trzycyfrowe czytane pionowo

byly podzielne przez 4. Uzasadnij podzie1ność.






LIGA MATEMATYCZNA

im. Zdzisława Matuskiego

LISTOPAD 2018

SZKOLA PODSTAWOWA

(klasy IV - VI)

ZADANIE I.

Ile jest róznych prostokątów, których dlugości boków wyraz $\mathrm{a}\mathrm{j}\mathrm{a}$ się cafkowitą liczbą centyme-

trów, a pole jest równe 2002 $\mathrm{c}\mathrm{m}^{2}$?

ZADANIE 2.

Znajd $\acute{\mathrm{z}}$ najmniejsza liczbę calkowitą dodatnią, która w zapisie dziesiętnym ma tylko 0 $\mathrm{i}1$ oraz

jest podzielna przez 225.

ZADANIE 3.

$\mathrm{W}$ pewnym dziewięciopiętrowym bloku w Slupsku na $\mathrm{k}\mathrm{a}\dot{\mathrm{z}}$ dym poziomie znajdują się trzy miesz-

kania. $\mathrm{W}\dot{\mathrm{z}}$ adnym mieszkaniu nie mieszka więcej $\mathrm{n}\mathrm{i}\dot{\mathrm{z}}$ troje dzieci. Na $\mathrm{k}\mathrm{a}\dot{\mathrm{z}}$ dym piętrze mieszka

inna liczba dzieci. Ile dzieci mieszka w tym bloku?

ZADANIE 4.

Ze 123 czerwonych i 123 białych sześcianików o krawędzi o długości 1 cm budujemy sześciany

o krawędzi dłuzszej $\mathrm{n}\mathrm{i}\dot{\mathrm{z}}1$ cm tak, aby $\dot{\mathrm{z}}$ adne dwa nie byly tego samego rozmiaru i by powierzch-

nia $\mathrm{k}\mathrm{a}\dot{\mathrm{z}}$ dego sześcianu byfa jednokolorowa. Ile najwięcej sześcianów $\mathrm{m}\mathrm{o}\dot{\mathrm{z}}\mathrm{e}$ powstać? Nie trzeba

wykorzystać wszystkich klocków.

ZADANIE 5.

Sześciokąt, w którym wszystkie kąty mają miarę $120^{\mathrm{o}}$, wpisano w trójkat tak, jak na rysunku.

Wyznacz dlugości boków tego trójkąta.






LIGA MATEMATYCZNA

GRUD Z$\mathrm{I}\mathrm{E}\acute{\mathrm{N}}$ 2009

SZKOLA PODSTAWOWA

ZADANIE I.

Monia, Tonia, Ponia i Sonia są kolezankami. Dwie z nich są rówieśniczkami. Monia bylaby

starsza od Toni, gdyby nie była młodsza od Soni. Ponia byłaby młodsza od Toni, gdyby nie

byla starsza od Soni. Kto jest rówieśniczką Toni: Ponia, Monia czy Sonia?

ZADANIE 2.

Przez stację kolejową przejechały trzy pociągi wojskowe. $\mathrm{W}$ pierwszym było 462 $\dot{\mathrm{z}}$ ofnierzy,

w drugim 546, w trzecim 630. Czy $\mathrm{m}\mathrm{o}\dot{\mathrm{z}}$ na obliczyć z ilu wagonów składał się $\mathrm{k}\mathrm{a}\dot{\mathrm{z}}\mathrm{d}\mathrm{y}$ z pocią-

gów, $\mathrm{j}\mathrm{e}\dot{\mathrm{z}}$ eli wiadomo, $\dot{\mathrm{z}}\mathrm{e}$ w $\mathrm{k}\mathrm{a}\dot{\mathrm{z}}$ dym wagonie była jednakowa ilość $\dot{\mathrm{z}}$ ofnierzy i $\dot{\mathrm{z}}\mathrm{e}$ ta liczba była

największa ze wszystkich $\mathrm{m}\mathrm{o}\dot{\mathrm{z}}$ liwych?

ZADANIE 3.

Figura przedstawiona na rysunku sklada się z siedmiu kwadratów. Dlugości boków dwóch spo-

šród tych kwadratów zostały podane. Iloma kwadratami typu $B\mathrm{m}\mathrm{o}\dot{\mathrm{z}}$ na wypelnić kwadrat $A$?
\begin{center}
\includegraphics[width=42.168mm,height=34.140mm]{./LigaMatematycznaMatuskiego_SP_Zestaw3_2009_2010_page0_images/image001.eps}
\end{center}
A

B

2

3

ZADANIE 4.

Prostokqt podzielono na cztery mniejsze prostokąty. Pola trzech z nich są równe odpowiednio

3, 4, 5. Jakie jest pole czwartego prostokąta?
\begin{center}
\begin{tabular}{|l|l|}
\hline
\multicolumn{1}{|l|}{$3$}&	\multicolumn{1}{|l|}{ $4$}	\\
\hline
\multicolumn{1}{|l|}{ $\mathrm{x}$}&	\multicolumn{1}{|l|}{ $5$}	\\
\hline
\end{tabular}

\end{center}
ZADANIE 5.

Mamy trzy rodzaje patyczków: 16 patyczków o d1ugości 1 cm, 15 patyczków o długości 2 cm

$\mathrm{i} 15$ patyczków o długošci 3 cm. Czy $\mathrm{m}\mathrm{o}\dot{\mathrm{z}}$ na zbudować ze wszystkich patyczków prostokąt?

Patyczki tworzące boki prostokąta nie mogą zachodzić na siebie i nie $\mathrm{m}\mathrm{o}\dot{\mathrm{z}}$ na ich lamać.






LIGA MATEMATYCZNA

GRUD Z$\mathrm{I}\mathrm{E}\acute{\mathrm{N}}$ 2010

SZKOLA PODSTAWOWA

ZADANIE I.

Gwarno będzie w šwięta $\mathrm{u}$ dziadków, gdy zjadą się dzieci z wlasnym potomstwem. Które dzieci

sq czyje, ješli:

$\bullet$ Barbara ma więcej dzieci $\mathrm{n}\mathrm{i}\dot{\mathrm{z}}$ brat;

$\bullet$ Piotrek i Oleńka mówią do Jerzego- wujku, a Magda do Haliny- ciociu;

$\bullet$ Misia nie jest siostrą ani Grzesia, ani Arka, który ma troje rodzeństwa:

i dwie siostry;

brata Mačka

$\bullet$ Halina ma córkę i syna.

ZADANIE 2.

Zepsuty kalkulator nie wyświetla cyfry 5. Na przyk1ad, jeś1i napiszemy 1iczbę 3535, to pokazuje

on liczbę 33 bez $\dot{\mathrm{z}}$ adnych odstępów między cyframi. Michał napisaf na tym kalkulatorze pewną

liczbę sześciocyfrową i na wyświetlaczu kalkulatora pojawiła się liczba 2010. D1a i1u 1iczb mog1o

się tak zdarzyć?

ZADANIE 3.

Lączna pojemność butelki i szklanki jest równa pojemnošci dzbanka. Pojemność butelki jest

równa lącznej pojemności szklanki i kufla. Lączna pojemność trzech kufii jest równa łącznej

pojemności dwóch dzbanków. Ile szklanek ma łączną pojemność jednego kufla?

ZADANIE 4.

Osiem zer i osiemjedynek ustaw w tablicy $4\times 4$ tak, aby sumy liczb w $\mathrm{k}\mathrm{a}\dot{\mathrm{z}}$ dym wierszu i w $\mathrm{k}\mathrm{a}\dot{\mathrm{z}}$ dej

kolumnie były nieparzyste.

ZADANIE 5.

$\mathrm{W}$ pomieszczeniu, które ma ksztalt prostopadlościanu, osiem pająków mieszka w ośmiu na-

$\mathrm{r}\mathrm{o}\dot{\mathrm{z}}$ ach. Jeden z nich postanowił odwiedzić wszystkich swoich kolegów, a następnie wrócić

do swojego naroznika wybierajqc drogę wzdluz krawędzi. Odleglości do najblizszych sąsiadów

są równe odpowiednio 4 $\mathrm{m}, 6\mathrm{m}, 8\mathrm{m}. \mathrm{W}$ jakiej kolejności pająk powinien odwiedzić swoich

s$\mathfrak{B}$iadów, aby jego spacer był najkrótszy? Jak długi będzie ten spacer?






LIGA MATEMATYCZNA

GRUD Z$\mathrm{I}\mathrm{E}\acute{\mathrm{N}}$ 2011

SZKOLA PODSTAWOWA

ZADANIE I.

Dziewięciu Mikofajów w 30 minut rozdaje 60 prezentów.

w ciągu trzech godzin?

Ile prezentów rozda 36 Mikofajów

ZADANIE 2.

$K\mathrm{a}\dot{\mathrm{z}}\mathrm{d}\mathrm{y}$ uczeń pewnej klasy interesuje się matematyka, historią lub geografią. Tylko jednego

ucznia pasjonują wszystkie te dziedziny nauki. Matematykę i geografię zglębia troje uczniów.

Tych, którzy nie lubią matematyki, ale poszerzają swoją wiedzę historyczną jest dziesięcioro.

Geografią i historia interesuje się pięcioro. Zapalonych matematyków jest 19. Historia i ma-

tematyka to ulubione przedmioty ośmiorga. Ilu jest milośników geografii, skoro wszystkich

uczniów jest 36?

ZADANIE 3.

Ulóz kwadrat z trzech kwadratów o boku l, trzech kwadratów o boku 2, dwóch kwadratów

o boku 3 i jednego o boku 4.

ZADANIE 4.

$\mathrm{W}$ trzech jednakowych puszkach znajduje się mleko, cukier i sól. Niestety, pomylono nalepki

i $\dot{\mathrm{z}}$ adna nie opisuje poprawnie zawartości puszki. Potrząsajac tylko jedną z nich ustal, co zawiera

$\mathrm{k}\mathrm{a}\dot{\mathrm{z}}$ da puszka.

ZADANIE 5.

Róznica liczby sześciocyfrowej i liczby pięciocyfrowej jest równa 6.

wszystkie rozwiązania.

Wyznacz te liczby. Podaj






LIGA MATEMATYCZNA

im. Zdzisława Matuskiego

GRUD Z$\mathrm{I}\mathrm{E}\acute{\mathrm{N}}$ 2012

SZKOLA PODSTAWOWA

ZADANIE I.

Mikołaj złowil karpie. Wypuścił jednego do rzeki, a połowę pozostalych dal Adamowi. Potem

znów wypuścił jednego karpia i połowę pozostalych ofiarował Ewie. Zostalo mu jeszcze 6 karpi.

Ile karpi zlowil Mikofaj?

ZADANIE 2.

Obwód trójkąta równoramiennego jest równy 56. $\acute{\mathrm{S}}$ rodek jednego z ramion polączono z wierz-

chołkiem przeciwleglego kąta. Powstaly w ten sposób dwa nowe trójkąty, z których jeden (ten,

który zawiera podstawę trójkąta równoramiennego) ma obwód o 10 krótszy $\mathrm{n}\mathrm{i}\dot{\mathrm{z}}$ drugi. Oblicz

długošci boków trójkqta równoramiennego.

ZADANIE 3.

Za siedmioma górami, za siedmioma lasami, za siedmioma morzami rosła czarodziejska wierzba.

$\mathrm{Z}$ jej grubego pnia wyrastala pewna liczba konarów. $\mathrm{Z}\mathrm{k}\mathrm{a}\dot{\mathrm{z}}$ dego konara wyrastalo tyle galęzi,

ile było wszystkich konarów. Na $\mathrm{k}\mathrm{a}\dot{\mathrm{z}}$ dej galęzi rosło dwa razy więcej magicznych gruszek $\mathrm{n}\mathrm{i}\dot{\mathrm{z}}$

było wszystkich galęzi na tym drzewie. Magicznych gruszek bylo 1250. I1e konarów wyrasta1o

z pnia czarodziejskiej wierzby?

ZADANIE 4.

$\mathrm{W}$ prawej i lewej kieszeni Karol mial lącznie 38 monet. $\mathrm{J}\mathrm{e}\dot{\mathrm{z}}$ eli przefozy z prawej do lewej

kieszeni tyle monet, ile jest w lewej, a następnie z lewej do prawej tyle monet, ile będzie

w prawej kieszeni po pierwszym przełozeniu, to w prawej będzie miał o 2 monety więcej $\mathrm{n}\mathrm{i}\dot{\mathrm{z}}$

w lewej kieszeni. Ile monet miał Karol na początku w $\mathrm{k}\mathrm{a}\dot{\mathrm{z}}$ dej kieszeni?

ZADANIE 5.

Liczbę naturalną nazywamy palindromiczną, $\mathrm{j}\mathrm{e}\dot{\mathrm{z}}$ eli jej zapis dziesiętny czytany od lewej strony

do prawej jest taki sam, jak czytany od prawej strony do lewej. Podaj wszystkie pary liczb

pięciocyfrowych palindromicznych, których róznica jest równa ll.






LIGA MATEMATYCZNA

im. Zdzisława Matuskiego

GRUD Z$\mathrm{I}\mathrm{E}\acute{\mathrm{N}}$ 2013

SZKOLA PODSTAWOWA

ZADANIE I.

Kwadrat o boku dfugości 10 cm podzie1ono na mniejszy kwadrat i cztery jednakowe prostokąty.

$K\mathrm{a}\dot{\mathrm{z}}$ da z pięciu części ma taki sam obwód. Oblicz pole malego kwadratu.

ZADANIE 2.

Liczbę naturalną nazywamy dobrą, $\mathrm{j}\mathrm{e}\dot{\mathrm{z}}$ eli $\mathrm{m}\mathrm{o}\dot{\mathrm{z}}$ na ją zapisač przy pomocy róznych cyfr, których

iloczyn jest równy 360. Podaj co najmniej dwie takie 1iczby natura1ne. Wyznacz największą

dobrą liczbę naturalną.

ZADANIE 3.

W pięciu rzutach kostką do gry otrzymano 27 punktów. I1e razy maksyma1nie mog1o wypaść 6

oczek? Ile razy mogfo wypaść 5 oczek?

ZADANIE 4.

Do sklepu dostarczono $\mathrm{g}\mathrm{w}\mathrm{o}\acute{\mathrm{z}}$dzie w sześciu skrzynkach. $\mathrm{G}\mathrm{w}\mathrm{o}\acute{\mathrm{z}}$dzie $\mathrm{w}\mathrm{a}\dot{\mathrm{z}}$ yly 2 kg, 3 kg, 5 kg, 8 kg,

9 kg i l0 kg. Którą skrzynkę sprzedawca powinien sprzedać panu Nowakowi, aby pozostale 5

skrzynek $\mathrm{m}\mathrm{o}\dot{\mathrm{z}}$ na byfo rozdzielić między dwóch klientów tak, aby $\mathrm{g}\mathrm{w}\mathrm{o}\acute{\mathrm{z}}$dzie w zakupionych przez

nich skrzynkach $\mathrm{w}\mathrm{a}\dot{\mathrm{z}}$ yfy tyle samo?

ZADANIE 5.

Joanna, Paweł, Michal, Piotr i Jarek mieszkają w jednym bloku. $\mathrm{K}\mathrm{a}\dot{\mathrm{z}}\mathrm{d}\mathrm{y}$ z nich prowadzi inny

samochód: peugeot, renault (samochody francuskie), fiat (samochód wloski), ligier (francuski

samochód wyścigowy) i ferrari (wloski samochód wyścigowy). Ustal jaki samochód posiada

$\mathrm{k}\mathrm{a}\dot{\mathrm{z}}\mathrm{d}\mathrm{y}$ z nich, $\mathrm{j}\mathrm{e}\dot{\mathrm{z}}$ eli:

$\bullet$ Pawel i Michaf mają samochody francuskie;

$\bullet$ Joanna i Jarek lubią bezpiecznąjazdę i nie mają samochodów wyścigowych;

$\bullet$ Michal, Joanna i Jarek wsiadają często do renaulta, ale go nie prowadzą;

$\bullet$ Joanna, Paweł i właściciel peugeota sa przyjaciółmi.






LIGA MATEMATYCZNA

im. Zdzisława Matuskiego

GRUD Z$\mathrm{I}\mathrm{E}\acute{\mathrm{N}}$ 2014

SZKOLA PODSTAWOWA

ZADANIE I.

Mikołaj przydzielal siedmiu elfom prezenty do rozdania. Kolejno $\mathrm{k}\mathrm{a}\dot{\mathrm{z}}\mathrm{d}\mathrm{y}$ otrzymywal po jednej

paczce. Kiedy $\mathrm{k}\mathrm{a}\dot{\mathrm{z}}\mathrm{d}\mathrm{y}$ elf miał $\mathrm{j}\mathrm{u}\dot{\mathrm{z}} 18$ prezentów pozostała reszta, która nie wystarczyla, by

$\mathrm{k}\mathrm{a}\dot{\mathrm{z}}\mathrm{d}\mathrm{y}$ otrzymał jeszcze po jednym. Resztę oddali Mikołajowi. Ile prezentów było do rozdania?

ZADANIE 2.

Na kiermaszu przedświątecznym zaplanowano sprzedaz 400 bombek choinkowych po 16 zł za

sztukę. Po sprzedaniu 30\% bombek okaza1o się, $\dot{\mathrm{z}}\mathrm{e}$ część popękała w czasie transportu. Odło-

$\dot{\mathrm{z}}$ ono więc te bombki. Aby uzyskać zaplanowany przychód, pozostale bombki zostały sprzedane

po 20 zł za sztukę. I1e bombek popęka1o podczas transportu?

ZADANIE 3.

Uzywając tylko cyfr 2 $\mathrm{i}8$ (być $\mathrm{m}\mathrm{o}\dot{\mathrm{z}}\mathrm{e}$ kilkukrotnie) zapisz najmniejszą liczbę naturalną podzielną

przez 9 oraz najmniejszą 1iczbę natura1ną podzie1ną przez 12.

ZADANIE 4.

Rozetnij kwadrat na cztery jednakowe części tak, aby w $\mathrm{k}\mathrm{a}\dot{\mathrm{z}}$ dej znalazly się trzy kólka.
\begin{center}
\begin{tabular}{|l|l|l|l|l|l|}
\hline
\multicolumn{1}{|l|}{}&	\multicolumn{1}{|l|}{}&	\multicolumn{1}{|l|}{}&	\multicolumn{1}{|l|}{}&	\multicolumn{1}{|l|}{}&	\multicolumn{1}{|l|}{}	\\
\hline
\multicolumn{1}{|l|}{}&	\multicolumn{1}{|l|}{}&	\multicolumn{1}{|l|}{}&	\multicolumn{1}{|l|}{}&	\multicolumn{1}{|l|}{}&	\multicolumn{1}{|l|}{}	\\
\hline
\multicolumn{1}{|l|}{}&	\multicolumn{1}{|l|}{}&	\multicolumn{1}{|l|}{}&	\multicolumn{1}{|l|}{}&	\multicolumn{1}{|l|}{}&	\multicolumn{1}{|l|}{}	\\
\hline
\multicolumn{1}{|l|}{}&	\multicolumn{1}{|l|}{}&	\multicolumn{1}{|l|}{}&	\multicolumn{1}{|l|}{}&	\multicolumn{1}{|l|}{}&	\multicolumn{1}{|l|}{}	\\
\hline
\multicolumn{1}{|l|}{}&	\multicolumn{1}{|l|}{}&	\multicolumn{1}{|l|}{}&	\multicolumn{1}{|l|}{}&	\multicolumn{1}{|l|}{}&	\multicolumn{1}{|l|}{}	\\
\hline
\multicolumn{1}{|l|}{}&	\multicolumn{1}{|l|}{}&	\multicolumn{1}{|l|}{}&	\multicolumn{1}{|l|}{}&	\multicolumn{1}{|l|}{}&	\multicolumn{1}{|l|}{}	\\
\hline
\end{tabular}
\end{center}
ZADANIE 5.

$\mathrm{Z}$ trzech trójkątów prostokątnych równoramiennych zbudowano choinkę, jak na rysunku. Pod-

stawa największego trójkąta ma długość 20 cm. Podstawa następnego jest umieszczona w po-

łowie wysokości poprzedniego trójkąta. $\mathrm{K}\mathrm{a}\dot{\mathrm{z}}\mathrm{d}\mathrm{y}$ kolejny trójkąt jest o 2 cm $\mathrm{n}\mathrm{i}\dot{\mathrm{z}}$ szy. Oblicz pole

powierzchni największego i najmniejszego trójkąta.






LIGA MATEMATYCZNA

im. Zdzisława Matuskiego

GRUD Z$\mathrm{I}\mathrm{E}\acute{\mathrm{N}}$ 2015

SZKOLA PODSTAWOWA

ZADANIE I.

Do zapakowania jest więcej $\mathrm{n}\mathrm{i}\dot{\mathrm{z}}150$ bombek, ale mniej $\mathrm{n}\mathrm{i}\dot{\mathrm{z}}200$. Mamy dwa rodzaje opakowań

do wyboru. Gdy wlozymy do pudefek po 10 sztuk, to zostana 4 bombki, a gdy zapakujemy po

8 sztuk, to $\mathrm{t}\mathrm{e}\dot{\mathrm{z}}$ zostaną 4. I1e bombek jest do zapakowania? I1e na1ezy wziąć pude1ek $\mathrm{k}\mathrm{a}\dot{\mathrm{z}}$ dego

rodzaju, abyje zapelnić i aby wszystkie bombki byly zapakowane? Podaj wszystkie $\mathrm{m}\mathrm{o}\dot{\mathrm{z}}$ liwości.

ZADANIE 2.

Wykaz$\cdot, \dot{\mathrm{z}}\mathrm{e}$ liczba $10^{49}+20$ jest podzielna przez 12.

ZADANIE 3.

Pan Jan hoduje koty. Ma ich tyle, $\dot{\mathrm{z}}\mathrm{e}$ gdy dodaf liczbę kocich ogonów, uszu i fapek, to otrzymaf

ponad 100. Gdy zsumowa1 ty1ko 1iczbę ogonów i 1ap, otrzyma1 mniej $\mathrm{n}\mathrm{i}\dot{\mathrm{z}}80$. Ile kotów ma pan

Jan?

ZADANIE 4.

Czarodziej podarowaf Ani zaczarowaną szkatufkę i podal dwa zaklęcia. Szkatulka ta na jedno

z zaklęć powiększa swoją zawartość ojednego denara, a na drugie podwaja liczbę denarów znaj-

dujących się w niej. Podaj najmniejszą liczbę zaklęć, jakie trzeba wypowiedzieć, aby w szkatulce

(na początku pustej) znalazlo się 40 denarów (nie wyjmujemy monet ze szkatu1ki).

ZADANIE 5.

$\mathrm{Z}$ ilu najmniejszych kwadracików (dwa z nich zaznaczono na czarno) składa się $\mathrm{d}\mathrm{u}\dot{\mathrm{z}}\mathrm{y}$ kwadrat

o czarnym obwodzie?






LIGA MATEMATYCZNA

im. Zdzisława Matuskiego

GRUD Z$\mathrm{I}\mathrm{E}\acute{\mathrm{N}}$ 2016

SZKOLA PODSTAWOWA

ZADANIE I.

Mikołaj napisal kolejne liczby naturalne $\mathrm{u}\dot{\mathrm{z}}$ ywajac łącznie siedmiu cyfr. Znajdz' te liczby wie-

dząc, $\dot{\mathrm{z}}\mathrm{e}$ ponad polowa spośród $\mathrm{u}\dot{\mathrm{z}}$ ytych cyfr byla taka sama.

ZADANIE 2.

$K\mathrm{a}\dot{\mathrm{z}}\mathrm{d}\mathrm{y}$ uczestnik mikolajkowego turnieju dostaje dziesięć punktów na starcie i musi odpowie-

dzieć na 10 pytań. Za dobrą odpowied $\acute{\mathrm{z}}$ dostaje l punkt, za zlą $\mathrm{o}\mathrm{d}\mathrm{p}\mathrm{o}\mathrm{w}\mathrm{i}\mathrm{e}\mathrm{d}\acute{\mathrm{z}}$ lub jej brak traci

l punkt. Mikolaj ukończył turniej z 14 punktami. I1u dobrych odpowiedzi udzie1ił?

ZADANIE 3.

Mianownik pewnego ulamka jest o 3 większy od 1icznika. $\mathrm{J}\mathrm{e}\dot{\mathrm{z}}$ eli jego licznik zwiększymy o 10,

a mianownik zwiększymy o l, to otrzymany ulamek będzie odwrotnością poszukiwanego. Jaki

to ulamek?

ZADANIE 4.

Ania rozpoczęla czytanie ksiązki w sobotę. Przez pierwsze cztery dni czytala $\mathrm{k}\mathrm{a}\dot{\mathrm{z}}$ dego dnia śred-

nio po 12 stron. Przez następne dni czytafa dziennie po 20 stron. Ostatniego dnia przeczytafa

ostatnie 10 stron ksiązki. Okaza1o się, $\dot{\mathrm{z}}\mathrm{e}$ gdyby czytala po 14 stron dziennie, to cafą ksiązkę

przeczytalaby w tym samym czasie. Ile dni zajęfo Ani przeczytanie tej ksiązki? $\mathrm{W}$ którym

dniu tygodnia skończyła czytać?

ZADANIE 5.

Prostokąt przedstawiony na rysunku podzielono na sześć figur. Czworokąty $A, B, C, D$ sa

kwadratami. Pole kwadratu $A$ jest równe 9 $\mathrm{c}\mathrm{m}^{2}$, pole B- $4\mathrm{c}\mathrm{m}^{2}$, a pole $D$ - $49\mathrm{c}\mathrm{m}^{2}$ Oblicz

pole tego prostokąta.

{\it A}

{\it B}

{\it D}

{\it C}






LIGA MATEMATYCZNA

Szkoła Podstawowa

Półfinał

201utego 2009

ZADANIE I.

Kasię, Ewę i Anię poczęstowano trzema czekoladkami z orzechami, z rodzynkami i truskawkami.

Kasia nie lubi orzechów, Ewa-rodzynek, Ania jest uczulona na truskawki. Na ile sposobów

$\mathrm{m}\mathrm{o}\dot{\mathrm{z}}$ na podzielić te czekoladki między dziewczynki tak, aby były zadowolone?

ZADANIE 2.

$\mathrm{W}$ trójkącie równoramiennym $ABC$, w którym $|AC| = |BC|$, poprowadzono wysokość CD.

Oblicz dlugość tej wysokości, $\mathrm{j}\mathrm{e}\dot{\mathrm{z}}$ eli obwód trójkąta $ABC$ jest równy 32 cm, a obwód trójkąta

$ADC$ jest o 6 cm krótszy od obwodu trójkąta $ABC.$

ZADANIE 3.

Ania i Basia wazą łącznie 44 kg, Basia i Ce1ina-47 kg, Ce1ina i Dorota-46 kg, Dorota i Ewa

$-49$ kg, Ewa i Ania-48 kg. I1e wazy Ania?

ZADANIE 4.

Przy uzyciu cyfr: 1, 2, 3, 4, 5, 6, Tomek napisał dwie 1iczby całkowite dodatnie takie, $\dot{\mathrm{z}}\mathrm{e}\mathrm{k}\mathrm{a}\dot{\mathrm{z}}$ da

z cyfr występowała tylko w jednej z dwóch liczb, i to dokladnie raz. Gdy liczby te dodał,

otrzymał 750. Jakie 1iczby napisał Tomek? Podaj wszystkie pary tych 1iczb.

ZADANIE 5.

Ala pomaga cioci w prowadzeniu sklepu cukierniczego. Po zamknięciu sklepu dziewczynka po-

liczyla, ile tabliczek czekolady pozostalo na pólkach, ale przez roztargnienie wpisała do zeszytu

otrzymaną liczbę bez ostatniej cyfry. Na drugi dzień stwierdzila ze zdziwieniem, $\dot{\mathrm{z}}\mathrm{e}$ liczba tabli-

czek czekolady na pólkach jest o 89 większa od 1iczby wpisanej do zeszytu. Jaką 1iczbę powinna

byla wpisać Ala?






LIGA MATEMATYCZNA

PÓLFINAL

51utego 20l0

SZKOLA PODSTAWOWA

ZADANIE I.

Arbuz jest o $\displaystyle \frac{4}{5}$ kg cięzszy od $\displaystyle \frac{4}{5}$ tego arbuza. Ile wazy arbuz?

ZADANIE 2.

$\mathrm{W}$ pewnej rodzinie jest czworo dzieci w wieku 5, 8, 13 $\mathrm{i} 15$ lat. Imiona tych dzieci to Ania,

Bartek, Czesia i Daria. Ile lat ma $\mathrm{k}\mathrm{a}\dot{\mathrm{z}}$ de z nich, $\mathrm{j}\mathrm{e}\dot{\mathrm{z}}$ eli jedna dziewczynka chodzi do przedszkola,

Ania jest starsza od Bartka, a suma lat Ani i Czesi dzieli się przez 3?

ZADANIE 3.

Do hurtowni nadszedł transport trzech gatunków herbaty. $K\mathrm{a}\dot{\mathrm{z}}$ dego gatunku było tyle samo,

a caly transport wazył mniej $\mathrm{n}\mathrm{i}\dot{\mathrm{z}}4$ tony. Pierwszy gatunek herbaty przysłano w 76jednakowych

paczkach, drugi w 57 jednakowych paczkach, a trzeci w 60 jednakowych paczkach. $\mathrm{W}\mathrm{k}\mathrm{a}\dot{\mathrm{z}}$-

dej paczce byla całkowita liczba kilogramów herbaty. Ile nadeszło herbaty $\mathrm{k}\mathrm{a}\dot{\mathrm{z}}$ dego gatunku?

Ile herbaty było w $\mathrm{k}\mathrm{a}\dot{\mathrm{z}}$ dej z trzech rodzajów paczek?

ZADANIE 4.

Tablicę $3\times 3$ podzielono na dziewięć jednakowych kwadratów. $\mathrm{W}\mathrm{k}\mathrm{a}\dot{\mathrm{z}}\mathrm{d}\mathrm{y}$ z nich wpisano liczbę

-llub 0, 1ub 1. Wykaz, $\dot{\mathrm{z}}\mathrm{e}$ wśród wszystkich sum z trzech wierszy, trzech kolumn i dwóch

gfównych przekątnych co najmniej dwie są równe.

ZADANIE 5.

Cztery kwadratowe plytki ulozono jak na rysunku. Dfugości boków dwóch z tych plytek za-

znaczono na rysunku. Jaka jest długość boku największej płytki?
\begin{center}
\includegraphics[width=56.184mm,height=32.616mm]{./LigaMatematycznaMatuskiego_SP_Zestaw4_2009_2010_page0_images/image001.eps}
\end{center}
1

40






LIGA MATEMATYCZNA

PÓLFINAL

181utego 20ll

SZKOLA PODSTAWOWA

ZADANIE I.

$\mathrm{W}$ maratonie startowalo 2011 zawodników. Które miejsce zajqf Michaf, $\mathrm{j}\mathrm{e}\dot{\mathrm{z}}$ eli wiadomo, $\dot{\mathrm{z}}\mathrm{e}$ liczba

uczestników, którzy przybiegli na metę przed nim jest cztery razy mniejsza od liczby uczestni-

ków, którzy przybiegli po nim?

ZADANIE 2.

Uzupelniamy tablicę wpisując w $\mathrm{k}\mathrm{a}\dot{\mathrm{z}}$ de jej pole 01ub 1 tak, aby sumy 1iczb w $\mathrm{k}\mathrm{a}\dot{\mathrm{z}}$ dym wierszu

i w $\mathrm{k}\mathrm{a}\dot{\mathrm{z}}$ dej kolumnie były równe 2. Jakie są wartości a $\mathrm{i}b$?
\begin{center}
\begin{tabular}{|l|l|l|l|}
\hline
\multicolumn{1}{|l|}{$a$}&	\multicolumn{1}{|l|}{}&	\multicolumn{1}{|l|}{}&	\multicolumn{1}{|l|}{}	\\
\hline
\multicolumn{1}{|l|}{}&	\multicolumn{1}{|l|}{ $b$}&	\multicolumn{1}{|l|}{}&	\multicolumn{1}{|l|}{ $1$}	\\
\hline
\multicolumn{1}{|l|}{}&	\multicolumn{1}{|l|}{}&	\multicolumn{1}{|l|}{ $0$}&	\multicolumn{1}{|l|}{}	\\
\hline
\multicolumn{1}{|l|}{ $0$}&	\multicolumn{1}{|l|}{}&	\multicolumn{1}{|l|}{ $0$}&	\multicolumn{1}{|l|}{}	\\
\hline
\end{tabular}

\end{center}
ZADANIE 3.

Na skraju lasu stoi siedem domków. $K\mathrm{a}\dot{\mathrm{z}}\mathrm{d}\mathrm{y}$ domek zamieszkiwany jest przez inną liczbę miesz-

kańców oraz $\dot{\mathrm{z}}$ aden domek nie jest pusty. Ile osób mieszka w poszczególnych domkach, $\mathrm{j}\mathrm{e}\dot{\mathrm{z}}$ eli

wszystkich mieszkańców jest 29?

ZADANIE 4.

$K\mathrm{a}\dot{\mathrm{z}}$ dą z dwóch identycznych prostokątnych kartek rozcięto na dwie części. $\mathrm{Z}$ pierwszej kartki

otrzymano dwa prostokąty o obwodach 40 cm $\mathrm{k}\mathrm{a}\dot{\mathrm{z}}$ dy, z drugiej - dwa prostokąty o obwodach

50 cm $\mathrm{k}\mathrm{a}\dot{\mathrm{z}}$ dy. Oblicz obwód $\mathrm{k}\mathrm{a}\dot{\mathrm{z}}$ dej z wyjściowych kartek.

ZADANIE 5.

Adaś dostał pod choinkę modele trzech samochodów: mercedesa, opla i fiata. Wszystkie są róz-

nej wielkości i w róznych kolorach: bialym, czerwonym i czarnym. Mercedes nie jest biały ani

czarny, opel nie jest šredni, a fiat nie jest $\mathrm{d}\mathrm{u}\dot{\mathrm{z}}\mathrm{y}$ ani czarny. Okrešl wielkošć i kolor $\mathrm{k}\mathrm{a}\dot{\mathrm{z}}$ dego

samochodu, jeśli wiadomo, $\dot{\mathrm{z}}\mathrm{e}$ mały samochód jest czarny.






LIGA MATEMATYCZNA

PÓLFINAL

161utego 20l2

SZKOLA PODSTAWOWA

ZADANIE I.

Trzy kolejne liczby trzycyfrowe zapisano obok siebie, bez odstępów, otrzymując liczbę dzie-

więciocyfrową podzielną przez 4 $\mathrm{i}25.$ Znajd $\acute{\mathrm{z}}$ te liczby wiedząc, $\dot{\mathrm{z}}\mathrm{e}$ w ich zapisie dziesiętnym

występują jedynie trzy rózne cyfry.

ZADANIE 2.

$K\mathrm{a}\dot{\mathrm{z}}\mathrm{d}\mathrm{y}$ uczeń pewnej klasy sportowej uprawia $\dot{\mathrm{z}}$ eglarstwo, plywanie lub judo. Tylko dwóch

uczniów uprawia wszystkie te dyscypliny sportu. Judo i $\dot{\mathrm{z}}$ eglarstwem zajmuje się czworo

uczniów. Dziesięciu uprawiających pfywanie nie zajmuje się $\dot{\mathrm{z}}$ eglarstwem. Pięciu uprawia ply-

wanie i judo. $\dot{\mathrm{Z}}$ eglarstwem pasjonuje się 19 uczniów. Pływanie i $\dot{\mathrm{z}}$ eglarstwo to dyscypliny

uprawiane przez 8 uczniów. I1u uczniów uprawia judo w tej k1asie, skoro wszystkich uczniów

jest 36? I1u uczniów uprawia ty1ko jedną dyscyp1inę sportu?

ZADANIE 3.

Ola, Basia, Ewa i Kasia wybrały się na grzyby. Ola i Basia zebraly razem 40 grzybów, Ewa

i Kasia 42, a O1a i Kasia 30 grzybów. I1e grzybów zebra1y 1ącznie Basia i Ewa?

ZADANIE 4.

Rozpoczynając od pewnej liczby naturalnej, wypisano pięć kolejnych jej wielokrotnošci. Suma

trzech najmniejszych jest równa 33330. Ob1icz sumę trzech największych.

ZADANIE 5.

Prostokąt został podzielony na kwadratyjak na rysunku. Pole kwadratu zaznaczonego ciemnym

kolorem jest równe l. Wyznacz dlugošci boków wszystkich kwadratów oraz oblicz pole tego

prostokąta.






LIGA MATEMATYCZNA

im. Zdzisława Matuskiego

PÓLFINAL

ll lutego 2014

SZKOLA PODSTAWOWA

ZADANIE I.

Uzupelnij diagram w taki sposób, aby liczba w $\mathrm{k}\mathrm{a}\dot{\mathrm{z}}$ dym polu w rzędzie $\mathrm{w}\mathrm{y}\dot{\mathrm{z}}$ szym była sumą

dwóch liczb z pól $\mathrm{n}\mathrm{i}\dot{\mathrm{z}}$ szego rzędu sąsiadujących z nim. Oblicz sumę liczb w podstawie diagramu.
\begin{center}
\includegraphics[width=33.276mm,height=29.712mm]{./LigaMatematycznaMatuskiego_SP_Zestaw4_2014_2015_page0_images/image001.eps}
\end{center}
32

23  10

14  2

ZADANIE 2.

Krzyś pomnozył trzy liczby naturalne i otrzymał 5400. Liczby pierwsza i druga nie dzie1ą się

przez 2, druga i trzecia- nie dzie1ą się przez 3, a pierwsza i trzecia- nie dzie1ą się przez 5. Jakie

to liczby?

ZADANIE 3.

Suma cyfr pewnej liczby dziewięciocyfrowej jest równa 9. Cyfra 2 występuje w niej ty1ko raz.

Oblicz iloczyn cyfr tej liczby.

ZADANIE 4.

$\mathrm{O}$ czterech kolegach wiadomo, $\dot{\mathrm{z}}\mathrm{e}$:

$\bullet$ Mirek i kierowca są starsi od Pawfa;

$\bullet$ Leszek i policjant trenujq boks;

$\bullet$ elektryk jest najmłodszy z całej czwórki;

$\bullet$ w soboty Zbyszek i piekarz grają w brydza przeciw Pawłowi i elektrykowi.

Jaki zawód wykonuje $\mathrm{k}\mathrm{a}\dot{\mathrm{z}}\mathrm{d}\mathrm{y}$ z przyjaciól?

ZADANIE 5.

$\mathrm{W}$ kredensie stoją 24 dzbany, w tym 8 z nich jest pustych, 11 wype1nionych miodem do połowy,

5 pełnych miodu. $\mathrm{W}$ jaki sposób $\mathrm{m}\mathrm{o}\dot{\mathrm{z}}$ na podzielić miód i dzbany między trzech braci tak, aby

$\mathrm{k}\mathrm{a}\dot{\mathrm{z}}\mathrm{d}\mathrm{y}$ z nich otrzyma18 dzbanów zjednakową zawartością w nich miodu? (Nie wo1no prze1ewać

miodu z dzbana do dzbana.)






AHADEMIA POMORSHA

III SLUPSHU
\begin{center}
\includegraphics[width=40.740mm,height=4.476mm]{./LigaMatematycznaMatuskiego_SP_Zestaw4_2015_2016_page0_images/image001.eps}
\end{center}
LIGA MATEMATYCZNA

im. Zdzisława Matuskiego

PÓLFINAL
\begin{center}
\includegraphics[width=34.548mm,height=42.576mm]{./LigaMatematycznaMatuskiego_SP_Zestaw4_2015_2016_page0_images/image002.eps}
\end{center}
2 marca 20l5

SZKOLA PODSTAWOWA

ZADANIE I.

Do liczby 18 dopisz jedną cyfrę na końcu 1ub na początku, 1ub w środku tak, aby otrzymana

liczba trzycyfrowa była podzielna przez 6. Wyznacz wszystkie takie 1iczby.

ZADANIE 2.

W trójkącie równoramiennym ABC, o ramionach AC i BC, połqczono środek E boku AC

z wierzcholkiem B oraz środek D boku BC z wierzcholkiem A. Obwód trójkąta ABC jest

równy 50, a obwód trójkąta ABE jest o 8 większy od obwodu trójkąta ADC. Ob1icz d1ugości

boków trójkąta ABC.

ZADANIE 3.

Przy ognisku na kocach siedziały elfy i skrzaty. Wszystkich duszków leśnych bylo mniej $\mathrm{n}\mathrm{i}\dot{\mathrm{z}}$

400. Dla $\mathrm{k}\mathrm{a}\dot{\mathrm{z}}$ dego elfa przygotowano jedną porcję nektaru, a dla $\mathrm{k}\mathrm{a}\dot{\mathrm{z}}$ dego skrzata dwie porcje.

Wszyscy siedzieli na 51 kocach, na $\mathrm{k}\mathrm{a}\dot{\mathrm{z}}$ dym taka sama liczba duszków leśnych, przy czym elfy

stanowiły $\displaystyle \frac{7}{12}$ wszystkich. Ile porcji nektaru przygotowano?

ZADANIE 4.

Uczeń klasy VI kupił cztery podręczniki: do języka polskiego, do języka angielskiego, do ma-

tematyki i przyrody. Wszystkie ksiązki bez podręcznika do języka polskiego kosztowafy 42 z1,

wszystkie bez języka angielskiego 40 zł, wszystkie bez matematyki 38 $\mathrm{z}l$, a wszystkie bez przy-

rody 36 zf. I1e kosztowa1 $\mathrm{k}\mathrm{a}\dot{\mathrm{z}}\mathrm{d}\mathrm{y}$ podręcznik?

ZADANIE 5.

Na spacerze Ania robiła zdjęcia Bartkowi i jego psu. Lącznie zrobiła 24 zdjęcia. Bartek jest

na 18 zdjęciach, a pies na 14. Jaką częšć wszystkich zdjęć stanowią te, na których jest Bartek

razem z psem?






LIGA MATEMATYCZNA

im. Zdzisława Matuskiego

PÓLFINAL

291utego 20l6

SZKOLA PODSTAWOWA

ZADANIE I.

Kwadrat, którego dlugość boku jest równa 12, podzie1ono na mniejsze kwadraty (najmniejszy

ma bok o dfugości l) i prostokąty, które nie są kwadratami. Na rysunku podano dfugości

boków niektórych figur. Które figury- kwadraty czy prostokąty- zajmuja większa część $\mathrm{d}\mathrm{u}\dot{\mathrm{z}}$ ego

kwadratu?
\begin{center}
\includegraphics[width=44.556mm,height=45.816mm]{./LigaMatematycznaMatuskiego_SP_Zestaw4_2016_2017_page0_images/image001.eps}
\end{center}
3

2

2

1

3

6

2 2 4

ZADANIE 2.

Do liczby 36 dopisz po jednej cyfrze na końcu i na początku tak, aby otrzymana 1iczba cztero-

cyfrowa byla podzielna przez 36. Podaj wszystkie $\mathrm{m}\mathrm{o}\dot{\mathrm{z}}$ liwości.

ZADANIE 3.

Mama postawila na stole tacę z cukierkami dla Ani, Bartka i Czarka. Ania wzięla trzecią część

wszystkich cukierków, Bartek- trzecia część tego, co zostalo na tacy. Na końcu Czarek wzial

trzecią część reszty cukierków. Na tacy pozostaly 24 cukierki. $\mathrm{W}$ jaki sposób mama powinna

podzielić pozostałe cukierki, aby $\mathrm{k}\mathrm{a}\dot{\mathrm{z}}$ de dziecko otrzymalo jedną trzecią wszystkich cukierków?

ZADANIE 4.

Wojtek wypisa112 ko1ejnych 1iczb natura1nych w porządku rosnącym. Gdy dodaf co drugą

z nich, zaczynajac od drugiej, otrzyma13330. Jaką sumę uzyska, gdy doda co trzecią 1iczbę,

zaczynając od trzeciej?

ZADANIE 5.

Wykaz, $\dot{\mathrm{z}}\mathrm{e}$ liczba 13333$\ldots$ 35 jest podzielna przez l5.
\begin{center}
\includegraphics[width=20.172mm,height=6.852mm]{./LigaMatematycznaMatuskiego_SP_Zestaw4_2016_2017_page0_images/image002.eps}
\end{center}
$2016$ cyfr 3






LIGA MATEMATYCZNA

im. Zdzisława Matuskiego

PÓLFINAL

161utego 20l7

SZKOLA PODSTAWOWA

ZADANIE I.

Monika wykonala pięćdziesiąt rzutów sześcienną kostką i otrzymala w sumie 100 oczek. I1e co naj-

$\mathrm{w}\mathrm{y}\dot{\mathrm{z}}$ ej razy mogla wypaść,,piątka''?

ZADANIE 2.

$\mathrm{W}$ pewnej grze komputerowej Bartek zdobyl najpierw 157 punktów, potem ki1ka razy straci1

po 19 punktów, a następnie odrobi1 pofowę strat i skończy1 grę z rezu1tatem 100 punktów.

Ile razy poniósl stratę?

ZADANIE 3.

Piotrek wypisaf wszystkie rózne liczby zapisane za pomocą trzech trójek i trzech zer.

sumę tych liczb.

Oblicz

ZADANIE 4.

Prostokąt o polu 100 podzie1ono na trzy prostokąty, z których jeden ma obwód 21 i dfugość 8,

a drugi ma obwód 23 i szerokość 1, 5. Ob1icz po1e trzeciego prostokąta.

ZADANIE 5.

Liczba A jest podzielna przez 42 i 45. Czy 1iczba A dzie1i się przez 210?






LIGA MATEMATYCZNA

im. Zdzisława Matuskiego

PÓLFINAL

211utego 20l8

SZKOLA PODSTAWOWA

(klasy IV - VI)

ZADANIE I.

$\mathrm{Z}$ czterocyfrowej liczby pierwszej Adam wymazal jedną cyfrę i otrzyma1630. Wyznacz tę 1iczbę

czterocyfrową.

ZADANIE 2.

$K\mathrm{a}\dot{\mathrm{z}}\mathrm{d}\mathrm{y}$ z trzech synów państwa Malinowskich ma calkowitą liczbę lat. Iloczyn ich lat jest równy

18, a za rok wyniesie 60. Podaj wiek $\mathrm{k}\mathrm{a}\dot{\mathrm{z}}$ dego z nich.

ZADANIE 3.

$\mathrm{W}$ trapezie równoramiennym przekątna dzieli kąt ostry na pofowy. Dluzsza podstawa trapezu

ma dfugość 24, a obwód jest równy 54. Ob1icz d1ugości pozosta1ych boków trapezu.

ZADANIE 4.

Adam, Bartek i Czarek mają razem 30 pi1ek. Gdy Bartek da15 pifek Czarkowi, Czarek daf

4 pifki Adamowi, Adam dal 2 pifki Bartkowi, to okazalo się, $\dot{\mathrm{z}}\mathrm{e}$ wszyscy mająjednakową liczbę

pifek. Ile pifek mieli na początku?

ZADANIE 5.

$\mathrm{W}$ pewnym trójkącie równoramiennym kąt między dwusiecznymi jednakowych kątów jest trzy

razy większy $\mathrm{n}\mathrm{i}\dot{\mathrm{z}}\mathrm{k}\mathrm{a}\mathrm{t}$ między ramionami trójkąta. Wyznacz miary katów trójkąta.






50

$\rightarrow\not\subset \mathrm{D}\vdash$

flkademia

P omorskawStupsku

LIGA MATEMATYCZNA

im. Zdzisława Matuskiego

PÓLFINAL 261utego 2019

SZKOLA PODSTAWOWA

(klasy IV- VI)

ZADANIE I.

$\mathrm{W}$ pojedynczym ruchu $\mathrm{m}\mathrm{o}\dot{\mathrm{z}}$ na albo wrzucić do urny jedną kulkę, albo podwoić liczbę kulek

znajdujących się w urnie. Wyznacz najmniejszą liczbę ruchów pozwalającą zamienić pustą

urnę w urnę zawierającą 200 ku1ek.

ZADANIE 2.

Bartek, jego ojciec i $\mathrm{k}\mathrm{a}\dot{\mathrm{z}}\mathrm{d}\mathrm{y}$ z dwóch braci: Darek i Czarek obchodzą urodziny w lutym. Wia-

domo, $\dot{\mathrm{z}}\mathrm{e}$ mnoząc liczby lat obu braci Bartka otrzymujemy wiek Bartka, a mnoząc liczby lat

całej trójki rodzeństwa dostajemy wiek taty. Ile lat ma tata Bartka, $\mathrm{j}\mathrm{e}\dot{\mathrm{z}}$ eli wiadomo, $\dot{\mathrm{z}}\mathrm{e}$ ma

mniej $\mathrm{n}\mathrm{i}\dot{\mathrm{z}}50$ lat i $\mathrm{k}\mathrm{a}\dot{\mathrm{z}}$ de z jego dzieci jest w innym wieku?

ZADANIE 3.

Ile jest liczb trzycyfrowych podzielnych przez 9, które $\mathrm{m}\mathrm{o}\dot{\mathrm{z}}$ na ulozyć z cyfr 2, 7, 9, 0 wykorzy-

stując w jednej liczbie $\mathrm{k}\mathrm{a}\dot{\mathrm{z}}$ dą z cyfr co najwyzej raz? Podaj wszystkie $\mathrm{m}\mathrm{o}\dot{\mathrm{z}}$ liwości.

ZADANIE 4.

$\mathrm{W}$ równolegfoboku ABCD bok $AB$ jest dwa razy dluzszy od boku $BC$. Punkt $M$, dzielący

bok $AB$ na polowy, polączono z punktami $C\mathrm{i}D$. Oblicz miarę kąta $CMD.$

ZADANIE 5.

Obwód pięciokąta wypuklego ABCDE jest równy 74, obwód czworokąta ABCD jest równy

56, a czworokata ACDE- 37. Oblicz obwód trójkąta $ACD.$
\begin{center}
\includegraphics[width=74.268mm,height=54.912mm]{./LigaMatematycznaMatuskiego_SP_Zestaw4_2019_2020_page0_images/image001.eps}
\end{center}
D

E

c

A

B






flkademia

P omorskamStupsku

LIGA MATEMATYCZNA

im. Zdzisława Matuskiego

PÓLFINAL 16 marca 2022

SZKOLA PODSTAWOWA

klasy VII- VIII

ZADANIE I.

Adam narysowaf dwa jednakowe kwadraty. Następnie jeden podzielif na osiem, a drugi na

trzynaście mniejszych kwadratów. Oblicz stosunek pola największego kwadratu z podzialu na

osiem części do pola najmniejszego kwadratu z podzialu na trzynaście części.

ZADANIE 2.

Suma 20221iczb ca1kowitych jest 1iczbą nieparzystą. Uzasadnij, $\dot{\mathrm{z}}\mathrm{e}$ iloczyn tych liczb jest liczbą

parzysta.

ZADANIE 3.

Ania podzielila liczbę 13 przez 10 róznych 1iczb natura1nych nie większych $\mathrm{n}\mathrm{i}\dot{\mathrm{z}}13$ i otrzymafa

reszty, których suma jest równa 13. Przez które 1iczby dzie1ifa?

ZADANIE 4.

Cyfrą jedności liczby trzycyfrowej $A$ jest 3. $\mathrm{J}\mathrm{e}\dot{\mathrm{z}}$ eli do liczby $A$ dodamy 3 i uzyskana sumę

podzielimy przez 3, to powstanie 1iczba trzycyfrowa, która na miejscu setek ma 1, a ko1ejne jej

cyfry są pierwszą i drugą cyfrą liczby $A$ (licząc od lewej strony). Wyznacz $A.$

ZADANIE 5.

Szyfr otwierający zamek kuferka Basi składa się z czterech róznych cyfr. Liczba czterocyfrowa

tworząca szyfr dzieli się przez 17 $\mathrm{i}137$, a suma jej cyfr jest $\mathrm{m}\mathrm{o}\dot{\mathrm{z}}$ liwie najmniejsza. Znajd $\acute{\mathrm{z}}$ ten

szyfr.






LIGA MATEMATYCZNA

FINAL

25 kwietnia 2009

SZKOLA PODSTAWOWA

ZADANIE I.

Dziesięć śliwek $\mathrm{w}\mathrm{a}\dot{\mathrm{z}}\mathrm{y}$ tyle, co trzy jablka i gruszka. Jabfko i gruszka wazą tyle, co sześć śliwek.

Waga ilu śliwek jest równa wadze jednej gruszki?

ZADANIE 2.

Na prostej obrano kolejno pięć punktów $A, B, C, D, E$. Wiadomo, $\dot{\mathrm{z}}\mathrm{e}AB=19$ cm, $CE=97$ cm,

$AC=BD$. Znajdz' długošć odcinka $DE.$

ZADANIE 3.

$\mathrm{W}$ prostokącie jeden z boków stanowi $\displaystyle \frac{2}{3}$ drugiego. $\mathrm{Z}$ wierzcholka prostokąta do środka dluzszego

boku poprowadzono odcinek. Dzieli on prostokąt na dwie figury: trójkąt o obwodzie równym

12 cm i trapez o obwodzie l8 cm. Oblicz obwód prostokąta.

ZADANIE 4.

Adam, Bartek i Witek uczą się w tej samej klasie. Jeden z nich dojezdza do szkoły autobusem,

drugi tramwajem, a trzeci rowerem. Pewnego dnia Adam odprowadzal kolegę na przystanek

autobusowy. $\mathrm{W}$ tym samym czasie obok nich przejechał rowerem trzeci kolega i zawolal:

,,Bartek, zostawifeś zeszyt w szkole'' Jakim środkiem lokomocji dojez $\mathrm{d}\dot{\mathrm{z}}$ a $\mathrm{k}\mathrm{a}\dot{\mathrm{z}}\mathrm{d}\mathrm{y}$ z nich?

ZADANIE 5.

Państwo Kowalscy i Wiśniewscy mają po dwóch synów, z których $\mathrm{k}\mathrm{a}\dot{\mathrm{z}}\mathrm{d}\mathrm{y}$ ma mniej $\mathrm{n}\mathrm{i}\dot{\mathrm{z}}9$ lat.

$\mathrm{W}\mathrm{k}\mathrm{a}\dot{\mathrm{z}}$ dej rodzinie jeden z synów ma więcej $\mathrm{n}\mathrm{i}\dot{\mathrm{z}} 5$ lat, a drugi mniej. Andrzej jest o 31ata

mlodszy od swojego brata. Wojtek jest najstarszy ze wszystkich chlopców. Krzyś jest o 21ata

młodszy od młodszego syna państwa Kowalskich, a Robert jest 51at starszy od młodszego syna

państwa Wiśniewskich. Podaj imiona i nazwiska chlopców oraz ich wiek.






LIGA MATEMATYCZNA

FINAL

26 marca 20l0

SZKOLA PODSTAWOWA

ZADANIE I.

Dziesięć pająków zjada dziesięć much w ciągu dwudziestu sekund. Ile czasu potrzeba stu

pająkom na zjedzenie stu much?

ZADANIE 2.

Suma trzynastu róznych liczb całkowitych dodatnich jest równa 92. Wyznacz te 1iczby.

ZADANIE 3.

Kamila, Ania i Marek twierdzą, $\dot{\mathrm{z}}\mathrm{e}$ suma ich lat jest równa 35. Jednak $\dot{\mathrm{z}}$ adne z nich nie podało

prawdziwego wieku. Kamila zanizyła swój wiek o 31ata, Ania zawyzyła o 21ata, a Marek

postarzył się o 41ata. Kiedy naprawdę suma ich 1at będzie równa 35?

ZADANIE 4.

$\acute{\mathrm{S}}$limak wspina się na drzewo o wysokości 10 $\mathrm{m}. \mathrm{W}$ ciągu dnia podnosi się o 4 metry, a w nocy

obsuwa się o 3 metry. Po i1u dniach ś1imak dostanie się na wierzcho1ek drzewa?

ZADANIE 5.

Prostokqt podzielono na dziewięć róznych kwadratów (rysunek na odwrocie). Długość boku

najmniejszego kwadratu jest równa l.

(a) Znajd $\acute{\mathrm{z}}$ długość boku zamalowanego kwadratu.

(b) Znajdz'dlugošci boków prostokąta.




1







LIGA MATEMATYCZNA

FINAL

30 marca 20ll

SZKOLA PODSTAWOWA

ZADANIE I.

Trapez o obwodzie 72 cm podzie1ono wysokošciami na dwa trójkąty i prostokąt. Suma obwodów

tych trzech figur jest równa 142 cm. Ob1icz wysokość tego trapezu.

ZADANIE 2.

$1=1^{2}-0^{2}$

$3=2^{2}-1^{2}$

$5=3^{2}-2^{2}$

Napisz trzy kolejne wiersze. Przedstaw liczbę 2011 w postaci róznicy kwadratów dwóch 1iczb

naturalnych.

ZADANIE 3.

Skrzynia, kufry i pudelka mają zamki. $\mathrm{W}$ skrzyni jest sześć kufrów, w $\mathrm{k}\mathrm{a}\dot{\mathrm{z}}$ dym kufrze są po trzy

pudełka, a w $\mathrm{k}\mathrm{a}\dot{\mathrm{z}}$ dym pudełku po trzy zlote monety. Jaka jest najmniejsza liczba zamków, które

trzeba otworzyć, aby wyjqć 22 z1ote monety?

ZADANIE 4.

$\mathrm{W}$ pewnej klasie jest 30 uczniów. Wšród nich pięciu ma brata i siostrę, a siedmiu nie ma brata

ani siostry. Ilu uczniów tej klasy ma brata, $\mathrm{j}\mathrm{e}\dot{\mathrm{z}}$ eli wiadomo, $\dot{\mathrm{z}}\mathrm{e}$ trzynastu ma siostrę?

ZADANIE 5.

$\mathrm{W}$ dane kólka wpisz liczby tak, aby suma liczb w $\mathrm{k}\mathrm{a}\dot{\mathrm{z}}$ dych trzech kolejnych kólkach była

równa 15.
\begin{center}
\includegraphics[width=168.804mm,height=10.872mm]{./LigaMatematycznaMatuskiego_SP_Zestaw5_2010_2011_page0_images/image001.eps}
\end{center}
4






l LiceumOgóloksztalcacewSlpsku

AkadmiPomorskawSiupsku

LIGA MATEMATYCZNA

FINAL

ll kwietnia 2012

SZKOLA PODSTAWOWA

ZADANIE I.

Piszemy liczbę 9, następnie 8 i znowu 8. Potem piszemy największą jednocyfrową 1iczbę ca1ko-

witą dodatniq, która nie wystąpiła na trzech poprzednich miejscach. Znowu piszemy największą

jednocyfrową liczbę calkowitą dodatniq, która nie wystąpila na trzech poprzednich miejscach.

Wypisywanie liczb o tej wlasności kontynuujemy. Jaka liczba będzie na 2012 miejscu?

ZADANIE 2.

$\mathrm{W}$ domach przy ulicy Owocowej mieszkają Jabłońscy, S$\acute{}$liwińscy i Wiśniewscy. Jablońscy miesz-

kają w 12 domach, S$\acute{}$1iwińscy w 16, Wiśniewscy w 14. $\mathrm{W}8$ domach mieszkają S$\acute{}$liwińscy i Wi-

śniewscy, $\mathrm{w}7$- Jablońscy i Wiśniewscy, $\mathrm{w}5$- Jabfońscy i S$\acute{}$liwińscy, a w 4- Jabłońscy, S$\acute{}$1iwińscy

i Wiśniewscy. Ile jest domów przy tej ulicy? $\mathrm{W}$ ilu domach mieszkają rodziny o tylko jednym

nazwisku?

ZADANIE 3.

$\mathrm{W}\mathrm{k}\mathrm{a}\dot{\mathrm{z}}$ dym z siedmiu kolejnych lat, zawsze ll kwietnia, urodzif się jeden krasnoludek.

najmłodsze krasnoludki mają razem 421ata. I1e 1at mają razem trzy najstarsze?

Trzy

ZADANIE 4.

Ania i Jarek stoją w kolejce po bilety na koncert. Jarek jest blizej kasy $\mathrm{n}\mathrm{i}\dot{\mathrm{z}}$ Ania. Między nimi

stoją trzy osoby. Za Jarkiem ustawifo się 10 osób, a przed Anią 8 osób. I1e osób stoi w ko1ejce?

Które miejsce w kolejce zajmuje Ania, a które Jarek?

ZADANIE 5.

Przedstawiony na rysunku prostokąt sklada się z szešciu kwadratów.

bok dfugości 2 cm. Ob1icz po1e prostokąta.

Najmniejszy z nich ma






LIGA MATEMATYCZNA

im. Zdzisława Matuskiego

FINAL

10 kwietnia 20l3

SZKOLA PODSTAWOWA

ZADANIE I.

Kwadrat ABCD podzielono na mniejsze kwadraty tak, jak na rysunku. Ile ijakich kwadratów

trzeba zamalować, aby powierzchnia zamalowana stanowila piątą część powierzchni kwadratu

ABCD? Kwadratów nie $\mathrm{m}\mathrm{o}\dot{\mathrm{z}}$ na dzielić na mniejsze części.
\begin{center}
\includegraphics[width=47.604mm,height=45.420mm]{./LigaMatematycznaMatuskiego_SP_Zestaw5_2012_2013_page0_images/image001.eps}
\end{center}
D  c

A  B

ZADANIE 2.

Wyznacz cztery rózne liczby naturalne parzyste, których iloczyn jest równy 8880.

ZADANIE 3.

Dany jest kwadrat ABCD oraz trójkąt równoboczny ABE, gdzie bok AB jest wspólny dla obu

figur. Wyznacz miarę kąta DEC. Rozwaz wszystkie przypadki pofozenia punktu E.

ZADANIE 4.

Liczba uczniów pewnej szkoly podstawowej jest zawarta między 500 a 1000. Gdy grupujemy

ich po 181ub po 20, 1ub po 24, to za $\mathrm{k}\mathrm{a}\dot{\mathrm{z}}$ dym razem pozostaje 9 uczniów. I1u uczniów uczęszcza

do tej szkoly?

ZADANIE 5.

$\mathrm{W}$ dwóch workach bylo 216 kg mąki. Jeś1i z pierwszego worka przesypiemy do drugiego 93 kg,

a następnie z drugiego worka przesypiemy do pierwszego tyle, aby jego zawartość podwoiła

się, to w obu workach będzie tyle samo mąki. Ile kilogramów mąki bylo w $\mathrm{k}\mathrm{a}\dot{\mathrm{z}}$ dym worku

na początku?






LIGA MATEMATYCZNA

im. Zdzisława Matuskiego

PÓLFINAL

71utego 20l3

SZKOLA PODSTAWOWA

ZADANIE I.

Kwadrat ABCD podzielono na czteryjednakowe większe prostokaty, czteryjednakowe mniejsze

prostokąty oraz kwadrat. Kwadraty ABCD, EFGH, IJKL mają obwody równe odpowiednio

360 cm, l20 cm oraz 40 cm. Wyznacz obwody mniejszych i większych prostokatów.

$\displaystyle \bigcap_{d}$

H

K

G

J

E

F

{\it 3}

ZADANIE 2.

$\mathrm{W}$ trójkącie równobocznym odcięto od jednego naroza trójkąt równoramienny, a od drugiego

- trójkąt prostokatny tak, $\dot{\mathrm{z}}\mathrm{e}$ pozostala część jest pięciokatem. Wyznacz miary kątów tego

pięciokąta.

ZADANIE 3.

Wojtek ma trzy patyczki: czerwony, zielony i niebieski. Ich dlugości to 2 cm, 3 cm oraz 5 cm

(kolejność tych liczb nie musi odpowiadać kolejności kolorów). Za pomocą $\mathrm{k}\mathrm{a}\dot{\mathrm{z}}$ dego patyczka

Wojtek zmierzyl dlugość krawędzi stolu. Zielony patyk zmieścif się 75 razy, a niebieski 50 razy.

Czerwony takze zmieścil się calkowitą liczbę razy- ile?

ZADANIE 4.

Wyznacz trzy kolejne liczby naturalne, których iloczyn jest sto razy większy od największej

liczby czterocyfrowej.

ZADANIE 5.

W Tfusty Czwartek mama kupifa mini-pączki dla swojej licznej rodzinki. Wojtek z Darkiem

otrzymali trzecią część wszystkich i jeszcze trzy mini-pączki. Agnieszka z Basią wzięfy trze-

cią częśč pozostalych i jeszcze dwa. Pofowę pozostalych mini-pączków mama dala Jarkowi,

a ostatnie sześć zjadla z tata do kawy. Ile mini-paczków kupila mama?






LIGA MATEMATYCZNA

im. Zdzisława Matuskiego

FINAL

15 kwietnia 20l4

SZKOLA PODSTAWOWA

ZADANIE I.

Znajd $\acute{\mathrm{z}}$ najmniejsza liczbę naturalną większą od 2014 i majacą taką samą sumę cyfr jak 2014.

ZADANIE 2.

Sześć dziewczynek $\mathrm{w}\mathrm{a}\dot{\mathrm{z}}$ acych 18 kg, 19 kg, 20 kg, 23 kg, 38 kg oraz 42 kg ma podzie1ić się na

dwie grupy w taki sposób, aby lączna waga dziewczynek w $\mathrm{k}\mathrm{a}\dot{\mathrm{z}}$ dej grupie byfa taka sama. Na ile

sposobów $\mathrm{m}\mathrm{o}\dot{\mathrm{z}}$ na dokonač takiego podziafu?

ZADANIE 3.

W rodzinie Wojtka są cztery osoby. Suma ich lat jest równa 100. Wojtek jest o cztery 1ata

starszy od Asi, a tata jest o sześć lat starszy od mamy. Asia poprosifa zlotą rybkę, aby cof-

nęla czas o calkowitą liczbę lat do takiego momentu, w którym Asia byla sześć razy młodsza

od mamy. Zfota rybka zastanowila się i cofnęla czas o pięć lat. Ile lat mają czlonkowie rodziny

po cofnięciu czasu?

ZADANIE 4.

Obwód kwadratu jest równy 32 cm. $\acute{\mathrm{S}}$ rodki dwóch kolejnych boków tego kwadratu polączono

ze soba i z wierzchołkiem nie nalezącym do tych boków. Oblicz pole otrzymanego w ten sposób

trójkąta.

ZADANIE 5.

Adam, Bartek, Czarek i Darek lubią milo spędzać czas wolny. $K\mathrm{a}\dot{\mathrm{z}}\mathrm{d}\mathrm{y}$ wybiera swoje ulubione

miejsce i dociera tam w inny sposób. Odkryj, kto dokad wychodzi i jak się tam dostaje, $\mathrm{j}\mathrm{e}\dot{\mathrm{z}}$ eli:

$\bullet$ Bartek nigdy nie chodzi do kina i zawsze $\mathrm{j}\mathrm{e}\acute{\mathrm{z}}\mathrm{d}\mathrm{z}\mathrm{i}$ pociągiem;

$\bullet$ Adam nie opuszcza $\dot{\mathrm{z}}$ adnego koncertu w filharmonii, ale nie ma roweru;

$\bullet$ Czarek jest bywalcem muzeów. Udaje się tam na piechotę;

$\bullet$ Jeden z kolegów spędza wieczory w teatrze, a inny wszędzie $\mathrm{j}\mathrm{e}\acute{\mathrm{z}}\mathrm{d}\mathrm{z}\mathrm{i}$ samochodem.






AHADEMIA POMORSHA

III SLUPSHU
\begin{center}
\includegraphics[width=40.740mm,height=4.476mm]{./LigaMatematycznaMatuskiego_SP_Zestaw5_2015_2016_page0_images/image001.eps}
\end{center}
LIGA MATEMATYCZNA

im. Zdzislawa Matuskiego

FINAL
\begin{center}
\includegraphics[width=34.548mm,height=42.576mm]{./LigaMatematycznaMatuskiego_SP_Zestaw5_2015_2016_page0_images/image002.eps}
\end{center}
16 kwietnia 20l5

SZKOLA PODSTAWOWA

ZADANIE I.

W pewnym biurowcu w Słupsku jest 200 okien. Rano otwartych było 60 okien. Po połu-

dniu zamknięto co drugie okno, a następnie otwarto co drugie okno zamknięte. Ile okien jest

otwartych?

ZADANIE 2.

Dwie kostki, piramida i walec $\mathrm{w}\mathrm{a}\dot{\mathrm{z}}$ a 17 kg. Kostka, dwie piramidy i wa1ec wazą 14 kg. Kostka,

piramida i dwa walce $\mathrm{w}\mathrm{a}\dot{\mathrm{z}}$ a 13 kg. Usta1, i1e $\mathrm{w}\mathrm{a}\dot{\mathrm{z}}\mathrm{y}\mathrm{k}\mathrm{a}\dot{\mathrm{z}}\mathrm{d}\mathrm{y}$ przedmiot.

ZADANIE 3.

$\mathrm{W}$ pięciokącie jedna przekątna ma 7 cm d1ugości, a druga- wychodząca z tego samego wierz-

chofka- ma 8 cm d1ugości. Przekątne te podzie1i1y pięciokąt na trzy trójkąty, $\mathrm{k}\mathrm{a}\dot{\mathrm{z}}\mathrm{d}\mathrm{y}$ o obwodzie

równym 20 cm. Ob1icz obwód pięciokąta.

ZADANIE 4.

W trójkącie równoramiennymjeden z kątów jest cztery razy większy od drugiego. Oblicz miary

kątów tego trójkąta. Rozwaz wszystkie przypadki.

ZADANIE 5.

Ania zbiera pocztówki z kwiatami. Ma ich więcej $\mathrm{n}\mathrm{i}\dot{\mathrm{z}}500$, ale mniej $\mathrm{n}\mathrm{i}\dot{\mathrm{z}}900$. Chce je umieścić

w kopertach, w $\mathrm{k}\mathrm{a}\dot{\mathrm{z}}$ dej tę samą ilość. Gdy wklada po 16, zostają 2 pocztówki. Tak samo, gdy

wkfada po 24 i po 30. I1e pocztówek ma Ania? Zaproponuj takie rozłozenie kartek, aby $\dot{\mathrm{z}}$ adna

nie zostafa i aby $\mathrm{u}\dot{\mathrm{z}}$ yć jak najmniej kopert. $\mathrm{W}$ jednej kopercie nie zmieści się więcej $\mathrm{n}\mathrm{i}\dot{\mathrm{z}} 30$

pocztówek.






LIGA MATEMATYCZNA

im. Zdzisława Matuskiego

FINAL

25 kwietnia 20l6

SZKOLA PODSTAWOWA

ZADANIE I.

Znajd $\acute{\mathrm{z}}$ wszystkie liczby czterocyfrowe podzielne przez 4 o sumie cyfr równej 4.

ZADANIE 2.

$\mathrm{D}\mathrm{u}\dot{\mathrm{z}}\mathrm{y}$ prostokat o obwodzie 136 cm podzie1ono na siedem przystających prostokątów tak, jak

na rysunku. Oblicz pole $\mathrm{d}\mathrm{u}\dot{\mathrm{z}}$ ego prostokata.

ZADANIE 3.

Wykaz$\cdot, \dot{\mathrm{z}}\mathrm{e}$ liczba $10^{45}+2$ jest podzielna przez 6.

ZADANIE 4.

Tysiąc punktów umieszczono równomiernie na okręgu i ponumerowano kolejno od l do 1000.

Jaki numer ma punkt lezący naprzeciw punktu o numerze 657?

ZADANIE 5.

Ramię trapezu równoramiennego ma dlugość 5 cm. Obwód trapezu jest równy 28 cm. Prosta

przechodząca przez środki podstaw podzielila ten trapez na dwie figury o obwodach po 18 cm.

Oblicz pole trapezu.






LIGA MATEMATYCZNA

im. Zdzisława Matuskiego

FINAL

24 kwietnia 20l7

SZKOLA PODSTAWOWA

ZADANIE I.

Ania miafa 96 jednakowych patyczków i zbudowa1a z nich kwadraty i trójkąty. Boki wszyst-

kich figur mialy dfugość jednego patyczka. Powstafo 27 roz1ącznych figur przy wykorzystaniu

wszystkich patyczków. Ile kwadratów i ile trójkątów zbudowala Ania?

ZADANIE 2.

Wszystkie figury znajdujące się wewnątrz prostokąta są kwadratami. Czarny kwadrat ma pole l,

kwadrat $A$ ma pole 81. Wyznacz po1e kwadratu $X$ oraz oblicz obwód $\mathrm{d}\mathrm{u}\dot{\mathrm{z}}$ ego prostokąta.

{\it A}

{\it X}

ZADANIE 3.

Ania i Bartek odrabiają pracę domową. Ania ma znalez$\acute{}$ć najmniejszą liczbę naturalną podzielną

przez sześć kolejnych liczb nieparzystych, a Bartek - najmniejszą liczbę naturalna podzielną

przez osiem kolejnych liczb parzystych. Czyja liczba będzie mniejsza?

ZADANIE 4.

W pewnej rodzinie jest pięć córek: Ania, Basia, Czesia, Daria i Ela. Rodzily się one w podanej

kolejności co trzy lata. Najstarsza Aniajest siedem razy starsza od najmlodszej Eli. Ile lat ma

Czesia?

ZADANIE 5.

Przez jaką liczbę nalezy podzielić liczby 331 i 459, aby w obu przypadkach otrzymać resztę

z dzielenia równą ll? Podaj wszystkie rozwiązania.






LIGA MATEMATYCZNA

im. Zdzisława Matuskiego

FINAL

16 kwietnia 20l8

SZKOLA PODSTAWOWA

(klasy IV - VI)

ZADANIE I.

W rodzinie Bartka są cztery osoby. Suma ich lat jest równa 100. Bartek jest o 41ata starszy

od Ani, a tata jest o 61at starszy od mamy. Ania poprosi1a z1otą rybkę, aby cofnęfa czas

o calkowitą liczbę lat do momentu, w którym Ania byfa sześč razy młodsza od mamy. Zlota

rybka zastanowifa się i cofnęfa czas o pięč lat. Ile lat mają czlonkowie rodziny po cofnięciu

czasu?

ZADANIE 2.

Podaj wszystkie liczby trzycyfrowe o sumie cyfr równej 9 i cyfrze setek podzie1nej przez 4.

ZADANIE 3.

Sierzant przygotowywal oddzial $\dot{\mathrm{z}}$ ołnierzy do defilady. Próbowal ustawiać ich trójkami, ale

jeden $\dot{\mathrm{z}}$ olnierz pozostawał. Takz $\mathrm{e}$ po ustawieniu czwórkami, piatkami i szóstkamijeden $\dot{\mathrm{z}}$ ołnierz

zostawal. $\mathrm{W}$ końcu ustawil ich siódemkami i wtedy siódemki byly kompletne. Jaka mogla być

najmniejsza liczba $\dot{\mathrm{z}}$ ofnierzy w tym oddziale?

ZADANIE 4.

Wykaz, $\dot{\mathrm{z}}\mathrm{e}$ liczba $10^{20}+19^{3}-2$ jest podzielna przez 9.

ZADANIE 5.

Punkt $S$ jest środkiem okręgu opisanego na trójkącie ostrokątnym $ABC$. Miara kąta $ACS$ jest

trzy razy większa od miary kąta BAS, a miara kąta $CBS$ jest dwa razy większa od miary kąta

BAS. Wyznacz miary kątów trójkąta $ABC.$
\begin{center}
\includegraphics[width=37.344mm,height=36.576mm]{./LigaMatematycznaMatuskiego_SP_Zestaw5_2018_2019_page0_images/image001.eps}
\end{center}
C





\begin{center}
\includegraphics[width=20.628mm,height=30.024mm]{./LigaMatematycznaMatuskiego_SP_Zestaw5_2019_2020_page0_images/image001.eps}
\end{center}
0

flkademia

P omorskawStupsku

LIGA MATEMATYCZNA

im. Zdzisława Matuskiego

FINAL 26 marca 2019

SZKOLA PODSTAWOWA

(klasy IV - VI)

ZADANIE I.

Sakiewka Adama zawiera monety srebrne i zlote. Wszystkie monety srebrne są jednakowe

i wszystkie monety zlote są jednakowe. Dwie monety srebrne i cztery złote wazą fącznie 72 g,

a dwie zlote i cztery srebrne wazą 66 g. Ob1icz 1ączną wagę trzech zfotych i trzech srebrnych

monet.

ZADANIE 2.

Znajd $\acute{\mathrm{z}}$ wszystkie liczby trzycyfrowe, których iloczyn cyfr jest liczbą pierwszą.

ZADANIE 3.

Jedna z przekątnych dzieli pewien czworokąt na dwa trójkąty o obwodach 16 $\mathrm{i}18$, druga prze-

kątna dzieli ten czworokąt na trójkaty o obwodach 12 $\mathrm{i}20$. Oblicz róznicę dlugości przekątnych

tego czworokąta.

ZADANIE 4.

Iloczyn trzech liczb naturalnych jest równy 36. Nawet gdy podamy ich sumę, nie będzie wia-

domo, co to za liczby. Oblicz ich sumę.

ZADANIE 5.

Numer szyfru do szkatulki Basi sklada się z dziewięciu róznych cyfr i róznych od 0. $K\mathrm{a}\dot{\mathrm{z}}$ de dwie

kolejne cyfry szyfru róznią się o 31ub 5. Pierwszą cyfrą kodu jest 3. Wyznacz przedostatnią

cyfrę szyfru.




\end{document}