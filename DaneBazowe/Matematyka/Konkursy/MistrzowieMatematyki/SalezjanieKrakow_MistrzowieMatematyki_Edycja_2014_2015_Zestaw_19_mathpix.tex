\documentclass[10pt]{article}
\usepackage[polish]{babel}
\usepackage[utf8]{inputenc}
\usepackage[T1]{fontenc}
\usepackage{amsmath}
\usepackage{amsfonts}
\usepackage{amssymb}
\usepackage[version=4]{mhchem}
\usepackage{stmaryrd}

\title{GIMNAZJUM }

\author{}
\date{}


\begin{document}
\maketitle
\begin{enumerate}
  \item Na tablicy zapisujemy liczby od 1 do 10. Ścieramy dwie liczby i w ich miejsce wpisujemy ich sumę pomniejszoną o 1. Wykonujemy tę operację tyle razy, aż na tablicy zostanie tylko jedna liczba. Udowodnij, że niezależnie od tego, jak będziemy ścierać liczby, na końcu zawsze otrzymamy tę samą liczbę i podaj, co to za liczba.
  \item Dany jest okrąg \(O_{1}\) o środku \(S\) oraz okrąg \(O_{2}\), przechodzący przez \(S\) i przecinający okrąg \(O_{1}\) w punktach \(A\) i \(B\). Z punktu \(A\) poprowadzono prostą, przecinającą okrąg \(O_{1}\) w punkcie \(C\), a okrąg \(O_{2}\) w punkcie \(D\). Udowodnij, że trójkąt \(B C D\) jest równoramienny.
  \item O liczbach \(a, b, c, d\) wiadomo, że spełniają układ równań:
\end{enumerate}

\[
\left\{\begin{array}{c}
a+b+c+d=101 \\
a b+c d=200
\end{array}\right.
\]

Udowodnij, że dokładnie jedna z tych liczb jest nieparzysta.

\section*{LICEUM}
\begin{enumerate}
  \item Dany jest trójkąt prostokątny o przyprostokątnych długości odpowiednio \(a\) i \(b\). Na pierwszej z tych przyprostokątnych wybrano punkt \(P\), a na drugiej punkt \(Q\). Niech \(K\) i \(H\) będą rzutami prostokątnymi odpowiednio punktów \(P\) i \(Q\) na przeciwprostokątną. Jaka jest najmniejsza możliwa wartość sumy \(|K P|+|P Q|+|Q H|\) ? Odpowiedź uzasadnij.
  \item Mamy 17 liczb rzeczywistych. Wiadomo, że suma dowolnych dziewięciu spośród tych liczb jest większa od sumy pozostałych ośmiu. Wykaż, że wszystkie te liczby są dodatnie.
  \item Wyznacz wszystkie liczby całkowite nieujemne \(n\), dla których liczba \(7^{n}+2 \cdot 4^{n}\) jest liczbą pierwszą.
\end{enumerate}

Rozwiązania należy oddać do piątku 12 czerwca do godziny 12.30 koordynatorowi konkursu panu Jarostawowi Szczepaniakowi lub swojemu nauczycielowi matematyki.

Na stronie internetowej szkoły w zakładce Konkursy i olimpiady można znaleźć wyniki dotychczasowych rund i rozwiązania zadań.


\end{document}