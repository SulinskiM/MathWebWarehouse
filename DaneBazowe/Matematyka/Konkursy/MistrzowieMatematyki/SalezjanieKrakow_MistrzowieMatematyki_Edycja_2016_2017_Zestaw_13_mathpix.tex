\documentclass[10pt]{article}
\usepackage[polish]{babel}
\usepackage[utf8]{inputenc}
\usepackage[T1]{fontenc}
\usepackage{amsmath}
\usepackage{amsfonts}
\usepackage{amssymb}
\usepackage[version=4]{mhchem}
\usepackage{stmaryrd}

\title{GIMNAZJUM }

\author{}
\date{}


\begin{document}
\maketitle
\begin{enumerate}
  \item Ile dzielników ma liczba 2494800?
  \item Udowodnij, że wśród dowolnych 17 podzbiorów zbioru pięcioelementowego zawsze znajdą się dwa podzbiory rozłączne.
  \item Na tablicy napisano liczby od 1 do n. Dwóch graczy gra w grę, w której na przemian wykonują następujące czynności: każdy w swoim ruchu zmazuje dwie wybrane liczby z tablicy i zastępuje je ich (nieujemną) różnicą. Gra kończy się, gdy na tablicy zostanie tylko jedna liczba. Jeśli ona jest parzysta, wygrywa gracz pierwszy, a jeśli nieparzysta drugi. Kto wygra gre?
\end{enumerate}

\section*{LICEUM}
\begin{enumerate}
  \item Na ile sposobów zbiór \(\{1,2, \ldots, n\}\), gdzie \(n \geq 3\), można podzielić na trzy niepuste podzbiory?
  \item Wybrano 51 różnych liczb naturalnych mniejszych od 100. Udowodnić, że istnieją wśród nich takie dwie liczby, że pierwsza dzieli drugą.
  \item Na każdym polu szachownicy \(8 \times 8\) siedzi chrząszcz. 7 Chrząszczy choruje na pewną chorobą zakaźną. Zdrowy chrząszcz, którego pole sąsiaduje (bokiem) z co najmniej dwoma polami zarażonych chrząszczy, sam zostaje zarażony. Czy istnieje takie początkowe ustawienie siedmiu chorych chrząszczy, że po pewnym czasie choroba dopadnie wszystkich mieszkańców szachownicy?
\end{enumerate}

\end{document}