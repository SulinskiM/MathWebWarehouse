\documentclass[10pt]{article}
\usepackage[polish]{babel}
\usepackage[utf8]{inputenc}
\usepackage[T1]{fontenc}
\usepackage{amsmath}
\usepackage{amsfonts}
\usepackage{amssymb}
\usepackage[version=4]{mhchem}
\usepackage{stmaryrd}

\title{GIMNAZJUM }

\author{}
\date{}


\begin{document}
\maketitle
\begin{enumerate}
  \item Jacek chce się wybrać na plażę. Ma on w swojej szafie następujące (rozróżnialne) stroje: 6 par kąpielówek, 2 kapelusze słomkowe, 4 pary okularów przeciwsłonecznych oraz 3 podkoszulki. Regulamin plaży wymaga założenia kąpielówek, jednak noszenie kapeluszy, okularów przeciwsłonecznych czy podkoszulek nie jest obowiązkowe. Na ile sposobów może Jacek przygotować strój zgodnie z powyższymi warunkami?
  \item Ania miała fatalne wakacje. Nie było ani jednego dnia bez deszczu. Każdego dnia deszcz padał albo rano, albo po południu, albo cały dzień. W czasie jej wakacji było dokładnie 13 dni, kiedy było kilka godzin bez opadów, dokładnie 11 deszczowych poranków i dokładnie 12 deszczowych popołudni. Ile dni trwały wakacje Ani?
  \item Ile jest różnych płaszczyzn zawierających dokładnie cztery wierzchołki danego prostopadłościanu?
\end{enumerate}

\section*{LICEUM}
\begin{enumerate}
  \item Każda z istot zamieszkujących planetę Mati ma sześć, siedem lub osiem nóg. Wszyscy, którzy mają 7 nóg zawsze kłamią, a ci, którzy mają 6 lub 8 nóg zawsze mówią prawdé. Pewnego dnia król planety Mati przywołał swoich czterech poddanych i zapytał ich ile ich czwórka ma łącznie nóg. Pierwszy odpowiedział, że 25, drugi, że 26, trzeci, że 27, a czwarty, że 28. Ile tak naprawdę nóg miała łącznie ta czwórka poddanych?
  \item Niech \(f(x)=\frac{x^{2}}{1+x^{2}}\)
\end{enumerate}

Oblicz sumę \(f\left(\frac{1}{1}\right)+f\left(\frac{1}{2}\right)+\cdots+f\left(\frac{1}{10}\right)+f\left(\frac{2}{1}\right)+f\left(\frac{2}{2}\right)+\cdots+f\left(\frac{2}{10}\right)+f\left(\frac{3}{1}\right)+\) \(f\left(\frac{3}{2}\right)+. .+f\left(\frac{3}{10}\right)+\cdots+f\left(\frac{10}{1}\right) f\left(\frac{10}{2}\right)+\cdots+f\left(\frac{1}{1}\right)\)\\
3. Rozwiąż w liczbach całkowitych równanie

\[
2^{n} \cdot(4-n)=2 n+4
\]


\end{document}