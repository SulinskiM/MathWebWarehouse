\documentclass[10pt]{article}
\usepackage[polish]{babel}
\usepackage[utf8]{inputenc}
\usepackage[T1]{fontenc}
\usepackage{amsmath}
\usepackage{amsfonts}
\usepackage{amssymb}
\usepackage[version=4]{mhchem}
\usepackage{stmaryrd}

\title{GIMNAZJUM }

\author{}
\date{}


\newcommand\Varangle{\mathop{{<\!\!\!\!\!\text{\small)}}\:}\nolimits}

\begin{document}
\maketitle
\begin{enumerate}
  \item Na bokach \(B C\) i \(C D\) kwadratu \(A B C D\) wybrano odpowiednio takie punkty \(P\) i \(Q\), że \(B P+D Q=P Q\). Odcinki \(A P\) i \(A Q\) przecinają przekątną \(B D\) kwadratu \(A B C D\) w punktach odpowiednio \(M\) i \(N\). Wykazać, że \(M N^{2}=B M^{2}+D N^{2}\).
  \item Dodatnie liczby rzeczywiste \(a, b\) mają tę własność, że liczba \(\frac{a-b}{a+b}\) jest wymierna. Udowodnij, że również liczba \(\frac{2 a-b}{2 a+b}\) jest wymierna.
  \item Liczby \(a+b, b+c, c+d, d+e\) oraz \(e+a\) są wymierne. Czy możemy stąd wnioskować, że liczby \(a, b, c, d, e\) są wymierne?
\end{enumerate}

\section*{LICEUM}
\begin{enumerate}
  \item Punkty \(P\) i \(Q\) leżą odpowiednio na bokach \(B C\) i \(C D\) kwadratu \(A B C D\), przy czym \(\Varangle P A Q=45^{\circ}\). Punkt \(E\) jest rzutem prostokątnym punktu \(A\) na odcinek \(P Q\), a odcinki \(A P \mathrm{i}\) \(A Q\) przecinają przekątną \(B D\) kwadratu \(A B C D\) w punktach odpowiednio \(M\) i \(N\). Wykazać, że proste \(P N, Q M\) i \(A E\) przecinają się w jednym punkcie.
  \item Dane są różne dodatnie liczby wymierne \(x\) i \(y\), dla których liczba
\end{enumerate}

\[
w=\frac{x+\sqrt{y}}{y+\sqrt{x}}
\]

jest wymierna. Wykaż, że obie liczby \(x\) i \(y\) są kwadratami liczb wymiernych.\\
3. Liczby \(p, q, r\) są takimi liczbami wymiernymi, że \(p q+q r+r p=1\). Wykaż, że \(\sqrt{\left(1+p^{2}\right)\left(1+q^{2}\right)\left(1+r^{2}\right)}\) jest liczbą wymierną.

\section*{Uwaga zmiana! Rozwiązania można przesyłać do soboty.}

\end{document}