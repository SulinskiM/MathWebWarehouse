\documentclass[a4paper,12pt]{article}
\usepackage{latexsym}
\usepackage{amsmath}
\usepackage{amssymb}
\usepackage{graphicx}
\usepackage{wrapfig}
\pagestyle{plain}
\usepackage{fancybox}
\usepackage{bm}

\begin{document}

KORESPONDENCYJNY KURS PRZYGOTOWAWCZY Z

MATEMATYKI

PRACA KONTROLNA nr l

$\mathrm{p}\mathrm{a}\acute{\mathrm{z}}$dziernik 1999 $\mathrm{r}$

l. Stop składa się $\mathrm{z}$ 40\% srebra próby 0,6, 30\% srebra próby 0,7 oraz l kg srebra próby 0,8.

Jaka jest waga $\mathrm{i}$ jaka jest próba tego stopu?

2. Rozwiązač równanie

$3^{x}+1+3^{-x}+\ldots=4,$

którego lewa strona jest $\mathrm{s}\mathrm{u}\mathrm{m}\Phi$ nieskończonego ciągu geometrycznego.

3. $\mathrm{W}$ trójkącie $ABC$ znane są wierzcholki $A(0,0)$ oraz $B(4,-1)$. Wiadomo, $\dot{\mathrm{z}}\mathrm{e}\mathrm{w}$ punkcie

$H(3,2)$ przecinają się proste zawierające wysokości tego trójkąta. Wyznaczyč wspólrzędne

wierzchofka $C$. Wykonač odpowiedni rysunek.

4. Rozwiqzač równanie

$\cos 4x=\sin 3x.$

5. Wykonač staranny wykres funkcji

$f(x)=|\log_{2}(x-2)^{2}|.$

6. Rozwiązač nierównośč

$\displaystyle \frac{1}{x^{2}}\geq\frac{1}{x+6}.$

7. $\mathrm{W}$ ostrosłupie prawidłowym sześciokatnym krawęd $\acute{\mathrm{z}}$ podstawy ma długośč $p$, a krawędz/

boczna dfugośč $2p$. Obliczyč cosinus $\mathrm{k}_{\Phi^{\mathrm{t}\mathrm{a}}}$ dwuściennego między sąsiednimi ścianami bocz-

nymi tego ostrosłupa.

8. Wyznaczyč równania wszystkich prostych stycznych do wykresu funkcji $y=\displaystyle \frac{2x+10}{x+4}$, które

są równolegfe do prostej stycznej do wykresu funkcji $y = \sqrt{1-x}\mathrm{w}$ punkcie $x = 0.$

Rozwiqzanie zilustrowač rysunkiem.

1




PRACA KONTROLNA nr 2

listopad $1999\mathrm{r}$

l. Udowodnič, $\dot{\mathrm{z}}\mathrm{e}$ dla $\mathrm{k}\mathrm{a}\dot{\mathrm{z}}$ dego $n$ naturalnego wielomian $x^{4n-2}+1$ jest podzielny przez trój-

mian kwadratowy $x^{2}+1.$

2. $\mathrm{W}$ równoramienny trójkąt prostokątny $0$ polu powierzchni $S=10\mathrm{c}\mathrm{m}^{2}$ wpisano prostokąty

$\mathrm{w}$ ten sposób, $\dot{\mathrm{z}}\mathrm{e}$ jeden $\mathrm{z}$ jego boków $\mathrm{l}\mathrm{e}\dot{\mathrm{z}}\mathrm{y}$ na przeciwprostokątnej, a pozostale wierzchof-

ki znajdują się na przyprostokątnych. Znalez/č ten $\mathrm{z}$ prostokątów, który ma najkrótszą

przekątną $\mathrm{i}$ obliczyč jej długośč.

3. Rozwiązač nierównośč

log125 $3\cdot\log_{x}5+\log_{9}8\cdot\log_{4}x>1.$

4. Znalez$\acute{}$č wszystkie wartości parametru $p$, dla których wykres funkcji $y=x^{2}+4x+3\mathrm{l}\mathrm{e}\dot{\mathrm{z}}\mathrm{y}$

nad prostą $y=px+1.$

5. Zbadač liczbę rozwiązań równania

$||x+5|-1|=m$

$\mathrm{w}$ zalezności od parametru $m.$

6. Rozwiazač układ równań

$\left\{\begin{array}{l}
x^{2}+y^{2}=50\\
(x-2)(y+2)=-9
\end{array}\right.$

Podač interpretację $\mathrm{g}\mathrm{e}\mathrm{o}\mathrm{m}\mathrm{e}\mathrm{t}\mathrm{r}\mathrm{y}\mathrm{c}\mathrm{z}\mathrm{n}\Phi$ tego ukladu $\mathrm{i}$ wykonač odpowiedni rysunek.

7. Wyznaczyč na osi x-ów punkty A $\mathrm{i}\mathrm{B}, \mathrm{z}$ których okrąg $x^{2}+y^{2}-4x+2y=20$ widač pod

kątem prostym $\mathrm{t}\mathrm{z}\mathrm{n}$. styczne do okręgu wychodzące $\mathrm{z}\mathrm{k}\mathrm{a}\dot{\mathrm{z}}$ dego $\mathrm{z}$ tych punktów są do siebie

prostopadle. Obliczyč pole figury ograniczonej stycznymi do okręgu przechodzącymi przez

punkty A $\mathrm{i}$ B. Wykonač staranny rysunek.

8. $\mathrm{W}$ przedziale $[0,2\pi]$ rozwiązač równanie

$1-\mathrm{t}\mathrm{g}^{2}x+\mathrm{t}\mathrm{g}^{4}x-\mathrm{t}\mathrm{g}^{6}x+\ldots=\sin^{2}3x.$

2





PRACA KONTROLNA nr 3

grudzień $1999\mathrm{r}$

l. Nie korzystając $\mathrm{z}$ metod rachunku rózniczkowego wyznaczyč dziedzinę $\mathrm{i}$ zbiór wartości

funkcji

$y=\sqrt{2+\sqrt{x}-x}.$

2. Jednym $\mathrm{z}$ wierzchofków rombu $0$ polu 20 $\mathrm{c}\mathrm{m}^{2}$ jest $A(6,3)$, ajedna $\mathrm{z}$ przekątnych zawiera

się $\mathrm{w}$ prostej $0$ równaniu $2x+y=5$. Wyznaczyč równania prostych, $\mathrm{w}$ których zawierają

się boki $\overline{AB} \mathrm{i} \overline{AD}.$

3. Stosując zasadę indukcji matematycznej udowodnič prawdziwośč wzoru

$3(1^{5}+2^{5}+\displaystyle \ldots+n^{5})+(1^{3}+2^{3}+\ldots+n^{3})=\frac{n^{3}(n+1)^{3}}{2}.$

4. Ostrosłup prawidłowy trójkątny ma pole powierzchni całkowitej $P = 12\sqrt{3}\mathrm{c}\mathrm{m}^{2}$, a kąt

nachylenia ściany bocznej do płaszczyzny podstawy $\alpha = 60^{0}$ Obliczyč objętośč tego

ostrosfupa.

5. Wśród trójkątów równoramiennych wpisanych $\mathrm{w}$ koło $0$ promieniu $R$ znalez/č ten, który

ma największe pole.

6. Przeprowadzič badanie przebiegu funkcji $y=\displaystyle \frac{1}{2}x^{2}\sqrt{5-2x}\mathrm{i}$ wykonač jej staranny wykres.

7. $\mathrm{W}$ trapezie równoramiennym dane $\mathrm{s}\Phi$ ramię $r$, kąt ostry przy podstawie $\alpha$ oraz suma

długości przekątnej $\mathrm{i}$ dluzszej podstawy wynosząca $d$. Obliczyč pole trapezu oraz pro-

mień okręgu opisanego na tym trapezie. Ustalič warunki istnienia rozwiązania. Następnie

podstawič $\alpha=30^{0}, r=\sqrt{3}$ cm $\mathrm{i} d=6$ cm.

8. Rozwiązač nierównośč

$|\cos x+\sqrt{3}\sin x|\leq\sqrt{2},x\in[0,3\pi].$

3





PRACA KONTROLNA nr 4

styczeń $2000\mathrm{r}$

l. Rozwiązač równanie $16+19+22+\cdots+x=2000$, którego lewa strona jest sumq pewnej

liczby kolejnych wyrazów ciqgu arytmetycznego.

2. Spośród cyfr $0,1,\cdots,9$ losujemy bez zwracania pięč cyfr. Obliczyč prawdopodobieństwo

tego, $\dot{\mathrm{z}}\mathrm{e}\mathrm{z}$ otrzymanych cyfr $\mathrm{m}\mathrm{o}\dot{\mathrm{z}}$ na utworzyč liczbę podzielną przez 5.

3. Zbadač, czy istnieje pochodna funkcji $f(x)=\sqrt{1-\cos x}\mathrm{w}$ punkcie $x=0$. Wynik zilu-

strowač na wykresie funkcji $f(x).$

4. Udowodnič, $\dot{\mathrm{z}}\mathrm{e}$ dwusieczne kątów wewnętrznych równolegfoboku tworzą prostokąt, którego

przekątna ma dlugośč równą róznicy długości sąsiednich boków równoległoboku.

5. Rozwiązač uklad nierówności

$\left\{\begin{array}{l}
x+y\leq 3\\
\log_{y}(2^{x+1}+32)\leq 2\log_{y}(8-2^{x})
\end{array}\right.$

$\mathrm{i}$ zaznaczyč zbiór jego rozwiązań na p{\it l}aszczy $\acute{\mathrm{z}}\mathrm{n}\mathrm{i}\mathrm{e}.$

6. Wyznaczyč równanie zbioru wszystkich punktów pfaszczyzny Oxy będących środkami

okręgów stycznych wewnętrznie do okręgu $x^{2} +y^{2} = 25 \mathrm{i}$ równocześnie stycznych

zewnetrznie do okręgu $(x+2)^{2}+y^{2}= 1$. Jaką linię przedstawia znalezione równanie?

Sporządzič staranny rysunek.

7. Zbadač iloczyn pierwiastków rzeczywistych równania

$m^{2}x^{2}+8mx+4m-4=0$

jako funkcję parametru $\mathrm{m}$. Sporządzič wykres tej funkcji.

8. Podstawą czworościanu ABCD jest trójk$\Phi$t równoboczny ABC $0$ boku $\mathrm{a}$, ściana bocz-

na BCD jest trójkątem równoramiennym prostopadfym do pfaszczyzny podstawy, a kąt

płaski ściany bocznej przy wierzchołku A jest równy $\alpha$. Obliczyč pole powierzchni kuli

opisanej na tym czworościanie.

4





PRACA KONTROLNA nr 5

luty 2000r

l. Narysowač na płaszczy $\acute{\mathrm{z}}\mathrm{n}\mathrm{i}\mathrm{e}$ zbiór $A$ wszystkich punktów $(x,y)$, których wspófrzędne spef-

niają warunki

$||x| -y| \leq 1,$

$-1\leq x\leq 2,$

$\mathrm{i}$ znalez/č punkt zbioru $A\mathrm{l}\mathrm{e}\dot{\mathrm{z}}$ ący najblizej punktu $P(0,4).$

2. Obliczyč $\sin^{3}\alpha+\cos^{3}\alpha$ wiedząc, $\displaystyle \dot{\mathrm{z}}\mathrm{e}\sin 2\alpha=\frac{1}{4}$ oraz $\alpha\in (0,2\pi).$

3. Rozwazmy rodzinę prostych przechodzących przez punkt $P(0,-1) \mathrm{i}$ przecinających pa-

rabolę $y = \displaystyle \frac{1}{4}x^{2} \mathrm{w}$ dwóch punktach. Wyznaczyč równanie środków powstalych $\mathrm{w}$ ten

sposób cięciw paraboli. Sporządzič rysunek $\mathrm{i}$ opisač otrzymaną krzywq.

4. Rozwiązač równanie

$\sqrt{x+\sqrt{x^{2}-x+2}}-\sqrt{x-\sqrt{x^{2}-x+2}}=4.$

5. Dwóch strzelców wykonuje strzelanie. Pierwszy trafia do celu $\mathrm{z}$ prawdopodobieństwem $\displaystyle \frac{2}{3}$

$\mathrm{w}\mathrm{k}\mathrm{a}\dot{\mathrm{z}}$ dym strzale $\mathrm{i}$ wykonuje 4 strza1y, a drugi trafia $\mathrm{z}$ prawdpodobieństwem $\displaystyle \frac{1}{3}\mathrm{i}$ wykonuje

8 strzałów. Który ze strzelców ma większe prawdopodobieństwo uzyskania co najmniej

trzech trafień do celu, jeśli wyniki kolejnych strzafów są wzajemnie niezalezne?

6. Do naczynia $\mathrm{w}$ ksztalcie walca $0$ promieniu podstawy $\mathrm{R}$ wrzucono trzy jednakowe kulki

$0$ promieniu $\mathrm{r}$, przy czym $R< 2r < 2R$. Okazafo się, $\dot{\mathrm{z}}\mathrm{e}$ płaska pokrywa naczynia jest

styczna do kulki znajdującej się najwyzej $\mathrm{w}$ naczyniu. Obliczyč wysokośč naczynia.

7. Dla jakich wartości parametru $m$ funkcja

$f(x)=\displaystyle \frac{x^{3}}{mx^{2}+6x+m}$

jest określona $\mathrm{i}$ rosnąca na całej prostej rzeczywistej.

8. Dany jest trójk$\Phi$t $0$ wierzchofkach $A(-2,1), B(-1,-6), C(2,5)$. Poslugując się rachun-

kiem wektorowym obliczyč cosinus kąta pomiędzy dwusieczną kąta $A\mathrm{i}$ środkową boku

$\overline{BC}$. Wykonač rysunek.

5





PRACA KONTROLNA nr 6

marzec 2000r

l. Rozwiązač równanie

xlog2 $(2x-1)+\log_{2}(x+2) =\underline{1}$

$X^{2}.$

2. Styczna do okręgu $x^{2}+y^{2}-4x$ -- $2y = 5\mathrm{w}$ punkcie $\mathrm{M}(-1,2)$, prosta $l0$ równaniu

$24x+5y$ -- $12 =0$ oraz oś Ox tworzą trójk$\Phi$t. Obliczyč pole tego trójkąta $\mathrm{i}$ wykonač

rysunek.

3. Udowodnič prawdziwośč $\mathrm{t}\mathrm{o}\dot{\mathrm{z}}$ samości

COS $\alpha+$ COS $\displaystyle \beta+\cos\gamma=4\cos\frac{\alpha+\beta}{2}\cos\frac{\beta+\gamma}{2}$ COS $\displaystyle \frac{\gamma+\alpha}{2}$)

gdzie $\alpha, \beta, \gamma \mathrm{s}\Phi$ kątami ostrymi, których suma wynosi $\displaystyle \frac{\pi}{2}$

4. Dfugości krawędzi prostopadfościanu $0$ objętości $V = 8$ tworzą ciąg geometryczny, $\mathrm{a}$

stosunek długości przekątnej prostopadłościanu do najdłuzszej $\mathrm{z}$ przekątnych ścian tej

bryły wynosi $\displaystyle \frac{3}{4}\sqrt{2}$. Obliczyč pole powierzchni cafkowitej prostopadfościanu.

5. $\mathrm{Z}$ urny zawierającej siedem kul czarnych $\mathrm{i}$ trzy biafe wybrano losowo trzy kule $\mathrm{i}$ przełozono

do drugiej, pustej urny. Jakie jest prawdopodobieństwo wylosowania kuli białej $\mathrm{z}$ drugiej

urny?

6. Prostokąt obraca się wokół swojej przekątnej. Obliczyč objętośč powstałej bryły, jeśli

przekątna ma długośč $d$, a $\mathrm{k}\mathrm{a}\mathrm{t}$ pomiędzy przekątną, a dfuzszym bokiem ma miarę $\alpha.$

Wykonač odpowiedni rysunek.

7. Wyznaczyč największq $\mathrm{i}$ najmniejszą wartośč funkcji

$f(x) =x^{5/2}$ -- $10x^{3/2}+40x^{1/2}$

w przedziale [1,5].

8. Stosunek promienia okręgu wpisanego do promienia okręgu opisanego na trójkącie prosto-

kątnym jest równy k. Obliczyč w jakim stosunku środek okręgu wpisanego w ten trójkąt

dzieli dwusieczną kata prostego. Określič dziedzine dla parametru k.

6





PRACA KONTROLNA nr 7

kwiecień 2000r

l. Rozwiązač nierównośč

$|9^{x}-2|<3^{x+1}-2.$

2. Wyznaczyč równanie krzywej $\mathrm{b}\text{ę} \mathrm{d}_{\Phi}\mathrm{c}\mathrm{e}\mathrm{j}$ obrazem okręgu $(x+1)^{2}+(y-6)^{2}=4\mathrm{w}$ po-

winowactwie prostokqtnym $0$ osi $\mathrm{O}\mathrm{x}\mathrm{i}$ stosunku $k=\displaystyle \frac{1}{2}$. Obliczyč pole figury ograniczonej

$\mathrm{t}_{\Phi}$ krzywą. Wykonač staranny rysunek.

3. Pewien zbiór zawiera dokładnie 67 podzbiorów $0$ co najwyzej dwóch elementach. Ile

podzbiorów siedmioelementowych zawiera ten zbiór?

4. Na kole $0$ promieniu $R$ opisano trapez $0$ kątach przy dfuzszej podstawie $15^{0} \mathrm{i} 45^{0}$

Obliczyč stosunek pola koła do pola tego trapezu.

5. Rozwiązač uklad równań

$\left\{\begin{array}{l}
mx\\
2x
\end{array}\right.$

$+$

$6y$

$(m-7)y$

$=3$

$=m-1$

w zalezności od parametru rzeczywistego m. Podač wszystkie rozwiązania

(i odpowiadające im wartości parametru m), dla których x jest równe y.

6. Rozwiązač nierównośč

$\sin 2x<\sin x$

$\mathrm{w}$ przedziale $[-\displaystyle \frac{\pi}{2},\frac{\pi}{2}]. \mathrm{R}\mathrm{o}\mathrm{z}\mathrm{w}\mathrm{i}_{\Phi}$zanie zilustrowač starannym wykresem.

7. Ostroslup przecięto na trzy części dwiema plaszczyznami równoległymi do jego podstawy.

Pierwsza pfaszczyznajest połozona $\mathrm{w}$ odlegfości $d_{1} =2$ cm, a druga $\mathrm{w}$ odlegfości $d_{2}=3$

cm od podstawy. Pola przekrojów ostroslupa tymi plaszczyznami równe są odpowiednio

$S_{1} = 25 \mathrm{c}\mathrm{m}^{2}$ oraz $S_{2} = 16 \mathrm{c}\mathrm{m}^{2}$ Obliczyč objętośč tego ostrosłupa oraz objętośč

najmniejszej części.

8. Trylogię skladającą się $\mathrm{z}$ dwóch powieści dwutomowych oraz jednej jednotomowej usta-

wiono przypadkowo na półce. Jakie jest prawdopodobieństwo tego, $\dot{\mathrm{z}}\mathrm{e}$ tomy

a) obydwu, b) co najmniej jednej $\mathrm{z}$ dwutomowych powieści znajdują się obok siebie $\mathrm{i}$ przy

tym tom I $\mathrm{z}$ lewej, a tom II $\mathrm{z}$ prawej strony.

7



\end{document}