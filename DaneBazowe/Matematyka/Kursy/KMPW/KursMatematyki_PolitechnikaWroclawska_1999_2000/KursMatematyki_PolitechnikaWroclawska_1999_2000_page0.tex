\documentclass[a4paper,12pt]{article}
\usepackage{latexsym}
\usepackage{amsmath}
\usepackage{amssymb}
\usepackage{graphicx}
\usepackage{wrapfig}
\pagestyle{plain}
\usepackage{fancybox}
\usepackage{bm}

\begin{document}

KORESPONDENCYJNY KURS PRZYGOTOWAWCZY Z

MATEMATYKI

PRACA KONTROLNA nr l

$\mathrm{p}\mathrm{a}\acute{\mathrm{z}}$dziernik 1999 $\mathrm{r}$

l. Stop składa się $\mathrm{z}$ 40\% srebra próby 0,6, 30\% srebra próby 0,7 oraz l kg srebra próby 0,8.

Jaka jest waga $\mathrm{i}$ jaka jest próba tego stopu?

2. Rozwiązač równanie

$3^{x}+1+3^{-x}+\ldots=4,$

którego lewa strona jest $\mathrm{s}\mathrm{u}\mathrm{m}\Phi$ nieskończonego ciągu geometrycznego.

3. $\mathrm{W}$ trójkącie $ABC$ znane są wierzcholki $A(0,0)$ oraz $B(4,-1)$. Wiadomo, $\dot{\mathrm{z}}\mathrm{e}\mathrm{w}$ punkcie

$H(3,2)$ przecinają się proste zawierające wysokości tego trójkąta. Wyznaczyč wspólrzędne

wierzchofka $C$. Wykonač odpowiedni rysunek.

4. Rozwiqzač równanie

$\cos 4x=\sin 3x.$

5. Wykonač staranny wykres funkcji

$f(x)=|\log_{2}(x-2)^{2}|.$

6. Rozwiązač nierównośč

$\displaystyle \frac{1}{x^{2}}\geq\frac{1}{x+6}.$

7. $\mathrm{W}$ ostrosłupie prawidłowym sześciokatnym krawęd $\acute{\mathrm{z}}$ podstawy ma długośč $p$, a krawędz/

boczna dfugośč $2p$. Obliczyč cosinus $\mathrm{k}_{\Phi^{\mathrm{t}\mathrm{a}}}$ dwuściennego między sąsiednimi ścianami bocz-

nymi tego ostrosłupa.

8. Wyznaczyč równania wszystkich prostych stycznych do wykresu funkcji $y=\displaystyle \frac{2x+10}{x+4}$, które

są równolegfe do prostej stycznej do wykresu funkcji $y = \sqrt{1-x}\mathrm{w}$ punkcie $x = 0.$

Rozwiqzanie zilustrowač rysunkiem.

1
\end{document}
