\documentclass[a4paper,12pt]{article}
\usepackage{latexsym}
\usepackage{amsmath}
\usepackage{amssymb}
\usepackage{graphicx}
\usepackage{wrapfig}
\pagestyle{plain}
\usepackage{fancybox}
\usepackage{bm}

\begin{document}

PRACA KONTROLNA nr 2

listopad $1999\mathrm{r}$

l. Udowodnič, $\dot{\mathrm{z}}\mathrm{e}$ dla $\mathrm{k}\mathrm{a}\dot{\mathrm{z}}$ dego $n$ naturalnego wielomian $x^{4n-2}+1$ jest podzielny przez trój-

mian kwadratowy $x^{2}+1.$

2. $\mathrm{W}$ równoramienny trójkąt prostokątny $0$ polu powierzchni $S=10\mathrm{c}\mathrm{m}^{2}$ wpisano prostokąty

$\mathrm{w}$ ten sposób, $\dot{\mathrm{z}}\mathrm{e}$ jeden $\mathrm{z}$ jego boków $\mathrm{l}\mathrm{e}\dot{\mathrm{z}}\mathrm{y}$ na przeciwprostokątnej, a pozostale wierzchof-

ki znajdują się na przyprostokątnych. Znalez/č ten $\mathrm{z}$ prostokątów, który ma najkrótszą

przekątną $\mathrm{i}$ obliczyč jej długośč.

3. Rozwiązač nierównośč

log125 $3\cdot\log_{x}5+\log_{9}8\cdot\log_{4}x>1.$

4. Znalez$\acute{}$č wszystkie wartości parametru $p$, dla których wykres funkcji $y=x^{2}+4x+3\mathrm{l}\mathrm{e}\dot{\mathrm{z}}\mathrm{y}$

nad prostą $y=px+1.$

5. Zbadač liczbę rozwiązań równania

$||x+5|-1|=m$

$\mathrm{w}$ zalezności od parametru $m.$

6. Rozwiazač układ równań

$\left\{\begin{array}{l}
x^{2}+y^{2}=50\\
(x-2)(y+2)=-9
\end{array}\right.$

Podač interpretację $\mathrm{g}\mathrm{e}\mathrm{o}\mathrm{m}\mathrm{e}\mathrm{t}\mathrm{r}\mathrm{y}\mathrm{c}\mathrm{z}\mathrm{n}\Phi$ tego ukladu $\mathrm{i}$ wykonač odpowiedni rysunek.

7. Wyznaczyč na osi x-ów punkty A $\mathrm{i}\mathrm{B}, \mathrm{z}$ których okrąg $x^{2}+y^{2}-4x+2y=20$ widač pod

kątem prostym $\mathrm{t}\mathrm{z}\mathrm{n}$. styczne do okręgu wychodzące $\mathrm{z}\mathrm{k}\mathrm{a}\dot{\mathrm{z}}$ dego $\mathrm{z}$ tych punktów są do siebie

prostopadle. Obliczyč pole figury ograniczonej stycznymi do okręgu przechodzącymi przez

punkty A $\mathrm{i}$ B. Wykonač staranny rysunek.

8. $\mathrm{W}$ przedziale $[0,2\pi]$ rozwiązač równanie

$1-\mathrm{t}\mathrm{g}^{2}x+\mathrm{t}\mathrm{g}^{4}x-\mathrm{t}\mathrm{g}^{6}x+\ldots=\sin^{2}3x.$

2
\end{document}
