\documentclass[a4paper,12pt]{article}
\usepackage{latexsym}
\usepackage{amsmath}
\usepackage{amssymb}
\usepackage{graphicx}
\usepackage{wrapfig}
\pagestyle{plain}
\usepackage{fancybox}
\usepackage{bm}

\begin{document}

PRACA KONTROLNA nr 3

grudzień $1999\mathrm{r}$

l. Nie korzystając $\mathrm{z}$ metod rachunku rózniczkowego wyznaczyč dziedzinę $\mathrm{i}$ zbiór wartości

funkcji

$y=\sqrt{2+\sqrt{x}-x}.$

2. Jednym $\mathrm{z}$ wierzchofków rombu $0$ polu 20 $\mathrm{c}\mathrm{m}^{2}$ jest $A(6,3)$, ajedna $\mathrm{z}$ przekątnych zawiera

się $\mathrm{w}$ prostej $0$ równaniu $2x+y=5$. Wyznaczyč równania prostych, $\mathrm{w}$ których zawierają

się boki $\overline{AB} \mathrm{i} \overline{AD}.$

3. Stosując zasadę indukcji matematycznej udowodnič prawdziwośč wzoru

$3(1^{5}+2^{5}+\displaystyle \ldots+n^{5})+(1^{3}+2^{3}+\ldots+n^{3})=\frac{n^{3}(n+1)^{3}}{2}.$

4. Ostrosłup prawidłowy trójkątny ma pole powierzchni całkowitej $P = 12\sqrt{3}\mathrm{c}\mathrm{m}^{2}$, a kąt

nachylenia ściany bocznej do płaszczyzny podstawy $\alpha = 60^{0}$ Obliczyč objętośč tego

ostrosfupa.

5. Wśród trójkątów równoramiennych wpisanych $\mathrm{w}$ koło $0$ promieniu $R$ znalez/č ten, który

ma największe pole.

6. Przeprowadzič badanie przebiegu funkcji $y=\displaystyle \frac{1}{2}x^{2}\sqrt{5-2x}\mathrm{i}$ wykonač jej staranny wykres.

7. $\mathrm{W}$ trapezie równoramiennym dane $\mathrm{s}\Phi$ ramię $r$, kąt ostry przy podstawie $\alpha$ oraz suma

długości przekątnej $\mathrm{i}$ dluzszej podstawy wynosząca $d$. Obliczyč pole trapezu oraz pro-

mień okręgu opisanego na tym trapezie. Ustalič warunki istnienia rozwiązania. Następnie

podstawič $\alpha=30^{0}, r=\sqrt{3}$ cm $\mathrm{i} d=6$ cm.

8. Rozwiązač nierównośč

$|\cos x+\sqrt{3}\sin x|\leq\sqrt{2},x\in[0,3\pi].$

3
\end{document}
