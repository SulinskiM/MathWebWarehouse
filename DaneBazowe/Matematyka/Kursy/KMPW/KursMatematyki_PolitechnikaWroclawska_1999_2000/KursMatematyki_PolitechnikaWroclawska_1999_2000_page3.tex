\documentclass[a4paper,12pt]{article}
\usepackage{latexsym}
\usepackage{amsmath}
\usepackage{amssymb}
\usepackage{graphicx}
\usepackage{wrapfig}
\pagestyle{plain}
\usepackage{fancybox}
\usepackage{bm}

\begin{document}

PRACA KONTROLNA nr 4

styczeń $2000\mathrm{r}$

l. Rozwiązač równanie $16+19+22+\cdots+x=2000$, którego lewa strona jest sumq pewnej

liczby kolejnych wyrazów ciqgu arytmetycznego.

2. Spośród cyfr $0,1,\cdots,9$ losujemy bez zwracania pięč cyfr. Obliczyč prawdopodobieństwo

tego, $\dot{\mathrm{z}}\mathrm{e}\mathrm{z}$ otrzymanych cyfr $\mathrm{m}\mathrm{o}\dot{\mathrm{z}}$ na utworzyč liczbę podzielną przez 5.

3. Zbadač, czy istnieje pochodna funkcji $f(x)=\sqrt{1-\cos x}\mathrm{w}$ punkcie $x=0$. Wynik zilu-

strowač na wykresie funkcji $f(x).$

4. Udowodnič, $\dot{\mathrm{z}}\mathrm{e}$ dwusieczne kątów wewnętrznych równolegfoboku tworzą prostokąt, którego

przekątna ma dlugośč równą róznicy długości sąsiednich boków równoległoboku.

5. Rozwiązač uklad nierówności

$\left\{\begin{array}{l}
x+y\leq 3\\
\log_{y}(2^{x+1}+32)\leq 2\log_{y}(8-2^{x})
\end{array}\right.$

$\mathrm{i}$ zaznaczyč zbiór jego rozwiązań na p{\it l}aszczy $\acute{\mathrm{z}}\mathrm{n}\mathrm{i}\mathrm{e}.$

6. Wyznaczyč równanie zbioru wszystkich punktów pfaszczyzny Oxy będących środkami

okręgów stycznych wewnętrznie do okręgu $x^{2} +y^{2} = 25 \mathrm{i}$ równocześnie stycznych

zewnetrznie do okręgu $(x+2)^{2}+y^{2}= 1$. Jaką linię przedstawia znalezione równanie?

Sporządzič staranny rysunek.

7. Zbadač iloczyn pierwiastków rzeczywistych równania

$m^{2}x^{2}+8mx+4m-4=0$

jako funkcję parametru $\mathrm{m}$. Sporządzič wykres tej funkcji.

8. Podstawą czworościanu ABCD jest trójk$\Phi$t równoboczny ABC $0$ boku $\mathrm{a}$, ściana bocz-

na BCD jest trójkątem równoramiennym prostopadfym do pfaszczyzny podstawy, a kąt

płaski ściany bocznej przy wierzchołku A jest równy $\alpha$. Obliczyč pole powierzchni kuli

opisanej na tym czworościanie.

4
\end{document}
