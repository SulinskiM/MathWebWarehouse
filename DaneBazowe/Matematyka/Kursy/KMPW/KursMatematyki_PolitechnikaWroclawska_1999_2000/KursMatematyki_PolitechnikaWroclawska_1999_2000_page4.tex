\documentclass[a4paper,12pt]{article}
\usepackage{latexsym}
\usepackage{amsmath}
\usepackage{amssymb}
\usepackage{graphicx}
\usepackage{wrapfig}
\pagestyle{plain}
\usepackage{fancybox}
\usepackage{bm}

\begin{document}

PRACA KONTROLNA nr 5

luty 2000r

l. Narysowač na płaszczy $\acute{\mathrm{z}}\mathrm{n}\mathrm{i}\mathrm{e}$ zbiór $A$ wszystkich punktów $(x,y)$, których wspófrzędne spef-

niają warunki

$||x| -y| \leq 1,$

$-1\leq x\leq 2,$

$\mathrm{i}$ znalez/č punkt zbioru $A\mathrm{l}\mathrm{e}\dot{\mathrm{z}}$ ący najblizej punktu $P(0,4).$

2. Obliczyč $\sin^{3}\alpha+\cos^{3}\alpha$ wiedząc, $\displaystyle \dot{\mathrm{z}}\mathrm{e}\sin 2\alpha=\frac{1}{4}$ oraz $\alpha\in (0,2\pi).$

3. Rozwazmy rodzinę prostych przechodzących przez punkt $P(0,-1) \mathrm{i}$ przecinających pa-

rabolę $y = \displaystyle \frac{1}{4}x^{2} \mathrm{w}$ dwóch punktach. Wyznaczyč równanie środków powstalych $\mathrm{w}$ ten

sposób cięciw paraboli. Sporządzič rysunek $\mathrm{i}$ opisač otrzymaną krzywq.

4. Rozwiązač równanie

$\sqrt{x+\sqrt{x^{2}-x+2}}-\sqrt{x-\sqrt{x^{2}-x+2}}=4.$

5. Dwóch strzelców wykonuje strzelanie. Pierwszy trafia do celu $\mathrm{z}$ prawdopodobieństwem $\displaystyle \frac{2}{3}$

$\mathrm{w}\mathrm{k}\mathrm{a}\dot{\mathrm{z}}$ dym strzale $\mathrm{i}$ wykonuje 4 strza1y, a drugi trafia $\mathrm{z}$ prawdpodobieństwem $\displaystyle \frac{1}{3}\mathrm{i}$ wykonuje

8 strzałów. Który ze strzelców ma większe prawdopodobieństwo uzyskania co najmniej

trzech trafień do celu, jeśli wyniki kolejnych strzafów są wzajemnie niezalezne?

6. Do naczynia $\mathrm{w}$ ksztalcie walca $0$ promieniu podstawy $\mathrm{R}$ wrzucono trzy jednakowe kulki

$0$ promieniu $\mathrm{r}$, przy czym $R< 2r < 2R$. Okazafo się, $\dot{\mathrm{z}}\mathrm{e}$ płaska pokrywa naczynia jest

styczna do kulki znajdującej się najwyzej $\mathrm{w}$ naczyniu. Obliczyč wysokośč naczynia.

7. Dla jakich wartości parametru $m$ funkcja

$f(x)=\displaystyle \frac{x^{3}}{mx^{2}+6x+m}$

jest określona $\mathrm{i}$ rosnąca na całej prostej rzeczywistej.

8. Dany jest trójk$\Phi$t $0$ wierzchofkach $A(-2,1), B(-1,-6), C(2,5)$. Poslugując się rachun-

kiem wektorowym obliczyč cosinus kąta pomiędzy dwusieczną kąta $A\mathrm{i}$ środkową boku

$\overline{BC}$. Wykonač rysunek.

5
\end{document}
