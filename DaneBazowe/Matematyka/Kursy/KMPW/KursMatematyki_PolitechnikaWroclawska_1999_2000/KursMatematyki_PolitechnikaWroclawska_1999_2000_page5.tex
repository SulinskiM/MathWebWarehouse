\documentclass[a4paper,12pt]{article}
\usepackage{latexsym}
\usepackage{amsmath}
\usepackage{amssymb}
\usepackage{graphicx}
\usepackage{wrapfig}
\pagestyle{plain}
\usepackage{fancybox}
\usepackage{bm}

\begin{document}

PRACA KONTROLNA nr 6

marzec 2000r

l. Rozwiązač równanie

xlog2 $(2x-1)+\log_{2}(x+2) =\underline{1}$

$X^{2}.$

2. Styczna do okręgu $x^{2}+y^{2}-4x$ -- $2y = 5\mathrm{w}$ punkcie $\mathrm{M}(-1,2)$, prosta $l0$ równaniu

$24x+5y$ -- $12 =0$ oraz oś Ox tworzą trójk$\Phi$t. Obliczyč pole tego trójkąta $\mathrm{i}$ wykonač

rysunek.

3. Udowodnič prawdziwośč $\mathrm{t}\mathrm{o}\dot{\mathrm{z}}$ samości

COS $\alpha+$ COS $\displaystyle \beta+\cos\gamma=4\cos\frac{\alpha+\beta}{2}\cos\frac{\beta+\gamma}{2}$ COS $\displaystyle \frac{\gamma+\alpha}{2}$)

gdzie $\alpha, \beta, \gamma \mathrm{s}\Phi$ kątami ostrymi, których suma wynosi $\displaystyle \frac{\pi}{2}$

4. Dfugości krawędzi prostopadfościanu $0$ objętości $V = 8$ tworzą ciąg geometryczny, $\mathrm{a}$

stosunek długości przekątnej prostopadłościanu do najdłuzszej $\mathrm{z}$ przekątnych ścian tej

bryły wynosi $\displaystyle \frac{3}{4}\sqrt{2}$. Obliczyč pole powierzchni cafkowitej prostopadfościanu.

5. $\mathrm{Z}$ urny zawierającej siedem kul czarnych $\mathrm{i}$ trzy biafe wybrano losowo trzy kule $\mathrm{i}$ przełozono

do drugiej, pustej urny. Jakie jest prawdopodobieństwo wylosowania kuli białej $\mathrm{z}$ drugiej

urny?

6. Prostokąt obraca się wokół swojej przekątnej. Obliczyč objętośč powstałej bryły, jeśli

przekątna ma długośč $d$, a $\mathrm{k}\mathrm{a}\mathrm{t}$ pomiędzy przekątną, a dfuzszym bokiem ma miarę $\alpha.$

Wykonač odpowiedni rysunek.

7. Wyznaczyč największq $\mathrm{i}$ najmniejszą wartośč funkcji

$f(x) =x^{5/2}$ -- $10x^{3/2}+40x^{1/2}$

w przedziale [1,5].

8. Stosunek promienia okręgu wpisanego do promienia okręgu opisanego na trójkącie prosto-

kątnym jest równy k. Obliczyč w jakim stosunku środek okręgu wpisanego w ten trójkąt

dzieli dwusieczną kata prostego. Określič dziedzine dla parametru k.

6
\end{document}
