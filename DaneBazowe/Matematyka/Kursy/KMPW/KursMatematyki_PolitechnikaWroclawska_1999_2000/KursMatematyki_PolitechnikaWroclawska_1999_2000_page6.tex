\documentclass[a4paper,12pt]{article}
\usepackage{latexsym}
\usepackage{amsmath}
\usepackage{amssymb}
\usepackage{graphicx}
\usepackage{wrapfig}
\pagestyle{plain}
\usepackage{fancybox}
\usepackage{bm}

\begin{document}

PRACA KONTROLNA nr 7

kwiecień 2000r

l. Rozwiązač nierównośč

$|9^{x}-2|<3^{x+1}-2.$

2. Wyznaczyč równanie krzywej $\mathrm{b}\text{ę} \mathrm{d}_{\Phi}\mathrm{c}\mathrm{e}\mathrm{j}$ obrazem okręgu $(x+1)^{2}+(y-6)^{2}=4\mathrm{w}$ po-

winowactwie prostokqtnym $0$ osi $\mathrm{O}\mathrm{x}\mathrm{i}$ stosunku $k=\displaystyle \frac{1}{2}$. Obliczyč pole figury ograniczonej

$\mathrm{t}_{\Phi}$ krzywą. Wykonač staranny rysunek.

3. Pewien zbiór zawiera dokładnie 67 podzbiorów $0$ co najwyzej dwóch elementach. Ile

podzbiorów siedmioelementowych zawiera ten zbiór?

4. Na kole $0$ promieniu $R$ opisano trapez $0$ kątach przy dfuzszej podstawie $15^{0} \mathrm{i} 45^{0}$

Obliczyč stosunek pola koła do pola tego trapezu.

5. Rozwiązač uklad równań

$\left\{\begin{array}{l}
mx\\
2x
\end{array}\right.$

$+$

$6y$

$(m-7)y$

$=3$

$=m-1$

w zalezności od parametru rzeczywistego m. Podač wszystkie rozwiązania

(i odpowiadające im wartości parametru m), dla których x jest równe y.

6. Rozwiązač nierównośč

$\sin 2x<\sin x$

$\mathrm{w}$ przedziale $[-\displaystyle \frac{\pi}{2},\frac{\pi}{2}]. \mathrm{R}\mathrm{o}\mathrm{z}\mathrm{w}\mathrm{i}_{\Phi}$zanie zilustrowač starannym wykresem.

7. Ostroslup przecięto na trzy części dwiema plaszczyznami równoległymi do jego podstawy.

Pierwsza pfaszczyznajest połozona $\mathrm{w}$ odlegfości $d_{1} =2$ cm, a druga $\mathrm{w}$ odlegfości $d_{2}=3$

cm od podstawy. Pola przekrojów ostroslupa tymi plaszczyznami równe są odpowiednio

$S_{1} = 25 \mathrm{c}\mathrm{m}^{2}$ oraz $S_{2} = 16 \mathrm{c}\mathrm{m}^{2}$ Obliczyč objętośč tego ostrosłupa oraz objętośč

najmniejszej części.

8. Trylogię skladającą się $\mathrm{z}$ dwóch powieści dwutomowych oraz jednej jednotomowej usta-

wiono przypadkowo na półce. Jakie jest prawdopodobieństwo tego, $\dot{\mathrm{z}}\mathrm{e}$ tomy

a) obydwu, b) co najmniej jednej $\mathrm{z}$ dwutomowych powieści znajdują się obok siebie $\mathrm{i}$ przy

tym tom I $\mathrm{z}$ lewej, a tom II $\mathrm{z}$ prawej strony.

7
\end{document}
