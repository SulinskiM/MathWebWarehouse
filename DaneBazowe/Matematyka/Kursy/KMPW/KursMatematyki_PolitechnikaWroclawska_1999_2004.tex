\documentclass[a4paper,12pt]{article}
\usepackage{latexsym}
\usepackage{amsmath}
\usepackage{amssymb}
\usepackage{graphicx}
\usepackage{wrapfig}
\pagestyle{plain}
\usepackage{fancybox}
\usepackage{bm}

\begin{document}

Tadeusz Inglot

KURS

KORESPONDENCYJNY

MATEMATYKA

Zbiór $\mathrm{z}\mathrm{a}\mathrm{d}\mathrm{a}\acute{\mathrm{r}}\perp 1999-2004$




Recenzenci

Roicistaw RABCZUK

$\mathrm{Z}\mathrm{b}_{\mathrm{f}}$gniew ROMANOWICZ

Opracowanie redakcyjne

Atina KACZAK

Projekt okfadki

Zofia l- Danusz GODLEWSCY

\copyright Copyright by Oficyna Wydawnicza Politechniki Wroctawskiej, Wrociaw 2005

OFICYNA WYDAWNICZA POLITECHNIKI WROCLAWSKIEJ

Wybrzeze Wyspiańskiego 27, 50-370 Wroctaw

ISBN 83-7085-871-6

Diukarnia $\mathrm{O}\Gamma$icyny Wydawniczej Politechniki Wrociawskiej. $\mathrm{Z}\mathrm{a}\mathrm{m}_{\wedge}$ nr $403/200\overline{\supset}.$





14

Praca kontrolna nr 6

6.1. Rozwiazač równanie

$\displaystyle \mathrm{x}^{\log_{2}(2x-1)+\log_{2}(x+2)}=\frac{1}{x^{2}}.$

6.2. Styczna do okregu $x^{2}+y^{2}-4x-2y=5\mathrm{w}$ punkcie $M(-1,2)$, prosta

$l0$ równaniu $24x+5y-12=0$ oraz oś $Ox$ tworza trójkat. Obliczyč

pole tego trójkata. Sporzadzič rysunek.

6.3. Udowodnič prawdziwośč $\mathrm{t}\mathrm{o}\dot{\mathrm{z}}$ samości

$\displaystyle \cos\alpha+\cos\beta+\cos\gamma=4\cos\frac{\alpha+\beta}{2}\cos\frac{\beta+\gamma}{2}\cos\frac{\gamma+\alpha}{2},$

gdzie $\alpha, \beta, \gamma$ sa katami ostrymi, których suma wynosi $\displaystyle \frac{\pi}{2}.$

6.4. Dlugości krawedzi prostopadlościanu $0$ objetości $V = 8$ tworza ciag

geometryczny, a stosunek dlugości przekatnej prostopadlościanu do

najdluzszej $\mathrm{z}$ przekatnych jego ścian wynosi $\displaystyle \frac{3}{4}\sqrt{2}$. Obliczyč pole

powierzchni calkowitej prostopadlościanu.

6.5. $\mathrm{Z}$ urny zawierajacej siedem kul czarnych $\mathrm{i}$ trzy biale wybrano losowo

trzy kule $\mathrm{i}$ przelozono do drugiej, pustej urny. Jakie jest prawdopodo-

bieństwo wylosowania kuli bialej $\mathrm{z}$ drugiej urny?

6.6. Prostokat obraca $\mathrm{s}\mathrm{i}\mathrm{e}$ wokól swojej przekatnej. Obliczyč objetośč pow-

stalej bryly, jeśli przekatna ma dlugośč $d$, a $\mathrm{k}\mathrm{a}\mathrm{t}$ pomiedzy przekatna

$\mathrm{i}$ dluzszym bokiem ma miare $\alpha$. Sporzadzič odpowiedni rysunek.

6.7. Wyznaczyč najwieksza $\mathrm{i}$ najmniejsza wartośč funkcji

$f(x)=x^{5/2}-10x^{3/2}+40x^{1/2}$

w przedziale [1, 5].

6.8. Stosunek promienia okregu wpisanego w trójkat prostokatny do pro-

mienia okregu opisanego na tym trójkacie jest równy k. W jakim

stosunku środek okregu wpisanego w ten trójkat dzieli dwusieczna

kata prostego? Określič dziedzine dla parametru k.





118

20.4. Stosowač wzór na odleglośč punktu od prostej. Pamietač, $\dot{\mathrm{z}}\mathrm{e}$

rozwazamy tylko punkty wewnatrz danego trójkata. Nazwač wyznaczona

krzywa.

20.5. Rozwazyč przypadki $x > 1 \mathrm{i}x < 1 \mathrm{i}$ uprościč wzór określajacy

funkcje. Podczas rysowania wykresu pamietač $0$ dziedzinie funkcji.

20.6. Napisač $\displaystyle \frac{1}{x^{2}} = |x|^{-2}\mathrm{i}$ rozwazyč przypadki $|x| =1, |x| < 1$ oraz

$|x|>1$. Nie stosowač bezpośrednio definicji wartości bezwzglednej.

20.7. Warunek zadania oznacza, $\dot{\mathrm{z}}\mathrm{e}$ rozwazane styczne maja wspólczyn-

niki kierunkowe $+1\mathrm{l}\mathrm{u}\mathrm{b}-1$. Obliczyč pochodna funkcji $f$, przyrównač jej

wartośč bezwzgledna do l $\mathrm{i}$ rozwiazač otrzymane równanie niewymierne.

20.8. Oznaczyč $x=|AD|$ oraz $y=|AE|$. Ze stosunku pól obliczyč $xy,$

a $\mathrm{z}$ twierdzenia sinusów $\mathrm{w}$ trójkacie $ADE$ iloraz $\displaystyle \frac{x}{y}$. Nie wyznaczač jawnie

$x\mathrm{i}y$, lecz tylko sume $x+y$ (korzystač ze wzoru skróconego mnozenia).

21.1. Oznaczyč przez $x, y$ krawedzie mniejszych sześcianów. Napisač

uklad równań $\mathrm{z}$ niewiadomymi $x\mathrm{i}y\mathrm{i}$ nie wyznaczajac ich jawnie, obliczyč

tylko $x^{2} +y^{2}$ za pomoca wzorów skróconego mnozenia. Stad od razu

otrzymač odpowied $\acute{\mathrm{z}}.$

21.2. Wyznaczyč wektory $\vec{AC}\mathrm{i}\vec{BD}\mathrm{i}$ zastosowač iloczyn skalarny oraz

$\mathrm{t}\mathrm{o}\dot{\mathrm{z}}$ samośč podana we wskazówce do $\mathrm{z}\mathrm{a}\mathrm{d}$. 2.8.

21.3. Wyznaczyč skale podobieństwa trójkatów i wyrazič przeciwpros-

tokatna przez promień okregu r. Stad obliczyč sume przyprostokatnych

wyjściowego trójkata iw konsekwencji sume cosinusów katów ostrych trój-

kata. Podnoszac te równośč do kwadratu obliczyč oba cosinusy.

21.4. Przenieśč niewymiernośč do mianownika $\mathrm{i}$ podzielič licznik $\mathrm{i}$ mia-

nownik przez $n$. Korzystač $\mathrm{z}$ faktu, $\dot{\mathrm{z}}\mathrm{e}$ zlozenie funkcji malejacych jest

funkcja rosnaca.

21.5. Korzystač ze wzoru podanego we wskazówce do zadania 3.8.





119

21.6. Napisač warunki określajace dziedzine, ale nie wyznaczač dziedzi-

ny w sposóbjawny. Sprowadzič logarytmy do wspólnej podstawy 4 i przejśč

do równania algebraicznego trzeciego stopnia. Obliczyč jego pierwiastki

i wybrač te, które naleza do dziedziny.

21.7. Narysowač przekrój osiowy stozka. Objetośč wyrazičjako funkcje

wysokości stozka. Nie mylič tego zadania z zagadnieniem wyznaczania

ekstremów lokalnych.

21.8. Obie parabole lacznie ze stycznymi tworza figure majaca środek

symetrii S (dlaczego?). Wiec szukane styczne przechodza przez punkt S.

Wyznaczyč S. Napisač równanie peku prostych przechodzacych przez S

i z warunku styczności (wyróznik odpowiedniego równania kwadratowego

równy zeru) obliczyč wspólczynniki kierunkowe szukanych stycznych.

22.1. Wykorzystač parzystośč funkcji.

zwrócič uwage na otoczenie punktu $x=0.$

Podczas rysowania wykresu

22.2. Uzasadnič, $\dot{\mathrm{z}}\mathrm{e}$ liczby metrów sześciennych wody wplywajace do

basenu $\mathrm{w}$ kolejnych minutach tworza ciag arytmetyczny. We wszystkich

obliczeniach przyjač $minut_{G}$ jako jednostke czasu. Dane liczbowe podstawič

na końcu.

22.3. Oznaczyč średnice obu podstaw przez $x\mathrm{i}y$. Ulozyč uklad równań

$\mathrm{z}$ niewiadomymi $x, y \mathrm{i}$ przejśč od razu do alternatywy ukladów równań

liniowych.

22.4. $\mathrm{Z}$ twierdzenia sinusów wynika, $\dot{\mathrm{z}}\mathrm{e}$ znany jest takze bok $|BC|.$

$\mathrm{W}$ okregu $0$ promieniu $R$ zaznaczyč cieciwe $0$ dlugości $|BC| \mathrm{i}$ rozwazač

katy wpisane oparte na luku wyznaczonym przez $\mathrm{t}\mathrm{e}$ cieciwe. Wybrač takie

polozenie (polozenia) wierzcholka $A$, które daje $|AB|=\displaystyle \frac{8}{5}R. \mathrm{W}$ zalezności

od wielkości kata $\alpha$ (czyli dlugości cieciwy $|BC|$) mamy rózne przypadki,

które nalezy kolejno rozpatrzyč.

22.5. Od razu zlogarytmowač obie strony,

logarytmu liczbe 8.

przyjmujac za podstawe





120

22.6. Wyrazič wektory

iloczynu skalarnego.

$\vec{CB}$

i

$\vec{CD}$ przez $\vec{AB}=\vec{u} \mathrm{i} \vec{BD}=\vec{v}\mathrm{i}\mathrm{u}\dot{\mathrm{z}}$ yč

22.7. Wyznaczyč dziedzine równania. Pomnozyč obie strony przez

wyrazenie $(\sin x\cos x) \mathrm{i}$ doprowadzič do równania elementarnego postaci

$\sin(f(x))=\sin(g(x))$. Rozwiazania zapisač $\mathrm{w}$ postaci jednej serii.

22.8. Napisač równanie stycznej $\mathrm{w}$ punkcie $x_{0}$, wyznaczyč punkty prze-

cieč tej stycznej $\mathrm{z}$ osiami ukladu wspólrzednych $\mathrm{i}$ wyrazič kwadrat dlugości

odcinka stycznej jako funkcje $x_{0}$. Do rózniczkowania pozostawič $\mathrm{t}\mathrm{e}$ funkcje

$\mathrm{w}$ postaci sumy funkcji potegowych. Nie mylič postawionego pytania $\mathrm{z}$ za-

gadnieniem wyznaczania ekstremów lokalnych.

23.1. Liczba,,slów'' utworzonych $\mathrm{z}$ danych liter odpowiada liczbie per-

mutacji $\mathrm{z}$ powtórzeniami.

23.2. Zadanie rozwiazač bez dzielenia wielomianów. Zauwazyč, $\dot{\mathrm{z}}\mathrm{e}$ i10-

raz danych wielomianów ma postač $ x+\alpha \mathrm{i}$ wyznaczyč najpierw niewia-

doma $\alpha.$

23.3. Wykorzystač symetrie figury $\mathrm{i}$ twierdzenie $0$ okregach wzajemnie

stycznych.

23.4. Przez punkty $K \mathrm{i} L$ poprowadzič plaszczyzny prostopadle do

plaszczyzny podstawy $\mathrm{i}$ równolegle do $BC$. Obliczač oddzielnie objetości

$\mathrm{k}\mathrm{a}\dot{\mathrm{z}}$ dej $\mathrm{z}$ tak otrzymanych bryl (dwie $\mathrm{z}$ nich sa identyczne). Por. $\mathrm{z}\mathrm{a}\mathrm{d}$. 15.3.

23.5. Wyznaczyč dziedzine nierówności. Rozpatrzyč najpierw oczy-

wisty przypadek $x < 0$. Dla $x > 0$ podnieśč obie strony nierówności

do kwadratu $\mathrm{i}$ rozwiazač nierównośč dwukwadratowa. Wykresem funkcji

$\mathrm{z}$ prawej strony nierówności nie jest luk paraboli lecz inna dobrze znana

krzywa (por. wskazówka do $\mathrm{z}\mathrm{a}\mathrm{d}$. 13.7).

23.6. Dowód kroku indukcyjnego przeprowadzič wprost. Nie stosowač

niewygodnej metody redukcji. Dbač $0$ logiczna poprawnośč zapisu dowodu.





121

23.7. Punkt $M(y_{0}^{2},y_{0}), y_{0}>0, \mathrm{l}\mathrm{e}\dot{\mathrm{z}}\mathrm{y}$ najblizej $P$, gdy odcinek $PM$ jest

prostopadly do stycznej do danej krzywej $\mathrm{w}$ punkcie $M. \mathrm{U}\dot{\mathrm{z}}$ yč rachunku

wektorowego.

23.8. Poniewaz wspólczynnik przy $x^{2}$ jest dodatni, wiec pierwiastki

trójmianu kwadratowego beda $\mathrm{l}\mathrm{e}\dot{\mathrm{z}}$ eč $\mathrm{w}$ odcinku $(0,1)$, gdy odcieta wierz-

cholka paraboli bedacej jego wykresem znajdzie $\mathrm{s}\mathrm{i}\mathrm{e} \mathrm{w}$ tym przedziale,

a wartości trójmianu dla $x=0\mathrm{i}x=1$ beda dodatnie. Otrzymane nierów-

ności trygonometryczne rozwiazač analitycznie. Ewentualny rysunek sluzy

do ilustracji rozwiazania.

24.1. Pamietač 0 warunku istnienia sumy nieskończonego ciagu geo-

metrycznego.

24.2. Zaczač od określenia modelu probabilistycznego, $\mathrm{t}\mathrm{j}$. zbioru zdarzeń

elementarnych $\Omega$ oraz prawdopodobieństwa $P$. Oznaczyč przez $A$ zdarze-

nie polegajace na $\mathrm{t}\mathrm{y}\mathrm{m}, \dot{\mathrm{z}}\mathrm{e}$ kości pasuja do siebie, a przez $A_{i}$ zdarzenie, $\dot{\mathrm{z}}\mathrm{e}$

na jednym $\mathrm{z}$ pól obu kości jest $i$ oczek, a na pozostalych polach cokolwiek,

$i=0$, 6. Wtedy $ A=A_{0}\cup \cup A_{6} \mathrm{i}$ skladniki parami wykluczaja $\mathrm{s}\mathrm{i}\mathrm{e}$

(dlaczego?). Obliczyč $P(A_{i})\mathrm{i}$ skorzystač $\mathrm{z}$ wlasności prawdopodobieństwa.

24.3. Wykazač, $\dot{\mathrm{z}}\mathrm{e}$ dla $m=10$ uklad jest sprzeczny, a dla $m\neq 10$ ma

jedno rozwiazanie. Zauwazyč, $\dot{\mathrm{z}}\mathrm{e}$ dla $\dot{\mathrm{z}}$ adnego $m \in \mathrm{R}$ para (l, l) nie jest

rozwiazaniem ukladu.

24.4. Określič dziedzine dla kata $\alpha$ porównujac ten $\mathrm{k}\mathrm{a}\mathrm{t}\mathrm{z}$ jego rzutem

prostokatnym na podstawe. $\mathrm{Z}$ twierdzenia $0$ trzech prostopadlych uza-

sadnič, $\dot{\mathrm{z}}\mathrm{e}$ {\it AB} $\perp BD'$. Wywnioskowač stad, $\dot{\mathrm{z}}\mathrm{e} \mathrm{k}\mathrm{a}\mathrm{t} DBD'$ jest katem

plaskim kata dwuściennego miedzy plaszczyzna ABD'E' $\mathrm{i}$ podstawa gra-

niastoslupa.

24.5. Rozwazyč przypadki $x < 1$ oraz $x > 1 \mathrm{i}$ pomnozyč obie strony

przez mianownik (dodatni lub ujemny, odpowiednio). Jedna $\mathrm{z}$ nierówności

podwójnych jest automatycznie spelniona, a druga, przez podstawienie

$2^{x}=t$, sprowadza $\mathrm{s}\mathrm{i}\mathrm{e}$ do do nierówności kwadratowej. Nie potrzeba rozwazač

nierówności $\mathrm{w}\mathrm{y}\dot{\mathrm{z}}$ szego stopnia.





122

24.6. Zauwazyč, $\dot{\mathrm{z}}\mathrm{e}$ tg $82^{\circ}30' = \displaystyle \frac{1}{\mathrm{t}\mathrm{g}7^{\circ}30'}$ oraz $\dot{\mathrm{z}}\mathrm{e}82^{\circ}30'-7^{\circ}30' = 75^{\circ}$

$\mathrm{i}$ zastosowač wzór na tangens róznicy katów. Nastepnie korzystač $\mathrm{z}$ rów-

ności $75^{\circ}=45^{\circ}+30^{\circ}$

24.7. Skorzystač ze wskazówki do zadania 6.2, a w drugiej cześci rozwia-

zania ze wskazówki do zad. 5.8.

24.8. Przypadek $\alpha=1$ wymaga oddzielnego rozpatrzenia (dlaczego?).

Pochodna funkcji $\displaystyle \frac{b}{x^{2}-1}=b(x^{2}-1)^{-1}$ wygodniej jest obliczač za pomoca

reguly rózniczkowania funkcji zlozonej. Zauwazyč, $\dot{\mathrm{z}}\mathrm{e}$ dla $\alpha= 3, b= 32,$

gwarantujacych ciaglośči rózniczkowalnośč $f(x)$, punkt $P(3$, 4$)$ jestjej punk-

tem przegiecia.

25.1.

$t\neq 0.$

Najpierw rozpatrzyč oczywisty przypadek $t = 0$, a nastepnie

25.2. Korzystajac $\mathrm{z}$ twierdzenia Talesa wykazač, $\dot{\mathrm{z}}\mathrm{e}$ przekrój jest równo-

leglobokiem. Nastepnie prowadzič plaszczyzne symetrii czworościanu $\mathrm{i}$ sto-

sujac twierdzenie $0$ trzech prostopadlych, wykazač, $\dot{\mathrm{z}}\mathrm{e}$ przekrój jest pros-

tokatem.

25.3. Określič dziedzine nierówności. Zauwazyč, $\dot{\mathrm{z}}\mathrm{e}$ szukany zbiór jest

symetryczny wzgledem poczatku ukladu, co pozwala ograniczyč rozwazania

do I čwiartki ukladu. Rozpatrzyč przypadki $xy>1$ oraz $xy<1.$

25.4. Pólprosta wychodzaca ze środka okregu $\mathrm{i}$ zawierajaca dany punkt

$A$ przecina ten okrag $\mathrm{w}$ punkcie $A'\mathrm{l}\mathrm{e}\dot{\mathrm{z}}$ acym najblizej punktu $A$. Stad $|AA'|$

jest odleglościa punktu $A$ od danego okregu. Prowadzac rozwazania geo-

metryczne uzasadnič, $\dot{\mathrm{z}}\mathrm{e}$ dla punktów $\mathrm{l}\mathrm{e}\dot{\mathrm{z}}$ acych wewnatrz okregu zachodzi

relacja $OA+PA=10$, co oznacza, $\dot{\mathrm{z}}\mathrm{e}A\mathrm{l}\mathrm{e}\dot{\mathrm{z}}\mathrm{y}$ na elipsie $0$ ogniskach $O\mathrm{i}P$

(por. wskazówka do $\mathrm{z}\mathrm{a}\mathrm{d}$. 4.6). Inaczej jest, gdy $A\mathrm{l}\mathrm{e}\dot{\mathrm{z}}\mathrm{y}$ na zewnatrz danego

okregu.

25.5. Wszystkie przeprowadzane losowania sa wzajemnie niezalezne,

wiec ich kolejnośč nie ma wplywu na prawdopodobieństwo rozwazanego

zdarzenia. Oznaczyč przez $K, N$ zdarzenia polegajace na $\mathrm{t}\mathrm{y}\mathrm{m}, \dot{\mathrm{z}}\mathrm{e}$ dziecko,

odpowiednio, Kowalskich, Nowakowskich zostalo wybrane przedstawicielem.





123

Wówczas $ K\cap N=\emptyset$ oraz $ K\cup N=\Omega$. Nastepnie zastosowač wzór na praw-

dopodobieństwo calkowite.

25.6. Unikač niewygodnego dowodu redukcyjnego, ajeśli $\mathrm{s}\mathrm{i}\mathrm{e}$ go stosuje,

pamietač $0$ odpowiednim zakończeniu potrzebnym dla poprawności rozu-

mowania.

25.7. Nie tracič czasu na badanie wlasności, których ta funkcja nie $\mathrm{m}\mathrm{o}\dot{\mathrm{z}}\mathrm{e}$

mieč (np. asymptoty ukośne). Do obliczania pochodnej przedstawič funkcje

$\mathrm{w}$ postaci iloczynu funkcji potegowych, $\mathrm{t}\mathrm{j}. f(x)=\sqrt{3}(x-1)^{1/2}(5-x)^{-1/2}$

$\mathrm{i}$ zastosowač regule rózniczkowania iloczynu. Zauwazyč, a nastepnie wyka-

zač, $\dot{\mathrm{z}}\mathrm{e}$ prosta $x= 1$ jest styczna do wykresu $f(x) \mathrm{w}$ punkcie $x= 1$ (por.

wskazówka do $\mathrm{z}\mathrm{a}\mathrm{d}$. 3.6).

25.8. Wykazač, $\dot{\mathrm{z}}\mathrm{e}$ kolejne odcinki lamanej tworza ciag geometryczny

$0$ ilorazie mniejszym od l. Nastepnie zastosowač wzór na sume wyrazów

nieskończonego ciagu geometrycznego lub uzasadnič, $\dot{\mathrm{z}}\mathrm{e}$ suma tajest równa

obwodowi danego trójkata.

26.1. Odcinek pasa laczacy oba kola jest styczny do $\mathrm{k}\mathrm{a}\dot{\mathrm{z}}$ dego $\mathrm{z}$ nich,

wiec prostopadly do promieni poprowadzonych do punktów styczności. Nie

$\mathrm{u}\dot{\mathrm{z}}$ ywač zapisu postaci $ 26\displaystyle \frac{2}{3}\pi$ cm który jest niejednoznaczny.

26.2. Zachowač podana $\mathrm{w}$ zadaniu kolejnośč obliczeń.

26.3. Wygodnie jest posluzyč $\mathrm{s}\mathrm{i}\mathrm{e}$ rachunkiem wektorowym. Oznaczyč

przez $A, B$ punkty przeciecia $\mathrm{s}\mathrm{i}\mathrm{e}$ szukanej prostej $l$ odpowiednio $\mathrm{z}$ prosta $k$

$\mathrm{i}m$. Wówczas mamy $A(x,x+3)$. Wyrazič $\vec{AP}\mathrm{i}\vec{AB}=2\vec{AP}$ przy pomocy

niewiadomej $x \mathrm{i}$ korzystajac $\mathrm{z}$ faktu, $\dot{\mathrm{z}}\mathrm{e} B \mathrm{l}\mathrm{e}\dot{\mathrm{z}}\mathrm{y}$ na prostej $m$ obliczyč $x.$

$\rightarrow$

Wektor normalny prostej $l$ jest prostopadly do $AB.$

26.4. Wierzcholek $C$ kata prostego, spodek $O$ wysokości ostroslupa

ijego rzuty prostokatne $K, L$ na przyprostokatne podstawy tworza kwadrat

$0$ boku $r$. Stad wynika, $\dot{\mathrm{z}}\mathrm{e}$ rzuty prostokatne punktów $K\mathrm{i}L$ na krawed $\acute{\mathrm{z}}DC$

pokrywaja $\mathrm{s}\mathrm{i}\mathrm{e}$ ($\mathrm{w}$ punkcie $E$), zatem $\beta=\angle KEL$. Wyznaczyč dziedzine dla

$\beta$. Wysokośč czworościanu obliczyč $\mathrm{z}$ podobieństwa odpowiednich trójkatów

$\mathrm{w}$ przekroju plaszczyzna $ODC.$





124

26.5. Skorzystač $\mathrm{z}$ parzystości funkcji oraz ze wzoru $\log_{c}\alpha^{2}=2\log_{c}|\alpha|,$

$c>0, c\neq 1, \alpha\neq 0$. Wykres funkcji $f$ otrzymujemy $\mathrm{z}$ wykresu standardowej

krzywej $y=\log_{2}x$ przez translacje $\mathrm{i}$ odbicia symetryczne.

26.6. Zastosowač wzór $\displaystyle \cos 2x=\frac{1-\mathrm{t}\mathrm{g}^{2}x}{1+\mathrm{t}\mathrm{g}^{2}x}\mathrm{i}$ podstawič tg $x=t.$

26.7. Dwie funkcje $\mathrm{m}\mathrm{o}\dot{\mathrm{z}}$ na zlozyč wtedy $\mathrm{i}$ tylko wtedy, gdy zbiór wartości

funkcji wewnetrznej jest zawarty $\mathrm{w}$ dziedzinie funkcji zewnetrznej. Przy-

padek $\alpha=0$ rozpatrzyč oddzielnie.

26.8. Patrz wskazówka do zadania l2.8.

27.1. Wyznaczyč dziedzine równania. Aby istnialo rozwiazanie, prawa

strona musi byč nieujemna. Wtedy obie strony $\mathrm{m}\mathrm{o}\dot{\mathrm{z}}$ na podnieśč do kwadratu.

Przypadek $p=0$ rozpatrzyč oddzielnie.

27.2. Zauwazyč, $\dot{\mathrm{z}}\mathrm{e}$ środki okregów $K\mathrm{i}K_{1}$ oraz punkt $S\mathrm{l}\mathrm{e}\dot{\mathrm{z}}$ a na prostej

prostopadlej do danej prostej. Nastepnie korzystač $\mathrm{z}$ zalezności miedzy

promieniami rozwazanych okregów.

27.3. Dane określaja jednoznacznie przekatna $AC$ trapezu, na której,

jako na cieciwie okregu, jest oparty $\mathrm{k}\mathrm{a}\mathrm{t}$ ostry przy wierzcholku $B$ podstawy.

Przez zmiane polozenia punktu $B$ na okregu, poczynajac od punktu $C,$

otrzymujemy rózne trapezy (tj. $0$ róznych wartościach $d$). Minimalne $d$

odpowiada sytuacji, gdy krótsza podstawa trapezu jest równa zeru $\mathrm{i}$ trapez

staje $\mathrm{s}\mathrm{i}\mathrm{e}$ trójkatem, a maksymalne, gdy $B$ pokrywa $\mathrm{s}\mathrm{i}\mathrm{e} \mathrm{z} C$. Wysokośč

trapezu obliczyč $\mathrm{z}$ twierdzenia Pitagorasa $\mathrm{i}$ stad bezpośrednio ramie trapezu.

27.4. Najpierw ustalič dziedzine dla kata $\beta$ (porównujac go $\mathrm{z}$ rzutem

prostokatnym na podstawe). $\mathrm{Z}$ twierdzenia $0$ trzech prostopadlych wywnios-

kowač, $\dot{\mathrm{z}}\mathrm{e}$ przekrój ostroslupa jest deltoidem. $\mathrm{W}$ obliczeniach korzystač

$\mathrm{z}$ podobieństwa trójkatów $\mathrm{i}$ twierdzenia $0$ środkowych $\mathrm{w}$ trójkacie.

27.5. Przenieśč niewymiernośč do mianownika, stosujac wzór na róznice

sześcianów $\mathrm{i}$ podzielič licznik $\mathrm{i}$ mianownik przez $n^{13/5}$ Skorzystač $\mathrm{z}$ faktu,

$\dot{\mathrm{z}}\mathrm{e}\alpha<3.$





125

27.6. Stosujac definicje logarytmu sprowadzič dana nierównośč do pros-

tej nierówności trygonometrycznej. Od razu ograniczyč $\mathrm{s}\mathrm{i}\mathrm{e}$ do dziedziny

(I čwiartka, cosinus dodatni), co pozwala latwo rozwiazač $\mathrm{t}\mathrm{e}$ nierównośč.

27.7. Rozwazmy losowanie jednej liczby $\mathrm{i}$ odpowiadajacy mu model

probabilistyczny $\Omega_{0} \mathrm{i}P_{0}$. Niech $ A\subset \Omega_{0}$ oznacza zdarzenie, $\dot{\mathrm{z}}\mathrm{e}$ liczba czy-

tana od strony lewej do prawej jest podzielna przez 4, a $B$ zdarzenie, $\dot{\mathrm{z}}\mathrm{e}$

liczba czytana od strony prawej do lewej jest podzielna przez 4. Wówczas

zdarzenia $A, B$ sa niezalezne (dlaczego?). $P_{0}(A\cup B)$ obliczyč, znajac

$P_{0}(A)\mathrm{i}P_{0}(B)$. Zauwazyč, $\dot{\mathrm{z}}\mathrm{e}P_{0}(A\cup B)$ jest prawdopodobieństwem sukcesu

$\mathrm{w}$ schemacie czterech prób Bernoulliego.

27.8. Szukany zbiór jest przekrojem pasa pomiedzy dwiema prostymi

równoleglymi $\mathrm{i}$ zbioru punktów $\mathrm{l}\mathrm{e}\dot{\mathrm{z}}$ acych pod wykresem $\mathrm{i}$ na wykresie funkcji

$f(x) = \sqrt[3]{x}$. Zwrócič uwage na przebieg tej funkcji $\mathrm{w}$ otoczeniu punktu

$x = 0. \mathrm{W}$ dwóch punktach wykres funkcji $f(x)$ jest styczny do danych

prostych, a $\mathrm{w}$ dwóch innych przecina te proste pod tym samym katem

(dlaczego?). Do obliczenia tangensa tego kata $\mathrm{u}\dot{\mathrm{z}}$ yč pochodnej.

28.1. Nie wyznaczač predkości obu punktów, lecz od razu ich stosunek.

28.2. Aby nierównośč byla spelniona dla $\mathrm{k}\mathrm{a}\dot{\mathrm{z}}$ dego $x\in \mathrm{R}$, mianownik nie

$\mathrm{m}\mathrm{o}\dot{\mathrm{z}}\mathrm{e}$ mieč pierwiastków rzeczywistych, czyli jest dodatni na calej prostej.

Wtedy $\mathrm{m}\mathrm{o}\dot{\mathrm{z}}$ na obie strony pomnozyč przez ten mianownik, zachowujac znak

nierówności $\mathrm{i}$ badač nieujemnośč otrzymanego trójmianu kwadratowego.

Przypadek $p=1$ rozpatrzyč oddzielnie.

28.3. Zastosowač twierdzenie cosinusów. Nie wyznaczač dlugości boków,

lecz od razu ich iloczyn. Określič dziedzine dla $\alpha, r\mathrm{i}d.$

28.4. Przekrój plaszczyzna symetrii zawiera środek kuli, środek jed-

nej nózki oraz środek odcinka laczacego pozostale nózki. Wykonač rysunek

tego przekroju, przyjmujac $r$ bardzo male $\mathrm{w}$ porównaniu $\mathrm{z}R$. Korzystač

$\mathrm{z}$ twierdzenia $0$ okregach stycznych zewnetrznie.

28.5. Rozwiazanie $\mathrm{w}$ przedziale (-00, 0) wyznaczyč bezpośrednio, ko-

rzystajac ze wzoru na sześcian sumy. $\mathrm{W}(0,\infty)$ wyznaczyč przedzialy mono-





126

toniczności, ekstrema lokalne oraz wartośč wielomianu $\mathrm{w} x = 0$. Stad

$\mathrm{i}\mathrm{z}$ wlasności Darboux określič liczbe rozwiazań (nie wyznaczač ich jawnie).

28.6. Skorzystač ze wzoru $\alpha^{n}-b^{n}=(\alpha-b)(\alpha^{n-1}+\alpha^{n-2}b++b^{n-1})$

$\mathrm{i}$ rozwazyč oddzielnie $n$ parzyste $\mathrm{i}$ nieparzyste.

28.7. Napisač warunki określajace dziedzine (warunek istnienia sum

obu nieskończonych ciagów geometrycznych), nie wyznaczajac jej $\mathrm{w}$ sposób

jawny. Podstawič $\cos x = t \mathrm{i}$ wyeliminowač pierwiastki nie nalezace do

dziedziny.

28.8. Za pomoca pochodnej napisač równanie stycznej $\mathrm{w}$ punkcie

$P(x_{0},\displaystyle \frac{x_{0}^{2}}{2}) \mathrm{l}\mathrm{e}\dot{\mathrm{z}}$ acym na danej paraboli $\mathrm{i}$ bezpośrednio stad równanie prostej

prostopadlej do stycznej przechodzacej przez $P$. Wyznaczyč wspólrzedne

środka rozwazanego odcinka tej normalnej $\mathrm{i}$ po wyeliminowaniu parametru

$x_{0}$ otrzymač równanie krzywej. Zauwazyč, $\dot{\mathrm{z}}\mathrm{e}x_{0}$ nie $\mathrm{m}\mathrm{o}\dot{\mathrm{z}}\mathrm{e}$ byč równe zeru

(dlaczego?).

29.1. Oznaczyč $C(x,3x-14)$. Wyznaczyč środek $S$ odcinka AB. Ko-

$\rightarrow$

$\rightarrow$

rzystajac $\mathrm{z}$ prostopadlości wektorów AB $\mathrm{i}SC$ (iloczyn skalarny równy zeru)

wyznaczyč niewiadoma $x.$

29.2. Oznaczyč przez $x$ liczbe pieciocyfrowa powstala po skreśleniu pier-

wszej cyfry $\mathrm{i}$ ulozyč równanie liniowe $\mathrm{z}$ niewiadoma $x.$

29.3. Wyrazič promień okregu wpisanego za pomoca krótszego ramie-

nia $c$. Uzasadnič, $\dot{\mathrm{z}}\mathrm{e}$ środek $O$ okregu wpisanego $\mathrm{i}$ krótsza podstawa $CD$

wyznaczaja trójkat, $\mathrm{w}$ którym wysokośč do boku $CD$ tworzy $\mathrm{z}$ odcinkami

$OC\mathrm{i}OD$ katy $\displaystyle \alpha \mathrm{i}\frac{\alpha}{2}$. Stad wyznaczyč $|CD|.$

29.4. Najpierw rozpatrzyč przypadek oczywisty, gdy $x^{2}-x-2<0$. Po-

zostale przypadki, przez odwrócenie ulamków po obu stronach nierówności,

prowadza do nierówności kwadratowych (uwaga na znak nierówności).

29.5. Ustalič dziedzine nierówności $\mathrm{i}$ rozpatrzyč przypadki $x< 1$ oraz

$x>1$. Wykres funkcji $f(x)=1+\sqrt[3]{x-1}$ jest translacja standardowej krzy-





127

wej $y=\sqrt[3]{x}\mathrm{i}$ powinien byč sporzadzony dokladnie, szczególnie $\mathrm{w}$ otoczeniu

punktu $x=1$. Opisač, które cześci brzegu wyznaczonego zbioru naleza do

tego zbioru.

29.6. Wygodna metoda przeksztalcania obu stron jest przejście do cos-

inusów podwojonych katów $(2\sin^{2}\gamma = 1-\cos 2\gamma, \mathrm{p}\mathrm{o}\mathrm{r}$. wskazówka do

$\mathrm{z}\mathrm{a}\mathrm{d}$. 4.3$)$. Otrzymane serie rozwiazań polaczyč $\mathrm{w}$ dwie serie.

29.7. Uzasadnič, $\dot{\mathrm{z}}\mathrm{e}$ dziedzina szukanego kata jest przedzial $(\displaystyle \frac{\pi}{2},\pi).$

Poprowadzič przekrój plaszczyzna symetrii przechodzaca przez wierzcholek

ostroslupa $\mathrm{i}$ środki przeciwleglych krawedzi podstawy $\mathrm{i}$ korzystač $\mathrm{z}$ podobień-

stwa odpowiednich trójkatów. Cosinus szukanego kata wyznaczyč za po-

moca twierdzenia cosinusów.

29.8. Wyznaczyč dziedzine $D$ funkcji $S(x)$, pamietač $0x=-1$. Posluzyč

$\mathrm{s}\mathrm{i}\mathrm{e}$ pochodna funkcji, ale nie wyznaczač ekstremów lokalnych, lecz ograniczyč

$\mathrm{s}\mathrm{i}\mathrm{e}$ do podania wartości najwiekszej $\mathrm{i}$ najmniejszej funkcji $S(x)\mathrm{w}D.$

30.1. Objetośč rozwazanej bryly jest róznica objetości dwóch stozków

$0$ wspólnej podstawie. Oznaczyč dluzsza przyprostokatna przez $\alpha$, krótsza

przez $b$, a objetośč stozka powstalego $\mathrm{z}$ obrotu trójkata wokól krótszej

przyprostokatnej przez $V_{1}$. Wtedy $V_{1} \geq V_{2}$. Nie wyznaczač przypros-

tokatnych ani innych wielkości liniowych, lecz od razu objetośč $\mathrm{i}$ po wye-

liminowaniu $\alpha \mathrm{i}b$ wyrazič $\mathrm{j}\mathrm{a}$ przez $V_{1}\mathrm{i}V_{2}.$

30.2. Przyjač wysokośč najmniejszej nagrody, róznice ciagu oraz liczbe

nagród $n$ za niewiadome. Ulozyč uklad dwóch równań $\mathrm{i}$ wykazač, $\dot{\mathrm{z}}\mathrm{e}$

$4\leq n\leq 6$. Rozwiazania wyznaczamy przez bezpośrednie sprawdzenie.

30.3. Równania okregów, których środki $\mathrm{l}\mathrm{e}\dot{\mathrm{z}}$ a na prostej $y = 1$, wyz-

naczyč bezpośrednio $\mathrm{z}$ twierdzenia $0$ okregach stycznych zewnetrznie lub

wewnetrznie. Środki pozostalych okregów otrzymujemy po rozwiazaniu

odpowiedniego ukladu równań.

30.4. Korzystamy $\mathrm{z}$ twierdzenia cosinusów. Nie wyznaczamy boków

równolegloboku, lecz tylko ich iloczyn $\mathrm{i}$ przez porównanie dwóch wyrazeń

na pole równolegloboku otrzymujemy od razu tangens szukanego kata.





15

Praca kontrolna

nr 7

7.1. Rozwiazač nierównośč

$|9^{x}-2|<3^{x+1}-2.$

7.2. Wyznaczyč równanie krzywej bedacej obrazem okregu

$(x+1)^{2}+(y-6)^{2}=4\mathrm{w}$ powinowactwie prostokatnym $0$ osi $Ox\mathrm{i}$ sto-

sunku $k=\displaystyle \frac{1}{2}$. Obliczyč pole figury ograniczonej ta krzywa. Sporzadzič

staranny rysunek.

7.3. Pewien zbiór zawiera dokladnie 67 podzbiorów $0$ co najwyzej dwóch

elementach. Ile podzbiorów siedmioelementowych zawiera ten zbiór?

7.4. Trapez $0$ katach przy podstawie wynoszacych $15^{\circ}\mathrm{i}45^{\circ}$ opisano na kole

$0$ promieniu $R$. Obliczyč stosunek pola kola do pola tego trapezu.

7.5. Rozwiazač uklad równań

$\left\{\begin{array}{l}
mx-6y=3\\
2x+(m-7)y=m-1
\end{array}\right.$

$\mathrm{w}$ zalezności od parametru rzeczywistego $m$. Podač wszystkie rozwia-

zania ($\mathrm{i}$ odpowiadajace im wartości parametru $m$), dla których $x$ jest

równe $y.$

7.6. Rozwiazač nierównośč $\sin 2x<\sin x\mathrm{w}$ przedziale $[-\displaystyle \frac{\pi}{2},\frac{\pi}{2}]$.

zanie zilustrowač starannym wykresem.

Rozwia-

7.7. Ostroslup podzielono na trzy cześci dwiema plaszczyznami równolegly-

mi do jego podstawy. Pierwsza plaszczyznajest polozona $\mathrm{w}$ odleglości

$d_{1} = 2$ cm, a druga $\mathrm{w}$ odleglości $d_{2} = 3$ cm od podstawy. Pola

przekrojów ostroslupa tymi plaszczyznami równe sa odpowiednio

$S_{1} = 25 \mathrm{c}\mathrm{m}^{2}$ oraz $S_{2} = 16 \mathrm{c}\mathrm{m}^{2}$ Obliczyč objetośč tego ostroslupa

oraz objetośč najmniejszej cześci.

7.8. Trylogie skladajaca $\mathrm{s}\mathrm{i}\mathrm{e} \mathrm{z}$ dwóch powieści dwutomowych oraz jednej

jednotomowej ustawiono na pólce $\mathrm{w}$ przypadkowej kolejności. Jakie

jest prawdopodobieństwo tego, $\dot{\mathrm{z}}\mathrm{e}$ tomy a) obydwu, b) co najmniej

jednej $\mathrm{z}$ dwutomowych powieści znajduja $\mathrm{s}\mathrm{i}\mathrm{e}$ obok siebie $\mathrm{i}$ przy tym

tom I $\mathrm{z}$ lewej, a tom II $\mathrm{z}$ prawej strony.





128

30.5. Określič dziedzine równania. Poniewaz $\mathrm{w}$ dziedzinie obie strony

równania sa dodatnie $\mathrm{m}\mathrm{o}\dot{\mathrm{z}}$ na podnieśč je do kwadratu.

30.6. Wyznaczyč wszystkie (trzy) pierwiastki danego wielomianu. Je-

den $\mathrm{z}$ nich nie $\mathrm{m}\mathrm{o}\dot{\mathrm{z}}\mathrm{e}$ byč pierwiastkiem trójmianu $2x^{2}+\alpha x+b$ (dlaczego?).

Znajac dwa pierwiastki, napisač ten trójmian $\mathrm{i}$ odczytač niewiadome $\alpha \mathrm{i}b.$

30.7. Podstawič $2^{x}=t$. Korzystač $\mathrm{z}\mathrm{t}\mathrm{o}\dot{\mathrm{z}}$ samości podanej we wskazówce

do $\mathrm{z}\mathrm{a}\mathrm{d}$. 5.1. Wykresy obu stron otrzymač przez translacje $\mathrm{i}$ odbicia syme-

tryczne standardowej krzywej $y=2^{x}$

30.8. Wyznaczyč miejsca zerowe pochodnej $\mathrm{i}$ za pomoca wykresu rozwia-

zač odpowiednia nierównośč trygonometryczna.

31.1. Określič model probabilistyczny. Rozwazane zdarzenie przed-

stawič $\mathrm{w}$ postaci sumy rozlacznych zdarzeń $B_{1}\cup B_{2}\cup B_{3}\cup B_{4}$, gdzie $B_{i}$ jest

zdarzeniem polegajacym na otrzymaniu przez gracza 4 kart $\mathrm{w}i$-tym kolorze

$\mathrm{z}$ asem, królem $\mathrm{i}$ dama. $P(B_{i})$ obliczyč bezpośrednio, pamietajac, $\dot{\mathrm{z}}\mathrm{e}P$ jest

prawdopodobieństwem klasycznym $\mathrm{i}$ skorzystač $\mathrm{z}$ wlasności prawdopodo-

bieństwa.

31.2. Patrz wskazówka do zad. 7.7.

31.3. Określič dziedzine ukladu. Zwrócič uwage najej asymetrie wzgle-

dem zmiennych $x\mathrm{i}y$. Dodajac $\mathrm{i}$ odejmujac równania stronami przejśč do

ukladów równań liniowych.

31.4. Najpierw określič ($\mathrm{i}$ uzasadnič geometrycznie) dziedzine dla kata

$\alpha$ oraz wyznaczyč katy trójkata $ABC.$

31.5. $\mathrm{W}$ dowodzie kroku indukcyjnego przeksztalcač lewa strong $\mathrm{i}$ do-

prowadzič $\mathrm{j}\mathrm{a}$ do równości $\mathrm{z}$ prawa strong. Korzystač ze wzoru na róznice

sinusów. Nie prowadzič dowodu metoda redukcji.

31.6. Pomnozyč licznik $\mathrm{i}$ mianownik przez $\sqrt{n} + \sqrt{n+\sqrt[3]{4n^{2}}},$

a nastepnie podzielič je przez $n^{2/3}$, zamieniajac wcześniej pierwiastki na

potegi $0$ wykladnikach ulamkowych.





129

31.7. Korzystač $\mathrm{z}$ nastepujacej wlasności wektorów na plaszczy $\acute{\mathrm{z}}\mathrm{n}\mathrm{i}\mathrm{e}$

(uzasadnič $\mathrm{j}\mathrm{a}$):

{\it Jeśli} $\text{{\it ũ}}\perp\vec{v},\ ||${\it ũ}$|| =||\vec{v}||$ {\it oraz ũ}$= (\alpha,b)$, {\it to} $\vec{v}=(b,-\alpha) lub\vec{v}=(-b,\alpha).$

Sugeruje ona, $\dot{\mathrm{z}}\mathrm{e}$ zadanie ma dwa rozwiazania.

31.8. Zapisač funkcje $\mathrm{w}$ postaci $f(x) = x^{1/2} +x^{-1/2} \mathrm{i}$ obliczyč

pochodna ze wzoru na pochodna funkcji potegowej. Zauwazyč, $\dot{\mathrm{z}}\mathrm{e}$

$\displaystyle \lim_{x\rightarrow+\infty}(f(x)-\sqrt{x})=0$. Jaka wlasnośč geometryczna wykresu funkcji $f(x)$

opisuje ta równośč?

32.1. Oznaczyč przez $x$ predkośč statku, przez $y$ predkośč wody, a przez

$d$ odleglośč $\mathrm{z}$ Wroclawia do Szczecina. Zapisač odpowiednie równania $\mathrm{i}$ nie

wyznaczajac niewiadomych, odpowiedzieč tylko na postawione pytanie.

32.2. Sprowadzič wszystkie logarytmy do tej samej podstawy 2 lub 8

$\mathrm{i}$ skorzystač $\mathrm{z}$ definicji ciagu geometrycznego.

32.3. Narysowač przekrój pionowy wanny $\mathrm{z}\mathrm{l}\mathrm{e}\dot{\mathrm{z}}\mathrm{a}\mathrm{c}\mathrm{a}$ na dnie belka. Ponie-

$\mathrm{w}\mathrm{a}\dot{\mathrm{z}}$ średnica belki jest polowa promienia wanny, wjej przekroju pionowym

pojawiaja $\mathrm{s}\mathrm{i}\mathrm{e}$ trójkaty równoboczne.

32.4. Zarówno $v(x)$, jak $\mathrm{i} w(x)$ musza mieč dwa rózne pierwiastki

rzeczywiste. To daje dziedzine dla parametru $m$. Obliczyč pierwiastki $x_{1},$

$x_{2}$ wielomianu $w(x)$. Jeśli wierzcholek paraboli $0$ równaniu $y=v(x) \mathrm{l}\mathrm{e}\dot{\mathrm{z}}\mathrm{y}$

pomiedzy $x_{1}\mathrm{i}x_{2}$ oraz $v(x_{1})\mathrm{i}v(x_{2})$ sa dodatnie, to wymagany warunek jest

spelniony.

32.5. Rozwazyč nastepujace zdarzenia: $C -$ wylosowano co najmniej

dwie kule biale, $D \mathrm{z}$ urny $\mathrm{B}$ wylosowano kule biala, $E_{i} - \mathrm{z}$ urny $\mathrm{A}$

wylosowano $i$ kul bialych, $i=0$, 1, 2, 3, 4. Wówczas $C'=E_{0}\cup D'\cap E_{1}.$

Skorzystač $\mathrm{z}$ niezalezności zdarzeń $D, E_{i}$, rozlaczności zdarzeń $E_{0}, D'\cap E_{1}$

oraz ze schematu Bernoulliego.

32.6. Wyznaczyč dziedzine równania. Pomnozyč obie strony przez $\cos x$

$\mathrm{i}$ po zastosowaniu wzorów $\sin 2x = 2\sin x\cos x$ oraz $\cos 2x = 1-2\sin^{2}x$

rozlozyč wyrazenie na czynniki, wylaczajac przed nawias czynnik

$(\sin x-\cos x).$





130

32.7. Napisač równanie stycznej $\mathrm{w}$ punkcie $S(x_{0},x_{0}^{4}-2x_{0}^{2})$, gdzie

$x_{0} \in \mathrm{R}$, nastepnie wyznaczyč wszystkie $x_{0}$, dla których $P \mathrm{l}\mathrm{e}\dot{\mathrm{z}}\mathrm{y}$ na sty-

cznej (trzy punkty). Dwa $\mathrm{z}$ nich wyznaczaja $\mathrm{t}\mathrm{e}$ sama styczna, a trzeci inna.

Sporzadzič wykres funkcji $f(x)$, korzystajac zjej parzystości oraz informacji

zebranych przy wyznaczaniu stycznych bez dalszego badania jej przebiegu.

32.8. $\mathrm{Z}$ twierdzenia $0$ trzech prostopadlych wywnioskowač, $\dot{\mathrm{z}}\mathrm{e}$ plasz-

czyzna $SCD$ jest plaszczyzna symetrii ostroslupa, a wiec zawiera środek kuli

opisanej. $\mathrm{L}\mathrm{e}\dot{\mathrm{z}}\mathrm{y}$ on na prostej prostopadlej do podstawy ostroslupa wysta-

wionej $\mathrm{w}$ środku okregu opisanego na podstawie. Wykazač, $\dot{\mathrm{z}}\mathrm{e}\triangle SCD$ jest

równoboczny $\mathrm{i}$ stad określič polozenie środka kuli.

33.1. Zastosowač wzór Newtona. Liczba $x$ jest wieksza od $y$

$\mathrm{g}\mathrm{d}\mathrm{y}x= (1+\displaystyle \frac{p}{100})y.$

0

{\it p}\%,

33.2. Zastosowač wzór na odleglośč punktu od prostej. Nalezy za-

uwazyč, $\dot{\mathrm{z}}\mathrm{e}$ punkt przeciecia $\mathrm{s}\mathrm{i}\mathrm{e}$ prostych $k\mathrm{i}l$ nie spelnia $\dot{\mathrm{z}}$ adanego warunku.

33.3. Skorzystač $\mathrm{z}$ twierdzenia $0$ dwusiecznej kata $\mathrm{w}$ trójkacie oraz ze

wzoru Herona.

33.4. Iloraz $q$ ciagu $(\alpha_{n})$ jest mniejszy od l, wiec droga przebyta przez

czastke jest skończona $\mathrm{i}$ ruch czastki kończy $si_{G} \mathrm{w}$ punkcie $P$. Znajac

wspólrzedne tego punktu, ulozyč dwa równania $\mathrm{z}$ niewiadomymi $\alpha_{1}\mathrm{i}q.$

33.5. Nie $\mathrm{u}\dot{\mathrm{z}}$ ywač algorytmu dzielenia wielomianów, lecz umiejetnie

stosowač rozklad na czynniki np. $x^{4} +x^{2} + 1 = (x^{2}+1)^{2} - x^{2} =$

$=(x^{2}+x+1)(x^{2}-x+1)$. Podobnie postepowač $\mathrm{w}$ dowodzie kroku induk-

cyjnego.

33.6. Oddzielnie rozwazyč przedzialy $(0,\infty)$ oraz (-00, 0). Wykresy

$\mathrm{w}$ tych przedzialach sa istotnie róznymi krzywymi. Nazwač je. Dokladnie

stosowač definicje asymptoty ukośnej prawostronnej $\mathrm{i}$ lewostronnej.

33.7. Przypadek $|\cos x| =$ l jest oczywisty. Dla przypadku

$0< |\cos x| <1$ przejśč do porównania wykladników obu stron. Rozwiazač

odpowiednie równanie trygonometryczne $\mathrm{i}$ za pomoca wykresu wyznaczyč





131

zbiór rozwiazań nierówności.

metrycznym.

Wygodnie jest posluzyč sie kolem trygono-

33.8. Wykonač przekrój osiowy stozka przechodzacy przez jedna

$\mathrm{z}$ krawedzi graniastoslupa. Wyrazič stosunek objetości bryl jako funkcje

zmiennej $x =$ tg $\alpha \in (0,\infty)$. Nie mylič postawionego pytania $\mathrm{z}$ zagad-

nieniem wyznaczania ekstremów lokalnych.

34.1. Napisač uklad równań $\mathrm{z}$ niewiadomymi przyprostokatnymi $\alpha \mathrm{i}b.$

Nie wyznaczač ich oddzielnie, lecz tylko sume $\alpha+b$ potrzebna do obliczenia

obwodu.

34.2. Skorzystač ze wzoru na sume sześcianów oraz ze wzorów na $\sin 2\gamma$

$\mathrm{i}\cos 2\gamma.$

34.3. Warunkiem stycznościjest istnienie pierwiastka podwójnego odpo-

wiedniego trójmianu kwadratowego. Zadanie ma wiecej $\mathrm{n}\mathrm{i}\dot{\mathrm{z}}$ jedno rozwia-

zanie.

34.4. Wektory (swobodne) $\vec{u}\mathrm{i}\vec{v}$ sa równolegle, gdy $\vec{v}=c\vec{u}\mathrm{d}\mathrm{l}\mathrm{a}$ pewnego

skalara $c$. Prostopadlośč wektorów wyrazič za pomoca iloczynu skalarnego.

34.5. Oznaczyč przez $B_{i}$ zdarzenie polegajace na $\mathrm{t}\mathrm{y}\mathrm{m}, \dot{\mathrm{z}}\mathrm{e}$ za pierwszym

razem wylosowano monete $i \mathrm{z}l, i = 1$, 2, 5. Wtedy $B_{1}\cup B_{2}\cup B_{5} = \Omega$

$\mathrm{i}$ skladniki sa rozlaczne. Prawdopodobieństwo zdarzenia, $\dot{\mathrm{z}}\mathrm{e}$ Jaś wyciagnie

dokladnie dwie monety obliczyč ze wzoru na prawdopodobieństwo calkowite,

a prawdopodobieństwo, $\dot{\mathrm{z}}\mathrm{e}$ Jaś wyciagnie tylko jedna monete (czyli 5 $\mathrm{z}l$)

wynosi $\displaystyle \frac{1}{6}$. Stad otrzymač odpowied $\acute{\mathrm{z}}.$

34.6. Zastosowač wzór $\sqrt{\alpha^{2}}=|\alpha|$. Uzasadnič, $\dot{\mathrm{z}}\mathrm{e}$ krzywa $K\mathrm{o}$ równaniu

$y=\sqrt{4x-x^{2}}$ jest górna polowa okregu $0$ środku $S(2,0)\mathrm{i}$ promieniu 2. Przy

obliczaniu odleglości $P$ od brzegu $\mathcal{F}$ ograniczyč $\mathrm{s}\mathrm{i}\mathrm{e}$ do porównania odleglości

$P$ od krzywej $K$ oraz od odcinka prostej $y=1-x,  x\in (1,4)$. Pozostale

cześci brzegu $\mathcal{F}$ sa znacznie dalej polozone, co wystarczy uzasadnič przez

powolanie $\mathrm{s}\mathrm{i}\mathrm{e}$ na (staranny) rysunek.





132

34.7. Wzór określajacy $f(x)$ sprowadzič do najprostszej postaci $\mathrm{i}$ za-

uwazyč, $\dot{\mathrm{z}}\mathrm{e}$ jest ona zlozeniem dwóch funkcji rosnacych ($\mathrm{w}$ dziedzinie!).

Dziedzina $f^{-1}$ jest zbiór wartości $f\mathrm{i}$ na odwrót.

34.8. Do obliczenia krawedzi podstawy $\alpha$ wykorzystač wskazówke do

zadania 3.4. Poprowadzič przekrój ostros1upa p1aszczyzna symetrii prze-

chodzaca przez wierzcholek ostroslupa $\mathrm{i}$ odpowiednia przekatna podstawy

$\mathrm{i}$ korzystač wielokrotnie $\mathrm{z}$ podobieństwa trójkatów. Objetośč wyrazič naj-

pierw przez $\alpha \mathrm{i}$ dopiero na końcu podstawič $c$. Zadanie ma sens, gdy krawed $\acute{\mathrm{z}}$

bocznajest nachylona do podstawy pod katem co najmniej $45^{\circ}$ (dlaczego?).

Stad warunek na $\alpha.$

35.1. Wykluczyč $p = 0 \mathrm{i} \mathrm{z}$ warunku istnienia sumy nieskończonego

ciagu geometrycznego wyznaczyč $\alpha_{1}\mathrm{i}q.$

35.2. $K\mathrm{a}\mathrm{t}$ miedzy prostymi jest równy katowi miedzy ich wektorami

normalnymi (odpowiednio zorientowanymi). Napisač równania danych

prostych $\mathrm{w}$ postaci ogólnej $\mathrm{i}\mathrm{u}\dot{\mathrm{z}}$ yč iloczynu skalarnego.

35.3. Rozwazyč przekrój sześcianu plaszczyzna symetrii (zawierajacy

środek $\mathrm{i}$ kolo wielkie danej kuli oraz przekroje czterech narozników). Szu-

kana krawed $\acute{\mathrm{z}}$ obliczyč za pomoca twierdzenia Pitagorasa dla odpowiedniego

trójkata $\mathrm{w}$ tym przekroju.

35.4. Uzasadnič, $\dot{\mathrm{z}}\mathrm{e}\mathrm{w}$ przedziale [-l, l] obie strony nierówności sa nieu-

jemne $\mathrm{i}$ podnieśč je do kwadratu. Wykresy nalezy wykonač dokladnie (leza

blisko siebie), zwracajac uwage na otoczenia punktów $x=0\mathrm{i}x=-1.$

35.5. Wyznaczyč dziedzine równania. Pomnozyč obie strony przez

$\sin 2x$. Zastosowač wzór na iloczyn sinusów $\mathrm{i}\mathrm{z}$ równości dwóch cosinusów

przejśč od razu do porównywania katów.

35.6. Napisač wzór na styczna do okregu $\mathrm{w}$ punkcie $\mathrm{l}\mathrm{e}\dot{\mathrm{z}}$ acym na nim

(por. wskazówka do zadania 6.2) $\mathrm{i}$ po podstawieniu wspólrzednych punktu $P$

wyznaczyč punkt styczności, dla którego styczna ma dodatni wspólczynnik

kierunkowy.





133

35.7. Dane $r\mathrm{i}d$ jednoznacznie określaja $\mathrm{k}\mathrm{a}\mathrm{t}\alpha$ przy podstawie trapezu,

przy czym $\displaystyle \alpha>\frac{\pi}{3}$. Obwód wyrazič jako funkcje wysokości trapezu. Ustalič

dziedzine. Wartośč najwieksza funkcji wyznaczyč, badajac jej przedzialy

monotoniczności.

35.8. Wyznaczyč $y \mathrm{z}$ pierwszego równania $\mathrm{i}$ podstawič do drugiego.

Nastepnie skorzystač $\mathrm{z}\mathrm{t}\mathrm{o}\dot{\mathrm{z}}$ samości $(|\alpha|=b)\Leftrightarrow$($\alpha=b$ lub $\alpha=-b$) prawdzi-

wej dla $b\geq 0$. Otrzymane alternatywy prowadza do czterech przypadków

$m = -\displaystyle \frac{1}{2}, m = \displaystyle \frac{1}{2}, m = 1$ oraz pozostale $m$. Na rysunku zaznaczyč

odpowiednio wybrane proste $\mathrm{z}$ peku prostych (któremu odpowiada pierwsze

równanie ukladu) przechodzacych przez $P(0$, 2$).$





12  przykIadowych

rozwiązań





137

Rozwiazanie zadania 2.1

I sposób. Rozwazmy wielomian $p_{n}(y) =y^{2n-1}+1$. Poniewaz $2n-1$

jest liczba nieparzysta dla $\mathrm{k}\mathrm{a}\dot{\mathrm{z}}$ dego $n\in N$, wiec $p(-1)=(-1)^{2n-1}+1=0.$

$\mathrm{Z}$ twierdzenia Bézouta wynika wiec, $\dot{\mathrm{z}}\mathrm{e}p_{n}(y)$ jest podzielny przez dwumian

$y+1, \mathrm{t}\mathrm{z}\mathrm{n}$. istnieje wielomian $q_{n}(y)$ stopnia $2n-2$ taki, $\dot{\mathrm{z}}\mathrm{e}$

$p_{n}(y)=y^{2n-1}+1=(y+1)q_{n}(y).$

(1)

Zauwazmy, $\dot{\mathrm{z}}\mathrm{e} w_{n}(x) = x^{4n-2}+1 =p_{n}(x^{2})$. Stad $\mathrm{i} \mathrm{z}$ (l) wynika, $\dot{\mathrm{z}}\mathrm{e}$

$w_{n}(x)=(x^{2}+1)q_{n}(x^{2})$. Poniewaz $q_{n}(x^{2})$ jest wielomianem stopnia $4n-4,$

wiec równośč ta dowodzi prawdziwości tezy.

Uwaga. Stosujac wzór skróconego mnozenia

$\alpha^{2n-1}+b^{2n-1}=(\alpha+b)(\alpha^{2n-2}-\alpha^{2n-3}b+-\alpha b^{2n-3}+b^{2n-2})$

$\mathrm{m}\mathrm{o}\dot{\mathrm{z}}$ na wielomian $q_{n}(y)$ napisač $\mathrm{w}$ postacijawnej. Niejest tojednak koniecz-

ne dla poprawności dowodu.

II sposób. Dowód indukcyjny. Rozwazmy funkcje zdaniowa zmien-

nej naturalnej $n$

$T(n)$ : $w_{n}(x)=x^{4n-2}+1$ jest podzielny przez $x^{2}+1.$

Sprawdzimy teraz, $\dot{\mathrm{z}}\mathrm{e}$ dla $T(n)$ obydwa zalozenia zasady indukcji mate-

matycznej sa spelnione.

$1^{\mathrm{O}}$ Sprawdzenie prawdziwości zdania $T(1).$

Mamy $w_{1}(x)=x^{4\cdot 1-2}+1=x^{2}+1$, czyli oczywiście dzieli $\mathrm{s}\mathrm{i}\mathrm{e}$ przez $x^{2}+1,$

a wiec $T(1)$ jest prawdziwe.

$2^{\circ}$ Wykazemy, $\dot{\mathrm{z}}\mathrm{e}$ implikacja $(T(n)\Rightarrow T(n+1))$ jest prawdziwa dla

$\mathrm{k}\mathrm{a}\dot{\mathrm{z}}$ dego $n\in N.$

Dowód. Niech $n$ bedzie dowolna ustalona liczba naturalna. Zalózmy,

$\dot{\mathrm{z}}\mathrm{e}$ zdanie $T(n)$ jest prawdziwe $\mathrm{t}\mathrm{z}\mathrm{n}$. istnieje wielomian $v_{n}(x)$ taki, $\dot{\mathrm{z}}\mathrm{e}$

$w_{n}(x) =x^{4n-2}+1 = (x^{2}+1)v_{n}(x)$. Wówczas korzystajac $\mathrm{z}$ tej równości

mamy

$w_{n+1}(x)=x^{4(n+1)-2}+1=x^{4n+2}+1=(x^{4n+2}+x^{4})-(x^{4}-1)$





138

$=x^{4}(x^{4n-2}+1)-(x^{2}+1)(x^{2}-1)=x^{4}(x^{2}+1)v_{n}(x)-(x^{2}+1)(x^{2}-1)$

$=(x^{2}+1)(x^{4}v_{n}(x)-x^{2}+1).$

Poniewaz $x^{4}v_{n}(x)-x^{2}+1$ jest wielomianem, wiec powyzsza równośč

oznacza, $\dot{\mathrm{z}}\mathrm{e}w_{n+1}(x)$ dzieli $\mathrm{s}\mathrm{i}\mathrm{e}$ przez $x^{2}+1$. To kończy dowód $2^{\circ}$

$\mathrm{Z}$ wykazanej prawdziwości warunków $1^{\circ} \mathrm{i} 2^{\circ}$ oraz $\mathrm{z}$ zasady indukcji

matematycznej wynika, $\dot{\mathrm{z}}\mathrm{e} T(n)$ jest prawdziwe dla $\mathrm{k}\mathrm{a}\dot{\mathrm{z}}$ dej liczby

naturalnej $n.$

Rozwiazanie zadania 3.8

Dziedzina nierówności jest R.

cosinus róznicy katów mamy

Poniewaz $\sqrt{3}=$ tg $\displaystyle \frac{\pi}{3}$, wiec ze wzoru na

$\cos x+\sqrt{3}\sin x=\cos x +$ tg $\displaystyle \frac{\pi}{3}\sin x=$

$\displaystyle \frac{\cos x\cos\frac{\pi}{3}+\sin x\sin\frac{\pi}{3}}{\cos\frac{\pi}{3}}=2\cos(x-\frac{\pi}{3})$

Nierównośč przyjmuje zatem postač $|2\displaystyle \cos(x-\frac{\pi}{3})| \leq \sqrt{2}$. Obie strony

nierówności sa nieujemne, wiec po podniesieniu do kwadratu dostajemy

nierównośč równowazna 2$\displaystyle \cos^{2}(x-\frac{\pi}{3}) \leq 1$. Stosujemy wzór l$+ \cos 2\gamma=$

$2\cos^{2}\gamma \mathrm{i}$ przeksztalcamy $\mathrm{j}\mathrm{a}$ do prostszej postaci $\displaystyle \cos(2x-\frac{2\pi}{3}) \leq 0$. Wiemy,

$\dot{\mathrm{z}}\mathrm{e}$ cosinus jest ujemny $\mathrm{w}$ II $\mathrm{i}$ III čwiartce, otrzymujemy wiec

$\displaystyle \frac{\pi}{2}+2k\pi\leq 2x-\frac{2\pi}{3}\leq\frac{3\pi}{2}+2k\pi$, czyli

$\displaystyle \frac{7\pi}{12}+k\pi\leq x\leq\frac{13\pi}{12}+k\pi, k\in \mathrm{Z}.$

(2)

Wyznaczamy cześč wspólna zbioru rozwiazań (2) $\mathrm{i}$ przedzialu $[0,3\pi]$, dosta-

jemy (podstawiamy kolejno $k=-1, 0$, 1, 2) odpowied $\acute{\mathrm{z}}.$

Odp. $ x\in [0,\displaystyle \frac{\pi}{12}]\cup [\displaystyle \frac{7\pi}{12},\frac{13\pi}{12}]\cup [\displaystyle \frac{19\pi}{12},\frac{25\pi}{12}]\cup [\displaystyle \frac{31\pi}{12},3\pi].$





139

Rozwiazanie zadania 12.6

Oznaczmy przez $O$ środek okregu opisanego na trapezie, a przez $h$

wysokośč trapezu (rys. 25). Wówczas $P= \displaystyle \frac{1}{2}sh$, czyli $h= \displaystyle \frac{2P}{s}. \mathrm{Z}$ twier-

dzenia Pitagorasa $\mathrm{w} \triangle DEB$ otrzymujemy $d^{2}=h^{2}+\displaystyle \frac{s^{2}}{4}=\frac{16P^{2}+s^{4}}{4s^{2}}.$
\begin{center}
\includegraphics[width=78.492mm,height=74.472mm]{./KursMatematyki_PolitechnikaWroclawska_1999_2004_page119_images/image001.eps}
\end{center}
' $l D C$

{\it M}

$c r$

$c h$

$kA E s 2 B$

$\mathrm{R}\mathrm{y}\mathrm{s}$. 25

Z drugiej strony $\mathrm{z}$ twierdzenia

$0$ kacie wpisanym $\mathrm{w}$ okrag wynika

$\dot{\mathrm{z}}\mathrm{e} \angle DAE = \displaystyle \frac{1}{2}\angle DOB = \angle DOM,$

zatem trójkaty prostokatne $\triangle DAE$

$\mathrm{i} \triangle DOM$ maja identyczne katy,

czyli sa podobne. To pozwala napi-

sač proporcje $\displaystyle \frac{h}{c}= \displaystyle \frac{d}{2r}$, skad otrzy-

$2rh$

mujemy $c =$ Po podstawie-

$d$

niu obliczonej wartości $d$ mamy

$8Pr$

$c= \sqrt{16P^{2}+s^{4}}$. Ostatecznie ob-

wód wynosi $0=s+2c=s+\displaystyle \frac{16Pr}{\sqrt{16P^{2}+s^{4}}}$. Dane $P\mathrm{i}s$ wyznaczaja jedno-

znacznie $h\mathrm{i}d$. Zadanie ma zatem rozwiazanie, gdy promień $r$ jest wystar-

czajaco $\mathrm{d}\mathrm{u}\dot{\mathrm{z}}\mathrm{y}$, aby powstal trójkat $\triangle DOM, \mathrm{t}\mathrm{z}\mathrm{n}. r\displaystyle \geq\frac{1}{2}d=\frac{\sqrt{16P^{2}+s^{4}}}{4s}.$

Poprawnośč tego warunku, jak ijednoznacznośč rozwiazania, najlepiej widač

$\mathrm{z}$ opisu konstrukcji trapezu, który dla kompletności przedstawiamy ponizej.

Opis konstrukcji trapezu

1. $\mathrm{Z}$ odcinków $h \mathrm{i} \displaystyle \frac{s}{2}$, jako przyprostokatnych, konstruujemy trójkat

prostokatny $DEB$. Odcinek $BE$ przedluzamy $\mathrm{i}$ otrzymujemy prosta $k,$

a przez punkt $D$ prowadzimy prosta $l$ równolegla do $k.$

2. $\mathrm{Z}$ punktów $B\mathrm{i}D$ kreślimy luki okregów $0$ promieniu $r$, które przeci-

najac $\mathrm{s}\mathrm{i}\mathrm{e}$ daja środek okregu opisanego $O (\mathrm{z}$ dwóch punktów, $\mathrm{w}$ których

przecinaja $\mathrm{s}\mathrm{i}\mathrm{e}$ te luki, wybieramy $\mathrm{l}\mathrm{e}\dot{\mathrm{z}}\mathrm{a}\mathrm{c}\mathrm{y}$ blizej prostej $k$, która ma zawierač

dluzsza podstawe trapezu).





Edycja

XXX

2000/2001





140
\begin{center}
\includegraphics[width=73.512mm,height=64.968mm]{./KursMatematyki_PolitechnikaWroclawska_1999_2004_page120_images/image001.eps}
\end{center}
3. Ze środka $O$ kreślimy okrag

$0$ promieniu $r$. Okrag ten prze-

cinajac prosta $k$, wyznacza wierz-

cho ek $A$ (oraz przechodzi przez

$B)$. Podobnie, okrag ten przeci-

najac prosta $l$, wyznacza wierz-

cho ek $C (\mathrm{i}$ równocześnie prze-

chodzi przez $D$). Na rysunku 26

przedstawiono konstrukcje

trapezu dla danych liczbo-

wych zadania, $\mathrm{t}\mathrm{j}. P = 12 \mathrm{c}\mathrm{m}^{2},$

$h=3$ cmi $s=8$ cm.

$\mathrm{R}\mathrm{y}\mathrm{s}$. 26

$16Pr$

Odp. Obwód wynosi $s+$ a zadanie ma rozwiazanie, gdy

$\sqrt{16P^{2}+s^{4}}$'

{\it r}2$\geq$ --{\it Ps}22$+$--1{\it s}62.

Rozwiazanie zadania 21.7

Logarytm jest określony dla liczb dodatnich, wiec dziedzine równania

wyznaczaja warunki:

$D$: 

czyli $D$ : $(x\in(-4,1))\wedge(x^{3}-x^{2}-3x+5>0). \mathrm{W}$ celu rozwiazania równania

wszystkie skladniki zapiszemy jako logarytmy $0$ podstawie 4. D1a $x \in D$

jest $\displaystyle \log_{2}(1-x)=\frac{\log_{4}(1-x)}{\log_{4}2}=2\log_{4}(1-x)=\log_{4}(x-1)^{2}$ oraz $\displaystyle \frac{1}{2}=\log_{4}2$

$\mathrm{i}$ równanie przyjmuje postač

$\log_{4}(x-1)^{2}+\log_{4}(x+4)=\log_{4}(x^{3}-x^{2}-3x+5)+\log_{4}2.$

(3)

Korzystajac $\mathrm{z}$ wlasności logarytmu oraz $\mathrm{z}$ róznowartościowości funkcji

logarytmicznej, widzimy, $\dot{\mathrm{z}}\mathrm{e}$ równanie (3) jest równowazne ($\mathrm{w}$ dziedzinie)

równaniu algebraicznemu $(x-1)^{2}(x+4)=2(x^{3}-x^{2}-3x+5)$. Po wykonaniu

dzialań $\mathrm{i}$ przeniesieniu wszystkich skladników na jedna strong dostajemy

$x^{3}-4x^{2}+x+6=0.$

(4)





141

Pierwiastki calkowite równania (4) sa podzie1nikami wyrazu wo1nego, $\mathrm{t}\mathrm{j}.$

liczby 6. Przez podstawienie sprawdzamy bezpośrednio, $\dot{\mathrm{z}}\mathrm{e}$ liczby $-1, 2\mathrm{i}3$

spelniaja (4), czy1i sa wszystkimi pierwiastkami tego równania (majac dwa

pierwiastki, np. $-1\mathrm{i}2$, trzeci $\mathrm{m}\mathrm{o}\dot{\mathrm{z}}$ na znalez$\acute{}$č $\mathrm{z}$ relacji $x_{1}x_{2}x_{3}=-6$). Liczby

2 $\mathrm{i}3$ znajduja $\mathrm{s}\mathrm{i}\mathrm{e}$ poza przedzialem $(-4,1)$, czyli $\mathrm{l}\mathrm{e}\dot{\mathrm{z}}$ a poza $D$. Natomiast

$-1\in(-4,1)$ oraz $(-1)^{3}-(-1)^{2}-3(-1)+5=6>0$, czyli liczba $-1$ jest

jedynym pierwiastkiem danego równania.

Odp. Równanie ma tylko jeden pierwiastek $\mathrm{i}$ jest nim liczba $-1.$

Rozwiazanie zadania 22.7

Dziedzine równania określaja warunki

$D$: 

czyli warunki $ x\neq k\pi$ oraz $ x\displaystyle \neq\frac{\pi}{2}+k\pi$. To daje ostatecznie

$D:x\displaystyle \neq k\frac{\pi}{2},$

$ k\in$ Z.

Dla $x\in D$ mnozymy obie strony równania przez $(\sin x\cos x)\mathrm{i}$ otrzymujemy

równanie równowazne

$\sin x+\cos x=\sqrt{8}\sin x\cos x.$

(5)

Korzystajac ze wzoru redukcyjnego oraz wzoru na róznice cosinusów,

mamy $\sin x+\cos x = \displaystyle \cos x-\cos(x+\frac{\pi}{2}) = -2\displaystyle \sin(-\frac{\pi}{4})\sin(x+\frac{\pi}{4}).$

Ponadto $\sqrt{8}\sin x\cos x = \sqrt{2}\sin 2x$, zatem równanie (5), po podzie1eniu

obu stron przez $\sqrt{2}, \mathrm{m}\mathrm{o}\dot{\mathrm{z}}$ na zapisač $\mathrm{w}$ postaci

$\displaystyle \sin(x+\frac{\pi}{4})=\sin 2x.$





142

Rys. 27

Stad otrzymujemy alternatywe równań li-

niowych $x+\displaystyle \frac{\pi}{4} =  2x+2k\pi$ lub $x+\displaystyle \frac{\pi}{4} =$

$\pi-2x+2k\pi$, gdzie $k \in$ Z. Po standardo-

wych przeksztalceniach mamy $x= \displaystyle \frac{\pi}{4}+2k\pi$

lub $x = \displaystyle \frac{\pi}{4}+k\frac{2\pi}{3}$. Zauwazmy, $\dot{\mathrm{z}}\mathrm{e}$ pierwsza

seria zawiera $\mathrm{s}\mathrm{i}\mathrm{e} \mathrm{w}$ drugiej (rys. 27), a ta

$\mathrm{z}$ kolei jest zawarta $\mathrm{w}$ dziedzinie równania.

Odp. $x=\displaystyle \frac{\pi}{4}+k\frac{2\pi}{3},  k\in$ Z.

Rozwiazanie zadania 26.4

Oznaczmy przez $O$ spodek wysokości czworościanu, a przez $K, L$

jego rzuty prostokatne odpowiednio na przyprostokatne $BC\mathrm{i}AC$ podstawy

(rys. 28). Poniewaz $O$ jest środkiem okregu wpisanego $\mathrm{w} \triangle ABC$, wiec

{\it D} $|OK| = |OL| = r$, czyli punkty

$L |KC|=|LC|$.   (6)

$r$ Mamy $\triangle DOK \equiv \triangle DOL$, gdyz
\begin{center}
\includegraphics[width=66.240mm,height=81.792mm]{./KursMatematyki_PolitechnikaWroclawska_1999_2004_page122_images/image001.eps}
\end{center}
{\it E}

{\it A} $\cdot C$

$\alpha$

{\it S}

{\it O} $r K$

{\it B}

$O, K, L \mathrm{i}$ wierzcho ek kata

prostego $C$ tworza kwadrat $0$ bo-

ku $r$. Stad

oba sa prostokatne $\mathrm{i}$ maja takie

same przyprostokatne. Stad

$|DK| = |DL|$. Poniewaz wyso-

kośč DO jest prostopad a do pod-

stawy, wiec DO $\perp BC$. Ponad-

{\it to OK} $\perp BC. \mathrm{Z}$ twierdzenia

$0$ trzech prostopad ych wniosku-

$\mathrm{R}\mathrm{y}\mathrm{s}$. 28

jemy, $\dot{\mathrm{z}}\mathrm{e} DK \perp BC$. Analogi-

cznie stwierdzamy, $\dot{\mathrm{z}}\mathrm{e} DL\perp AC.$

Wynika stad, $\dot{\mathrm{z}}\mathrm{e}\triangle DKC\mathrm{i}\triangle DLC$ sa przystajacymi trójkatami prostokat-

nymi $\mathrm{i}\mathrm{w}$ konsekwencji

$\angle DCK=\angle DCL$.   (7)

Niech $E$ oznacza rzut prostokatny punktu $K$ na krawed $\acute{\mathrm{z}} DC$. Ze

wzorów (6) $\mathrm{i}$ (7) oraz $\mathrm{z}$ II cechy przystawania trójkatów (bkb) wynika, $\dot{\mathrm{z}}\mathrm{e}$

$\triangle KCE \equiv \triangle LCE$, a stad $ LE\perp DC$. To oznacza, $\dot{\mathrm{z}}\mathrm{e}$ krawed $\acute{\mathrm{z}} DC$ jest





143

prostopadla do plaszczyzny wyznaczonej przez punkty $K, L\mathrm{i}E\mathrm{i}\mathrm{w}$ kon-

sekwencji $\angle KEL=\beta. \mathrm{Z}$ porównania trójkatow równoramiennych $\triangle KCL$

$\mathrm{i} \triangle KEL$, majacych wspólna podstawe oraz $|KE| < |KC| (|KE|$ jest

przyprostokatna), wnioskujemy, $\dot{\mathrm{z}}\mathrm{e} \angle KEL = \beta > \angle KCL = \displaystyle \frac{\pi}{2}$, zatem

dziedzina dla kata $\beta$ jest przedzial $(\displaystyle \frac{\pi}{2},\pi).$
\begin{center}
\includegraphics[width=51.720mm,height=54.408mm]{./KursMatematyki_PolitechnikaWroclawska_1999_2004_page123_images/image001.eps}
\end{center}
{\it D}

$(\cdot E$

{\it O S}

Rys. 29

{\it C}
\begin{center}
\includegraphics[width=66.804mm,height=90.168mm]{./KursMatematyki_PolitechnikaWroclawska_1999_2004_page123_images/image002.eps}
\end{center}
{\it A  L C}

$\alpha$  {\it r}

{\it r}

{\it S}

{\it O K}

{\it r}

{\it B}

Rys. 30

$\mathrm{W}$ celu wyznaczenia wysokości czworościanu oznaczmy przez $S$ środek

kwadratu OKCL. Wówczas $|ES|= |SK|\displaystyle \mathrm{c}\mathrm{t}\mathrm{g}\frac{\beta}{2}=\frac{r}{\sqrt{2}}\mathrm{c}\mathrm{t}\mathrm{g}\frac{\beta}{2}.$ Poprowad $\acute{\mathrm{z}}\mathrm{m}\mathrm{y}$

plaszczyzne przechodzaca przez $DO$ oraz przez $C$. Przekrój czworościanu ta

plaszczyzna pokazano na rysunku 29. $\mathrm{Z}$ twierdzenia Pitagorasa $\mathrm{w}\triangle ESC$

mamy $|EC|^{2}=|SC|^{2}-|ES|^{2}=\displaystyle \frac{r^{2}}{2}-\frac{r^{2}}{2}\mathrm{c}\mathrm{t}\mathrm{g}^{2}\frac{\beta}{2}=r^{2}\frac{-\cos\beta}{2\sin^{2}\frac{\beta}{2}}. \mathrm{Z}$ podobieństwa

trójkatów $\triangle ESC\mathrm{i}\triangle DOC$ dostajemy proporcje $\displaystyle \frac{H}{|OC|} = \displaystyle \frac{|ES|}{|EC|}$. Stad

$H=\displaystyle \frac{|OC||ES|}{|EC|}=\frac{r^{2}\mathrm{c}\mathrm{t}\mathrm{g}\frac{\beta}{2}}{r\sqrt{\frac{-\cos\beta}{2\sin^{2}\frac{\beta}{2}}}}=r\sqrt{2}\frac{\cos\frac{\beta}{2}}{\sqrt{-\cos\beta}}.$

(8)





144

Dla obliczenia pola podstawy czworościanu (rys. 30) zauwazmy, $\dot{\mathrm{z}}\mathrm{e}$

$|AC|=|LC|+|AL|=r+r\displaystyle \mathrm{c}\mathrm{t}\mathrm{g}\frac{\alpha}{2}$ oraz $|BC|=|AC|$ tg $\alpha$. Stad mamy

$P_{p}=\displaystyle \frac{1}{2}|AC|$

$|BC|=\displaystyle \frac{1}{2}r^{2}(\mathrm{c}\mathrm{t}\mathrm{g}\frac{\alpha}{2}+1)^{2}$ tg $\displaystyle \alpha=\frac{1}{2}r^{2}\frac{(\sin\frac{\alpha}{2}+\cos\frac{\alpha}{2})^{2}}{\sin^{2}\frac{\alpha}{2}}$ tg $\alpha$

$\mathrm{i}$ ostatecznie

$P_{p}=r^{2}\displaystyle \frac{1+\sin\alpha}{\cos\alpha}$ ctg $\displaystyle \frac{\alpha}{2}.$

(9)

$\mathrm{Z}$ równości (8) $\mathrm{i}$ (9) otrzymujemy

$V=\displaystyle \frac{1}{3}P_{p}H=\frac{\sqrt{2}}{3}r^{3}\frac{1+\sin\alpha}{\cos\alpha\sqrt{-\cos\beta}}\cos\frac{\beta}{2}$ ctg $\displaystyle \frac{\alpha}{2}.$

Odp. Objetośč czworościanu wynosi $\displaystyle \frac{\sqrt{2}}{3}r^{3}\frac{1+\sin\alpha}{\cos\alpha\sqrt{-\cos\beta}}\cos\frac{\beta}{2}\mathrm{c}\mathrm{t}\mathrm{g}\frac{\alpha}{2}.$

Rozwiazanie zadania 28.2

Aby nierównośč

$\displaystyle \frac{2px^{2}+2px+1}{x^{2}+x+2-p^{2}}\geq 2$

(10)

byla spelniona dla $\mathrm{k}\mathrm{a}\dot{\mathrm{z}}$ dej liczby rzeczywistej, jej dziedzina musi byč $\mathrm{R}$, czyli

trójmian kwadratowy $\mathrm{w}$ mianowniku nie $\mathrm{m}\mathrm{o}\dot{\mathrm{z}}\mathrm{e}$ mieč pierwiastków rzeczy-

wistych. Stad otrzymujemy warunek $\triangle_{0} = 1-4(2-p^{2}) =4p^{2}-7< 0.$

Nierównośč ta jest spelniona dla

$p\displaystyle \in(-\frac{\sqrt{7}}{2},\frac{\sqrt{7}}{2})$

(11)

Dla parametru $p$ spelniajacego warunek (ll) mianownik lewej strony

(10) jest dodatni na calej prostej, wiec po pomnozeniu obu stron

(10) przez ten mianownik otrzymujemy nierównośč równowazna

$2px^{2}+2px+1\geq 2x^{2}+2x+4-2p^{2}$, a po uporzadkowaniu

$2(p-1)x^{2}+2(p-1)x+2p^{2}-3\geq 0.$

(12)





145

Dla $p=1$lewa strona (12) jest równa $-1\mathrm{i}$ nierównośč niejest spelniona dla

$\dot{\mathrm{z}}$ adnego $x$. Gdy $p<1\mathrm{t}\mathrm{z}\mathrm{n}$. wspólczynnik przy $x^{2}$ jest ujemny, nierównośč

(12) nie $\mathrm{m}\mathrm{o}\dot{\mathrm{z}}\mathrm{e}$ byč spelniona dla wszystkich $x$ (gdyz,,ramiona paraboli sa

skierowane $\mathrm{w}$ dól''). Natomiast dla $p>1$, nierównośč (12) bedzie spe1niona

dla wszystkich liczb rzeczywistych wtedy $\mathrm{i}$ tylko wtedy, gdy

$\triangle_{1} =4(p-1)^{2}-8(p-1)(2p^{2}-3) \leq 0$. Po podzieleniu obu stron przez

wyrazenie dodatnie $4(p-1)$ otrzymujemy $-4p^{2}+p+5\leq 0$, skad od razu

mamy $ p\leq$ -llub $ p\geq \displaystyle \frac{5}{4}$. Poniewaz $\displaystyle \frac{5}{4}< \displaystyle \frac{\sqrt{7}}{2}\mathrm{i}$ zalozyliśmy, $\dot{\mathrm{z}}\mathrm{e}p>1$, wiec

laczac wszystkie otrzymane warunki dostajemy ostatecznie $ p\in [\displaystyle \frac{5}{4},\frac{\sqrt{7}}{2}$).

Odp. Nierównośč jest spelniona dla $\mathrm{k}\mathrm{a}\dot{\mathrm{z}}$ dej liczby rzeczywistej, gdy

$ p\in [\displaystyle \frac{5}{4},\frac{\sqrt{7}}{2}).$

Rozwiazanie zadania 29.8

$\mathrm{Z}$ postaci ciagu odczytujemy wyraz poczatkowy $\alpha_{0}=x+1$ oraz iloraz

$q = -x^{2}$ Jeśli $x = -1$, to wszystkie wyrazy ciagu sa zerami $\mathrm{i}$ suma

$S(-1)=0$. Gdy $x\neq-1$, wówczas warunkiem istnienia sumy nieskończonego

ciagu geometrycznegojest $|q|<1$, czyli $|-x^{2}|=x^{2}<1$, skad od razu otrzy-

mujemy $x\in(-1,1)$. Ostatecznie dziedzina sumy $S(x)$ jest $D=[-1$, 1).

Korzystajac ze wzoru na sume nieskończonego ciagu geometrycznego,

dostajemy $S(x) =\displaystyle \frac{x+1}{1-(-x^{2})}=\frac{x+1}{x^{2}+1},  x\in (-1,1)$. Wzór ten pozostaje

prawdziwy takze dla $x=-1$. Dlatego $\mathrm{m}\mathrm{o}\dot{\mathrm{z}}$ na napisač

$S(x)=\displaystyle \frac{x+1}{x^{2}+1}, x\in D=[-1$, 1$)$.   (13)

Dalsze postepowanie sprowadza $\mathrm{s}\mathrm{i}\mathrm{e}$ do wyznaczenia wartości namniejszej

$\mathrm{i}$ najwiekszej funkcji wymiernej $S(x)$, danej wzorem (13). Zauwazmy, $\dot{\mathrm{z}}\mathrm{e}$

mianownik jest dodatni, a licznik nieujemny, zatem $S(x) \geq 0$ dla wszyst-

kich $x$. Stad wynika, $\dot{\mathrm{z}}\mathrm{e}$ najmniejsza wartościa tej funkcji jest 0 $\mathrm{i}$ jest ona

osiagana dla $x=-1.$

Dla znalezienia wartości najwiekszej wykorzystamy pochodna funkcji

$S(x).$

$S'(x)=\displaystyle \frac{1\cdot(x^{2}+1)-2x(x+1)}{(x^{2}+1)^{2}}=\frac{-x^{2}-2x+1}{(x^{2}+1)^{2}}.$





146

Miejsca zerowe pochodnej spelniaja równanie $-x^{2}-2x+1=0$. Stad dosta-

jemy $\triangle= 8$ oraz $x_{1} = -1-\sqrt{2}, x_{2} = -1+\sqrt{2}$. Tylko $x_{2} \in D$. Mamy

$S(x_{2}) = \displaystyle \frac{\sqrt{2}}{1+2-2\sqrt{2}+1} = \displaystyle \frac{1+\sqrt{2}}{2}$ oraz $\displaystyle \lim_{x\rightarrow 1-}S(x) = \displaystyle \lim_{x\rightarrow 1-}\frac{x+1}{x^{2}+1} = 1.$

Poniewaz $\displaystyle \frac{1+\sqrt{2}}{2}>1$, wiec najwieksza wartościa funkcji jest $\displaystyle \frac{1+\sqrt{2}}{2}.$

Odp. Wartośč najmniejsza sumy danego nieskończonego ciagu geo-

metrycznego wynosi 0 $\mathrm{i}$ jest osiagana dla $x = -1$, a wartośč najwieksza

tej sumy wynosi $\displaystyle \frac{1+\sqrt{2}}{2}\mathrm{i}$ jest osiagana dla $x=-1+\sqrt{2}.$

Rozwiazanie zadania 30.7

Dziedzina nierówności

$|2^{x}-3|\leq 2^{1-x}$

(14)

jest R. Nierównośč $\mathrm{t}\mathrm{e}\mathrm{r}$ozwia $\dot{\mathrm{z}}$ emy przez podstawienie $2^{x}=t, t>0$. Mamy

$2^{1-x}=22^{-x}=2\displaystyle \frac{1}{t},$ wiec po podstawieniu nierównośč (14) przyjmie postač

$|t-3|\displaystyle \leq\frac{2}{t}$. Stad od razu przechodzimy do nierówności podwójnej

$-\displaystyle \frac{2}{t}\leq t-3\leq\frac{2}{t},$

$t>0.$

(15)

Ze wzgledu na dodatni znak niewiadomej $t \mathrm{m}\mathrm{o}\dot{\mathrm{z}}$ emy $\mathrm{t}\mathrm{e}$ nierównośč pomnozyč

przez $t\mathrm{i}$ otrzymamy nastepujacy uklad nierówności kwadratowych

$\left\{\begin{array}{l}
t^{2}-3t+2\geq 0\\
t^{2}-3t-2\leq 0
\end{array}\right.$

$t>0.$

Pierwsza nierównośč powyzszego ukladu jest spelniona dla $t \leq 1 \mathrm{i}t \geq 2,$

czyli po uwzglednieniu warunku $t> 0$ dla $ t\in (0,1]\cup[2,\infty)$. Dla drugiej

nierówności mamy $\triangle_{2} = 17, t_{1}'' = \displaystyle \frac{3-\sqrt{17}}{2} < 0, t_{2}'' = \displaystyle \frac{3+\sqrt{17}}{2} \in (3,4).$

Druga nierównośč jest zatem spelniona dla $t \in (0,t_{2}''$]. Cześč wspólna

zbiorów rozwiazań obu nierówności ma postač $(0,1[\cup[2,t_{2}'']$. Poniewaz funkcja

$t = 2^{x}$ jest rosnaca, wiec zbiór rozwiazań nierówności (14) ma postač

$(-\infty,0]\cup[1,x_{0}]$, gdzie $x_{0}=\log_{2}t_{2}''\in(1,2).$





147

Wykresy funkcji wystepujacych po obu stronach nierówności (14) otrzy-

mujemy przez translacje $\mathrm{i}$ odbicia symetryczne standardowej krzywej

$\mathrm{I}^{\urcorner}$ : $y = 2^{x}$ Wykres krzywej $y = |2^{x}-3|$ dostajemy przez translacje $\mathrm{I}^{\urcorner}$

$0$ wektor $[0,-3]$, a nastepnie odbicie symetryczne cześci $\mathrm{l}\mathrm{e}\dot{\mathrm{z}}$ acej pod osia

odcietych wzgledem tej osi. Krzywa ta ma asymptote pozioma lewostronna

$y=3$. Natomiast krzywa $y=2^{1-x}$ dostajemy przez odbicie symetryczne $\mathrm{I}^{\urcorner}$

wzgledem osi rzednych, a nastepnie translacje $0$ wektor $($1, $0)$. Wykresy sa

przedstawione na rysunku 31.
\begin{center}
\includegraphics[width=110.484mm,height=74.472mm]{./KursMatematyki_PolitechnikaWroclawska_1999_2004_page127_images/image001.eps}
\end{center}
{\it y}

$y=2^{1-x}  \Gamma$

3  $y=|2^{x}-3|$

2

1

$-1$  0 1  $x_{0}$  3 4  {\it x}

Rys. 31

Odp. Zbiorem rozwiazań

(-00, $ 0]\cup [1,\displaystyle \log_{2}\frac{3+\sqrt{17}}{2}].$

nierówności jest

suma

przedzialów

Rozwiazanie zadania 31.7

Przy rozwiazywaniu zadania skorzystamy nastepujacej wlasności wek-

torów na plaszczy $\acute{\mathrm{z}}\mathrm{n}\mathrm{i}\mathrm{e}$:

$\mathrm{T}\mathrm{w}\mathrm{i}\mathrm{e}\mathrm{r}\mathrm{d}\mathrm{z}\mathrm{e}\mathrm{n}\mathrm{i}\mathrm{e}$. {\it Jeśli wektory} $\vec{u}i\vec{v}sq_{f}$ {\it prostopadte} $i$ {\it majq} $t_{G}$

$samq_{f}$ {\it dtugośč oraz} $\vec{u}=(\alpha,b)$, {\it to} $\vec{v}=(b,-\alpha) lub\vec{v}=(-b,\alpha).$

Przez $B$ oznaczmy wierzcholek kwadratu $\mathrm{l}\mathrm{e}\dot{\mathrm{z}}\mathrm{a}\mathrm{c}\mathrm{y}$ na prostej $l$, a przez

$D$ jego wierzcholek $\mathrm{l}\mathrm{e}\dot{\mathrm{z}}\mathrm{a}\mathrm{c}\mathrm{y}$ na prostej $k$. Korzystajac $\mathrm{z}$ równań prostych,

$\mathrm{m}\mathrm{o}\dot{\mathrm{z}}$ emy napisač $B(2y-1,y), D(4-3z,z)$, gdzie $y, z$ sa niezna-

$\rightarrow$

nymi rzednymi tych wierzcholków, zatem $AB= [2y-7,y-1]$ oraz





148

$\vec{AD}= [-3z-2,z-1]$. Poniewaz $\vec{AB}\perp\vec{AD}$ oraz oba wektory sa tej samej

dlugości, wiec $\mathrm{z}$ powyzszego twierdzenia otrzymujemy $\mathrm{z}$ porównania odpo-

wiednich wspólrzednych dwa uklady równań liniowych:

$\left\{\begin{array}{l}
2y-7=-z+1\\
-3z-2=y-1
\end{array}\right.$

oraz

$\left\{\begin{array}{l}
2y-7=z-1\\
3z+2=y-1
\end{array}\right.$

Po rozwiazaniu pierwszego ukladu dostajemy $y=3, z=0$, czyli $B_{1}(5,3),$

$\rightarrow$

$\rightarrow$

$\rightarrow$

$D_{1}(4,0)$ oraz $AB= [-1,2]$. Poniewaz $AB=DC$, wiec $C_{1}(3,2)$. Rozwiaza-

niem drugiego ukladu jest $y = 5, z = -2$, czyli $B_{2}(9,5), D_{2}(10,-2)$

$\rightarrow$

$\rightarrow$

$\mathrm{i}$ podobnie jak poprzednio $AB=DC=[4,-3]$, skad $C_{2}(13,2)$. Rozwiazanie

ilustruje rysunek 32.

Uwaga. Ze wzgledu na ogólnie przyjety sposób oznaczania wierzcholków

wielokatów na rysunku 32 przestawiono 1itery $B \mathrm{i} D$, oznaczajac

$B_{2}(10,-2)\mathrm{i}D_{2}(9,5).$
\begin{center}
\includegraphics[width=108.108mm,height=65.028mm]{./KursMatematyki_PolitechnikaWroclawska_1999_2004_page128_images/image001.eps}
\end{center}
{\it y}

5

$D_{2}$

$B_{1}$

3  $C_{1}$

$C_{2}$

1  {\it A}

0 1  $D_{1}$  7  10  13 x

$B_{2}$

Rys. 32

Odp. Istnieja dwa kwadraty spelniajace warunki zadania. Ich wierz-

cholkami, oprócz wierzcholka $A$, sa punkty $B_{1}(5,3), C_{1}(3,2), D_{1}(4,0)$ oraz

$B_{2}(10,-2), C_{2}(13,2), D_{2}(9,5).$

Rozwiazanie zadania 32.7

Równanie stycznej do wykresu funkcji $f(x)\mathrm{w}$ punkcie $S(x_{0},f(x_{0}))$ ma

postač ogólna $y-f(x_{0}) =f'(x_{0})(x-x_{0})$. Poniewaz $\mathrm{w}$ naszym przypadku





149

jest $f(x)=x^{4}-2x^{2}$ oraz $f'(x)=4x^{3}-4x$, wiec równanie stycznej przyjmie

postač

$y-(x_{0}^{4}-2x_{0}^{2})=(4x_{0}^{3}-4x_{0})(x-x_{0})$.   (16)

Punkt $P(1,-1)\mathrm{l}\mathrm{e}\dot{\mathrm{z}}\mathrm{y}$ na tej stycznej, wiec niewiadoma $x_{0}$ spelnia równanie

$-1+2x_{0}^{2}-x_{0}^{4}=4x_{0}(x_{0}^{2}-1)(1-x_{0})$. Po wylaczeniu wspólnych czynników

$\mathrm{i}$ uporzadkowaniu dostajemy $(x_{0}^{2}-1)(x_{0}-1)(3x_{0}-1)=0$. Równanie (16)

ma wiec trzy pierwiastki $-1$, l oraz $\displaystyle \frac{1}{3}.$

Po podstawieniu do równania (16) pierwiastków $x_{0} = -1 \mathrm{i} x_{0} = 1$

otrzymujemy $\mathrm{t}\mathrm{e}$ sama prosta $p$: $y+1=0$. Prosta ta jest wiec styczna do

wykresu $f$ równocześnie $\mathrm{w}$ punktach $P(-1,1)$ oraz $Q(-1,-1)$. Poniewaz

$f(x)=x^{4}-2x^{2}\geq-1$ (inaczej $(x^{2}-1)^{2}\geq 0$) dla wszystkich $x\mathrm{i}$ równośč ma

miejsce jedynie dla $x=-1\mathrm{i}x=1$, wiec styczna $p$ ma dwa punkty wspólne

$\mathrm{z}$ wykresem $f.$

Dla $x_{0}=\displaystyle \frac{1}{3}$ równanie (16) przyjmuje postač $l$ : $32x+27y-5=0. \mathrm{W}$ celu

określenia liczby punktów wspólnych stycznej $l\mathrm{z}$ wykresem $f$ nalezy określič

liczbe róznych pierwiastków równania $x^{4}-2x^{2} = \displaystyle \frac{5-32x}{27}, \mathrm{t}\mathrm{j}$. równania

$27x^{4}-54x^{2}+32x-5 = 0$. Ze wzgledu na stycznośč $\mathrm{w}$ punkcie $x_{0} = \displaystyle \frac{1}{3}$

równanie to ma podwójny pierwiastek $\displaystyle \frac{1}{3}$ oraz pierwiastek l (punkt $P\mathrm{l}\mathrm{e}\dot{\mathrm{z}}\mathrm{y}$

na wykresie $f$), zatem, jako równanie czwartego stopnia, ma takze czwarty

pierwiastek rzeczywisty, który obliczamy $\mathrm{z}$ równości $x_{1}x_{2}x_{3}x_{4}=\displaystyle \frac{-5}{27}$, czyli

$\mathrm{w}$ naszym przypadku $\displaystyle \frac{1}{9}x_{4} = \displaystyle \frac{-5}{27}$, skad $x_{4} = \displaystyle \frac{-5}{3}$. Styczna $l$ ma zatem

trzy punkty wspólne $\mathrm{z}$ wykresem $f$: $P, S(\displaystyle \frac{1}{3},-\frac{17}{81})$ oraz $A(-\displaystyle \frac{5}{3},\frac{175}{81}).$

$\mathrm{W}$ punkcie $A$ styczna $l$ przecina wykres $f$. Dla sporzadzenia rysunku

zauwazmy, $\dot{\mathrm{z}}\mathrm{e}f(x)$ jest funkcja parzysta. Liczba $x=0$ jest pierwiastkiem

podwójnym równania $x^{4}-2x^{2}=0$, co oznacza, $\dot{\mathrm{z}}\mathrm{e}$ wykres $f$ jest styczny do

osi odcietych $\mathrm{w}$ poczatku ukladu. Pozostale miejsca zerowe funkcji to $-\sqrt{2}$

$\mathrm{i}\sqrt{2}$. Kreślac styczne $y=0, l$ oraz $p\mathrm{i}$ zaznaczajac punkty styczności oraz

punkty $(\sqrt{2},0), B(\displaystyle \frac{5}{3},\frac{175}{81}), \mathrm{m}\mathrm{o}\dot{\mathrm{z}}$ emy narysowač wykres funkcji na $(0,\infty),$

a przez odbicie symetryczne takze $\mathrm{w}$ (-00, 0). Wykres przedstawiono na ry-

sunku 33.





19

Praca kontrolna

nr l

8.1. Suma wszystkich wyrazów nieskończonego ciagu geometrycznego

wynosi 2040. Jeś1i pierwszy wyraz tego ciagu zmniejszymy $0 172,$

a jego iloraz zwiekszymy 3-krotnie, to suma wszystkich wyrazów tak

otrzymanego ciagu wyniesie 2000. Wyznaczyč i1oraz $\mathrm{i}$ pierwszy wyraz

danego ciagu.

8.2. Obliczyč wszystkie te skladniki rozwiniecia dwumianu $(\sqrt{3}+\sqrt[3]{2})^{11}$,

które sa liczbami calkowitymi.

8.3. Narysowač staranny wykres funkcji $f(x) = |x^{2}-2|x|-3| \mathrm{i}$ na jego

podstawie podač ekstrema lokalne oraz przedzialy monotoniczności tej

funkcji.

8.4. Rozwiazač nierównośč

$x+1\geq\log_{2}(4^{x}-8).$

8.5. $\mathrm{W}$ ostroslupie prawidlowym trójkatnym krawed $\acute{\mathrm{z}}$ podstawy ma dlugośč

$\alpha$, a polowa kata plaskiego przy wierzcholkujest równa katowi nachyle-

nia ściany bocznej do podstawy. Obliczyč objetośč ostroslupa. Sporza-

dzič odpowiednie rysunki.

8.6. Znalez$\acute{}$č wszystkie wartości parametru $p$, dla których trójkat $KLM$

$0$ wierzcholkach $K(1,1), L(5,0)\mathrm{i}M(p,p-1)$ jest prostokatny. Roz-

wiazanie zilustrowač rysunkiem.

8.7. Rozwiazač równanie

--ssiinn35{\it xx}$=$--ssiinn46{\it xx}.

8.8. Przez punkt $P\mathrm{l}\mathrm{e}\dot{\mathrm{z}}\mathrm{a}\mathrm{c}\mathrm{y}$ wewnatrz trójkata $ABC$ poprowadzono proste

równolegle do wszystkich boków trójkata. Pola utworzonych $\mathrm{w}$ ten

sposób trzech mniejszych trójkatów $0$ wspólnym wierzcholku $P$ wyno-

$\mathrm{s}\mathrm{z}\mathrm{a}$ odpowiednio $S_{1}, S_{2}, S_{3}$. Obliczyč pole $S$ trójkata $ABC.$





150
\begin{center}
\includegraphics[width=90.012mm,height=83.916mm]{./KursMatematyki_PolitechnikaWroclawska_1999_2004_page130_images/image001.eps}
\end{center}
{\it y}

3

{\it l}

{\it A  B}

$1_{11}$  2  111

1

$-2  -1$  0 1  {\it x}

{\it S}  $\sqrt{2}2$

{\it p  Q  P}

Rys. 33

Odp. $\mathrm{s}_{\mathrm{a}}$ dwie takie styczne jedna $0$ równaniu $y= -1$, która ma dwa

punkty wspólne $\mathrm{z}$ wykresem funkcji $f(x)$, oraz druga $0$ równaniu

$32x+27y-5=0$ majaca trzy punkty wspólne $\mathrm{z}$ wykresem.

Rozwiazanie zadania 34.5

Wprowad $\acute{\mathrm{z}}\mathrm{m}\mathrm{y}$ nastepujace zdarzenia:

$A-$ Jaś wyciagnie co najmniej trzy monety;

$B_{i}-$ za pierwszym razem zostanie wylosowana moneta $0$ nominale $i \mathrm{z}l,$

$i=1$, 2, 5;

$C_{j}-$ dla uiszczenia zaplaty Jaś wyciagnie $j$ monet, $j=1$, 2, 3, 4.

Wówczas $A' = C_{1}\cup C_{2} \mathrm{i}$ oba skladniki sa rozlaczne. Zauwazmy, $\dot{\mathrm{z}}\mathrm{e}$

$C_{1} = B_{5}$ oraz $B_{1}\cup B_{2}\cup B_{5} = \Omega$. Ponadto $P(B_{1}) = \displaystyle \frac{1}{2}, P(B_{2}) = \displaystyle \frac{1}{3}$

$\displaystyle \mathrm{i}P(B_{5})=P(C_{1})=\frac{1}{6}$. Ze wzoru na prawdopodobieństwo calkowite mamy

$P(C_{2})=P(C_{2}|B_{1})P(B_{1})+P(C_{2}|B_{2})P(B_{2})+P(C_{2}|B_{5})P(B_{5})$.   (17)

Mamy $P(C_{2}|B_{1})=\displaystyle \frac{1}{5}$, gdyz za drugim razem Jaś musi wyciagnač mone-

$\mathrm{t}\mathrm{e} 5 \mathrm{z}l$ spośród 5 monet $\mathrm{w}$ portmonetce. Podobnie $P(C_{2}|B_{2}) = \displaystyle \frac{2}{5}$ (za





151

drugim razem musi byč wyciagnieta moneta 5 $\mathrm{z}l$ lub pozostala dostepna

moneta 2 $\mathrm{z}l$) oraz $P(C_{2}|B_{5}) =0$ (nie ma drugiego losowania, gdy $\mathrm{w}$ pier-

wszym byla moneta 5 $\mathrm{z}l$ lub inaczej $ C_{2}\cap B_{5}=\emptyset$). Po podstawieniu tych

wartości do wzoru (17) dostajemy $P(C_{2})=\displaystyle \frac{7}{30}\mathrm{i}$ ostatecznie

$P(A)=1-P(C_{1})-P(C_{2})=1-\displaystyle \frac{7}{30}-\frac{1}{6}=\frac{6}{10}.$

Odp. Prawdopodobieństwo tego, $\dot{\mathrm{z}}\mathrm{e}$ Jaś wyciagnie co najmniej trzy

monety wynosi $\displaystyle \frac{3}{5}.$





20

Praca kontrolna nr 2

9.1. Promień kuli powiekszono $\mathrm{t}\mathrm{a}\mathrm{k},\ \dot{\mathrm{z}}\mathrm{e}$ pole jej powierzchni wzroslo $0$ 44\%.

$\mathrm{O}$ ile procent wzrosla jej objetośč?

9.2. Wyznaczyč równanie krzywej utworzonej przez środki odcinków maja-

cych obydwa końce na osiach ukladu wspólrzednych $\mathrm{i}$ zawierajacych

punkt $P(2,1)$. Sporzadzič dokladny wykres $\mathrm{i}$ podač nazwe otrzymanej

krzywej.

9.3. Znalez$\acute{}$č wszystkie wartości parametru $m$, dla których równanie

$(m-1)9^{x}-4\cdot 3^{x}+m+2=0$

ma dwa rózne pierwiastki.

9.4. Róznica promienia kuli opisanej na czworościanie foremnym $\mathrm{i}$ promienia

kuli wpisanej $\mathrm{w}$ niegojest równa l. Obliczyč objetośč tego czworościanu.

9.5. Rozwiazač nierównośč

$\displaystyle \frac{2}{|x^{2}-9|}\geq\frac{1}{x+3}.$

9.6. Stosunek dlugości przyprostokatnych trójkata prostokatnego wynosi k.

Obliczyč stosunek dlugości dwusiecznych katów ostrych tego trójkata.

Zastosowač odpowiednie wzory trygonometryczne.

9.7. Zbadač przebieg zmienności i narysowač wykres funkcji

$f(x)=\displaystyle \frac{x^{2}+4}{(x-2)^{2}}.$

9.8. Znalez$\acute{}$č równania wszystkich prostych przechodzacych przez punkt

$A(\displaystyle \frac{7}{5},-2)\mathrm{i}$ stycznych do wykresu funkcji $f(x)=x^{3}-2x$. Rozwiazanie

zilustrowač rysunkiem.





21

Praca kontrolna nr 3

10.1. Stosujac zasade indukcji matematycznej, udowodnič, $\dot{\mathrm{z}}\mathrm{e}$ dla $\mathrm{k}\mathrm{a}\dot{\mathrm{z}}$ dej

liczby naturalnej $n$ suma $2^{n+1}+3^{2n-1}$ jest podzielna przez 7.

10.2. Tworzaca stozka ma dlugośč $l \mathrm{i}$ widač $\mathrm{j}\mathrm{a}$ ze środka kuli wpisanej

$\mathrm{w}$ ten stozek pod katem $\alpha$. Obliczyč objetośč $\mathrm{i}\mathrm{k}\mathrm{a}\mathrm{t}$ rozwarcia stozka.

Określič dziedzine dla kata $\alpha.$

10.3. Bez stosowania metod rachunku rózniczkowego wyznaczyč dziedzine

$\mathrm{i}$ zbiór wartości funkcji

$y=\sqrt{2+\sqrt{x}-x}.$

10.4. $\mathrm{Z}$ talii 24 kart wy1osowano (bez zwracania) cztery karty. Jakie jest

prawdopodobieństwo, $\dot{\mathrm{z}}\mathrm{e}$ otrzymano dokladnie trzy karty $\mathrm{z}$ jednego

koloru ($\mathrm{z}$ czterech $\mathrm{m}\mathrm{o}\dot{\mathrm{z}}$ liwych)?

10.5. Rozwiazač nierównośč

$\log_{1/3}$ (log2 $4x$) $\geq\log_{1/3}(2-\log_{2x}4)-1.$

10.6. $\mathrm{Z}$ punktu $C(1,0)$ poprowadzono styczne do okregu $x^{2}+y^{2} = r^{2},$

$ r\in (0,1)$. Punkty styczności oznaczono przez A $\mathrm{i}B$. Wyrazič pole

trójkata $ABC$ jako funkcje promienia $r\mathrm{i}$ znalez$\acute{}$č najwieksza wartośč

tego pola.

10.7. Rozwiazač uklad równań

$\left\{\begin{array}{l}
x^{2}+y^{2}=5|x|\\
|4y-3x+10|=10.
\end{array}\right.$

Podač interpretacje geometryczna $\mathrm{k}\mathrm{a}\dot{\mathrm{z}}$ dego $\mathrm{z}$ równań $\mathrm{i}$ sporzadzič sta-

ranny rysunek.

10.8. Rozwiazač $\mathrm{w}$ przedziale $[0,\pi]$ równanie

1$+ \sin 2x=2\sin^{2}x,$

a nastepnie nierównośč l$+ \sin 2x>2\sin^{2}x.$





22

Praca kontrolna nr 4

$\mathrm{W}$ celu przyblizenia sluchaczom, jakie wymagania byly stawiane ich

starszym kolegom przed ponad dwudziestu laty, niniejszy zestaw

zadań jest powtórzeniem pracy kontrolnej ze stycznia 1979 $\mathrm{r}.$

ll.l. Przez środek boku trójkata równobocznego przeprowadzono prosta,

tworzaca $\mathrm{z}$ tym bokiem $\mathrm{k}\mathrm{a}\mathrm{t}$ ostry $\alpha \mathrm{i}$ dzielaca ten trójkat na dwie

figury, których stosunek pól jest równy 1 : 7. Ob1iczyč miare kata $\alpha.$

11.2. $\mathrm{W}$ kule $0$ promieniu $R$ wpisano graniastoslup trójkatny prawidlowy

$0$ krawedzi podstawy równej promieniowi kuli. Obliczyč wysokośč

tego graniastoslupa.

11.3. Wyznaczyč wartości parametru $\alpha$, dla których funkcja

$f(x)=\displaystyle \frac{\alpha x}{1+x^{2}}$

osiaga maksimum równe 2.

11.4. Rozwiazač nierównośč

$\cos^{2}x+\cos^{3}x+\ldots+\cos^{n+1}x+\ldots<1+\cos x$

dla $x\in[0,2\pi].$

11.5. Wykazač, $\dot{\mathrm{z}}\mathrm{e}$ dla $\mathrm{k}\mathrm{a}\dot{\mathrm{z}}$ dej liczby naturalnej $n$

równośč

$\geq$

2 prawdziwa jest

$1^{2}+2^{2}++n^{2}=\left(\begin{array}{lll}
n & + & 1\\
 & 2 & 
\end{array}\right)+2[\left(\begin{array}{l}
n\\
2
\end{array}\right)+\left(\begin{array}{lll}
n & - & 1\\
 & 2 & 
\end{array}\right)+\ldots+\left(\begin{array}{l}
2\\
2
\end{array}\right)]$

11.6. Wyznaczyč równanie linii bedacej zbiorem środków wszystkich okre-

gów stycznych do prostej $y=0$ ijednocześnie stycznych zewnetrznie

do okregu $(x+2)^{2}+y^{2}=4$. Narysowač $\mathrm{t}\mathrm{e}$ linie.

11.7. Wyznaczyč wartości parametru $m$, dla których równanie

$9x^{2}-3x\log_{3}m+1=0$ ma dwa rózne pierwiastki rzeczywiste $x_{1}, x_{2}$

spelniajace warunek $x_{1}^{2}+x_{2}^{2}=1.$

11.8. Rozwiazač nierównośč

$\displaystyle \frac{\sqrt{30+x-x^{2}}}{x}<\frac{\sqrt{10}}{5}.$





23

Praca kontrolna nr 5

12.1. Za pomoca odpowiedniego wykresu wykazač, $\dot{\mathrm{z}}\mathrm{e}$ równanie

$\sqrt{x-3}+x = 4$ ma dokladnie jeden pierwiastek. Nastepnie wyz-

naczyč ten pierwiastek analitycznie.

12.2. Wiadomo, $\dot{\mathrm{z}}\mathrm{e}$ wielomian $w(x) = 3x^{3}-5x+1$ ma trzy pierwiastki

rzeczywiste $x_{1}, x_{2}, x_{3}$. Bez wyznaczania tych pierwiastków obliczyč

wartośč wyrazenia $(1+x_{1})(1+x_{2})(1+x_{3})$

12.3. Rzucono jeden raz kostka, a nastepnie moneta tyle razy, ile oczek

pokazala kostka. Obliczyč prawdopodobieństwo tego, $\dot{\mathrm{z}}\mathrm{e}$ rzuty mone-

ta daly co najmniej jednego orla.

12.4. Wyznaczyč równania wszystkich okregów stycznych do obu osi ukladu

wspólrzednych oraz do prostej $3x+4y=12.$

12.5. $\mathrm{W}$ ostroslupie prawidlowym czworokatnym dana jest odleglośč $d$

środka podstawy od krawedzi bocznej oraz $\mathrm{k}\mathrm{a}\mathrm{t}2\alpha$ miedzy sasiednimi

ścianami bocznymi. Obliczyč objetośč ostroslupa.

12.6. $\mathrm{W}$ trapezie równoramiennym $0$ polu $P$ dane sa promień okregu opisa-

nego $r$ oraz suma dlugości obu podstaw $s$. Obliczyč obwód tego tra-

pezu. Podač warunki rozwiazalności zadania. Sporzadzič rysunek dla

$P=12\mathrm{c}\mathrm{m}^{2}, r=3$ cmi $s=8$ cm.

12.7. Rozwiazač uklad równań

$\left\{\begin{array}{l}
px+y=3p^{2}-3p-2\\
(p+2)x+py=4p
\end{array}\right.$

$\mathrm{w}$ zalezności od parametru rzeczywistego $p$. Podač wszystkie rozwia-

zania ($\mathrm{i}$ odpowiadajace im wartości parametru $p$), dla których obie

niewiadome sa liczbami calkowitymi $0$ wartości bezwzglednej mniej-

szej od 3.

12.8. Odcinek AB $0$ końcach $A(0,\displaystyle \frac{3}{2}) \mathrm{i}B(1,y)$, gdzie $ y\in [0,\displaystyle \frac{3}{2}]$, obraca

$\mathrm{s}\mathrm{i}\mathrm{e}$ wokól osi $Ox$. Wyrazič pole powstalej powierzchni jako funkcje

zmiennej $y\mathrm{i}$ znalez$\acute{}$č najmniejsza wartośč tego pola. Sporzadzič ry-

sunek.





24

Praca kontrolna nr 6

13.1. Wykazač, $\dot{\mathrm{z}}\mathrm{e}$ dla $\mathrm{k}\mathrm{a}\dot{\mathrm{z}}$ dego kata $\alpha$ prawdziwa jest nierównośč

$\sqrt{3}\sin\alpha+\sqrt{6}\cos\alpha\leq 3.$

13.2. Dane sa punkty $A(2,2)\mathrm{i}B(-1,4)$. Wyznaczyč dlugośč rzutu prosto-

katnego odcinka $AB$ na prosta $0$ równaniu $12x+5y=30$. Sporzadzič

rysunek.

13.3. Niech $f(m)$ bedzie suma odwrotności pierwiatków rzeczywistych rów-

nania kwadratowego

$(2^{m}-7)x^{2}-|2^{m+1}-8|x+2^{m}=0,$

gdzie m jest parametrem rzeczywistym.

f(m) i narysowač wykres tej funkcji.

Napisač wzór określajacy

13.4. Dwóch strzelców strzela równocześnie do tego samego celu niezaleznie

od siebie. Pierwszy strzelec trafia za $\mathrm{k}\mathrm{a}\dot{\mathrm{z}}$ dym razem $\mathrm{z}$ prawdopodobień-

stwem $\displaystyle \frac{2}{3}\mathrm{i}$ oddaje 2 strza1y, a drugi trafia $\mathrm{z}$ prawdopodobieństwem $\displaystyle \frac{1}{2}$

$\mathrm{i}$ oddaje 5 strza1ów. Ob1iczyč prawdopodobieństwo, $\dot{\mathrm{z}}\mathrm{e}$ cel zostanie

trafiony dokladnie 3 razy.

13.5. Liczby $\alpha_{1}, \alpha_{2}, \alpha_{n}, n\geq 3$, tworza ciag arytmetyczny. Suma wyrazów

tego ciagu wynosi 28, suma wyrazów $0$ numerach nieparzystych wyno-

si 16, a i1oczyn $\alpha_{2}\cdot\alpha_{3}=48$. Wyznaczyč te liczby.

13.6. $\mathrm{W}$ trójkacie $ABC, \mathrm{w}$ którym $|AB| = 7$ oraz $|AC| =9$, a $\mathrm{k}\mathrm{a}\mathrm{t}$ przy

wierzcholku $A$ jest dwa razy wiekszy $\mathrm{n}\mathrm{i}\dot{\mathrm{z}} \mathrm{k}\mathrm{a}\mathrm{t}$ przy wierzcholku $B.$

Obliczyč stosunek promienia kola wpisanego $\mathrm{w}$ trójkat do promienia

kola opisanego na tym trójkacie. Rozwiazanie zilustrowač rysunkiem.

13.7. Zaznaczyč na plaszczy $\acute{\mathrm{z}}\mathrm{n}\mathrm{i}\mathrm{e}$ nastepujace zbiory punktów

$A=\{(x,y):x+y-2\geq|x-2|\},$

$B=\{(x,y):y\leq\sqrt{4x-x^{2}}\}.$

Nastepnie znalez$\acute{}$č na brzegu zbioru $A\cap B$ punkt $Q$, którego odleglośč

od punktu $P(\displaystyle \frac{5}{2},1)$ jest najmniejsza.

13.8. Zbadač przebieg zmienności $\mathrm{i}$ narysowač wykres funkcji

$f(x)=\displaystyle \frac{1}{2}x^{2}-4+\sqrt{8-x^{2}}.$





25

Praca kontrolna nr 7

14.1. Ile elementów ma zbiór $A$, jeśli liczba jego podzbiorów trójelemen-

towych jest wieksza $048$ od liczby podzbiorów dwuelementowych?

14.2. $\mathrm{W}$ sześciokat foremny $0$ boku l wpisano okrag. Nastepnie $\mathrm{w}$ otrzy-

many okrag wpisano sześciokat foremny, $\mathrm{w}$ który znów wpisano okrag

$\mathrm{i}\mathrm{t}\mathrm{d}$. Obliczyč sume obwodów wszystkich otrzymanych $\mathrm{w}$ taki sposób

okregów.

14.3. Dana jest rodzina prostych $0$ równaniach $2x+my-m-2 = 0,$

$m\in R$. Które $\mathrm{z}$ prostych tej rodziny sa:

a) prostopadle do prostej $x+4y+2=0,$

b) równolegle do prostej $3x+2y=0,$

c) tworza $\mathrm{z}$ prosta $x-\displaystyle \sqrt{3}y-1=0\mathrm{k}\mathrm{a}\mathrm{t}\frac{\pi}{3}.$

14.4. Sprawdzič $\mathrm{t}\mathrm{o}\dot{\mathrm{z}}$ samośč tg $(x-\displaystyle \frac{\pi}{4})-1=\frac{-2}{\mathrm{t}\mathrm{g}x+1}$. Korzystajac $\mathrm{z}$ niej,

sporzadzič wykres funkcji $f(x)=\displaystyle \frac{1}{\mathrm{t}\mathrm{g}x+1}\mathrm{w}$ przedziale $[0,\pi].$

14.5. Dany jest okrag $K\mathrm{o}$ równaniu $x^{2}+y^{2}-6y=27$. Wyznaczyč równanie

krzywej $\Gamma$ bedacej obrazem okregu $K\mathrm{w}$ powinowactwie prostokatnym

$0$ osi $ox \mathrm{i}$ skali $k = \displaystyle \frac{1}{3}$. Obliczyč pole figury ograniczonej lukiem

okregu $K\mathrm{i}$ krzywej $\Gamma, \mathrm{l}\mathrm{e}\dot{\mathrm{z}}$ acej pod osia odcietych. Wykonač rysunek.

14.6. Korzystajac $\mathrm{z}$ nierówności $2\sqrt{\alpha b} \leq \alpha+b, \alpha, b > 0$, obliczyč gra-

nice $\displaystyle \lim_{n\rightarrow\infty}(\frac{\log_{5}16}{\log_{2}3})^{n}$

14.7. Trylogie skladajaca $\mathrm{s}\mathrm{i}\mathrm{e}\mathrm{z}$ dwóch powieści dwutomowych oraz jednej

jednotomowej ustawiono na pólce $\mathrm{w}$ przypadkowej kolejności. Jakie

jest prawdopodobieństwo tego, $\dot{\mathrm{z}}\mathrm{e}$ tomy a) obydwu, b) co najmniej

jednej $\mathrm{z}$ dwutomowych powieści znajduja $\mathrm{s}\mathrm{i}\mathrm{e}$ obok siebie $\mathrm{i}$ przy tym

tom I $\mathrm{z}$ lewej, a tom II $\mathrm{z}$ prawej strony.

14.8. $\mathrm{W}$ ostroslupie prawidlowym czworokatnym krawed $\acute{\mathrm{z}}$ boczna jest na-

chylona do plaszczyzny podstawy pod katem $\alpha$, a krawed $\acute{\mathrm{z}}$ podstawy

ma dlugośč $\alpha$. Obliczyč promień kuli stycznej do wszystkich krawedzi

tego ostroslupa. Sporzadzič odpowiednie rysunki.





SPIS TREŚCI

l. Przedmowa $\ldots\ldots\ldots\ldots\ldots\ldots\ldots\ldots\ldots\ldots\ldots\ldots\ldots\ldots\ldots\ldots$.. 5

2. Zadania $\ldots\ldots\ldots\ldots\ldots\ldots\ldots\ldots\ldots\ldots\ldots\ldots\ldots\ldots\ldots\ldots\ldots$.. 7

7

3. Indeks tematyczny $\ldots\ldots\ldots\ldots\ldots\ldots\ldots\ldots\ldots\ldots\ldots\ldots\ldots\ldots$. 57

4. Odpowiedzi do zadań $\ldots\ldots\ldots\ldots\ldots\ldots\ldots\ldots\ldots\ldots\ldots\ldots\ldots$. 65

5. Wskazówki do zadań $\ldots\ldots\ldots\ldots\ldots\ldots\ldots\ldots\ldots\ldots\ldots\ldots\ldots$.. 97

6. 12 przykladowych rozwiazań

135





Edycja

XXXI

2001/2002





29

Praca kontrolna nr l

15.1. Dwaj rowerzyści wyruszyli jednocześnie $\mathrm{w}$ droge, jeden $\mathrm{z}$ A do $\mathrm{B},$

drugi $\mathrm{z} \mathrm{B}$ do A $\mathrm{i}$ spotkali $\mathrm{s}\mathrm{i}\mathrm{e}$ po jednej godzinie. Pierwszy $\mathrm{z}$ nich

przebywal $\mathrm{w}$ ciagu godziny $03$ km wiecej $\mathrm{n}\mathrm{i}\dot{\mathrm{z}}$ drugi $\mathrm{i}$ przyjechal do celu

$027$ minut wcześniej $\mathrm{n}\mathrm{i}\dot{\mathrm{z}}$ drugi. Jakie byly predkości obu rowerzystów

$\mathrm{i}$ jaka jest odleglośč AB?

15.2. Rozwiazač nierównośč $\displaystyle \sqrt{x^{2}-3}>\frac{2}{x}.$

15.3. Rysunek przedstawia dach budynku $\mathrm{w}$ rzucie poziomym.

$\mathrm{z}\mathrm{p}$ aszczyznjest nachylona do $\mathrm{p}$ aszczyzny

poziomej pod katem $30^{\mathrm{O}} \mathrm{D}$ ugośč dachu

wynosi 18 $\mathrm{m}$, a szerokośč 9 $\mathrm{m}$. Obliczyč

pole powierzchni dachu oraz ca kowita ku-

bature strychu $\mathrm{w}$ tym budynku.

$K\mathrm{a}\dot{\mathrm{z}}$ da
\begin{center}
\includegraphics[width=48.204mm,height=24.228mm]{./KursMatematyki_PolitechnikaWroclawska_1999_2004_page21_images/image001.eps}
\end{center}
15.4. Pewna firma przeprowadza co kwartal regulacje plac dla swoich pra-

cowników, waloryzujac je zgodnie ze wska $\acute{\mathrm{z}}\mathrm{n}\mathrm{i}\mathrm{k}\mathrm{i}\mathrm{e}\mathrm{m}$ inflacji, który jest

staly $\mathrm{i}$ wynosi 1,5\% kwarta1nie, oraz do1iczajac sta1a kwote podwyzki

16 $\mathrm{z}l. \mathrm{W}$ styczniu 2001 pan Kowa1ski otrzyma1 wynagrodzenie 1600

$\mathrm{z}l$. Jaka pensje otrzyma $\mathrm{w}$ kwietniu 2002? Wyznaczyč wzór ogó1ny

na pensje $w_{n}$ pana Kowalskiego $\mathrm{w}n$-tym kwartale, przyjmujac, $\dot{\mathrm{z}}\mathrm{e}$

$w_{1}=1600$ jest placa $\mathrm{w}$ pierwszym kwartale 2001. Ob1iczyč średnia

miesieczna place pana Kowalskiego $\mathrm{w}$ 2002 roku.

15.5. Wyznaczyč funkcje odwrotna do $f(x) = x^{3}, x \in R$. Nastepnie

narysowač wykres funkcji $h(x) = \sqrt[3]{(|x|-1)}+1$, wyrazajac $\mathrm{j}\mathrm{a}$

za pomoca $f^{-1}$

15.6. Rozwiazač równanie $\displaystyle \frac{\sin 2x}{\cos 4x}=1.$

15.7. Dany jest trójkat $0$ wierzcholkach $A(-2,1), B(-1,-6), C(2,5).$

Za pomoca rachunku wektorowego obliczyč cosinus kata miedzy dwu-

sieczna kata $A\mathrm{i}$ środkowa boku $BC$. Sporzadzič rysunek.

15.8. Zbadač przebieg zmienności $\mathrm{i}$ narysowač wykres funkcji

$f(x)=x+\displaystyle \frac{x}{x-1}+\frac{x}{(x-1)^{2}}+\frac{x}{(x-1)^{3}}+$





30

Praca kontrolna nr 2

16.1. Cena llitra paliwa zostala obnizona $0$ 15\%. Po dwóch tygodniach

dokonano kolejnej zmiany ceny llitra paliwa, podwyzszajacja $0$ 15\%.

$\mathrm{O}$ ile procent końcowa cena paliwa rózni $\mathrm{s}\mathrm{i}\mathrm{e}$ od poczatkowej?

16.2. Wyznaczyč $\mathrm{i}$ narysowač zbiór zlozony $\mathrm{z}$ punktów $(x,y)$ plaszczyzny

spelniajacych warunek

$x^{2}+y^{2}=8|x|+6|y|.$

16.3. Wysokośč ostroslupa trójkatnego prawidlowego wynosi $h$, a $\mathrm{k}\mathrm{a}\mathrm{t}$ mie-

dzy wysokościami ścian bocznych poprowadzonymi $\mathrm{z}$ wierzcholka

ostroslupa jest równy $ 2\alpha$. Obliczyč pole powierzchni bocznej tego

ostroslupa. Sporzadzič odpowiednie rysunki.

16.4. $\mathrm{Z}$ arkusza blachy $\mathrm{w}$ ksztalcie równolegloboku $0$ bokach 30 cm $\mathrm{i}60$ cm

$\mathrm{i}$ kacie ostrym $60^{\mathrm{o}}$ nalezy odciač dwa przeciwlegle trójkatne narozniki

$\mathrm{t}\mathrm{a}\mathrm{k}$, aby powstal romb $\mathrm{o}\mathrm{m}\mathrm{o}\dot{\mathrm{z}}$ liwie najwiekszym polu. Określič przez

który punkt na dluzszym boku równolegloboku nalezy przeprowadzič

ciecie oraz obliczyč $\mathrm{k}\mathrm{a}\mathrm{t}$ ostry otrzymanego rombu. Wynik zaokraglič

do jednej minuty katowej.

16.5. Rozwiazač równanie

$2^{\log_{\sqrt{2}^{X}}}=(\sqrt{2})^{\log_{x}2}$

16.6. Wyznaczyč dziedzine $\mathrm{i}$ zbiór wartości funkcji

$f(x)=\displaystyle \frac{4}{\sin x+2\cos x+3}.$

16.7. Znalez$\acute{}$č wszystkie wartości parametru $p$, dla których równanie

$px^{4}-4x^{2}+p+1=0$

ma dwa rózne pierwiastki.

16.8. Wyznaczyč tangens kata, pod którym styczna do wykresu funkcji

$f(x)=\displaystyle \frac{8}{x^{2}+3}$

$\mathrm{w}$ punkcie $A($3, $\displaystyle \frac{2}{3})$ przecina ten wykres.





31

Praca kontrolna nr 3

17.1. Dla jakich wartości $\sin x$ liczby $\sin x, \cos x, \sin 2x (\mathrm{w}$ podanym

porzadku) sa kolejnymi wyrazami ciagu geometrycznego? Wyznaczyč

czwarty wyraz tego ciagu dla $\mathrm{k}\mathrm{a}\dot{\mathrm{z}}$ dego $\mathrm{z}$ rozwiazań.

17.2. $\mathrm{W}$ pewnych zawodach sportowych startuje 16 druzyn. $\mathrm{W}$ elimina-

cjach sa one losowo dzielone na 4 grupy po 4 druzyny $\mathrm{w}\mathrm{k}\mathrm{a}\dot{\mathrm{z}}$ dej grupie.

Obliczyč prawdopodobieństwo tego, $\dot{\mathrm{z}}\mathrm{e}$ trzy zwycieskie druzyny $\mathrm{z}$ po-

przednich zawodów znajda $\mathrm{s}\mathrm{i}\mathrm{e}\mathrm{w}$ trzech róznych grupach.

17.3. Nie wykonujac dzielenia, udowodnič, $\dot{\mathrm{z}}\mathrm{e}$ wielomian

$(x^{2}+x+1)^{3}-x^{6}-x^{3}-1$

jest podzielny przez trójmian $(x+1)^{2}$

17.4. Wyznaczyč równanie okregu $0$ promieniu $r$ stycznego do paraboli

$y=x^{2}\mathrm{w}$ dwóch punktach. Dla jakiego $r$ zadanie ma rozwiazanie?

Sporzadzič rysunek, przyjmujac $r=3/2.$

17.5. Stosujac zasade indukcji matematycznej, udowodnič prawdziwośč

wzoru

$\left(\begin{array}{l}
2\\
2
\end{array}\right) - \left(\begin{array}{l}
3\\
2
\end{array}\right)+\left(\begin{array}{l}
4\\
2
\end{array}\right) - \left(\begin{array}{l}
5\\
2
\end{array}\right)+\ldots+\left(\begin{array}{l}
2n\\
2
\end{array}\right) =n^{2},$

$n\geq 1.$

17.6. Rozwiazač nierównośč

$\log_{x}(1-6x^{2})\geq 1.$

17.7. $\mathrm{W}$ trapezie ABCD opisanym na okregu $0$ środku $S$ dane sa ramie

$|AD| =c$ oraz $|AS| =d$. Punkt styczności okregu $\mathrm{z}$ podstawa $AB$

dzieli $\mathrm{j}\mathrm{a}\mathrm{w}$ stosunku 1 : 2. Ob1iczyč po1e tego trapezu. Sporzadzič

rysunek dla $c=5\mathrm{i}d=4.$

17.8. Wszystkie ściany równoleglościanu sa rombami $0$ boku $\alpha \mathrm{i}$ kacie

ostrym $\beta$. Obliczyč objetośč tego równoleglościanu. Sporzadzič

rysunek. Obliczenia odpowiednio uzasadnič.





32

Praca kontrolna nr 4

18.1. Obliczyč granice ciagu 0 wyrazie ogólnym

$\displaystyle \alpha_{n}=\frac{2^{n}+2^{n+1}+..+2^{2n}}{2^{2}+2^{4}+\ldots+2^{2n}}.$

18.2. Wyznaczyč równanie prostej prostopadlej do prostej $0$ równaniu

$2x+3y+3 = 0 \mathrm{i} \mathrm{l}\mathrm{e}\dot{\mathrm{z}}$ acej $\mathrm{w}$ równej odleglości od dwóch danych

punktów $A(-1,1)\mathrm{i}B(3,3)$. Sporzadzič rysunek.

18.3. Tworzaca stozka ma dlugośč $l \mathrm{i}$ widač $\mathrm{j}\mathrm{a}$ ze środka kuli wpisanej

$\mathrm{w}$ ten stozek pod katem $\alpha$. Obliczyč objetośč $\mathrm{i}\mathrm{k}\mathrm{a}\mathrm{t}$ rozwarcia stozka.

Określič dziedzine kata $\alpha.$

18.4. Bolek kupil jeden dlugopis $\mathrm{i}k$ zeszytów, zaplacil $k\mathrm{z}l\mathrm{i}50$ gr, a Lolek

kupil $k$ dlugopisów $\mathrm{i}4$ zeszyty, $\mathrm{i}$ zaplaci12, 5 $k\mathrm{z}l$. Wyznaczyč cene

dlugopisu $\mathrm{i}$ zeszytu $\mathrm{w}$ zalezności od parametru $k$. Znalez$\acute{}$č wszystkie

$\mathrm{m}\mathrm{o}\dot{\mathrm{z}}$ liwe wartości tych cen wiedzac, $\dot{\mathrm{z}}\mathrm{e}$ zeszyt kosztuje nie mniej $\mathrm{n}\mathrm{i}\dot{\mathrm{z}}$

50 gr, dlugopis jest drozszy od zeszytu, a ceny obydwu artykulów

wyrazaja $\mathrm{s}\mathrm{i}\mathrm{e}\mathrm{w}$ pelnych zlotych $\mathrm{i}$ dziesiatkach groszy.

18.5. Rozwiazač nierównośč $\mathrm{t}\mathrm{g}^{3}x\geq\sin 2x.$

18.6. $\dot{\mathrm{Z}}$ arówki sa sprzedawane $\mathrm{w}$ opakowaniach po 6 sztuk. Prawdopodo-

bieństwo, $\dot{\mathrm{z}}\mathrm{e}$ pojedyncza $\dot{\mathrm{z}}$ arówka jest dobra wynosi $\displaystyle \frac{2}{3}$. Jakie jest

prawdopodobieństwo tego, $\dot{\mathrm{z}}\mathrm{e} \mathrm{w}$ jednym opakowaniu znajda $\mathrm{s}\mathrm{i}\mathrm{e}$ co

najmniej 4 dobre $\dot{\mathrm{z}}$ arówki. $\mathrm{O}$ ile zwiekszy $\mathrm{s}\mathrm{i}\mathrm{e}$ prawdopodobieństwo

tego zdarzenia, jeśli jedna, wylosowana $\mathrm{z}$ opakowania $\dot{\mathrm{z}}$ arówka, oka-

zala $\mathrm{s}\mathrm{i}\mathrm{e}$ dobra.

18.7. Prosta styczna $\mathrm{w}$ punkcie $P$ do okregu $0$ promieniu 2 $\mathrm{i}$ pólprosta

wychodzaca ze środka okregu majaca $\mathrm{z}$ okregiem punkt wspólny $S$

przecinaja $\mathrm{s}\mathrm{i}\mathrm{e}\mathrm{w}$ punkcie $A$ pod katem $60^{\circ}$ Znalez$\acute{}$č promień okregu

stycznego do odcinków $AP$, {\it AS} $\mathrm{i}$ luku $PS$. Sporzadzič rysunek.

18.8. $\mathrm{W}$ ostroslupie prawidlowym, którego podstawa jest kwadrat, pole

$\mathrm{k}\mathrm{a}\dot{\mathrm{z}}$ dej $\mathrm{z}$ pieciu ścian wynosi l. Ostroslup ten ścieto plaszczyzna

równolegla do podstawy $\mathrm{t}\mathrm{a}\mathrm{k}$, aby uzyskač maksymalny stosunek obje-

tości do pola powierzchni calkowitej. Obliczyč pole powierzchni calko-

witej otrzymanego ostroslupa ścietego. Sporzadzič rysunek.





33

Praca kontrolna nr 5

19.1. $\mathrm{W}$ czworokacie ABCD dane sa wktory $\vec{AB}= [2,-1], \vec{BC}= [3$, 3$],$

$\vec{CD}=[-4,1]$. Punkty $K\mathrm{i}M$ sa środkami boków $CD$ oraz $AD$. Za

pomoca rachunku wektorowego obliczyč pole trójkata $KMB.$

Sporzadzič rysunek.

19.2. Trzy rózne krawedzie oraz przekatna prostopadlościanu tworza cztery

kolejne wyrazy ciagu arytmetycznego. Wyznaczyč sume dlugości

wszystkich krawedzi tego prostopadlościanu, jeśli przekatna ma dlu-

gośč 7 cm.

19.3. Na plaszczy $\acute{\mathrm{z}}\mathrm{n}\mathrm{i}\mathrm{e}Oxy$ dane sa zbiory $A = \{(x,y):y\leq\sqrt{5x-x^{2}}\}$

oraz $B_{s} = \{(x,y):3x+4y=s\}$. Dla jakich wartości parametru $s$

zbiór $A\cap B_{s}$ nie jest pusty? Sporzadzič rysunek.

19.4. Dzialka gruntu ma ksztalt trapezu $0$ bokach 20 $\mathrm{m}, 30\mathrm{m}, 40\mathrm{m}\mathrm{i}60$

$\mathrm{m}$. Wlaściciel dzialki twierdzi, $\dot{\mathrm{z}}\mathrm{e}$ powierzchnia jego dzialki wynosi

ponad ll arów. Czy wlaściciel ma racje? Jeśli $\mathrm{t}\mathrm{a}\mathrm{k}$, to narysowač plan

dzialki $\mathrm{w}$ skali 1:1000 $\mathrm{i}$ podač jej dokladna powierzchnie.

19.5. Dane jest równanie kwadratowe $\mathrm{z}$ parametrem $m$

$(m+2)x^{2}+4\sqrt{m}x+(m-3)=0.$

Dlajakiej wartości parametru $m$ kwadrat róznicy pierwiastków rzeczy-

wistych tego równaniajest najwiekszy. Podač $\mathrm{t}\mathrm{e}$ najwieksza wartośč.

19.6. Stosujac zasade indukcji matematycznej, udowodnič, $\dot{\mathrm{z}}\mathrm{e}$ dla $\mathrm{k}\mathrm{a}\dot{\mathrm{z}}$ dego

$n\geq 2$ liczba $2^{2^{n}}-6$ jest podzielna przez 10.

19.7. Rozwiazač uklad równań

$\left\{\begin{array}{l}
\mathrm{t}\mathrm{g}x+\mathrm{t}\mathrm{g}y=4\\
\cos(x+y)+\cos(x-y)=\frac{1}{2}
\end{array}\right.$

dla $x, y\in[-\pi,\pi].$

19.8. Równoramienny trójkat prostokatny $ABC$ zgieto wzdluz środkowej

$CD$ wychodzacej $\mathrm{z}$ wierzcholka kata prostego $C\mathrm{t}\mathrm{a}\mathrm{k}, \dot{\mathrm{z}}\mathrm{e}$ obie polowy

tego trójkata utworzyly $\mathrm{k}\mathrm{a}\mathrm{t}60^{\circ}$ Obliczyč sinusy wszystkich katów

dwuściennych otrzymanego czworościanu ABCD. Rozwiazanie zilus-

trowač odpowiednimi rysunkami, a obliczenia uzasadnič.





34

Praca kontrolna nr 6

20.1. Wyznaczyč wszystkie wartości parametru rzeczywistego

których prosta $x = m$ jest osia symetrii wykresu

$p(x)=(m^{2}-2m)x^{2}-(2m-4)x+3$. Sporzadzič rysunek.

m, dla

funkcji

20.2. $\mathrm{Z}$ kuli $0$ promieniu $R$ wycieto ósma cześč trzema wzajemnie prosto-

padlymi plaszczyznami przechodzacymi przez środek kuli. $\mathrm{W}$ tak

otrzymana bryle wpisano inna kule. Obliczyč stosunek pola powierz-

chni tej kuli do pola powierzchni bryly.

20.3. $\mathrm{W}$ trzech pustych urnach $K, L, M$ rozmieszczamy losowo 4 rózne

kule. Obliczyč prawdopodobieństwo tego, $\dot{\mathrm{z}}\mathrm{e}\dot{\mathrm{z}}$ adna $\mathrm{z}$ urn $K\mathrm{i}L$ nie

pozostanie pusta.

20.4. Dane sa punkty $A(2,6), B(-2,6) \mathrm{i} C(0,0)$. Wyznaczyč równanie

linii zawierajacej wszystkie punkty trójkata $ABC$, dla których suma

kwadratów ich odleglości od trzech boków jest stala $\mathrm{i}$ wynosi 9. Spo-

rzadzič rysunek.

20.5. Narysowač dokladny wykres $\mathrm{i}$ napisač równania asymptot funkcji

$f(x)=\displaystyle \frac{(x+1)^{2}-1}{x|x-1|}$

nie badajac jej przebiegu.

20.6. Rozwiazač nierównośč

$|x|^{2x-1}\displaystyle \leq\frac{1}{x^{2}}.$

20.7. Styczna do wykresu funkcji $f(x) = \sqrt{3+x}+\sqrt{3-x} \mathrm{w}$ punkcie

$A(x_{0},f(x_{0}))$ przecina oś $Ox \mathrm{w}$ punkcie $P$, a oś $Oy \mathrm{w}$ punkcie $Q$

$\mathrm{t}\mathrm{a}\mathrm{k}, \dot{\mathrm{z}}\mathrm{e} |OP|=|OQ|$. Wyznaczyč $x_{0}.$

20.8. Trójkat równoboczny $0$ boku $\alpha$ podzielono prosta $l$ na dwie figury,

których stosunek pól jest równy 1 : 5. Prosta ta przecina bok $AC$

$\mathrm{w}$ punkcie $D$ pod katem $15^{\circ}$, a bok AB $\mathrm{w}$ punkcie $E$. Wykazač, $\dot{\mathrm{z}}\mathrm{e}$

$|AD|+|AE|=\alpha.$





35

Praca kontrolna nr 7

21.1. Sześcian $0$ krawedzi 3 cm ma objetośč taka sama jak dwa sześciany,

których suma obydwu krawedzi wynosi 4 cm. $\mathrm{O}$ ile $\mathrm{c}\mathrm{m}^{2}$ pole powierz-

chni wiekszego sześcianujest mniejsze od sumy pól powierzchni dwóch

mniejszych sześcianów.

21.2. Obliczyč tangens kata utworzonego przez przekatne czworokata

$0$ wierzcholkach $A(1,1), B(2,0), C(2,4), D(0,6)$. Rozwiazanie zilu-

strowač rysunkiem.

21.3. $\mathrm{W}$ trójkat prostokatny wpisano okrag, a $\mathrm{w}$ okrag ten wpisano podobny

trójkat prostokatny. Wyznaczyč cosinusy katów ostrych trójkata,

jeśli wiadomo, $\dot{\mathrm{z}}\mathrm{e}$ stosunek pól obu trójkatów wynosi 9.

21.4. Wykazač, $\dot{\mathrm{z}}\mathrm{e}$ ciag

granice.

$\alpha_{n}=\sqrt{n(n+1)}-n$ jest rosnacy. Obliczyč jego

21.5. Rozwiazač nierównośč

$2\displaystyle \cos^{2}\frac{x}{4}>1.$

21.6. Rozwiazač równanie

$\displaystyle \log_{2}(1-x)+\log_{4}(x+4)=\log_{4}(x^{3}-x^{2}-3x+5)+\frac{1}{2}.$

Nie wyznaczač dziedziny równania $\mathrm{w}$ sposób jawny.

21.7. $\mathrm{W}$ kule $0$ promieniu $R$ wpisano stozek $0$ najwiekszej objetości. Wyz-

naczyč promień podstawy $r \mathrm{i}$ wysokośč $h$ tego stozka. Sporzadzič

rysunek.

21.8. Znalez$\acute{}$č równania wszystkich prostych, które sa styczne jednocześnie

do krzywych

$y=-x^{2},y=x^{2}-8x+18.$

Sporzadzič rysunek.





Edycja

XXXII

2002/2003





39

Praca kontrolna nr l

22.1. Narysowač wykres funkcji $y = 4+2|x| -x^{2}$ Na podstawie tego

wykresu określič liczbe rozwiazań równania $4 + 2|x| - x^{2} = p$

$\mathrm{w}$ zalezności od parametru rzeczywistego $p.$

22.2. Pompa napelniajaca pusty basen $\mathrm{w}$ pierwszej minucie pracy miala

wydajnośč 0,2 $\mathrm{m}^{3}/\mathrm{s}$, a $\mathrm{w}\mathrm{k}\mathrm{a}\dot{\mathrm{z}}$ dej kolejnej minuciejej wydajnośč zwiek-

szano $0 0,01 \mathrm{m}^{3}/\mathrm{s}$. Polowa basenu zostala napelniona po $2n$ mi-

nutach, a caly basen po kolejnych $n$ minutach, gdzie $n$ jest liczba

naturalna. Wyznaczyč czas napelniania basenu oraz jego pojemnośč.

22.3. Stozek ścietyjest opisany na kuli $0$ promieniu $r=2$ cm. Objetośč kuli

stanowi 25\% objetości stozka. Wyznaczyč średnice podstaw $\mathrm{i}$ dlu-

gośč tworzacej tego stozka.

22.4. $\mathrm{W}$ trójkacie $ABC$ dane sa promień okregu opisanego $R, \mathrm{k}\mathrm{a}\mathrm{t}\angle A=\alpha$

oraz $|AB|=\displaystyle \frac{8}{5}R$. Obliczyč pole tego trójkata.

22.5. Rozwiazač nierównośč

$(\sqrt{x})^{\log_{8}x}\geq\sqrt[3]{16x}.$

22.6. $\mathrm{W}$ czworokacie ABCD odcinki AB $\mathrm{i} BD$ sa prostopadle,

$|AD| = 2|AB| = \alpha$ oraz $\vec{AC}= \displaystyle \frac{5}{3} \vec{AB} +\displaystyle \frac{1}{3} \vec{AD}$. Wyznaczyč cosi-

nus kata $\angle BCD = \alpha$ oraz obwód czworokata ABCD. Sporzadzič

rysunek.

22.7. Rozwiazač równanie

$\displaystyle \frac{1}{\sin x}+\frac{1}{\cos x}=\sqrt{8}.$

22.8. Wyznaczyč równanie prostej stycznej do wykresu funkcji $y = \displaystyle \frac{1}{x^{2}}$

$\mathrm{w}$ punkcie $P(x_{0},y_{0}), x_{0}>0$, takim, $\dot{\mathrm{z}}$ eby odcinek tej stycznej zawarty

$\mathrm{w}$ pierwszej čwiartce ukladu wspólrzednych byl najkrótszy. Rozwia-

zanie zilustrowač odpowiednim wykresem.





Przedmowa

{\it Zbiór obejmuje zadania Korespondencyjnego Kursu} $\mathrm{z}$ {\it Matematyki} $\mathrm{z}$ lat 1999-

2004. Kurs tenjest prowadzony przez Instytut Matematyki Politechniki Wroclaw-

{\it skiej. Jest kontynuacja Korespondencyjnego Kursu Przygotowawczego} $\mathrm{z}$ {\it Mate}-

{\it matyki, który} $\mathrm{w}$ latach $1972-1999$ byl prowadzony wspólnie $\mathrm{z}$ Instytutem Mate-

matyki Politechniki Warszawskiej.

Opracowujac niniejszy zbiór, autor pragnal ulatwič szerokiemu gronu matu-

rzystów $\mathrm{i}$ kandydatów na studia dostęp do materialów kursu ujętych $\mathrm{w}$ wygodna,

zwarta formę $\mathrm{i}\mathrm{w}$ ten sposób pomóc im $\mathrm{w}$ lepszym opanowaniu wiadomości $\mathrm{z}$

matematyki $\mathrm{w}$ zakresie szkoly średniej oraz dač jeszcze jedna okazję do powtó-

rzenia materialu.

Większośč zadań jest oryginalna, częśč pochodzi $\mathrm{z}$ egzaminów wstępnych na

Politechnikę Wroclawska $\mathrm{z}$ ostatnich 201at, a ty1ko niewie1ka 1iczba $\mathrm{z}$ innych

z$\acute{}$ródel ($\mathrm{w}$ tym powtórzenia zadań $\mathrm{z}$ lat ubieglych). Dla udogodnienia samodzielnej

pracy $\mathrm{i}$ zachęcenia do korzystania $\mathrm{z}$ tego zbioru, podano odpowiedzi, a $\mathrm{w}$ oddziel-

nym rozdziale takze wskazówki do wszystkich zadań. $\mathrm{W}$ końcowej części zbioru

przedstawiono 12 przyk1adowych rozwiazań róznorodnych zadań wybranych ze

wszystkich dzialów. Celem ich zamieszczenia jest pokazanie najwazniejszych

metod $\mathrm{i}$ narzędzi $\mathrm{u}\dot{\mathrm{z}}$ ywanych do rozwiazywania zadań. Moga więc sluzyč jako

wzorzec do przygotowania innych rozwiazań. Zastosowano dwuczlonowa nume-

rację zadań uwzględniajaca chronologię kursu. Pierwsza liczba jest kolejnym

numerem pracy kontrolnej ($\mathrm{z}$ okresu $1999-2004$), a druga podaje numer tematu

$\mathrm{w}$ danym zestawie. Dolaczony indeks tematyczny pozwala na szybkie wyszukanie

zadań $\mathrm{z}$ dowolnie wybranego dzialu matematyki.

Ze zbioru moga korzystač zarówno osoby zdajace maturę na poziomie podsta-

wowym, jak $\mathrm{i}$ rozszerzonym. Poczynajac od XXXI edycji, $\mathrm{t}\mathrm{j}$. od pracy kontrolnej

$0$ numerze 15, pierwsze cztery zadania $\mathrm{w}\mathrm{k}\mathrm{a}\dot{\mathrm{z}}$ dym zestawie odpowiadaja zakresowi

podstawowemu, a cztery następne zakresowi rozszerzonemu. Podzial ten dotyczy

$\mathrm{w}$ przyblizeniu takze wcześniejszych prac kontrolnych.

Kurs ma swoja strong internetowa, na której $\mathrm{m}\mathrm{o}\dot{\mathrm{z}}$ na znalez$\acute{}$č zarówno mate-

rialy biezace, jak $\mathrm{i}$ archiwum zawierajace tematy $\mathrm{z}$ lat ubieglych. Dostęp do niej

$\mathrm{m}\mathrm{o}\dot{\mathrm{z}}$ na uzyskač przez strong glówna Politechniki Wroclawskiej: $\mathrm{w}\mathrm{w}\mathrm{w}$. pwr. wroc. pl.

Następnie nalezy wybrač dzial Rekrutacja $\mathrm{i}\mathrm{w}$ nim wyszukač pozycję Korespon-

{\it dencyjny Kurs} $\mathrm{z}$ {\it Matematyki}.

Serdecznie dziękuję Recenzentom Docentowi Zbigniewowi Romanowiczowi

oraz Doktorowi Rościslawowi Rabczukowi za cenne uwagi, które pozwolily usunač

usterki $\mathrm{i}$ ulepszyč pierwotna wersję ksia $\dot{\mathrm{Z}}$ ki.

Wroclaw, marzec 2005

{\it Tadeusz Inglot}





40

Praca kontrolna nr 2

23.1. Czy liczby róznych,,slów'', jakie $\mathrm{m}\mathrm{o}\dot{\mathrm{z}}$ na utworzyč zmieniajac kole-

jnośč liter $\mathrm{w},$,slowach'' TANATAN $\mathrm{i}$ AKABARA, sa takie same?

Uzasadnič odpowied $\acute{\mathrm{z}}$. Przez,,slowo'' rozumiemy tutaj dowolny ciag

liter.

23.2. Reszta $\mathrm{z}$ dzielenia wielomianu $x^{3}+px^{2}-x+q$ przez trójmian $(x+2)^{2}$

wynosi $(-x+1)$. Wyznaczyč pierwiastki tego wielomianu.
\begin{center}
\includegraphics[width=33.324mm,height=23.724mm]{./KursMatematyki_PolitechnikaWroclawska_1999_2004_page30_images/image001.eps}
\end{center}
A $\mathrm{B}$

$A, B$, oraz $\mathrm{z}$ odcinka AB $0$ dugości $\alpha.$

Obliczyč promień okregu stycznego do obu

uków oraz do odcinka $AB.$

23.3. Figura na rysunku sklada $\mathrm{s}\mathrm{i}\mathrm{e}\mathrm{Z}$ luków $BC, CA$ okregów $0$ promie-

C niu $\alpha \mathrm{i}$ środkach odpowiednio $\mathrm{w}$ punktach

23.4. Podstawa pryzmy przedstawionej na rysunku jest prostokat

ABCD, którego bok $AB$ ma

D $\mathrm{C}$ gdzie $\alpha > b$. Wszystkie ściany

$b$ boczne pryzmy sa nachylone pod
\begin{center}
\includegraphics[width=60.552mm,height=30.480mm]{./KursMatematyki_PolitechnikaWroclawska_1999_2004_page30_images/image002.eps}
\end{center}
{\it K} $\mathrm{L}$

A $\alpha \mathrm{B}$

$\mathrm{d}$ ugośč $\alpha$, a bok $BC \mathrm{d}$ ugośč $b,$

katem $\alpha$ do $\mathrm{p}$ aszczyzny podstawy.

Obliczyč objetośč tej pryzmy.

23.5. Rozwiazač nierównośč

$- x2<\sqrt{}$5-{\it x}2.

Rozwiazanie zilustrowač wykresami funkcji wystepujacych po obu

stronach nierówności. Zaznaczyč na rysunku otrzymany zbiór rozwia-

zań.

23.6. Ciag $(\alpha_{n})$ jest określony rekurencyjnie warunkami $\alpha_{1} =$ 4,

$\alpha_{n+1} = 1+2\sqrt{\alpha_{n}}, n \geq 1$. Stosujac zasade indukcji matematycznej,

wykazač, $\dot{\mathrm{z}}\mathrm{e}$ ciag $(\alpha_{n})$ jest rosnacy oraz $4\leq\alpha_{n}<6$ dla $n\geq 1.$

23.7. Na krzywej $0$ równaniu $y=\sqrt{x}$ znalez$\acute{}$č punkt $\mathrm{l}\mathrm{e}\dot{\mathrm{z}}\mathrm{a}\mathrm{c}\mathrm{y}$ najblizej punktu

$P(0,3)$. Sporzadzič rysunek.

23.8. Wykazač, $\dot{\mathrm{z}}\mathrm{e}$ dla $\mathrm{k}\mathrm{a}\dot{\mathrm{z}}$ dej wartości parametru $\alpha \in \mathrm{R}$ równanie

kwadratowe $3x^{2}+4x\sin\alpha-\cos 2\alpha=0$ ma dwa rózne pierwiastki

rzeczywiste. Wyznaczyč te wartości parametru $\alpha$, dla których oba

pierwiastki $\mathrm{l}\mathrm{e}\dot{\mathrm{z}}$ a $\mathrm{w}$ przedziale $(0,1).$





41

Praca kontrolna nr 3

24.1. Suma wyrazów nieskończonego ciagu geometrycznego zmniejszy $\mathrm{s}\mathrm{i}\mathrm{e}$

$0$ 25\%, jeśli wykreślimy $\mathrm{z}$ niej skladniki $0$ numerach parzystych niepo-

dzielnych przez 4. Ob1iczyč sume wszystkich wyrazów tego ciagu,

wiedzac, $\dot{\mathrm{z}}\mathrm{e}$ jego drugi wyraz wynosi l.

24.2. $\mathrm{Z}$ kompletu 28 kości do gry $\mathrm{w}$ domino wylosowano dwie kości (bez

zwracania). Obliczyč prawdopodobieństwo tego, $\dot{\mathrm{z}}\mathrm{e}$ kości $pasujq_{f}$ do

siebie, $\mathrm{t}\mathrm{z}\mathrm{n}$. najednym $\mathrm{z}$ pól obu kości wystepuje ta sama liczba oczek.

24.3. Rozwiazač uklad równań

$\left\{\begin{array}{l}
x+2y=3\\
5x+my=m
\end{array}\right.$

$\mathrm{w}$ zalezności od parametru rzeczywistego $m$. Wyznaczyč $\mathrm{i}$ narysowač

zbiór, jaki tworza rozwiazania $(x(m),y(m))$ tego ukladu, gdy $m$

przebiega zbiór liczb rzeczywistych.

24.4. $\mathrm{W}$ graniastoslupie prawidlowym sześciokatnym krawed $\acute{\mathrm{z}}$ dolnej pod-

stawy $AB$ widač ze środka górnej podstawy $P$ pod katem $\alpha$. Wyz-

naczyč cosinus kata utworzonego przez plaszczyzne podstawy $\mathrm{i}$ plasz-

czyzne zawierajaca krawed $\acute{\mathrm{z}}$ AB oraz przeciwlegla do niej $\mathrm{k}\mathrm{r}\mathrm{a}\mathrm{w}\mathrm{e}\mathrm{d}\acute{\mathrm{z}}$

$D'E'$ górnej podstawy. Uzasadnič odpowiednio obliczenia.

24.5. Rozwiazač nierównośč $-1\displaystyle \leq\frac{2^{x+1/2}}{4^{x}-4}\leq 1.$

24.6. Bez $\mathrm{u}\dot{\mathrm{z}}$ ycia tablic wykazač, $\dot{\mathrm{z}}\mathrm{e}$ tg $82^{\circ}30'$ -tg $7^{\circ}30'=4+2\sqrt{3}.$

24.7. Napisač równanie prostej $k$ stycznej $\mathrm{w}$ punkcie $P(2,3)$ do okregu

$x^{2}+y^{2}-2x-2y-3=0$ Nastepnie wyznaczyč równania wszystkich

prostych stycznych do tego okregu, które tworza $\mathrm{z}$ prosta $k\mathrm{k}\mathrm{a}\mathrm{t}45^{\circ}$

24.8. Dobrač parametry $\alpha>0 \mathrm{i} b\in R \mathrm{t}\mathrm{a}\mathrm{k}$, aby funkcja

$f(x)=$

dla $x\leq\alpha,$

dla $x>\alpha,$

byla ciagla $\mathrm{i}$ miala pochodna $\mathrm{w}$ punkcie $x =\alpha$. Sporzadzič wykres

funkcji $f(x)$ oraz stycznej do wykresu $\mathrm{w}$ punkcie $P(\alpha,f(\alpha))$ bez

szczególowego badania jej przebiegu.





42

Praca kontrolna nr 4

25.1. Dla jakich wartości parametru rzeczywistego

$x+3=-(tx+1)^{2}$ ma dokladnie jeden pierwiastek.

t równanie

25.2. Czworościan foremny $0$ krawedzi $\alpha$ przecieto plaszczyzna równolegla

do dwóch przeciwleglych krawedzi. Wyrazič pole otrzymanego prze-

krojujako funkcje dlugości odcinka wyznaczonego przez ten przekrój

najednej $\mathrm{z}$ pozostalych krawedzi. Uzasadnič postepowanie. Przedsta-

wič znaleziona funkcje na wykresie $\mathrm{i}$ podač jej najwieksza wartośč.

25.3. Zaznaczyč na wykresie zbiór punktów $(x,y)$ plaszczyzny spelniaja-

cych warunek $\log_{xy}|y|\geq 1.$

25.4. Wyznaczyč równanie linii utworzonej przez wszystkie punkty plasz-

czyzny, których odleglośč od okregu $x^{2}+y^{2}=81$ jest $01$ mniejsza

$\mathrm{n}\mathrm{i}\dot{\mathrm{z}}$ od punktu $P(8,0)$. Sporzadzič rysunek.

25.5. Na dziesiatym pietrze pewnego bloku mieszkaja Kowalscy $\mathrm{i}$ Nowa-

kowie. Kowalscy maja dwóch synów $\mathrm{i}$ dwie córki, a Nowakowie jed-

nego syna $\mathrm{i}$ dwie córki. Postanowili oni wybrač mlodziezowego przed-

stawiciela swojego pietra. $\mathrm{W}$ tym celu Kowalscy wybrali losowo jedno

ze swoich dzieci $\mathrm{i}$ Nowakowie jedno ze swoich. Nastepnie spośród tej

dwójki wylosowano jedna osobe. Obliczyč prawdopodobieństwo, $\dot{\mathrm{z}}\mathrm{e}$

przedstawicielem zostal chlopiec.

25.6. Uzasadnič prawdziwośč nierówności $n+\displaystyle \frac{1}{2}\geq\sqrt{n(n+1)}, n\geq 1.$ {\it Ko}-

rzystajac $\mathrm{z}$ niej oraz $\mathrm{z}$ zasady indukcji matematycznej, udowodnič,

$\dot{\mathrm{z}}\mathrm{e}$

$\displaystyle \left(\begin{array}{l}
2n\\
n
\end{array}\right)\geq\frac{4^{n}}{2\sqrt{n}}$

dla $\mathrm{k}\mathrm{a}\dot{\mathrm{z}}$ dej liczby naturalnej $n.$

25.7. Zbadač przebieg zmienności $\mathrm{i}$ narysowač wykres funkcji

$f(x)=\sqrt{\frac{3x-3}{5-x}}.$

25.8. $\mathrm{W}$ trójkacie $ABC \mathrm{k}\mathrm{a}\mathrm{t} A$ ma miare $\alpha, \mathrm{k}\mathrm{a}\mathrm{t} B$ miare $2\alpha,$

$\mathrm{a}|BC|=\alpha$. Oznaczmy kolejno przez $A_{1}$ punkt na boku $AC$ taki, $\dot{\mathrm{z}}\mathrm{e}$

BAl jest dwusieczna kata $B;B_{1}$ punkt na boku $BC$ taki, $\dot{\mathrm{z}}\mathrm{e}$ AlBl jest

dwusieczna kata $A_{1}, \mathrm{i}\mathrm{t}\mathrm{d}$. Wyznaczyč dlugośč nieskończonej lamanej

$ABA_{1}B_{1}A_{2}$





43

Praca kontrolna nr 5

26.1. Jakiej dlugości powinien byč pas transmisyjny, aby $\mathrm{m}\mathrm{o}\dot{\mathrm{z}}$ na go bylo

$\mathrm{u}\dot{\mathrm{z}}$ yč do polaczenia dwóch kól $0$ promieniach 20 cm $\mathrm{i}5$ cm, jeśli od-

leglośč środków tych kól wynosi 30 cm?

26.2. Umowa określa wynagrodzenie na kwote 4000 $\mathrm{z}l$. Skladka na ubez-

pieczenie spoleczne wynosi 18,7\% tej kwoty, a sk1adka na Kase Cho-

rych 7,75\% kwoty pozosta1ej po od1iczeniu sk1adki na ubezpiecze-

nie spoleczne. $\mathrm{W}$ celu obliczenia podatku nalezy od 80\% wyjściowej

kwoty umowy odjač skladke na ubezpieczenie spoleczne $\mathrm{i}$ wyznaczyč

19\% pozostalej sumy. Podatek jest róznica tak otrzymanej liczby

$\mathrm{i}$ skladki na Kase Chorych. Ile wynosi podatek?.

26.3. Przez punkt $P(1,3)$ poprowadzič prosta $l$, tak aby odcinek tej prostej

zawarty miedzy prostymi $x-y+3=0\mathrm{i}x+2y-12=0$ dzielil $\mathrm{s}\mathrm{i}\mathrm{e}$

$\mathrm{w}$ punkcie $P$ na polowy. Wyznaczyč równanie ogólne prostej $l\mathrm{i}$ obli-

czyč pole trójkata, jaki prosta $l$ tworzy $\mathrm{z}$ danymi prostymi.

26.4. Podstawa czworościanu ABCD jest trójkat prostokatny $ABC\mathrm{o}$ kacie

ostrym $\alpha \mathrm{i}$ promieniu okregu wpisanego $r$. Spodek wysokości opusz-

czonej $\mathrm{z}$ wierzcholka $D \mathrm{l}\mathrm{e}\dot{\mathrm{z}}\mathrm{y}\mathrm{w}$ punkcie przeciecia $\mathrm{s}\mathrm{i}\mathrm{e}$ dwusiecznych

trójkata $ABC$, a ściany boczne wychodzace $\mathrm{z}$ wierzcholka kata pros-

tego podstawy tworza $\mathrm{k}\mathrm{a}\mathrm{t}\beta$. Obliczyč objetośč tego ostroslupa.

26.5. Sporzadzič wykres funkcji $f(x)=\log_{4}(2|x|-4)^{2}$ Odczytač $\mathrm{z}$ wykre-

su wszystkie ekstrema lokalne tej funkcji.

26.6. Rozwiazač równanie $\displaystyle \cos 2x+\frac{\mathrm{t}\mathrm{g}x}{\sqrt{3}+\mathrm{t}\mathrm{g}x}=0.$

26.7. Dla jakich wartości parametru $\alpha \in \mathrm{R} \mathrm{m}\mathrm{o}\dot{\mathrm{z}}$ na określič funkcje

$g(x) = f(f(x))$, gdzie $f(x) = \displaystyle \frac{x^{2}}{\alpha x-1}$. Napisač wzór funkcji $g(x).$

Wyznaczyč asymptoty funkcji $g(x)$ dla najwiekszej $\mathrm{m}\mathrm{o}\dot{\mathrm{z}}$ liwej calko-

witej wartości parametru $\alpha.$

26.8. Odcinek $0$ końcach $A(0,3), B(2,y), y \in [0$, 3$]$, obraca $\mathrm{s}\mathrm{i}\mathrm{e}$ wokól

osi $Ox$. Wyznaczyč pole powierzchni bocznej powstalej bryly jako

funkcje $y \mathrm{i}$ znalez$\acute{}$č najmniejsza wartośč tego pola. Sporzadzič ry-

sunek.





44

Praca kontrolna nr 6

27.1. Znalez$\acute{}$č wszystkie wartości parametru rzeczywistego $p$, dla których

równanie $\sqrt{x+8p}=\sqrt{x}+2p$ ma rozwiazanie.

27.2. Obrazem okregu $K\mathrm{w}$ jednokladności $0$ środku $S(0,1)\mathrm{i}$ skali $k=-3$

jest okrag $K_{1}$, natomiast obrazem $K_{1} \mathrm{w}$ symetrii wzgledem prostej

$0$ równaniu $2x+y+3 = 0$ jest okrag $0$ tym samym środku co

okrag $K$. Wyznaczyč równanie okregu $K$, jeśli wiadomo, $\dot{\mathrm{z}}\mathrm{e}$ okregi

$K\mathrm{i}K_{1}$ sa styczne zewnetrznie.

27.3. $\mathrm{W}$ trapezie równoramiennym dane sa promień okregu opisanego $r,$

$\mathrm{k}\mathrm{a}\mathrm{t}$ ostry przy podstawie $\alpha$ oraz suma dlugości obu podstaw $d$. Obli-

czyč dlugośč ramienia tego trapezu. Zbadač warunki rozwiazalności

zadania. Sporzadzič rysunek dla $\alpha=60^{\circ}, d=\displaystyle \frac{5}{2}r.$

27.4. $\mathrm{W}$ ostroslupie prawidlowym czworokatnym $\mathrm{k}\mathrm{a}\mathrm{t}$ plaski ściany bocznej

przy wierzcholku wynosi $ 2\beta$. Przez wierzcholek $A$ podstawy oraz

środek przeciwleglej krawedzi bocznej poprowadzono plaszczyzne

równolegla do przekatnej podstawy wyznaczajaca przekrój plaski

ostroslupa. Obliczyč objetośč ostroslupa, wiedzac, $\dot{\mathrm{z}}\mathrm{e}$ pole przekroju

wynosi $S.$

27.5. Obliczyč granice

$\displaystyle \lim_{n\rightarrow\infty}\frac{n-\sqrt[3]{n^{3}+n^{\alpha}}}{\sqrt[5]{n^{3}}},$

jeśli $\alpha$ jest najmniejszym

$2\cos\alpha=-\sqrt{3}.$

dodatnim pierwiastkiem

równania

27.6. Rozwiazač nierównośč

$2^{1+2\log_{2}\cos x}-\displaystyle \frac{3}{4}\geq 9^{0}$' $5+\log_{3}\sin x$

27.7. Wylosowano, ze zwracaniem, 4 liczby czterocyfrowe (cyfra tysiecy

nie $\mathrm{m}\mathrm{o}\dot{\mathrm{z}}\mathrm{e}$ byč zerem!). Obliczyč prawdopodobieństwo tego, $\dot{\mathrm{z}}\mathrm{e}$ co

najmniej dwie $\mathrm{z}$ tych liczb czytane od strony lewej do prawej lub od

strony prawej do lewej beda podzielne przez 4.

27.8. Zaznaczyč na rysunku zbiór punktów $(x,y)$ plaszczyzny określony

warunkami $|x-3y| < 2$ oraz $y^{3} \leq x$. Obliczyč tangens kata, pod

którym przecinaja $\mathrm{s}\mathrm{i}\mathrm{e}$ linie tworzace brzeg tego zbioru.





45

Praca kontrolna nr 7

28.1. Dwa punkty poruszaja $\mathrm{s}\mathrm{i}\mathrm{e}$ ruchem jednostajnym po okregu $\mathrm{w}$ tym

samym kierunku, przy czym jeden $\mathrm{z}$ nich wyprzedza drugi co 44

sekundy. $\mathrm{J}\mathrm{e}\dot{\mathrm{z}}$ eli zmienič kierunek ruchu jednego $\mathrm{z}$ tych punktów na

przeciwny, to beda $\mathrm{s}\mathrm{i}\mathrm{e}$ one spotykač co 8 sekund. Ob1iczyč stosunek

predkości tych punktów.

28.2. Dlajakich wartości parametru $p$ nierównośč

$\displaystyle \frac{2px^{2}+2px+1}{x^{2}+x+2-p^{2}}\geq 2$

jest spelniona dla $\mathrm{k}\mathrm{a}\dot{\mathrm{z}}$ dej liczby rzeczywistej $x$?

28.3. $\mathrm{W}$ równolegloboku dane sa $\mathrm{k}\mathrm{a}\mathrm{t}$ ostry $\alpha$, dluzsza przekatna $d$ oraz

róznica boków $r$. Obliczyč pole równolegloboku.

28.4. Naczynie $\mathrm{w}$ ksztalcie pólkuli $0$ promieniu $R$ ma trzy nózki $\mathrm{w}$ ksztalcie

kulek $0$ promieniu $r, 4r < R$, przymocowanych do naczynia $\mathrm{w}$ ten

sposób, $\dot{\mathrm{z}}\mathrm{e}$ ich środki tworza trójkat równoboczny, a naczynie posta-

wione na plaskiej powierzchni dotyka $\mathrm{j}\mathrm{a}$ wjednym punkcie. Obliczyč

wzajemna odleglośč punktów przymocowania kulek. Sporzadzič od-

powiednie rysunki.

28.5. Za pomoca metod rachunku rózniczkowego określič liczbe rozwiazań

równania $2x^{3}+1=6|x|-6x^{2}$

28.6. Bez stosowania zasady indukcji matematycznej wykazač, $\dot{\mathrm{z}}\mathrm{e} \displaystyle \frac{n^{n}-1}{n-1}$

jest nieparzysta liczba naturalna dla wszystkich $n\geq 2.$

28.7. Rozwiazač równanie

$\displaystyle \frac{8}{3}(\sin^{2}x+\sin^{4}x+\ldots)=4-2\cos x+3\cos^{2}x-\frac{9}{2}\cos^{3}x+$

28.8. Rozwazmy rodzine prostych normalnych do paraboli $0$ równaniu

$2y = x^{2}$ (tj. prostopadlych do stycznych $\mathrm{w}$ punktach styczności).

Znalez$\acute{}$č równanie krzywej utworzonej ze środków odcinków tych nor-

malnych zawartych miedzy osia rzednych $\mathrm{i}$ wyznaczajacymi je punk-

tami paraboli. Sporzadzič rysunek.





Edycja

XXXIII

2003/2004





49

Praca kontrolna nr l

29.1. Podstawa trójkata równoramiennego jest odcinek AB $0$ końcach

$A(-1,3), B(1,-1)$, a wierzcholek $C$ tego trójkata $\mathrm{l}\mathrm{e}\dot{\mathrm{z}}\mathrm{y}$ na prostej

$l\mathrm{o}$ równaniu $3x-y-14=0$. Obliczyč pole trójkata $ABC.$

29.2. Pewna liczba sześciocyfrowa zaczyna $\mathrm{s}\mathrm{i}\mathrm{e}$ ($\mathrm{z}$ lewej strony) cyfra 3. Jeś1i

cyfre $\mathrm{t}\mathrm{e}$ przestawimy $\mathrm{z}$ pierwszej pozycji na ostatnia, to otrzymamy

liczbe stanowiaca 25\% 1iczby pierwotnej. Zna1ez$\acute{}$č $\mathrm{t}\mathrm{e}$ liczbe.

29.3. $\mathrm{W}$ trapezie opisanym na okregu katy ostre przy podstawie maja

miary $\alpha \mathrm{i}2\alpha$, a dlugośč krótszego ramienia wynosi $c$. Obliczyč dlugośč

krótszej podstawy tego trapezu. Wynik przedstawič $\mathrm{w}$ najprostszej

postaci.

29.4. Rozwiazač nierównośč

$\displaystyle \frac{1}{x^{2}-x-2}\leq\frac{1}{|x|}.$

29.5. Zaznaczyč na plaszczy $\acute{\mathrm{z}}\mathrm{n}\mathrm{i}\mathrm{e}$ zbiór wszystkich punktów $(x,y)$ spelnia-

jacych nierównośč $\log_{x}(1+(y-1)^{3})\leq 1.$

29.6. Rozwiazač równanie $\sin^{2}3x-\sin^{2}2x=\sin^{2}x.$

29.7. Wysokośč ostroslupa prawidlowego czworokatnegojest trzy razy dluz-

sza od promienia kuli wpisanej $\mathrm{w}$ ten ostroslup. Obliczyč cosinus kata

miedzy sasiednimi ścianami bocznymi tego ostroslupa.

29.8. Dany jest nieskończony ciag geometryczny

$x+1,-x^{2}(x+1),x^{4}(x+1),$

Wyznaczyč najmniejsza $\mathrm{i}$ najwieksza wartośč funkcji $S(x)$ bedacej

suma wszystkich wyrazów tego ciagu.





50

Praca kontrolna nr 2

30.1. Trójkat prostokatny, obracajac $\mathrm{s}\mathrm{i}\mathrm{e}$ wokól jednej $\mathrm{i}$ drugiej przypros-

tokatnej tworzy bryly $0$ objetościach odpowiednio $V_{1}\mathrm{i}V_{2}$. Obliczyč

objetośč bryly powstalej $\mathrm{z}$ obrotu tego trójkata wokól dwusiecznej

kata prostego.

30.2. Czy $\mathrm{m}\mathrm{o}\dot{\mathrm{z}}$ na sume $42000$ zlotych podzielič na pewna liczbe nagród,

tak aby kwoty tych nagród wyrazaly $\mathrm{s}\mathrm{i}\mathrm{e}\mathrm{w}$ pelnych setkach zlotych,

tworzyly ciag arytmetyczny oraz $\dot{\mathrm{z}}$ eby najwyzsza nagroda wynosila

13000 $\mathrm{z}l$? Jeśli $\mathrm{t}\mathrm{a}\mathrm{k}$, to podač liczbe $\mathrm{i}$ wysokości tych nagród.

30.3 Dane sa okregi $0$ równaniach $(x-1)^{2} + (y-1)^{2} =$ l oraz

$(x-2)^{2}+(y-1)^{2}=16$. Wyznaczyč równania wszystkich okregów

stycznych równocześnie do obu danych okregów oraz do osi $Oy.$

Sporzadzič rysunek.

30.4. $\mathrm{W}$ równolegloboku $\mathrm{k}\mathrm{a}\mathrm{t}$ ostry miedzy przekatnymi ma miare $\beta$, a sto-

sunek dlugości dluzszej przekatnej do krótszej przekatnej wynosi $k.$

Obliczyč tangens kata ostrego tego równolegloboku.

30.5. Rozwiazač równanie $\sqrt{4x-3}-3=\sqrt{2x-10}.$

30.6. Dobrač liczby calkowite $\alpha, b, \mathrm{t}\mathrm{a}\mathrm{k}$ aby wielomian $6x^{3}-7x^{2}+1$ dzielil

siebez reszty przez trójmian kwadratowy $2x^{2}+\alpha x+b.$

30.7. Rozwiazač nierównośč $|2^{x}-3| \leq 2^{1-x}$ Sporzadzič wykresy funkcji

wystepujacych po obu stronach tej nierówności oraz zaznaczyč na

rysunku zbiór rozwiazań.

30.8. Wyznaczyč przedzialy monotoniczności funkcji

$f(x)=\displaystyle \sin^{2}x+\frac{\sqrt{3}}{2}x,$

$x\in[-\pi,\pi].$





51

Praca kontrolna nr 3

31.1. $\mathrm{Z}$ talii 24 kart do gry wy1osowano 7 kart. Jakie jest prawdopodobień-

stwo otrzymania dokladnie czterech kart wjednym $\mathrm{z}$ czterech kolorów,

$\mathrm{w}$ tym asa, króla $\mathrm{i}$ dame.

31.2. Pewien ostroslup podzielono na trzy cześci dwiema plaszczyznami

równoleglymi do jego podstawy. Pierwsza plaszczyzna $\mathrm{l}\mathrm{e}\dot{\mathrm{z}}\mathrm{y} \mathrm{w}$ od-

leglości $d_{1} = 2$ cm, a druga $\mathrm{w}$ odleglości $d_{2} = 3$ cm od podstawy.

Pola przekrojów ostroslupa tymi plaszczyznami równe sa odpowied-

nio $S_{1}=25\mathrm{c}\mathrm{m}^{2}$ oraz $S_{2}=16\mathrm{c}\mathrm{m}^{2}$ Obliczyč objetośč tego ostroslupa

oraz objetośč najmniejszej cześci.

31.3. Rozwiazač uklad równań

$\left\{\begin{array}{l}
x^{2}+y^{2}=24\\
\frac{2\log x+\log y^{2}}{\log(x+y)}=2.
\end{array}\right.$

31.4. $\mathrm{W}$ trójkacie równoramiennym $ABC$ odleglośč środka okregu wpisane-

go od wierzcholka $C$ wynosi $d$, a podstawe $AB$ widač ze środka okregu

wpisanego pod katem $\alpha$. Obliczyč pole tego trójkata.

31.5. Stosujac zasade indukcji matematycznej, udowodnič prawdziwośč dla

$n\geq 1$ wzoru

$\displaystyle \cos x+\cos 3x++\cos(2n-1)x=\frac{\sin 2nx}{2\sin x},\sin x\neq 0.$

31.6. Wyznaczyč granice ciagu $0$ wyrazie ogólnym

$\displaystyle \alpha_{n}=\frac{\sqrt[6]{4n}}{\sqrt{n}-\sqrt{n+\sqrt[3]{4n^{2}}}},$

$n\geq 1.$

31.7. Dany jest wierzcholek $A(6,1)$ kwadratu. Wyznaczyč pozostale wierz-

cholki tego kwadratu, gdy wierzcholki sasiadujace $\mathrm{z}A\mathrm{l}\mathrm{e}\dot{\mathrm{z}}$ a jeden na

prostej $l$ : $x-2y+1 = 0$, a drugi na prostej $k$ : $x+3y-4 = 0.$

Sporzadzič rysunek.

31.8. Zbadač przebieg zmienności $\mathrm{i}$ narysowač wykres funkcji

$f(x)=\displaystyle \frac{x+1}{\sqrt{x}}.$





Edycja

XXIX

1999/2000





52

Praca kontrolna nr 4

32.1. Statek $\mathrm{z}$ Wroclawia do Szczecina plynie 3 $\mathrm{d}\mathrm{n}\mathrm{i}$, a ze Szczecina do

Wroclawia 5 $\mathrm{d}\mathrm{n}\mathrm{i}$. Jak dlugo $\mathrm{z}$ Wroclawia do Szczecina plynie woda?

32.2. Dla jakich wartości rzeczywistych $x$ liczby $1 +$ log23, $\log_{x}36,$

$\displaystyle \frac{4}{3}$ log86 sa trzema ko1ejnymi wyrazami ciagu geometrycznego.

32.3. Wanna $0$ pojemności 2001 majaca kszta1t po1owy wa1ca (rozcietego

wzdluz osi) $\mathrm{l}\mathrm{e}\dot{\mathrm{z}}\mathrm{y}$ poziomo na ziemi $\mathrm{i}$ zawiera pewna ilośč wody. Do

wanny wlozono belke ($\mathrm{c}\mathrm{i}\dot{\mathrm{z}}\mathrm{s}\mathrm{z}$ od wody) $\mathrm{w}$ ksztalcie walca $0$ średnicy

cztery razy mniejszej $\mathrm{n}\mathrm{i}\dot{\mathrm{z}}$ średnica wanny $\mathrm{i}$ dlugości równej polowie

dlugości wanny. Okazalo $\mathrm{s}\mathrm{i}\mathrm{e}, \dot{\mathrm{z}}\mathrm{e}$ lustro wody styka $\mathrm{s}\mathrm{i}\mathrm{e}\mathrm{Z}$ powierzchnia

belki zanurzonej $\mathrm{w}$ wodzie. Podač, $\mathrm{z}$ dokladnościa do 0,11, i1e wody

znajduje $\mathrm{s}\mathrm{i}\mathrm{e}\mathrm{w}$ wannie?

32.4. Wyznaczyč wszystkie wartości parametru $m$, dla których obydwa

pierwiastki trójmianu kwadratowego $v(x) = x^{2}+mx-m^{2} \mathrm{l}\mathrm{e}\dot{\mathrm{z}}\mathrm{a}$

miedzy pierwiastkami trójmianu $w(x)=x^{2}-(m-1)x-m.$

32.5. Urna A zawiera trzy kule biale $\mathrm{i}$ dwie czarne, a urna $\mathrm{B}$ dwie kule biale

$\mathrm{i}$ trzy czarne. Wylosowano cztery razy jedna kule (ze zwracaniem)

$\mathrm{z}$ urny A oraz jedna kule $\mathrm{z}$ urny B. Obliczyč prawdopodobieństwo

tego, $\dot{\mathrm{z}}\mathrm{e}$ wśród pieciu wylosowanych kul sa co najmniej dwie kule

biale.

32.6. Rozwiazač równanie 2 $\sin 2x+2\cos 2x+\mathrm{t}\mathrm{g}x=3.$

32.7. Danajest funkcja $f(x)=x^{4}-2x^{2}$ Wyznaczyč wszystkie proste sty-

czne do wykresu $\mathrm{t}\mathrm{e}\mathrm{j}$ funkcji zawierajace punkt $P(1,-1)$. Ile punktów

wspólnych $\mathrm{z}$ wykresem $\mathrm{t}\mathrm{e}\mathrm{j}$ funkcji maja wyznaczone styczne? Rozwia-

zanie zilustrowač rysunkiem.

32.8. Podstawa ostroslupa ABCS jest trójkat równoramienny, którego $\mathrm{k}\mathrm{a}\mathrm{t}$

przy wierzcholku $C$ ma miare $\alpha$, a ramie $BC$ ma dlugośč $b$. Spodek

wysokości ostroslupa $\mathrm{l}\mathrm{e}\dot{\mathrm{z}}\mathrm{y}\mathrm{w}$ środku wysokości $CD$ podstawy, a $\mathrm{k}\mathrm{a}\mathrm{t}$

plaski ściany bocznej $ABS$ przy wierzcholku ma miare $\alpha$. Obliczyč

promień kuli opisanej $\mathrm{n}\mathrm{a}\mathrm{t}\mathrm{y}\mathrm{m}$ ostroslupie oraz cosinusy katów nachyle-

nia ścian bocznych do podstawy.





53

Praca kontrolna nr 5

33.1. Piaty wyraz rozwiniecia dwumianu $(\alpha+b)^{18}$, gdzie $\alpha, b > 0,$

jest $0$ 180\% wiekszy od wyrazu trzeciego. $\mathrm{O}$ ile procent wyraz ósmy

tego rozwiniecia jest mniejszy $\mathrm{b}\mathrm{a}\mathrm{d}\acute{\mathrm{z}}$ wiekszy od wyrazu czwartego?

33.2. Wyznaczyč równanie linii utworzonej przez wszystkie punkty

plaszczyzny, dla których stosunek kwadratu odleglości od prostej

$k:x-2y+3=0$ do kwadratu odleglości od prostej $l:3x+y+2=0$

wynosi 2. Sporzadzič rysunek.

33.3. Obwód trójkata $ABC$ wynosi 15 cm, a dwusieczna kata $A$ dzieli bok

przeciwlegly na odcinki dlugości 3 cm oraz 2 cm. Ob1iczyč po1e ko1a

wpisanego $\mathrm{w}$ ten trójkat.

33.4. Czastka startuje $\mathrm{z}$ poczatku ukladu wspólrzednych $\mathrm{i}$ porusza $\mathrm{s}\mathrm{i}\mathrm{e}$ ze

sta a predkościa po nieskończo-

nej amanej jak na rysunku, któ- 

rej $\mathrm{d}$ ugości kolejnych odcinków
\begin{center}
\includegraphics[width=44.352mm,height=37.488mm]{./KursMatematyki_PolitechnikaWroclawska_1999_2004_page41_images/image001.eps}
\end{center}
$\alpha_{2}$

$\alpha_{3}$

$\alpha_{1}$

$\alpha_{4}$

{\it O}

{\it P}

tworza ciag geometryczny maleja-

cy. Po pewnym czasie czastka za-

trzyma a $\mathrm{s}\mathrm{i}\mathrm{e} \mathrm{w}$ punkcie $P(10,3).$

Jaka droge przeby a czastka?

33.5. Stosujac zasade indukcji matematycznej, udowodnič, $\dot{\mathrm{z}}\mathrm{e}$ dla wszyst-

kich $n \geq 1$ wielomian $x^{3n+1}+x^{3n-1}+1$ dzieli $\mathrm{s}\mathrm{i}\mathrm{e}$ bez reszty przez

wielomian $x^{2}+x+1.$

33.6. Narysowač wykres funkcji $f(x) = \displaystyle \frac{|x-2|}{x-|x|+2}$ bez badania jej prze-

biegu. Podač równania asymptot $\mathrm{i}$ ekstrema lokalne tej funkcji.

33.7. Rozwiazač nierównośč

$|\cos x|^{1+\sqrt{2}\sin x+\sqrt{2}\cos x}\leq 1,$

$x\in[-\pi,\pi].$

33.8. $\mathrm{W}$ stozek wpisano graniastoslup trójkatny prawidlowy $0$ wszystkich

krawedziach tej samej dlugości, tak $\dot{\mathrm{z}}\mathrm{e}$ dolna podstawa $\mathrm{l}\mathrm{e}\dot{\mathrm{z}}\mathrm{y}$ na pod-

stawie stozka. Przy jakim kacie rozwarcia stozka stosunek objetości

graniastoslupa do objetości stozka jest najwiekszy?





54

Praca kontrolna nr 6

34.1. $\mathrm{W}$ kolo $0$ polu $\displaystyle \frac{5}{4}\pi$ wpisano trójkat prostokatny $0$ polu l.

obwód tego trójkata.

Obliczyč

34.2. Sprowadzič do najprostszej postaci wyrazenie

2(sin6 $\alpha+\cos^{6}\alpha$)$-7(\sin^{4}\alpha+\cos^{4}\alpha)+\cos 4\alpha.$

34.3. Wyznaczyč trójmian kwadratowy, którego wykresem jest parabola

styczna do prostej $y=x+2$, przechodzaca przez punkt $P(-2,-2)$

oraz symetryczna wzgledem prostej $x=1$. Sporzadzič rysunek.

34.4. $\mathrm{W}$ trapezie ABCD, $\mathrm{w}$ którym AB $\Vert CD$, dane sa $\vec{AC}= [4$, 7$]$ oraz

$\vec{BD}=[-6,2]$. Za pomoca rachunku wektorowego wyznaczyč wektory

$\vec{AB}\mathrm{i}\vec{CD}$, jeśli wiadomo, $\dot{\mathrm{z}}\mathrm{e} \vec{AD}\perp\vec{BD}.$

34.5. Jaś ma $\mathrm{w}$ portmonetce 3 monetyjednoz1otowe, 2 monety dwuz1otowe

$\mathrm{i}$ jedna pieciozlotowa. Kupujac zeszyt $\mathrm{w}$ cenie 4 $\mathrm{z}l$, wyciaga losowo

$\mathrm{z}$ portmonetki po jednej monecie tak dlugo, $\mathrm{a}\dot{\mathrm{z}}$ uzbiera $\mathrm{s}\mathrm{i}\mathrm{e}$ suma wys-

tarczajaca na kupno zeszytu. Obliczyč prawdopodobieństwo, $\dot{\mathrm{z}}\mathrm{e}$ Jaś

wyciagnie co najmniej trzy monety. Podač odpowiednie uzasadnienie

(nie jest nim $\mathrm{t}\mathrm{z}\mathrm{w}$. drzewko).

34.6. Narysowač na plaszczy $\acute{\mathrm{z}}\mathrm{n}\mathrm{i}\mathrm{e}$ zbiór punktów określony nastepujaco

$\mathcal{F}=\{(x,y):\sqrt{4x-x^{2}}\leq y\leq 4-\sqrt{1-2x+x^{2}}\}.$

Wjakiej odleglości od brzegu figury $\mathcal{F}$ znajduje $\mathrm{s}\mathrm{i}\mathrm{e}$ punkt $P(\displaystyle \frac{3}{2},\frac{5}{2})$ ?

34.7. Danajest funkcja $f(x)=\log_{2}(1-x^{2})-\log_{2}(x^{2}-x)$. Bez stosowania

metod rachunku rózniczkowego wykazač, $\dot{\mathrm{z}}\mathrm{e}f$ jest rosnaca $\mathrm{w}$ swojej

dziedzinie oraz, $\dot{\mathrm{z}}\mathrm{e}g(x) = f(x-\displaystyle \frac{1}{2})$ jest nieparzysta. Wyznaczyč

funkcje odwrotna $f^{-1}$, jej dziedzine $\mathrm{i}$ zbiór wartości.

34.8. Pole powierzchni bocznej ostroslupa prawidlowego czworokatnego wy-

nosi $c^{2}$, a $\mathrm{k}\mathrm{a}\mathrm{t}$ nachylenia ściany bocznej do podstawy ma miare $\alpha.$

Ostroslup przecieto na dwie cześci plaszczyzna przechodzaca przez

jeden $\mathrm{z}$ wierzcholków podstawy $\mathrm{i}$ prostopadla do przeciwleglej krawe-

dzi bocznej. Obliczyč objetośč cześci zawierajacej wierzcholek ostro-

slupa. Podač warunek rozwiazalności zadania.





55

Praca kontrolna nr 7

35.1. Dwa pierwsze wyrazy nieskończonego ciagu geometrycznego sa pier-

wiastkami równania $4x^{2}-4px-3p^{2}=0$, gdzie $p$ jest nieznana liczba.

Wyznaczyč ten ciag, jeśli suma wszystkich jego wyrazów wynosi 3.

35.2. Wiedzac, $\dot{\mathrm{z}}\mathrm{e} \cos\varphi=\sqrt{\frac{2}{3}}$ oraz $\varphi\in (\displaystyle \frac{3}{2}\pi,2\pi)$, obliczyč cosinus kata

pomiedzy prostymi $y= (\displaystyle \sin\frac{\varphi}{2})x, y= (\displaystyle \cos\frac{\varphi}{2})x.$

35.3. Kostka sześcienna ma krawed $\acute{\mathrm{z}}  2\alpha$. Aby zmieścič $\mathrm{j}\mathrm{a}\mathrm{w}$ pojemniku

$\mathrm{w}$ ksztalcie kuli $0$ średnicy $ 3\alpha$, ze wszystkich narozników odcieto

$\mathrm{w}$ minimalny sposób jednakowe ostroslupy prawidlowe trójkatne.

Obliczyč dlugośč krawedzi bocznych odcietych czworościanów?

35.4. Udowodnič prawdziwośč nierówności

$1+\displaystyle \frac{x}{2}\geq\sqrt{1+x}\geq 1+\frac{x}{2}-\frac{x^{2}}{2}$ dla $x\in[-1,1].$

Zilustrowač $\mathrm{j}\mathrm{a}$ na odpowiednim wykresie.

35.5. Rozwiazač równanie

--csions25{\it xx}$=$-sin3{\it x}.

35.6. Dany jest okrag $\mathcal{K}$ : $x^{2}-4x+y^{2}+6y = 0$. Znalez$\acute{}$č równanie

okregu symetrycznego do $\mathcal{K}$ wzgledem stycznej do $\mathcal{K}$ poprowadzonej

$\mathrm{z}$ punktu $P(3,5)\mathrm{i}$ majacej dodatni wspólczynnik kierunkowy.

35.7. W okrag 0 promieniu r wpisano trapez 0 przekatnej d,

i najwiekszym obwodzie. Obliczyč pole tego trapezu.

$ d\geq r\sqrt{3},$

35.8. Metoda analityczna określič dla jakich wartości parametru $m$ uklad

równań

$\left\{\begin{array}{l}
mx-y+2=0\\
x-2|y|+2=0
\end{array}\right.$

ma dokladnie jedno rozwiazanie. Wyznaczyč to rozwiazanie $\mathrm{w}$ zalez-

ności od $m$. Sporzadzič rysunek.





Indeks

tematyczny





59

l. Liczby rzeczywiste

$\bullet$ Obliczenia procentowe: 1.1, 9.1, 16.1, 26.2, 33.1.

$\bullet$ Zasada indukcji matematycznej:

23.6, 25.6, 31.5, 33.5.

2.1, 3.3, 10.1, 11.5, 17.5, 19.6,

$\bullet$ Inne: 8.2, 28.6, 33.1.

2. Funkcja liniowa

$\bullet$ Uklady równań liniowych $\mathrm{z}$ parametrem: 7.5, 12.7, 18.4, 24.3, 35.8.

$\bullet$ Wartośč bezwzgledna: 2.5, 5.1, 13.7, 27.8, 35.8.

$\bullet$ Inne: 15.1, 28.1, 29.2, 32.1.

3. Funkcja kwadratowa

$\bullet$ Równania kwadratowe $\mathrm{z}$ parametrem: 2.4, 4.7, 5.7, 9.3, 11.7, 13.3,

16.7, 19.5, 20.1, 23.8, 25.1, 28.2, 32.4.

$\bullet$ Uklady równań drugiego stopnia: 2.6, 10.7, 21.1, 31.3.

$\bullet$ Inne: 2.2, 3.1 (10.3), 8.3, 21.8, 22.1, 34.3, 35.1, 35.4.

4. Wielomiany

$\bullet$ 2.1, 12.2, 17.3, 21.1, 23.2, 28.5, 30.6, 33.5.





60

5. Funkcje wymierne i niewymierne

$\bullet$ Wykresy funkcji: 12.1, 13.7, 15.5, 19.3, 20.5, 26.7, 33.6.

$\bullet$ Równania $\mathrm{i}$ nierówności: 1.6, 5.4, 9.5, 11.8, 15.2, 23.5, 27.1, 29.4,

30.5.

6.

Funkcje potegowe, wykIadnicze

i logarytmiczne

$\bullet$ Wykresy funkcji: 1.5, 13.3, 26.5.

$\bullet$ Przeksztalcanie wyrazeń: 14.6, 27.5, 31.6, 32.2.

$\bullet$ Równania: 1.2, 6.1, 9.3, 16.5, 21.6, 31.3.

$\bullet$ Nierówności: 2.3, 4.5, 7.1, 8.4, 10.5, 11.7, 17.6, 20.6, 22.5, 24.5,

25.3, 27.6, 29.5, 30.7, 33.7.

7. Funkcje trygonometryczne

$\bullet \mathrm{T}\mathrm{o}\dot{\mathrm{z}}$ samości, przeksztalcanie wyrazeń: 4.3, 5.2, 6.3, 13.1, 14.4, 16.6,

17.1, 24.6, 31.5, 34.2, 35.2.

$\bullet$ Równania: 1.4, 2.8, 8.7, 10.8, 15.6, 19.7, 22.7, 23.8, 26.6, 27.5,

28.7, 29.6, 32.6, 35.5.

$\bullet$ Nierówności: 3.8, 7.6, 10.8, 11.4, 18.5, 21.5, 23.8, 27.6, 33.7.





61

8. WIasności funkcji

$\bullet$ 3.1 (10.3), 15.5, 16.6, 26.7, 34.7.

9. Ciagi liczbowe

$\bullet$ Ciag arytmetyczny: 4.1, 13.5, 19.2, 22.2, 30.2.

$\bullet$ Ciag geometryczny: 6.4, 15.4, 17.1, 18.1, 32.2.

$\bullet$ Ciag geometryczny nieskończony:

24.1, 25.8, 28.7, 29.8, 33.4, 35.1.

1.2, 2.8, 8.1, 11.4, 14.2, 15.8,

$\bullet$ Granica ciagu: 14.6, 18.1, 21.4, 27.5, 31.6.

$\bullet$ Wlasności ciagu: 21.4, 23.6.

10.

Granica i ciagIośč funkcji.

Pochodna, styczna

$\bullet$ 1.8, 4.3, 9.8, 16.8, 20.7, 21.8, 22.8, 23.7, 24.8, 27.8, 32.7.

ll. Zastosowania pochodnej

$\bullet$ Badanie przebiegu funkcji: 3.6, 4.7, 9.7, 13.8, 15.8, 25.7, 31.8.

$\bullet$ Ekstrema lokalne: 8.3, 11.3, 26.5, 28.5, 33.6.

$\bullet$ Wartośč najmniejsza $\mathrm{i}$ najwieksza funkcji $\mathrm{w}$ zbiorze: 2.2, 3.1 (10.3),

3.5, 6.7, 10.6, 12.8 (26.8), 18.8, 19.5, 21.7, 22.8, 25.2, 29.8, 33.8, 35.7.

$\bullet$ Inne: 5.7, 24.8, 30.8.





62

12. Geometria analityczna

$\bullet$ Rachunek wektorowy, iloczyn skalarny: 1.3, 5.8 (15.7), 8.6, 13.2,

14.3, 19.1, 21.2, 22.6, 23.7, 29.1, 31.7, 34.4.

$\bullet$ Prosta: 3.2, 14.3, 18.2, 26.3, 35.2.

$\bullet$ Okrag: 2.7, 6.2, 10.6, 12.4, 16.2, 24.7, 27.2, 30.3, 35.6.

$\bullet$ Zbiory punktów $0$ danej wlasności (miejsca geometryczne punktów):

4.6, 5.3, 9.2, 11.6, 20.4, 25.4, 28.8, 33.2.

$\bullet$ Geometryczna interpretacja ukladów równań $\mathrm{i}$ nierówności: 2.6, 4.5,

5.1, 10.7, 13.7, 19.3, 23.5, 25.3, 27.8, 29.5, 34.6.

$\bullet$ Inne: 7.2, 14.5, 17.4, 21.8, 34.3.

13. Planimetria

$\bullet$ Trójkat: 3.5, 6.8, 8.8, 9.6, 11.1, 13.6, 20.8, 21.3, 22.4, 23.3, 25.8,

31.4, 33.3, 34.1.

$\bullet$ Trapez: 3.7, 7.4, 12.6, 17.7, 19.4, 27.3, 29.3, 35.7.

$\bullet$ Inne figury: 4.4, 14.2, 16.4, 18.7, 23.3, 26.1, 28.3, 30.4.





63

14. Stereometria

$\bullet$ Graniastoslupy: 6.4, 11.2, 19.2, 21.1, 24.4, 35.3.

$\bullet$ Ostroslupy: 1.7, 3.4, 4.8, 7.7 (31.2), 8.5, 9.4, 12.5, 14.8, 16.3,

18.8, 19.8, 25.2, 26.4, 27.4, 29.7, 32.8, 34.8.

$\bullet$ Bryly obrotowe: 5.6, 6.6, 10.2 (18.3), 12.8 (26.8), 20.2, 21.7, 22.3,

28.4, 30.1, 32.3, 33.8.

$\bullet$ Inne bryly: 15.3, 17.8, 23.4.

15.

Rachunek prawdopodobieństwa

i kombinatoryka

$\bullet$ Prawdopodobieństwo klasyczne:

24.2, 31.1.

4.2, 7.8 (14.7), 10.4, 17.2, 20.3,

$\bullet$ Wzór na prawdopodobieństwo calkowite: 6.5, 12.3, 25.5, 34.5.

$\bullet$ Niezaleznośč $\mathrm{i}$ schemat Bernoulliego: 5.5, 13.4, 18.6, 27.7, 32.5.

$\bullet$ Kombinatoryka: 7.3, 14.1, 23.1.





9

Praca kontrolna nr l

l.l. Stop sklada $\mathrm{s}\mathrm{i}\mathrm{e} \mathrm{z}$ 40\% srebra próby 0,6, 30\% srebra próby 0,7 oraz

l kg srebra próby 0,8. Jaka jest masa $\mathrm{i}$ jaka jest próba tego stopu?

1.2. Rozwiazač równanie

$3^{x}+1+3^{-x}+=4,$

którego lewa strona jest suma nieskończonego ciagu geometrycznego.

1.3. $\mathrm{W}$ trójkacie $ABC$ znane sa wierzcholki $A(0,0)$ oraz $B(4,-1)$. Wiado-

mo, $\dot{\mathrm{z}}\mathrm{e}\mathrm{w}$ punkcie $H(3,2)$ przecinaja $\mathrm{s}\mathrm{i}\mathrm{e}$ proste zawierajace wysokości

tego trójkata. Wyznaczyč wspólrzedne wierzcholka $C$. Sporzadzič

rysunek.

1.4. Rozwiazač równanie

$\cos 4x=\sin 3x.$

1.5. Narysowač staranny wykres funkcji

$f(x)=|\log_{2}(x-2)^{2}|.$

1.6. Rozwiazač nierównośč

$\displaystyle \frac{1}{x^{2}}\geq\frac{1}{x+6}.$

1.7. $\mathrm{W}$ ostroslupie prawidlowym sześciokatnym krawed $\acute{\mathrm{z}}$ podstawy ma dlu-

gośč $p$, a krawed $\acute{\mathrm{z}}$ boczna dlugośč $2p$. Obliczyč cosinus kata dwuścien-

nego miedzy sasiednimi ścianami bocznymi tego ostroslupa.

1.8. Wyznaczyč równania wszystkich prostych stycznych do wykresu funkcji

$f(x)=\displaystyle \frac{2x+10}{x+4},$

które sa równolegle do prostej stycznej do wykresu funkcji

$g(x)=\sqrt{1-x}\mathrm{w}$ punkcie $x=0$. Rozwiazanie zilustrowač rysunkiem.





Odpowiedzi

do

zadań





67

l.l. Masa stopu 2,3 kg, próba stopu 0,690.

1.2. 1og32.

1.3. {\it C}(--3101'--1101).

1.4. $\displaystyle \frac{\pi}{14}+k\frac{2\pi}{7},  k\in$ Z.

1.5. Rysunek l.
\begin{center}
\includegraphics[width=108.000mm,height=84.228mm]{./KursMatematyki_PolitechnikaWroclawska_1999_2004_page51_images/image001.eps}
\end{center}
{\it y}  í í

4

2

$-2$  0 1  3 4  6 {\it x}

Rys. l

1.6. $(- 00,-6)\cup[-2,0)\cup(0$, 3$].$

1.7. $-\displaystyle \frac{3}{5}.$

1.8. $\mathrm{s}_{\mathrm{a}}$ dwie takie styczne $\mathrm{i}$ maja równania $y = -\displaystyle \frac{1}{2}x +2$ oraz

$y=-\displaystyle \frac{1}{2}x-2.$

2.2. Podstawa prostokata $\alpha = \displaystyle \frac{2}{5}\sqrt{10}$ cm, wysokośč $b = \displaystyle \frac{4}{5}\sqrt{10}$ cm,

przekatna $p=2\sqrt{2}$ cm.





68

2.3. $(1,\sqrt[3]{9})\cup(\sqrt[3]{9},\infty).$

2.4. $4-2\sqrt{2}<p<4+2\sqrt{2}.$

2.5. Dla $m<0$ brak rozwiazań,

dla $m=0$ lub $m>1$ sa dwa rozwiazania,

dla $m=1$ sa trzy rozwiazania,

dla $0<m<1$ sa cztery rozwiazania.

2.6. Uklad ma trzy rozwiazania:

$\left\{\begin{array}{l}
x_{1}=-7\\
y_{1}=-1,
\end{array}\right.$

2.7. $S=\displaystyle \frac{1225}{12}.$

$\left\{\begin{array}{l}
x_{2}=1\\
y_{2}=7,
\end{array}\right.$

$\left\{\begin{array}{l}
x_{3}=5\\
y_{3}=-5.
\end{array}\right.$

2.8. -$\pi$8'--78$\pi$, --98$\pi$, --158$\pi$.

3.1. Dziedzina jest przedzial $[0$, 4$]$, a zbiorem wartości przedzial $[0,\displaystyle \frac{3}{2}]$.

3.2. Prosta $AB$ ma równanie $y=3$, a prosta $AD$ równanie $4x-3y=15$.

3.4. $\displaystyle \frac{8}{3}\sqrt{3}\mathrm{c}\mathrm{m}^{3}$

3.5. Trójkat równoboczny $0$ boku $R\sqrt{3}\mathrm{i}$ polu $\displaystyle \frac{3\sqrt{3}}{4}R^{2}$

3.6. $D = (-\displaystyle \infty,\frac{5}{2}]$; miejsca zerowe $0, \displaystyle \frac{5}{2}$; minimum lokalne 0

dla $x = 0$; maksimum lokalne 2 d1a $x = 2$; funkcja rosnaca $\mathrm{w} (0,2)$ ;

malejaca $\mathrm{w} (-\infty,0)$ oraz $\mathrm{w} (2,\displaystyle \frac{5}{2})$ ; wypukla $\mathrm{w} (-\displaystyle \infty,2-\frac{\sqrt{6}}{3})$ ;

wklesla $\mathrm{w} (2-\displaystyle \frac{\sqrt{6}}{3},\frac{5}{2})$ ; punkt przegiecia $P(2-\displaystyle \frac{\sqrt{6}}{3},\sqrt{\frac{62\sqrt{6}-117}{27}})$ ;

prosta $x= \displaystyle \frac{5}{2}$ jest styczna do wykresu funkcji $\mathrm{w}$ punkcie $(\displaystyle \frac{5}{2},0)$. Wykres

funkcji przedstawiono na rysunku 2.





69
\begin{center}
\includegraphics[width=66.036mm,height=56.544mm]{./KursMatematyki_PolitechnikaWroclawska_1999_2004_page53_images/image001.eps}
\end{center}
{\it y}

2

1  {\it P}

$-1$  0 1  2  3 {\it x}

Rys. 2

3.7.

$S = \displaystyle \frac{d^{2}-2dr\cos\alpha+r^{2}\cos 2\alpha}{2(d-r\cos\alpha)}r\sin\alpha$; $R = \displaystyle \frac{d^{2}-2dr\cos\alpha+r^{2}}{4(d-r\cos\alpha)\sin\alpha}$;

rozwiazanie istnieje, gdy $d\geq r(1+2\cos\alpha)$. Wynik liczbowy $S=\displaystyle \frac{13}{12}\sqrt{3}\mathrm{c}\mathrm{m}^{2},$

$R=\displaystyle \frac{7}{3}$ cm.

3.8.[0,-1$\pi$2]$\cup$[-71$\pi$2'-1132$\pi$]$\cup$[-1192$\pi$,-2152$\pi$]$\cup$[-3112$\pi$,3$\pi$].

4.1.109.

4.2.-97.

lub $\{$

4.3. Pochodna nie istnieje.

4.5. $\{x\leq 11<y\leq 3-x$

$0<y<1$

$1\leq x\leq 3-y.$

4.6. Elipsa $0$ równaniu $\displaystyle \frac{(x+4)^{2}}{36}+\frac{y^{2}}{20}=1$, środku $M(-4,0)\mathrm{i}$ pólosiach

$\alpha=6, b=2\sqrt{5}.$





70

4.7. $y = f(m) = \displaystyle \frac{4}{m}-\frac{4}{m^{2}}. D = ($-00, $0)\cup(0,5]$; miejsce zerowe l;

asymptota pionowa obustronna $m = 0$; asymptota pozioma lewostronna

$y=0$; maksimum lokalne l dla $m=2$; funkcja rosnaca $\mathrm{w} (0,2)$ ; malejaca

$\mathrm{w} (-\infty,0)$ oraz $\mathrm{w} (2,5)$ ; wypukla $\mathrm{w} (3,5)$ ; wklesla $\mathrm{w} (-\infty,0)$ oraz

$\mathrm{w}(0,3)$ ; punkt przegiecia $P(3,\displaystyle \frac{8}{9})$. Wykres przedstawiono na rysunku 3.
\begin{center}
\includegraphics[width=135.588mm,height=70.608mm]{./KursMatematyki_PolitechnikaWroclawska_1999_2004_page54_images/image001.eps}
\end{center}
$y$

1

{\it P}

$-4  -2$  0 2 3  {\it 5 m}

$-1$

$-3$

Rys. 3

4.8. $S=\displaystyle \frac{\pi}{16}\alpha^{2_{\cos^{2}\alpha(3-4\cos^{2}\alpha)}}27-32\cos^{2}\alpha, \alpha\in \displaystyle \frac{\pi}{6}, \displaystyle \frac{\pi}{2}$

5.1. Zbiór $A$ przedstawiono na rysunku 4.

najblizej punktu P.

Punkt $Q \displaystyle \frac{3}{2}, \displaystyle \frac{5}{2}$

$\mathrm{l}\mathrm{e}\dot{\mathrm{z}}\mathrm{y}$
\begin{center}
\includegraphics[width=60.192mm,height=72.084mm]{./KursMatematyki_PolitechnikaWroclawska_1999_2004_page54_images/image002.eps}
\end{center}
{\it y}

{\it P}

{\it Q}

2

$-1$  2  {\it x}

Rys. 4





71

5.2. $\displaystyle \frac{7}{16}\sqrt{5}$ lub - $\displaystyle \frac{7}{16}\sqrt{5}.$

5.3. Szukana krzywa stanowia dwie galezie paraboli $y= \displaystyle \frac{1}{2}x^{2}-1$ dla

$x\geq 2$ oraz dla $x\leq 2.$

5.4. 11.

5.5. Pierwszy.

5.6. $2r+4\sqrt{2Rr-R^{2}}.$

5.7. Dla $ m\in [2\sqrt{3},\infty$).

5.8. $\displaystyle \frac{9}{85}\sqrt{85}.$

6.1. $\displaystyle \frac{1}{4}(-3+3\sqrt{3}).$

6.2. $\displaystyle \frac{13}{3}.$

6.4. $8+(1+\sqrt{33})^{3/2}$

6.5. $\displaystyle \frac{3}{10}.$

6.6. $\displaystyle \frac{\pi}{12}d^{3}\mathrm{t}\mathrm{g}^{2}\alpha(8\cos^{4}\alpha-1).$

6.7. Wartośč najmniejsza 3l, a najwieksza $24\sqrt{2}$.

6.8. Stosunek wynosi $1+k$, a dziedzina $k$ jest przedzial $(0,\sqrt{2}-1$].

7.1. $(0,1).$

7.2.

Elipsa $0$ równaniu $\displaystyle \frac{(x+1)^{2}}{4} + \displaystyle \frac{(y-3)^{2}}{1} =$

l, środku $S(-1,3)$

$\mathrm{i}$ pólosiach $\alpha=2, b=1$. Pole figury wynosi $2\pi.$





72

7.3. 330.

7.4. $\displaystyle \frac{\pi}{4}(\sqrt{8}-\sqrt{6}).$

7.5. Dla $m$ róznych od 3 $\mathrm{i}4$ jedno rozwiazanie $x=\displaystyle \frac{9}{m-4},\ y=\displaystyle \frac{m+2}{m-4}$.

Dla $m = 4$ uklad sprzeczny. Dla $m = 3$ nieskończenie wiele rozwiazań

spelniajacych warunek $x-2y = 1$, gdzie $x$ dowolne rzeczywiste. $\mathrm{s}_{\mathrm{a}}$ dwa

rozwiazania spelniajace warunek $x=y$: dla $m=7 (x=y=3)$ oraz dla

$m=3 (x=y=-1).$

7.6. $(-\displaystyle \frac{\pi}{3},0)\cup(\frac{\pi}{3},\frac{\pi}{2}].$

7.7. Objetośč ostroslupa wynosi $\displaystyle \frac{343}{3}\mathrm{c}\mathrm{m}^{3}$, a objetośč najmniejszej cześci

$\displaystyle \frac{61}{3}\mathrm{c}\mathrm{m}^{3}$

7.8. a) $\displaystyle \frac{1}{20};\mathrm{b}) \displaystyle \frac{7}{20}.$

8.1. $q=\displaystyle \frac{1}{30}, \alpha_{1}=1972.$

8.2. $\mathrm{s}_{\mathrm{a}}$ dwa takie skladniki 26730 oraz 1320.

8.3. Wykres funkcji przedstawiono na rysunku 5.

Rys. 5





73

Maksima lokalne 4 d1a $x = 1 \mathrm{i}x = -1$; minima lokalne 0 d1a $x = 3$

$\mathrm{i} x = -3$ oraz 3 d1a $x = 0$. Funkcja rosnaca $\mathrm{w}$ przedzialach $(-3,-1),$

$(0,1), ($3, $\infty)$ ; malejaca $\mathrm{w}$ przedzialach $(-\infty,-3), (-1,0)$, (1, 3).

8.4.

(-23, 2].

8.5. $\displaystyle \frac{1}{48}\alpha^{3}\sqrt{\sqrt{52}-2}.$

8.6. -32, 1, -27, --139.

8.7. $\displaystyle \frac{\pi}{9}+k\frac{\pi}{3}$ lub $\displaystyle \frac{2\pi}{9}+k\frac{\pi}{3},  k\in$ Z.

8.8. $(\sqrt{S_{1}}+\sqrt{S_{2}}+\sqrt{S_{3}})^{2}$

9.1. $\mathrm{O}$ 72,8\%.

9.2. Prawa gala $\acute{\mathrm{z}}$ hiperboli $0$ równaniu $y=\displaystyle \frac{1}{2}+\frac{1}{2(x-1)}$,

$x>1.$

9.3. $m\in(1,2).$

9.4. $\sqrt{3}.$

9.5. $(- 00,-3)\cup[1,3)\cup(3$, 5$].$

9.6. $\displaystyle \frac{\sqrt{1+k^{2}}-1+k}{k^{2}\sqrt{2}}.$

9.7. $D=\mathrm{R}\backslash \{2\}$; asymptota pionowa obustronna $x= 2$; asymptota

pozioma obustronna $y=1$; minimum lokalne $\displaystyle \frac{1}{2}$ dla $x=-2$; funkcja rosnaca

$\mathrm{w} (-2,2)$ ; malejaca $\mathrm{w} (- 00,-2)$ oraz $\mathrm{w} (2,\infty)$ ; wypukla $\mathrm{w} (-4,2)$ oraz

$\mathrm{w}(2,\infty)$ ; wklesla $\mathrm{w}(-\infty,-4)$ ; punkt przegiecia $P(-4,\displaystyle \frac{5}{9})$. Wykres funkcji

przedstawiono na rysunku 6.





74
\begin{center}
\includegraphics[width=144.324mm,height=79.656mm]{./KursMatematyki_PolitechnikaWroclawska_1999_2004_page58_images/image001.eps}
\end{center}
{\it y}

5

1

$-4$

{\it P}

$-2$

2 4  6 {\it x}

Rys. 6

9.8. $y=10x-16, y=-\displaystyle \frac{5}{4}x-\frac{1}{4}, y=-\displaystyle \frac{38}{25}x+\frac{16}{125}.$

10.2. $V=-\displaystyle \frac{\pi}{6}l^{3}\sin 4\alpha\cos 2\alpha, \varphi=3\pi-4\alpha, \alpha\in (\displaystyle \frac{\pi}{2},\frac{3\pi}{4}).$

10.3. Dziedzinajest przedzial $[0$, 4$]$, a zbiorem wartości przedzial $[0,\displaystyle \frac{3}{2}].$

10.4. $\displaystyle \frac{240}{1771}\approx 0$, 136.

10.5. $(\displaystyle \frac{1}{4},\frac{1}{2})\cup[2$, 4$].$

10.6.

$r=\displaystyle \frac{1}{2}.$

$S(r) = r(1-r^{2})^{3/2}, r \in (0,1)$. Wartośč najwieksza $\displaystyle \frac{3\sqrt{3}}{16}$ dla

10.7. Uklad ma cztery rozwiazania:

$\left\{\begin{array}{l}
x_{1}=0\\
y_{1}=0,
\end{array}\right.$

$\left\{\begin{array}{l}
x_{2}=\frac{16}{5}\\
y_{2}=\frac{12}{5},
\end{array}\right.$

$\left\{\begin{array}{l}
x_{3}=-\frac{16}{5}\\
y_{3}=-\frac{12}{5},
\end{array}\right.$

$\left\{\begin{array}{l}
x_{4}=4\\
y_{4}=-2.
\end{array}\right.$





75

Pierwsze równanie przedstawia dwa okregi styczne do osi $Oy 0$ środkach

$S_{1}(\displaystyle \frac{5}{2},0), S_{2}(-\displaystyle \frac{5}{2},0) \mathrm{i}$ promieniu $\displaystyle \frac{5}{2}$. Drugie równanie przedstawia dwie

proste równolegle.

10.8. Rozwiazania równania: $\displaystyle \frac{3\pi}{8}, \displaystyle \frac{7\pi}{8}.$

$[0,\displaystyle \frac{3\pi}{8})\cup(\frac{7\pi}{8},\pi].$

11.1. $\displaystyle \frac{\pi}{6}.$

Zbiór rozwiazań nierówności:

11.2. $\displaystyle \frac{2\sqrt{6}}{3}R.$

11.3. 4, $-4.$

11.4. $(\displaystyle \frac{\pi}{4},\frac{3\pi}{4})\cup(\frac{5\pi}{4},\frac{7\pi}{4}).$

11.6. $|y|=\displaystyle \frac{(x+2)^{2}}{4}-1, x\in(-\infty,-4)\cup(0,\infty).$

11.7. $3^{\sqrt{11}}, 3^{-\sqrt{11}}.$

11.8. $[-5,0)\cup(5$, 6$].$

12.1. $\displaystyle \frac{9-\sqrt{5}}{2}.$

12.2. $-1.$

12.3. $\displaystyle \frac{107}{128}\approx 0$, 836.

12.4. $(x-1)^{2}+(y-1)^{2}=1, (x-6)^{2}+(y-6)^{2}=36, (x+2)^{2}+(y-2)^{2}=4,$

$(x-3)^{2}+(y+3)^{2}=9.$

12.5. -32{\it d}3--$\sqrt{}$tgtg2$\alpha$3$\alpha$-1' $\alpha\in$ (-$\pi$4'-$\pi$2).

12.6. $s+\displaystyle \frac{16Pr}{\sqrt{16P^{2}+s^{4}}}$. Warunek rozwiazalności $r\displaystyle \geq\frac{\sqrt{16P^{2}+s^{4}}}{4s}.$





10

Praca kontrolna nr 2

2.1. Udowodnič, $\dot{\mathrm{z}}\mathrm{e}$ dla $\mathrm{k}\mathrm{a}\dot{\mathrm{z}}$ dego $n$ naturalnego wielomian $x^{4n-2}+1$ jest

podzielny przez trójmian kwadratowy $x^{2}+1.$

2.2. $\mathrm{W}$ równoramienny trójkat prostokatny $0$ polu $S = 10\mathrm{c}\mathrm{m}^{2}$ wpisano

prostokat $\mathrm{w}$ taki sposób, aby jeden $\mathrm{z}$ jego boków $\mathrm{l}\mathrm{e}\dot{\mathrm{z}}\mathrm{a}l$ na przeci-

wprostokatnej trójkata, a pozostale dwa wierzcholki znalazly $\mathrm{s}\mathrm{i}\mathrm{e}$ na

przyprostokatnych $\mathrm{i}$ równocześnie $\mathrm{t}\mathrm{a}\mathrm{k}$, aby mial on najkrótsza prze-

katna. Obliczyč dlugośč przekatnej tego prostokata.

2.3. Rozwiazač nierównośč

log1253 $\log_{x}5+\log_{9}8\log_{4}x>1.$

2.4. Znalez$\acute{}$č wszystkie wartości parametru $p$, dla których wykres funkcji

$y=x^{2}+4x+3\mathrm{l}\mathrm{e}\dot{\mathrm{z}}\mathrm{y}$ nad prosta $y=px+1.$

2.5. Zbadač liczbe rozwiazań równania

$||x+5|-1|=m$

$\mathrm{w}$ zalezności od parametru $m.$

2.6. Rozwiazač uklad równań

$\left\{\begin{array}{l}
x^{2}+y^{2}=50\\
(x-2)(y+2)=-9.
\end{array}\right.$

Podač interpretacje geometryczna tego ukladu $\mathrm{i}$ sporzadzič odpowiedni

rysunek.

2.7. Wyznaczyč na osi odcietych punkty A $\mathrm{i} B,\ \mathrm{z}$ których okrag

$x^{2}+y^{2}-4x+2y= 20$ widač pod katem prostym, $\mathrm{t}\mathrm{z}\mathrm{n}$. styczne do

okregu wychodzace $\mathrm{z}\mathrm{k}\mathrm{a}\dot{\mathrm{z}}$ dego $\mathrm{z}$ tych punktów sa do siebie prostopadle.

Obliczyč pole figury ograniczonej stycznymi do okregu przechodzacymi

przez punkty A $\mathrm{i}B$. Rozwiazanie zilustrowač rysunkiem.

2.8. $\mathrm{W}$ przedziale $[0,2\pi]$ rozwiazač równanie

$1-\mathrm{t}\mathrm{g}^{2}x+\mathrm{t}\mathrm{g}^{4}x-\mathrm{t}\mathrm{g}^{6}x+\ldots=\sin^{2}3x.$





76

12.7. Dla parametrów $p$ róznych od 2 $\mathrm{i}-1$ jedno rozwiazanie $x=3p,$

$y=-3p-2$. Dla $p=2$ nieskończenie wiele rozwiazań spelniajacych waru-

nek $2x+y-4=0$, gdzie $x$ dowolne rzeczywiste. Dla $p=-1$ nieskończenie

wiele rozwiazań spelniajacych warunek $x-y+4 = 0$, gdzie $x$ dowolne

rzeczywiste. Rozwiazania $0$ wspólrzednych calkowitych:

$\left\{\begin{array}{l}
x=-2\\
y=2
\end{array}\right.$

, $p=-1$;

$\left\{\begin{array}{l}
x=-2\\
y=0
\end{array}\right.$

, {\it p}$=$ - -32;

$\left\{\begin{array}{l}
x=-1\\
y=-1
\end{array}\right.$

, {\it p}$=$ - -31;

$\left\{\begin{array}{l}
x=0\\
y=-2
\end{array}\right.$

, $p=0$;

$\left\{\begin{array}{l}
x=1\\
y=2
\end{array}\right.$

, $p=2$;

$\left\{\begin{array}{l}
x=2\\
y=0
\end{array}\right.$

, $p=2.$

12.8. $S(y) = \displaystyle \pi(y+\frac{3}{2})\sqrt{1+(y-\frac{3}{2})^{2}},$

mniejsza $\displaystyle \frac{3\sqrt{13}}{4}\pi$ dla $y=0.$

$y \in$

[0, -23].

Wartośč naj-

13.2. 3.

13.3. $f(m)=|2^{-(m-3)}-2|,$

przedstawiono na rysunku 7.

$ m\in$ (-00, 4$]$ \{log27\}. Wykres funkcji $f$
\begin{center}
\includegraphics[width=60.252mm,height=108.048mm]{./KursMatematyki_PolitechnikaWroclawska_1999_2004_page60_images/image001.eps}
\end{center}
$y$

6

$m_{0}=\log_{2}7$

4

2

0 1  2  $m_{0}$  4{\it m}

Rys. 7





77

13.4. $\displaystyle \frac{35}{144}\approx 0$, 243.

13.5.

20, 28.

10, 8, 6, 4 1ub

10, 8, 6, 4, 2, $0, -2$ lub $-20, -12, -4$, 4, 12,

13.6. $\displaystyle \frac{10}{27}.$

13.7. $Q(2+\displaystyle \frac{2}{5}\sqrt{5},\frac{4}{5}\sqrt{5})$

na rysunku 8.

Zbiory A, B oraz

$A\cap B$ przedstawiono

Rys. 8
\begin{center}
\includegraphics[width=161.640mm,height=77.520mm]{./KursMatematyki_PolitechnikaWroclawska_1999_2004_page61_images/image001.eps}
\end{center}
13.8. Funkcja jest parzysta. $D = [-\sqrt{8},\sqrt{8}]$; miejsca zerowe $-2\mathrm{i}2$;

maksima lokalne $\displaystyle \frac{1}{2}$ dla $x = -\sqrt{7}$ oraz dla $x = \sqrt{7}$; minimum lokalne

$y$

1

0,5

$-3  -2  -1$  1 2  3  {\it x}

$-1$

Rys. 9





78

$-4+\sqrt{8}$ dla $x=0$; funkcja rosnaca $\mathrm{w}(-\sqrt{8},-\sqrt{7})$ oraz $\mathrm{w}(0,\sqrt{7})$ ; malejaca

$\mathrm{w}(-\sqrt{7},0)$ oraz $\mathrm{w}(\sqrt{7},\sqrt{8})$ ; wypukla $\mathrm{w}(-2,2)$ ; wklesla $\mathrm{w}(-\sqrt{8},-2)$ oraz

$\mathrm{w}(2,\sqrt{8})$ ; punkty przegiecia $(-2,0), ($2, $0)$, proste $x=-\sqrt{8}$ oraz $x=\sqrt{8}$

styczne do wykresu funkcji. Wykres funkcji przedstawiono na rysunku 9.

14.1. 9.

14.2. $2\pi(3+2\sqrt{3}).$

14.3. a) $m=-\displaystyle \frac{1}{2}$; b$)m=\displaystyle \frac{4}{3}$; c$)m=01\mathrm{u}\mathrm{b}m=2\sqrt{3}.$

14.5. Elipsa $0$ równaniu $\displaystyle \frac{x^{2}}{36}+\frac{(y-1)^{2}}{4}=1$, środku $S(0,1)\mathrm{i}$ pólosiach

$\alpha=6, b=2$. Pole figury wynosi $8\pi-6\sqrt{3}.$

14.6. $+\infty.$

14.7. a) $\displaystyle \frac{1}{20};\mathrm{b}) \displaystyle \frac{7}{20}.$

14.8. $\displaystyle \frac{\sqrt{2}-\cos\alpha}{2\sin\alpha}\alpha,$

$\alpha\in (0,\displaystyle \frac{\pi}{2}).$

15.1. 12 $\mathrm{k}\mathrm{m}/\mathrm{h}, 15\mathrm{k}\mathrm{m}/\mathrm{h}, AB=27$ km.

15.2. (-00, $-\sqrt{3}]\cup(2,\infty).$

15.3. $108\sqrt{3}\mathrm{m}^{2}, \displaystyle \frac{405}{4}\sqrt{3}\mathrm{m}^{3}$

15.4. $w_{n}=1600+\displaystyle \frac{8000}{3}((\frac{203}{200})^{n-1}-1)$, pensja $\mathrm{w}$ kwietniu 2002 roku

wynosi 1806,09 $\mathrm{z}l$, średnia pensja $\mathrm{w}$ 2002 roku wynosi l827,96 $\mathrm{z}l.$

15.5. $f^{-1}(x) = \sqrt[3]{x},$

sunku 10.

$x \in$ R. Wykres funkcji $h$ przedstawiono na ry-

15.6. $\displaystyle \frac{\pi}{12}+k\frac{\pi}{3},$

$k \in$ Z.





79
\begin{center}
\includegraphics[width=130.044mm,height=59.484mm]{./KursMatematyki_PolitechnikaWroclawska_1999_2004_page63_images/image001.eps}
\end{center}
{\it y}

2

1

$-2  -1$  1 2  {\it x}

Rys. 10

15.7. $\displaystyle \frac{9}{85}\sqrt{85}.$

15.8. $f(x) = \displaystyle \frac{x^{2}-x}{x-2} = x+ 1 + \displaystyle \frac{2}{x-2}$; $D = (-\infty,0]\cup(2,\infty)$ ;

asymptota pionowa prawostronna $x = 2$; asymptota ukośna obustronna

$y = x+1$; minimum lokalne $3+2\sqrt{2}$ dla $x_{0} = 2+\sqrt{2}$; funkcja rosnaca

$\mathrm{w}$ (-00, 0) oraz $\mathrm{w}(2+\sqrt{2},\infty)$ ; malejaca $\mathrm{w}(2,2+\sqrt{2})$ ; wypukla $\mathrm{w}(2,\infty)$ ;

wklesla $\mathrm{w}(-\infty,0)$. Wykres funkcji przedstawiono na rysunku ll.
\begin{center}
\includegraphics[width=93.012mm,height=100.992mm]{./KursMatematyki_PolitechnikaWroclawska_1999_2004_page63_images/image002.eps}
\end{center}
{\it y}

6

4

{\it S}

2

$-2$  2 $x_{0}4$ 6  {\it x}

$-2$

Rys. ll





80

16.1. Cena mniejsza od poczatkowej 02,25\%.

16.2. Zbiór sklada sieZ luków czterech okregów oraz punktu (0,0) ijest

przedstawiony na rysunku 12.
\begin{center}
\includegraphics[width=91.740mm,height=77.220mm]{./KursMatematyki_PolitechnikaWroclawska_1999_2004_page64_images/image001.eps}
\end{center}
$K_{2}$  {\it y}  $K_{1}$

6

$S_{2}  S_{1}$

2

2 4  8  {\it x}

$S_{3}  S_{4}$

$K_{3}  K_{4}$

16.3. $18h^{2}\displaystyle \frac{\sin^{2}\alpha}{\sin 3\alpha},$

Rys. 12

$\alpha\in (0,\displaystyle \frac{\pi}{3}).$

16.4. 18 cm od wierzcholka kata rozwartego, $\alpha=38^{\circ}13'.$

16.5. $\sqrt{2},$

$\displaystyle \frac{\sqrt{2}}{2}.$

16.6. Dziedzina jest $\mathrm{R}$, a zbiorem wartości przedzial $[3-\sqrt{5},3+\sqrt{5}].$

16.7. $(-1,0]\displaystyle \cup\{\frac{\sqrt{17}-1}{2}\}.$

16.8. 2.

17.1. 1, $-1, \sqrt{\frac{\sqrt{17}-1}{8}}, -\sqrt{\frac{\sqrt{17}-1}{8}}$; wyraz czwarty 0 1ub $\displaystyle \frac{9-\sqrt{17}}{4}.$

17.2. $\displaystyle \frac{16}{35}\approx 0$, 457.





81

17.4. $x^{2}+(y-r^{2}-\displaystyle \frac{1}{4})^{2}=r^{2}$ Rozwiazanie istnieje dla $r>\displaystyle \frac{1}{2}.$

17.6. $[\displaystyle \frac{1}{3},\frac{\sqrt{6}}{6}).$

17.7. $\displaystyle \frac{3d(c^{2}+d^{2})}{2c^{2}}\sqrt{c^{2}-d^{2}}$

lub $\displaystyle \frac{3d(2c^{2}-d^{2})}{2c^{2}}\sqrt{c^{2}-d^{2}},$

$c>d.$

17.8. Gdy $\mathrm{w}$ równoleglościanie sa dwa wierzcholki trójścienne $0$ trzech

katach plaskich $\beta$, to objetośč wynosi $2\alpha^{3}\sqrt{\sin\frac{3}{2}\beta\sin^{3}\frac{1}{2}\beta}$. Gdy $\beta\in (\displaystyle \frac{\pi}{3},\frac{\pi}{2})$

$\mathrm{i}\mathrm{w}$ równoleglościanie sa dwa wierzcholki trójścienne $0$ trzech katach plaskich

$\pi-\beta$, to objetośč wynosi $2\alpha^{3}\sqrt{-\cos\frac{3}{2}\beta\cos^{3}\frac{1}{2}\beta},$

18.1. $\displaystyle \frac{3}{2}.$

18.2. $3x-2y+1=0.$

18.3. $V=-\displaystyle \frac{\pi}{6}l^{3}\sin 4\alpha\cos 2\alpha, \varphi=3\pi-4\alpha,$

$\alpha\in (\displaystyle \frac{\pi}{2},\frac{3\pi}{4}).$

18.4. Niech $x$ oznacza cene dlugopisu, a $y$ cene zeszytu. Dla $k\neq 2$ jest

$x=\displaystyle \frac{5k+2}{2k+2}, y=\displaystyle \frac{k}{k+2}$. Dla $k=2$ spelnionajest relacja $2x+4y=5$. Ceny

dlugopisu $\mathrm{i}$ zeszytu moga byč nastepujace:

$\left\{\begin{array}{l}
x=2,3\\
y=0,9;
\end{array}\right.$

$\left\{\begin{array}{l}
x=2,1\\
y=0,8;
\end{array}\right.$

$\left\{\begin{array}{l}
x=1,7\\
y=0,6;
\end{array}\right.$

$\left\{\begin{array}{l}
x=1,5\\
y=0,5;
\end{array}\right.$

$\left\{\begin{array}{l}
x=1,3\\
y=0,6;
\end{array}\right.$

$\left\{\begin{array}{l}
x=1,1\\
y=0,7;
\end{array}\right.$

$\left\{\begin{array}{l}
x=0,9\\
y=0,8.
\end{array}\right.$

18.5.

$[-\displaystyle \frac{\pi}{4}+k\pi,k\pi]\cup[\frac{\pi}{4}+k\pi,\frac{\pi}{2}+k\pi),$

$ k\in$ Z.

18.6. $\displaystyle \frac{496}{729}\approx 0$, 680; $0 \displaystyle \frac{496}{728\cdot 729}\approx 0$, 001.

18.7. $2-\displaystyle \frac{4}{3}\sqrt{2}.$





82

18.8. $6\sqrt{2}-4.$

19.1. $\displaystyle \frac{31}{8}.$

19.2. $12+24\sqrt{2}$ cm.

19.3. $s\leq 20.$

19.4. $\displaystyle \frac{3}{2}\sqrt{55}$ arów. Plan dzialki $\mathrm{w}$ skali 1:1000 przedstawia rysunek 13.
\begin{center}
\includegraphics[width=72.444mm,height=42.828mm]{./KursMatematyki_PolitechnikaWroclawska_1999_2004_page66_images/image001.eps}
\end{center}
20

40  30

60

Rys. 13

19.5. Wartośč najwieksza 6 dla $m=0.$

19.7.

$\left\{\begin{array}{l}
x1=-- 51\pi 2\\
y_{1}=\frac{\pi}{12},
\end{array}\right.$

$\left\{\begin{array}{l}
x_{2}=\frac{\pi}{12}\\
y2=-- 51\pi 2,
\end{array}\right.$

$\left\{\begin{array}{l}
x3=--- 71\pi 2\\
y_{3}=-\frac{11\pi}{12},
\end{array}\right.$

$\left\{\begin{array}{l}
x_{4}=-\frac{11\pi}{12}\\
y4=--- 71\pi 2^{\cdot}
\end{array}\right.$

19.8. 1, 1, $\displaystyle \frac{\sqrt{3}}{2}, \displaystyle \frac{2\sqrt{7}}{7}, \displaystyle \frac{\sqrt{42}}{7}, \displaystyle \frac{\sqrt{42}}{7}.$

20.1. $-1$, 1, 2.

20.2. $\displaystyle \frac{8}{5}(2-\sqrt{3}).$

20.3. $\displaystyle \frac{50}{81}\approx 0$, 617.





83

20.4. Cześč elipsy $0$ równaniu $\displaystyle \frac{x^{2}}{\frac{5}{3}}+\frac{(y-5)^{2}}{\frac{5}{2}}=1$ dla $y\leq 6.$

20.5. Asymptota pionowa obustronna $x=1$; asymptota pozioma lewo-

stronna $y=-1$; asymptota pozioma prawostronna $y=1$. Wykres funkcji

przedstawiono na rysunku 14.
\begin{center}
\includegraphics[width=120.348mm,height=78.840mm]{./KursMatematyki_PolitechnikaWroclawska_1999_2004_page67_images/image001.eps}
\end{center}
$y$

4

2

$-2$  0 2  4  {\it x}

Rys. 14

20.6. $(-\displaystyle \infty,-1]\cup[-\frac{1}{2},0)\cup(0,1].$

20.7. $-\sqrt{8}$ lub $\sqrt{8}.$

21.1. $\mathrm{O} 5\mathrm{c}\mathrm{m}^{2}$

21.2. $\displaystyle \frac{3}{4}.$

21.3. $\displaystyle \frac{4-\sqrt{2}}{6}, \displaystyle \frac{4+\sqrt{2}}{6}.$

21.4. Granica ciagu wynosi $\displaystyle \frac{1}{2}.$

21.5. $(-\pi+4k\pi,\pi+4k\pi),  k\in$ Z.

21.6. $-1.$





84

21.7. $r=\displaystyle \frac{2\sqrt{2}}{3}R, h=\displaystyle \frac{4}{3}R.$

21.8. $y=1-(4+2\sqrt{5})(x-2),$

$y=1-(4-2\sqrt{5})(x-2).$
\begin{center}
\includegraphics[width=116.892mm,height=81.276mm]{./KursMatematyki_PolitechnikaWroclawska_1999_2004_page68_images/image001.eps}
\end{center}
{\it y}

5

$y=p_{1}$

3

$y=p_{2}$

1

$-4  -x_{0}  -1$  1 2  $x_{0}$  4 {\it x}

Rys. 15

22.1. Wykres funkcji przedstawiono na rysunku l5, gdzie $x_{0}=1+\sqrt{5}.$

Niech $f(p)$ oznacza liczbe rozwiazań równania $4+2|x|-x^{2}=p$. Wtedy

$f(p)=$

dla

dla

dla

dla

$p>5,$

$p<4$ lub $p=5,$

$p=4,$

$4<p<5.$

22.2. 117 minut; 5475,6 $\mathrm{m}^{3}$

22.3. Średnice podstaw $6+2\sqrt{5}$ cm oraz $6-2\sqrt{5}$ cm; tworzaca 6 cm.

22.4. Gdy $\mathrm{k}\mathrm{a}\mathrm{t}\alpha$ jest ostry $\mathrm{i}\sin\alpha < \displaystyle \frac{4}{5}$, wówczas sa dwa rozwiazania:

$ P_{1}=\displaystyle \frac{8}{25}R^{2}(4\cos\alpha-3\sin\alpha)\sin\alpha$ oraz $P_{2}=\displaystyle \frac{8}{25}R^{2}(4\cos\alpha+3\sin\alpha)\sin\alpha.$





85

Jeśli $\displaystyle \sin\alpha\geq\frac{4}{5}$, to jest jedno rozwiazanie $P_{2}$, a jeśli $\alpha$ rozwarty $\displaystyle \mathrm{i}\sin\alpha<\frac{4}{5},$

to brak rozwiazań.

22.5. $(0,\displaystyle \frac{1}{4}]\cup[16,\infty).$

22.6. $\displaystyle \cos\alpha=\frac{\sqrt{7}}{14}$, obwód $\displaystyle \frac{1}{6}(9+\sqrt{12}+\sqrt{21})\alpha.$

22.7. $\displaystyle \frac{\pi}{4}+k\frac{2\pi}{3},  k\in$ Z.

22.8. $\sqrt{2}x+2y-3=0.$

23.1. Tak. $\mathrm{W}$ obu przypadkach liczba,,slów'' wynosi 210.

23.2. $-3, -1$, 1.

23.3. $\displaystyle \frac{3}{8}\alpha.$

23.4. $\displaystyle \frac{1}{12}b^{2}(3\alpha-b)\mathrm{t}\mathrm{g}\alpha.$

23.5. $[-\sqrt{5},0)\cup(1$, 2$).$

23.7. Punkt $Q(1,1).$

23.8. $(\displaystyle \frac{5\pi}{4}+2k\pi,\frac{3\pi}{2}+2k\pi)\cup(\frac{3\pi}{2}+2k\pi,\frac{7\pi}{4}+2k\pi),$

24.1. $2+\displaystyle \frac{3}{2}\sqrt{2}.$

$ k\in$ Z.

24.2. $\displaystyle \frac{7}{18}\approx 0$, 389.

24.3. Dla $ m\neq 10$ jedno rozwiazanie $x= \displaystyle \frac{m}{m-10}, y= \displaystyle \frac{m-15}{m-10}$. Dla

$m= 10$ uklad sprzeczny. Rozwiazania tworza prosta $x+2y-3=0$ bez

punktu $P(1,1).$

24.4. $\sqrt{\frac{6-6\cos\alpha}{5-4\cos\alpha}},$

$\alpha\in (0,\displaystyle \frac{\pi}{3}).$





11

Praca kontrolna

nr 3

3.1. Bez stosowania metod rachunku rózniczkowego wyznaczyč dziedzine

i zbiór wartości funkcji

$f(x)=\sqrt{2+\sqrt{x}-x}.$

3.2. Jednym $\mathrm{z}$ wierzcholków rombu $0$ polu 20 $\mathrm{c}\mathrm{m}^{2}$ jest punkt $A(6,3)$,

a jedna $\mathrm{z}$ przekatnych zawiera $\mathrm{s}\mathrm{i}\mathrm{e}\mathrm{w}$ prostej $0$ równaniu $2x+y=5.$

Wyznaczyč równania prostych, $\mathrm{w}$ których zawieraja $\mathrm{s}\mathrm{i}\mathrm{e}$ boki AB $\mathrm{i}AD.$

3.3. Stosujac zasade indukcji matematycznej, wykazač prawdziwośč wzoru

3 $(1^{5}+2^{5}+\displaystyle \ldots+n^{5})+(1^{3}+2^{3}+\ldots+n^{3})=\frac{n^{3}(n+1)^{3}}{2},$

$n\geq 1.$

3.4. Ostroslup prawidlowy trójkatny ma pole powierzchni calkowitej

$P=12\sqrt{3}\mathrm{c}\mathrm{m}^{2}$, a $\mathrm{k}\mathrm{a}\mathrm{t}$ nachylenia ściany bocznej do plaszczyzny pod-

stawy $\alpha=60^{\circ}$ Obliczyč objetośč tego ostroslupa.

3.5. Wśród trójkatów równoramiennych wpisanych $\mathrm{w}$ kolo $0$ promieniu $R$

znalez$\acute{}$č ten, który ma najwieksze pole.

3.6. Zbadač przebieg zmienności $\mathrm{i}$ narysowač wykres funkcji

$f(x)=\displaystyle \frac{1}{2}x^{2}\sqrt{5-2x}.$

3.7. $\mathrm{W}$ trapezie równoramiennym dane sa ramie $r, \mathrm{k}\mathrm{a}\mathrm{t}$ ostry przy pod-

stawie $\alpha$ oraz suma $d$ dlugości przekatnej $\mathrm{i}$ dluzszej podstawy. Wyz-

naczyč pole trapezu oraz promień okregu opisanego na tym trapezie.

Podač warunki istnienia rozwiazania. Nastepnie przeprowadzič obli-

czenia dla $\alpha=30^{\circ}, r=\sqrt{3}$ cm $\mathrm{i} d=6$ cm.

3.8. Rozwiazač nierównośč

$|\cos x+\sqrt{3}\sin x|\leq\sqrt{2},x\in[0,3\pi].$





86

24.5. (-00, $\displaystyle \frac{1}{2}]\cup[\frac{3}{2},\infty).$

24.7. Równanie prostej $k$: $x+2y-8 = 0$. Równania stycznych

tworzacych $\mathrm{z} k \mathrm{k}\mathrm{a}\mathrm{t} 45^{\circ}$: $x-3y+2+5\sqrt{2}= 0, x-3y+2-5\sqrt{2}= 0,$

$3x+y-4+5\sqrt{2}=0, 3x+y-4-5\sqrt{2}=0.$

24.8. $\alpha=3, b=32$. Styczna $y= -3x+13.$

stawiono na rysunku 16.

Wykres funkcji przed-
\begin{center}
\includegraphics[width=109.224mm,height=96.768mm]{./KursMatematyki_PolitechnikaWroclawska_1999_2004_page70_images/image001.eps}
\end{center}
{\it y}

6

4

2

0 1  3 5  7  {\it x}

Rys. 16

25.1. -$\displaystyle \frac{1}{6}, 0, \displaystyle \frac{1}{2}.$

25.2. $S(x)=x(\alpha-x), x\in(0,\alpha)$. Wartośč najwieksza $\displaystyle \frac{\alpha^{2}}{4}$ dla $x=\displaystyle \frac{\alpha}{2}.$

25.3. Rysunek l7.

25.4. Elipsa $0$ równaniu $\displaystyle \frac{(x-4)^{2}}{25}+\frac{y^{2}}{9}=1$ oraz cześč prostej $y=0$ dla

$x>9.$





87

Rys. 17

25.5. $\displaystyle \frac{5}{12}\approx 0$, 417.

25.7. $D = [1$, 5$)$ ; asymptota pionowa lewostronna $x = 5$; funkcja

rosnaca $\mathrm{w}(1,5)$ ; wypukla $\mathrm{w}(2,5)$ ; wklesla $\mathrm{w}(1,2)$ ; punkt przegiecia $P(2,1)$ ;

prosta $x=1$ styczna do wykresu funkcji. Wykres funkcji przedstawiono na

rysunku 18.
\begin{center}
\includegraphics[width=72.036mm,height=72.084mm]{./KursMatematyki_PolitechnikaWroclawska_1999_2004_page71_images/image001.eps}
\end{center}
{\it y}

3

1  {\it P}

0 1  2  {\it 5 x}

Rys. 18





88

25.8. $2\alpha\cos\alpha(1+2\cos\alpha), \alpha\in (0,\displaystyle \frac{\pi}{3}).$

26.1. 30 $(\pi+\sqrt{3})$ cm.

26.2. 213 $\mathrm{z}l\mathrm{i}85$ gr.

26.3. $4x-7y+17=0$; pole $\displaystyle \frac{10}{3}.$

26.4. $\displaystyle \frac{\sqrt{2}}{3}r^{3}\frac{(1+\sin\alpha)\cos\frac{\beta}{2}\mathrm{c}\mathrm{t}\mathrm{g}\frac{\alpha}{2}}{\cos\alpha\sqrt{-\cos\beta}},$

$\beta\in (\displaystyle \frac{\pi}{2},\pi).$

26.5. Maksimum lokalne 2 dla $x=0$. Wykres funkcji przedstawiono na

rysunku 19.
\begin{center}
\includegraphics[width=156.612mm,height=66.240mm]{./KursMatematyki_PolitechnikaWroclawska_1999_2004_page72_images/image001.eps}
\end{center}
{\it y}

3

1

$-6  -3  -1$  1 3 4  6 {\it x}

Rys. 19

26.6. $\displaystyle \frac{\pi}{3}+k\frac{\pi}{2},$

$ k\in$ Z.

26.7. Dla $\alpha\in[0$, 4). Wtedy

$g(x)=$

dla

dla

$\alpha=0,$

$0<\alpha<4.$

Dla $\alpha=3$ asymptota pionowa obustronna $x= \displaystyle \frac{1}{3}$, asymptota ukośna obu-

stronna $3x-27y+4=0.$





89

26.8. $S(y)=\pi(y+3)\sqrt{4+(y-3)^{2}},$

dla $y=0$ wynoszaca $3\pi\sqrt{13}.$

$ y\in [0$, 3$]$. Wartośč najmniejsza

27.1. $p\in[-2,2].$

27.2. $(x-\displaystyle \frac{8}{5})^{2}+(y-\frac{9}{5})^{2}=\frac{16}{5}.$

27.3. $\displaystyle \frac{\sqrt{16r^{2}\sin^{2}\alpha-d^{2}}}{2\sin\alpha},$

$4r\sin\alpha\cos\alpha<d<4r\sin\alpha.$
\begin{center}
\includegraphics[width=36.216mm,height=15.192mm]{./KursMatematyki_PolitechnikaWroclawska_1999_2004_page73_images/image001.eps}
\end{center}
$(17+\mathrm{c}\mathrm{t}\mathrm{g}^{2}\beta)^{3}$

27.4. $2S^{3}24 18(\mathrm{c}\mathrm{t}\mathrm{g}^{2}\beta-1)^{2}$

27.5. $-\infty.$

27.6. $(2k\displaystyle \pi,\frac{\pi}{6}+2k\pi],$

27.7. $\displaystyle \frac{425}{768}\approx 0$, 553.

$ k\in$ Z.

27.8. Tangens kata przeciecia linii wynosi $\displaystyle \frac{9}{37}$. Szukany zbiór pokazano

na rysunku 20.
\begin{center}
\includegraphics[width=151.380mm,height=57.300mm]{./KursMatematyki_PolitechnikaWroclawska_1999_2004_page73_images/image002.eps}
\end{center}
{\it y}

2

1

$-8  -1$  2  8  {\it x}

$-1$

$-2$

Rys. 20

28.1. $\displaystyle \frac{13}{9}.$





90

28.2. $ p\in [\displaystyle \frac{5}{4},\frac{\sqrt{7}}{2}).$

28.3. $\displaystyle \frac{d^{2}-r^{2}}{2}\mathrm{t}\mathrm{g}\frac{\alpha}{2}, r<d.$

28.4. $\displaystyle \frac{2R}{R+r}\sqrt{3Rr}.$

28.5. Trzy pierwiastki, $\mathrm{w}$ tym jeden ujemny $\mathrm{i}$ dwa dodatnie.

28.7. $\displaystyle \frac{2\pi}{3}+2k\pi$ lub $\displaystyle \frac{4\pi}{3}+2k\pi,  k\in$ Z.

28.8. Szukana krzywa jest parabola $0$ równaniu $y = 2x^{2}+ \displaystyle \frac{1}{2}$ bez

punktu $W(0,\displaystyle \frac{1}{2}).$

29.1. 15.

29.2. 307692.

29.3. $c(\cos\alpha-\cos 2\alpha), \alpha\in (0,\displaystyle \frac{\pi}{4}).$

29.4. (-00, $-\sqrt{2}]\cup(-1,0)\cup(0,2)\cup[1+\sqrt{3},\infty).$

29.5. Rysunek 2l.
\begin{center}
\includegraphics[width=91.032mm,height=66.648mm]{./KursMatematyki_PolitechnikaWroclawska_1999_2004_page74_images/image001.eps}
\end{center}
Rys. 21





91

29.6. $\displaystyle \frac{\pi}{6}+k\frac{\pi}{3}$

29.7. $-\displaystyle \frac{1}{4}.$

lub

$k\pi,$

$ k\in$ Z.

29.8. Wartośč najmniejsza 0 dla $x =$

$\displaystyle \frac{1+\sqrt{2}}{2}$ dla $x=-1+\sqrt{2}.$

30.1. $\displaystyle \frac{\sqrt{2}}{2}\frac{V_{1}^{2}}{V_{1}+V_{2}}$, gdzie $V_{1}\geq V_{2}.$

$-1$, a wartośč najwieksza

30.2. Tak, na dwa sposoby: 3800, 6l00, 8400, l0700 i l3000 zl

1000, 3400, 5800, 8200, 10600 i l3000zl.

lub

30.3. $\mathrm{s}_{\mathrm{a}}$ cztery takie okregi $\mathrm{i}$ maja równania:

$(x-\displaystyle \frac{3}{2})^{2} + (y-1-\sqrt{6})^{2} = \displaystyle \frac{9}{4}, (x-\displaystyle \frac{3}{2})^{2} + (y-1+\sqrt{6})^{2}$

-49,

$(x+1)^{2}+(y-1)^{2}=1, (x-3)^{2}+(y-1)^{2}=9.$

30.4. $\displaystyle \frac{2k}{k^{2}-1}\sin\beta, k>1.$

30.5. 7, 13.

30.6. $\alpha=-3, b=1.$

30.7. $(-\infty,0]\cup [1,\displaystyle \log_{2}\frac{3+\sqrt{17}}{2}].$

30.8. $(-\displaystyle \pi,-\frac{2\pi}{6}), (-\displaystyle \frac{\pi}{6},\frac{4\pi}{6}), (\displaystyle \frac{5\pi}{6},\pi).$

31.1. $\displaystyle \frac{136}{4807}\approx 0$, 028.

31.2. Objestośč ostroslupa wynosi $\displaystyle \frac{343}{3} \mathrm{c}\mathrm{m}^{3}$, a objetośč najmniejszej

cześci $\displaystyle \frac{61}{3}\mathrm{c}\mathrm{m}^{3}$

31.3. Uklad ma trzy rozwiazania:

$\left\{\begin{array}{l}
x_{1}=3+\sqrt{3}\\
y_{1}=3-\sqrt{3},
\end{array}\right.$

$\left\{\begin{array}{l}
x_{1}=3-\sqrt{3}\\
y_{1}=3+\sqrt{3},
\end{array}\right.$

$\left\{\begin{array}{l}
x_{1}=2+2\sqrt{2}\\
y_{1}=2-2\sqrt{2}.
\end{array}\right.$





92

31.4. - $\displaystyle \frac{1}{2}d^{2}\sin 2\alpha \mathrm{t}\mathrm{g}^{2}\frac{\alpha}{2}, \alpha\in (\displaystyle \frac{\pi}{2},\pi).$

31.6. $-\sqrt[3]{4}.$

31.7. $B_{1}(5,3), C_{1}(3,2), D_{1}(4,0)$ lub $B_{2}(10,-2), C_{2}(13,2), D_{2}(9,5).$

31.8. $D= (0,\infty)$ ; asymptota pionowa prawostronna $x=0$; minimum

lokalne 2 $\mathrm{d}\mathrm{l}\mathrm{a}x=1$; funkcja rosnaca $\mathrm{w}(1,\infty)$ ; malejaca $\mathrm{w} (0,1)$, wypukla

$\mathrm{w}(0,3)$ ; wklesla $\mathrm{w}(3,\infty)$ ; punkt przegiecia $P(3,\displaystyle \frac{4}{3}\sqrt{3})$ ; krzywa asympto-

tyczna $(\mathrm{w}+\infty) y=\sqrt{x}$. Wykres funkcji przedstawiono na rysunku 22.
\begin{center}
\includegraphics[width=116.436mm,height=54.864mm]{./KursMatematyki_PolitechnikaWroclawska_1999_2004_page76_images/image001.eps}
\end{center}
{\it y}

3

{\it P}

2

0 1  3  8  {\it x}

Rys. 22

32.1. 15 $\mathrm{d}\mathrm{n}\mathrm{i}.$

32.2. 8, $\displaystyle \frac{1}{8}.$

32.3. 65,71itra.

32.4. $ m\in (0,\displaystyle \frac{\sqrt{5}-1}{2}).$

32.5. $\displaystyle \frac{2757}{3125}\approx 0$, 882.





93

32.6. $\displaystyle \frac{\pi}{4}+k\pi$ lub $\displaystyle \frac{\pi}{12}+k\pi$

lub $\displaystyle \frac{5\pi}{12}+k\pi,  k\in$ Z.

32.7. $y=-1$ (dwa punkty wspólne), $32x+27y-5=0$ (trzy punkty

wspólne).

32.8. $R = \displaystyle \frac{1}{3}b \sqrt{\frac{9+3\cos^{2}\alpha}{2+2\cos\alpha}}$. Cosinusy katów nachylenia ścian

bocznych wynosza $\displaystyle \frac{1}{2}$

oraz $\sqrt{\frac{1-\cos\alpha}{7-\cos\alpha}}.$

33.1. Mniejszy 023,56\%.

33.2. Szukana linie stanowia dwie proste $0$ równaniach $2x+3y-1=0$

oraz $4x-y+5=0$ bez punktu ich przeciecia $P(-1,1).$

33.3. $\displaystyle \frac{7\pi}{4}.$

33.4. 2 $(7+\sqrt{19}).$
\begin{center}
\includegraphics[width=120.192mm,height=90.420mm]{./KursMatematyki_PolitechnikaWroclawska_1999_2004_page77_images/image001.eps}
\end{center}
{\it y}

4

2

$-4  -1$  0 2  4  {\it x}

Rys. 23





94

33.6. Asymptota pionowa obustronna $x=1$; asymptota pozioma lewo-

stronna $y= -\displaystyle \frac{1}{2}$; asymptota ukośna prawostronna $y= \displaystyle \frac{1}{2}x-1$; minimum

lokalne 0 d1a $x=2$. Wykres funkcji przedstawiono na rysunku 23.

33.7. $[-\displaystyle \frac{5\pi}{12},\frac{\pi}{2})\cup(\frac{\pi}{2},\frac{11\pi}{12}]\cup\{-\pi,\pi\}.$

33.8. Cosinus kata rozwarcia wynosi $\displaystyle \frac{11}{13}.$

34.1. $3+\sqrt{5}.$

34.2. $-4.$

34.3. - $\displaystyle \frac{1}{2}x^{2}+x+2$ lub - $\displaystyle \frac{1}{18}x^{2}+\frac{1}{9}x-\frac{14}{9}.$

34.4. $\vec{AB}=[8$, 4$], \vec{CD}=[-2,-1].$

34.5. $\displaystyle \frac{6}{10}.$

34.6. Odleglośč $P$ od brzegu $\mathcal{F}$ wynosi $\displaystyle \frac{\sqrt{26}}{2}-2$. Zbiór $\mathcal{F}$ przedstawiono

na rysunku 24.
\begin{center}
\includegraphics[width=66.036mm,height=60.096mm]{./KursMatematyki_PolitechnikaWroclawska_1999_2004_page78_images/image001.eps}
\end{center}
{\it y}

4

2

0 2  4  {\it x}

Rys. 24

34.7. $f^{-1}(y)=-\displaystyle \frac{1}{1+2^{y}}, D_{f^{-1}}=\mathrm{R}, W_{f^{-1}}=(-1,0).$





95

34.8. $\displaystyle \frac{1}{6}c^{3}\frac{1-3\cos^{2}\alpha}{(1+\cos^{2}\alpha)\sin\alpha}\sqrt{\cos\alpha}$. Zadanie ma sens, gdy $\displaystyle \cos\alpha<\frac{\sqrt{3}}{3}.$

35.1. 4, $-\displaystyle \frac{4}{3}, \displaystyle \frac{4}{9}, -\displaystyle \frac{4}{27},$

35.2. $\displaystyle \frac{2\sqrt{3}-1}{5}.$

35.3. $\displaystyle \frac{\alpha}{2}.$

35.5. $\displaystyle \frac{\pi}{4}+k\frac{\pi}{2}$

lub

$\displaystyle \frac{\pi}{6}+k\pi$

lub $\displaystyle \frac{5\pi}{6}+k\pi,  k\in$ Z.

35.6. $(x+4)^{2}+(y-1)^{2}=13.$

35.7. $\displaystyle \frac{2rd^{3}}{4r^{2}+d^{2}}.$

35.8. $ m\in [-\displaystyle \frac{1}{2},\frac{1}{2})\cup\{1\}.$





12

Praca kontrolna nr 4

4.1. Rozwiazač równanie $16+19+22+\ldots+x=2000$, którego lewa strona

jest suma pewnej liczby kolejnych wyrazów ciagu arytmetycznego.

4.2. Ze zbioru $\{0$, 1, $\ldots$, 9$\}$ losujemy bez zwracania pieč cyfr. Obliczyč

prawdopodobieństwo tego, $\dot{\mathrm{z}}\mathrm{e}\mathrm{m}\mathrm{o}\dot{\mathrm{z}}$ na $\mathrm{z}$ nich utworzyč liczbe podzielna

przez 5.

4.3. Zbadač, czy istnieje pochodna funkcji $f(x) = \sqrt{1-\cos x}\mathrm{w}$ punkcie

$x=0$. Wynik zilustrowač na wykresie funkcji $f(x).$

4.4. Udowodnič, $\dot{\mathrm{z}}\mathrm{e}$ dwusieczne katów wewnetrznych równolegloboku two-

$\mathrm{r}\mathrm{z}\mathrm{a}$ prostokat, którego przekatna ma dlugośč równa róznicy dlugości

sasiednich boków równolegloboku.

4.5. Rozwiazač uklad nierówności

$\left\{\begin{array}{l}
x+y\leq 3\\
\log_{y}(2^{x+1}+32)\leq 2\log_{y}(8-2^{x})
\end{array}\right.$

$\mathrm{i}$ zaznaczyč zbiór jego rozwiazań na plaszczy $\acute{\mathrm{z}}\mathrm{n}\mathrm{i}\mathrm{e}.$

4.6. Znalez$\acute{}$č równanie zbioru wszystkich punktów plaszczyzny $Oxy$, które

sa środkami okregów stycznych wewnetrznie do okregu $x^{2}+y^{2}=121$

$\mathrm{i}$ równocześnie stycznych zewnetrznie do okregu $(x+8)^{2}+y^{2}=1$. Jaka

linie przedstawia znalezione równanie? Sporzadzič staranny rysunek.

4.7. Zbadač iloczyn pierwiastków rzeczywistych równania

$m^{2}x^{2}+8mx+4m-4=0$

jako funkcje parametru $m$. Sporzadzič wykres tej funkcji.

4.8. Podstawa czworościanu ABCD jest trójkat równoboczny $ABC\mathrm{o}$ boku

$\alpha$, ściana boczna $BCD$ jest trójkatem równoramiennym prostopadlym

do plaszczyzny podstawy, a $\mathrm{k}\mathrm{a}\mathrm{t}$ plaski ściany bocznej przy wierzcholku

$A$ jest równy $\alpha$. Obliczyč pole powierzchni kuli opisanej na tym

czworościanie.





Wskazówki

do

zadań





99

Uwaga. Podano wskazówki do wszystkich zadań. Zaproponowano

pewna metodę rozwiazania $\mathrm{k}\mathrm{a}\dot{\mathrm{z}}$ dego $\mathrm{z}$ zadań, najczęściej nie jedyna

$\mathrm{i}\mathrm{z}$ pewnościa nie zawsze najprostsza.

l.l. Najpierw obliczyč oddzielnie mase stopu $\mathrm{i}$ mase srebra $\mathrm{w}$ stopie.

1.2. Pamietač $0$ wyznaczeniu dziedziny równania.

1.3. Oznaczyč nieznane wspólrzedne punktu $C$ przez $x \mathrm{i} y$, zapisač

wektory $\vec{AC}\mathrm{i}\vec{BC}$ za pomoca $x\mathrm{i}y\mathrm{i}$ korzystač $\mathrm{z}$ prostopadlości $\vec{AC}\perp\vec{BH}$

oraz $\vec{BC}\perp\vec{AH}. \mathrm{U}\dot{\mathrm{z}}$ yč iloczynu skalarnego.

1.4. Zamienič sinus na cosinus, stosujac odpowiedni wzór redukcyjny

$\mathrm{i}$ od razu przejśč do porównywania katów. Odpowied $\acute{\mathrm{z}}$ zapisač $\mathrm{w}$ postaci

jednej serii rozwiazań.

1.5. Pamietač, $\dot{\mathrm{z}}\mathrm{e} \log_{2}\alpha^{2} = 2\log_{2}|\alpha| \mathrm{i}$ skorzystač $\mathrm{z}$ symetrii wykresu

wzgledem prostej $x = 2$. Wykres otrzymač przez odbicia symetryczne

$\mathrm{i}$ translacje standardowej krzywej $y=\log_{2}x.$

1.6. Najpierw rozwazyč przypadek oczywisty $x < -6$. Dla $x > -6$

porównač odwrotności obu stron $\mathrm{i}$ przejśč do nierówności kwadratowej.

Pamietač $0$ dziedzinie nierówności.

1.7. Zastosowač twierdzenie cosinusów. Podczas wykonywania rysunku

pamietač, $\dot{\mathrm{z}}\mathrm{e}\mathrm{w}$ rzucie równoleglym zachowuje $\mathrm{s}\mathrm{i}\mathrm{e}$ równoleglośč oraz propor-

cje odcinków równoleglych.

1.8. Proste równolegle maja takie same wspólczynniki kierunkowe.

Wspólczynniki te wyznaczyč za pomoca pochodnych obu funkcji. Przy

kreśleniu wykresu krzywej $y=\sqrt{1-x}$ zwrócič uwage na lewostronne otocze-

nie punktu $x=1.$

2.1. Wystarczy pokazač, $\dot{\mathrm{z}}\mathrm{e}$ dla $\mathrm{k}\mathrm{a}\dot{\mathrm{z}}$ dego $n$ naturalnego wielomian

$y^{2n-1}+1$ jest podzielny przez dwumian $y+1.$

2.2. Kwadrat dlugości przekatnej wyrazič jako funkcje wysokości pros-

tokata wpisanego $\mathrm{w}$ trójkat. Jest to funkcja kwadratowa $\mathrm{i}$ do jej badania

nie jest potrzebna pochodna.





100

2.3. Przypadek $0 < x < 1$ jest oczywisty. Dla $x >$

logarytmy do wspólnej podstawy 3 $\mathrm{i}$ przyjač $\log_{3}x=t.$

l sprowadzič

2.4. Warunek geometryczny zapisač $\mathrm{w}$ jezyku nierówności kwadratowej

$\mathrm{z}$ parametrem.

2.5. Podstawič $x+5 =t\mathrm{i}$ badač równanie $||t|-1| =m$. Przypadki

$m<0\mathrm{i}m=0$ rozpatrzeč bezpośrednio, a dla $m>0$ korzystač $\mathrm{z}\mathrm{t}\mathrm{o}\dot{\mathrm{z}}$ samości

$(|\alpha|=b)\Leftrightarrow$($\alpha=b$ lub $\alpha=-b$) prawdziwej dla $b\geq 0.$

2.6. Pomnozyč drugie równanie przez 2 $\mathrm{i}$ nastepnie odjač oba równania

stronami. Podstawienie $x-y = t$ prowadzi do równania kwadratowego

$\mathrm{z}$ niewiadoma $t.$

2.7. Uzasadnič, $\dot{\mathrm{z}}\mathrm{e}$ szukane punkty $A\mathrm{i}B\mathrm{l}\mathrm{e}\dot{\mathrm{z}}$ a na osi $Ox\mathrm{w}$ odleglości

$5\sqrt{2}$ od środka danego okregu. Przy obliczaniu pola figury (która jest

deltoid), najprościej jest korzystač $\mathrm{z}$ podobieństwa odpowiednich trzech

trójkatów prostokatnych.

2.8. Dziedzine równania określaja warunek istnienia tangensa $\mathrm{i}$ warunek

istnienia sumy nieskończonego ciagu geometrycznego. Korzystajac ze wzoru

$1+\mathrm{t}\mathrm{g}^{2}\gamma= \displaystyle \frac{1}{\cos^{2}\gamma}$ oraz ze wzorów podanych we wskazówkach do $\mathrm{z}\mathrm{a}\mathrm{d}$. 3.8

$\mathrm{i}4.3$, przeksztalcič obie strony do równości dwóch cosinusów lub sinusów

$\mathrm{i}$ przejśč od razu do porównywania katów.

3.1. Podstawič$\sqrt{}$x$=t\mathrm{i}$ korzystač $\mathrm{z}$ wlasności funkcji kwdratowej oraz

$\mathrm{z}$ monotoniczności pierwiastka kwadratowego.

3.2. Wyznaczyč środek $S$ rombu korzystajac $\mathrm{z}$ relacji $S\in l$ oraz $\vec{AS}\perp l$

$\mathrm{t}\mathrm{z}\mathrm{n}. \vec{AS}= \alpha$ñ, gdzie ñ $= [2$, 1$]$ jest wektorem prostopadlym do prostej $l.$

$\mathrm{Z}$ warunku $\vec{AS}\perp\vec{SB}$ wynika, $\dot{\mathrm{z}}\mathrm{e}\vec{SB}= - \vec{SD}=c[1,-2]$. Dane pole rombu

pozwala wyznaczyč skalar $c \mathrm{i}$ stad od razu otrzymujemy wspól-

rzedne wierzcholków $B\mathrm{i}D.$

3.3. $\mathrm{W}$ dowodzie kroku indukcyjnego przeksztalcajac lewa strong do-

prowadzič do równości $\mathrm{z}$ prawa. Unikač dowodu metoda redukcji.





101

3.4. Pole podstawy obliczyč korzystajac $\mathrm{z}$ nastepujacego twierdzenia

$0$ zmianie pola figury plaskiej $\mathrm{w}$ rzucie prostokatnym:

{\it Pole rzutu} $prostokq_{f}tnego$ {\it figury ptaskiej jest równe polu} $tej$ {\it figury po}-

{\it mnozonemu przez cosinus} $kq_{f}tami_{G}dzy$ {\it ptaszczyznami figury} $ijej$ {\it rzutu}.

3.5. Kwadrat pola trójkata wyrazič jako funkcje wysokości trójkata.

Funkcja ta jest wielomianem. Nie mylič tego zadania $\mathrm{z}$ zagadnieniem wy-

znaczania ekstremów lokalnych.

3.6. Zauwazyč, $\dot{\mathrm{z}}\mathrm{e}$ granica lewostronna pochodnej $y'(x) \mathrm{w}$ punkcie

$x = \displaystyle \frac{5}{2}$ jest równa $-\mathrm{o}\mathrm{o}$ co oznacza, $\dot{\mathrm{z}}\mathrm{e}$ wykres jest $\mathrm{w}$ punkcie $(\displaystyle \frac{5}{2},0)$

styczny (lewostronnie) do prostej $x=\displaystyle \frac{5}{2}.$

3.7. Dla danych $ r\mathrm{i}\alpha$ najmniejsze $d$ jest wtedy, gdy krótsza podstawa

trapezu ma dlugośč 0, $\mathrm{t}\mathrm{z}\mathrm{n}$. trapez staje $\mathrm{s}\mathrm{i}\mathrm{e}$ trójkatem. Stad otrzymač

dziedzine dla $d$. Analiza otrzymanych wzorów na pole $\mathrm{i}$ promień okregu

opisanego na trapezie prowadzi do blednej dziedziny. $\mathrm{W}$ obliczeniach przyjač

jako niewiadoma polowe sumy obu podstaw $\mathrm{i}$ wyznaczyč $\mathrm{j}\mathrm{a}\mathrm{z}$ twierdzenia

Pitagorasa $\mathrm{w}$ trójkacie zawierajacym przekatna $\mathrm{i}$ wysokośč trapezu. Promień

okregu opisanego wyznaczyč stosujac twierdzenie sinusów.

3.8. Wyrazenie znajdujace $\mathrm{s}\mathrm{i}\mathrm{e}$ pod wartościa bezwzgledna przedstawič

jako $\alpha\cos(x-\alpha)$ dla odpowiedniego $\alpha \mathrm{i}\alpha$, podnieśč obie strony do kwadratu

$\mathrm{i}$ skorzystač ze wzoru 2 $\cos^{2}\gamma=1+\cos 2\gamma.$

4.1. Wyrazič $x$ przez niewiadoma liczbe skladników $n \mathrm{i}$ rozwiazač

równanie kwadratowe $\mathrm{z}$ ta niewiadoma.

4.2. Zbudowač model probabilistyczny doświadczenia, $\mathrm{t}\mathrm{j}$. określič zbiór

$\Omega \mathrm{i}$ prawdopodobieństwo $P$. Wygodniej jest obliczač prawdopodobieństwo

zdarzenia przeciwnego, $\mathrm{t}\mathrm{j}. \dot{\mathrm{z}}\mathrm{e} \mathrm{z}$ wylosowanych cyfr nie $\mathrm{m}\mathrm{o}\dot{\mathrm{z}}$ na utworzyč

liczby podzielnej przez 5.

4.3. Korzystač ze wzorów l- $\cos 2\gamma=2\sin^{2}\gamma$ oraz $\sqrt{\alpha^{2}}=|\alpha|$. Obliczyč

pochodnejednostronne bezpośrednio $\mathrm{z}$ definicji. Podczas rysowania wykresu

zwrócič uwage na otoczenie punktu $x=0.$





102

4.4. Korzystač $\mathrm{z}$ wlasności katów $\mathrm{w}$ równolegloboku. Nastepnie $\mathrm{z}$ przy-

stawania odpowiednich (trzech) trójkatów wywnioskowač, $\dot{\mathrm{z}}\mathrm{e}$ przekatna

utworzonego prostokata jest równolegla do dluzszych boków równoleglo-

boku.

4.5. Podczas rozwiazywania drugiej nierówności rozpatrzyč przypadki

$ y\in (0,1)$ oraz $ y\in (1,\infty)$. Po podstawieniu $2^{x}=t$ przejśč do nierówności

kwadratowych zmiennej $t$. Nie zapomnieč $0$ dziedzinie ukladu $\mathrm{i}$ szczególo-

wym ustaleniu, które punkty brzegu naleza do rozwazanego zbioru.

4.6. Korzystajac $\mathrm{z}$ wlasności okregów stycznych zewnetrznie $\mathrm{i}$ wewne-

trznie, wykazač, $\dot{\mathrm{z}}\mathrm{e}$ suma odleglości rozwazanych punktów od środków obu

danych okregów jest stala $\mathrm{i}$ wynosi 12. Nastepnie zastosowač geometryczna

definicje elipsy.

4.7. Dziedzina funkcji jest określona przez warunki istnienia dwóch

pierwiastków rzeczywistych równania (ale niekoniecznie róznych). $\mathrm{U}\dot{\mathrm{z}}$ yč

wzorów Viète'a. Do rózniczkowania przedstawič otrzymana funkcje jako

sume funkcji potegowych. Ze wzgledu na postač dziedziny nie $\mathrm{m}\mathrm{o}\dot{\mathrm{z}}$ na mówič

$0$ asymptocie ukośnej prawostronnej. Pamietač, $\dot{\mathrm{z}}\mathrm{e},$,przyleganie'' wykresu

funkcji do asymptoty pionowej $\mathrm{m}\mathrm{o}\dot{\mathrm{z}}\mathrm{e}$ byč inne $\mathrm{z}\mathrm{k}\mathrm{a}\dot{\mathrm{z}}$ dej strony tej asymptoty.

4.8. Zauwazyč, $\dot{\mathrm{z}}\mathrm{e}$ czworościan ma plaszczyzne symetrii, która prze-

chodzi przez wierzcholki $A, D\mathrm{i}$ środek krawedzi $BC$. Środek kuli opisanej

$\mathrm{l}\mathrm{e}\dot{\mathrm{z}}\mathrm{y}$ na tej plaszczy $\acute{\mathrm{z}}\mathrm{n}\mathrm{i}\mathrm{e}\mathrm{w}$ punkcie przeciecia $\mathrm{s}\mathrm{i}\mathrm{e}$ prostej prostopadlej do pod-

stawy wystawionej $\mathrm{w}$ środku okregu opisanego na podstawie $\mathrm{z}$ symetralna

krawedzi $AD$. Dziedzine kata $\alpha$ ustalič poprzez rozwazania geometryczne

($\mathrm{k}\mathrm{a}\mathrm{t}\alpha$ musi byč wiekszy od jego rzutu prostokatnego na podstawe).

5.1. Korzystač $\mathrm{z} \mathrm{t}\mathrm{o}\dot{\mathrm{z}}$ samości $(|\alpha|\leq b) \Leftrightarrow (-b\leq\alpha\leq b)$.

narysowač za pomoca translacji standardowej krzywej $y=|x|.$

Zbiór A

5.2. Wyznaczyč najpierw $\sin\alpha+\cos\alpha \mathrm{i}$ stosowač wzór na sume sześcia-

nów.

5.3. Rozwazyč rodzine prostych przechodzacych przez punkt P. Proste

te przecinajac dana parabole, wyznaczaja cieciwy. Napisač uklad rów-

nań, który spelniaja końce cieciw i nie rozwiazujac go, wyznaczyč środki





103

tych cieciw ze wzorów Viète'a. Zwrócič uwage na dziedzine (szukana krzywa

nie jest cala parabola!).

5.4. Wyznaczyč dziedzine $\mathrm{i}$ podnieśč obie strony równania do kwadratu,

otrzymujac proste równanie równowazne wyjściowemu.

5.5. Korzystajac ze schematu Bernoulliego, obliczyč odpowiednie praw-

dopodobieństwa dla obu strzelców. Dla drugiego strzelca najpierw obliczyč

prawdopodobieństwo zdarzenia przeciwnego.

5.6. Jeśli $R$ jest nieduzo wieksze $\mathrm{n}\mathrm{i}\dot{\mathrm{z}}r$, to środki kulek $\mathrm{l}\mathrm{e}\dot{\mathrm{z}}$ a na przekroju

osiowym walca, gdyz kulki zajmuja $\mathrm{m}\mathrm{o}\dot{\mathrm{z}}$ liwie najnizsze polozenie. Najwiek-

sze $R$ (przy ustalonym $r$), przy którym kulki przyjmuja takie polozenie jest

wtedy, gdy trzecia kulka (tj. $\mathrm{l}\mathrm{e}\dot{\mathrm{z}}\mathrm{a}\mathrm{c}\mathrm{a}$ najwyzej) bedzie styczna $\mathrm{z}$ pierwsza

(tj. $\mathrm{l}\mathrm{e}\dot{\mathrm{z}}\mathrm{a}\mathrm{c}\mathrm{a}$ na dnie naczynia). To odpowiada warunkowi $r<R\displaystyle \leq r+\frac{r\sqrt{3}}{2}.$

Narysowač przekrój osiowy walca, zaznaczajac na nim przekroje kulek.

Korzystač $\mathrm{z}$ twierdzenia $0$ okregach stycznych zewnetrznie $\mathrm{i}\mathrm{z}$ twierdzenia

Pitagorasa.

5.7. Przypadek $m=0$ rozpatrzeč oddzielnie. Dla $m\neq 0$ badač mono-

tonicznośč rozwazajac znak pochodnej. Prowadzi to do warunków, przy

których odpowiedni trójmian kwadratowy $\mathrm{w}$ liczniku pochodnej jest nieu-

jemny na R. Pamietač, $\dot{\mathrm{z}}\mathrm{e}$ funkcja jest rosnaca $\mathrm{w}$ pewnym przedziale takze

wtedy, gdy jej pochodna jest nieujemna $\mathrm{i}$ zeruje $\mathrm{s}\mathrm{i}\mathrm{e}\mathrm{w}$ skończonej liczbie

punktów.

5.8. Przekatne $\mathrm{w}$ rombie sa równocześnie dwusiecznymi jego katów.

Jeśli wiec dwa wektory sa równej dlugości, to ich suma wyznacza kierunek

dwusiecznej kata miedzy tymi wektorami.

6.1. Zauwazyč, $\dot{\mathrm{z}}\mathrm{e} x = 1$ spelnia równanie, a dla $x \neq 1$ przejśč do

porównania wykladników. Pamietač $0$ wyznaczeniu dziedziny równania.

6.2. Równanie stycznej do okregu $(x-x_{0})^{2}+(y-y_{0})^{2}=r^{2}\mathrm{w}$ punkcie

$A(x_{1},y_{1})\mathrm{l}\mathrm{e}\dot{\mathrm{z}}$ acym na tym okregu ma postač

$(x_{1}-x_{0})(x-x_{0})+(y_{1}-y_{0})(y-y_{0})=r^{2}$





104

6.3. Korzystač ze wzoru na sume cosinusów oraz ze wzorów reduk-

cyjnych. Przeksztalcač tylko lewa strong $\mathrm{i}$ doprowadzič do równości $\mathrm{z}$ prawa.

6.4. Przyjač, $\dot{\mathrm{z}}\mathrm{e}$ iloraz $q$ ciagujest wiekszy od l. Zauwazyč, $\dot{\mathrm{z}}\mathrm{e}$ środkowy

wyraz ciagu jest równy 2 $\mathrm{i}$ ulozyč równanie $\mathrm{z}$ niewiadoma $q.$

6.5. Oznaczyč przez $A_{i}$ zdarzenie polegajace na wylosowaniu $\mathrm{z}$ pierwszej

urny $i$ kul bialych, $i=0$, 1, 2, 3, $\mathrm{i}$ zastosowač wzór na prawdopodobieństwo

calkowite.

6.6. Zauwazyč, $\dot{\mathrm{z}}\mathrm{e}$ bryle $\mathrm{m}\mathrm{o}\dot{\mathrm{z}}$ na podzielič na dwie (identyczne) polowy

odpowiednia plaszczyzna prostopadla do osi obrotu, a $\mathrm{k}\mathrm{a}\dot{\mathrm{z}}$ da polowa sklada

$\mathrm{s}\mathrm{i}\mathrm{e}$ ze stozka oraz stozka ścietego $0$ wspólnej podstawie.

6.7. Wyznaczyč tylko miejsca zerowe pochodnej $\mathrm{i}$ porównač wartości

funkcji $\mathrm{w}$ tych punktach zjej wartościami na końcach przedzialu. Nie tracič

czasu na wyznaczanie ekstremów lokalnych.

6.8. Maksymalna wartośč $k$ jest osiagana wtedy, gdy trójkat jest równo-

ramienny. Stad ustalič dziedzine $k$. Korzystač $\mathrm{z}$ podobieństwa odpowied-

nich trójkatów $\mathrm{i}\mathrm{z}$ nastepujacej wlasności trójkata prostokatnego:

{\it Suma} $przyprostokq_{f}tnych$ {\it jest równa sumie średnic okrGgów}

{\it wpisanego} $i$ {\it opisanego}.

7.1. Podstawič $3^{x}=t\mathrm{i}$ korzystač $\mathrm{z}\mathrm{t}\mathrm{o}\dot{\mathrm{z}}$ samości podanej we wskazówce

do zadania 5.1.

7.2. Wykorzystač zwiazek wspólrzednych punktu ijego obrazu w powino-

wactwie prostokatnym oraz zwiazek pól figury i jej obrazu w tym przek-

sztalceniu.

7.3. Liczba $k$-elementowych podzbiorów zbioru $n$-elementowego wynosi

$\left(\begin{array}{l}
n\\
k
\end{array}\right)$. Nie pominač zbioru pustego, który jest podzbiorem $\mathrm{k}\mathrm{a}\dot{\mathrm{z}}$ dego zbioru.

7.4. Korzystač $\mathrm{z}$ twierdzenia $0$ czworokacie opisanym na okregu. Do

wyznaczenia $\sin 15^{\mathrm{O}}$ oraz $\cos 15^{\circ}$ nie korzystač $\mathrm{z}$ tablic, lecz przeksztal-

cič wyrazenie $\mathrm{t}\mathrm{a}\mathrm{k}$, aby otrzymač funkcje kata $30^{\circ}$ (por. wskazówka do

$\mathrm{z}\mathrm{a}\mathrm{d}$. 3.8$).$





105

7.5. Rozwiazań, dla których $x=y$, szukač takze wśród nieskończenie

wielu rozwiazań ukladu dla przypadku $m=3.$

7.6. Rozwazyč oddzielnie przedzialy $[-\displaystyle \frac{\pi}{2},0$) oraz $[0,\displaystyle \frac{\pi}{2}],\ \mathrm{w}$ których

$\sin x$ ma staly znak, a funkcja cosinus jest monotoniczna. Zbiór rozwiazań

zaznaczyč na wykresie jako podzbiór osi odcietych.

7.7. Korzystač $\mathrm{z}$ zalezności miedzy polami $\mathrm{i}$ objetościami figur $\mathrm{i}$ bryl

podobnych.

7.8. Skonstruowač model probabilistyczny, czyli określič zbiór $\Omega$ oraz

prawdopodobieństwo $P$. Oznaczyč przez $A_{\mathrm{I}}, A_{\mathrm{I}\mathrm{I}}$ zdarzenia polegajace na

$\mathrm{t}\mathrm{y}\mathrm{m}, \dot{\mathrm{z}}\mathrm{e}$ oba tomy odpowiednio I, II powieści znajduja $\mathrm{s}\mathrm{i}\mathrm{e}$ obok siebie

$\mathrm{i}$ we wlaściwej kolejności. Interesuja nas zdarzenia $A_{\mathrm{I}} \cap A_{\mathrm{I}\mathrm{I}}$ oraz

$A_{\mathrm{I}}\cup A_{\mathrm{I}\mathrm{I}}$. Prawdopodobieństwo tego drugiego obliczyč, stosujac wzór na

prawdopodobieństwo sumy dwóch dowolnych zdarzeń.

8.1. Pamietač $0$ warunku istnienia sumy nieskończonego ciagu geome-

trycznego.

8.2. Skladnik $\left(\begin{array}{l}
11\\
i
\end{array}\right)3^{i/3}2^{(11-i)/2}$ bedzie liczba calkowita wtedy $\mathrm{i}$ tylko

wtedy, gdy $i$ bedzie podzielne przez 3, a $11-i$ bedzie parzyste.

8.3. Korzystač $\mathrm{z}$ parzystości funkcji. Narysowač $\mathrm{w}$ przedziale $[0,\infty$)

wykres funkcji $g(x)=x^{2}-2x-3\mathrm{i}$ zastosowač geometryczna interpretacje

nalozenia na $\mathrm{n}\mathrm{i}\mathrm{a}$ wartości bezwzglednej.

8.4. Najpierw określič dziedzine nierówności. Napisač $x+1=\log_{2}2^{x+1}$,

podstawič $2^{x}=t\mathrm{i}$ przejśč do nierówności kwadratowej.

8.5. Do obliczenia objetości potrzebny jest tylko tangens kata nachyle-

nia ściany bocznej do podstawy $ t=\mathrm{t}\mathrm{g}\alpha$. Warunek podany $\mathrm{w}$ zadaniu zapi-

sač $\mathrm{w}$ postaci równania $\mathrm{z}$ niewiadoma $t. \mathrm{U}\dot{\mathrm{z}}$ yč $\mathrm{t}\mathrm{o}\dot{\mathrm{z}}$ samości $\displaystyle \frac{1}{\cos^{2}\alpha}=1+\mathrm{t}\mathrm{g}^{2}\alpha.$

8.6. $K\mathrm{a}\mathrm{t}$ prosty $\mathrm{m}\mathrm{o}\dot{\mathrm{z}}\mathrm{e}\mathrm{s}\mathrm{i}\mathrm{e}$ znajdowač wjednym $\mathrm{z}$ trzech podanych wierz-

cholków trójkata. Zastosowač iloczyn skalarny.





106

8.7. Po wymnozeniu $\mathrm{n}\mathrm{a}$ krzyz'' skorzystač ze wzoru na iloczyn si-

nusów, doprowadzič do równości dwóch cosinusów $\mathrm{i}$ stad od razu przejśč do

porównania katów. Nie zapomnieč $0$ uwzglednieniu dziedziny.

8.8. Korzystač $\mathrm{z}$ twierdzenia $0$ stosunku pól figur podobnych. Zauwazyč

$\mathrm{i}$ uzasadnič, $\dot{\mathrm{z}}\mathrm{e}$ suma skal podobieństwa trzech mniejszych trójkatów jest

równa l.

9.1. Pole powierzchni powiekszonej kuli jest l,44 razy wieksze od pola

kuli wyjściowej.

9.2. Napisač równanie peku prostych przechodzacych przez punkt $P$

$\mathrm{i}$ majacych ujemny wspólczynnik kierunkowy $m$ (dlaczego?). Wyznaczyč

wspólrzedne punktów $A, B$ przeciecia $\mathrm{s}\mathrm{i}\mathrm{e}$ tych prostych $\mathrm{z}$ osiami ukladu

oraz środków odcinków AB $\mathrm{w}$ zalezności od $m$. Eliminujac parametr $m$

zapisač równanie krzywej $\mathrm{w}$ postaci $y=f(x).$

9.3. Po podstawieniu $3^{x} = t$ zadanie sprowadza $\mathrm{s}\mathrm{i}\mathrm{e}$ do znalezienia

warunków, przy których równanie kwadratowe $\mathrm{z}$ niewiadoma $t$ ma dwa rózne

pierwiastki dodatnie.

9.4. Rozwazyč przekrój czworościanu plaszczyzna symetrii. Korzystajac

$\mathrm{z}$ podobieństwa odpowiednich dwóch trójkatów $\mathrm{w}$ tym przekroju, wykazač,

$\dot{\mathrm{z}}\mathrm{e}$ stosunek promieni kuli opisanej do wpisanej wynosi 3. Stad ob1iczyč

wysokośč czworościanu, a nastepnie kolejno krawed $\acute{\mathrm{z}} \mathrm{i}$ objetośč.

9.5. Dla $x<-3$ lewa stronajest dodatnia, a prawa ujemna $\mathrm{i}$ nierównośč

jest oczywiście spelniona. Dla $x > -3, x \neq 3$, obie strony sa dodat-

nie. Pomnozyč je przez $(x+3)|x-3|$. Po uproszczeniu dostajemy prosta

nierównośč, do której zastosowač $\mathrm{t}\mathrm{o}\dot{\mathrm{z}}$ samośč $(|\alpha|\leq b)\Leftrightarrow(-b\leq\alpha\leq b).$

9.6. Przyjač $k \geq 1$ oraz oznaczyč przez $\alpha$ polowe wiekszego $\mathrm{z}$ katów

ostrych trójkata. Stosunek dwusiecznych wyrazič za pomoca $k$ oraz funkcji

trygonometrycznych kata $\alpha \mathrm{i}$ przeksztalcič $\mathrm{t}\mathrm{a}\mathrm{k}$, aby wystapil tylko tg $\alpha.$

Wartośč tg $\alpha$ obliczyč, wiedzac, $\dot{\mathrm{z}}\mathrm{e}$ tg $2\alpha=k.$





107

9.7. Przedstawič funkcje $\mathrm{w}$ postaci $f(x)=1+\displaystyle \frac{4}{x-2}+\frac{8}{(x-2)^{2}}\mathrm{i}\mathrm{w}$ tej

postaci $\mathrm{j}\mathrm{a}$ rózniczkowač. Zauwazyč, $\dot{\mathrm{z}}\mathrm{e}$ wykres jest wyra $\acute{\mathrm{z}}\mathrm{n}\mathrm{i}\mathrm{e}$ asymetryczny

wzgledem asymptoty $x=2.$

9.8. Napisač równanie stycznej $\mathrm{w}$ punkcie $(x_{0},f(x_{0}))$. Po podstawieniu

do niego wspólrzednych punktu $A$ otrzymujemy równanie trzeciego stop-

nia $\mathrm{z}$ niewiadoma $x_{0}$. Równanie to ma trzy pierwiastki wymierne. Przez

bezpośrednie sprawdzenie wystarczy znalez$\acute{}$č $\mathrm{d}\mathrm{w}\mathrm{a}$. Trzeci $\mathrm{m}\mathrm{o}\dot{\mathrm{z}}$ na obliczyč,

wiedzac, $\dot{\mathrm{z}}\mathrm{e}$ iloczyn pierwiastków wyraza $\mathrm{s}\mathrm{i}\mathrm{e}$ przez wyraz wolny $\mathrm{i}$ wspól-

czynnik przy najwyzszej potedze $x_{0}$. Podczas rysowania wykresu korzystač

$\mathrm{z}$ nieparzystości funkcji $f \mathrm{i}\mathrm{j}\mathrm{u}\dot{\mathrm{z}}$ wyznaczonych stycznych. Dodatkowe ba-

danie nie jest potrzebne.

10.1. Patrz wskazówka do zadania 3.3.

10.2. $K\mathrm{a}\mathrm{t}$ widzenia odcinka AB $\mathrm{z}$ punktu $C$ niewspólliniowego $\mathrm{z}A\mathrm{i}B$

to $\mathrm{k}\mathrm{a}\mathrm{t}\angle ACB$. Dany $\mathrm{w}$ zadaniu $\mathrm{k}\mathrm{a}\mathrm{t}$ zaznaczyč na przekroju osiowym stozka.

Objetośč wyrazič przez $l$ oraz funkcje trygonometryczne wielokrotności kata

$\alpha$. Uwaznie stosowač wzory redukcyjne $\mathrm{i}$ nie bač $\mathrm{s}\mathrm{i}\mathrm{e}$ napisač znaku minus

we wzorze na objetośč.

10.3. Patrz wskazówka do $\mathrm{z}\mathrm{a}\mathrm{d}$. 3.1.

10.4. Najpierw określič model probabilistyczny $\mathrm{t}\mathrm{j}. \Omega \mathrm{i} P$. Zdarze-

nie określone $\mathrm{w}$ treści zadania jest suma czterech rozlacznych (dlaczego?)

zdarzeń $A_{i}, i=1$, 2, 3, 4, gdzie $A_{i}$ oznacza otrzymanie trzech kart $\mathrm{w}i$-tym

kolorze $\mathrm{i}$ jednej $\mathrm{z}$ innego koloru. $P(A_{i})$ obliczyč bezpośrednio, korzystajac

$\mathrm{z}$ tego, $\dot{\mathrm{z}}\mathrm{e}P$ jest prawdopodobieństwem klasycznym.

10.5. Wyznaczyč dziedzine nierówności. Podstawič $\log_{2}x=t \mathrm{i}$ korzy-

stajac $\mathrm{z}$ monotoniczności funkcji logarytmicznej $0$ podstawie $\displaystyle \frac{1}{3}$, przejśč do

nierówności wymiernej.

10.6. Skorzystač ze wskazówki do zadania 6.2 $\mathrm{i}$ wyrazič wspólrzedne

punktów stycznościjako funkcje zmiennej $r$. Wygodniej jest szukač wartości

najwiekszej kwadratu pola, który jest funkcja wymierna.





13

Praca kontrolna

nr 5

5.1. Narysowač na plaszczy $\acute{\mathrm{z}}\mathrm{n}\mathrm{i}\mathrm{e}$ zbiór

$A=\{(x,y)$ : $||x|-y|\leq 1,$

$-1\leq x\leq 2\}$

$\mathrm{i}$ znalez$\acute{}$č punkt zbioru $A\mathrm{l}\mathrm{e}\dot{\mathrm{z}}\mathrm{a}\mathrm{c}\mathrm{y}$ najblizej punktu $P(0,4).$

5.2. Obliczyč $\sin^{3}\alpha+\cos^{3}\alpha,\ \mathrm{m}\mathrm{a}\mathrm{j}_{s}\mathrm{a}\mathrm{c}$ dane $\displaystyle \sin 2\alpha=\frac{1}{4},\ \alpha\in(0,2\pi)$.

5.3. Rozwazmy rodzine prostych przechodzacych przez punkt $P(0,-1)$

$\mathrm{i}$ przecinajacych parabole $y = \displaystyle \frac{1}{4}x^{2} \mathrm{w}$ dwóch punktach. Wyznaczyč

równanie środków powstalych $\mathrm{w}$ ten sposób cieciw paraboli. Sporza-

dzič rysunek $\mathrm{i}$ opisač otrzymana krzywa.

5.4. Rozwiazač równanie

$\sqrt{x+\sqrt{x^{2}-x+2}}-\sqrt{x-\sqrt{x^{2}-x+2}}=4.$

5.5. Dwaj strzelcy strzelaja do tarczy. Pierwszy trafia $\mathrm{z}$ prawdopodo-

bieństwem $\displaystyle \frac{2}{3} \mathrm{w} \mathrm{k}\mathrm{a}\dot{\mathrm{z}}$ dym strzale $\mathrm{i}$ wykonuje 4 strza1y, a drugi trafia

$\mathrm{z}$ prawdopodobieństwem $\displaystyle \frac{1}{3} \mathrm{i}$ oddaje 8 strza1ów. Który ze strze1ców

ma wieksze prawdopodobieństwo uzyskania co najmniej trzech trafień,

jeśli wyniki kolejnych strzalów sa wzajemnie niezalezne?

5.6. Do naczynia $\mathrm{w}$ ksztalcie walca $0$ promieniu podstawy $R$ wrzucono trzy

jednakowe kulki $0$ promieniu $r$, gdzie $2r<2R\leq r(2+\sqrt{3})$. Okazalo

$\mathrm{s}\mathrm{i}\mathrm{e}, \dot{\mathrm{z}}\mathrm{e}$ plaska pokrywa naczynia jest styczna do kulki znajdujacej $\mathrm{s}\mathrm{i}\mathrm{e}$

najwyzej $\mathrm{w}$ naczyniu. Obliczyč wysokośč naczynia.

5.7. Dlajakich wartości parametru $m$ funkcja

$f(x)=\displaystyle \frac{x^{3}}{mx^{2}+6x+m}$

jest określona $\mathrm{i}$ rosnaca na calej prostej rzeczywistej.

5.8. Dany jest trójkat $0$ wierzcholkach $A(-2,1),\ B(-1,-6),\ C(2,5)$.

Za pomoca rachunku wektorowego obliczyč cosinus kata miedzy dwu-

sieczna kata $A\mathrm{i}$ środkowa boku $BC$. Sporzadzič rysunek.





108

10.7. Korzystajac $\mathrm{z}\mathrm{t}\mathrm{o}\dot{\mathrm{z}}$ samości podanej we wskazówce do zadania 2.5,

wyznaczyč $y\mathrm{z}$ drugiego równania, otrzymujac dwa przypadki $y=\displaystyle \frac{3}{4}x$ oraz

$y= \displaystyle \frac{3}{4}x-5\mathrm{i}$ podstawič kolejno do pierwszego równania. Krzywa opisana

pierwszym równaniem jest symetryczna wzgledem osi rzednych, a drugie

równanie przedstawia dwie proste równolegle.

10.8. Przenieśč wszystkie wyrazy na lewa strong, $\mathrm{u}\dot{\mathrm{z}}$ yč wzoru podanego

we wskazówce do zadania 4.3, a nastepnie wzoru na sume sinusów.

ll.l. Jedna $\mathrm{z}$ figur jest trójkat, którego pole stanowi ósma cześč pola

calego trójkata (dlaczego?). Stad wywnioskowač, $\displaystyle \dot{\mathrm{z}}\mathrm{e}\alpha=\frac{\pi}{6}.$

11.2. Plaszczyzna przechodzaca przez jedna $\mathrm{z}$ krawedzi bocznych

$\mathrm{i}$ środek kuli jest plaszczyzna symetrii $\mathrm{i}$ przecina podstawy graniastoslupa

wzdluz ich wysokości. Wybierajac odpowiedni trójkat, obliczyč szukana

wysokośč. (Mozna $\mathrm{t}\mathrm{e}\dot{\mathrm{z}}$ argumentowač inaczej zauwazajac, $\dot{\mathrm{z}}\mathrm{e}$ środek kuli

opisanej oraz wierzcholki podstawy tworza czworościan foremny, którego

wysokośč stanowi polowe szukanej wysokości graniastoslupa.)

11.3. Najpierw wyznaczyč ekstrema lokalne funkcji $g(x) = \displaystyle \frac{x}{1+x^{2}}.$

Poniewaz $f(x) = \alpha g(x)$, wiec dobór $\alpha$ jest natychmiastowy. Trzeba tylko

pamietač, $\dot{\mathrm{z}}\mathrm{e}\alpha \mathrm{m}\mathrm{o}\dot{\mathrm{z}}\mathrm{e}$ byč takze ujemne $\mathrm{i}$ wtedy maksimum $f$ jest osiagane

tam, gdzie $g$ ma minimum.

11.4. Wyznaczyč dziedzine (warunek istnienia sumy nieskończonego

ciagu geometrycznego) $\mathrm{i}$ pamietač, $\dot{\mathrm{z}}\mathrm{e} \mathrm{w}$ niej mianownik sumy po lewej

stronie jest dodatni. Pomnozyč obie strony przez ten mianownik $\mathrm{i}$ sko-

rzystač ze wzoru podanego we wskazówce do zadania 3.8.

11.5. $\mathrm{U}\dot{\mathrm{z}}$ ycie indukcji matematycznej nie jest potrzebne. Przeksztalcič

prawa strong piszac 2 $\left(\begin{array}{l}
i\\
2
\end{array}\right) = i(i-1) = i^{2}-i \mathrm{i}$ pogrupowač skladniki

kwadratowe oddzielnie, a liniowe zsumowač jako kolejne liczby naturalne.

11.6. Oznaczyč środek jednego $\mathrm{z}$ rozwazanych okregów przez $A(x,y).$

Stycznośč do osi $Ox$ oznacza, $\dot{\mathrm{z}}\mathrm{e}$ promień tego okregu wynosi $|y|$, czyli





109

odleglośč $A$ od środka $S$ danego okregu wynosi $|AS| = 2+ |y|$ (stycz-

nośč zewnetrzna!). Odleglośč $|AS|$ wyrazič bezpośrednio za pomoca $x\mathrm{i}y$

$\mathrm{i}$ tak otrzymač szukane równanie. Nazwač otrzymana krzywa. Pamietač,

$\dot{\mathrm{z}}\mathrm{e}$ środki okregów $\mathrm{l}\mathrm{e}\dot{\mathrm{z}}$ a na zewnatrz danego okregu.

11.7. Przyjač $\log_{3}m=t\mathrm{i}$ korzystač ze wzorów Viète'a.

11.8. Najpierw wyznaczyč dziedzine nierówności. Przypadek $x<0$jest

oczywisty, a dla $x>0\mathrm{m}\mathrm{o}\dot{\mathrm{z}}$ na podnieśč obie strony do kwadratu, nastepnie

pomnozyč przez $x^{2}$, otrzymujac nierównośč kwadratowa.

12.1. Narysowač krzywe $y = \sqrt{x-3}$ oraz $y = 4-x \mathrm{i}$ za pomoca

rysunku uzasadnič, $\dot{\mathrm{z}}\mathrm{e}$ równanie to ma tylko jeden pierwiastek oraz $\dot{\mathrm{z}}\mathrm{e}\mathrm{l}\mathrm{e}\dot{\mathrm{z}}\mathrm{y}$

$\mathrm{w}$ przedziale (3, 4). Ob1iczyč go przez podniesienie obu stron równania do

kwadratu.

12.2. Napisač rozklad $w(x)$ na czynniki $\mathrm{i}$ podstawič do obu stron

równości $x=-1.$

12.3. Niech $A_{i}$ oznacza zdarzenie polegajace na wypadnieciu $i$ oczek na

kostce. Wówczas $\Omega =  A_{1}\cup \cup A_{6} \mathrm{i}$ skladniki sa parami rozlaczne. Za-

stosowač wzór na prawdopodobieństwo calkowite. Dla wygody obliczyč

najpierw prawdopodobieństwo zdarzenia przeciwnego do określonego

$\mathrm{w}$ zadaniu, polegajacego na $\mathrm{t}\mathrm{y}\mathrm{m}, \dot{\mathrm{z}}\mathrm{e}$ rzuty moneta nie daly $\dot{\mathrm{z}}$ adnego orla.

12.4. Zauwazyč, $\dot{\mathrm{z}}\mathrm{e}$ sa cztery takie okregi; dwa $\mathrm{w}$ I čwiartce $\mathrm{i}$ pojednym

$\mathrm{w}$ II $\mathrm{i}$ IV čwiartce. Środek szukanego okregu ma $\mathrm{w}$ I čwiartce postač $S(r,r),$

$\mathrm{w}$ II čwiartce $S(-r,r)$, a $\mathrm{w}$ IV $S(r,-r)$, gdzie $r>0$jest nieznanym promie-

niem rozwazanego okregu. $\mathrm{W} \mathrm{k}\mathrm{a}\dot{\mathrm{z}}$ dym przypadku niewiadoma $r$ wyzna-

czyč ze wzoru na odleglośč punktu od danej prostej, $\mathrm{t}\mathrm{j}. 3x+4y=12.$

12.5. Poprowadzič wysokości sasiednich ścian bocznych do ich wspólnej

krawedzi. Tworza one wraz $\mathrm{z}$ przekatna podstawy trójkat równoramienny,

którego $\mathrm{k}\mathrm{a}\mathrm{t}$ przy wierzcholku wynosi $ 2\alpha (\mathrm{z}$ twierdzenia $0$ trzech prostopa-

dlych), a wysokośč jest równa $d.$

12.6. Znajac $P \mathrm{i} s$, obliczamy wysokośč trapezu, a nastepnie jego

przekatna $\mathrm{z}$ twierdzenia Pitagorasa, gdyz rzut prostokatny przekatnej na





110

podstawe ma dlugośč $\displaystyle \frac{s}{2}$. Ramie trapezu wyznaczamy $\mathrm{z}$ podobieństwa odpo-

wiednich trójkatów. Przekatna trapezu nie $\mathrm{m}\mathrm{o}\dot{\mathrm{z}}\mathrm{e}$ przekroczyč średnicy okre-

gu. Stad wynika warunek rozwiazalności zadania.

12.7. Dla $p=-1\mathrm{i}p=2$ ukladjest nieoznaczony $\mathrm{t}\mathrm{z}\mathrm{n}$. ma nieskończenie

wiele rozwiazań. Rozwiazania te tworza dwie proste. Dla $\mathrm{k}\mathrm{a}\dot{\mathrm{z}}$ dego $\mathrm{z}$ po-

zostalych $p$ uklad ma jedno rozwiazanie, które przy zmieniajacym $\mathrm{s}\mathrm{i}\mathrm{e}p$

przebiega trzecia prosta. Na tych trzech prostych znalez$\acute{}$č punkty $0$ podanej

wlasności.

12.8. Badač kwadrat pola powierzchni jako funkcje $y$. Jest ona wielo-

mianem. Nie mylič postawionego pytania $\mathrm{z}$ zagadnieniem wyznaczania

ekstremów lokalnych. Wartośč najmniejsza jest osiagana $\mathrm{w}$ punkcie $y=0,$

a nie $\mathrm{w}$ minimum lokalnym. (Wynik ten klóci $\mathrm{s}\mathrm{i}\mathrm{e}\mathrm{z}$ intuicja, gdyz $\mathrm{w}$ tym

przypadku tworzaca stozka jest najdluzsza.)

13.1. Korzystajac ze wzoru na cosinus róznicy katów przedstawič lewa

strong $\mathrm{w}$ postaci $\alpha\cos(x-\varphi)$ dla odpowiednio dobranego kata $\varphi.$

13.2. Wektor [l2, 5] jest wektorem normalnym prostej $l$, czyli wektor

$\vec{v}= [5,-12]$ jest do niej równolegly (por. wskazówka do zadania 31.7.).

$\mathrm{Z}$ definicji iloczynu skalarnego wynika, $\dot{\mathrm{z}}\mathrm{e}$ liczba $\displaystyle \frac{|\vec{AB}0\vec{v}|}{|\vec{v}|}$ jest dlugościa

rzutu prostokatnego odcinka $AB$ na prosta $l.$

13.3. Wyznaczyč dziedzine (nie zapomnieč $0$ warunku $2^{m}\neq 7$) $\mathrm{i}\mathrm{u}\dot{\mathrm{z}}$ yč

wzorów Viète'a. Wykres $f$ otrzymač ze standardowej krzywej $y=2^{m}$ przez

translacje $\mathrm{i}$ odbicie symetryczne.

13.4. Oznaczyč przez $B_{i}$ zdarzenie polegajace na $\mathrm{t}\mathrm{y}\mathrm{m}, \dot{\mathrm{z}}\mathrm{e}$ pierwszy

strzelec trafil $i$ razy, $i=0$, 1, 2, a przez $C_{j}$ zdarzenie, $\dot{\mathrm{z}}\mathrm{e}$ drugi strzelec

trafil $j$ razy, $j = 0$, 1, 5. Wtedy rozwazane zdarzenie ma postač

$(B_{0}\cap C_{3})\cup(B_{1}\cap C_{2})\cup(B_{2}\cap C_{1})$. Korzystač ze schematu Bernoulliego

$\mathrm{i}$ niezalezności par zdarzeń $B_{i}, C_{j}.$

13.5. Oddzielnie rozwazyč $n$ parzyste $\mathrm{i}$ nieparzyste. Zapisač warunki

na sumy wyrazów tego ciagu $\mathrm{i}$ eliminujac niewiadome wyrazič $\alpha_{2}$ oraz $\alpha_{3}$





111

tylko przez róznice tego ciagu.

$\alpha_{2}\alpha_{3}=48.$

Nastepnie obliczyč te róznice z równania

13.6. Poprowadzič dwusieczna $AD \mathrm{i}$ wyznaczyč $|BC|$, korzystajac

$\mathrm{z}$ podobieństwa trójkatów $ABC \mathrm{i} ADC$. Dalej korzystač $\mathrm{z}$ twierdzenia

sinusów oraz ze wzoru na promień okregu wpisanego $\mathrm{w}$ trójkat $r=\displaystyle \frac{S}{p}.$

13.7. Zbiór $A$ wyznaczyč korzystajac ze wskazówki do zadania 5.1.

Uzasadnič (podnoszac obie strony do kwadratu), $\dot{\mathrm{z}}\mathrm{e}$ krzywa $0$ równaniu

$y=\sqrt{4x-x^{2}}$ nie jest lukiem paraboli lecz pólokregiem. Obliczyč odleglośč

punktu $P$ od $\mathrm{k}\mathrm{a}\dot{\mathrm{z}}$ dej $\mathrm{z}$ trzech cześci brzegu zbioru $A\cap B\mathrm{i}$ porównač je.

13.8. Korzystač $\mathrm{z}$ parzystości funkcji. $\mathrm{Z}$ postaci dziedziny wynika, $\dot{\mathrm{z}}\mathrm{e}$

funkcja nie $\mathrm{m}\mathrm{o}\dot{\mathrm{z}}\mathrm{e}$ mieč asymptot (dlaczego?). Granica lewostronna pochod-

nej $\mathrm{w}$ punkcie $x = \sqrt{8}$ wynosi $-\infty$, wiec prosta $x = \sqrt{8}$ jest styczna do

wykresu funkcji $f(x)$. Dla sporzadzenia wykresu dobrač odpowiednia jed-

nostke na obu osiach ukladu wspólrzednych.

14.1. Korzystač ze wskazówki do $\mathrm{z}\mathrm{a}\mathrm{d}$. 7.3. Otrzymane wyrazenie jest

ciagiem rosnacym $\mathrm{i}$ zadanie $\mathrm{m}\mathrm{o}\dot{\mathrm{z}}\mathrm{e}$ mieč co najwyzej jedno rozwiazanie.

14.2. Uzasadnič, $\dot{\mathrm{z}}\mathrm{e}$ promienie kolejnych okregów tworza ciag geome-

tryczny, którego iloraz jest równy pierwszemu wyrazowi ciagu.

14.3. Korzystač $\mathrm{z}$ rachunku wektorowego $\mathrm{i}$ iloczynu skalarnego. Za-

uwazyč, $\dot{\mathrm{z}}\mathrm{e}$ wszystkie proste $\mathrm{z}$ danej rodziny przechodza przez punkt $P(1,1).$

14.4. Stosowač wzór na tangens róznicy katów.

tej $\mathrm{t}\mathrm{o}\dot{\mathrm{z}}$ samości $\mathrm{i}$ funkcji $f(x).$

Wyznaczyč dziedzine

14.5. Skorzystač ze wskazówki do zadania 7.2. Rozwazana figura jest

róznica odcinka danego kola, wyznaczonego przez oś odcietych, oraz jego

obrazu $\mathrm{w}$ powinowactwie określonym $\mathrm{w}$ zadaniu.

14.6. Zastosowač podana nierównośč $\mathrm{i}$ sprowadzič logarytmy do pod-

stawy 2. Nastepnie wykazač, $\dot{\mathrm{z}}\mathrm{e}$ iloraz rozwazanego ciagu geometrycznego

jest wiekszy od l $\mathrm{i}$ stad od razu otrzymač odpowied $\acute{\mathrm{z}}.$

14.7. Patrz wskazówka do zadania 7.8.





112

14.8. Rozwazyč przekrój plaszczyzna przechodzaca przez przekatna

podstawy $\mathrm{i}$ wierzcholek ostroslupa. $\mathrm{Z}$ twierdzenia $0$ odcinkach stycznych

do kuli poprowadzonych $\mathrm{z}$ ustalonego punktu wynika, $\dot{\mathrm{z}}\mathrm{e}$ punkt styczności

kuli $\mathrm{z}$ krawedzia boczna $\mathrm{l}\mathrm{e}\dot{\mathrm{z}}\mathrm{y}\mathrm{w}$ odleglości $\displaystyle \frac{\alpha}{2}$ od wierzcholka podstawy. Ko-

rzystajac $\mathrm{z}$ tej obserwacji obliczyč krawed $\acute{\mathrm{z}}$ boczna na dwa sposoby $\mathrm{i}$ stad

wyznaczyč promień kuli.

15.1. Oznaczyč przez x odleglośč miejscowości, a przez y predkośč

drugiego rowerzysty. Ulozyč uklad dwóch równań z tymi niewiadomymi,

zapisujac informacje podane w treści zadania.

15.2. Określič dziedzine nierówności. Przypadek $x<0$ jest oczywisty.

Dla $x>0$ podnieśč obie strony do kwadratu, po pomnozeniu przez $x^{2}$ otrzy-

mujemy nierównośč dwukwadratowa.

15.3. Pole powierzchni dachu obliczyč z twierdzenia podanego we wska-

zówce do zadania 3.4. Objetośč dachu ob1iczyč, dzie1ac bry1e p1aszczyznami

pionowymi na dwa ostroslupy i graniastoslup.

15.4. Wyrazič $w_{n+1}$ przez $w_{n}$, korzystajac $\mathrm{z}$ danych zadania. Uzasadnič,

$\dot{\mathrm{z}}\mathrm{e}$ ciag $\triangle_{n}=w_{n+1}-w_{n}$ jest ciagiem geometrycznym $0$ ilorazie 1,015 oraz

$\dot{\mathrm{z}}\mathrm{e}w_{n}=w_{1}+\triangle_{1}+ +\triangle_{n-1}$. Pensja $\mathrm{w}$ kwietniu 2002 roku jest równa $w_{6}.$

15.5. Funkcja $f(x)$ jest rosnaca $\mathrm{i}$ zbiorem jej wartości jest R. Stad

$f^{-1}(x)=\sqrt[3]{x}$ jest określona na $\mathrm{R}$, ajej wykres jest odbiciem symetrycznym

wykresu $f(x)$ wzgledem prostej $0$ równaniu $y = x$. Wykres funkcji $h(x)$

$\mathrm{w}$ przedziale $(0,\infty)$ otrzymač przez translacje cześci wykresu funkcji $f^{-1}(x)$

$\mathrm{i}$ korzystajac $\mathrm{z}$ parzystości funkcji $h(x).$

15.6. Wyznaczyč dziedzine, pomnozyč obie strony przez mianownik,

przejśč za pomoca wzoru redukcyjnego do równości dwóch cosinusów i stad

od razu do porównania katów. Wynik zapisač w postaci jednej serii.

15.7. Patrz wskazówka do zadania 5.8.





113

15.8. Wyznaczyč dziedzine funkcji; nie pominač punktu $x=0$. Sume

wyrazów nieskończonego ciagu geometrycznego zapisač $\mathrm{w}$ postaci

$f(x)=x+1+\displaystyle \frac{2}{x-2}, \mathrm{z}$ której od razu odczytač równania asymptot (uwazač

na dziedzine). Ta postač jest takze wygodna do rózniczkowania (nie jest

celowe stosowanie wzoru na pochodna ilorazu). Podczas rysowania wykresu

jeszcze raz uwazač na dziedzine.

16.1. Oznaczyč przez $x, y, z$ ceny odpowiednio poczatkowa, po obnizce

$\mathrm{i}$ po podwyzce. Wyrazič $y$ przez $x$ oraz $z$ przez $y\mathrm{i}\mathrm{w}$ konsekwencji $z$ przez $x.$

16.2. Punkt $(0,0)$ rozpatrzyč oddzielnie. Zauwazyč, $\dot{\mathrm{z}}\mathrm{e}$ zbiór jest syme-

tryczny wzgledem obu osi ukladu wspólrzednych $\mathrm{i}$ wyznaczyč (oraz opisač)

najpierw $\mathrm{t}\mathrm{e}$ cześč, która $\mathrm{l}\mathrm{e}\dot{\mathrm{z}}\mathrm{y}\mathrm{w}$ I čwiartce.

16.3. Wysokości ścian bocznych oraz odcinek laczacy środki dwóch

odpowiadajacych im krawedzi podstawy tworza trójkat równoramienny

$0$ kacie przy wierzcholku $ 2\alpha \mathrm{i}$ podstawie $\displaystyle \frac{\alpha}{2}$ (dlaczego?). Podstawa tego

trójkata nie przechodzi przez spodek wysokości ostroslupa. Przez porów-

nanie tego trójkata $\mathrm{z}$ jego rzutem prostokatnym na podstawe ostroslupa,

określič dziedzine dla kata $\alpha.$

16.4. $\mathrm{M}\mathrm{o}\dot{\mathrm{z}}$ na odciač narozniki zawierajace wierzcholki katów ostrych

równolegloboku lub zawierajace wierzcholki katów rozwartych. Nalezy wy-

brač ($\mathrm{i}$ uzasadnič odpowiednim rachunkiem) to ciecie, które daje romb

$0$ wiekszym polu, $\mathrm{t}\mathrm{j}$. odciač narozniki zawierajace katy rozwarte. Punkt,

przez który nalezy poprowadzič ciecie wyznaczyč $\mathrm{z}$ twierdzenia cosinusów.

16.5. $\sqrt{2}$ zamienič na potege $0$ podstawie 2 $\mathrm{i}$ wykladniku ulamkowym,

skorzystač $\mathrm{z}$ regul dzialań na potegach, przejśč do porównania wykladników

$\mathrm{i}$ podstawič $\log_{2}x=t.$

16.6. Wyrazenie $\mathrm{w}$ mianowniku zapisač $\mathrm{w}$ postaci $3+\alpha\cos(x-\alpha)$ (por.

wskazówka do zadania 3.8). Wykazač, $\dot{\mathrm{z}}\mathrm{e} |\alpha| <3$, co oznacza, $\dot{\mathrm{z}}\mathrm{e}$ dziedzina

funkcji $f(x)$ jest $\mathrm{R}$, a mianownik jest dodatni $\mathrm{w}$ R. Wartośč najmniejsza

funkcji $f(x)$ jest osiagana $\mathrm{w}$ tym punkcie, $\mathrm{w}$ którym mianownik jest naj-

wiekszy $\mathrm{i}$ na odwrót. $\mathrm{U}\dot{\mathrm{z}}$ ycie pochodnej jest zbedne.





114

16.7. Oddzielnie rozpatrzyč przypadek $p = 0$. Dla $p \neq 0$ równanie

dwukwadratowe ma dokladnie dwa rózne pierwiastki, gdy odpowiadajace

mu równanie kwadratowe ma wyróznik równy zeru $\mathrm{b}\mathrm{a}\mathrm{d}\acute{\mathrm{z}}$ ma wyróznik do-

datni $\mathrm{i}$ jednocześnie jeden $\mathrm{z}$ pierwiastków ujemny.

16.8. Napisač równanie stycznej $\mathrm{w}$ punkcie $A$, korzystajac $\mathrm{z}$ pochod-

nej funkcji. Styczna ta przecina wykres funkcji $\mathrm{w}$ innym punkcie $B$. Przy

wyznaczaniu tego punktu otrzymujemy równanie trzeciego stopnia, które

ze wzgledu na stycznośč $\mathrm{w}$ punkcie $A$ ma pierwiastek podwójny 3 $\mathrm{i}$ tylko

trzeba znalez$\acute{}$č trzeci pierwiastek (por. wskazówka do $\mathrm{z}\mathrm{a}\mathrm{d}$. 9.8).

17.1. Najpierw rozwazyč przypadek, gdy iloraz równy zeru, $\mathrm{t}\mathrm{z}\mathrm{n}.$

$\cos x = 0$. Wtedy wszystkie dalsze wyrazy ciagu sa równe zeru. Jeśli

$\cos x \neq 0$, to liczby $\sin x, \cos x, \sin 2x$ tworza ciag geometryczny wtedy

$\mathrm{i}$ tylko wtedy, gdy kwadrat liczby środkowej jest iloczynem liczb skrajnych,

$\mathrm{t}\mathrm{z}\mathrm{n}$. gdy $\cos x=2\sin^{2}x$. Podstawič $\cos x=t.$

17.2. Losowe dzielenie druzyn na grupy interpretowač jako permuta-

cje numerów wszystkich druzyn, $\mathrm{t}\mathrm{j}$. liczb 1, 2, 16, gdzie ko1ejne czwórki

wyrazów permutacji wyznaczaja sklad kolejnych grup. Pamietač $0$ określeniu

na poczatku modelu probabilistycznego, $\mathrm{t}\mathrm{j}. \Omega \mathrm{i}P.$

17.3. Zauwazyč, $\dot{\mathrm{z}}\mathrm{e}$ dane wyrazenie $\mathrm{m}\mathrm{o}\dot{\mathrm{z}}$ na zapisač $\mathrm{w}$ postaci

$[(x^{2}+x+1)^{3}+x^{3}] - [x^{6}+2x^{3}+1] \mathrm{i}$ stosujac wzór na sume sześcianów,

wykazač, $\dot{\mathrm{z}}\mathrm{e}$ oba skladniki tej sumy dziela $\mathrm{s}\mathrm{i}\mathrm{e}$ przez $(x+1)^{2}$

17.4. $\mathrm{Z}$ symetrii figury wynika, $\dot{\mathrm{z}}\mathrm{e}$ środek $S$ okregu stycznego $\mathrm{w}$ dwóch

punktach do danej paraboli $\mathrm{l}\mathrm{e}\dot{\mathrm{z}}\mathrm{y}$ na osi rzednych, $\mathrm{t}\mathrm{z}\mathrm{n}$. mamy $S(0,y_{0})$, przy

czym $y_{0} > r$. Stycznośč oznacza, $\dot{\mathrm{z}}\mathrm{e}$ równanie kwadratowe $\mathrm{z}$ niewiadoma

rzedna $y$ punktu styczności powinno mieč dodatni pierwiastek podwójny, co

jest spelnione, gdy wyróznik tego równaniajest równy zeru, a wspólczynnik

przy niewiadomej $y$ jest ujemny.

17.5. Dbač $0$ logiczna poprawnośč zapisu calego dowodu indukcyjnego.

$\mathrm{W}$ dowodzie kroku indukcyjnego przeksztalcač tylko lewa strong, pamietajac,

$\dot{\mathrm{z}}\mathrm{e}$ zwiekszenie $n\mathrm{o}1$ powoduje pojawienie $\mathrm{s}\mathrm{i}\mathrm{e}$ dwóch dodatkowych skladników.





115

17.6. Ustalič dziedzine nierówności $\mathrm{i}$ korzystač $\mathrm{z}$ wlasności logarytmu

$0$ podstawie mniejszej od l (dziedzina jest zawarta $\mathrm{w}$ odcinku $(0,1)$).

17.7. Uzasadnič, $\dot{\mathrm{z}}\mathrm{e} \angle ASD$ jest prosty. To oznacza, $\dot{\mathrm{z}}\mathrm{e}$ dane $c= |AD|$

oraz $d = |AS|, d\sqrt{2} \geq c > d$, jednoznacznie określaja trójkat $ASD$

oraz promień okregu $r \mathrm{i}\mathrm{k}\mathrm{a}\mathrm{t} \angle DAB$ trapezu. Wyznaczyč $r$ oraz dlugości

odcinków, na które punkt styczności dzieli $AD. \mathrm{M}\mathrm{o}\dot{\mathrm{z}}$ liwe sa dwa przy-

padki: albo podzial $AB$, liczac od wierzcholka $A$, jest $\mathrm{w}$ stosunku 2:1, a1bo

$\mathrm{w}$ stosunku 1:2. $\mathrm{W}$ drugim przypadku $\mathrm{m}\mathrm{o}\dot{\mathrm{z}}\mathrm{e}\mathrm{s}\mathrm{i}\mathrm{e}$ zdarzyč, $\dot{\mathrm{z}}\mathrm{e}\mathrm{k}\mathrm{a}\mathrm{t}$ przy wierz-

cholku $B$ jest rozwarty.

17.8. $\mathrm{M}\mathrm{o}\dot{\mathrm{z}}$ liwe sa dwa przypadki: albo $\mathrm{w}$ jednym $\mathrm{z}$ wierzcholków pod-

stawy wszystkie katy plaskie kata trójściennego wychodzacego $\mathrm{z}$ tego wierz-

cholka sa ostre, albo wszystkie sa rozwarte. $\mathrm{W}$ obu przypadkach poprowa-

dzič plaszczyzne symetrii przez ten wierzcholek $\mathrm{i}$ przeciwlegly wierzcholek

drugiej podstawy oraz przez odpowiednie przekatne obu podstaw. Niezna-

$\mathrm{n}\mathrm{a}$ wysokośč równoleglościanu obliczamy $\mathrm{z}$ twierdzenia $0$ trzech prostopa-

dlych. Obliczamy najpierw wysokośč rombu tworzacego $\mathrm{k}\mathrm{a}\dot{\mathrm{z}}\mathrm{d}\mathrm{a}$ ściane równo-

leglościanu, nastepnie odleglośč spodka wysokości równoleglościanu od kra-

wedzi podstawy $\mathrm{i}$ wreszcie $\mathrm{z}$ twierdzenia Pitagorasa wysokośč równoleglo-

ścianu. $\mathrm{W}$ obu przypadkach obliczenia sa analogiczne.

18.1. Zarówno licznik jak $\mathrm{i}$ mianownik sa sumami skończenie wielu

(ustalič $\mathrm{i}\mathrm{l}\mathrm{u}$) wyrazów dwóch ciagów geometrycznych. Obliczyč te sumy

$\mathrm{i}$ podzielič licznik $\mathrm{i}$ mianownik przez $2^{2n}$

18.2. Szukana prosta przechodzi przez środek odcinka $AB$ ijest prosto-

padla do danej prostej. Stad od razu $\mathrm{m}\mathrm{o}\dot{\mathrm{z}}$ na napisač równanie tej prostej.

18.3. Patrz wskazówka do zadania l0.2.

18.4. Oznaczyč przez $x, y$ ceny dlugopisu $\mathrm{i}$ zeszytu. Wtedy

$x >y\geq 0$, 50. Ulozyč uklad dwóch równań $\mathrm{z}$ niewiadomymi $x, y\mathrm{i}$ para-

metrem $k$. Oddzielnie rozwazyč przypadek $k = 2$, dla którego uklad jest

nieoznaczony, oraz $k\neq 2$, gdy uklad ma jedno rozwiazanie. $\mathrm{W}$ pierwszym

przypadku wybrač wszystkie $k$, dla których $x\mathrm{i}y$ wyrazaja $\mathrm{s}\mathrm{i}\mathrm{e}\mathrm{w}$ pelnych

dziesiatkach groszy $\mathrm{i}$ spelniaja warunek $x>y\geq 0$, 50. Odpowiedni rysunek

ulatwia znalezienie wszystkich rozwiazań.





116

18.5. Korzystač ze wzoru $\displaystyle \sin 2\gamma=\frac{2\mathrm{t}\mathrm{g}\gamma}{1+\mathrm{t}\mathrm{g}^{2}\gamma} \mathrm{i}$ podstawič $\mathrm{t}\mathrm{g}\gamma=t.$

18.6. Stosowač schemat Bernoulliego. Drugie pytanie dotyczy praw-

dopodobieństwa warunkowego rozwazanego zdarzenia przy warunku, $\dot{\mathrm{z}}\mathrm{e}$ co

najmniej jedna $\dot{\mathrm{z}}$ arówka jest dobra.

18.7. Poniewaz promień szukanego okregu jest bardzo many, nalezy

przyjač $\mathrm{n}\mathrm{a}$ rysunku $\mathrm{d}\mathrm{u}\dot{\mathrm{z}}$ ajednostke $\mathrm{i}$ narysowač tylko odpowiedni $\mathrm{l}\mathrm{u}\mathrm{k}$ danego

okregu. $\mathrm{W}$ obliczeniach korzystač $\mathrm{z}$ twierdzenia $0$ okregach stycznych ze-

wnetrznie oraz $\mathrm{z}$ twierdzenia Pitagorasa $\mathrm{w}$ trójkacie, którego wierzcholkami

sa środki obu okregów oraz rzut prostokatny środka malego okregu na od-

cinek $AS.$

18.8. Pole $\mathrm{i}$ objetośč ostroslupa ścietego wyrazičjako funkcje dlugości $x$

krawedzi górnej podstawy tego ostroslupa, $0 < x <$ l. Korzystač

$\mathrm{z}$ twierdzenia $0$ stosunku pól $\mathrm{i}$ objetości figur $\mathrm{i}$ bryl podobnych. Wyznaczyč

miejsce zerowe pochodnej znalezionej funkcji, zbadač znak pochodnej $\mathrm{i}$ uza-

sadnič, $\dot{\mathrm{z}}\mathrm{e}\mathrm{w}$ tym punkcie funkcja osiaga nie tylko ekstremum lokalne, ale

takze wartośč najwieksza.

19.1. Wektory $\vec{BM}$ oraz $\vec{BK}$ wyrazič za pomoca wektorów $\vec{AB}, \vec{BC}$

oraz $\vec{CD}$. Majac wspólrzedne tych wektorów, od razu obliczyč pole $\triangle KMB.$

19.2. Napisač zwiazek przekatnej prostopadlościanu $\mathrm{z}$ dlugościami jego

krawedzi $\mathrm{i}$ stad obliczyč nieznana róznice ciagu. Odrzucič to rozwiazanie,

które prowadzi do ujemnych dlugości krawedzi.

19.3. Zbiór $A$ wyznaczyč korzystajac ze wskazówki do zadania 13.7

($\mathrm{w}$ cześci dotyczacej zbioru $B \mathrm{w}$ tamtej wskazówce). Dobrač $s \mathrm{t}\mathrm{a}\mathrm{k}$, aby

prosta $B_{s}$ miala jeden punkt wspólny ze zbiorem $A$ (co to znaczy geome-

trycznie?) $\mathrm{i}$ stad od razu podač odpowied $\acute{\mathrm{z}}.$

19.4. Korzystajac $\mathrm{z}$ nierówności trójkata, ustalič, które pary odcinków

moga byč podstawami trapezu. $\mathrm{s}_{\mathrm{a}}$ trzy takie $\mathrm{m}\mathrm{o}\dot{\mathrm{z}}$ liwości (spośród sześciu).

$\mathrm{W}$ dwóch przypadkach pole trapezu jest mniejsze od ll arów. Wykazač to,

zauwazajac, $\dot{\mathrm{z}}\mathrm{e}$ wysokośč trapezu jest mniejsza od $\mathrm{k}\mathrm{a}\dot{\mathrm{z}}$ dego $\mathrm{z}$ jego ramion.

$\mathrm{W}$ trzecim przypadku nalezy obliczyč pole $\mathrm{i}$ wykazač, $\dot{\mathrm{z}}\mathrm{e}$ przekracza ono

ll arów.





117

19.5. Ustalič dziedzine dla parametru $m\mathrm{i}$ stosowač wzory Viète'a. Za

pomoca pochodnej wykazač, $\dot{\mathrm{z}}\mathrm{e}$ kwadrat róznicy pierwiastków, jako funkcja

zmiennej $m$, jest malejacy $\mathrm{w}$ wyznaczonej dziedzinie.

19.6. $\mathrm{W}$ dowodzie kroku indukcyjnego uwaznie stosowač reguly dzialań

na potegach.

19.7. Sumy $\mathrm{z}$ lewych stron przeksztalcič na iloczyny. Stad wywniosko-

wač, $\dot{\mathrm{z}}\mathrm{e}\sin(x+y)=1$, czyli $\mathrm{z}$ drugiego równania $\displaystyle \cos(x-y)=\frac{1}{2}\mathrm{i}$ od razu

przejśč do ukladów równań liniowych $\mathrm{z}$ niewiadomymi $x\mathrm{i}y.$

19.8. Oznaczyč $|CD| = |CA| = |CB| = \alpha$. Poniewaz $CD \perp AD$

oraz $CD \perp BD$, wiec dwie ściany boczne sa prostopadle do podstawy

$ABD$ (bedacej trójkatem równobocznym) $\mathrm{i}$ tworza ze soba $\mathrm{k}\mathrm{a}\mathrm{t} 60^{\circ} K\mathrm{a}\mathrm{t}$

miedzy podstawa $\mathrm{i}$ plaszczyzna $ABC$ wyznaczamy, przecinajac czworościan

plaszczyzna symetrii $CDE$, gdzie $E$ jest środkiem $AB$. Dla wyznaczenia

kata miedzy plaszczyzna $ABC\mathrm{i}$ plaszczyzna $BCD$ ($\mathrm{i}$ równocześnie $ACD$)

nalezy $\mathrm{z}$ wierzcholka $D$ poprowadzič wysokośč czworościanu na ściane $ABC.$

Poniewaz $\triangle BCD$ jest prostokatny $\mathrm{i}$ równoramienny, wiec $\mathrm{z}$ twierdzenia

$0$ trzech prostopadlych wynika, $\dot{\mathrm{z}}\mathrm{e}$ spodek tej wysokości $\mathrm{l}\mathrm{e}\dot{\mathrm{z}}\mathrm{y} \mathrm{w}$ środku

okregu opisanego na trójkacie $ABC$. Wyrazič $\mathrm{t}\mathrm{e}$ wysokośč przez $\alpha,$

obliczajac objetośč czworościanu na dwa sposoby $\mathrm{i}$ stad od razu wyznaczyč

sinus rozwazanego kata.

20.1. Oddzielnie rozpatrzeč $m=0\mathrm{i}m=2$. Dla pozostalych parametrów

$m$ korzystač $\mathrm{z}$ faktu, $\dot{\mathrm{z}}\mathrm{e}$ oś symetrii paraboli przechodzi przezjej wierzcholek.

20.2. Uzasadnič, $\dot{\mathrm{z}}\mathrm{e}$ środek danej kuli $\mathrm{i}$ środek kuli wpisanej $\mathrm{w}$ dana

bryle tworza przekatna sześcianu $0$ krawedzi równej promieniowi kuli wpisa-

nej. Rozwazyč przekrój plaszczyzna symetrii zawierajaca środki obu $\mathrm{k}\mathrm{u}\mathrm{l}.$

20.3. Określič model probabilistyczny, $\mathrm{t}\mathrm{j}. \Omega \mathrm{i} P$. Obliczyč prawdo-

podobieństwo zdarzenia przeciwnego, korzystajac ze wzoru na prawdopodo-

bieństwo sumy dwóch dowolnych zdarzeń.



\end{document}