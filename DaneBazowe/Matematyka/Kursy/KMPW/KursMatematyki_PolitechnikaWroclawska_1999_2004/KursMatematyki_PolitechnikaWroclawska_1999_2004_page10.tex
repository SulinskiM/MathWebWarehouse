\documentclass[a4paper,12pt]{article}
\usepackage{latexsym}
\usepackage{amsmath}
\usepackage{amssymb}
\usepackage{graphicx}
\usepackage{wrapfig}
\pagestyle{plain}
\usepackage{fancybox}
\usepackage{bm}

\begin{document}

14

Praca kontrolna nr 6

6.1. Rozwiazač równanie

$\displaystyle \mathrm{x}^{\log_{2}(2x-1)+\log_{2}(x+2)}=\frac{1}{x^{2}}.$

6.2. Styczna do okregu $x^{2}+y^{2}-4x-2y=5\mathrm{w}$ punkcie $M(-1,2)$, prosta

$l0$ równaniu $24x+5y-12=0$ oraz oś $Ox$ tworza trójkat. Obliczyč

pole tego trójkata. Sporzadzič rysunek.

6.3. Udowodnič prawdziwośč $\mathrm{t}\mathrm{o}\dot{\mathrm{z}}$ samości

$\displaystyle \cos\alpha+\cos\beta+\cos\gamma=4\cos\frac{\alpha+\beta}{2}\cos\frac{\beta+\gamma}{2}\cos\frac{\gamma+\alpha}{2},$

gdzie $\alpha, \beta, \gamma$ sa katami ostrymi, których suma wynosi $\displaystyle \frac{\pi}{2}.$

6.4. Dlugości krawedzi prostopadlościanu $0$ objetości $V = 8$ tworza ciag

geometryczny, a stosunek dlugości przekatnej prostopadlościanu do

najdluzszej $\mathrm{z}$ przekatnych jego ścian wynosi $\displaystyle \frac{3}{4}\sqrt{2}$. Obliczyč pole

powierzchni calkowitej prostopadlościanu.

6.5. $\mathrm{Z}$ urny zawierajacej siedem kul czarnych $\mathrm{i}$ trzy biale wybrano losowo

trzy kule $\mathrm{i}$ przelozono do drugiej, pustej urny. Jakie jest prawdopodo-

bieństwo wylosowania kuli bialej $\mathrm{z}$ drugiej urny?

6.6. Prostokat obraca $\mathrm{s}\mathrm{i}\mathrm{e}$ wokól swojej przekatnej. Obliczyč objetośč pow-

stalej bryly, jeśli przekatna ma dlugośč $d$, a $\mathrm{k}\mathrm{a}\mathrm{t}$ pomiedzy przekatna

$\mathrm{i}$ dluzszym bokiem ma miare $\alpha$. Sporzadzič odpowiedni rysunek.

6.7. Wyznaczyč najwieksza $\mathrm{i}$ najmniejsza wartośč funkcji

$f(x)=x^{5/2}-10x^{3/2}+40x^{1/2}$

w przedziale [1, 5].

6.8. Stosunek promienia okregu wpisanego w trójkat prostokatny do pro-

mienia okregu opisanego na tym trójkacie jest równy k. W jakim

stosunku środek okregu wpisanego w ten trójkat dzieli dwusieczna

kata prostego? Określič dziedzine dla parametru k.
\end{document}
