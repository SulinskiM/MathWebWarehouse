\documentclass[a4paper,12pt]{article}
\usepackage{latexsym}
\usepackage{amsmath}
\usepackage{amssymb}
\usepackage{graphicx}
\usepackage{wrapfig}
\pagestyle{plain}
\usepackage{fancybox}
\usepackage{bm}

\begin{document}

118

20.4. Stosowač wzór na odleglośč punktu od prostej. Pamietač, $\dot{\mathrm{z}}\mathrm{e}$

rozwazamy tylko punkty wewnatrz danego trójkata. Nazwač wyznaczona

krzywa.

20.5. Rozwazyč przypadki $x > 1 \mathrm{i}x < 1 \mathrm{i}$ uprościč wzór określajacy

funkcje. Podczas rysowania wykresu pamietač $0$ dziedzinie funkcji.

20.6. Napisač $\displaystyle \frac{1}{x^{2}} = |x|^{-2}\mathrm{i}$ rozwazyč przypadki $|x| =1, |x| < 1$ oraz

$|x|>1$. Nie stosowač bezpośrednio definicji wartości bezwzglednej.

20.7. Warunek zadania oznacza, $\dot{\mathrm{z}}\mathrm{e}$ rozwazane styczne maja wspólczyn-

niki kierunkowe $+1\mathrm{l}\mathrm{u}\mathrm{b}-1$. Obliczyč pochodna funkcji $f$, przyrównač jej

wartośč bezwzgledna do l $\mathrm{i}$ rozwiazač otrzymane równanie niewymierne.

20.8. Oznaczyč $x=|AD|$ oraz $y=|AE|$. Ze stosunku pól obliczyč $xy,$

a $\mathrm{z}$ twierdzenia sinusów $\mathrm{w}$ trójkacie $ADE$ iloraz $\displaystyle \frac{x}{y}$. Nie wyznaczač jawnie

$x\mathrm{i}y$, lecz tylko sume $x+y$ (korzystač ze wzoru skróconego mnozenia).

21.1. Oznaczyč przez $x, y$ krawedzie mniejszych sześcianów. Napisač

uklad równań $\mathrm{z}$ niewiadomymi $x\mathrm{i}y\mathrm{i}$ nie wyznaczajac ich jawnie, obliczyč

tylko $x^{2} +y^{2}$ za pomoca wzorów skróconego mnozenia. Stad od razu

otrzymač odpowied $\acute{\mathrm{z}}.$

21.2. Wyznaczyč wektory $\vec{AC}\mathrm{i}\vec{BD}\mathrm{i}$ zastosowač iloczyn skalarny oraz

$\mathrm{t}\mathrm{o}\dot{\mathrm{z}}$ samośč podana we wskazówce do $\mathrm{z}\mathrm{a}\mathrm{d}$. 2.8.

21.3. Wyznaczyč skale podobieństwa trójkatów i wyrazič przeciwpros-

tokatna przez promień okregu r. Stad obliczyč sume przyprostokatnych

wyjściowego trójkata iw konsekwencji sume cosinusów katów ostrych trój-

kata. Podnoszac te równośč do kwadratu obliczyč oba cosinusy.

21.4. Przenieśč niewymiernośč do mianownika $\mathrm{i}$ podzielič licznik $\mathrm{i}$ mia-

nownik przez $n$. Korzystač $\mathrm{z}$ faktu, $\dot{\mathrm{z}}\mathrm{e}$ zlozenie funkcji malejacych jest

funkcja rosnaca.

21.5. Korzystač ze wzoru podanego we wskazówce do zadania 3.8.
\end{document}
