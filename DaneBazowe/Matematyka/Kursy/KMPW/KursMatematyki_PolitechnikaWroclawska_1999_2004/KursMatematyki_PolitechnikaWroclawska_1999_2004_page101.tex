\documentclass[a4paper,12pt]{article}
\usepackage{latexsym}
\usepackage{amsmath}
\usepackage{amssymb}
\usepackage{graphicx}
\usepackage{wrapfig}
\pagestyle{plain}
\usepackage{fancybox}
\usepackage{bm}

\begin{document}

119

21.6. Napisač warunki określajace dziedzine, ale nie wyznaczač dziedzi-

ny w sposóbjawny. Sprowadzič logarytmy do wspólnej podstawy 4 i przejśč

do równania algebraicznego trzeciego stopnia. Obliczyč jego pierwiastki

i wybrač te, które naleza do dziedziny.

21.7. Narysowač przekrój osiowy stozka. Objetośč wyrazičjako funkcje

wysokości stozka. Nie mylič tego zadania z zagadnieniem wyznaczania

ekstremów lokalnych.

21.8. Obie parabole lacznie ze stycznymi tworza figure majaca środek

symetrii S (dlaczego?). Wiec szukane styczne przechodza przez punkt S.

Wyznaczyč S. Napisač równanie peku prostych przechodzacych przez S

i z warunku styczności (wyróznik odpowiedniego równania kwadratowego

równy zeru) obliczyč wspólczynniki kierunkowe szukanych stycznych.

22.1. Wykorzystač parzystośč funkcji.

zwrócič uwage na otoczenie punktu $x=0.$

Podczas rysowania wykresu

22.2. Uzasadnič, $\dot{\mathrm{z}}\mathrm{e}$ liczby metrów sześciennych wody wplywajace do

basenu $\mathrm{w}$ kolejnych minutach tworza ciag arytmetyczny. We wszystkich

obliczeniach przyjač $minut_{G}$ jako jednostke czasu. Dane liczbowe podstawič

na końcu.

22.3. Oznaczyč średnice obu podstaw przez $x\mathrm{i}y$. Ulozyč uklad równań

$\mathrm{z}$ niewiadomymi $x, y \mathrm{i}$ przejśč od razu do alternatywy ukladów równań

liniowych.

22.4. $\mathrm{Z}$ twierdzenia sinusów wynika, $\dot{\mathrm{z}}\mathrm{e}$ znany jest takze bok $|BC|.$

$\mathrm{W}$ okregu $0$ promieniu $R$ zaznaczyč cieciwe $0$ dlugości $|BC| \mathrm{i}$ rozwazač

katy wpisane oparte na luku wyznaczonym przez $\mathrm{t}\mathrm{e}$ cieciwe. Wybrač takie

polozenie (polozenia) wierzcholka $A$, które daje $|AB|=\displaystyle \frac{8}{5}R. \mathrm{W}$ zalezności

od wielkości kata $\alpha$ (czyli dlugości cieciwy $|BC|$) mamy rózne przypadki,

które nalezy kolejno rozpatrzyč.

22.5. Od razu zlogarytmowač obie strony,

logarytmu liczbe 8.

przyjmujac za podstawe
\end{document}
