\documentclass[a4paper,12pt]{article}
\usepackage{latexsym}
\usepackage{amsmath}
\usepackage{amssymb}
\usepackage{graphicx}
\usepackage{wrapfig}
\pagestyle{plain}
\usepackage{fancybox}
\usepackage{bm}

\begin{document}

120

22.6. Wyrazič wektory

iloczynu skalarnego.

$\vec{CB}$

i

$\vec{CD}$ przez $\vec{AB}=\vec{u} \mathrm{i} \vec{BD}=\vec{v}\mathrm{i}\mathrm{u}\dot{\mathrm{z}}$ yč

22.7. Wyznaczyč dziedzine równania. Pomnozyč obie strony przez

wyrazenie $(\sin x\cos x) \mathrm{i}$ doprowadzič do równania elementarnego postaci

$\sin(f(x))=\sin(g(x))$. Rozwiazania zapisač $\mathrm{w}$ postaci jednej serii.

22.8. Napisač równanie stycznej $\mathrm{w}$ punkcie $x_{0}$, wyznaczyč punkty prze-

cieč tej stycznej $\mathrm{z}$ osiami ukladu wspólrzednych $\mathrm{i}$ wyrazič kwadrat dlugości

odcinka stycznej jako funkcje $x_{0}$. Do rózniczkowania pozostawič $\mathrm{t}\mathrm{e}$ funkcje

$\mathrm{w}$ postaci sumy funkcji potegowych. Nie mylič postawionego pytania $\mathrm{z}$ za-

gadnieniem wyznaczania ekstremów lokalnych.

23.1. Liczba,,slów'' utworzonych $\mathrm{z}$ danych liter odpowiada liczbie per-

mutacji $\mathrm{z}$ powtórzeniami.

23.2. Zadanie rozwiazač bez dzielenia wielomianów. Zauwazyč, $\dot{\mathrm{z}}\mathrm{e}$ i10-

raz danych wielomianów ma postač $ x+\alpha \mathrm{i}$ wyznaczyč najpierw niewia-

doma $\alpha.$

23.3. Wykorzystač symetrie figury $\mathrm{i}$ twierdzenie $0$ okregach wzajemnie

stycznych.

23.4. Przez punkty $K \mathrm{i} L$ poprowadzič plaszczyzny prostopadle do

plaszczyzny podstawy $\mathrm{i}$ równolegle do $BC$. Obliczač oddzielnie objetości

$\mathrm{k}\mathrm{a}\dot{\mathrm{z}}$ dej $\mathrm{z}$ tak otrzymanych bryl (dwie $\mathrm{z}$ nich sa identyczne). Por. $\mathrm{z}\mathrm{a}\mathrm{d}$. 15.3.

23.5. Wyznaczyč dziedzine nierówności. Rozpatrzyč najpierw oczy-

wisty przypadek $x < 0$. Dla $x > 0$ podnieśč obie strony nierówności

do kwadratu $\mathrm{i}$ rozwiazač nierównośč dwukwadratowa. Wykresem funkcji

$\mathrm{z}$ prawej strony nierówności nie jest luk paraboli lecz inna dobrze znana

krzywa (por. wskazówka do $\mathrm{z}\mathrm{a}\mathrm{d}$. 13.7).

23.6. Dowód kroku indukcyjnego przeprowadzič wprost. Nie stosowač

niewygodnej metody redukcji. Dbač $0$ logiczna poprawnośč zapisu dowodu.
\end{document}
