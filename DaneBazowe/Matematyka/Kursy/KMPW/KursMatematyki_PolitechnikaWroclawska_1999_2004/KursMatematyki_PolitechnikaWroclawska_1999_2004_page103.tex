\documentclass[a4paper,12pt]{article}
\usepackage{latexsym}
\usepackage{amsmath}
\usepackage{amssymb}
\usepackage{graphicx}
\usepackage{wrapfig}
\pagestyle{plain}
\usepackage{fancybox}
\usepackage{bm}

\begin{document}

121

23.7. Punkt $M(y_{0}^{2},y_{0}), y_{0}>0, \mathrm{l}\mathrm{e}\dot{\mathrm{z}}\mathrm{y}$ najblizej $P$, gdy odcinek $PM$ jest

prostopadly do stycznej do danej krzywej $\mathrm{w}$ punkcie $M. \mathrm{U}\dot{\mathrm{z}}$ yč rachunku

wektorowego.

23.8. Poniewaz wspólczynnik przy $x^{2}$ jest dodatni, wiec pierwiastki

trójmianu kwadratowego beda $\mathrm{l}\mathrm{e}\dot{\mathrm{z}}$ eč $\mathrm{w}$ odcinku $(0,1)$, gdy odcieta wierz-

cholka paraboli bedacej jego wykresem znajdzie $\mathrm{s}\mathrm{i}\mathrm{e} \mathrm{w}$ tym przedziale,

a wartości trójmianu dla $x=0\mathrm{i}x=1$ beda dodatnie. Otrzymane nierów-

ności trygonometryczne rozwiazač analitycznie. Ewentualny rysunek sluzy

do ilustracji rozwiazania.

24.1. Pamietač 0 warunku istnienia sumy nieskończonego ciagu geo-

metrycznego.

24.2. Zaczač od określenia modelu probabilistycznego, $\mathrm{t}\mathrm{j}$. zbioru zdarzeń

elementarnych $\Omega$ oraz prawdopodobieństwa $P$. Oznaczyč przez $A$ zdarze-

nie polegajace na $\mathrm{t}\mathrm{y}\mathrm{m}, \dot{\mathrm{z}}\mathrm{e}$ kości pasuja do siebie, a przez $A_{i}$ zdarzenie, $\dot{\mathrm{z}}\mathrm{e}$

na jednym $\mathrm{z}$ pól obu kości jest $i$ oczek, a na pozostalych polach cokolwiek,

$i=0$, 6. Wtedy $ A=A_{0}\cup \cup A_{6} \mathrm{i}$ skladniki parami wykluczaja $\mathrm{s}\mathrm{i}\mathrm{e}$

(dlaczego?). Obliczyč $P(A_{i})\mathrm{i}$ skorzystač $\mathrm{z}$ wlasności prawdopodobieństwa.

24.3. Wykazač, $\dot{\mathrm{z}}\mathrm{e}$ dla $m=10$ uklad jest sprzeczny, a dla $m\neq 10$ ma

jedno rozwiazanie. Zauwazyč, $\dot{\mathrm{z}}\mathrm{e}$ dla $\dot{\mathrm{z}}$ adnego $m \in \mathrm{R}$ para (l, l) nie jest

rozwiazaniem ukladu.

24.4. Określič dziedzine dla kata $\alpha$ porównujac ten $\mathrm{k}\mathrm{a}\mathrm{t}\mathrm{z}$ jego rzutem

prostokatnym na podstawe. $\mathrm{Z}$ twierdzenia $0$ trzech prostopadlych uza-

sadnič, $\dot{\mathrm{z}}\mathrm{e}$ {\it AB} $\perp BD'$. Wywnioskowač stad, $\dot{\mathrm{z}}\mathrm{e} \mathrm{k}\mathrm{a}\mathrm{t} DBD'$ jest katem

plaskim kata dwuściennego miedzy plaszczyzna ABD'E' $\mathrm{i}$ podstawa gra-

niastoslupa.

24.5. Rozwazyč przypadki $x < 1$ oraz $x > 1 \mathrm{i}$ pomnozyč obie strony

przez mianownik (dodatni lub ujemny, odpowiednio). Jedna $\mathrm{z}$ nierówności

podwójnych jest automatycznie spelniona, a druga, przez podstawienie

$2^{x}=t$, sprowadza $\mathrm{s}\mathrm{i}\mathrm{e}$ do do nierówności kwadratowej. Nie potrzeba rozwazač

nierówności $\mathrm{w}\mathrm{y}\dot{\mathrm{z}}$ szego stopnia.
\end{document}
