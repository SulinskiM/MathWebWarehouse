\documentclass[a4paper,12pt]{article}
\usepackage{latexsym}
\usepackage{amsmath}
\usepackage{amssymb}
\usepackage{graphicx}
\usepackage{wrapfig}
\pagestyle{plain}
\usepackage{fancybox}
\usepackage{bm}

\begin{document}

122

24.6. Zauwazyč, $\dot{\mathrm{z}}\mathrm{e}$ tg $82^{\circ}30' = \displaystyle \frac{1}{\mathrm{t}\mathrm{g}7^{\circ}30'}$ oraz $\dot{\mathrm{z}}\mathrm{e}82^{\circ}30'-7^{\circ}30' = 75^{\circ}$

$\mathrm{i}$ zastosowač wzór na tangens róznicy katów. Nastepnie korzystač $\mathrm{z}$ rów-

ności $75^{\circ}=45^{\circ}+30^{\circ}$

24.7. Skorzystač ze wskazówki do zadania 6.2, a w drugiej cześci rozwia-

zania ze wskazówki do zad. 5.8.

24.8. Przypadek $\alpha=1$ wymaga oddzielnego rozpatrzenia (dlaczego?).

Pochodna funkcji $\displaystyle \frac{b}{x^{2}-1}=b(x^{2}-1)^{-1}$ wygodniej jest obliczač za pomoca

reguly rózniczkowania funkcji zlozonej. Zauwazyč, $\dot{\mathrm{z}}\mathrm{e}$ dla $\alpha= 3, b= 32,$

gwarantujacych ciaglośči rózniczkowalnośč $f(x)$, punkt $P(3$, 4$)$ jestjej punk-

tem przegiecia.

25.1.

$t\neq 0.$

Najpierw rozpatrzyč oczywisty przypadek $t = 0$, a nastepnie

25.2. Korzystajac $\mathrm{z}$ twierdzenia Talesa wykazač, $\dot{\mathrm{z}}\mathrm{e}$ przekrój jest równo-

leglobokiem. Nastepnie prowadzič plaszczyzne symetrii czworościanu $\mathrm{i}$ sto-

sujac twierdzenie $0$ trzech prostopadlych, wykazač, $\dot{\mathrm{z}}\mathrm{e}$ przekrój jest pros-

tokatem.

25.3. Określič dziedzine nierówności. Zauwazyč, $\dot{\mathrm{z}}\mathrm{e}$ szukany zbiór jest

symetryczny wzgledem poczatku ukladu, co pozwala ograniczyč rozwazania

do I čwiartki ukladu. Rozpatrzyč przypadki $xy>1$ oraz $xy<1.$

25.4. Pólprosta wychodzaca ze środka okregu $\mathrm{i}$ zawierajaca dany punkt

$A$ przecina ten okrag $\mathrm{w}$ punkcie $A'\mathrm{l}\mathrm{e}\dot{\mathrm{z}}$ acym najblizej punktu $A$. Stad $|AA'|$

jest odleglościa punktu $A$ od danego okregu. Prowadzac rozwazania geo-

metryczne uzasadnič, $\dot{\mathrm{z}}\mathrm{e}$ dla punktów $\mathrm{l}\mathrm{e}\dot{\mathrm{z}}$ acych wewnatrz okregu zachodzi

relacja $OA+PA=10$, co oznacza, $\dot{\mathrm{z}}\mathrm{e}A\mathrm{l}\mathrm{e}\dot{\mathrm{z}}\mathrm{y}$ na elipsie $0$ ogniskach $O\mathrm{i}P$

(por. wskazówka do $\mathrm{z}\mathrm{a}\mathrm{d}$. 4.6). Inaczej jest, gdy $A\mathrm{l}\mathrm{e}\dot{\mathrm{z}}\mathrm{y}$ na zewnatrz danego

okregu.

25.5. Wszystkie przeprowadzane losowania sa wzajemnie niezalezne,

wiec ich kolejnośč nie ma wplywu na prawdopodobieństwo rozwazanego

zdarzenia. Oznaczyč przez $K, N$ zdarzenia polegajace na $\mathrm{t}\mathrm{y}\mathrm{m}, \dot{\mathrm{z}}\mathrm{e}$ dziecko,

odpowiednio, Kowalskich, Nowakowskich zostalo wybrane przedstawicielem.
\end{document}
