\documentclass[a4paper,12pt]{article}
\usepackage{latexsym}
\usepackage{amsmath}
\usepackage{amssymb}
\usepackage{graphicx}
\usepackage{wrapfig}
\pagestyle{plain}
\usepackage{fancybox}
\usepackage{bm}

\begin{document}

123

Wówczas $ K\cap N=\emptyset$ oraz $ K\cup N=\Omega$. Nastepnie zastosowač wzór na praw-

dopodobieństwo calkowite.

25.6. Unikač niewygodnego dowodu redukcyjnego, ajeśli $\mathrm{s}\mathrm{i}\mathrm{e}$ go stosuje,

pamietač $0$ odpowiednim zakończeniu potrzebnym dla poprawności rozu-

mowania.

25.7. Nie tracič czasu na badanie wlasności, których ta funkcja nie $\mathrm{m}\mathrm{o}\dot{\mathrm{z}}\mathrm{e}$

mieč (np. asymptoty ukośne). Do obliczania pochodnej przedstawič funkcje

$\mathrm{w}$ postaci iloczynu funkcji potegowych, $\mathrm{t}\mathrm{j}. f(x)=\sqrt{3}(x-1)^{1/2}(5-x)^{-1/2}$

$\mathrm{i}$ zastosowač regule rózniczkowania iloczynu. Zauwazyč, a nastepnie wyka-

zač, $\dot{\mathrm{z}}\mathrm{e}$ prosta $x= 1$ jest styczna do wykresu $f(x) \mathrm{w}$ punkcie $x= 1$ (por.

wskazówka do $\mathrm{z}\mathrm{a}\mathrm{d}$. 3.6).

25.8. Wykazač, $\dot{\mathrm{z}}\mathrm{e}$ kolejne odcinki lamanej tworza ciag geometryczny

$0$ ilorazie mniejszym od l. Nastepnie zastosowač wzór na sume wyrazów

nieskończonego ciagu geometrycznego lub uzasadnič, $\dot{\mathrm{z}}\mathrm{e}$ suma tajest równa

obwodowi danego trójkata.

26.1. Odcinek pasa laczacy oba kola jest styczny do $\mathrm{k}\mathrm{a}\dot{\mathrm{z}}$ dego $\mathrm{z}$ nich,

wiec prostopadly do promieni poprowadzonych do punktów styczności. Nie

$\mathrm{u}\dot{\mathrm{z}}$ ywač zapisu postaci $ 26\displaystyle \frac{2}{3}\pi$ cm który jest niejednoznaczny.

26.2. Zachowač podana $\mathrm{w}$ zadaniu kolejnośč obliczeń.

26.3. Wygodnie jest posluzyč $\mathrm{s}\mathrm{i}\mathrm{e}$ rachunkiem wektorowym. Oznaczyč

przez $A, B$ punkty przeciecia $\mathrm{s}\mathrm{i}\mathrm{e}$ szukanej prostej $l$ odpowiednio $\mathrm{z}$ prosta $k$

$\mathrm{i}m$. Wówczas mamy $A(x,x+3)$. Wyrazič $\vec{AP}\mathrm{i}\vec{AB}=2\vec{AP}$ przy pomocy

niewiadomej $x \mathrm{i}$ korzystajac $\mathrm{z}$ faktu, $\dot{\mathrm{z}}\mathrm{e} B \mathrm{l}\mathrm{e}\dot{\mathrm{z}}\mathrm{y}$ na prostej $m$ obliczyč $x.$

$\rightarrow$

Wektor normalny prostej $l$ jest prostopadly do $AB.$

26.4. Wierzcholek $C$ kata prostego, spodek $O$ wysokości ostroslupa

ijego rzuty prostokatne $K, L$ na przyprostokatne podstawy tworza kwadrat

$0$ boku $r$. Stad wynika, $\dot{\mathrm{z}}\mathrm{e}$ rzuty prostokatne punktów $K\mathrm{i}L$ na krawed $\acute{\mathrm{z}}DC$

pokrywaja $\mathrm{s}\mathrm{i}\mathrm{e}$ ($\mathrm{w}$ punkcie $E$), zatem $\beta=\angle KEL$. Wyznaczyč dziedzine dla

$\beta$. Wysokośč czworościanu obliczyč $\mathrm{z}$ podobieństwa odpowiednich trójkatów

$\mathrm{w}$ przekroju plaszczyzna $ODC.$
\end{document}
