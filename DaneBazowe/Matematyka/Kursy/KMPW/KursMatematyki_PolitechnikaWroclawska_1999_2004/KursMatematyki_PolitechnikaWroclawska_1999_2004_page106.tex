\documentclass[a4paper,12pt]{article}
\usepackage{latexsym}
\usepackage{amsmath}
\usepackage{amssymb}
\usepackage{graphicx}
\usepackage{wrapfig}
\pagestyle{plain}
\usepackage{fancybox}
\usepackage{bm}

\begin{document}

124

26.5. Skorzystač $\mathrm{z}$ parzystości funkcji oraz ze wzoru $\log_{c}\alpha^{2}=2\log_{c}|\alpha|,$

$c>0, c\neq 1, \alpha\neq 0$. Wykres funkcji $f$ otrzymujemy $\mathrm{z}$ wykresu standardowej

krzywej $y=\log_{2}x$ przez translacje $\mathrm{i}$ odbicia symetryczne.

26.6. Zastosowač wzór $\displaystyle \cos 2x=\frac{1-\mathrm{t}\mathrm{g}^{2}x}{1+\mathrm{t}\mathrm{g}^{2}x}\mathrm{i}$ podstawič tg $x=t.$

26.7. Dwie funkcje $\mathrm{m}\mathrm{o}\dot{\mathrm{z}}$ na zlozyč wtedy $\mathrm{i}$ tylko wtedy, gdy zbiór wartości

funkcji wewnetrznej jest zawarty $\mathrm{w}$ dziedzinie funkcji zewnetrznej. Przy-

padek $\alpha=0$ rozpatrzyč oddzielnie.

26.8. Patrz wskazówka do zadania l2.8.

27.1. Wyznaczyč dziedzine równania. Aby istnialo rozwiazanie, prawa

strona musi byč nieujemna. Wtedy obie strony $\mathrm{m}\mathrm{o}\dot{\mathrm{z}}$ na podnieśč do kwadratu.

Przypadek $p=0$ rozpatrzyč oddzielnie.

27.2. Zauwazyč, $\dot{\mathrm{z}}\mathrm{e}$ środki okregów $K\mathrm{i}K_{1}$ oraz punkt $S\mathrm{l}\mathrm{e}\dot{\mathrm{z}}$ a na prostej

prostopadlej do danej prostej. Nastepnie korzystač $\mathrm{z}$ zalezności miedzy

promieniami rozwazanych okregów.

27.3. Dane określaja jednoznacznie przekatna $AC$ trapezu, na której,

jako na cieciwie okregu, jest oparty $\mathrm{k}\mathrm{a}\mathrm{t}$ ostry przy wierzcholku $B$ podstawy.

Przez zmiane polozenia punktu $B$ na okregu, poczynajac od punktu $C,$

otrzymujemy rózne trapezy (tj. $0$ róznych wartościach $d$). Minimalne $d$

odpowiada sytuacji, gdy krótsza podstawa trapezu jest równa zeru $\mathrm{i}$ trapez

staje $\mathrm{s}\mathrm{i}\mathrm{e}$ trójkatem, a maksymalne, gdy $B$ pokrywa $\mathrm{s}\mathrm{i}\mathrm{e} \mathrm{z} C$. Wysokośč

trapezu obliczyč $\mathrm{z}$ twierdzenia Pitagorasa $\mathrm{i}$ stad bezpośrednio ramie trapezu.

27.4. Najpierw ustalič dziedzine dla kata $\beta$ (porównujac go $\mathrm{z}$ rzutem

prostokatnym na podstawe). $\mathrm{Z}$ twierdzenia $0$ trzech prostopadlych wywnios-

kowač, $\dot{\mathrm{z}}\mathrm{e}$ przekrój ostroslupa jest deltoidem. $\mathrm{W}$ obliczeniach korzystač

$\mathrm{z}$ podobieństwa trójkatów $\mathrm{i}$ twierdzenia $0$ środkowych $\mathrm{w}$ trójkacie.

27.5. Przenieśč niewymiernośč do mianownika, stosujac wzór na róznice

sześcianów $\mathrm{i}$ podzielič licznik $\mathrm{i}$ mianownik przez $n^{13/5}$ Skorzystač $\mathrm{z}$ faktu,

$\dot{\mathrm{z}}\mathrm{e}\alpha<3.$
\end{document}
