\documentclass[a4paper,12pt]{article}
\usepackage{latexsym}
\usepackage{amsmath}
\usepackage{amssymb}
\usepackage{graphicx}
\usepackage{wrapfig}
\pagestyle{plain}
\usepackage{fancybox}
\usepackage{bm}

\begin{document}

125

27.6. Stosujac definicje logarytmu sprowadzič dana nierównośč do pros-

tej nierówności trygonometrycznej. Od razu ograniczyč $\mathrm{s}\mathrm{i}\mathrm{e}$ do dziedziny

(I čwiartka, cosinus dodatni), co pozwala latwo rozwiazač $\mathrm{t}\mathrm{e}$ nierównośč.

27.7. Rozwazmy losowanie jednej liczby $\mathrm{i}$ odpowiadajacy mu model

probabilistyczny $\Omega_{0} \mathrm{i}P_{0}$. Niech $ A\subset \Omega_{0}$ oznacza zdarzenie, $\dot{\mathrm{z}}\mathrm{e}$ liczba czy-

tana od strony lewej do prawej jest podzielna przez 4, a $B$ zdarzenie, $\dot{\mathrm{z}}\mathrm{e}$

liczba czytana od strony prawej do lewej jest podzielna przez 4. Wówczas

zdarzenia $A, B$ sa niezalezne (dlaczego?). $P_{0}(A\cup B)$ obliczyč, znajac

$P_{0}(A)\mathrm{i}P_{0}(B)$. Zauwazyč, $\dot{\mathrm{z}}\mathrm{e}P_{0}(A\cup B)$ jest prawdopodobieństwem sukcesu

$\mathrm{w}$ schemacie czterech prób Bernoulliego.

27.8. Szukany zbiór jest przekrojem pasa pomiedzy dwiema prostymi

równoleglymi $\mathrm{i}$ zbioru punktów $\mathrm{l}\mathrm{e}\dot{\mathrm{z}}$ acych pod wykresem $\mathrm{i}$ na wykresie funkcji

$f(x) = \sqrt[3]{x}$. Zwrócič uwage na przebieg tej funkcji $\mathrm{w}$ otoczeniu punktu

$x = 0. \mathrm{W}$ dwóch punktach wykres funkcji $f(x)$ jest styczny do danych

prostych, a $\mathrm{w}$ dwóch innych przecina te proste pod tym samym katem

(dlaczego?). Do obliczenia tangensa tego kata $\mathrm{u}\dot{\mathrm{z}}$ yč pochodnej.

28.1. Nie wyznaczač predkości obu punktów, lecz od razu ich stosunek.

28.2. Aby nierównośč byla spelniona dla $\mathrm{k}\mathrm{a}\dot{\mathrm{z}}$ dego $x\in \mathrm{R}$, mianownik nie

$\mathrm{m}\mathrm{o}\dot{\mathrm{z}}\mathrm{e}$ mieč pierwiastków rzeczywistych, czyli jest dodatni na calej prostej.

Wtedy $\mathrm{m}\mathrm{o}\dot{\mathrm{z}}$ na obie strony pomnozyč przez ten mianownik, zachowujac znak

nierówności $\mathrm{i}$ badač nieujemnośč otrzymanego trójmianu kwadratowego.

Przypadek $p=1$ rozpatrzyč oddzielnie.

28.3. Zastosowač twierdzenie cosinusów. Nie wyznaczač dlugości boków,

lecz od razu ich iloczyn. Określič dziedzine dla $\alpha, r\mathrm{i}d.$

28.4. Przekrój plaszczyzna symetrii zawiera środek kuli, środek jed-

nej nózki oraz środek odcinka laczacego pozostale nózki. Wykonač rysunek

tego przekroju, przyjmujac $r$ bardzo male $\mathrm{w}$ porównaniu $\mathrm{z}R$. Korzystač

$\mathrm{z}$ twierdzenia $0$ okregach stycznych zewnetrznie.

28.5. Rozwiazanie $\mathrm{w}$ przedziale (-00, 0) wyznaczyč bezpośrednio, ko-

rzystajac ze wzoru na sześcian sumy. $\mathrm{W}(0,\infty)$ wyznaczyč przedzialy mono-
\end{document}
