\documentclass[a4paper,12pt]{article}
\usepackage{latexsym}
\usepackage{amsmath}
\usepackage{amssymb}
\usepackage{graphicx}
\usepackage{wrapfig}
\pagestyle{plain}
\usepackage{fancybox}
\usepackage{bm}

\begin{document}

126

toniczności, ekstrema lokalne oraz wartośč wielomianu $\mathrm{w} x = 0$. Stad

$\mathrm{i}\mathrm{z}$ wlasności Darboux określič liczbe rozwiazań (nie wyznaczač ich jawnie).

28.6. Skorzystač ze wzoru $\alpha^{n}-b^{n}=(\alpha-b)(\alpha^{n-1}+\alpha^{n-2}b++b^{n-1})$

$\mathrm{i}$ rozwazyč oddzielnie $n$ parzyste $\mathrm{i}$ nieparzyste.

28.7. Napisač warunki określajace dziedzine (warunek istnienia sum

obu nieskończonych ciagów geometrycznych), nie wyznaczajac jej $\mathrm{w}$ sposób

jawny. Podstawič $\cos x = t \mathrm{i}$ wyeliminowač pierwiastki nie nalezace do

dziedziny.

28.8. Za pomoca pochodnej napisač równanie stycznej $\mathrm{w}$ punkcie

$P(x_{0},\displaystyle \frac{x_{0}^{2}}{2}) \mathrm{l}\mathrm{e}\dot{\mathrm{z}}$ acym na danej paraboli $\mathrm{i}$ bezpośrednio stad równanie prostej

prostopadlej do stycznej przechodzacej przez $P$. Wyznaczyč wspólrzedne

środka rozwazanego odcinka tej normalnej $\mathrm{i}$ po wyeliminowaniu parametru

$x_{0}$ otrzymač równanie krzywej. Zauwazyč, $\dot{\mathrm{z}}\mathrm{e}x_{0}$ nie $\mathrm{m}\mathrm{o}\dot{\mathrm{z}}\mathrm{e}$ byč równe zeru

(dlaczego?).

29.1. Oznaczyč $C(x,3x-14)$. Wyznaczyč środek $S$ odcinka AB. Ko-

$\rightarrow$

$\rightarrow$

rzystajac $\mathrm{z}$ prostopadlości wektorów AB $\mathrm{i}SC$ (iloczyn skalarny równy zeru)

wyznaczyč niewiadoma $x.$

29.2. Oznaczyč przez $x$ liczbe pieciocyfrowa powstala po skreśleniu pier-

wszej cyfry $\mathrm{i}$ ulozyč równanie liniowe $\mathrm{z}$ niewiadoma $x.$

29.3. Wyrazič promień okregu wpisanego za pomoca krótszego ramie-

nia $c$. Uzasadnič, $\dot{\mathrm{z}}\mathrm{e}$ środek $O$ okregu wpisanego $\mathrm{i}$ krótsza podstawa $CD$

wyznaczaja trójkat, $\mathrm{w}$ którym wysokośč do boku $CD$ tworzy $\mathrm{z}$ odcinkami

$OC\mathrm{i}OD$ katy $\displaystyle \alpha \mathrm{i}\frac{\alpha}{2}$. Stad wyznaczyč $|CD|.$

29.4. Najpierw rozpatrzyč przypadek oczywisty, gdy $x^{2}-x-2<0$. Po-

zostale przypadki, przez odwrócenie ulamków po obu stronach nierówności,

prowadza do nierówności kwadratowych (uwaga na znak nierówności).

29.5. Ustalič dziedzine nierówności $\mathrm{i}$ rozpatrzyč przypadki $x< 1$ oraz

$x>1$. Wykres funkcji $f(x)=1+\sqrt[3]{x-1}$ jest translacja standardowej krzy-
\end{document}
