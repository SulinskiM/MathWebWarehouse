\documentclass[a4paper,12pt]{article}
\usepackage{latexsym}
\usepackage{amsmath}
\usepackage{amssymb}
\usepackage{graphicx}
\usepackage{wrapfig}
\pagestyle{plain}
\usepackage{fancybox}
\usepackage{bm}

\begin{document}

127

wej $y=\sqrt[3]{x}\mathrm{i}$ powinien byč sporzadzony dokladnie, szczególnie $\mathrm{w}$ otoczeniu

punktu $x=1$. Opisač, które cześci brzegu wyznaczonego zbioru naleza do

tego zbioru.

29.6. Wygodna metoda przeksztalcania obu stron jest przejście do cos-

inusów podwojonych katów $(2\sin^{2}\gamma = 1-\cos 2\gamma, \mathrm{p}\mathrm{o}\mathrm{r}$. wskazówka do

$\mathrm{z}\mathrm{a}\mathrm{d}$. 4.3$)$. Otrzymane serie rozwiazań polaczyč $\mathrm{w}$ dwie serie.

29.7. Uzasadnič, $\dot{\mathrm{z}}\mathrm{e}$ dziedzina szukanego kata jest przedzial $(\displaystyle \frac{\pi}{2},\pi).$

Poprowadzič przekrój plaszczyzna symetrii przechodzaca przez wierzcholek

ostroslupa $\mathrm{i}$ środki przeciwleglych krawedzi podstawy $\mathrm{i}$ korzystač $\mathrm{z}$ podobień-

stwa odpowiednich trójkatów. Cosinus szukanego kata wyznaczyč za po-

moca twierdzenia cosinusów.

29.8. Wyznaczyč dziedzine $D$ funkcji $S(x)$, pamietač $0x=-1$. Posluzyč

$\mathrm{s}\mathrm{i}\mathrm{e}$ pochodna funkcji, ale nie wyznaczač ekstremów lokalnych, lecz ograniczyč

$\mathrm{s}\mathrm{i}\mathrm{e}$ do podania wartości najwiekszej $\mathrm{i}$ najmniejszej funkcji $S(x)\mathrm{w}D.$

30.1. Objetośč rozwazanej bryly jest róznica objetości dwóch stozków

$0$ wspólnej podstawie. Oznaczyč dluzsza przyprostokatna przez $\alpha$, krótsza

przez $b$, a objetośč stozka powstalego $\mathrm{z}$ obrotu trójkata wokól krótszej

przyprostokatnej przez $V_{1}$. Wtedy $V_{1} \geq V_{2}$. Nie wyznaczač przypros-

tokatnych ani innych wielkości liniowych, lecz od razu objetośč $\mathrm{i}$ po wye-

liminowaniu $\alpha \mathrm{i}b$ wyrazič $\mathrm{j}\mathrm{a}$ przez $V_{1}\mathrm{i}V_{2}.$

30.2. Przyjač wysokośč najmniejszej nagrody, róznice ciagu oraz liczbe

nagród $n$ za niewiadome. Ulozyč uklad dwóch równań $\mathrm{i}$ wykazač, $\dot{\mathrm{z}}\mathrm{e}$

$4\leq n\leq 6$. Rozwiazania wyznaczamy przez bezpośrednie sprawdzenie.

30.3. Równania okregów, których środki $\mathrm{l}\mathrm{e}\dot{\mathrm{z}}$ a na prostej $y = 1$, wyz-

naczyč bezpośrednio $\mathrm{z}$ twierdzenia $0$ okregach stycznych zewnetrznie lub

wewnetrznie. Środki pozostalych okregów otrzymujemy po rozwiazaniu

odpowiedniego ukladu równań.

30.4. Korzystamy $\mathrm{z}$ twierdzenia cosinusów. Nie wyznaczamy boków

równolegloboku, lecz tylko ich iloczyn $\mathrm{i}$ przez porównanie dwóch wyrazeń

na pole równolegloboku otrzymujemy od razu tangens szukanego kata.
\end{document}
