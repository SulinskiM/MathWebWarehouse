\documentclass[a4paper,12pt]{article}
\usepackage{latexsym}
\usepackage{amsmath}
\usepackage{amssymb}
\usepackage{graphicx}
\usepackage{wrapfig}
\pagestyle{plain}
\usepackage{fancybox}
\usepackage{bm}

\begin{document}

15

Praca kontrolna

nr 7

7.1. Rozwiazač nierównośč

$|9^{x}-2|<3^{x+1}-2.$

7.2. Wyznaczyč równanie krzywej bedacej obrazem okregu

$(x+1)^{2}+(y-6)^{2}=4\mathrm{w}$ powinowactwie prostokatnym $0$ osi $Ox\mathrm{i}$ sto-

sunku $k=\displaystyle \frac{1}{2}$. Obliczyč pole figury ograniczonej ta krzywa. Sporzadzič

staranny rysunek.

7.3. Pewien zbiór zawiera dokladnie 67 podzbiorów $0$ co najwyzej dwóch

elementach. Ile podzbiorów siedmioelementowych zawiera ten zbiór?

7.4. Trapez $0$ katach przy podstawie wynoszacych $15^{\circ}\mathrm{i}45^{\circ}$ opisano na kole

$0$ promieniu $R$. Obliczyč stosunek pola kola do pola tego trapezu.

7.5. Rozwiazač uklad równań

$\left\{\begin{array}{l}
mx-6y=3\\
2x+(m-7)y=m-1
\end{array}\right.$

$\mathrm{w}$ zalezności od parametru rzeczywistego $m$. Podač wszystkie rozwia-

zania ($\mathrm{i}$ odpowiadajace im wartości parametru $m$), dla których $x$ jest

równe $y.$

7.6. Rozwiazač nierównośč $\sin 2x<\sin x\mathrm{w}$ przedziale $[-\displaystyle \frac{\pi}{2},\frac{\pi}{2}]$.

zanie zilustrowač starannym wykresem.

Rozwia-

7.7. Ostroslup podzielono na trzy cześci dwiema plaszczyznami równolegly-

mi do jego podstawy. Pierwsza plaszczyznajest polozona $\mathrm{w}$ odleglości

$d_{1} = 2$ cm, a druga $\mathrm{w}$ odleglości $d_{2} = 3$ cm od podstawy. Pola

przekrojów ostroslupa tymi plaszczyznami równe sa odpowiednio

$S_{1} = 25 \mathrm{c}\mathrm{m}^{2}$ oraz $S_{2} = 16 \mathrm{c}\mathrm{m}^{2}$ Obliczyč objetośč tego ostroslupa

oraz objetośč najmniejszej cześci.

7.8. Trylogie skladajaca $\mathrm{s}\mathrm{i}\mathrm{e} \mathrm{z}$ dwóch powieści dwutomowych oraz jednej

jednotomowej ustawiono na pólce $\mathrm{w}$ przypadkowej kolejności. Jakie

jest prawdopodobieństwo tego, $\dot{\mathrm{z}}\mathrm{e}$ tomy a) obydwu, b) co najmniej

jednej $\mathrm{z}$ dwutomowych powieści znajduja $\mathrm{s}\mathrm{i}\mathrm{e}$ obok siebie $\mathrm{i}$ przy tym

tom I $\mathrm{z}$ lewej, a tom II $\mathrm{z}$ prawej strony.
\end{document}
