\documentclass[a4paper,12pt]{article}
\usepackage{latexsym}
\usepackage{amsmath}
\usepackage{amssymb}
\usepackage{graphicx}
\usepackage{wrapfig}
\pagestyle{plain}
\usepackage{fancybox}
\usepackage{bm}

\begin{document}

128

30.5. Określič dziedzine równania. Poniewaz $\mathrm{w}$ dziedzinie obie strony

równania sa dodatnie $\mathrm{m}\mathrm{o}\dot{\mathrm{z}}$ na podnieśč je do kwadratu.

30.6. Wyznaczyč wszystkie (trzy) pierwiastki danego wielomianu. Je-

den $\mathrm{z}$ nich nie $\mathrm{m}\mathrm{o}\dot{\mathrm{z}}\mathrm{e}$ byč pierwiastkiem trójmianu $2x^{2}+\alpha x+b$ (dlaczego?).

Znajac dwa pierwiastki, napisač ten trójmian $\mathrm{i}$ odczytač niewiadome $\alpha \mathrm{i}b.$

30.7. Podstawič $2^{x}=t$. Korzystač $\mathrm{z}\mathrm{t}\mathrm{o}\dot{\mathrm{z}}$ samości podanej we wskazówce

do $\mathrm{z}\mathrm{a}\mathrm{d}$. 5.1. Wykresy obu stron otrzymač przez translacje $\mathrm{i}$ odbicia syme-

tryczne standardowej krzywej $y=2^{x}$

30.8. Wyznaczyč miejsca zerowe pochodnej $\mathrm{i}$ za pomoca wykresu rozwia-

zač odpowiednia nierównośč trygonometryczna.

31.1. Określič model probabilistyczny. Rozwazane zdarzenie przed-

stawič $\mathrm{w}$ postaci sumy rozlacznych zdarzeń $B_{1}\cup B_{2}\cup B_{3}\cup B_{4}$, gdzie $B_{i}$ jest

zdarzeniem polegajacym na otrzymaniu przez gracza 4 kart $\mathrm{w}i$-tym kolorze

$\mathrm{z}$ asem, królem $\mathrm{i}$ dama. $P(B_{i})$ obliczyč bezpośrednio, pamietajac, $\dot{\mathrm{z}}\mathrm{e}P$ jest

prawdopodobieństwem klasycznym $\mathrm{i}$ skorzystač $\mathrm{z}$ wlasności prawdopodo-

bieństwa.

31.2. Patrz wskazówka do zad. 7.7.

31.3. Określič dziedzine ukladu. Zwrócič uwage najej asymetrie wzgle-

dem zmiennych $x\mathrm{i}y$. Dodajac $\mathrm{i}$ odejmujac równania stronami przejśč do

ukladów równań liniowych.

31.4. Najpierw określič ($\mathrm{i}$ uzasadnič geometrycznie) dziedzine dla kata

$\alpha$ oraz wyznaczyč katy trójkata $ABC.$

31.5. $\mathrm{W}$ dowodzie kroku indukcyjnego przeksztalcač lewa strong $\mathrm{i}$ do-

prowadzič $\mathrm{j}\mathrm{a}$ do równości $\mathrm{z}$ prawa strong. Korzystač ze wzoru na róznice

sinusów. Nie prowadzič dowodu metoda redukcji.

31.6. Pomnozyč licznik $\mathrm{i}$ mianownik przez $\sqrt{n} + \sqrt{n+\sqrt[3]{4n^{2}}},$

a nastepnie podzielič je przez $n^{2/3}$, zamieniajac wcześniej pierwiastki na

potegi $0$ wykladnikach ulamkowych.
\end{document}
