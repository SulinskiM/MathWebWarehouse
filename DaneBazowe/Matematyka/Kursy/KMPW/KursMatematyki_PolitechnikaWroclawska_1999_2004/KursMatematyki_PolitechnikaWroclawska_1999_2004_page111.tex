\documentclass[a4paper,12pt]{article}
\usepackage{latexsym}
\usepackage{amsmath}
\usepackage{amssymb}
\usepackage{graphicx}
\usepackage{wrapfig}
\pagestyle{plain}
\usepackage{fancybox}
\usepackage{bm}

\begin{document}

129

31.7. Korzystač $\mathrm{z}$ nastepujacej wlasności wektorów na plaszczy $\acute{\mathrm{z}}\mathrm{n}\mathrm{i}\mathrm{e}$

(uzasadnič $\mathrm{j}\mathrm{a}$):

{\it Jeśli} $\text{{\it ũ}}\perp\vec{v},\ ||${\it ũ}$|| =||\vec{v}||$ {\it oraz ũ}$= (\alpha,b)$, {\it to} $\vec{v}=(b,-\alpha) lub\vec{v}=(-b,\alpha).$

Sugeruje ona, $\dot{\mathrm{z}}\mathrm{e}$ zadanie ma dwa rozwiazania.

31.8. Zapisač funkcje $\mathrm{w}$ postaci $f(x) = x^{1/2} +x^{-1/2} \mathrm{i}$ obliczyč

pochodna ze wzoru na pochodna funkcji potegowej. Zauwazyč, $\dot{\mathrm{z}}\mathrm{e}$

$\displaystyle \lim_{x\rightarrow+\infty}(f(x)-\sqrt{x})=0$. Jaka wlasnośč geometryczna wykresu funkcji $f(x)$

opisuje ta równośč?

32.1. Oznaczyč przez $x$ predkośč statku, przez $y$ predkośč wody, a przez

$d$ odleglośč $\mathrm{z}$ Wroclawia do Szczecina. Zapisač odpowiednie równania $\mathrm{i}$ nie

wyznaczajac niewiadomych, odpowiedzieč tylko na postawione pytanie.

32.2. Sprowadzič wszystkie logarytmy do tej samej podstawy 2 lub 8

$\mathrm{i}$ skorzystač $\mathrm{z}$ definicji ciagu geometrycznego.

32.3. Narysowač przekrój pionowy wanny $\mathrm{z}\mathrm{l}\mathrm{e}\dot{\mathrm{z}}\mathrm{a}\mathrm{c}\mathrm{a}$ na dnie belka. Ponie-

$\mathrm{w}\mathrm{a}\dot{\mathrm{z}}$ średnica belki jest polowa promienia wanny, wjej przekroju pionowym

pojawiaja $\mathrm{s}\mathrm{i}\mathrm{e}$ trójkaty równoboczne.

32.4. Zarówno $v(x)$, jak $\mathrm{i} w(x)$ musza mieč dwa rózne pierwiastki

rzeczywiste. To daje dziedzine dla parametru $m$. Obliczyč pierwiastki $x_{1},$

$x_{2}$ wielomianu $w(x)$. Jeśli wierzcholek paraboli $0$ równaniu $y=v(x) \mathrm{l}\mathrm{e}\dot{\mathrm{z}}\mathrm{y}$

pomiedzy $x_{1}\mathrm{i}x_{2}$ oraz $v(x_{1})\mathrm{i}v(x_{2})$ sa dodatnie, to wymagany warunek jest

spelniony.

32.5. Rozwazyč nastepujace zdarzenia: $C -$ wylosowano co najmniej

dwie kule biale, $D \mathrm{z}$ urny $\mathrm{B}$ wylosowano kule biala, $E_{i} - \mathrm{z}$ urny $\mathrm{A}$

wylosowano $i$ kul bialych, $i=0$, 1, 2, 3, 4. Wówczas $C'=E_{0}\cup D'\cap E_{1}.$

Skorzystač $\mathrm{z}$ niezalezności zdarzeń $D, E_{i}$, rozlaczności zdarzeń $E_{0}, D'\cap E_{1}$

oraz ze schematu Bernoulliego.

32.6. Wyznaczyč dziedzine równania. Pomnozyč obie strony przez $\cos x$

$\mathrm{i}$ po zastosowaniu wzorów $\sin 2x = 2\sin x\cos x$ oraz $\cos 2x = 1-2\sin^{2}x$

rozlozyč wyrazenie na czynniki, wylaczajac przed nawias czynnik

$(\sin x-\cos x).$
\end{document}
