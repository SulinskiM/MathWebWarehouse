\documentclass[a4paper,12pt]{article}
\usepackage{latexsym}
\usepackage{amsmath}
\usepackage{amssymb}
\usepackage{graphicx}
\usepackage{wrapfig}
\pagestyle{plain}
\usepackage{fancybox}
\usepackage{bm}

\begin{document}

130

32.7. Napisač równanie stycznej $\mathrm{w}$ punkcie $S(x_{0},x_{0}^{4}-2x_{0}^{2})$, gdzie

$x_{0} \in \mathrm{R}$, nastepnie wyznaczyč wszystkie $x_{0}$, dla których $P \mathrm{l}\mathrm{e}\dot{\mathrm{z}}\mathrm{y}$ na sty-

cznej (trzy punkty). Dwa $\mathrm{z}$ nich wyznaczaja $\mathrm{t}\mathrm{e}$ sama styczna, a trzeci inna.

Sporzadzič wykres funkcji $f(x)$, korzystajac zjej parzystości oraz informacji

zebranych przy wyznaczaniu stycznych bez dalszego badania jej przebiegu.

32.8. $\mathrm{Z}$ twierdzenia $0$ trzech prostopadlych wywnioskowač, $\dot{\mathrm{z}}\mathrm{e}$ plasz-

czyzna $SCD$ jest plaszczyzna symetrii ostroslupa, a wiec zawiera środek kuli

opisanej. $\mathrm{L}\mathrm{e}\dot{\mathrm{z}}\mathrm{y}$ on na prostej prostopadlej do podstawy ostroslupa wysta-

wionej $\mathrm{w}$ środku okregu opisanego na podstawie. Wykazač, $\dot{\mathrm{z}}\mathrm{e}\triangle SCD$ jest

równoboczny $\mathrm{i}$ stad określič polozenie środka kuli.

33.1. Zastosowač wzór Newtona. Liczba $x$ jest wieksza od $y$

$\mathrm{g}\mathrm{d}\mathrm{y}x= (1+\displaystyle \frac{p}{100})y.$

0

{\it p}\%,

33.2. Zastosowač wzór na odleglośč punktu od prostej. Nalezy za-

uwazyč, $\dot{\mathrm{z}}\mathrm{e}$ punkt przeciecia $\mathrm{s}\mathrm{i}\mathrm{e}$ prostych $k\mathrm{i}l$ nie spelnia $\dot{\mathrm{z}}$ adanego warunku.

33.3. Skorzystač $\mathrm{z}$ twierdzenia $0$ dwusiecznej kata $\mathrm{w}$ trójkacie oraz ze

wzoru Herona.

33.4. Iloraz $q$ ciagu $(\alpha_{n})$ jest mniejszy od l, wiec droga przebyta przez

czastke jest skończona $\mathrm{i}$ ruch czastki kończy $si_{G} \mathrm{w}$ punkcie $P$. Znajac

wspólrzedne tego punktu, ulozyč dwa równania $\mathrm{z}$ niewiadomymi $\alpha_{1}\mathrm{i}q.$

33.5. Nie $\mathrm{u}\dot{\mathrm{z}}$ ywač algorytmu dzielenia wielomianów, lecz umiejetnie

stosowač rozklad na czynniki np. $x^{4} +x^{2} + 1 = (x^{2}+1)^{2} - x^{2} =$

$=(x^{2}+x+1)(x^{2}-x+1)$. Podobnie postepowač $\mathrm{w}$ dowodzie kroku induk-

cyjnego.

33.6. Oddzielnie rozwazyč przedzialy $(0,\infty)$ oraz (-00, 0). Wykresy

$\mathrm{w}$ tych przedzialach sa istotnie róznymi krzywymi. Nazwač je. Dokladnie

stosowač definicje asymptoty ukośnej prawostronnej $\mathrm{i}$ lewostronnej.

33.7. Przypadek $|\cos x| =$ l jest oczywisty. Dla przypadku

$0< |\cos x| <1$ przejśč do porównania wykladników obu stron. Rozwiazač

odpowiednie równanie trygonometryczne $\mathrm{i}$ za pomoca wykresu wyznaczyč
\end{document}
