\documentclass[a4paper,12pt]{article}
\usepackage{latexsym}
\usepackage{amsmath}
\usepackage{amssymb}
\usepackage{graphicx}
\usepackage{wrapfig}
\pagestyle{plain}
\usepackage{fancybox}
\usepackage{bm}

\begin{document}

131

zbiór rozwiazań nierówności.

metrycznym.

Wygodnie jest posluzyč sie kolem trygono-

33.8. Wykonač przekrój osiowy stozka przechodzacy przez jedna

$\mathrm{z}$ krawedzi graniastoslupa. Wyrazič stosunek objetości bryl jako funkcje

zmiennej $x =$ tg $\alpha \in (0,\infty)$. Nie mylič postawionego pytania $\mathrm{z}$ zagad-

nieniem wyznaczania ekstremów lokalnych.

34.1. Napisač uklad równań $\mathrm{z}$ niewiadomymi przyprostokatnymi $\alpha \mathrm{i}b.$

Nie wyznaczač ich oddzielnie, lecz tylko sume $\alpha+b$ potrzebna do obliczenia

obwodu.

34.2. Skorzystač ze wzoru na sume sześcianów oraz ze wzorów na $\sin 2\gamma$

$\mathrm{i}\cos 2\gamma.$

34.3. Warunkiem stycznościjest istnienie pierwiastka podwójnego odpo-

wiedniego trójmianu kwadratowego. Zadanie ma wiecej $\mathrm{n}\mathrm{i}\dot{\mathrm{z}}$ jedno rozwia-

zanie.

34.4. Wektory (swobodne) $\vec{u}\mathrm{i}\vec{v}$ sa równolegle, gdy $\vec{v}=c\vec{u}\mathrm{d}\mathrm{l}\mathrm{a}$ pewnego

skalara $c$. Prostopadlośč wektorów wyrazič za pomoca iloczynu skalarnego.

34.5. Oznaczyč przez $B_{i}$ zdarzenie polegajace na $\mathrm{t}\mathrm{y}\mathrm{m}, \dot{\mathrm{z}}\mathrm{e}$ za pierwszym

razem wylosowano monete $i \mathrm{z}l, i = 1$, 2, 5. Wtedy $B_{1}\cup B_{2}\cup B_{5} = \Omega$

$\mathrm{i}$ skladniki sa rozlaczne. Prawdopodobieństwo zdarzenia, $\dot{\mathrm{z}}\mathrm{e}$ Jaś wyciagnie

dokladnie dwie monety obliczyč ze wzoru na prawdopodobieństwo calkowite,

a prawdopodobieństwo, $\dot{\mathrm{z}}\mathrm{e}$ Jaś wyciagnie tylko jedna monete (czyli 5 $\mathrm{z}l$)

wynosi $\displaystyle \frac{1}{6}$. Stad otrzymač odpowied $\acute{\mathrm{z}}.$

34.6. Zastosowač wzór $\sqrt{\alpha^{2}}=|\alpha|$. Uzasadnič, $\dot{\mathrm{z}}\mathrm{e}$ krzywa $K\mathrm{o}$ równaniu

$y=\sqrt{4x-x^{2}}$ jest górna polowa okregu $0$ środku $S(2,0)\mathrm{i}$ promieniu 2. Przy

obliczaniu odleglości $P$ od brzegu $\mathcal{F}$ ograniczyč $\mathrm{s}\mathrm{i}\mathrm{e}$ do porównania odleglości

$P$ od krzywej $K$ oraz od odcinka prostej $y=1-x,  x\in (1,4)$. Pozostale

cześci brzegu $\mathcal{F}$ sa znacznie dalej polozone, co wystarczy uzasadnič przez

powolanie $\mathrm{s}\mathrm{i}\mathrm{e}$ na (staranny) rysunek.
\end{document}
