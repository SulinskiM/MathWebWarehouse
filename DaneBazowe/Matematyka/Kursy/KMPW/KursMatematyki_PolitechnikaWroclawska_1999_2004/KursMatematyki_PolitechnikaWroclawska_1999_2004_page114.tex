\documentclass[a4paper,12pt]{article}
\usepackage{latexsym}
\usepackage{amsmath}
\usepackage{amssymb}
\usepackage{graphicx}
\usepackage{wrapfig}
\pagestyle{plain}
\usepackage{fancybox}
\usepackage{bm}

\begin{document}

132

34.7. Wzór określajacy $f(x)$ sprowadzič do najprostszej postaci $\mathrm{i}$ za-

uwazyč, $\dot{\mathrm{z}}\mathrm{e}$ jest ona zlozeniem dwóch funkcji rosnacych ($\mathrm{w}$ dziedzinie!).

Dziedzina $f^{-1}$ jest zbiór wartości $f\mathrm{i}$ na odwrót.

34.8. Do obliczenia krawedzi podstawy $\alpha$ wykorzystač wskazówke do

zadania 3.4. Poprowadzič przekrój ostros1upa p1aszczyzna symetrii prze-

chodzaca przez wierzcholek ostroslupa $\mathrm{i}$ odpowiednia przekatna podstawy

$\mathrm{i}$ korzystač wielokrotnie $\mathrm{z}$ podobieństwa trójkatów. Objetośč wyrazič naj-

pierw przez $\alpha \mathrm{i}$ dopiero na końcu podstawič $c$. Zadanie ma sens, gdy krawed $\acute{\mathrm{z}}$

bocznajest nachylona do podstawy pod katem co najmniej $45^{\circ}$ (dlaczego?).

Stad warunek na $\alpha.$

35.1. Wykluczyč $p = 0 \mathrm{i} \mathrm{z}$ warunku istnienia sumy nieskończonego

ciagu geometrycznego wyznaczyč $\alpha_{1}\mathrm{i}q.$

35.2. $K\mathrm{a}\mathrm{t}$ miedzy prostymi jest równy katowi miedzy ich wektorami

normalnymi (odpowiednio zorientowanymi). Napisač równania danych

prostych $\mathrm{w}$ postaci ogólnej $\mathrm{i}\mathrm{u}\dot{\mathrm{z}}$ yč iloczynu skalarnego.

35.3. Rozwazyč przekrój sześcianu plaszczyzna symetrii (zawierajacy

środek $\mathrm{i}$ kolo wielkie danej kuli oraz przekroje czterech narozników). Szu-

kana krawed $\acute{\mathrm{z}}$ obliczyč za pomoca twierdzenia Pitagorasa dla odpowiedniego

trójkata $\mathrm{w}$ tym przekroju.

35.4. Uzasadnič, $\dot{\mathrm{z}}\mathrm{e}\mathrm{w}$ przedziale [-l, l] obie strony nierówności sa nieu-

jemne $\mathrm{i}$ podnieśč je do kwadratu. Wykresy nalezy wykonač dokladnie (leza

blisko siebie), zwracajac uwage na otoczenia punktów $x=0\mathrm{i}x=-1.$

35.5. Wyznaczyč dziedzine równania. Pomnozyč obie strony przez

$\sin 2x$. Zastosowač wzór na iloczyn sinusów $\mathrm{i}\mathrm{z}$ równości dwóch cosinusów

przejśč od razu do porównywania katów.

35.6. Napisač wzór na styczna do okregu $\mathrm{w}$ punkcie $\mathrm{l}\mathrm{e}\dot{\mathrm{z}}$ acym na nim

(por. wskazówka do zadania 6.2) $\mathrm{i}$ po podstawieniu wspólrzednych punktu $P$

wyznaczyč punkt styczności, dla którego styczna ma dodatni wspólczynnik

kierunkowy.
\end{document}
