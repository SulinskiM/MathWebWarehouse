\documentclass[a4paper,12pt]{article}
\usepackage{latexsym}
\usepackage{amsmath}
\usepackage{amssymb}
\usepackage{graphicx}
\usepackage{wrapfig}
\pagestyle{plain}
\usepackage{fancybox}
\usepackage{bm}

\begin{document}

133

35.7. Dane $r\mathrm{i}d$ jednoznacznie określaja $\mathrm{k}\mathrm{a}\mathrm{t}\alpha$ przy podstawie trapezu,

przy czym $\displaystyle \alpha>\frac{\pi}{3}$. Obwód wyrazič jako funkcje wysokości trapezu. Ustalič

dziedzine. Wartośč najwieksza funkcji wyznaczyč, badajac jej przedzialy

monotoniczności.

35.8. Wyznaczyč $y \mathrm{z}$ pierwszego równania $\mathrm{i}$ podstawič do drugiego.

Nastepnie skorzystač $\mathrm{z}\mathrm{t}\mathrm{o}\dot{\mathrm{z}}$ samości $(|\alpha|=b)\Leftrightarrow$($\alpha=b$ lub $\alpha=-b$) prawdzi-

wej dla $b\geq 0$. Otrzymane alternatywy prowadza do czterech przypadków

$m = -\displaystyle \frac{1}{2}, m = \displaystyle \frac{1}{2}, m = 1$ oraz pozostale $m$. Na rysunku zaznaczyč

odpowiednio wybrane proste $\mathrm{z}$ peku prostych (któremu odpowiada pierwsze

równanie ukladu) przechodzacych przez $P(0$, 2$).$
\end{document}
