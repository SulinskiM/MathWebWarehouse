\documentclass[a4paper,12pt]{article}
\usepackage{latexsym}
\usepackage{amsmath}
\usepackage{amssymb}
\usepackage{graphicx}
\usepackage{wrapfig}
\pagestyle{plain}
\usepackage{fancybox}
\usepackage{bm}

\begin{document}

137

Rozwiazanie zadania 2.1

I sposób. Rozwazmy wielomian $p_{n}(y) =y^{2n-1}+1$. Poniewaz $2n-1$

jest liczba nieparzysta dla $\mathrm{k}\mathrm{a}\dot{\mathrm{z}}$ dego $n\in N$, wiec $p(-1)=(-1)^{2n-1}+1=0.$

$\mathrm{Z}$ twierdzenia Bézouta wynika wiec, $\dot{\mathrm{z}}\mathrm{e}p_{n}(y)$ jest podzielny przez dwumian

$y+1, \mathrm{t}\mathrm{z}\mathrm{n}$. istnieje wielomian $q_{n}(y)$ stopnia $2n-2$ taki, $\dot{\mathrm{z}}\mathrm{e}$

$p_{n}(y)=y^{2n-1}+1=(y+1)q_{n}(y).$

(1)

Zauwazmy, $\dot{\mathrm{z}}\mathrm{e} w_{n}(x) = x^{4n-2}+1 =p_{n}(x^{2})$. Stad $\mathrm{i} \mathrm{z}$ (l) wynika, $\dot{\mathrm{z}}\mathrm{e}$

$w_{n}(x)=(x^{2}+1)q_{n}(x^{2})$. Poniewaz $q_{n}(x^{2})$ jest wielomianem stopnia $4n-4,$

wiec równośč ta dowodzi prawdziwości tezy.

Uwaga. Stosujac wzór skróconego mnozenia

$\alpha^{2n-1}+b^{2n-1}=(\alpha+b)(\alpha^{2n-2}-\alpha^{2n-3}b+-\alpha b^{2n-3}+b^{2n-2})$

$\mathrm{m}\mathrm{o}\dot{\mathrm{z}}$ na wielomian $q_{n}(y)$ napisač $\mathrm{w}$ postacijawnej. Niejest tojednak koniecz-

ne dla poprawności dowodu.

II sposób. Dowód indukcyjny. Rozwazmy funkcje zdaniowa zmien-

nej naturalnej $n$

$T(n)$ : $w_{n}(x)=x^{4n-2}+1$ jest podzielny przez $x^{2}+1.$

Sprawdzimy teraz, $\dot{\mathrm{z}}\mathrm{e}$ dla $T(n)$ obydwa zalozenia zasady indukcji mate-

matycznej sa spelnione.

$1^{\mathrm{O}}$ Sprawdzenie prawdziwości zdania $T(1).$

Mamy $w_{1}(x)=x^{4\cdot 1-2}+1=x^{2}+1$, czyli oczywiście dzieli $\mathrm{s}\mathrm{i}\mathrm{e}$ przez $x^{2}+1,$

a wiec $T(1)$ jest prawdziwe.

$2^{\circ}$ Wykazemy, $\dot{\mathrm{z}}\mathrm{e}$ implikacja $(T(n)\Rightarrow T(n+1))$ jest prawdziwa dla

$\mathrm{k}\mathrm{a}\dot{\mathrm{z}}$ dego $n\in N.$

Dowód. Niech $n$ bedzie dowolna ustalona liczba naturalna. Zalózmy,

$\dot{\mathrm{z}}\mathrm{e}$ zdanie $T(n)$ jest prawdziwe $\mathrm{t}\mathrm{z}\mathrm{n}$. istnieje wielomian $v_{n}(x)$ taki, $\dot{\mathrm{z}}\mathrm{e}$

$w_{n}(x) =x^{4n-2}+1 = (x^{2}+1)v_{n}(x)$. Wówczas korzystajac $\mathrm{z}$ tej równości

mamy

$w_{n+1}(x)=x^{4(n+1)-2}+1=x^{4n+2}+1=(x^{4n+2}+x^{4})-(x^{4}-1)$
\end{document}
