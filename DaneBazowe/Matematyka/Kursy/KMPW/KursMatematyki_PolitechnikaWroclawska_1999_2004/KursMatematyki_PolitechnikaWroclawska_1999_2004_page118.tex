\documentclass[a4paper,12pt]{article}
\usepackage{latexsym}
\usepackage{amsmath}
\usepackage{amssymb}
\usepackage{graphicx}
\usepackage{wrapfig}
\pagestyle{plain}
\usepackage{fancybox}
\usepackage{bm}

\begin{document}

138

$=x^{4}(x^{4n-2}+1)-(x^{2}+1)(x^{2}-1)=x^{4}(x^{2}+1)v_{n}(x)-(x^{2}+1)(x^{2}-1)$

$=(x^{2}+1)(x^{4}v_{n}(x)-x^{2}+1).$

Poniewaz $x^{4}v_{n}(x)-x^{2}+1$ jest wielomianem, wiec powyzsza równośč

oznacza, $\dot{\mathrm{z}}\mathrm{e}w_{n+1}(x)$ dzieli $\mathrm{s}\mathrm{i}\mathrm{e}$ przez $x^{2}+1$. To kończy dowód $2^{\circ}$

$\mathrm{Z}$ wykazanej prawdziwości warunków $1^{\circ} \mathrm{i} 2^{\circ}$ oraz $\mathrm{z}$ zasady indukcji

matematycznej wynika, $\dot{\mathrm{z}}\mathrm{e} T(n)$ jest prawdziwe dla $\mathrm{k}\mathrm{a}\dot{\mathrm{z}}$ dej liczby

naturalnej $n.$

Rozwiazanie zadania 3.8

Dziedzina nierówności jest R.

cosinus róznicy katów mamy

Poniewaz $\sqrt{3}=$ tg $\displaystyle \frac{\pi}{3}$, wiec ze wzoru na

$\cos x+\sqrt{3}\sin x=\cos x +$ tg $\displaystyle \frac{\pi}{3}\sin x=$

$\displaystyle \frac{\cos x\cos\frac{\pi}{3}+\sin x\sin\frac{\pi}{3}}{\cos\frac{\pi}{3}}=2\cos(x-\frac{\pi}{3})$

Nierównośč przyjmuje zatem postač $|2\displaystyle \cos(x-\frac{\pi}{3})| \leq \sqrt{2}$. Obie strony

nierówności sa nieujemne, wiec po podniesieniu do kwadratu dostajemy

nierównośč równowazna 2$\displaystyle \cos^{2}(x-\frac{\pi}{3}) \leq 1$. Stosujemy wzór l$+ \cos 2\gamma=$

$2\cos^{2}\gamma \mathrm{i}$ przeksztalcamy $\mathrm{j}\mathrm{a}$ do prostszej postaci $\displaystyle \cos(2x-\frac{2\pi}{3}) \leq 0$. Wiemy,

$\dot{\mathrm{z}}\mathrm{e}$ cosinus jest ujemny $\mathrm{w}$ II $\mathrm{i}$ III čwiartce, otrzymujemy wiec

$\displaystyle \frac{\pi}{2}+2k\pi\leq 2x-\frac{2\pi}{3}\leq\frac{3\pi}{2}+2k\pi$, czyli

$\displaystyle \frac{7\pi}{12}+k\pi\leq x\leq\frac{13\pi}{12}+k\pi, k\in \mathrm{Z}.$

(2)

Wyznaczamy cześč wspólna zbioru rozwiazań (2) $\mathrm{i}$ przedzialu $[0,3\pi]$, dosta-

jemy (podstawiamy kolejno $k=-1, 0$, 1, 2) odpowied $\acute{\mathrm{z}}.$

Odp. $ x\in [0,\displaystyle \frac{\pi}{12}]\cup [\displaystyle \frac{7\pi}{12},\frac{13\pi}{12}]\cup [\displaystyle \frac{19\pi}{12},\frac{25\pi}{12}]\cup [\displaystyle \frac{31\pi}{12},3\pi].$
\end{document}
