\documentclass[a4paper,12pt]{article}
\usepackage{latexsym}
\usepackage{amsmath}
\usepackage{amssymb}
\usepackage{graphicx}
\usepackage{wrapfig}
\pagestyle{plain}
\usepackage{fancybox}
\usepackage{bm}

\begin{document}

139

Rozwiazanie zadania 12.6

Oznaczmy przez $O$ środek okregu opisanego na trapezie, a przez $h$

wysokośč trapezu (rys. 25). Wówczas $P= \displaystyle \frac{1}{2}sh$, czyli $h= \displaystyle \frac{2P}{s}. \mathrm{Z}$ twier-

dzenia Pitagorasa $\mathrm{w} \triangle DEB$ otrzymujemy $d^{2}=h^{2}+\displaystyle \frac{s^{2}}{4}=\frac{16P^{2}+s^{4}}{4s^{2}}.$
\begin{center}
\includegraphics[width=78.492mm,height=74.472mm]{./KursMatematyki_PolitechnikaWroclawska_1999_2004_page119_images/image001.eps}
\end{center}
' $l D C$

{\it M}

$c r$

$c h$

$kA E s 2 B$

$\mathrm{R}\mathrm{y}\mathrm{s}$. 25

Z drugiej strony $\mathrm{z}$ twierdzenia

$0$ kacie wpisanym $\mathrm{w}$ okrag wynika

$\dot{\mathrm{z}}\mathrm{e} \angle DAE = \displaystyle \frac{1}{2}\angle DOB = \angle DOM,$

zatem trójkaty prostokatne $\triangle DAE$

$\mathrm{i} \triangle DOM$ maja identyczne katy,

czyli sa podobne. To pozwala napi-

sač proporcje $\displaystyle \frac{h}{c}= \displaystyle \frac{d}{2r}$, skad otrzy-

$2rh$

mujemy $c =$ Po podstawie-

$d$

niu obliczonej wartości $d$ mamy

$8Pr$

$c= \sqrt{16P^{2}+s^{4}}$. Ostatecznie ob-

wód wynosi $0=s+2c=s+\displaystyle \frac{16Pr}{\sqrt{16P^{2}+s^{4}}}$. Dane $P\mathrm{i}s$ wyznaczaja jedno-

znacznie $h\mathrm{i}d$. Zadanie ma zatem rozwiazanie, gdy promień $r$ jest wystar-

czajaco $\mathrm{d}\mathrm{u}\dot{\mathrm{z}}\mathrm{y}$, aby powstal trójkat $\triangle DOM, \mathrm{t}\mathrm{z}\mathrm{n}. r\displaystyle \geq\frac{1}{2}d=\frac{\sqrt{16P^{2}+s^{4}}}{4s}.$

Poprawnośč tego warunku, jak ijednoznacznośč rozwiazania, najlepiej widač

$\mathrm{z}$ opisu konstrukcji trapezu, który dla kompletności przedstawiamy ponizej.

Opis konstrukcji trapezu

1. $\mathrm{Z}$ odcinków $h \mathrm{i} \displaystyle \frac{s}{2}$, jako przyprostokatnych, konstruujemy trójkat

prostokatny $DEB$. Odcinek $BE$ przedluzamy $\mathrm{i}$ otrzymujemy prosta $k,$

a przez punkt $D$ prowadzimy prosta $l$ równolegla do $k.$

2. $\mathrm{Z}$ punktów $B\mathrm{i}D$ kreślimy luki okregów $0$ promieniu $r$, które przeci-

najac $\mathrm{s}\mathrm{i}\mathrm{e}$ daja środek okregu opisanego $O (\mathrm{z}$ dwóch punktów, $\mathrm{w}$ których

przecinaja $\mathrm{s}\mathrm{i}\mathrm{e}$ te luki, wybieramy $\mathrm{l}\mathrm{e}\dot{\mathrm{z}}\mathrm{a}\mathrm{c}\mathrm{y}$ blizej prostej $k$, która ma zawierač

dluzsza podstawe trapezu).
\end{document}
