\documentclass[a4paper,12pt]{article}
\usepackage{latexsym}
\usepackage{amsmath}
\usepackage{amssymb}
\usepackage{graphicx}
\usepackage{wrapfig}
\pagestyle{plain}
\usepackage{fancybox}
\usepackage{bm}

\begin{document}

140
\begin{center}
\includegraphics[width=73.512mm,height=64.968mm]{./KursMatematyki_PolitechnikaWroclawska_1999_2004_page120_images/image001.eps}
\end{center}
3. Ze środka $O$ kreślimy okrag

$0$ promieniu $r$. Okrag ten prze-

cinajac prosta $k$, wyznacza wierz-

cho ek $A$ (oraz przechodzi przez

$B)$. Podobnie, okrag ten przeci-

najac prosta $l$, wyznacza wierz-

cho ek $C (\mathrm{i}$ równocześnie prze-

chodzi przez $D$). Na rysunku 26

przedstawiono konstrukcje

trapezu dla danych liczbo-

wych zadania, $\mathrm{t}\mathrm{j}. P = 12 \mathrm{c}\mathrm{m}^{2},$

$h=3$ cmi $s=8$ cm.

$\mathrm{R}\mathrm{y}\mathrm{s}$. 26

$16Pr$

Odp. Obwód wynosi $s+$ a zadanie ma rozwiazanie, gdy

$\sqrt{16P^{2}+s^{4}}$'

{\it r}2$\geq$ --{\it Ps}22$+$--1{\it s}62.

Rozwiazanie zadania 21.7

Logarytm jest określony dla liczb dodatnich, wiec dziedzine równania

wyznaczaja warunki:

$D$: 

czyli $D$ : $(x\in(-4,1))\wedge(x^{3}-x^{2}-3x+5>0). \mathrm{W}$ celu rozwiazania równania

wszystkie skladniki zapiszemy jako logarytmy $0$ podstawie 4. D1a $x \in D$

jest $\displaystyle \log_{2}(1-x)=\frac{\log_{4}(1-x)}{\log_{4}2}=2\log_{4}(1-x)=\log_{4}(x-1)^{2}$ oraz $\displaystyle \frac{1}{2}=\log_{4}2$

$\mathrm{i}$ równanie przyjmuje postač

$\log_{4}(x-1)^{2}+\log_{4}(x+4)=\log_{4}(x^{3}-x^{2}-3x+5)+\log_{4}2.$

(3)

Korzystajac $\mathrm{z}$ wlasności logarytmu oraz $\mathrm{z}$ róznowartościowości funkcji

logarytmicznej, widzimy, $\dot{\mathrm{z}}\mathrm{e}$ równanie (3) jest równowazne ($\mathrm{w}$ dziedzinie)

równaniu algebraicznemu $(x-1)^{2}(x+4)=2(x^{3}-x^{2}-3x+5)$. Po wykonaniu

dzialań $\mathrm{i}$ przeniesieniu wszystkich skladników na jedna strong dostajemy

$x^{3}-4x^{2}+x+6=0.$

(4)
\end{document}
