\documentclass[a4paper,12pt]{article}
\usepackage{latexsym}
\usepackage{amsmath}
\usepackage{amssymb}
\usepackage{graphicx}
\usepackage{wrapfig}
\pagestyle{plain}
\usepackage{fancybox}
\usepackage{bm}

\begin{document}

141

Pierwiastki calkowite równania (4) sa podzie1nikami wyrazu wo1nego, $\mathrm{t}\mathrm{j}.$

liczby 6. Przez podstawienie sprawdzamy bezpośrednio, $\dot{\mathrm{z}}\mathrm{e}$ liczby $-1, 2\mathrm{i}3$

spelniaja (4), czy1i sa wszystkimi pierwiastkami tego równania (majac dwa

pierwiastki, np. $-1\mathrm{i}2$, trzeci $\mathrm{m}\mathrm{o}\dot{\mathrm{z}}$ na znalez$\acute{}$č $\mathrm{z}$ relacji $x_{1}x_{2}x_{3}=-6$). Liczby

2 $\mathrm{i}3$ znajduja $\mathrm{s}\mathrm{i}\mathrm{e}$ poza przedzialem $(-4,1)$, czyli $\mathrm{l}\mathrm{e}\dot{\mathrm{z}}$ a poza $D$. Natomiast

$-1\in(-4,1)$ oraz $(-1)^{3}-(-1)^{2}-3(-1)+5=6>0$, czyli liczba $-1$ jest

jedynym pierwiastkiem danego równania.

Odp. Równanie ma tylko jeden pierwiastek $\mathrm{i}$ jest nim liczba $-1.$

Rozwiazanie zadania 22.7

Dziedzine równania określaja warunki

$D$: 

czyli warunki $ x\neq k\pi$ oraz $ x\displaystyle \neq\frac{\pi}{2}+k\pi$. To daje ostatecznie

$D:x\displaystyle \neq k\frac{\pi}{2},$

$ k\in$ Z.

Dla $x\in D$ mnozymy obie strony równania przez $(\sin x\cos x)\mathrm{i}$ otrzymujemy

równanie równowazne

$\sin x+\cos x=\sqrt{8}\sin x\cos x.$

(5)

Korzystajac ze wzoru redukcyjnego oraz wzoru na róznice cosinusów,

mamy $\sin x+\cos x = \displaystyle \cos x-\cos(x+\frac{\pi}{2}) = -2\displaystyle \sin(-\frac{\pi}{4})\sin(x+\frac{\pi}{4}).$

Ponadto $\sqrt{8}\sin x\cos x = \sqrt{2}\sin 2x$, zatem równanie (5), po podzie1eniu

obu stron przez $\sqrt{2}, \mathrm{m}\mathrm{o}\dot{\mathrm{z}}$ na zapisač $\mathrm{w}$ postaci

$\displaystyle \sin(x+\frac{\pi}{4})=\sin 2x.$
\end{document}
