\documentclass[a4paper,12pt]{article}
\usepackage{latexsym}
\usepackage{amsmath}
\usepackage{amssymb}
\usepackage{graphicx}
\usepackage{wrapfig}
\pagestyle{plain}
\usepackage{fancybox}
\usepackage{bm}

\begin{document}

142

Rys. 27

Stad otrzymujemy alternatywe równań li-

niowych $x+\displaystyle \frac{\pi}{4} =  2x+2k\pi$ lub $x+\displaystyle \frac{\pi}{4} =$

$\pi-2x+2k\pi$, gdzie $k \in$ Z. Po standardo-

wych przeksztalceniach mamy $x= \displaystyle \frac{\pi}{4}+2k\pi$

lub $x = \displaystyle \frac{\pi}{4}+k\frac{2\pi}{3}$. Zauwazmy, $\dot{\mathrm{z}}\mathrm{e}$ pierwsza

seria zawiera $\mathrm{s}\mathrm{i}\mathrm{e} \mathrm{w}$ drugiej (rys. 27), a ta

$\mathrm{z}$ kolei jest zawarta $\mathrm{w}$ dziedzinie równania.

Odp. $x=\displaystyle \frac{\pi}{4}+k\frac{2\pi}{3},  k\in$ Z.

Rozwiazanie zadania 26.4

Oznaczmy przez $O$ spodek wysokości czworościanu, a przez $K, L$

jego rzuty prostokatne odpowiednio na przyprostokatne $BC\mathrm{i}AC$ podstawy

(rys. 28). Poniewaz $O$ jest środkiem okregu wpisanego $\mathrm{w} \triangle ABC$, wiec

{\it D} $|OK| = |OL| = r$, czyli punkty

$L |KC|=|LC|$.   (6)

$r$ Mamy $\triangle DOK \equiv \triangle DOL$, gdyz
\begin{center}
\includegraphics[width=66.240mm,height=81.792mm]{./KursMatematyki_PolitechnikaWroclawska_1999_2004_page122_images/image001.eps}
\end{center}
{\it E}

{\it A} $\cdot C$

$\alpha$

{\it S}

{\it O} $r K$

{\it B}

$O, K, L \mathrm{i}$ wierzcho ek kata

prostego $C$ tworza kwadrat $0$ bo-

ku $r$. Stad

oba sa prostokatne $\mathrm{i}$ maja takie

same przyprostokatne. Stad

$|DK| = |DL|$. Poniewaz wyso-

kośč DO jest prostopad a do pod-

stawy, wiec DO $\perp BC$. Ponad-

{\it to OK} $\perp BC. \mathrm{Z}$ twierdzenia

$0$ trzech prostopad ych wniosku-

$\mathrm{R}\mathrm{y}\mathrm{s}$. 28

jemy, $\dot{\mathrm{z}}\mathrm{e} DK \perp BC$. Analogi-

cznie stwierdzamy, $\dot{\mathrm{z}}\mathrm{e} DL\perp AC.$

Wynika stad, $\dot{\mathrm{z}}\mathrm{e}\triangle DKC\mathrm{i}\triangle DLC$ sa przystajacymi trójkatami prostokat-

nymi $\mathrm{i}\mathrm{w}$ konsekwencji

$\angle DCK=\angle DCL$.   (7)

Niech $E$ oznacza rzut prostokatny punktu $K$ na krawed $\acute{\mathrm{z}} DC$. Ze

wzorów (6) $\mathrm{i}$ (7) oraz $\mathrm{z}$ II cechy przystawania trójkatów (bkb) wynika, $\dot{\mathrm{z}}\mathrm{e}$

$\triangle KCE \equiv \triangle LCE$, a stad $ LE\perp DC$. To oznacza, $\dot{\mathrm{z}}\mathrm{e}$ krawed $\acute{\mathrm{z}} DC$ jest
\end{document}
