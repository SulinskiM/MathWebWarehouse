\documentclass[a4paper,12pt]{article}
\usepackage{latexsym}
\usepackage{amsmath}
\usepackage{amssymb}
\usepackage{graphicx}
\usepackage{wrapfig}
\pagestyle{plain}
\usepackage{fancybox}
\usepackage{bm}

\begin{document}

143

prostopadla do plaszczyzny wyznaczonej przez punkty $K, L\mathrm{i}E\mathrm{i}\mathrm{w}$ kon-

sekwencji $\angle KEL=\beta. \mathrm{Z}$ porównania trójkatow równoramiennych $\triangle KCL$

$\mathrm{i} \triangle KEL$, majacych wspólna podstawe oraz $|KE| < |KC| (|KE|$ jest

przyprostokatna), wnioskujemy, $\dot{\mathrm{z}}\mathrm{e} \angle KEL = \beta > \angle KCL = \displaystyle \frac{\pi}{2}$, zatem

dziedzina dla kata $\beta$ jest przedzial $(\displaystyle \frac{\pi}{2},\pi).$
\begin{center}
\includegraphics[width=51.720mm,height=54.408mm]{./KursMatematyki_PolitechnikaWroclawska_1999_2004_page123_images/image001.eps}
\end{center}
{\it D}

$(\cdot E$

{\it O S}

Rys. 29

{\it C}
\begin{center}
\includegraphics[width=66.804mm,height=90.168mm]{./KursMatematyki_PolitechnikaWroclawska_1999_2004_page123_images/image002.eps}
\end{center}
{\it A  L C}

$\alpha$  {\it r}

{\it r}

{\it S}

{\it O K}

{\it r}

{\it B}

Rys. 30

$\mathrm{W}$ celu wyznaczenia wysokości czworościanu oznaczmy przez $S$ środek

kwadratu OKCL. Wówczas $|ES|= |SK|\displaystyle \mathrm{c}\mathrm{t}\mathrm{g}\frac{\beta}{2}=\frac{r}{\sqrt{2}}\mathrm{c}\mathrm{t}\mathrm{g}\frac{\beta}{2}.$ Poprowad $\acute{\mathrm{z}}\mathrm{m}\mathrm{y}$

plaszczyzne przechodzaca przez $DO$ oraz przez $C$. Przekrój czworościanu ta

plaszczyzna pokazano na rysunku 29. $\mathrm{Z}$ twierdzenia Pitagorasa $\mathrm{w}\triangle ESC$

mamy $|EC|^{2}=|SC|^{2}-|ES|^{2}=\displaystyle \frac{r^{2}}{2}-\frac{r^{2}}{2}\mathrm{c}\mathrm{t}\mathrm{g}^{2}\frac{\beta}{2}=r^{2}\frac{-\cos\beta}{2\sin^{2}\frac{\beta}{2}}. \mathrm{Z}$ podobieństwa

trójkatów $\triangle ESC\mathrm{i}\triangle DOC$ dostajemy proporcje $\displaystyle \frac{H}{|OC|} = \displaystyle \frac{|ES|}{|EC|}$. Stad

$H=\displaystyle \frac{|OC||ES|}{|EC|}=\frac{r^{2}\mathrm{c}\mathrm{t}\mathrm{g}\frac{\beta}{2}}{r\sqrt{\frac{-\cos\beta}{2\sin^{2}\frac{\beta}{2}}}}=r\sqrt{2}\frac{\cos\frac{\beta}{2}}{\sqrt{-\cos\beta}}.$

(8)
\end{document}
