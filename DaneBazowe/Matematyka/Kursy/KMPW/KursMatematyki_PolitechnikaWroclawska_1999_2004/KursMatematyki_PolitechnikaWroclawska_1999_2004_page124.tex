\documentclass[a4paper,12pt]{article}
\usepackage{latexsym}
\usepackage{amsmath}
\usepackage{amssymb}
\usepackage{graphicx}
\usepackage{wrapfig}
\pagestyle{plain}
\usepackage{fancybox}
\usepackage{bm}

\begin{document}

144

Dla obliczenia pola podstawy czworościanu (rys. 30) zauwazmy, $\dot{\mathrm{z}}\mathrm{e}$

$|AC|=|LC|+|AL|=r+r\displaystyle \mathrm{c}\mathrm{t}\mathrm{g}\frac{\alpha}{2}$ oraz $|BC|=|AC|$ tg $\alpha$. Stad mamy

$P_{p}=\displaystyle \frac{1}{2}|AC|$

$|BC|=\displaystyle \frac{1}{2}r^{2}(\mathrm{c}\mathrm{t}\mathrm{g}\frac{\alpha}{2}+1)^{2}$ tg $\displaystyle \alpha=\frac{1}{2}r^{2}\frac{(\sin\frac{\alpha}{2}+\cos\frac{\alpha}{2})^{2}}{\sin^{2}\frac{\alpha}{2}}$ tg $\alpha$

$\mathrm{i}$ ostatecznie

$P_{p}=r^{2}\displaystyle \frac{1+\sin\alpha}{\cos\alpha}$ ctg $\displaystyle \frac{\alpha}{2}.$

(9)

$\mathrm{Z}$ równości (8) $\mathrm{i}$ (9) otrzymujemy

$V=\displaystyle \frac{1}{3}P_{p}H=\frac{\sqrt{2}}{3}r^{3}\frac{1+\sin\alpha}{\cos\alpha\sqrt{-\cos\beta}}\cos\frac{\beta}{2}$ ctg $\displaystyle \frac{\alpha}{2}.$

Odp. Objetośč czworościanu wynosi $\displaystyle \frac{\sqrt{2}}{3}r^{3}\frac{1+\sin\alpha}{\cos\alpha\sqrt{-\cos\beta}}\cos\frac{\beta}{2}\mathrm{c}\mathrm{t}\mathrm{g}\frac{\alpha}{2}.$

Rozwiazanie zadania 28.2

Aby nierównośč

$\displaystyle \frac{2px^{2}+2px+1}{x^{2}+x+2-p^{2}}\geq 2$

(10)

byla spelniona dla $\mathrm{k}\mathrm{a}\dot{\mathrm{z}}$ dej liczby rzeczywistej, jej dziedzina musi byč $\mathrm{R}$, czyli

trójmian kwadratowy $\mathrm{w}$ mianowniku nie $\mathrm{m}\mathrm{o}\dot{\mathrm{z}}\mathrm{e}$ mieč pierwiastków rzeczy-

wistych. Stad otrzymujemy warunek $\triangle_{0} = 1-4(2-p^{2}) =4p^{2}-7< 0.$

Nierównośč ta jest spelniona dla

$p\displaystyle \in(-\frac{\sqrt{7}}{2},\frac{\sqrt{7}}{2})$

(11)

Dla parametru $p$ spelniajacego warunek (ll) mianownik lewej strony

(10) jest dodatni na calej prostej, wiec po pomnozeniu obu stron

(10) przez ten mianownik otrzymujemy nierównośč równowazna

$2px^{2}+2px+1\geq 2x^{2}+2x+4-2p^{2}$, a po uporzadkowaniu

$2(p-1)x^{2}+2(p-1)x+2p^{2}-3\geq 0.$

(12)
\end{document}
