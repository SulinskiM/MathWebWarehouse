\documentclass[a4paper,12pt]{article}
\usepackage{latexsym}
\usepackage{amsmath}
\usepackage{amssymb}
\usepackage{graphicx}
\usepackage{wrapfig}
\pagestyle{plain}
\usepackage{fancybox}
\usepackage{bm}

\begin{document}

145

Dla $p=1$lewa strona (12) jest równa $-1\mathrm{i}$ nierównośč niejest spelniona dla

$\dot{\mathrm{z}}$ adnego $x$. Gdy $p<1\mathrm{t}\mathrm{z}\mathrm{n}$. wspólczynnik przy $x^{2}$ jest ujemny, nierównośč

(12) nie $\mathrm{m}\mathrm{o}\dot{\mathrm{z}}\mathrm{e}$ byč spelniona dla wszystkich $x$ (gdyz,,ramiona paraboli sa

skierowane $\mathrm{w}$ dól''). Natomiast dla $p>1$, nierównośč (12) bedzie spe1niona

dla wszystkich liczb rzeczywistych wtedy $\mathrm{i}$ tylko wtedy, gdy

$\triangle_{1} =4(p-1)^{2}-8(p-1)(2p^{2}-3) \leq 0$. Po podzieleniu obu stron przez

wyrazenie dodatnie $4(p-1)$ otrzymujemy $-4p^{2}+p+5\leq 0$, skad od razu

mamy $ p\leq$ -llub $ p\geq \displaystyle \frac{5}{4}$. Poniewaz $\displaystyle \frac{5}{4}< \displaystyle \frac{\sqrt{7}}{2}\mathrm{i}$ zalozyliśmy, $\dot{\mathrm{z}}\mathrm{e}p>1$, wiec

laczac wszystkie otrzymane warunki dostajemy ostatecznie $ p\in [\displaystyle \frac{5}{4},\frac{\sqrt{7}}{2}$).

Odp. Nierównośč jest spelniona dla $\mathrm{k}\mathrm{a}\dot{\mathrm{z}}$ dej liczby rzeczywistej, gdy

$ p\in [\displaystyle \frac{5}{4},\frac{\sqrt{7}}{2}).$

Rozwiazanie zadania 29.8

$\mathrm{Z}$ postaci ciagu odczytujemy wyraz poczatkowy $\alpha_{0}=x+1$ oraz iloraz

$q = -x^{2}$ Jeśli $x = -1$, to wszystkie wyrazy ciagu sa zerami $\mathrm{i}$ suma

$S(-1)=0$. Gdy $x\neq-1$, wówczas warunkiem istnienia sumy nieskończonego

ciagu geometrycznegojest $|q|<1$, czyli $|-x^{2}|=x^{2}<1$, skad od razu otrzy-

mujemy $x\in(-1,1)$. Ostatecznie dziedzina sumy $S(x)$ jest $D=[-1$, 1).

Korzystajac ze wzoru na sume nieskończonego ciagu geometrycznego,

dostajemy $S(x) =\displaystyle \frac{x+1}{1-(-x^{2})}=\frac{x+1}{x^{2}+1},  x\in (-1,1)$. Wzór ten pozostaje

prawdziwy takze dla $x=-1$. Dlatego $\mathrm{m}\mathrm{o}\dot{\mathrm{z}}$ na napisač

$S(x)=\displaystyle \frac{x+1}{x^{2}+1}, x\in D=[-1$, 1$)$.   (13)

Dalsze postepowanie sprowadza $\mathrm{s}\mathrm{i}\mathrm{e}$ do wyznaczenia wartości namniejszej

$\mathrm{i}$ najwiekszej funkcji wymiernej $S(x)$, danej wzorem (13). Zauwazmy, $\dot{\mathrm{z}}\mathrm{e}$

mianownik jest dodatni, a licznik nieujemny, zatem $S(x) \geq 0$ dla wszyst-

kich $x$. Stad wynika, $\dot{\mathrm{z}}\mathrm{e}$ najmniejsza wartościa tej funkcji jest 0 $\mathrm{i}$ jest ona

osiagana dla $x=-1.$

Dla znalezienia wartości najwiekszej wykorzystamy pochodna funkcji

$S(x).$

$S'(x)=\displaystyle \frac{1\cdot(x^{2}+1)-2x(x+1)}{(x^{2}+1)^{2}}=\frac{-x^{2}-2x+1}{(x^{2}+1)^{2}}.$
\end{document}
