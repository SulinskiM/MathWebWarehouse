\documentclass[a4paper,12pt]{article}
\usepackage{latexsym}
\usepackage{amsmath}
\usepackage{amssymb}
\usepackage{graphicx}
\usepackage{wrapfig}
\pagestyle{plain}
\usepackage{fancybox}
\usepackage{bm}

\begin{document}

146

Miejsca zerowe pochodnej spelniaja równanie $-x^{2}-2x+1=0$. Stad dosta-

jemy $\triangle= 8$ oraz $x_{1} = -1-\sqrt{2}, x_{2} = -1+\sqrt{2}$. Tylko $x_{2} \in D$. Mamy

$S(x_{2}) = \displaystyle \frac{\sqrt{2}}{1+2-2\sqrt{2}+1} = \displaystyle \frac{1+\sqrt{2}}{2}$ oraz $\displaystyle \lim_{x\rightarrow 1-}S(x) = \displaystyle \lim_{x\rightarrow 1-}\frac{x+1}{x^{2}+1} = 1.$

Poniewaz $\displaystyle \frac{1+\sqrt{2}}{2}>1$, wiec najwieksza wartościa funkcji jest $\displaystyle \frac{1+\sqrt{2}}{2}.$

Odp. Wartośč najmniejsza sumy danego nieskończonego ciagu geo-

metrycznego wynosi 0 $\mathrm{i}$ jest osiagana dla $x = -1$, a wartośč najwieksza

tej sumy wynosi $\displaystyle \frac{1+\sqrt{2}}{2}\mathrm{i}$ jest osiagana dla $x=-1+\sqrt{2}.$

Rozwiazanie zadania 30.7

Dziedzina nierówności

$|2^{x}-3|\leq 2^{1-x}$

(14)

jest R. Nierównośč $\mathrm{t}\mathrm{e}\mathrm{r}$ozwia $\dot{\mathrm{z}}$ emy przez podstawienie $2^{x}=t, t>0$. Mamy

$2^{1-x}=22^{-x}=2\displaystyle \frac{1}{t},$ wiec po podstawieniu nierównośč (14) przyjmie postač

$|t-3|\displaystyle \leq\frac{2}{t}$. Stad od razu przechodzimy do nierówności podwójnej

$-\displaystyle \frac{2}{t}\leq t-3\leq\frac{2}{t},$

$t>0.$

(15)

Ze wzgledu na dodatni znak niewiadomej $t \mathrm{m}\mathrm{o}\dot{\mathrm{z}}$ emy $\mathrm{t}\mathrm{e}$ nierównośč pomnozyč

przez $t\mathrm{i}$ otrzymamy nastepujacy uklad nierówności kwadratowych

$\left\{\begin{array}{l}
t^{2}-3t+2\geq 0\\
t^{2}-3t-2\leq 0
\end{array}\right.$

$t>0.$

Pierwsza nierównośč powyzszego ukladu jest spelniona dla $t \leq 1 \mathrm{i}t \geq 2,$

czyli po uwzglednieniu warunku $t> 0$ dla $ t\in (0,1]\cup[2,\infty)$. Dla drugiej

nierówności mamy $\triangle_{2} = 17, t_{1}'' = \displaystyle \frac{3-\sqrt{17}}{2} < 0, t_{2}'' = \displaystyle \frac{3+\sqrt{17}}{2} \in (3,4).$

Druga nierównośč jest zatem spelniona dla $t \in (0,t_{2}''$]. Cześč wspólna

zbiorów rozwiazań obu nierówności ma postač $(0,1[\cup[2,t_{2}'']$. Poniewaz funkcja

$t = 2^{x}$ jest rosnaca, wiec zbiór rozwiazań nierówności (14) ma postač

$(-\infty,0]\cup[1,x_{0}]$, gdzie $x_{0}=\log_{2}t_{2}''\in(1,2).$
\end{document}
