\documentclass[a4paper,12pt]{article}
\usepackage{latexsym}
\usepackage{amsmath}
\usepackage{amssymb}
\usepackage{graphicx}
\usepackage{wrapfig}
\pagestyle{plain}
\usepackage{fancybox}
\usepackage{bm}

\begin{document}

147

Wykresy funkcji wystepujacych po obu stronach nierówności (14) otrzy-

mujemy przez translacje $\mathrm{i}$ odbicia symetryczne standardowej krzywej

$\mathrm{I}^{\urcorner}$ : $y = 2^{x}$ Wykres krzywej $y = |2^{x}-3|$ dostajemy przez translacje $\mathrm{I}^{\urcorner}$

$0$ wektor $[0,-3]$, a nastepnie odbicie symetryczne cześci $\mathrm{l}\mathrm{e}\dot{\mathrm{z}}$ acej pod osia

odcietych wzgledem tej osi. Krzywa ta ma asymptote pozioma lewostronna

$y=3$. Natomiast krzywa $y=2^{1-x}$ dostajemy przez odbicie symetryczne $\mathrm{I}^{\urcorner}$

wzgledem osi rzednych, a nastepnie translacje $0$ wektor $($1, $0)$. Wykresy sa

przedstawione na rysunku 31.
\begin{center}
\includegraphics[width=110.484mm,height=74.472mm]{./KursMatematyki_PolitechnikaWroclawska_1999_2004_page127_images/image001.eps}
\end{center}
{\it y}

$y=2^{1-x}  \Gamma$

3  $y=|2^{x}-3|$

2

1

$-1$  0 1  $x_{0}$  3 4  {\it x}

Rys. 31

Odp. Zbiorem rozwiazań

(-00, $ 0]\cup [1,\displaystyle \log_{2}\frac{3+\sqrt{17}}{2}].$

nierówności jest

suma

przedzialów

Rozwiazanie zadania 31.7

Przy rozwiazywaniu zadania skorzystamy nastepujacej wlasności wek-

torów na plaszczy $\acute{\mathrm{z}}\mathrm{n}\mathrm{i}\mathrm{e}$:

$\mathrm{T}\mathrm{w}\mathrm{i}\mathrm{e}\mathrm{r}\mathrm{d}\mathrm{z}\mathrm{e}\mathrm{n}\mathrm{i}\mathrm{e}$. {\it Jeśli wektory} $\vec{u}i\vec{v}sq_{f}$ {\it prostopadte} $i$ {\it majq} $t_{G}$

$samq_{f}$ {\it dtugośč oraz} $\vec{u}=(\alpha,b)$, {\it to} $\vec{v}=(b,-\alpha) lub\vec{v}=(-b,\alpha).$

Przez $B$ oznaczmy wierzcholek kwadratu $\mathrm{l}\mathrm{e}\dot{\mathrm{z}}\mathrm{a}\mathrm{c}\mathrm{y}$ na prostej $l$, a przez

$D$ jego wierzcholek $\mathrm{l}\mathrm{e}\dot{\mathrm{z}}\mathrm{a}\mathrm{c}\mathrm{y}$ na prostej $k$. Korzystajac $\mathrm{z}$ równań prostych,

$\mathrm{m}\mathrm{o}\dot{\mathrm{z}}$ emy napisač $B(2y-1,y), D(4-3z,z)$, gdzie $y, z$ sa niezna-

$\rightarrow$

nymi rzednymi tych wierzcholków, zatem $AB= [2y-7,y-1]$ oraz
\end{document}
