\documentclass[a4paper,12pt]{article}
\usepackage{latexsym}
\usepackage{amsmath}
\usepackage{amssymb}
\usepackage{graphicx}
\usepackage{wrapfig}
\pagestyle{plain}
\usepackage{fancybox}
\usepackage{bm}

\begin{document}

148

$\vec{AD}= [-3z-2,z-1]$. Poniewaz $\vec{AB}\perp\vec{AD}$ oraz oba wektory sa tej samej

dlugości, wiec $\mathrm{z}$ powyzszego twierdzenia otrzymujemy $\mathrm{z}$ porównania odpo-

wiednich wspólrzednych dwa uklady równań liniowych:

$\left\{\begin{array}{l}
2y-7=-z+1\\
-3z-2=y-1
\end{array}\right.$

oraz

$\left\{\begin{array}{l}
2y-7=z-1\\
3z+2=y-1
\end{array}\right.$

Po rozwiazaniu pierwszego ukladu dostajemy $y=3, z=0$, czyli $B_{1}(5,3),$

$\rightarrow$

$\rightarrow$

$\rightarrow$

$D_{1}(4,0)$ oraz $AB= [-1,2]$. Poniewaz $AB=DC$, wiec $C_{1}(3,2)$. Rozwiaza-

niem drugiego ukladu jest $y = 5, z = -2$, czyli $B_{2}(9,5), D_{2}(10,-2)$

$\rightarrow$

$\rightarrow$

$\mathrm{i}$ podobnie jak poprzednio $AB=DC=[4,-3]$, skad $C_{2}(13,2)$. Rozwiazanie

ilustruje rysunek 32.

Uwaga. Ze wzgledu na ogólnie przyjety sposób oznaczania wierzcholków

wielokatów na rysunku 32 przestawiono 1itery $B \mathrm{i} D$, oznaczajac

$B_{2}(10,-2)\mathrm{i}D_{2}(9,5).$
\begin{center}
\includegraphics[width=108.108mm,height=65.028mm]{./KursMatematyki_PolitechnikaWroclawska_1999_2004_page128_images/image001.eps}
\end{center}
{\it y}

5

$D_{2}$

$B_{1}$

3  $C_{1}$

$C_{2}$

1  {\it A}

0 1  $D_{1}$  7  10  13 x

$B_{2}$

Rys. 32

Odp. Istnieja dwa kwadraty spelniajace warunki zadania. Ich wierz-

cholkami, oprócz wierzcholka $A$, sa punkty $B_{1}(5,3), C_{1}(3,2), D_{1}(4,0)$ oraz

$B_{2}(10,-2), C_{2}(13,2), D_{2}(9,5).$

Rozwiazanie zadania 32.7

Równanie stycznej do wykresu funkcji $f(x)\mathrm{w}$ punkcie $S(x_{0},f(x_{0}))$ ma

postač ogólna $y-f(x_{0}) =f'(x_{0})(x-x_{0})$. Poniewaz $\mathrm{w}$ naszym przypadku
\end{document}
