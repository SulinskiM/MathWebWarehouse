\documentclass[a4paper,12pt]{article}
\usepackage{latexsym}
\usepackage{amsmath}
\usepackage{amssymb}
\usepackage{graphicx}
\usepackage{wrapfig}
\pagestyle{plain}
\usepackage{fancybox}
\usepackage{bm}

\begin{document}

149

jest $f(x)=x^{4}-2x^{2}$ oraz $f'(x)=4x^{3}-4x$, wiec równanie stycznej przyjmie

postač

$y-(x_{0}^{4}-2x_{0}^{2})=(4x_{0}^{3}-4x_{0})(x-x_{0})$.   (16)

Punkt $P(1,-1)\mathrm{l}\mathrm{e}\dot{\mathrm{z}}\mathrm{y}$ na tej stycznej, wiec niewiadoma $x_{0}$ spelnia równanie

$-1+2x_{0}^{2}-x_{0}^{4}=4x_{0}(x_{0}^{2}-1)(1-x_{0})$. Po wylaczeniu wspólnych czynników

$\mathrm{i}$ uporzadkowaniu dostajemy $(x_{0}^{2}-1)(x_{0}-1)(3x_{0}-1)=0$. Równanie (16)

ma wiec trzy pierwiastki $-1$, l oraz $\displaystyle \frac{1}{3}.$

Po podstawieniu do równania (16) pierwiastków $x_{0} = -1 \mathrm{i} x_{0} = 1$

otrzymujemy $\mathrm{t}\mathrm{e}$ sama prosta $p$: $y+1=0$. Prosta ta jest wiec styczna do

wykresu $f$ równocześnie $\mathrm{w}$ punktach $P(-1,1)$ oraz $Q(-1,-1)$. Poniewaz

$f(x)=x^{4}-2x^{2}\geq-1$ (inaczej $(x^{2}-1)^{2}\geq 0$) dla wszystkich $x\mathrm{i}$ równośč ma

miejsce jedynie dla $x=-1\mathrm{i}x=1$, wiec styczna $p$ ma dwa punkty wspólne

$\mathrm{z}$ wykresem $f.$

Dla $x_{0}=\displaystyle \frac{1}{3}$ równanie (16) przyjmuje postač $l$ : $32x+27y-5=0. \mathrm{W}$ celu

określenia liczby punktów wspólnych stycznej $l\mathrm{z}$ wykresem $f$ nalezy określič

liczbe róznych pierwiastków równania $x^{4}-2x^{2} = \displaystyle \frac{5-32x}{27}, \mathrm{t}\mathrm{j}$. równania

$27x^{4}-54x^{2}+32x-5 = 0$. Ze wzgledu na stycznośč $\mathrm{w}$ punkcie $x_{0} = \displaystyle \frac{1}{3}$

równanie to ma podwójny pierwiastek $\displaystyle \frac{1}{3}$ oraz pierwiastek l (punkt $P\mathrm{l}\mathrm{e}\dot{\mathrm{z}}\mathrm{y}$

na wykresie $f$), zatem, jako równanie czwartego stopnia, ma takze czwarty

pierwiastek rzeczywisty, który obliczamy $\mathrm{z}$ równości $x_{1}x_{2}x_{3}x_{4}=\displaystyle \frac{-5}{27}$, czyli

$\mathrm{w}$ naszym przypadku $\displaystyle \frac{1}{9}x_{4} = \displaystyle \frac{-5}{27}$, skad $x_{4} = \displaystyle \frac{-5}{3}$. Styczna $l$ ma zatem

trzy punkty wspólne $\mathrm{z}$ wykresem $f$: $P, S(\displaystyle \frac{1}{3},-\frac{17}{81})$ oraz $A(-\displaystyle \frac{5}{3},\frac{175}{81}).$

$\mathrm{W}$ punkcie $A$ styczna $l$ przecina wykres $f$. Dla sporzadzenia rysunku

zauwazmy, $\dot{\mathrm{z}}\mathrm{e}f(x)$ jest funkcja parzysta. Liczba $x=0$ jest pierwiastkiem

podwójnym równania $x^{4}-2x^{2}=0$, co oznacza, $\dot{\mathrm{z}}\mathrm{e}$ wykres $f$ jest styczny do

osi odcietych $\mathrm{w}$ poczatku ukladu. Pozostale miejsca zerowe funkcji to $-\sqrt{2}$

$\mathrm{i}\sqrt{2}$. Kreślac styczne $y=0, l$ oraz $p\mathrm{i}$ zaznaczajac punkty styczności oraz

punkty $(\sqrt{2},0), B(\displaystyle \frac{5}{3},\frac{175}{81}), \mathrm{m}\mathrm{o}\dot{\mathrm{z}}$ emy narysowač wykres funkcji na $(0,\infty),$

a przez odbicie symetryczne takze $\mathrm{w}$ (-00, 0). Wykres przedstawiono na ry-

sunku 33.
\end{document}
