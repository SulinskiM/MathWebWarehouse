\documentclass[a4paper,12pt]{article}
\usepackage{latexsym}
\usepackage{amsmath}
\usepackage{amssymb}
\usepackage{graphicx}
\usepackage{wrapfig}
\pagestyle{plain}
\usepackage{fancybox}
\usepackage{bm}

\begin{document}

19

Praca kontrolna

nr l

8.1. Suma wszystkich wyrazów nieskończonego ciagu geometrycznego

wynosi 2040. Jeś1i pierwszy wyraz tego ciagu zmniejszymy $0 172,$

a jego iloraz zwiekszymy 3-krotnie, to suma wszystkich wyrazów tak

otrzymanego ciagu wyniesie 2000. Wyznaczyč i1oraz $\mathrm{i}$ pierwszy wyraz

danego ciagu.

8.2. Obliczyč wszystkie te skladniki rozwiniecia dwumianu $(\sqrt{3}+\sqrt[3]{2})^{11}$,

które sa liczbami calkowitymi.

8.3. Narysowač staranny wykres funkcji $f(x) = |x^{2}-2|x|-3| \mathrm{i}$ na jego

podstawie podač ekstrema lokalne oraz przedzialy monotoniczności tej

funkcji.

8.4. Rozwiazač nierównośč

$x+1\geq\log_{2}(4^{x}-8).$

8.5. $\mathrm{W}$ ostroslupie prawidlowym trójkatnym krawed $\acute{\mathrm{z}}$ podstawy ma dlugośč

$\alpha$, a polowa kata plaskiego przy wierzcholkujest równa katowi nachyle-

nia ściany bocznej do podstawy. Obliczyč objetośč ostroslupa. Sporza-

dzič odpowiednie rysunki.

8.6. Znalez$\acute{}$č wszystkie wartości parametru $p$, dla których trójkat $KLM$

$0$ wierzcholkach $K(1,1), L(5,0)\mathrm{i}M(p,p-1)$ jest prostokatny. Roz-

wiazanie zilustrowač rysunkiem.

8.7. Rozwiazač równanie

--ssiinn35{\it xx}$=$--ssiinn46{\it xx}.

8.8. Przez punkt $P\mathrm{l}\mathrm{e}\dot{\mathrm{z}}\mathrm{a}\mathrm{c}\mathrm{y}$ wewnatrz trójkata $ABC$ poprowadzono proste

równolegle do wszystkich boków trójkata. Pola utworzonych $\mathrm{w}$ ten

sposób trzech mniejszych trójkatów $0$ wspólnym wierzcholku $P$ wyno-

$\mathrm{s}\mathrm{z}\mathrm{a}$ odpowiednio $S_{1}, S_{2}, S_{3}$. Obliczyč pole $S$ trójkata $ABC.$
\end{document}
