\documentclass[a4paper,12pt]{article}
\usepackage{latexsym}
\usepackage{amsmath}
\usepackage{amssymb}
\usepackage{graphicx}
\usepackage{wrapfig}
\pagestyle{plain}
\usepackage{fancybox}
\usepackage{bm}

\begin{document}

150
\begin{center}
\includegraphics[width=90.012mm,height=83.916mm]{./KursMatematyki_PolitechnikaWroclawska_1999_2004_page130_images/image001.eps}
\end{center}
{\it y}

3

{\it l}

{\it A  B}

$1_{11}$  2  111

1

$-2  -1$  0 1  {\it x}

{\it S}  $\sqrt{2}2$

{\it p  Q  P}

Rys. 33

Odp. $\mathrm{s}_{\mathrm{a}}$ dwie takie styczne jedna $0$ równaniu $y= -1$, która ma dwa

punkty wspólne $\mathrm{z}$ wykresem funkcji $f(x)$, oraz druga $0$ równaniu

$32x+27y-5=0$ majaca trzy punkty wspólne $\mathrm{z}$ wykresem.

Rozwiazanie zadania 34.5

Wprowad $\acute{\mathrm{z}}\mathrm{m}\mathrm{y}$ nastepujace zdarzenia:

$A-$ Jaś wyciagnie co najmniej trzy monety;

$B_{i}-$ za pierwszym razem zostanie wylosowana moneta $0$ nominale $i \mathrm{z}l,$

$i=1$, 2, 5;

$C_{j}-$ dla uiszczenia zaplaty Jaś wyciagnie $j$ monet, $j=1$, 2, 3, 4.

Wówczas $A' = C_{1}\cup C_{2} \mathrm{i}$ oba skladniki sa rozlaczne. Zauwazmy, $\dot{\mathrm{z}}\mathrm{e}$

$C_{1} = B_{5}$ oraz $B_{1}\cup B_{2}\cup B_{5} = \Omega$. Ponadto $P(B_{1}) = \displaystyle \frac{1}{2}, P(B_{2}) = \displaystyle \frac{1}{3}$

$\displaystyle \mathrm{i}P(B_{5})=P(C_{1})=\frac{1}{6}$. Ze wzoru na prawdopodobieństwo calkowite mamy

$P(C_{2})=P(C_{2}|B_{1})P(B_{1})+P(C_{2}|B_{2})P(B_{2})+P(C_{2}|B_{5})P(B_{5})$.   (17)

Mamy $P(C_{2}|B_{1})=\displaystyle \frac{1}{5}$, gdyz za drugim razem Jaś musi wyciagnač mone-

$\mathrm{t}\mathrm{e} 5 \mathrm{z}l$ spośród 5 monet $\mathrm{w}$ portmonetce. Podobnie $P(C_{2}|B_{2}) = \displaystyle \frac{2}{5}$ (za
\end{document}
