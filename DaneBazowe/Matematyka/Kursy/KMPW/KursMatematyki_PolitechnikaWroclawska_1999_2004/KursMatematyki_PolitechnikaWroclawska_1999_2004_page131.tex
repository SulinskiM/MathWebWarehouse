\documentclass[a4paper,12pt]{article}
\usepackage{latexsym}
\usepackage{amsmath}
\usepackage{amssymb}
\usepackage{graphicx}
\usepackage{wrapfig}
\pagestyle{plain}
\usepackage{fancybox}
\usepackage{bm}

\begin{document}

151

drugim razem musi byč wyciagnieta moneta 5 $\mathrm{z}l$ lub pozostala dostepna

moneta 2 $\mathrm{z}l$) oraz $P(C_{2}|B_{5}) =0$ (nie ma drugiego losowania, gdy $\mathrm{w}$ pier-

wszym byla moneta 5 $\mathrm{z}l$ lub inaczej $ C_{2}\cap B_{5}=\emptyset$). Po podstawieniu tych

wartości do wzoru (17) dostajemy $P(C_{2})=\displaystyle \frac{7}{30}\mathrm{i}$ ostatecznie

$P(A)=1-P(C_{1})-P(C_{2})=1-\displaystyle \frac{7}{30}-\frac{1}{6}=\frac{6}{10}.$

Odp. Prawdopodobieństwo tego, $\dot{\mathrm{z}}\mathrm{e}$ Jaś wyciagnie co najmniej trzy

monety wynosi $\displaystyle \frac{3}{5}.$
\end{document}
