\documentclass[a4paper,12pt]{article}
\usepackage{latexsym}
\usepackage{amsmath}
\usepackage{amssymb}
\usepackage{graphicx}
\usepackage{wrapfig}
\pagestyle{plain}
\usepackage{fancybox}
\usepackage{bm}

\begin{document}

20

Praca kontrolna nr 2

9.1. Promień kuli powiekszono $\mathrm{t}\mathrm{a}\mathrm{k},\ \dot{\mathrm{z}}\mathrm{e}$ pole jej powierzchni wzroslo $0$ 44\%.

$\mathrm{O}$ ile procent wzrosla jej objetośč?

9.2. Wyznaczyč równanie krzywej utworzonej przez środki odcinków maja-

cych obydwa końce na osiach ukladu wspólrzednych $\mathrm{i}$ zawierajacych

punkt $P(2,1)$. Sporzadzič dokladny wykres $\mathrm{i}$ podač nazwe otrzymanej

krzywej.

9.3. Znalez$\acute{}$č wszystkie wartości parametru $m$, dla których równanie

$(m-1)9^{x}-4\cdot 3^{x}+m+2=0$

ma dwa rózne pierwiastki.

9.4. Róznica promienia kuli opisanej na czworościanie foremnym $\mathrm{i}$ promienia

kuli wpisanej $\mathrm{w}$ niegojest równa l. Obliczyč objetośč tego czworościanu.

9.5. Rozwiazač nierównośč

$\displaystyle \frac{2}{|x^{2}-9|}\geq\frac{1}{x+3}.$

9.6. Stosunek dlugości przyprostokatnych trójkata prostokatnego wynosi k.

Obliczyč stosunek dlugości dwusiecznych katów ostrych tego trójkata.

Zastosowač odpowiednie wzory trygonometryczne.

9.7. Zbadač przebieg zmienności i narysowač wykres funkcji

$f(x)=\displaystyle \frac{x^{2}+4}{(x-2)^{2}}.$

9.8. Znalez$\acute{}$č równania wszystkich prostych przechodzacych przez punkt

$A(\displaystyle \frac{7}{5},-2)\mathrm{i}$ stycznych do wykresu funkcji $f(x)=x^{3}-2x$. Rozwiazanie

zilustrowač rysunkiem.
\end{document}
