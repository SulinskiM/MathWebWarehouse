\documentclass[a4paper,12pt]{article}
\usepackage{latexsym}
\usepackage{amsmath}
\usepackage{amssymb}
\usepackage{graphicx}
\usepackage{wrapfig}
\pagestyle{plain}
\usepackage{fancybox}
\usepackage{bm}

\begin{document}

21

Praca kontrolna nr 3

10.1. Stosujac zasade indukcji matematycznej, udowodnič, $\dot{\mathrm{z}}\mathrm{e}$ dla $\mathrm{k}\mathrm{a}\dot{\mathrm{z}}$ dej

liczby naturalnej $n$ suma $2^{n+1}+3^{2n-1}$ jest podzielna przez 7.

10.2. Tworzaca stozka ma dlugośč $l \mathrm{i}$ widač $\mathrm{j}\mathrm{a}$ ze środka kuli wpisanej

$\mathrm{w}$ ten stozek pod katem $\alpha$. Obliczyč objetośč $\mathrm{i}\mathrm{k}\mathrm{a}\mathrm{t}$ rozwarcia stozka.

Określič dziedzine dla kata $\alpha.$

10.3. Bez stosowania metod rachunku rózniczkowego wyznaczyč dziedzine

$\mathrm{i}$ zbiór wartości funkcji

$y=\sqrt{2+\sqrt{x}-x}.$

10.4. $\mathrm{Z}$ talii 24 kart wy1osowano (bez zwracania) cztery karty. Jakie jest

prawdopodobieństwo, $\dot{\mathrm{z}}\mathrm{e}$ otrzymano dokladnie trzy karty $\mathrm{z}$ jednego

koloru ($\mathrm{z}$ czterech $\mathrm{m}\mathrm{o}\dot{\mathrm{z}}$ liwych)?

10.5. Rozwiazač nierównośč

$\log_{1/3}$ (log2 $4x$) $\geq\log_{1/3}(2-\log_{2x}4)-1.$

10.6. $\mathrm{Z}$ punktu $C(1,0)$ poprowadzono styczne do okregu $x^{2}+y^{2} = r^{2},$

$ r\in (0,1)$. Punkty styczności oznaczono przez A $\mathrm{i}B$. Wyrazič pole

trójkata $ABC$ jako funkcje promienia $r\mathrm{i}$ znalez$\acute{}$č najwieksza wartośč

tego pola.

10.7. Rozwiazač uklad równań

$\left\{\begin{array}{l}
x^{2}+y^{2}=5|x|\\
|4y-3x+10|=10.
\end{array}\right.$

Podač interpretacje geometryczna $\mathrm{k}\mathrm{a}\dot{\mathrm{z}}$ dego $\mathrm{z}$ równań $\mathrm{i}$ sporzadzič sta-

ranny rysunek.

10.8. Rozwiazač $\mathrm{w}$ przedziale $[0,\pi]$ równanie

1$+ \sin 2x=2\sin^{2}x,$

a nastepnie nierównośč l$+ \sin 2x>2\sin^{2}x.$
\end{document}
