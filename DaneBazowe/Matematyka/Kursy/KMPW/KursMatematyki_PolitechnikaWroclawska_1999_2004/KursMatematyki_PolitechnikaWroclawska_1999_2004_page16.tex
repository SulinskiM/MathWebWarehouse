\documentclass[a4paper,12pt]{article}
\usepackage{latexsym}
\usepackage{amsmath}
\usepackage{amssymb}
\usepackage{graphicx}
\usepackage{wrapfig}
\pagestyle{plain}
\usepackage{fancybox}
\usepackage{bm}

\begin{document}

22

Praca kontrolna nr 4

$\mathrm{W}$ celu przyblizenia sluchaczom, jakie wymagania byly stawiane ich

starszym kolegom przed ponad dwudziestu laty, niniejszy zestaw

zadań jest powtórzeniem pracy kontrolnej ze stycznia 1979 $\mathrm{r}.$

ll.l. Przez środek boku trójkata równobocznego przeprowadzono prosta,

tworzaca $\mathrm{z}$ tym bokiem $\mathrm{k}\mathrm{a}\mathrm{t}$ ostry $\alpha \mathrm{i}$ dzielaca ten trójkat na dwie

figury, których stosunek pól jest równy 1 : 7. Ob1iczyč miare kata $\alpha.$

11.2. $\mathrm{W}$ kule $0$ promieniu $R$ wpisano graniastoslup trójkatny prawidlowy

$0$ krawedzi podstawy równej promieniowi kuli. Obliczyč wysokośč

tego graniastoslupa.

11.3. Wyznaczyč wartości parametru $\alpha$, dla których funkcja

$f(x)=\displaystyle \frac{\alpha x}{1+x^{2}}$

osiaga maksimum równe 2.

11.4. Rozwiazač nierównośč

$\cos^{2}x+\cos^{3}x+\ldots+\cos^{n+1}x+\ldots<1+\cos x$

dla $x\in[0,2\pi].$

11.5. Wykazač, $\dot{\mathrm{z}}\mathrm{e}$ dla $\mathrm{k}\mathrm{a}\dot{\mathrm{z}}$ dej liczby naturalnej $n$

równośč

$\geq$

2 prawdziwa jest

$1^{2}+2^{2}++n^{2}=\left(\begin{array}{lll}
n & + & 1\\
 & 2 & 
\end{array}\right)+2[\left(\begin{array}{l}
n\\
2
\end{array}\right)+\left(\begin{array}{lll}
n & - & 1\\
 & 2 & 
\end{array}\right)+\ldots+\left(\begin{array}{l}
2\\
2
\end{array}\right)]$

11.6. Wyznaczyč równanie linii bedacej zbiorem środków wszystkich okre-

gów stycznych do prostej $y=0$ ijednocześnie stycznych zewnetrznie

do okregu $(x+2)^{2}+y^{2}=4$. Narysowač $\mathrm{t}\mathrm{e}$ linie.

11.7. Wyznaczyč wartości parametru $m$, dla których równanie

$9x^{2}-3x\log_{3}m+1=0$ ma dwa rózne pierwiastki rzeczywiste $x_{1}, x_{2}$

spelniajace warunek $x_{1}^{2}+x_{2}^{2}=1.$

11.8. Rozwiazač nierównośč

$\displaystyle \frac{\sqrt{30+x-x^{2}}}{x}<\frac{\sqrt{10}}{5}.$
\end{document}
