\documentclass[a4paper,12pt]{article}
\usepackage{latexsym}
\usepackage{amsmath}
\usepackage{amssymb}
\usepackage{graphicx}
\usepackage{wrapfig}
\pagestyle{plain}
\usepackage{fancybox}
\usepackage{bm}

\begin{document}

23

Praca kontrolna nr 5

12.1. Za pomoca odpowiedniego wykresu wykazač, $\dot{\mathrm{z}}\mathrm{e}$ równanie

$\sqrt{x-3}+x = 4$ ma dokladnie jeden pierwiastek. Nastepnie wyz-

naczyč ten pierwiastek analitycznie.

12.2. Wiadomo, $\dot{\mathrm{z}}\mathrm{e}$ wielomian $w(x) = 3x^{3}-5x+1$ ma trzy pierwiastki

rzeczywiste $x_{1}, x_{2}, x_{3}$. Bez wyznaczania tych pierwiastków obliczyč

wartośč wyrazenia $(1+x_{1})(1+x_{2})(1+x_{3})$

12.3. Rzucono jeden raz kostka, a nastepnie moneta tyle razy, ile oczek

pokazala kostka. Obliczyč prawdopodobieństwo tego, $\dot{\mathrm{z}}\mathrm{e}$ rzuty mone-

ta daly co najmniej jednego orla.

12.4. Wyznaczyč równania wszystkich okregów stycznych do obu osi ukladu

wspólrzednych oraz do prostej $3x+4y=12.$

12.5. $\mathrm{W}$ ostroslupie prawidlowym czworokatnym dana jest odleglośč $d$

środka podstawy od krawedzi bocznej oraz $\mathrm{k}\mathrm{a}\mathrm{t}2\alpha$ miedzy sasiednimi

ścianami bocznymi. Obliczyč objetośč ostroslupa.

12.6. $\mathrm{W}$ trapezie równoramiennym $0$ polu $P$ dane sa promień okregu opisa-

nego $r$ oraz suma dlugości obu podstaw $s$. Obliczyč obwód tego tra-

pezu. Podač warunki rozwiazalności zadania. Sporzadzič rysunek dla

$P=12\mathrm{c}\mathrm{m}^{2}, r=3$ cmi $s=8$ cm.

12.7. Rozwiazač uklad równań

$\left\{\begin{array}{l}
px+y=3p^{2}-3p-2\\
(p+2)x+py=4p
\end{array}\right.$

$\mathrm{w}$ zalezności od parametru rzeczywistego $p$. Podač wszystkie rozwia-

zania ($\mathrm{i}$ odpowiadajace im wartości parametru $p$), dla których obie

niewiadome sa liczbami calkowitymi $0$ wartości bezwzglednej mniej-

szej od 3.

12.8. Odcinek AB $0$ końcach $A(0,\displaystyle \frac{3}{2}) \mathrm{i}B(1,y)$, gdzie $ y\in [0,\displaystyle \frac{3}{2}]$, obraca

$\mathrm{s}\mathrm{i}\mathrm{e}$ wokól osi $Ox$. Wyrazič pole powstalej powierzchni jako funkcje

zmiennej $y\mathrm{i}$ znalez$\acute{}$č najmniejsza wartośč tego pola. Sporzadzič ry-

sunek.
\end{document}
