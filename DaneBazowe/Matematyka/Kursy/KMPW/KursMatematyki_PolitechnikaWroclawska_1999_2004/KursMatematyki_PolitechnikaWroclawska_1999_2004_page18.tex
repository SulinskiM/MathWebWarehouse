\documentclass[a4paper,12pt]{article}
\usepackage{latexsym}
\usepackage{amsmath}
\usepackage{amssymb}
\usepackage{graphicx}
\usepackage{wrapfig}
\pagestyle{plain}
\usepackage{fancybox}
\usepackage{bm}

\begin{document}

24

Praca kontrolna nr 6

13.1. Wykazač, $\dot{\mathrm{z}}\mathrm{e}$ dla $\mathrm{k}\mathrm{a}\dot{\mathrm{z}}$ dego kata $\alpha$ prawdziwa jest nierównośč

$\sqrt{3}\sin\alpha+\sqrt{6}\cos\alpha\leq 3.$

13.2. Dane sa punkty $A(2,2)\mathrm{i}B(-1,4)$. Wyznaczyč dlugośč rzutu prosto-

katnego odcinka $AB$ na prosta $0$ równaniu $12x+5y=30$. Sporzadzič

rysunek.

13.3. Niech $f(m)$ bedzie suma odwrotności pierwiatków rzeczywistych rów-

nania kwadratowego

$(2^{m}-7)x^{2}-|2^{m+1}-8|x+2^{m}=0,$

gdzie m jest parametrem rzeczywistym.

f(m) i narysowač wykres tej funkcji.

Napisač wzór określajacy

13.4. Dwóch strzelców strzela równocześnie do tego samego celu niezaleznie

od siebie. Pierwszy strzelec trafia za $\mathrm{k}\mathrm{a}\dot{\mathrm{z}}$ dym razem $\mathrm{z}$ prawdopodobień-

stwem $\displaystyle \frac{2}{3}\mathrm{i}$ oddaje 2 strza1y, a drugi trafia $\mathrm{z}$ prawdopodobieństwem $\displaystyle \frac{1}{2}$

$\mathrm{i}$ oddaje 5 strza1ów. Ob1iczyč prawdopodobieństwo, $\dot{\mathrm{z}}\mathrm{e}$ cel zostanie

trafiony dokladnie 3 razy.

13.5. Liczby $\alpha_{1}, \alpha_{2}, \alpha_{n}, n\geq 3$, tworza ciag arytmetyczny. Suma wyrazów

tego ciagu wynosi 28, suma wyrazów $0$ numerach nieparzystych wyno-

si 16, a i1oczyn $\alpha_{2}\cdot\alpha_{3}=48$. Wyznaczyč te liczby.

13.6. $\mathrm{W}$ trójkacie $ABC, \mathrm{w}$ którym $|AB| = 7$ oraz $|AC| =9$, a $\mathrm{k}\mathrm{a}\mathrm{t}$ przy

wierzcholku $A$ jest dwa razy wiekszy $\mathrm{n}\mathrm{i}\dot{\mathrm{z}} \mathrm{k}\mathrm{a}\mathrm{t}$ przy wierzcholku $B.$

Obliczyč stosunek promienia kola wpisanego $\mathrm{w}$ trójkat do promienia

kola opisanego na tym trójkacie. Rozwiazanie zilustrowač rysunkiem.

13.7. Zaznaczyč na plaszczy $\acute{\mathrm{z}}\mathrm{n}\mathrm{i}\mathrm{e}$ nastepujace zbiory punktów

$A=\{(x,y):x+y-2\geq|x-2|\},$

$B=\{(x,y):y\leq\sqrt{4x-x^{2}}\}.$

Nastepnie znalez$\acute{}$č na brzegu zbioru $A\cap B$ punkt $Q$, którego odleglośč

od punktu $P(\displaystyle \frac{5}{2},1)$ jest najmniejsza.

13.8. Zbadač przebieg zmienności $\mathrm{i}$ narysowač wykres funkcji

$f(x)=\displaystyle \frac{1}{2}x^{2}-4+\sqrt{8-x^{2}}.$
\end{document}
