\documentclass[a4paper,12pt]{article}
\usepackage{latexsym}
\usepackage{amsmath}
\usepackage{amssymb}
\usepackage{graphicx}
\usepackage{wrapfig}
\pagestyle{plain}
\usepackage{fancybox}
\usepackage{bm}

\begin{document}

25

Praca kontrolna nr 7

14.1. Ile elementów ma zbiór $A$, jeśli liczba jego podzbiorów trójelemen-

towych jest wieksza $048$ od liczby podzbiorów dwuelementowych?

14.2. $\mathrm{W}$ sześciokat foremny $0$ boku l wpisano okrag. Nastepnie $\mathrm{w}$ otrzy-

many okrag wpisano sześciokat foremny, $\mathrm{w}$ który znów wpisano okrag

$\mathrm{i}\mathrm{t}\mathrm{d}$. Obliczyč sume obwodów wszystkich otrzymanych $\mathrm{w}$ taki sposób

okregów.

14.3. Dana jest rodzina prostych $0$ równaniach $2x+my-m-2 = 0,$

$m\in R$. Które $\mathrm{z}$ prostych tej rodziny sa:

a) prostopadle do prostej $x+4y+2=0,$

b) równolegle do prostej $3x+2y=0,$

c) tworza $\mathrm{z}$ prosta $x-\displaystyle \sqrt{3}y-1=0\mathrm{k}\mathrm{a}\mathrm{t}\frac{\pi}{3}.$

14.4. Sprawdzič $\mathrm{t}\mathrm{o}\dot{\mathrm{z}}$ samośč tg $(x-\displaystyle \frac{\pi}{4})-1=\frac{-2}{\mathrm{t}\mathrm{g}x+1}$. Korzystajac $\mathrm{z}$ niej,

sporzadzič wykres funkcji $f(x)=\displaystyle \frac{1}{\mathrm{t}\mathrm{g}x+1}\mathrm{w}$ przedziale $[0,\pi].$

14.5. Dany jest okrag $K\mathrm{o}$ równaniu $x^{2}+y^{2}-6y=27$. Wyznaczyč równanie

krzywej $\Gamma$ bedacej obrazem okregu $K\mathrm{w}$ powinowactwie prostokatnym

$0$ osi $ox \mathrm{i}$ skali $k = \displaystyle \frac{1}{3}$. Obliczyč pole figury ograniczonej lukiem

okregu $K\mathrm{i}$ krzywej $\Gamma, \mathrm{l}\mathrm{e}\dot{\mathrm{z}}$ acej pod osia odcietych. Wykonač rysunek.

14.6. Korzystajac $\mathrm{z}$ nierówności $2\sqrt{\alpha b} \leq \alpha+b, \alpha, b > 0$, obliczyč gra-

nice $\displaystyle \lim_{n\rightarrow\infty}(\frac{\log_{5}16}{\log_{2}3})^{n}$

14.7. Trylogie skladajaca $\mathrm{s}\mathrm{i}\mathrm{e}\mathrm{z}$ dwóch powieści dwutomowych oraz jednej

jednotomowej ustawiono na pólce $\mathrm{w}$ przypadkowej kolejności. Jakie

jest prawdopodobieństwo tego, $\dot{\mathrm{z}}\mathrm{e}$ tomy a) obydwu, b) co najmniej

jednej $\mathrm{z}$ dwutomowych powieści znajduja $\mathrm{s}\mathrm{i}\mathrm{e}$ obok siebie $\mathrm{i}$ przy tym

tom I $\mathrm{z}$ lewej, a tom II $\mathrm{z}$ prawej strony.

14.8. $\mathrm{W}$ ostroslupie prawidlowym czworokatnym krawed $\acute{\mathrm{z}}$ boczna jest na-

chylona do plaszczyzny podstawy pod katem $\alpha$, a krawed $\acute{\mathrm{z}}$ podstawy

ma dlugośč $\alpha$. Obliczyč promień kuli stycznej do wszystkich krawedzi

tego ostroslupa. Sporzadzič odpowiednie rysunki.
\end{document}
