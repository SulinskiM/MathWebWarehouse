\documentclass[a4paper,12pt]{article}
\usepackage{latexsym}
\usepackage{amsmath}
\usepackage{amssymb}
\usepackage{graphicx}
\usepackage{wrapfig}
\pagestyle{plain}
\usepackage{fancybox}
\usepackage{bm}

\begin{document}

29

Praca kontrolna nr l

15.1. Dwaj rowerzyści wyruszyli jednocześnie $\mathrm{w}$ droge, jeden $\mathrm{z}$ A do $\mathrm{B},$

drugi $\mathrm{z} \mathrm{B}$ do A $\mathrm{i}$ spotkali $\mathrm{s}\mathrm{i}\mathrm{e}$ po jednej godzinie. Pierwszy $\mathrm{z}$ nich

przebywal $\mathrm{w}$ ciagu godziny $03$ km wiecej $\mathrm{n}\mathrm{i}\dot{\mathrm{z}}$ drugi $\mathrm{i}$ przyjechal do celu

$027$ minut wcześniej $\mathrm{n}\mathrm{i}\dot{\mathrm{z}}$ drugi. Jakie byly predkości obu rowerzystów

$\mathrm{i}$ jaka jest odleglośč AB?

15.2. Rozwiazač nierównośč $\displaystyle \sqrt{x^{2}-3}>\frac{2}{x}.$

15.3. Rysunek przedstawia dach budynku $\mathrm{w}$ rzucie poziomym.

$\mathrm{z}\mathrm{p}$ aszczyznjest nachylona do $\mathrm{p}$ aszczyzny

poziomej pod katem $30^{\mathrm{O}} \mathrm{D}$ ugośč dachu

wynosi 18 $\mathrm{m}$, a szerokośč 9 $\mathrm{m}$. Obliczyč

pole powierzchni dachu oraz ca kowita ku-

bature strychu $\mathrm{w}$ tym budynku.

$K\mathrm{a}\dot{\mathrm{z}}$ da
\begin{center}
\includegraphics[width=48.204mm,height=24.228mm]{./KursMatematyki_PolitechnikaWroclawska_1999_2004_page21_images/image001.eps}
\end{center}
15.4. Pewna firma przeprowadza co kwartal regulacje plac dla swoich pra-

cowników, waloryzujac je zgodnie ze wska $\acute{\mathrm{z}}\mathrm{n}\mathrm{i}\mathrm{k}\mathrm{i}\mathrm{e}\mathrm{m}$ inflacji, który jest

staly $\mathrm{i}$ wynosi 1,5\% kwarta1nie, oraz do1iczajac sta1a kwote podwyzki

16 $\mathrm{z}l. \mathrm{W}$ styczniu 2001 pan Kowa1ski otrzyma1 wynagrodzenie 1600

$\mathrm{z}l$. Jaka pensje otrzyma $\mathrm{w}$ kwietniu 2002? Wyznaczyč wzór ogó1ny

na pensje $w_{n}$ pana Kowalskiego $\mathrm{w}n$-tym kwartale, przyjmujac, $\dot{\mathrm{z}}\mathrm{e}$

$w_{1}=1600$ jest placa $\mathrm{w}$ pierwszym kwartale 2001. Ob1iczyč średnia

miesieczna place pana Kowalskiego $\mathrm{w}$ 2002 roku.

15.5. Wyznaczyč funkcje odwrotna do $f(x) = x^{3}, x \in R$. Nastepnie

narysowač wykres funkcji $h(x) = \sqrt[3]{(|x|-1)}+1$, wyrazajac $\mathrm{j}\mathrm{a}$

za pomoca $f^{-1}$

15.6. Rozwiazač równanie $\displaystyle \frac{\sin 2x}{\cos 4x}=1.$

15.7. Dany jest trójkat $0$ wierzcholkach $A(-2,1), B(-1,-6), C(2,5).$

Za pomoca rachunku wektorowego obliczyč cosinus kata miedzy dwu-

sieczna kata $A\mathrm{i}$ środkowa boku $BC$. Sporzadzič rysunek.

15.8. Zbadač przebieg zmienności $\mathrm{i}$ narysowač wykres funkcji

$f(x)=x+\displaystyle \frac{x}{x-1}+\frac{x}{(x-1)^{2}}+\frac{x}{(x-1)^{3}}+$
\end{document}
