\documentclass[a4paper,12pt]{article}
\usepackage{latexsym}
\usepackage{amsmath}
\usepackage{amssymb}
\usepackage{graphicx}
\usepackage{wrapfig}
\pagestyle{plain}
\usepackage{fancybox}
\usepackage{bm}

\begin{document}

30

Praca kontrolna nr 2

16.1. Cena llitra paliwa zostala obnizona $0$ 15\%. Po dwóch tygodniach

dokonano kolejnej zmiany ceny llitra paliwa, podwyzszajacja $0$ 15\%.

$\mathrm{O}$ ile procent końcowa cena paliwa rózni $\mathrm{s}\mathrm{i}\mathrm{e}$ od poczatkowej?

16.2. Wyznaczyč $\mathrm{i}$ narysowač zbiór zlozony $\mathrm{z}$ punktów $(x,y)$ plaszczyzny

spelniajacych warunek

$x^{2}+y^{2}=8|x|+6|y|.$

16.3. Wysokośč ostroslupa trójkatnego prawidlowego wynosi $h$, a $\mathrm{k}\mathrm{a}\mathrm{t}$ mie-

dzy wysokościami ścian bocznych poprowadzonymi $\mathrm{z}$ wierzcholka

ostroslupa jest równy $ 2\alpha$. Obliczyč pole powierzchni bocznej tego

ostroslupa. Sporzadzič odpowiednie rysunki.

16.4. $\mathrm{Z}$ arkusza blachy $\mathrm{w}$ ksztalcie równolegloboku $0$ bokach 30 cm $\mathrm{i}60$ cm

$\mathrm{i}$ kacie ostrym $60^{\mathrm{o}}$ nalezy odciač dwa przeciwlegle trójkatne narozniki

$\mathrm{t}\mathrm{a}\mathrm{k}$, aby powstal romb $\mathrm{o}\mathrm{m}\mathrm{o}\dot{\mathrm{z}}$ liwie najwiekszym polu. Określič przez

który punkt na dluzszym boku równolegloboku nalezy przeprowadzič

ciecie oraz obliczyč $\mathrm{k}\mathrm{a}\mathrm{t}$ ostry otrzymanego rombu. Wynik zaokraglič

do jednej minuty katowej.

16.5. Rozwiazač równanie

$2^{\log_{\sqrt{2}^{X}}}=(\sqrt{2})^{\log_{x}2}$

16.6. Wyznaczyč dziedzine $\mathrm{i}$ zbiór wartości funkcji

$f(x)=\displaystyle \frac{4}{\sin x+2\cos x+3}.$

16.7. Znalez$\acute{}$č wszystkie wartości parametru $p$, dla których równanie

$px^{4}-4x^{2}+p+1=0$

ma dwa rózne pierwiastki.

16.8. Wyznaczyč tangens kata, pod którym styczna do wykresu funkcji

$f(x)=\displaystyle \frac{8}{x^{2}+3}$

$\mathrm{w}$ punkcie $A($3, $\displaystyle \frac{2}{3})$ przecina ten wykres.
\end{document}
