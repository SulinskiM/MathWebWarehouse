\documentclass[a4paper,12pt]{article}
\usepackage{latexsym}
\usepackage{amsmath}
\usepackage{amssymb}
\usepackage{graphicx}
\usepackage{wrapfig}
\pagestyle{plain}
\usepackage{fancybox}
\usepackage{bm}

\begin{document}

31

Praca kontrolna nr 3

17.1. Dla jakich wartości $\sin x$ liczby $\sin x, \cos x, \sin 2x (\mathrm{w}$ podanym

porzadku) sa kolejnymi wyrazami ciagu geometrycznego? Wyznaczyč

czwarty wyraz tego ciagu dla $\mathrm{k}\mathrm{a}\dot{\mathrm{z}}$ dego $\mathrm{z}$ rozwiazań.

17.2. $\mathrm{W}$ pewnych zawodach sportowych startuje 16 druzyn. $\mathrm{W}$ elimina-

cjach sa one losowo dzielone na 4 grupy po 4 druzyny $\mathrm{w}\mathrm{k}\mathrm{a}\dot{\mathrm{z}}$ dej grupie.

Obliczyč prawdopodobieństwo tego, $\dot{\mathrm{z}}\mathrm{e}$ trzy zwycieskie druzyny $\mathrm{z}$ po-

przednich zawodów znajda $\mathrm{s}\mathrm{i}\mathrm{e}\mathrm{w}$ trzech róznych grupach.

17.3. Nie wykonujac dzielenia, udowodnič, $\dot{\mathrm{z}}\mathrm{e}$ wielomian

$(x^{2}+x+1)^{3}-x^{6}-x^{3}-1$

jest podzielny przez trójmian $(x+1)^{2}$

17.4. Wyznaczyč równanie okregu $0$ promieniu $r$ stycznego do paraboli

$y=x^{2}\mathrm{w}$ dwóch punktach. Dla jakiego $r$ zadanie ma rozwiazanie?

Sporzadzič rysunek, przyjmujac $r=3/2.$

17.5. Stosujac zasade indukcji matematycznej, udowodnič prawdziwośč

wzoru

$\left(\begin{array}{l}
2\\
2
\end{array}\right) - \left(\begin{array}{l}
3\\
2
\end{array}\right)+\left(\begin{array}{l}
4\\
2
\end{array}\right) - \left(\begin{array}{l}
5\\
2
\end{array}\right)+\ldots+\left(\begin{array}{l}
2n\\
2
\end{array}\right) =n^{2},$

$n\geq 1.$

17.6. Rozwiazač nierównośč

$\log_{x}(1-6x^{2})\geq 1.$

17.7. $\mathrm{W}$ trapezie ABCD opisanym na okregu $0$ środku $S$ dane sa ramie

$|AD| =c$ oraz $|AS| =d$. Punkt styczności okregu $\mathrm{z}$ podstawa $AB$

dzieli $\mathrm{j}\mathrm{a}\mathrm{w}$ stosunku 1 : 2. Ob1iczyč po1e tego trapezu. Sporzadzič

rysunek dla $c=5\mathrm{i}d=4.$

17.8. Wszystkie ściany równoleglościanu sa rombami $0$ boku $\alpha \mathrm{i}$ kacie

ostrym $\beta$. Obliczyč objetośč tego równoleglościanu. Sporzadzič

rysunek. Obliczenia odpowiednio uzasadnič.
\end{document}
