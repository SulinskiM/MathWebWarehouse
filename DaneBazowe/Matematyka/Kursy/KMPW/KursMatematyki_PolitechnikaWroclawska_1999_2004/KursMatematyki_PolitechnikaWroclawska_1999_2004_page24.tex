\documentclass[a4paper,12pt]{article}
\usepackage{latexsym}
\usepackage{amsmath}
\usepackage{amssymb}
\usepackage{graphicx}
\usepackage{wrapfig}
\pagestyle{plain}
\usepackage{fancybox}
\usepackage{bm}

\begin{document}

32

Praca kontrolna nr 4

18.1. Obliczyč granice ciagu 0 wyrazie ogólnym

$\displaystyle \alpha_{n}=\frac{2^{n}+2^{n+1}+..+2^{2n}}{2^{2}+2^{4}+\ldots+2^{2n}}.$

18.2. Wyznaczyč równanie prostej prostopadlej do prostej $0$ równaniu

$2x+3y+3 = 0 \mathrm{i} \mathrm{l}\mathrm{e}\dot{\mathrm{z}}$ acej $\mathrm{w}$ równej odleglości od dwóch danych

punktów $A(-1,1)\mathrm{i}B(3,3)$. Sporzadzič rysunek.

18.3. Tworzaca stozka ma dlugośč $l \mathrm{i}$ widač $\mathrm{j}\mathrm{a}$ ze środka kuli wpisanej

$\mathrm{w}$ ten stozek pod katem $\alpha$. Obliczyč objetośč $\mathrm{i}\mathrm{k}\mathrm{a}\mathrm{t}$ rozwarcia stozka.

Określič dziedzine kata $\alpha.$

18.4. Bolek kupil jeden dlugopis $\mathrm{i}k$ zeszytów, zaplacil $k\mathrm{z}l\mathrm{i}50$ gr, a Lolek

kupil $k$ dlugopisów $\mathrm{i}4$ zeszyty, $\mathrm{i}$ zaplaci12, 5 $k\mathrm{z}l$. Wyznaczyč cene

dlugopisu $\mathrm{i}$ zeszytu $\mathrm{w}$ zalezności od parametru $k$. Znalez$\acute{}$č wszystkie

$\mathrm{m}\mathrm{o}\dot{\mathrm{z}}$ liwe wartości tych cen wiedzac, $\dot{\mathrm{z}}\mathrm{e}$ zeszyt kosztuje nie mniej $\mathrm{n}\mathrm{i}\dot{\mathrm{z}}$

50 gr, dlugopis jest drozszy od zeszytu, a ceny obydwu artykulów

wyrazaja $\mathrm{s}\mathrm{i}\mathrm{e}\mathrm{w}$ pelnych zlotych $\mathrm{i}$ dziesiatkach groszy.

18.5. Rozwiazač nierównośč $\mathrm{t}\mathrm{g}^{3}x\geq\sin 2x.$

18.6. $\dot{\mathrm{Z}}$ arówki sa sprzedawane $\mathrm{w}$ opakowaniach po 6 sztuk. Prawdopodo-

bieństwo, $\dot{\mathrm{z}}\mathrm{e}$ pojedyncza $\dot{\mathrm{z}}$ arówka jest dobra wynosi $\displaystyle \frac{2}{3}$. Jakie jest

prawdopodobieństwo tego, $\dot{\mathrm{z}}\mathrm{e} \mathrm{w}$ jednym opakowaniu znajda $\mathrm{s}\mathrm{i}\mathrm{e}$ co

najmniej 4 dobre $\dot{\mathrm{z}}$ arówki. $\mathrm{O}$ ile zwiekszy $\mathrm{s}\mathrm{i}\mathrm{e}$ prawdopodobieństwo

tego zdarzenia, jeśli jedna, wylosowana $\mathrm{z}$ opakowania $\dot{\mathrm{z}}$ arówka, oka-

zala $\mathrm{s}\mathrm{i}\mathrm{e}$ dobra.

18.7. Prosta styczna $\mathrm{w}$ punkcie $P$ do okregu $0$ promieniu 2 $\mathrm{i}$ pólprosta

wychodzaca ze środka okregu majaca $\mathrm{z}$ okregiem punkt wspólny $S$

przecinaja $\mathrm{s}\mathrm{i}\mathrm{e}\mathrm{w}$ punkcie $A$ pod katem $60^{\circ}$ Znalez$\acute{}$č promień okregu

stycznego do odcinków $AP$, {\it AS} $\mathrm{i}$ luku $PS$. Sporzadzič rysunek.

18.8. $\mathrm{W}$ ostroslupie prawidlowym, którego podstawa jest kwadrat, pole

$\mathrm{k}\mathrm{a}\dot{\mathrm{z}}$ dej $\mathrm{z}$ pieciu ścian wynosi l. Ostroslup ten ścieto plaszczyzna

równolegla do podstawy $\mathrm{t}\mathrm{a}\mathrm{k}$, aby uzyskač maksymalny stosunek obje-

tości do pola powierzchni calkowitej. Obliczyč pole powierzchni calko-

witej otrzymanego ostroslupa ścietego. Sporzadzič rysunek.
\end{document}
