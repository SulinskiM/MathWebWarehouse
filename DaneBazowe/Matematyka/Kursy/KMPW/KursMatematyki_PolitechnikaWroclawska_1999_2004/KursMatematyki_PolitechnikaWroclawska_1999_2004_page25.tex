\documentclass[a4paper,12pt]{article}
\usepackage{latexsym}
\usepackage{amsmath}
\usepackage{amssymb}
\usepackage{graphicx}
\usepackage{wrapfig}
\pagestyle{plain}
\usepackage{fancybox}
\usepackage{bm}

\begin{document}

33

Praca kontrolna nr 5

19.1. $\mathrm{W}$ czworokacie ABCD dane sa wktory $\vec{AB}= [2,-1], \vec{BC}= [3$, 3$],$

$\vec{CD}=[-4,1]$. Punkty $K\mathrm{i}M$ sa środkami boków $CD$ oraz $AD$. Za

pomoca rachunku wektorowego obliczyč pole trójkata $KMB.$

Sporzadzič rysunek.

19.2. Trzy rózne krawedzie oraz przekatna prostopadlościanu tworza cztery

kolejne wyrazy ciagu arytmetycznego. Wyznaczyč sume dlugości

wszystkich krawedzi tego prostopadlościanu, jeśli przekatna ma dlu-

gośč 7 cm.

19.3. Na plaszczy $\acute{\mathrm{z}}\mathrm{n}\mathrm{i}\mathrm{e}Oxy$ dane sa zbiory $A = \{(x,y):y\leq\sqrt{5x-x^{2}}\}$

oraz $B_{s} = \{(x,y):3x+4y=s\}$. Dla jakich wartości parametru $s$

zbiór $A\cap B_{s}$ nie jest pusty? Sporzadzič rysunek.

19.4. Dzialka gruntu ma ksztalt trapezu $0$ bokach 20 $\mathrm{m}, 30\mathrm{m}, 40\mathrm{m}\mathrm{i}60$

$\mathrm{m}$. Wlaściciel dzialki twierdzi, $\dot{\mathrm{z}}\mathrm{e}$ powierzchnia jego dzialki wynosi

ponad ll arów. Czy wlaściciel ma racje? Jeśli $\mathrm{t}\mathrm{a}\mathrm{k}$, to narysowač plan

dzialki $\mathrm{w}$ skali 1:1000 $\mathrm{i}$ podač jej dokladna powierzchnie.

19.5. Dane jest równanie kwadratowe $\mathrm{z}$ parametrem $m$

$(m+2)x^{2}+4\sqrt{m}x+(m-3)=0.$

Dlajakiej wartości parametru $m$ kwadrat róznicy pierwiastków rzeczy-

wistych tego równaniajest najwiekszy. Podač $\mathrm{t}\mathrm{e}$ najwieksza wartośč.

19.6. Stosujac zasade indukcji matematycznej, udowodnič, $\dot{\mathrm{z}}\mathrm{e}$ dla $\mathrm{k}\mathrm{a}\dot{\mathrm{z}}$ dego

$n\geq 2$ liczba $2^{2^{n}}-6$ jest podzielna przez 10.

19.7. Rozwiazač uklad równań

$\left\{\begin{array}{l}
\mathrm{t}\mathrm{g}x+\mathrm{t}\mathrm{g}y=4\\
\cos(x+y)+\cos(x-y)=\frac{1}{2}
\end{array}\right.$

dla $x, y\in[-\pi,\pi].$

19.8. Równoramienny trójkat prostokatny $ABC$ zgieto wzdluz środkowej

$CD$ wychodzacej $\mathrm{z}$ wierzcholka kata prostego $C\mathrm{t}\mathrm{a}\mathrm{k}, \dot{\mathrm{z}}\mathrm{e}$ obie polowy

tego trójkata utworzyly $\mathrm{k}\mathrm{a}\mathrm{t}60^{\circ}$ Obliczyč sinusy wszystkich katów

dwuściennych otrzymanego czworościanu ABCD. Rozwiazanie zilus-

trowač odpowiednimi rysunkami, a obliczenia uzasadnič.
\end{document}
