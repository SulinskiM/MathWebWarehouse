\documentclass[a4paper,12pt]{article}
\usepackage{latexsym}
\usepackage{amsmath}
\usepackage{amssymb}
\usepackage{graphicx}
\usepackage{wrapfig}
\pagestyle{plain}
\usepackage{fancybox}
\usepackage{bm}

\begin{document}

34

Praca kontrolna nr 6

20.1. Wyznaczyč wszystkie wartości parametru rzeczywistego

których prosta $x = m$ jest osia symetrii wykresu

$p(x)=(m^{2}-2m)x^{2}-(2m-4)x+3$. Sporzadzič rysunek.

m, dla

funkcji

20.2. $\mathrm{Z}$ kuli $0$ promieniu $R$ wycieto ósma cześč trzema wzajemnie prosto-

padlymi plaszczyznami przechodzacymi przez środek kuli. $\mathrm{W}$ tak

otrzymana bryle wpisano inna kule. Obliczyč stosunek pola powierz-

chni tej kuli do pola powierzchni bryly.

20.3. $\mathrm{W}$ trzech pustych urnach $K, L, M$ rozmieszczamy losowo 4 rózne

kule. Obliczyč prawdopodobieństwo tego, $\dot{\mathrm{z}}\mathrm{e}\dot{\mathrm{z}}$ adna $\mathrm{z}$ urn $K\mathrm{i}L$ nie

pozostanie pusta.

20.4. Dane sa punkty $A(2,6), B(-2,6) \mathrm{i} C(0,0)$. Wyznaczyč równanie

linii zawierajacej wszystkie punkty trójkata $ABC$, dla których suma

kwadratów ich odleglości od trzech boków jest stala $\mathrm{i}$ wynosi 9. Spo-

rzadzič rysunek.

20.5. Narysowač dokladny wykres $\mathrm{i}$ napisač równania asymptot funkcji

$f(x)=\displaystyle \frac{(x+1)^{2}-1}{x|x-1|}$

nie badajac jej przebiegu.

20.6. Rozwiazač nierównośč

$|x|^{2x-1}\displaystyle \leq\frac{1}{x^{2}}.$

20.7. Styczna do wykresu funkcji $f(x) = \sqrt{3+x}+\sqrt{3-x} \mathrm{w}$ punkcie

$A(x_{0},f(x_{0}))$ przecina oś $Ox \mathrm{w}$ punkcie $P$, a oś $Oy \mathrm{w}$ punkcie $Q$

$\mathrm{t}\mathrm{a}\mathrm{k}, \dot{\mathrm{z}}\mathrm{e} |OP|=|OQ|$. Wyznaczyč $x_{0}.$

20.8. Trójkat równoboczny $0$ boku $\alpha$ podzielono prosta $l$ na dwie figury,

których stosunek pól jest równy 1 : 5. Prosta ta przecina bok $AC$

$\mathrm{w}$ punkcie $D$ pod katem $15^{\circ}$, a bok AB $\mathrm{w}$ punkcie $E$. Wykazač, $\dot{\mathrm{z}}\mathrm{e}$

$|AD|+|AE|=\alpha.$
\end{document}
