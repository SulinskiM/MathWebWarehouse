\documentclass[a4paper,12pt]{article}
\usepackage{latexsym}
\usepackage{amsmath}
\usepackage{amssymb}
\usepackage{graphicx}
\usepackage{wrapfig}
\pagestyle{plain}
\usepackage{fancybox}
\usepackage{bm}

\begin{document}

35

Praca kontrolna nr 7

21.1. Sześcian $0$ krawedzi 3 cm ma objetośč taka sama jak dwa sześciany,

których suma obydwu krawedzi wynosi 4 cm. $\mathrm{O}$ ile $\mathrm{c}\mathrm{m}^{2}$ pole powierz-

chni wiekszego sześcianujest mniejsze od sumy pól powierzchni dwóch

mniejszych sześcianów.

21.2. Obliczyč tangens kata utworzonego przez przekatne czworokata

$0$ wierzcholkach $A(1,1), B(2,0), C(2,4), D(0,6)$. Rozwiazanie zilu-

strowač rysunkiem.

21.3. $\mathrm{W}$ trójkat prostokatny wpisano okrag, a $\mathrm{w}$ okrag ten wpisano podobny

trójkat prostokatny. Wyznaczyč cosinusy katów ostrych trójkata,

jeśli wiadomo, $\dot{\mathrm{z}}\mathrm{e}$ stosunek pól obu trójkatów wynosi 9.

21.4. Wykazač, $\dot{\mathrm{z}}\mathrm{e}$ ciag

granice.

$\alpha_{n}=\sqrt{n(n+1)}-n$ jest rosnacy. Obliczyč jego

21.5. Rozwiazač nierównośč

$2\displaystyle \cos^{2}\frac{x}{4}>1.$

21.6. Rozwiazač równanie

$\displaystyle \log_{2}(1-x)+\log_{4}(x+4)=\log_{4}(x^{3}-x^{2}-3x+5)+\frac{1}{2}.$

Nie wyznaczač dziedziny równania $\mathrm{w}$ sposób jawny.

21.7. $\mathrm{W}$ kule $0$ promieniu $R$ wpisano stozek $0$ najwiekszej objetości. Wyz-

naczyč promień podstawy $r \mathrm{i}$ wysokośč $h$ tego stozka. Sporzadzič

rysunek.

21.8. Znalez$\acute{}$č równania wszystkich prostych, które sa styczne jednocześnie

do krzywych

$y=-x^{2},y=x^{2}-8x+18.$

Sporzadzič rysunek.
\end{document}
