\documentclass[a4paper,12pt]{article}
\usepackage{latexsym}
\usepackage{amsmath}
\usepackage{amssymb}
\usepackage{graphicx}
\usepackage{wrapfig}
\pagestyle{plain}
\usepackage{fancybox}
\usepackage{bm}

\begin{document}

39

Praca kontrolna nr l

22.1. Narysowač wykres funkcji $y = 4+2|x| -x^{2}$ Na podstawie tego

wykresu określič liczbe rozwiazań równania $4 + 2|x| - x^{2} = p$

$\mathrm{w}$ zalezności od parametru rzeczywistego $p.$

22.2. Pompa napelniajaca pusty basen $\mathrm{w}$ pierwszej minucie pracy miala

wydajnośč 0,2 $\mathrm{m}^{3}/\mathrm{s}$, a $\mathrm{w}\mathrm{k}\mathrm{a}\dot{\mathrm{z}}$ dej kolejnej minuciejej wydajnośč zwiek-

szano $0 0,01 \mathrm{m}^{3}/\mathrm{s}$. Polowa basenu zostala napelniona po $2n$ mi-

nutach, a caly basen po kolejnych $n$ minutach, gdzie $n$ jest liczba

naturalna. Wyznaczyč czas napelniania basenu oraz jego pojemnośč.

22.3. Stozek ścietyjest opisany na kuli $0$ promieniu $r=2$ cm. Objetośč kuli

stanowi 25\% objetości stozka. Wyznaczyč średnice podstaw $\mathrm{i}$ dlu-

gośč tworzacej tego stozka.

22.4. $\mathrm{W}$ trójkacie $ABC$ dane sa promień okregu opisanego $R, \mathrm{k}\mathrm{a}\mathrm{t}\angle A=\alpha$

oraz $|AB|=\displaystyle \frac{8}{5}R$. Obliczyč pole tego trójkata.

22.5. Rozwiazač nierównośč

$(\sqrt{x})^{\log_{8}x}\geq\sqrt[3]{16x}.$

22.6. $\mathrm{W}$ czworokacie ABCD odcinki AB $\mathrm{i} BD$ sa prostopadle,

$|AD| = 2|AB| = \alpha$ oraz $\vec{AC}= \displaystyle \frac{5}{3} \vec{AB} +\displaystyle \frac{1}{3} \vec{AD}$. Wyznaczyč cosi-

nus kata $\angle BCD = \alpha$ oraz obwód czworokata ABCD. Sporzadzič

rysunek.

22.7. Rozwiazač równanie

$\displaystyle \frac{1}{\sin x}+\frac{1}{\cos x}=\sqrt{8}.$

22.8. Wyznaczyč równanie prostej stycznej do wykresu funkcji $y = \displaystyle \frac{1}{x^{2}}$

$\mathrm{w}$ punkcie $P(x_{0},y_{0}), x_{0}>0$, takim, $\dot{\mathrm{z}}$ eby odcinek tej stycznej zawarty

$\mathrm{w}$ pierwszej čwiartce ukladu wspólrzednych byl najkrótszy. Rozwia-

zanie zilustrowač odpowiednim wykresem.
\end{document}
