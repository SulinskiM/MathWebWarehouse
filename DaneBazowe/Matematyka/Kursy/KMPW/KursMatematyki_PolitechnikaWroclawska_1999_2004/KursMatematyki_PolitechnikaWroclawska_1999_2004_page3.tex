\documentclass[a4paper,12pt]{article}
\usepackage{latexsym}
\usepackage{amsmath}
\usepackage{amssymb}
\usepackage{graphicx}
\usepackage{wrapfig}
\pagestyle{plain}
\usepackage{fancybox}
\usepackage{bm}

\begin{document}

Przedmowa

{\it Zbiór obejmuje zadania Korespondencyjnego Kursu} $\mathrm{z}$ {\it Matematyki} $\mathrm{z}$ lat 1999-

2004. Kurs tenjest prowadzony przez Instytut Matematyki Politechniki Wroclaw-

{\it skiej. Jest kontynuacja Korespondencyjnego Kursu Przygotowawczego} $\mathrm{z}$ {\it Mate}-

{\it matyki, który} $\mathrm{w}$ latach $1972-1999$ byl prowadzony wspólnie $\mathrm{z}$ Instytutem Mate-

matyki Politechniki Warszawskiej.

Opracowujac niniejszy zbiór, autor pragnal ulatwič szerokiemu gronu matu-

rzystów $\mathrm{i}$ kandydatów na studia dostęp do materialów kursu ujętych $\mathrm{w}$ wygodna,

zwarta formę $\mathrm{i}\mathrm{w}$ ten sposób pomóc im $\mathrm{w}$ lepszym opanowaniu wiadomości $\mathrm{z}$

matematyki $\mathrm{w}$ zakresie szkoly średniej oraz dač jeszcze jedna okazję do powtó-

rzenia materialu.

Większośč zadań jest oryginalna, częśč pochodzi $\mathrm{z}$ egzaminów wstępnych na

Politechnikę Wroclawska $\mathrm{z}$ ostatnich 201at, a ty1ko niewie1ka 1iczba $\mathrm{z}$ innych

z$\acute{}$ródel ($\mathrm{w}$ tym powtórzenia zadań $\mathrm{z}$ lat ubieglych). Dla udogodnienia samodzielnej

pracy $\mathrm{i}$ zachęcenia do korzystania $\mathrm{z}$ tego zbioru, podano odpowiedzi, a $\mathrm{w}$ oddziel-

nym rozdziale takze wskazówki do wszystkich zadań. $\mathrm{W}$ końcowej części zbioru

przedstawiono 12 przyk1adowych rozwiazań róznorodnych zadań wybranych ze

wszystkich dzialów. Celem ich zamieszczenia jest pokazanie najwazniejszych

metod $\mathrm{i}$ narzędzi $\mathrm{u}\dot{\mathrm{z}}$ ywanych do rozwiazywania zadań. Moga więc sluzyč jako

wzorzec do przygotowania innych rozwiazań. Zastosowano dwuczlonowa nume-

rację zadań uwzględniajaca chronologię kursu. Pierwsza liczba jest kolejnym

numerem pracy kontrolnej ($\mathrm{z}$ okresu $1999-2004$), a druga podaje numer tematu

$\mathrm{w}$ danym zestawie. Dolaczony indeks tematyczny pozwala na szybkie wyszukanie

zadań $\mathrm{z}$ dowolnie wybranego dzialu matematyki.

Ze zbioru moga korzystač zarówno osoby zdajace maturę na poziomie podsta-

wowym, jak $\mathrm{i}$ rozszerzonym. Poczynajac od XXXI edycji, $\mathrm{t}\mathrm{j}$. od pracy kontrolnej

$0$ numerze 15, pierwsze cztery zadania $\mathrm{w}\mathrm{k}\mathrm{a}\dot{\mathrm{z}}$ dym zestawie odpowiadaja zakresowi

podstawowemu, a cztery następne zakresowi rozszerzonemu. Podzial ten dotyczy

$\mathrm{w}$ przyblizeniu takze wcześniejszych prac kontrolnych.

Kurs ma swoja strong internetowa, na której $\mathrm{m}\mathrm{o}\dot{\mathrm{z}}$ na znalez$\acute{}$č zarówno mate-

rialy biezace, jak $\mathrm{i}$ archiwum zawierajace tematy $\mathrm{z}$ lat ubieglych. Dostęp do niej

$\mathrm{m}\mathrm{o}\dot{\mathrm{z}}$ na uzyskač przez strong glówna Politechniki Wroclawskiej: $\mathrm{w}\mathrm{w}\mathrm{w}$. pwr. wroc. pl.

Następnie nalezy wybrač dzial Rekrutacja $\mathrm{i}\mathrm{w}$ nim wyszukač pozycję Korespon-

{\it dencyjny Kurs} $\mathrm{z}$ {\it Matematyki}.

Serdecznie dziękuję Recenzentom Docentowi Zbigniewowi Romanowiczowi

oraz Doktorowi Rościslawowi Rabczukowi za cenne uwagi, które pozwolily usunač

usterki $\mathrm{i}$ ulepszyč pierwotna wersję ksia $\dot{\mathrm{Z}}$ ki.

Wroclaw, marzec 2005

{\it Tadeusz Inglot}
\end{document}
