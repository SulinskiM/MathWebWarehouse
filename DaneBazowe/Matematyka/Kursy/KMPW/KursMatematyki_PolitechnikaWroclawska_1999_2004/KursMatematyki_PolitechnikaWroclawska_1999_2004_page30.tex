\documentclass[a4paper,12pt]{article}
\usepackage{latexsym}
\usepackage{amsmath}
\usepackage{amssymb}
\usepackage{graphicx}
\usepackage{wrapfig}
\pagestyle{plain}
\usepackage{fancybox}
\usepackage{bm}

\begin{document}

40

Praca kontrolna nr 2

23.1. Czy liczby róznych,,slów'', jakie $\mathrm{m}\mathrm{o}\dot{\mathrm{z}}$ na utworzyč zmieniajac kole-

jnośč liter $\mathrm{w},$,slowach'' TANATAN $\mathrm{i}$ AKABARA, sa takie same?

Uzasadnič odpowied $\acute{\mathrm{z}}$. Przez,,slowo'' rozumiemy tutaj dowolny ciag

liter.

23.2. Reszta $\mathrm{z}$ dzielenia wielomianu $x^{3}+px^{2}-x+q$ przez trójmian $(x+2)^{2}$

wynosi $(-x+1)$. Wyznaczyč pierwiastki tego wielomianu.
\begin{center}
\includegraphics[width=33.324mm,height=23.724mm]{./KursMatematyki_PolitechnikaWroclawska_1999_2004_page30_images/image001.eps}
\end{center}
A $\mathrm{B}$

$A, B$, oraz $\mathrm{z}$ odcinka AB $0$ dugości $\alpha.$

Obliczyč promień okregu stycznego do obu

uków oraz do odcinka $AB.$

23.3. Figura na rysunku sklada $\mathrm{s}\mathrm{i}\mathrm{e}\mathrm{Z}$ luków $BC, CA$ okregów $0$ promie-

C niu $\alpha \mathrm{i}$ środkach odpowiednio $\mathrm{w}$ punktach

23.4. Podstawa pryzmy przedstawionej na rysunku jest prostokat

ABCD, którego bok $AB$ ma

D $\mathrm{C}$ gdzie $\alpha > b$. Wszystkie ściany

$b$ boczne pryzmy sa nachylone pod
\begin{center}
\includegraphics[width=60.552mm,height=30.480mm]{./KursMatematyki_PolitechnikaWroclawska_1999_2004_page30_images/image002.eps}
\end{center}
{\it K} $\mathrm{L}$

A $\alpha \mathrm{B}$

$\mathrm{d}$ ugośč $\alpha$, a bok $BC \mathrm{d}$ ugośč $b,$

katem $\alpha$ do $\mathrm{p}$ aszczyzny podstawy.

Obliczyč objetośč tej pryzmy.

23.5. Rozwiazač nierównośč

$- x2<\sqrt{}$5-{\it x}2.

Rozwiazanie zilustrowač wykresami funkcji wystepujacych po obu

stronach nierówności. Zaznaczyč na rysunku otrzymany zbiór rozwia-

zań.

23.6. Ciag $(\alpha_{n})$ jest określony rekurencyjnie warunkami $\alpha_{1} =$ 4,

$\alpha_{n+1} = 1+2\sqrt{\alpha_{n}}, n \geq 1$. Stosujac zasade indukcji matematycznej,

wykazač, $\dot{\mathrm{z}}\mathrm{e}$ ciag $(\alpha_{n})$ jest rosnacy oraz $4\leq\alpha_{n}<6$ dla $n\geq 1.$

23.7. Na krzywej $0$ równaniu $y=\sqrt{x}$ znalez$\acute{}$č punkt $\mathrm{l}\mathrm{e}\dot{\mathrm{z}}\mathrm{a}\mathrm{c}\mathrm{y}$ najblizej punktu

$P(0,3)$. Sporzadzič rysunek.

23.8. Wykazač, $\dot{\mathrm{z}}\mathrm{e}$ dla $\mathrm{k}\mathrm{a}\dot{\mathrm{z}}$ dej wartości parametru $\alpha \in \mathrm{R}$ równanie

kwadratowe $3x^{2}+4x\sin\alpha-\cos 2\alpha=0$ ma dwa rózne pierwiastki

rzeczywiste. Wyznaczyč te wartości parametru $\alpha$, dla których oba

pierwiastki $\mathrm{l}\mathrm{e}\dot{\mathrm{z}}$ a $\mathrm{w}$ przedziale $(0,1).$
\end{document}
