\documentclass[a4paper,12pt]{article}
\usepackage{latexsym}
\usepackage{amsmath}
\usepackage{amssymb}
\usepackage{graphicx}
\usepackage{wrapfig}
\pagestyle{plain}
\usepackage{fancybox}
\usepackage{bm}

\begin{document}

41

Praca kontrolna nr 3

24.1. Suma wyrazów nieskończonego ciagu geometrycznego zmniejszy $\mathrm{s}\mathrm{i}\mathrm{e}$

$0$ 25\%, jeśli wykreślimy $\mathrm{z}$ niej skladniki $0$ numerach parzystych niepo-

dzielnych przez 4. Ob1iczyč sume wszystkich wyrazów tego ciagu,

wiedzac, $\dot{\mathrm{z}}\mathrm{e}$ jego drugi wyraz wynosi l.

24.2. $\mathrm{Z}$ kompletu 28 kości do gry $\mathrm{w}$ domino wylosowano dwie kości (bez

zwracania). Obliczyč prawdopodobieństwo tego, $\dot{\mathrm{z}}\mathrm{e}$ kości $pasujq_{f}$ do

siebie, $\mathrm{t}\mathrm{z}\mathrm{n}$. najednym $\mathrm{z}$ pól obu kości wystepuje ta sama liczba oczek.

24.3. Rozwiazač uklad równań

$\left\{\begin{array}{l}
x+2y=3\\
5x+my=m
\end{array}\right.$

$\mathrm{w}$ zalezności od parametru rzeczywistego $m$. Wyznaczyč $\mathrm{i}$ narysowač

zbiór, jaki tworza rozwiazania $(x(m),y(m))$ tego ukladu, gdy $m$

przebiega zbiór liczb rzeczywistych.

24.4. $\mathrm{W}$ graniastoslupie prawidlowym sześciokatnym krawed $\acute{\mathrm{z}}$ dolnej pod-

stawy $AB$ widač ze środka górnej podstawy $P$ pod katem $\alpha$. Wyz-

naczyč cosinus kata utworzonego przez plaszczyzne podstawy $\mathrm{i}$ plasz-

czyzne zawierajaca krawed $\acute{\mathrm{z}}$ AB oraz przeciwlegla do niej $\mathrm{k}\mathrm{r}\mathrm{a}\mathrm{w}\mathrm{e}\mathrm{d}\acute{\mathrm{z}}$

$D'E'$ górnej podstawy. Uzasadnič odpowiednio obliczenia.

24.5. Rozwiazač nierównośč $-1\displaystyle \leq\frac{2^{x+1/2}}{4^{x}-4}\leq 1.$

24.6. Bez $\mathrm{u}\dot{\mathrm{z}}$ ycia tablic wykazač, $\dot{\mathrm{z}}\mathrm{e}$ tg $82^{\circ}30'$ -tg $7^{\circ}30'=4+2\sqrt{3}.$

24.7. Napisač równanie prostej $k$ stycznej $\mathrm{w}$ punkcie $P(2,3)$ do okregu

$x^{2}+y^{2}-2x-2y-3=0$ Nastepnie wyznaczyč równania wszystkich

prostych stycznych do tego okregu, które tworza $\mathrm{z}$ prosta $k\mathrm{k}\mathrm{a}\mathrm{t}45^{\circ}$

24.8. Dobrač parametry $\alpha>0 \mathrm{i} b\in R \mathrm{t}\mathrm{a}\mathrm{k}$, aby funkcja

$f(x)=$

dla $x\leq\alpha,$

dla $x>\alpha,$

byla ciagla $\mathrm{i}$ miala pochodna $\mathrm{w}$ punkcie $x =\alpha$. Sporzadzič wykres

funkcji $f(x)$ oraz stycznej do wykresu $\mathrm{w}$ punkcie $P(\alpha,f(\alpha))$ bez

szczególowego badania jej przebiegu.
\end{document}
