\documentclass[a4paper,12pt]{article}
\usepackage{latexsym}
\usepackage{amsmath}
\usepackage{amssymb}
\usepackage{graphicx}
\usepackage{wrapfig}
\pagestyle{plain}
\usepackage{fancybox}
\usepackage{bm}

\begin{document}

42

Praca kontrolna nr 4

25.1. Dla jakich wartości parametru rzeczywistego

$x+3=-(tx+1)^{2}$ ma dokladnie jeden pierwiastek.

t równanie

25.2. Czworościan foremny $0$ krawedzi $\alpha$ przecieto plaszczyzna równolegla

do dwóch przeciwleglych krawedzi. Wyrazič pole otrzymanego prze-

krojujako funkcje dlugości odcinka wyznaczonego przez ten przekrój

najednej $\mathrm{z}$ pozostalych krawedzi. Uzasadnič postepowanie. Przedsta-

wič znaleziona funkcje na wykresie $\mathrm{i}$ podač jej najwieksza wartośč.

25.3. Zaznaczyč na wykresie zbiór punktów $(x,y)$ plaszczyzny spelniaja-

cych warunek $\log_{xy}|y|\geq 1.$

25.4. Wyznaczyč równanie linii utworzonej przez wszystkie punkty plasz-

czyzny, których odleglośč od okregu $x^{2}+y^{2}=81$ jest $01$ mniejsza

$\mathrm{n}\mathrm{i}\dot{\mathrm{z}}$ od punktu $P(8,0)$. Sporzadzič rysunek.

25.5. Na dziesiatym pietrze pewnego bloku mieszkaja Kowalscy $\mathrm{i}$ Nowa-

kowie. Kowalscy maja dwóch synów $\mathrm{i}$ dwie córki, a Nowakowie jed-

nego syna $\mathrm{i}$ dwie córki. Postanowili oni wybrač mlodziezowego przed-

stawiciela swojego pietra. $\mathrm{W}$ tym celu Kowalscy wybrali losowo jedno

ze swoich dzieci $\mathrm{i}$ Nowakowie jedno ze swoich. Nastepnie spośród tej

dwójki wylosowano jedna osobe. Obliczyč prawdopodobieństwo, $\dot{\mathrm{z}}\mathrm{e}$

przedstawicielem zostal chlopiec.

25.6. Uzasadnič prawdziwośč nierówności $n+\displaystyle \frac{1}{2}\geq\sqrt{n(n+1)}, n\geq 1.$ {\it Ko}-

rzystajac $\mathrm{z}$ niej oraz $\mathrm{z}$ zasady indukcji matematycznej, udowodnič,

$\dot{\mathrm{z}}\mathrm{e}$

$\displaystyle \left(\begin{array}{l}
2n\\
n
\end{array}\right)\geq\frac{4^{n}}{2\sqrt{n}}$

dla $\mathrm{k}\mathrm{a}\dot{\mathrm{z}}$ dej liczby naturalnej $n.$

25.7. Zbadač przebieg zmienności $\mathrm{i}$ narysowač wykres funkcji

$f(x)=\sqrt{\frac{3x-3}{5-x}}.$

25.8. $\mathrm{W}$ trójkacie $ABC \mathrm{k}\mathrm{a}\mathrm{t} A$ ma miare $\alpha, \mathrm{k}\mathrm{a}\mathrm{t} B$ miare $2\alpha,$

$\mathrm{a}|BC|=\alpha$. Oznaczmy kolejno przez $A_{1}$ punkt na boku $AC$ taki, $\dot{\mathrm{z}}\mathrm{e}$

BAl jest dwusieczna kata $B;B_{1}$ punkt na boku $BC$ taki, $\dot{\mathrm{z}}\mathrm{e}$ AlBl jest

dwusieczna kata $A_{1}, \mathrm{i}\mathrm{t}\mathrm{d}$. Wyznaczyč dlugośč nieskończonej lamanej

$ABA_{1}B_{1}A_{2}$
\end{document}
