\documentclass[a4paper,12pt]{article}
\usepackage{latexsym}
\usepackage{amsmath}
\usepackage{amssymb}
\usepackage{graphicx}
\usepackage{wrapfig}
\pagestyle{plain}
\usepackage{fancybox}
\usepackage{bm}

\begin{document}

43

Praca kontrolna nr 5

26.1. Jakiej dlugości powinien byč pas transmisyjny, aby $\mathrm{m}\mathrm{o}\dot{\mathrm{z}}$ na go bylo

$\mathrm{u}\dot{\mathrm{z}}$ yč do polaczenia dwóch kól $0$ promieniach 20 cm $\mathrm{i}5$ cm, jeśli od-

leglośč środków tych kól wynosi 30 cm?

26.2. Umowa określa wynagrodzenie na kwote 4000 $\mathrm{z}l$. Skladka na ubez-

pieczenie spoleczne wynosi 18,7\% tej kwoty, a sk1adka na Kase Cho-

rych 7,75\% kwoty pozosta1ej po od1iczeniu sk1adki na ubezpiecze-

nie spoleczne. $\mathrm{W}$ celu obliczenia podatku nalezy od 80\% wyjściowej

kwoty umowy odjač skladke na ubezpieczenie spoleczne $\mathrm{i}$ wyznaczyč

19\% pozostalej sumy. Podatek jest róznica tak otrzymanej liczby

$\mathrm{i}$ skladki na Kase Chorych. Ile wynosi podatek?.

26.3. Przez punkt $P(1,3)$ poprowadzič prosta $l$, tak aby odcinek tej prostej

zawarty miedzy prostymi $x-y+3=0\mathrm{i}x+2y-12=0$ dzielil $\mathrm{s}\mathrm{i}\mathrm{e}$

$\mathrm{w}$ punkcie $P$ na polowy. Wyznaczyč równanie ogólne prostej $l\mathrm{i}$ obli-

czyč pole trójkata, jaki prosta $l$ tworzy $\mathrm{z}$ danymi prostymi.

26.4. Podstawa czworościanu ABCD jest trójkat prostokatny $ABC\mathrm{o}$ kacie

ostrym $\alpha \mathrm{i}$ promieniu okregu wpisanego $r$. Spodek wysokości opusz-

czonej $\mathrm{z}$ wierzcholka $D \mathrm{l}\mathrm{e}\dot{\mathrm{z}}\mathrm{y}\mathrm{w}$ punkcie przeciecia $\mathrm{s}\mathrm{i}\mathrm{e}$ dwusiecznych

trójkata $ABC$, a ściany boczne wychodzace $\mathrm{z}$ wierzcholka kata pros-

tego podstawy tworza $\mathrm{k}\mathrm{a}\mathrm{t}\beta$. Obliczyč objetośč tego ostroslupa.

26.5. Sporzadzič wykres funkcji $f(x)=\log_{4}(2|x|-4)^{2}$ Odczytač $\mathrm{z}$ wykre-

su wszystkie ekstrema lokalne tej funkcji.

26.6. Rozwiazač równanie $\displaystyle \cos 2x+\frac{\mathrm{t}\mathrm{g}x}{\sqrt{3}+\mathrm{t}\mathrm{g}x}=0.$

26.7. Dla jakich wartości parametru $\alpha \in \mathrm{R} \mathrm{m}\mathrm{o}\dot{\mathrm{z}}$ na określič funkcje

$g(x) = f(f(x))$, gdzie $f(x) = \displaystyle \frac{x^{2}}{\alpha x-1}$. Napisač wzór funkcji $g(x).$

Wyznaczyč asymptoty funkcji $g(x)$ dla najwiekszej $\mathrm{m}\mathrm{o}\dot{\mathrm{z}}$ liwej calko-

witej wartości parametru $\alpha.$

26.8. Odcinek $0$ końcach $A(0,3), B(2,y), y \in [0$, 3$]$, obraca $\mathrm{s}\mathrm{i}\mathrm{e}$ wokól

osi $Ox$. Wyznaczyč pole powierzchni bocznej powstalej bryly jako

funkcje $y \mathrm{i}$ znalez$\acute{}$č najmniejsza wartośč tego pola. Sporzadzič ry-

sunek.
\end{document}
