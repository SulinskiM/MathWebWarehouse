\documentclass[a4paper,12pt]{article}
\usepackage{latexsym}
\usepackage{amsmath}
\usepackage{amssymb}
\usepackage{graphicx}
\usepackage{wrapfig}
\pagestyle{plain}
\usepackage{fancybox}
\usepackage{bm}

\begin{document}

44

Praca kontrolna nr 6

27.1. Znalez$\acute{}$č wszystkie wartości parametru rzeczywistego $p$, dla których

równanie $\sqrt{x+8p}=\sqrt{x}+2p$ ma rozwiazanie.

27.2. Obrazem okregu $K\mathrm{w}$ jednokladności $0$ środku $S(0,1)\mathrm{i}$ skali $k=-3$

jest okrag $K_{1}$, natomiast obrazem $K_{1} \mathrm{w}$ symetrii wzgledem prostej

$0$ równaniu $2x+y+3 = 0$ jest okrag $0$ tym samym środku co

okrag $K$. Wyznaczyč równanie okregu $K$, jeśli wiadomo, $\dot{\mathrm{z}}\mathrm{e}$ okregi

$K\mathrm{i}K_{1}$ sa styczne zewnetrznie.

27.3. $\mathrm{W}$ trapezie równoramiennym dane sa promień okregu opisanego $r,$

$\mathrm{k}\mathrm{a}\mathrm{t}$ ostry przy podstawie $\alpha$ oraz suma dlugości obu podstaw $d$. Obli-

czyč dlugośč ramienia tego trapezu. Zbadač warunki rozwiazalności

zadania. Sporzadzič rysunek dla $\alpha=60^{\circ}, d=\displaystyle \frac{5}{2}r.$

27.4. $\mathrm{W}$ ostroslupie prawidlowym czworokatnym $\mathrm{k}\mathrm{a}\mathrm{t}$ plaski ściany bocznej

przy wierzcholku wynosi $ 2\beta$. Przez wierzcholek $A$ podstawy oraz

środek przeciwleglej krawedzi bocznej poprowadzono plaszczyzne

równolegla do przekatnej podstawy wyznaczajaca przekrój plaski

ostroslupa. Obliczyč objetośč ostroslupa, wiedzac, $\dot{\mathrm{z}}\mathrm{e}$ pole przekroju

wynosi $S.$

27.5. Obliczyč granice

$\displaystyle \lim_{n\rightarrow\infty}\frac{n-\sqrt[3]{n^{3}+n^{\alpha}}}{\sqrt[5]{n^{3}}},$

jeśli $\alpha$ jest najmniejszym

$2\cos\alpha=-\sqrt{3}.$

dodatnim pierwiastkiem

równania

27.6. Rozwiazač nierównośč

$2^{1+2\log_{2}\cos x}-\displaystyle \frac{3}{4}\geq 9^{0}$' $5+\log_{3}\sin x$

27.7. Wylosowano, ze zwracaniem, 4 liczby czterocyfrowe (cyfra tysiecy

nie $\mathrm{m}\mathrm{o}\dot{\mathrm{z}}\mathrm{e}$ byč zerem!). Obliczyč prawdopodobieństwo tego, $\dot{\mathrm{z}}\mathrm{e}$ co

najmniej dwie $\mathrm{z}$ tych liczb czytane od strony lewej do prawej lub od

strony prawej do lewej beda podzielne przez 4.

27.8. Zaznaczyč na rysunku zbiór punktów $(x,y)$ plaszczyzny określony

warunkami $|x-3y| < 2$ oraz $y^{3} \leq x$. Obliczyč tangens kata, pod

którym przecinaja $\mathrm{s}\mathrm{i}\mathrm{e}$ linie tworzace brzeg tego zbioru.
\end{document}
