\documentclass[a4paper,12pt]{article}
\usepackage{latexsym}
\usepackage{amsmath}
\usepackage{amssymb}
\usepackage{graphicx}
\usepackage{wrapfig}
\pagestyle{plain}
\usepackage{fancybox}
\usepackage{bm}

\begin{document}

45

Praca kontrolna nr 7

28.1. Dwa punkty poruszaja $\mathrm{s}\mathrm{i}\mathrm{e}$ ruchem jednostajnym po okregu $\mathrm{w}$ tym

samym kierunku, przy czym jeden $\mathrm{z}$ nich wyprzedza drugi co 44

sekundy. $\mathrm{J}\mathrm{e}\dot{\mathrm{z}}$ eli zmienič kierunek ruchu jednego $\mathrm{z}$ tych punktów na

przeciwny, to beda $\mathrm{s}\mathrm{i}\mathrm{e}$ one spotykač co 8 sekund. Ob1iczyč stosunek

predkości tych punktów.

28.2. Dlajakich wartości parametru $p$ nierównośč

$\displaystyle \frac{2px^{2}+2px+1}{x^{2}+x+2-p^{2}}\geq 2$

jest spelniona dla $\mathrm{k}\mathrm{a}\dot{\mathrm{z}}$ dej liczby rzeczywistej $x$?

28.3. $\mathrm{W}$ równolegloboku dane sa $\mathrm{k}\mathrm{a}\mathrm{t}$ ostry $\alpha$, dluzsza przekatna $d$ oraz

róznica boków $r$. Obliczyč pole równolegloboku.

28.4. Naczynie $\mathrm{w}$ ksztalcie pólkuli $0$ promieniu $R$ ma trzy nózki $\mathrm{w}$ ksztalcie

kulek $0$ promieniu $r, 4r < R$, przymocowanych do naczynia $\mathrm{w}$ ten

sposób, $\dot{\mathrm{z}}\mathrm{e}$ ich środki tworza trójkat równoboczny, a naczynie posta-

wione na plaskiej powierzchni dotyka $\mathrm{j}\mathrm{a}$ wjednym punkcie. Obliczyč

wzajemna odleglośč punktów przymocowania kulek. Sporzadzič od-

powiednie rysunki.

28.5. Za pomoca metod rachunku rózniczkowego określič liczbe rozwiazań

równania $2x^{3}+1=6|x|-6x^{2}$

28.6. Bez stosowania zasady indukcji matematycznej wykazač, $\dot{\mathrm{z}}\mathrm{e} \displaystyle \frac{n^{n}-1}{n-1}$

jest nieparzysta liczba naturalna dla wszystkich $n\geq 2.$

28.7. Rozwiazač równanie

$\displaystyle \frac{8}{3}(\sin^{2}x+\sin^{4}x+\ldots)=4-2\cos x+3\cos^{2}x-\frac{9}{2}\cos^{3}x+$

28.8. Rozwazmy rodzine prostych normalnych do paraboli $0$ równaniu

$2y = x^{2}$ (tj. prostopadlych do stycznych $\mathrm{w}$ punktach styczności).

Znalez$\acute{}$č równanie krzywej utworzonej ze środków odcinków tych nor-

malnych zawartych miedzy osia rzednych $\mathrm{i}$ wyznaczajacymi je punk-

tami paraboli. Sporzadzič rysunek.
\end{document}
