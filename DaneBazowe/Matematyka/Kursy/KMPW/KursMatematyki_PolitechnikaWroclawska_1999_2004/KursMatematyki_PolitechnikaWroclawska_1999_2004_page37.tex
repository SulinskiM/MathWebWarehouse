\documentclass[a4paper,12pt]{article}
\usepackage{latexsym}
\usepackage{amsmath}
\usepackage{amssymb}
\usepackage{graphicx}
\usepackage{wrapfig}
\pagestyle{plain}
\usepackage{fancybox}
\usepackage{bm}

\begin{document}

49

Praca kontrolna nr l

29.1. Podstawa trójkata równoramiennego jest odcinek AB $0$ końcach

$A(-1,3), B(1,-1)$, a wierzcholek $C$ tego trójkata $\mathrm{l}\mathrm{e}\dot{\mathrm{z}}\mathrm{y}$ na prostej

$l\mathrm{o}$ równaniu $3x-y-14=0$. Obliczyč pole trójkata $ABC.$

29.2. Pewna liczba sześciocyfrowa zaczyna $\mathrm{s}\mathrm{i}\mathrm{e}$ ($\mathrm{z}$ lewej strony) cyfra 3. Jeś1i

cyfre $\mathrm{t}\mathrm{e}$ przestawimy $\mathrm{z}$ pierwszej pozycji na ostatnia, to otrzymamy

liczbe stanowiaca 25\% 1iczby pierwotnej. Zna1ez$\acute{}$č $\mathrm{t}\mathrm{e}$ liczbe.

29.3. $\mathrm{W}$ trapezie opisanym na okregu katy ostre przy podstawie maja

miary $\alpha \mathrm{i}2\alpha$, a dlugośč krótszego ramienia wynosi $c$. Obliczyč dlugośč

krótszej podstawy tego trapezu. Wynik przedstawič $\mathrm{w}$ najprostszej

postaci.

29.4. Rozwiazač nierównośč

$\displaystyle \frac{1}{x^{2}-x-2}\leq\frac{1}{|x|}.$

29.5. Zaznaczyč na plaszczy $\acute{\mathrm{z}}\mathrm{n}\mathrm{i}\mathrm{e}$ zbiór wszystkich punktów $(x,y)$ spelnia-

jacych nierównośč $\log_{x}(1+(y-1)^{3})\leq 1.$

29.6. Rozwiazač równanie $\sin^{2}3x-\sin^{2}2x=\sin^{2}x.$

29.7. Wysokośč ostroslupa prawidlowego czworokatnegojest trzy razy dluz-

sza od promienia kuli wpisanej $\mathrm{w}$ ten ostroslup. Obliczyč cosinus kata

miedzy sasiednimi ścianami bocznymi tego ostroslupa.

29.8. Dany jest nieskończony ciag geometryczny

$x+1,-x^{2}(x+1),x^{4}(x+1),$

Wyznaczyč najmniejsza $\mathrm{i}$ najwieksza wartośč funkcji $S(x)$ bedacej

suma wszystkich wyrazów tego ciagu.
\end{document}
