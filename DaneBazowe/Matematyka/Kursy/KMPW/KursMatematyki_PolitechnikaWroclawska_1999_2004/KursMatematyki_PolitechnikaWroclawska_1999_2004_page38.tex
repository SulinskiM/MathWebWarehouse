\documentclass[a4paper,12pt]{article}
\usepackage{latexsym}
\usepackage{amsmath}
\usepackage{amssymb}
\usepackage{graphicx}
\usepackage{wrapfig}
\pagestyle{plain}
\usepackage{fancybox}
\usepackage{bm}

\begin{document}

50

Praca kontrolna nr 2

30.1. Trójkat prostokatny, obracajac $\mathrm{s}\mathrm{i}\mathrm{e}$ wokól jednej $\mathrm{i}$ drugiej przypros-

tokatnej tworzy bryly $0$ objetościach odpowiednio $V_{1}\mathrm{i}V_{2}$. Obliczyč

objetośč bryly powstalej $\mathrm{z}$ obrotu tego trójkata wokól dwusiecznej

kata prostego.

30.2. Czy $\mathrm{m}\mathrm{o}\dot{\mathrm{z}}$ na sume $42000$ zlotych podzielič na pewna liczbe nagród,

tak aby kwoty tych nagród wyrazaly $\mathrm{s}\mathrm{i}\mathrm{e}\mathrm{w}$ pelnych setkach zlotych,

tworzyly ciag arytmetyczny oraz $\dot{\mathrm{z}}$ eby najwyzsza nagroda wynosila

13000 $\mathrm{z}l$? Jeśli $\mathrm{t}\mathrm{a}\mathrm{k}$, to podač liczbe $\mathrm{i}$ wysokości tych nagród.

30.3 Dane sa okregi $0$ równaniach $(x-1)^{2} + (y-1)^{2} =$ l oraz

$(x-2)^{2}+(y-1)^{2}=16$. Wyznaczyč równania wszystkich okregów

stycznych równocześnie do obu danych okregów oraz do osi $Oy.$

Sporzadzič rysunek.

30.4. $\mathrm{W}$ równolegloboku $\mathrm{k}\mathrm{a}\mathrm{t}$ ostry miedzy przekatnymi ma miare $\beta$, a sto-

sunek dlugości dluzszej przekatnej do krótszej przekatnej wynosi $k.$

Obliczyč tangens kata ostrego tego równolegloboku.

30.5. Rozwiazač równanie $\sqrt{4x-3}-3=\sqrt{2x-10}.$

30.6. Dobrač liczby calkowite $\alpha, b, \mathrm{t}\mathrm{a}\mathrm{k}$ aby wielomian $6x^{3}-7x^{2}+1$ dzielil

siebez reszty przez trójmian kwadratowy $2x^{2}+\alpha x+b.$

30.7. Rozwiazač nierównośč $|2^{x}-3| \leq 2^{1-x}$ Sporzadzič wykresy funkcji

wystepujacych po obu stronach tej nierówności oraz zaznaczyč na

rysunku zbiór rozwiazań.

30.8. Wyznaczyč przedzialy monotoniczności funkcji

$f(x)=\displaystyle \sin^{2}x+\frac{\sqrt{3}}{2}x,$

$x\in[-\pi,\pi].$
\end{document}
