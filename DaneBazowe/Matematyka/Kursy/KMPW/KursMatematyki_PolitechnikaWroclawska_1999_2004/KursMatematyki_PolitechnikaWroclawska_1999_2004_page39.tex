\documentclass[a4paper,12pt]{article}
\usepackage{latexsym}
\usepackage{amsmath}
\usepackage{amssymb}
\usepackage{graphicx}
\usepackage{wrapfig}
\pagestyle{plain}
\usepackage{fancybox}
\usepackage{bm}

\begin{document}

51

Praca kontrolna nr 3

31.1. $\mathrm{Z}$ talii 24 kart do gry wy1osowano 7 kart. Jakie jest prawdopodobień-

stwo otrzymania dokladnie czterech kart wjednym $\mathrm{z}$ czterech kolorów,

$\mathrm{w}$ tym asa, króla $\mathrm{i}$ dame.

31.2. Pewien ostroslup podzielono na trzy cześci dwiema plaszczyznami

równoleglymi do jego podstawy. Pierwsza plaszczyzna $\mathrm{l}\mathrm{e}\dot{\mathrm{z}}\mathrm{y} \mathrm{w}$ od-

leglości $d_{1} = 2$ cm, a druga $\mathrm{w}$ odleglości $d_{2} = 3$ cm od podstawy.

Pola przekrojów ostroslupa tymi plaszczyznami równe sa odpowied-

nio $S_{1}=25\mathrm{c}\mathrm{m}^{2}$ oraz $S_{2}=16\mathrm{c}\mathrm{m}^{2}$ Obliczyč objetośč tego ostroslupa

oraz objetośč najmniejszej cześci.

31.3. Rozwiazač uklad równań

$\left\{\begin{array}{l}
x^{2}+y^{2}=24\\
\frac{2\log x+\log y^{2}}{\log(x+y)}=2.
\end{array}\right.$

31.4. $\mathrm{W}$ trójkacie równoramiennym $ABC$ odleglośč środka okregu wpisane-

go od wierzcholka $C$ wynosi $d$, a podstawe $AB$ widač ze środka okregu

wpisanego pod katem $\alpha$. Obliczyč pole tego trójkata.

31.5. Stosujac zasade indukcji matematycznej, udowodnič prawdziwośč dla

$n\geq 1$ wzoru

$\displaystyle \cos x+\cos 3x++\cos(2n-1)x=\frac{\sin 2nx}{2\sin x},\sin x\neq 0.$

31.6. Wyznaczyč granice ciagu $0$ wyrazie ogólnym

$\displaystyle \alpha_{n}=\frac{\sqrt[6]{4n}}{\sqrt{n}-\sqrt{n+\sqrt[3]{4n^{2}}}},$

$n\geq 1.$

31.7. Dany jest wierzcholek $A(6,1)$ kwadratu. Wyznaczyč pozostale wierz-

cholki tego kwadratu, gdy wierzcholki sasiadujace $\mathrm{z}A\mathrm{l}\mathrm{e}\dot{\mathrm{z}}$ a jeden na

prostej $l$ : $x-2y+1 = 0$, a drugi na prostej $k$ : $x+3y-4 = 0.$

Sporzadzič rysunek.

31.8. Zbadač przebieg zmienności $\mathrm{i}$ narysowač wykres funkcji

$f(x)=\displaystyle \frac{x+1}{\sqrt{x}}.$
\end{document}
