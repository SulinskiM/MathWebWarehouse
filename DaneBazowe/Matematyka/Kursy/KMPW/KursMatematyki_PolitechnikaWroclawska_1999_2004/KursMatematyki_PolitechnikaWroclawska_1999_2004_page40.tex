\documentclass[a4paper,12pt]{article}
\usepackage{latexsym}
\usepackage{amsmath}
\usepackage{amssymb}
\usepackage{graphicx}
\usepackage{wrapfig}
\pagestyle{plain}
\usepackage{fancybox}
\usepackage{bm}

\begin{document}

52

Praca kontrolna nr 4

32.1. Statek $\mathrm{z}$ Wroclawia do Szczecina plynie 3 $\mathrm{d}\mathrm{n}\mathrm{i}$, a ze Szczecina do

Wroclawia 5 $\mathrm{d}\mathrm{n}\mathrm{i}$. Jak dlugo $\mathrm{z}$ Wroclawia do Szczecina plynie woda?

32.2. Dla jakich wartości rzeczywistych $x$ liczby $1 +$ log23, $\log_{x}36,$

$\displaystyle \frac{4}{3}$ log86 sa trzema ko1ejnymi wyrazami ciagu geometrycznego.

32.3. Wanna $0$ pojemności 2001 majaca kszta1t po1owy wa1ca (rozcietego

wzdluz osi) $\mathrm{l}\mathrm{e}\dot{\mathrm{z}}\mathrm{y}$ poziomo na ziemi $\mathrm{i}$ zawiera pewna ilośč wody. Do

wanny wlozono belke ($\mathrm{c}\mathrm{i}\dot{\mathrm{z}}\mathrm{s}\mathrm{z}$ od wody) $\mathrm{w}$ ksztalcie walca $0$ średnicy

cztery razy mniejszej $\mathrm{n}\mathrm{i}\dot{\mathrm{z}}$ średnica wanny $\mathrm{i}$ dlugości równej polowie

dlugości wanny. Okazalo $\mathrm{s}\mathrm{i}\mathrm{e}, \dot{\mathrm{z}}\mathrm{e}$ lustro wody styka $\mathrm{s}\mathrm{i}\mathrm{e}\mathrm{Z}$ powierzchnia

belki zanurzonej $\mathrm{w}$ wodzie. Podač, $\mathrm{z}$ dokladnościa do 0,11, i1e wody

znajduje $\mathrm{s}\mathrm{i}\mathrm{e}\mathrm{w}$ wannie?

32.4. Wyznaczyč wszystkie wartości parametru $m$, dla których obydwa

pierwiastki trójmianu kwadratowego $v(x) = x^{2}+mx-m^{2} \mathrm{l}\mathrm{e}\dot{\mathrm{z}}\mathrm{a}$

miedzy pierwiastkami trójmianu $w(x)=x^{2}-(m-1)x-m.$

32.5. Urna A zawiera trzy kule biale $\mathrm{i}$ dwie czarne, a urna $\mathrm{B}$ dwie kule biale

$\mathrm{i}$ trzy czarne. Wylosowano cztery razy jedna kule (ze zwracaniem)

$\mathrm{z}$ urny A oraz jedna kule $\mathrm{z}$ urny B. Obliczyč prawdopodobieństwo

tego, $\dot{\mathrm{z}}\mathrm{e}$ wśród pieciu wylosowanych kul sa co najmniej dwie kule

biale.

32.6. Rozwiazač równanie 2 $\sin 2x+2\cos 2x+\mathrm{t}\mathrm{g}x=3.$

32.7. Danajest funkcja $f(x)=x^{4}-2x^{2}$ Wyznaczyč wszystkie proste sty-

czne do wykresu $\mathrm{t}\mathrm{e}\mathrm{j}$ funkcji zawierajace punkt $P(1,-1)$. Ile punktów

wspólnych $\mathrm{z}$ wykresem $\mathrm{t}\mathrm{e}\mathrm{j}$ funkcji maja wyznaczone styczne? Rozwia-

zanie zilustrowač rysunkiem.

32.8. Podstawa ostroslupa ABCS jest trójkat równoramienny, którego $\mathrm{k}\mathrm{a}\mathrm{t}$

przy wierzcholku $C$ ma miare $\alpha$, a ramie $BC$ ma dlugośč $b$. Spodek

wysokości ostroslupa $\mathrm{l}\mathrm{e}\dot{\mathrm{z}}\mathrm{y}\mathrm{w}$ środku wysokości $CD$ podstawy, a $\mathrm{k}\mathrm{a}\mathrm{t}$

plaski ściany bocznej $ABS$ przy wierzcholku ma miare $\alpha$. Obliczyč

promień kuli opisanej $\mathrm{n}\mathrm{a}\mathrm{t}\mathrm{y}\mathrm{m}$ ostroslupie oraz cosinusy katów nachyle-

nia ścian bocznych do podstawy.
\end{document}
