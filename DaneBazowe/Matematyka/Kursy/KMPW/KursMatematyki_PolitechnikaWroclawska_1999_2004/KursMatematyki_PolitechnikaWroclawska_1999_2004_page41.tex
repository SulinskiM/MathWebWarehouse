\documentclass[a4paper,12pt]{article}
\usepackage{latexsym}
\usepackage{amsmath}
\usepackage{amssymb}
\usepackage{graphicx}
\usepackage{wrapfig}
\pagestyle{plain}
\usepackage{fancybox}
\usepackage{bm}

\begin{document}

53

Praca kontrolna nr 5

33.1. Piaty wyraz rozwiniecia dwumianu $(\alpha+b)^{18}$, gdzie $\alpha, b > 0,$

jest $0$ 180\% wiekszy od wyrazu trzeciego. $\mathrm{O}$ ile procent wyraz ósmy

tego rozwiniecia jest mniejszy $\mathrm{b}\mathrm{a}\mathrm{d}\acute{\mathrm{z}}$ wiekszy od wyrazu czwartego?

33.2. Wyznaczyč równanie linii utworzonej przez wszystkie punkty

plaszczyzny, dla których stosunek kwadratu odleglości od prostej

$k:x-2y+3=0$ do kwadratu odleglości od prostej $l:3x+y+2=0$

wynosi 2. Sporzadzič rysunek.

33.3. Obwód trójkata $ABC$ wynosi 15 cm, a dwusieczna kata $A$ dzieli bok

przeciwlegly na odcinki dlugości 3 cm oraz 2 cm. Ob1iczyč po1e ko1a

wpisanego $\mathrm{w}$ ten trójkat.

33.4. Czastka startuje $\mathrm{z}$ poczatku ukladu wspólrzednych $\mathrm{i}$ porusza $\mathrm{s}\mathrm{i}\mathrm{e}$ ze

sta a predkościa po nieskończo-

nej amanej jak na rysunku, któ- 

rej $\mathrm{d}$ ugości kolejnych odcinków
\begin{center}
\includegraphics[width=44.352mm,height=37.488mm]{./KursMatematyki_PolitechnikaWroclawska_1999_2004_page41_images/image001.eps}
\end{center}
$\alpha_{2}$

$\alpha_{3}$

$\alpha_{1}$

$\alpha_{4}$

{\it O}

{\it P}

tworza ciag geometryczny maleja-

cy. Po pewnym czasie czastka za-

trzyma a $\mathrm{s}\mathrm{i}\mathrm{e} \mathrm{w}$ punkcie $P(10,3).$

Jaka droge przeby a czastka?

33.5. Stosujac zasade indukcji matematycznej, udowodnič, $\dot{\mathrm{z}}\mathrm{e}$ dla wszyst-

kich $n \geq 1$ wielomian $x^{3n+1}+x^{3n-1}+1$ dzieli $\mathrm{s}\mathrm{i}\mathrm{e}$ bez reszty przez

wielomian $x^{2}+x+1.$

33.6. Narysowač wykres funkcji $f(x) = \displaystyle \frac{|x-2|}{x-|x|+2}$ bez badania jej prze-

biegu. Podač równania asymptot $\mathrm{i}$ ekstrema lokalne tej funkcji.

33.7. Rozwiazač nierównośč

$|\cos x|^{1+\sqrt{2}\sin x+\sqrt{2}\cos x}\leq 1,$

$x\in[-\pi,\pi].$

33.8. $\mathrm{W}$ stozek wpisano graniastoslup trójkatny prawidlowy $0$ wszystkich

krawedziach tej samej dlugości, tak $\dot{\mathrm{z}}\mathrm{e}$ dolna podstawa $\mathrm{l}\mathrm{e}\dot{\mathrm{z}}\mathrm{y}$ na pod-

stawie stozka. Przy jakim kacie rozwarcia stozka stosunek objetości

graniastoslupa do objetości stozka jest najwiekszy?
\end{document}
