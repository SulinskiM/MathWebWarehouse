\documentclass[a4paper,12pt]{article}
\usepackage{latexsym}
\usepackage{amsmath}
\usepackage{amssymb}
\usepackage{graphicx}
\usepackage{wrapfig}
\pagestyle{plain}
\usepackage{fancybox}
\usepackage{bm}

\begin{document}

54

Praca kontrolna nr 6

34.1. $\mathrm{W}$ kolo $0$ polu $\displaystyle \frac{5}{4}\pi$ wpisano trójkat prostokatny $0$ polu l.

obwód tego trójkata.

Obliczyč

34.2. Sprowadzič do najprostszej postaci wyrazenie

2(sin6 $\alpha+\cos^{6}\alpha$)$-7(\sin^{4}\alpha+\cos^{4}\alpha)+\cos 4\alpha.$

34.3. Wyznaczyč trójmian kwadratowy, którego wykresem jest parabola

styczna do prostej $y=x+2$, przechodzaca przez punkt $P(-2,-2)$

oraz symetryczna wzgledem prostej $x=1$. Sporzadzič rysunek.

34.4. $\mathrm{W}$ trapezie ABCD, $\mathrm{w}$ którym AB $\Vert CD$, dane sa $\vec{AC}= [4$, 7$]$ oraz

$\vec{BD}=[-6,2]$. Za pomoca rachunku wektorowego wyznaczyč wektory

$\vec{AB}\mathrm{i}\vec{CD}$, jeśli wiadomo, $\dot{\mathrm{z}}\mathrm{e} \vec{AD}\perp\vec{BD}.$

34.5. Jaś ma $\mathrm{w}$ portmonetce 3 monetyjednoz1otowe, 2 monety dwuz1otowe

$\mathrm{i}$ jedna pieciozlotowa. Kupujac zeszyt $\mathrm{w}$ cenie 4 $\mathrm{z}l$, wyciaga losowo

$\mathrm{z}$ portmonetki po jednej monecie tak dlugo, $\mathrm{a}\dot{\mathrm{z}}$ uzbiera $\mathrm{s}\mathrm{i}\mathrm{e}$ suma wys-

tarczajaca na kupno zeszytu. Obliczyč prawdopodobieństwo, $\dot{\mathrm{z}}\mathrm{e}$ Jaś

wyciagnie co najmniej trzy monety. Podač odpowiednie uzasadnienie

(nie jest nim $\mathrm{t}\mathrm{z}\mathrm{w}$. drzewko).

34.6. Narysowač na plaszczy $\acute{\mathrm{z}}\mathrm{n}\mathrm{i}\mathrm{e}$ zbiór punktów określony nastepujaco

$\mathcal{F}=\{(x,y):\sqrt{4x-x^{2}}\leq y\leq 4-\sqrt{1-2x+x^{2}}\}.$

Wjakiej odleglości od brzegu figury $\mathcal{F}$ znajduje $\mathrm{s}\mathrm{i}\mathrm{e}$ punkt $P(\displaystyle \frac{3}{2},\frac{5}{2})$ ?

34.7. Danajest funkcja $f(x)=\log_{2}(1-x^{2})-\log_{2}(x^{2}-x)$. Bez stosowania

metod rachunku rózniczkowego wykazač, $\dot{\mathrm{z}}\mathrm{e}f$ jest rosnaca $\mathrm{w}$ swojej

dziedzinie oraz, $\dot{\mathrm{z}}\mathrm{e}g(x) = f(x-\displaystyle \frac{1}{2})$ jest nieparzysta. Wyznaczyč

funkcje odwrotna $f^{-1}$, jej dziedzine $\mathrm{i}$ zbiór wartości.

34.8. Pole powierzchni bocznej ostroslupa prawidlowego czworokatnego wy-

nosi $c^{2}$, a $\mathrm{k}\mathrm{a}\mathrm{t}$ nachylenia ściany bocznej do podstawy ma miare $\alpha.$

Ostroslup przecieto na dwie cześci plaszczyzna przechodzaca przez

jeden $\mathrm{z}$ wierzcholków podstawy $\mathrm{i}$ prostopadla do przeciwleglej krawe-

dzi bocznej. Obliczyč objetośč cześci zawierajacej wierzcholek ostro-

slupa. Podač warunek rozwiazalności zadania.
\end{document}
