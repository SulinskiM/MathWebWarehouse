\documentclass[a4paper,12pt]{article}
\usepackage{latexsym}
\usepackage{amsmath}
\usepackage{amssymb}
\usepackage{graphicx}
\usepackage{wrapfig}
\pagestyle{plain}
\usepackage{fancybox}
\usepackage{bm}

\begin{document}

55

Praca kontrolna nr 7

35.1. Dwa pierwsze wyrazy nieskończonego ciagu geometrycznego sa pier-

wiastkami równania $4x^{2}-4px-3p^{2}=0$, gdzie $p$ jest nieznana liczba.

Wyznaczyč ten ciag, jeśli suma wszystkich jego wyrazów wynosi 3.

35.2. Wiedzac, $\dot{\mathrm{z}}\mathrm{e} \cos\varphi=\sqrt{\frac{2}{3}}$ oraz $\varphi\in (\displaystyle \frac{3}{2}\pi,2\pi)$, obliczyč cosinus kata

pomiedzy prostymi $y= (\displaystyle \sin\frac{\varphi}{2})x, y= (\displaystyle \cos\frac{\varphi}{2})x.$

35.3. Kostka sześcienna ma krawed $\acute{\mathrm{z}}  2\alpha$. Aby zmieścič $\mathrm{j}\mathrm{a}\mathrm{w}$ pojemniku

$\mathrm{w}$ ksztalcie kuli $0$ średnicy $ 3\alpha$, ze wszystkich narozników odcieto

$\mathrm{w}$ minimalny sposób jednakowe ostroslupy prawidlowe trójkatne.

Obliczyč dlugośč krawedzi bocznych odcietych czworościanów?

35.4. Udowodnič prawdziwośč nierówności

$1+\displaystyle \frac{x}{2}\geq\sqrt{1+x}\geq 1+\frac{x}{2}-\frac{x^{2}}{2}$ dla $x\in[-1,1].$

Zilustrowač $\mathrm{j}\mathrm{a}$ na odpowiednim wykresie.

35.5. Rozwiazač równanie

--csions25{\it xx}$=$-sin3{\it x}.

35.6. Dany jest okrag $\mathcal{K}$ : $x^{2}-4x+y^{2}+6y = 0$. Znalez$\acute{}$č równanie

okregu symetrycznego do $\mathcal{K}$ wzgledem stycznej do $\mathcal{K}$ poprowadzonej

$\mathrm{z}$ punktu $P(3,5)\mathrm{i}$ majacej dodatni wspólczynnik kierunkowy.

35.7. W okrag 0 promieniu r wpisano trapez 0 przekatnej d,

i najwiekszym obwodzie. Obliczyč pole tego trapezu.

$ d\geq r\sqrt{3},$

35.8. Metoda analityczna określič dla jakich wartości parametru $m$ uklad

równań

$\left\{\begin{array}{l}
mx-y+2=0\\
x-2|y|+2=0
\end{array}\right.$

ma dokladnie jedno rozwiazanie. Wyznaczyč to rozwiazanie $\mathrm{w}$ zalez-

ności od $m$. Sporzadzič rysunek.
\end{document}
