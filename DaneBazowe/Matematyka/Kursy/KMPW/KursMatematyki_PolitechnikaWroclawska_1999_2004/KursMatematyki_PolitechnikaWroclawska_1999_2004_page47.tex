\documentclass[a4paper,12pt]{article}
\usepackage{latexsym}
\usepackage{amsmath}
\usepackage{amssymb}
\usepackage{graphicx}
\usepackage{wrapfig}
\pagestyle{plain}
\usepackage{fancybox}
\usepackage{bm}

\begin{document}

61

8. WIasności funkcji

$\bullet$ 3.1 (10.3), 15.5, 16.6, 26.7, 34.7.

9. Ciagi liczbowe

$\bullet$ Ciag arytmetyczny: 4.1, 13.5, 19.2, 22.2, 30.2.

$\bullet$ Ciag geometryczny: 6.4, 15.4, 17.1, 18.1, 32.2.

$\bullet$ Ciag geometryczny nieskończony:

24.1, 25.8, 28.7, 29.8, 33.4, 35.1.

1.2, 2.8, 8.1, 11.4, 14.2, 15.8,

$\bullet$ Granica ciagu: 14.6, 18.1, 21.4, 27.5, 31.6.

$\bullet$ Wlasności ciagu: 21.4, 23.6.

10.

Granica i ciagIośč funkcji.

Pochodna, styczna

$\bullet$ 1.8, 4.3, 9.8, 16.8, 20.7, 21.8, 22.8, 23.7, 24.8, 27.8, 32.7.

ll. Zastosowania pochodnej

$\bullet$ Badanie przebiegu funkcji: 3.6, 4.7, 9.7, 13.8, 15.8, 25.7, 31.8.

$\bullet$ Ekstrema lokalne: 8.3, 11.3, 26.5, 28.5, 33.6.

$\bullet$ Wartośč najmniejsza $\mathrm{i}$ najwieksza funkcji $\mathrm{w}$ zbiorze: 2.2, 3.1 (10.3),

3.5, 6.7, 10.6, 12.8 (26.8), 18.8, 19.5, 21.7, 22.8, 25.2, 29.8, 33.8, 35.7.

$\bullet$ Inne: 5.7, 24.8, 30.8.
\end{document}
