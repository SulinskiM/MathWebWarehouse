\documentclass[a4paper,12pt]{article}
\usepackage{latexsym}
\usepackage{amsmath}
\usepackage{amssymb}
\usepackage{graphicx}
\usepackage{wrapfig}
\pagestyle{plain}
\usepackage{fancybox}
\usepackage{bm}

\begin{document}

9

Praca kontrolna nr l

l.l. Stop sklada $\mathrm{s}\mathrm{i}\mathrm{e} \mathrm{z}$ 40\% srebra próby 0,6, 30\% srebra próby 0,7 oraz

l kg srebra próby 0,8. Jaka jest masa $\mathrm{i}$ jaka jest próba tego stopu?

1.2. Rozwiazač równanie

$3^{x}+1+3^{-x}+=4,$

którego lewa strona jest suma nieskończonego ciagu geometrycznego.

1.3. $\mathrm{W}$ trójkacie $ABC$ znane sa wierzcholki $A(0,0)$ oraz $B(4,-1)$. Wiado-

mo, $\dot{\mathrm{z}}\mathrm{e}\mathrm{w}$ punkcie $H(3,2)$ przecinaja $\mathrm{s}\mathrm{i}\mathrm{e}$ proste zawierajace wysokości

tego trójkata. Wyznaczyč wspólrzedne wierzcholka $C$. Sporzadzič

rysunek.

1.4. Rozwiazač równanie

$\cos 4x=\sin 3x.$

1.5. Narysowač staranny wykres funkcji

$f(x)=|\log_{2}(x-2)^{2}|.$

1.6. Rozwiazač nierównośč

$\displaystyle \frac{1}{x^{2}}\geq\frac{1}{x+6}.$

1.7. $\mathrm{W}$ ostroslupie prawidlowym sześciokatnym krawed $\acute{\mathrm{z}}$ podstawy ma dlu-

gośč $p$, a krawed $\acute{\mathrm{z}}$ boczna dlugośč $2p$. Obliczyč cosinus kata dwuścien-

nego miedzy sasiednimi ścianami bocznymi tego ostroslupa.

1.8. Wyznaczyč równania wszystkich prostych stycznych do wykresu funkcji

$f(x)=\displaystyle \frac{2x+10}{x+4},$

które sa równolegle do prostej stycznej do wykresu funkcji

$g(x)=\sqrt{1-x}\mathrm{w}$ punkcie $x=0$. Rozwiazanie zilustrowač rysunkiem.
\end{document}
