\documentclass[a4paper,12pt]{article}
\usepackage{latexsym}
\usepackage{amsmath}
\usepackage{amssymb}
\usepackage{graphicx}
\usepackage{wrapfig}
\pagestyle{plain}
\usepackage{fancybox}
\usepackage{bm}

\begin{document}

67

l.l. Masa stopu 2,3 kg, próba stopu 0,690.

1.2. 1og32.

1.3. {\it C}(--3101'--1101).

1.4. $\displaystyle \frac{\pi}{14}+k\frac{2\pi}{7},  k\in$ Z.

1.5. Rysunek l.
\begin{center}
\includegraphics[width=108.000mm,height=84.228mm]{./KursMatematyki_PolitechnikaWroclawska_1999_2004_page51_images/image001.eps}
\end{center}
{\it y}  í í

4

2

$-2$  0 1  3 4  6 {\it x}

Rys. l

1.6. $(- 00,-6)\cup[-2,0)\cup(0$, 3$].$

1.7. $-\displaystyle \frac{3}{5}.$

1.8. $\mathrm{s}_{\mathrm{a}}$ dwie takie styczne $\mathrm{i}$ maja równania $y = -\displaystyle \frac{1}{2}x +2$ oraz

$y=-\displaystyle \frac{1}{2}x-2.$

2.2. Podstawa prostokata $\alpha = \displaystyle \frac{2}{5}\sqrt{10}$ cm, wysokośč $b = \displaystyle \frac{4}{5}\sqrt{10}$ cm,

przekatna $p=2\sqrt{2}$ cm.
\end{document}
