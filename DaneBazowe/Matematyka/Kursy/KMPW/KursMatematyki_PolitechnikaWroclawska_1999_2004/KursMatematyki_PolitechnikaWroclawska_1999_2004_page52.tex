\documentclass[a4paper,12pt]{article}
\usepackage{latexsym}
\usepackage{amsmath}
\usepackage{amssymb}
\usepackage{graphicx}
\usepackage{wrapfig}
\pagestyle{plain}
\usepackage{fancybox}
\usepackage{bm}

\begin{document}

68

2.3. $(1,\sqrt[3]{9})\cup(\sqrt[3]{9},\infty).$

2.4. $4-2\sqrt{2}<p<4+2\sqrt{2}.$

2.5. Dla $m<0$ brak rozwiazań,

dla $m=0$ lub $m>1$ sa dwa rozwiazania,

dla $m=1$ sa trzy rozwiazania,

dla $0<m<1$ sa cztery rozwiazania.

2.6. Uklad ma trzy rozwiazania:

$\left\{\begin{array}{l}
x_{1}=-7\\
y_{1}=-1,
\end{array}\right.$

2.7. $S=\displaystyle \frac{1225}{12}.$

$\left\{\begin{array}{l}
x_{2}=1\\
y_{2}=7,
\end{array}\right.$

$\left\{\begin{array}{l}
x_{3}=5\\
y_{3}=-5.
\end{array}\right.$

2.8. -$\pi$8'--78$\pi$, --98$\pi$, --158$\pi$.

3.1. Dziedzina jest przedzial $[0$, 4$]$, a zbiorem wartości przedzial $[0,\displaystyle \frac{3}{2}]$.

3.2. Prosta $AB$ ma równanie $y=3$, a prosta $AD$ równanie $4x-3y=15$.

3.4. $\displaystyle \frac{8}{3}\sqrt{3}\mathrm{c}\mathrm{m}^{3}$

3.5. Trójkat równoboczny $0$ boku $R\sqrt{3}\mathrm{i}$ polu $\displaystyle \frac{3\sqrt{3}}{4}R^{2}$

3.6. $D = (-\displaystyle \infty,\frac{5}{2}]$; miejsca zerowe $0, \displaystyle \frac{5}{2}$; minimum lokalne 0

dla $x = 0$; maksimum lokalne 2 d1a $x = 2$; funkcja rosnaca $\mathrm{w} (0,2)$ ;

malejaca $\mathrm{w} (-\infty,0)$ oraz $\mathrm{w} (2,\displaystyle \frac{5}{2})$ ; wypukla $\mathrm{w} (-\displaystyle \infty,2-\frac{\sqrt{6}}{3})$ ;

wklesla $\mathrm{w} (2-\displaystyle \frac{\sqrt{6}}{3},\frac{5}{2})$ ; punkt przegiecia $P(2-\displaystyle \frac{\sqrt{6}}{3},\sqrt{\frac{62\sqrt{6}-117}{27}})$ ;

prosta $x= \displaystyle \frac{5}{2}$ jest styczna do wykresu funkcji $\mathrm{w}$ punkcie $(\displaystyle \frac{5}{2},0)$. Wykres

funkcji przedstawiono na rysunku 2.
\end{document}
