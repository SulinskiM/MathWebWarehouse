\documentclass[a4paper,12pt]{article}
\usepackage{latexsym}
\usepackage{amsmath}
\usepackage{amssymb}
\usepackage{graphicx}
\usepackage{wrapfig}
\pagestyle{plain}
\usepackage{fancybox}
\usepackage{bm}

\begin{document}

69
\begin{center}
\includegraphics[width=66.036mm,height=56.544mm]{./KursMatematyki_PolitechnikaWroclawska_1999_2004_page53_images/image001.eps}
\end{center}
{\it y}

2

1  {\it P}

$-1$  0 1  2  3 {\it x}

Rys. 2

3.7.

$S = \displaystyle \frac{d^{2}-2dr\cos\alpha+r^{2}\cos 2\alpha}{2(d-r\cos\alpha)}r\sin\alpha$; $R = \displaystyle \frac{d^{2}-2dr\cos\alpha+r^{2}}{4(d-r\cos\alpha)\sin\alpha}$;

rozwiazanie istnieje, gdy $d\geq r(1+2\cos\alpha)$. Wynik liczbowy $S=\displaystyle \frac{13}{12}\sqrt{3}\mathrm{c}\mathrm{m}^{2},$

$R=\displaystyle \frac{7}{3}$ cm.

3.8.[0,-1$\pi$2]$\cup$[-71$\pi$2'-1132$\pi$]$\cup$[-1192$\pi$,-2152$\pi$]$\cup$[-3112$\pi$,3$\pi$].

4.1.109.

4.2.-97.

lub $\{$

4.3. Pochodna nie istnieje.

4.5. $\{x\leq 11<y\leq 3-x$

$0<y<1$

$1\leq x\leq 3-y.$

4.6. Elipsa $0$ równaniu $\displaystyle \frac{(x+4)^{2}}{36}+\frac{y^{2}}{20}=1$, środku $M(-4,0)\mathrm{i}$ pólosiach

$\alpha=6, b=2\sqrt{5}.$
\end{document}
