\documentclass[a4paper,12pt]{article}
\usepackage{latexsym}
\usepackage{amsmath}
\usepackage{amssymb}
\usepackage{graphicx}
\usepackage{wrapfig}
\pagestyle{plain}
\usepackage{fancybox}
\usepackage{bm}

\begin{document}

70

4.7. $y = f(m) = \displaystyle \frac{4}{m}-\frac{4}{m^{2}}. D = ($-00, $0)\cup(0,5]$; miejsce zerowe l;

asymptota pionowa obustronna $m = 0$; asymptota pozioma lewostronna

$y=0$; maksimum lokalne l dla $m=2$; funkcja rosnaca $\mathrm{w} (0,2)$ ; malejaca

$\mathrm{w} (-\infty,0)$ oraz $\mathrm{w} (2,5)$ ; wypukla $\mathrm{w} (3,5)$ ; wklesla $\mathrm{w} (-\infty,0)$ oraz

$\mathrm{w}(0,3)$ ; punkt przegiecia $P(3,\displaystyle \frac{8}{9})$. Wykres przedstawiono na rysunku 3.
\begin{center}
\includegraphics[width=135.588mm,height=70.608mm]{./KursMatematyki_PolitechnikaWroclawska_1999_2004_page54_images/image001.eps}
\end{center}
$y$

1

{\it P}

$-4  -2$  0 2 3  {\it 5 m}

$-1$

$-3$

Rys. 3

4.8. $S=\displaystyle \frac{\pi}{16}\alpha^{2_{\cos^{2}\alpha(3-4\cos^{2}\alpha)}}27-32\cos^{2}\alpha, \alpha\in \displaystyle \frac{\pi}{6}, \displaystyle \frac{\pi}{2}$

5.1. Zbiór $A$ przedstawiono na rysunku 4.

najblizej punktu P.

Punkt $Q \displaystyle \frac{3}{2}, \displaystyle \frac{5}{2}$

$\mathrm{l}\mathrm{e}\dot{\mathrm{z}}\mathrm{y}$
\begin{center}
\includegraphics[width=60.192mm,height=72.084mm]{./KursMatematyki_PolitechnikaWroclawska_1999_2004_page54_images/image002.eps}
\end{center}
{\it y}

{\it P}

{\it Q}

2

$-1$  2  {\it x}

Rys. 4
\end{document}
