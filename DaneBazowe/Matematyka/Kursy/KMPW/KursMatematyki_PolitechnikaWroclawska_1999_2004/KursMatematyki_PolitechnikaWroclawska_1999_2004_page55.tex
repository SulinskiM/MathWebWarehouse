\documentclass[a4paper,12pt]{article}
\usepackage{latexsym}
\usepackage{amsmath}
\usepackage{amssymb}
\usepackage{graphicx}
\usepackage{wrapfig}
\pagestyle{plain}
\usepackage{fancybox}
\usepackage{bm}

\begin{document}

71

5.2. $\displaystyle \frac{7}{16}\sqrt{5}$ lub - $\displaystyle \frac{7}{16}\sqrt{5}.$

5.3. Szukana krzywa stanowia dwie galezie paraboli $y= \displaystyle \frac{1}{2}x^{2}-1$ dla

$x\geq 2$ oraz dla $x\leq 2.$

5.4. 11.

5.5. Pierwszy.

5.6. $2r+4\sqrt{2Rr-R^{2}}.$

5.7. Dla $ m\in [2\sqrt{3},\infty$).

5.8. $\displaystyle \frac{9}{85}\sqrt{85}.$

6.1. $\displaystyle \frac{1}{4}(-3+3\sqrt{3}).$

6.2. $\displaystyle \frac{13}{3}.$

6.4. $8+(1+\sqrt{33})^{3/2}$

6.5. $\displaystyle \frac{3}{10}.$

6.6. $\displaystyle \frac{\pi}{12}d^{3}\mathrm{t}\mathrm{g}^{2}\alpha(8\cos^{4}\alpha-1).$

6.7. Wartośč najmniejsza 3l, a najwieksza $24\sqrt{2}$.

6.8. Stosunek wynosi $1+k$, a dziedzina $k$ jest przedzial $(0,\sqrt{2}-1$].

7.1. $(0,1).$

7.2.

Elipsa $0$ równaniu $\displaystyle \frac{(x+1)^{2}}{4} + \displaystyle \frac{(y-3)^{2}}{1} =$

l, środku $S(-1,3)$

$\mathrm{i}$ pólosiach $\alpha=2, b=1$. Pole figury wynosi $2\pi.$
\end{document}
