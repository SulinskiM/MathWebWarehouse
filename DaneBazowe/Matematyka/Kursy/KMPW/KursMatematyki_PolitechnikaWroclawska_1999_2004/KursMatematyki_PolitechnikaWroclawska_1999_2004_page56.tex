\documentclass[a4paper,12pt]{article}
\usepackage{latexsym}
\usepackage{amsmath}
\usepackage{amssymb}
\usepackage{graphicx}
\usepackage{wrapfig}
\pagestyle{plain}
\usepackage{fancybox}
\usepackage{bm}

\begin{document}

72

7.3. 330.

7.4. $\displaystyle \frac{\pi}{4}(\sqrt{8}-\sqrt{6}).$

7.5. Dla $m$ róznych od 3 $\mathrm{i}4$ jedno rozwiazanie $x=\displaystyle \frac{9}{m-4},\ y=\displaystyle \frac{m+2}{m-4}$.

Dla $m = 4$ uklad sprzeczny. Dla $m = 3$ nieskończenie wiele rozwiazań

spelniajacych warunek $x-2y = 1$, gdzie $x$ dowolne rzeczywiste. $\mathrm{s}_{\mathrm{a}}$ dwa

rozwiazania spelniajace warunek $x=y$: dla $m=7 (x=y=3)$ oraz dla

$m=3 (x=y=-1).$

7.6. $(-\displaystyle \frac{\pi}{3},0)\cup(\frac{\pi}{3},\frac{\pi}{2}].$

7.7. Objetośč ostroslupa wynosi $\displaystyle \frac{343}{3}\mathrm{c}\mathrm{m}^{3}$, a objetośč najmniejszej cześci

$\displaystyle \frac{61}{3}\mathrm{c}\mathrm{m}^{3}$

7.8. a) $\displaystyle \frac{1}{20};\mathrm{b}) \displaystyle \frac{7}{20}.$

8.1. $q=\displaystyle \frac{1}{30}, \alpha_{1}=1972.$

8.2. $\mathrm{s}_{\mathrm{a}}$ dwa takie skladniki 26730 oraz 1320.

8.3. Wykres funkcji przedstawiono na rysunku 5.

Rys. 5
\end{document}
