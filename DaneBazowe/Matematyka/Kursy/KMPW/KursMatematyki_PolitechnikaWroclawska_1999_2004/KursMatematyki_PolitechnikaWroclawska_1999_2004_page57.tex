\documentclass[a4paper,12pt]{article}
\usepackage{latexsym}
\usepackage{amsmath}
\usepackage{amssymb}
\usepackage{graphicx}
\usepackage{wrapfig}
\pagestyle{plain}
\usepackage{fancybox}
\usepackage{bm}

\begin{document}

73

Maksima lokalne 4 d1a $x = 1 \mathrm{i}x = -1$; minima lokalne 0 d1a $x = 3$

$\mathrm{i} x = -3$ oraz 3 d1a $x = 0$. Funkcja rosnaca $\mathrm{w}$ przedzialach $(-3,-1),$

$(0,1), ($3, $\infty)$ ; malejaca $\mathrm{w}$ przedzialach $(-\infty,-3), (-1,0)$, (1, 3).

8.4.

(-23, 2].

8.5. $\displaystyle \frac{1}{48}\alpha^{3}\sqrt{\sqrt{52}-2}.$

8.6. -32, 1, -27, --139.

8.7. $\displaystyle \frac{\pi}{9}+k\frac{\pi}{3}$ lub $\displaystyle \frac{2\pi}{9}+k\frac{\pi}{3},  k\in$ Z.

8.8. $(\sqrt{S_{1}}+\sqrt{S_{2}}+\sqrt{S_{3}})^{2}$

9.1. $\mathrm{O}$ 72,8\%.

9.2. Prawa gala $\acute{\mathrm{z}}$ hiperboli $0$ równaniu $y=\displaystyle \frac{1}{2}+\frac{1}{2(x-1)}$,

$x>1.$

9.3. $m\in(1,2).$

9.4. $\sqrt{3}.$

9.5. $(- 00,-3)\cup[1,3)\cup(3$, 5$].$

9.6. $\displaystyle \frac{\sqrt{1+k^{2}}-1+k}{k^{2}\sqrt{2}}.$

9.7. $D=\mathrm{R}\backslash \{2\}$; asymptota pionowa obustronna $x= 2$; asymptota

pozioma obustronna $y=1$; minimum lokalne $\displaystyle \frac{1}{2}$ dla $x=-2$; funkcja rosnaca

$\mathrm{w} (-2,2)$ ; malejaca $\mathrm{w} (- 00,-2)$ oraz $\mathrm{w} (2,\infty)$ ; wypukla $\mathrm{w} (-4,2)$ oraz

$\mathrm{w}(2,\infty)$ ; wklesla $\mathrm{w}(-\infty,-4)$ ; punkt przegiecia $P(-4,\displaystyle \frac{5}{9})$. Wykres funkcji

przedstawiono na rysunku 6.
\end{document}
