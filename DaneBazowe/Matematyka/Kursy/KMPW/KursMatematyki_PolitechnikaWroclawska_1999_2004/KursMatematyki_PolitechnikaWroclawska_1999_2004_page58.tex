\documentclass[a4paper,12pt]{article}
\usepackage{latexsym}
\usepackage{amsmath}
\usepackage{amssymb}
\usepackage{graphicx}
\usepackage{wrapfig}
\pagestyle{plain}
\usepackage{fancybox}
\usepackage{bm}

\begin{document}

74
\begin{center}
\includegraphics[width=144.324mm,height=79.656mm]{./KursMatematyki_PolitechnikaWroclawska_1999_2004_page58_images/image001.eps}
\end{center}
{\it y}

5

1

$-4$

{\it P}

$-2$

2 4  6 {\it x}

Rys. 6

9.8. $y=10x-16, y=-\displaystyle \frac{5}{4}x-\frac{1}{4}, y=-\displaystyle \frac{38}{25}x+\frac{16}{125}.$

10.2. $V=-\displaystyle \frac{\pi}{6}l^{3}\sin 4\alpha\cos 2\alpha, \varphi=3\pi-4\alpha, \alpha\in (\displaystyle \frac{\pi}{2},\frac{3\pi}{4}).$

10.3. Dziedzinajest przedzial $[0$, 4$]$, a zbiorem wartości przedzial $[0,\displaystyle \frac{3}{2}].$

10.4. $\displaystyle \frac{240}{1771}\approx 0$, 136.

10.5. $(\displaystyle \frac{1}{4},\frac{1}{2})\cup[2$, 4$].$

10.6.

$r=\displaystyle \frac{1}{2}.$

$S(r) = r(1-r^{2})^{3/2}, r \in (0,1)$. Wartośč najwieksza $\displaystyle \frac{3\sqrt{3}}{16}$ dla

10.7. Uklad ma cztery rozwiazania:

$\left\{\begin{array}{l}
x_{1}=0\\
y_{1}=0,
\end{array}\right.$

$\left\{\begin{array}{l}
x_{2}=\frac{16}{5}\\
y_{2}=\frac{12}{5},
\end{array}\right.$

$\left\{\begin{array}{l}
x_{3}=-\frac{16}{5}\\
y_{3}=-\frac{12}{5},
\end{array}\right.$

$\left\{\begin{array}{l}
x_{4}=4\\
y_{4}=-2.
\end{array}\right.$
\end{document}
