\documentclass[a4paper,12pt]{article}
\usepackage{latexsym}
\usepackage{amsmath}
\usepackage{amssymb}
\usepackage{graphicx}
\usepackage{wrapfig}
\pagestyle{plain}
\usepackage{fancybox}
\usepackage{bm}

\begin{document}

75

Pierwsze równanie przedstawia dwa okregi styczne do osi $Oy 0$ środkach

$S_{1}(\displaystyle \frac{5}{2},0), S_{2}(-\displaystyle \frac{5}{2},0) \mathrm{i}$ promieniu $\displaystyle \frac{5}{2}$. Drugie równanie przedstawia dwie

proste równolegle.

10.8. Rozwiazania równania: $\displaystyle \frac{3\pi}{8}, \displaystyle \frac{7\pi}{8}.$

$[0,\displaystyle \frac{3\pi}{8})\cup(\frac{7\pi}{8},\pi].$

11.1. $\displaystyle \frac{\pi}{6}.$

Zbiór rozwiazań nierówności:

11.2. $\displaystyle \frac{2\sqrt{6}}{3}R.$

11.3. 4, $-4.$

11.4. $(\displaystyle \frac{\pi}{4},\frac{3\pi}{4})\cup(\frac{5\pi}{4},\frac{7\pi}{4}).$

11.6. $|y|=\displaystyle \frac{(x+2)^{2}}{4}-1, x\in(-\infty,-4)\cup(0,\infty).$

11.7. $3^{\sqrt{11}}, 3^{-\sqrt{11}}.$

11.8. $[-5,0)\cup(5$, 6$].$

12.1. $\displaystyle \frac{9-\sqrt{5}}{2}.$

12.2. $-1.$

12.3. $\displaystyle \frac{107}{128}\approx 0$, 836.

12.4. $(x-1)^{2}+(y-1)^{2}=1, (x-6)^{2}+(y-6)^{2}=36, (x+2)^{2}+(y-2)^{2}=4,$

$(x-3)^{2}+(y+3)^{2}=9.$

12.5. -32{\it d}3--$\sqrt{}$tgtg2$\alpha$3$\alpha$-1' $\alpha\in$ (-$\pi$4'-$\pi$2).

12.6. $s+\displaystyle \frac{16Pr}{\sqrt{16P^{2}+s^{4}}}$. Warunek rozwiazalności $r\displaystyle \geq\frac{\sqrt{16P^{2}+s^{4}}}{4s}.$
\end{document}
