\documentclass[a4paper,12pt]{article}
\usepackage{latexsym}
\usepackage{amsmath}
\usepackage{amssymb}
\usepackage{graphicx}
\usepackage{wrapfig}
\pagestyle{plain}
\usepackage{fancybox}
\usepackage{bm}

\begin{document}

10

Praca kontrolna nr 2

2.1. Udowodnič, $\dot{\mathrm{z}}\mathrm{e}$ dla $\mathrm{k}\mathrm{a}\dot{\mathrm{z}}$ dego $n$ naturalnego wielomian $x^{4n-2}+1$ jest

podzielny przez trójmian kwadratowy $x^{2}+1.$

2.2. $\mathrm{W}$ równoramienny trójkat prostokatny $0$ polu $S = 10\mathrm{c}\mathrm{m}^{2}$ wpisano

prostokat $\mathrm{w}$ taki sposób, aby jeden $\mathrm{z}$ jego boków $\mathrm{l}\mathrm{e}\dot{\mathrm{z}}\mathrm{a}l$ na przeci-

wprostokatnej trójkata, a pozostale dwa wierzcholki znalazly $\mathrm{s}\mathrm{i}\mathrm{e}$ na

przyprostokatnych $\mathrm{i}$ równocześnie $\mathrm{t}\mathrm{a}\mathrm{k}$, aby mial on najkrótsza prze-

katna. Obliczyč dlugośč przekatnej tego prostokata.

2.3. Rozwiazač nierównośč

log1253 $\log_{x}5+\log_{9}8\log_{4}x>1.$

2.4. Znalez$\acute{}$č wszystkie wartości parametru $p$, dla których wykres funkcji

$y=x^{2}+4x+3\mathrm{l}\mathrm{e}\dot{\mathrm{z}}\mathrm{y}$ nad prosta $y=px+1.$

2.5. Zbadač liczbe rozwiazań równania

$||x+5|-1|=m$

$\mathrm{w}$ zalezności od parametru $m.$

2.6. Rozwiazač uklad równań

$\left\{\begin{array}{l}
x^{2}+y^{2}=50\\
(x-2)(y+2)=-9.
\end{array}\right.$

Podač interpretacje geometryczna tego ukladu $\mathrm{i}$ sporzadzič odpowiedni

rysunek.

2.7. Wyznaczyč na osi odcietych punkty A $\mathrm{i} B,\ \mathrm{z}$ których okrag

$x^{2}+y^{2}-4x+2y= 20$ widač pod katem prostym, $\mathrm{t}\mathrm{z}\mathrm{n}$. styczne do

okregu wychodzace $\mathrm{z}\mathrm{k}\mathrm{a}\dot{\mathrm{z}}$ dego $\mathrm{z}$ tych punktów sa do siebie prostopadle.

Obliczyč pole figury ograniczonej stycznymi do okregu przechodzacymi

przez punkty A $\mathrm{i}B$. Rozwiazanie zilustrowač rysunkiem.

2.8. $\mathrm{W}$ przedziale $[0,2\pi]$ rozwiazač równanie

$1-\mathrm{t}\mathrm{g}^{2}x+\mathrm{t}\mathrm{g}^{4}x-\mathrm{t}\mathrm{g}^{6}x+\ldots=\sin^{2}3x.$
\end{document}
