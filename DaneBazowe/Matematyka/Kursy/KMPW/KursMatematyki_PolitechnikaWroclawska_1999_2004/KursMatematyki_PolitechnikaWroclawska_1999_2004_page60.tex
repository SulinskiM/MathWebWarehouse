\documentclass[a4paper,12pt]{article}
\usepackage{latexsym}
\usepackage{amsmath}
\usepackage{amssymb}
\usepackage{graphicx}
\usepackage{wrapfig}
\pagestyle{plain}
\usepackage{fancybox}
\usepackage{bm}

\begin{document}

76

12.7. Dla parametrów $p$ róznych od 2 $\mathrm{i}-1$ jedno rozwiazanie $x=3p,$

$y=-3p-2$. Dla $p=2$ nieskończenie wiele rozwiazań spelniajacych waru-

nek $2x+y-4=0$, gdzie $x$ dowolne rzeczywiste. Dla $p=-1$ nieskończenie

wiele rozwiazań spelniajacych warunek $x-y+4 = 0$, gdzie $x$ dowolne

rzeczywiste. Rozwiazania $0$ wspólrzednych calkowitych:

$\left\{\begin{array}{l}
x=-2\\
y=2
\end{array}\right.$

, $p=-1$;

$\left\{\begin{array}{l}
x=-2\\
y=0
\end{array}\right.$

, {\it p}$=$ - -32;

$\left\{\begin{array}{l}
x=-1\\
y=-1
\end{array}\right.$

, {\it p}$=$ - -31;

$\left\{\begin{array}{l}
x=0\\
y=-2
\end{array}\right.$

, $p=0$;

$\left\{\begin{array}{l}
x=1\\
y=2
\end{array}\right.$

, $p=2$;

$\left\{\begin{array}{l}
x=2\\
y=0
\end{array}\right.$

, $p=2.$

12.8. $S(y) = \displaystyle \pi(y+\frac{3}{2})\sqrt{1+(y-\frac{3}{2})^{2}},$

mniejsza $\displaystyle \frac{3\sqrt{13}}{4}\pi$ dla $y=0.$

$y \in$

[0, -23].

Wartośč naj-

13.2. 3.

13.3. $f(m)=|2^{-(m-3)}-2|,$

przedstawiono na rysunku 7.

$ m\in$ (-00, 4$]$ \{log27\}. Wykres funkcji $f$
\begin{center}
\includegraphics[width=60.252mm,height=108.048mm]{./KursMatematyki_PolitechnikaWroclawska_1999_2004_page60_images/image001.eps}
\end{center}
$y$

6

$m_{0}=\log_{2}7$

4

2

0 1  2  $m_{0}$  4{\it m}

Rys. 7
\end{document}
