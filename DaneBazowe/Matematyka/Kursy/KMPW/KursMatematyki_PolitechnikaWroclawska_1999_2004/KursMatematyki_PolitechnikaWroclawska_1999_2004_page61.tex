\documentclass[a4paper,12pt]{article}
\usepackage{latexsym}
\usepackage{amsmath}
\usepackage{amssymb}
\usepackage{graphicx}
\usepackage{wrapfig}
\pagestyle{plain}
\usepackage{fancybox}
\usepackage{bm}

\begin{document}

77

13.4. $\displaystyle \frac{35}{144}\approx 0$, 243.

13.5.

20, 28.

10, 8, 6, 4 1ub

10, 8, 6, 4, 2, $0, -2$ lub $-20, -12, -4$, 4, 12,

13.6. $\displaystyle \frac{10}{27}.$

13.7. $Q(2+\displaystyle \frac{2}{5}\sqrt{5},\frac{4}{5}\sqrt{5})$

na rysunku 8.

Zbiory A, B oraz

$A\cap B$ przedstawiono

Rys. 8
\begin{center}
\includegraphics[width=161.640mm,height=77.520mm]{./KursMatematyki_PolitechnikaWroclawska_1999_2004_page61_images/image001.eps}
\end{center}
13.8. Funkcja jest parzysta. $D = [-\sqrt{8},\sqrt{8}]$; miejsca zerowe $-2\mathrm{i}2$;

maksima lokalne $\displaystyle \frac{1}{2}$ dla $x = -\sqrt{7}$ oraz dla $x = \sqrt{7}$; minimum lokalne

$y$

1

0,5

$-3  -2  -1$  1 2  3  {\it x}

$-1$

Rys. 9
\end{document}
