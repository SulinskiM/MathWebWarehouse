\documentclass[a4paper,12pt]{article}
\usepackage{latexsym}
\usepackage{amsmath}
\usepackage{amssymb}
\usepackage{graphicx}
\usepackage{wrapfig}
\pagestyle{plain}
\usepackage{fancybox}
\usepackage{bm}

\begin{document}

78

$-4+\sqrt{8}$ dla $x=0$; funkcja rosnaca $\mathrm{w}(-\sqrt{8},-\sqrt{7})$ oraz $\mathrm{w}(0,\sqrt{7})$ ; malejaca

$\mathrm{w}(-\sqrt{7},0)$ oraz $\mathrm{w}(\sqrt{7},\sqrt{8})$ ; wypukla $\mathrm{w}(-2,2)$ ; wklesla $\mathrm{w}(-\sqrt{8},-2)$ oraz

$\mathrm{w}(2,\sqrt{8})$ ; punkty przegiecia $(-2,0), ($2, $0)$, proste $x=-\sqrt{8}$ oraz $x=\sqrt{8}$

styczne do wykresu funkcji. Wykres funkcji przedstawiono na rysunku 9.

14.1. 9.

14.2. $2\pi(3+2\sqrt{3}).$

14.3. a) $m=-\displaystyle \frac{1}{2}$; b$)m=\displaystyle \frac{4}{3}$; c$)m=01\mathrm{u}\mathrm{b}m=2\sqrt{3}.$

14.5. Elipsa $0$ równaniu $\displaystyle \frac{x^{2}}{36}+\frac{(y-1)^{2}}{4}=1$, środku $S(0,1)\mathrm{i}$ pólosiach

$\alpha=6, b=2$. Pole figury wynosi $8\pi-6\sqrt{3}.$

14.6. $+\infty.$

14.7. a) $\displaystyle \frac{1}{20};\mathrm{b}) \displaystyle \frac{7}{20}.$

14.8. $\displaystyle \frac{\sqrt{2}-\cos\alpha}{2\sin\alpha}\alpha,$

$\alpha\in (0,\displaystyle \frac{\pi}{2}).$

15.1. 12 $\mathrm{k}\mathrm{m}/\mathrm{h}, 15\mathrm{k}\mathrm{m}/\mathrm{h}, AB=27$ km.

15.2. (-00, $-\sqrt{3}]\cup(2,\infty).$

15.3. $108\sqrt{3}\mathrm{m}^{2}, \displaystyle \frac{405}{4}\sqrt{3}\mathrm{m}^{3}$

15.4. $w_{n}=1600+\displaystyle \frac{8000}{3}((\frac{203}{200})^{n-1}-1)$, pensja $\mathrm{w}$ kwietniu 2002 roku

wynosi 1806,09 $\mathrm{z}l$, średnia pensja $\mathrm{w}$ 2002 roku wynosi l827,96 $\mathrm{z}l.$

15.5. $f^{-1}(x) = \sqrt[3]{x},$

sunku 10.

$x \in$ R. Wykres funkcji $h$ przedstawiono na ry-

15.6. $\displaystyle \frac{\pi}{12}+k\frac{\pi}{3},$

$k \in$ Z.
\end{document}
