\documentclass[a4paper,12pt]{article}
\usepackage{latexsym}
\usepackage{amsmath}
\usepackage{amssymb}
\usepackage{graphicx}
\usepackage{wrapfig}
\pagestyle{plain}
\usepackage{fancybox}
\usepackage{bm}

\begin{document}

79
\begin{center}
\includegraphics[width=130.044mm,height=59.484mm]{./KursMatematyki_PolitechnikaWroclawska_1999_2004_page63_images/image001.eps}
\end{center}
{\it y}

2

1

$-2  -1$  1 2  {\it x}

Rys. 10

15.7. $\displaystyle \frac{9}{85}\sqrt{85}.$

15.8. $f(x) = \displaystyle \frac{x^{2}-x}{x-2} = x+ 1 + \displaystyle \frac{2}{x-2}$; $D = (-\infty,0]\cup(2,\infty)$ ;

asymptota pionowa prawostronna $x = 2$; asymptota ukośna obustronna

$y = x+1$; minimum lokalne $3+2\sqrt{2}$ dla $x_{0} = 2+\sqrt{2}$; funkcja rosnaca

$\mathrm{w}$ (-00, 0) oraz $\mathrm{w}(2+\sqrt{2},\infty)$ ; malejaca $\mathrm{w}(2,2+\sqrt{2})$ ; wypukla $\mathrm{w}(2,\infty)$ ;

wklesla $\mathrm{w}(-\infty,0)$. Wykres funkcji przedstawiono na rysunku ll.
\begin{center}
\includegraphics[width=93.012mm,height=100.992mm]{./KursMatematyki_PolitechnikaWroclawska_1999_2004_page63_images/image002.eps}
\end{center}
{\it y}

6

4

{\it S}

2

$-2$  2 $x_{0}4$ 6  {\it x}

$-2$

Rys. ll
\end{document}
