\documentclass[a4paper,12pt]{article}
\usepackage{latexsym}
\usepackage{amsmath}
\usepackage{amssymb}
\usepackage{graphicx}
\usepackage{wrapfig}
\pagestyle{plain}
\usepackage{fancybox}
\usepackage{bm}

\begin{document}

80

16.1. Cena mniejsza od poczatkowej 02,25\%.

16.2. Zbiór sklada sieZ luków czterech okregów oraz punktu (0,0) ijest

przedstawiony na rysunku 12.
\begin{center}
\includegraphics[width=91.740mm,height=77.220mm]{./KursMatematyki_PolitechnikaWroclawska_1999_2004_page64_images/image001.eps}
\end{center}
$K_{2}$  {\it y}  $K_{1}$

6

$S_{2}  S_{1}$

2

2 4  8  {\it x}

$S_{3}  S_{4}$

$K_{3}  K_{4}$

16.3. $18h^{2}\displaystyle \frac{\sin^{2}\alpha}{\sin 3\alpha},$

Rys. 12

$\alpha\in (0,\displaystyle \frac{\pi}{3}).$

16.4. 18 cm od wierzcholka kata rozwartego, $\alpha=38^{\circ}13'.$

16.5. $\sqrt{2},$

$\displaystyle \frac{\sqrt{2}}{2}.$

16.6. Dziedzina jest $\mathrm{R}$, a zbiorem wartości przedzial $[3-\sqrt{5},3+\sqrt{5}].$

16.7. $(-1,0]\displaystyle \cup\{\frac{\sqrt{17}-1}{2}\}.$

16.8. 2.

17.1. 1, $-1, \sqrt{\frac{\sqrt{17}-1}{8}}, -\sqrt{\frac{\sqrt{17}-1}{8}}$; wyraz czwarty 0 1ub $\displaystyle \frac{9-\sqrt{17}}{4}.$

17.2. $\displaystyle \frac{16}{35}\approx 0$, 457.
\end{document}
