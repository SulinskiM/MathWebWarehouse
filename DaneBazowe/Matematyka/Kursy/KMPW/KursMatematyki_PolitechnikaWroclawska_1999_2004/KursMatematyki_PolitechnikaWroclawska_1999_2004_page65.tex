\documentclass[a4paper,12pt]{article}
\usepackage{latexsym}
\usepackage{amsmath}
\usepackage{amssymb}
\usepackage{graphicx}
\usepackage{wrapfig}
\pagestyle{plain}
\usepackage{fancybox}
\usepackage{bm}

\begin{document}

81

17.4. $x^{2}+(y-r^{2}-\displaystyle \frac{1}{4})^{2}=r^{2}$ Rozwiazanie istnieje dla $r>\displaystyle \frac{1}{2}.$

17.6. $[\displaystyle \frac{1}{3},\frac{\sqrt{6}}{6}).$

17.7. $\displaystyle \frac{3d(c^{2}+d^{2})}{2c^{2}}\sqrt{c^{2}-d^{2}}$

lub $\displaystyle \frac{3d(2c^{2}-d^{2})}{2c^{2}}\sqrt{c^{2}-d^{2}},$

$c>d.$

17.8. Gdy $\mathrm{w}$ równoleglościanie sa dwa wierzcholki trójścienne $0$ trzech

katach plaskich $\beta$, to objetośč wynosi $2\alpha^{3}\sqrt{\sin\frac{3}{2}\beta\sin^{3}\frac{1}{2}\beta}$. Gdy $\beta\in (\displaystyle \frac{\pi}{3},\frac{\pi}{2})$

$\mathrm{i}\mathrm{w}$ równoleglościanie sa dwa wierzcholki trójścienne $0$ trzech katach plaskich

$\pi-\beta$, to objetośč wynosi $2\alpha^{3}\sqrt{-\cos\frac{3}{2}\beta\cos^{3}\frac{1}{2}\beta},$

18.1. $\displaystyle \frac{3}{2}.$

18.2. $3x-2y+1=0.$

18.3. $V=-\displaystyle \frac{\pi}{6}l^{3}\sin 4\alpha\cos 2\alpha, \varphi=3\pi-4\alpha,$

$\alpha\in (\displaystyle \frac{\pi}{2},\frac{3\pi}{4}).$

18.4. Niech $x$ oznacza cene dlugopisu, a $y$ cene zeszytu. Dla $k\neq 2$ jest

$x=\displaystyle \frac{5k+2}{2k+2}, y=\displaystyle \frac{k}{k+2}$. Dla $k=2$ spelnionajest relacja $2x+4y=5$. Ceny

dlugopisu $\mathrm{i}$ zeszytu moga byč nastepujace:

$\left\{\begin{array}{l}
x=2,3\\
y=0,9;
\end{array}\right.$

$\left\{\begin{array}{l}
x=2,1\\
y=0,8;
\end{array}\right.$

$\left\{\begin{array}{l}
x=1,7\\
y=0,6;
\end{array}\right.$

$\left\{\begin{array}{l}
x=1,5\\
y=0,5;
\end{array}\right.$

$\left\{\begin{array}{l}
x=1,3\\
y=0,6;
\end{array}\right.$

$\left\{\begin{array}{l}
x=1,1\\
y=0,7;
\end{array}\right.$

$\left\{\begin{array}{l}
x=0,9\\
y=0,8.
\end{array}\right.$

18.5.

$[-\displaystyle \frac{\pi}{4}+k\pi,k\pi]\cup[\frac{\pi}{4}+k\pi,\frac{\pi}{2}+k\pi),$

$ k\in$ Z.

18.6. $\displaystyle \frac{496}{729}\approx 0$, 680; $0 \displaystyle \frac{496}{728\cdot 729}\approx 0$, 001.

18.7. $2-\displaystyle \frac{4}{3}\sqrt{2}.$
\end{document}
