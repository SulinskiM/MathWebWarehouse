\documentclass[a4paper,12pt]{article}
\usepackage{latexsym}
\usepackage{amsmath}
\usepackage{amssymb}
\usepackage{graphicx}
\usepackage{wrapfig}
\pagestyle{plain}
\usepackage{fancybox}
\usepackage{bm}

\begin{document}

82

18.8. $6\sqrt{2}-4.$

19.1. $\displaystyle \frac{31}{8}.$

19.2. $12+24\sqrt{2}$ cm.

19.3. $s\leq 20.$

19.4. $\displaystyle \frac{3}{2}\sqrt{55}$ arów. Plan dzialki $\mathrm{w}$ skali 1:1000 przedstawia rysunek 13.
\begin{center}
\includegraphics[width=72.444mm,height=42.828mm]{./KursMatematyki_PolitechnikaWroclawska_1999_2004_page66_images/image001.eps}
\end{center}
20

40  30

60

Rys. 13

19.5. Wartośč najwieksza 6 dla $m=0.$

19.7.

$\left\{\begin{array}{l}
x1=-- 51\pi 2\\
y_{1}=\frac{\pi}{12},
\end{array}\right.$

$\left\{\begin{array}{l}
x_{2}=\frac{\pi}{12}\\
y2=-- 51\pi 2,
\end{array}\right.$

$\left\{\begin{array}{l}
x3=--- 71\pi 2\\
y_{3}=-\frac{11\pi}{12},
\end{array}\right.$

$\left\{\begin{array}{l}
x_{4}=-\frac{11\pi}{12}\\
y4=--- 71\pi 2^{\cdot}
\end{array}\right.$

19.8. 1, 1, $\displaystyle \frac{\sqrt{3}}{2}, \displaystyle \frac{2\sqrt{7}}{7}, \displaystyle \frac{\sqrt{42}}{7}, \displaystyle \frac{\sqrt{42}}{7}.$

20.1. $-1$, 1, 2.

20.2. $\displaystyle \frac{8}{5}(2-\sqrt{3}).$

20.3. $\displaystyle \frac{50}{81}\approx 0$, 617.
\end{document}
