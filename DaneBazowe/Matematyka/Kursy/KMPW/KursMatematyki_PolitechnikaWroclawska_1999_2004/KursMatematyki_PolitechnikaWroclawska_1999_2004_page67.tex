\documentclass[a4paper,12pt]{article}
\usepackage{latexsym}
\usepackage{amsmath}
\usepackage{amssymb}
\usepackage{graphicx}
\usepackage{wrapfig}
\pagestyle{plain}
\usepackage{fancybox}
\usepackage{bm}

\begin{document}

83

20.4. Cześč elipsy $0$ równaniu $\displaystyle \frac{x^{2}}{\frac{5}{3}}+\frac{(y-5)^{2}}{\frac{5}{2}}=1$ dla $y\leq 6.$

20.5. Asymptota pionowa obustronna $x=1$; asymptota pozioma lewo-

stronna $y=-1$; asymptota pozioma prawostronna $y=1$. Wykres funkcji

przedstawiono na rysunku 14.
\begin{center}
\includegraphics[width=120.348mm,height=78.840mm]{./KursMatematyki_PolitechnikaWroclawska_1999_2004_page67_images/image001.eps}
\end{center}
$y$

4

2

$-2$  0 2  4  {\it x}

Rys. 14

20.6. $(-\displaystyle \infty,-1]\cup[-\frac{1}{2},0)\cup(0,1].$

20.7. $-\sqrt{8}$ lub $\sqrt{8}.$

21.1. $\mathrm{O} 5\mathrm{c}\mathrm{m}^{2}$

21.2. $\displaystyle \frac{3}{4}.$

21.3. $\displaystyle \frac{4-\sqrt{2}}{6}, \displaystyle \frac{4+\sqrt{2}}{6}.$

21.4. Granica ciagu wynosi $\displaystyle \frac{1}{2}.$

21.5. $(-\pi+4k\pi,\pi+4k\pi),  k\in$ Z.

21.6. $-1.$
\end{document}
