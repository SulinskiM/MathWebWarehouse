\documentclass[a4paper,12pt]{article}
\usepackage{latexsym}
\usepackage{amsmath}
\usepackage{amssymb}
\usepackage{graphicx}
\usepackage{wrapfig}
\pagestyle{plain}
\usepackage{fancybox}
\usepackage{bm}

\begin{document}

84

21.7. $r=\displaystyle \frac{2\sqrt{2}}{3}R, h=\displaystyle \frac{4}{3}R.$

21.8. $y=1-(4+2\sqrt{5})(x-2),$

$y=1-(4-2\sqrt{5})(x-2).$
\begin{center}
\includegraphics[width=116.892mm,height=81.276mm]{./KursMatematyki_PolitechnikaWroclawska_1999_2004_page68_images/image001.eps}
\end{center}
{\it y}

5

$y=p_{1}$

3

$y=p_{2}$

1

$-4  -x_{0}  -1$  1 2  $x_{0}$  4 {\it x}

Rys. 15

22.1. Wykres funkcji przedstawiono na rysunku l5, gdzie $x_{0}=1+\sqrt{5}.$

Niech $f(p)$ oznacza liczbe rozwiazań równania $4+2|x|-x^{2}=p$. Wtedy

$f(p)=$

dla

dla

dla

dla

$p>5,$

$p<4$ lub $p=5,$

$p=4,$

$4<p<5.$

22.2. 117 minut; 5475,6 $\mathrm{m}^{3}$

22.3. Średnice podstaw $6+2\sqrt{5}$ cm oraz $6-2\sqrt{5}$ cm; tworzaca 6 cm.

22.4. Gdy $\mathrm{k}\mathrm{a}\mathrm{t}\alpha$ jest ostry $\mathrm{i}\sin\alpha < \displaystyle \frac{4}{5}$, wówczas sa dwa rozwiazania:

$ P_{1}=\displaystyle \frac{8}{25}R^{2}(4\cos\alpha-3\sin\alpha)\sin\alpha$ oraz $P_{2}=\displaystyle \frac{8}{25}R^{2}(4\cos\alpha+3\sin\alpha)\sin\alpha.$
\end{document}
