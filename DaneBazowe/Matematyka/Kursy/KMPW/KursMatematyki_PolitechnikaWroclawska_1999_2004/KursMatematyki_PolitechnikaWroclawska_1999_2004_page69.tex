\documentclass[a4paper,12pt]{article}
\usepackage{latexsym}
\usepackage{amsmath}
\usepackage{amssymb}
\usepackage{graphicx}
\usepackage{wrapfig}
\pagestyle{plain}
\usepackage{fancybox}
\usepackage{bm}

\begin{document}

85

Jeśli $\displaystyle \sin\alpha\geq\frac{4}{5}$, to jest jedno rozwiazanie $P_{2}$, a jeśli $\alpha$ rozwarty $\displaystyle \mathrm{i}\sin\alpha<\frac{4}{5},$

to brak rozwiazań.

22.5. $(0,\displaystyle \frac{1}{4}]\cup[16,\infty).$

22.6. $\displaystyle \cos\alpha=\frac{\sqrt{7}}{14}$, obwód $\displaystyle \frac{1}{6}(9+\sqrt{12}+\sqrt{21})\alpha.$

22.7. $\displaystyle \frac{\pi}{4}+k\frac{2\pi}{3},  k\in$ Z.

22.8. $\sqrt{2}x+2y-3=0.$

23.1. Tak. $\mathrm{W}$ obu przypadkach liczba,,slów'' wynosi 210.

23.2. $-3, -1$, 1.

23.3. $\displaystyle \frac{3}{8}\alpha.$

23.4. $\displaystyle \frac{1}{12}b^{2}(3\alpha-b)\mathrm{t}\mathrm{g}\alpha.$

23.5. $[-\sqrt{5},0)\cup(1$, 2$).$

23.7. Punkt $Q(1,1).$

23.8. $(\displaystyle \frac{5\pi}{4}+2k\pi,\frac{3\pi}{2}+2k\pi)\cup(\frac{3\pi}{2}+2k\pi,\frac{7\pi}{4}+2k\pi),$

24.1. $2+\displaystyle \frac{3}{2}\sqrt{2}.$

$ k\in$ Z.

24.2. $\displaystyle \frac{7}{18}\approx 0$, 389.

24.3. Dla $ m\neq 10$ jedno rozwiazanie $x= \displaystyle \frac{m}{m-10}, y= \displaystyle \frac{m-15}{m-10}$. Dla

$m= 10$ uklad sprzeczny. Rozwiazania tworza prosta $x+2y-3=0$ bez

punktu $P(1,1).$

24.4. $\sqrt{\frac{6-6\cos\alpha}{5-4\cos\alpha}},$

$\alpha\in (0,\displaystyle \frac{\pi}{3}).$
\end{document}
