\documentclass[a4paper,12pt]{article}
\usepackage{latexsym}
\usepackage{amsmath}
\usepackage{amssymb}
\usepackage{graphicx}
\usepackage{wrapfig}
\pagestyle{plain}
\usepackage{fancybox}
\usepackage{bm}

\begin{document}

11

Praca kontrolna

nr 3

3.1. Bez stosowania metod rachunku rózniczkowego wyznaczyč dziedzine

i zbiór wartości funkcji

$f(x)=\sqrt{2+\sqrt{x}-x}.$

3.2. Jednym $\mathrm{z}$ wierzcholków rombu $0$ polu 20 $\mathrm{c}\mathrm{m}^{2}$ jest punkt $A(6,3)$,

a jedna $\mathrm{z}$ przekatnych zawiera $\mathrm{s}\mathrm{i}\mathrm{e}\mathrm{w}$ prostej $0$ równaniu $2x+y=5.$

Wyznaczyč równania prostych, $\mathrm{w}$ których zawieraja $\mathrm{s}\mathrm{i}\mathrm{e}$ boki AB $\mathrm{i}AD.$

3.3. Stosujac zasade indukcji matematycznej, wykazač prawdziwośč wzoru

3 $(1^{5}+2^{5}+\displaystyle \ldots+n^{5})+(1^{3}+2^{3}+\ldots+n^{3})=\frac{n^{3}(n+1)^{3}}{2},$

$n\geq 1.$

3.4. Ostroslup prawidlowy trójkatny ma pole powierzchni calkowitej

$P=12\sqrt{3}\mathrm{c}\mathrm{m}^{2}$, a $\mathrm{k}\mathrm{a}\mathrm{t}$ nachylenia ściany bocznej do plaszczyzny pod-

stawy $\alpha=60^{\circ}$ Obliczyč objetośč tego ostroslupa.

3.5. Wśród trójkatów równoramiennych wpisanych $\mathrm{w}$ kolo $0$ promieniu $R$

znalez$\acute{}$č ten, który ma najwieksze pole.

3.6. Zbadač przebieg zmienności $\mathrm{i}$ narysowač wykres funkcji

$f(x)=\displaystyle \frac{1}{2}x^{2}\sqrt{5-2x}.$

3.7. $\mathrm{W}$ trapezie równoramiennym dane sa ramie $r, \mathrm{k}\mathrm{a}\mathrm{t}$ ostry przy pod-

stawie $\alpha$ oraz suma $d$ dlugości przekatnej $\mathrm{i}$ dluzszej podstawy. Wyz-

naczyč pole trapezu oraz promień okregu opisanego na tym trapezie.

Podač warunki istnienia rozwiazania. Nastepnie przeprowadzič obli-

czenia dla $\alpha=30^{\circ}, r=\sqrt{3}$ cm $\mathrm{i} d=6$ cm.

3.8. Rozwiazač nierównośč

$|\cos x+\sqrt{3}\sin x|\leq\sqrt{2},x\in[0,3\pi].$
\end{document}
