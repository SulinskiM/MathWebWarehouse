\documentclass[a4paper,12pt]{article}
\usepackage{latexsym}
\usepackage{amsmath}
\usepackage{amssymb}
\usepackage{graphicx}
\usepackage{wrapfig}
\pagestyle{plain}
\usepackage{fancybox}
\usepackage{bm}

\begin{document}

86

24.5. (-00, $\displaystyle \frac{1}{2}]\cup[\frac{3}{2},\infty).$

24.7. Równanie prostej $k$: $x+2y-8 = 0$. Równania stycznych

tworzacych $\mathrm{z} k \mathrm{k}\mathrm{a}\mathrm{t} 45^{\circ}$: $x-3y+2+5\sqrt{2}= 0, x-3y+2-5\sqrt{2}= 0,$

$3x+y-4+5\sqrt{2}=0, 3x+y-4-5\sqrt{2}=0.$

24.8. $\alpha=3, b=32$. Styczna $y= -3x+13.$

stawiono na rysunku 16.

Wykres funkcji przed-
\begin{center}
\includegraphics[width=109.224mm,height=96.768mm]{./KursMatematyki_PolitechnikaWroclawska_1999_2004_page70_images/image001.eps}
\end{center}
{\it y}

6

4

2

0 1  3 5  7  {\it x}

Rys. 16

25.1. -$\displaystyle \frac{1}{6}, 0, \displaystyle \frac{1}{2}.$

25.2. $S(x)=x(\alpha-x), x\in(0,\alpha)$. Wartośč najwieksza $\displaystyle \frac{\alpha^{2}}{4}$ dla $x=\displaystyle \frac{\alpha}{2}.$

25.3. Rysunek l7.

25.4. Elipsa $0$ równaniu $\displaystyle \frac{(x-4)^{2}}{25}+\frac{y^{2}}{9}=1$ oraz cześč prostej $y=0$ dla

$x>9.$
\end{document}
