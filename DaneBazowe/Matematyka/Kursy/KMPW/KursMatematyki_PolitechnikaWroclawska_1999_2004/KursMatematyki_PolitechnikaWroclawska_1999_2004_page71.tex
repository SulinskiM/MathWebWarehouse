\documentclass[a4paper,12pt]{article}
\usepackage{latexsym}
\usepackage{amsmath}
\usepackage{amssymb}
\usepackage{graphicx}
\usepackage{wrapfig}
\pagestyle{plain}
\usepackage{fancybox}
\usepackage{bm}

\begin{document}

87

Rys. 17

25.5. $\displaystyle \frac{5}{12}\approx 0$, 417.

25.7. $D = [1$, 5$)$ ; asymptota pionowa lewostronna $x = 5$; funkcja

rosnaca $\mathrm{w}(1,5)$ ; wypukla $\mathrm{w}(2,5)$ ; wklesla $\mathrm{w}(1,2)$ ; punkt przegiecia $P(2,1)$ ;

prosta $x=1$ styczna do wykresu funkcji. Wykres funkcji przedstawiono na

rysunku 18.
\begin{center}
\includegraphics[width=72.036mm,height=72.084mm]{./KursMatematyki_PolitechnikaWroclawska_1999_2004_page71_images/image001.eps}
\end{center}
{\it y}

3

1  {\it P}

0 1  2  {\it 5 x}

Rys. 18
\end{document}
