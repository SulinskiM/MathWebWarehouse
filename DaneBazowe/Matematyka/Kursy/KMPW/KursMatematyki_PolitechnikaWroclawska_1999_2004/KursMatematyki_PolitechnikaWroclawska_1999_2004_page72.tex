\documentclass[a4paper,12pt]{article}
\usepackage{latexsym}
\usepackage{amsmath}
\usepackage{amssymb}
\usepackage{graphicx}
\usepackage{wrapfig}
\pagestyle{plain}
\usepackage{fancybox}
\usepackage{bm}

\begin{document}

88

25.8. $2\alpha\cos\alpha(1+2\cos\alpha), \alpha\in (0,\displaystyle \frac{\pi}{3}).$

26.1. 30 $(\pi+\sqrt{3})$ cm.

26.2. 213 $\mathrm{z}l\mathrm{i}85$ gr.

26.3. $4x-7y+17=0$; pole $\displaystyle \frac{10}{3}.$

26.4. $\displaystyle \frac{\sqrt{2}}{3}r^{3}\frac{(1+\sin\alpha)\cos\frac{\beta}{2}\mathrm{c}\mathrm{t}\mathrm{g}\frac{\alpha}{2}}{\cos\alpha\sqrt{-\cos\beta}},$

$\beta\in (\displaystyle \frac{\pi}{2},\pi).$

26.5. Maksimum lokalne 2 dla $x=0$. Wykres funkcji przedstawiono na

rysunku 19.
\begin{center}
\includegraphics[width=156.612mm,height=66.240mm]{./KursMatematyki_PolitechnikaWroclawska_1999_2004_page72_images/image001.eps}
\end{center}
{\it y}

3

1

$-6  -3  -1$  1 3 4  6 {\it x}

Rys. 19

26.6. $\displaystyle \frac{\pi}{3}+k\frac{\pi}{2},$

$ k\in$ Z.

26.7. Dla $\alpha\in[0$, 4). Wtedy

$g(x)=$

dla

dla

$\alpha=0,$

$0<\alpha<4.$

Dla $\alpha=3$ asymptota pionowa obustronna $x= \displaystyle \frac{1}{3}$, asymptota ukośna obu-

stronna $3x-27y+4=0.$
\end{document}
