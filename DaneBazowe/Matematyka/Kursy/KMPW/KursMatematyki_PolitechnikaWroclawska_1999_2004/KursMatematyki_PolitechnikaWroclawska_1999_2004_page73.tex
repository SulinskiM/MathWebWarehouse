\documentclass[a4paper,12pt]{article}
\usepackage{latexsym}
\usepackage{amsmath}
\usepackage{amssymb}
\usepackage{graphicx}
\usepackage{wrapfig}
\pagestyle{plain}
\usepackage{fancybox}
\usepackage{bm}

\begin{document}

89

26.8. $S(y)=\pi(y+3)\sqrt{4+(y-3)^{2}},$

dla $y=0$ wynoszaca $3\pi\sqrt{13}.$

$ y\in [0$, 3$]$. Wartośč najmniejsza

27.1. $p\in[-2,2].$

27.2. $(x-\displaystyle \frac{8}{5})^{2}+(y-\frac{9}{5})^{2}=\frac{16}{5}.$

27.3. $\displaystyle \frac{\sqrt{16r^{2}\sin^{2}\alpha-d^{2}}}{2\sin\alpha},$

$4r\sin\alpha\cos\alpha<d<4r\sin\alpha.$
\begin{center}
\includegraphics[width=36.216mm,height=15.192mm]{./KursMatematyki_PolitechnikaWroclawska_1999_2004_page73_images/image001.eps}
\end{center}
$(17+\mathrm{c}\mathrm{t}\mathrm{g}^{2}\beta)^{3}$

27.4. $2S^{3}24 18(\mathrm{c}\mathrm{t}\mathrm{g}^{2}\beta-1)^{2}$

27.5. $-\infty.$

27.6. $(2k\displaystyle \pi,\frac{\pi}{6}+2k\pi],$

27.7. $\displaystyle \frac{425}{768}\approx 0$, 553.

$ k\in$ Z.

27.8. Tangens kata przeciecia linii wynosi $\displaystyle \frac{9}{37}$. Szukany zbiór pokazano

na rysunku 20.
\begin{center}
\includegraphics[width=151.380mm,height=57.300mm]{./KursMatematyki_PolitechnikaWroclawska_1999_2004_page73_images/image002.eps}
\end{center}
{\it y}

2

1

$-8  -1$  2  8  {\it x}

$-1$

$-2$

Rys. 20

28.1. $\displaystyle \frac{13}{9}.$
\end{document}
