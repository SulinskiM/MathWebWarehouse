\documentclass[a4paper,12pt]{article}
\usepackage{latexsym}
\usepackage{amsmath}
\usepackage{amssymb}
\usepackage{graphicx}
\usepackage{wrapfig}
\pagestyle{plain}
\usepackage{fancybox}
\usepackage{bm}

\begin{document}

90

28.2. $ p\in [\displaystyle \frac{5}{4},\frac{\sqrt{7}}{2}).$

28.3. $\displaystyle \frac{d^{2}-r^{2}}{2}\mathrm{t}\mathrm{g}\frac{\alpha}{2}, r<d.$

28.4. $\displaystyle \frac{2R}{R+r}\sqrt{3Rr}.$

28.5. Trzy pierwiastki, $\mathrm{w}$ tym jeden ujemny $\mathrm{i}$ dwa dodatnie.

28.7. $\displaystyle \frac{2\pi}{3}+2k\pi$ lub $\displaystyle \frac{4\pi}{3}+2k\pi,  k\in$ Z.

28.8. Szukana krzywa jest parabola $0$ równaniu $y = 2x^{2}+ \displaystyle \frac{1}{2}$ bez

punktu $W(0,\displaystyle \frac{1}{2}).$

29.1. 15.

29.2. 307692.

29.3. $c(\cos\alpha-\cos 2\alpha), \alpha\in (0,\displaystyle \frac{\pi}{4}).$

29.4. (-00, $-\sqrt{2}]\cup(-1,0)\cup(0,2)\cup[1+\sqrt{3},\infty).$

29.5. Rysunek 2l.
\begin{center}
\includegraphics[width=91.032mm,height=66.648mm]{./KursMatematyki_PolitechnikaWroclawska_1999_2004_page74_images/image001.eps}
\end{center}
Rys. 21
\end{document}
