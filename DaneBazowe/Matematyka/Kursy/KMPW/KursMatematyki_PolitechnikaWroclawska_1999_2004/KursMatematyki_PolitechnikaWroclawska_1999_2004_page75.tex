\documentclass[a4paper,12pt]{article}
\usepackage{latexsym}
\usepackage{amsmath}
\usepackage{amssymb}
\usepackage{graphicx}
\usepackage{wrapfig}
\pagestyle{plain}
\usepackage{fancybox}
\usepackage{bm}

\begin{document}

91

29.6. $\displaystyle \frac{\pi}{6}+k\frac{\pi}{3}$

29.7. $-\displaystyle \frac{1}{4}.$

lub

$k\pi,$

$ k\in$ Z.

29.8. Wartośč najmniejsza 0 dla $x =$

$\displaystyle \frac{1+\sqrt{2}}{2}$ dla $x=-1+\sqrt{2}.$

30.1. $\displaystyle \frac{\sqrt{2}}{2}\frac{V_{1}^{2}}{V_{1}+V_{2}}$, gdzie $V_{1}\geq V_{2}.$

$-1$, a wartośč najwieksza

30.2. Tak, na dwa sposoby: 3800, 6l00, 8400, l0700 i l3000 zl

1000, 3400, 5800, 8200, 10600 i l3000zl.

lub

30.3. $\mathrm{s}_{\mathrm{a}}$ cztery takie okregi $\mathrm{i}$ maja równania:

$(x-\displaystyle \frac{3}{2})^{2} + (y-1-\sqrt{6})^{2} = \displaystyle \frac{9}{4}, (x-\displaystyle \frac{3}{2})^{2} + (y-1+\sqrt{6})^{2}$

-49,

$(x+1)^{2}+(y-1)^{2}=1, (x-3)^{2}+(y-1)^{2}=9.$

30.4. $\displaystyle \frac{2k}{k^{2}-1}\sin\beta, k>1.$

30.5. 7, 13.

30.6. $\alpha=-3, b=1.$

30.7. $(-\infty,0]\cup [1,\displaystyle \log_{2}\frac{3+\sqrt{17}}{2}].$

30.8. $(-\displaystyle \pi,-\frac{2\pi}{6}), (-\displaystyle \frac{\pi}{6},\frac{4\pi}{6}), (\displaystyle \frac{5\pi}{6},\pi).$

31.1. $\displaystyle \frac{136}{4807}\approx 0$, 028.

31.2. Objestośč ostroslupa wynosi $\displaystyle \frac{343}{3} \mathrm{c}\mathrm{m}^{3}$, a objetośč najmniejszej

cześci $\displaystyle \frac{61}{3}\mathrm{c}\mathrm{m}^{3}$

31.3. Uklad ma trzy rozwiazania:

$\left\{\begin{array}{l}
x_{1}=3+\sqrt{3}\\
y_{1}=3-\sqrt{3},
\end{array}\right.$

$\left\{\begin{array}{l}
x_{1}=3-\sqrt{3}\\
y_{1}=3+\sqrt{3},
\end{array}\right.$

$\left\{\begin{array}{l}
x_{1}=2+2\sqrt{2}\\
y_{1}=2-2\sqrt{2}.
\end{array}\right.$
\end{document}
