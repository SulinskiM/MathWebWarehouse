\documentclass[a4paper,12pt]{article}
\usepackage{latexsym}
\usepackage{amsmath}
\usepackage{amssymb}
\usepackage{graphicx}
\usepackage{wrapfig}
\pagestyle{plain}
\usepackage{fancybox}
\usepackage{bm}

\begin{document}

92

31.4. - $\displaystyle \frac{1}{2}d^{2}\sin 2\alpha \mathrm{t}\mathrm{g}^{2}\frac{\alpha}{2}, \alpha\in (\displaystyle \frac{\pi}{2},\pi).$

31.6. $-\sqrt[3]{4}.$

31.7. $B_{1}(5,3), C_{1}(3,2), D_{1}(4,0)$ lub $B_{2}(10,-2), C_{2}(13,2), D_{2}(9,5).$

31.8. $D= (0,\infty)$ ; asymptota pionowa prawostronna $x=0$; minimum

lokalne 2 $\mathrm{d}\mathrm{l}\mathrm{a}x=1$; funkcja rosnaca $\mathrm{w}(1,\infty)$ ; malejaca $\mathrm{w} (0,1)$, wypukla

$\mathrm{w}(0,3)$ ; wklesla $\mathrm{w}(3,\infty)$ ; punkt przegiecia $P(3,\displaystyle \frac{4}{3}\sqrt{3})$ ; krzywa asympto-

tyczna $(\mathrm{w}+\infty) y=\sqrt{x}$. Wykres funkcji przedstawiono na rysunku 22.
\begin{center}
\includegraphics[width=116.436mm,height=54.864mm]{./KursMatematyki_PolitechnikaWroclawska_1999_2004_page76_images/image001.eps}
\end{center}
{\it y}

3

{\it P}

2

0 1  3  8  {\it x}

Rys. 22

32.1. 15 $\mathrm{d}\mathrm{n}\mathrm{i}.$

32.2. 8, $\displaystyle \frac{1}{8}.$

32.3. 65,71itra.

32.4. $ m\in (0,\displaystyle \frac{\sqrt{5}-1}{2}).$

32.5. $\displaystyle \frac{2757}{3125}\approx 0$, 882.
\end{document}
