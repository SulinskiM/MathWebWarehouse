\documentclass[a4paper,12pt]{article}
\usepackage{latexsym}
\usepackage{amsmath}
\usepackage{amssymb}
\usepackage{graphicx}
\usepackage{wrapfig}
\pagestyle{plain}
\usepackage{fancybox}
\usepackage{bm}

\begin{document}

93

32.6. $\displaystyle \frac{\pi}{4}+k\pi$ lub $\displaystyle \frac{\pi}{12}+k\pi$

lub $\displaystyle \frac{5\pi}{12}+k\pi,  k\in$ Z.

32.7. $y=-1$ (dwa punkty wspólne), $32x+27y-5=0$ (trzy punkty

wspólne).

32.8. $R = \displaystyle \frac{1}{3}b \sqrt{\frac{9+3\cos^{2}\alpha}{2+2\cos\alpha}}$. Cosinusy katów nachylenia ścian

bocznych wynosza $\displaystyle \frac{1}{2}$

oraz $\sqrt{\frac{1-\cos\alpha}{7-\cos\alpha}}.$

33.1. Mniejszy 023,56\%.

33.2. Szukana linie stanowia dwie proste $0$ równaniach $2x+3y-1=0$

oraz $4x-y+5=0$ bez punktu ich przeciecia $P(-1,1).$

33.3. $\displaystyle \frac{7\pi}{4}.$

33.4. 2 $(7+\sqrt{19}).$
\begin{center}
\includegraphics[width=120.192mm,height=90.420mm]{./KursMatematyki_PolitechnikaWroclawska_1999_2004_page77_images/image001.eps}
\end{center}
{\it y}

4

2

$-4  -1$  0 2  4  {\it x}

Rys. 23
\end{document}
