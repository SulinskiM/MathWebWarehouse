\documentclass[a4paper,12pt]{article}
\usepackage{latexsym}
\usepackage{amsmath}
\usepackage{amssymb}
\usepackage{graphicx}
\usepackage{wrapfig}
\pagestyle{plain}
\usepackage{fancybox}
\usepackage{bm}

\begin{document}

94

33.6. Asymptota pionowa obustronna $x=1$; asymptota pozioma lewo-

stronna $y= -\displaystyle \frac{1}{2}$; asymptota ukośna prawostronna $y= \displaystyle \frac{1}{2}x-1$; minimum

lokalne 0 d1a $x=2$. Wykres funkcji przedstawiono na rysunku 23.

33.7. $[-\displaystyle \frac{5\pi}{12},\frac{\pi}{2})\cup(\frac{\pi}{2},\frac{11\pi}{12}]\cup\{-\pi,\pi\}.$

33.8. Cosinus kata rozwarcia wynosi $\displaystyle \frac{11}{13}.$

34.1. $3+\sqrt{5}.$

34.2. $-4.$

34.3. - $\displaystyle \frac{1}{2}x^{2}+x+2$ lub - $\displaystyle \frac{1}{18}x^{2}+\frac{1}{9}x-\frac{14}{9}.$

34.4. $\vec{AB}=[8$, 4$], \vec{CD}=[-2,-1].$

34.5. $\displaystyle \frac{6}{10}.$

34.6. Odleglośč $P$ od brzegu $\mathcal{F}$ wynosi $\displaystyle \frac{\sqrt{26}}{2}-2$. Zbiór $\mathcal{F}$ przedstawiono

na rysunku 24.
\begin{center}
\includegraphics[width=66.036mm,height=60.096mm]{./KursMatematyki_PolitechnikaWroclawska_1999_2004_page78_images/image001.eps}
\end{center}
{\it y}

4

2

0 2  4  {\it x}

Rys. 24

34.7. $f^{-1}(y)=-\displaystyle \frac{1}{1+2^{y}}, D_{f^{-1}}=\mathrm{R}, W_{f^{-1}}=(-1,0).$
\end{document}
