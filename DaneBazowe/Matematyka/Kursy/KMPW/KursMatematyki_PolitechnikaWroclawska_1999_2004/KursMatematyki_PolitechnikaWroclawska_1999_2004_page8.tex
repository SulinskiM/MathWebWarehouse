\documentclass[a4paper,12pt]{article}
\usepackage{latexsym}
\usepackage{amsmath}
\usepackage{amssymb}
\usepackage{graphicx}
\usepackage{wrapfig}
\pagestyle{plain}
\usepackage{fancybox}
\usepackage{bm}

\begin{document}

12

Praca kontrolna nr 4

4.1. Rozwiazač równanie $16+19+22+\ldots+x=2000$, którego lewa strona

jest suma pewnej liczby kolejnych wyrazów ciagu arytmetycznego.

4.2. Ze zbioru $\{0$, 1, $\ldots$, 9$\}$ losujemy bez zwracania pieč cyfr. Obliczyč

prawdopodobieństwo tego, $\dot{\mathrm{z}}\mathrm{e}\mathrm{m}\mathrm{o}\dot{\mathrm{z}}$ na $\mathrm{z}$ nich utworzyč liczbe podzielna

przez 5.

4.3. Zbadač, czy istnieje pochodna funkcji $f(x) = \sqrt{1-\cos x}\mathrm{w}$ punkcie

$x=0$. Wynik zilustrowač na wykresie funkcji $f(x).$

4.4. Udowodnič, $\dot{\mathrm{z}}\mathrm{e}$ dwusieczne katów wewnetrznych równolegloboku two-

$\mathrm{r}\mathrm{z}\mathrm{a}$ prostokat, którego przekatna ma dlugośč równa róznicy dlugości

sasiednich boków równolegloboku.

4.5. Rozwiazač uklad nierówności

$\left\{\begin{array}{l}
x+y\leq 3\\
\log_{y}(2^{x+1}+32)\leq 2\log_{y}(8-2^{x})
\end{array}\right.$

$\mathrm{i}$ zaznaczyč zbiór jego rozwiazań na plaszczy $\acute{\mathrm{z}}\mathrm{n}\mathrm{i}\mathrm{e}.$

4.6. Znalez$\acute{}$č równanie zbioru wszystkich punktów plaszczyzny $Oxy$, które

sa środkami okregów stycznych wewnetrznie do okregu $x^{2}+y^{2}=121$

$\mathrm{i}$ równocześnie stycznych zewnetrznie do okregu $(x+8)^{2}+y^{2}=1$. Jaka

linie przedstawia znalezione równanie? Sporzadzič staranny rysunek.

4.7. Zbadač iloczyn pierwiastków rzeczywistych równania

$m^{2}x^{2}+8mx+4m-4=0$

jako funkcje parametru $m$. Sporzadzič wykres tej funkcji.

4.8. Podstawa czworościanu ABCD jest trójkat równoboczny $ABC\mathrm{o}$ boku

$\alpha$, ściana boczna $BCD$ jest trójkatem równoramiennym prostopadlym

do plaszczyzny podstawy, a $\mathrm{k}\mathrm{a}\mathrm{t}$ plaski ściany bocznej przy wierzcholku

$A$ jest równy $\alpha$. Obliczyč pole powierzchni kuli opisanej na tym

czworościanie.
\end{document}
