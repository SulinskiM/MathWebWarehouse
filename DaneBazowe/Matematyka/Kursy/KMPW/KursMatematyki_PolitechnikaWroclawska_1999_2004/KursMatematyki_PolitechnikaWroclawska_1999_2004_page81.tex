\documentclass[a4paper,12pt]{article}
\usepackage{latexsym}
\usepackage{amsmath}
\usepackage{amssymb}
\usepackage{graphicx}
\usepackage{wrapfig}
\pagestyle{plain}
\usepackage{fancybox}
\usepackage{bm}

\begin{document}

99

Uwaga. Podano wskazówki do wszystkich zadań. Zaproponowano

pewna metodę rozwiazania $\mathrm{k}\mathrm{a}\dot{\mathrm{z}}$ dego $\mathrm{z}$ zadań, najczęściej nie jedyna

$\mathrm{i}\mathrm{z}$ pewnościa nie zawsze najprostsza.

l.l. Najpierw obliczyč oddzielnie mase stopu $\mathrm{i}$ mase srebra $\mathrm{w}$ stopie.

1.2. Pamietač $0$ wyznaczeniu dziedziny równania.

1.3. Oznaczyč nieznane wspólrzedne punktu $C$ przez $x \mathrm{i} y$, zapisač

wektory $\vec{AC}\mathrm{i}\vec{BC}$ za pomoca $x\mathrm{i}y\mathrm{i}$ korzystač $\mathrm{z}$ prostopadlości $\vec{AC}\perp\vec{BH}$

oraz $\vec{BC}\perp\vec{AH}. \mathrm{U}\dot{\mathrm{z}}$ yč iloczynu skalarnego.

1.4. Zamienič sinus na cosinus, stosujac odpowiedni wzór redukcyjny

$\mathrm{i}$ od razu przejśč do porównywania katów. Odpowied $\acute{\mathrm{z}}$ zapisač $\mathrm{w}$ postaci

jednej serii rozwiazań.

1.5. Pamietač, $\dot{\mathrm{z}}\mathrm{e} \log_{2}\alpha^{2} = 2\log_{2}|\alpha| \mathrm{i}$ skorzystač $\mathrm{z}$ symetrii wykresu

wzgledem prostej $x = 2$. Wykres otrzymač przez odbicia symetryczne

$\mathrm{i}$ translacje standardowej krzywej $y=\log_{2}x.$

1.6. Najpierw rozwazyč przypadek oczywisty $x < -6$. Dla $x > -6$

porównač odwrotności obu stron $\mathrm{i}$ przejśč do nierówności kwadratowej.

Pamietač $0$ dziedzinie nierówności.

1.7. Zastosowač twierdzenie cosinusów. Podczas wykonywania rysunku

pamietač, $\dot{\mathrm{z}}\mathrm{e}\mathrm{w}$ rzucie równoleglym zachowuje $\mathrm{s}\mathrm{i}\mathrm{e}$ równoleglośč oraz propor-

cje odcinków równoleglych.

1.8. Proste równolegle maja takie same wspólczynniki kierunkowe.

Wspólczynniki te wyznaczyč za pomoca pochodnych obu funkcji. Przy

kreśleniu wykresu krzywej $y=\sqrt{1-x}$ zwrócič uwage na lewostronne otocze-

nie punktu $x=1.$

2.1. Wystarczy pokazač, $\dot{\mathrm{z}}\mathrm{e}$ dla $\mathrm{k}\mathrm{a}\dot{\mathrm{z}}$ dego $n$ naturalnego wielomian

$y^{2n-1}+1$ jest podzielny przez dwumian $y+1.$

2.2. Kwadrat dlugości przekatnej wyrazič jako funkcje wysokości pros-

tokata wpisanego $\mathrm{w}$ trójkat. Jest to funkcja kwadratowa $\mathrm{i}$ do jej badania

nie jest potrzebna pochodna.
\end{document}
