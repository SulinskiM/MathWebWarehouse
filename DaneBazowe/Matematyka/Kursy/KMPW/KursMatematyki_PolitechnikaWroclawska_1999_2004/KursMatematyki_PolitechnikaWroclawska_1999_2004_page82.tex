\documentclass[a4paper,12pt]{article}
\usepackage{latexsym}
\usepackage{amsmath}
\usepackage{amssymb}
\usepackage{graphicx}
\usepackage{wrapfig}
\pagestyle{plain}
\usepackage{fancybox}
\usepackage{bm}

\begin{document}

100

2.3. Przypadek $0 < x < 1$ jest oczywisty. Dla $x >$

logarytmy do wspólnej podstawy 3 $\mathrm{i}$ przyjač $\log_{3}x=t.$

l sprowadzič

2.4. Warunek geometryczny zapisač $\mathrm{w}$ jezyku nierówności kwadratowej

$\mathrm{z}$ parametrem.

2.5. Podstawič $x+5 =t\mathrm{i}$ badač równanie $||t|-1| =m$. Przypadki

$m<0\mathrm{i}m=0$ rozpatrzeč bezpośrednio, a dla $m>0$ korzystač $\mathrm{z}\mathrm{t}\mathrm{o}\dot{\mathrm{z}}$ samości

$(|\alpha|=b)\Leftrightarrow$($\alpha=b$ lub $\alpha=-b$) prawdziwej dla $b\geq 0.$

2.6. Pomnozyč drugie równanie przez 2 $\mathrm{i}$ nastepnie odjač oba równania

stronami. Podstawienie $x-y = t$ prowadzi do równania kwadratowego

$\mathrm{z}$ niewiadoma $t.$

2.7. Uzasadnič, $\dot{\mathrm{z}}\mathrm{e}$ szukane punkty $A\mathrm{i}B\mathrm{l}\mathrm{e}\dot{\mathrm{z}}$ a na osi $Ox\mathrm{w}$ odleglości

$5\sqrt{2}$ od środka danego okregu. Przy obliczaniu pola figury (która jest

deltoid), najprościej jest korzystač $\mathrm{z}$ podobieństwa odpowiednich trzech

trójkatów prostokatnych.

2.8. Dziedzine równania określaja warunek istnienia tangensa $\mathrm{i}$ warunek

istnienia sumy nieskończonego ciagu geometrycznego. Korzystajac ze wzoru

$1+\mathrm{t}\mathrm{g}^{2}\gamma= \displaystyle \frac{1}{\cos^{2}\gamma}$ oraz ze wzorów podanych we wskazówkach do $\mathrm{z}\mathrm{a}\mathrm{d}$. 3.8

$\mathrm{i}4.3$, przeksztalcič obie strony do równości dwóch cosinusów lub sinusów

$\mathrm{i}$ przejśč od razu do porównywania katów.

3.1. Podstawič$\sqrt{}$x$=t\mathrm{i}$ korzystač $\mathrm{z}$ wlasności funkcji kwdratowej oraz

$\mathrm{z}$ monotoniczności pierwiastka kwadratowego.

3.2. Wyznaczyč środek $S$ rombu korzystajac $\mathrm{z}$ relacji $S\in l$ oraz $\vec{AS}\perp l$

$\mathrm{t}\mathrm{z}\mathrm{n}. \vec{AS}= \alpha$ñ, gdzie ñ $= [2$, 1$]$ jest wektorem prostopadlym do prostej $l.$

$\mathrm{Z}$ warunku $\vec{AS}\perp\vec{SB}$ wynika, $\dot{\mathrm{z}}\mathrm{e}\vec{SB}= - \vec{SD}=c[1,-2]$. Dane pole rombu

pozwala wyznaczyč skalar $c \mathrm{i}$ stad od razu otrzymujemy wspól-

rzedne wierzcholków $B\mathrm{i}D.$

3.3. $\mathrm{W}$ dowodzie kroku indukcyjnego przeksztalcajac lewa strong do-

prowadzič do równości $\mathrm{z}$ prawa. Unikač dowodu metoda redukcji.
\end{document}
