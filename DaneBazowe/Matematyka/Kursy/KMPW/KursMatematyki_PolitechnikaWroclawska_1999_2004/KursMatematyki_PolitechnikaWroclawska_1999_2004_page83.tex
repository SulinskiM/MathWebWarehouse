\documentclass[a4paper,12pt]{article}
\usepackage{latexsym}
\usepackage{amsmath}
\usepackage{amssymb}
\usepackage{graphicx}
\usepackage{wrapfig}
\pagestyle{plain}
\usepackage{fancybox}
\usepackage{bm}

\begin{document}

101

3.4. Pole podstawy obliczyč korzystajac $\mathrm{z}$ nastepujacego twierdzenia

$0$ zmianie pola figury plaskiej $\mathrm{w}$ rzucie prostokatnym:

{\it Pole rzutu} $prostokq_{f}tnego$ {\it figury ptaskiej jest równe polu} $tej$ {\it figury po}-

{\it mnozonemu przez cosinus} $kq_{f}tami_{G}dzy$ {\it ptaszczyznami figury} $ijej$ {\it rzutu}.

3.5. Kwadrat pola trójkata wyrazič jako funkcje wysokości trójkata.

Funkcja ta jest wielomianem. Nie mylič tego zadania $\mathrm{z}$ zagadnieniem wy-

znaczania ekstremów lokalnych.

3.6. Zauwazyč, $\dot{\mathrm{z}}\mathrm{e}$ granica lewostronna pochodnej $y'(x) \mathrm{w}$ punkcie

$x = \displaystyle \frac{5}{2}$ jest równa $-\mathrm{o}\mathrm{o}$ co oznacza, $\dot{\mathrm{z}}\mathrm{e}$ wykres jest $\mathrm{w}$ punkcie $(\displaystyle \frac{5}{2},0)$

styczny (lewostronnie) do prostej $x=\displaystyle \frac{5}{2}.$

3.7. Dla danych $ r\mathrm{i}\alpha$ najmniejsze $d$ jest wtedy, gdy krótsza podstawa

trapezu ma dlugośč 0, $\mathrm{t}\mathrm{z}\mathrm{n}$. trapez staje $\mathrm{s}\mathrm{i}\mathrm{e}$ trójkatem. Stad otrzymač

dziedzine dla $d$. Analiza otrzymanych wzorów na pole $\mathrm{i}$ promień okregu

opisanego na trapezie prowadzi do blednej dziedziny. $\mathrm{W}$ obliczeniach przyjač

jako niewiadoma polowe sumy obu podstaw $\mathrm{i}$ wyznaczyč $\mathrm{j}\mathrm{a}\mathrm{z}$ twierdzenia

Pitagorasa $\mathrm{w}$ trójkacie zawierajacym przekatna $\mathrm{i}$ wysokośč trapezu. Promień

okregu opisanego wyznaczyč stosujac twierdzenie sinusów.

3.8. Wyrazenie znajdujace $\mathrm{s}\mathrm{i}\mathrm{e}$ pod wartościa bezwzgledna przedstawič

jako $\alpha\cos(x-\alpha)$ dla odpowiedniego $\alpha \mathrm{i}\alpha$, podnieśč obie strony do kwadratu

$\mathrm{i}$ skorzystač ze wzoru 2 $\cos^{2}\gamma=1+\cos 2\gamma.$

4.1. Wyrazič $x$ przez niewiadoma liczbe skladników $n \mathrm{i}$ rozwiazač

równanie kwadratowe $\mathrm{z}$ ta niewiadoma.

4.2. Zbudowač model probabilistyczny doświadczenia, $\mathrm{t}\mathrm{j}$. określič zbiór

$\Omega \mathrm{i}$ prawdopodobieństwo $P$. Wygodniej jest obliczač prawdopodobieństwo

zdarzenia przeciwnego, $\mathrm{t}\mathrm{j}. \dot{\mathrm{z}}\mathrm{e} \mathrm{z}$ wylosowanych cyfr nie $\mathrm{m}\mathrm{o}\dot{\mathrm{z}}$ na utworzyč

liczby podzielnej przez 5.

4.3. Korzystač ze wzorów l- $\cos 2\gamma=2\sin^{2}\gamma$ oraz $\sqrt{\alpha^{2}}=|\alpha|$. Obliczyč

pochodnejednostronne bezpośrednio $\mathrm{z}$ definicji. Podczas rysowania wykresu

zwrócič uwage na otoczenie punktu $x=0.$
\end{document}
