\documentclass[a4paper,12pt]{article}
\usepackage{latexsym}
\usepackage{amsmath}
\usepackage{amssymb}
\usepackage{graphicx}
\usepackage{wrapfig}
\pagestyle{plain}
\usepackage{fancybox}
\usepackage{bm}

\begin{document}

102

4.4. Korzystač $\mathrm{z}$ wlasności katów $\mathrm{w}$ równolegloboku. Nastepnie $\mathrm{z}$ przy-

stawania odpowiednich (trzech) trójkatów wywnioskowač, $\dot{\mathrm{z}}\mathrm{e}$ przekatna

utworzonego prostokata jest równolegla do dluzszych boków równoleglo-

boku.

4.5. Podczas rozwiazywania drugiej nierówności rozpatrzyč przypadki

$ y\in (0,1)$ oraz $ y\in (1,\infty)$. Po podstawieniu $2^{x}=t$ przejśč do nierówności

kwadratowych zmiennej $t$. Nie zapomnieč $0$ dziedzinie ukladu $\mathrm{i}$ szczególo-

wym ustaleniu, które punkty brzegu naleza do rozwazanego zbioru.

4.6. Korzystajac $\mathrm{z}$ wlasności okregów stycznych zewnetrznie $\mathrm{i}$ wewne-

trznie, wykazač, $\dot{\mathrm{z}}\mathrm{e}$ suma odleglości rozwazanych punktów od środków obu

danych okregów jest stala $\mathrm{i}$ wynosi 12. Nastepnie zastosowač geometryczna

definicje elipsy.

4.7. Dziedzina funkcji jest określona przez warunki istnienia dwóch

pierwiastków rzeczywistych równania (ale niekoniecznie róznych). $\mathrm{U}\dot{\mathrm{z}}$ yč

wzorów Viète'a. Do rózniczkowania przedstawič otrzymana funkcje jako

sume funkcji potegowych. Ze wzgledu na postač dziedziny nie $\mathrm{m}\mathrm{o}\dot{\mathrm{z}}$ na mówič

$0$ asymptocie ukośnej prawostronnej. Pamietač, $\dot{\mathrm{z}}\mathrm{e},$,przyleganie'' wykresu

funkcji do asymptoty pionowej $\mathrm{m}\mathrm{o}\dot{\mathrm{z}}\mathrm{e}$ byč inne $\mathrm{z}\mathrm{k}\mathrm{a}\dot{\mathrm{z}}$ dej strony tej asymptoty.

4.8. Zauwazyč, $\dot{\mathrm{z}}\mathrm{e}$ czworościan ma plaszczyzne symetrii, która prze-

chodzi przez wierzcholki $A, D\mathrm{i}$ środek krawedzi $BC$. Środek kuli opisanej

$\mathrm{l}\mathrm{e}\dot{\mathrm{z}}\mathrm{y}$ na tej plaszczy $\acute{\mathrm{z}}\mathrm{n}\mathrm{i}\mathrm{e}\mathrm{w}$ punkcie przeciecia $\mathrm{s}\mathrm{i}\mathrm{e}$ prostej prostopadlej do pod-

stawy wystawionej $\mathrm{w}$ środku okregu opisanego na podstawie $\mathrm{z}$ symetralna

krawedzi $AD$. Dziedzine kata $\alpha$ ustalič poprzez rozwazania geometryczne

($\mathrm{k}\mathrm{a}\mathrm{t}\alpha$ musi byč wiekszy od jego rzutu prostokatnego na podstawe).

5.1. Korzystač $\mathrm{z} \mathrm{t}\mathrm{o}\dot{\mathrm{z}}$ samości $(|\alpha|\leq b) \Leftrightarrow (-b\leq\alpha\leq b)$.

narysowač za pomoca translacji standardowej krzywej $y=|x|.$

Zbiór A

5.2. Wyznaczyč najpierw $\sin\alpha+\cos\alpha \mathrm{i}$ stosowač wzór na sume sześcia-

nów.

5.3. Rozwazyč rodzine prostych przechodzacych przez punkt P. Proste

te przecinajac dana parabole, wyznaczaja cieciwy. Napisač uklad rów-

nań, który spelniaja końce cieciw i nie rozwiazujac go, wyznaczyč środki
\end{document}
