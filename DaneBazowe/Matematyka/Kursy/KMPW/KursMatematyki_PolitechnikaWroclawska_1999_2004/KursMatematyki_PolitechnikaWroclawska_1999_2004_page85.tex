\documentclass[a4paper,12pt]{article}
\usepackage{latexsym}
\usepackage{amsmath}
\usepackage{amssymb}
\usepackage{graphicx}
\usepackage{wrapfig}
\pagestyle{plain}
\usepackage{fancybox}
\usepackage{bm}

\begin{document}

103

tych cieciw ze wzorów Viète'a. Zwrócič uwage na dziedzine (szukana krzywa

nie jest cala parabola!).

5.4. Wyznaczyč dziedzine $\mathrm{i}$ podnieśč obie strony równania do kwadratu,

otrzymujac proste równanie równowazne wyjściowemu.

5.5. Korzystajac ze schematu Bernoulliego, obliczyč odpowiednie praw-

dopodobieństwa dla obu strzelców. Dla drugiego strzelca najpierw obliczyč

prawdopodobieństwo zdarzenia przeciwnego.

5.6. Jeśli $R$ jest nieduzo wieksze $\mathrm{n}\mathrm{i}\dot{\mathrm{z}}r$, to środki kulek $\mathrm{l}\mathrm{e}\dot{\mathrm{z}}$ a na przekroju

osiowym walca, gdyz kulki zajmuja $\mathrm{m}\mathrm{o}\dot{\mathrm{z}}$ liwie najnizsze polozenie. Najwiek-

sze $R$ (przy ustalonym $r$), przy którym kulki przyjmuja takie polozenie jest

wtedy, gdy trzecia kulka (tj. $\mathrm{l}\mathrm{e}\dot{\mathrm{z}}\mathrm{a}\mathrm{c}\mathrm{a}$ najwyzej) bedzie styczna $\mathrm{z}$ pierwsza

(tj. $\mathrm{l}\mathrm{e}\dot{\mathrm{z}}\mathrm{a}\mathrm{c}\mathrm{a}$ na dnie naczynia). To odpowiada warunkowi $r<R\displaystyle \leq r+\frac{r\sqrt{3}}{2}.$

Narysowač przekrój osiowy walca, zaznaczajac na nim przekroje kulek.

Korzystač $\mathrm{z}$ twierdzenia $0$ okregach stycznych zewnetrznie $\mathrm{i}\mathrm{z}$ twierdzenia

Pitagorasa.

5.7. Przypadek $m=0$ rozpatrzeč oddzielnie. Dla $m\neq 0$ badač mono-

tonicznośč rozwazajac znak pochodnej. Prowadzi to do warunków, przy

których odpowiedni trójmian kwadratowy $\mathrm{w}$ liczniku pochodnej jest nieu-

jemny na R. Pamietač, $\dot{\mathrm{z}}\mathrm{e}$ funkcja jest rosnaca $\mathrm{w}$ pewnym przedziale takze

wtedy, gdy jej pochodna jest nieujemna $\mathrm{i}$ zeruje $\mathrm{s}\mathrm{i}\mathrm{e}\mathrm{w}$ skończonej liczbie

punktów.

5.8. Przekatne $\mathrm{w}$ rombie sa równocześnie dwusiecznymi jego katów.

Jeśli wiec dwa wektory sa równej dlugości, to ich suma wyznacza kierunek

dwusiecznej kata miedzy tymi wektorami.

6.1. Zauwazyč, $\dot{\mathrm{z}}\mathrm{e} x = 1$ spelnia równanie, a dla $x \neq 1$ przejśč do

porównania wykladników. Pamietač $0$ wyznaczeniu dziedziny równania.

6.2. Równanie stycznej do okregu $(x-x_{0})^{2}+(y-y_{0})^{2}=r^{2}\mathrm{w}$ punkcie

$A(x_{1},y_{1})\mathrm{l}\mathrm{e}\dot{\mathrm{z}}$ acym na tym okregu ma postač

$(x_{1}-x_{0})(x-x_{0})+(y_{1}-y_{0})(y-y_{0})=r^{2}$
\end{document}
