\documentclass[a4paper,12pt]{article}
\usepackage{latexsym}
\usepackage{amsmath}
\usepackage{amssymb}
\usepackage{graphicx}
\usepackage{wrapfig}
\pagestyle{plain}
\usepackage{fancybox}
\usepackage{bm}

\begin{document}

104

6.3. Korzystač ze wzoru na sume cosinusów oraz ze wzorów reduk-

cyjnych. Przeksztalcač tylko lewa strong $\mathrm{i}$ doprowadzič do równości $\mathrm{z}$ prawa.

6.4. Przyjač, $\dot{\mathrm{z}}\mathrm{e}$ iloraz $q$ ciagujest wiekszy od l. Zauwazyč, $\dot{\mathrm{z}}\mathrm{e}$ środkowy

wyraz ciagu jest równy 2 $\mathrm{i}$ ulozyč równanie $\mathrm{z}$ niewiadoma $q.$

6.5. Oznaczyč przez $A_{i}$ zdarzenie polegajace na wylosowaniu $\mathrm{z}$ pierwszej

urny $i$ kul bialych, $i=0$, 1, 2, 3, $\mathrm{i}$ zastosowač wzór na prawdopodobieństwo

calkowite.

6.6. Zauwazyč, $\dot{\mathrm{z}}\mathrm{e}$ bryle $\mathrm{m}\mathrm{o}\dot{\mathrm{z}}$ na podzielič na dwie (identyczne) polowy

odpowiednia plaszczyzna prostopadla do osi obrotu, a $\mathrm{k}\mathrm{a}\dot{\mathrm{z}}$ da polowa sklada

$\mathrm{s}\mathrm{i}\mathrm{e}$ ze stozka oraz stozka ścietego $0$ wspólnej podstawie.

6.7. Wyznaczyč tylko miejsca zerowe pochodnej $\mathrm{i}$ porównač wartości

funkcji $\mathrm{w}$ tych punktach zjej wartościami na końcach przedzialu. Nie tracič

czasu na wyznaczanie ekstremów lokalnych.

6.8. Maksymalna wartośč $k$ jest osiagana wtedy, gdy trójkat jest równo-

ramienny. Stad ustalič dziedzine $k$. Korzystač $\mathrm{z}$ podobieństwa odpowied-

nich trójkatów $\mathrm{i}\mathrm{z}$ nastepujacej wlasności trójkata prostokatnego:

{\it Suma} $przyprostokq_{f}tnych$ {\it jest równa sumie średnic okrGgów}

{\it wpisanego} $i$ {\it opisanego}.

7.1. Podstawič $3^{x}=t\mathrm{i}$ korzystač $\mathrm{z}\mathrm{t}\mathrm{o}\dot{\mathrm{z}}$ samości podanej we wskazówce

do zadania 5.1.

7.2. Wykorzystač zwiazek wspólrzednych punktu ijego obrazu w powino-

wactwie prostokatnym oraz zwiazek pól figury i jej obrazu w tym przek-

sztalceniu.

7.3. Liczba $k$-elementowych podzbiorów zbioru $n$-elementowego wynosi

$\left(\begin{array}{l}
n\\
k
\end{array}\right)$. Nie pominač zbioru pustego, który jest podzbiorem $\mathrm{k}\mathrm{a}\dot{\mathrm{z}}$ dego zbioru.

7.4. Korzystač $\mathrm{z}$ twierdzenia $0$ czworokacie opisanym na okregu. Do

wyznaczenia $\sin 15^{\mathrm{O}}$ oraz $\cos 15^{\circ}$ nie korzystač $\mathrm{z}$ tablic, lecz przeksztal-

cič wyrazenie $\mathrm{t}\mathrm{a}\mathrm{k}$, aby otrzymač funkcje kata $30^{\circ}$ (por. wskazówka do

$\mathrm{z}\mathrm{a}\mathrm{d}$. 3.8$).$
\end{document}
