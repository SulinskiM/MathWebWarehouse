\documentclass[a4paper,12pt]{article}
\usepackage{latexsym}
\usepackage{amsmath}
\usepackage{amssymb}
\usepackage{graphicx}
\usepackage{wrapfig}
\pagestyle{plain}
\usepackage{fancybox}
\usepackage{bm}

\begin{document}

105

7.5. Rozwiazań, dla których $x=y$, szukač takze wśród nieskończenie

wielu rozwiazań ukladu dla przypadku $m=3.$

7.6. Rozwazyč oddzielnie przedzialy $[-\displaystyle \frac{\pi}{2},0$) oraz $[0,\displaystyle \frac{\pi}{2}],\ \mathrm{w}$ których

$\sin x$ ma staly znak, a funkcja cosinus jest monotoniczna. Zbiór rozwiazań

zaznaczyč na wykresie jako podzbiór osi odcietych.

7.7. Korzystač $\mathrm{z}$ zalezności miedzy polami $\mathrm{i}$ objetościami figur $\mathrm{i}$ bryl

podobnych.

7.8. Skonstruowač model probabilistyczny, czyli określič zbiór $\Omega$ oraz

prawdopodobieństwo $P$. Oznaczyč przez $A_{\mathrm{I}}, A_{\mathrm{I}\mathrm{I}}$ zdarzenia polegajace na

$\mathrm{t}\mathrm{y}\mathrm{m}, \dot{\mathrm{z}}\mathrm{e}$ oba tomy odpowiednio I, II powieści znajduja $\mathrm{s}\mathrm{i}\mathrm{e}$ obok siebie

$\mathrm{i}$ we wlaściwej kolejności. Interesuja nas zdarzenia $A_{\mathrm{I}} \cap A_{\mathrm{I}\mathrm{I}}$ oraz

$A_{\mathrm{I}}\cup A_{\mathrm{I}\mathrm{I}}$. Prawdopodobieństwo tego drugiego obliczyč, stosujac wzór na

prawdopodobieństwo sumy dwóch dowolnych zdarzeń.

8.1. Pamietač $0$ warunku istnienia sumy nieskończonego ciagu geome-

trycznego.

8.2. Skladnik $\left(\begin{array}{l}
11\\
i
\end{array}\right)3^{i/3}2^{(11-i)/2}$ bedzie liczba calkowita wtedy $\mathrm{i}$ tylko

wtedy, gdy $i$ bedzie podzielne przez 3, a $11-i$ bedzie parzyste.

8.3. Korzystač $\mathrm{z}$ parzystości funkcji. Narysowač $\mathrm{w}$ przedziale $[0,\infty$)

wykres funkcji $g(x)=x^{2}-2x-3\mathrm{i}$ zastosowač geometryczna interpretacje

nalozenia na $\mathrm{n}\mathrm{i}\mathrm{a}$ wartości bezwzglednej.

8.4. Najpierw określič dziedzine nierówności. Napisač $x+1=\log_{2}2^{x+1}$,

podstawič $2^{x}=t\mathrm{i}$ przejśč do nierówności kwadratowej.

8.5. Do obliczenia objetości potrzebny jest tylko tangens kata nachyle-

nia ściany bocznej do podstawy $ t=\mathrm{t}\mathrm{g}\alpha$. Warunek podany $\mathrm{w}$ zadaniu zapi-

sač $\mathrm{w}$ postaci równania $\mathrm{z}$ niewiadoma $t. \mathrm{U}\dot{\mathrm{z}}$ yč $\mathrm{t}\mathrm{o}\dot{\mathrm{z}}$ samości $\displaystyle \frac{1}{\cos^{2}\alpha}=1+\mathrm{t}\mathrm{g}^{2}\alpha.$

8.6. $K\mathrm{a}\mathrm{t}$ prosty $\mathrm{m}\mathrm{o}\dot{\mathrm{z}}\mathrm{e}\mathrm{s}\mathrm{i}\mathrm{e}$ znajdowač wjednym $\mathrm{z}$ trzech podanych wierz-

cholków trójkata. Zastosowač iloczyn skalarny.
\end{document}
