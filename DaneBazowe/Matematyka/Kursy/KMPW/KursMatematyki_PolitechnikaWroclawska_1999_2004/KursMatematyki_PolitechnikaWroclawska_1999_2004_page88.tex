\documentclass[a4paper,12pt]{article}
\usepackage{latexsym}
\usepackage{amsmath}
\usepackage{amssymb}
\usepackage{graphicx}
\usepackage{wrapfig}
\pagestyle{plain}
\usepackage{fancybox}
\usepackage{bm}

\begin{document}

106

8.7. Po wymnozeniu $\mathrm{n}\mathrm{a}$ krzyz'' skorzystač ze wzoru na iloczyn si-

nusów, doprowadzič do równości dwóch cosinusów $\mathrm{i}$ stad od razu przejśč do

porównania katów. Nie zapomnieč $0$ uwzglednieniu dziedziny.

8.8. Korzystač $\mathrm{z}$ twierdzenia $0$ stosunku pól figur podobnych. Zauwazyč

$\mathrm{i}$ uzasadnič, $\dot{\mathrm{z}}\mathrm{e}$ suma skal podobieństwa trzech mniejszych trójkatów jest

równa l.

9.1. Pole powierzchni powiekszonej kuli jest l,44 razy wieksze od pola

kuli wyjściowej.

9.2. Napisač równanie peku prostych przechodzacych przez punkt $P$

$\mathrm{i}$ majacych ujemny wspólczynnik kierunkowy $m$ (dlaczego?). Wyznaczyč

wspólrzedne punktów $A, B$ przeciecia $\mathrm{s}\mathrm{i}\mathrm{e}$ tych prostych $\mathrm{z}$ osiami ukladu

oraz środków odcinków AB $\mathrm{w}$ zalezności od $m$. Eliminujac parametr $m$

zapisač równanie krzywej $\mathrm{w}$ postaci $y=f(x).$

9.3. Po podstawieniu $3^{x} = t$ zadanie sprowadza $\mathrm{s}\mathrm{i}\mathrm{e}$ do znalezienia

warunków, przy których równanie kwadratowe $\mathrm{z}$ niewiadoma $t$ ma dwa rózne

pierwiastki dodatnie.

9.4. Rozwazyč przekrój czworościanu plaszczyzna symetrii. Korzystajac

$\mathrm{z}$ podobieństwa odpowiednich dwóch trójkatów $\mathrm{w}$ tym przekroju, wykazač,

$\dot{\mathrm{z}}\mathrm{e}$ stosunek promieni kuli opisanej do wpisanej wynosi 3. Stad ob1iczyč

wysokośč czworościanu, a nastepnie kolejno krawed $\acute{\mathrm{z}} \mathrm{i}$ objetośč.

9.5. Dla $x<-3$ lewa stronajest dodatnia, a prawa ujemna $\mathrm{i}$ nierównośč

jest oczywiście spelniona. Dla $x > -3, x \neq 3$, obie strony sa dodat-

nie. Pomnozyč je przez $(x+3)|x-3|$. Po uproszczeniu dostajemy prosta

nierównośč, do której zastosowač $\mathrm{t}\mathrm{o}\dot{\mathrm{z}}$ samośč $(|\alpha|\leq b)\Leftrightarrow(-b\leq\alpha\leq b).$

9.6. Przyjač $k \geq 1$ oraz oznaczyč przez $\alpha$ polowe wiekszego $\mathrm{z}$ katów

ostrych trójkata. Stosunek dwusiecznych wyrazič za pomoca $k$ oraz funkcji

trygonometrycznych kata $\alpha \mathrm{i}$ przeksztalcič $\mathrm{t}\mathrm{a}\mathrm{k}$, aby wystapil tylko tg $\alpha.$

Wartośč tg $\alpha$ obliczyč, wiedzac, $\dot{\mathrm{z}}\mathrm{e}$ tg $2\alpha=k.$
\end{document}
