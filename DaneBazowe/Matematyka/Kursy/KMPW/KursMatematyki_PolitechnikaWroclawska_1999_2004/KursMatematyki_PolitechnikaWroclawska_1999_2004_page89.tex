\documentclass[a4paper,12pt]{article}
\usepackage{latexsym}
\usepackage{amsmath}
\usepackage{amssymb}
\usepackage{graphicx}
\usepackage{wrapfig}
\pagestyle{plain}
\usepackage{fancybox}
\usepackage{bm}

\begin{document}

107

9.7. Przedstawič funkcje $\mathrm{w}$ postaci $f(x)=1+\displaystyle \frac{4}{x-2}+\frac{8}{(x-2)^{2}}\mathrm{i}\mathrm{w}$ tej

postaci $\mathrm{j}\mathrm{a}$ rózniczkowač. Zauwazyč, $\dot{\mathrm{z}}\mathrm{e}$ wykres jest wyra $\acute{\mathrm{z}}\mathrm{n}\mathrm{i}\mathrm{e}$ asymetryczny

wzgledem asymptoty $x=2.$

9.8. Napisač równanie stycznej $\mathrm{w}$ punkcie $(x_{0},f(x_{0}))$. Po podstawieniu

do niego wspólrzednych punktu $A$ otrzymujemy równanie trzeciego stop-

nia $\mathrm{z}$ niewiadoma $x_{0}$. Równanie to ma trzy pierwiastki wymierne. Przez

bezpośrednie sprawdzenie wystarczy znalez$\acute{}$č $\mathrm{d}\mathrm{w}\mathrm{a}$. Trzeci $\mathrm{m}\mathrm{o}\dot{\mathrm{z}}$ na obliczyč,

wiedzac, $\dot{\mathrm{z}}\mathrm{e}$ iloczyn pierwiastków wyraza $\mathrm{s}\mathrm{i}\mathrm{e}$ przez wyraz wolny $\mathrm{i}$ wspól-

czynnik przy najwyzszej potedze $x_{0}$. Podczas rysowania wykresu korzystač

$\mathrm{z}$ nieparzystości funkcji $f \mathrm{i}\mathrm{j}\mathrm{u}\dot{\mathrm{z}}$ wyznaczonych stycznych. Dodatkowe ba-

danie nie jest potrzebne.

10.1. Patrz wskazówka do zadania 3.3.

10.2. $K\mathrm{a}\mathrm{t}$ widzenia odcinka AB $\mathrm{z}$ punktu $C$ niewspólliniowego $\mathrm{z}A\mathrm{i}B$

to $\mathrm{k}\mathrm{a}\mathrm{t}\angle ACB$. Dany $\mathrm{w}$ zadaniu $\mathrm{k}\mathrm{a}\mathrm{t}$ zaznaczyč na przekroju osiowym stozka.

Objetośč wyrazič przez $l$ oraz funkcje trygonometryczne wielokrotności kata

$\alpha$. Uwaznie stosowač wzory redukcyjne $\mathrm{i}$ nie bač $\mathrm{s}\mathrm{i}\mathrm{e}$ napisač znaku minus

we wzorze na objetośč.

10.3. Patrz wskazówka do $\mathrm{z}\mathrm{a}\mathrm{d}$. 3.1.

10.4. Najpierw określič model probabilistyczny $\mathrm{t}\mathrm{j}. \Omega \mathrm{i} P$. Zdarze-

nie określone $\mathrm{w}$ treści zadania jest suma czterech rozlacznych (dlaczego?)

zdarzeń $A_{i}, i=1$, 2, 3, 4, gdzie $A_{i}$ oznacza otrzymanie trzech kart $\mathrm{w}i$-tym

kolorze $\mathrm{i}$ jednej $\mathrm{z}$ innego koloru. $P(A_{i})$ obliczyč bezpośrednio, korzystajac

$\mathrm{z}$ tego, $\dot{\mathrm{z}}\mathrm{e}P$ jest prawdopodobieństwem klasycznym.

10.5. Wyznaczyč dziedzine nierówności. Podstawič $\log_{2}x=t \mathrm{i}$ korzy-

stajac $\mathrm{z}$ monotoniczności funkcji logarytmicznej $0$ podstawie $\displaystyle \frac{1}{3}$, przejśč do

nierówności wymiernej.

10.6. Skorzystač ze wskazówki do zadania 6.2 $\mathrm{i}$ wyrazič wspólrzedne

punktów stycznościjako funkcje zmiennej $r$. Wygodniej jest szukač wartości

najwiekszej kwadratu pola, który jest funkcja wymierna.
\end{document}
