\documentclass[a4paper,12pt]{article}
\usepackage{latexsym}
\usepackage{amsmath}
\usepackage{amssymb}
\usepackage{graphicx}
\usepackage{wrapfig}
\pagestyle{plain}
\usepackage{fancybox}
\usepackage{bm}

\begin{document}

13

Praca kontrolna

nr 5

5.1. Narysowač na plaszczy $\acute{\mathrm{z}}\mathrm{n}\mathrm{i}\mathrm{e}$ zbiór

$A=\{(x,y)$ : $||x|-y|\leq 1,$

$-1\leq x\leq 2\}$

$\mathrm{i}$ znalez$\acute{}$č punkt zbioru $A\mathrm{l}\mathrm{e}\dot{\mathrm{z}}\mathrm{a}\mathrm{c}\mathrm{y}$ najblizej punktu $P(0,4).$

5.2. Obliczyč $\sin^{3}\alpha+\cos^{3}\alpha,\ \mathrm{m}\mathrm{a}\mathrm{j}_{s}\mathrm{a}\mathrm{c}$ dane $\displaystyle \sin 2\alpha=\frac{1}{4},\ \alpha\in(0,2\pi)$.

5.3. Rozwazmy rodzine prostych przechodzacych przez punkt $P(0,-1)$

$\mathrm{i}$ przecinajacych parabole $y = \displaystyle \frac{1}{4}x^{2} \mathrm{w}$ dwóch punktach. Wyznaczyč

równanie środków powstalych $\mathrm{w}$ ten sposób cieciw paraboli. Sporza-

dzič rysunek $\mathrm{i}$ opisač otrzymana krzywa.

5.4. Rozwiazač równanie

$\sqrt{x+\sqrt{x^{2}-x+2}}-\sqrt{x-\sqrt{x^{2}-x+2}}=4.$

5.5. Dwaj strzelcy strzelaja do tarczy. Pierwszy trafia $\mathrm{z}$ prawdopodo-

bieństwem $\displaystyle \frac{2}{3} \mathrm{w} \mathrm{k}\mathrm{a}\dot{\mathrm{z}}$ dym strzale $\mathrm{i}$ wykonuje 4 strza1y, a drugi trafia

$\mathrm{z}$ prawdopodobieństwem $\displaystyle \frac{1}{3} \mathrm{i}$ oddaje 8 strza1ów. Który ze strze1ców

ma wieksze prawdopodobieństwo uzyskania co najmniej trzech trafień,

jeśli wyniki kolejnych strzalów sa wzajemnie niezalezne?

5.6. Do naczynia $\mathrm{w}$ ksztalcie walca $0$ promieniu podstawy $R$ wrzucono trzy

jednakowe kulki $0$ promieniu $r$, gdzie $2r<2R\leq r(2+\sqrt{3})$. Okazalo

$\mathrm{s}\mathrm{i}\mathrm{e}, \dot{\mathrm{z}}\mathrm{e}$ plaska pokrywa naczynia jest styczna do kulki znajdujacej $\mathrm{s}\mathrm{i}\mathrm{e}$

najwyzej $\mathrm{w}$ naczyniu. Obliczyč wysokośč naczynia.

5.7. Dlajakich wartości parametru $m$ funkcja

$f(x)=\displaystyle \frac{x^{3}}{mx^{2}+6x+m}$

jest określona $\mathrm{i}$ rosnaca na calej prostej rzeczywistej.

5.8. Dany jest trójkat $0$ wierzcholkach $A(-2,1),\ B(-1,-6),\ C(2,5)$.

Za pomoca rachunku wektorowego obliczyč cosinus kata miedzy dwu-

sieczna kata $A\mathrm{i}$ środkowa boku $BC$. Sporzadzič rysunek.
\end{document}
