\documentclass[a4paper,12pt]{article}
\usepackage{latexsym}
\usepackage{amsmath}
\usepackage{amssymb}
\usepackage{graphicx}
\usepackage{wrapfig}
\pagestyle{plain}
\usepackage{fancybox}
\usepackage{bm}

\begin{document}

108

10.7. Korzystajac $\mathrm{z}\mathrm{t}\mathrm{o}\dot{\mathrm{z}}$ samości podanej we wskazówce do zadania 2.5,

wyznaczyč $y\mathrm{z}$ drugiego równania, otrzymujac dwa przypadki $y=\displaystyle \frac{3}{4}x$ oraz

$y= \displaystyle \frac{3}{4}x-5\mathrm{i}$ podstawič kolejno do pierwszego równania. Krzywa opisana

pierwszym równaniem jest symetryczna wzgledem osi rzednych, a drugie

równanie przedstawia dwie proste równolegle.

10.8. Przenieśč wszystkie wyrazy na lewa strong, $\mathrm{u}\dot{\mathrm{z}}$ yč wzoru podanego

we wskazówce do zadania 4.3, a nastepnie wzoru na sume sinusów.

ll.l. Jedna $\mathrm{z}$ figur jest trójkat, którego pole stanowi ósma cześč pola

calego trójkata (dlaczego?). Stad wywnioskowač, $\displaystyle \dot{\mathrm{z}}\mathrm{e}\alpha=\frac{\pi}{6}.$

11.2. Plaszczyzna przechodzaca przez jedna $\mathrm{z}$ krawedzi bocznych

$\mathrm{i}$ środek kuli jest plaszczyzna symetrii $\mathrm{i}$ przecina podstawy graniastoslupa

wzdluz ich wysokości. Wybierajac odpowiedni trójkat, obliczyč szukana

wysokośč. (Mozna $\mathrm{t}\mathrm{e}\dot{\mathrm{z}}$ argumentowač inaczej zauwazajac, $\dot{\mathrm{z}}\mathrm{e}$ środek kuli

opisanej oraz wierzcholki podstawy tworza czworościan foremny, którego

wysokośč stanowi polowe szukanej wysokości graniastoslupa.)

11.3. Najpierw wyznaczyč ekstrema lokalne funkcji $g(x) = \displaystyle \frac{x}{1+x^{2}}.$

Poniewaz $f(x) = \alpha g(x)$, wiec dobór $\alpha$ jest natychmiastowy. Trzeba tylko

pamietač, $\dot{\mathrm{z}}\mathrm{e}\alpha \mathrm{m}\mathrm{o}\dot{\mathrm{z}}\mathrm{e}$ byč takze ujemne $\mathrm{i}$ wtedy maksimum $f$ jest osiagane

tam, gdzie $g$ ma minimum.

11.4. Wyznaczyč dziedzine (warunek istnienia sumy nieskończonego

ciagu geometrycznego) $\mathrm{i}$ pamietač, $\dot{\mathrm{z}}\mathrm{e} \mathrm{w}$ niej mianownik sumy po lewej

stronie jest dodatni. Pomnozyč obie strony przez ten mianownik $\mathrm{i}$ sko-

rzystač ze wzoru podanego we wskazówce do zadania 3.8.

11.5. $\mathrm{U}\dot{\mathrm{z}}$ ycie indukcji matematycznej nie jest potrzebne. Przeksztalcič

prawa strong piszac 2 $\left(\begin{array}{l}
i\\
2
\end{array}\right) = i(i-1) = i^{2}-i \mathrm{i}$ pogrupowač skladniki

kwadratowe oddzielnie, a liniowe zsumowač jako kolejne liczby naturalne.

11.6. Oznaczyč środek jednego $\mathrm{z}$ rozwazanych okregów przez $A(x,y).$

Stycznośč do osi $Ox$ oznacza, $\dot{\mathrm{z}}\mathrm{e}$ promień tego okregu wynosi $|y|$, czyli
\end{document}
