\documentclass[a4paper,12pt]{article}
\usepackage{latexsym}
\usepackage{amsmath}
\usepackage{amssymb}
\usepackage{graphicx}
\usepackage{wrapfig}
\pagestyle{plain}
\usepackage{fancybox}
\usepackage{bm}

\begin{document}

109

odleglośč $A$ od środka $S$ danego okregu wynosi $|AS| = 2+ |y|$ (stycz-

nośč zewnetrzna!). Odleglośč $|AS|$ wyrazič bezpośrednio za pomoca $x\mathrm{i}y$

$\mathrm{i}$ tak otrzymač szukane równanie. Nazwač otrzymana krzywa. Pamietač,

$\dot{\mathrm{z}}\mathrm{e}$ środki okregów $\mathrm{l}\mathrm{e}\dot{\mathrm{z}}$ a na zewnatrz danego okregu.

11.7. Przyjač $\log_{3}m=t\mathrm{i}$ korzystač ze wzorów Viète'a.

11.8. Najpierw wyznaczyč dziedzine nierówności. Przypadek $x<0$jest

oczywisty, a dla $x>0\mathrm{m}\mathrm{o}\dot{\mathrm{z}}$ na podnieśč obie strony do kwadratu, nastepnie

pomnozyč przez $x^{2}$, otrzymujac nierównośč kwadratowa.

12.1. Narysowač krzywe $y = \sqrt{x-3}$ oraz $y = 4-x \mathrm{i}$ za pomoca

rysunku uzasadnič, $\dot{\mathrm{z}}\mathrm{e}$ równanie to ma tylko jeden pierwiastek oraz $\dot{\mathrm{z}}\mathrm{e}\mathrm{l}\mathrm{e}\dot{\mathrm{z}}\mathrm{y}$

$\mathrm{w}$ przedziale (3, 4). Ob1iczyč go przez podniesienie obu stron równania do

kwadratu.

12.2. Napisač rozklad $w(x)$ na czynniki $\mathrm{i}$ podstawič do obu stron

równości $x=-1.$

12.3. Niech $A_{i}$ oznacza zdarzenie polegajace na wypadnieciu $i$ oczek na

kostce. Wówczas $\Omega =  A_{1}\cup \cup A_{6} \mathrm{i}$ skladniki sa parami rozlaczne. Za-

stosowač wzór na prawdopodobieństwo calkowite. Dla wygody obliczyč

najpierw prawdopodobieństwo zdarzenia przeciwnego do określonego

$\mathrm{w}$ zadaniu, polegajacego na $\mathrm{t}\mathrm{y}\mathrm{m}, \dot{\mathrm{z}}\mathrm{e}$ rzuty moneta nie daly $\dot{\mathrm{z}}$ adnego orla.

12.4. Zauwazyč, $\dot{\mathrm{z}}\mathrm{e}$ sa cztery takie okregi; dwa $\mathrm{w}$ I čwiartce $\mathrm{i}$ pojednym

$\mathrm{w}$ II $\mathrm{i}$ IV čwiartce. Środek szukanego okregu ma $\mathrm{w}$ I čwiartce postač $S(r,r),$

$\mathrm{w}$ II čwiartce $S(-r,r)$, a $\mathrm{w}$ IV $S(r,-r)$, gdzie $r>0$jest nieznanym promie-

niem rozwazanego okregu. $\mathrm{W} \mathrm{k}\mathrm{a}\dot{\mathrm{z}}$ dym przypadku niewiadoma $r$ wyzna-

czyč ze wzoru na odleglośč punktu od danej prostej, $\mathrm{t}\mathrm{j}. 3x+4y=12.$

12.5. Poprowadzič wysokości sasiednich ścian bocznych do ich wspólnej

krawedzi. Tworza one wraz $\mathrm{z}$ przekatna podstawy trójkat równoramienny,

którego $\mathrm{k}\mathrm{a}\mathrm{t}$ przy wierzcholku wynosi $ 2\alpha (\mathrm{z}$ twierdzenia $0$ trzech prostopa-

dlych), a wysokośč jest równa $d.$

12.6. Znajac $P \mathrm{i} s$, obliczamy wysokośč trapezu, a nastepnie jego

przekatna $\mathrm{z}$ twierdzenia Pitagorasa, gdyz rzut prostokatny przekatnej na
\end{document}
