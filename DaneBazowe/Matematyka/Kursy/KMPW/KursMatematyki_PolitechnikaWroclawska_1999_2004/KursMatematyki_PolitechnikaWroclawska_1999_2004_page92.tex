\documentclass[a4paper,12pt]{article}
\usepackage{latexsym}
\usepackage{amsmath}
\usepackage{amssymb}
\usepackage{graphicx}
\usepackage{wrapfig}
\pagestyle{plain}
\usepackage{fancybox}
\usepackage{bm}

\begin{document}

110

podstawe ma dlugośč $\displaystyle \frac{s}{2}$. Ramie trapezu wyznaczamy $\mathrm{z}$ podobieństwa odpo-

wiednich trójkatów. Przekatna trapezu nie $\mathrm{m}\mathrm{o}\dot{\mathrm{z}}\mathrm{e}$ przekroczyč średnicy okre-

gu. Stad wynika warunek rozwiazalności zadania.

12.7. Dla $p=-1\mathrm{i}p=2$ ukladjest nieoznaczony $\mathrm{t}\mathrm{z}\mathrm{n}$. ma nieskończenie

wiele rozwiazań. Rozwiazania te tworza dwie proste. Dla $\mathrm{k}\mathrm{a}\dot{\mathrm{z}}$ dego $\mathrm{z}$ po-

zostalych $p$ uklad ma jedno rozwiazanie, które przy zmieniajacym $\mathrm{s}\mathrm{i}\mathrm{e}p$

przebiega trzecia prosta. Na tych trzech prostych znalez$\acute{}$č punkty $0$ podanej

wlasności.

12.8. Badač kwadrat pola powierzchni jako funkcje $y$. Jest ona wielo-

mianem. Nie mylič postawionego pytania $\mathrm{z}$ zagadnieniem wyznaczania

ekstremów lokalnych. Wartośč najmniejsza jest osiagana $\mathrm{w}$ punkcie $y=0,$

a nie $\mathrm{w}$ minimum lokalnym. (Wynik ten klóci $\mathrm{s}\mathrm{i}\mathrm{e}\mathrm{z}$ intuicja, gdyz $\mathrm{w}$ tym

przypadku tworzaca stozka jest najdluzsza.)

13.1. Korzystajac ze wzoru na cosinus róznicy katów przedstawič lewa

strong $\mathrm{w}$ postaci $\alpha\cos(x-\varphi)$ dla odpowiednio dobranego kata $\varphi.$

13.2. Wektor [l2, 5] jest wektorem normalnym prostej $l$, czyli wektor

$\vec{v}= [5,-12]$ jest do niej równolegly (por. wskazówka do zadania 31.7.).

$\mathrm{Z}$ definicji iloczynu skalarnego wynika, $\dot{\mathrm{z}}\mathrm{e}$ liczba $\displaystyle \frac{|\vec{AB}0\vec{v}|}{|\vec{v}|}$ jest dlugościa

rzutu prostokatnego odcinka $AB$ na prosta $l.$

13.3. Wyznaczyč dziedzine (nie zapomnieč $0$ warunku $2^{m}\neq 7$) $\mathrm{i}\mathrm{u}\dot{\mathrm{z}}$ yč

wzorów Viète'a. Wykres $f$ otrzymač ze standardowej krzywej $y=2^{m}$ przez

translacje $\mathrm{i}$ odbicie symetryczne.

13.4. Oznaczyč przez $B_{i}$ zdarzenie polegajace na $\mathrm{t}\mathrm{y}\mathrm{m}, \dot{\mathrm{z}}\mathrm{e}$ pierwszy

strzelec trafil $i$ razy, $i=0$, 1, 2, a przez $C_{j}$ zdarzenie, $\dot{\mathrm{z}}\mathrm{e}$ drugi strzelec

trafil $j$ razy, $j = 0$, 1, 5. Wtedy rozwazane zdarzenie ma postač

$(B_{0}\cap C_{3})\cup(B_{1}\cap C_{2})\cup(B_{2}\cap C_{1})$. Korzystač ze schematu Bernoulliego

$\mathrm{i}$ niezalezności par zdarzeń $B_{i}, C_{j}.$

13.5. Oddzielnie rozwazyč $n$ parzyste $\mathrm{i}$ nieparzyste. Zapisač warunki

na sumy wyrazów tego ciagu $\mathrm{i}$ eliminujac niewiadome wyrazič $\alpha_{2}$ oraz $\alpha_{3}$
\end{document}
