\documentclass[a4paper,12pt]{article}
\usepackage{latexsym}
\usepackage{amsmath}
\usepackage{amssymb}
\usepackage{graphicx}
\usepackage{wrapfig}
\pagestyle{plain}
\usepackage{fancybox}
\usepackage{bm}

\begin{document}

111

tylko przez róznice tego ciagu.

$\alpha_{2}\alpha_{3}=48.$

Nastepnie obliczyč te róznice z równania

13.6. Poprowadzič dwusieczna $AD \mathrm{i}$ wyznaczyč $|BC|$, korzystajac

$\mathrm{z}$ podobieństwa trójkatów $ABC \mathrm{i} ADC$. Dalej korzystač $\mathrm{z}$ twierdzenia

sinusów oraz ze wzoru na promień okregu wpisanego $\mathrm{w}$ trójkat $r=\displaystyle \frac{S}{p}.$

13.7. Zbiór $A$ wyznaczyč korzystajac ze wskazówki do zadania 5.1.

Uzasadnič (podnoszac obie strony do kwadratu), $\dot{\mathrm{z}}\mathrm{e}$ krzywa $0$ równaniu

$y=\sqrt{4x-x^{2}}$ nie jest lukiem paraboli lecz pólokregiem. Obliczyč odleglośč

punktu $P$ od $\mathrm{k}\mathrm{a}\dot{\mathrm{z}}$ dej $\mathrm{z}$ trzech cześci brzegu zbioru $A\cap B\mathrm{i}$ porównač je.

13.8. Korzystač $\mathrm{z}$ parzystości funkcji. $\mathrm{Z}$ postaci dziedziny wynika, $\dot{\mathrm{z}}\mathrm{e}$

funkcja nie $\mathrm{m}\mathrm{o}\dot{\mathrm{z}}\mathrm{e}$ mieč asymptot (dlaczego?). Granica lewostronna pochod-

nej $\mathrm{w}$ punkcie $x = \sqrt{8}$ wynosi $-\infty$, wiec prosta $x = \sqrt{8}$ jest styczna do

wykresu funkcji $f(x)$. Dla sporzadzenia wykresu dobrač odpowiednia jed-

nostke na obu osiach ukladu wspólrzednych.

14.1. Korzystač ze wskazówki do $\mathrm{z}\mathrm{a}\mathrm{d}$. 7.3. Otrzymane wyrazenie jest

ciagiem rosnacym $\mathrm{i}$ zadanie $\mathrm{m}\mathrm{o}\dot{\mathrm{z}}\mathrm{e}$ mieč co najwyzej jedno rozwiazanie.

14.2. Uzasadnič, $\dot{\mathrm{z}}\mathrm{e}$ promienie kolejnych okregów tworza ciag geome-

tryczny, którego iloraz jest równy pierwszemu wyrazowi ciagu.

14.3. Korzystač $\mathrm{z}$ rachunku wektorowego $\mathrm{i}$ iloczynu skalarnego. Za-

uwazyč, $\dot{\mathrm{z}}\mathrm{e}$ wszystkie proste $\mathrm{z}$ danej rodziny przechodza przez punkt $P(1,1).$

14.4. Stosowač wzór na tangens róznicy katów.

tej $\mathrm{t}\mathrm{o}\dot{\mathrm{z}}$ samości $\mathrm{i}$ funkcji $f(x).$

Wyznaczyč dziedzine

14.5. Skorzystač ze wskazówki do zadania 7.2. Rozwazana figura jest

róznica odcinka danego kola, wyznaczonego przez oś odcietych, oraz jego

obrazu $\mathrm{w}$ powinowactwie określonym $\mathrm{w}$ zadaniu.

14.6. Zastosowač podana nierównośč $\mathrm{i}$ sprowadzič logarytmy do pod-

stawy 2. Nastepnie wykazač, $\dot{\mathrm{z}}\mathrm{e}$ iloraz rozwazanego ciagu geometrycznego

jest wiekszy od l $\mathrm{i}$ stad od razu otrzymač odpowied $\acute{\mathrm{z}}.$

14.7. Patrz wskazówka do zadania 7.8.
\end{document}
