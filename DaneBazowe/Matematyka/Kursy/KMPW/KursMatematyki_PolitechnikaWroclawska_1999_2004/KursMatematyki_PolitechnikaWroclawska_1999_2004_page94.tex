\documentclass[a4paper,12pt]{article}
\usepackage{latexsym}
\usepackage{amsmath}
\usepackage{amssymb}
\usepackage{graphicx}
\usepackage{wrapfig}
\pagestyle{plain}
\usepackage{fancybox}
\usepackage{bm}

\begin{document}

112

14.8. Rozwazyč przekrój plaszczyzna przechodzaca przez przekatna

podstawy $\mathrm{i}$ wierzcholek ostroslupa. $\mathrm{Z}$ twierdzenia $0$ odcinkach stycznych

do kuli poprowadzonych $\mathrm{z}$ ustalonego punktu wynika, $\dot{\mathrm{z}}\mathrm{e}$ punkt styczności

kuli $\mathrm{z}$ krawedzia boczna $\mathrm{l}\mathrm{e}\dot{\mathrm{z}}\mathrm{y}\mathrm{w}$ odleglości $\displaystyle \frac{\alpha}{2}$ od wierzcholka podstawy. Ko-

rzystajac $\mathrm{z}$ tej obserwacji obliczyč krawed $\acute{\mathrm{z}}$ boczna na dwa sposoby $\mathrm{i}$ stad

wyznaczyč promień kuli.

15.1. Oznaczyč przez x odleglośč miejscowości, a przez y predkośč

drugiego rowerzysty. Ulozyč uklad dwóch równań z tymi niewiadomymi,

zapisujac informacje podane w treści zadania.

15.2. Określič dziedzine nierówności. Przypadek $x<0$ jest oczywisty.

Dla $x>0$ podnieśč obie strony do kwadratu, po pomnozeniu przez $x^{2}$ otrzy-

mujemy nierównośč dwukwadratowa.

15.3. Pole powierzchni dachu obliczyč z twierdzenia podanego we wska-

zówce do zadania 3.4. Objetośč dachu ob1iczyč, dzie1ac bry1e p1aszczyznami

pionowymi na dwa ostroslupy i graniastoslup.

15.4. Wyrazič $w_{n+1}$ przez $w_{n}$, korzystajac $\mathrm{z}$ danych zadania. Uzasadnič,

$\dot{\mathrm{z}}\mathrm{e}$ ciag $\triangle_{n}=w_{n+1}-w_{n}$ jest ciagiem geometrycznym $0$ ilorazie 1,015 oraz

$\dot{\mathrm{z}}\mathrm{e}w_{n}=w_{1}+\triangle_{1}+ +\triangle_{n-1}$. Pensja $\mathrm{w}$ kwietniu 2002 roku jest równa $w_{6}.$

15.5. Funkcja $f(x)$ jest rosnaca $\mathrm{i}$ zbiorem jej wartości jest R. Stad

$f^{-1}(x)=\sqrt[3]{x}$ jest określona na $\mathrm{R}$, ajej wykres jest odbiciem symetrycznym

wykresu $f(x)$ wzgledem prostej $0$ równaniu $y = x$. Wykres funkcji $h(x)$

$\mathrm{w}$ przedziale $(0,\infty)$ otrzymač przez translacje cześci wykresu funkcji $f^{-1}(x)$

$\mathrm{i}$ korzystajac $\mathrm{z}$ parzystości funkcji $h(x).$

15.6. Wyznaczyč dziedzine, pomnozyč obie strony przez mianownik,

przejśč za pomoca wzoru redukcyjnego do równości dwóch cosinusów i stad

od razu do porównania katów. Wynik zapisač w postaci jednej serii.

15.7. Patrz wskazówka do zadania 5.8.
\end{document}
