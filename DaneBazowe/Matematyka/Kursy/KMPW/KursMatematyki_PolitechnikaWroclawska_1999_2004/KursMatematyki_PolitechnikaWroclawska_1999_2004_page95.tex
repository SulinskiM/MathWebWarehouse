\documentclass[a4paper,12pt]{article}
\usepackage{latexsym}
\usepackage{amsmath}
\usepackage{amssymb}
\usepackage{graphicx}
\usepackage{wrapfig}
\pagestyle{plain}
\usepackage{fancybox}
\usepackage{bm}

\begin{document}

113

15.8. Wyznaczyč dziedzine funkcji; nie pominač punktu $x=0$. Sume

wyrazów nieskończonego ciagu geometrycznego zapisač $\mathrm{w}$ postaci

$f(x)=x+1+\displaystyle \frac{2}{x-2}, \mathrm{z}$ której od razu odczytač równania asymptot (uwazač

na dziedzine). Ta postač jest takze wygodna do rózniczkowania (nie jest

celowe stosowanie wzoru na pochodna ilorazu). Podczas rysowania wykresu

jeszcze raz uwazač na dziedzine.

16.1. Oznaczyč przez $x, y, z$ ceny odpowiednio poczatkowa, po obnizce

$\mathrm{i}$ po podwyzce. Wyrazič $y$ przez $x$ oraz $z$ przez $y\mathrm{i}\mathrm{w}$ konsekwencji $z$ przez $x.$

16.2. Punkt $(0,0)$ rozpatrzyč oddzielnie. Zauwazyč, $\dot{\mathrm{z}}\mathrm{e}$ zbiór jest syme-

tryczny wzgledem obu osi ukladu wspólrzednych $\mathrm{i}$ wyznaczyč (oraz opisač)

najpierw $\mathrm{t}\mathrm{e}$ cześč, która $\mathrm{l}\mathrm{e}\dot{\mathrm{z}}\mathrm{y}\mathrm{w}$ I čwiartce.

16.3. Wysokości ścian bocznych oraz odcinek laczacy środki dwóch

odpowiadajacych im krawedzi podstawy tworza trójkat równoramienny

$0$ kacie przy wierzcholku $ 2\alpha \mathrm{i}$ podstawie $\displaystyle \frac{\alpha}{2}$ (dlaczego?). Podstawa tego

trójkata nie przechodzi przez spodek wysokości ostroslupa. Przez porów-

nanie tego trójkata $\mathrm{z}$ jego rzutem prostokatnym na podstawe ostroslupa,

określič dziedzine dla kata $\alpha.$

16.4. $\mathrm{M}\mathrm{o}\dot{\mathrm{z}}$ na odciač narozniki zawierajace wierzcholki katów ostrych

równolegloboku lub zawierajace wierzcholki katów rozwartych. Nalezy wy-

brač ($\mathrm{i}$ uzasadnič odpowiednim rachunkiem) to ciecie, które daje romb

$0$ wiekszym polu, $\mathrm{t}\mathrm{j}$. odciač narozniki zawierajace katy rozwarte. Punkt,

przez który nalezy poprowadzič ciecie wyznaczyč $\mathrm{z}$ twierdzenia cosinusów.

16.5. $\sqrt{2}$ zamienič na potege $0$ podstawie 2 $\mathrm{i}$ wykladniku ulamkowym,

skorzystač $\mathrm{z}$ regul dzialań na potegach, przejśč do porównania wykladników

$\mathrm{i}$ podstawič $\log_{2}x=t.$

16.6. Wyrazenie $\mathrm{w}$ mianowniku zapisač $\mathrm{w}$ postaci $3+\alpha\cos(x-\alpha)$ (por.

wskazówka do zadania 3.8). Wykazač, $\dot{\mathrm{z}}\mathrm{e} |\alpha| <3$, co oznacza, $\dot{\mathrm{z}}\mathrm{e}$ dziedzina

funkcji $f(x)$ jest $\mathrm{R}$, a mianownik jest dodatni $\mathrm{w}$ R. Wartośč najmniejsza

funkcji $f(x)$ jest osiagana $\mathrm{w}$ tym punkcie, $\mathrm{w}$ którym mianownik jest naj-

wiekszy $\mathrm{i}$ na odwrót. $\mathrm{U}\dot{\mathrm{z}}$ ycie pochodnej jest zbedne.
\end{document}
