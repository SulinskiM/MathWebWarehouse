\documentclass[a4paper,12pt]{article}
\usepackage{latexsym}
\usepackage{amsmath}
\usepackage{amssymb}
\usepackage{graphicx}
\usepackage{wrapfig}
\pagestyle{plain}
\usepackage{fancybox}
\usepackage{bm}

\begin{document}

114

16.7. Oddzielnie rozpatrzyč przypadek $p = 0$. Dla $p \neq 0$ równanie

dwukwadratowe ma dokladnie dwa rózne pierwiastki, gdy odpowiadajace

mu równanie kwadratowe ma wyróznik równy zeru $\mathrm{b}\mathrm{a}\mathrm{d}\acute{\mathrm{z}}$ ma wyróznik do-

datni $\mathrm{i}$ jednocześnie jeden $\mathrm{z}$ pierwiastków ujemny.

16.8. Napisač równanie stycznej $\mathrm{w}$ punkcie $A$, korzystajac $\mathrm{z}$ pochod-

nej funkcji. Styczna ta przecina wykres funkcji $\mathrm{w}$ innym punkcie $B$. Przy

wyznaczaniu tego punktu otrzymujemy równanie trzeciego stopnia, które

ze wzgledu na stycznośč $\mathrm{w}$ punkcie $A$ ma pierwiastek podwójny 3 $\mathrm{i}$ tylko

trzeba znalez$\acute{}$č trzeci pierwiastek (por. wskazówka do $\mathrm{z}\mathrm{a}\mathrm{d}$. 9.8).

17.1. Najpierw rozwazyč przypadek, gdy iloraz równy zeru, $\mathrm{t}\mathrm{z}\mathrm{n}.$

$\cos x = 0$. Wtedy wszystkie dalsze wyrazy ciagu sa równe zeru. Jeśli

$\cos x \neq 0$, to liczby $\sin x, \cos x, \sin 2x$ tworza ciag geometryczny wtedy

$\mathrm{i}$ tylko wtedy, gdy kwadrat liczby środkowej jest iloczynem liczb skrajnych,

$\mathrm{t}\mathrm{z}\mathrm{n}$. gdy $\cos x=2\sin^{2}x$. Podstawič $\cos x=t.$

17.2. Losowe dzielenie druzyn na grupy interpretowač jako permuta-

cje numerów wszystkich druzyn, $\mathrm{t}\mathrm{j}$. liczb 1, 2, 16, gdzie ko1ejne czwórki

wyrazów permutacji wyznaczaja sklad kolejnych grup. Pamietač $0$ określeniu

na poczatku modelu probabilistycznego, $\mathrm{t}\mathrm{j}. \Omega \mathrm{i}P.$

17.3. Zauwazyč, $\dot{\mathrm{z}}\mathrm{e}$ dane wyrazenie $\mathrm{m}\mathrm{o}\dot{\mathrm{z}}$ na zapisač $\mathrm{w}$ postaci

$[(x^{2}+x+1)^{3}+x^{3}] - [x^{6}+2x^{3}+1] \mathrm{i}$ stosujac wzór na sume sześcianów,

wykazač, $\dot{\mathrm{z}}\mathrm{e}$ oba skladniki tej sumy dziela $\mathrm{s}\mathrm{i}\mathrm{e}$ przez $(x+1)^{2}$

17.4. $\mathrm{Z}$ symetrii figury wynika, $\dot{\mathrm{z}}\mathrm{e}$ środek $S$ okregu stycznego $\mathrm{w}$ dwóch

punktach do danej paraboli $\mathrm{l}\mathrm{e}\dot{\mathrm{z}}\mathrm{y}$ na osi rzednych, $\mathrm{t}\mathrm{z}\mathrm{n}$. mamy $S(0,y_{0})$, przy

czym $y_{0} > r$. Stycznośč oznacza, $\dot{\mathrm{z}}\mathrm{e}$ równanie kwadratowe $\mathrm{z}$ niewiadoma

rzedna $y$ punktu styczności powinno mieč dodatni pierwiastek podwójny, co

jest spelnione, gdy wyróznik tego równaniajest równy zeru, a wspólczynnik

przy niewiadomej $y$ jest ujemny.

17.5. Dbač $0$ logiczna poprawnośč zapisu calego dowodu indukcyjnego.

$\mathrm{W}$ dowodzie kroku indukcyjnego przeksztalcač tylko lewa strong, pamietajac,

$\dot{\mathrm{z}}\mathrm{e}$ zwiekszenie $n\mathrm{o}1$ powoduje pojawienie $\mathrm{s}\mathrm{i}\mathrm{e}$ dwóch dodatkowych skladników.
\end{document}
