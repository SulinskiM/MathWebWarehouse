\documentclass[a4paper,12pt]{article}
\usepackage{latexsym}
\usepackage{amsmath}
\usepackage{amssymb}
\usepackage{graphicx}
\usepackage{wrapfig}
\pagestyle{plain}
\usepackage{fancybox}
\usepackage{bm}

\begin{document}

115

17.6. Ustalič dziedzine nierówności $\mathrm{i}$ korzystač $\mathrm{z}$ wlasności logarytmu

$0$ podstawie mniejszej od l (dziedzina jest zawarta $\mathrm{w}$ odcinku $(0,1)$).

17.7. Uzasadnič, $\dot{\mathrm{z}}\mathrm{e} \angle ASD$ jest prosty. To oznacza, $\dot{\mathrm{z}}\mathrm{e}$ dane $c= |AD|$

oraz $d = |AS|, d\sqrt{2} \geq c > d$, jednoznacznie określaja trójkat $ASD$

oraz promień okregu $r \mathrm{i}\mathrm{k}\mathrm{a}\mathrm{t} \angle DAB$ trapezu. Wyznaczyč $r$ oraz dlugości

odcinków, na które punkt styczności dzieli $AD. \mathrm{M}\mathrm{o}\dot{\mathrm{z}}$ liwe sa dwa przy-

padki: albo podzial $AB$, liczac od wierzcholka $A$, jest $\mathrm{w}$ stosunku 2:1, a1bo

$\mathrm{w}$ stosunku 1:2. $\mathrm{W}$ drugim przypadku $\mathrm{m}\mathrm{o}\dot{\mathrm{z}}\mathrm{e}\mathrm{s}\mathrm{i}\mathrm{e}$ zdarzyč, $\dot{\mathrm{z}}\mathrm{e}\mathrm{k}\mathrm{a}\mathrm{t}$ przy wierz-

cholku $B$ jest rozwarty.

17.8. $\mathrm{M}\mathrm{o}\dot{\mathrm{z}}$ liwe sa dwa przypadki: albo $\mathrm{w}$ jednym $\mathrm{z}$ wierzcholków pod-

stawy wszystkie katy plaskie kata trójściennego wychodzacego $\mathrm{z}$ tego wierz-

cholka sa ostre, albo wszystkie sa rozwarte. $\mathrm{W}$ obu przypadkach poprowa-

dzič plaszczyzne symetrii przez ten wierzcholek $\mathrm{i}$ przeciwlegly wierzcholek

drugiej podstawy oraz przez odpowiednie przekatne obu podstaw. Niezna-

$\mathrm{n}\mathrm{a}$ wysokośč równoleglościanu obliczamy $\mathrm{z}$ twierdzenia $0$ trzech prostopa-

dlych. Obliczamy najpierw wysokośč rombu tworzacego $\mathrm{k}\mathrm{a}\dot{\mathrm{z}}\mathrm{d}\mathrm{a}$ ściane równo-

leglościanu, nastepnie odleglośč spodka wysokości równoleglościanu od kra-

wedzi podstawy $\mathrm{i}$ wreszcie $\mathrm{z}$ twierdzenia Pitagorasa wysokośč równoleglo-

ścianu. $\mathrm{W}$ obu przypadkach obliczenia sa analogiczne.

18.1. Zarówno licznik jak $\mathrm{i}$ mianownik sa sumami skończenie wielu

(ustalič $\mathrm{i}\mathrm{l}\mathrm{u}$) wyrazów dwóch ciagów geometrycznych. Obliczyč te sumy

$\mathrm{i}$ podzielič licznik $\mathrm{i}$ mianownik przez $2^{2n}$

18.2. Szukana prosta przechodzi przez środek odcinka $AB$ ijest prosto-

padla do danej prostej. Stad od razu $\mathrm{m}\mathrm{o}\dot{\mathrm{z}}$ na napisač równanie tej prostej.

18.3. Patrz wskazówka do zadania l0.2.

18.4. Oznaczyč przez $x, y$ ceny dlugopisu $\mathrm{i}$ zeszytu. Wtedy

$x >y\geq 0$, 50. Ulozyč uklad dwóch równań $\mathrm{z}$ niewiadomymi $x, y\mathrm{i}$ para-

metrem $k$. Oddzielnie rozwazyč przypadek $k = 2$, dla którego uklad jest

nieoznaczony, oraz $k\neq 2$, gdy uklad ma jedno rozwiazanie. $\mathrm{W}$ pierwszym

przypadku wybrač wszystkie $k$, dla których $x\mathrm{i}y$ wyrazaja $\mathrm{s}\mathrm{i}\mathrm{e}\mathrm{w}$ pelnych

dziesiatkach groszy $\mathrm{i}$ spelniaja warunek $x>y\geq 0$, 50. Odpowiedni rysunek

ulatwia znalezienie wszystkich rozwiazań.
\end{document}
