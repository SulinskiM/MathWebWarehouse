\documentclass[a4paper,12pt]{article}
\usepackage{latexsym}
\usepackage{amsmath}
\usepackage{amssymb}
\usepackage{graphicx}
\usepackage{wrapfig}
\pagestyle{plain}
\usepackage{fancybox}
\usepackage{bm}

\begin{document}

116

18.5. Korzystač ze wzoru $\displaystyle \sin 2\gamma=\frac{2\mathrm{t}\mathrm{g}\gamma}{1+\mathrm{t}\mathrm{g}^{2}\gamma} \mathrm{i}$ podstawič $\mathrm{t}\mathrm{g}\gamma=t.$

18.6. Stosowač schemat Bernoulliego. Drugie pytanie dotyczy praw-

dopodobieństwa warunkowego rozwazanego zdarzenia przy warunku, $\dot{\mathrm{z}}\mathrm{e}$ co

najmniej jedna $\dot{\mathrm{z}}$ arówka jest dobra.

18.7. Poniewaz promień szukanego okregu jest bardzo many, nalezy

przyjač $\mathrm{n}\mathrm{a}$ rysunku $\mathrm{d}\mathrm{u}\dot{\mathrm{z}}$ ajednostke $\mathrm{i}$ narysowač tylko odpowiedni $\mathrm{l}\mathrm{u}\mathrm{k}$ danego

okregu. $\mathrm{W}$ obliczeniach korzystač $\mathrm{z}$ twierdzenia $0$ okregach stycznych ze-

wnetrznie oraz $\mathrm{z}$ twierdzenia Pitagorasa $\mathrm{w}$ trójkacie, którego wierzcholkami

sa środki obu okregów oraz rzut prostokatny środka malego okregu na od-

cinek $AS.$

18.8. Pole $\mathrm{i}$ objetośč ostroslupa ścietego wyrazičjako funkcje dlugości $x$

krawedzi górnej podstawy tego ostroslupa, $0 < x <$ l. Korzystač

$\mathrm{z}$ twierdzenia $0$ stosunku pól $\mathrm{i}$ objetości figur $\mathrm{i}$ bryl podobnych. Wyznaczyč

miejsce zerowe pochodnej znalezionej funkcji, zbadač znak pochodnej $\mathrm{i}$ uza-

sadnič, $\dot{\mathrm{z}}\mathrm{e}\mathrm{w}$ tym punkcie funkcja osiaga nie tylko ekstremum lokalne, ale

takze wartośč najwieksza.

19.1. Wektory $\vec{BM}$ oraz $\vec{BK}$ wyrazič za pomoca wektorów $\vec{AB}, \vec{BC}$

oraz $\vec{CD}$. Majac wspólrzedne tych wektorów, od razu obliczyč pole $\triangle KMB.$

19.2. Napisač zwiazek przekatnej prostopadlościanu $\mathrm{z}$ dlugościami jego

krawedzi $\mathrm{i}$ stad obliczyč nieznana róznice ciagu. Odrzucič to rozwiazanie,

które prowadzi do ujemnych dlugości krawedzi.

19.3. Zbiór $A$ wyznaczyč korzystajac ze wskazówki do zadania 13.7

($\mathrm{w}$ cześci dotyczacej zbioru $B \mathrm{w}$ tamtej wskazówce). Dobrač $s \mathrm{t}\mathrm{a}\mathrm{k}$, aby

prosta $B_{s}$ miala jeden punkt wspólny ze zbiorem $A$ (co to znaczy geome-

trycznie?) $\mathrm{i}$ stad od razu podač odpowied $\acute{\mathrm{z}}.$

19.4. Korzystajac $\mathrm{z}$ nierówności trójkata, ustalič, które pary odcinków

moga byč podstawami trapezu. $\mathrm{s}_{\mathrm{a}}$ trzy takie $\mathrm{m}\mathrm{o}\dot{\mathrm{z}}$ liwości (spośród sześciu).

$\mathrm{W}$ dwóch przypadkach pole trapezu jest mniejsze od ll arów. Wykazač to,

zauwazajac, $\dot{\mathrm{z}}\mathrm{e}$ wysokośč trapezu jest mniejsza od $\mathrm{k}\mathrm{a}\dot{\mathrm{z}}$ dego $\mathrm{z}$ jego ramion.

$\mathrm{W}$ trzecim przypadku nalezy obliczyč pole $\mathrm{i}$ wykazač, $\dot{\mathrm{z}}\mathrm{e}$ przekracza ono

ll arów.
\end{document}
