\documentclass[a4paper,12pt]{article}
\usepackage{latexsym}
\usepackage{amsmath}
\usepackage{amssymb}
\usepackage{graphicx}
\usepackage{wrapfig}
\pagestyle{plain}
\usepackage{fancybox}
\usepackage{bm}

\begin{document}

117

19.5. Ustalič dziedzine dla parametru $m\mathrm{i}$ stosowač wzory Viète'a. Za

pomoca pochodnej wykazač, $\dot{\mathrm{z}}\mathrm{e}$ kwadrat róznicy pierwiastków, jako funkcja

zmiennej $m$, jest malejacy $\mathrm{w}$ wyznaczonej dziedzinie.

19.6. $\mathrm{W}$ dowodzie kroku indukcyjnego uwaznie stosowač reguly dzialań

na potegach.

19.7. Sumy $\mathrm{z}$ lewych stron przeksztalcič na iloczyny. Stad wywniosko-

wač, $\dot{\mathrm{z}}\mathrm{e}\sin(x+y)=1$, czyli $\mathrm{z}$ drugiego równania $\displaystyle \cos(x-y)=\frac{1}{2}\mathrm{i}$ od razu

przejśč do ukladów równań liniowych $\mathrm{z}$ niewiadomymi $x\mathrm{i}y.$

19.8. Oznaczyč $|CD| = |CA| = |CB| = \alpha$. Poniewaz $CD \perp AD$

oraz $CD \perp BD$, wiec dwie ściany boczne sa prostopadle do podstawy

$ABD$ (bedacej trójkatem równobocznym) $\mathrm{i}$ tworza ze soba $\mathrm{k}\mathrm{a}\mathrm{t} 60^{\circ} K\mathrm{a}\mathrm{t}$

miedzy podstawa $\mathrm{i}$ plaszczyzna $ABC$ wyznaczamy, przecinajac czworościan

plaszczyzna symetrii $CDE$, gdzie $E$ jest środkiem $AB$. Dla wyznaczenia

kata miedzy plaszczyzna $ABC\mathrm{i}$ plaszczyzna $BCD$ ($\mathrm{i}$ równocześnie $ACD$)

nalezy $\mathrm{z}$ wierzcholka $D$ poprowadzič wysokośč czworościanu na ściane $ABC.$

Poniewaz $\triangle BCD$ jest prostokatny $\mathrm{i}$ równoramienny, wiec $\mathrm{z}$ twierdzenia

$0$ trzech prostopadlych wynika, $\dot{\mathrm{z}}\mathrm{e}$ spodek tej wysokości $\mathrm{l}\mathrm{e}\dot{\mathrm{z}}\mathrm{y} \mathrm{w}$ środku

okregu opisanego na trójkacie $ABC$. Wyrazič $\mathrm{t}\mathrm{e}$ wysokośč przez $\alpha,$

obliczajac objetośč czworościanu na dwa sposoby $\mathrm{i}$ stad od razu wyznaczyč

sinus rozwazanego kata.

20.1. Oddzielnie rozpatrzeč $m=0\mathrm{i}m=2$. Dla pozostalych parametrów

$m$ korzystač $\mathrm{z}$ faktu, $\dot{\mathrm{z}}\mathrm{e}$ oś symetrii paraboli przechodzi przezjej wierzcholek.

20.2. Uzasadnič, $\dot{\mathrm{z}}\mathrm{e}$ środek danej kuli $\mathrm{i}$ środek kuli wpisanej $\mathrm{w}$ dana

bryle tworza przekatna sześcianu $0$ krawedzi równej promieniowi kuli wpisa-

nej. Rozwazyč przekrój plaszczyzna symetrii zawierajaca środki obu $\mathrm{k}\mathrm{u}\mathrm{l}.$

20.3. Określič model probabilistyczny, $\mathrm{t}\mathrm{j}. \Omega \mathrm{i} P$. Obliczyč prawdo-

podobieństwo zdarzenia przeciwnego, korzystajac ze wzoru na prawdopodo-

bieństwo sumy dwóch dowolnych zdarzeń.
\end{document}
