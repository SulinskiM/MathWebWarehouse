\documentclass[a4paper,12pt]{article}
\usepackage{latexsym}
\usepackage{amsmath}
\usepackage{amssymb}
\usepackage{graphicx}
\usepackage{wrapfig}
\pagestyle{plain}
\usepackage{fancybox}
\usepackage{bm}

\begin{document}

XLIV

KORESPONDENCYJNY KURS

Z MATEMATYKI

wrzesień 2014 r.

PRACA KONTROLNA nr l- POZIOM PODSTAWOWY

l. Wykaz, $\dot{\mathrm{z}}\mathrm{e}$ róznica kwadratów dwóch liczb nieparzystych jest podzielna przez 8.

2. Wlaściciel hurtowni sprzedaf $\displaystyle \frac{1}{3}$ partii bananów po zalozonej przez siebie cenie. Poniewaz

pozostałe owoce zaczęły zbyt szybko dojrzewač, więc obnizył ich cenę $0$ 30\%. Dzięki te-

mu sprzedaf 60\% aktua1nego stanu. Resztę bananów udało mu się sprzedač dopiero, gdy

ustalif ich cenę na poziomie $\displaystyle \frac{1}{5}$ ceny $\mathrm{P}^{\mathrm{O}\mathrm{C}\mathrm{Z}}\Phi^{\mathrm{t}\mathrm{k}\mathrm{o}\mathrm{w}\mathrm{e}\mathrm{j}}$. Ile procent zaplanowanego zysku sta-

nowi kwota uzyskana ze sprzedazy? Wjakiej cenie ($\mathrm{w}$ porównaniu $\mathrm{z}$ zalozonq) powinien

sprzedač pierwszą partię towaru, $\dot{\mathrm{z}}$ eby jednokrotna obnizka ich ceny $0$ 25\% pozwolifa na

sprzedanie wszystkich owoców $\mathrm{i}$ uzyskanie zaplanowanego początkowo zysku?

3. Narysuj wykres funkcji

$f(x)=\displaystyle \frac{|x-1|+x}{|x+1|}.$

Następnie rozwiąz nierównośč $f(x)\geq 1 \mathrm{i}$, korzystając $\mathrm{z}$ wykresu, podaj jej interpretację

graficzną.

4. Wykresem funkcji $f(x) =x^{2}+bx+c$ jest parabola $0$ wierzcholku $\mathrm{w}$ punkcie $(3,-1).$

Podaj wzór funkcji, której wykres jest obrazem symetrycznym tej paraboli:

a) względem prostej $x=1,$

b) względem punktu $($1, $0).$

$\mathrm{s}_{\mathrm{P}^{\mathrm{o}\mathrm{r}\mathrm{z}}\Phi^{\mathrm{d}\acute{\mathrm{z}}}}$ staranne wykresy wszystkich funkcji.

5. Oblicz

2$\sin^{3}\alpha+3\sin\alpha\cos^{2}\alpha$

$\sin\alpha\sqrt{\cos\alpha}+\cos\alpha\sqrt{\sin\alpha}'$

wiedząc, $\dot{\mathrm{z}}\mathrm{e}$ tg $\displaystyle \alpha=\frac{1}{2}$. Wynik podaj bez niewymierności $\mathrm{w}$ mianowniku.

6. $\mathrm{Z}$ miejscowości $A\mathrm{i}B$ odległych $090$ kilometrów wyruszyli dwaj rowerzyści. Adam wy-

jechaf $\mathrm{z} A0$ godzinę wcześniej $\mathrm{n}\mathrm{i}\dot{\mathrm{z}}$ Bartek $\mathrm{z} B$. Od momentu spotkania Adam jechaf

do $B90$ minut, a Bartek dotarl do $A$ po 4 godzinach. $\mathrm{Z}$ jaką prędkością jechał $\mathrm{k}\mathrm{a}\dot{\mathrm{z}}\mathrm{d}\mathrm{y}\mathrm{z}$

rowerzystów?
\end{document}
