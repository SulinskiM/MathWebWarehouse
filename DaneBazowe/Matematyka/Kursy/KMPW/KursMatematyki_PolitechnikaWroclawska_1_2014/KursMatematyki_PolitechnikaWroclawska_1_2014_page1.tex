\documentclass[a4paper,12pt]{article}
\usepackage{latexsym}
\usepackage{amsmath}
\usepackage{amssymb}
\usepackage{graphicx}
\usepackage{wrapfig}
\pagestyle{plain}
\usepackage{fancybox}
\usepackage{bm}

\begin{document}

PRACA KONTROLNA nr l- POZ1OM ROZSZERZONY

l. Wykaz, $\dot{\mathrm{z}}\mathrm{e}$ róznica czwartych potęg dwóch liczb nieparzystych jest podzielna przez 16.

2. 31 grudnia Kowalski zaciągnąf $\mathrm{p}\mathrm{o}\dot{\mathrm{z}}$ yczkę 4000 zfotych oprocentowaną $\mathrm{w}$ wysokości 16\%

$\mathrm{w}$ skali roku. Zobowiązaf się spfacič ją $\mathrm{w}$ ciągu roku $\mathrm{w}$ czterech równych ratach platnych

31 marca, 30 czerwca, 30 września $\mathrm{i}31$ grudnia. Oprocentowanie $\mathrm{p}\mathrm{o}\dot{\mathrm{z}}$ yczki liczy się od

l stycznia, a odsetki od kredytu naliczane $\mathrm{s}\Phi^{\mathrm{W}}$ terminach pfatności rat. Oblicz wysokośč

tych rat $\mathrm{w}$ zaokrągleniu do pelnych groszy.

3. Narysuj wykres funkcji

$f(x)=\displaystyle \frac{|x+1|+x}{|x-1|}$

$\mathrm{i}$ wyznacz zbiór jej wartości. Następnie rozwiąz nierównośč $f(x-1) < x \mathrm{i}$ podaj jej

interpretację graficzną.

4. Dla jakich wartości parametru rzeczywistego $m$ równanie kwadratowe

$2x^{2}-mx+m+2=0$

ma dwa pierwiastki rzeczywiste $x_{1}, x_{2}$, których suma odwrotności jest nieujemna? Spo-

rząd $\acute{\mathrm{z}}$ wykres funkcji $f(m)=\displaystyle \frac{1}{x_{1}}+\frac{1}{x_{2}}.$

5. Odcinek $0$ końcach $A(\displaystyle \frac{5}{2},\frac{\sqrt{3}}{2}), B(\displaystyle \frac{5}{2},\frac{3\sqrt{3}}{2})$ jest bokiem $\mathrm{w}\mathrm{i}\mathrm{e}\mathrm{l}\mathrm{o}\mathrm{k}_{\Phi}\mathrm{t}\mathrm{a}$ foremnego wpi-

sanego $\mathrm{w}$ okrąg styczny do osi $Ox$. Wyznacz równanie tego okręgu $\mathrm{i}$ wspólrzędne pozo-

stałych wierzchofków $\mathrm{w}\mathrm{i}\mathrm{e}\mathrm{l}\mathrm{o}\mathrm{k}_{\Phi}\mathrm{t}\mathrm{a}$. Ile rozwi$\mathfrak{B}$ań ma to zadanie? Sporząd $\acute{\mathrm{z}}$ rysunek.

6. $\mathrm{Z}$ wierzcholków podstawy $AB$ trójkąta równobocznego $0$ boku $\alpha$ rozpoczęfy ruch dwa

punkty. Poruszajq się one wzdłuz boków $AC\mathrm{i}BC\mathrm{z}$ prędkościami odpowiednio $v_{1}\mathrm{i}v_{2}.$

Po jakim czasie odlegfośč między nimi będzie równa wysokości trójkąta?

Rozwiązania (rękopis) zadań z wybranego poziomu prosimy nadsylač do

na adres:

28 września 20l4r.

Instytut Matematyki $\mathrm{i}$ Informatyki

Politechniki Wrocfawskiej

Wybrzez $\mathrm{e}$ Wyspiańskiego 27

$50-370$ WROCLAW.

Na kopercie prosimy koniecznie zaznaczyč wybrany poziom! (np. poziom podsta-

wowy lub rozszerzony). Do rozwiązań nalez $\mathrm{y}$ dołączyč zaadresowaną do siebie koperte

zwrotną $\mathrm{z}$ naklejonym znaczkiem, odpowiednim do wagi listu. Prace niespelniające po-

danych warunków nie bedą poprawiane ani odsyłane.

Adres internetowy Kursu: http://www. im.pwr.edu.pl/kurs
\end{document}
