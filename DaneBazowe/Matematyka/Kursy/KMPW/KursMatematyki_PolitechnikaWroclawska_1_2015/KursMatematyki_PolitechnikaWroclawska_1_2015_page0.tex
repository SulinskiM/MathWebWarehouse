\documentclass[a4paper,12pt]{article}
\usepackage{latexsym}
\usepackage{amsmath}
\usepackage{amssymb}
\usepackage{graphicx}
\usepackage{wrapfig}
\pagestyle{plain}
\usepackage{fancybox}
\usepackage{bm}

\begin{document}

XLV

KORESPONDENCYJNY KURS

Z MATEMATYKI

wrzesień 2015 r.

PRACA KONTROLNA $\mathrm{n}\mathrm{r} 1 -$ POZIOM PODSTAWOWY

l. Dla pewnego kąta ostrego $\alpha$ zachodzi równośč $\cos\alpha =  2\sin\alpha$. Wyznaczyč wartości

wszystkich funkcji trygonometrycznych tego $\mathrm{k}_{\Phi^{\mathrm{t}\mathrm{a}}}.$

2. Po modernizacji linii kolejowej łączącej Walbrzych $\mathrm{z}$ Wrocławiem średnia prędkośč po-

ciągu wzrosla $014\mathrm{k}\mathrm{m}/\mathrm{h}$, a czas przejazdu 70 km skrócił się $025$ minut. $\mathrm{Z}$ jaką średnią

prędkością jedzie teraz pociąg na tej linii?

3. Wyznaczyč dziedzinę oraz najmniejszą wartośč funkcji

$f(x)=\displaystyle \frac{1}{\sqrt{10+8x^{2}-x^{4}}}.$

4. Wyznaczyč wzory tych funkcji kwadratowych $f(x)=ax^{2}+bx+c$, dla których najmniej-

szą wartości$\Phi$ jest - $\displaystyle \frac{9}{2}, f(0) =-4$, a jednym $\mathrm{z}$ miejsc zerowych jest $x=4$. Narysowač

wykresy tych funkcji.

5. Uprościč wyrazenie (dla tych $a, b$, dla których ma ono sens)

$(\displaystyle \frac{1}{b}+\frac{2}{\sqrt[6]{a^{2}b^{3}}}+\frac{1}{\sqrt[3]{a^{2}}})(\sqrt[3]{a^{2}}(\sqrt[3]{a}+\sqrt{b})-\frac{a(2\sqrt{b}+\sqrt[3]{a})}{\sqrt[3]{a}+\sqrt{b}})$

Następnie obliczyč jego wartośč dla $a=5\sqrt{5}\mathrm{i}b=14-6\sqrt{5}.$

6. Dane są zbiory $A=\{(x,y):4|x|-4\leq 2|y|\leq|x|+2\}$ oraz $B=\displaystyle \{(x,y):|x|+|y|\leq\frac{5}{2}\}.$

Obliczyč pole zbioru $A\cap B$. Wykonač staranny rysunek.
\end{document}
