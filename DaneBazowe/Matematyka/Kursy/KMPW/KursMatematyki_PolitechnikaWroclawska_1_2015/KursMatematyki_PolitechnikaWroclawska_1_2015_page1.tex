\documentclass[a4paper,12pt]{article}
\usepackage{latexsym}
\usepackage{amsmath}
\usepackage{amssymb}
\usepackage{graphicx}
\usepackage{wrapfig}
\pagestyle{plain}
\usepackage{fancybox}
\usepackage{bm}

\begin{document}

PRACA KONTROLNA nr l- POZ1OM ROZSZERZONY

1. $\mathrm{W}\mathrm{i}\mathrm{e}\mathrm{d}\mathrm{z}\Phi^{\mathrm{C}}, \dot{\mathrm{z}}\mathrm{e}$ dla wypukłego $\mathrm{k}_{\Phi^{\mathrm{t}\mathrm{a}}} \alpha$ zachodzi równośč $\cos\alpha-\sin\alpha = \displaystyle \frac{1}{3}$, wyznaczyč

wszystkie funkcje trygonometryczne tego kąta.

2. Dlajakich wartości parametru $p$ suma kwadratów pierwiastków trójmianu $px^{2}-2px+2$

jest większa od 3?

3. Cięzarówka $0$ długości $16\mathrm{m}$ jedzie ze stałą prędkością $70\mathrm{k}\mathrm{m}/\mathrm{h}$. Wyprzedza ją samo-

chód osobowy $0$ dfugości $4\mathrm{m}$ jadąc ze stafą prędkości$\Phi 100\mathrm{k}\mathrm{m}/\mathrm{h}$. Manewr wyprzedzania

rozpoczyna od zjazdu na lewy pas dokładnie $20\mathrm{m}$ za $\mathrm{c}\mathrm{i}\mathrm{e}\dot{\mathrm{z}}$ arówką, a kończy, powraca-

jqc na prawy pas jezdni dokładnie $20\mathrm{m}$ przed $\mathrm{n}\mathrm{i}\mathrm{a}$ (odstęp między pojazdami wynosi $\mathrm{w}$

tych momentach $20\mathrm{m}$). $\mathrm{Z}$ naprzeciwka nadjez $\mathrm{d}\dot{\mathrm{z}}$ a inny samochód osobowy $\mathrm{z}$ prędkością

$105\mathrm{k}\mathrm{m}/\mathrm{h}$. Jaka powinna byč odległośč między oboma samochodami osobowymi na po-

czątku manewru wyprzedzania, $\dot{\mathrm{z}}$ eby zakończyf się on bezpiecznie (bez zmiany prędkości

obu samochodów)?

4. Narysowač wykres funkcji

$f(x)=$

dla

dla

$x\leq 1,$

$x>1.$

Posfugując się nim, podač wzór funkcji $g(m)$ określającej liczbe rozwiązań równania

$f(x)=m$, gdzie $m$ jest parametrem rzeczywistym.

5. Uprościč wyrazenie (dla tych $a, b$, dla których ma ono sens)

$(\displaystyle \frac{\sqrt[4]{a}}{\sqrt{b}}-\frac{b}{\sqrt{a}}+\frac{3\sqrt{b}}{\sqrt[4]{a}}-3)(\sqrt[4]{ab^{2}}-b+\frac{2b\sqrt[4]{\alpha}-\sqrt{b^{3}}}{\sqrt[4]{a}-\sqrt{b}})$

Następnie obliczyč jego wartośč dla $a=28-16\sqrt{3}\mathrm{i}b=3.$

6. Dane są zbiory $A=\{(x,y):x^{2}+y^{2}<16\}$ oraz $B=\{(x,y):x^{2}+y^{2}<4||x|-|y||\}.$

Narysowač zbiór $A\backslash B$ oraz obliczyč jego pole.

Rozwiązania (rękopis) zadań z wybranego poziomu prosimy nadsyfač do

na adres:

28 września 20l5r.

Katedra Matematyki WPPT

Politechniki Wrocfawskiej

Wybrzez $\mathrm{e}$ Wyspiańskiego 27

$50-370$ WROCLAW.

Na kopercie prosimy $\underline{\mathrm{k}\mathrm{o}\mathrm{n}\mathrm{i}\mathrm{e}\mathrm{c}\mathrm{z}\mathrm{n}\mathrm{i}\mathrm{e}}$ zaznaczyč wybrany poziom! (np. poziom podsta-

wowy lub rozszerzony). Do rozwiązań nalez $\mathrm{y}$ dołaczyč zaadresowana do siebie kopertę

zwrotną $\mathrm{z}$ naklejonym znaczkiem, odpowiednim do wagi listu. Prace niespelniające po-

danych warunków nie będą poprawiane ani odsyłane.

Adres internetowy Kursu: http://www.im.pwr.wroc.pl/kurs
\end{document}
