\documentclass[a4paper,12pt]{article}
\usepackage{latexsym}
\usepackage{amsmath}
\usepackage{amssymb}
\usepackage{graphicx}
\usepackage{wrapfig}
\pagestyle{plain}
\usepackage{fancybox}
\usepackage{bm}

\begin{document}

XLVI

KORESPONDENCYJNY KURS

Z MATEMATYKI

wrzesień 2016 r.

PRACA KONTROLNA $\mathrm{n}\mathrm{r} 1-$ POZIOM PODSTAWOWY

1. $\mathrm{Z}$ miast A $\mathrm{i}\mathrm{B}$ odległych $0700$ km $0$ tej samej godzinie wyruszajq naprzeciw siebie (po

dwu równoleglych torach) dwa pociągi. Pociąg pospieszny, który wyjezdza $\mathrm{z}\mathrm{B}$, jedzie

$\mathrm{z}$ prędkością $035\mathrm{k}\mathrm{m}/\mathrm{h}$ większą $\mathrm{n}\mathrm{i}\dot{\mathrm{z}}$ wyjez $\mathrm{d}\dot{\mathrm{z}}$ ający $\mathrm{z}$ A pociąg osobowy $\mathrm{i}$ przyjez $\mathrm{d}\dot{\mathrm{z}}$ a do

A godzinę wcześniej $\mathrm{n}\mathrm{i}\dot{\mathrm{z}}$ pociag osobowy osiąga B. $\mathrm{Z}$ jakimi prędkościami poruszają się

pociągi $\mathrm{i}\mathrm{w}$ jakiej odlegfości od A się minęly.

2. Wyznaczyč dziedziny funkcji $f(x)=\sqrt{\frac{|x-1|-4}{x+2}}$ oraz $g(x)=f(x+1) \mathrm{i}h(x)=f(|x|).$

3. Liczby

{\it p}$=$--($\sqrt{}$354-2)(9$\sqrt{}\sqrt{}$334$++$(61$\sqrt{}$3$+$2$\sqrt{}+$34))2-(2-$\sqrt{}$3)3 i {\it q}$=$--$\sqrt{}$6346-314$\sqrt{}\sqrt{}$88$+$(18$+$-31$\sqrt{}\sqrt{}$624)

są miejscami zerowym trójmianu kwadratowego $f(x)=x^{2}+ax+b$. Znalez/č najmniejszą

$\mathrm{i}$ największq wartośč $f(x)$ na przedziale $[0$, 5$].$

4. Niech $f(x) = x^{2}$ Narysowač wykres funkcji $g(x) = |f(x-1) -4| \mathrm{i}$ określič liczbę

rozwiązań równania $g(x)=m\mathrm{w}$ zalezności $0$ parametru $m.$

5. Wykresy funkcji $f(x) = \displaystyle \frac{m-1}{m+2}x+1\mathrm{i}g(x) =\displaystyle \frac{m+2}{m-3}x+1$ są prostymi prostopadlymi.

Obliczyč pole trójkata ograniczonego wykresami tych funkcji $\mathrm{i}$ osią $Ox$. Podač równanie

okręgu opisanego na tym trójkqcie. Sporządzič rysunek.

6. $\mathrm{W}$ kwadrat ABCD wpisano kwadrat EFGH, ktory zajmuje
\begin{center}
\includegraphics[width=29.772mm,height=29.364mm]{./KursMatematyki_PolitechnikaWroclawska_1_2016_page0_images/image001.eps}
\end{center}
{\it D}

{\it G} $C$

{\it H}

{\it F}

{\it A E  B}

34 jego powierzchni. Wyznaczyc wartosci wszystkich funkcji

trygonometrycznych mniejszego $\mathrm{z}\mathrm{k}$ tów trójk ta $EBF.$
\end{document}
