\documentclass[a4paper,12pt]{article}
\usepackage{latexsym}
\usepackage{amsmath}
\usepackage{amssymb}
\usepackage{graphicx}
\usepackage{wrapfig}
\pagestyle{plain}
\usepackage{fancybox}
\usepackage{bm}

\begin{document}

PRACA KONTROLNA nr l- POZ1OM ROZSZERZONY

l. Statek wyrusza ($\mathrm{z}$ biegiem rzeki) $\mathrm{z}$ przystani A do odległej $0 140$ km przystani B. Po

uplywie l godziny wyrusza za nim łódz/ motorowa, dopędza statek $\mathrm{w}$ pofowie drogi,

po czym wraca do przystani A $\mathrm{w}$ tym samym momencie, $\mathrm{w}$ którym statek przybija do

przystani B. Wyznaczyč prędkośč statku $\mathrm{i}$ prędkośč lodzi $\mathrm{w}$ wodzie stojącej, wiedzqc, $\dot{\mathrm{z}}\mathrm{e}$

prędkośč nurtu rzeki wynosi 4 $\mathrm{k}\mathrm{m}/$godz.

2. Narysowač wykres funkcji $f(x)=\displaystyle \min\{x^{3},\frac{1}{x}\}\mathrm{i}$ wyznaczyčjej dziedzinę oraz zbiór warto-

ści. Podač wzór funkcji $h(x)$, której wykres jest symetryczny do wykresu $f(x)$ względem

punktu $(0,0)$. Określič liczbę rozwiązań równania $f(x)=m\mathrm{w}$ zalezności $0$ parametru $m.$

3. Dla jakich wartości rzeczywistego parametru $p$ równanie $(p-1)x^{2}-(p+1)x-1=0$

ma dwa pierwiastki tego samego znaku odległe co najwyzej $01$?

4. Wykresy funkcji $f(x)=(m-1)x+1\displaystyle \mathrm{i}g(x)=\frac{m}{m-1}x+b$ są prostymi prostopadłymi,

a pole trójkata ograniczonego wykresami tych funkcji $\mathrm{i}$ osią $Ox$ jest równe polu trójkąta

ograniczonego tymi wykresami $\mathrm{i}$ osia $Oy$. Wyznaczyč wzory funkcji $f\mathrm{i}g\mathrm{i}$ obliczyč pole

rozwazanych trójkątów. Sporządzič rysunek.

5. Obliczyč wartości

$p=\displaystyle \sqrt{19-8\sqrt{3}}-\sqrt[3]{26-15\sqrt{3}}\mathrm{i}q=\frac{14\log_{9}\frac{1}{2}-\log_{\sqrt[3]{3}}\frac{1}{4}}{\log_{9}8+\log_{\sqrt{3}}\frac{1}{2}}.$

Następnie wyznaczyč wzór $\mathrm{i}$ narysowač wykres funkcji $f(x) =\displaystyle \frac{ax+b}{cx+d}$, wiedząc, $\dot{\mathrm{z}}\mathrm{e}$ jest

on symetryczny względem punktu $(p,q)\mathrm{i}$ przechodzi przez punkt $(0,0).$

6. Punkt $D$ dzieli bok $AB$ trójkąta równobocznego $ABC\mathrm{w}$ stosunku 2:1. Wyznaczyč stosu-

nek dlugości promienia okręgu wpisanego $\mathrm{w}$ trójkąt $ADC$ do dfugości promienia okręgu

wpisanego $\mathrm{w}$ trójkąt $DBC.$

Rozwiązania (rękopis) zadań z wybranego poziomu prosimy nadsyfač do

na adres:

28 września 20l6r.

Wydziaf Matematyki

Politechnika Wrocfawska

Wybrzez $\mathrm{e}$ Wyspiańskiego 27

$50-370$ WROCLAW.

Na kopercie prosimy $\underline{\mathrm{k}\mathrm{o}\mathrm{n}\mathrm{i}\mathrm{e}\mathrm{c}\mathrm{z}\mathrm{n}\mathrm{i}\mathrm{e}}$ zaznaczyč wybrany poziom! (np. poziom podsta-

wowy lub rozszerzony). Do rozwiązań nalez $\mathrm{y}$ dołączyč zaadresowana do siebie kopertę

zwrotną $\mathrm{z}$ naklejonym znaczkiem, odpowiednim do wagi listu. Prace niespelniające po-

danych warunków nie będą poprawiane ani odsyłane.

Adres internetowy Kursu: http: //www. im. pwr. edu. pl/kurs
\end{document}
