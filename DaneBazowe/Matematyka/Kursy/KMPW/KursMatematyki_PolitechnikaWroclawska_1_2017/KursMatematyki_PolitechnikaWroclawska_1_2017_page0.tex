\documentclass[a4paper,12pt]{article}
\usepackage{latexsym}
\usepackage{amsmath}
\usepackage{amssymb}
\usepackage{graphicx}
\usepackage{wrapfig}
\pagestyle{plain}
\usepackage{fancybox}
\usepackage{bm}

\begin{document}

XLVII

KORESPONDENCYJNY KURS

Z MATEMATYKI

wrzesień 2017 r.

PRACA KONTROLNA $\mathrm{n}\mathrm{r} 1 -$ POZIOM PODSTAWOWY

l. Uprościč następujace wyrazenie, określiwszy uprzednio jego dziedzine:

$\displaystyle \frac{1}{\sqrt[6]{a^{3}b^{2}}-\sqrt[6]{b^{5}}}(\sqrt[3]{a^{2}}-\frac{b}{\sqrt[3]{a}})+\frac{1}{\sqrt{a}+\sqrt{b}}$ : $\displaystyle \frac{\sqrt[3]{ab}}{a-b}$

Obliczyč wartośč tego wyrazenia, przyjmując $a=3+2\sqrt{2} \mathrm{i} b=1+\sqrt{2}.$

2. Niech $B$ oznacza dziedzinę funkcji $f(x)=\displaystyle \frac{1}{\sqrt{3+2x-x^{2}}}$, a $A=\displaystyle \{x\in 1\mathrm{R}:\frac{1}{|x^{2}-1|}\geq 4\}.$

Wyznaczyč $\mathrm{i}$ zaznaczyč na osi liczbowej zbiory $A, B, A\cap B, A\cup B$ oraz $(A\backslash B)\cup(B\backslash A).$

3. Podač wzór funkcji kwadratowej, której wykres jest symetrycznym odbiciem wykresu

funkcji $f(x)=x^{2}+2x$ względem: a) prostej $x=1$, b) punktu $(0,0)$, c) punktu $($1, $0).$

Odpowiedz/uzasadnič, przeprowadzając odpowiednie obliczenia. Sporzqdzič staranne wy-

kresy wszystkich rozwazanych funkcji.

4. $\mathrm{W}$ pewnym ciągu arytmetycznym róznica piętnastego $\mathrm{i}$ drugiego wyrazu jest równa 13.

Oblicz $\alpha_{30}-a_{4}$ oraz sumę pierwszych dziesięciu wyrazów $0$ numerach nieparzystych,

wiedząc, $\dot{\mathrm{z}}\mathrm{e}$ suma pierwszych dziesięciu wyrazów $0$ numerach parzystych jest równa 125.

5. Przekqtne trapezu prostokatnego $0$ podstawach 3 $\mathrm{i}4$ przecinają się pod kątem prostym.

Obliczyč obwód $\mathrm{i}$ pole trapezu. Sporz$\Phi$dzič rysunek.

6. Ostrosłup prawidłowy, którego podstawą jest kwadrat $0$ boku $a$, przecięto płaszczyzną

przechodzącą przez wysokośč ostrosfupa $\mathrm{i}$ przekatną podstawy. Pole otrzymanego prze-

kroju jest równe polu podstawy. Wyznaczyč pole powierzchni cafkowitej ostrosfupa oraz

cosinus kąta nachylenia ściany bocznej do podstawy.
\end{document}
