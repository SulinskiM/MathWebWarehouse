\documentclass[a4paper,12pt]{article}
\usepackage{latexsym}
\usepackage{amsmath}
\usepackage{amssymb}
\usepackage{graphicx}
\usepackage{wrapfig}
\pagestyle{plain}
\usepackage{fancybox}
\usepackage{bm}

\begin{document}

PRACA KONTROLNA nr l- POZ1OM ROZSZERZONY

l. Uprościč następujące wyrazenie, określiwszy uprzednio jego dziedzinę:

$\displaystyle \frac{\sqrt[3]{a}-\sqrt[3]{b}}{\sqrt[3]{a^{2}}+\sqrt[3]{ab}+\sqrt[3]{b^{2}}} \displaystyle \frac{a-b}{\sqrt[3]{a^{2}}-\sqrt[3]{b^{2}}} (1+\displaystyle \frac{\sqrt[3]{b}}{\sqrt[3]{a}-\sqrt[3]{b}}-\frac{1+\sqrt[3]{b}}{\sqrt[3]{b}})$ : $\displaystyle \frac{\sqrt[3]{b}(1+\sqrt[3]{b})-\sqrt[3]{a}}{\sqrt[3]{b}}$

Obliczyč wartośč tego wyrazenia dla $a=7+5\sqrt{2} \mathrm{i} b=7-5\sqrt{2}.$

2. Dla jakiego rzeczywistego parametru $m$ równanie

--{\it mm}$+${\it x}1--{\it mx}$=$1$+$-{\it mx}

ma dwa pierwiastki będqce sinusem $\mathrm{i}$ cosinusem kąta $\mathrm{z}$ przedziału $(\displaystyle \frac{\pi}{2},\pi)$ ?

3. Dane są liczby: $m= \displaystyle \frac{\left(\begin{array}{l}
6\\
4
\end{array}\right)\left(\begin{array}{l}
8\\
2
\end{array}\right)}{\left(\begin{array}{l}
7\\
3
\end{array}\right)}, n= \displaystyle \frac{(\sqrt{2})^{-4}(\frac{1}{4})^{-\frac{5}{2}}\sqrt[4]{3}}{(\sqrt[4]{4})^{5}\cdot\sqrt{32}\cdot 27^{-\frac{1}{4}}}$. Wyznaczyč $k$ tak, by liczby

$m, k, n$ byly odpowiednio: pierwszym, drugim $\mathrm{i}$ trzecim wyrazem ciqgu geometrycznego,

a następnie wyznaczyč sumę wszystkich wyrazów nieskończonego ciągu geometrycznego,

którego pierwszymi trzema wyrazami są $m, k, n$. Ile wyrazów tego ciągu nalezy wziąč,

by ich suma przekroczyła 95\% sumy wszystkich wyrazów?

4. Podač wzór funkcji homograficznej, której wykres jest symetrycznym odbiciem wykresu

funkcji $f(x)=\displaystyle \frac{x-1}{x+1}$ względem: a) prostej $x=1$, b) punktu $(0,0)$, c) punktu $($1, $0).$

Odpowiedz/uzasadnič, przeprowadzaj $\Phi^{\mathrm{C}}$ odpowiednie obliczenia. Sporządzič staranne wy-

kresy wszystkich rozwazanych funkcji.

5. $\mathrm{W}$ czworokącie wypukfym ABCD, $\mathrm{w}$ którym $AB= 1, BC = 2, CD = 4, DA = 3,$

cosinus kąta przy wierzchofku $B$ jest równy - $\displaystyle \frac{5}{7}$. Wykazač, $\dot{\mathrm{z}}\mathrm{e}$ czworokąt ten $\mathrm{m}\mathrm{o}\dot{\mathrm{z}}$ na

wpisač $\mathrm{w}$ okrąg $\mathrm{i}$ obliczyč promień $R$ tego okręgu. Sprawdzič, czy $\mathrm{w}$ rozwazany czworokąt

$\mathrm{m}\mathrm{o}\dot{\mathrm{z}}$ na wpisač okrqg. $\mathrm{J}\mathrm{e}\dot{\mathrm{z}}$ eli $\mathrm{t}\mathrm{a}\mathrm{k}$, to obliczyč jego promień.

6. $\mathrm{W}$ ostrosłupie prawidfowym czworokątnym, $\mathrm{w}$ którym wszystkie krawędzie $\mathrm{s}\Phi$ równe,

poprowadzono płaszczyznę przechodzącą przez jedną $\mathrm{z}$ krawędzi podstawy oraz środ-

ki dwu przeciwleglych do niej krawędzi bocznych. Obliczyč stosunek pola powierzchni

przekroju do pola podstawy oraz stosunek objętości brył, najakie płaszczyzna podzieliła

ostrosłup.

Rozwiązania (rękopis) zadań z wybranego poziomu prosimy nadsyłač do

na adres:

28 września 20l7r.

Wydziaf Matematyki

Politechnika Wrocfawska

Wybrzez $\mathrm{e}$ Wyspiańskiego 27

$50-370$ WROCLAW.

Na kopercie prosimy $\underline{\mathrm{k}\mathrm{o}\mathrm{n}\mathrm{i}\mathrm{e}\mathrm{c}\mathrm{z}\mathrm{n}\mathrm{i}\mathrm{e}}$ zaznaczyč wybrany poziom! (np. poziom podsta-

wowy lub rozszerzony). Do rozwiązań nalez $\mathrm{y}$ dołączyč zaadresowaną do siebie kopertę

zwrotną $\mathrm{z}$ naklejonym znaczkiem, odpowiednim do wagi listu. Prace niespelniajace po-

danych warunków nie będą poprawiane ani odsylane.

Adres internetowy Kursu: http: //www. im. pwr. edu. pl/kurs
\end{document}
