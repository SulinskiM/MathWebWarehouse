\documentclass[a4paper,12pt]{article}
\usepackage{latexsym}
\usepackage{amsmath}
\usepackage{amssymb}
\usepackage{graphicx}
\usepackage{wrapfig}
\pagestyle{plain}
\usepackage{fancybox}
\usepackage{bm}

\begin{document}

XLVIII

KORESPONDENCYJNY KURS

Z MATEMATYKI

wrzesień 2018 r.

PRACA KONTROLNA $\mathrm{n}\mathrm{r} 1 -$ POZIOM PODSTAWOWY

l. Promień podstawy stozka obrotowego zmniejszono $0$ 20\%. $\mathrm{O}$ ile procent trzeba zwiększyč

wysokośč tego stozka, $\dot{\mathrm{z}}$ eby jego objętośč nie ulegfa zmianie?

2. Dla jakich wartości parametru $m$ nierównośč

$mx^{2}+(m+1)x+2m<0$

jest spelniona dla wszystkich $x\in \mathbb{R}$?

3. Określič dziedzinę $\mathrm{i}$ uprościč nastepujące wyrazenie:

-(($\sqrt{}\sqrt{}$5{\it aa}$\sqrt{}$3-43{\it a})2{\it b}-)-234:[-($\sqrt{}$4$\sqrt{}$5{\it aa}$\sqrt{}$-{\it b}4)2]3

Następnie obliczyč wartośč tego wyrazenia dla $a=\sqrt{3}+\sqrt{2} \mathrm{i} b=5-2\sqrt{6}.$

4. Niech $f(x) = x^{2}$ Narysowač wykres funkcji $g(x) = |f(x+1) -4| \mathrm{i}$ określič liczbę

rozwiązań równania $g(x)=m\mathrm{w}$ zalezności od parametru $m.$

5. Obliczyč pole koła wpisanego $\mathrm{w}$ romb $0$ polu 10 $\mathrm{i}$ kacie ostrym $30^{\mathrm{o}}$

6. Niech $A = \displaystyle \{x\in \mathbb{R}:\frac{3}{2x^{2}+x-6}\geq\frac{1}{2x-3}\}$ oraz $B = \{x\in \mathbb{R}:\sqrt{x^{2}-4x+4}<x\}.$

Wyznaczyč $\mathrm{i}$ narysowač na osi liczbowej zbiory $A, B$ oraz $A\backslash B, B\backslash A.$
\end{document}
