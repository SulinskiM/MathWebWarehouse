\documentclass[a4paper,12pt]{article}
\usepackage{latexsym}
\usepackage{amsmath}
\usepackage{amssymb}
\usepackage{graphicx}
\usepackage{wrapfig}
\pagestyle{plain}
\usepackage{fancybox}
\usepackage{bm}

\begin{document}

PRACA KONTROLNA nr l- POZ1OM ROZSZERZONY

l. Pewna liczba pieciocyfrowa zaczyna się ($\mathrm{z}$ lewej strony) cyfrą 8. Jeś1i cyfrę tę przestawimy

$\mathrm{z}$ pierwszej pozycji na ostatnią, to otrzymamy liczbę stanowiąca 16\% 1iczby pierwotnej.

Znalez/č tę liczbę.

2. Określič dziedzinę $\mathrm{i}$ uprościč następujące wyrazenie:

$\displaystyle \frac{(\sqrt{a}+\sqrt{b})^{2}-4b}{(a-b)(\sqrt{\frac{1}{b}}+3\sqrt{\frac{1}{a}})^{-1}}$ : $\displaystyle \frac{a+9b+6\sqrt{ab}}{\frac{1}{\sqrt{b}}+\frac{1}{\sqrt{a}}}.$

Następnie wyznaczyč jego wartośč dla $a=\sqrt{4-2\sqrt{3}} \mathrm{i} b=\sqrt{3}+1.$

3. Narysowač wykres funkcji $f(x) = \displaystyle \min\{\frac{2x}{x-1},x^{2}\}$. Podač wzór funkcji, której wykres

jest symetryczny do wykresu funkcji $f(x)$ względem początku ukladu wspófrzędnych.

Określič liczbę rozwiązań równania $f(x)=m\mathrm{w}$ zalezności od parametru $m.$

4. Dfugości boków trójkąta prostokątnego tworzą ciąg arytmetyczny $0$ róznicy $p>0$. Ob-

liczyč stosunek promienia okręgu opisanego na $\mathrm{t}\mathrm{y}\mathrm{m}$ trójkącie do promienia okręgu wpi-

sanego $\mathrm{w}$ ten trójkąt.

5. Dla jakich wartości parametru $m$ suma sześcianów pierwiastków równania

$x^{2}+(m-1)x+m=\displaystyle \frac{7}{4}$

nalez $\mathrm{y}$ do przedzialu $[-\displaystyle \frac{1}{2},0$)?

6. Dane sa zbiory

$A=\{(x,y)\in \mathbb{R}^{2}:9-4\sqrt{2}\leq x^{2}+y^{2}<9+4\sqrt{2}\}$

oraz

$B=\{(x,y)\in \mathbb{R}^{2}:x^{2}+y^{2}<4|x|+4|y|-7\}.$

Narysowač starannie zbiór $A\backslash B \mathrm{i}$ wyznaczyč jego pole. Zadbač $0$ odpowiednią skalę

$\mathrm{i}$ czytelnośč rysunku.

Rozwiązania (rękopis) zadań $\mathrm{z}$ wybranego poziomu prosimy nadsyłač do 28 września $2018\mathrm{r}.$

na adres:

Wydziaf Matematyki

Politechnika Wrocfawska

Wybrzez $\mathrm{e}$ Wyspiańskiego 27

$50-370$ WROCLAW.

Na kopercie prosimy $\underline{\mathrm{k}\mathrm{o}\mathrm{n}\mathrm{i}\mathrm{e}\mathrm{c}\mathrm{z}\mathrm{n}\mathrm{i}\mathrm{e}}$ zaznaczyč wybrany poziom! (np. poziom podsta-

wowy lub rozszerzony). Do rozwiązań nalez $\mathrm{y}$ dołączyč zaadresowaną do siebie koperte

zwrotną $\mathrm{z}$ naklejonym znaczkiem, odpowiednim do wagi listu. Prace niespelniające po-

danych warunków nie będą poprawiane ani odsylane.

Uwaga. Wysylajac nam rozwiązania zadań uczestnik Kursu udostępnia nam swoje dane osobo-

we, które przetwarzamy wyłącznie $\mathrm{w}$ zakresie niezbędnym do jego prowadzenia (odesfanie zadań,

prowadzenie statystyki). Szczególowe informacje $0$ przetwarzaniu przez nas danych osobowych są

dostępne na stronie internetowej Kursu.

Adres internetowy Kursu: http: //www. im. pwr. edu. pl/kurs
\end{document}
