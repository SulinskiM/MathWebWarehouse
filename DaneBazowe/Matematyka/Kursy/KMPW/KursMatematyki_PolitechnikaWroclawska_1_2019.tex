\documentclass[a4paper,12pt]{article}
\usepackage{latexsym}
\usepackage{amsmath}
\usepackage{amssymb}
\usepackage{graphicx}
\usepackage{wrapfig}
\pagestyle{plain}
\usepackage{fancybox}
\usepackage{bm}

\begin{document}

XLIX

KORESPONDENCYJNY KURS

Z MATEMATYKI

wrzesień 2019 r.

PRACA KONTROLNA $\mathrm{n}\mathrm{r} 1 -$ POZIOM PODSTAWOWY

l. Pan Kowalski załozył dwie lokaty, wplacając do banku $\mathrm{w}$ sumie 10120 $\mathrm{z}1$. Pierwsza $\mathrm{z}$ nich

ma oprocentowanie 12\% $\mathrm{w}$ skali roku $\mathrm{z}$ pófroczn$\Phi$ kapitalizacją odsetek, a druga daje 18\%

zysku, przy czym odsetki są naliczane dopiero po roku. Okazało się, $\dot{\mathrm{z}}\mathrm{e}$ na obu kontach

przybyła mu taka sama kwota. Jakie sumy wplacif na $\mathrm{k}\mathrm{a}\dot{\mathrm{z}}$ dą $\mathrm{z}$ lokat ijaki osiagnąf zysk?

Jaki byfby zysk pana Kowalskiego, gdyby na $\mathrm{k}\mathrm{a}\dot{\mathrm{z}}$ dą $\mathrm{z}$ lokat wpfacif tę $\mathrm{s}\mathrm{a}\mathrm{m}\Phi$ sumę 5060

$\mathrm{z}l.$?

2. Niech $A=\displaystyle \{x\in 1\mathrm{R}:\frac{1}{\sqrt{5-x}}\geq\frac{2}{\sqrt{x+1}}\}$ oraz $B=\{x\in 1\mathrm{R}:|x|+|x-1|\geq 3\}.$

Znalez/č $\mathrm{i}$ zaznaczyč na osi liczbowej zbiory $A, B$ oraz $(A\backslash B)\cup(B\backslash A).$

3. Uprościč wyrazenie (dla tych $a, b$, dla których ma ono sens)

( -{\it b}1 $+$ -$\sqrt{}$6 {\it a}22{\it b}3 $+$ -$\sqrt{}$31{\it a}2) : -$\sqrt{}$3 {\it ba}$\sqrt{}$3$+${\it a}2$\sqrt{}${\it b}.

Następnie obliczyč jego wartośč dla $a=5\sqrt{5}\mathrm{i}b=14-6\sqrt{5}.$

4. Odcinek $AB$ jest średnic$\Phi$ okręgu. Styczna $\mathrm{w}$ punkcie $A\mathrm{i}$ prosta, na której $\mathrm{l}\mathrm{e}\dot{\mathrm{z}}\mathrm{y}$ cięciwa

$BC$ przecinają się $\mathrm{w}$ punkcie $P$ odległym od A $04\sqrt{3}$. Wyznaczyč promień okręgu oraz

długośč cięciwy $BC$, wiedzqc, $\dot{\mathrm{z}}\mathrm{e}$ pole trójkata $ABP$ jest równe $8\sqrt{3}.$

5. Pole trójk$\Phi$ta równobocznego $ABX$ zbudowanego na przeciwprostokątnej $AB$ trójk$\Phi$ta

prostokqtnego $ABC$ jest dwa razy większe od pola wyjściowego trójkąta. Niech $D$ będzie

środkiem boku $AB$. Wykazač, $\dot{\mathrm{z}}\mathrm{e}$ trójkąty $ABC\mathrm{i}ADX$ są $\mathrm{p}\mathrm{r}\mathrm{z}\mathrm{y}\mathrm{s}\mathrm{t}\mathrm{a}\mathrm{j}_{\Phi}\mathrm{c}\mathrm{e}.$

6. Pole powierzchni bocznej stozka jest trzy razy większe $\mathrm{n}\mathrm{i}\dot{\mathrm{z}}$ pole jego podstawy. $\mathrm{W}$ stozek

wpisano walec, którego dolna podstawa jest zawarta $\mathrm{w}$ podstawie stozka, a przekrój

plaszczyzną zawierającą oś stozka jest kwadratem. Wyznaczyč stosunek objętości walca

do objętości stozka.




PRACA KONTROLNA nr l- POZ1OM ROZSZERZONY

l. Określič dziedzinę i uprościč nastepujące wyrazenie

$[\displaystyle \frac{y\sqrt[3]{x}}{\sqrt[3]{x}+\sqrt{y}}-\frac{x-y\sqrt{y}}{x+y\sqrt{y}}\frac{y\sqrt[3]{x^{2}}-y\sqrt{y}\sqrt[3]{x}+y^{2}}{\sqrt[3]{x^{2}}-y}]$ : $\displaystyle \frac{y^{2}}{\sqrt[3]{x}+\sqrt{y}}$

Następnie wyznaczyč jego wartośč dla $x=6\sqrt{3}-10 \mathrm{i} y=12-6\sqrt{3}.$

2. Wyznaczyč sinus kąta przy wierzchofku $C\mathrm{w}$ trójkącie równoramiennym, $\mathrm{w}$ którym środ-

kowe ramion $AC\mathrm{i}BC$ przecinają się pod kqtem prostym.

3. Narysowač obszar $D = \{(x,y):|y|\leq x\leq 4-y^{2}\}$. Obliczyč pole kwadratu, którego

boki są równolegfe do osi ukfadu wspófrzędnych, a wszystkie wierzchofki $\mathrm{l}\mathrm{e}\dot{\mathrm{z}}$ ą $\mathrm{n}\mathrm{a}$ krzywej

ograniczającej obszar $D.$

4. $\mathrm{W}$ trójkącie $ABC$ dane są: $|BC|=a, |AB|=c, \angle ABC=\beta$. Okrąg $\mathrm{P}^{\mathrm{r}\mathrm{z}\mathrm{e}\mathrm{c}\mathrm{h}\mathrm{o}\mathrm{d}\mathrm{z}}\Phi^{\mathrm{c}\mathrm{y}}$ przez

punkty $B\mathrm{i}C$ przecina boki AB $\mathrm{i}AC\mathrm{w}$ takich punktach $D\mathrm{i}E, \dot{\mathrm{z}}\mathrm{e}$ pole czworokąta

BCDE stanowi 75\% po1a trójkąta $ABC$. Wyznaczyč obwód $\mathrm{i}$ pole czworokąta.

5. Basen $\mathrm{m}\mathrm{o}\dot{\mathrm{z}}$ na napefnič, $\mathrm{o}\mathrm{t}\mathrm{w}\mathrm{i}\mathrm{e}\mathrm{r}\mathrm{a}\mathrm{j}_{\Phi}\mathrm{c}$ którykolwiek $\mathrm{z}$ trzech zaworów. Otwarcie pierwszych

$\mathrm{d}\mathrm{w}\mathrm{u}$ pozwala napelnič basen $\mathrm{w}$ czasie $02$ godziny dłuzszym $\mathrm{n}\mathrm{i}\dot{\mathrm{z}}$ otwarcie drugiego $\mathrm{i}$ trze-

ciego zaworu, natomiast otwarcie zaworów pierwszego $\mathrm{i}$ trzeciego pozwala napefnič basen

$\mathrm{w}$ czasie $\mathrm{d}\mathrm{w}\mathrm{a}$ razy krótszym $\mathrm{n}\mathrm{i}\dot{\mathrm{z}}$ otwarcie $\mathrm{d}\mathrm{w}\mathrm{u}$ pierwszych. Napełnienie basenu, gdy otwar-

te są wszystkie trzy zawory, trwa 2 godziny 40 minut. I1e trwa napełnienie basenu, gdy

otwarty jest tylko jeden zawór?

6. $\mathrm{W}$ ostrosłupie prawidlowym czworokątnym przekrój plaszczyzną przechodzącą przez

wierzchofek ostrosfupa $\mathrm{i}$ środki dwu przeciwlegfych krawędzi podstawy jest trójkątem

równobocznym. Ostrosfup przecięto pfaszczyzną przechodzącą przez jedną $\mathrm{z}$ krawędzi

podstawy prostopadła do przeciwleglej ściany bocznej. Obliczyč stosunek objętości brył,

$\mathrm{n}\mathrm{a}$ jakie pfaszczyzna ta dzieli ostroslup.

Rozwiązania (rękopis) zadań z wybranego poziomu prosimy nadsyłač do

na adres:

28 września 20l9r.

Wydziaf Matematyki

Politechnika Wrocfawska

Wybrzez $\mathrm{e}$ Wyspiańskiego 27

$50-370$ WROCLAW.

Na kopercie prosimy $\underline{\mathrm{k}\mathrm{o}\mathrm{n}\mathrm{i}\mathrm{e}\mathrm{c}\mathrm{z}\mathrm{n}\mathrm{i}\mathrm{e}}$ zaznaczyč wybrany poziom! (np. poziom podsta-

wowy lub rozszerzony). Do rozwiązań nalez $\mathrm{y}$ dołączyč zaadresowaną do siebie koperte

zwrotną $\mathrm{z}$ naklejonym znaczkiem, odpowiednim do wagi listu. Prace niespełniające po-

danych warunków nie będą poprawiane ani odsyłane.

Uwaga. Wysylając nam rozwiazania zadań uczestnik Kursu udostępnia Politechnice Wrocfawskiej

swoje dane osobowe, które przetwarzamy wyłącznie $\mathrm{w}$ zakresie niezbędnym do jego prowadzenia

(odesłanie zadań, prowadzenie statystyki). Szczegófowe informacje $0$ przetwarzaniu przez nas danych

osobowych są dostępne na stronie internetowej Kursu.

Adres internetowy Kursu: http: //www. im. pwr. edu. pl/kurs



\end{document}