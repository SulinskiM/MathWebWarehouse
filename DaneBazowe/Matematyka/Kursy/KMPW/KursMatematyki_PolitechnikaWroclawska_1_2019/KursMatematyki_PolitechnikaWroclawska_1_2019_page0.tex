\documentclass[a4paper,12pt]{article}
\usepackage{latexsym}
\usepackage{amsmath}
\usepackage{amssymb}
\usepackage{graphicx}
\usepackage{wrapfig}
\pagestyle{plain}
\usepackage{fancybox}
\usepackage{bm}

\begin{document}

XLIX

KORESPONDENCYJNY KURS

Z MATEMATYKI

wrzesień 2019 r.

PRACA KONTROLNA $\mathrm{n}\mathrm{r} 1 -$ POZIOM PODSTAWOWY

l. Pan Kowalski załozył dwie lokaty, wplacając do banku $\mathrm{w}$ sumie 10120 $\mathrm{z}1$. Pierwsza $\mathrm{z}$ nich

ma oprocentowanie 12\% $\mathrm{w}$ skali roku $\mathrm{z}$ pófroczn$\Phi$ kapitalizacją odsetek, a druga daje 18\%

zysku, przy czym odsetki są naliczane dopiero po roku. Okazało się, $\dot{\mathrm{z}}\mathrm{e}$ na obu kontach

przybyła mu taka sama kwota. Jakie sumy wplacif na $\mathrm{k}\mathrm{a}\dot{\mathrm{z}}$ dą $\mathrm{z}$ lokat ijaki osiagnąf zysk?

Jaki byfby zysk pana Kowalskiego, gdyby na $\mathrm{k}\mathrm{a}\dot{\mathrm{z}}$ dą $\mathrm{z}$ lokat wpfacif tę $\mathrm{s}\mathrm{a}\mathrm{m}\Phi$ sumę 5060

$\mathrm{z}l.$?

2. Niech $A=\displaystyle \{x\in 1\mathrm{R}:\frac{1}{\sqrt{5-x}}\geq\frac{2}{\sqrt{x+1}}\}$ oraz $B=\{x\in 1\mathrm{R}:|x|+|x-1|\geq 3\}.$

Znalez/č $\mathrm{i}$ zaznaczyč na osi liczbowej zbiory $A, B$ oraz $(A\backslash B)\cup(B\backslash A).$

3. Uprościč wyrazenie (dla tych $a, b$, dla których ma ono sens)

( -{\it b}1 $+$ -$\sqrt{}$6 {\it a}22{\it b}3 $+$ -$\sqrt{}$31{\it a}2) : -$\sqrt{}$3 {\it ba}$\sqrt{}$3$+${\it a}2$\sqrt{}${\it b}.

Następnie obliczyč jego wartośč dla $a=5\sqrt{5}\mathrm{i}b=14-6\sqrt{5}.$

4. Odcinek $AB$ jest średnic$\Phi$ okręgu. Styczna $\mathrm{w}$ punkcie $A\mathrm{i}$ prosta, na której $\mathrm{l}\mathrm{e}\dot{\mathrm{z}}\mathrm{y}$ cięciwa

$BC$ przecinają się $\mathrm{w}$ punkcie $P$ odległym od A $04\sqrt{3}$. Wyznaczyč promień okręgu oraz

długośč cięciwy $BC$, wiedzqc, $\dot{\mathrm{z}}\mathrm{e}$ pole trójkata $ABP$ jest równe $8\sqrt{3}.$

5. Pole trójk$\Phi$ta równobocznego $ABX$ zbudowanego na przeciwprostokątnej $AB$ trójk$\Phi$ta

prostokqtnego $ABC$ jest dwa razy większe od pola wyjściowego trójkąta. Niech $D$ będzie

środkiem boku $AB$. Wykazač, $\dot{\mathrm{z}}\mathrm{e}$ trójkąty $ABC\mathrm{i}ADX$ są $\mathrm{p}\mathrm{r}\mathrm{z}\mathrm{y}\mathrm{s}\mathrm{t}\mathrm{a}\mathrm{j}_{\Phi}\mathrm{c}\mathrm{e}.$

6. Pole powierzchni bocznej stozka jest trzy razy większe $\mathrm{n}\mathrm{i}\dot{\mathrm{z}}$ pole jego podstawy. $\mathrm{W}$ stozek

wpisano walec, którego dolna podstawa jest zawarta $\mathrm{w}$ podstawie stozka, a przekrój

plaszczyzną zawierającą oś stozka jest kwadratem. Wyznaczyč stosunek objętości walca

do objętości stozka.
\end{document}
