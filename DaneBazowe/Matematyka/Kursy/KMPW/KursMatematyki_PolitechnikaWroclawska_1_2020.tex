\documentclass[a4paper,12pt]{article}
\usepackage{latexsym}
\usepackage{amsmath}
\usepackage{amssymb}
\usepackage{graphicx}
\usepackage{wrapfig}
\pagestyle{plain}
\usepackage{fancybox}
\usepackage{bm}

\begin{document}

L

KORESPONDENCYJNY KURS

Z MATEMATYKI

wrzesień 2020 r.

PRACA KONTROLNA $\mathrm{n}\mathrm{r} 1 -$ POZIOM PODSTAWOWY

1. $\mathrm{W}$ pierwszym naczyniu było $\alpha$ litrów $p$-procentowego kwasu siarkowego, $\mathrm{w}$ drugim na-

tomiast $b$ litrów $q$-procentowego kwasu siarkowego. $\mathrm{Z}\mathrm{k}\mathrm{a}\dot{\mathrm{z}}$ dego $\mathrm{z}$ naczyń odlano czwartą

częśč objętości roztworu, a następnie roztwór odlany $\mathrm{z}$ drugiego naczynia wlano do pierw-

szego, a odlany $\mathrm{z}$ pierwszego wlano do drugiego naczynia. Okazafo się, $\dot{\mathrm{z}}\mathrm{e}$ po wymieszaniu

stęzenia roztworów $\mathrm{w}$ obu naczyniach byly równe. Wyznacz stosunek stęzeń wyjściowych

roztworów.

2. Uprośč następujące wyrazenie, określiwszy uprzednio jego dziedzinę:

$\displaystyle \frac{1}{\sqrt[6]{x^{3}y^{2}}-\sqrt[6]{y^{5}}}(\sqrt[3]{x^{2}}-\frac{y}{\sqrt[3]{x}})+\frac{1}{\sqrt{x}+\sqrt{y}}$ : $\displaystyle \frac{\sqrt[3]{xy}}{x-y}$

Oblicz wartośč tego wyrazenia, przyjmując $x=3+2\sqrt{2} \mathrm{i} y=1+\sqrt{2}.$

3. Narysuj wykres funkcji $f(x)=(\displaystyle \sin x+\frac{1}{2}\cos x)^{2}+(\frac{1}{2}\sin x+\cos x)^{2}$

wartości $\mathrm{i}$ rozwiqz nierównośč $f(x)\displaystyle \geq\frac{5}{4}.$

Wyznacz zbiór jej

4. Niech $A=\{(x,y)\in \mathbb{R}^{2}:|x|\leq 2,|y|\leq 2\}$ oraz $B=\{(x,y)\in \mathbb{R}^{2}$ :

Zaznacz na płaszczyz$\acute{}$nie zbiory $A\backslash B$ oraz $A\backslash (A\backslash B).$

$|x-y|\leq|x|+1\}.$

5. $\mathrm{W}$ kwadrat wpisano trójkąt równoboczny $\mathrm{w}$ taki sposób, $\dot{\mathrm{z}}\mathrm{e}$ jeden $\mathrm{z}$ jego wierzchofków

jest $\mathrm{w}$ wierzchołku kwadratu, a dwa pozostałe lezą na przeciwległych bokach kwadratu.

Wyznacz stosunek pola trójkąta do pola kwadratu.

6. $\mathrm{W}$ ostrosfupie prawidlowym trójkątnym podstawa ma dfugośč $a$, a krawęd $\acute{\mathrm{z}}$ boczna jest

do niej nachylona pod kątem $\alpha$. Oblicz objętośč $\mathrm{i}$ pole powierzchni bocznej bryły.




PRACA KONTROLNA nrl -P0Zi0M R0ZSZERZ0NY

1. $\mathrm{W}$ pierwszym naczyniu było $a$ litrów $p$-procentowego kwasu siarkowego, $\mathrm{w}$ drugim nato-

miast $b$ litrów $q$-procentowego kwasu siarkowego. $\mathrm{Z}$ obu naczyń odlano równe objętości

roztworów, a następnie roztwór odlany $\mathrm{z}$ drugiego naczynia wlano do pierwszego, a od-

lany $\mathrm{z}$ pierwszego wlano do drugiego naczynia. Okazało się, $\dot{\mathrm{z}}\mathrm{e}$ po wymieszaniu stęzenia

roztworów $\mathrm{w}$ obu naczyniach byfy równe. Jakie ilości roztworów odlano $\mathrm{z}\mathrm{k}\mathrm{a}\dot{\mathrm{z}}$ dego $\mathrm{z}$ na-

czyń?

2. Uprośč wyrazenie (dla tych $x, y$, dla których ma ono sens)

$(\displaystyle \frac{1}{\sqrt[3]{x}-\sqrt[3]{y}}-\frac{3\sqrt[3]{xy}}{x-y}-\frac{\sqrt[3]{y}-\sqrt[3]{x}}{\sqrt[3]{x^{2}}+\sqrt[3]{xy}+\sqrt[3]{y^{2}}})\frac{x-y}{4\sqrt[3]{xy}}.$

Następnie oblicz jego wartośč dla $x=5\sqrt{2}-7\mathrm{i}y=5\sqrt{2}+7.$

3. Narysuj wykres funkcji $f(x)=\sin^{2}x+\sin x\cos x$. Wyznacz zbiór jej wartości $\mathrm{i}$ rozwiąz

nierównośč $f(x)\geq 1.$

4. Niech $A=\{(x,y)\in \mathbb{R}^{2}:|x-1|+|y-1|\leq 3\}$ oraz $B=\{(x,y)\in \mathbb{R}^{2}$ :

Zaznacz na pfaszczy $\acute{\mathrm{z}}\mathrm{n}\mathrm{i}\mathrm{e}$ zbiór $A\cap B\mathrm{i}$ oblicz jego pole.

$|x-y|\leq|x+y|\}.$

5. $\mathrm{W}$ {\it romb ABCD} $0$ boku $a\mathrm{i}$ kącie ostrym $\alpha$ wpisano trójkąt $APQ\mathrm{t}\mathrm{a}\mathrm{k}, \dot{\mathrm{z}}\mathrm{e}$ punkt $P\mathrm{l}\mathrm{e}\dot{\mathrm{z}}\mathrm{y}$

na boku $BC$ a punkt $Q$ na boku $DC$, przy czym $|PC|=|DQ|=x$. Dla jakiego $x$ pole

trójkąta jest najmniejsze?

6. $\mathrm{W}$ ostrosłupie prawidłowym trójkątnym ściana boczna jest nachylona do podstawy pod

$\mathrm{k}_{\Phi}\mathrm{t}\mathrm{e}\mathrm{m}\alpha$. Wyznacz kąt między ścianami bocznymi.

Rozwiązania (rękopis) zadań z wybranego poziomu prosimy nadsyfač do

na adres:

28 września 2020r.

Wydziaf Matematyki

Politechnika Wrocfawska

Wybrzez $\mathrm{e}$ Wyspiańskiego 27

$50-370$ WROCLAW.

Na kopercie prosimy $\underline{\mathrm{k}\mathrm{o}\mathrm{n}\mathrm{i}\mathrm{e}\mathrm{c}\mathrm{z}\mathrm{n}\mathrm{i}\mathrm{e}}$ zaznaczyč wybrany poziom! (np. poziom podsta-

wowy lub rozszerzony). Do rozwiązań nalez $\mathrm{y}$ dołaczyč zaadresowaną do siebie kopertę

zwrotną $\mathrm{z}$ naklejonym znaczkiem, odpowiednim do formatu listu. Polecamy stosowanie

kopert formatu C5 $(160\mathrm{x}230\mathrm{m}\mathrm{m})$ ze znaczkiem $0$ wartości 3,30 zł. Na $\mathrm{k}\mathrm{a}\dot{\mathrm{z}}$ dą większą

kopertę nalez $\mathrm{y}$ nakleič $\mathrm{d}\mathrm{r}\mathrm{o}\dot{\mathrm{z}}$ szy znaczek. Prace niespełniające podanych warunków nie

bedą poprawiane ani odsyłane.

Uwaga. Wysyfajac nam rozwiazania zadań uczestnik Kursu udostępnia Politechnice Wrocfawskiej

swoje dane osobowe, które przetwarzamy wyłącznie $\mathrm{w}$ zakresie niezbędnym do jego prowadzenia

(odeslanie zadań, prowadzenie statystyki). Szczegófowe informacje $0$ przetwarzaniu przez nas danych

osobowych $\mathrm{S}\otimes$ dostępne na stronie internetowej Kursu.

Adres internetowy Kursu: http: //www. im. pwr. edu. pl/kurs



\end{document}