\documentclass[a4paper,12pt]{article}
\usepackage{latexsym}
\usepackage{amsmath}
\usepackage{amssymb}
\usepackage{graphicx}
\usepackage{wrapfig}
\pagestyle{plain}
\usepackage{fancybox}
\usepackage{bm}

\begin{document}

L

KORESPONDENCYJNY KURS

Z MATEMATYKI

wrzesień 2020 r.

PRACA KONTROLNA $\mathrm{n}\mathrm{r} 1 -$ POZIOM PODSTAWOWY

1. $\mathrm{W}$ pierwszym naczyniu było $\alpha$ litrów $p$-procentowego kwasu siarkowego, $\mathrm{w}$ drugim na-

tomiast $b$ litrów $q$-procentowego kwasu siarkowego. $\mathrm{Z}\mathrm{k}\mathrm{a}\dot{\mathrm{z}}$ dego $\mathrm{z}$ naczyń odlano czwartą

częśč objętości roztworu, a następnie roztwór odlany $\mathrm{z}$ drugiego naczynia wlano do pierw-

szego, a odlany $\mathrm{z}$ pierwszego wlano do drugiego naczynia. Okazafo się, $\dot{\mathrm{z}}\mathrm{e}$ po wymieszaniu

stęzenia roztworów $\mathrm{w}$ obu naczyniach byly równe. Wyznacz stosunek stęzeń wyjściowych

roztworów.

2. Uprośč następujące wyrazenie, określiwszy uprzednio jego dziedzinę:

$\displaystyle \frac{1}{\sqrt[6]{x^{3}y^{2}}-\sqrt[6]{y^{5}}}(\sqrt[3]{x^{2}}-\frac{y}{\sqrt[3]{x}})+\frac{1}{\sqrt{x}+\sqrt{y}}$ : $\displaystyle \frac{\sqrt[3]{xy}}{x-y}$

Oblicz wartośč tego wyrazenia, przyjmując $x=3+2\sqrt{2} \mathrm{i} y=1+\sqrt{2}.$

3. Narysuj wykres funkcji $f(x)=(\displaystyle \sin x+\frac{1}{2}\cos x)^{2}+(\frac{1}{2}\sin x+\cos x)^{2}$

wartości $\mathrm{i}$ rozwiqz nierównośč $f(x)\displaystyle \geq\frac{5}{4}.$

Wyznacz zbiór jej

4. Niech $A=\{(x,y)\in \mathbb{R}^{2}:|x|\leq 2,|y|\leq 2\}$ oraz $B=\{(x,y)\in \mathbb{R}^{2}$ :

Zaznacz na płaszczyz$\acute{}$nie zbiory $A\backslash B$ oraz $A\backslash (A\backslash B).$

$|x-y|\leq|x|+1\}.$

5. $\mathrm{W}$ kwadrat wpisano trójkąt równoboczny $\mathrm{w}$ taki sposób, $\dot{\mathrm{z}}\mathrm{e}$ jeden $\mathrm{z}$ jego wierzchofków

jest $\mathrm{w}$ wierzchołku kwadratu, a dwa pozostałe lezą na przeciwległych bokach kwadratu.

Wyznacz stosunek pola trójkąta do pola kwadratu.

6. $\mathrm{W}$ ostrosfupie prawidlowym trójkątnym podstawa ma dfugośč $a$, a krawęd $\acute{\mathrm{z}}$ boczna jest

do niej nachylona pod kątem $\alpha$. Oblicz objętośč $\mathrm{i}$ pole powierzchni bocznej bryły.
\end{document}
