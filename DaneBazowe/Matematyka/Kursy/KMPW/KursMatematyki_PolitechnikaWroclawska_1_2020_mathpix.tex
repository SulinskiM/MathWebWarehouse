\documentclass[10pt]{article}
\usepackage[polish]{babel}
\usepackage[utf8]{inputenc}
\usepackage[T1]{fontenc}
\usepackage{amsmath}
\usepackage{amsfonts}
\usepackage{amssymb}
\usepackage[version=4]{mhchem}
\usepackage{stmaryrd}
\usepackage{bbold}
\usepackage{hyperref}
\hypersetup{colorlinks=true, linkcolor=blue, filecolor=magenta, urlcolor=cyan,}
\urlstyle{same}

\title{PRACA KONTROLNA nr 1 - POZIOM PODSTAWOWY }

\author{}
\date{}


\begin{document}
\maketitle
\begin{enumerate}
  \item W pierwszym naczyniu było $a$ litrów $p$-procentowego kwasu siarkowego, w drugim natomiast $b$ litrów $q$-procentowego kwasu siarkowego. Z każdego z naczyń odlano czwartą część objętości roztworu, a następnie roztwór odlany z drugiego naczynia wlano do pierwszego, a odlany z pierwszego wlano do drugiego naczynia. Okazało się, że po wymieszaniu stężenia roztworów w obu naczyniach były równe. Wyznacz stosunek stężeń wyjściowych roztworów.
  \item Uprość następujące wyrażenie, określiwszy uprzednio jego dziedzinę:
\end{enumerate}

$$
\frac{1}{\sqrt[6]{x^{3} y^{2}}-\sqrt[6]{y^{5}}}\left(\sqrt[3]{x^{2}}-\frac{y}{\sqrt[3]{x}}\right)+\frac{1}{\sqrt{x}+\sqrt{y}}: \frac{\sqrt[3]{x y}}{x-y}
$$

Oblicz wartość tego wyrażenia, przyjmując $x=3+2 \sqrt{2}$ i $y=1+\sqrt{2}$.\\
3. Narysuj wykres funkcji $f(x)=\left(\sin x+\frac{1}{2} \cos x\right)^{2}+\left(\frac{1}{2} \sin x+\cos x\right)^{2}$. Wyznacz zbiór jej wartości i rozwiąż nierówność $f(x) \geqslant \frac{5}{4}$.\\
4. Niech $A=\left\{(x, y) \in \mathbb{R}^{2}:|x| \leqslant 2,|y| \leqslant 2\right\}$ oraz $B=\left\{(x, y) \in \mathbb{R}^{2}:|x-y| \leqslant|x|+1\right\}$. Zaznacz na płaszczyźnie zbiory $A \backslash B$ oraz $A \backslash(A \backslash B)$.\\
5. W kwadrat wpisano trójkąt równoboczny w taki sposób, że jeden z jego wierzchołków jest w wierzchołku kwadratu, a dwa pozostałe leżą na przeciwległych bokach kwadratu. Wyznacz stosunek pola trójkąta do pola kwadratu.\\
6. W ostrosłupie prawidłowym trójkątnym podstawa ma długość $a$, a krawędź boczna jest do niej nachylona pod kątem $\alpha$. Oblicz objętość i pole powierzchni bocznej bryły.

\section*{PRACA KONTROLNA nr 1 - POZIOM ROZSZERZONY}
\begin{enumerate}
  \item W pierwszym naczyniu było $a$ litrów $p$-procentowego kwasu siarkowego, w drugim natomiast $b$ litrów $q$-procentowego kwasu siarkowego. Z obu naczyń odlano równe objętości roztworów, a następnie roztwór odlany z drugiego naczynia wlano do pierwszego, a odlany z pierwszego wlano do drugiego naczynia. Okazało się, że po wymieszaniu stężenia roztworów w obu naczyniach były równe. Jakie ilości roztworów odlano z każdego z naczyń?
  \item Uprość wyrażenie (dla tych $x, y$, dla których ma ono sens)
\end{enumerate}

$$
\left(\frac{1}{\sqrt[3]{x}-\sqrt[3]{y}}-\frac{3 \sqrt[3]{x y}}{x-y}-\frac{\sqrt[3]{y}-\sqrt[3]{x}}{\sqrt[3]{x^{2}}+\sqrt[3]{x y}+\sqrt[3]{y^{2}}}\right) \cdot \frac{x-y}{4 \sqrt[3]{x y}}
$$

Następnie oblicz jego wartość dla $x=5 \sqrt{2}-7$ i $y=5 \sqrt{2}+7$.\\
3. Narysuj wykres funkcji $f(x)=\sin ^{2} x+\sin x \cos x$. Wyznacz zbiór jej wartości i rozwiąż nierówność $f(x) \geqslant 1$.\\
4. Niech $A=\left\{(x, y) \in \mathbb{R}^{2}:|x-1|+|y-1| \leqslant 3\right\}$ oraz $B=\left\{(x, y) \in \mathbb{R}^{2}:|x-y| \leqslant|x+y|\right\}$. Zaznacz na płaszczyźnie zbiór $A \cap B$ i oblicz jego pole.\\
5. W romb $A B C D$ o boku $a$ i kącie ostrym $\alpha$ wpisano trójkąt $A P Q$ tak, że punkt $P$ leży na boku $B C$ a punkt $Q$ na boku $D C$, przy czym $|P C|=|D Q|=x$. Dla jakiego $x$ pole trójkąta jest najmniejsze?\\
6. W ostrosłupie prawidłowym trójkątnym ściana boczna jest nachylona do podstawy pod kątem $\alpha$. Wyznacz kąt między ścianami bocznymi.

Rozwiązania (rękopis) zadań z wybranego poziomu prosimy nadsyłać do 28 września 2020r. na adres:

\begin{verbatim}
Wydział Matematyki
Politechnika Wrocławska
Wybrzeże Wyspiańskiego 27
50-370 WROCEAW.
\end{verbatim}

Na kopercie prosimy koniecznie zaznaczyć wybrany poziom! (np. poziom podstawowy lub rozszerzony). Do rozwiązań należy dołączyć zaadresowaną do siebie kopertę zwrotną z naklejonym znaczkiem, odpowiednim do formatu listu. Polecamy stosowanie kopert formatu C5 ( $160 \times 230 \mathrm{~mm}$ ) ze znaczkiem o wartości $3,30 \mathrm{zł}$. Na każdą większą kopertę należy nakleić droższy znaczek. Prace niespełniające podanych warunków nie będą poprawiane ani odsyłane.

Uwaga. Wysyłając nam rozwiązania zadań uczestnik Kursu udostępnia Politechnice Wrocławskiej swoje dane osobowe, które przetwarzamy wyłącznie w zakresie niezbędnym do jego prowadzenia (odesłanie zadań, prowadzenie statystyki). Szczegółowe informacje o przetwarzaniu przez nas danych osobowych są dostępne na stronie internetowej Kursu.\\
Adres internetowy Kursu: \href{http://www.im.pwr.edu.pl/kurs}{http://www.im.pwr.edu.pl/kurs}


\end{document}