\documentclass[a4paper,12pt]{article}
\usepackage{latexsym}
\usepackage{amsmath}
\usepackage{amssymb}
\usepackage{graphicx}
\usepackage{wrapfig}
\pagestyle{plain}
\usepackage{fancybox}
\usepackage{bm}

\begin{document}

LI KORESPONDENCYJNY KURS

Z MATEMATYKI

wrzesień 2021 r.

PRACA KONTROLNA nr l- POZIOM PODSTAWOWY

l. Wykaz$\cdot, \dot{\mathrm{z}}\mathrm{e}$ róznica kwadratów dwóch liczb nieparzystych jest podzielna przez 8.

2. Określ dziedzinę wyrazenia $w(x,y)= [\displaystyle \frac{\sqrt{x}+\sqrt{y}}{\sqrt{x}-\sqrt{y}}-\frac{4\sqrt{x}\sqrt{y}}{x-y}]$ : $[\displaystyle \frac{1}{\sqrt{x}+\sqrt{y}}-\frac{1}{x-y}]$

Sprowad $\acute{\mathrm{z}}$ je do najprostszej postaci $\mathrm{i}$ oblicz $w(3+2\sqrt{2},3-2\sqrt{2}).$

3. Dwie druzyny harcerskie postanowiły zebrač dla ogrodu zoologicznego określoną ilośč

$\dot{\mathrm{z}}$ ofędzi. Pierwsza $\mathrm{z}$ nich rozpoczęfa pracę póltora dnia wcześniej. $\mathrm{W}\mathrm{c}\mathrm{i}_{\Phi \mathrm{g}}\mathrm{u}$ siedmiu na-

stępnych dni pracowały razem $\mathrm{i}$ zebrały zaplanowaną ilośč $\dot{\mathrm{z}}$ olędzi. Gdyby $\mathrm{k}\mathrm{a}\dot{\mathrm{z}}$ da $\mathrm{z}$ druzyn

pracowafa oddzielnie, to druga wykonalaby calą pracę $03$ dni wcześniej od pierwszej.

Ile dni potrzebuje $\mathrm{k}\mathrm{a}\dot{\mathrm{z}}$ da $\mathrm{z}$ druzyn na zebranie tej ilości $\dot{\mathrm{z}}$ ołędzi?

4. Wyznacz wartości wszystkich funkcji trygonometrycznych kata ostrego $\alpha$, wiedząc, $\dot{\mathrm{z}}\mathrm{e}$

spefnione jest równanie

$\displaystyle \frac{2\sin\alpha+3\cos\alpha}{\cos\alpha}=2$ ctg $\alpha.$

5. Funkcja liniowa $f(x)=ax+b$ spelnia warunek $f(5)-f(3)=4$. Wyznaczjej wzór, wiedząc,

$\dot{\mathrm{z}}\mathrm{e}$ pole obszaru ograniczonego wykresami funkcji $g(x)=a|x|-b$ oraz $h(x)=-a|x|+b$

jest równe 16. $\mathrm{s}_{\mathrm{P}^{\mathrm{o}\mathrm{r}\mathrm{z}}\Phi^{\mathrm{d}\acute{\mathrm{z}}\mathrm{r}\mathrm{y}\mathrm{s}\mathrm{u}\mathrm{n}\mathrm{e}\mathrm{k}}}.$

6. Niech $A= \{(x,y):|x|\leq 2,|y|\leq 2\}$ oraz $B_{p}= \{(x,y):|x|+|y|\leq p\}$ dla $p> 2.$

Narysuj $\mathrm{w}$ jednym układzie współrzędnych zbiory A $\mathrm{i}B_{3}$. Oblicz pole zbiorów $A\cap B_{3}$

$\mathrm{i} A\cup B_{3}$. Dla jakiego $p$ zbiór $A\cap B_{p}$ jest wielokątem foremnym?
\end{document}
