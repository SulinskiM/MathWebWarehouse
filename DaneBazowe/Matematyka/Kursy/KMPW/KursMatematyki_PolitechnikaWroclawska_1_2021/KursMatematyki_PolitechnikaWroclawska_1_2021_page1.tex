\documentclass[a4paper,12pt]{article}
\usepackage{latexsym}
\usepackage{amsmath}
\usepackage{amssymb}
\usepackage{graphicx}
\usepackage{wrapfig}
\pagestyle{plain}
\usepackage{fancybox}
\usepackage{bm}

\begin{document}

PRACA KONTROLNA $\mathrm{n}\mathrm{r} 1 -$ POZIOM ROZSZERZONY

l. Wykaz, $\dot{\mathrm{z}}$ ejezeli $p$ jest liczbą pierwszą większ$\Phi \mathrm{n}\mathrm{i}\dot{\mathrm{z}}3$, to jej czwarta potęga pomniejszona

$01$ jest wielokrotnością 48.

2. Określ dziedzinę wyrazenia:

$w(x,y)=(\displaystyle \frac{\sqrt[6]{y}}{\sqrt{y}-\sqrt[6]{x^{3}y^{2}}}-\frac{x}{\sqrt{xy}-x\sqrt[3]{y}})[\frac{1}{\sqrt{x}-\sqrt{y}}(\sqrt[6]{x^{5}}-\frac{y}{\sqrt[6]{x}})-\frac{x-y}{\sqrt[3]{x^{2}}+\sqrt[6]{x}\sqrt{y}}]$

$\mathrm{i}$ sprowad $\acute{\mathrm{z}}$ je do najprostszej postaci. Oblicz $w(7+5\sqrt{2},-7+5\sqrt{2}).$

Wskazówka: Oblicz najpierw $(\sqrt{2}+1)^{3}$

3. Trzech informatyków podjęfo się naprawy awarii $\mathrm{d}\mathrm{u}\dot{\mathrm{z}}$ ego systemu komputerowego. $\mathrm{Z}$ wcze-

śniejszych doświadczeń wiadomo, $\dot{\mathrm{z}}\mathrm{e}$ pierwszy $\mathrm{z}$ nich potrzebowałby na realizację tego

zlecenia 4 godziny więcej $\mathrm{n}\mathrm{i}\dot{\mathrm{z}}$ drugi, a trzeci pracowafby nad nim dwa razy krócej $\mathrm{n}\mathrm{i}\dot{\mathrm{z}}$

pierwszy. $\mathrm{W}$ jakim czasie wykonałby to zadanie $\mathrm{k}\mathrm{a}\dot{\mathrm{z}}\mathrm{d}\mathrm{y}\mathrm{z}$ informatyków, $\mathrm{j}\mathrm{e}\dot{\mathrm{z}}$ eli wiadomo,

$\dot{\mathrm{z}}\mathrm{e}$, pracując razem, naprawili awarię $\mathrm{w}$ ciągu 2 godzin $\mathrm{i}40$ minut?

4. Wyznacz wartości wszystkich funkcji trygonometrycznych $\mathrm{k}_{\Phi^{\mathrm{t}\mathrm{a}}}\alpha \in (\displaystyle \frac{\pi}{2},\pi)$, wiedząc,

$\dot{\mathrm{z}}\mathrm{e}$ spełnione jest równanie

3 $\displaystyle \cos\alpha-\frac{1}{\cos\alpha}=\sin\alpha.$

5. Dla jakich wartości parametru rzeczywistego $m$ wielomian

$w(x)=2x^{3}-(2+m)x^{2}+(2m+2)x-m-2$

ma trzy parami rózne pierwiastki rzeczywiste $x_{1}, x_{2}, x_{3}$, których suma odwrotności jest

nieujemna? $\mathrm{s}_{\mathrm{P}^{\mathrm{o}\mathrm{r}\mathrm{z}}\Phi^{\mathrm{d}\acute{\mathrm{z}}}}$ wykres funkcji $f(m)=\displaystyle \frac{1}{x_{1}}+\frac{1}{x_{2}}+\frac{1}{x_{3}}.$

6. Niech $A = \{(x,y):\sqrt{3}|x|+|y|\leq\sqrt{3}\}, B = \{(x,y):(|x|-1)^{2}+y^{2}\leq 1\}$ oraz

$C=\{(x,y):x^{2}+(|y|-\sqrt{3})^{2}\leq 1\}$. Narysuj $\mathrm{w}$ jednym ukfadzie wspófrzędnych zbiory

$A, B\mathrm{i}C$. Oblicz pole zbioru $A\backslash (B\cup C).$

$\mathrm{R}\mathrm{o}\mathrm{z}\mathrm{w}\mathrm{i}_{\Phi}$zania (rękopis) zadań $\mathrm{z}$ wybranego poziomu prosimy nadsyfač do 28.$09.2021\mathrm{r}.$

adres:

na

Wydziaf Matematyki

Politechnika Wrocfawska

Wybrzeže Wyspiańskiego 27

$50-370$ WROCLAW.

Na kopercie prosimy $\underline{\mathrm{k}\mathrm{o}\mathrm{n}\mathrm{i}\mathrm{e}\mathrm{c}\mathrm{z}\mathrm{n}\mathrm{i}\mathrm{e}}$ zaznaczyč wybrany poziom! (np. poziom podsta-

wowy lub rozszerzony). Do rozwiązań nalez $\mathrm{y}$ dołączyč zaadresowaną do siebie koperte

zwrotną $\mathrm{z}$ naklejonym znaczkiem, odpowiednim do formatu listu. Polecamy stosowanie

kopert formatu C5 $(160\mathrm{x}230\mathrm{m}\mathrm{m})$ ze znaczkiem $0$ wartości 3,30 zł. Na $\mathrm{k}\mathrm{a}\dot{\mathrm{z}}$ dą wiekszą

kopertę nalez $\mathrm{y}$ nakleič $\mathrm{d}\mathrm{r}\mathrm{o}\dot{\mathrm{z}}$ szy znaczek. Prace niespełniające podanych warunków nie

będą poprawiane ani odsyłane.

Uwaga. Wysylajac nam rozwiazania zadań uczestnik Kursu udostępnia Politechnice Wroclawskiej

swoje dane osobowe, które przetwarzamy wyłącznie $\mathrm{w}$ zakresie niezbednym do jego prowadzenia

(odesfanie zadań, prowadzenie statystyki). Szczegófowe informacje $0$ przetwarzaniu przez nas danych

osobowych są dostępne na stronie internetowej Kursu.

Adres internetowy Kursu: http: //www. im. pwr. edu. pl/kurs
\end{document}
