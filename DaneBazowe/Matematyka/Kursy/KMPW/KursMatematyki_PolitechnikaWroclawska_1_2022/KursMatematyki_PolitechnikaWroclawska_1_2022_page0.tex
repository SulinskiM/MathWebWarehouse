\documentclass[a4paper,12pt]{article}
\usepackage{latexsym}
\usepackage{amsmath}
\usepackage{amssymb}
\usepackage{graphicx}
\usepackage{wrapfig}
\pagestyle{plain}
\usepackage{fancybox}
\usepackage{bm}

\begin{document}

LII

KORESPONDENCYJNY KURS

Z MATEMATYKI

wrzesień 2022 r.

PRACA KONTROLNA $\mathrm{n}\mathrm{r} 1-$ POZIOM PODSTAWOWY

l. Uprośč wyrazenie

$\displaystyle \frac{x^{-1}-a^{-1}}{a^{-1}-b(ax)^{-1}},$

jeśli

$x=\displaystyle \frac{1}{(\alpha+b)^{-1}}- (\displaystyle \frac{a+b}{a^{2}+b^{2}})^{-1}$

2. $\mathrm{W}$ jakim stosunku nalez $\mathrm{y}$ zmieszač dwa roztwory cukru $0$ stęzeniach 5\% oraz 23\%, aby

otrzymač roztwór 17\%?

3. Rozwiąz nierównośč

$x-|5x-2|<0.$

4. Dla jakich wartości parametru $a$ nierównośč

$(a^{2}-1)x^{2}+2(a-1)x+2>0$

jest spełniona dla $\mathrm{k}\mathrm{a}\dot{\mathrm{z}}$ dego $x\in \mathbb{R}$?

5. $\mathrm{W}\mathrm{i}\mathrm{e}\mathrm{d}\mathrm{z}\Phi^{\mathrm{C}}, \dot{\mathrm{z}}\mathrm{e}1\mathrm{i}3$ są pierwiastkami równania

$x^{3}+mx^{2}+23x+n=0,$

oblicz $m, n\mathrm{i}$ wyznacz trzeci pierwiastek równania.

6. Narysuj wykres funkcji

$f(x)=$

dla

dla

$|2x-2|\leq 4,$

$|2x-2|>4.$

Wykorzystuj $\otimes \mathrm{C}$ wykres, wyznacz zbiór wartości funkcji $f(x)$ oraz $\mathrm{n}\mathrm{a}\mathrm{j}\mathrm{m}\mathrm{n}\mathrm{i}\mathrm{e}\mathrm{j}_{\mathrm{S}\mathrm{Z}\Phi}\mathrm{i}$ najwięk-

szą wartośč funkcji $\mathrm{w}$ przedziale $[0$, 4$].$
\end{document}
