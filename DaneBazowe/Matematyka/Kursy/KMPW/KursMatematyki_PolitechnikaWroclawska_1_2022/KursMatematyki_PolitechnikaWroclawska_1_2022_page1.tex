\documentclass[a4paper,12pt]{article}
\usepackage{latexsym}
\usepackage{amsmath}
\usepackage{amssymb}
\usepackage{graphicx}
\usepackage{wrapfig}
\pagestyle{plain}
\usepackage{fancybox}
\usepackage{bm}

\begin{document}

PRACA KONTROLNA $\mathrm{n}\mathrm{r} 1 -$ POZIOM ROZSZERZONY

l. Dla jakich wartości parametru $a$ równanie

$2x^{2}-ax+a+2=0$

ma pierwiastki spefniające warunek $|x_{2}-x_{1}|=1$?

2. $\mathrm{W}$ sali ustawiono krzesla $\mathrm{i}$ trzyosobowe ławki, $\mathrm{w}$ lącznej liczbie 268. Do sa1i weszło 480

osób. Po zajeciu wszystkich miejsc siedzących proporcja osób stojących do siedzących

okazafa się większa $\displaystyle \mathrm{n}\mathrm{i}\dot{\mathrm{z}}\frac{39}{160}$, ale mniejsza $\displaystyle \mathrm{n}\mathrm{i}\dot{\mathrm{z}}\frac{41}{160}$. Ile fawek $\mathrm{i}$ ile krzesel bylo $\mathrm{w}$ sali?

3. Rozwiąz nierównośč

$|||||x|-1|-2|-1|-2|\leq 3.$

4. Oblicz

$x^{4}+y^{4}+z^{4},$

jeśli $x+y+z=0$

oraz

$x^{2}+y^{2}+z^{2}=3.$

5. Rozwiąz układ równań

$\left\{\begin{array}{l}
x-|y+1|=1,\\
x^{2}+y=10.
\end{array}\right.$

Podaj jego interpretację geometryczną (narysuj starannie obie dane powyzszymi równa-

niami krzywe).

6. Wyznacz wartości parametru $p$, dla których równanie

$(p-1)x^{4}-2(p+4)x^{2}+p=0$

ma cztery pierwiastki rózne od 0.

$\mathrm{R}\mathrm{o}\mathrm{z}\mathrm{w}\mathrm{i}_{\Phi}$zania (rękopis) zadań $\mathrm{z}$ wybranego poziomu prosimy nadsyfač do $28.09.2022\mathrm{r}$. na

adres:

Wydziaf Matematyki

Politechnika Wrocfawska

Wybrzez $\mathrm{e}$ Wyspiańskiego 27

$50-370$ WROCLAW.

lub elektronicznie, za pośrednictwem portalu talent. $\mathrm{p}\mathrm{w}\mathrm{r}$. edu. pl

Na kopercie prosimy $\underline{\mathrm{k}\mathrm{o}\mathrm{n}\mathrm{i}\mathrm{e}\mathrm{c}\mathrm{z}\mathrm{n}\mathrm{i}\mathrm{e}}$ zaznaczyč wybrany poziom! (np. poziom podsta-

wowy lub rozszerzony). Do rozwiązań nalez $\mathrm{y}$ dołaczyč zaadresowaną do siebie koperte

zwrotną $\mathrm{z}$ naklejonym znaczkiem, odpowiednim do formatu listu. Prace niespełniające

podanych warunków nie będą poprawiane ani odsyłane.

Uwaga. Wysyfajac nam rozwiazania zadań uczestnik Kursu udostępnia Politechnice Wroclawskiej

swoje dane osobowe, które przetwarzamy wyłącznie $\mathrm{w}$ zakresie niezbędnym do jego prowadzenia

(odeslanie zadań, prowadzenie statystyki). Szczegófowe informacje $0$ przetwarzaniu przez nas danych

osobowych $\mathrm{S}\otimes$ dostępne na stronie internetowej Kursu.

Adres internetowy Kursu: http: //www. im. pwr. edu. pl/kurs
\end{document}
