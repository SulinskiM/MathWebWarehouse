\documentclass[a4paper,12pt]{article}
\usepackage{latexsym}
\usepackage{amsmath}
\usepackage{amssymb}
\usepackage{graphicx}
\usepackage{wrapfig}
\pagestyle{plain}
\usepackage{fancybox}
\usepackage{bm}

\begin{document}

KORESPONDENCYJNY KURS Z MATEMATYKI

PRACA KONTROLNA nr l

$\mathrm{p}\mathrm{a}\acute{\mathrm{z}}$dziernik 2$000\mathrm{r}$

l. Suma wszystkich wyrazów nieskończonego ciągu geometrycznego wynosi 2040. Jeś1i

pierwszy wyraz tego ciągu zmniejszymy $0172$, a jego iloraz zwiększymy 3-krotnie,

to suma wszystkich wyrazów tak otrzymanego ciągu wyniesie 2000. Wyznaczyč

iloraz $\mathrm{i}$ pierwszy wyraz danego ciągu.

2. Obliczyč wszystkie te skfadniki rozwinięcia dwumianu $(\sqrt{3}+\sqrt[3]{2})^{11}$, które są

liczbami całkowitymi.

3. Wykonač staranny wykres funkcji

$f(x)=|x^{2}-2|x|-3|$

i na jego podstawie podač ekstrema lokalne oraz przedzialy monotoniczności tej

funkcji.

4. Rozwiązač nierównośč

$x+1\geq\log_{2}(4^{x}-8).$

5. $\mathrm{W}$ ostrosfupie prawidfowym trójkątnym krawędz/ podstawy ma dfugośč $a$, a polowa

kąta płaskiego przy wierzchołku jest równa kątowi nachylenia ściany bocznej do

podstawy. Obliczyč objętośč ostroslupa. Sporządzič odpowiednie rysunki.

6. Znalez/č wszystkie wartości parametru $p$, dla których trójk$\Phi$t KLM $0$ wierzchofkach

$\mathrm{K}(1,1), \mathrm{L}(5,0)\mathrm{i}\mathrm{M}(\mathrm{p},\mathrm{p}-1)$ jest prostokątny. Rozwiązanie zilustrowač rysunkiem.

7. Rozwiązač równanie

$\sin 5x\sin 4x$

$\overline{\sin 3x}^{=}\overline{\sin 6x}.$

8. Przez punkt $P$ lezący wewnątrz trójkąta $ABC$ poprowadzono proste równolegle

do wszystkich boków trójkąta. Pola utworzonych $\mathrm{w}$ ten sposób trzech mniejszych

trójkatów $0$ wspólnym wierzchołku $P$ wynosza $S_{1}, S_{2}, S_{3}$. Obliczyč pole $S$

trójkąta $ABC.$

1




PRACA KONTROLNA nr 2

listopad 2000 $\mathrm{r}$

l. Promień kuli zwiększono $\mathrm{t}\mathrm{a}\mathrm{k}, \dot{\mathrm{z}}\mathrm{e}$ pole jej powierzchni wzrosło $0$ 44\%. $\mathrm{O}$ ile procent

wzrosfa jej objętośč?

2. Wyznaczyč równanie krzywej utworzonej przez środki odcinków majqcych obydwa

końce na osiach ukfadu wspófrzędnych $\mathrm{i}$ zawierających punkt $\mathrm{P}(2,1)$. Sporządzič

dokładny wykres $\mathrm{i}$ podač nazwę otrzymanej krzywej.

3. Znalez$\acute{}$č wszystkie wartości parametru $m$, dla których równanie

$(m-1)9^{x}-4\cdot 3^{x}+m+2=0$

ma dwa rózne rozwiazania.

4. Róznica promienia kuli opisanej na czworościanie foremnym $\mathrm{i}$ promienia kuli wpi-

sanej $\mathrm{w}$ niego jest równa l. Obliczyč objętośč tego czworościanu.

5. Rozwiązač nierównośč

$\displaystyle \frac{2}{|x^{2}-9|}\geq\frac{1}{x+3}$

6. Stosunek dfugości $\mathrm{p}\mathrm{r}\mathrm{z}\mathrm{y}\mathrm{p}\mathrm{r}\mathrm{o}\mathrm{s}\mathrm{t}\mathrm{o}\mathrm{k}_{\Phi^{\mathrm{t}}}$nych trójkąta $\mathrm{P}^{\mathrm{r}\mathrm{o}\mathrm{s}\mathrm{t}\mathrm{o}\mathrm{k}}\Phi^{\mathrm{t}\mathrm{n}\mathrm{e}\mathrm{g}\mathrm{o}}$ wynosi $k$. Obliczyč sto-

sunek dlugości dwusiecznych kątów ostrych tego trójkąta. $\mathrm{U}\dot{\mathrm{z}}$ yč odpowiednich wzo-

rów trygonometrycznych.

7. Zbadač przebieg zmienności funkcji

$f(x)=\displaystyle \frac{x^{2}+4}{(x-2)^{2}}$

$\mathrm{i}$ wykonač jej staranny wykres.

8. Wyznaczyč równania wszystkich prostych stycznych do wykresu funkcji $f(x) =$

$x^{3}-2x\mathrm{i}$ przechodzących przez punkt $A(\displaystyle \frac{7}{5},-2)$. Wykonač odpowiedni rysunek.

2





PRACA KONTROLNA nr 3

grudzień 2000 $\mathrm{r}$

l. Stosując zasadę indukcji matematycznej udowodnič, $\dot{\mathrm{z}}\mathrm{e}$ dla $\mathrm{k}\mathrm{a}\dot{\mathrm{z}}$ dej liczby naturalnej

$n$ suma $2^{n+1}+3^{2n-1}$ jest podzielna przez 7.

2. Tworząca stozka ma długośč $l\mathrm{i}$ widač ją ze środka kuli wpisanej $\mathrm{w}$ ten stozek pod

$\mathrm{k}_{\Phi}\mathrm{t}\mathrm{e}\mathrm{m}\alpha$. Obliczyč objętośč $\mathrm{i}$ kąt rozwarcia stozka. Określič dziedzinę dla kąta $\alpha.$

3. Nie korzystajqc $\mathrm{z}$ metod rachunku rózniczkowego wyznaczyč dziedzinę $\mathrm{i}$ zbiór war-

tości funkcji

$y=\sqrt{2+\sqrt{x}-x}.$

4. $\mathrm{Z}$ talii 24 kart wy1osowano (bez zwracania) cztery karty. Jakie jest prawdopodobień-

stwo, $\dot{\mathrm{z}}\mathrm{e}$ otrzymano dokładnie trzy karty $\mathrm{z}$ jednego koloru ($\mathrm{z}$ czterech $\mathrm{m}\mathrm{o}\dot{\mathrm{z}}$ liwych)?

5. Rozwiązač nierównośč

$\log_{1/3}$ (log2 $4x$)$\geq\log_{1/3}(2-\log_{2x}4)-1.$

6. $\mathrm{Z}$ punktu $C(1,0)$ poprowadzono styczne do okręgu $x^{2}+y^{2} =r^{2}, r \in (0,1).$

Punkty styczności oznaczono przez A $\mathrm{i}B$. Wyrazič pole trójkąta ABC jako funkcję

promienia $r\mathrm{i}$ znalez/č największą wartośč tego pola.

7. Rozwiązač ukfad równań

$\left\{\begin{array}{l}
x^{2}+y^{2}\\
|4y-3x+10|
\end{array}\right.$

$=5|x|$

$=10$

Podač interpretację geometryczną $\mathrm{k}\mathrm{a}\dot{\mathrm{z}}$ dego $\mathrm{z}$ równań $\mathrm{i}$ wykonač staranny rysunek.

8. Rozwiązač $\mathrm{w}$ przedziale $[0,\pi]$ równanie

1$+ \sin 2x=2\sin^{2}x,$

a następnie nierównośč $1+\sin 2x>2\sin^{2}x.$

3





PRACA KONTROLNA nr 4

styczeń 2001 $\mathrm{r}$

$\mathrm{W}$ celu przyblizenia sfuchaczom Kursu, jakie wymagania były stawiane ich starszym

kolegom przed ponad dwudziestu laty, niniejszy zestaw zadań jest dokladnym powtórze-

niem pracy kontrolnej ze stycznia 1979 $\mathrm{r}.$

l. Przez środek boku trójk$\Phi$ta równobocznego przeprowadzono prostą, $\mathrm{t}\mathrm{w}\mathrm{o}\mathrm{r}\mathrm{z}\text{ą}^{\mathrm{C}}\Phi \mathrm{z}$ tym

bokiem kąt ostry $\alpha \mathrm{i}$ dzielącą ten trójkąt na dwie figury, których stosunek pól jest

równy 1 : 7. Ob1iczyč miarę kata $\alpha.$

2. $\mathrm{W}$ kulę $0$ promieniu $R$ wpisano graniastosłup trójkątny prawidfowy $0$ krawędzi pod-

stawy równej $R$. Obliczyč wysokośč tego graniastoslupa.

3. Wyznaczyč wartości parametru $a$, dla których funkcja $f(x) = \displaystyle \frac{ax}{1+x^{2}}$ osiąga maksi-

mum równe 2.

4. Rozwiązač nierównośč

$\cos^{2}x+\cos^{3}x+\ldots+\cos^{n+1}x+\ldots<1+\cos x$

dla $x\in[0,2\pi].$

5. Wykazač, $\dot{\mathrm{z}}\mathrm{e}$ dla $\mathrm{k}\mathrm{a}\dot{\mathrm{z}}$ dej liczby naturalnej $n\geq 2$ prawdziwa jest równośč

$1^{2}+2^{2}+\ldots+n^{2}= \left(\begin{array}{lll}
n & + & 1\\
 & 2 & 
\end{array}\right)+2[\left(\begin{array}{l}
n\\
2
\end{array}\right)+\left(\begin{array}{ll}
n & -1\\
 & 2
\end{array}\right)+\ldots+ \left(\begin{array}{l}
2\\
2
\end{array}\right)]$

6. Wyznaczyč równanie linii bedącej zbiorem środków wszystkich okręgów stycznych

do prostej $y=0\mathrm{i}$ jednocześnie stycznych zewnętrznie do okręgu $(x+2)^{2}+y^{2}=4.$

Narysowač tę linię.

7. Wyznaczyč wartości parametru $m$, dla których równanie $9x^{2}-3x\log_{3}m+1 =0$

ma dwa rózne pierwiastki rzeczywiste $x_{1}, x_{2}$ spelniające warunek $x_{1}^{2}+x_{2}^{2}=1.$

8. Rozwiązač nierównośč

$\displaystyle \frac{\sqrt{30+x-x^{2}}}{x}<\frac{\sqrt{10}}{5}.$

4





PRACA KONTROLNA nr 5

luty 2001 $\mathrm{r}$

l. Posfugując się odpowiednim wykresem wykazač, $\dot{\mathrm{z}}\mathrm{e}$ równanie

$\sqrt{x-3}+x=4$

posiada dokfadnie jedno $\mathrm{r}\mathrm{o}\mathrm{z}\mathrm{w}\mathrm{i}_{\Phi}$zanie. Następnie wyznaczyč to rozwiązanie anali-

tycznie.

2. Wiadomo, $\dot{\mathrm{z}}\mathrm{e}$ wielomian $w(x) = 3x^{3}-5x+1$ ma trzy pierwiastki rzeczywiste

$x_{1}, x_{2}, x_{3}$. Nie wyznaczając tych pierwiastków obliczyč wartośč wyrazenia

$(1+x_{1})(1+x_{2})(1+x_{3}).$

3. Rzucamy jeden raz kostką, a nastepnie monetą tyle razy, ile oczek pokazała kostka.

Obliczyč prawdopodobieństwo tego, $\dot{\mathrm{z}}\mathrm{e}$ rzuty monetą dały co najmniej jednego orfa.

4. Wyznaczyč równania wszystkich okręgów stycznych do obu osi układu współrzęd-

nych oraz do prostej $3x+4y=12.$

5. $\mathrm{W}$ ostrosfupie prawidlowym czworokątnym dana jest odlegfośč $d$ środka podstawy

od krawędzi bocznej oraz $\mathrm{k}\mathrm{a}\mathrm{t}  2\alpha$ miedzy sąsiednimi ścianami bocznymi. Obliczyč

objętośč ostrosfupa.

6. $\mathrm{W}$ trapezie równoramiennym $0$ polu $P$ dane są promień okręgu opisanego $r$ oraz

suma długości obu podstaw $s$. Obliczyč obwód tego trapezu. Podač warunki roz-

wiązalności zadania. Wykonač rysunek dla $P=12\mathrm{c}\mathrm{m}^{2}, r=3$ cm $\mathrm{i}s=8$ cm.

7. Rozwiązač uklad równań

$\left\{\begin{array}{l}
px\\
(p+2)x
\end{array}\right.$

$+$

$+$

{\it y}

{\it py}

$3p^{2}-3p-2$

$4p$

$\mathrm{w}$ zalezności od parametru rzeczywistego $p$. Podač wszystkie rozwiązania $(\mathrm{i}$ od-

powiadające im wartości parametru $p$), dla których obie niewiadome są liczbami

całkowitymi $0$ wartości bezwzględnej mniejszej od 3.

8. Odcinek $\overline{AB}\mathrm{o}$ końcach $A(0,\displaystyle \frac{3}{2}) \mathrm{i} B(1,y), y \in [0,\displaystyle \frac{3}{2}]$, obraca się wokól osi Ox.

Wyrazič pole powstałej powierzchni jako funkcje $y\mathrm{i}$ znalez/č najmniejszą wartośč

tego pola. Sporządzič rysunek.

5





PRACA KONTROLNA nr 6

marzec 2001 r

l. Wykazač, $\dot{\mathrm{z}}\mathrm{e}$ dla $\mathrm{k}\mathrm{a}\dot{\mathrm{z}}$ dego $\mathrm{k}_{\Phi^{\mathrm{t}\mathrm{a}}} \alpha$ prawdziwa jest nierównośč

$\sqrt{3}\sin\alpha+\sqrt{6}\cos\alpha\leq 3.$

2. Dane są punkty $A(2,2) \mathrm{i} B(-1,4)$. Wyznaczyč długośč rzutu prostopadłego

odcinka $\overline{AB}$ na prostą $0$ równaniu $12x+5y=30$. Sporz$\Phi$dzič rysunek.

3. Niech $f(m)$ będzie sumą odwrotności pierwiatków rzeczywistych równania kwadra-

towego $(2^{m}-7)x^{2}-2|2^{m}-4|x+2^{m}=0$, gdzie $m$ jest parametrem rzeczywistym.

Napisač wzór określający $f(m)\mathrm{i}$ narysowač wykres tej funkcji.

4. Dwóch strzelców strzela równocześnie do tego samego celu niezaleznie od siebie.

Pierwszy strzelec trafia za $\mathrm{k}\mathrm{a}\dot{\mathrm{z}}$ dym razem $\mathrm{z}$ prawdopodobieństwem $\displaystyle \frac{2}{3} \mathrm{i}$ oddaje 2

strzały, a drugi trafia $\mathrm{z}$ prawdopodobieństwem $\displaystyle \frac{1}{2} \mathrm{i}$ oddaje 5 strzałów. Ob1iczyč

prawdopodobieństwo, $\dot{\mathrm{z}}\mathrm{e}$ cel zostanie trafiony dokładnie 3 razy.

5. Liczby $a_{1}, a_{2}, a_{n},  n\geq 3$, tworzą ciąg arytmetyczny. Suma wyrazów tego $\mathrm{c}\mathrm{i}_{\Phi \mathrm{g}}\mathrm{u}$

wynosi 28, suma wyrazów $0$ numerach nieparzystych wynosi 16, a $a_{2}\cdot a_{3}=48.$

Wyznaczyč te liczby.

6. $\mathrm{W}$ trójkącie $ABC, \mathrm{w}$ którym $AB=7$ oraz $AC=9$, a kąt przy wierzchołku $A$ jest

dwa razy większy $\mathrm{n}\mathrm{i}\dot{\mathrm{z}}$ kąt przy wierzchołku $B$. Obliczyč stosunek promienia kola

wpisanego do promienia kola opisanego na tym trójk$\Phi$cie. Rozwiązanie zilustrowač

rysunkiem.

7. Zaznaczyč na p{\it l}aszczy $\acute{\mathrm{z}}\mathrm{n}\mathrm{i}\mathrm{e}$ nastepujące zbiory punktów:

$A=\{(x,y):x+y-2\geq|x-2|\},$

$B=\{(x,y):y\leq\sqrt{4x-x^{2}}\}.$

Następnie znalez/č na brzegu zbioru

$P(\displaystyle \frac{5}{2},1)$ jest najmniejsza.

$A\cap B$ punkt $\mathrm{Q}$, którego odleglośč od punktu

8. Przeprowadzič badanie przebiegu i sporządzič wykres funkcji

$f(x)=\displaystyle \frac{1}{2}x^{2}-4+\sqrt{8-x^{2}}.$

6





PRACA KONTROLNA nr 7

kwiecień 2001 $\mathrm{r}$

l. Ile elementów ma zbiór $A$, jeśli liczbajego podzbiorów trójelementowych jest większa

od liczby podzbiorów dwuelementowych $048$ ?

2. $\mathrm{W}$ sześciokqt foremny $0$ boku l wpisano okrąg. $\mathrm{W}$ otrzymany okrqg wpisano sześcio-

$\mathrm{k}_{\Phi^{\mathrm{t}}}$ foremny, $\mathrm{w}$ który znów wpisano okrąg, itd. Obliczyč sumę obwodów wszystkich

otrzymanych okręgów.

3. Dana jest rodzina prostych $0$ równaniach $2x+my-m-2=0,$

prostych tej rodziny są:

a) prostopadłe do prostej $x+4y+2=0,$

b) równoległe do prostej $3x+2y=0,$

c) tworzą $\mathrm{z}$ prostą $x-\sqrt{3}y-1=0$ kąt $\displaystyle \frac{\pi}{3}.$

$m\in R$. Które $\mathrm{z}$

4. Sprawdzič $\mathrm{t}\mathrm{o}\dot{\mathrm{z}}$ samośč: $tg(x-\displaystyle \frac{\pi}{4})-1=\frac{-2}{tgx+1}$. Korzystajqc $\mathrm{z}$ niej sporzadzič wykres

funkcji $f(x)=\displaystyle \frac{1}{tgx+1}\mathrm{w}$ przedziale $[0,\pi].$

5. Dany jest okrąg $K\mathrm{o}$ równaniu $x^{2}+y^{2}-6y=27$. Wyznaczyč równanie krzywej $\Gamma$

bedącej obrazem okręgu $K\mathrm{w}$ powinowactwie prostokątnym $0$ osi Ox $\mathrm{i}$ skali $k=\displaystyle \frac{1}{3}.$

Obliczyč pole figury lezącej ponizej osi odciętych $\mathrm{i}$ ograniczonej łukiem okręgu $\mathrm{K}\mathrm{i}$

$\mathrm{k}\mathrm{r}\mathrm{z}\mathrm{y}\mathrm{w}\Phi^{\Gamma}$. Wykonač rysunek.

6. Wykorzystując nierównośč $2\sqrt{ab}\leq a+b, a, b>0$, wyznaczyč granicę

$\displaystyle \lim_{n\rightarrow\infty}(\frac{\log_{5}16}{\log_{2}3})^{n}$

7. Trylogię skfadającą się $\mathrm{z}$ dwóch powieści dwutomowych oraz jednej jednotomowej

ustawiono przypadkowo na pólce. Jakie jest prawdopodobieństwo tego, $\dot{\mathrm{z}}\mathrm{e}$ tomy a)

obydwu, b) co najmniej jednej, $\mathrm{z}$ dwutomowych powieści znajdujq się obok siebie $\mathrm{i}$

przy tym tom I $\mathrm{z}$ lewej, a tom II $\mathrm{z}$ prawej strony.

8. $\mathrm{W}$ ostrosłupie prawidlowym czworokątnym krawęd $\acute{\mathrm{z}}$ bocznajest nachylona do plasz-

czyzny podstawy pod kqtem $\alpha$, a krawędz$\acute{}$ podstawy ma długośč $a$. Obliczyč pro-

mień kuli stycznej do wszystkich krawędzi tego ostrosfupa. Wykonač odpowiednie

rysunki.

7



\end{document}