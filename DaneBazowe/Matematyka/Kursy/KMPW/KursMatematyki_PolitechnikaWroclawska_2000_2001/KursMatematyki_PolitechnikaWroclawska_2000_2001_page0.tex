\documentclass[a4paper,12pt]{article}
\usepackage{latexsym}
\usepackage{amsmath}
\usepackage{amssymb}
\usepackage{graphicx}
\usepackage{wrapfig}
\pagestyle{plain}
\usepackage{fancybox}
\usepackage{bm}

\begin{document}

KORESPONDENCYJNY KURS Z MATEMATYKI

PRACA KONTROLNA nr l

$\mathrm{p}\mathrm{a}\acute{\mathrm{z}}$dziernik 2$000\mathrm{r}$

l. Suma wszystkich wyrazów nieskończonego ciągu geometrycznego wynosi 2040. Jeś1i

pierwszy wyraz tego ciągu zmniejszymy $0172$, a jego iloraz zwiększymy 3-krotnie,

to suma wszystkich wyrazów tak otrzymanego ciągu wyniesie 2000. Wyznaczyč

iloraz $\mathrm{i}$ pierwszy wyraz danego ciągu.

2. Obliczyč wszystkie te skfadniki rozwinięcia dwumianu $(\sqrt{3}+\sqrt[3]{2})^{11}$, które są

liczbami całkowitymi.

3. Wykonač staranny wykres funkcji

$f(x)=|x^{2}-2|x|-3|$

i na jego podstawie podač ekstrema lokalne oraz przedzialy monotoniczności tej

funkcji.

4. Rozwiązač nierównośč

$x+1\geq\log_{2}(4^{x}-8).$

5. $\mathrm{W}$ ostrosfupie prawidfowym trójkątnym krawędz/ podstawy ma dfugośč $a$, a polowa

kąta płaskiego przy wierzchołku jest równa kątowi nachylenia ściany bocznej do

podstawy. Obliczyč objętośč ostroslupa. Sporządzič odpowiednie rysunki.

6. Znalez/č wszystkie wartości parametru $p$, dla których trójk$\Phi$t KLM $0$ wierzchofkach

$\mathrm{K}(1,1), \mathrm{L}(5,0)\mathrm{i}\mathrm{M}(\mathrm{p},\mathrm{p}-1)$ jest prostokątny. Rozwiązanie zilustrowač rysunkiem.

7. Rozwiązač równanie

$\sin 5x\sin 4x$

$\overline{\sin 3x}^{=}\overline{\sin 6x}.$

8. Przez punkt $P$ lezący wewnątrz trójkąta $ABC$ poprowadzono proste równolegle

do wszystkich boków trójkąta. Pola utworzonych $\mathrm{w}$ ten sposób trzech mniejszych

trójkatów $0$ wspólnym wierzchołku $P$ wynosza $S_{1}, S_{2}, S_{3}$. Obliczyč pole $S$

trójkąta $ABC.$

1
\end{document}
