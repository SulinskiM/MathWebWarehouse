\documentclass[a4paper,12pt]{article}
\usepackage{latexsym}
\usepackage{amsmath}
\usepackage{amssymb}
\usepackage{graphicx}
\usepackage{wrapfig}
\pagestyle{plain}
\usepackage{fancybox}
\usepackage{bm}

\begin{document}

PRACA KONTROLNA nr 2

listopad 2000 $\mathrm{r}$

l. Promień kuli zwiększono $\mathrm{t}\mathrm{a}\mathrm{k}, \dot{\mathrm{z}}\mathrm{e}$ pole jej powierzchni wzrosło $0$ 44\%. $\mathrm{O}$ ile procent

wzrosfa jej objętośč?

2. Wyznaczyč równanie krzywej utworzonej przez środki odcinków majqcych obydwa

końce na osiach ukfadu wspófrzędnych $\mathrm{i}$ zawierających punkt $\mathrm{P}(2,1)$. Sporządzič

dokładny wykres $\mathrm{i}$ podač nazwę otrzymanej krzywej.

3. Znalez$\acute{}$č wszystkie wartości parametru $m$, dla których równanie

$(m-1)9^{x}-4\cdot 3^{x}+m+2=0$

ma dwa rózne rozwiazania.

4. Róznica promienia kuli opisanej na czworościanie foremnym $\mathrm{i}$ promienia kuli wpi-

sanej $\mathrm{w}$ niego jest równa l. Obliczyč objętośč tego czworościanu.

5. Rozwiązač nierównośč

$\displaystyle \frac{2}{|x^{2}-9|}\geq\frac{1}{x+3}$

6. Stosunek dfugości $\mathrm{p}\mathrm{r}\mathrm{z}\mathrm{y}\mathrm{p}\mathrm{r}\mathrm{o}\mathrm{s}\mathrm{t}\mathrm{o}\mathrm{k}_{\Phi^{\mathrm{t}}}$nych trójkąta $\mathrm{P}^{\mathrm{r}\mathrm{o}\mathrm{s}\mathrm{t}\mathrm{o}\mathrm{k}}\Phi^{\mathrm{t}\mathrm{n}\mathrm{e}\mathrm{g}\mathrm{o}}$ wynosi $k$. Obliczyč sto-

sunek dlugości dwusiecznych kątów ostrych tego trójkąta. $\mathrm{U}\dot{\mathrm{z}}$ yč odpowiednich wzo-

rów trygonometrycznych.

7. Zbadač przebieg zmienności funkcji

$f(x)=\displaystyle \frac{x^{2}+4}{(x-2)^{2}}$

$\mathrm{i}$ wykonač jej staranny wykres.

8. Wyznaczyč równania wszystkich prostych stycznych do wykresu funkcji $f(x) =$

$x^{3}-2x\mathrm{i}$ przechodzących przez punkt $A(\displaystyle \frac{7}{5},-2)$. Wykonač odpowiedni rysunek.

2
\end{document}
