\documentclass[a4paper,12pt]{article}
\usepackage{latexsym}
\usepackage{amsmath}
\usepackage{amssymb}
\usepackage{graphicx}
\usepackage{wrapfig}
\pagestyle{plain}
\usepackage{fancybox}
\usepackage{bm}

\begin{document}

PRACA KONTROLNA nr 3

grudzień 2000 $\mathrm{r}$

l. Stosując zasadę indukcji matematycznej udowodnič, $\dot{\mathrm{z}}\mathrm{e}$ dla $\mathrm{k}\mathrm{a}\dot{\mathrm{z}}$ dej liczby naturalnej

$n$ suma $2^{n+1}+3^{2n-1}$ jest podzielna przez 7.

2. Tworząca stozka ma długośč $l\mathrm{i}$ widač ją ze środka kuli wpisanej $\mathrm{w}$ ten stozek pod

$\mathrm{k}_{\Phi}\mathrm{t}\mathrm{e}\mathrm{m}\alpha$. Obliczyč objętośč $\mathrm{i}$ kąt rozwarcia stozka. Określič dziedzinę dla kąta $\alpha.$

3. Nie korzystajqc $\mathrm{z}$ metod rachunku rózniczkowego wyznaczyč dziedzinę $\mathrm{i}$ zbiór war-

tości funkcji

$y=\sqrt{2+\sqrt{x}-x}.$

4. $\mathrm{Z}$ talii 24 kart wy1osowano (bez zwracania) cztery karty. Jakie jest prawdopodobień-

stwo, $\dot{\mathrm{z}}\mathrm{e}$ otrzymano dokładnie trzy karty $\mathrm{z}$ jednego koloru ($\mathrm{z}$ czterech $\mathrm{m}\mathrm{o}\dot{\mathrm{z}}$ liwych)?

5. Rozwiązač nierównośč

$\log_{1/3}$ (log2 $4x$)$\geq\log_{1/3}(2-\log_{2x}4)-1.$

6. $\mathrm{Z}$ punktu $C(1,0)$ poprowadzono styczne do okręgu $x^{2}+y^{2} =r^{2}, r \in (0,1).$

Punkty styczności oznaczono przez A $\mathrm{i}B$. Wyrazič pole trójkąta ABC jako funkcję

promienia $r\mathrm{i}$ znalez/č największą wartośč tego pola.

7. Rozwiązač ukfad równań

$\left\{\begin{array}{l}
x^{2}+y^{2}\\
|4y-3x+10|
\end{array}\right.$

$=5|x|$

$=10$

Podač interpretację geometryczną $\mathrm{k}\mathrm{a}\dot{\mathrm{z}}$ dego $\mathrm{z}$ równań $\mathrm{i}$ wykonač staranny rysunek.

8. Rozwiązač $\mathrm{w}$ przedziale $[0,\pi]$ równanie

1$+ \sin 2x=2\sin^{2}x,$

a następnie nierównośč $1+\sin 2x>2\sin^{2}x.$

3
\end{document}
