\documentclass[a4paper,12pt]{article}
\usepackage{latexsym}
\usepackage{amsmath}
\usepackage{amssymb}
\usepackage{graphicx}
\usepackage{wrapfig}
\pagestyle{plain}
\usepackage{fancybox}
\usepackage{bm}

\begin{document}

PRACA KONTROLNA nr 4

styczeń 2001 $\mathrm{r}$

$\mathrm{W}$ celu przyblizenia sfuchaczom Kursu, jakie wymagania były stawiane ich starszym

kolegom przed ponad dwudziestu laty, niniejszy zestaw zadań jest dokladnym powtórze-

niem pracy kontrolnej ze stycznia 1979 $\mathrm{r}.$

l. Przez środek boku trójk$\Phi$ta równobocznego przeprowadzono prostą, $\mathrm{t}\mathrm{w}\mathrm{o}\mathrm{r}\mathrm{z}\text{ą}^{\mathrm{C}}\Phi \mathrm{z}$ tym

bokiem kąt ostry $\alpha \mathrm{i}$ dzielącą ten trójkąt na dwie figury, których stosunek pól jest

równy 1 : 7. Ob1iczyč miarę kata $\alpha.$

2. $\mathrm{W}$ kulę $0$ promieniu $R$ wpisano graniastosłup trójkątny prawidfowy $0$ krawędzi pod-

stawy równej $R$. Obliczyč wysokośč tego graniastoslupa.

3. Wyznaczyč wartości parametru $a$, dla których funkcja $f(x) = \displaystyle \frac{ax}{1+x^{2}}$ osiąga maksi-

mum równe 2.

4. Rozwiązač nierównośč

$\cos^{2}x+\cos^{3}x+\ldots+\cos^{n+1}x+\ldots<1+\cos x$

dla $x\in[0,2\pi].$

5. Wykazač, $\dot{\mathrm{z}}\mathrm{e}$ dla $\mathrm{k}\mathrm{a}\dot{\mathrm{z}}$ dej liczby naturalnej $n\geq 2$ prawdziwa jest równośč

$1^{2}+2^{2}+\ldots+n^{2}= \left(\begin{array}{lll}
n & + & 1\\
 & 2 & 
\end{array}\right)+2[\left(\begin{array}{l}
n\\
2
\end{array}\right)+\left(\begin{array}{ll}
n & -1\\
 & 2
\end{array}\right)+\ldots+ \left(\begin{array}{l}
2\\
2
\end{array}\right)]$

6. Wyznaczyč równanie linii bedącej zbiorem środków wszystkich okręgów stycznych

do prostej $y=0\mathrm{i}$ jednocześnie stycznych zewnętrznie do okręgu $(x+2)^{2}+y^{2}=4.$

Narysowač tę linię.

7. Wyznaczyč wartości parametru $m$, dla których równanie $9x^{2}-3x\log_{3}m+1 =0$

ma dwa rózne pierwiastki rzeczywiste $x_{1}, x_{2}$ spelniające warunek $x_{1}^{2}+x_{2}^{2}=1.$

8. Rozwiązač nierównośč

$\displaystyle \frac{\sqrt{30+x-x^{2}}}{x}<\frac{\sqrt{10}}{5}.$

4
\end{document}
