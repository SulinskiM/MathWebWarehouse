\documentclass[a4paper,12pt]{article}
\usepackage{latexsym}
\usepackage{amsmath}
\usepackage{amssymb}
\usepackage{graphicx}
\usepackage{wrapfig}
\pagestyle{plain}
\usepackage{fancybox}
\usepackage{bm}

\begin{document}

PRACA KONTROLNA nr 5

luty 2001 $\mathrm{r}$

l. Posfugując się odpowiednim wykresem wykazač, $\dot{\mathrm{z}}\mathrm{e}$ równanie

$\sqrt{x-3}+x=4$

posiada dokfadnie jedno $\mathrm{r}\mathrm{o}\mathrm{z}\mathrm{w}\mathrm{i}_{\Phi}$zanie. Następnie wyznaczyč to rozwiązanie anali-

tycznie.

2. Wiadomo, $\dot{\mathrm{z}}\mathrm{e}$ wielomian $w(x) = 3x^{3}-5x+1$ ma trzy pierwiastki rzeczywiste

$x_{1}, x_{2}, x_{3}$. Nie wyznaczając tych pierwiastków obliczyč wartośč wyrazenia

$(1+x_{1})(1+x_{2})(1+x_{3}).$

3. Rzucamy jeden raz kostką, a nastepnie monetą tyle razy, ile oczek pokazała kostka.

Obliczyč prawdopodobieństwo tego, $\dot{\mathrm{z}}\mathrm{e}$ rzuty monetą dały co najmniej jednego orfa.

4. Wyznaczyč równania wszystkich okręgów stycznych do obu osi układu współrzęd-

nych oraz do prostej $3x+4y=12.$

5. $\mathrm{W}$ ostrosfupie prawidlowym czworokątnym dana jest odlegfośč $d$ środka podstawy

od krawędzi bocznej oraz $\mathrm{k}\mathrm{a}\mathrm{t}  2\alpha$ miedzy sąsiednimi ścianami bocznymi. Obliczyč

objętośč ostrosfupa.

6. $\mathrm{W}$ trapezie równoramiennym $0$ polu $P$ dane są promień okręgu opisanego $r$ oraz

suma długości obu podstaw $s$. Obliczyč obwód tego trapezu. Podač warunki roz-

wiązalności zadania. Wykonač rysunek dla $P=12\mathrm{c}\mathrm{m}^{2}, r=3$ cm $\mathrm{i}s=8$ cm.

7. Rozwiązač uklad równań

$\left\{\begin{array}{l}
px\\
(p+2)x
\end{array}\right.$

$+$

$+$

{\it y}

{\it py}

$3p^{2}-3p-2$

$4p$

$\mathrm{w}$ zalezności od parametru rzeczywistego $p$. Podač wszystkie rozwiązania $(\mathrm{i}$ od-

powiadające im wartości parametru $p$), dla których obie niewiadome są liczbami

całkowitymi $0$ wartości bezwzględnej mniejszej od 3.

8. Odcinek $\overline{AB}\mathrm{o}$ końcach $A(0,\displaystyle \frac{3}{2}) \mathrm{i} B(1,y), y \in [0,\displaystyle \frac{3}{2}]$, obraca się wokól osi Ox.

Wyrazič pole powstałej powierzchni jako funkcje $y\mathrm{i}$ znalez/č najmniejszą wartośč

tego pola. Sporządzič rysunek.

5
\end{document}
