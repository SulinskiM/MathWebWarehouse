\documentclass[a4paper,12pt]{article}
\usepackage{latexsym}
\usepackage{amsmath}
\usepackage{amssymb}
\usepackage{graphicx}
\usepackage{wrapfig}
\pagestyle{plain}
\usepackage{fancybox}
\usepackage{bm}

\begin{document}

PRACA KONTROLNA nr 6

marzec 2001 r

l. Wykazač, $\dot{\mathrm{z}}\mathrm{e}$ dla $\mathrm{k}\mathrm{a}\dot{\mathrm{z}}$ dego $\mathrm{k}_{\Phi^{\mathrm{t}\mathrm{a}}} \alpha$ prawdziwa jest nierównośč

$\sqrt{3}\sin\alpha+\sqrt{6}\cos\alpha\leq 3.$

2. Dane są punkty $A(2,2) \mathrm{i} B(-1,4)$. Wyznaczyč długośč rzutu prostopadłego

odcinka $\overline{AB}$ na prostą $0$ równaniu $12x+5y=30$. Sporz$\Phi$dzič rysunek.

3. Niech $f(m)$ będzie sumą odwrotności pierwiatków rzeczywistych równania kwadra-

towego $(2^{m}-7)x^{2}-2|2^{m}-4|x+2^{m}=0$, gdzie $m$ jest parametrem rzeczywistym.

Napisač wzór określający $f(m)\mathrm{i}$ narysowač wykres tej funkcji.

4. Dwóch strzelców strzela równocześnie do tego samego celu niezaleznie od siebie.

Pierwszy strzelec trafia za $\mathrm{k}\mathrm{a}\dot{\mathrm{z}}$ dym razem $\mathrm{z}$ prawdopodobieństwem $\displaystyle \frac{2}{3} \mathrm{i}$ oddaje 2

strzały, a drugi trafia $\mathrm{z}$ prawdopodobieństwem $\displaystyle \frac{1}{2} \mathrm{i}$ oddaje 5 strzałów. Ob1iczyč

prawdopodobieństwo, $\dot{\mathrm{z}}\mathrm{e}$ cel zostanie trafiony dokładnie 3 razy.

5. Liczby $a_{1}, a_{2}, a_{n},  n\geq 3$, tworzą ciąg arytmetyczny. Suma wyrazów tego $\mathrm{c}\mathrm{i}_{\Phi \mathrm{g}}\mathrm{u}$

wynosi 28, suma wyrazów $0$ numerach nieparzystych wynosi 16, a $a_{2}\cdot a_{3}=48.$

Wyznaczyč te liczby.

6. $\mathrm{W}$ trójkącie $ABC, \mathrm{w}$ którym $AB=7$ oraz $AC=9$, a kąt przy wierzchołku $A$ jest

dwa razy większy $\mathrm{n}\mathrm{i}\dot{\mathrm{z}}$ kąt przy wierzchołku $B$. Obliczyč stosunek promienia kola

wpisanego do promienia kola opisanego na tym trójk$\Phi$cie. Rozwiązanie zilustrowač

rysunkiem.

7. Zaznaczyč na p{\it l}aszczy $\acute{\mathrm{z}}\mathrm{n}\mathrm{i}\mathrm{e}$ nastepujące zbiory punktów:

$A=\{(x,y):x+y-2\geq|x-2|\},$

$B=\{(x,y):y\leq\sqrt{4x-x^{2}}\}.$

Następnie znalez/č na brzegu zbioru

$P(\displaystyle \frac{5}{2},1)$ jest najmniejsza.

$A\cap B$ punkt $\mathrm{Q}$, którego odleglośč od punktu

8. Przeprowadzič badanie przebiegu i sporządzič wykres funkcji

$f(x)=\displaystyle \frac{1}{2}x^{2}-4+\sqrt{8-x^{2}}.$

6
\end{document}
