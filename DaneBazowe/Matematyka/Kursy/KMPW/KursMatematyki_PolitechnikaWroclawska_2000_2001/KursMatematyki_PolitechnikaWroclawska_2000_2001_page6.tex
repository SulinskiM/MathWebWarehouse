\documentclass[a4paper,12pt]{article}
\usepackage{latexsym}
\usepackage{amsmath}
\usepackage{amssymb}
\usepackage{graphicx}
\usepackage{wrapfig}
\pagestyle{plain}
\usepackage{fancybox}
\usepackage{bm}

\begin{document}

PRACA KONTROLNA nr 7

kwiecień 2001 $\mathrm{r}$

l. Ile elementów ma zbiór $A$, jeśli liczbajego podzbiorów trójelementowych jest większa

od liczby podzbiorów dwuelementowych $048$ ?

2. $\mathrm{W}$ sześciokqt foremny $0$ boku l wpisano okrąg. $\mathrm{W}$ otrzymany okrqg wpisano sześcio-

$\mathrm{k}_{\Phi^{\mathrm{t}}}$ foremny, $\mathrm{w}$ który znów wpisano okrąg, itd. Obliczyč sumę obwodów wszystkich

otrzymanych okręgów.

3. Dana jest rodzina prostych $0$ równaniach $2x+my-m-2=0,$

prostych tej rodziny są:

a) prostopadłe do prostej $x+4y+2=0,$

b) równoległe do prostej $3x+2y=0,$

c) tworzą $\mathrm{z}$ prostą $x-\sqrt{3}y-1=0$ kąt $\displaystyle \frac{\pi}{3}.$

$m\in R$. Które $\mathrm{z}$

4. Sprawdzič $\mathrm{t}\mathrm{o}\dot{\mathrm{z}}$ samośč: $tg(x-\displaystyle \frac{\pi}{4})-1=\frac{-2}{tgx+1}$. Korzystajqc $\mathrm{z}$ niej sporzadzič wykres

funkcji $f(x)=\displaystyle \frac{1}{tgx+1}\mathrm{w}$ przedziale $[0,\pi].$

5. Dany jest okrąg $K\mathrm{o}$ równaniu $x^{2}+y^{2}-6y=27$. Wyznaczyč równanie krzywej $\Gamma$

bedącej obrazem okręgu $K\mathrm{w}$ powinowactwie prostokątnym $0$ osi Ox $\mathrm{i}$ skali $k=\displaystyle \frac{1}{3}.$

Obliczyč pole figury lezącej ponizej osi odciętych $\mathrm{i}$ ograniczonej łukiem okręgu $\mathrm{K}\mathrm{i}$

$\mathrm{k}\mathrm{r}\mathrm{z}\mathrm{y}\mathrm{w}\Phi^{\Gamma}$. Wykonač rysunek.

6. Wykorzystując nierównośč $2\sqrt{ab}\leq a+b, a, b>0$, wyznaczyč granicę

$\displaystyle \lim_{n\rightarrow\infty}(\frac{\log_{5}16}{\log_{2}3})^{n}$

7. Trylogię skfadającą się $\mathrm{z}$ dwóch powieści dwutomowych oraz jednej jednotomowej

ustawiono przypadkowo na pólce. Jakie jest prawdopodobieństwo tego, $\dot{\mathrm{z}}\mathrm{e}$ tomy a)

obydwu, b) co najmniej jednej, $\mathrm{z}$ dwutomowych powieści znajdujq się obok siebie $\mathrm{i}$

przy tym tom I $\mathrm{z}$ lewej, a tom II $\mathrm{z}$ prawej strony.

8. $\mathrm{W}$ ostrosłupie prawidlowym czworokątnym krawęd $\acute{\mathrm{z}}$ bocznajest nachylona do plasz-

czyzny podstawy pod kqtem $\alpha$, a krawędz$\acute{}$ podstawy ma długośč $a$. Obliczyč pro-

mień kuli stycznej do wszystkich krawędzi tego ostrosfupa. Wykonač odpowiednie

rysunki.

7
\end{document}
