\documentclass[a4paper,12pt]{article}
\usepackage{latexsym}
\usepackage{amsmath}
\usepackage{amssymb}
\usepackage{graphicx}
\usepackage{wrapfig}
\pagestyle{plain}
\usepackage{fancybox}
\usepackage{bm}

\begin{document}

KORESPONDENCYJNY KURS Z MATEMATYKI

PRACA KONTROLNA nr l

$\mathrm{p}\mathrm{a}\acute{\mathrm{z}}$dziernik 2$001\mathrm{r}$

l. Dwaj rowerzyści wyruszyli jednocześnie $\mathrm{w}$ drogę, jeden $\mathrm{z}$ A do $\mathrm{B}$, drugi $\mathrm{z}\mathrm{B}$ do $\mathrm{A}$

$\mathrm{i}$ spotkali się po jednej godzinie. Pierwszy $\mathrm{z}$ nich przebywał $\mathrm{w}$ ciągu godziny $03$

km więcej $\mathrm{n}\mathrm{i}\dot{\mathrm{z}}$ drugi $\mathrm{i}$ przyjechal do celu $027$ minut wcześniej $\mathrm{n}\mathrm{i}\dot{\mathrm{z}}$ drugi. Jakie byfy

prędkości obu rowerzystów $\mathrm{i}$ jaka jest odległośč AB?

2. Rozwiązač nierównośč:

$\sqrt{x^{2}-3}>\underline{2}.$

$x$

3. Rysunek przedstawia dach budynku $\mathrm{w}$ rzucie poziomym. $\mathrm{K}\mathrm{a}\dot{\mathrm{z}}$ da $\mathrm{z}$ plaszczyzn nachy-
\begin{center}
\includegraphics[width=48.156mm,height=24.132mm]{./KursMatematyki_PolitechnikaWroclawska_2001_2002_page0_images/image001.eps}
\end{center}
lona jest do płaszczyzny poziomej pod $\mathrm{k}$ tem $30^{0}$ Dłu-

gossc dachu wynosi 18 $\mathrm{m}$, a szerokosc 9 $\mathrm{m}$. Obliczyc po-

le powierzchni dachu oraz cafkowit kubaturę strychu $\mathrm{w}$

tym budynku.

4. Pewna firma przeprowadza co kwartal regulację plac dla swoich pracowników rewa-

$1\mathrm{o}\mathrm{r}\mathrm{y}\mathrm{z}\mathrm{u}\mathrm{j}_{\Phi}\mathrm{c}$ je zgodnie ze wska $\acute{\mathrm{z}}\mathrm{n}\mathrm{i}\mathrm{k}\mathrm{i}\mathrm{e}\mathrm{m}$ inflacji, który jest stafy $\mathrm{i}$ wynosi 1,5\% kwar-

talnie, oraz doliczając stałą kwotę podwyzki 16 $\mathrm{z}\mathrm{l}\mathrm{p}. \mathrm{W}$ styczniu 2001 pan Kowa1ski

otrzymał wynagrodzenie 1600 $\mathrm{z}\mathrm{l}\mathrm{p}$. Jaką pensję otrzyma $\mathrm{w}$ kwietniu 2002? Wyzna-

czyč wzór ogólny na pensję $w_{n}$ pana Kowalskiego $\mathrm{w}\mathrm{n}$-tym kwartale przyjmując, $\dot{\mathrm{z}}\mathrm{e}$

$w_{1}=1600$jest placą $\mathrm{w}$ pierwszym kwartale 2001. Ob1iczyč średniq miesięcznq płacę

pana Kowalskiego $\mathrm{w}$ 2002 roku.

5. Wyznaczyč funkcję odwrotną do $f(x) =x^{3}, x\in R$. Korzystając $\mathrm{z}$ tego wykonač

staranny wykres funkcji $h(x) =$
\begin{center}
\includegraphics[width=36.732mm,height=7.620mm]{./KursMatematyki_PolitechnikaWroclawska_2001_2002_page0_images/image002.eps}
\end{center}
$(|x|-- 1)+1.$

6. Rozwiązač równanie:

-cs  oins 24{\it xx} $=$ 1.

7. Dany jest trójkąt $0$ wierzcholkach $A(-2,1), B(-1,-6), C(2,5)$. Posługując się

rachunkiem wektorowym obliczyč cosinus kąta pomiędzy dwusieczną kata $A\mathrm{i}$ środ-

kową boku $\overline{BC}$. Wykonač rysunek.

8. Przeprowadzič badanie przebiegu $\mathrm{i}$ wykonač wykres funkcji

$f(x)=x+\displaystyle \frac{x}{x-1}+\frac{x}{(x-1)^{2}}+\frac{x}{(x-1)^{3}}+$




PRACA KONTROLNA nr 2

listopad $2001\mathrm{r}$

l. Cena ll paliwa została zmniejszona $0$ 15\%. Po dwóch tygodniach dokonano kolej-

nej zmiany ceny paliwa zwiększając ją $0$ 15\%. $\mathrm{O}$ ile procent końcowa cena paliwa

rózni $\mathrm{s}\mathrm{i}_{9}$ od początkowej?

2. Wyznaczyč $\mathrm{i}$ narysowač zbiór złozony $\mathrm{z}$ punktów $(x,y)$ płaszczyzny spełniających

warunek

$x^{2}+y^{2}=8|x|+6|y|.$

3. Wysokośč ostrosfupa trójkątnego prawidfowego wynosi $h$, a kąt między wysokościa-

mi ścian bocznych jest równy $ 2\alpha$. Obliczyč pole powierzchni bocznej tego ostrosłupa.

Sporządzič odpowiednie rysunki.

4. $\mathrm{Z}$ arkusza blachy $\mathrm{w}$ kształcie równoległoboku $0$ bokach 30 cm $\mathrm{i}60$ cm $\mathrm{i}$ kącie ostrym

$60^{0}$ nalezy odciąč dwa przeciwlegfe trójkqtne naroza $\mathrm{t}\mathrm{a}\mathrm{k}$, aby powstaf romb $0\mathrm{m}\mathrm{o}\dot{\mathrm{z}}$-

liwie największym polu. Określič przez który punkt dfuzszego boku nalez $\mathrm{y}$ prze-

prowadzič cięcie oraz obliczyč kąt ostry otrzymanego rombu zaokrqglajqc wynik do

jednej minuty kątowej.

5. Rozwiązač równanie

$2^{\log_{\sqrt{2}}x}=(\sqrt{2})^{\log_{x}2}$

6. Wyznaczyč dziedzinę i zbiór wartości funkcji

$f(x)=\displaystyle \frac{4}{\sin x+2\cos x+3}.$

7. Znalez$\acute{}$č wszystkie wartości parametru $p$, dla których równanie

$px^{4}-4x^{2}+p+1=0$

ma dwa rózne rozwiązania.

8. Wyznaczyč tangens $\mathrm{k}_{\Phi^{\mathrm{t}\mathrm{a}}}$, pod którym styczna do wykresu funkcji $f(x) = \displaystyle \frac{8}{x^{2}+3} \mathrm{w}$

punkcie $A(3,\displaystyle \frac{2}{3})$ przecina wykres tej funkcji.





PRACA KONTROLNA nr 3

grudzień $2001\mathrm{r}$

l. Dla jakich wartości $\sin x$ liczby $\sin x, \cos x, \sin 2x$ ($\mathrm{w}$ podanym porządku) są ko-

lejnymi wyrazami ciągu geometrycznego. Wyznaczyč czwarte wyrazy tych ciągów.

2. $\mathrm{W}$ pewnych zawodach sportowych startuje 16 druzyn. $\mathrm{W}$ eliminacjach są one losowo

dzielone na 4 grupy po 4 druzyny $\mathrm{k}\mathrm{a}\dot{\mathrm{z}}$ da grupa. Obliczyč prawdopodobieństwo tego,

$\dot{\mathrm{z}}\mathrm{e}$ trzy zwycięskie druzyny $\mathrm{z}$ poprzednich zawodów $\mathrm{z}\mathrm{n}\mathrm{a}\mathrm{j}\mathrm{d}_{\Phi}$ się $\mathrm{k}\mathrm{a}\dot{\mathrm{z}}$ da $\mathrm{w}$ innej grupie.

3. Nie wykonując dzielenia udowodnič, $\dot{\mathrm{z}}\mathrm{e}$ wielomian $(x^{2}+x+1)^{3}-x^{6}-x^{3}-1$ dzieli

się bez reszty przez trójmian $(x+1)^{2}$

4. Wyznaczyč równanie okręgu $0$ promieniu $r$ stycznego do paraboli $y=x^{2}\mathrm{w}$ dwóch

punktach. Dla jakiego $r$ zadanie ma rozwiqzanie? Sporządzič rysunek przyjmujac

$r=3/2.$

5. Stosując zasadę indukcji matematycznej udowodnič prawdziwośč wzoru

$\left(\begin{array}{l}
2\\
2
\end{array}\right) - \left(\begin{array}{l}
3\\
2
\end{array}\right) + \left(\begin{array}{l}
4\\
2
\end{array}\right) - \left(\begin{array}{l}
5\\
2
\end{array}\right) +\ldots+\left(\begin{array}{l}
2n\\
2
\end{array}\right) =n^{2},$

$n\geq 1.$

6. Rozwiązač nierównośč:

$\log_{x}(1-6x^{2})\geq 1.$

7. Środek $S$ okręgu wpisanego $\mathrm{w}$ trapez ABCD jest odlegfy od wierzchofka $B\mathrm{o}SB=$

$\mathrm{d}$, a krótsze ramię $\overline{BC}$ ma dlugośč $BC = \mathrm{c}$. Punkt styczności okręgu $\mathrm{z}$ krótszą

podstawą dzieli ją $\mathrm{w}$ stosunku 1:2. Ob1iczyč po1e tego trapezu. Wykonač rysunek

dla $\mathrm{c}=5\mathrm{i}\mathrm{d}=4.$

8. Wszystkie ściany równoległościanu są rombami $0$ boku $a\mathrm{i}$ kącie ostrym $\beta$. Obliczyč

objętośč tego równoleglościanu. Sporz$\Phi$dzič rysunek. Obliczenia poprzeč stosownym

dowodem.





PRACA KONTROLNA nr 4

styczeń 2002r

l. Obliczyč granicę ciągu 0 wyrazie ogó1nym

{\it an} $=$ -2{\it n}22$++$22{\it n}$+$4 1$++$.. .. . $+.\ +$222{\it n}2{\it n}.

2. Wyznaczyč równanie prostej prostopadfej do danej $2x+3y+3=0\mathrm{i}\mathrm{l}\mathrm{e}\dot{\mathrm{z}}$ qcej $\mathrm{w}$ równej

odleglości od dwóch danych punktów $A(-1,1)\mathrm{i}B(3,3)$. Sporządzič rysunek.

3. Tworząca stozka ma dfugośč $l\mathrm{i}$ widač ją ze środka kuli wpisanej $\mathrm{w}$ ten stozek pod

kątem $\alpha$. Obliczyč objętośč $\mathrm{i}$ kąt rozwarcia stozka. Określič dziedzinę kąta $\alpha.$

4. Bolek kupil jeden długopis $\mathrm{i} k$ zeszytów $\mathrm{i}$ zapłacił $k\mathrm{z}l\mathrm{i}$ 50 gr, a Lolek kupil $k$

dlugopisów $\mathrm{i} 4$ zeszyty $\mathrm{i}$ zapfacif 2, $5k$ zł. Wyznaczyč cenę dfugopisu $\mathrm{i}$ zeszytu $\mathrm{w}$

zalezności od parametru $k$. Znalez/č wszystkie $\mathrm{m}\mathrm{o}\dot{\mathrm{z}}$ liwe wartości tych cen wiedzqc, $\dot{\mathrm{z}}\mathrm{e}$

zeszyt kosztuje nie mniej $\mathrm{n}\mathrm{i}\dot{\mathrm{z}} 50$ gr, długopis jest drozszy od zeszytu, a ceny obydwu

artykułów wyrazają się $\mathrm{w}$ pełnych złotych $\mathrm{i}$ dziesiątkach groszy.

5. Rozwiazač nierównośč:

$\mathrm{t}\mathrm{g}^{3}x\geq\sin 2x.$

6. $\dot{\mathrm{Z}}$ arówki są sprzedawane $\mathrm{w}$ opakowaniach po 6 sztuk. Prawdopodobieństwo, $\dot{\mathrm{z}}\mathrm{e}$ po-

jedyncza $\dot{\mathrm{z}}$ arówka jest sprawna wynosi $\displaystyle \frac{2}{3}$. Jakie jest prawdopodobieństwo tego, $\dot{\mathrm{z}}\mathrm{e}$

$\mathrm{w}$ jednym opakowaniu znajdą się co najmniej 4 sprawne $\dot{\mathrm{z}}$ arówki. $\mathrm{O}$ ile wzrośnie

to prawdopodobieństwo, jeśli jedna, wylosowana $\mathrm{z}$ opakowania $\dot{\mathrm{z}}$ arówka okazafa się

sprawna.

7. Prosta styczna $\mathrm{w}$ punkcie $P$ do okręgu $0$ promieniu 2 $\mathrm{i}$ pólprosta wychodząca ze

środka okręgu mająca $\mathrm{z}$ okręgiem punkt wspólny $S$ przecinają się $\mathrm{w}$ punkcie $A$ pod

kątem $60^{0}$ Znalez/č promień okręgu stycznego do odcinków $AP$, {\it AS} $\mathrm{i}$ łuku $PS.$

Wykonač odpowiedni rysunek.

8. $\mathrm{W}$ ostrosłupie prawidłowym, którego podstawą jest kwadrat, pole $\mathrm{k}\mathrm{a}\dot{\mathrm{z}}$ dej $\mathrm{z}$ pięciu

ścian wynosi l. Ostrosfup ten ścięto pfaszczyzną równolegfq do podstawy $\mathrm{t}\mathrm{a}\mathrm{k}$, aby

uzyskač maksymalny stosunek objętości do pola powierzchni cafkowitej. Obliczyč

pole powierzchni całkowitej otrzymanego ostrosłupa ściętego. Rozwiazanie zilustro-

wač rysunkiem.





PRACA KONTROLNA nr 5

luty $2002\mathrm{r}$

1. $\mathrm{W}$ czworokącie ABCD dane są wktory $AB=\rightarrow(2,-1), BC=\rightarrow(3,3), c^{\rightarrow}D=(-4,1).$

Punkty $K\mathrm{i}M$ są środkami boków $\overline{CD}$ oraz $\overline{AD}$. Posługując się rachunkiem wekto-

rowym obliczyč pole trójkata $KMB$. Wykonač rysunek.

2. Krawędzie oraz przekątna prostopadlościanu $\mathrm{t}\mathrm{w}\mathrm{o}\mathrm{r}\mathrm{z}\Phi$ cztery kolejne wyrazy ciągu

arytmetycznego. Wyznaczyč sumę długości wszystkich krawędzi tego prostopadło-

ścianu, jeśli przekątna ma dfugośč 7 cm.

3. Na pfaszczy $\acute{\mathrm{z}}\mathrm{n}\mathrm{i}\mathrm{e}$ Oxy dane są zbiory:

$A=\{(x,y):y\leq\sqrt{5x-x^{2}}\},B_{s}=\{(x,y):3x+4y=s\}.$

Dla jakich wartości parametru $s$ zbiór $A\cap B_{s}$ nie jest pusty? Sporządzič rysunek.

4. Działka gruntu ma kształt trapezu $0$ bokach 20 $\mathrm{m}, 30\mathrm{m}, 40\mathrm{m}\mathrm{i}60\mathrm{m}$. Właściciel

dziafki twierdzi, $\dot{\mathrm{z}}\mathrm{e}$ polejego dzialki wynosi ponad ll arów. Czy wfaściciel ma rację?

Jeśli tak, to narysowač plan działki $\mathrm{w}$ skali 1:1000 $\mathrm{i}$ podač dokladną wartośčjej pola.

5. Dane jest równanie kwadratowe $\mathrm{z}$ parametrem $m$:

$(m+2)x^{2}+4\sqrt{m}x+(m-3)=0.$

Dla jakiej wartości parametru $m$ kwadrat róznicy pierwiastków rzeczywistych tego

równania jest największy. Podač tę największą wartośč.

6. Stosując zasadę indukcji matematycznej udowodnič, $\dot{\mathrm{z}}\mathrm{e}$ dla $\mathrm{k}\mathrm{a}\dot{\mathrm{z}}$ dego $n \geq 2$ liczba

$2^{2^{n}}-6$ jest podzielna przez 10.

7. Rozwiązač uklad równań

$\left\{\begin{array}{l}
\mathrm{t}\mathrm{g}x+\mathrm{t}\mathrm{g}y=4\\
\cos(x+y)+\cos(x-y)=\frac{1}{2}
\end{array}\right.$

dla $x, y\in[-\pi,\pi].$

8. Równoramienny trójkqt prostokqtny $ABC$ zgięto wzdłuz środkowej $\overline{CD}$ wychodzą-

cej $\mathrm{z}$ wierzchofka kąta prostego $C\mathrm{t}\mathrm{a}\mathrm{k}$, aby obie pofowy tego trójk$\Phi$ta utworzyfy

kąt $60^{0}$ Obliczyč sinusy wszystkich kątów dwuściennych otrzymanego czworościanu

ABCD. Wykonač odpowiednie rysunki $\mathrm{i}$ uzasadnič obliczenia.





PRACA KONTROLNA nr 6

marzec 2002r

l. Wyznaczyč wszystkie wartości parametru rzeczywistego $m$, dla których osią symetrii

wykresu funkcji $p(x)=(m^{2}-2m)x^{2}-(2m-4)x+3$ jest prosta $x=m$. Wykonač

rysunek.

2. $\mathrm{Z}$ kuli $0$ środku $\mathrm{w}$ zerze $\mathrm{i}$ promieniu $R$ wycięto ósmą jej częśč trzema płaszczyznami

ukfadu wspófrzędnych. $\mathrm{W}$ tak $\mathrm{o}\mathrm{t}\mathrm{r}\mathrm{z}\mathrm{y}\mathrm{m}\mathrm{a}\mathrm{n}\Phi$ bryfę wpisano kulę. Obliczyč stosunek

pola powierzchni tej kuli do pola powierzchni bryły.

3. $\mathrm{W}$ trzech pustych urnach $K, \mathrm{L}, \mathrm{M}$ rozmieszczamy losowo 4 rózne ku1e. Ob1iczyč

prawdopodobieństwo tego, $\dot{\mathrm{z}}\mathrm{e}\dot{\mathrm{z}}$ adna $\mathrm{z}$ urn $K\mathrm{i}\mathrm{L}$ nie pozostanie pusta.

4. Dane sa punkty $A(2,6), B(-2,6)\mathrm{i}C(0,0)$, Wyznaczyč równanie linii zawierajacej

wszystkie punkty trójkąta $ABC$, dla których suma kwadratów ich odlegfości od

trzech boków jest stala $\mathrm{i}$ wynosi 9. Sporządzič rysunek.

5. Sporządzič dokładny wykres $\mathrm{i}$ napisač równania asymptot funkcji

$f(x)=\displaystyle \frac{(x+1)^{2}-1}{x|x-1|}$

nie przeprowadzając badania jej przebiegu.

6. Rozwiqzač nierównośč:

$|x|^{2x-1}\displaystyle \leq\frac{1}{x^{2}}.$

7. Styczna do wykresu funkcji $f(x)=\sqrt{3+x}+\sqrt{3-x}\mathrm{w}$ punkcie $A(x_{0},f(x_{0}))$ przecina

oś $\mathrm{x}\mathrm{w}$ punkcie $P$, a oś $\mathrm{y}\mathrm{w}$ punkcie $Q\mathrm{t}\mathrm{a}\mathrm{k}, \dot{\mathrm{z}}\mathrm{e}OP=OQ$. Wyznaczyč $x_{0}.$

8. Trójkat równoboczny $0$ boku $a$ przecięto prostq $l$ na dwie figury, których stosunek

pól jest równy 1:5. Prosta ta przecina bok $\overline{AC}\mathrm{w}$ punkcie $D$ pod kątem $15^{0}$, a bok

$\overline{AB}\mathrm{w}$ punkcie $E$. Wykazač, $\dot{\mathrm{z}}\mathrm{e}AD+AE=a.$





PRACA KONTROLNA nr 7

kwiecień $2002\mathrm{r}$

l. Sześcian $0$ krawędzi dlugości 3 cm ma taką samą objętośčjak dwa sześciany, których

suma dfugości obydwu krawędzi wynosi 4 cm. $\mathrm{O}$ ile $\mathrm{c}\mathrm{m}^{2}$ pole powierzchni $\mathrm{d}\mathrm{u}\dot{\mathrm{z}}$ ego

sześcianu jest mniejsze od sumy pól powierzchni dwóch mniejszych sześcianów.

2. ObliczyČ tangens kąta utworzonego przez przekątne czworokata $0$ wierzchołkach

$\mathrm{A}(1,1), \mathrm{B}(2,0), \mathrm{C}(2,4), \mathrm{D}(0,6)$. Rozwiązanie zilustrowaČ rysunkiem.

3. $\mathrm{W}$ trójkąt prostokątny wpisano okrąg, a $\mathrm{w}$ okrqg ten wpisano podobny trójkąt pro-

stokątny. Wyznaczyč cosinusy kątów ostrych trójk$\Phi$ta, jeśli wiadomo, $\dot{\mathrm{z}}\mathrm{e}$ stosunek

pól obu trójkątów wynosi 9.

4. Wykazač, $\dot{\mathrm{z}}\mathrm{e}$ ciag $a_{n}=\sqrt{n(n+1)}-n$ jest rosnący. Obliczyč jego granice.

5. Rozwiązač nierównośč:

$2\displaystyle \cos^{2}\frac{x}{4}>1.$

6. Rozwiązač równanie

$\displaystyle \log_{2}(1-x)+\log_{4}(x+4)=\log_{4}(x^{3}-x^{2}-3x+5)+\frac{1}{2}$

nie wyznaczając dziedziny $\mathrm{w}$ sposób jawny.

7. $\mathrm{W}$ kulę $0$ promieniu $R$ wpisano stozek $0$ największej objętości. Wyznaczyč promień

podstawy $r\mathrm{i}$ wysokośč $h$ tego stozka. Sporzqdzič rysunek.

8. Znalez/č równania wszystkich prostych, które są styczne jednocześnie do krzywych

$y=-x^{2},y=x^{2}-8x+18.$

Sporządzič rysunek.



\end{document}