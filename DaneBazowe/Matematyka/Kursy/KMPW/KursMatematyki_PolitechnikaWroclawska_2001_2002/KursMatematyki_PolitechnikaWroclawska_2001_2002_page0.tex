\documentclass[a4paper,12pt]{article}
\usepackage{latexsym}
\usepackage{amsmath}
\usepackage{amssymb}
\usepackage{graphicx}
\usepackage{wrapfig}
\pagestyle{plain}
\usepackage{fancybox}
\usepackage{bm}

\begin{document}

KORESPONDENCYJNY KURS Z MATEMATYKI

PRACA KONTROLNA nr l

$\mathrm{p}\mathrm{a}\acute{\mathrm{z}}$dziernik 2$001\mathrm{r}$

l. Dwaj rowerzyści wyruszyli jednocześnie $\mathrm{w}$ drogę, jeden $\mathrm{z}$ A do $\mathrm{B}$, drugi $\mathrm{z}\mathrm{B}$ do $\mathrm{A}$

$\mathrm{i}$ spotkali się po jednej godzinie. Pierwszy $\mathrm{z}$ nich przebywał $\mathrm{w}$ ciągu godziny $03$

km więcej $\mathrm{n}\mathrm{i}\dot{\mathrm{z}}$ drugi $\mathrm{i}$ przyjechal do celu $027$ minut wcześniej $\mathrm{n}\mathrm{i}\dot{\mathrm{z}}$ drugi. Jakie byfy

prędkości obu rowerzystów $\mathrm{i}$ jaka jest odległośč AB?

2. Rozwiązač nierównośč:

$\sqrt{x^{2}-3}>\underline{2}.$

$x$

3. Rysunek przedstawia dach budynku $\mathrm{w}$ rzucie poziomym. $\mathrm{K}\mathrm{a}\dot{\mathrm{z}}$ da $\mathrm{z}$ plaszczyzn nachy-
\begin{center}
\includegraphics[width=48.156mm,height=24.132mm]{./KursMatematyki_PolitechnikaWroclawska_2001_2002_page0_images/image001.eps}
\end{center}
lona jest do płaszczyzny poziomej pod $\mathrm{k}$ tem $30^{0}$ Dłu-

gossc dachu wynosi 18 $\mathrm{m}$, a szerokosc 9 $\mathrm{m}$. Obliczyc po-

le powierzchni dachu oraz cafkowit kubaturę strychu $\mathrm{w}$

tym budynku.

4. Pewna firma przeprowadza co kwartal regulację plac dla swoich pracowników rewa-

$1\mathrm{o}\mathrm{r}\mathrm{y}\mathrm{z}\mathrm{u}\mathrm{j}_{\Phi}\mathrm{c}$ je zgodnie ze wska $\acute{\mathrm{z}}\mathrm{n}\mathrm{i}\mathrm{k}\mathrm{i}\mathrm{e}\mathrm{m}$ inflacji, który jest stafy $\mathrm{i}$ wynosi 1,5\% kwar-

talnie, oraz doliczając stałą kwotę podwyzki 16 $\mathrm{z}\mathrm{l}\mathrm{p}. \mathrm{W}$ styczniu 2001 pan Kowa1ski

otrzymał wynagrodzenie 1600 $\mathrm{z}\mathrm{l}\mathrm{p}$. Jaką pensję otrzyma $\mathrm{w}$ kwietniu 2002? Wyzna-

czyč wzór ogólny na pensję $w_{n}$ pana Kowalskiego $\mathrm{w}\mathrm{n}$-tym kwartale przyjmując, $\dot{\mathrm{z}}\mathrm{e}$

$w_{1}=1600$jest placą $\mathrm{w}$ pierwszym kwartale 2001. Ob1iczyč średniq miesięcznq płacę

pana Kowalskiego $\mathrm{w}$ 2002 roku.

5. Wyznaczyč funkcję odwrotną do $f(x) =x^{3}, x\in R$. Korzystając $\mathrm{z}$ tego wykonač

staranny wykres funkcji $h(x) =$
\begin{center}
\includegraphics[width=36.732mm,height=7.620mm]{./KursMatematyki_PolitechnikaWroclawska_2001_2002_page0_images/image002.eps}
\end{center}
$(|x|-- 1)+1.$

6. Rozwiązač równanie:

-cs  oins 24{\it xx} $=$ 1.

7. Dany jest trójkąt $0$ wierzcholkach $A(-2,1), B(-1,-6), C(2,5)$. Posługując się

rachunkiem wektorowym obliczyč cosinus kąta pomiędzy dwusieczną kata $A\mathrm{i}$ środ-

kową boku $\overline{BC}$. Wykonač rysunek.

8. Przeprowadzič badanie przebiegu $\mathrm{i}$ wykonač wykres funkcji

$f(x)=x+\displaystyle \frac{x}{x-1}+\frac{x}{(x-1)^{2}}+\frac{x}{(x-1)^{3}}+$
\end{document}
