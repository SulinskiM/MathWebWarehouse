\documentclass[a4paper,12pt]{article}
\usepackage{latexsym}
\usepackage{amsmath}
\usepackage{amssymb}
\usepackage{graphicx}
\usepackage{wrapfig}
\pagestyle{plain}
\usepackage{fancybox}
\usepackage{bm}

\begin{document}

PRACA KONTROLNA nr 2

listopad $2001\mathrm{r}$

l. Cena ll paliwa została zmniejszona $0$ 15\%. Po dwóch tygodniach dokonano kolej-

nej zmiany ceny paliwa zwiększając ją $0$ 15\%. $\mathrm{O}$ ile procent końcowa cena paliwa

rózni $\mathrm{s}\mathrm{i}_{9}$ od początkowej?

2. Wyznaczyč $\mathrm{i}$ narysowač zbiór złozony $\mathrm{z}$ punktów $(x,y)$ płaszczyzny spełniających

warunek

$x^{2}+y^{2}=8|x|+6|y|.$

3. Wysokośč ostrosfupa trójkątnego prawidfowego wynosi $h$, a kąt między wysokościa-

mi ścian bocznych jest równy $ 2\alpha$. Obliczyč pole powierzchni bocznej tego ostrosłupa.

Sporządzič odpowiednie rysunki.

4. $\mathrm{Z}$ arkusza blachy $\mathrm{w}$ kształcie równoległoboku $0$ bokach 30 cm $\mathrm{i}60$ cm $\mathrm{i}$ kącie ostrym

$60^{0}$ nalezy odciąč dwa przeciwlegfe trójkqtne naroza $\mathrm{t}\mathrm{a}\mathrm{k}$, aby powstaf romb $0\mathrm{m}\mathrm{o}\dot{\mathrm{z}}$-

liwie największym polu. Określič przez który punkt dfuzszego boku nalez $\mathrm{y}$ prze-

prowadzič cięcie oraz obliczyč kąt ostry otrzymanego rombu zaokrqglajqc wynik do

jednej minuty kątowej.

5. Rozwiązač równanie

$2^{\log_{\sqrt{2}}x}=(\sqrt{2})^{\log_{x}2}$

6. Wyznaczyč dziedzinę i zbiór wartości funkcji

$f(x)=\displaystyle \frac{4}{\sin x+2\cos x+3}.$

7. Znalez$\acute{}$č wszystkie wartości parametru $p$, dla których równanie

$px^{4}-4x^{2}+p+1=0$

ma dwa rózne rozwiązania.

8. Wyznaczyč tangens $\mathrm{k}_{\Phi^{\mathrm{t}\mathrm{a}}}$, pod którym styczna do wykresu funkcji $f(x) = \displaystyle \frac{8}{x^{2}+3} \mathrm{w}$

punkcie $A(3,\displaystyle \frac{2}{3})$ przecina wykres tej funkcji.
\end{document}
