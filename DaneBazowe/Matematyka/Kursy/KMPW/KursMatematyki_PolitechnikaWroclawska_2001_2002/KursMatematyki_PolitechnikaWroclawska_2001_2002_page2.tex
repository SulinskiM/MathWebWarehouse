\documentclass[a4paper,12pt]{article}
\usepackage{latexsym}
\usepackage{amsmath}
\usepackage{amssymb}
\usepackage{graphicx}
\usepackage{wrapfig}
\pagestyle{plain}
\usepackage{fancybox}
\usepackage{bm}

\begin{document}

PRACA KONTROLNA nr 3

grudzień $2001\mathrm{r}$

l. Dla jakich wartości $\sin x$ liczby $\sin x, \cos x, \sin 2x$ ($\mathrm{w}$ podanym porządku) są ko-

lejnymi wyrazami ciągu geometrycznego. Wyznaczyč czwarte wyrazy tych ciągów.

2. $\mathrm{W}$ pewnych zawodach sportowych startuje 16 druzyn. $\mathrm{W}$ eliminacjach są one losowo

dzielone na 4 grupy po 4 druzyny $\mathrm{k}\mathrm{a}\dot{\mathrm{z}}$ da grupa. Obliczyč prawdopodobieństwo tego,

$\dot{\mathrm{z}}\mathrm{e}$ trzy zwycięskie druzyny $\mathrm{z}$ poprzednich zawodów $\mathrm{z}\mathrm{n}\mathrm{a}\mathrm{j}\mathrm{d}_{\Phi}$ się $\mathrm{k}\mathrm{a}\dot{\mathrm{z}}$ da $\mathrm{w}$ innej grupie.

3. Nie wykonując dzielenia udowodnič, $\dot{\mathrm{z}}\mathrm{e}$ wielomian $(x^{2}+x+1)^{3}-x^{6}-x^{3}-1$ dzieli

się bez reszty przez trójmian $(x+1)^{2}$

4. Wyznaczyč równanie okręgu $0$ promieniu $r$ stycznego do paraboli $y=x^{2}\mathrm{w}$ dwóch

punktach. Dla jakiego $r$ zadanie ma rozwiqzanie? Sporządzič rysunek przyjmujac

$r=3/2.$

5. Stosując zasadę indukcji matematycznej udowodnič prawdziwośč wzoru

$\left(\begin{array}{l}
2\\
2
\end{array}\right) - \left(\begin{array}{l}
3\\
2
\end{array}\right) + \left(\begin{array}{l}
4\\
2
\end{array}\right) - \left(\begin{array}{l}
5\\
2
\end{array}\right) +\ldots+\left(\begin{array}{l}
2n\\
2
\end{array}\right) =n^{2},$

$n\geq 1.$

6. Rozwiązač nierównośč:

$\log_{x}(1-6x^{2})\geq 1.$

7. Środek $S$ okręgu wpisanego $\mathrm{w}$ trapez ABCD jest odlegfy od wierzchofka $B\mathrm{o}SB=$

$\mathrm{d}$, a krótsze ramię $\overline{BC}$ ma dlugośč $BC = \mathrm{c}$. Punkt styczności okręgu $\mathrm{z}$ krótszą

podstawą dzieli ją $\mathrm{w}$ stosunku 1:2. Ob1iczyč po1e tego trapezu. Wykonač rysunek

dla $\mathrm{c}=5\mathrm{i}\mathrm{d}=4.$

8. Wszystkie ściany równoległościanu są rombami $0$ boku $a\mathrm{i}$ kącie ostrym $\beta$. Obliczyč

objętośč tego równoleglościanu. Sporz$\Phi$dzič rysunek. Obliczenia poprzeč stosownym

dowodem.
\end{document}
