\documentclass[a4paper,12pt]{article}
\usepackage{latexsym}
\usepackage{amsmath}
\usepackage{amssymb}
\usepackage{graphicx}
\usepackage{wrapfig}
\pagestyle{plain}
\usepackage{fancybox}
\usepackage{bm}

\begin{document}

PRACA KONTROLNA nr 4

styczeń 2002r

l. Obliczyč granicę ciągu 0 wyrazie ogó1nym

{\it an} $=$ -2{\it n}22$++$22{\it n}$+$4 1$++$.. .. . $+.\ +$222{\it n}2{\it n}.

2. Wyznaczyč równanie prostej prostopadfej do danej $2x+3y+3=0\mathrm{i}\mathrm{l}\mathrm{e}\dot{\mathrm{z}}$ qcej $\mathrm{w}$ równej

odleglości od dwóch danych punktów $A(-1,1)\mathrm{i}B(3,3)$. Sporządzič rysunek.

3. Tworząca stozka ma dfugośč $l\mathrm{i}$ widač ją ze środka kuli wpisanej $\mathrm{w}$ ten stozek pod

kątem $\alpha$. Obliczyč objętośč $\mathrm{i}$ kąt rozwarcia stozka. Określič dziedzinę kąta $\alpha.$

4. Bolek kupil jeden długopis $\mathrm{i} k$ zeszytów $\mathrm{i}$ zapłacił $k\mathrm{z}l\mathrm{i}$ 50 gr, a Lolek kupil $k$

dlugopisów $\mathrm{i} 4$ zeszyty $\mathrm{i}$ zapfacif 2, $5k$ zł. Wyznaczyč cenę dfugopisu $\mathrm{i}$ zeszytu $\mathrm{w}$

zalezności od parametru $k$. Znalez/č wszystkie $\mathrm{m}\mathrm{o}\dot{\mathrm{z}}$ liwe wartości tych cen wiedzqc, $\dot{\mathrm{z}}\mathrm{e}$

zeszyt kosztuje nie mniej $\mathrm{n}\mathrm{i}\dot{\mathrm{z}} 50$ gr, długopis jest drozszy od zeszytu, a ceny obydwu

artykułów wyrazają się $\mathrm{w}$ pełnych złotych $\mathrm{i}$ dziesiątkach groszy.

5. Rozwiazač nierównośč:

$\mathrm{t}\mathrm{g}^{3}x\geq\sin 2x.$

6. $\dot{\mathrm{Z}}$ arówki są sprzedawane $\mathrm{w}$ opakowaniach po 6 sztuk. Prawdopodobieństwo, $\dot{\mathrm{z}}\mathrm{e}$ po-

jedyncza $\dot{\mathrm{z}}$ arówka jest sprawna wynosi $\displaystyle \frac{2}{3}$. Jakie jest prawdopodobieństwo tego, $\dot{\mathrm{z}}\mathrm{e}$

$\mathrm{w}$ jednym opakowaniu znajdą się co najmniej 4 sprawne $\dot{\mathrm{z}}$ arówki. $\mathrm{O}$ ile wzrośnie

to prawdopodobieństwo, jeśli jedna, wylosowana $\mathrm{z}$ opakowania $\dot{\mathrm{z}}$ arówka okazafa się

sprawna.

7. Prosta styczna $\mathrm{w}$ punkcie $P$ do okręgu $0$ promieniu 2 $\mathrm{i}$ pólprosta wychodząca ze

środka okręgu mająca $\mathrm{z}$ okręgiem punkt wspólny $S$ przecinają się $\mathrm{w}$ punkcie $A$ pod

kątem $60^{0}$ Znalez/č promień okręgu stycznego do odcinków $AP$, {\it AS} $\mathrm{i}$ łuku $PS.$

Wykonač odpowiedni rysunek.

8. $\mathrm{W}$ ostrosłupie prawidłowym, którego podstawą jest kwadrat, pole $\mathrm{k}\mathrm{a}\dot{\mathrm{z}}$ dej $\mathrm{z}$ pięciu

ścian wynosi l. Ostrosfup ten ścięto pfaszczyzną równolegfq do podstawy $\mathrm{t}\mathrm{a}\mathrm{k}$, aby

uzyskač maksymalny stosunek objętości do pola powierzchni cafkowitej. Obliczyč

pole powierzchni całkowitej otrzymanego ostrosłupa ściętego. Rozwiazanie zilustro-

wač rysunkiem.
\end{document}
