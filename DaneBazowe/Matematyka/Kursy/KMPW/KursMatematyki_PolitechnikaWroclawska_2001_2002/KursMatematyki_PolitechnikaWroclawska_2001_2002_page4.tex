\documentclass[a4paper,12pt]{article}
\usepackage{latexsym}
\usepackage{amsmath}
\usepackage{amssymb}
\usepackage{graphicx}
\usepackage{wrapfig}
\pagestyle{plain}
\usepackage{fancybox}
\usepackage{bm}

\begin{document}

PRACA KONTROLNA nr 5

luty $2002\mathrm{r}$

1. $\mathrm{W}$ czworokącie ABCD dane są wktory $AB=\rightarrow(2,-1), BC=\rightarrow(3,3), c^{\rightarrow}D=(-4,1).$

Punkty $K\mathrm{i}M$ są środkami boków $\overline{CD}$ oraz $\overline{AD}$. Posługując się rachunkiem wekto-

rowym obliczyč pole trójkata $KMB$. Wykonač rysunek.

2. Krawędzie oraz przekątna prostopadlościanu $\mathrm{t}\mathrm{w}\mathrm{o}\mathrm{r}\mathrm{z}\Phi$ cztery kolejne wyrazy ciągu

arytmetycznego. Wyznaczyč sumę długości wszystkich krawędzi tego prostopadło-

ścianu, jeśli przekątna ma dfugośč 7 cm.

3. Na pfaszczy $\acute{\mathrm{z}}\mathrm{n}\mathrm{i}\mathrm{e}$ Oxy dane są zbiory:

$A=\{(x,y):y\leq\sqrt{5x-x^{2}}\},B_{s}=\{(x,y):3x+4y=s\}.$

Dla jakich wartości parametru $s$ zbiór $A\cap B_{s}$ nie jest pusty? Sporządzič rysunek.

4. Działka gruntu ma kształt trapezu $0$ bokach 20 $\mathrm{m}, 30\mathrm{m}, 40\mathrm{m}\mathrm{i}60\mathrm{m}$. Właściciel

dziafki twierdzi, $\dot{\mathrm{z}}\mathrm{e}$ polejego dzialki wynosi ponad ll arów. Czy wfaściciel ma rację?

Jeśli tak, to narysowač plan działki $\mathrm{w}$ skali 1:1000 $\mathrm{i}$ podač dokladną wartośčjej pola.

5. Dane jest równanie kwadratowe $\mathrm{z}$ parametrem $m$:

$(m+2)x^{2}+4\sqrt{m}x+(m-3)=0.$

Dla jakiej wartości parametru $m$ kwadrat róznicy pierwiastków rzeczywistych tego

równania jest największy. Podač tę największą wartośč.

6. Stosując zasadę indukcji matematycznej udowodnič, $\dot{\mathrm{z}}\mathrm{e}$ dla $\mathrm{k}\mathrm{a}\dot{\mathrm{z}}$ dego $n \geq 2$ liczba

$2^{2^{n}}-6$ jest podzielna przez 10.

7. Rozwiązač uklad równań

$\left\{\begin{array}{l}
\mathrm{t}\mathrm{g}x+\mathrm{t}\mathrm{g}y=4\\
\cos(x+y)+\cos(x-y)=\frac{1}{2}
\end{array}\right.$

dla $x, y\in[-\pi,\pi].$

8. Równoramienny trójkqt prostokqtny $ABC$ zgięto wzdłuz środkowej $\overline{CD}$ wychodzą-

cej $\mathrm{z}$ wierzchofka kąta prostego $C\mathrm{t}\mathrm{a}\mathrm{k}$, aby obie pofowy tego trójk$\Phi$ta utworzyfy

kąt $60^{0}$ Obliczyč sinusy wszystkich kątów dwuściennych otrzymanego czworościanu

ABCD. Wykonač odpowiednie rysunki $\mathrm{i}$ uzasadnič obliczenia.
\end{document}
