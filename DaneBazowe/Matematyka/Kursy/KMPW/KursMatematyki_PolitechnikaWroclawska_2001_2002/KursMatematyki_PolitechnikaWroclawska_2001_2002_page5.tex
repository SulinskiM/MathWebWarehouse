\documentclass[a4paper,12pt]{article}
\usepackage{latexsym}
\usepackage{amsmath}
\usepackage{amssymb}
\usepackage{graphicx}
\usepackage{wrapfig}
\pagestyle{plain}
\usepackage{fancybox}
\usepackage{bm}

\begin{document}

PRACA KONTROLNA nr 6

marzec 2002r

l. Wyznaczyč wszystkie wartości parametru rzeczywistego $m$, dla których osią symetrii

wykresu funkcji $p(x)=(m^{2}-2m)x^{2}-(2m-4)x+3$ jest prosta $x=m$. Wykonač

rysunek.

2. $\mathrm{Z}$ kuli $0$ środku $\mathrm{w}$ zerze $\mathrm{i}$ promieniu $R$ wycięto ósmą jej częśč trzema płaszczyznami

ukfadu wspófrzędnych. $\mathrm{W}$ tak $\mathrm{o}\mathrm{t}\mathrm{r}\mathrm{z}\mathrm{y}\mathrm{m}\mathrm{a}\mathrm{n}\Phi$ bryfę wpisano kulę. Obliczyč stosunek

pola powierzchni tej kuli do pola powierzchni bryły.

3. $\mathrm{W}$ trzech pustych urnach $K, \mathrm{L}, \mathrm{M}$ rozmieszczamy losowo 4 rózne ku1e. Ob1iczyč

prawdopodobieństwo tego, $\dot{\mathrm{z}}\mathrm{e}\dot{\mathrm{z}}$ adna $\mathrm{z}$ urn $K\mathrm{i}\mathrm{L}$ nie pozostanie pusta.

4. Dane sa punkty $A(2,6), B(-2,6)\mathrm{i}C(0,0)$, Wyznaczyč równanie linii zawierajacej

wszystkie punkty trójkąta $ABC$, dla których suma kwadratów ich odlegfości od

trzech boków jest stala $\mathrm{i}$ wynosi 9. Sporządzič rysunek.

5. Sporządzič dokładny wykres $\mathrm{i}$ napisač równania asymptot funkcji

$f(x)=\displaystyle \frac{(x+1)^{2}-1}{x|x-1|}$

nie przeprowadzając badania jej przebiegu.

6. Rozwiqzač nierównośč:

$|x|^{2x-1}\displaystyle \leq\frac{1}{x^{2}}.$

7. Styczna do wykresu funkcji $f(x)=\sqrt{3+x}+\sqrt{3-x}\mathrm{w}$ punkcie $A(x_{0},f(x_{0}))$ przecina

oś $\mathrm{x}\mathrm{w}$ punkcie $P$, a oś $\mathrm{y}\mathrm{w}$ punkcie $Q\mathrm{t}\mathrm{a}\mathrm{k}, \dot{\mathrm{z}}\mathrm{e}OP=OQ$. Wyznaczyč $x_{0}.$

8. Trójkat równoboczny $0$ boku $a$ przecięto prostq $l$ na dwie figury, których stosunek

pól jest równy 1:5. Prosta ta przecina bok $\overline{AC}\mathrm{w}$ punkcie $D$ pod kątem $15^{0}$, a bok

$\overline{AB}\mathrm{w}$ punkcie $E$. Wykazač, $\dot{\mathrm{z}}\mathrm{e}AD+AE=a.$
\end{document}
