\documentclass[a4paper,12pt]{article}
\usepackage{latexsym}
\usepackage{amsmath}
\usepackage{amssymb}
\usepackage{graphicx}
\usepackage{wrapfig}
\pagestyle{plain}
\usepackage{fancybox}
\usepackage{bm}

\begin{document}

PRACA KONTROLNA nr 7

kwiecień $2002\mathrm{r}$

l. Sześcian $0$ krawędzi dlugości 3 cm ma taką samą objętośčjak dwa sześciany, których

suma dfugości obydwu krawędzi wynosi 4 cm. $\mathrm{O}$ ile $\mathrm{c}\mathrm{m}^{2}$ pole powierzchni $\mathrm{d}\mathrm{u}\dot{\mathrm{z}}$ ego

sześcianu jest mniejsze od sumy pól powierzchni dwóch mniejszych sześcianów.

2. ObliczyČ tangens kąta utworzonego przez przekątne czworokata $0$ wierzchołkach

$\mathrm{A}(1,1), \mathrm{B}(2,0), \mathrm{C}(2,4), \mathrm{D}(0,6)$. Rozwiązanie zilustrowaČ rysunkiem.

3. $\mathrm{W}$ trójkąt prostokątny wpisano okrąg, a $\mathrm{w}$ okrqg ten wpisano podobny trójkąt pro-

stokątny. Wyznaczyč cosinusy kątów ostrych trójk$\Phi$ta, jeśli wiadomo, $\dot{\mathrm{z}}\mathrm{e}$ stosunek

pól obu trójkątów wynosi 9.

4. Wykazač, $\dot{\mathrm{z}}\mathrm{e}$ ciag $a_{n}=\sqrt{n(n+1)}-n$ jest rosnący. Obliczyč jego granice.

5. Rozwiązač nierównośč:

$2\displaystyle \cos^{2}\frac{x}{4}>1.$

6. Rozwiązač równanie

$\displaystyle \log_{2}(1-x)+\log_{4}(x+4)=\log_{4}(x^{3}-x^{2}-3x+5)+\frac{1}{2}$

nie wyznaczając dziedziny $\mathrm{w}$ sposób jawny.

7. $\mathrm{W}$ kulę $0$ promieniu $R$ wpisano stozek $0$ największej objętości. Wyznaczyč promień

podstawy $r\mathrm{i}$ wysokośč $h$ tego stozka. Sporzqdzič rysunek.

8. Znalez/č równania wszystkich prostych, które są styczne jednocześnie do krzywych

$y=-x^{2},y=x^{2}-8x+18.$

Sporządzič rysunek.
\end{document}
