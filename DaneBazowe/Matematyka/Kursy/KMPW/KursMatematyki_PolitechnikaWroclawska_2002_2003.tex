\documentclass[a4paper,12pt]{article}
\usepackage{latexsym}
\usepackage{amsmath}
\usepackage{amssymb}
\usepackage{graphicx}
\usepackage{wrapfig}
\pagestyle{plain}
\usepackage{fancybox}
\usepackage{bm}

\begin{document}

KORESPONDENCYJNY KURS Z MATEMATYKI

PRACA KONTROLNA nr l

$\mathrm{p}\mathrm{a}\acute{\mathrm{z}}$dziernik 2$002\mathrm{r}$

l. Narysowač wykres funkcji $y=4+2|x|-x^{2}$ Korzystając $\mathrm{z}$ tego wykresu określič

liczbę rozwi$\Phi$zań równania $4+2|x|-x^{2}=p\mathrm{w}$ zalezności od parametru rzeczywistego

$p.$

2. Pompa napełniająca pusty basen $\mathrm{w}$ pierwszej minucie pracy miała wydajnośč 0,2

$\mathrm{m}^{3}/\mathrm{s}$, a $\mathrm{w}\mathrm{k}\mathrm{a}\dot{\mathrm{z}}$ dej kolejnej minucie jej wydajnośč zwiększano $00,01 \mathrm{m}^{3}/\mathrm{s}$. Pofowa

basenu zostala napełniona po $2n$ minutach, a caly basen po kolejnych $n$ minutach,

gdzie $n$ jest liczbą naturalnq. Wyznaczyč czas napefniania basenu oraz jego pojem-

nośč.

3. Stozek ścięty jest opisany na kuli $0$ promieniu $r=2$ cm. Objętośč kuli stanowi 25\%

objętości stozka. Wyznaczyč średnice podstaw $\mathrm{i}$ dfugośč tworzącej tego stozka.

4. $\mathrm{W}$ trójkącie $ABC$ dane są promień okręgu opisanego $R$, kąt $\angle A=\alpha$ oraz $AB=\displaystyle \frac{8}{5}R.$

Obliczyč pole tego trójkqta.

5. Rozwiązač nierównośč:

$(\sqrt{x})^{\log_{8}x}\geq\sqrt[3]{16x}.$

6. $\mathrm{W}$ czworokącie ABCD odcinki $\overline{AB}\mathrm{i}\overline{BD}$ są prostopadle, $AD = 2AB =a$ oraz

$ AC=\rightarrow \displaystyle \frac{5}{3}AB\rightarrow+\frac{1}{3}AD\rightarrow$. Wyznaczyč cosinus kąta $\angle BCD=\alpha$ oraz obwód czworokąta

ABCD. Sporządzič rysunek.

7. Rozwiązač równanie:

$\displaystyle \frac{1}{\sin x}+\frac{1}{\cos x}=\sqrt{8}.$

8. Wyznaczyč równanie prostej stycznej do wykresu funkcji $y=\displaystyle \frac{1}{x^{2}}\mathrm{w}$ punkcie $P(x_{0},y_{0}),$

$x_{0}>0$, takim, $\dot{\mathrm{z}}\mathrm{e}$ odcinek tej stycznej zawarty $\mathrm{w}$ I čwiartce układu wspólrzędnych

jest najkrótszy. Rozwiązanie zilustrowač stosownym wykresem.

1




PRACA KONTROLNA nr 2

listopad $2002\mathrm{r}$

l. Czy liczby róznych `sfów', jakie $\mathrm{m}\mathrm{o}\dot{\mathrm{z}}$ na utworzyč zmieniając kolejnośč liter $\mathrm{w}$ s{\it l}o-

wach' TANATAN $\mathrm{i}$ AKABARA, są takie same? Uzasadnič odpowied $\acute{\mathrm{z}}$. Przez `sfowo'

rozumiemy tutaj dowolny ciag liter.

2. Reszta $\mathrm{z}$ dzielenia wielomianu $x^{3}+px^{2}-x+q$ przez trójmian $(x+2)^{2}$ wynosi $-x+1.$

Wyznaczyč pierwiastki tego wielomianu.

3. Figura na rysunku ponizej składa się $\mathrm{z}$ łuków $BC, CA$ okręgów $0$ promieniu $a\mathrm{i}$

środkach odpowiednio $\mathrm{w}$ punktach $A, B$, oraz $\mathrm{z}$ odcinka $\overline{AB}0$ dfugości $a$. Obliczyč

promień okręgu stycznego do obu łuków oraz do odcinka $\overline{AB}.$

4. Podstawą pryzmy przedstawionej na rysunku ponizej jest $\mathrm{p}\mathrm{r}\mathrm{o}\mathrm{s}\mathrm{t}\mathrm{o}\mathrm{k}_{\Phi}\mathrm{t}$ ABCD, którego

bok $\overline{AB}$ ma długośč $a$, a bok $\overline{BC}$ długośč $b$, gdzie $a>b$. Wszystkie ściany boczne

pryzmy są nachylone pod kątem $\alpha$ do płaszczyzny podstawy. Obliczyč objetośč tej

pryzmy.

5. Rozwiązač nierównośč

-{\it x}2$<\sqrt{}$5-{\it x}2.

Rozwiązanie zilustrowač wykresami funkcji wystepujqcych po obu stronach nierów-

ności. Zaznaczyč na rysunku otrzymany zbiór rozwiązań.

6. Ciąg $(a_{n})$ jest określony warunkami $\alpha_{1}=4, a_{n+1}=1+2\sqrt{a_{n}}, n\geq 1$. Stosując zasadę

indukcji matematycznej wykazač, $\dot{\mathrm{z}}\mathrm{e}$ ciag $(a_{n})$ jest rosnący oraz dla $n\geq 1$ spefniona

jest nierównośč: $4\leq a_{n}<6.$

7. Na krzywej $0$ równaniu $y=\sqrt{x}$ znalez/č miejsce, którejest połozone najblizej punktu

$P(0,3)$. Sporz$\Phi$dzič rysunek.

8. Wykazač, $\dot{\mathrm{z}}\mathrm{e}$ dla $\mathrm{k}\mathrm{a}\dot{\mathrm{z}}$ dej wartości parametru $\alpha\in R$ równanie kwadratowe

$3x^{2}+4x\sin\alpha-\cos 2\alpha=0$

ma dwa rózne pierwiastki rzeczywiste. Wyznaczyč te wartości parametru $\alpha$, dla

których oba pierwiastki $\mathrm{l}\mathrm{e}\dot{\mathrm{z}}$ ą $\mathrm{w}$ przedziale $(0,1).$

2





PRACA KONTROLNA nr 3

grudzień $2002\mathrm{r}$

l. Suma wyrazów nieskończonego ciqgu geometrycznego zmniejszy się $0$ 25\%, jeśli wy-

kreślimy $\mathrm{z}$ niej skladniki $0$ numerach parzystych niepodzielnych przez 4. Ob1iczyč

sume wszystkich wyrazów tego ciągu wiedząc, $\dot{\mathrm{z}}\mathrm{e}$ jego drugi wyraz wynosi l.

2. $\mathrm{Z}$ kompletu 28 kości do gry $\mathrm{w}$ domino wylosowano dwie kości (bez zwracania).

Obliczyč prawdopodobieństwo tego, $\dot{\mathrm{z}}\mathrm{e}$ kości pasujq do siebie $\mathrm{t}\mathrm{z}\mathrm{n}$. na jednym $\mathrm{z}$ pól

obu kości wystepuje ta sama liczba oczek.

3. Rozwiązač ukfad równań

$\left\{\begin{array}{l}
x\\
5x
\end{array}\right.$

$+2y$

$+my$

3

{\it m}

$\mathrm{w}$ zalezności od parametru rzeczywistego $m$. Wyznaczyč $\mathrm{i}$ narysowač zbiór, jaki

tworzq rozwiązania $(x(m),y(m))$ tego układu, gdy $m$ przebiega zbiór liczb rzeczy-

wistych.

4. $\mathrm{W}$ graniastoslupie prawidłowym sześciokątnym krawędz/ dolnej podstawy $\overline{AB}$ jest

widoczna ze środka górnej podstawy $P$ pod $\mathrm{k}_{\Phi}\mathrm{t}\mathrm{e}\mathrm{m}\alpha$. Wyznaczyč cosinus kąta utwo-

rzonego przez płaszczyznę podstawy $\mathrm{i}$ płaszczyznę zawierającą $\overline{AB}$ oraz przeciwległą

do niej krawęd $\acute{\mathrm{z}}\overline{D'E'}$ górnej podstawy. Obliczenia odpowiednio uzasadnič.

5. Rozwiązač nierównośč

$-1\displaystyle \leq\frac{2^{x+1/2}}{4^{x}-4}\leq 1.$

6. Nie posfugując się tablicami wykazač, $\dot{\mathrm{z}}\mathrm{e}$ tg82030' -tg $7^{0}30'=4+2\sqrt{3}.$

7. Napisač równanie prostej $k$ stycznej do okręgu $x^{2}+y^{2}-2x-2y-3=0\mathrm{w}$ punk-

cie $P(2,3)$. Następnie wyznaczyč równania wszystkich prostych stycznych do tego

okręgu, które tworzą $\mathrm{z}$ prostą $k$ kąt $45^{0}$

8. Dobrač parametry $a>0\mathrm{i}b\in R\mathrm{t}\mathrm{a}\mathrm{k}$, aby funkcja

$f(x)=\displaystyle \{\frac{(a+b}{x^{2}-1}1)+ax-x^{2}$

dla $x\leq a,$

dla $x>a,$

była ciagła i miala pochodnq w punkcie a. Nie przeprowadzając dalszego badania

sporządzič wykres funkcji f(x) oraz stycznej do jej wykresu w punkcie P(a, f(a)).

3





PRACA KONTROLNA nr 4

styczeń $2003\mathrm{r}$

l. Dla jakich wartości parametru rzeczywistego $t$ równanie

$x+3=-(tx+1)^{2}$

ma dokfadnie jedno rozwiązanie.

2. Czworościan foremny $0$ krawędzi $a$ przecięto płaszczyzną równoległą do dwóch prze-

ciwległych krawędzi. Wyrazič pole otrzymanego przekrojujako funkcję długości od-

cinka wyznaczonego przez ten przekrój na jednej $\mathrm{z}$ pozostafych krawędzi. Uzasad-

nič postępowanie. Przedstawič znalezioną funkcję na wykresie $\mathrm{i}$ podačjej największą

wartośč.

3. Zaznaczyč na wykresie zbiór punktów $(x,y)$ pfaszczyzny spelniajqcych warunek

$\log_{xy}|y|\geq 1.$

4. Wyznaczyč równanie linii utworzonej przez wszystkie punkty plaszczyzny, których

odległośč od okręgu $x^{2}+y^{2}=81$ jest $01$ mniejsza $\mathrm{n}\mathrm{i}\dot{\mathrm{z}}$ od punktu $P(8,0)$. Sporządzič

rysunek.

5. Na dziesiątym piętrze pewnego bloku mieszkają Kowalscy $\mathrm{i}$ Nowakowie. Kowalscy

maja dwóch synów $\mathrm{i}$ dwie córki, a Nowakowie jednego syna $\mathrm{i}$ dwie córki. Postanowili

oni wybrač mfodziezowego przedstawiciela swojego piętra. $\mathrm{W}$ tym celu Kowalscy wy-

brali losowo jedno ze swoich dzieci, a Nowakowie jedno ze swoich. Nastepnie spośród

tej dwójki wylosowano jedną osobę. Obliczyč prawdopodobieństwo, $\dot{\mathrm{z}}\mathrm{e}$ przedstawi-

cielem zostal chlopiec.

6. Uzasadnič prawdziwośč nierówności $n+\displaystyle \frac{1}{2}\geq\sqrt{n(n+1)}, n\geq 1$. Korzystając $\mathrm{z}$ niej

oraz $\mathrm{z}$ zasady indukcji matematycznej udowodnič, $\dot{\mathrm{z}}\mathrm{e}$ dla wszystkich $n\geq 1$ jest

$\displaystyle \left(\begin{array}{l}
2n\\
n
\end{array}\right)\geq\frac{4^{n}}{2\sqrt{n}}.$

7. Przeprowadzič badanie przebiegu zmienności funkcji $f(x) = \sqrt{\frac{3x-3}{5-x}}\mathrm{i}$ wykonač jej

wykres.

8. $\mathrm{W}$ trójkacie $ABC$ kąt $A$ ma miarę $\alpha$, kąt $B$ miarę $ 2\alpha$, a $BC=a$. Oznaczmy kolejno

przez $A_{1}$ punkt na boku $\overline{AC}$ taki, $\dot{\mathrm{z}}\mathrm{e}\overline{BA_{1}}$ jest dwusieczną kąta $B$; $B_{1}$ punkt na

boku $\overline{BC}$ taki, $\dot{\mathrm{z}}\mathrm{e}\overline{A_{1}B_{1}}$ jest dwusieczną kąta $A_{1}$, itd. Wyznaczyč długośč łamanej

nieskończonej ABAlBlA2$\ldots.$

4





PRACA KONTROLNA nr 5

luty $2003\mathrm{r}$

l. Jakiej dfugości powinien byč pas napędowy, aby $\mathrm{m}\mathrm{o}\dot{\mathrm{z}}$ na go było $\mathrm{u}\dot{\mathrm{z}}$ yč do pofączenia

dwóch kół $0$ promieniach 20 cm $\mathrm{i}5$ cm, jeśli odlegfośč środków tych kól wynosi 30

cm?

2. Umowa określa wynagrodzenie na kwotę 4000 $\mathrm{z}\mathrm{f}$. Skladka na ubezpieczenie spofeczne

wynosi 18,7\% tej kwoty, a składka na Kasę Chorych 7,75\% kwoty pozostałej po

odliczeniu składki na ubezpieczenie spofeczne. $\mathrm{W}$ celu obliczenia podatku nalezy od

80\% wyjściowej kwoty umowy odjąč skfadkę na ubezpieczenie spoleczne $\mathrm{i}$ wyznaczyč

19\% pozostałej sumy. Podatek jest róznicą tak otrzymanej liczby $\mathrm{i}$ kwoty składki na

Kasę Chorych. Ile wynosi podatek?.

3. Przez punkt $P(1,3)$ poprowadzič prostą $l\mathrm{t}\mathrm{a}\mathrm{k}$, aby odcinek tej prostej zawarty po-

między dwiema danymi prostymi $x-y+3 = 0 \mathrm{i}x+2y-12 = 0$ dzielil się $\mathrm{w}$

punkcie $P$ na polowy. Wyznaczyč równanie ogólne prostej $l\mathrm{i}$ obliczyč pole trójk$\Phi$ta,

jaki prosta $l$ tworzy $\mathrm{z}$ danymi prostymi.

4. Podstawą czworościanu jest trójkąt prostokątny $ABC\mathrm{o}$ kącie ostrym $\alpha \mathrm{i}$ promieniu

okręgu wpisanego $r$. Spodek wysokości opuszczonej $\mathrm{z}$ wierzchołka $D\mathrm{l}\mathrm{e}\dot{\mathrm{z}}\mathrm{y}\mathrm{w}$ punkcie

przecięcia się dwusiecznych trójkąta $ABC$, a ściany boczne wychodzące $\mathrm{z}$ wierzchoł-

ka kąta prostego podstawy tworzą kąt $\beta$. Obliczyč objętośč tego ostrosfupa.

5. Sporzqdzič wykres funkcji

$f(x)=\log_{4}(2|x|-4)^{2}$

Odczytač $\mathrm{z}$ wykresu wszystkie ekstrema lokalne tej funkcji.

6. Rozwiązač równanie $\displaystyle \cos 2x+\frac{\mathrm{t}\mathrm{g}x}{\sqrt{3}+\mathrm{t}\mathrm{g}x}=0.$

7. Dla jakich wartości parametru $a\in R\mathrm{m}\mathrm{o}\dot{\mathrm{z}}$ na określič funkcję $g(x)=f(f(x))$, gdzie

$f(x)=\displaystyle \frac{x^{2}}{ax-1}$. Napisač funkcję $g(x)\mathrm{w}$ jawnej postaci. Wyznaczyč asymptoty funkcji

$g(x)$ dla największej $\mathrm{m}\mathrm{o}\dot{\mathrm{z}}$ liwej cafkowitej wartości parametru $a.$

8. Odcinek $0$ końcach $A(0,3), B(2,y), y \in [0$, 3$]$, obraca się wokół osi Ox. Wyzna-

czyč pole powierzchni bocznej powstałej bryły jako funkcję $y\mathrm{i}$ znalez$\acute{}$č najmniejszq

wartośč tego pola. Sporz$\Phi$dzič rysunek.

5





PRACA KONTROLNA nr 6

marzec $2003\mathrm{r}$

l. Dlajakich wartości parametru rzeczywistego $p$ równanie $\sqrt{x+8p}=\sqrt{x}+2p$ posiada

rozwiązanie?

2. Obrazem okręgu $K$ wjednokładności $0$ środku $S(0,1)\mathrm{i}$ skali $k=-3$ jest okrqg $K_{1}.$

Natomiast obrazem $K_{1} \mathrm{w}$ symetrii względem prostej $0$ równaniu $2x+y+3 = 0$

jest okrąg $0$ tym samym środku co okrąg $K$. Wyznaczyč równanie okręgu $K$, jeśli

wiadomo, $\dot{\mathrm{z}}\mathrm{e}$ okręgi $K\mathrm{i}K_{1}$ są styczne zewnętrznie.

3. $\mathrm{W}$ trapezie równoramiennym dane są promień okręgu opisanego $r$, kąt ostry przy

podstawie $\alpha$ oraz suma długości obu podstaw $d$. Obliczyč dlugośč ramienia tego tra-

pezu. Zbadač warunki rozwiązalności zadania. Wykonač rysunek dla $\alpha=60^{0}, d=$

-25{\it r}.

4. $\mathrm{W}$ ostrosłupie prawidłowym czworokqtnym $\mathrm{k}\mathrm{a}\mathrm{t}$ płaski ściany bocznej przy wierz-

chofku wynosi $ 2\beta$. Przez wierzchofek $A$ podstawy oraz środek przeciwlegfej krawę-

dzi bocznej poprowadzono płaszczyznę równoległą do przekqtnej podstawy wyzna-

czającą przekrój płaski ostrosfupa. Obliczyč objętośč ostroslupa $\mathrm{w}\mathrm{i}\mathrm{e}\mathrm{d}\mathrm{z}\Phi^{\mathrm{C}}, \dot{\mathrm{z}}\mathrm{e}$ pole

przekroju wynosi $S.$

5. Obliczyč granicę

$\displaystyle \lim_{n\rightarrow\infty}\frac{n-\sqrt[3]{n^{3}+n^{\alpha}}}{\sqrt[5]{n^{3}}},$

gdzie $\alpha$ jest najmniejszym dodatnim pierwiastkiem równania 2 $\cos\alpha=-\sqrt{3}.$

6. Rozwiązač nierównośč

$2^{1+2\log_{2}\cos x}-\displaystyle \frac{3}{4}\geq 9^{0.5+\log_{3}\sin x}$

7. Wybrano losowo 4 liczby czterocyfrowe (cyfra tysiecy nie $\mathrm{m}\mathrm{o}\dot{\mathrm{z}}\mathrm{e}$ byč zerem!). Obliczyč

prawdopodobieństwo tego, $\dot{\mathrm{z}}\mathrm{e}$ co najmniej dwie $\mathrm{z}$ tych liczb czytane od przodu lub

od końca będą podzielne przez 4.

8. Zaznaczyč na rysunku zbiór punktów $(x,y)$ płaszczyzny określony warunkami

$|x-3y| <2$ oraz $y^{3}\leq x$. Obliczyč tangens $\mathrm{k}_{\Phi^{\mathrm{t}\mathrm{a}}}$, pod którym przecinają się linie

tworzqce brzeg tego zbioru.

6





PRACA KONTROLNA nr 7

kwiecień 2003r

l. Dwa punkty poruszają się ruchem jednostajnym po okręgu $\mathrm{w}$ tym samym kierunku,

przy czym jeden $\mathrm{z}$ nich wyprzedza drugi co 44 sekund. $\mathrm{J}\mathrm{e}\dot{\mathrm{z}}$ eli zmienič kierunek ruchu

jednego $\mathrm{z}$ tych punktów, to bedą się one spotykač co 8 sekund. Ob1iczyč stosunek

prędkości tych punktów.

2. Dla jakich wartości parametru $p$ nierównośč

$\displaystyle \frac{2px^{2}+2px+1}{x^{2}+x+2-p^{2}}\geq 2$

jest spełniona dla $\mathrm{k}\mathrm{a}\dot{\mathrm{z}}$ dej liczby rzeczywistej $x$?

3. $\mathrm{W}$ równolegloboku dane są $\mathrm{k}\mathrm{a}\mathrm{t}$ ostry $\alpha$, dłuzsza przekątna $d$ oraz róznica boków $r.$

Obliczyč pole równolegloboku.

4. Naczynie $\mathrm{w}$ kształcie półkuli $0$ promieniu $R$ ma trzy nózki $\mathrm{w}$ kształcie kulek $0$

promieniu $r, 4r < R$, przymocowanych do naczynia $\mathrm{w}$ ten sposób, $\dot{\mathrm{z}}\mathrm{e}$ ich środki

tworzą trójkąt równoboczny, a naczynie postawione na płaskiej powierzchni dotyka

ją wjednym punkcie. Obliczyč wzajemnq odleglośč punktów przymocowania kulek.

Wykonač odpowiednie rysunki.

5. Poslugując się rachunkiem rózniczkowym określič liczbę rozwiązań równania

$2x^{3}+1=6|x|-6x^{2}$

6. Nie $\mathrm{s}\mathrm{t}\mathrm{o}\mathrm{s}\mathrm{u}\mathrm{j}_{\Phi}\mathrm{c}$ zasady indukcji matematycznej wykazač, $\dot{\mathrm{z}}\mathrm{e}\mathrm{j}\mathrm{e}\dot{\mathrm{z}}$ eli $n \geq 2$ jest liczbą

naturalną, to $\displaystyle \frac{n^{n}-1}{n-1}$ jest nieparzystą liczbą naturalną.

7. Rozwiązač równanie

$\displaystyle \frac{8}{3}(\sin^{2}x+\sin^{4}x+\ldots)=4-2\cos x+3\cos^{2}x-\frac{9}{2}\cos^{3}x+\ldots$

8. Rozwazmy rodzine prostych normalnych (tj. prostopadfych do stycznych $\mathrm{w}$ punk-

tach styczności) do paraboli $0$ równaniu $2y=x^{2}$ Znalez$\acute{}$č równanie krzywej utwo-

rzonej ze środków odcinków tych normalnych zawartych pomiędzy osią rzędnych $\mathrm{i}$

$\mathrm{w}\mathrm{y}\mathrm{z}\mathrm{n}\mathrm{a}\mathrm{c}\mathrm{z}\mathrm{a}\mathrm{j}_{\Phi}$cymi je punktami paraboli. Sporz$\Phi$dzič rysunek.

7



\end{document}