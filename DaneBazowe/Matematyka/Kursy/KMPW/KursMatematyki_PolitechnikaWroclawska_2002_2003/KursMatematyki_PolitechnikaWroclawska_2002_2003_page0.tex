\documentclass[a4paper,12pt]{article}
\usepackage{latexsym}
\usepackage{amsmath}
\usepackage{amssymb}
\usepackage{graphicx}
\usepackage{wrapfig}
\pagestyle{plain}
\usepackage{fancybox}
\usepackage{bm}

\begin{document}

KORESPONDENCYJNY KURS Z MATEMATYKI

PRACA KONTROLNA nr l

$\mathrm{p}\mathrm{a}\acute{\mathrm{z}}$dziernik 2$002\mathrm{r}$

l. Narysowač wykres funkcji $y=4+2|x|-x^{2}$ Korzystając $\mathrm{z}$ tego wykresu określič

liczbę rozwi$\Phi$zań równania $4+2|x|-x^{2}=p\mathrm{w}$ zalezności od parametru rzeczywistego

$p.$

2. Pompa napełniająca pusty basen $\mathrm{w}$ pierwszej minucie pracy miała wydajnośč 0,2

$\mathrm{m}^{3}/\mathrm{s}$, a $\mathrm{w}\mathrm{k}\mathrm{a}\dot{\mathrm{z}}$ dej kolejnej minucie jej wydajnośč zwiększano $00,01 \mathrm{m}^{3}/\mathrm{s}$. Pofowa

basenu zostala napełniona po $2n$ minutach, a caly basen po kolejnych $n$ minutach,

gdzie $n$ jest liczbą naturalnq. Wyznaczyč czas napefniania basenu oraz jego pojem-

nośč.

3. Stozek ścięty jest opisany na kuli $0$ promieniu $r=2$ cm. Objętośč kuli stanowi 25\%

objętości stozka. Wyznaczyč średnice podstaw $\mathrm{i}$ dfugośč tworzącej tego stozka.

4. $\mathrm{W}$ trójkącie $ABC$ dane są promień okręgu opisanego $R$, kąt $\angle A=\alpha$ oraz $AB=\displaystyle \frac{8}{5}R.$

Obliczyč pole tego trójkqta.

5. Rozwiązač nierównośč:

$(\sqrt{x})^{\log_{8}x}\geq\sqrt[3]{16x}.$

6. $\mathrm{W}$ czworokącie ABCD odcinki $\overline{AB}\mathrm{i}\overline{BD}$ są prostopadle, $AD = 2AB =a$ oraz

$ AC=\rightarrow \displaystyle \frac{5}{3}AB\rightarrow+\frac{1}{3}AD\rightarrow$. Wyznaczyč cosinus kąta $\angle BCD=\alpha$ oraz obwód czworokąta

ABCD. Sporządzič rysunek.

7. Rozwiązač równanie:

$\displaystyle \frac{1}{\sin x}+\frac{1}{\cos x}=\sqrt{8}.$

8. Wyznaczyč równanie prostej stycznej do wykresu funkcji $y=\displaystyle \frac{1}{x^{2}}\mathrm{w}$ punkcie $P(x_{0},y_{0}),$

$x_{0}>0$, takim, $\dot{\mathrm{z}}\mathrm{e}$ odcinek tej stycznej zawarty $\mathrm{w}$ I čwiartce układu wspólrzędnych

jest najkrótszy. Rozwiązanie zilustrowač stosownym wykresem.

1
\end{document}
