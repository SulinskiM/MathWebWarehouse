\documentclass[a4paper,12pt]{article}
\usepackage{latexsym}
\usepackage{amsmath}
\usepackage{amssymb}
\usepackage{graphicx}
\usepackage{wrapfig}
\pagestyle{plain}
\usepackage{fancybox}
\usepackage{bm}

\begin{document}

PRACA KONTROLNA nr 2

listopad $2002\mathrm{r}$

l. Czy liczby róznych `sfów', jakie $\mathrm{m}\mathrm{o}\dot{\mathrm{z}}$ na utworzyč zmieniając kolejnośč liter $\mathrm{w}$ s{\it l}o-

wach' TANATAN $\mathrm{i}$ AKABARA, są takie same? Uzasadnič odpowied $\acute{\mathrm{z}}$. Przez `sfowo'

rozumiemy tutaj dowolny ciag liter.

2. Reszta $\mathrm{z}$ dzielenia wielomianu $x^{3}+px^{2}-x+q$ przez trójmian $(x+2)^{2}$ wynosi $-x+1.$

Wyznaczyč pierwiastki tego wielomianu.

3. Figura na rysunku ponizej składa się $\mathrm{z}$ łuków $BC, CA$ okręgów $0$ promieniu $a\mathrm{i}$

środkach odpowiednio $\mathrm{w}$ punktach $A, B$, oraz $\mathrm{z}$ odcinka $\overline{AB}0$ dfugości $a$. Obliczyč

promień okręgu stycznego do obu łuków oraz do odcinka $\overline{AB}.$

4. Podstawą pryzmy przedstawionej na rysunku ponizej jest $\mathrm{p}\mathrm{r}\mathrm{o}\mathrm{s}\mathrm{t}\mathrm{o}\mathrm{k}_{\Phi}\mathrm{t}$ ABCD, którego

bok $\overline{AB}$ ma długośč $a$, a bok $\overline{BC}$ długośč $b$, gdzie $a>b$. Wszystkie ściany boczne

pryzmy są nachylone pod kątem $\alpha$ do płaszczyzny podstawy. Obliczyč objetośč tej

pryzmy.

5. Rozwiązač nierównośč

-{\it x}2$<\sqrt{}$5-{\it x}2.

Rozwiązanie zilustrowač wykresami funkcji wystepujqcych po obu stronach nierów-

ności. Zaznaczyč na rysunku otrzymany zbiór rozwiązań.

6. Ciąg $(a_{n})$ jest określony warunkami $\alpha_{1}=4, a_{n+1}=1+2\sqrt{a_{n}}, n\geq 1$. Stosując zasadę

indukcji matematycznej wykazač, $\dot{\mathrm{z}}\mathrm{e}$ ciag $(a_{n})$ jest rosnący oraz dla $n\geq 1$ spefniona

jest nierównośč: $4\leq a_{n}<6.$

7. Na krzywej $0$ równaniu $y=\sqrt{x}$ znalez/č miejsce, którejest połozone najblizej punktu

$P(0,3)$. Sporz$\Phi$dzič rysunek.

8. Wykazač, $\dot{\mathrm{z}}\mathrm{e}$ dla $\mathrm{k}\mathrm{a}\dot{\mathrm{z}}$ dej wartości parametru $\alpha\in R$ równanie kwadratowe

$3x^{2}+4x\sin\alpha-\cos 2\alpha=0$

ma dwa rózne pierwiastki rzeczywiste. Wyznaczyč te wartości parametru $\alpha$, dla

których oba pierwiastki $\mathrm{l}\mathrm{e}\dot{\mathrm{z}}$ ą $\mathrm{w}$ przedziale $(0,1).$

2
\end{document}
