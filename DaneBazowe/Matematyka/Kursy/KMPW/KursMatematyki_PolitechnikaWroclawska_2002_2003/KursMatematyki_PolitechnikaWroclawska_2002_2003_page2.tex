\documentclass[a4paper,12pt]{article}
\usepackage{latexsym}
\usepackage{amsmath}
\usepackage{amssymb}
\usepackage{graphicx}
\usepackage{wrapfig}
\pagestyle{plain}
\usepackage{fancybox}
\usepackage{bm}

\begin{document}

PRACA KONTROLNA nr 3

grudzień $2002\mathrm{r}$

l. Suma wyrazów nieskończonego ciqgu geometrycznego zmniejszy się $0$ 25\%, jeśli wy-

kreślimy $\mathrm{z}$ niej skladniki $0$ numerach parzystych niepodzielnych przez 4. Ob1iczyč

sume wszystkich wyrazów tego ciągu wiedząc, $\dot{\mathrm{z}}\mathrm{e}$ jego drugi wyraz wynosi l.

2. $\mathrm{Z}$ kompletu 28 kości do gry $\mathrm{w}$ domino wylosowano dwie kości (bez zwracania).

Obliczyč prawdopodobieństwo tego, $\dot{\mathrm{z}}\mathrm{e}$ kości pasujq do siebie $\mathrm{t}\mathrm{z}\mathrm{n}$. na jednym $\mathrm{z}$ pól

obu kości wystepuje ta sama liczba oczek.

3. Rozwiązač ukfad równań

$\left\{\begin{array}{l}
x\\
5x
\end{array}\right.$

$+2y$

$+my$

3

{\it m}

$\mathrm{w}$ zalezności od parametru rzeczywistego $m$. Wyznaczyč $\mathrm{i}$ narysowač zbiór, jaki

tworzq rozwiązania $(x(m),y(m))$ tego układu, gdy $m$ przebiega zbiór liczb rzeczy-

wistych.

4. $\mathrm{W}$ graniastoslupie prawidłowym sześciokątnym krawędz/ dolnej podstawy $\overline{AB}$ jest

widoczna ze środka górnej podstawy $P$ pod $\mathrm{k}_{\Phi}\mathrm{t}\mathrm{e}\mathrm{m}\alpha$. Wyznaczyč cosinus kąta utwo-

rzonego przez płaszczyznę podstawy $\mathrm{i}$ płaszczyznę zawierającą $\overline{AB}$ oraz przeciwległą

do niej krawęd $\acute{\mathrm{z}}\overline{D'E'}$ górnej podstawy. Obliczenia odpowiednio uzasadnič.

5. Rozwiązač nierównośč

$-1\displaystyle \leq\frac{2^{x+1/2}}{4^{x}-4}\leq 1.$

6. Nie posfugując się tablicami wykazač, $\dot{\mathrm{z}}\mathrm{e}$ tg82030' -tg $7^{0}30'=4+2\sqrt{3}.$

7. Napisač równanie prostej $k$ stycznej do okręgu $x^{2}+y^{2}-2x-2y-3=0\mathrm{w}$ punk-

cie $P(2,3)$. Następnie wyznaczyč równania wszystkich prostych stycznych do tego

okręgu, które tworzą $\mathrm{z}$ prostą $k$ kąt $45^{0}$

8. Dobrač parametry $a>0\mathrm{i}b\in R\mathrm{t}\mathrm{a}\mathrm{k}$, aby funkcja

$f(x)=\displaystyle \{\frac{(a+b}{x^{2}-1}1)+ax-x^{2}$

dla $x\leq a,$

dla $x>a,$

była ciagła i miala pochodnq w punkcie a. Nie przeprowadzając dalszego badania

sporządzič wykres funkcji f(x) oraz stycznej do jej wykresu w punkcie P(a, f(a)).

3
\end{document}
