\documentclass[a4paper,12pt]{article}
\usepackage{latexsym}
\usepackage{amsmath}
\usepackage{amssymb}
\usepackage{graphicx}
\usepackage{wrapfig}
\pagestyle{plain}
\usepackage{fancybox}
\usepackage{bm}

\begin{document}

PRACA KONTROLNA nr 4

styczeń $2003\mathrm{r}$

l. Dla jakich wartości parametru rzeczywistego $t$ równanie

$x+3=-(tx+1)^{2}$

ma dokfadnie jedno rozwiązanie.

2. Czworościan foremny $0$ krawędzi $a$ przecięto płaszczyzną równoległą do dwóch prze-

ciwległych krawędzi. Wyrazič pole otrzymanego przekrojujako funkcję długości od-

cinka wyznaczonego przez ten przekrój na jednej $\mathrm{z}$ pozostafych krawędzi. Uzasad-

nič postępowanie. Przedstawič znalezioną funkcję na wykresie $\mathrm{i}$ podačjej największą

wartośč.

3. Zaznaczyč na wykresie zbiór punktów $(x,y)$ pfaszczyzny spelniajqcych warunek

$\log_{xy}|y|\geq 1.$

4. Wyznaczyč równanie linii utworzonej przez wszystkie punkty plaszczyzny, których

odległośč od okręgu $x^{2}+y^{2}=81$ jest $01$ mniejsza $\mathrm{n}\mathrm{i}\dot{\mathrm{z}}$ od punktu $P(8,0)$. Sporządzič

rysunek.

5. Na dziesiątym piętrze pewnego bloku mieszkają Kowalscy $\mathrm{i}$ Nowakowie. Kowalscy

maja dwóch synów $\mathrm{i}$ dwie córki, a Nowakowie jednego syna $\mathrm{i}$ dwie córki. Postanowili

oni wybrač mfodziezowego przedstawiciela swojego piętra. $\mathrm{W}$ tym celu Kowalscy wy-

brali losowo jedno ze swoich dzieci, a Nowakowie jedno ze swoich. Nastepnie spośród

tej dwójki wylosowano jedną osobę. Obliczyč prawdopodobieństwo, $\dot{\mathrm{z}}\mathrm{e}$ przedstawi-

cielem zostal chlopiec.

6. Uzasadnič prawdziwośč nierówności $n+\displaystyle \frac{1}{2}\geq\sqrt{n(n+1)}, n\geq 1$. Korzystając $\mathrm{z}$ niej

oraz $\mathrm{z}$ zasady indukcji matematycznej udowodnič, $\dot{\mathrm{z}}\mathrm{e}$ dla wszystkich $n\geq 1$ jest

$\displaystyle \left(\begin{array}{l}
2n\\
n
\end{array}\right)\geq\frac{4^{n}}{2\sqrt{n}}.$

7. Przeprowadzič badanie przebiegu zmienności funkcji $f(x) = \sqrt{\frac{3x-3}{5-x}}\mathrm{i}$ wykonač jej

wykres.

8. $\mathrm{W}$ trójkacie $ABC$ kąt $A$ ma miarę $\alpha$, kąt $B$ miarę $ 2\alpha$, a $BC=a$. Oznaczmy kolejno

przez $A_{1}$ punkt na boku $\overline{AC}$ taki, $\dot{\mathrm{z}}\mathrm{e}\overline{BA_{1}}$ jest dwusieczną kąta $B$; $B_{1}$ punkt na

boku $\overline{BC}$ taki, $\dot{\mathrm{z}}\mathrm{e}\overline{A_{1}B_{1}}$ jest dwusieczną kąta $A_{1}$, itd. Wyznaczyč długośč łamanej

nieskończonej ABAlBlA2$\ldots.$

4
\end{document}
