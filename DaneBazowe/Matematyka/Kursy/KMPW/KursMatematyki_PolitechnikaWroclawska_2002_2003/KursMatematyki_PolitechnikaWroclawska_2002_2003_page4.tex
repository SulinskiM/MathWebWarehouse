\documentclass[a4paper,12pt]{article}
\usepackage{latexsym}
\usepackage{amsmath}
\usepackage{amssymb}
\usepackage{graphicx}
\usepackage{wrapfig}
\pagestyle{plain}
\usepackage{fancybox}
\usepackage{bm}

\begin{document}

PRACA KONTROLNA nr 5

luty $2003\mathrm{r}$

l. Jakiej dfugości powinien byč pas napędowy, aby $\mathrm{m}\mathrm{o}\dot{\mathrm{z}}$ na go było $\mathrm{u}\dot{\mathrm{z}}$ yč do pofączenia

dwóch kół $0$ promieniach 20 cm $\mathrm{i}5$ cm, jeśli odlegfośč środków tych kól wynosi 30

cm?

2. Umowa określa wynagrodzenie na kwotę 4000 $\mathrm{z}\mathrm{f}$. Skladka na ubezpieczenie spofeczne

wynosi 18,7\% tej kwoty, a składka na Kasę Chorych 7,75\% kwoty pozostałej po

odliczeniu składki na ubezpieczenie spofeczne. $\mathrm{W}$ celu obliczenia podatku nalezy od

80\% wyjściowej kwoty umowy odjąč skfadkę na ubezpieczenie spoleczne $\mathrm{i}$ wyznaczyč

19\% pozostałej sumy. Podatek jest róznicą tak otrzymanej liczby $\mathrm{i}$ kwoty składki na

Kasę Chorych. Ile wynosi podatek?.

3. Przez punkt $P(1,3)$ poprowadzič prostą $l\mathrm{t}\mathrm{a}\mathrm{k}$, aby odcinek tej prostej zawarty po-

między dwiema danymi prostymi $x-y+3 = 0 \mathrm{i}x+2y-12 = 0$ dzielil się $\mathrm{w}$

punkcie $P$ na polowy. Wyznaczyč równanie ogólne prostej $l\mathrm{i}$ obliczyč pole trójk$\Phi$ta,

jaki prosta $l$ tworzy $\mathrm{z}$ danymi prostymi.

4. Podstawą czworościanu jest trójkąt prostokątny $ABC\mathrm{o}$ kącie ostrym $\alpha \mathrm{i}$ promieniu

okręgu wpisanego $r$. Spodek wysokości opuszczonej $\mathrm{z}$ wierzchołka $D\mathrm{l}\mathrm{e}\dot{\mathrm{z}}\mathrm{y}\mathrm{w}$ punkcie

przecięcia się dwusiecznych trójkąta $ABC$, a ściany boczne wychodzące $\mathrm{z}$ wierzchoł-

ka kąta prostego podstawy tworzą kąt $\beta$. Obliczyč objętośč tego ostrosfupa.

5. Sporzqdzič wykres funkcji

$f(x)=\log_{4}(2|x|-4)^{2}$

Odczytač $\mathrm{z}$ wykresu wszystkie ekstrema lokalne tej funkcji.

6. Rozwiązač równanie $\displaystyle \cos 2x+\frac{\mathrm{t}\mathrm{g}x}{\sqrt{3}+\mathrm{t}\mathrm{g}x}=0.$

7. Dla jakich wartości parametru $a\in R\mathrm{m}\mathrm{o}\dot{\mathrm{z}}$ na określič funkcję $g(x)=f(f(x))$, gdzie

$f(x)=\displaystyle \frac{x^{2}}{ax-1}$. Napisač funkcję $g(x)\mathrm{w}$ jawnej postaci. Wyznaczyč asymptoty funkcji

$g(x)$ dla największej $\mathrm{m}\mathrm{o}\dot{\mathrm{z}}$ liwej cafkowitej wartości parametru $a.$

8. Odcinek $0$ końcach $A(0,3), B(2,y), y \in [0$, 3$]$, obraca się wokół osi Ox. Wyzna-

czyč pole powierzchni bocznej powstałej bryły jako funkcję $y\mathrm{i}$ znalez$\acute{}$č najmniejszq

wartośč tego pola. Sporz$\Phi$dzič rysunek.

5
\end{document}
