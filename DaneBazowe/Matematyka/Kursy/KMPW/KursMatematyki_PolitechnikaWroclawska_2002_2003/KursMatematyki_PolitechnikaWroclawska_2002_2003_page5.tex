\documentclass[a4paper,12pt]{article}
\usepackage{latexsym}
\usepackage{amsmath}
\usepackage{amssymb}
\usepackage{graphicx}
\usepackage{wrapfig}
\pagestyle{plain}
\usepackage{fancybox}
\usepackage{bm}

\begin{document}

PRACA KONTROLNA nr 6

marzec $2003\mathrm{r}$

l. Dlajakich wartości parametru rzeczywistego $p$ równanie $\sqrt{x+8p}=\sqrt{x}+2p$ posiada

rozwiązanie?

2. Obrazem okręgu $K$ wjednokładności $0$ środku $S(0,1)\mathrm{i}$ skali $k=-3$ jest okrqg $K_{1}.$

Natomiast obrazem $K_{1} \mathrm{w}$ symetrii względem prostej $0$ równaniu $2x+y+3 = 0$

jest okrąg $0$ tym samym środku co okrąg $K$. Wyznaczyč równanie okręgu $K$, jeśli

wiadomo, $\dot{\mathrm{z}}\mathrm{e}$ okręgi $K\mathrm{i}K_{1}$ są styczne zewnętrznie.

3. $\mathrm{W}$ trapezie równoramiennym dane są promień okręgu opisanego $r$, kąt ostry przy

podstawie $\alpha$ oraz suma długości obu podstaw $d$. Obliczyč dlugośč ramienia tego tra-

pezu. Zbadač warunki rozwiązalności zadania. Wykonač rysunek dla $\alpha=60^{0}, d=$

-25{\it r}.

4. $\mathrm{W}$ ostrosłupie prawidłowym czworokqtnym $\mathrm{k}\mathrm{a}\mathrm{t}$ płaski ściany bocznej przy wierz-

chofku wynosi $ 2\beta$. Przez wierzchofek $A$ podstawy oraz środek przeciwlegfej krawę-

dzi bocznej poprowadzono płaszczyznę równoległą do przekqtnej podstawy wyzna-

czającą przekrój płaski ostrosfupa. Obliczyč objętośč ostroslupa $\mathrm{w}\mathrm{i}\mathrm{e}\mathrm{d}\mathrm{z}\Phi^{\mathrm{C}}, \dot{\mathrm{z}}\mathrm{e}$ pole

przekroju wynosi $S.$

5. Obliczyč granicę

$\displaystyle \lim_{n\rightarrow\infty}\frac{n-\sqrt[3]{n^{3}+n^{\alpha}}}{\sqrt[5]{n^{3}}},$

gdzie $\alpha$ jest najmniejszym dodatnim pierwiastkiem równania 2 $\cos\alpha=-\sqrt{3}.$

6. Rozwiązač nierównośč

$2^{1+2\log_{2}\cos x}-\displaystyle \frac{3}{4}\geq 9^{0.5+\log_{3}\sin x}$

7. Wybrano losowo 4 liczby czterocyfrowe (cyfra tysiecy nie $\mathrm{m}\mathrm{o}\dot{\mathrm{z}}\mathrm{e}$ byč zerem!). Obliczyč

prawdopodobieństwo tego, $\dot{\mathrm{z}}\mathrm{e}$ co najmniej dwie $\mathrm{z}$ tych liczb czytane od przodu lub

od końca będą podzielne przez 4.

8. Zaznaczyč na rysunku zbiór punktów $(x,y)$ płaszczyzny określony warunkami

$|x-3y| <2$ oraz $y^{3}\leq x$. Obliczyč tangens $\mathrm{k}_{\Phi^{\mathrm{t}\mathrm{a}}}$, pod którym przecinają się linie

tworzqce brzeg tego zbioru.

6
\end{document}
