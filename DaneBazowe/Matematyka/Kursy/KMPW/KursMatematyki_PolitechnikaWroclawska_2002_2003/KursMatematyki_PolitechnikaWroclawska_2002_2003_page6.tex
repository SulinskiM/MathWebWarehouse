\documentclass[a4paper,12pt]{article}
\usepackage{latexsym}
\usepackage{amsmath}
\usepackage{amssymb}
\usepackage{graphicx}
\usepackage{wrapfig}
\pagestyle{plain}
\usepackage{fancybox}
\usepackage{bm}

\begin{document}

PRACA KONTROLNA nr 7

kwiecień 2003r

l. Dwa punkty poruszają się ruchem jednostajnym po okręgu $\mathrm{w}$ tym samym kierunku,

przy czym jeden $\mathrm{z}$ nich wyprzedza drugi co 44 sekund. $\mathrm{J}\mathrm{e}\dot{\mathrm{z}}$ eli zmienič kierunek ruchu

jednego $\mathrm{z}$ tych punktów, to bedą się one spotykač co 8 sekund. Ob1iczyč stosunek

prędkości tych punktów.

2. Dla jakich wartości parametru $p$ nierównośč

$\displaystyle \frac{2px^{2}+2px+1}{x^{2}+x+2-p^{2}}\geq 2$

jest spełniona dla $\mathrm{k}\mathrm{a}\dot{\mathrm{z}}$ dej liczby rzeczywistej $x$?

3. $\mathrm{W}$ równolegloboku dane są $\mathrm{k}\mathrm{a}\mathrm{t}$ ostry $\alpha$, dłuzsza przekątna $d$ oraz róznica boków $r.$

Obliczyč pole równolegloboku.

4. Naczynie $\mathrm{w}$ kształcie półkuli $0$ promieniu $R$ ma trzy nózki $\mathrm{w}$ kształcie kulek $0$

promieniu $r, 4r < R$, przymocowanych do naczynia $\mathrm{w}$ ten sposób, $\dot{\mathrm{z}}\mathrm{e}$ ich środki

tworzą trójkąt równoboczny, a naczynie postawione na płaskiej powierzchni dotyka

ją wjednym punkcie. Obliczyč wzajemnq odleglośč punktów przymocowania kulek.

Wykonač odpowiednie rysunki.

5. Poslugując się rachunkiem rózniczkowym określič liczbę rozwiązań równania

$2x^{3}+1=6|x|-6x^{2}$

6. Nie $\mathrm{s}\mathrm{t}\mathrm{o}\mathrm{s}\mathrm{u}\mathrm{j}_{\Phi}\mathrm{c}$ zasady indukcji matematycznej wykazač, $\dot{\mathrm{z}}\mathrm{e}\mathrm{j}\mathrm{e}\dot{\mathrm{z}}$ eli $n \geq 2$ jest liczbą

naturalną, to $\displaystyle \frac{n^{n}-1}{n-1}$ jest nieparzystą liczbą naturalną.

7. Rozwiązač równanie

$\displaystyle \frac{8}{3}(\sin^{2}x+\sin^{4}x+\ldots)=4-2\cos x+3\cos^{2}x-\frac{9}{2}\cos^{3}x+\ldots$

8. Rozwazmy rodzine prostych normalnych (tj. prostopadfych do stycznych $\mathrm{w}$ punk-

tach styczności) do paraboli $0$ równaniu $2y=x^{2}$ Znalez$\acute{}$č równanie krzywej utwo-

rzonej ze środków odcinków tych normalnych zawartych pomiędzy osią rzędnych $\mathrm{i}$

$\mathrm{w}\mathrm{y}\mathrm{z}\mathrm{n}\mathrm{a}\mathrm{c}\mathrm{z}\mathrm{a}\mathrm{j}_{\Phi}$cymi je punktami paraboli. Sporz$\Phi$dzič rysunek.

7
\end{document}
