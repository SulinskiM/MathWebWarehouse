\documentclass[a4paper,12pt]{article}
\usepackage{latexsym}
\usepackage{amsmath}
\usepackage{amssymb}
\usepackage{graphicx}
\usepackage{wrapfig}
\pagestyle{plain}
\usepackage{fancybox}
\usepackage{bm}

\begin{document}

XXXIII

KORESPONDENCYJNY KURS Z MATEMATYKI

$\mathrm{p}\mathrm{a}\acute{\mathrm{z}}$dziernik 2$003\mathrm{r}.$

PRACA KONTROLNA nr l

l. Podstawą trójk$\Phi$ta równoramiennegojest odcinek $\overline{AB}0$ końcach $A(-1,3), B(1,-1),$

a wierzchołek $C$ tego trójkąta $\mathrm{l}\mathrm{e}\dot{\mathrm{z}}\mathrm{y}$ na prostej $l\mathrm{o}$ równaniu $3x-y-14=0$. Obliczyč

pole trójkąta $ABC.$

2. Pewna liczba sześciocyfrowa zaczyna się ($\mathrm{z}$ lewej strony) cyfrą 3. Jeś1i cyfrę tę

przestawimy $\mathrm{z}$ pierwszej pozycji na ostatnią, to otrzymamy liczbę stanowiacą 25\%

liczby pierwotnej. Znalez/č tę liczbę.

3. $\mathrm{W}$ trapezie opisanym na okregu kąty ostre przy podstawie mają miary $\alpha \mathrm{i}2\alpha, \mathrm{a}$

dlugośč krótszego ramienia wynosi $c$. Obliczyč długośč krótszej podstawy tego

trapezu. Wynik doprowadzič do najprostszej postaci.

4. Rozwiązač nierównośč:

$\displaystyle \frac{1}{x^{2}-x-2}\leq\frac{1}{|x|}.$

5. Zaznaczyč na pfaszczy $\acute{\mathrm{z}}\mathrm{n}\mathrm{i}\mathrm{e}$ zbiór wszystkich punktów $(x,y)$ spelniających nierów-

nośč $\log_{x}(1+(y-1)^{3})\leq 1.$

6. Rozwiązač równanie:

$\sin^{2}3x$ -sin2 $2x=\sin^{2}x.$

7. Wysokośč ostroslupa prawidfowego czworokątnego jest trzy razy dfuzsza od pro-

mienia kuli wpisanej $\mathrm{w}$ ten ostroslup Obliczyč cosinus kata pomiędzy sąsiednimi

ścianami bocznymi tego ostrosłupa.

8. Dany jest nieskończony ciąg geometryczny: $x+1, -x^{2}(x+1), x^{4}(x+1), \ldots$ Wyzna-

czyč najmniejszą $\mathrm{i}$ największą wartośč funkcji $S(x)$ oznaczającej sumę wszystkich

wyrazów tego ciągu.

1




listopad 2003r.

PRACA KONTROLNA nr 2

l. Trójkąt $\mathrm{P}^{\mathrm{r}\mathrm{o}\mathrm{s}\mathrm{t}\mathrm{o}\mathrm{k}}\Phi^{\mathrm{t}\mathrm{n}\mathrm{y}} \mathrm{o}\mathrm{b}\mathrm{r}\mathrm{a}\mathrm{c}\mathrm{a}\mathrm{j}_{\Phi}\mathrm{c}$ się wokół jednej $\mathrm{i}$ drugiej przyprostokątnej daje

bryły $0$ objętościach $V_{1} \mathrm{i} V_{2}$, odpowiednio. Obliczyč objętośč bryły powstałej $\mathrm{z}$

obrotu tego trójkąta wokół dwusiecznej kąta prostego.

2. Czy $\mathrm{m}\mathrm{o}\dot{\mathrm{z}}$ na sumę 42000 z1otych podzie1ič na pewną 1iczbę nagród $\mathrm{t}\mathrm{a}\mathrm{k}$, aby kwoty

tych nagród wyrazaly się $\mathrm{w}$ pelnych setkach złotych, tworzyly ciąg arytmetyczny

oraz najwyzsza nagroda wynosifa 13000 $\mathrm{z}\mathrm{f}$? Jeśli $\mathrm{t}\mathrm{a}\mathrm{k}$, to podač liczbę $\mathrm{i}$ wysokości

tych nagród.

3. Dane sq okregi $0$ równaniach $(x-1)^{2}+(y-1)^{2}=1$ oraz $(x-2)^{2}+(y-1)^{2}=16.$

Wyznaczyč równania wszystkich okręgów stycznych równocześnie do obu danych

okręgów oraz do osi Oy. Sporządzič rysunek.

4. $\mathrm{W}$ równolegloboku kąt ostry miedzy przekqtnymi ma miarę $\beta$, a stosunek dfugości

dfuzszej przekątnej do krótszej przekątnej wynosi $k$. Obliczyč tangens kąta ostrego

tego równoległoboku.

5. Rozwiązač równanie $\sqrt{4x-3}-3=\sqrt{2x-10}.$

6. Dobrač liczby calkowite a,b $\mathrm{t}\mathrm{a}\mathrm{k}$, aby wielomian $6x^{3}-7x^{2}+1$ dzielil się bez reszty

przez trójmian kwadratowy $2x^{2}+ax+b.$

7. Rozwiązač nierównośč $|2^{x}-3|\leq 2^{1-x}$ Rozwiązanie zilustrowač na rysunku wyko-

nując wykresy funkcji występujqcych po obu stronach tej nierówności.

8. Wyznaczyč przedziały monotoniczności funkcji

$f(x)=\displaystyle \sin^{2}x+\frac{\sqrt{3}}{2}x,$

$x\in[-\pi,\pi].$

2





grudzień 2003r.

PRACA KONTROLNA nr 3

l. Obliczyč prawdopodobieństwo tego, $\dot{\mathrm{z}}\mathrm{e}$ gracz losując 7 kart $\mathrm{z}$ talii 24 kart do gry

otrzyma dokładnie cztery karty $\mathrm{w}$ jednym kolorze $\mathrm{w}$ tym asa, króla $\mathrm{i}$ damę.

2. Pewien ostroslup przecięto na trzy części dwiema płaszczyznami równoległymi do

jego podstawy. Pierwsza pfaszczyznajest polozona $\mathrm{w}$ odlegfości $d_{1}=2$ cm, a druga

$\mathrm{w}$ odległości $d_{2}=3$ cm od podstawy. Pola przekrojów ostroslupa tymi plaszczy-

znami równe są odpowiednio $S_{1}=25\mathrm{c}\mathrm{m}^{2}$ oraz $S_{2}=16\mathrm{c}\mathrm{m}^{2}$ Obliczyč objętośč tego

ostroslupa oraz objętośč najmniejszej części.

3. Rozwiązač ukfad równań:

$\left\{\begin{array}{l}
x^{2}+y^{2}=24\\
\frac{2\log x+\log y^{2}}{\log(x+y)}=2
\end{array}\right.$

4. $\mathrm{W}$ trójkącie równoramiennym $ABC$ odległośč środka okręgu wpisanego od wierz-

chofka $C$ wynosi $d$, a podstawę $\overline{AB}$ widač ze środka okręgu wpisanego pod $\mathrm{k}_{\Phi^{\mathrm{t}\mathrm{e}\mathrm{m}}}$

$\alpha$. Obliczyč pole tego trójkąta.

5. Stosując zasadę indukcji matematycznej udowodnič prawdziwośč dla $n\geq 1$ wzoru

$\displaystyle \cos x+\cos 3x+\ldots+\cos(2n-1)x=\frac{\sin 2nx}{2\sin x},\sin x\neq 0.$

6. Wyznaczyč granicę ciągu 0 wyrazie ogólnym

$a_{n}=\displaystyle \frac{\sqrt[6]{4n}}{\sqrt{n}-\sqrt{n+\sqrt[3]{4n^{2}}}},$

$n\geq 1.$

7. Dany jest wierzcholek $A(6,1)$ kwadratu. Wyznaczyč pozostałe wierzchołki tego

kwadratu wiedząc, $\dot{\mathrm{z}}\mathrm{e}$ wierzchofki sąsiadujące $\mathrm{z}A\mathrm{l}\mathrm{e}\mathrm{z}\Phi$jeden na prostej $l:x-2y+1=$

$0$, a jeden na prostej $k:x+3y-4=0$. Sporządzič rysunek.

8. Przeprowadzič badanie $\mathrm{i}$ wykonač wykres funkcji

$f(x)=\displaystyle \frac{x+1}{\sqrt{x}}.$

3





styczeń 2004r.

PRACA KONTROLNA nr 4

l. Statek plynie z Wrocfawia do Szczecina 3 dni, a ze Szczecina do Wrocfawia 5 dni.

Jak długo z Wrocławia do Szczecina płynie woda?

2. Dla jakich wartości rzeczywistych parametru x liczby

$1+\log_{2}3, \log_{x}36,$

$\displaystyle \frac{4}{3}\log_{8}6$

są trzema kolejnymi wyrazami pewnego ciagu geometrycznego.

3. Wanna $0$ pojemności 2001 mająca kszta1t pofowy wa1ca (rozciętego wzdfuz osi) $\mathrm{l}\mathrm{e}\dot{\mathrm{z}}\mathrm{y}$

poziomo na ziemi $\mathrm{i}$ zawiera pewną ilośč wody. Do wanny włozono belkę $\mathrm{w}$ kształcie

walca $0$ średnicy cztery razy mniejszej $\mathrm{n}\mathrm{i}\dot{\mathrm{z}}$ średnica wanny $\mathrm{i}$ długości równej polowie

dlugości wanny. Okazafo się, $\dot{\mathrm{z}}\mathrm{e}$ lustro wody styka się $\mathrm{z}$ belką $\mathrm{z}\mathrm{a}\mathrm{n}\mathrm{u}\mathrm{r}\mathrm{z}\mathrm{o}\mathrm{n}\Phi^{\mathrm{W}}$ wodzie.

Ile wody znajduje się $\mathrm{w}$ wannie? Podač $\mathrm{z}$ dokładnością do 0,11.

4. Wyznaczyč wszystkie wartości parametru $m$, dla których obydwa pierwiastki trój-

mianu kwadratowego $v(x)=x^{2}+mx-m^{2}\mathrm{l}\mathrm{e}\dot{\mathrm{z}}$ ą pomiędzy pierwiastkami trójmianu

$w(x)=x^{2}-(m-1)x-m.$

5. Urna A zawiera trzy kule biafe $\mathrm{i}$ dwie czarne, a urna $\mathrm{B}$ dwie biafe $\mathrm{i}$ trzy czarne.

Wylosowano cztery razy jedną kulę ze zwracaniem $\mathrm{z}$ urny A oraz jedną kulę $\mathrm{z}$ urny

B. Obliczyč prawdopodobieństwo tego, $\dot{\mathrm{z}}\mathrm{e}$ wśród pięciu wylosowanych kul są co

najmniej dwie kule biafe.

6. Rozwiązač równanie:

2 $\sin 2x+2\cos 2x+\mathrm{t}\mathrm{g}x=3.$

7. Danajest funkcja $f(x)=x^{4}-2x^{2}$. Wyznaczyč wszystkie proste styczne do wykresu

tej funkcji zawierające punkt $P(1,-1)$. Określič ile punktów wspólnych $\mathrm{z}$ wykresem

tej funkcji mają wyznaczone styczne. Rozwiązanie zilustrowač rysunkiem.

8. Podstawą ostroslupa ABCS jest trójkąt równoramienny, którego kąt przy wierz-

chołku $C$ ma miarę $\alpha$, a ramię ma długośč $BC=b$. Spodek wysokości ostrosłupa

$\mathrm{l}\mathrm{e}\dot{\mathrm{z}}\mathrm{y}\mathrm{w}$ środku wysokości $\overline{CD}$ podstawy, a kąt pfaski ściany bocznej $ABS$ przy

wierzchofku ma miarę $\alpha$. Obliczyč promień kuli opisanej na tym ostrosfupie oraz

cosinusy katów nachylenia ścian bocznych do podstawy.

4





luty 2004r.

PRACA KONTROLNA nr 5

l. Piąty wyraz rozwinięcia dwumianu $(a+b)^{18}$ jest $0$ 180\% większy od wyrazu trze-

ciego. $\mathrm{O}$ ile procent wyraz ósmy tego rozwinięcia jest mniejszy $\mathrm{b}\text{ą} \mathrm{d}\acute{\mathrm{z}}$ większy od

wyrazu czwartego?

2. Wyznaczyč równanie linii utworzonej przez wszystkie punkty plaszczyzny, dla któ-

rych stosunek kwadratu odległości od prostej $k$ : $x-2y+3 = 0$ do kwadratu

odlegfości od prostej $l:3x+y+2=0$ wynosi 2. Sporządzič rysunek.

3. Obwód trójkąta $ABC$ wynosi 15, a dwusieczna kąta $A$ dzieli bok przeciwlegfy na

odcinki długości 3 oraz 2. Ob1iczyč po1e koła wpisanego $\mathrm{w}$ ten trójkąt.
\begin{center}
\includegraphics[width=182.316mm,height=37.488mm]{./KursMatematyki_PolitechnikaWroclawska_2003_2004_page4_images/image001.eps}
\end{center}
$a_{2}$

$a_{3}$

$a_{4}$

{\it O}

po nieskonczonej famanej jak na rysunku obok,

$p$ ktorej długosci kolejnych odcinkow tworz ci $\mathrm{g}$

cz stka zatrzymała się $\mathrm{w}$ punkcie $P(10,3)$. Jaką

drogę przebyla cz stka?

4. $\mathrm{C}_{\mathrm{Z}\Phi}$stka startuje $\mathrm{z}\mathrm{P}^{\mathrm{o}\mathrm{c}\mathrm{z}}\Phi^{\mathrm{t}\mathrm{k}\mathrm{u}}$ ukfadu wspólrzędnych $\mathrm{i}$ porusza się ze stafą prędkością

$\alpha_{1}$

5. Stosując zasadę indukcji matematycznej udowodnič, $\dot{\mathrm{z}}\mathrm{e}\mathrm{d}\mathrm{l}\mathrm{a}$ wszystkich $n\geq 1$ wie-

lomian $x^{3n+1}+x^{3n-1}+1$ dzieli się $\mathrm{b}\mathrm{e}\mathrm{z}$ reszty przez wielomian $x^{2}+x+1.$

6. Nie przeprowadzajqc badania przebiegu wykonač wykres funkcji

$f(x)=\displaystyle \frac{|x-2|}{x-|x|+2}.$

Podač równania asymptot i ekstrema lokalne tej funkcji.

7. Rozwi$\Phi$zač nierównośč

$|\cos x|^{1+\sqrt{2}\sin x+\sqrt{2}\cos x}\leq 1,$

$x\in[-\pi,\pi].$

8. W stozek wpisano graniastosfup trójkątny prawidłowy 0 wszystkich krawedziach tej

samej dfugości. Przyjakim kącie rozwarcia stozka stosunek objętości graniastosłupa

do objetości stozka jest największy?

5





marzec 2004r.

PRACA KONTROLNA nr 6

1. $\mathrm{W}$ kolo $0$ powierzchni $\displaystyle \frac{5}{4}\pi$ wpisano trójkąt prostokątny $0$ polu l. Obliczyč obwód

tego trójkąta.

2. Sprowadzič do najprostszej postaci wyrazenie

2(sin6 $\alpha+\cos^{6}\alpha$)$-7(\sin^{4}\alpha+\cos^{4}\alpha)+\cos 4\alpha.$

3. Wyznaczyč trójmian kwadratowy, którego wykresem jest parabola styczna do pro-

stej $y=x+2$, przechodząca przez punkt $P(-2,-2)$ oraz symetryczna względem

prostej $x=1$. Sporzqdzič rysunek.

4. $\mathrm{W}$ trapezie ABCD, $\mathrm{w}$ którym $\overline{AB}\Vert\overline{CD}$, dane są $\vec{AC}=(4,7)\rightarrow$ oraz $\vec{BD}=\rightarrow\rightarrow(-6,2).$

Posfugując się rachunkiem wektorowym wyznaczyč wektory AB $\mathrm{i}\vec{CD}$, jeśli $AD\perp BD.$

5. Jaś ma $\mathrm{w}$ portmonetce 3 monetyjednozłotowe, 2 monety dwuzłotowe ijedną pięcio-

złotową. Kupujac zeszyt $\mathrm{w}$ cenie 4 zł wyciaga 1osowo $\mathrm{z}$ portmonetki po jednej mo-

necie tak dlugo, $\mathrm{a}\dot{\mathrm{z}}$ nazbiera się suma wystarczająca do zaplaty za zeszyt. Obliczyč

prawdopodobieństwo, $\dot{\mathrm{z}}\mathrm{e}$ wyciągnie co najmniej trzy monety. Podač odpowiednie

uzasadnienie (nie jest nim $\mathrm{t}\mathrm{z}\mathrm{w}$. drzewko).

6. Narysowač na pfaszczy $\acute{\mathrm{z}}\mathrm{n}\mathrm{i}\mathrm{e}$ zbiór punktów określony następująco

$\mathcal{F}=\{(x,y):\sqrt{4x-x^{2}}\leq y\leq 4-\sqrt{1-2x+x^{2}}\}.$

$\mathrm{W}$ jakiej odleglości od brzegu figury $\mathcal{F}$ znajduje się punkt $P(\displaystyle \frac{3}{2},\frac{5}{2})$ ?

7. Dana jest funkcja $f(x) = \log_{2}(1-x^{2})-\log_{2}(x^{2}-x)$. Nie korzystając $\mathrm{z}$ metod

rachunku rózniczkowego wykazač, $\dot{\mathrm{z}}\mathrm{e}f$ jest rosnąca $\mathrm{w}$ swojej dziedzinie oraz, $\dot{\mathrm{z}}\mathrm{e}$

$g(x)=f(x-\displaystyle \frac{1}{2})$ jest nieparzysta. Wyznaczyč funkcję odwrotną $f^{-1}$, jej dziedzinę

$\mathrm{i}$ zbiór wartości.

8. Pole powierzchni bocznej ostrosłupa prawidfowego czworokątnego wynosi $c^{2}$, a kąt

nachylenia ściany bocznej do podstawy ma miarę $\alpha$. Ostrosłup rozcieto na dwie

części pfaszczyzną przechodzącą przez jeden $\mathrm{z}$ wierzchołków podstawy $\mathrm{i}$ prostopa-

dłą do przeciwległej krawędzi bocznej. Obliczyč objętośč części zawierającej wierz-

cholek ostrosłupa. Kiedy zadanie ma sens?

6





kwiecień 2004r.

PRACA KONTROLNA nr 7

l. Pierwsze dwa wyrazy ciągu geometrycznego są rozwiazaniami równania

$4x^{2}-4px-3p^{2}=0$, gdzie $p$ jest nieznaną $1\mathrm{i}\mathrm{c}\mathrm{z}\mathrm{b}_{\Phi}$. Wyznaczyč ten ciąg, jeśli suma

wszystkich jego wyrazów wynosi 3.

2. Wiedząc, $\dot{\mathrm{z}}\mathrm{e} \cos\varphi = \sqrt{\frac{2}{3}}$ oraz $\varphi \in (\displaystyle \frac{3}{2}\pi,2\pi)$, obliczyč cosinus kąta pomiędzy

prostymi $y=(\displaystyle \sin\frac{\varphi}{2})x, y=(\displaystyle \cos\frac{\varphi}{2})x.$

3. Kostka sześcienna ma krawęd $\acute{\mathrm{z}} 2a$. Aby zmieścič ją $\mathrm{w}$ pojemniku $\mathrm{w}$ kształcie kuli

$0$ średnicy $3a$, ze wszystkich narozy odcięto $\mathrm{w}$ minimalny sposób jednakowe ostro-

slupy prawidfowe trójk$\Phi$tne. Obliczyč dlugośč krawędzi bocznej odciętych czworo-

ścianów?

4. Udowodnič prawdziwośč nierówności

$1+\displaystyle \frac{x}{2}\geq\sqrt{1+x}\geq 1+\frac{x}{2}-\frac{x^{2}}{2}$ dla $x\in[-1,1].$

Zilustrowač $\mathrm{j}_{\Phi}$ na odpowiednim wykresie.

5. Rozwiązač równanie:

--csoins25{\it xx}$=$-sin3{\it x}.

6. Znalez/č równanie okręgu symetrycznego do okręgu $x^{2}-4x+y^{2}+6y=0$ wzglę-

dem stycznej do tego okręgu poprowadzonej $\mathrm{z}$ punktu $P(3,5) \mathrm{i}$ majqcej dodatni

wspófczynnik kierunkowy.

7. $\mathrm{W}$ okrąg $0$ promieniu $r$ wpisano trapez $0$ przekątnej $d\geq r\sqrt{3}\mathrm{i}$ największym ob-

wodzie. Obliczyč pole tego trapezu.

8. Metodą analityczną określič dla jakich wartości parametru $m$ układ równań

$\left\{\begin{array}{l}
mx\\
x
\end{array}\right.$

$-y$

$-2|y|$

$+2=0$

$+2=0$

ma dokladnie jedno rozwiązanie? Wyznaczyč to rozwiązanie w zalezności od m.

Sporządzič rysunek.

7



\end{document}