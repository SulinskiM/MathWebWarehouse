\documentclass[a4paper,12pt]{article}
\usepackage{latexsym}
\usepackage{amsmath}
\usepackage{amssymb}
\usepackage{graphicx}
\usepackage{wrapfig}
\pagestyle{plain}
\usepackage{fancybox}
\usepackage{bm}

\begin{document}

XXXIII

KORESPONDENCYJNY KURS Z MATEMATYKI

$\mathrm{p}\mathrm{a}\acute{\mathrm{z}}$dziernik 2$003\mathrm{r}.$

PRACA KONTROLNA nr l

l. Podstawą trójk$\Phi$ta równoramiennegojest odcinek $\overline{AB}0$ końcach $A(-1,3), B(1,-1),$

a wierzchołek $C$ tego trójkąta $\mathrm{l}\mathrm{e}\dot{\mathrm{z}}\mathrm{y}$ na prostej $l\mathrm{o}$ równaniu $3x-y-14=0$. Obliczyč

pole trójkąta $ABC.$

2. Pewna liczba sześciocyfrowa zaczyna się ($\mathrm{z}$ lewej strony) cyfrą 3. Jeś1i cyfrę tę

przestawimy $\mathrm{z}$ pierwszej pozycji na ostatnią, to otrzymamy liczbę stanowiacą 25\%

liczby pierwotnej. Znalez/č tę liczbę.

3. $\mathrm{W}$ trapezie opisanym na okregu kąty ostre przy podstawie mają miary $\alpha \mathrm{i}2\alpha, \mathrm{a}$

dlugośč krótszego ramienia wynosi $c$. Obliczyč długośč krótszej podstawy tego

trapezu. Wynik doprowadzič do najprostszej postaci.

4. Rozwiązač nierównośč:

$\displaystyle \frac{1}{x^{2}-x-2}\leq\frac{1}{|x|}.$

5. Zaznaczyč na pfaszczy $\acute{\mathrm{z}}\mathrm{n}\mathrm{i}\mathrm{e}$ zbiór wszystkich punktów $(x,y)$ spelniających nierów-

nośč $\log_{x}(1+(y-1)^{3})\leq 1.$

6. Rozwiązač równanie:

$\sin^{2}3x$ -sin2 $2x=\sin^{2}x.$

7. Wysokośč ostroslupa prawidfowego czworokątnego jest trzy razy dfuzsza od pro-

mienia kuli wpisanej $\mathrm{w}$ ten ostroslup Obliczyč cosinus kata pomiędzy sąsiednimi

ścianami bocznymi tego ostrosłupa.

8. Dany jest nieskończony ciąg geometryczny: $x+1, -x^{2}(x+1), x^{4}(x+1), \ldots$ Wyzna-

czyč najmniejszą $\mathrm{i}$ największą wartośč funkcji $S(x)$ oznaczającej sumę wszystkich

wyrazów tego ciągu.

1
\end{document}
