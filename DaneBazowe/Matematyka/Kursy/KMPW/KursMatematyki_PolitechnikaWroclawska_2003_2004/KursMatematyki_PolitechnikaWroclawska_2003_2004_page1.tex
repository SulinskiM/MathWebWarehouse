\documentclass[a4paper,12pt]{article}
\usepackage{latexsym}
\usepackage{amsmath}
\usepackage{amssymb}
\usepackage{graphicx}
\usepackage{wrapfig}
\pagestyle{plain}
\usepackage{fancybox}
\usepackage{bm}

\begin{document}

listopad 2003r.

PRACA KONTROLNA nr 2

l. Trójkąt $\mathrm{P}^{\mathrm{r}\mathrm{o}\mathrm{s}\mathrm{t}\mathrm{o}\mathrm{k}}\Phi^{\mathrm{t}\mathrm{n}\mathrm{y}} \mathrm{o}\mathrm{b}\mathrm{r}\mathrm{a}\mathrm{c}\mathrm{a}\mathrm{j}_{\Phi}\mathrm{c}$ się wokół jednej $\mathrm{i}$ drugiej przyprostokątnej daje

bryły $0$ objętościach $V_{1} \mathrm{i} V_{2}$, odpowiednio. Obliczyč objętośč bryły powstałej $\mathrm{z}$

obrotu tego trójkąta wokół dwusiecznej kąta prostego.

2. Czy $\mathrm{m}\mathrm{o}\dot{\mathrm{z}}$ na sumę 42000 z1otych podzie1ič na pewną 1iczbę nagród $\mathrm{t}\mathrm{a}\mathrm{k}$, aby kwoty

tych nagród wyrazaly się $\mathrm{w}$ pelnych setkach złotych, tworzyly ciąg arytmetyczny

oraz najwyzsza nagroda wynosifa 13000 $\mathrm{z}\mathrm{f}$? Jeśli $\mathrm{t}\mathrm{a}\mathrm{k}$, to podač liczbę $\mathrm{i}$ wysokości

tych nagród.

3. Dane sq okregi $0$ równaniach $(x-1)^{2}+(y-1)^{2}=1$ oraz $(x-2)^{2}+(y-1)^{2}=16.$

Wyznaczyč równania wszystkich okręgów stycznych równocześnie do obu danych

okręgów oraz do osi Oy. Sporządzič rysunek.

4. $\mathrm{W}$ równolegloboku kąt ostry miedzy przekqtnymi ma miarę $\beta$, a stosunek dfugości

dfuzszej przekątnej do krótszej przekątnej wynosi $k$. Obliczyč tangens kąta ostrego

tego równoległoboku.

5. Rozwiązač równanie $\sqrt{4x-3}-3=\sqrt{2x-10}.$

6. Dobrač liczby calkowite a,b $\mathrm{t}\mathrm{a}\mathrm{k}$, aby wielomian $6x^{3}-7x^{2}+1$ dzielil się bez reszty

przez trójmian kwadratowy $2x^{2}+ax+b.$

7. Rozwiązač nierównośč $|2^{x}-3|\leq 2^{1-x}$ Rozwiązanie zilustrowač na rysunku wyko-

nując wykresy funkcji występujqcych po obu stronach tej nierówności.

8. Wyznaczyč przedziały monotoniczności funkcji

$f(x)=\displaystyle \sin^{2}x+\frac{\sqrt{3}}{2}x,$

$x\in[-\pi,\pi].$

2
\end{document}
