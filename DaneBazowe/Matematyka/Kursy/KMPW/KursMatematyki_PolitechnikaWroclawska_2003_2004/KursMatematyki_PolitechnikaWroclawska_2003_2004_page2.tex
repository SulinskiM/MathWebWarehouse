\documentclass[a4paper,12pt]{article}
\usepackage{latexsym}
\usepackage{amsmath}
\usepackage{amssymb}
\usepackage{graphicx}
\usepackage{wrapfig}
\pagestyle{plain}
\usepackage{fancybox}
\usepackage{bm}

\begin{document}

grudzień 2003r.

PRACA KONTROLNA nr 3

l. Obliczyč prawdopodobieństwo tego, $\dot{\mathrm{z}}\mathrm{e}$ gracz losując 7 kart $\mathrm{z}$ talii 24 kart do gry

otrzyma dokładnie cztery karty $\mathrm{w}$ jednym kolorze $\mathrm{w}$ tym asa, króla $\mathrm{i}$ damę.

2. Pewien ostroslup przecięto na trzy części dwiema płaszczyznami równoległymi do

jego podstawy. Pierwsza pfaszczyznajest polozona $\mathrm{w}$ odlegfości $d_{1}=2$ cm, a druga

$\mathrm{w}$ odległości $d_{2}=3$ cm od podstawy. Pola przekrojów ostroslupa tymi plaszczy-

znami równe są odpowiednio $S_{1}=25\mathrm{c}\mathrm{m}^{2}$ oraz $S_{2}=16\mathrm{c}\mathrm{m}^{2}$ Obliczyč objętośč tego

ostroslupa oraz objętośč najmniejszej części.

3. Rozwiązač ukfad równań:

$\left\{\begin{array}{l}
x^{2}+y^{2}=24\\
\frac{2\log x+\log y^{2}}{\log(x+y)}=2
\end{array}\right.$

4. $\mathrm{W}$ trójkącie równoramiennym $ABC$ odległośč środka okręgu wpisanego od wierz-

chofka $C$ wynosi $d$, a podstawę $\overline{AB}$ widač ze środka okręgu wpisanego pod $\mathrm{k}_{\Phi^{\mathrm{t}\mathrm{e}\mathrm{m}}}$

$\alpha$. Obliczyč pole tego trójkąta.

5. Stosując zasadę indukcji matematycznej udowodnič prawdziwośč dla $n\geq 1$ wzoru

$\displaystyle \cos x+\cos 3x+\ldots+\cos(2n-1)x=\frac{\sin 2nx}{2\sin x},\sin x\neq 0.$

6. Wyznaczyč granicę ciągu 0 wyrazie ogólnym

$a_{n}=\displaystyle \frac{\sqrt[6]{4n}}{\sqrt{n}-\sqrt{n+\sqrt[3]{4n^{2}}}},$

$n\geq 1.$

7. Dany jest wierzcholek $A(6,1)$ kwadratu. Wyznaczyč pozostałe wierzchołki tego

kwadratu wiedząc, $\dot{\mathrm{z}}\mathrm{e}$ wierzchofki sąsiadujące $\mathrm{z}A\mathrm{l}\mathrm{e}\mathrm{z}\Phi$jeden na prostej $l:x-2y+1=$

$0$, a jeden na prostej $k:x+3y-4=0$. Sporządzič rysunek.

8. Przeprowadzič badanie $\mathrm{i}$ wykonač wykres funkcji

$f(x)=\displaystyle \frac{x+1}{\sqrt{x}}.$

3
\end{document}
