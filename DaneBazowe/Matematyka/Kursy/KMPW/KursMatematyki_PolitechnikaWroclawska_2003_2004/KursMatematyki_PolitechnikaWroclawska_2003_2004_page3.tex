\documentclass[a4paper,12pt]{article}
\usepackage{latexsym}
\usepackage{amsmath}
\usepackage{amssymb}
\usepackage{graphicx}
\usepackage{wrapfig}
\pagestyle{plain}
\usepackage{fancybox}
\usepackage{bm}

\begin{document}

styczeń 2004r.

PRACA KONTROLNA nr 4

l. Statek plynie z Wrocfawia do Szczecina 3 dni, a ze Szczecina do Wrocfawia 5 dni.

Jak długo z Wrocławia do Szczecina płynie woda?

2. Dla jakich wartości rzeczywistych parametru x liczby

$1+\log_{2}3, \log_{x}36,$

$\displaystyle \frac{4}{3}\log_{8}6$

są trzema kolejnymi wyrazami pewnego ciagu geometrycznego.

3. Wanna $0$ pojemności 2001 mająca kszta1t pofowy wa1ca (rozciętego wzdfuz osi) $\mathrm{l}\mathrm{e}\dot{\mathrm{z}}\mathrm{y}$

poziomo na ziemi $\mathrm{i}$ zawiera pewną ilośč wody. Do wanny włozono belkę $\mathrm{w}$ kształcie

walca $0$ średnicy cztery razy mniejszej $\mathrm{n}\mathrm{i}\dot{\mathrm{z}}$ średnica wanny $\mathrm{i}$ długości równej polowie

dlugości wanny. Okazafo się, $\dot{\mathrm{z}}\mathrm{e}$ lustro wody styka się $\mathrm{z}$ belką $\mathrm{z}\mathrm{a}\mathrm{n}\mathrm{u}\mathrm{r}\mathrm{z}\mathrm{o}\mathrm{n}\Phi^{\mathrm{W}}$ wodzie.

Ile wody znajduje się $\mathrm{w}$ wannie? Podač $\mathrm{z}$ dokładnością do 0,11.

4. Wyznaczyč wszystkie wartości parametru $m$, dla których obydwa pierwiastki trój-

mianu kwadratowego $v(x)=x^{2}+mx-m^{2}\mathrm{l}\mathrm{e}\dot{\mathrm{z}}$ ą pomiędzy pierwiastkami trójmianu

$w(x)=x^{2}-(m-1)x-m.$

5. Urna A zawiera trzy kule biafe $\mathrm{i}$ dwie czarne, a urna $\mathrm{B}$ dwie biafe $\mathrm{i}$ trzy czarne.

Wylosowano cztery razy jedną kulę ze zwracaniem $\mathrm{z}$ urny A oraz jedną kulę $\mathrm{z}$ urny

B. Obliczyč prawdopodobieństwo tego, $\dot{\mathrm{z}}\mathrm{e}$ wśród pięciu wylosowanych kul są co

najmniej dwie kule biafe.

6. Rozwiązač równanie:

2 $\sin 2x+2\cos 2x+\mathrm{t}\mathrm{g}x=3.$

7. Danajest funkcja $f(x)=x^{4}-2x^{2}$. Wyznaczyč wszystkie proste styczne do wykresu

tej funkcji zawierające punkt $P(1,-1)$. Określič ile punktów wspólnych $\mathrm{z}$ wykresem

tej funkcji mają wyznaczone styczne. Rozwiązanie zilustrowač rysunkiem.

8. Podstawą ostroslupa ABCS jest trójkąt równoramienny, którego kąt przy wierz-

chołku $C$ ma miarę $\alpha$, a ramię ma długośč $BC=b$. Spodek wysokości ostrosłupa

$\mathrm{l}\mathrm{e}\dot{\mathrm{z}}\mathrm{y}\mathrm{w}$ środku wysokości $\overline{CD}$ podstawy, a kąt pfaski ściany bocznej $ABS$ przy

wierzchofku ma miarę $\alpha$. Obliczyč promień kuli opisanej na tym ostrosfupie oraz

cosinusy katów nachylenia ścian bocznych do podstawy.

4
\end{document}
