\documentclass[a4paper,12pt]{article}
\usepackage{latexsym}
\usepackage{amsmath}
\usepackage{amssymb}
\usepackage{graphicx}
\usepackage{wrapfig}
\pagestyle{plain}
\usepackage{fancybox}
\usepackage{bm}

\begin{document}

luty 2004r.

PRACA KONTROLNA nr 5

l. Piąty wyraz rozwinięcia dwumianu $(a+b)^{18}$ jest $0$ 180\% większy od wyrazu trze-

ciego. $\mathrm{O}$ ile procent wyraz ósmy tego rozwinięcia jest mniejszy $\mathrm{b}\text{ą} \mathrm{d}\acute{\mathrm{z}}$ większy od

wyrazu czwartego?

2. Wyznaczyč równanie linii utworzonej przez wszystkie punkty plaszczyzny, dla któ-

rych stosunek kwadratu odległości od prostej $k$ : $x-2y+3 = 0$ do kwadratu

odlegfości od prostej $l:3x+y+2=0$ wynosi 2. Sporządzič rysunek.

3. Obwód trójkąta $ABC$ wynosi 15, a dwusieczna kąta $A$ dzieli bok przeciwlegfy na

odcinki długości 3 oraz 2. Ob1iczyč po1e koła wpisanego $\mathrm{w}$ ten trójkąt.
\begin{center}
\includegraphics[width=182.316mm,height=37.488mm]{./KursMatematyki_PolitechnikaWroclawska_2003_2004_page4_images/image001.eps}
\end{center}
$a_{2}$

$a_{3}$

$a_{4}$

{\it O}

po nieskonczonej famanej jak na rysunku obok,

$p$ ktorej długosci kolejnych odcinkow tworz ci $\mathrm{g}$

cz stka zatrzymała się $\mathrm{w}$ punkcie $P(10,3)$. Jaką

drogę przebyla cz stka?

4. $\mathrm{C}_{\mathrm{Z}\Phi}$stka startuje $\mathrm{z}\mathrm{P}^{\mathrm{o}\mathrm{c}\mathrm{z}}\Phi^{\mathrm{t}\mathrm{k}\mathrm{u}}$ ukfadu wspólrzędnych $\mathrm{i}$ porusza się ze stafą prędkością

$\alpha_{1}$

5. Stosując zasadę indukcji matematycznej udowodnič, $\dot{\mathrm{z}}\mathrm{e}\mathrm{d}\mathrm{l}\mathrm{a}$ wszystkich $n\geq 1$ wie-

lomian $x^{3n+1}+x^{3n-1}+1$ dzieli się $\mathrm{b}\mathrm{e}\mathrm{z}$ reszty przez wielomian $x^{2}+x+1.$

6. Nie przeprowadzajqc badania przebiegu wykonač wykres funkcji

$f(x)=\displaystyle \frac{|x-2|}{x-|x|+2}.$

Podač równania asymptot i ekstrema lokalne tej funkcji.

7. Rozwi$\Phi$zač nierównośč

$|\cos x|^{1+\sqrt{2}\sin x+\sqrt{2}\cos x}\leq 1,$

$x\in[-\pi,\pi].$

8. W stozek wpisano graniastosfup trójkątny prawidłowy 0 wszystkich krawedziach tej

samej dfugości. Przyjakim kącie rozwarcia stozka stosunek objętości graniastosłupa

do objetości stozka jest największy?

5
\end{document}
