\documentclass[a4paper,12pt]{article}
\usepackage{latexsym}
\usepackage{amsmath}
\usepackage{amssymb}
\usepackage{graphicx}
\usepackage{wrapfig}
\pagestyle{plain}
\usepackage{fancybox}
\usepackage{bm}

\begin{document}

marzec 2004r.

PRACA KONTROLNA nr 6

1. $\mathrm{W}$ kolo $0$ powierzchni $\displaystyle \frac{5}{4}\pi$ wpisano trójkąt prostokątny $0$ polu l. Obliczyč obwód

tego trójkąta.

2. Sprowadzič do najprostszej postaci wyrazenie

2(sin6 $\alpha+\cos^{6}\alpha$)$-7(\sin^{4}\alpha+\cos^{4}\alpha)+\cos 4\alpha.$

3. Wyznaczyč trójmian kwadratowy, którego wykresem jest parabola styczna do pro-

stej $y=x+2$, przechodząca przez punkt $P(-2,-2)$ oraz symetryczna względem

prostej $x=1$. Sporzqdzič rysunek.

4. $\mathrm{W}$ trapezie ABCD, $\mathrm{w}$ którym $\overline{AB}\Vert\overline{CD}$, dane są $\vec{AC}=(4,7)\rightarrow$ oraz $\vec{BD}=\rightarrow\rightarrow(-6,2).$

Posfugując się rachunkiem wektorowym wyznaczyč wektory AB $\mathrm{i}\vec{CD}$, jeśli $AD\perp BD.$

5. Jaś ma $\mathrm{w}$ portmonetce 3 monetyjednozłotowe, 2 monety dwuzłotowe ijedną pięcio-

złotową. Kupujac zeszyt $\mathrm{w}$ cenie 4 zł wyciaga 1osowo $\mathrm{z}$ portmonetki po jednej mo-

necie tak dlugo, $\mathrm{a}\dot{\mathrm{z}}$ nazbiera się suma wystarczająca do zaplaty za zeszyt. Obliczyč

prawdopodobieństwo, $\dot{\mathrm{z}}\mathrm{e}$ wyciągnie co najmniej trzy monety. Podač odpowiednie

uzasadnienie (nie jest nim $\mathrm{t}\mathrm{z}\mathrm{w}$. drzewko).

6. Narysowač na pfaszczy $\acute{\mathrm{z}}\mathrm{n}\mathrm{i}\mathrm{e}$ zbiór punktów określony następująco

$\mathcal{F}=\{(x,y):\sqrt{4x-x^{2}}\leq y\leq 4-\sqrt{1-2x+x^{2}}\}.$

$\mathrm{W}$ jakiej odleglości od brzegu figury $\mathcal{F}$ znajduje się punkt $P(\displaystyle \frac{3}{2},\frac{5}{2})$ ?

7. Dana jest funkcja $f(x) = \log_{2}(1-x^{2})-\log_{2}(x^{2}-x)$. Nie korzystając $\mathrm{z}$ metod

rachunku rózniczkowego wykazač, $\dot{\mathrm{z}}\mathrm{e}f$ jest rosnąca $\mathrm{w}$ swojej dziedzinie oraz, $\dot{\mathrm{z}}\mathrm{e}$

$g(x)=f(x-\displaystyle \frac{1}{2})$ jest nieparzysta. Wyznaczyč funkcję odwrotną $f^{-1}$, jej dziedzinę

$\mathrm{i}$ zbiór wartości.

8. Pole powierzchni bocznej ostrosłupa prawidfowego czworokątnego wynosi $c^{2}$, a kąt

nachylenia ściany bocznej do podstawy ma miarę $\alpha$. Ostrosłup rozcieto na dwie

części pfaszczyzną przechodzącą przez jeden $\mathrm{z}$ wierzchołków podstawy $\mathrm{i}$ prostopa-

dłą do przeciwległej krawędzi bocznej. Obliczyč objętośč części zawierającej wierz-

cholek ostrosłupa. Kiedy zadanie ma sens?

6
\end{document}
