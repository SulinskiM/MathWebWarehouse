\documentclass[a4paper,12pt]{article}
\usepackage{latexsym}
\usepackage{amsmath}
\usepackage{amssymb}
\usepackage{graphicx}
\usepackage{wrapfig}
\pagestyle{plain}
\usepackage{fancybox}
\usepackage{bm}

\begin{document}

kwiecień 2004r.

PRACA KONTROLNA nr 7

l. Pierwsze dwa wyrazy ciągu geometrycznego są rozwiazaniami równania

$4x^{2}-4px-3p^{2}=0$, gdzie $p$ jest nieznaną $1\mathrm{i}\mathrm{c}\mathrm{z}\mathrm{b}_{\Phi}$. Wyznaczyč ten ciąg, jeśli suma

wszystkich jego wyrazów wynosi 3.

2. Wiedząc, $\dot{\mathrm{z}}\mathrm{e} \cos\varphi = \sqrt{\frac{2}{3}}$ oraz $\varphi \in (\displaystyle \frac{3}{2}\pi,2\pi)$, obliczyč cosinus kąta pomiędzy

prostymi $y=(\displaystyle \sin\frac{\varphi}{2})x, y=(\displaystyle \cos\frac{\varphi}{2})x.$

3. Kostka sześcienna ma krawęd $\acute{\mathrm{z}} 2a$. Aby zmieścič ją $\mathrm{w}$ pojemniku $\mathrm{w}$ kształcie kuli

$0$ średnicy $3a$, ze wszystkich narozy odcięto $\mathrm{w}$ minimalny sposób jednakowe ostro-

slupy prawidfowe trójk$\Phi$tne. Obliczyč dlugośč krawędzi bocznej odciętych czworo-

ścianów?

4. Udowodnič prawdziwośč nierówności

$1+\displaystyle \frac{x}{2}\geq\sqrt{1+x}\geq 1+\frac{x}{2}-\frac{x^{2}}{2}$ dla $x\in[-1,1].$

Zilustrowač $\mathrm{j}_{\Phi}$ na odpowiednim wykresie.

5. Rozwiązač równanie:

--csoins25{\it xx}$=$-sin3{\it x}.

6. Znalez/č równanie okręgu symetrycznego do okręgu $x^{2}-4x+y^{2}+6y=0$ wzglę-

dem stycznej do tego okręgu poprowadzonej $\mathrm{z}$ punktu $P(3,5) \mathrm{i}$ majqcej dodatni

wspófczynnik kierunkowy.

7. $\mathrm{W}$ okrąg $0$ promieniu $r$ wpisano trapez $0$ przekątnej $d\geq r\sqrt{3}\mathrm{i}$ największym ob-

wodzie. Obliczyč pole tego trapezu.

8. Metodą analityczną określič dla jakich wartości parametru $m$ układ równań

$\left\{\begin{array}{l}
mx\\
x
\end{array}\right.$

$-y$

$-2|y|$

$+2=0$

$+2=0$

ma dokladnie jedno rozwiązanie? Wyznaczyč to rozwiązanie w zalezności od m.

Sporządzič rysunek.

7
\end{document}
