\documentclass[a4paper,12pt]{article}
\usepackage{latexsym}
\usepackage{amsmath}
\usepackage{amssymb}
\usepackage{graphicx}
\usepackage{wrapfig}
\pagestyle{plain}
\usepackage{fancybox}
\usepackage{bm}

\begin{document}

XXXIV

KORESPONDENCYJNY KURS Z MATEMATYKI

PRACA KONTROLNA nr l

$\mathrm{p}\mathrm{a}\acute{\mathrm{z}}$dziernik 2$004\mathrm{r}.$

l. Staś kupił zeszyty 32-kartkowe po 80 gr za sztukę $\mathrm{i}$ zeszyty 60-kartkowe po 1,20 zł za sztukę

$\mathrm{i}$ zapłacił 13,20 $\mathrm{z}l$. Ile zeszytów 60-kartkowych kupił Staś, jeś1i by1o ich więcej $\mathrm{n}\mathrm{i}\dot{\mathrm{z}}$ zeszytów

32-kartkowych?

2. Rozwiązač nierównośč

--{\it xx}32$+$-{\it xx}$\leq$1.

3. Dana jest parabola $0$ równaniu $y = -x^{2}+2x+3$. Znalez/č równanie paraboli, która jest

symetryczna do danej względem punktu $S(2,1)$, oraz wyznaczyč punkty, $\mathrm{w}$ których przecina

ona osie ukladu wspólrzędnych. Sporządzič rysunek.

4. $\mathrm{W}$ trójkącie prostokątnym równoramiennym $ABC$ dany jest wierzchołek kata prostego $C(1,1),$

a bok $\overline{AB}\mathrm{l}\mathrm{e}\dot{\mathrm{z}}\mathrm{y}$ na prostej $x+5y+7=0$. Wyznaczyč współrzędne wierzcholków A $\mathrm{i}B.$

5. $\mathrm{W}$ ostroslupie prawidlowym sześciokątnym kąty płaskie ścian bocznych przy wierzchołku są

równe $\alpha$. Wyznaczyč cosinus kata między sąsiednimi ścianami bocznymi tego ostroslupa.

6. Dany jest trójkąt równoramienny $0$ kqcie przy podstawie $\alpha \mathrm{i}$ ramieniu $b$. Ramiona tego

trójkąta przecięto prostą odcinając $\mathrm{z}$ niego deltoid. Wyznaczyč katy pozostałego mniejszego

trójkąta oraz jego pole. Kiedy zadanie ma rozwiazanie?

7. Rozwiązač nierównośč

$\sqrt{2^{x-2}+3}\leq 2^{x}-2.$

8. Wyznaczyč dziedzinę oraz narysowač wykres funkcji $s(x)$ danej wzorem

$s(x)=\log_{2}(1-x+x^{2}-x^{3}+\ldots).$

Przy pomocy wykresu określič zbiór wartości tej funkcji.

9. Rozwiązač równanie

tg 3{\it x}$=$ -csoins 42{\it xx}.




PRACA KONTROLNA nr 2

listopad 2004r.

l. Liczby $0$ 45\% mniejsza $\mathrm{i} 0$ 32\% większa od ułamka okresowego 0,(60) sq pierwiastkami

trójmianu kwadratowego $0$ współczynnikach całkowitych względnie pierwszych. Obliczyč

resztę $\mathrm{z}$ dzielenia tego trójmianu przez dwumian $(x-1).$

2. Wykres funkcji $f$ : $[0,5]\rightarrow R$ jest przed-

stawiony na rysunku obok. Narysowac

wykres funkcji $g(x)=f(x)-f(5-x)$

$\mathrm{i}$ zapisač $\mathrm{j}$ wzorem.
\begin{center}
\includegraphics[width=58.524mm,height=35.616mm]{./KursMatematyki_PolitechnikaWroclawska_2004_2005_page1_images/image001.eps}
\end{center}
$y$

2

1

0 1 3 5 $x$

$-1$

3. Obliczyc wartosci $\sin\alpha \mathrm{i}\cos\alpha$, jeśli wiadomo, $\dot{\mathrm{z}}\mathrm{e}$

$\displaystyle \sin\alpha+3\cos\alpha=\frac{1}{\cos\alpha},$

$\displaystyle \alpha\in[0,\pi]\backslash \{\frac{\pi}{2}\}.$

4. Suma 20 pierwszych wyrazów pewnego ciągu arytmetycznego jest równa zeru, a iloczyn dzie-

siqtego $\mathrm{i}$ jedenastego wyrazu wynosi $-1$. Dla jakich liczb naturalnych $n$ suma $n$ pierwszych

wyrazów tego ciągu przekracza 77?

5. Trapez równoramienny jest wpisany $\mathrm{w}$ okrąg $0$ promieniu $R$, a jednq $\mathrm{z}$ jego podstaw jest

średnica tego okręgu. $\mathrm{W}$ trapez ten daje się wpisač okrąg. Wyznaczyč jego promień.

6. Środek kuli opisanej na ostrosłupie prawidłowym trójkatnym $\mathrm{l}\mathrm{e}\dot{\mathrm{z}}\mathrm{y}\mathrm{w}$ odległości $d$ ponad

podstawą ostrosłupa, a kąt nachylenia krawędzi bocznej do podstawy wynosi $\alpha$. Obliczyč

objętośč ostrosłupa.

7. Wyznaczyč wszystkie wartości parametru rzeczywistego $m$, dla których funkcja

$f(x)=\displaystyle \frac{x+1}{x^{2}+mx+4}$

jest dodatnia i rosnąca na odcinku (0,1).

8. Nie korzystając z rachunku rózniczkowego wyznaczyč dziedzinę i zbiór wartości funkcji

$f(x)=\sqrt{\sqrt{2}-\cos x-\sqrt{3}\sin x},$

$x\in[0,\pi].$

9. Rozwiązač uklad równań

$\left\{\begin{array}{l}
|x+1|y\\
x^{2}-4|x|+2y-1
\end{array}\right.$

$=4$

$=0$

Przedstawič ilustrację graficzną obu równań i zaznaczyč na rysunku znalezione rozwiązania.





PRACA KONTROLNA nr 3

grudzień 2004r.

l. W pewnej szkole zapytano uczniów klas maturalnych ile razy w ostatnim miesiqcu ucze-

liczba uczniow stniczyli w imprezie kulturalnej. Wyniki przed-

10 uczniów jest w klasach maturalnych tej szkoly; b)
\begin{center}
\includegraphics[width=78.588mm,height=35.304mm]{./KursMatematyki_PolitechnikaWroclawska_2004_2005_page2_images/image001.eps}
\end{center}
15

5

stawiono na diagramie obok. Obliczyc: a) Ilu

Ile razy średnio w miesi cu uczeń był na imprezie

kulturalnej. Sporz dzic diagram kołowy przedsta-

0 1 2 3 4 5 6 7 wiaj cy procentowo otrzymane wyniki.

2. Turysta zauwazyl, $\dot{\mathrm{z}}\mathrm{e}\mathrm{w}$ pewnym miejscu na odcinku 10 $\mathrm{m}$ potok górski płynie $\mathrm{w}$ korycie

skalnym, które $\mathrm{w}$ przekroju pionowym tworzy trapez $0$ dolnej podstawie 2 $\mathrm{m}\mathrm{i}$ górnej 3 $\mathrm{m}.$

Wysokośč koryta wynosi 50 cm, przy czym woda wypełnia koryto jedynie na głębokośč 10

cm. Turysta ustalil równiez, $\dot{\mathrm{z}}\mathrm{e}$ czas przepływu wody przez koryto wynosi 3 sekundy. I1e

litrów wody przeplywa przez ten potok $\mathrm{w}$ ciągu jednej sekundy?

3. Wykazač, $\dot{\mathrm{z}}\mathrm{e}$ dla dowolnych liczb dodatnich $a, b$ prawdziwa jest nierównośč

$(a+b)^{3}\leq 4(a^{3}+b^{3}).$

Wsk. Podzielič obie strony przez $b^{3}\mathrm{i}$ wprowadzič jednq zmiennq.

4. Boki $\overline{AB}\mathrm{i}\overline{AD}$ równoległoboku $\mathrm{l}\mathrm{e}\dot{\mathrm{z}}$ ą odpowiednio na prostych $3x+4y-7=0\mathrm{i}x-2y+1=0.$

Wyznaczyč współrzędne wierzcholka $C$ tego równolegloboku wiedząc, $\dot{\mathrm{z}}\mathrm{e}$ jego wysokośč do

boku $\overline{AB}$ wynosi 2, a wierzcho1ek $B$ ma współrzędne $(5,-2).$

5. $\mathrm{W}$ trójkącie ostrokątnym $ABC$ dane sq bok $BC=\displaystyle \frac{5}{2}\sqrt{5}$ cm oraz wysokości $BD=\displaystyle \frac{11}{2}$ cm $\mathrm{i}$

$CE=5$ cm. Obliczyč obwód tego trójkąta oraz cosinus kata $\angle BAC.$

6. Spośród dwudziestu najmniejszych, nieparzystych liczb naturalnych wylosowano (bez zwra-

cania) dwie. Obliczyč prawdopodobieństwo, $\dot{\mathrm{z}}\mathrm{e}$ otrzymano: a) dwie liczby pierwsze; b) dwie

liczby względnie pierwsze.

7. Rozwiązač nierównośč $\log_{2} x^{\log_{4}x}\geq\log_{x}16.$

8. Niech $f(m)$ oznacza sumę trzecich potęg pierwiastków rzeczywistych równania kwadratowego

$x^{2}+(m+3)x+m^{2}=0\mathrm{z}$ parametrem $m$. Wyznaczyč wzór funkcji $f(m)$ oraz najmniejszq $\mathrm{i}$

największq wartośč tej funkcji.

9. $\mathrm{W}$ ostrosłupie prawidłowym czworokątnym $\mathrm{k}\mathrm{a}\mathrm{t}$ nachylenia krawędzi bocznej do podstawy

wynosi $\alpha$, a odległosc krawędzi podstawy od przeciwległej sciany

bocznej jest równa $d=3$ cm. Obliczyc wysokosc sciany bocznej.

Czy siatka tego ostrosłupa, jak na rysunku obok, zmiesci $\mathrm{s}\mathrm{i}_{9}$

na arkuszu papieru $\mathrm{w}$ ksztalcie kwadratu $0$ boku 16 cm, jes1i

wiadomo, $\dot{\mathrm{z}}\mathrm{e}\mathrm{t}\mathrm{g}\alpha=2$? Sporz dzic rysunek.
\begin{center}
\includegraphics[width=30.324mm,height=30.384mm]{./KursMatematyki_PolitechnikaWroclawska_2004_2005_page2_images/image002.eps}
\end{center}




PRACA KONTROLNA nr 4

styczeń $2005\mathrm{r}.$

l. Krawędzie oraz przekątna prostopadłościanu tworzq cztery kolejne wyrazy ciągu arytmetycz-

nego, przy czym przekątna ma długośč 7 cm. Jaką najkrótszq drogę musi przebyč mucha,

aby wędrujqc po krawędziach tego prostopadłościanu odwiedzila wszystkie jego wierzchołki.

2. Dany jest wielomian $w(x)=x^{4}-2x^{2}-x+2$. Rozłozyc na czynniki $\mathrm{m}\mathrm{o}\dot{\mathrm{z}}$ liwie najnizszego

stopnia wielomian $p(x)=w(x+1)-w(x).$
\begin{center}
\includegraphics[width=60.504mm,height=64.824mm]{./KursMatematyki_PolitechnikaWroclawska_2004_2005_page3_images/image001.eps}
\end{center}
{\it y}

2

0 2 $x$

3. Na rysunku obok przedstawiono fragment mapy $\mathrm{w}$ ska-

li 1:25000, który zawiera obszar 1asu $L$ ograniczony

czterema drogami. Na mapę jest naniesiona siatka ki-

lometrowa, a dodatkowo umieszczono na niej układ

współrzędnych pokrywaj cy się $\mathrm{z}$ wybranymi liniami

siatki. Zapisac obszar $L\mathrm{w}$ postaci układu nierownosci

liniowych ($\mathrm{w}$ skali mapy). Obliczyc rzeczywiste pole

obszaru $L$ wyrazaj $\mathrm{c}$ go $\mathrm{w}$ hektarach.

4. Na ile sposobów $\mathrm{m}\mathrm{o}\dot{\mathrm{z}}\mathrm{e}$ Krzyś rozdzielič 12 jednakowych cukierków pomiędzy siebie $\mathrm{i}$ trójkę

rodzeństwa, jeśli $\mathrm{k}\mathrm{a}\dot{\mathrm{z}}\mathrm{d}\mathrm{y}$ ma otrzymač co najmniej dwa cukierki.

5. $\mathrm{W}$ stozek wpisano sześcian $0$ krawędzi $a$. Rozwinięcie powierzchni bocznej stozka tworzy

wycinek koła $0$ kącie środkowym 1200. Ob1iczyč tangens kąta podjakim tworzącą tego stozka

widač ze środka sześcianu.

6. $\mathrm{W}$ trójkącie $ABC$ dane $\mathrm{S}\otimes$katy $\alpha \mathrm{i}\beta$ przy podstawie $\overline{AB}$ oraz środkowa $CD=s$ podstawy.

Obliczyč pole tego trójkąta.

7. Rozwiązač równanie $3^{\sin x}+9^{\sin x}+27^{\sin x}+\ldots= \displaystyle \frac{\sqrt{3}+1}{2},$

nieskończonego ciągu geometrycznego.

którego lewa strona jest sumą

8. Stosując zasadę indukcji matematycznej udowodnič nierównośč:

$1-\sqrt{2}+\sqrt{3}-\ldots+\sqrt{2n-1}>\sqrt{\frac{n}{2}},n\geq 1.$

9. Wyznaczyč wszystkie wartości parametru rzeczywistego $p, \mathrm{d}\mathrm{l}\mathrm{a}$ których krzywe $0$ równaniach

$y=\sqrt[3]{x}, y=x^{p}$ przecinajq się $\mathrm{w}$ pewnym punkcie pod kqtem $45^{0}$ Rozwiqzanie zilustrowač

odpowiednim rysunkiem.





PRACA KONTROLNA nr 5

luty $2005\mathrm{r}.$

l. Firma otrzymała zlecenie na wyprodukowanie 80000 sztuk pewnego wyrobu $\mathrm{w}$ terminie 60

$\mathrm{d}\mathrm{n}\mathrm{i}. \mathrm{K}\mathrm{a}\dot{\mathrm{z}}\mathrm{d}\mathrm{y} \mathrm{z} 20$ pracowników firmy $\mathrm{m}\mathrm{o}\dot{\mathrm{z}}\mathrm{e}$ wykonač $\mathrm{w}$ ciągu dnia 50 sztuk tego wyrobu.

Reszta zamówienia $\mathrm{m}\mathrm{o}\dot{\mathrm{z}}\mathrm{e}$ byč zrealizowana przez dotychczasowq załogę, ale za dodatkową

pracę nalezy zapłacič podwójnie. $\mathrm{M}\mathrm{o}\dot{\mathrm{z}}$ na $\mathrm{t}\mathrm{e}\dot{\mathrm{z}}$ zatrudnič pewną liczbę nowych pracowników,

którzy otrzymają 80\% wynagrodzenia sta1ych pracowników. Nowy pracownik $\mathrm{m}\mathrm{o}\dot{\mathrm{z}}\mathrm{e}$ po 4

dniach szkolenia wykonač 26 sztuk wyrobów $\mathrm{w}$ pierwszym dniu $\mathrm{i}$ zwiększač wydajnośč $0$

l sztukę dziennie $\mathrm{a}\dot{\mathrm{z}}$ do osiągnięcia 50 sztuk. I1u nowych pracowników na1ezałoby zatrudnič

wybierajqc drugi wariant $\mathrm{i}$ który wariant jest korzystniejszy dla firmy?

2. Wyznaczyč wszystkie liczby rzeczywiste a $\mathrm{i}b$, których iloczyn oraz róznica kwadratów są

równe ich sumie.

3. Dane są zbiory na p{\it l}aszczy $\acute{\mathrm{z}}\mathrm{n}\mathrm{i}\mathrm{e}A=\{(x,y):(x+y)(y-2x)\leq 0\}$ oraz $B=$

$\{(x,y):y(3-x)\geq x\}$. Zaznaczyč na rysunku zbiór $C=A\cap B$. Podač wszystkie punkty

zbioru $C$, których obie wspólrzędne są liczbami naturalnymi.

4. $\mathrm{W}$ czworokącie wypukłym ABCD przekqtne $\vec{AC}= [7,-1] \mathrm{i} \vec{BD}= [3$, 3$]$ przecinajq się

$\mathrm{w}$ punkcie $O$ odległym $0\sqrt{8}$ od wierzchołków $C\mathrm{i}D$. Wyznaczyč wektory $\overline{AB}\succ \mathrm{i}\overline{B}C\succ$ oraz

narysowač ten czworokąt.

5. Wazon $\mathrm{w}$ ksztalcie graniastosłupa prawidłowego trójkątnego $0$ krawędzi podstawy 4 cm $\mathrm{i}$

wysokości 25 cm napełniono ca1kowicie wodą. Następnie wy1ano częśč wody przechy1ajqc

wazon $\mathrm{w}$ taki sposób, $\dot{\mathrm{z}}\mathrm{e}$ poziom wody na dwóch krawędziach bocznych znajdował się $\mathrm{w}$

odległości 4 cm $\mathrm{i}3$ cm od górnego brzegu wazonu. Jaka wysokośč będzie miał słup wody $\mathrm{w}$

wazonie po ustawieniu go $\mathrm{z}$ powrotem $\mathrm{w}$ pozycji pionowej?

6. Zbadač monotonicznośč ciągu $0$ wyrazie ogólnym

$a_{n}=\displaystyle \frac{2^{n}+2^{n+1}+\ldots+2^{2n+1}}{2+2^{3}+\ldots+2^{2n+1}}.$

7. Sporządzič wykres funkcji $f(x)=\sqrt{5x-x^{2}}-2$ nie przeprowadzając badania jej przebiegu $\mathrm{i}$

podač nazwę otrzymanej krzywej. Na podstawie wykresu określič liczbę rozwiązań równania

$|\sqrt{5x-x^{2}}-2|=p\mathrm{w}$ zalezności od parametru rzeczywistego $p.$

8. Wykazač, $\dot{\mathrm{z}}\mathrm{e}$ równanie kwadratowe $ 3x^{2}+4x\sin\alpha-\cos 2\alpha = 0$ ma dla $\mathrm{k}\mathrm{a}\dot{\mathrm{z}}$ dej wartości

parametru $\alpha$ dwa rózne pierwiastki rzeczywiste. Wyznaczyč wszystkie wartości parametru

$\alpha\in[0,2\pi]$, dla których suma odwrotności pierwiastków tego równania jest nieujemna.

9. Wyznaczyč asymptoty, przedziały monotoniczności oraz ekstrema lokalne funkcji

$f(x)=|x-2|+\displaystyle \frac{5x-4}{2x^{3}}.$





PRACA KONTROLNA nr 6

marzec $2005\mathrm{r}.$

l. Suma cyfr liczby trzycyfrowej wynosi 9. Cyfra setek jest równa 1/81iczby złozonej $\mathrm{z}$ dwu

pozostałych cyfr, a cyfra jednostek jest takze równa 1/81iczby złozonej $\mathrm{z}$ dwu pozostałych

cyfr. Co to za liczba?

2. Obliczyč $\mathrm{t}\mathrm{g}\beta$, gdzie $\beta\in[0,\pi]$, wiedzqc, $\dot{\mathrm{z}}\mathrm{e}\cos\beta=\sin\alpha+\cos\alpha$ oraz $\dot{\mathrm{z}}\mathrm{e}$

tg $\displaystyle \alpha=-\frac{3}{4}, \alpha\in[0,\pi]. \mathrm{W}$ której čwiartce $\mathrm{l}\mathrm{e}\dot{\mathrm{z}}\mathrm{y}$ kąt $\alpha+\beta?$Odpowied $\acute{\mathrm{z}}$ uzasadnič nie wykonujac

obliczeń przyblizonych.

3. Wyznaczyč równania wszystkich parabol przechodzących przez punkt $P(1,\sqrt{3})$, których

wierzchołek $\mathrm{i}$ punkty przecięcia $\mathrm{z}\mathrm{o}\mathrm{s}\mathrm{i}\otimes Ox$ tworzq trójkąt równoboczny $0$ polu $\sqrt{3}$. Sporządzič

rysunek.

4. Rzucamy trzy razy kostką do gry. Jakie jest prawdopodobieństwo, $\dot{\mathrm{z}}\mathrm{e}$ wyniki kolejnych

rzutów utworzą a) ciąg arytmetyczny; b) ciąg rosnący?

5. $\mathrm{Z}$ punktu $P \mathrm{l}\mathrm{e}\dot{\mathrm{z}}$ qcego $\mathrm{w}$ odległości $R$ od powierzchni kuli $0$ promieniu $R$ poprowadzono

trzy pólproste styczne do tej kuli tworzące kąt trójścienny $0$ jednakowych kątach plaskich.

Obliczyč cosinus kata plaskiego tego trójścianu.

6. Okrąg $0$ promieniu $r$ przecina $\mathrm{k}\mathrm{a}\dot{\mathrm{z}}$ de $\mathrm{z}$ ramion kąta ostrego $2\gamma \mathrm{w}$ dwóch punktach $\mathrm{w}$ taki

sposób, $\dot{\mathrm{z}}\mathrm{e}$ wyznaczajq one dwie cięciwy jednakowej długości, a czworokąt utworzony przez

te cztery punkty ma największe pole. Obliczyč odleglośč środka okręgu od wierzcholka kąta?

7. Rozwiązač nierównośč

$\displaystyle \log_{x}\frac{1-2x}{2-x}\geq 1.$

8. Wyznaczyč $\mathrm{i}$ narysowač zbiór wszystkich punktów płaszczyzny, których suma odległości od

osi $Ox\mathrm{i}$ od okręgu $x^{2}+(y-1)^{2}=1$ wynosi 2.

9. Dana jest funkcja $f(x) =\displaystyle \cos 2x+\frac{2}{3}\sin x. |\sin x|$. a) Korzystając $\mathrm{z}$ definicji uzasadnič, $\dot{\mathrm{z}}\mathrm{e}$

$f'(0) = 0$. b) Znalez/č wszystkie punkty $\mathrm{z}$ przedzialu $[-\pi,\pi], \mathrm{w}$ których styczna do wy-

kresu funkcji $f(x)$ jest równoległa do stycznej $\mathrm{w}$ punkcie $x = \displaystyle \frac{\pi}{4}$. Rozwiqzanie zilustrowač

odpowiednim rysunkiem.





PRACA KONTROLNA nr 7

kwiecień $2005\mathrm{r}.$

l. Liczba czteroelementowych podzbiorów zbioru $A$ jest ll razy większa od liczby jego pod-

zbiorów dwuelementowych, a zbiór $B\subset A$ ma tyle samo podzbiorów czteroelementowych co

dwuelementowych. Ile podzbiorów co najwyzej trzyelementowych ma zbiór $A\backslash B$?

2. Reszta $\mathrm{z}$ dzielenia wielomianu $x^{3}+px^{2}-x+q$ przez trójmian $(x+2)^{2}$ wynosi $(-x+1).$

Obliczyč pierwiastki tego wielomianu.

3. Kula $\mathcal{K}$ jest styczna do wszystkich krawędzi czworościanu foremnego $0$ objętości 64 $\mathrm{c}\mathrm{m}^{3}$

Czworościan ten przecięto płaszczyznq równoległą do jednej ze ścian $\mathrm{i}$ styczną do kuli $\mathcal{K}.$

Obliczyč objętośč otrzymanego ostrosłupa ściętego.

4. Znalez$\acute{}$č wszystkie wartości parametru $p$, dla których przedział [1, 2] jest zawarty $\mathrm{w}$ dziedzinie

funkcji

$f(x)=\displaystyle \frac{\sqrt{x^{2}-3px+2p^{2}}}{\sqrt{x+p}}.$

5. Ze zbioru liczb czterocyfrowych wylosowano (ze zwracaniem) 4 liczby. Obliczyč prawdopodo-

bieństwo tego, $\dot{\mathrm{z}}\mathrm{e}$ co najmniej dwie $\mathrm{z}$ wylosowanych liczb czytane od strony lewej do prawej

lub od strony prawej do lewej są podzielne przez 4.

6. Nalezy wykonač stolik $0$ symetrycznym owalnym blacie, jak pokazano na rysunku obok,
\begin{center}
\includegraphics[width=49.128mm,height=29.004mm]{./KursMatematyki_PolitechnikaWroclawska_2004_2005_page6_images/image001.eps}
\end{center}
$0$ długosci l $\mathrm{m}\mathrm{i}$ szerokosci 60 cm. Projektant przyj ł, $\dot{\mathrm{z}}\mathrm{e}$

brzeg blatu będzie się składał $\mathrm{z}$ czterech łuków okręgow,

$\mathrm{k}\mathrm{a}\dot{\mathrm{z}}\mathrm{d}\mathrm{y}0\mathrm{k}$ cie srodkowym $90^{0}$ Jakie powinny byc pro-

mienie tych łuków, aby brzeg blatu $\mathrm{b}\mathrm{y}l$ krzyw gładk?

Podac powierzchnię blatu $\mathrm{z}$ dokladnością do l $\mathrm{c}\mathrm{m}^{2}$

7. Styczna do okręgu $x^{2} + y^{2} 4x 2y 5 = 0 \mathrm{w}$ punkcie $A(-1,2)$, prosta

$3x+4y-10=0$ oraz oś $Ox$ tworzq trójkąt. Obliczyč jego pole $\mathrm{i}$ sporządzič rysunek.

8. Rozwiązač równanie

$\displaystyle \mathrm{c}\mathrm{t}\mathrm{g}^{2}x-\mathrm{c}\mathrm{t}\mathrm{g}^{4}x+\mathrm{c}\mathrm{t}\mathrm{g}^{6}x-\ldots=\frac{1+\cos 3x}{2},$

którego lewa strona jest sumą nieskończonego ciągu geometrycznego.

9. Na walcu obrotowym $0$ wysokości równej średnicy podstawy opisano ostroslup prawidlo-

wy trójkątny $0$ najmniejszej objętości $\mathrm{i}$ taki, $\dot{\mathrm{z}}\mathrm{e}$ jedna $\mathrm{z}$ podstaw walca $\mathrm{l}\mathrm{e}\dot{\mathrm{z}}\mathrm{y}$ na podstawie

ostrosłupa. Obliczyč tangens kąta nachylenia ściany bocznej tego ostroslupa do podstawy.



\end{document}