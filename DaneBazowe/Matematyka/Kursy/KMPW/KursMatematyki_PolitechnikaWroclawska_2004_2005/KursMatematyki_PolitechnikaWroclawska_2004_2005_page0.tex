\documentclass[a4paper,12pt]{article}
\usepackage{latexsym}
\usepackage{amsmath}
\usepackage{amssymb}
\usepackage{graphicx}
\usepackage{wrapfig}
\pagestyle{plain}
\usepackage{fancybox}
\usepackage{bm}

\begin{document}

XXXIV

KORESPONDENCYJNY KURS Z MATEMATYKI

PRACA KONTROLNA nr l

$\mathrm{p}\mathrm{a}\acute{\mathrm{z}}$dziernik 2$004\mathrm{r}.$

l. Staś kupił zeszyty 32-kartkowe po 80 gr za sztukę $\mathrm{i}$ zeszyty 60-kartkowe po 1,20 zł za sztukę

$\mathrm{i}$ zapłacił 13,20 $\mathrm{z}l$. Ile zeszytów 60-kartkowych kupił Staś, jeś1i by1o ich więcej $\mathrm{n}\mathrm{i}\dot{\mathrm{z}}$ zeszytów

32-kartkowych?

2. Rozwiązač nierównośč

--{\it xx}32$+$-{\it xx}$\leq$1.

3. Dana jest parabola $0$ równaniu $y = -x^{2}+2x+3$. Znalez/č równanie paraboli, która jest

symetryczna do danej względem punktu $S(2,1)$, oraz wyznaczyč punkty, $\mathrm{w}$ których przecina

ona osie ukladu wspólrzędnych. Sporządzič rysunek.

4. $\mathrm{W}$ trójkącie prostokątnym równoramiennym $ABC$ dany jest wierzchołek kata prostego $C(1,1),$

a bok $\overline{AB}\mathrm{l}\mathrm{e}\dot{\mathrm{z}}\mathrm{y}$ na prostej $x+5y+7=0$. Wyznaczyč współrzędne wierzcholków A $\mathrm{i}B.$

5. $\mathrm{W}$ ostroslupie prawidlowym sześciokątnym kąty płaskie ścian bocznych przy wierzchołku są

równe $\alpha$. Wyznaczyč cosinus kata między sąsiednimi ścianami bocznymi tego ostroslupa.

6. Dany jest trójkąt równoramienny $0$ kqcie przy podstawie $\alpha \mathrm{i}$ ramieniu $b$. Ramiona tego

trójkąta przecięto prostą odcinając $\mathrm{z}$ niego deltoid. Wyznaczyč katy pozostałego mniejszego

trójkąta oraz jego pole. Kiedy zadanie ma rozwiazanie?

7. Rozwiązač nierównośč

$\sqrt{2^{x-2}+3}\leq 2^{x}-2.$

8. Wyznaczyč dziedzinę oraz narysowač wykres funkcji $s(x)$ danej wzorem

$s(x)=\log_{2}(1-x+x^{2}-x^{3}+\ldots).$

Przy pomocy wykresu określič zbiór wartości tej funkcji.

9. Rozwiązač równanie

tg 3{\it x}$=$ -csoins 42{\it xx}.
\end{document}
