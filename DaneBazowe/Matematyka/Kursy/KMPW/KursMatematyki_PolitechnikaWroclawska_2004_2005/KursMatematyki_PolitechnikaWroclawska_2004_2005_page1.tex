\documentclass[a4paper,12pt]{article}
\usepackage{latexsym}
\usepackage{amsmath}
\usepackage{amssymb}
\usepackage{graphicx}
\usepackage{wrapfig}
\pagestyle{plain}
\usepackage{fancybox}
\usepackage{bm}

\begin{document}

PRACA KONTROLNA nr 2

listopad 2004r.

l. Liczby $0$ 45\% mniejsza $\mathrm{i} 0$ 32\% większa od ułamka okresowego 0,(60) sq pierwiastkami

trójmianu kwadratowego $0$ współczynnikach całkowitych względnie pierwszych. Obliczyč

resztę $\mathrm{z}$ dzielenia tego trójmianu przez dwumian $(x-1).$

2. Wykres funkcji $f$ : $[0,5]\rightarrow R$ jest przed-

stawiony na rysunku obok. Narysowac

wykres funkcji $g(x)=f(x)-f(5-x)$

$\mathrm{i}$ zapisač $\mathrm{j}$ wzorem.
\begin{center}
\includegraphics[width=58.524mm,height=35.616mm]{./KursMatematyki_PolitechnikaWroclawska_2004_2005_page1_images/image001.eps}
\end{center}
$y$

2

1

0 1 3 5 $x$

$-1$

3. Obliczyc wartosci $\sin\alpha \mathrm{i}\cos\alpha$, jeśli wiadomo, $\dot{\mathrm{z}}\mathrm{e}$

$\displaystyle \sin\alpha+3\cos\alpha=\frac{1}{\cos\alpha},$

$\displaystyle \alpha\in[0,\pi]\backslash \{\frac{\pi}{2}\}.$

4. Suma 20 pierwszych wyrazów pewnego ciągu arytmetycznego jest równa zeru, a iloczyn dzie-

siqtego $\mathrm{i}$ jedenastego wyrazu wynosi $-1$. Dla jakich liczb naturalnych $n$ suma $n$ pierwszych

wyrazów tego ciągu przekracza 77?

5. Trapez równoramienny jest wpisany $\mathrm{w}$ okrąg $0$ promieniu $R$, a jednq $\mathrm{z}$ jego podstaw jest

średnica tego okręgu. $\mathrm{W}$ trapez ten daje się wpisač okrąg. Wyznaczyč jego promień.

6. Środek kuli opisanej na ostrosłupie prawidłowym trójkatnym $\mathrm{l}\mathrm{e}\dot{\mathrm{z}}\mathrm{y}\mathrm{w}$ odległości $d$ ponad

podstawą ostrosłupa, a kąt nachylenia krawędzi bocznej do podstawy wynosi $\alpha$. Obliczyč

objętośč ostrosłupa.

7. Wyznaczyč wszystkie wartości parametru rzeczywistego $m$, dla których funkcja

$f(x)=\displaystyle \frac{x+1}{x^{2}+mx+4}$

jest dodatnia i rosnąca na odcinku (0,1).

8. Nie korzystając z rachunku rózniczkowego wyznaczyč dziedzinę i zbiór wartości funkcji

$f(x)=\sqrt{\sqrt{2}-\cos x-\sqrt{3}\sin x},$

$x\in[0,\pi].$

9. Rozwiązač uklad równań

$\left\{\begin{array}{l}
|x+1|y\\
x^{2}-4|x|+2y-1
\end{array}\right.$

$=4$

$=0$

Przedstawič ilustrację graficzną obu równań i zaznaczyč na rysunku znalezione rozwiązania.
\end{document}
