\documentclass[a4paper,12pt]{article}
\usepackage{latexsym}
\usepackage{amsmath}
\usepackage{amssymb}
\usepackage{graphicx}
\usepackage{wrapfig}
\pagestyle{plain}
\usepackage{fancybox}
\usepackage{bm}

\begin{document}

PRACA KONTROLNA nr 3

grudzień 2004r.

l. W pewnej szkole zapytano uczniów klas maturalnych ile razy w ostatnim miesiqcu ucze-

liczba uczniow stniczyli w imprezie kulturalnej. Wyniki przed-

10 uczniów jest w klasach maturalnych tej szkoly; b)
\begin{center}
\includegraphics[width=78.588mm,height=35.304mm]{./KursMatematyki_PolitechnikaWroclawska_2004_2005_page2_images/image001.eps}
\end{center}
15

5

stawiono na diagramie obok. Obliczyc: a) Ilu

Ile razy średnio w miesi cu uczeń był na imprezie

kulturalnej. Sporz dzic diagram kołowy przedsta-

0 1 2 3 4 5 6 7 wiaj cy procentowo otrzymane wyniki.

2. Turysta zauwazyl, $\dot{\mathrm{z}}\mathrm{e}\mathrm{w}$ pewnym miejscu na odcinku 10 $\mathrm{m}$ potok górski płynie $\mathrm{w}$ korycie

skalnym, które $\mathrm{w}$ przekroju pionowym tworzy trapez $0$ dolnej podstawie 2 $\mathrm{m}\mathrm{i}$ górnej 3 $\mathrm{m}.$

Wysokośč koryta wynosi 50 cm, przy czym woda wypełnia koryto jedynie na głębokośč 10

cm. Turysta ustalil równiez, $\dot{\mathrm{z}}\mathrm{e}$ czas przepływu wody przez koryto wynosi 3 sekundy. I1e

litrów wody przeplywa przez ten potok $\mathrm{w}$ ciągu jednej sekundy?

3. Wykazač, $\dot{\mathrm{z}}\mathrm{e}$ dla dowolnych liczb dodatnich $a, b$ prawdziwa jest nierównośč

$(a+b)^{3}\leq 4(a^{3}+b^{3}).$

Wsk. Podzielič obie strony przez $b^{3}\mathrm{i}$ wprowadzič jednq zmiennq.

4. Boki $\overline{AB}\mathrm{i}\overline{AD}$ równoległoboku $\mathrm{l}\mathrm{e}\dot{\mathrm{z}}$ ą odpowiednio na prostych $3x+4y-7=0\mathrm{i}x-2y+1=0.$

Wyznaczyč współrzędne wierzcholka $C$ tego równolegloboku wiedząc, $\dot{\mathrm{z}}\mathrm{e}$ jego wysokośč do

boku $\overline{AB}$ wynosi 2, a wierzcho1ek $B$ ma współrzędne $(5,-2).$

5. $\mathrm{W}$ trójkącie ostrokątnym $ABC$ dane sq bok $BC=\displaystyle \frac{5}{2}\sqrt{5}$ cm oraz wysokości $BD=\displaystyle \frac{11}{2}$ cm $\mathrm{i}$

$CE=5$ cm. Obliczyč obwód tego trójkąta oraz cosinus kata $\angle BAC.$

6. Spośród dwudziestu najmniejszych, nieparzystych liczb naturalnych wylosowano (bez zwra-

cania) dwie. Obliczyč prawdopodobieństwo, $\dot{\mathrm{z}}\mathrm{e}$ otrzymano: a) dwie liczby pierwsze; b) dwie

liczby względnie pierwsze.

7. Rozwiązač nierównośč $\log_{2} x^{\log_{4}x}\geq\log_{x}16.$

8. Niech $f(m)$ oznacza sumę trzecich potęg pierwiastków rzeczywistych równania kwadratowego

$x^{2}+(m+3)x+m^{2}=0\mathrm{z}$ parametrem $m$. Wyznaczyč wzór funkcji $f(m)$ oraz najmniejszq $\mathrm{i}$

największq wartośč tej funkcji.

9. $\mathrm{W}$ ostrosłupie prawidłowym czworokątnym $\mathrm{k}\mathrm{a}\mathrm{t}$ nachylenia krawędzi bocznej do podstawy

wynosi $\alpha$, a odległosc krawędzi podstawy od przeciwległej sciany

bocznej jest równa $d=3$ cm. Obliczyc wysokosc sciany bocznej.

Czy siatka tego ostrosłupa, jak na rysunku obok, zmiesci $\mathrm{s}\mathrm{i}_{9}$

na arkuszu papieru $\mathrm{w}$ ksztalcie kwadratu $0$ boku 16 cm, jes1i

wiadomo, $\dot{\mathrm{z}}\mathrm{e}\mathrm{t}\mathrm{g}\alpha=2$? Sporz dzic rysunek.
\begin{center}
\includegraphics[width=30.324mm,height=30.384mm]{./KursMatematyki_PolitechnikaWroclawska_2004_2005_page2_images/image002.eps}
\end{center}\end{document}
