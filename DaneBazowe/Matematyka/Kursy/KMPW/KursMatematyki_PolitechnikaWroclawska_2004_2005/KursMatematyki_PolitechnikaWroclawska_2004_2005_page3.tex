\documentclass[a4paper,12pt]{article}
\usepackage{latexsym}
\usepackage{amsmath}
\usepackage{amssymb}
\usepackage{graphicx}
\usepackage{wrapfig}
\pagestyle{plain}
\usepackage{fancybox}
\usepackage{bm}

\begin{document}

PRACA KONTROLNA nr 4

styczeń $2005\mathrm{r}.$

l. Krawędzie oraz przekątna prostopadłościanu tworzq cztery kolejne wyrazy ciągu arytmetycz-

nego, przy czym przekątna ma długośč 7 cm. Jaką najkrótszq drogę musi przebyč mucha,

aby wędrujqc po krawędziach tego prostopadłościanu odwiedzila wszystkie jego wierzchołki.

2. Dany jest wielomian $w(x)=x^{4}-2x^{2}-x+2$. Rozłozyc na czynniki $\mathrm{m}\mathrm{o}\dot{\mathrm{z}}$ liwie najnizszego

stopnia wielomian $p(x)=w(x+1)-w(x).$
\begin{center}
\includegraphics[width=60.504mm,height=64.824mm]{./KursMatematyki_PolitechnikaWroclawska_2004_2005_page3_images/image001.eps}
\end{center}
{\it y}

2

0 2 $x$

3. Na rysunku obok przedstawiono fragment mapy $\mathrm{w}$ ska-

li 1:25000, który zawiera obszar 1asu $L$ ograniczony

czterema drogami. Na mapę jest naniesiona siatka ki-

lometrowa, a dodatkowo umieszczono na niej układ

współrzędnych pokrywaj cy się $\mathrm{z}$ wybranymi liniami

siatki. Zapisac obszar $L\mathrm{w}$ postaci układu nierownosci

liniowych ($\mathrm{w}$ skali mapy). Obliczyc rzeczywiste pole

obszaru $L$ wyrazaj $\mathrm{c}$ go $\mathrm{w}$ hektarach.

4. Na ile sposobów $\mathrm{m}\mathrm{o}\dot{\mathrm{z}}\mathrm{e}$ Krzyś rozdzielič 12 jednakowych cukierków pomiędzy siebie $\mathrm{i}$ trójkę

rodzeństwa, jeśli $\mathrm{k}\mathrm{a}\dot{\mathrm{z}}\mathrm{d}\mathrm{y}$ ma otrzymač co najmniej dwa cukierki.

5. $\mathrm{W}$ stozek wpisano sześcian $0$ krawędzi $a$. Rozwinięcie powierzchni bocznej stozka tworzy

wycinek koła $0$ kącie środkowym 1200. Ob1iczyč tangens kąta podjakim tworzącą tego stozka

widač ze środka sześcianu.

6. $\mathrm{W}$ trójkącie $ABC$ dane $\mathrm{S}\otimes$katy $\alpha \mathrm{i}\beta$ przy podstawie $\overline{AB}$ oraz środkowa $CD=s$ podstawy.

Obliczyč pole tego trójkąta.

7. Rozwiązač równanie $3^{\sin x}+9^{\sin x}+27^{\sin x}+\ldots= \displaystyle \frac{\sqrt{3}+1}{2},$

nieskończonego ciągu geometrycznego.

którego lewa strona jest sumą

8. Stosując zasadę indukcji matematycznej udowodnič nierównośč:

$1-\sqrt{2}+\sqrt{3}-\ldots+\sqrt{2n-1}>\sqrt{\frac{n}{2}},n\geq 1.$

9. Wyznaczyč wszystkie wartości parametru rzeczywistego $p, \mathrm{d}\mathrm{l}\mathrm{a}$ których krzywe $0$ równaniach

$y=\sqrt[3]{x}, y=x^{p}$ przecinajq się $\mathrm{w}$ pewnym punkcie pod kqtem $45^{0}$ Rozwiqzanie zilustrowač

odpowiednim rysunkiem.
\end{document}
