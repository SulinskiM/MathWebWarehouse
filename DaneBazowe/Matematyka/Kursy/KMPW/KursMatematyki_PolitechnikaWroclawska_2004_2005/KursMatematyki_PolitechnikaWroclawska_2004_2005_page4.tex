\documentclass[a4paper,12pt]{article}
\usepackage{latexsym}
\usepackage{amsmath}
\usepackage{amssymb}
\usepackage{graphicx}
\usepackage{wrapfig}
\pagestyle{plain}
\usepackage{fancybox}
\usepackage{bm}

\begin{document}

PRACA KONTROLNA nr 5

luty $2005\mathrm{r}.$

l. Firma otrzymała zlecenie na wyprodukowanie 80000 sztuk pewnego wyrobu $\mathrm{w}$ terminie 60

$\mathrm{d}\mathrm{n}\mathrm{i}. \mathrm{K}\mathrm{a}\dot{\mathrm{z}}\mathrm{d}\mathrm{y} \mathrm{z} 20$ pracowników firmy $\mathrm{m}\mathrm{o}\dot{\mathrm{z}}\mathrm{e}$ wykonač $\mathrm{w}$ ciągu dnia 50 sztuk tego wyrobu.

Reszta zamówienia $\mathrm{m}\mathrm{o}\dot{\mathrm{z}}\mathrm{e}$ byč zrealizowana przez dotychczasowq załogę, ale za dodatkową

pracę nalezy zapłacič podwójnie. $\mathrm{M}\mathrm{o}\dot{\mathrm{z}}$ na $\mathrm{t}\mathrm{e}\dot{\mathrm{z}}$ zatrudnič pewną liczbę nowych pracowników,

którzy otrzymają 80\% wynagrodzenia sta1ych pracowników. Nowy pracownik $\mathrm{m}\mathrm{o}\dot{\mathrm{z}}\mathrm{e}$ po 4

dniach szkolenia wykonač 26 sztuk wyrobów $\mathrm{w}$ pierwszym dniu $\mathrm{i}$ zwiększač wydajnośč $0$

l sztukę dziennie $\mathrm{a}\dot{\mathrm{z}}$ do osiągnięcia 50 sztuk. I1u nowych pracowników na1ezałoby zatrudnič

wybierajqc drugi wariant $\mathrm{i}$ który wariant jest korzystniejszy dla firmy?

2. Wyznaczyč wszystkie liczby rzeczywiste a $\mathrm{i}b$, których iloczyn oraz róznica kwadratów są

równe ich sumie.

3. Dane są zbiory na p{\it l}aszczy $\acute{\mathrm{z}}\mathrm{n}\mathrm{i}\mathrm{e}A=\{(x,y):(x+y)(y-2x)\leq 0\}$ oraz $B=$

$\{(x,y):y(3-x)\geq x\}$. Zaznaczyč na rysunku zbiór $C=A\cap B$. Podač wszystkie punkty

zbioru $C$, których obie wspólrzędne są liczbami naturalnymi.

4. $\mathrm{W}$ czworokącie wypukłym ABCD przekqtne $\vec{AC}= [7,-1] \mathrm{i} \vec{BD}= [3$, 3$]$ przecinajq się

$\mathrm{w}$ punkcie $O$ odległym $0\sqrt{8}$ od wierzchołków $C\mathrm{i}D$. Wyznaczyč wektory $\overline{AB}\succ \mathrm{i}\overline{B}C\succ$ oraz

narysowač ten czworokąt.

5. Wazon $\mathrm{w}$ ksztalcie graniastosłupa prawidłowego trójkątnego $0$ krawędzi podstawy 4 cm $\mathrm{i}$

wysokości 25 cm napełniono ca1kowicie wodą. Następnie wy1ano częśč wody przechy1ajqc

wazon $\mathrm{w}$ taki sposób, $\dot{\mathrm{z}}\mathrm{e}$ poziom wody na dwóch krawędziach bocznych znajdował się $\mathrm{w}$

odległości 4 cm $\mathrm{i}3$ cm od górnego brzegu wazonu. Jaka wysokośč będzie miał słup wody $\mathrm{w}$

wazonie po ustawieniu go $\mathrm{z}$ powrotem $\mathrm{w}$ pozycji pionowej?

6. Zbadač monotonicznośč ciągu $0$ wyrazie ogólnym

$a_{n}=\displaystyle \frac{2^{n}+2^{n+1}+\ldots+2^{2n+1}}{2+2^{3}+\ldots+2^{2n+1}}.$

7. Sporządzič wykres funkcji $f(x)=\sqrt{5x-x^{2}}-2$ nie przeprowadzając badania jej przebiegu $\mathrm{i}$

podač nazwę otrzymanej krzywej. Na podstawie wykresu określič liczbę rozwiązań równania

$|\sqrt{5x-x^{2}}-2|=p\mathrm{w}$ zalezności od parametru rzeczywistego $p.$

8. Wykazač, $\dot{\mathrm{z}}\mathrm{e}$ równanie kwadratowe $ 3x^{2}+4x\sin\alpha-\cos 2\alpha = 0$ ma dla $\mathrm{k}\mathrm{a}\dot{\mathrm{z}}$ dej wartości

parametru $\alpha$ dwa rózne pierwiastki rzeczywiste. Wyznaczyč wszystkie wartości parametru

$\alpha\in[0,2\pi]$, dla których suma odwrotności pierwiastków tego równania jest nieujemna.

9. Wyznaczyč asymptoty, przedziały monotoniczności oraz ekstrema lokalne funkcji

$f(x)=|x-2|+\displaystyle \frac{5x-4}{2x^{3}}.$
\end{document}
