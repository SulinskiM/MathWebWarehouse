\documentclass[a4paper,12pt]{article}
\usepackage{latexsym}
\usepackage{amsmath}
\usepackage{amssymb}
\usepackage{graphicx}
\usepackage{wrapfig}
\pagestyle{plain}
\usepackage{fancybox}
\usepackage{bm}

\begin{document}

PRACA KONTROLNA nr 6

marzec $2005\mathrm{r}.$

l. Suma cyfr liczby trzycyfrowej wynosi 9. Cyfra setek jest równa 1/81iczby złozonej $\mathrm{z}$ dwu

pozostałych cyfr, a cyfra jednostek jest takze równa 1/81iczby złozonej $\mathrm{z}$ dwu pozostałych

cyfr. Co to za liczba?

2. Obliczyč $\mathrm{t}\mathrm{g}\beta$, gdzie $\beta\in[0,\pi]$, wiedzqc, $\dot{\mathrm{z}}\mathrm{e}\cos\beta=\sin\alpha+\cos\alpha$ oraz $\dot{\mathrm{z}}\mathrm{e}$

tg $\displaystyle \alpha=-\frac{3}{4}, \alpha\in[0,\pi]. \mathrm{W}$ której čwiartce $\mathrm{l}\mathrm{e}\dot{\mathrm{z}}\mathrm{y}$ kąt $\alpha+\beta?$Odpowied $\acute{\mathrm{z}}$ uzasadnič nie wykonujac

obliczeń przyblizonych.

3. Wyznaczyč równania wszystkich parabol przechodzących przez punkt $P(1,\sqrt{3})$, których

wierzchołek $\mathrm{i}$ punkty przecięcia $\mathrm{z}\mathrm{o}\mathrm{s}\mathrm{i}\otimes Ox$ tworzq trójkąt równoboczny $0$ polu $\sqrt{3}$. Sporządzič

rysunek.

4. Rzucamy trzy razy kostką do gry. Jakie jest prawdopodobieństwo, $\dot{\mathrm{z}}\mathrm{e}$ wyniki kolejnych

rzutów utworzą a) ciąg arytmetyczny; b) ciąg rosnący?

5. $\mathrm{Z}$ punktu $P \mathrm{l}\mathrm{e}\dot{\mathrm{z}}$ qcego $\mathrm{w}$ odległości $R$ od powierzchni kuli $0$ promieniu $R$ poprowadzono

trzy pólproste styczne do tej kuli tworzące kąt trójścienny $0$ jednakowych kątach plaskich.

Obliczyč cosinus kata plaskiego tego trójścianu.

6. Okrąg $0$ promieniu $r$ przecina $\mathrm{k}\mathrm{a}\dot{\mathrm{z}}$ de $\mathrm{z}$ ramion kąta ostrego $2\gamma \mathrm{w}$ dwóch punktach $\mathrm{w}$ taki

sposób, $\dot{\mathrm{z}}\mathrm{e}$ wyznaczajq one dwie cięciwy jednakowej długości, a czworokąt utworzony przez

te cztery punkty ma największe pole. Obliczyč odleglośč środka okręgu od wierzcholka kąta?

7. Rozwiązač nierównośč

$\displaystyle \log_{x}\frac{1-2x}{2-x}\geq 1.$

8. Wyznaczyč $\mathrm{i}$ narysowač zbiór wszystkich punktów płaszczyzny, których suma odległości od

osi $Ox\mathrm{i}$ od okręgu $x^{2}+(y-1)^{2}=1$ wynosi 2.

9. Dana jest funkcja $f(x) =\displaystyle \cos 2x+\frac{2}{3}\sin x. |\sin x|$. a) Korzystając $\mathrm{z}$ definicji uzasadnič, $\dot{\mathrm{z}}\mathrm{e}$

$f'(0) = 0$. b) Znalez/č wszystkie punkty $\mathrm{z}$ przedzialu $[-\pi,\pi], \mathrm{w}$ których styczna do wy-

kresu funkcji $f(x)$ jest równoległa do stycznej $\mathrm{w}$ punkcie $x = \displaystyle \frac{\pi}{4}$. Rozwiqzanie zilustrowač

odpowiednim rysunkiem.
\end{document}
