\documentclass[a4paper,12pt]{article}
\usepackage{latexsym}
\usepackage{amsmath}
\usepackage{amssymb}
\usepackage{graphicx}
\usepackage{wrapfig}
\pagestyle{plain}
\usepackage{fancybox}
\usepackage{bm}

\begin{document}

PRACA KONTROLNA nr 7

kwiecień $2005\mathrm{r}.$

l. Liczba czteroelementowych podzbiorów zbioru $A$ jest ll razy większa od liczby jego pod-

zbiorów dwuelementowych, a zbiór $B\subset A$ ma tyle samo podzbiorów czteroelementowych co

dwuelementowych. Ile podzbiorów co najwyzej trzyelementowych ma zbiór $A\backslash B$?

2. Reszta $\mathrm{z}$ dzielenia wielomianu $x^{3}+px^{2}-x+q$ przez trójmian $(x+2)^{2}$ wynosi $(-x+1).$

Obliczyč pierwiastki tego wielomianu.

3. Kula $\mathcal{K}$ jest styczna do wszystkich krawędzi czworościanu foremnego $0$ objętości 64 $\mathrm{c}\mathrm{m}^{3}$

Czworościan ten przecięto płaszczyznq równoległą do jednej ze ścian $\mathrm{i}$ styczną do kuli $\mathcal{K}.$

Obliczyč objętośč otrzymanego ostrosłupa ściętego.

4. Znalez$\acute{}$č wszystkie wartości parametru $p$, dla których przedział [1, 2] jest zawarty $\mathrm{w}$ dziedzinie

funkcji

$f(x)=\displaystyle \frac{\sqrt{x^{2}-3px+2p^{2}}}{\sqrt{x+p}}.$

5. Ze zbioru liczb czterocyfrowych wylosowano (ze zwracaniem) 4 liczby. Obliczyč prawdopodo-

bieństwo tego, $\dot{\mathrm{z}}\mathrm{e}$ co najmniej dwie $\mathrm{z}$ wylosowanych liczb czytane od strony lewej do prawej

lub od strony prawej do lewej są podzielne przez 4.

6. Nalezy wykonač stolik $0$ symetrycznym owalnym blacie, jak pokazano na rysunku obok,
\begin{center}
\includegraphics[width=49.128mm,height=29.004mm]{./KursMatematyki_PolitechnikaWroclawska_2004_2005_page6_images/image001.eps}
\end{center}
$0$ długosci l $\mathrm{m}\mathrm{i}$ szerokosci 60 cm. Projektant przyj ł, $\dot{\mathrm{z}}\mathrm{e}$

brzeg blatu będzie się składał $\mathrm{z}$ czterech łuków okręgow,

$\mathrm{k}\mathrm{a}\dot{\mathrm{z}}\mathrm{d}\mathrm{y}0\mathrm{k}$ cie srodkowym $90^{0}$ Jakie powinny byc pro-

mienie tych łuków, aby brzeg blatu $\mathrm{b}\mathrm{y}l$ krzyw gładk?

Podac powierzchnię blatu $\mathrm{z}$ dokladnością do l $\mathrm{c}\mathrm{m}^{2}$

7. Styczna do okręgu $x^{2} + y^{2} 4x 2y 5 = 0 \mathrm{w}$ punkcie $A(-1,2)$, prosta

$3x+4y-10=0$ oraz oś $Ox$ tworzq trójkąt. Obliczyč jego pole $\mathrm{i}$ sporządzič rysunek.

8. Rozwiązač równanie

$\displaystyle \mathrm{c}\mathrm{t}\mathrm{g}^{2}x-\mathrm{c}\mathrm{t}\mathrm{g}^{4}x+\mathrm{c}\mathrm{t}\mathrm{g}^{6}x-\ldots=\frac{1+\cos 3x}{2},$

którego lewa strona jest sumą nieskończonego ciągu geometrycznego.

9. Na walcu obrotowym $0$ wysokości równej średnicy podstawy opisano ostroslup prawidlo-

wy trójkątny $0$ najmniejszej objętości $\mathrm{i}$ taki, $\dot{\mathrm{z}}\mathrm{e}$ jedna $\mathrm{z}$ podstaw walca $\mathrm{l}\mathrm{e}\dot{\mathrm{z}}\mathrm{y}$ na podstawie

ostrosłupa. Obliczyč tangens kąta nachylenia ściany bocznej tego ostroslupa do podstawy.
\end{document}
