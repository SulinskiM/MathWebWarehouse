\documentclass[a4paper,12pt]{article}
\usepackage{latexsym}
\usepackage{amsmath}
\usepackage{amssymb}
\usepackage{graphicx}
\usepackage{wrapfig}
\pagestyle{plain}
\usepackage{fancybox}
\usepackage{bm}

\begin{document}

xxxv

KORESPONDENCYJNY KURS Z MATEMATYKI

PRACA KONTROLNA nr l

$\mathrm{p}\mathrm{a}\acute{\mathrm{z}}$dziernik 2$005\mathrm{r}.$

l. Niech $f(x) = x^{2}+bx+5$. Wyznaczyč wszystkie wartości parametru $b$, dla których:

a) wykres funkcji $f$ jest symetryczny względem prostej $x=2$, b) wierzchofek paraboli

będqcej wykresem funkcji $f\mathrm{l}\mathrm{e}\dot{\mathrm{z}}\mathrm{y}$ na prostej $x+y+1=0$. Sporządzič staranny rysunek.

2. Kilkoro dzieci dostafo torebkę cukierków do równego podziału. Gdyby liczba dzieci byfa

$01$ mniejsza, to $\mathrm{k}\mathrm{a}\dot{\mathrm{z}}$ de $\mathrm{z}$ nich dostafoby $02$ cukierki więcej. Gdyby cukierków byfo dwa

razy więcej, a dzieci $0$ dwoje więcej, to $\mathrm{k}\mathrm{a}\dot{\mathrm{z}}$ de dostałoby $05$ cukierków wiecej. Ile było

dzieci a ile cukierków?

3. Babcia zalozyła swemu rocznemu wnukowi lokatę $\mathrm{w}$ wysokości 1000 $\mathrm{z}l$ oprocentowanq $\mathrm{w}$

wysokości 6\% $\mathrm{w}$ skali roku $\mathrm{z}$ półroczną kapitalizacją odsetek $\mathrm{i}$ postanowiła co 6 miesięcy

wplacač na to konto 100 $\mathrm{z}\mathrm{f}$. Jaka sumę dostanie wnuczek $\mathrm{w}$ dniu swoich osiemnastych

urodzin?

4. Dane są wierzcholki $A(-3,2), C(4,2), D(0,4)$ trapezu równoramiennego ABCD, $\mathrm{w}$ któ-

rym $\overline{AB}||\overline{CD}$. Wyznaczyč współrzędne wierzchołka $B$ oraz pole trapezu. Sporządzič

rysunek.

5. Wyznaczyč stosunek dlugości przekątnych rombu wiedząc, $\dot{\mathrm{z}}\mathrm{e}$ stosunek pola kofa wpisa-

nego $\mathrm{w}$ ten romb do pola rombu wynosi $\displaystyle \frac{\pi}{5}.$

6. Podstawą prostopadłościanu jest prostokąt $0$ dluzszym boku $a$. Przekątna prostopadlo-

ścianu tworzy $\mathrm{z}$ przekątnymi ścian bocznych kąty $\alpha$ oraz $ 2\alpha$. Obliczyč objetośč tego

prostopadfościanu. Dla jakich kątów $\alpha$ zadanie ma rozwiązanie?

7. Dla jakich wartości parametru $p$ funkcja

$f(x)=\displaystyle \frac{x^{3}}{px^{2}+px+1}$

jest określona $\mathrm{i}$ rosnąca na calej prostej rzeczywistej?

8. Rozwiązač równanie

ctg $x=2\sqrt{3}\sin x.$

9. Liczby $a_{1} = (\sqrt{2})^{\log_{\frac{1}{2}}16}$ oraz $a_{2} = 16^{-\log_{\sqrt[3]{2}}\sqrt[4]{2}}$ są odpowiednio pierwszym $\mathrm{i}$ drugim

wyrazem pewnego ciqgu geometrycznego. Rozwiązač nierównośč

$(\sqrt{x})^{\log^{2}x-1}\geq 2S,$

gdzie S oznacza sumę wszystkich wyrazów tego ciągu.




PRACA KONTROLNA nr 2

listopad $2005\mathrm{r}.$

l. Stop zawiera 60\% srebra próby 0,6 $\mathrm{i}$ 30\% srebra próby 0,7 oraz 20 dkg srebra próby 0,8.

a) Ile srebra $\mathrm{i}$ jakiej próby nalez $\mathrm{y}$ dodač, by otrzymač 2,5 kg srebra próby 0,7?

b) Obliczyč próbę stopu, jakim nalezy zastapič połowę danego stopu, by otrzymač stop

$0$ próbie 0,75?

2. Wyznaczyč wszystkie punkty okręgu $0$ środku $(0,0)\mathrm{i}$ promieniu 5, których i1oczyn kwa-

dratów wspólrzędnych jest najmniejszą wspólną wielokrotnościa liczb 12 $\mathrm{i} 14$. Obliczyč

obwód wielokąta, którego wierzchofkami $\mathrm{s}\Phi$ znalezione punkty. Bez $\mathrm{u}\dot{\mathrm{z}}$ ywania kalkulatora

zbadač, czy jest on większy od 30.

3. Dla jakich wartości $a \mathrm{i} b$ wielomian $W(x) = x^{4}-3x^{3}+bx^{2}+ax+b$ jest podzielny

przez trójmian kwadratowy $(x^{2}-1)$ ? Dla znalezionych wartości wspófczynników $a\mathrm{i}b$

rozwiązač nierównośč $W(x)\leq 0.$

4. Wykorzystuj $\Phi^{\mathrm{C}}\mathrm{t}\mathrm{o}\dot{\mathrm{z}}$ samośč trygonometryczną $\displaystyle \sin\alpha+\sin\beta=2\sin\frac{\alpha+\beta}{2}\cos\frac{\alpha-\beta}{2}$ narysowač

staranny wykres funkcji $f(x)=|\sin x+\cos x|$. Korzystając $\mathrm{z}$ tego wykresu, wyznaczyč

najmniejszą $\mathrm{i}$ największa wartośč funkcji $f$ na przedziale $[-\displaystyle \frac{\pi}{2},\pi]$. Wyznaczyč rozwiązania

równania $f(x)=\displaystyle \frac{1}{\sqrt{2}}$ zawarte $\mathrm{w}$ tym przedziale.

5. Pole powierzchni całkowitej stozka jest dwa razy większe od pola powierzchni kuli wpi-

sanej $\mathrm{w}$ ten stozek. Znalez/č cosinus kąta nachylenia $\mathrm{t}\mathrm{w}\mathrm{o}\mathrm{r}\mathrm{z}\Phi^{\mathrm{C}\mathrm{e}\mathrm{j}}$ stozka do podstawy.

6. $\mathrm{W}$ trójkącie równoramiennym suma długości ramienia $\mathrm{i}$ promienia okręgu opisanego

na tym trójkącie równa jest $m$ a wysokośč trójkąta równa jest 2. Wyznaczyč długośč

ramienia jako funkcję parametru $m$ oraz wartośč $m$, dla której kąt przy wierzchofku

trójkąta równy jest $120^{\mathrm{o}}$? Dla jakich wartości $m$ zadanie ma rozwiązanie?

7. Narysowač zbiory $A=\{(x,y):x^{2}+2x+y^{2}\leq 0\}, B= \{(x,y):x^{2}+2y+y^{2}\leq 0\},$

$C=\{(x,y):x\leq 0,y\geq 0,x^{2}+y^{2}\leq 4\}$. Obliczyč pola figur $A\cap B, A\backslash B, C\backslash (A\cup B).$

Podač równania osi symetrii figury $A\cup B.$

8. Rozwiązač nierównośč $\displaystyle \frac{1}{\sqrt{4-x^{2}}}\leq\frac{1}{x-1}.$

9 $\mathrm{W}\mathrm{z}$naczyč równania wszystkich pstych stycznychktóre s$\text{ą} \mathrm{p}\mathrm{o}\mathrm{s}\mathrm{o}$padle d$\mathrm{o}\mathrm{p}$rostej orównaniu {\it x}$+y=0.$Obliczyčdo wykres p$\mathrm{o}1\mathrm{e}\mathrm{r}\text{ó} \mathrm{w}\mathrm{n}\mathrm{o}1\mathrm{e}\mathrm{g}1$obfunkc {\it f}$(x)=\displaystyle \frac{8x}{x^{2}+3,\mathrm{o}\mathrm{k}\mathrm{u}},$

którego wierzchołkami są punkty wspólne tych stycznych $\mathrm{z}$ wykresem funkcji $f(x).$





PRACA KONTROLNA nr 3

grudzień $2005\mathrm{r}$

l. Droge $\mathrm{z}$ miasta $A$ do miasta $B$ rowerzysta pokonuje $\mathrm{w}$ ciqgu 3 godzin. Po długotrwałych

deszczach stan $\displaystyle \frac{3}{5}$ drogi pogorszyf się na tyle, $\dot{\mathrm{z}}\mathrm{e}$ na tym odcinku rowerzysta $\mathrm{m}\mathrm{o}\dot{\mathrm{z}}\mathrm{e}$ jechač

$\mathrm{z}$ prędkością $04\mathrm{k}\mathrm{m}/\mathrm{h}$ mniejszą. By czas podrózy $\mathrm{z}A$ do $B$ nie uległ zmianie, zmuszony

jest na pozostafym odcinku zwiększyč prędkośč $012\mathrm{k}\mathrm{m}/\mathrm{h}$. Jaka jest odległośč $\mathrm{z}A$ do

$B\mathrm{i}\mathrm{z}$ jaką prędkością $\mathrm{j}\mathrm{e}\acute{\mathrm{z}}$dził rowerzysta przed ulewami?

2. Niech $f(x)=|4-|x-2||+1$. Sporządzič staranny wykres funkcji $f\mathrm{i}$ posługując $\mathrm{s}\mathrm{i}\mathrm{e}$ nim:

a) wyznaczyč najmniejszą $\mathrm{i}$ największ$\Phi$ wartośč funkcji $f\mathrm{w}$ przedziale $[0$, 7$]$, b) podač

równanie osi symetrii wykresu funkcji $f$, c) wyznaczyč $a>0\mathrm{t}\mathrm{a}\mathrm{k}$, aby pole figury ogra-

niczonej osiami układu, wykresem funkcji $f$ oraz prostą $x=a$ było równe 32.

3. Promień światfa przechodzi przez punkt $A(1,1)$, odbija się od prostej $0$ równaniu $y=$

$x-2$ (zgodnie $\mathrm{z}$ zasadą mówiącą, $\dot{\mathrm{z}}\mathrm{e}$ kąt padania jest równy kątowi odbicia) $\mathrm{i}$ przechodzi

przez punkt $B(4,6)$. Wyznaczyč wspófrzędne punktu odbicia $P$ oraz równania prostych,

po których biegnie promień przed $\mathrm{i}$ po odbiciu.

4. Na egzaminie uczeń wybiera losowo 4 pytania $\mathrm{z}$ zestawu egzaminacyjnego liczącego 40

pytań. Aby zdač egzamin nalez $\mathrm{y}$ poprawnie odpowiedzieč na co najmniej dwa pytania.

Jakie jest prawdopodobieństwo zdania egzaminu przez ucznia znajqcego odpowiedzi na

40\% pytań $\mathrm{z}$ zestawu egzaminacyjnego?

5. $\mathrm{W}$ ciągu arytmetycznym $(a_{n})$ mamy $a_{1}+a_{3}=3$ oraz $a_{1}a_{4}=1$. Dla jakich $n$ prawdziwa

jest nierównośč $a_{1}+a_{2}+a_{3}+\ldots+a_{n}\leq 93$?

6. Trójk$\Phi$t prostokątny $0$ przyprostokątnych $a, b$ obracamy wokóf środkowej najdluzszego

boku. Obliczyč objętośč otrzymanej bryly.

7. Korzystając $\mathrm{z}$ zasady indukcji matematycznej wykazač, $\dot{\mathrm{z}}\mathrm{e}$ dla $\mathrm{k}\mathrm{a}\dot{\mathrm{z}}$ dej liczby naturalnej

$n$ liczba $7^{n}-(-3)^{n}$ dzieli się przez 10.

8. Dla jakich wartości parametru rzeczywistego $m$ równanie

$2^{2x}-2(m-1)2^{x}+m^{2}-m-2=0$

ma dokładnie jeden pierwiastek rzeczywisty?

9. Wśród graniastosfupów prawidfowych sześciokątnych $0$ danym polu powierzchni cafkowi-

tej $S=27\sqrt{3}\mathrm{d}\mathrm{m}^{2}$ wskazač graniastosłup $0$ największej objętości. Podač objętośč tego

graniastosfupa $\mathrm{z}$ dokfadnością do l $\mathrm{m}1.$





PRACA KONTROLNA nr 4

styczeń 2006r.

l. Rozwiązač uklad równań

$\left\{\begin{array}{l}
x^{2}-y^{2}\\
x^{3}+y^{3}
\end{array}\right.$

$2(x-y)$

$6-(x-y)$

2. Dany jest punkt $P(3,2)$ oraz dwie proste $k\mathrm{i}l\mathrm{o}$ równaniach odpowiednio: $x+y+4=0$

$\mathrm{i}2x-3y-9=0$. Znalez/č taki punkt $Q$ na prostej $l$, aby środek odcinka $\overline{PQ}\mathrm{l}\mathrm{e}\dot{\mathrm{z}}$ af na

prostej $k$. Rozwiqzanie zilustrowač odpowiednim rysunkiem.

3. Dlajakich wartości parametru rzeczywistego $a\neq 0$ pierwiastki wielomianu $w(x)=a^{2}x^{3}-$

$a^{2}x^{2}-(a^{2}+1)x+a^{2}-1$ są trzema pierwszymi wyrazami pewnego ciągu arytmetycznego?

Dla $\mathrm{k}\mathrm{a}\dot{\mathrm{z}}$ dego otrzymanego przypadku obliczyč czwarty wyraz ciągu.

4. Znalez/č liczbę trzycyfrową wiedząc, $\dot{\mathrm{z}}\mathrm{e}$ iloraz $\mathrm{z}$ dzielenia tej liczby przez sumę jej cyfr

jest równy 48, a róznica szukanej 1iczby $\mathrm{i}$ liczby napisanej tymi samymi cyframi, ale $\mathrm{w}$

odwrotnym porządku wynosi 198.

5. $\mathrm{W}$ okrąg wpisano trapez $\mathrm{t}\mathrm{a}\mathrm{k}, \dot{\mathrm{z}}\mathrm{e}$ jedna $\mathrm{z}$ jego podstaw jest średnicą okręgu. Stosunek

długości obwodu trapezu do sumy długości jego podstaw jest równy $\displaystyle \frac{3}{2}$. Obliczyč cosinus

kąta ostrego $\mathrm{w}$ tym trapezie.

6. Na ostrosłupie prawidłowym trójkątnym opisano stozek, a na tym stozku opisano ku-

lę. $K_{\Phi^{\mathrm{t}}}$ przy wierzcholku przekroju osiowego stozka jest równy $\alpha$. Obliczyč stosunek

objętości kuli do objętości ostrosłupa.

7. Rozwiązač nierównośč

$-\infty^{1}1$ -$0_{m}11 m \rightarrow$ -{\it m} l

$1. \perp \mathrm{L}\cdot\Delta \mathrm{v}\mathrm{v}1*^{\Delta}\omega\vee\perp\perp 1\vee\perp\cdot \mathrm{v}\mathrm{v}\perp\perp\cdot \mathrm{o}\mathrm{c}$

$3^{x+\frac{1}{2}}-2^{2x+1}<4^{x}-5\cdot 3^{x-\frac{1}{2}}$

8. Zbadač przebieg zmienności $\mathrm{i}$ sporządzič staranny wykres funkcji $f(x) = \displaystyle \frac{4-x^{2}}{x_{/}^{2}-1}$. Na-

stępnie narysowač wykres funkcji $k=g(m)$, gdzie $k$ jest liczbą pierwiastków rownania

$|$--{\it x}4-2-{\it x}21$|=$ {\it m}.

9. Ze zbioru cyfr $\{0$, 1, 2, 3$\}$ wylosowano dwie $\mathrm{i}$ odrzucono. $\mathrm{Z}$ otrzymanego zbioru wyloso-

wano ze zwracaniem pięč cyfr. Jakie jest prawdopodobieństwo, $\dot{\mathrm{z}}\mathrm{e}$ liczba utworzona $\mathrm{z}$

tych cyfr jest podzielna przez 3?





PRACA KONTROLNA nr 5

luty $2006\mathrm{r}.$

l. Przyprostokqtne trójkąta prostokqtnego mają długości 6 $\mathrm{i}8$ cm. $\mathrm{W}$ trójkąt ten wpisano

kwadrat $\mathrm{t}\mathrm{a}\mathrm{k}, \dot{\mathrm{z}}\mathrm{e}$ dwa jego wierzchofki $ 1\mathrm{e}\mathrm{Z}\otimes$ na przeciwprostokątnej, a dwa pozostafe na

przyprostokątnych. Obliczyč pola figur, na jakie brzeg kwadratu dzieli dany trójkąt.

2. Niech $A$ będzie zbiorem tych punktów $x$ osi liczbowej, których suma odleglości od punk-

tów $-1\mathrm{i}5$jest mniejsza od 12, a $B=\{x\in R:\sqrt{x^{2}-25}-x<1\}$. Znalez/č $\mathrm{i}$ zaznaczyč

na osi liczbowej zbiory $A, B$ oraz $(A\backslash B)\cup(B\backslash A).$

3. Wykazač, $\dot{\mathrm{z}}\mathrm{e}$ liczba $x=\sqrt[3]{2\sqrt{6}+4}-\sqrt[3]{2\sqrt{6}-4}$

Wskazówka: obliczyč $x^{3}$

jest niewymierna.

4. Wyznaczyč zbiór wszystkich wartości parametru $m$, dla których równanie

$\displaystyle \cos x=\frac{3m}{m^{2}-4}$

ma rozwiązanie $\mathrm{w}$ przedziale $[-\displaystyle \frac{\pi}{3},\frac{\pi}{3}]$. Obliczyč ctg $x$ dla cafkowitych $m\mathrm{z}$ tego zbioru.

5. $\mathrm{W}$ ostrosłupie prawidłowym sześciokątnym przekrój $0$ najmniejszym polu płaszczyzną

zawieraj $\Phi^{\mathrm{C}}\Phi$ wysokośč ostrosłupa jest trójkątem równobocznym $0$ boku $2a$. Obliczyč co-

sinus kąta dwuściennego między ścianami bocznymi tego ostroslupa.

6. Dane jest pófkole $0$ średnicy AB $\mathrm{i}$ promieniu długości $|AO| = r$. Na promieniu $AO$

jako na średnicy wewnątrz danego pólkola zakreślono pófokrąg. Na większym pófokręgu

obrano punkt $P \mathrm{i}$ polączono go $\mathrm{z}$ punktami A $\mathrm{i} B$. Odcinek $AP$ przecina mniejszy

pólokrąg $\mathrm{w}$ punkcie $C$. Obliczyč dfugośč odcinka $AP, \mathrm{j}\mathrm{e}\dot{\mathrm{z}}$ eli wiadomo, $\dot{\mathrm{z}}\mathrm{e}|CP|+|PB|=1.$

Przeprowadzič analizę dla jakich wartości $r$ zadanie ma rozwiązanie.

7. Zbadač monotonicznośč ciągu $a_{n} = \displaystyle \frac{n-2}{\sqrt{n^{2}+1}}$. Obliczyč granicę tego ciągu, a następnie

znalez/č wszystkie jego wyrazy odlegle od granicy co najmniej $0\displaystyle \frac{1}{10}.$

8. Wykazač, $\dot{\mathrm{z}}\mathrm{e}$ pole trójkąta ograniczonego styczną do wykresu funkcji $y = \displaystyle \frac{2x-3}{x-2} \mathrm{i}$ jego

asymptotami jest stałe. Sporządzič rysunek.

9. Rozwiązač uklad równań

$\left\{\begin{array}{l}
\log_{(x-y)}[8(x+y)]\\
(x+y)^{\log_{4}(x-y)}
\end{array}\right.$

$-2$

-21



\end{document}