\documentclass[a4paper,12pt]{article}
\usepackage{latexsym}
\usepackage{amsmath}
\usepackage{amssymb}
\usepackage{graphicx}
\usepackage{wrapfig}
\pagestyle{plain}
\usepackage{fancybox}
\usepackage{bm}

\begin{document}

xxxv

KORESPONDENCYJNY KURS Z MATEMATYKI

PRACA KONTROLNA nr l

$\mathrm{p}\mathrm{a}\acute{\mathrm{z}}$dziernik 2$005\mathrm{r}.$

l. Niech $f(x) = x^{2}+bx+5$. Wyznaczyč wszystkie wartości parametru $b$, dla których:

a) wykres funkcji $f$ jest symetryczny względem prostej $x=2$, b) wierzchofek paraboli

będqcej wykresem funkcji $f\mathrm{l}\mathrm{e}\dot{\mathrm{z}}\mathrm{y}$ na prostej $x+y+1=0$. Sporządzič staranny rysunek.

2. Kilkoro dzieci dostafo torebkę cukierków do równego podziału. Gdyby liczba dzieci byfa

$01$ mniejsza, to $\mathrm{k}\mathrm{a}\dot{\mathrm{z}}$ de $\mathrm{z}$ nich dostafoby $02$ cukierki więcej. Gdyby cukierków byfo dwa

razy więcej, a dzieci $0$ dwoje więcej, to $\mathrm{k}\mathrm{a}\dot{\mathrm{z}}$ de dostałoby $05$ cukierków wiecej. Ile było

dzieci a ile cukierków?

3. Babcia zalozyła swemu rocznemu wnukowi lokatę $\mathrm{w}$ wysokości 1000 $\mathrm{z}l$ oprocentowanq $\mathrm{w}$

wysokości 6\% $\mathrm{w}$ skali roku $\mathrm{z}$ półroczną kapitalizacją odsetek $\mathrm{i}$ postanowiła co 6 miesięcy

wplacač na to konto 100 $\mathrm{z}\mathrm{f}$. Jaka sumę dostanie wnuczek $\mathrm{w}$ dniu swoich osiemnastych

urodzin?

4. Dane są wierzcholki $A(-3,2), C(4,2), D(0,4)$ trapezu równoramiennego ABCD, $\mathrm{w}$ któ-

rym $\overline{AB}||\overline{CD}$. Wyznaczyč współrzędne wierzchołka $B$ oraz pole trapezu. Sporządzič

rysunek.

5. Wyznaczyč stosunek dlugości przekątnych rombu wiedząc, $\dot{\mathrm{z}}\mathrm{e}$ stosunek pola kofa wpisa-

nego $\mathrm{w}$ ten romb do pola rombu wynosi $\displaystyle \frac{\pi}{5}.$

6. Podstawą prostopadłościanu jest prostokąt $0$ dluzszym boku $a$. Przekątna prostopadlo-

ścianu tworzy $\mathrm{z}$ przekątnymi ścian bocznych kąty $\alpha$ oraz $ 2\alpha$. Obliczyč objetośč tego

prostopadfościanu. Dla jakich kątów $\alpha$ zadanie ma rozwiązanie?

7. Dla jakich wartości parametru $p$ funkcja

$f(x)=\displaystyle \frac{x^{3}}{px^{2}+px+1}$

jest określona $\mathrm{i}$ rosnąca na calej prostej rzeczywistej?

8. Rozwiązač równanie

ctg $x=2\sqrt{3}\sin x.$

9. Liczby $a_{1} = (\sqrt{2})^{\log_{\frac{1}{2}}16}$ oraz $a_{2} = 16^{-\log_{\sqrt[3]{2}}\sqrt[4]{2}}$ są odpowiednio pierwszym $\mathrm{i}$ drugim

wyrazem pewnego ciqgu geometrycznego. Rozwiązač nierównośč

$(\sqrt{x})^{\log^{2}x-1}\geq 2S,$

gdzie S oznacza sumę wszystkich wyrazów tego ciągu.
\end{document}
