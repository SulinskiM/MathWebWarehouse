\documentclass[a4paper,12pt]{article}
\usepackage{latexsym}
\usepackage{amsmath}
\usepackage{amssymb}
\usepackage{graphicx}
\usepackage{wrapfig}
\pagestyle{plain}
\usepackage{fancybox}
\usepackage{bm}

\begin{document}

PRACA KONTROLNA nr 2

listopad $2005\mathrm{r}.$

l. Stop zawiera 60\% srebra próby 0,6 $\mathrm{i}$ 30\% srebra próby 0,7 oraz 20 dkg srebra próby 0,8.

a) Ile srebra $\mathrm{i}$ jakiej próby nalez $\mathrm{y}$ dodač, by otrzymač 2,5 kg srebra próby 0,7?

b) Obliczyč próbę stopu, jakim nalezy zastapič połowę danego stopu, by otrzymač stop

$0$ próbie 0,75?

2. Wyznaczyč wszystkie punkty okręgu $0$ środku $(0,0)\mathrm{i}$ promieniu 5, których i1oczyn kwa-

dratów wspólrzędnych jest najmniejszą wspólną wielokrotnościa liczb 12 $\mathrm{i} 14$. Obliczyč

obwód wielokąta, którego wierzchofkami $\mathrm{s}\Phi$ znalezione punkty. Bez $\mathrm{u}\dot{\mathrm{z}}$ ywania kalkulatora

zbadač, czy jest on większy od 30.

3. Dla jakich wartości $a \mathrm{i} b$ wielomian $W(x) = x^{4}-3x^{3}+bx^{2}+ax+b$ jest podzielny

przez trójmian kwadratowy $(x^{2}-1)$ ? Dla znalezionych wartości wspófczynników $a\mathrm{i}b$

rozwiązač nierównośč $W(x)\leq 0.$

4. Wykorzystuj $\Phi^{\mathrm{C}}\mathrm{t}\mathrm{o}\dot{\mathrm{z}}$ samośč trygonometryczną $\displaystyle \sin\alpha+\sin\beta=2\sin\frac{\alpha+\beta}{2}\cos\frac{\alpha-\beta}{2}$ narysowač

staranny wykres funkcji $f(x)=|\sin x+\cos x|$. Korzystając $\mathrm{z}$ tego wykresu, wyznaczyč

najmniejszą $\mathrm{i}$ największa wartośč funkcji $f$ na przedziale $[-\displaystyle \frac{\pi}{2},\pi]$. Wyznaczyč rozwiązania

równania $f(x)=\displaystyle \frac{1}{\sqrt{2}}$ zawarte $\mathrm{w}$ tym przedziale.

5. Pole powierzchni całkowitej stozka jest dwa razy większe od pola powierzchni kuli wpi-

sanej $\mathrm{w}$ ten stozek. Znalez/č cosinus kąta nachylenia $\mathrm{t}\mathrm{w}\mathrm{o}\mathrm{r}\mathrm{z}\Phi^{\mathrm{C}\mathrm{e}\mathrm{j}}$ stozka do podstawy.

6. $\mathrm{W}$ trójkącie równoramiennym suma długości ramienia $\mathrm{i}$ promienia okręgu opisanego

na tym trójkącie równa jest $m$ a wysokośč trójkąta równa jest 2. Wyznaczyč długośč

ramienia jako funkcję parametru $m$ oraz wartośč $m$, dla której kąt przy wierzchofku

trójkąta równy jest $120^{\mathrm{o}}$? Dla jakich wartości $m$ zadanie ma rozwiązanie?

7. Narysowač zbiory $A=\{(x,y):x^{2}+2x+y^{2}\leq 0\}, B= \{(x,y):x^{2}+2y+y^{2}\leq 0\},$

$C=\{(x,y):x\leq 0,y\geq 0,x^{2}+y^{2}\leq 4\}$. Obliczyč pola figur $A\cap B, A\backslash B, C\backslash (A\cup B).$

Podač równania osi symetrii figury $A\cup B.$

8. Rozwiązač nierównośč $\displaystyle \frac{1}{\sqrt{4-x^{2}}}\leq\frac{1}{x-1}.$

9 $\mathrm{W}\mathrm{z}$naczyč równania wszystkich pstych stycznychktóre s$\text{ą} \mathrm{p}\mathrm{o}\mathrm{s}\mathrm{o}$padle d$\mathrm{o}\mathrm{p}$rostej orównaniu {\it x}$+y=0.$Obliczyčdo wykres p$\mathrm{o}1\mathrm{e}\mathrm{r}\text{ó} \mathrm{w}\mathrm{n}\mathrm{o}1\mathrm{e}\mathrm{g}1$obfunkc {\it f}$(x)=\displaystyle \frac{8x}{x^{2}+3,\mathrm{o}\mathrm{k}\mathrm{u}},$

którego wierzchołkami są punkty wspólne tych stycznych $\mathrm{z}$ wykresem funkcji $f(x).$
\end{document}
