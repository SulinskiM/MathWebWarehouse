\documentclass[a4paper,12pt]{article}
\usepackage{latexsym}
\usepackage{amsmath}
\usepackage{amssymb}
\usepackage{graphicx}
\usepackage{wrapfig}
\pagestyle{plain}
\usepackage{fancybox}
\usepackage{bm}

\begin{document}

PRACA KONTROLNA nr 3

grudzień $2005\mathrm{r}$

l. Droge $\mathrm{z}$ miasta $A$ do miasta $B$ rowerzysta pokonuje $\mathrm{w}$ ciqgu 3 godzin. Po długotrwałych

deszczach stan $\displaystyle \frac{3}{5}$ drogi pogorszyf się na tyle, $\dot{\mathrm{z}}\mathrm{e}$ na tym odcinku rowerzysta $\mathrm{m}\mathrm{o}\dot{\mathrm{z}}\mathrm{e}$ jechač

$\mathrm{z}$ prędkością $04\mathrm{k}\mathrm{m}/\mathrm{h}$ mniejszą. By czas podrózy $\mathrm{z}A$ do $B$ nie uległ zmianie, zmuszony

jest na pozostafym odcinku zwiększyč prędkośč $012\mathrm{k}\mathrm{m}/\mathrm{h}$. Jaka jest odległośč $\mathrm{z}A$ do

$B\mathrm{i}\mathrm{z}$ jaką prędkością $\mathrm{j}\mathrm{e}\acute{\mathrm{z}}$dził rowerzysta przed ulewami?

2. Niech $f(x)=|4-|x-2||+1$. Sporządzič staranny wykres funkcji $f\mathrm{i}$ posługując $\mathrm{s}\mathrm{i}\mathrm{e}$ nim:

a) wyznaczyč najmniejszą $\mathrm{i}$ największ$\Phi$ wartośč funkcji $f\mathrm{w}$ przedziale $[0$, 7$]$, b) podač

równanie osi symetrii wykresu funkcji $f$, c) wyznaczyč $a>0\mathrm{t}\mathrm{a}\mathrm{k}$, aby pole figury ogra-

niczonej osiami układu, wykresem funkcji $f$ oraz prostą $x=a$ było równe 32.

3. Promień światfa przechodzi przez punkt $A(1,1)$, odbija się od prostej $0$ równaniu $y=$

$x-2$ (zgodnie $\mathrm{z}$ zasadą mówiącą, $\dot{\mathrm{z}}\mathrm{e}$ kąt padania jest równy kątowi odbicia) $\mathrm{i}$ przechodzi

przez punkt $B(4,6)$. Wyznaczyč wspófrzędne punktu odbicia $P$ oraz równania prostych,

po których biegnie promień przed $\mathrm{i}$ po odbiciu.

4. Na egzaminie uczeń wybiera losowo 4 pytania $\mathrm{z}$ zestawu egzaminacyjnego liczącego 40

pytań. Aby zdač egzamin nalez $\mathrm{y}$ poprawnie odpowiedzieč na co najmniej dwa pytania.

Jakie jest prawdopodobieństwo zdania egzaminu przez ucznia znajqcego odpowiedzi na

40\% pytań $\mathrm{z}$ zestawu egzaminacyjnego?

5. $\mathrm{W}$ ciągu arytmetycznym $(a_{n})$ mamy $a_{1}+a_{3}=3$ oraz $a_{1}a_{4}=1$. Dla jakich $n$ prawdziwa

jest nierównośč $a_{1}+a_{2}+a_{3}+\ldots+a_{n}\leq 93$?

6. Trójk$\Phi$t prostokątny $0$ przyprostokątnych $a, b$ obracamy wokóf środkowej najdluzszego

boku. Obliczyč objętośč otrzymanej bryly.

7. Korzystając $\mathrm{z}$ zasady indukcji matematycznej wykazač, $\dot{\mathrm{z}}\mathrm{e}$ dla $\mathrm{k}\mathrm{a}\dot{\mathrm{z}}$ dej liczby naturalnej

$n$ liczba $7^{n}-(-3)^{n}$ dzieli się przez 10.

8. Dla jakich wartości parametru rzeczywistego $m$ równanie

$2^{2x}-2(m-1)2^{x}+m^{2}-m-2=0$

ma dokładnie jeden pierwiastek rzeczywisty?

9. Wśród graniastosfupów prawidfowych sześciokątnych $0$ danym polu powierzchni cafkowi-

tej $S=27\sqrt{3}\mathrm{d}\mathrm{m}^{2}$ wskazač graniastosłup $0$ największej objętości. Podač objętośč tego

graniastosfupa $\mathrm{z}$ dokfadnością do l $\mathrm{m}1.$
\end{document}
