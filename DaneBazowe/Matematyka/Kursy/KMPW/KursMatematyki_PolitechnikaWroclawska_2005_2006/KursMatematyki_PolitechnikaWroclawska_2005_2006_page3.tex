\documentclass[a4paper,12pt]{article}
\usepackage{latexsym}
\usepackage{amsmath}
\usepackage{amssymb}
\usepackage{graphicx}
\usepackage{wrapfig}
\pagestyle{plain}
\usepackage{fancybox}
\usepackage{bm}

\begin{document}

PRACA KONTROLNA nr 4

styczeń 2006r.

l. Rozwiązač uklad równań

$\left\{\begin{array}{l}
x^{2}-y^{2}\\
x^{3}+y^{3}
\end{array}\right.$

$2(x-y)$

$6-(x-y)$

2. Dany jest punkt $P(3,2)$ oraz dwie proste $k\mathrm{i}l\mathrm{o}$ równaniach odpowiednio: $x+y+4=0$

$\mathrm{i}2x-3y-9=0$. Znalez/č taki punkt $Q$ na prostej $l$, aby środek odcinka $\overline{PQ}\mathrm{l}\mathrm{e}\dot{\mathrm{z}}$ af na

prostej $k$. Rozwiqzanie zilustrowač odpowiednim rysunkiem.

3. Dlajakich wartości parametru rzeczywistego $a\neq 0$ pierwiastki wielomianu $w(x)=a^{2}x^{3}-$

$a^{2}x^{2}-(a^{2}+1)x+a^{2}-1$ są trzema pierwszymi wyrazami pewnego ciągu arytmetycznego?

Dla $\mathrm{k}\mathrm{a}\dot{\mathrm{z}}$ dego otrzymanego przypadku obliczyč czwarty wyraz ciągu.

4. Znalez/č liczbę trzycyfrową wiedząc, $\dot{\mathrm{z}}\mathrm{e}$ iloraz $\mathrm{z}$ dzielenia tej liczby przez sumę jej cyfr

jest równy 48, a róznica szukanej 1iczby $\mathrm{i}$ liczby napisanej tymi samymi cyframi, ale $\mathrm{w}$

odwrotnym porządku wynosi 198.

5. $\mathrm{W}$ okrąg wpisano trapez $\mathrm{t}\mathrm{a}\mathrm{k}, \dot{\mathrm{z}}\mathrm{e}$ jedna $\mathrm{z}$ jego podstaw jest średnicą okręgu. Stosunek

długości obwodu trapezu do sumy długości jego podstaw jest równy $\displaystyle \frac{3}{2}$. Obliczyč cosinus

kąta ostrego $\mathrm{w}$ tym trapezie.

6. Na ostrosłupie prawidłowym trójkątnym opisano stozek, a na tym stozku opisano ku-

lę. $K_{\Phi^{\mathrm{t}}}$ przy wierzcholku przekroju osiowego stozka jest równy $\alpha$. Obliczyč stosunek

objętości kuli do objętości ostrosłupa.

7. Rozwiązač nierównośč

$-\infty^{1}1$ -$0_{m}11 m \rightarrow$ -{\it m} l

$1. \perp \mathrm{L}\cdot\Delta \mathrm{v}\mathrm{v}1*^{\Delta}\omega\vee\perp\perp 1\vee\perp\cdot \mathrm{v}\mathrm{v}\perp\perp\cdot \mathrm{o}\mathrm{c}$

$3^{x+\frac{1}{2}}-2^{2x+1}<4^{x}-5\cdot 3^{x-\frac{1}{2}}$

8. Zbadač przebieg zmienności $\mathrm{i}$ sporządzič staranny wykres funkcji $f(x) = \displaystyle \frac{4-x^{2}}{x_{/}^{2}-1}$. Na-

stępnie narysowač wykres funkcji $k=g(m)$, gdzie $k$ jest liczbą pierwiastków rownania

$|$--{\it x}4-2-{\it x}21$|=$ {\it m}.

9. Ze zbioru cyfr $\{0$, 1, 2, 3$\}$ wylosowano dwie $\mathrm{i}$ odrzucono. $\mathrm{Z}$ otrzymanego zbioru wyloso-

wano ze zwracaniem pięč cyfr. Jakie jest prawdopodobieństwo, $\dot{\mathrm{z}}\mathrm{e}$ liczba utworzona $\mathrm{z}$

tych cyfr jest podzielna przez 3?
\end{document}
