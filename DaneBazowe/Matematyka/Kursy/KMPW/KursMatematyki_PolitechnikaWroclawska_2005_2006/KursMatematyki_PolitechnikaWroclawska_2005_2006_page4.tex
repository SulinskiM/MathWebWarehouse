\documentclass[a4paper,12pt]{article}
\usepackage{latexsym}
\usepackage{amsmath}
\usepackage{amssymb}
\usepackage{graphicx}
\usepackage{wrapfig}
\pagestyle{plain}
\usepackage{fancybox}
\usepackage{bm}

\begin{document}

PRACA KONTROLNA nr 5

luty $2006\mathrm{r}.$

l. Przyprostokqtne trójkąta prostokqtnego mają długości 6 $\mathrm{i}8$ cm. $\mathrm{W}$ trójkąt ten wpisano

kwadrat $\mathrm{t}\mathrm{a}\mathrm{k}, \dot{\mathrm{z}}\mathrm{e}$ dwa jego wierzchofki $ 1\mathrm{e}\mathrm{Z}\otimes$ na przeciwprostokątnej, a dwa pozostafe na

przyprostokątnych. Obliczyč pola figur, na jakie brzeg kwadratu dzieli dany trójkąt.

2. Niech $A$ będzie zbiorem tych punktów $x$ osi liczbowej, których suma odleglości od punk-

tów $-1\mathrm{i}5$jest mniejsza od 12, a $B=\{x\in R:\sqrt{x^{2}-25}-x<1\}$. Znalez/č $\mathrm{i}$ zaznaczyč

na osi liczbowej zbiory $A, B$ oraz $(A\backslash B)\cup(B\backslash A).$

3. Wykazač, $\dot{\mathrm{z}}\mathrm{e}$ liczba $x=\sqrt[3]{2\sqrt{6}+4}-\sqrt[3]{2\sqrt{6}-4}$

Wskazówka: obliczyč $x^{3}$

jest niewymierna.

4. Wyznaczyč zbiór wszystkich wartości parametru $m$, dla których równanie

$\displaystyle \cos x=\frac{3m}{m^{2}-4}$

ma rozwiązanie $\mathrm{w}$ przedziale $[-\displaystyle \frac{\pi}{3},\frac{\pi}{3}]$. Obliczyč ctg $x$ dla cafkowitych $m\mathrm{z}$ tego zbioru.

5. $\mathrm{W}$ ostrosłupie prawidłowym sześciokątnym przekrój $0$ najmniejszym polu płaszczyzną

zawieraj $\Phi^{\mathrm{C}}\Phi$ wysokośč ostrosłupa jest trójkątem równobocznym $0$ boku $2a$. Obliczyč co-

sinus kąta dwuściennego między ścianami bocznymi tego ostroslupa.

6. Dane jest pófkole $0$ średnicy AB $\mathrm{i}$ promieniu długości $|AO| = r$. Na promieniu $AO$

jako na średnicy wewnątrz danego pólkola zakreślono pófokrąg. Na większym pófokręgu

obrano punkt $P \mathrm{i}$ polączono go $\mathrm{z}$ punktami A $\mathrm{i} B$. Odcinek $AP$ przecina mniejszy

pólokrąg $\mathrm{w}$ punkcie $C$. Obliczyč dfugośč odcinka $AP, \mathrm{j}\mathrm{e}\dot{\mathrm{z}}$ eli wiadomo, $\dot{\mathrm{z}}\mathrm{e}|CP|+|PB|=1.$

Przeprowadzič analizę dla jakich wartości $r$ zadanie ma rozwiązanie.

7. Zbadač monotonicznośč ciągu $a_{n} = \displaystyle \frac{n-2}{\sqrt{n^{2}+1}}$. Obliczyč granicę tego ciągu, a następnie

znalez/č wszystkie jego wyrazy odlegle od granicy co najmniej $0\displaystyle \frac{1}{10}.$

8. Wykazač, $\dot{\mathrm{z}}\mathrm{e}$ pole trójkąta ograniczonego styczną do wykresu funkcji $y = \displaystyle \frac{2x-3}{x-2} \mathrm{i}$ jego

asymptotami jest stałe. Sporządzič rysunek.

9. Rozwiązač uklad równań

$\left\{\begin{array}{l}
\log_{(x-y)}[8(x+y)]\\
(x+y)^{\log_{4}(x-y)}
\end{array}\right.$

$-2$

-21
\end{document}
