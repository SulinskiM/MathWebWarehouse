\documentclass[a4paper,12pt]{article}
\usepackage{latexsym}
\usepackage{amsmath}
\usepackage{amssymb}
\usepackage{graphicx}
\usepackage{wrapfig}
\pagestyle{plain}
\usepackage{fancybox}
\usepackage{bm}

\begin{document}

XXXVI

KORESPONDENCYJNY KURS Z MATEMATYKI

PRACA KONTROLNA $\mathrm{n}\mathrm{r}1-$ POZIOM PODSTAWOWY

$\mathrm{p}\mathrm{a}\acute{\mathrm{z}}$dziernik 2$006\mathrm{r}.$

l. Róznica pewnej liczby trzycyfrowej $\mathrm{i}$ liczby otrzymanej za pomocą tych samych cyfr

zapisanych $\mathrm{w}$ odwrotnej kolejności równa jest 495, a suma równa jest 1009. Jaka to

liczba.

2. Obliczyč $p=\displaystyle \frac{64^{\frac{1}{3}}\sqrt{8}+8^{\frac{1}{3}}\sqrt{64}}{\sqrt[3]{64\sqrt{8}}}$. Znalez/č wszystkie liczby naturalne, dla których spełniona

jest nierównośč $x^{3}-2x^{2}-p^{2}x+2p^{2}\leq 0.$

3. Polowę kolekcji letniej sprzedano po zafozonej cenie. Po obnizce ceny $0$ 50\% udalo się

sprzedač pofowę pozostalej części towaru $\mathrm{i}$ dopiero kolejna 50\%-owa obnizka pozwo1ifa

sklepowi pozbyč się produktu.

a) Ile procent zaplanowanego przychodu stanowi uzyskana ze sprzedaz $\mathrm{y}$ kwota?

b) $\mathrm{O}$ ile procent wyjściowa cena towaru powinna byla byč $\mathrm{w}\mathrm{y}\dot{\mathrm{z}}$ sza, by sklep uzyskaf

zaplanowany początkowo przychód? Wyniki podač $\mathrm{z}$ dokładnością do l promila.

4. Dach wiezy kościola ma ksztalt ostrosfupa, którego podstawq jest sześciokąt foremny $0$

boku 2 $\mathrm{m}$ a największy $\mathrm{z}$ przekrojów pfaszczyzną $\mathrm{z}\mathrm{a}\mathrm{w}\mathrm{i}\mathrm{e}\mathrm{r}\mathrm{a}\mathrm{j}_{\Phi}\mathrm{c}\text{ą}$ wysokośč jest trójkątem

równobocznym. Obliczyč kubaturę dachu wiezy kościoła. Ile 2-1itrowych puszek farby

antykorozyjnej trzeba kupič do pomalowania blachy, którą pokryty jest dach, $\mathrm{j}\mathrm{e}\dot{\mathrm{z}}$ eli wia-

domo, $\dot{\mathrm{z}}\mathrm{e} 1$ litr farby wystarcza do pomalowania 6 $\mathrm{m}^{2}$ blachy $\mathrm{i}$ trzeba uwzględnič 8\%

farby na ewentualne straty.

5. Niech

$f(x)=$

dla

dla

$x\leq 1,$

$x>1.$

a) Narysowač wykres funkcji $f\mathrm{i}$ na jego podstawie wyznaczyč zbiór wartości funkcji.

b) Obliczyč $f(\sqrt{3}-1)$ oraz $f(3-\sqrt{3}).$

c) Rozwiqzač nierównośč $2\sqrt{f(x)}\leq 3\mathrm{i}$ zaznaczyč na osi $\mathrm{o}x$ zbiór rozwiązań.

6. Punkt $A=(1,0)$ jest wierzchofkiem rombu $0$ kącie przy tym wierzchołku równym $60^{\mathrm{o}}$

Wyznaczyč wspófrzędne pozostafych wierzchofków rombu wiedząc, $\dot{\mathrm{z}}\mathrm{e}$ dwa $\mathrm{z}$ nich lezą

na prostej $l$ : $2x-y+3=0$. Obliczyč pole rombu. Ile rozwiązań ma to zadanie?




PRACA KONTROLNA $\mathrm{n}\mathrm{r}1-$ POZIOM ROZSZERZONY

l. Rozwiązač nierównośč $\displaystyle \frac{1}{\sqrt{4-x^{2}}}\geq\frac{1}{x-1}\mathrm{i}$ starannie zaznaczyč zbiór rozwiqzań na osi liczbo-

wej.

2. Rozwiązač równanie 2 $\sin 2x+2\sin x-2\cos x=1$. Następnie podač rozwiązania nalezące

do przedziału $[-\pi,\pi].$

3. $\mathrm{Z}$ przystani A wyrusza $\mathrm{z}$ biegiem rzeki statek do przystani $\mathrm{B}$, odlegfej od A $0140$ km. Po

upływie l godziny wyrusza za nim łódz$\acute{}$ motorowa, dopędza statek, po czym wraca do

przystani A $\mathrm{w}$ tym samym momencie, $\mathrm{w}$ którym statek przybija do przystani B. Znalez$\acute{}$č

prędkośč biegu rzeki, $\mathrm{j}\mathrm{e}\dot{\mathrm{z}}$ eli wiadomo, $\dot{\mathrm{z}}\mathrm{e}\mathrm{w}$ stojącej wodzie prędkośč statku wynosi 16

$\mathrm{k}\mathrm{m}/$godz, a prędkośč łodzi 24 $\mathrm{k}\mathrm{m}/$godz.

4. Dane są liczby: $m=\displaystyle \frac{(_{4}^{6})\cdot(_{2}^{8})}{(_{3}^{7})}, n=\displaystyle \frac{(\sqrt{2})^{-4}(\frac{1}{4})^{-\frac{5}{2}}\sqrt[4]{3}}{(\sqrt[4]{16})_{27^{-\frac{1}{4}}}^{3}}.$

a) Sprawdzič, wykonując odpowiednie obliczenia, $\dot{\mathrm{z}}\mathrm{e}m, n$ są liczbami naturalnymi.

b) Wyznaczyč $k\mathrm{t}\mathrm{a}\mathrm{k}$, by liczby $m, k, n$ były odpowiednio: pierwszym, drugim $\mathrm{i}$ trzecim

wyrazem ciągu geometrycznego.

c) Wyznaczyč sumę wszystkich wyrazów nieskończonego ciągu geometrycznego, któ-

rego pierwszymi trzema wyrazami są $m, k, n$. Ile wyrazów tego ciągu nalez $\mathrm{y}$ wziąč,

by ich suma przekroczyła 95\% sumy wszystkich wyrazów?

5. $\mathrm{Z}$ wierzchofka $A$ kwadratu ABCD $0$ boku $a$ poprowadzono dwie proste, które dzielą kąt

przy tym wierzchołku na trzy równe części $\mathrm{i}$ przecinają boki kwadratu $\mathrm{w}$ punktach $K\mathrm{i}$

$L$. Wyznaczyč długości odcinków, najakie te proste dzielą przekątną kwadratu. Znalez$\acute{}$č

promień okręgu wpisanego $\mathrm{w}$ deltoid AKCL.

6. Podstawą pryzmy przedstawionej na rysunku ponizej jest prostokąt ABCD,
\begin{center}
\includegraphics[width=87.432mm,height=33.528mm]{./KursMatematyki_PolitechnikaWroclawska_2006_2007_page1_images/image001.eps}
\end{center}
{\it K} $\mathrm{L}$

C

$\mathrm{b}$

zmy.

A a $\mathrm{B}$

gosč $b$, gdzie $a>b$. Wszystkie ściany boczne

pryzmy $\mathrm{s}$ nachylone pod $\mathrm{k}$ tem $\alpha$ do płasz-

czyzny podstawy. Obliczyc objętosc tej pry-

ktorego bok $AB$ ma dlugośc $a$, a bok $BC$ dfu-





PRACA KONTROLNA $\mathrm{n}\mathrm{r}6-$ POZIOM PODSTAWOWY

marzec 2007r.

l. Boki trójk$\Phi$ta prostokątnego $0$ polu 12 $\mathrm{t}\mathrm{w}\mathrm{o}\mathrm{r}\mathrm{z}\Phi$ ciąg arytmetyczny. Wyznaczyč promień

okręgu wpisanego $\mathrm{w}$ ten trójkąt.

2. Pan Kowalski zaciągnąf 3l grudnia $\mathrm{p}\mathrm{o}\dot{\mathrm{z}}$ yczkę 4000 z1otych oprocentowaną $\mathrm{w}$ wysokości

18\% $\mathrm{w}$ skali roku. Zobowiązaf się splacič ją $\mathrm{w}$ ciągu roku $\mathrm{w}$ trzech równych ratach

płatnych 30 kwietnia, 30 sierpnia $\mathrm{i}30$ grudnia. Oprocentowanie $\mathrm{p}\mathrm{o}\dot{\mathrm{z}}$ yczki liczy się od l

stycznia, a odsetki od kredytu naliczane są $\mathrm{w}$ terminach pfatności rat. Obliczyč wysokośč

tych rat $\mathrm{w}$ zaokrągleniu do pełnych groszy.

3. Narysowač wykres funkcji $f(x)=$

$\mathrm{i}$ na jego podstawie wyznaczyč:

dla

dla

dla

$x<0,$

$x=0,$

$x>0,$

a) zbiór, jaki tworzą wartości funkcji $f(x)$, gdy $x$ przebiega przedzial $(-2,1)$ ;

b) zbiór rozwi$\Phi$zań nierówności $\displaystyle \frac{1}{2}\leq f(x)\leq 2.$

4. Suma wysokości $h$ ostrosłupa prawidłowego czworokątnego $\mathrm{i}$ jego krawędzi bocznej $b$

równa jest 12. D1a jakiej wartości $h$ objętośč tego ostroslupa jest najwieksza? Obliczyč

pole powierzchni cafkowitej ostrosfupa dla tej wartości $h.$

5. Punkty $A(0,4) \mathrm{i}D(3,5)$ są wierzchołkami trapezu równoramiennego ABCD, którego

podstawy $\overline{AB}$ oraz $\overline{CD}$ są prostopadfe do prostej $k\mathrm{o}$ równaniu $x-y-2=0$. Wyznaczyč

wspólrzędne pozostałych wierzchołków wiedząc, $\dot{\mathrm{z}}\mathrm{e}$ wierzchołek $C \mathrm{l}\mathrm{e}\dot{\mathrm{z}}\mathrm{y}$ na prostej $k.$

Znalez$\acute{}$č współrzędne środka oraz promień okręgu opisanego na tym trapezie.

6. Na kole $0$ promieniu $r$ opisano romb. Punkty styczności są wierzcholkami $\mathrm{c}\mathrm{z}\mathrm{w}\mathrm{o}\mathrm{r}\mathrm{o}\mathrm{k}_{\Phi}\mathrm{t}\mathrm{a}$

ABCD. Zakładając, $\dot{\mathrm{z}}\mathrm{e}$ stosunek pola rombu do pola czworokqta równy jest $\displaystyle \frac{8}{3}$, obliczyč

dlugośč boku rombu ijego $\mathrm{p}\mathrm{r}\mathrm{z}\mathrm{e}\mathrm{k}_{\Phi^{\mathrm{t}}}$nych. Obliczyč pole jednego $\mathrm{z}$ obszarów ograniczonych

bokami rombu $\mathrm{i}$ okręgiem.





PRACA KONTROLNA $\mathrm{n}\mathrm{r}6-$ POZIOM ROZSZERZONY

l. Dla jakich wartości parametru $\alpha\in[0,2\pi]$ istnieje dodatnie maksimum funkcji

$ f(x)=(2\cos\alpha-1)x^{2}-2x+\cos\alpha$ ?

2. Granicą ciągu $0$ wyrazie ogólnym $a_{n}=\displaystyle \frac{\sqrt{n^{4}+an^{3}+bn}-n^{2}}{\sqrt{n^{2}+1}}$ jest większy $\mathrm{z}$ pierwiastków

równania $4x^{\log x}+10x^{-\log x}=41$. Wyznaczyč parametry a $\mathrm{i}b.$

3. Wyznaczyč równanie krzywej utworzonej przez punkty, których odlegfośč od osi $0x$ jest

taka sama, jak odległośč od pólokręgu $0$ równaniu $y=\sqrt{2x-x^{2}}$. Sporzqdzič rysunek.

4. $\mathrm{W}$ stozku ściętym $\mathrm{P}^{\mathrm{r}\mathrm{z}\mathrm{e}\mathrm{k}}\Phi^{\mathrm{t}\mathrm{n}\mathrm{e}}$ przekroju osiowego $\mathrm{p}\mathrm{r}\mathrm{z}\mathrm{e}\mathrm{c}\mathrm{i}\mathrm{n}\mathrm{a}\mathrm{j}_{\Phi}$ się pod $\mathrm{k}_{\Phi}\mathrm{t}\mathrm{e}\mathrm{m}$ prostym, $\mathrm{a}$

tworząca $0$ dfugości $l$ nachylona jest do płaszczyzny podstawy dolnej pod kątem $\alpha.$

Obliczyč pole powierzchni bocznej tego stozka ściętego oraz pole powierzchni opisanej

na nim kuli.

5. $\mathrm{W}$ trójkącie $\triangle ABC$ dane są podstawa $|AB|=a$, kąt ostry przy podstawie $\angle CAB=2\alpha$

$\mathrm{i}$ dwusieczna tego kąta $|AD|=d$. Obliczyč pole koła opisanego na tym trójkącie. Podač

warunek istnienia rozwiązania.

6. Zbadač przebieg zmienności funkcji określonej wzorem

$f(x)=\displaystyle \sqrt{x+1}+1+\frac{1}{\sqrt{x+1}}+\ldots,$

gdzie prawa stronajest sumą wyrazów nieskończonego ciągu geometrycznego. Narysowač

jej staranny wykres.





PRACA KONTROLNA $\mathrm{n}\mathrm{r}2-$ POZIOM PODSTAWOWY

listopad $2006\mathrm{r}.$

l. Liczba dwuelementowych podzbiorów zbioru $A$ jest 7 razy większa $\mathrm{n}\mathrm{i}\dot{\mathrm{z}}$ liczba dwuele-

mentowych podzbiorów zbioru $B$. Liczba dwuelementowych podzbiorów zbioru $A$ nie

zawierających ustalonego elementu $a\in A$ jest 5 razy większa $\mathrm{n}\mathrm{i}\dot{\mathrm{z}}$ liczba dwuelemento-

wych podzbiorów zbioru $B$. Ile elementów ma $\mathrm{k}\mathrm{a}\dot{\mathrm{z}}\mathrm{d}\mathrm{y}\mathrm{z}$ tych zbiorów? Ile $\mathrm{k}\mathrm{a}\dot{\mathrm{z}}\mathrm{d}\mathrm{y}\mathrm{z}$ tych

zbiorów ma podzbiorów trzyelementowych?

2. $A\cap B, A\backslash B\mathrm{i}B\backslash $apisa '$\mathrm{w}\mathrm{p}$ostaciNiech {\it A}$=\displaystyle \{x\in 1\mathrm{R}:\frac{1}{x^{2}+23,A\mathrm{z}}\geq\frac{1}{10x,\mathrm{C}}\}$oraz {\it B}$=\mathrm{p}\mathrm{r}!^{x\in 1\mathrm{R}:|x-2|<\frac{7}{2}\}.\mathrm{Z}\mathrm{b}\mathrm{i}\mathrm{o}\mathrm{r}\mathrm{y}A,B,A\cup B}\mathrm{e}\mathrm{d}\mathrm{z}\mathrm{i}\mathrm{a}1\text{ó} \mathrm{w}1$iczbowych izaznaczyč j$\mathrm{e}\mathrm{n}\mathrm{a}\mathrm{o}\mathrm{s}\mathrm{i}$

liczb owej.

3. Stosując wzory skróconego mnozenia sprowadzič do najprostszej postaci wyrazenie

$W=2$ (sin6 $\alpha+\cos^{6}\alpha$)$-(\sin^{4}\alpha+\cos^{4}\alpha).$

Wykorzystując wzór $\cos 2\alpha = \cos^{2}\alpha-\sin^{2}\alpha$

wyrazenie $W$ przyjmuje wartośč $\displaystyle \frac{1}{2}.$

obliczyč, dla jakich wartości kąta $\alpha$

4. Wiadomo, $\dot{\mathrm{z}}\mathrm{e}$ liczby $-1$, 3 są pierwiastkami wielomianu $W(x)=x^{4}-ax^{3}-4x^{2}+bx+3.$

Wyznaczyč $a, b\mathrm{i}$ rozwiązač nierównośč $\sqrt{W(x)}\leq x^{2}-x.$

5. Na kole $0$ promieniu $r$ opisano trapez równoramienny, $\mathrm{w}$ którym stosunek dlugości pod-

staw wynosi 4: 3. Ob1iczyč stosunek po1a kofa do po1a trapezu oraz cosinus kąta ostrego

$\mathrm{w}$ tym trapezie.

6. $\mathrm{W}$ ostroslupie prawidłowym $\mathrm{c}\mathrm{z}\mathrm{w}\mathrm{o}\mathrm{r}\mathrm{o}\mathrm{k}_{\Phi^{\mathrm{t}}}\mathrm{n}\mathrm{y}\mathrm{m}$ wszystkie krawędzie $\mathrm{s}\Phi$ równe $a$. Obliczyč

objętośč tego ostroslupa. Znalez/č cosinus kąta nachylenia ściany bocznej do podstawy

oraz cosinus kata między ścianami bocznymi tego ostrosłupa.





PRACA KONTROLNA $\mathrm{n}\mathrm{r}2-$ POZIOM ROZSZERZONY

l. Trzeci składnik rozwinięcia dwumianu $(\displaystyle \sqrt[3]{x}+\frac{1}{\sqrt{x}})^{n}$ ma współczynnik równy 45. Wyzna-

czyč wszystkie skladniki tego rozwinięcia, $\mathrm{w}$ których $x$ występuje $\mathrm{w}$ potędze $0$ wykfadniku

całkowitym.

2. Niech $A=\{(x,y):y\geq||x-2|-1|\}, B=\{(x,y):y+\sqrt{4x-x^{2}-3}\leq 2\}$. Narysowač

na pfaszczy $\acute{\mathrm{z}}\mathrm{n}\mathrm{i}\mathrm{e}$ zbiór $A\cap B\mathrm{i}$ obliczyč jego pole.

3. Niech $a_{n}=\displaystyle \frac{1+kn}{5+k^{2}n}.$

a) Określič monotonicznośč ciągu $(a_{n})\mathrm{w}$ zalezności od parametru $k.$

b) Niech $S(k)$ oznacza sumę nieskończonego ciągu geometrycznego $0$ pierwszym wyra-

zie $a_{1}=1 \mathrm{i}$ ilorazie $q_{k}=\displaystyle \lim_{n\rightarrow\infty}a_{n}$. Sporządzič wykres funkcji $S(k)\mathrm{i}$ na tej podstawie

wyznaczyč zbiór jej wartości.

4. Dana jest funkcja $f(x)=\cos x$. Wyznaczyč dziedzinę oraz zbiór wartości funkcji

$g(x)=\sqrt{f(\frac{\pi}{2}-x)+\sqrt{3}f(x)-1}.$

5. $\mathrm{C}\mathrm{z}\mathrm{w}\mathrm{o}\mathrm{r}\mathrm{o}\mathrm{k}_{\Phi^{\mathrm{t}}}$ wypukly ABCD, $\mathrm{w}$ którym $AB=1, BC=2, CD=4, DA=3$ jest wpisany

$\mathrm{w}$ okrąg. Obliczyč promień $R$ tego okręgu. Sprawdzič, czy $\mathrm{w}$ czworokąt ten $\mathrm{m}\mathrm{o}\dot{\mathrm{z}}$ na wpisač

$\mathrm{o}\mathrm{k}\mathrm{r}\Phi \mathrm{g}. \mathrm{J}\mathrm{e}\dot{\mathrm{z}}$ eli $\mathrm{t}\mathrm{a}\mathrm{k}$, to obliczyč promień $r$ tego okręgu.

6. Plaszczyzna przechodząca przez jeden $\mathrm{z}$ wierzcholków czworościanu foremnego $\mathrm{i}$ rów-

noległa do jednej $\mathrm{z}$ jego krawędzi dzieli ten czworościan na dwie bryły $0$ takiej samej

objętości. Wyznaczyč pole przekroju oraz cosinus kąta nachylenia tego przekroju do

plaszczyzny podstawy.





PRACA KONTROLNA $\mathrm{n}\mathrm{r}3-$ POZIOM PODSTAWOWY

grudzień $2006\mathrm{r}.$

1. $\mathrm{Z}$ talii 24 kart wy1osowano dwie. Jakie jest prawdopodobieństwo, $\dot{\mathrm{z}}\mathrm{e}$ obie $\mathrm{s}\Phi$ koloru czer-

wonego lub obie są figurami?

2. Panowie X $\mathrm{i}\mathrm{Y}$ zafozyli jednocześnie firmy $\mathrm{i}\mathrm{w}$ pierwszym miesiącu dziafalności $\mathrm{k}\mathrm{a}\dot{\mathrm{z}}$ da

$\mathrm{z}$ nich miała obrot równy 50000 zfotych. Po pięciu miesiącach okazafo się, $\dot{\mathrm{z}}\mathrm{e}$ obrót

firmy pana X rósł $\mathrm{z}$ miesiąca na miesiąc $0$ tę samą kwotę, a obrót firmy pana $\mathrm{Y}$ rósł co

miesiąc $\mathrm{w}$ postępie geometrycznym. Stwierdzili równiez, $\dot{\mathrm{z}}\mathrm{e}\mathrm{w}$ drugim $\mathrm{i}$ trzecim miesiącu

działalności firma pana X miała obrót większy od obrotu firmy pana $\mathrm{Y}\mathrm{o}$ 2000 zł.

a) Jakie były obroty $\mathrm{k}\mathrm{a}\dot{\mathrm{z}}$ dej $\mathrm{z}$ firm $\mathrm{w}$ pieciu początkowych miesiącach?

b) Która $\mathrm{z}$ firm miała większą sumę obrotów $\mathrm{w}$ pierwszych pięciu miesiącach $\mathrm{i}\mathrm{o}$ ile?

c) Po ilu miesiącach obrót jednej $\mathrm{z}$ firm (której?) przekroczy dwukrotnie obrót drugiej

firmy?

3. Tangens kąta ostrego $\alpha$ równy jest $\displaystyle \frac{a}{b}$, gdzie

$\alpha=(\sqrt{2+\sqrt{3}}-\sqrt{2-\sqrt{3}})^{2}b=(\sqrt{\sqrt{2}+1}-\sqrt{\sqrt{2}-1})^{2}$

Wyznaczyč wartości pozostałych funkcji trygonometrycznych tego kata. Wykorzystując

wzór $\sin 2\alpha=2\sin\alpha\cos\alpha$, obliczyč miarę kąta $\alpha.$

4. Narysowač wykres funkcji $f(x)=|2x-4|-\sqrt{x^{2}+4x+4}$. Dlajakiego $m$ pole trójkąta

ograniczonego wykresem funkcji $f$ oraz prostą $y=m$ równe jest 6?

5. Harcerze rozbili 2 namioty, jeden $\mathrm{w}$ odległości 5 $\mathrm{m}$, drugi - 17 $\mathrm{m}$ od prostoliniowego

brzegu rzeki. Odległośč między namiotami równajest 13 $\mathrm{m}. \mathrm{W}$ którym miejscu $\mathrm{n}\mathrm{a}$ samym

brzegu rzeki (licząc od punktu brzegu będqcego rzutem prostopadfym punktu polozenia

pierwszego namiotu) powinni umieścič maszt $\mathrm{z}$ flagą zastępu, by odległośč od masztu do

$\mathrm{k}\mathrm{a}\dot{\mathrm{z}}$ dego $\mathrm{z}$ namiotów byfa taka sama?

6. Wysokośč ostrosłupa trójkątnego prawidłowego wynosi $h$, a kąt między wysokościami

ścian bocznych poprowadzonymi $\mathrm{z}$ wierzchołka ostrosfupa jest równy $ 2\alpha$. Obliczyč pole

powierzchni bocznej $\mathrm{i}$ objętośč tego ostrosfupa.





PRACA KONTROLNA $\mathrm{n}\mathrm{r}3-$ POZIOM ROZSZERZONY

l. Dlajakich wartości rzeczywistego parametru $p$ równanie $(p-2)x^{2}-(p+1)x-p=0$ ma

dwa rózne pierwiastki: a) ujemne? b) będące sinusem $\mathrm{i}$ cosinusem tego samego kąta?

2. Jakie powinny byč wymiary puszki $\mathrm{w}$ kształcie walca $0$ pojemności jednego litra, by jej

pole powierzchni całkowitej bylo najmniejsze?

3. $\mathrm{Z}$ badań statystycznych wynika,$\dot{\mathrm{z}}\mathrm{e}$ 5\% $\mathrm{m}\text{ę}\dot{\mathrm{z}}$ czyzn $\mathrm{i}$ 0,2\% kobiet to daltoniści. Wiadomo,

$\dot{\mathrm{z}}\mathrm{e}$ 55\% mieszkańców Wrocławia stanowia kobiety. Jakie jest prawdopodobieństwo, $\dot{\mathrm{z}}\mathrm{e}$

wśród 31osowo wybranych osób przynajmniej dwie nie odrózniaj$\Phi$ ko1orów?

4. Rozwiązač nierównośč $\displaystyle \log_{x}\frac{2-7x}{2x-7}\geq a$, gdzie $a$ jest granicą ciagu $0$ wyrazach

$a_{n}=\displaystyle \frac{4n(\sqrt{n^{2}+n}-n)}{n+1}.$

5. Pary liczb spefniające uklad równań

$\left\{\begin{array}{l}
-4x^{2}+y^{2}+2y+1=0,\\
-x^{2}+y+4=0
\end{array}\right.$

są wspólrzędnymi wierzchofków czworokata wypukfego ABCD.

a) Wykazač, $\dot{\mathrm{z}}\mathrm{e}$ czworokąt ABCD jest trapezem równoramiennym.

b) Wyznaczyč równanie okręgu opisanego na czworokącie ABCD.

6. Piramida utworzona z pięciu kul, z których cztery maja taki sam promień, jest wpisana

w walec. Przekrój osiowy walca jest kwadratem 0 boku d. Wyznaczyč promienie tych

kul.





PRACA KONTROLNA $\mathrm{n}\mathrm{r}4-$ POZIOM PODSTAWOWY

styczeń $2007\mathrm{r}.$

l. Dwóch robotników $\mathrm{m}\mathrm{o}\dot{\mathrm{z}}\mathrm{e}$ razem wykonač $\mathrm{P}^{\mathrm{e}\mathrm{w}\mathrm{n}}\Phi$ pracę $\mathrm{w}\mathrm{c}\mathrm{i}_{\Phi \mathrm{g}}\mathrm{u}7$ dni pod warunkiem, $\dot{\mathrm{z}}\mathrm{e}$

pierwszy $\mathrm{z}$ nich rozpocznie pracę $0$ póltora dnia wcześniej Gdyby $\mathrm{k}\mathrm{a}\dot{\mathrm{z}}\mathrm{d}\mathrm{y}\mathrm{z}$ nich praco-

waf oddzielnie, to drugi wykonałby calą pracę $03$ dni wcześniej od pierwszego. Ile dni

potrzebuje $\mathrm{k}\mathrm{a}\dot{\mathrm{z}}\mathrm{d}\mathrm{y}\mathrm{z}$ robotników na wykonanie calej pracy?

2. Narysowač na płaszczyz$\acute{}$nie zbiór $\{(x,y):\sqrt{x-1}+x\leq 2,0\leq y^{3}\leq\sqrt{5}-2\}$

jego pole. Wsk. Obliczyč $a=(\displaystyle \frac{\sqrt{5}-1}{2})^{3}$

i obliczyč

3. Obliczyč $a=\mathrm{t}\mathrm{g}\alpha, \mathrm{j}\mathrm{e}\dot{\mathrm{z}}$ eli $\displaystyle \sin\alpha-\cos\alpha=\frac{1}{5}\mathrm{i}\mathrm{k}\mathrm{a}\mathrm{t}\alpha$ spefnia nierównośč $\displaystyle \frac{\pi}{4}<\alpha<\frac{\pi}{2}$. Wyznaczyč

wysokośč trójk$\Phi$ta prostokątnego, $\mathrm{w}$ którym tangens jednego $\mathrm{z}$ k$\Phi$tów ostrych jest równy

$a$ a pole koła opisanego na tym trójkącie wynosi $25\pi.$

4. Kopufa Bazyliki $\acute{\mathrm{S}}\mathrm{w}$. Piotra $\mathrm{w}$ Watykanie ma ksztalt pólsfery $0$ promieniu 28 $\mathrm{m}$. Przed

rozpoczęciem prac renowacyjnych, na centralnie ustawionym rusztowaniu, umocowano

poziomą platformę $\mathrm{w}$ ksztalcie kola. Największa odległośč tej platformy od sklepienia

równa jest 2, 5 $\mathrm{m}$. a najmniejsza 1, 5 $\mathrm{m}$. Jaka jest powierzchnia tej platformy?

5. Trójmian kwadratowy $f(x)=\alpha x^{2}+bx+c$ przyjmuje najmniejszą wartośč równą $-2\mathrm{w}$

punkcie $x=2$ a reszta $\mathrm{z}$ dzielenia tego trójmianu przez dwumian $(x-1)$ równa jest 4.

Wyznaczyč współczynniki $a, b, c$. Narysowač staranny wykres funkcji $g(x) = f(|x|) \mathrm{i}$

wyznaczyč najmniejszq $\mathrm{i}$ najwiekszą wartośč tej funkcji na przedziale [$-1,3].$

6. Pani Zosia odcięfa $\mathrm{z}$ kwadratowego kawafka materiafu $0$ boku l $\mathrm{m}$ wszystkie cztery

narozniki $\mathrm{i}$ otrzymala serwetę $\mathrm{w}$ kształcie ośmiokąta foremnego. Postanowila wykończyč

ją szydelkową koronkq $0$ szerokości 5 cm.

a) Obliczyč dfugośč boku serwety przed $\mathrm{i}$ po jej wykończeniu.

b) Wiedząc, $\dot{\mathrm{z}}\mathrm{e}$ na zrobienie 100 centymetrów kwadratowych koronki potrzebny jest

jeden motek kordonku obliczyč, ile motków musi kupič Pani Zosia, $\mathrm{j}\mathrm{e}\dot{\mathrm{z}}$ eli powinna

uwzględnič 2\% straty materiafu podczas pracy.





PRACA KONTROLNA $\mathrm{n}\mathrm{r}4-$ POZIOM ROZSZERZONY

l. Do zbiornika poprowadzono trzy rury. Pierwsza rura potrzebuje do napełnienia zbiornika

$04$ godziny więcej $\mathrm{n}\mathrm{i}\dot{\mathrm{z}}$ druga, a trzecia napełnia cafy zbiornik $\mathrm{w}$ czasie dwa razy krótszym

$\mathrm{n}\mathrm{i}\dot{\mathrm{z}}$ pierwsza. Wjakim czasie napelnia zbiornik $\mathrm{k}\mathrm{a}\dot{\mathrm{z}}$ da $\mathrm{z}\mathrm{r}\mathrm{u}\mathrm{r}, \mathrm{j}\mathrm{e}\dot{\mathrm{z}}$ eli wiadomo, $\dot{\mathrm{z}}\mathrm{e}$ wszystkie

trzy rury otwarte jednocześnie napefniajq zbiornik $\mathrm{w}$ ciągu 2 godzin $\mathrm{i}40$ minut?

2. Stosując zasadę indukcji matematycznej wykazač prawdziwośč następującego wzoru dla

wszystkich $n\geq 1$

$\displaystyle \frac{1^{2}}{1\cdot 3}+\frac{2^{2}}{3\cdot 5}+\frac{3^{2}}{5\cdot 7}+\ldots+\frac{n^{2}}{(2n-1)(2n+1)}=\frac{n(n+1)}{2(2n+1)}$

3. Nie wykorzystujqc metod rachunku rózniczkowego wyznaczyč przedziały zawarte $\mathrm{w}[0,2\pi],$

na których funkcja

$ f(x)=\cos x+2\cos^{2}x+4\cos^{3}x+8\cos^{4}x+\ldots$

jest rosnąca.

4. Narysowač zbiór $\{(x,y):|x|+|y|\leq 6,|y|\leq 2^{|x|},|y|\geq\log_{2}|x|\}\mathrm{i}$ napisač równaniajego

osi symetrii. Podač odpowiednie uzasadnienie.

5. Pole przekroju ostrosłupa prawidlowego czworokątnego plaszczyznq przechodzącą przez

$\mathrm{P}^{\mathrm{r}\mathrm{z}\mathrm{e}\mathrm{k}}\Phi^{\mathrm{t}\mathrm{n}\text{ą}}$ podstawy $\mathrm{i}$ wierzcholek ostroslupa jest trójk$\Phi$tem równobocznym $0$ polu $S.$

Wyznaczyč stosunek promienia kuli wpisanej $\mathrm{w}$ ten ostrosłup do promienia kuli opisanej

na tym ostroslupie.

6. Punkt $A(1,2)$ jest wierzchołkiem trójkąta równobocznego. Wyznaczyč dwa pozostałe

wierzchołki tego trójkqta wiedząc, $\dot{\mathrm{z}}\mathrm{e}$ jeden $\mathrm{z}$ nich $\mathrm{l}\mathrm{e}\dot{\mathrm{z}}\mathrm{y}$ na prostej $x-y-1=0$, ajeden

$\mathrm{z}$ boków jest równolegly do wektora $\vec{v}= [-1,2]$. Obliczyč pole tego trójkąta. Ile jest

trójkątów spelniających warunki zadania?





PRACA KONTROLNA $\mathrm{n}\mathrm{r}5-$ POZIOM PODSTAWOWY

luty 2007r.

l. Bolek $\mathrm{i}$ Lolek $\mathrm{z}$ okazji swoich 9 $\mathrm{i} 11$ urodzin otrzymali od babci 200 zf do podziafu.

Umówili się, $\dot{\mathrm{z}}\mathrm{e}$ starszy otrzyma większą sumę, ale nie więcej $\mathrm{n}\mathrm{i}\dot{\mathrm{z}}0$ połowę od otrzymanej

przez brata, a ponadto średnia geometryczna obu kwot nie przekroczy iloczynu ich lat

$\dot{\mathrm{z}}$ ycia. Jaką maksymalną $\mathrm{i}$ minimalną kwotę $\mathrm{m}\mathrm{o}\dot{\mathrm{z}}\mathrm{e}$ otrzymač starszy brat.

2. Rozwazmy zbiór wszystkich ciagów binarnych $0$ dlugości 7. Wy1osowano jeden ciąg.

a) Jakie jest prawdopodobieństwo, $\dot{\mathrm{z}}\mathrm{e}$ bedzie zawieraf co najmniej 3 jedynki.

b) Jakie jest prawdopodobieństwo, $\dot{\mathrm{z}}\mathrm{e}\mathrm{w}$ tym ciągu wystqpi seria samych zer lub sa-

mych jedynek $0$ dfugości co najmniej 4.

3. $\mathrm{W}$ trójkącie $ABC$ dane są $\displaystyle \angle CAB=\frac{\pi}{3}$, wysokośč $|CD| =h=5$ oraz $|BD| =d=\sqrt{2}.$

Obliczyč promień okręgu wpisanego $\mathrm{w}$ ten trójkąt.

4. Na jednym rysunku przedstawič staranne wykresy funkcji $f(x) = |\displaystyle \sin(x-\frac{\pi}{9})|$ oraz

$g(x)=-\displaystyle \cos(x+\frac{5\pi}{18})$ na przedziale $I=[-\pi,2\pi].$

a) Odczytač $\mathrm{z}$ wykresu kąt $x_{0}$ taki, $\dot{\mathrm{z}}\mathrm{e}g(x)=\sin(x-x_{0}).$

b) Korzystając $\mathrm{z}$ wykresu oraz punktu a) wyznaczyč wszystkie kąty $x\in I$, dla których

$f(x)=g(x)$ oraz przedziafy, dla których $g(x)>f(x).$

5. Na walcu $0$ wysokości 6 cm $\mathrm{i}$ średnicy podstawy 16 cm opisano stozek $0$ kqcie rozwarcia

$ 2\alpha$ tak, $\dot{\mathrm{z}}\mathrm{e}$ podstawa walca $\mathrm{l}\mathrm{e}\dot{\mathrm{z}}\mathrm{y}$ na podstawie stozka, przy czym $\mathrm{t}\mathrm{g}\alpha= \displaystyle \frac{4}{3}$. Wyznaczyč

minimalne wymiary prostokąta ($\mathrm{z}$ zaokrągleniem $\mathrm{w}$ górę do pelnych cm), $\mathrm{w}$ którym

$\mathrm{m}\mathrm{o}\dot{\mathrm{z}}$ na zmieścič rozciętą powierzchnię boczną stozka $\mathrm{i}$ obliczyč jaki procent pola tego

prostokąta stanowi powierzchnia boczna stozka.

6. Dane są proste $k$ : $2x-3y+6=0$ oraz $l$ : $2x+4y-7=0$. Na prostej $k$ znalez$\acute{}$č punkt,

którego obraz symetryczny względem prostej $l\mathrm{l}\mathrm{e}\dot{\mathrm{z}}\mathrm{y}$ na osi $\mathrm{O}y$. Sporządzič rysunek.





PRACA KONTROLNA $\mathrm{n}\mathrm{r}5-$ POZIOM ROZSZERZONY

l. Stosując zasadę indukcji matematycznej wykazač, $\dot{\mathrm{z}}\mathrm{e}$ liczba $7^{n}-(-3)^{n}$ jest podzielna

przez 10 d1a $\mathrm{k}\mathrm{a}\dot{\mathrm{z}}$ dego naturalnego $n.$

2. Rozwiązač nierównośč 4 logl6 $\cos 2x+2\log_{4}\sin x+\log_{2}\cos x+3<0$ dla $x\displaystyle \in(0,\frac{\pi}{4}).$

3. Róznica ciqgu arytmetycznego $(a_{n})$ jest liczbq mniejszq od l. Wyznaczyč najmniejszą

wartośč wyrazenia $\displaystyle \frac{a_{1}a49}{a_{50}}, \mathrm{w}\mathrm{i}\mathrm{e}\mathrm{d}\mathrm{z}\Phi^{\mathrm{C}}, \dot{\mathrm{z}}\mathrm{e}a_{51}=1.$

4. Cięciwa paraboli $0$ równaniu $y=-a^{2}x^{2}+5ax-4$ jest styczna do krzywej $y=\displaystyle \frac{1}{-x+1}$

$\mathrm{w}$ punkcie $0$ odciętej $x_{o}=2$, który dzieli $\mathrm{t}\mathrm{e}$ cięciwę na połowy. Wyznaczyč parametr $a.$

Podač ilustrację graficzną rozwiązania zadania.

5. Dana jest funkcja $f(x)=\displaystyle \frac{2x^{2}}{(2-x)^{2}}.$

a) Zbadač przebieg zmienności funkcji $f\mathrm{i}$ naszkicowač jej wykres.

b) Sporządzič wykres funkcji $k=g(m)$, gdzie $k$ jest liczbą rozwi$\Phi$zań równania

$\displaystyle \frac{2x^{2}}{(2-|x|)^{2}}=m$

$\mathrm{w}$ zalezności od parametru rzeczywistego $m.$

6. $\mathrm{W}$ kulę $0$ promieniu $R$ wpisano stozek, $\mathrm{w}$ którym tworząca jest równa średnicy pod-

stawy. Obydwie bryły przecieto płaszczyzną równoległą do podstawy stozka. Szerokośč

otrzymanego $\mathrm{w}$ przecięciu pierścienia kofowego zawartego między powierzchnią kulistą

a powierzchnią boczną stozka równa się $m.$

a) Znalez/č odlegfośč pfaszczyzny tnącej od wierzchołka stozka.

b) Przedyskutowač liczbę rozwiązań $\mathrm{w}$ zalezności od $m\mathrm{i}$ podač interpretację geome-

tryczną przypadków szczególnych.



\end{document}