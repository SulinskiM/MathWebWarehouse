\documentclass[a4paper,12pt]{article}
\usepackage{latexsym}
\usepackage{amsmath}
\usepackage{amssymb}
\usepackage{graphicx}
\usepackage{wrapfig}
\pagestyle{plain}
\usepackage{fancybox}
\usepackage{bm}

\begin{document}

XXXVI

KORESPONDENCYJNY KURS Z MATEMATYKI

PRACA KONTROLNA $\mathrm{n}\mathrm{r}1-$ POZIOM PODSTAWOWY

$\mathrm{p}\mathrm{a}\acute{\mathrm{z}}$dziernik 2$006\mathrm{r}.$

l. Róznica pewnej liczby trzycyfrowej $\mathrm{i}$ liczby otrzymanej za pomocą tych samych cyfr

zapisanych $\mathrm{w}$ odwrotnej kolejności równa jest 495, a suma równa jest 1009. Jaka to

liczba.

2. Obliczyč $p=\displaystyle \frac{64^{\frac{1}{3}}\sqrt{8}+8^{\frac{1}{3}}\sqrt{64}}{\sqrt[3]{64\sqrt{8}}}$. Znalez/č wszystkie liczby naturalne, dla których spełniona

jest nierównośč $x^{3}-2x^{2}-p^{2}x+2p^{2}\leq 0.$

3. Polowę kolekcji letniej sprzedano po zafozonej cenie. Po obnizce ceny $0$ 50\% udalo się

sprzedač pofowę pozostalej części towaru $\mathrm{i}$ dopiero kolejna 50\%-owa obnizka pozwo1ifa

sklepowi pozbyč się produktu.

a) Ile procent zaplanowanego przychodu stanowi uzyskana ze sprzedaz $\mathrm{y}$ kwota?

b) $\mathrm{O}$ ile procent wyjściowa cena towaru powinna byla byč $\mathrm{w}\mathrm{y}\dot{\mathrm{z}}$ sza, by sklep uzyskaf

zaplanowany początkowo przychód? Wyniki podač $\mathrm{z}$ dokładnością do l promila.

4. Dach wiezy kościola ma ksztalt ostrosfupa, którego podstawq jest sześciokąt foremny $0$

boku 2 $\mathrm{m}$ a największy $\mathrm{z}$ przekrojów pfaszczyzną $\mathrm{z}\mathrm{a}\mathrm{w}\mathrm{i}\mathrm{e}\mathrm{r}\mathrm{a}\mathrm{j}_{\Phi}\mathrm{c}\text{ą}$ wysokośč jest trójkątem

równobocznym. Obliczyč kubaturę dachu wiezy kościoła. Ile 2-1itrowych puszek farby

antykorozyjnej trzeba kupič do pomalowania blachy, którą pokryty jest dach, $\mathrm{j}\mathrm{e}\dot{\mathrm{z}}$ eli wia-

domo, $\dot{\mathrm{z}}\mathrm{e} 1$ litr farby wystarcza do pomalowania 6 $\mathrm{m}^{2}$ blachy $\mathrm{i}$ trzeba uwzględnič 8\%

farby na ewentualne straty.

5. Niech

$f(x)=$

dla

dla

$x\leq 1,$

$x>1.$

a) Narysowač wykres funkcji $f\mathrm{i}$ na jego podstawie wyznaczyč zbiór wartości funkcji.

b) Obliczyč $f(\sqrt{3}-1)$ oraz $f(3-\sqrt{3}).$

c) Rozwiqzač nierównośč $2\sqrt{f(x)}\leq 3\mathrm{i}$ zaznaczyč na osi $\mathrm{o}x$ zbiór rozwiązań.

6. Punkt $A=(1,0)$ jest wierzchofkiem rombu $0$ kącie przy tym wierzchołku równym $60^{\mathrm{o}}$

Wyznaczyč wspófrzędne pozostafych wierzchofków rombu wiedząc, $\dot{\mathrm{z}}\mathrm{e}$ dwa $\mathrm{z}$ nich lezą

na prostej $l$ : $2x-y+3=0$. Obliczyč pole rombu. Ile rozwiązań ma to zadanie?
\end{document}
