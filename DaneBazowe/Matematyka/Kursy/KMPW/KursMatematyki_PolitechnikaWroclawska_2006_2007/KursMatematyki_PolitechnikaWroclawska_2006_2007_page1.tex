\documentclass[a4paper,12pt]{article}
\usepackage{latexsym}
\usepackage{amsmath}
\usepackage{amssymb}
\usepackage{graphicx}
\usepackage{wrapfig}
\pagestyle{plain}
\usepackage{fancybox}
\usepackage{bm}

\begin{document}

PRACA KONTROLNA $\mathrm{n}\mathrm{r}1-$ POZIOM ROZSZERZONY

l. Rozwiązač nierównośč $\displaystyle \frac{1}{\sqrt{4-x^{2}}}\geq\frac{1}{x-1}\mathrm{i}$ starannie zaznaczyč zbiór rozwiqzań na osi liczbo-

wej.

2. Rozwiązač równanie 2 $\sin 2x+2\sin x-2\cos x=1$. Następnie podač rozwiązania nalezące

do przedziału $[-\pi,\pi].$

3. $\mathrm{Z}$ przystani A wyrusza $\mathrm{z}$ biegiem rzeki statek do przystani $\mathrm{B}$, odlegfej od A $0140$ km. Po

upływie l godziny wyrusza za nim łódz$\acute{}$ motorowa, dopędza statek, po czym wraca do

przystani A $\mathrm{w}$ tym samym momencie, $\mathrm{w}$ którym statek przybija do przystani B. Znalez$\acute{}$č

prędkośč biegu rzeki, $\mathrm{j}\mathrm{e}\dot{\mathrm{z}}$ eli wiadomo, $\dot{\mathrm{z}}\mathrm{e}\mathrm{w}$ stojącej wodzie prędkośč statku wynosi 16

$\mathrm{k}\mathrm{m}/$godz, a prędkośč łodzi 24 $\mathrm{k}\mathrm{m}/$godz.

4. Dane są liczby: $m=\displaystyle \frac{(_{4}^{6})\cdot(_{2}^{8})}{(_{3}^{7})}, n=\displaystyle \frac{(\sqrt{2})^{-4}(\frac{1}{4})^{-\frac{5}{2}}\sqrt[4]{3}}{(\sqrt[4]{16})_{27^{-\frac{1}{4}}}^{3}}.$

a) Sprawdzič, wykonując odpowiednie obliczenia, $\dot{\mathrm{z}}\mathrm{e}m, n$ są liczbami naturalnymi.

b) Wyznaczyč $k\mathrm{t}\mathrm{a}\mathrm{k}$, by liczby $m, k, n$ były odpowiednio: pierwszym, drugim $\mathrm{i}$ trzecim

wyrazem ciągu geometrycznego.

c) Wyznaczyč sumę wszystkich wyrazów nieskończonego ciągu geometrycznego, któ-

rego pierwszymi trzema wyrazami są $m, k, n$. Ile wyrazów tego ciągu nalez $\mathrm{y}$ wziąč,

by ich suma przekroczyła 95\% sumy wszystkich wyrazów?

5. $\mathrm{Z}$ wierzchofka $A$ kwadratu ABCD $0$ boku $a$ poprowadzono dwie proste, które dzielą kąt

przy tym wierzchołku na trzy równe części $\mathrm{i}$ przecinają boki kwadratu $\mathrm{w}$ punktach $K\mathrm{i}$

$L$. Wyznaczyč długości odcinków, najakie te proste dzielą przekątną kwadratu. Znalez$\acute{}$č

promień okręgu wpisanego $\mathrm{w}$ deltoid AKCL.

6. Podstawą pryzmy przedstawionej na rysunku ponizej jest prostokąt ABCD,
\begin{center}
\includegraphics[width=87.432mm,height=33.528mm]{./KursMatematyki_PolitechnikaWroclawska_2006_2007_page1_images/image001.eps}
\end{center}
{\it K} $\mathrm{L}$

C

$\mathrm{b}$

zmy.

A a $\mathrm{B}$

gosč $b$, gdzie $a>b$. Wszystkie ściany boczne

pryzmy $\mathrm{s}$ nachylone pod $\mathrm{k}$ tem $\alpha$ do płasz-

czyzny podstawy. Obliczyc objętosc tej pry-

ktorego bok $AB$ ma dlugośc $a$, a bok $BC$ dfu-
\end{document}
