\documentclass[a4paper,12pt]{article}
\usepackage{latexsym}
\usepackage{amsmath}
\usepackage{amssymb}
\usepackage{graphicx}
\usepackage{wrapfig}
\pagestyle{plain}
\usepackage{fancybox}
\usepackage{bm}

\begin{document}

PRACA KONTROLNA $\mathrm{n}\mathrm{r}6-$ POZIOM PODSTAWOWY

marzec 2007r.

l. Boki trójk$\Phi$ta prostokątnego $0$ polu 12 $\mathrm{t}\mathrm{w}\mathrm{o}\mathrm{r}\mathrm{z}\Phi$ ciąg arytmetyczny. Wyznaczyč promień

okręgu wpisanego $\mathrm{w}$ ten trójkąt.

2. Pan Kowalski zaciągnąf 3l grudnia $\mathrm{p}\mathrm{o}\dot{\mathrm{z}}$ yczkę 4000 z1otych oprocentowaną $\mathrm{w}$ wysokości

18\% $\mathrm{w}$ skali roku. Zobowiązaf się splacič ją $\mathrm{w}$ ciągu roku $\mathrm{w}$ trzech równych ratach

płatnych 30 kwietnia, 30 sierpnia $\mathrm{i}30$ grudnia. Oprocentowanie $\mathrm{p}\mathrm{o}\dot{\mathrm{z}}$ yczki liczy się od l

stycznia, a odsetki od kredytu naliczane są $\mathrm{w}$ terminach pfatności rat. Obliczyč wysokośč

tych rat $\mathrm{w}$ zaokrągleniu do pełnych groszy.

3. Narysowač wykres funkcji $f(x)=$

$\mathrm{i}$ na jego podstawie wyznaczyč:

dla

dla

dla

$x<0,$

$x=0,$

$x>0,$

a) zbiór, jaki tworzą wartości funkcji $f(x)$, gdy $x$ przebiega przedzial $(-2,1)$ ;

b) zbiór rozwi$\Phi$zań nierówności $\displaystyle \frac{1}{2}\leq f(x)\leq 2.$

4. Suma wysokości $h$ ostrosłupa prawidłowego czworokątnego $\mathrm{i}$ jego krawędzi bocznej $b$

równa jest 12. D1a jakiej wartości $h$ objętośč tego ostroslupa jest najwieksza? Obliczyč

pole powierzchni cafkowitej ostrosfupa dla tej wartości $h.$

5. Punkty $A(0,4) \mathrm{i}D(3,5)$ są wierzchołkami trapezu równoramiennego ABCD, którego

podstawy $\overline{AB}$ oraz $\overline{CD}$ są prostopadfe do prostej $k\mathrm{o}$ równaniu $x-y-2=0$. Wyznaczyč

wspólrzędne pozostałych wierzchołków wiedząc, $\dot{\mathrm{z}}\mathrm{e}$ wierzchołek $C \mathrm{l}\mathrm{e}\dot{\mathrm{z}}\mathrm{y}$ na prostej $k.$

Znalez$\acute{}$č współrzędne środka oraz promień okręgu opisanego na tym trapezie.

6. Na kole $0$ promieniu $r$ opisano romb. Punkty styczności są wierzcholkami $\mathrm{c}\mathrm{z}\mathrm{w}\mathrm{o}\mathrm{r}\mathrm{o}\mathrm{k}_{\Phi}\mathrm{t}\mathrm{a}$

ABCD. Zakładając, $\dot{\mathrm{z}}\mathrm{e}$ stosunek pola rombu do pola czworokqta równy jest $\displaystyle \frac{8}{3}$, obliczyč

dlugośč boku rombu ijego $\mathrm{p}\mathrm{r}\mathrm{z}\mathrm{e}\mathrm{k}_{\Phi^{\mathrm{t}}}$nych. Obliczyč pole jednego $\mathrm{z}$ obszarów ograniczonych

bokami rombu $\mathrm{i}$ okręgiem.
\end{document}
