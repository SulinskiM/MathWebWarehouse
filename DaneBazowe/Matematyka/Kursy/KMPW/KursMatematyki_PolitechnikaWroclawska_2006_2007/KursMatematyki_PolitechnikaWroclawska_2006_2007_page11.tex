\documentclass[a4paper,12pt]{article}
\usepackage{latexsym}
\usepackage{amsmath}
\usepackage{amssymb}
\usepackage{graphicx}
\usepackage{wrapfig}
\pagestyle{plain}
\usepackage{fancybox}
\usepackage{bm}

\begin{document}

PRACA KONTROLNA $\mathrm{n}\mathrm{r}6-$ POZIOM ROZSZERZONY

l. Dla jakich wartości parametru $\alpha\in[0,2\pi]$ istnieje dodatnie maksimum funkcji

$ f(x)=(2\cos\alpha-1)x^{2}-2x+\cos\alpha$ ?

2. Granicą ciągu $0$ wyrazie ogólnym $a_{n}=\displaystyle \frac{\sqrt{n^{4}+an^{3}+bn}-n^{2}}{\sqrt{n^{2}+1}}$ jest większy $\mathrm{z}$ pierwiastków

równania $4x^{\log x}+10x^{-\log x}=41$. Wyznaczyč parametry a $\mathrm{i}b.$

3. Wyznaczyč równanie krzywej utworzonej przez punkty, których odlegfośč od osi $0x$ jest

taka sama, jak odległośč od pólokręgu $0$ równaniu $y=\sqrt{2x-x^{2}}$. Sporzqdzič rysunek.

4. $\mathrm{W}$ stozku ściętym $\mathrm{P}^{\mathrm{r}\mathrm{z}\mathrm{e}\mathrm{k}}\Phi^{\mathrm{t}\mathrm{n}\mathrm{e}}$ przekroju osiowego $\mathrm{p}\mathrm{r}\mathrm{z}\mathrm{e}\mathrm{c}\mathrm{i}\mathrm{n}\mathrm{a}\mathrm{j}_{\Phi}$ się pod $\mathrm{k}_{\Phi}\mathrm{t}\mathrm{e}\mathrm{m}$ prostym, $\mathrm{a}$

tworząca $0$ dfugości $l$ nachylona jest do płaszczyzny podstawy dolnej pod kątem $\alpha.$

Obliczyč pole powierzchni bocznej tego stozka ściętego oraz pole powierzchni opisanej

na nim kuli.

5. $\mathrm{W}$ trójkącie $\triangle ABC$ dane są podstawa $|AB|=a$, kąt ostry przy podstawie $\angle CAB=2\alpha$

$\mathrm{i}$ dwusieczna tego kąta $|AD|=d$. Obliczyč pole koła opisanego na tym trójkącie. Podač

warunek istnienia rozwiązania.

6. Zbadač przebieg zmienności funkcji określonej wzorem

$f(x)=\displaystyle \sqrt{x+1}+1+\frac{1}{\sqrt{x+1}}+\ldots,$

gdzie prawa stronajest sumą wyrazów nieskończonego ciągu geometrycznego. Narysowač

jej staranny wykres.
\end{document}
