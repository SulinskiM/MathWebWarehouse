\documentclass[a4paper,12pt]{article}
\usepackage{latexsym}
\usepackage{amsmath}
\usepackage{amssymb}
\usepackage{graphicx}
\usepackage{wrapfig}
\pagestyle{plain}
\usepackage{fancybox}
\usepackage{bm}

\begin{document}

PRACA KONTROLNA $\mathrm{n}\mathrm{r}2-$ POZIOM PODSTAWOWY

listopad $2006\mathrm{r}.$

l. Liczba dwuelementowych podzbiorów zbioru $A$ jest 7 razy większa $\mathrm{n}\mathrm{i}\dot{\mathrm{z}}$ liczba dwuele-

mentowych podzbiorów zbioru $B$. Liczba dwuelementowych podzbiorów zbioru $A$ nie

zawierających ustalonego elementu $a\in A$ jest 5 razy większa $\mathrm{n}\mathrm{i}\dot{\mathrm{z}}$ liczba dwuelemento-

wych podzbiorów zbioru $B$. Ile elementów ma $\mathrm{k}\mathrm{a}\dot{\mathrm{z}}\mathrm{d}\mathrm{y}\mathrm{z}$ tych zbiorów? Ile $\mathrm{k}\mathrm{a}\dot{\mathrm{z}}\mathrm{d}\mathrm{y}\mathrm{z}$ tych

zbiorów ma podzbiorów trzyelementowych?

2. $A\cap B, A\backslash B\mathrm{i}B\backslash $apisa '$\mathrm{w}\mathrm{p}$ostaciNiech {\it A}$=\displaystyle \{x\in 1\mathrm{R}:\frac{1}{x^{2}+23,A\mathrm{z}}\geq\frac{1}{10x,\mathrm{C}}\}$oraz {\it B}$=\mathrm{p}\mathrm{r}!^{x\in 1\mathrm{R}:|x-2|<\frac{7}{2}\}.\mathrm{Z}\mathrm{b}\mathrm{i}\mathrm{o}\mathrm{r}\mathrm{y}A,B,A\cup B}\mathrm{e}\mathrm{d}\mathrm{z}\mathrm{i}\mathrm{a}1\text{ó} \mathrm{w}1$iczbowych izaznaczyč j$\mathrm{e}\mathrm{n}\mathrm{a}\mathrm{o}\mathrm{s}\mathrm{i}$

liczb owej.

3. Stosując wzory skróconego mnozenia sprowadzič do najprostszej postaci wyrazenie

$W=2$ (sin6 $\alpha+\cos^{6}\alpha$)$-(\sin^{4}\alpha+\cos^{4}\alpha).$

Wykorzystując wzór $\cos 2\alpha = \cos^{2}\alpha-\sin^{2}\alpha$

wyrazenie $W$ przyjmuje wartośč $\displaystyle \frac{1}{2}.$

obliczyč, dla jakich wartości kąta $\alpha$

4. Wiadomo, $\dot{\mathrm{z}}\mathrm{e}$ liczby $-1$, 3 są pierwiastkami wielomianu $W(x)=x^{4}-ax^{3}-4x^{2}+bx+3.$

Wyznaczyč $a, b\mathrm{i}$ rozwiązač nierównośč $\sqrt{W(x)}\leq x^{2}-x.$

5. Na kole $0$ promieniu $r$ opisano trapez równoramienny, $\mathrm{w}$ którym stosunek dlugości pod-

staw wynosi 4: 3. Ob1iczyč stosunek po1a kofa do po1a trapezu oraz cosinus kąta ostrego

$\mathrm{w}$ tym trapezie.

6. $\mathrm{W}$ ostroslupie prawidłowym $\mathrm{c}\mathrm{z}\mathrm{w}\mathrm{o}\mathrm{r}\mathrm{o}\mathrm{k}_{\Phi^{\mathrm{t}}}\mathrm{n}\mathrm{y}\mathrm{m}$ wszystkie krawędzie $\mathrm{s}\Phi$ równe $a$. Obliczyč

objętośč tego ostroslupa. Znalez/č cosinus kąta nachylenia ściany bocznej do podstawy

oraz cosinus kata między ścianami bocznymi tego ostrosłupa.
\end{document}
