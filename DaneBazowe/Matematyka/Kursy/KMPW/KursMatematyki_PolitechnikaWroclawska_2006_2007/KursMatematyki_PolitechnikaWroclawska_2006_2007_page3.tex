\documentclass[a4paper,12pt]{article}
\usepackage{latexsym}
\usepackage{amsmath}
\usepackage{amssymb}
\usepackage{graphicx}
\usepackage{wrapfig}
\pagestyle{plain}
\usepackage{fancybox}
\usepackage{bm}

\begin{document}

PRACA KONTROLNA $\mathrm{n}\mathrm{r}2-$ POZIOM ROZSZERZONY

l. Trzeci składnik rozwinięcia dwumianu $(\displaystyle \sqrt[3]{x}+\frac{1}{\sqrt{x}})^{n}$ ma współczynnik równy 45. Wyzna-

czyč wszystkie skladniki tego rozwinięcia, $\mathrm{w}$ których $x$ występuje $\mathrm{w}$ potędze $0$ wykfadniku

całkowitym.

2. Niech $A=\{(x,y):y\geq||x-2|-1|\}, B=\{(x,y):y+\sqrt{4x-x^{2}-3}\leq 2\}$. Narysowač

na pfaszczy $\acute{\mathrm{z}}\mathrm{n}\mathrm{i}\mathrm{e}$ zbiór $A\cap B\mathrm{i}$ obliczyč jego pole.

3. Niech $a_{n}=\displaystyle \frac{1+kn}{5+k^{2}n}.$

a) Określič monotonicznośč ciągu $(a_{n})\mathrm{w}$ zalezności od parametru $k.$

b) Niech $S(k)$ oznacza sumę nieskończonego ciągu geometrycznego $0$ pierwszym wyra-

zie $a_{1}=1 \mathrm{i}$ ilorazie $q_{k}=\displaystyle \lim_{n\rightarrow\infty}a_{n}$. Sporządzič wykres funkcji $S(k)\mathrm{i}$ na tej podstawie

wyznaczyč zbiór jej wartości.

4. Dana jest funkcja $f(x)=\cos x$. Wyznaczyč dziedzinę oraz zbiór wartości funkcji

$g(x)=\sqrt{f(\frac{\pi}{2}-x)+\sqrt{3}f(x)-1}.$

5. $\mathrm{C}\mathrm{z}\mathrm{w}\mathrm{o}\mathrm{r}\mathrm{o}\mathrm{k}_{\Phi^{\mathrm{t}}}$ wypukly ABCD, $\mathrm{w}$ którym $AB=1, BC=2, CD=4, DA=3$ jest wpisany

$\mathrm{w}$ okrąg. Obliczyč promień $R$ tego okręgu. Sprawdzič, czy $\mathrm{w}$ czworokąt ten $\mathrm{m}\mathrm{o}\dot{\mathrm{z}}$ na wpisač

$\mathrm{o}\mathrm{k}\mathrm{r}\Phi \mathrm{g}. \mathrm{J}\mathrm{e}\dot{\mathrm{z}}$ eli $\mathrm{t}\mathrm{a}\mathrm{k}$, to obliczyč promień $r$ tego okręgu.

6. Plaszczyzna przechodząca przez jeden $\mathrm{z}$ wierzcholków czworościanu foremnego $\mathrm{i}$ rów-

noległa do jednej $\mathrm{z}$ jego krawędzi dzieli ten czworościan na dwie bryły $0$ takiej samej

objętości. Wyznaczyč pole przekroju oraz cosinus kąta nachylenia tego przekroju do

plaszczyzny podstawy.
\end{document}
