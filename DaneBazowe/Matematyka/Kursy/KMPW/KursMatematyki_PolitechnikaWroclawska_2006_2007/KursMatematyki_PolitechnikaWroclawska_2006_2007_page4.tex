\documentclass[a4paper,12pt]{article}
\usepackage{latexsym}
\usepackage{amsmath}
\usepackage{amssymb}
\usepackage{graphicx}
\usepackage{wrapfig}
\pagestyle{plain}
\usepackage{fancybox}
\usepackage{bm}

\begin{document}

PRACA KONTROLNA $\mathrm{n}\mathrm{r}3-$ POZIOM PODSTAWOWY

grudzień $2006\mathrm{r}.$

1. $\mathrm{Z}$ talii 24 kart wy1osowano dwie. Jakie jest prawdopodobieństwo, $\dot{\mathrm{z}}\mathrm{e}$ obie $\mathrm{s}\Phi$ koloru czer-

wonego lub obie są figurami?

2. Panowie X $\mathrm{i}\mathrm{Y}$ zafozyli jednocześnie firmy $\mathrm{i}\mathrm{w}$ pierwszym miesiącu dziafalności $\mathrm{k}\mathrm{a}\dot{\mathrm{z}}$ da

$\mathrm{z}$ nich miała obrot równy 50000 zfotych. Po pięciu miesiącach okazafo się, $\dot{\mathrm{z}}\mathrm{e}$ obrót

firmy pana X rósł $\mathrm{z}$ miesiąca na miesiąc $0$ tę samą kwotę, a obrót firmy pana $\mathrm{Y}$ rósł co

miesiąc $\mathrm{w}$ postępie geometrycznym. Stwierdzili równiez, $\dot{\mathrm{z}}\mathrm{e}\mathrm{w}$ drugim $\mathrm{i}$ trzecim miesiącu

działalności firma pana X miała obrót większy od obrotu firmy pana $\mathrm{Y}\mathrm{o}$ 2000 zł.

a) Jakie były obroty $\mathrm{k}\mathrm{a}\dot{\mathrm{z}}$ dej $\mathrm{z}$ firm $\mathrm{w}$ pieciu początkowych miesiącach?

b) Która $\mathrm{z}$ firm miała większą sumę obrotów $\mathrm{w}$ pierwszych pięciu miesiącach $\mathrm{i}\mathrm{o}$ ile?

c) Po ilu miesiącach obrót jednej $\mathrm{z}$ firm (której?) przekroczy dwukrotnie obrót drugiej

firmy?

3. Tangens kąta ostrego $\alpha$ równy jest $\displaystyle \frac{a}{b}$, gdzie

$\alpha=(\sqrt{2+\sqrt{3}}-\sqrt{2-\sqrt{3}})^{2}b=(\sqrt{\sqrt{2}+1}-\sqrt{\sqrt{2}-1})^{2}$

Wyznaczyč wartości pozostałych funkcji trygonometrycznych tego kata. Wykorzystując

wzór $\sin 2\alpha=2\sin\alpha\cos\alpha$, obliczyč miarę kąta $\alpha.$

4. Narysowač wykres funkcji $f(x)=|2x-4|-\sqrt{x^{2}+4x+4}$. Dlajakiego $m$ pole trójkąta

ograniczonego wykresem funkcji $f$ oraz prostą $y=m$ równe jest 6?

5. Harcerze rozbili 2 namioty, jeden $\mathrm{w}$ odległości 5 $\mathrm{m}$, drugi - 17 $\mathrm{m}$ od prostoliniowego

brzegu rzeki. Odległośč między namiotami równajest 13 $\mathrm{m}. \mathrm{W}$ którym miejscu $\mathrm{n}\mathrm{a}$ samym

brzegu rzeki (licząc od punktu brzegu będqcego rzutem prostopadfym punktu polozenia

pierwszego namiotu) powinni umieścič maszt $\mathrm{z}$ flagą zastępu, by odległośč od masztu do

$\mathrm{k}\mathrm{a}\dot{\mathrm{z}}$ dego $\mathrm{z}$ namiotów byfa taka sama?

6. Wysokośč ostrosłupa trójkątnego prawidłowego wynosi $h$, a kąt między wysokościami

ścian bocznych poprowadzonymi $\mathrm{z}$ wierzchołka ostrosfupa jest równy $ 2\alpha$. Obliczyč pole

powierzchni bocznej $\mathrm{i}$ objętośč tego ostrosfupa.
\end{document}
