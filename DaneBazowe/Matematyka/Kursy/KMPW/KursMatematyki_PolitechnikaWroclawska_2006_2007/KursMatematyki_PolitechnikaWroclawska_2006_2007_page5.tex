\documentclass[a4paper,12pt]{article}
\usepackage{latexsym}
\usepackage{amsmath}
\usepackage{amssymb}
\usepackage{graphicx}
\usepackage{wrapfig}
\pagestyle{plain}
\usepackage{fancybox}
\usepackage{bm}

\begin{document}

PRACA KONTROLNA $\mathrm{n}\mathrm{r}3-$ POZIOM ROZSZERZONY

l. Dlajakich wartości rzeczywistego parametru $p$ równanie $(p-2)x^{2}-(p+1)x-p=0$ ma

dwa rózne pierwiastki: a) ujemne? b) będące sinusem $\mathrm{i}$ cosinusem tego samego kąta?

2. Jakie powinny byč wymiary puszki $\mathrm{w}$ kształcie walca $0$ pojemności jednego litra, by jej

pole powierzchni całkowitej bylo najmniejsze?

3. $\mathrm{Z}$ badań statystycznych wynika,$\dot{\mathrm{z}}\mathrm{e}$ 5\% $\mathrm{m}\text{ę}\dot{\mathrm{z}}$ czyzn $\mathrm{i}$ 0,2\% kobiet to daltoniści. Wiadomo,

$\dot{\mathrm{z}}\mathrm{e}$ 55\% mieszkańców Wrocławia stanowia kobiety. Jakie jest prawdopodobieństwo, $\dot{\mathrm{z}}\mathrm{e}$

wśród 31osowo wybranych osób przynajmniej dwie nie odrózniaj$\Phi$ ko1orów?

4. Rozwiązač nierównośč $\displaystyle \log_{x}\frac{2-7x}{2x-7}\geq a$, gdzie $a$ jest granicą ciagu $0$ wyrazach

$a_{n}=\displaystyle \frac{4n(\sqrt{n^{2}+n}-n)}{n+1}.$

5. Pary liczb spefniające uklad równań

$\left\{\begin{array}{l}
-4x^{2}+y^{2}+2y+1=0,\\
-x^{2}+y+4=0
\end{array}\right.$

są wspólrzędnymi wierzchofków czworokata wypukfego ABCD.

a) Wykazač, $\dot{\mathrm{z}}\mathrm{e}$ czworokąt ABCD jest trapezem równoramiennym.

b) Wyznaczyč równanie okręgu opisanego na czworokącie ABCD.

6. Piramida utworzona z pięciu kul, z których cztery maja taki sam promień, jest wpisana

w walec. Przekrój osiowy walca jest kwadratem 0 boku d. Wyznaczyč promienie tych

kul.
\end{document}
