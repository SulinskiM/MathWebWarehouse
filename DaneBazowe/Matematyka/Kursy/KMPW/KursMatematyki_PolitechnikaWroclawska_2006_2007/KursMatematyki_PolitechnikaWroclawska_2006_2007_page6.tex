\documentclass[a4paper,12pt]{article}
\usepackage{latexsym}
\usepackage{amsmath}
\usepackage{amssymb}
\usepackage{graphicx}
\usepackage{wrapfig}
\pagestyle{plain}
\usepackage{fancybox}
\usepackage{bm}

\begin{document}

PRACA KONTROLNA $\mathrm{n}\mathrm{r}4-$ POZIOM PODSTAWOWY

styczeń $2007\mathrm{r}.$

l. Dwóch robotników $\mathrm{m}\mathrm{o}\dot{\mathrm{z}}\mathrm{e}$ razem wykonač $\mathrm{P}^{\mathrm{e}\mathrm{w}\mathrm{n}}\Phi$ pracę $\mathrm{w}\mathrm{c}\mathrm{i}_{\Phi \mathrm{g}}\mathrm{u}7$ dni pod warunkiem, $\dot{\mathrm{z}}\mathrm{e}$

pierwszy $\mathrm{z}$ nich rozpocznie pracę $0$ póltora dnia wcześniej Gdyby $\mathrm{k}\mathrm{a}\dot{\mathrm{z}}\mathrm{d}\mathrm{y}\mathrm{z}$ nich praco-

waf oddzielnie, to drugi wykonałby calą pracę $03$ dni wcześniej od pierwszego. Ile dni

potrzebuje $\mathrm{k}\mathrm{a}\dot{\mathrm{z}}\mathrm{d}\mathrm{y}\mathrm{z}$ robotników na wykonanie calej pracy?

2. Narysowač na płaszczyz$\acute{}$nie zbiór $\{(x,y):\sqrt{x-1}+x\leq 2,0\leq y^{3}\leq\sqrt{5}-2\}$

jego pole. Wsk. Obliczyč $a=(\displaystyle \frac{\sqrt{5}-1}{2})^{3}$

i obliczyč

3. Obliczyč $a=\mathrm{t}\mathrm{g}\alpha, \mathrm{j}\mathrm{e}\dot{\mathrm{z}}$ eli $\displaystyle \sin\alpha-\cos\alpha=\frac{1}{5}\mathrm{i}\mathrm{k}\mathrm{a}\mathrm{t}\alpha$ spefnia nierównośč $\displaystyle \frac{\pi}{4}<\alpha<\frac{\pi}{2}$. Wyznaczyč

wysokośč trójk$\Phi$ta prostokątnego, $\mathrm{w}$ którym tangens jednego $\mathrm{z}$ k$\Phi$tów ostrych jest równy

$a$ a pole koła opisanego na tym trójkącie wynosi $25\pi.$

4. Kopufa Bazyliki $\acute{\mathrm{S}}\mathrm{w}$. Piotra $\mathrm{w}$ Watykanie ma ksztalt pólsfery $0$ promieniu 28 $\mathrm{m}$. Przed

rozpoczęciem prac renowacyjnych, na centralnie ustawionym rusztowaniu, umocowano

poziomą platformę $\mathrm{w}$ ksztalcie kola. Największa odległośč tej platformy od sklepienia

równa jest 2, 5 $\mathrm{m}$. a najmniejsza 1, 5 $\mathrm{m}$. Jaka jest powierzchnia tej platformy?

5. Trójmian kwadratowy $f(x)=\alpha x^{2}+bx+c$ przyjmuje najmniejszą wartośč równą $-2\mathrm{w}$

punkcie $x=2$ a reszta $\mathrm{z}$ dzielenia tego trójmianu przez dwumian $(x-1)$ równa jest 4.

Wyznaczyč współczynniki $a, b, c$. Narysowač staranny wykres funkcji $g(x) = f(|x|) \mathrm{i}$

wyznaczyč najmniejszq $\mathrm{i}$ najwiekszą wartośč tej funkcji na przedziale [$-1,3].$

6. Pani Zosia odcięfa $\mathrm{z}$ kwadratowego kawafka materiafu $0$ boku l $\mathrm{m}$ wszystkie cztery

narozniki $\mathrm{i}$ otrzymala serwetę $\mathrm{w}$ kształcie ośmiokąta foremnego. Postanowila wykończyč

ją szydelkową koronkq $0$ szerokości 5 cm.

a) Obliczyč dfugośč boku serwety przed $\mathrm{i}$ po jej wykończeniu.

b) Wiedząc, $\dot{\mathrm{z}}\mathrm{e}$ na zrobienie 100 centymetrów kwadratowych koronki potrzebny jest

jeden motek kordonku obliczyč, ile motków musi kupič Pani Zosia, $\mathrm{j}\mathrm{e}\dot{\mathrm{z}}$ eli powinna

uwzględnič 2\% straty materiafu podczas pracy.
\end{document}
