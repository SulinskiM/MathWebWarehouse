\documentclass[a4paper,12pt]{article}
\usepackage{latexsym}
\usepackage{amsmath}
\usepackage{amssymb}
\usepackage{graphicx}
\usepackage{wrapfig}
\pagestyle{plain}
\usepackage{fancybox}
\usepackage{bm}

\begin{document}

PRACA KONTROLNA $\mathrm{n}\mathrm{r}4-$ POZIOM ROZSZERZONY

l. Do zbiornika poprowadzono trzy rury. Pierwsza rura potrzebuje do napełnienia zbiornika

$04$ godziny więcej $\mathrm{n}\mathrm{i}\dot{\mathrm{z}}$ druga, a trzecia napełnia cafy zbiornik $\mathrm{w}$ czasie dwa razy krótszym

$\mathrm{n}\mathrm{i}\dot{\mathrm{z}}$ pierwsza. Wjakim czasie napelnia zbiornik $\mathrm{k}\mathrm{a}\dot{\mathrm{z}}$ da $\mathrm{z}\mathrm{r}\mathrm{u}\mathrm{r}, \mathrm{j}\mathrm{e}\dot{\mathrm{z}}$ eli wiadomo, $\dot{\mathrm{z}}\mathrm{e}$ wszystkie

trzy rury otwarte jednocześnie napefniajq zbiornik $\mathrm{w}$ ciągu 2 godzin $\mathrm{i}40$ minut?

2. Stosując zasadę indukcji matematycznej wykazač prawdziwośč następującego wzoru dla

wszystkich $n\geq 1$

$\displaystyle \frac{1^{2}}{1\cdot 3}+\frac{2^{2}}{3\cdot 5}+\frac{3^{2}}{5\cdot 7}+\ldots+\frac{n^{2}}{(2n-1)(2n+1)}=\frac{n(n+1)}{2(2n+1)}$

3. Nie wykorzystujqc metod rachunku rózniczkowego wyznaczyč przedziały zawarte $\mathrm{w}[0,2\pi],$

na których funkcja

$ f(x)=\cos x+2\cos^{2}x+4\cos^{3}x+8\cos^{4}x+\ldots$

jest rosnąca.

4. Narysowač zbiór $\{(x,y):|x|+|y|\leq 6,|y|\leq 2^{|x|},|y|\geq\log_{2}|x|\}\mathrm{i}$ napisač równaniajego

osi symetrii. Podač odpowiednie uzasadnienie.

5. Pole przekroju ostrosłupa prawidlowego czworokątnego plaszczyznq przechodzącą przez

$\mathrm{P}^{\mathrm{r}\mathrm{z}\mathrm{e}\mathrm{k}}\Phi^{\mathrm{t}\mathrm{n}\text{ą}}$ podstawy $\mathrm{i}$ wierzcholek ostroslupa jest trójk$\Phi$tem równobocznym $0$ polu $S.$

Wyznaczyč stosunek promienia kuli wpisanej $\mathrm{w}$ ten ostrosłup do promienia kuli opisanej

na tym ostroslupie.

6. Punkt $A(1,2)$ jest wierzchołkiem trójkąta równobocznego. Wyznaczyč dwa pozostałe

wierzchołki tego trójkqta wiedząc, $\dot{\mathrm{z}}\mathrm{e}$ jeden $\mathrm{z}$ nich $\mathrm{l}\mathrm{e}\dot{\mathrm{z}}\mathrm{y}$ na prostej $x-y-1=0$, ajeden

$\mathrm{z}$ boków jest równolegly do wektora $\vec{v}= [-1,2]$. Obliczyč pole tego trójkąta. Ile jest

trójkątów spelniających warunki zadania?
\end{document}
