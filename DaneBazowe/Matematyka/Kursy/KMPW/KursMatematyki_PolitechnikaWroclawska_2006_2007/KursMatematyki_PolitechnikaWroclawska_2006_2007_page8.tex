\documentclass[a4paper,12pt]{article}
\usepackage{latexsym}
\usepackage{amsmath}
\usepackage{amssymb}
\usepackage{graphicx}
\usepackage{wrapfig}
\pagestyle{plain}
\usepackage{fancybox}
\usepackage{bm}

\begin{document}

PRACA KONTROLNA $\mathrm{n}\mathrm{r}5-$ POZIOM PODSTAWOWY

luty 2007r.

l. Bolek $\mathrm{i}$ Lolek $\mathrm{z}$ okazji swoich 9 $\mathrm{i} 11$ urodzin otrzymali od babci 200 zf do podziafu.

Umówili się, $\dot{\mathrm{z}}\mathrm{e}$ starszy otrzyma większą sumę, ale nie więcej $\mathrm{n}\mathrm{i}\dot{\mathrm{z}}0$ połowę od otrzymanej

przez brata, a ponadto średnia geometryczna obu kwot nie przekroczy iloczynu ich lat

$\dot{\mathrm{z}}$ ycia. Jaką maksymalną $\mathrm{i}$ minimalną kwotę $\mathrm{m}\mathrm{o}\dot{\mathrm{z}}\mathrm{e}$ otrzymač starszy brat.

2. Rozwazmy zbiór wszystkich ciagów binarnych $0$ dlugości 7. Wy1osowano jeden ciąg.

a) Jakie jest prawdopodobieństwo, $\dot{\mathrm{z}}\mathrm{e}$ bedzie zawieraf co najmniej 3 jedynki.

b) Jakie jest prawdopodobieństwo, $\dot{\mathrm{z}}\mathrm{e}\mathrm{w}$ tym ciągu wystqpi seria samych zer lub sa-

mych jedynek $0$ dfugości co najmniej 4.

3. $\mathrm{W}$ trójkącie $ABC$ dane są $\displaystyle \angle CAB=\frac{\pi}{3}$, wysokośč $|CD| =h=5$ oraz $|BD| =d=\sqrt{2}.$

Obliczyč promień okręgu wpisanego $\mathrm{w}$ ten trójkąt.

4. Na jednym rysunku przedstawič staranne wykresy funkcji $f(x) = |\displaystyle \sin(x-\frac{\pi}{9})|$ oraz

$g(x)=-\displaystyle \cos(x+\frac{5\pi}{18})$ na przedziale $I=[-\pi,2\pi].$

a) Odczytač $\mathrm{z}$ wykresu kąt $x_{0}$ taki, $\dot{\mathrm{z}}\mathrm{e}g(x)=\sin(x-x_{0}).$

b) Korzystając $\mathrm{z}$ wykresu oraz punktu a) wyznaczyč wszystkie kąty $x\in I$, dla których

$f(x)=g(x)$ oraz przedziafy, dla których $g(x)>f(x).$

5. Na walcu $0$ wysokości 6 cm $\mathrm{i}$ średnicy podstawy 16 cm opisano stozek $0$ kqcie rozwarcia

$ 2\alpha$ tak, $\dot{\mathrm{z}}\mathrm{e}$ podstawa walca $\mathrm{l}\mathrm{e}\dot{\mathrm{z}}\mathrm{y}$ na podstawie stozka, przy czym $\mathrm{t}\mathrm{g}\alpha= \displaystyle \frac{4}{3}$. Wyznaczyč

minimalne wymiary prostokąta ($\mathrm{z}$ zaokrągleniem $\mathrm{w}$ górę do pelnych cm), $\mathrm{w}$ którym

$\mathrm{m}\mathrm{o}\dot{\mathrm{z}}$ na zmieścič rozciętą powierzchnię boczną stozka $\mathrm{i}$ obliczyč jaki procent pola tego

prostokąta stanowi powierzchnia boczna stozka.

6. Dane są proste $k$ : $2x-3y+6=0$ oraz $l$ : $2x+4y-7=0$. Na prostej $k$ znalez$\acute{}$č punkt,

którego obraz symetryczny względem prostej $l\mathrm{l}\mathrm{e}\dot{\mathrm{z}}\mathrm{y}$ na osi $\mathrm{O}y$. Sporządzič rysunek.
\end{document}
