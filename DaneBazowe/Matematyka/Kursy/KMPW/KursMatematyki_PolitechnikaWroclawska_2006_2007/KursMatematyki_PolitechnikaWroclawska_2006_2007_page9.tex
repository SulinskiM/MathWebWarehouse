\documentclass[a4paper,12pt]{article}
\usepackage{latexsym}
\usepackage{amsmath}
\usepackage{amssymb}
\usepackage{graphicx}
\usepackage{wrapfig}
\pagestyle{plain}
\usepackage{fancybox}
\usepackage{bm}

\begin{document}

PRACA KONTROLNA $\mathrm{n}\mathrm{r}5-$ POZIOM ROZSZERZONY

l. Stosując zasadę indukcji matematycznej wykazač, $\dot{\mathrm{z}}\mathrm{e}$ liczba $7^{n}-(-3)^{n}$ jest podzielna

przez 10 d1a $\mathrm{k}\mathrm{a}\dot{\mathrm{z}}$ dego naturalnego $n.$

2. Rozwiązač nierównośč 4 logl6 $\cos 2x+2\log_{4}\sin x+\log_{2}\cos x+3<0$ dla $x\displaystyle \in(0,\frac{\pi}{4}).$

3. Róznica ciqgu arytmetycznego $(a_{n})$ jest liczbq mniejszq od l. Wyznaczyč najmniejszą

wartośč wyrazenia $\displaystyle \frac{a_{1}a49}{a_{50}}, \mathrm{w}\mathrm{i}\mathrm{e}\mathrm{d}\mathrm{z}\Phi^{\mathrm{C}}, \dot{\mathrm{z}}\mathrm{e}a_{51}=1.$

4. Cięciwa paraboli $0$ równaniu $y=-a^{2}x^{2}+5ax-4$ jest styczna do krzywej $y=\displaystyle \frac{1}{-x+1}$

$\mathrm{w}$ punkcie $0$ odciętej $x_{o}=2$, który dzieli $\mathrm{t}\mathrm{e}$ cięciwę na połowy. Wyznaczyč parametr $a.$

Podač ilustrację graficzną rozwiązania zadania.

5. Dana jest funkcja $f(x)=\displaystyle \frac{2x^{2}}{(2-x)^{2}}.$

a) Zbadač przebieg zmienności funkcji $f\mathrm{i}$ naszkicowač jej wykres.

b) Sporządzič wykres funkcji $k=g(m)$, gdzie $k$ jest liczbą rozwi$\Phi$zań równania

$\displaystyle \frac{2x^{2}}{(2-|x|)^{2}}=m$

$\mathrm{w}$ zalezności od parametru rzeczywistego $m.$

6. $\mathrm{W}$ kulę $0$ promieniu $R$ wpisano stozek, $\mathrm{w}$ którym tworząca jest równa średnicy pod-

stawy. Obydwie bryły przecieto płaszczyzną równoległą do podstawy stozka. Szerokośč

otrzymanego $\mathrm{w}$ przecięciu pierścienia kofowego zawartego między powierzchnią kulistą

a powierzchnią boczną stozka równa się $m.$

a) Znalez/č odlegfośč pfaszczyzny tnącej od wierzchołka stozka.

b) Przedyskutowač liczbę rozwiązań $\mathrm{w}$ zalezności od $m\mathrm{i}$ podač interpretację geome-

tryczną przypadków szczególnych.
\end{document}
