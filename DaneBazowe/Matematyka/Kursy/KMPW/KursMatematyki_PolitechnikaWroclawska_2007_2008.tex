\documentclass[a4paper,12pt]{article}
\usepackage{latexsym}
\usepackage{amsmath}
\usepackage{amssymb}
\usepackage{graphicx}
\usepackage{wrapfig}
\pagestyle{plain}
\usepackage{fancybox}
\usepackage{bm}

\begin{document}

XXXVII

KORESPONDENCYJNY KURS Z MATEMATYKI

PRACA KONTROLNA $\mathrm{n}\mathrm{r}1-$ POZIOM PODSTAWOWY

$\mathrm{p}\mathrm{a}\acute{\mathrm{z}}$dziernik 2$007\mathrm{r}.$

l. Pan Kowalski wpłacił $\mathrm{P}^{\mathrm{e}\mathrm{w}\mathrm{n}}\Phi$ sumę na lokatę oprocentowaną $\mathrm{w}$ wysokości 8\% $\mathrm{w}$ skali

roku, przy czym odsetki naliczane sq kwartalnie. $\mathrm{W}$ ciągu rozwazanego roku inflacja

wyniosła 4\%. Jakie jest rea1ne roczne oprocentowanie 1okaty Pana Kowa1skiego, $\mathrm{t}\mathrm{z}\mathrm{n}. 0$

ile procent więcej warte $\mathrm{s}\Phi \mathrm{p}\mathrm{i}\mathrm{e}\mathrm{n}\mathrm{i}_{\Phi}\mathrm{d}\mathrm{z}\mathrm{e}$, które Pan Kowalski miał na koncie po roku od

tych, które wpłacił? Wynik podač $\mathrm{z}$ dokładnością do setnych części procenta.

2. Liczba $p=\displaystyle \frac{(\sqrt[3]{54}-2)(9\sqrt[3]{4}+6\sqrt[3]{2}+4)-(2-\sqrt{3})^{3}}{\sqrt{3}+(1+\sqrt{3})^{2}}$ jest miejscem zerowym funkcji

$f(x) = ax^{2}+bx+c$. Wyznaczyč wspólczynniki $a, b, c$ oraz drugie miejsce zerowe tej

funkcji $\mathrm{w}\mathrm{i}\mathrm{e}\mathrm{d}\mathrm{z}\Phi^{\mathrm{C}}, \dot{\mathrm{z}}\mathrm{e}$ największ$\Phi$ wartości$\Phi$ funkcji jest 4, a jej wykres jest symetryczny

względem prostej $x=1.$

3. Dwie styczne do okręgu $0$ promieniu 6 przecinają się pod kątem $60^{\mathrm{o}}$. Obliczyč pole obsza-

ru ograniczonego odcinkami tych stycznych $\mathrm{i}$ krótszym $\mathrm{z}$ łuków, najakie $\mathrm{o}\mathrm{k}\mathrm{r}\Phi \mathrm{g}$ podzielony

jest punktami styczności. Wyznaczyč promień okręgu wpisanego $\mathrm{w}$ ten obszar.

4. Niech

$f(x)=$

gdy

gdy

$|x-1|\geq 1,$

$|x-1|<1.$

a) Obliczyč $f(-\displaystyle \frac{2}{3}), f(\displaystyle \frac{1+\sqrt{3}}{2})$ oraz $f(\pi-1).$

b) Narysowač wykres funkcji $f\mathrm{i}$ na jego podstawie podač zbiór wartości funkcji.

c) Rozwi$\Phi$zač nierównośč $f(x)\displaystyle \geq-\frac{1}{2}\mathrm{i}$ zaznaczyč na osi $0x$ zbiór jej rozwi$\Phi$zań.

5. Pole przekroju graniastosłupa prawidlowego $0$ podstawie kwadratowej paszczyz$\Phi$ prze-

$\mathrm{c}\mathrm{h}\mathrm{o}\mathrm{d}\mathrm{z}\Phi^{\mathrm{C}}\Phi$ przez $\mathrm{P}^{\mathrm{r}\mathrm{z}\mathrm{e}\mathrm{k}}\Phi^{\mathrm{t}\mathrm{n}}\Phi$ graniastosłupa $\mathrm{i}$ środki przeciwległych krawędzi bocznych jest

3 razy większe $\mathrm{n}\mathrm{i}\dot{\mathrm{z}}$ pole podstawy. Wyznaczyč tangens kąta nachylenia $\mathrm{P}^{\mathrm{r}\mathrm{z}\mathrm{e}\mathrm{k}}\Phi^{\mathrm{t}\mathrm{n}\mathrm{e}\mathrm{j}}$ grania-

stosłupa do podstawy. Obliczyč pole powierzchni całkowitej tego graniastosłupa $\mathrm{w}\mathrm{i}\mathrm{e}\mathrm{d}\mathrm{z}\Phi^{\mathrm{C}},$

$\dot{\mathrm{z}}\mathrm{e}$ pole rozwazanego przekroju równe jest 10.

6. Jeden $\mathrm{z}$ wierzcholków trójk$\Phi$ta prostokątnego $0$ polu 7, 5 jest punktem przecięcia pro-

stych $k:x-y+3=0$ oraz $l$ : $2x+y=0$. Wyznaczyč pozostałe wierzchołki $\mathrm{w}\mathrm{i}\mathrm{e}\mathrm{d}\mathrm{z}\Phi^{\mathrm{C}},$

$\dot{\mathrm{z}}\mathrm{e}\mathrm{l}\mathrm{e}\mathrm{z}\Phi$ one na prostych $k\mathrm{i}l$, a wierzchołek $\mathrm{k}_{\Phi^{\mathrm{t}\mathrm{a}}}$ prostego jest na prostej $l$. Sporz$\Phi$dzič

staranny rysunek.




PRACA KONTROLNA $\mathrm{n}\mathrm{r} 1 -$ POZIOM ROZSZERZONY

l. Narysowač wykres funkcji $f(x)=$ 

$\mathrm{P}\mathrm{o}\mathrm{s}l\mathrm{u}\mathrm{g}\mathrm{u}\mathrm{j}_{\Phi}\mathrm{c}$ się nim podač

wzór $\mathrm{i}$ narysowač wykres funkcji $g(m)$ określaj$\Phi$cej liczbę rozwi$\Phi$zań równania $f(x)=m,$

gdzie $m$ jest parametrem rzeczywistym.

2. Rozwi$\Phi$zač równanie $\displaystyle \frac{\sin 3x}{\cos x}=$ ctg $x-\mathrm{t}\mathrm{g}x.$

3. Napisač równanie stycznej $k$ do wykresu funkcji $f(x)=x^{2}-4x+3\mathrm{w}$ punkcie $(x_{1},0),$

gdzie $x_{1}$ jest najmniejszym miejscem zerowym tej funkcji. Znalez$\acute{}$č punkt przecięcia tej

stycznej ze $\mathrm{s}\mathrm{t}\mathrm{y}\mathrm{c}\mathrm{z}\mathrm{n}\Phi$ do niej $\mathrm{p}\mathrm{r}\mathrm{o}\mathrm{s}\mathrm{t}\mathrm{o}\mathrm{p}\mathrm{a}\mathrm{d}\text{ł}_{\Phi}$ Sporządzič staranny rysunek.

4. Rozwi$\Phi$zač nierównośč $\log_{2}(x-1)-\log_{\frac{1}{2}}(4-x)-\log_{\sqrt{2}}(x-2)\leq 0.$

5. Rozwi$\Phi$zač nierównośč $\displaystyle \sqrt{x^{2}-1}+1+\frac{1}{\sqrt{x^{2}-1}}+\ldots\geq \displaystyle \frac{9}{2},$

wyrazów nieskończonego $\mathrm{c}\mathrm{i}_{\Phi \mathrm{g}}\mathrm{u}$ geometrycznego.

gdzie lewa strona jest $\mathrm{s}\mathrm{u}\mathrm{m}\Phi$

6. $\mathrm{W}$ stozek wpisano kulę, a następnie $\mathrm{w}$ obszar zawarty między $\mathrm{t}_{\Phi}\mathrm{k}\mathrm{u}1_{\Phi}\mathrm{i}$ wierzchołkiem

stozka wpisano kulę $0$ objętości 8 razy mniejszej. Ob1iczyč stosunek objętości stozka do

objętości kuli na nim opisanej.





PRACA KONTROLNA $\mathrm{n}\mathrm{r}6-$ POZIOM PODSTAWOWY

marzec 2008r.

l. Dwa naczynia zawieraj $\Phi^{\mathrm{W}}$ sumie 401itrów wody. Po prze1aniu pewnej części wody pierw-

szego naczynia do drugiego, $\mathrm{w}$ pierwszym naczyniu zostalo trzy razy mniej wody $\mathrm{n}\mathrm{i}\dot{\mathrm{z}}\mathrm{w}$

drugim. Gdy następnie przelano taką samą częśč wody drugiego naczynia do pierwszego,

okazało się, $\dot{\mathrm{z}}\mathrm{e}\mathrm{w}$ obu naczyniach jest tyle samo płynu. Obliczyč, ile wody było pierwotnie

$\mathrm{w}\mathrm{k}\mathrm{a}\dot{\mathrm{z}}$ dym naczyniu $\mathrm{i}\mathrm{j}\mathrm{a}\mathrm{k}_{\Phi}$ jej częśč przelewano.

2. Obwód trójk$\Phi$ta równoramiennego równy jest 20. Jakie powinny byč jego boki, by obję-

tośč bryły otrzymanej przez obrót tego trójkąta wokóf podstawy była największa?

3. Student opracował 28 spośród 45 przygotowanych na egzamin tematów. Losuje trzy

tematy. $\mathrm{J}\mathrm{e}\dot{\mathrm{z}}$ eli odpowie poprawnie na wszystkie, to dostanie ocenę bardzo $\mathrm{d}\mathrm{o}\mathrm{b}\mathrm{r}\Phi, \mathrm{j}\mathrm{e}\dot{\mathrm{z}}$ eli

na dwa- $\mathrm{d}\mathrm{o}\mathrm{b}\mathrm{r}\Phi$, a $\mathrm{j}\mathrm{e}\dot{\mathrm{z}}$ eli na jedno- $\mathrm{d}\mathrm{o}\mathrm{s}\mathrm{t}\mathrm{a}\mathrm{t}\mathrm{e}\mathrm{c}\mathrm{z}\mathrm{n}\Phi$. Jakie jest prawdopodobieństwo, $\dot{\mathrm{z}}\mathrm{e}$:

a) dostanie przynajmniej db? b) zda egzamin?

4. Narysowač staranny wykres funkcji $f(x)=x^{2}-2|x|-3$, wyznaczyč jej miejsca zerowe $\mathrm{i}$

zbiór wartości. $\mathrm{W}\mathrm{y}\mathrm{k}\mathrm{o}\mathrm{r}\mathrm{z}\mathrm{y}\mathrm{s}\mathrm{t}\mathrm{u}\mathrm{j}_{\Phi}\mathrm{c}$ wykres funkcji $f$:

a) narysowač wykres funkcji $h(x)=x^{2}-2x-2|x-1|-1.$

b) $\mathrm{p}\mathrm{o}\mathrm{s}\text{ł} \mathrm{u}\mathrm{g}\mathrm{u}\mathrm{j}_{\Phi}\mathrm{c}$ się powyzszymi wykresami określič, dla jakich wartości parametru rze-

czywistego $m$ równanie $f(x)=h(x)+m$ ma dokładnie jedno $\mathrm{r}\mathrm{o}\mathrm{z}\mathrm{w}\mathrm{i}_{\Phi}$zanie.

5. Państwo Kowalscy $\mathrm{s}\Phi$ właścicielami działki budowlanej $\mathrm{w}$

kształcie trójk$\Phi$ta prostokątnego $0$ przyprostokątnych dfugości

30 $\mathrm{m}\mathrm{i}40\mathrm{m}$. Postanowili podzielič ją na dwie równej wartości

części zgodnie ze schematem obok. Wyznaczyč długośč odcinka

$\overline{BK}\mathrm{w}\mathrm{i}\mathrm{e}\mathrm{d}\mathrm{z}\Phi^{\mathrm{C}}, \dot{\mathrm{z}}\mathrm{e}$ jeden metr kwadratowy działki $\mathrm{c}\mathrm{z}\mathrm{w}\mathrm{o}\mathrm{r}\mathrm{o}\mathrm{k}_{\Phi}$tnej

jest póltora raza drozszy $\mathrm{n}\mathrm{i}\dot{\mathrm{z}}$ jeden metr kwadratowy dzialki

trójk$\Phi$tnej. Która $\mathrm{z}$ działek ma większy obwód $\mathrm{i}0$ ile? Wynik

podač $\mathrm{z}$ dokładnościq do 10 cm.
\begin{center}
\includegraphics[width=45.312mm,height=37.740mm]{./KursMatematyki_PolitechnikaWroclawska_2007_2008_page10_images/image001.eps}
\end{center}
{\it A}

{\it L}

{\it B  K C}

6. Boki $\overline{AB}, \overline{AC}$ trójk$\Phi$ta zawarte są $\mathrm{w}$ prostych $l$ : $x-y-1=0$ oraz $k$ : $x+2y+2=0.$

Wyznaczyč współrzędne wierzcholków $B, C \mathrm{w}\mathrm{i}\mathrm{e}\mathrm{d}\mathrm{z}\Phi^{\mathrm{C}}, \dot{\mathrm{z}}\mathrm{e}$ punkt $P(1,1)$ jest środkiem

boku $\overline{BC}$. Wyznaczyč współrzędne wierzcholków trójk$\Phi$ta otrzymanego przez odbicie

symetryczne powyzszego trójk$\Phi$ta względem boku $\overline{BC}.$





PRACA KONTROLNA $\mathrm{n}\mathrm{r} 6-$ POZIOM ROZSZERZONY

l. Rozwi$\Phi$zač $\mathrm{i}$ zinterpretowač graficznie układ równań 

1,

1.

2. Niech $f(x)=\log_{2}x, g(x)=x+2, h(x)=|x|.$

a) Narysowač wykresy funkcji $f\mathrm{o}h\mathrm{o}g$ oraz $g0f0h$

b) Rozwi$\Phi$zač nierównośč $(f\mathrm{o}h\mathrm{o}g)(x)<(g\mathrm{o}f\mathrm{o}h)(x).$

3. Rzucamy kolejno trzy razy kostką do gry. Jakie jest prawdopodobieństwo, $\dot{\mathrm{z}}\mathrm{e}\mathrm{w}$ otrzy-

manym $\mathrm{c}\mathrm{i}_{\Phi \mathrm{g}}\mathrm{u}\mathrm{s}\Phi$ przynajmniej dwie,,szóstki'' lub suma oczek przekroczy 14?

4. Dany jest wielomian $W(x) = x^{3}+ax+b$, gdzie $b \neq 0$. Wykazač, $\dot{\mathrm{z}}\mathrm{e} W(x)$ posiada

pierwiastek podwójny wtedy $\mathrm{i}$ tylko wtedy, gdy spełniony jest warunek $4a^{3}+27b^{2}=0.$

Wyrazič pierwiastki za pomocą współczynnika $b.$

5. $\mathrm{W}$ ostrosłupie prawidłowym czworokątnym dany jest kąt $\alpha$ nachylenia ściany bocznej

do podstawy oraz obwód ściany bocznej równy $l$. Obliczyč objętośč tego ostrosłupa.

6. Narysowač staranny wykres funkcji $f(x)=\cos x-\sqrt{3}|\sin x| \mathrm{w}$ przedziale $[0,2\pi] \mathrm{i}$ wy-

znaczyč zbiór jej wartości.

a) $\mathrm{P}\mathrm{o}\mathrm{s}\text{ł} \mathrm{u}\mathrm{g}\mathrm{u}\mathrm{j}_{\Phi}\mathrm{c}$ się wykresem podač liczbę rozwi$\Phi$zań równania $f(x)=m\mathrm{w}$ zalezności

od parametru rzeczywistego $m.$

b) $\mathrm{R}\mathrm{o}\mathrm{z}\mathrm{w}\mathrm{i}_{\Phi}\mathrm{z}\mathrm{u}\mathrm{j}_{\Phi}\mathrm{c}$ odpowiednie równanie $\mathrm{i}$ korzystając $\mathrm{z}$ wykresu podač $\mathrm{r}\mathrm{o}\mathrm{z}\mathrm{w}\mathrm{i}_{\Phi}$zanie nie-

równości $f(x)\leq-\sqrt{2}.$





PRACA KONTROLNA $\mathrm{n}\mathrm{r}2-$ POZIOM PODSTAWOWY

listopad $2007\mathrm{r}.$

l. Trzy liczby dodatnie $\mathrm{t}\mathrm{w}\mathrm{o}\mathrm{r}\mathrm{z}\Phi \mathrm{c}\mathrm{i}_{\Phi \mathrm{g}}$ geometryczny. Suma tych liczb równa jest 26, a suma

ich odwrotności wynosi 0.7(2). Wyznaczyč te 1iczby.

2. Pole powierzchni bocznej ostrosłupa prawidłowego $\mathrm{c}\mathrm{z}\mathrm{w}\mathrm{o}\mathrm{r}\mathrm{o}\mathrm{k}_{\Phi^{\mathrm{t}}}$nego j$\mathrm{e}\mathrm{s}\mathrm{t}2$ razy większe $\mathrm{n}\mathrm{i}\dot{\mathrm{z}}$

pole podstawy. $\mathrm{W}$ trójk$\Phi$t otrzymany $\mathrm{w}$ przekroju ostrosfupa $\mathrm{p}\mathrm{a}$szczyz $\Phi \mathrm{p}\mathrm{r}\mathrm{z}\mathrm{e}\mathrm{c}\mathrm{h}\mathrm{o}\mathrm{d}\mathrm{z}\Phi^{\mathrm{C}}\Phi$

przez jego wysokośč $\mathrm{i} \mathrm{P}^{\mathrm{r}\mathrm{z}\mathrm{e}\mathrm{k}}\Phi^{\mathrm{t}\mathrm{n}\mathrm{q}}$ podstawy wpisano kwadrat, którego jeden bok jest

zawarty $\mathrm{w}\mathrm{P}^{\mathrm{r}\mathrm{z}\mathrm{e}\mathrm{k}}\Phi^{\mathrm{t}\mathrm{n}\mathrm{e}\mathrm{j}}$ podstawy. Obliczyč stosunek pola tego kwadratu do pola podstawy

ostrosłupa. Sporz$\Phi$dzič staranny rysunek.

3. Wykonač działania $\mathrm{i}$ zapisač $\mathrm{w}$ najprostszej postaci wyrazenie

$s(a,b)= (\displaystyle \frac{a^{2}+b^{2}}{a^{2}-b^{2}}-\frac{a^{3}+b^{3}}{a^{3}-b^{3}})$ : $(\displaystyle \frac{a^{2}}{a^{3}-b^{3}}-\frac{a}{a^{2}+ab+b^{2}})$

Wyznaczyč wysokośč trójk$\Phi$ta prostokątnego wpisanego $\mathrm{w}\mathrm{o}\mathrm{k}\mathrm{r}\Phi \mathrm{g}\mathrm{o}$ promieniu 6 opusz-

$\mathrm{c}\mathrm{z}\mathrm{o}\mathrm{n}\Phi \mathrm{z}$ wierzchołka $\mathrm{k}_{\Phi^{\mathrm{t}\mathrm{a}}}$ prostego wiedząc, $\dot{\mathrm{z}}\mathrm{e}$ tangens jednego $\mathrm{z}$ k$\Phi$tów ostrych tego

trójk$\Phi$ta równy jest $s(\sqrt{5}+\sqrt{3},\sqrt{5}-\sqrt{3}).$

4. Wielomian $W(x) =x^{3}-x^{2}+bx+c$ jest podzielny przez $(x+3)$, a reszta $\mathrm{z}$ dzielenia

tego wielomianu przez $(x-3)$ równa jest 6. Wyznaczyč $b\mathrm{i} c$, a następnie rozwi$\Phi$zač

nierównośč $(x+1)W(x-1)-(x+2)W(x-2)\leq 0.$

5. $\mathrm{W}$ ramach przygotowań do EURO 2012 zap1anowano budowe komp1eksu sportowego zło-

$\dot{\mathrm{z}}$ onego $\mathrm{z}$ czterech jednakowych hal sportowych $\mathrm{w}$ kształcie pófkul $0$ środkach $\mathrm{w}$ rogach

kwadratu $0$ boku 100 $\mathrm{m}\mathrm{i}$ piątej hali $\mathrm{w}$ ksztafcie pófkuli stycznej do czterech pozosta-

fych. Jakie powinny byč wymiary tych hal, by koszt ich budowy był najmniejszy, $\mathrm{j}\mathrm{e}\dot{\mathrm{z}}$ eli

wiadomo, $\dot{\mathrm{z}}\mathrm{e}$ jest on proporcjonalny do pola powierzchni dachu hali?

6. $\mathrm{W}$ trójk$\Phi$cie prostokątnym $0$ kącie prostym przy wierzchoku $C$ na przedłuzeniu przeciw-

$\mathrm{p}\mathrm{r}\mathrm{o}\mathrm{s}\mathrm{t}\mathrm{o}\mathrm{k}_{\Phi^{\mathrm{t}}}\mathrm{n}\mathrm{e}\mathrm{j}$ AB odmierzono odcinek $BD\mathrm{t}\mathrm{a}\mathrm{k}, \dot{\mathrm{z}}\mathrm{e}|BD|=|BC|$. Wyznaczyč $|CD|$ oraz

obliczyč pole trójkta $\triangle ACD, \mathrm{j}\mathrm{e}\dot{\mathrm{z}}$ eli $|BC|=5, |AC|=12$. Sporz$\Phi$dzič staranny rysunek.





PRACA KONTROLNA $\mathrm{n}\mathrm{r} 2-$ POZIOM ROZSZERZONY

l. Znalez/č wszystkie wartości parametru rzeczywistego $m$, dla których pierwiastki trójmia-

nu kwadratowego $f(x)=(m-2)x^{2}-(m+1)x-m \mathrm{s}\mathrm{p}\mathrm{e}\mathrm{f}\mathrm{n}\mathrm{i}\mathrm{a}\mathrm{j}_{\Phi}$ nierównośč $|x_{1}|+|x_{2}|\leq 1.$

2. Wyznaczyč dziedzinę funkcji

$f(x)=\displaystyle \frac{\sqrt{2^{4-x^{2}}-4^{x}}}{\log(2-x-x^{2}-\ldots)}.$

3. Grupa l75 robotników miala wykonač pewną pracę $\mathrm{w}$ określonym terminie. Po upływie

30 dni wspólnej pracy przesyłano codziennie po 3 robotników na inne stanowiska, wsku-

tek czego robota została wykonana $\mathrm{z}$ opóz/nieniem 21 $\mathrm{d}\mathrm{n}\mathrm{i}. \mathrm{W}$ ciągu ilu dni miała byč

wykonana praca według planu?

4. Wyznaczyč promień okręgu opisanego na czworokącie ABCD, $\mathrm{w}$ którym $\mathrm{k}_{\Phi^{\mathrm{t}}}$ przy wierz-

cholku $A$ ma miarę $\alpha, \mathrm{k}_{\Phi^{\mathrm{t}\mathrm{y}}}$ przy wierzchołkach $B,  D\mathrm{s}\Phi$ proste oraz $|BC|=a, |AD|=b.$

Sporz$\Phi$dzič staranny rysunek.

5. Narysowač staranny wykres funkcji $f(x)=\displaystyle \frac{\sin 2x-|\sin x|}{\sin x}.$

$\mathrm{W}$ przedziale $[0,\pi]$ wyznaczyč $\mathrm{r}\mathrm{o}\mathrm{z}\mathrm{w}\mathrm{i}_{\Phi}$zania nierówności $f(x)<2(\sqrt{2}-1)\cos^{2}x.$

6. Pole przekroju graniastosłupa prawidlowego $0$ podstawie kwadratowej paszczyz$\Phi$ prze-

$\mathrm{c}\mathrm{h}\mathrm{o}\mathrm{d}\mathrm{z}\Phi^{\mathrm{C}}\Phi$ przez $\mathrm{P}^{\mathrm{r}\mathrm{z}\mathrm{e}\mathrm{k}}\Phi^{\mathrm{t}\mathrm{n}}\Phi$ graniastosłupa $\mathrm{i}$ środek jednej $\mathrm{z}$ krawędzi podstawy jest 3 razy

większe $\mathrm{n}\mathrm{i}\dot{\mathrm{z}}$ pole podstawy. Wyznaczyč tangens $\mathrm{k}_{\Phi^{\mathrm{t}\mathrm{a}}}$ nachylenia $\mathrm{P}^{\mathrm{r}\mathrm{z}\mathrm{e}\mathrm{k}}\Phi^{\mathrm{t}\mathrm{n}\mathrm{e}\mathrm{j}}$ graniastosłu-

pa do podstawy. Obliczyč pole powierzchni całkowitej tego graniastosłupa $\mathrm{w}\mathrm{i}\mathrm{e}\mathrm{d}\mathrm{z}\Phi^{\mathrm{C}}, \dot{\mathrm{z}}\mathrm{e}$

pole rozwazanego przekroju równe jest 15. Sporządzič staranny rysunek.





PRACA KONTROLNA $\mathrm{n}\mathrm{r}3-$ POZIOM PODSTAWOWY

grudzień $2007\mathrm{r}.$

l. Rozwi$\Phi$zač równanie

$\sqrt{3-x}+\sqrt{3x-2}=2.$

2. Sześč kostek sześciennych $0$ objętościach 1, 2, 4, 8, 16 $\mathrm{i}32\mathrm{d}\mathrm{m}^{3}$ ustawiono $\mathrm{w}$ piramidę,

$\mathrm{u}\mathrm{k}l\mathrm{a}\mathrm{d}\mathrm{a}\mathrm{j}_{\Phi}\mathrm{c}\mathrm{j}\mathrm{e}\mathrm{d}\mathrm{n}\Phi$ kostkę na drugiej poczynając od największej. Czy wysokośč piramidy

przekroczy 120 cm? Odpowied $\acute{\mathrm{z}}$ uzasadnič bez prowadzenia obliczeń przyblizonych.

3. ()$\mathrm{P}\mathrm{a}\mathrm{n}\mathrm{W}$ wybrał się na spacer do parku mającego ksztalt $\mathrm{p}\mathrm{r}\mathrm{o}\mathrm{s}\mathrm{t}\mathrm{o}\mathrm{k}_{\Phi^{\mathrm{t}}}\mathrm{a}\mathrm{o}$ wymiarach 400 $\mathrm{m}$

na 300 $\mathrm{m}$, podzielonego alejkami na 12 kwadratów $0$ boku 100 $\mathrm{m}$, jak na rysunku ponizej.

Postanowił przejśč od punktu $A$ do $B, l_{\Phi}$cznie $700\mathrm{m}$, wybierajqc przypadkowo alejkę

na $\mathrm{k}\mathrm{a}\dot{\mathrm{z}}$ dym rozwidleniu. Jakie jest prawdopodobieństwo, $\dot{\mathrm{z}}\mathrm{e}$ Pan $\mathrm{W}$ przejdzie środkow$\Phi$

$\mathrm{a}\mathrm{l}\mathrm{e}\mathrm{j}\mathrm{k}_{\Phi^{\mathrm{O}\mathrm{Z}\mathrm{n}\mathrm{a}\mathrm{c}\mathrm{z}\mathrm{o}\mathrm{n}}\Phi}$ na rysunku $x$?
\begin{center}
\includegraphics[width=36.216mm,height=28.752mm]{./KursMatematyki_PolitechnikaWroclawska_2007_2008_page4_images/image001.eps}
\end{center}
$\sqrt{}^{B}$

4. $\mathrm{P}\mathrm{o}\mathrm{d}\mathrm{s}\mathrm{t}\mathrm{a}\mathrm{w}\Phi$ trójk$\Phi$ta równoramiennego jest odcinek AB $0$ końcach $A(-1,1), B(3,3), \mathrm{a}$

wierzchołek $C\mathrm{l}\mathrm{e}\dot{\mathrm{z}}\mathrm{y}$ na paraboli $0$ równaniu $y^{2}=x+1$. Wyznaczyč współrzędne wierz-

chołka $C$ oraz pole trójk$\Phi$ta $ABC$. Sporz$\Phi$dzič rysunek.

5. Na jednym rysunku sporz$\Phi$dzič dokładne wykresy funkcji $\sin x, \cos x$, tg $x$ oraz ctg $x$

$\mathrm{w}$ przedziale $(0,\displaystyle \frac{\pi}{2}) \mathrm{i}$ zaznaczyč na nich

ctg $(\displaystyle \cos\frac{\pi}{4}), \displaystyle \cos(\sin\frac{\pi}{3}), \displaystyle \sin(\cos\frac{\pi}{3})$, tg $(\displaystyle \sin\frac{\pi}{2})$

Uporz$\Phi$dkowač powyzsze liczby od najmniejszej do największej. Uzasadnič te relacje za

$\mathrm{P}^{\mathrm{o}\mathrm{m}\mathrm{o}\mathrm{c}}\Phi$ odpowiednich nierówności.

6. $\mathrm{W}$ ostrosłupie prawidłowym trójkątnym kąt między ścianami bocznymi ma miarę $\alpha, \mathrm{a}$

odległośč krawędzi podstawy od przeciwległej krawędzi bocznej jest równa $d$. Obliczyč

objętośč ostrosłupa.





PRACA KONTROLNA nr 3 -POZIOM ROZSZERZONY

1. $\mathrm{S}\mathrm{t}\mathrm{o}\mathrm{s}\mathrm{u}\mathrm{j}_{\Phi}\mathrm{c}$ zasadę indukcji matematycznej, udowodnič prawdziwośč wzoru

$\left(\begin{array}{l}
3\\
2
\end{array}\right) + \left(\begin{array}{l}
5\\
2
\end{array}\right) +\ldots+\left(\begin{array}{ll}
2n+ & 1\\
2 & 
\end{array}\right) =\displaystyle \frac{n(n+1)(4n+5)}{6}$

dla $n\geq 1.$

2. Wojtuś wylosował $\mathrm{j}\mathrm{e}\mathrm{d}\mathrm{n}\Phi$ monetę ze skarbonki zawierającej 3 złotówki, 4 dwuzłotówki $\mathrm{i}3$

pięciozlotówki. Następnie, $\mathrm{w}$ zalezności od wyniku pierwszego losowania, wylosował jesz-

cze trzy monety, gdy za pierwszym razem otrzymał złotówkę, dwie monety, gdy pierwsza

była dwuzlotówk$\Phi$ oraz jedną monetę, gdy $\mathrm{w}$ pierwszym losowaniu dostał pięciozłotów-

kę. Obliczyč prawdopodobieństwo, $\dot{\mathrm{z}}\mathrm{e}$, postępuj$\Phi$c $\mathrm{w}$ ten sposób, zgromadził $\text{ł}_{\Phi}$cznie co

najmniej 8 złotych.

3. Jednym $\mathrm{z}$ wierzchołków kwadratujest punkt $A(2,2)$, a środkiemjednego $\mathrm{z}$ przeciwległych

boków jest punkt $M(-\displaystyle \frac{1}{2},-\frac{1}{2})$. Wyznaczyč współrzędne pozostałych wierzchołków oraz

równanie okręgu opisanego na tym kwadracie.

4. Rozwi$\Phi$zač nierównośč

$\displaystyle \frac{1}{\sqrt{3^{x+1}-2}}\geq\frac{1}{4-(\sqrt{3})^{x+2}}.$

5. $\mathrm{W}$ ostrosłup prawidłowy trójkątny wpisano walec, którego podstawa $\mathrm{l}\mathrm{e}\dot{\mathrm{z}}\mathrm{y}$ na podstawie

ostrosłupa. Srednica podstawy walcajest równajego wysokości. Znalez/č tangens $\mathrm{k}_{\Phi^{\mathrm{t}\mathrm{a}}}$ na-

chylenia krawędzi bocznej ostrosłupa do podstawy, dla którego stosunek objętości walca

do objętości ostrosłupa jest największy. Podač ten największy stosunek $\mathrm{w}$ procentach.

6. Długości boków trapezu opisanego na okręgu $0$ promieniu $R\mathrm{t}\mathrm{w}\mathrm{o}\mathrm{r}\mathrm{z}\Phi \mathrm{c}\mathrm{i}_{\Phi \mathrm{g}}$ arytmetyczny,

przy czym najkrótszy bok ma długośč $\displaystyle \frac{3}{4}R$. Obliczyč długości obu podstaw trapezu oraz

cosinus $\mathrm{k}_{\Phi^{\mathrm{t}\mathrm{a}}}$ pomiędzy $\mathrm{P}^{\mathrm{r}\mathrm{z}\mathrm{e}\mathrm{k}}\Phi^{\mathrm{t}\mathrm{n}\mathrm{y}\mathrm{m}\mathrm{i}}$. Sporządzič rysunek $\mathrm{p}\mathrm{r}\mathrm{z}\mathrm{y}\mathrm{j}\mathrm{m}\mathrm{u}\mathrm{j}_{\Phi}\mathrm{c}R=2$ cm.





PRACA KONTROLNA $\mathrm{n}\mathrm{r}4-$ POZIOM PODSTAWOWY

styczeń 2008r.

l. Ramka z drutu 0 długości 1 ma kształt kwadratu zakończonego

trójk$\Phi$tem równoramiennym, jak na rysunku. Bok kwadratu wynosi

a, natomiast ramię trójkąta równe jest b. Wyznaczyč a i b tak, by

pola kwadratu i trójkąta byly jednakowe.

2. Niech

$A=\{(x,y):x\in \mathbb{R},y\in \mathbb{R},y=-x+a,a\in\langle-2,2\rangle\},$

$B=\displaystyle \{(x,y):x\in \mathbb{R},y\in \mathbb{R},y=kx,k\in\langle\frac{1}{2},1\rangle\}.$

$\mathrm{W}\mathrm{P}^{\mathrm{r}\mathrm{o}\mathrm{s}\mathrm{t}\mathrm{o}\mathrm{k}}\Phi^{\mathrm{t}\mathrm{n}\mathrm{y}\mathrm{m}}$ układzie współrzędnych narysowač zbiór $A\cap B\mathrm{i}$ obliczyč jego pole.

Sprawdzič, czy punkt $(\displaystyle \frac{1}{2},\frac{3}{4})$ nalezy do zbioru $A\cap B.$

3. Dany jest stozek ścięty, $\mathrm{w}$ którym pole dolnej podstawy jest 4 razy większe od po1a

górnej. $\mathrm{W}$ stozek wpisano walec $\mathrm{t}\mathrm{a}\mathrm{k}, \dot{\mathrm{z}}\mathrm{e}$ dolna podstawa walca $\mathrm{l}\mathrm{e}\dot{\mathrm{z}}\mathrm{y}$ na dolnej podstawie

stozka, a brzeg górnej podstawy $\mathrm{l}\mathrm{e}\dot{\mathrm{z}}\mathrm{y}$ na jego powierzchni bocznej. $\mathrm{J}\mathrm{a}\mathrm{k}_{\Phi}$ częśč objętości

stozka ściętego stanowi objętośč walca, $\mathrm{j}\mathrm{e}\dot{\mathrm{z}}$ eli wysokośč walca jest 3 razy mniejsza od

wysokości stozka? Odpowied $\acute{\mathrm{z}}$ podač $\mathrm{w}$ procentach $\mathrm{z}$ dokładności$\Phi$ do jednego promila.

Sporz$\Phi$dzič staranny rysunek przekroju osiowego bryly.

4. Rozwi$\Phi$zač nierównośč $f(x)+3x>1$, gdzie $f(x)=\displaystyle \frac{1-3x}{\sqrt{2-\frac{3x+1}{x-2}}}.$

5. Dane $\mathrm{s}\Phi$ dwa $\displaystyle \mathrm{c}\mathrm{i}_{\Phi \mathrm{g}}\mathrm{i}a_{n}=\frac{1}{n}$ oraz $b_{n}=\displaystyle \frac{n-2}{(n+2)(n+4)}$. Zbadač monotonicznośč ciqgu

$c_{n}=(n-1)a_{n+1}+2b_{2n}.$

Czy $\mathrm{c}\mathrm{i}_{\Phi \mathrm{g}}c_{n}$ jest ograniczony? Dla jakich $n$ spefniona jest nierównośč $\displaystyle \frac{3}{4}<c_{n}<1$?

6. Okręgi $0$ promieniach $r\mathrm{i}2r\mathrm{p}\mathrm{r}\mathrm{z}\mathrm{e}\mathrm{c}\mathrm{i}\mathrm{n}\mathrm{a}\mathrm{j}_{\Phi}$ się $\mathrm{w}$ punktach A $\mathrm{i}B, \mathrm{b}\text{ę} \mathrm{d}_{\Phi}$cych wierzchołkami

trójk$\Phi$ta równobocznego $ABC$ wpisanego $\mathrm{w}$ jeden $\mathrm{z}$ okręgów. Obliczyč pole deltoidu

ADBC, którego wierzchołek $D\mathrm{l}\mathrm{e}\dot{\mathrm{z}}\mathrm{y}$ na drugim okręgu oraz wyznaczyč sinus kąta przy

wierzchołku $D.$





PRACA KONTROLNA $\mathrm{n}\mathrm{r} 4-$ POZIOM ROZSZERZONY

l. Dany jest romb ABCD $0$ boku $a\mathrm{i}\mathrm{k}_{\Phi}\mathrm{c}\mathrm{i}\mathrm{e}$ ostrym $\alpha. \mathrm{Z}$ wierzcholka $A\mathrm{k}_{\Phi^{\mathrm{t}\mathrm{a}}}$ ostrego po-

prowadzono dwa jednakowej długości odcinki $0$ końcach zawartych $\mathrm{w}$ bokach $BC\mathrm{i}CD.$

Wyznaczyč długości tych odcinków oraz sinusy kątów, na jaki został podzielony $\mathrm{k}_{\Phi^{\mathrm{t}}}\alpha$

$\mathrm{w}\mathrm{i}\mathrm{e}\mathrm{d}\mathrm{z}\Phi^{\mathrm{C}}, \dot{\mathrm{z}}\mathrm{e}$ pole środkowego deltoidu jest równe połowie pola danego rombu.

2. Napisač równanie stycznej do krzywej $f(x)=\displaystyle \frac{x}{x^{2}-1} \mathrm{w}$ punkcie $x_{0} = 2$. Wykazač, $\dot{\mathrm{z}}\mathrm{e}$

obrazem tej stycznej $\mathrm{w}$ symetrii względem punktu $(0,0)$ jest prosta, która jest styczną

do tej samej krzywej. Wyznaczyč odległośč między tymi stycznymi.

3. Niech

$A=\{(x,y):x\in \mathbb{R},y\in \mathbb{R},|x-1|+x\geq y+|y-2|\},$

$B=\displaystyle \{(x,y):x\in \mathbb{R},y\in \mathbb{R},|x-1|+\frac{1}{4}|y|\leq 1\}.$

Na płaszczy $\acute{\mathrm{z}}\mathrm{n}\mathrm{i}\mathrm{e}OXY$ narysowač zbiory $A\cap B$ oraz $B'\backslash A.$

4. Dane jest równanie

8 $(\sin\alpha+4)x^{2}-8(\sin\alpha+1)x+1=0,$

gdzie $\alpha \in \langle 0,  2\pi\rangle$. Dla jakich wartości $\mathrm{k}_{\Phi^{\mathrm{t}\mathrm{a}}}\alpha$ suma odwrotności pierwiastków tego

równania jest równa co najmniej 8 $(\cos\alpha-(\cos\alpha)^{-1}+1)$ ?

5. Zbadač funkcję $f(m)=\displaystyle \frac{y}{x}$, gdzie para $x\mathrm{i}y$ jest $\mathrm{r}\mathrm{o}\mathrm{z}\mathrm{w}\mathrm{i}_{\Phi}$zaniem układu równań

$\left\{\begin{array}{l}
(m-2)x+(m+2)y=m^{2}-1\\
(m+2)x+(m-2)y=m^{2}+1,
\end{array}\right.$

$\mathrm{z}$ parametrem rzeczywistym $m$. Sporz$\Phi$dzič wykres funkcji $f(m).$

6. $\mathrm{W}$ stozek $0$ promieniu podstawy $r\mathrm{i}\mathrm{t}\mathrm{w}\mathrm{o}\mathrm{r}\mathrm{z}\Phi^{\mathrm{C}\mathrm{e}\mathrm{j}}l$ wpisano ostrosłup prawidłowy trójkątny

$\mathrm{t}\mathrm{a}\mathrm{k}, \dot{\mathrm{z}}\mathrm{e}$ wierzchołek tego ostrosłupa pokrywa się ze środkiem podstawy stozka, a pozo-

stałe wierzchołki $\mathrm{l}\mathrm{e}\mathrm{z}\Phi$ na ścianie bocznej stozka. Jaka jest maksymalna objętośč tego

ostrosłupa? Sporzqdzič staranny rysunek.





PRACA KONTROLNA $\mathrm{n}\mathrm{r}5-$ POZIOM PODSTAWOWY

luty $2008\mathrm{r}.$

l. Ile razy objętośč ostrosłupa trójkątnego prawidfowego opisanego na stozku $0$ objętości $V$

jest większa od objętości ostrosłupa trójkątnego prawidlowego wpisanego $\mathrm{w}$ ten stozek?

2. Rozwi$\Phi$zač nierównośč

$|4x^{2}-4|+2x\geq|1-x|+2.$

3. Kamilek ma 2 latka $\mathrm{i}85$ cm wzrostu. Przez kolejne 31ata będzie rósł średnio 1cm mie-

sięcznie. Potem $\mathrm{w}\mathrm{c}\mathrm{i}_{\Phi \mathrm{g}}\mathrm{u}\mathrm{k}\mathrm{a}\dot{\mathrm{z}}$ dych 10 miesięcy będzie rósł $0$ 10\% wolniej $\mathrm{n}\mathrm{i}\dot{\mathrm{z}}\mathrm{w}$ poprzednim

okresie. Jaki wzrost będzie miał chłopczyk $\mathrm{w}$ dniu swoich 15-tych urodzin? Wynik podač

$\mathrm{z}$ dokładności$\Phi$ do 5 mm.

4. Uzasadnič, wykonuj $\Phi^{\mathrm{C}}$ odpowiednie obliczenia, $\dot{\mathrm{z}}\mathrm{e}\mathrm{z}$ kartki papieru $\mathrm{w}$ kształcie sześciok$\Phi$ta

foremnego $0$ boku $a= 2(1+\sqrt{3}) \mathrm{m}\mathrm{o}\dot{\mathrm{z}}$ na wyci$\Phi$č 19 kółek $0$ promieniu l. Czy istnieje

mniejszy sześciok$\Phi$t foremny, $\mathrm{z}$ którego $\mathrm{m}\mathrm{o}\dot{\mathrm{z}}$ na wyciąč taką $\mathrm{s}\mathrm{a}\mathrm{m}\Phi$ ilośč identycznych kółek?

5. Punkty (l, l) $\mathrm{i} (5,4) \mathrm{s}\Phi$ dwoma wierzchołkami rombu $0$ polu 15. Opisač konstrukcje

wszystkich rombów spełniaj $\Phi^{\mathrm{C}}\mathrm{y}\mathrm{c}\mathrm{h}$ podane warunki. Wyznaczyč wspófrzędne pozostałych

wierzcholków, przy załozeniu, $\dot{\mathrm{z}}\mathrm{e}$ nie wszystkie wierzchołki $\mathrm{l}\mathrm{e}\mathrm{z}\Phi \mathrm{w}$ I čwiartce układu

wspólrzędnych.

6. Wyznaczyč równanie krzywej będ$\Phi$cej zbiorem wszystkich środków cięciw paraboli

$y=(x-1)^{2}+1 \mathrm{p}\mathrm{r}\mathrm{z}\mathrm{e}\mathrm{c}\mathrm{h}\mathrm{o}\mathrm{d}_{\mathrm{Z}\Phi}$cych przez punkt $P(-1,2).$

(Wsk. Zauwazyč, $\dot{\mathrm{z}}\mathrm{e}\mathrm{j}\mathrm{e}\dot{\mathrm{z}}$ eli $x_{1}, x_{2}$ są pierwiastkami trójmianu kwadratowego $y=ax^{2}+bx+c,$

to prawdziwa jest równośč $x_{1}+x_{2}=\displaystyle \frac{-b}{a}.$)





PRACA KONTROLNA $\mathrm{n}\mathrm{r} 5-$ POZIOM ROZSZERZONY

l. Rozwi$\Phi$zač równanie

$\displaystyle \mathrm{t}\mathrm{g}^{2}x+\mathrm{t}\mathrm{g}^{4}x+\cdots=\frac{1}{2},$

$\mathrm{w}$ którym lewa strona jest $\mathrm{s}\mathrm{u}\mathrm{m}\Phi$ wyrazów nieskończonego ciągu geometrycznego.

2. Pani Józefa wpłaciła do banku pewien kapitał $K_{0}$ na okres jednego roku na lokatę opro-

$\mathrm{c}\mathrm{e}\mathrm{n}\mathrm{t}\mathrm{o}\mathrm{w}\mathrm{a}\mathrm{n}\Phi$ {\it P}\% $\mathrm{w}$ skali roku, przy czym kapitalizacja odsetek następuje $N$ razy rocznie.

Uzasadnič indukcyjnie, $\dot{\mathrm{z}}\mathrm{e}$ wzór $K_{n}=K_{0}(1+\displaystyle \frac{P}{100N})^{n}$ okeśla stan konta pani Józefy po

$n$-tym okresie kapitalizacyjnym. Sprawdzič, jaki będzie stan konta pani Józefy po roku

przy załozeniu, $\dot{\mathrm{z}}\mathrm{e}$ wplaci ona 10.000, 00 zł na 6\%, a odsetki kapita1izowane będą co

$\mathrm{m}\mathrm{i}\mathrm{e}\mathrm{s}\mathrm{i}_{\Phi}\mathrm{c}.$

3. Zaznaczyč na płaszczy $\acute{\mathrm{z}}\mathrm{n}\mathrm{i}\mathrm{e}$ zbiór rozwi$\Phi$zań nierówności

$\log_{\frac{1}{2}}(3\log_{x}(2y))\geq 0.$

4. $\mathrm{W}$ koło $0$ promieniu $R$ wpisano trójkqt, którego pole stanowi czwartą częśč pola koła,

ajeden $\mathrm{z}$ k$\Phi$tów ma miarę $\alpha$. Obliczyč iloczyn oraz sumę kwadratów długości boków tego

trójk$\Phi$ta.

5. Wyznaczyč równanie krzywej będ$\Phi$cej zbiorem wszystkich środków okręgów stycznych do

prostej $y=2\mathrm{i}\mathrm{p}\mathrm{r}\mathrm{z}\mathrm{e}\mathrm{c}\mathrm{h}\mathrm{o}\mathrm{d}_{\mathrm{Z}\Phi}$cych przez $\mathrm{P}^{\mathrm{O}\mathrm{C}\mathrm{Z}}\Phi^{\mathrm{t}\mathrm{e}\mathrm{k}}$ układu współrzędnych. Spośród rozwaza-

nych okręgów narysowač wszystkie okręgi styczne do jednej $\mathrm{z}$ osi układu współrzędnych

$\mathrm{i}$ wyznaczyč równanie okręgu przechodzącego przez ich środki.

6. Na dnie naczynia $\mathrm{w}$ kształcie walca umieszczono 6 małych ku1ek $0$ promieniu $R\mathrm{w}$ taki

sposób, $\dot{\mathrm{z}}\mathrm{e}\mathrm{k}\mathrm{a}\dot{\mathrm{z}}$ da $\mathrm{z}$ nich jest styczna do dwu innych kulek $\mathrm{i}$ ściany bocznej naczynia.

Następnie umieszczono $\mathrm{w}$ nim kulę $0$ promieniu $ 2R\mathrm{s}\mathrm{t}\mathrm{y}\mathrm{c}\mathrm{z}\mathrm{n}\Phi$ do $\mathrm{k}\mathrm{a}\dot{\mathrm{z}}$ dej $\mathrm{z}$ małych kulek

oraz górnej podstawy walca. Sprawdzič, ile wody zmieści się $\mathrm{w}$ tak zapełnionym naczyniu.



\end{document}