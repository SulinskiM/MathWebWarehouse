\documentclass[a4paper,12pt]{article}
\usepackage{latexsym}
\usepackage{amsmath}
\usepackage{amssymb}
\usepackage{graphicx}
\usepackage{wrapfig}
\pagestyle{plain}
\usepackage{fancybox}
\usepackage{bm}

\begin{document}

XXXVII

KORESPONDENCYJNY KURS Z MATEMATYKI

PRACA KONTROLNA $\mathrm{n}\mathrm{r}1-$ POZIOM PODSTAWOWY

$\mathrm{p}\mathrm{a}\acute{\mathrm{z}}$dziernik 2$007\mathrm{r}.$

l. Pan Kowalski wpłacił $\mathrm{P}^{\mathrm{e}\mathrm{w}\mathrm{n}}\Phi$ sumę na lokatę oprocentowaną $\mathrm{w}$ wysokości 8\% $\mathrm{w}$ skali

roku, przy czym odsetki naliczane sq kwartalnie. $\mathrm{W}$ ciągu rozwazanego roku inflacja

wyniosła 4\%. Jakie jest rea1ne roczne oprocentowanie 1okaty Pana Kowa1skiego, $\mathrm{t}\mathrm{z}\mathrm{n}. 0$

ile procent więcej warte $\mathrm{s}\Phi \mathrm{p}\mathrm{i}\mathrm{e}\mathrm{n}\mathrm{i}_{\Phi}\mathrm{d}\mathrm{z}\mathrm{e}$, które Pan Kowalski miał na koncie po roku od

tych, które wpłacił? Wynik podač $\mathrm{z}$ dokładnością do setnych części procenta.

2. Liczba $p=\displaystyle \frac{(\sqrt[3]{54}-2)(9\sqrt[3]{4}+6\sqrt[3]{2}+4)-(2-\sqrt{3})^{3}}{\sqrt{3}+(1+\sqrt{3})^{2}}$ jest miejscem zerowym funkcji

$f(x) = ax^{2}+bx+c$. Wyznaczyč wspólczynniki $a, b, c$ oraz drugie miejsce zerowe tej

funkcji $\mathrm{w}\mathrm{i}\mathrm{e}\mathrm{d}\mathrm{z}\Phi^{\mathrm{C}}, \dot{\mathrm{z}}\mathrm{e}$ największ$\Phi$ wartości$\Phi$ funkcji jest 4, a jej wykres jest symetryczny

względem prostej $x=1.$

3. Dwie styczne do okręgu $0$ promieniu 6 przecinają się pod kątem $60^{\mathrm{o}}$. Obliczyč pole obsza-

ru ograniczonego odcinkami tych stycznych $\mathrm{i}$ krótszym $\mathrm{z}$ łuków, najakie $\mathrm{o}\mathrm{k}\mathrm{r}\Phi \mathrm{g}$ podzielony

jest punktami styczności. Wyznaczyč promień okręgu wpisanego $\mathrm{w}$ ten obszar.

4. Niech

$f(x)=$

gdy

gdy

$|x-1|\geq 1,$

$|x-1|<1.$

a) Obliczyč $f(-\displaystyle \frac{2}{3}), f(\displaystyle \frac{1+\sqrt{3}}{2})$ oraz $f(\pi-1).$

b) Narysowač wykres funkcji $f\mathrm{i}$ na jego podstawie podač zbiór wartości funkcji.

c) Rozwi$\Phi$zač nierównośč $f(x)\displaystyle \geq-\frac{1}{2}\mathrm{i}$ zaznaczyč na osi $0x$ zbiór jej rozwi$\Phi$zań.

5. Pole przekroju graniastosłupa prawidlowego $0$ podstawie kwadratowej paszczyz$\Phi$ prze-

$\mathrm{c}\mathrm{h}\mathrm{o}\mathrm{d}\mathrm{z}\Phi^{\mathrm{C}}\Phi$ przez $\mathrm{P}^{\mathrm{r}\mathrm{z}\mathrm{e}\mathrm{k}}\Phi^{\mathrm{t}\mathrm{n}}\Phi$ graniastosłupa $\mathrm{i}$ środki przeciwległych krawędzi bocznych jest

3 razy większe $\mathrm{n}\mathrm{i}\dot{\mathrm{z}}$ pole podstawy. Wyznaczyč tangens kąta nachylenia $\mathrm{P}^{\mathrm{r}\mathrm{z}\mathrm{e}\mathrm{k}}\Phi^{\mathrm{t}\mathrm{n}\mathrm{e}\mathrm{j}}$ grania-

stosłupa do podstawy. Obliczyč pole powierzchni całkowitej tego graniastosłupa $\mathrm{w}\mathrm{i}\mathrm{e}\mathrm{d}\mathrm{z}\Phi^{\mathrm{C}},$

$\dot{\mathrm{z}}\mathrm{e}$ pole rozwazanego przekroju równe jest 10.

6. Jeden $\mathrm{z}$ wierzcholków trójk$\Phi$ta prostokątnego $0$ polu 7, 5 jest punktem przecięcia pro-

stych $k:x-y+3=0$ oraz $l$ : $2x+y=0$. Wyznaczyč pozostałe wierzchołki $\mathrm{w}\mathrm{i}\mathrm{e}\mathrm{d}\mathrm{z}\Phi^{\mathrm{C}},$

$\dot{\mathrm{z}}\mathrm{e}\mathrm{l}\mathrm{e}\mathrm{z}\Phi$ one na prostych $k\mathrm{i}l$, a wierzchołek $\mathrm{k}_{\Phi^{\mathrm{t}\mathrm{a}}}$ prostego jest na prostej $l$. Sporz$\Phi$dzič

staranny rysunek.
\end{document}
