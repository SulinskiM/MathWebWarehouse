\documentclass[a4paper,12pt]{article}
\usepackage{latexsym}
\usepackage{amsmath}
\usepackage{amssymb}
\usepackage{graphicx}
\usepackage{wrapfig}
\pagestyle{plain}
\usepackage{fancybox}
\usepackage{bm}

\begin{document}

PRACA KONTROLNA $\mathrm{n}\mathrm{r} 1 -$ POZIOM ROZSZERZONY

l. Narysowač wykres funkcji $f(x)=$ 

$\mathrm{P}\mathrm{o}\mathrm{s}l\mathrm{u}\mathrm{g}\mathrm{u}\mathrm{j}_{\Phi}\mathrm{c}$ się nim podač

wzór $\mathrm{i}$ narysowač wykres funkcji $g(m)$ określaj$\Phi$cej liczbę rozwi$\Phi$zań równania $f(x)=m,$

gdzie $m$ jest parametrem rzeczywistym.

2. Rozwi$\Phi$zač równanie $\displaystyle \frac{\sin 3x}{\cos x}=$ ctg $x-\mathrm{t}\mathrm{g}x.$

3. Napisač równanie stycznej $k$ do wykresu funkcji $f(x)=x^{2}-4x+3\mathrm{w}$ punkcie $(x_{1},0),$

gdzie $x_{1}$ jest najmniejszym miejscem zerowym tej funkcji. Znalez$\acute{}$č punkt przecięcia tej

stycznej ze $\mathrm{s}\mathrm{t}\mathrm{y}\mathrm{c}\mathrm{z}\mathrm{n}\Phi$ do niej $\mathrm{p}\mathrm{r}\mathrm{o}\mathrm{s}\mathrm{t}\mathrm{o}\mathrm{p}\mathrm{a}\mathrm{d}\text{ł}_{\Phi}$ Sporządzič staranny rysunek.

4. Rozwi$\Phi$zač nierównośč $\log_{2}(x-1)-\log_{\frac{1}{2}}(4-x)-\log_{\sqrt{2}}(x-2)\leq 0.$

5. Rozwi$\Phi$zač nierównośč $\displaystyle \sqrt{x^{2}-1}+1+\frac{1}{\sqrt{x^{2}-1}}+\ldots\geq \displaystyle \frac{9}{2},$

wyrazów nieskończonego $\mathrm{c}\mathrm{i}_{\Phi \mathrm{g}}\mathrm{u}$ geometrycznego.

gdzie lewa strona jest $\mathrm{s}\mathrm{u}\mathrm{m}\Phi$

6. $\mathrm{W}$ stozek wpisano kulę, a następnie $\mathrm{w}$ obszar zawarty między $\mathrm{t}_{\Phi}\mathrm{k}\mathrm{u}1_{\Phi}\mathrm{i}$ wierzchołkiem

stozka wpisano kulę $0$ objętości 8 razy mniejszej. Ob1iczyč stosunek objętości stozka do

objętości kuli na nim opisanej.
\end{document}
