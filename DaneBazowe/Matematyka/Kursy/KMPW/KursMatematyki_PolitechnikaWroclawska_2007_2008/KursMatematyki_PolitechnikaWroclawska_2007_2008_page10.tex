\documentclass[a4paper,12pt]{article}
\usepackage{latexsym}
\usepackage{amsmath}
\usepackage{amssymb}
\usepackage{graphicx}
\usepackage{wrapfig}
\pagestyle{plain}
\usepackage{fancybox}
\usepackage{bm}

\begin{document}

PRACA KONTROLNA $\mathrm{n}\mathrm{r}6-$ POZIOM PODSTAWOWY

marzec 2008r.

l. Dwa naczynia zawieraj $\Phi^{\mathrm{W}}$ sumie 401itrów wody. Po prze1aniu pewnej części wody pierw-

szego naczynia do drugiego, $\mathrm{w}$ pierwszym naczyniu zostalo trzy razy mniej wody $\mathrm{n}\mathrm{i}\dot{\mathrm{z}}\mathrm{w}$

drugim. Gdy następnie przelano taką samą częśč wody drugiego naczynia do pierwszego,

okazało się, $\dot{\mathrm{z}}\mathrm{e}\mathrm{w}$ obu naczyniach jest tyle samo płynu. Obliczyč, ile wody było pierwotnie

$\mathrm{w}\mathrm{k}\mathrm{a}\dot{\mathrm{z}}$ dym naczyniu $\mathrm{i}\mathrm{j}\mathrm{a}\mathrm{k}_{\Phi}$ jej częśč przelewano.

2. Obwód trójk$\Phi$ta równoramiennego równy jest 20. Jakie powinny byč jego boki, by obję-

tośč bryły otrzymanej przez obrót tego trójkąta wokóf podstawy była największa?

3. Student opracował 28 spośród 45 przygotowanych na egzamin tematów. Losuje trzy

tematy. $\mathrm{J}\mathrm{e}\dot{\mathrm{z}}$ eli odpowie poprawnie na wszystkie, to dostanie ocenę bardzo $\mathrm{d}\mathrm{o}\mathrm{b}\mathrm{r}\Phi, \mathrm{j}\mathrm{e}\dot{\mathrm{z}}$ eli

na dwa- $\mathrm{d}\mathrm{o}\mathrm{b}\mathrm{r}\Phi$, a $\mathrm{j}\mathrm{e}\dot{\mathrm{z}}$ eli na jedno- $\mathrm{d}\mathrm{o}\mathrm{s}\mathrm{t}\mathrm{a}\mathrm{t}\mathrm{e}\mathrm{c}\mathrm{z}\mathrm{n}\Phi$. Jakie jest prawdopodobieństwo, $\dot{\mathrm{z}}\mathrm{e}$:

a) dostanie przynajmniej db? b) zda egzamin?

4. Narysowač staranny wykres funkcji $f(x)=x^{2}-2|x|-3$, wyznaczyč jej miejsca zerowe $\mathrm{i}$

zbiór wartości. $\mathrm{W}\mathrm{y}\mathrm{k}\mathrm{o}\mathrm{r}\mathrm{z}\mathrm{y}\mathrm{s}\mathrm{t}\mathrm{u}\mathrm{j}_{\Phi}\mathrm{c}$ wykres funkcji $f$:

a) narysowač wykres funkcji $h(x)=x^{2}-2x-2|x-1|-1.$

b) $\mathrm{p}\mathrm{o}\mathrm{s}\text{ł} \mathrm{u}\mathrm{g}\mathrm{u}\mathrm{j}_{\Phi}\mathrm{c}$ się powyzszymi wykresami określič, dla jakich wartości parametru rze-

czywistego $m$ równanie $f(x)=h(x)+m$ ma dokładnie jedno $\mathrm{r}\mathrm{o}\mathrm{z}\mathrm{w}\mathrm{i}_{\Phi}$zanie.

5. Państwo Kowalscy $\mathrm{s}\Phi$ właścicielami działki budowlanej $\mathrm{w}$

kształcie trójk$\Phi$ta prostokątnego $0$ przyprostokątnych dfugości

30 $\mathrm{m}\mathrm{i}40\mathrm{m}$. Postanowili podzielič ją na dwie równej wartości

części zgodnie ze schematem obok. Wyznaczyč długośč odcinka

$\overline{BK}\mathrm{w}\mathrm{i}\mathrm{e}\mathrm{d}\mathrm{z}\Phi^{\mathrm{C}}, \dot{\mathrm{z}}\mathrm{e}$ jeden metr kwadratowy działki $\mathrm{c}\mathrm{z}\mathrm{w}\mathrm{o}\mathrm{r}\mathrm{o}\mathrm{k}_{\Phi}$tnej

jest póltora raza drozszy $\mathrm{n}\mathrm{i}\dot{\mathrm{z}}$ jeden metr kwadratowy dzialki

trójk$\Phi$tnej. Która $\mathrm{z}$ działek ma większy obwód $\mathrm{i}0$ ile? Wynik

podač $\mathrm{z}$ dokładnościq do 10 cm.
\begin{center}
\includegraphics[width=45.312mm,height=37.740mm]{./KursMatematyki_PolitechnikaWroclawska_2007_2008_page10_images/image001.eps}
\end{center}
{\it A}

{\it L}

{\it B  K C}

6. Boki $\overline{AB}, \overline{AC}$ trójk$\Phi$ta zawarte są $\mathrm{w}$ prostych $l$ : $x-y-1=0$ oraz $k$ : $x+2y+2=0.$

Wyznaczyč współrzędne wierzcholków $B, C \mathrm{w}\mathrm{i}\mathrm{e}\mathrm{d}\mathrm{z}\Phi^{\mathrm{C}}, \dot{\mathrm{z}}\mathrm{e}$ punkt $P(1,1)$ jest środkiem

boku $\overline{BC}$. Wyznaczyč współrzędne wierzcholków trójk$\Phi$ta otrzymanego przez odbicie

symetryczne powyzszego trójk$\Phi$ta względem boku $\overline{BC}.$
\end{document}
