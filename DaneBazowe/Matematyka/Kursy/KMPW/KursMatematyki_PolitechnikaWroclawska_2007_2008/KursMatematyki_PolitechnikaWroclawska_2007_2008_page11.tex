\documentclass[a4paper,12pt]{article}
\usepackage{latexsym}
\usepackage{amsmath}
\usepackage{amssymb}
\usepackage{graphicx}
\usepackage{wrapfig}
\pagestyle{plain}
\usepackage{fancybox}
\usepackage{bm}

\begin{document}

PRACA KONTROLNA $\mathrm{n}\mathrm{r} 6-$ POZIOM ROZSZERZONY

l. Rozwi$\Phi$zač $\mathrm{i}$ zinterpretowač graficznie układ równań 

1,

1.

2. Niech $f(x)=\log_{2}x, g(x)=x+2, h(x)=|x|.$

a) Narysowač wykresy funkcji $f\mathrm{o}h\mathrm{o}g$ oraz $g0f0h$

b) Rozwi$\Phi$zač nierównośč $(f\mathrm{o}h\mathrm{o}g)(x)<(g\mathrm{o}f\mathrm{o}h)(x).$

3. Rzucamy kolejno trzy razy kostką do gry. Jakie jest prawdopodobieństwo, $\dot{\mathrm{z}}\mathrm{e}\mathrm{w}$ otrzy-

manym $\mathrm{c}\mathrm{i}_{\Phi \mathrm{g}}\mathrm{u}\mathrm{s}\Phi$ przynajmniej dwie,,szóstki'' lub suma oczek przekroczy 14?

4. Dany jest wielomian $W(x) = x^{3}+ax+b$, gdzie $b \neq 0$. Wykazač, $\dot{\mathrm{z}}\mathrm{e} W(x)$ posiada

pierwiastek podwójny wtedy $\mathrm{i}$ tylko wtedy, gdy spełniony jest warunek $4a^{3}+27b^{2}=0.$

Wyrazič pierwiastki za pomocą współczynnika $b.$

5. $\mathrm{W}$ ostrosłupie prawidłowym czworokątnym dany jest kąt $\alpha$ nachylenia ściany bocznej

do podstawy oraz obwód ściany bocznej równy $l$. Obliczyč objętośč tego ostrosłupa.

6. Narysowač staranny wykres funkcji $f(x)=\cos x-\sqrt{3}|\sin x| \mathrm{w}$ przedziale $[0,2\pi] \mathrm{i}$ wy-

znaczyč zbiór jej wartości.

a) $\mathrm{P}\mathrm{o}\mathrm{s}\text{ł} \mathrm{u}\mathrm{g}\mathrm{u}\mathrm{j}_{\Phi}\mathrm{c}$ się wykresem podač liczbę rozwi$\Phi$zań równania $f(x)=m\mathrm{w}$ zalezności

od parametru rzeczywistego $m.$

b) $\mathrm{R}\mathrm{o}\mathrm{z}\mathrm{w}\mathrm{i}_{\Phi}\mathrm{z}\mathrm{u}\mathrm{j}_{\Phi}\mathrm{c}$ odpowiednie równanie $\mathrm{i}$ korzystając $\mathrm{z}$ wykresu podač $\mathrm{r}\mathrm{o}\mathrm{z}\mathrm{w}\mathrm{i}_{\Phi}$zanie nie-

równości $f(x)\leq-\sqrt{2}.$
\end{document}
