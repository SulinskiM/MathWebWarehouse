\documentclass[a4paper,12pt]{article}
\usepackage{latexsym}
\usepackage{amsmath}
\usepackage{amssymb}
\usepackage{graphicx}
\usepackage{wrapfig}
\pagestyle{plain}
\usepackage{fancybox}
\usepackage{bm}

\begin{document}

PRACA KONTROLNA $\mathrm{n}\mathrm{r}2-$ POZIOM PODSTAWOWY

listopad $2007\mathrm{r}.$

l. Trzy liczby dodatnie $\mathrm{t}\mathrm{w}\mathrm{o}\mathrm{r}\mathrm{z}\Phi \mathrm{c}\mathrm{i}_{\Phi \mathrm{g}}$ geometryczny. Suma tych liczb równa jest 26, a suma

ich odwrotności wynosi 0.7(2). Wyznaczyč te 1iczby.

2. Pole powierzchni bocznej ostrosłupa prawidłowego $\mathrm{c}\mathrm{z}\mathrm{w}\mathrm{o}\mathrm{r}\mathrm{o}\mathrm{k}_{\Phi^{\mathrm{t}}}$nego j$\mathrm{e}\mathrm{s}\mathrm{t}2$ razy większe $\mathrm{n}\mathrm{i}\dot{\mathrm{z}}$

pole podstawy. $\mathrm{W}$ trójk$\Phi$t otrzymany $\mathrm{w}$ przekroju ostrosfupa $\mathrm{p}\mathrm{a}$szczyz $\Phi \mathrm{p}\mathrm{r}\mathrm{z}\mathrm{e}\mathrm{c}\mathrm{h}\mathrm{o}\mathrm{d}\mathrm{z}\Phi^{\mathrm{C}}\Phi$

przez jego wysokośč $\mathrm{i} \mathrm{P}^{\mathrm{r}\mathrm{z}\mathrm{e}\mathrm{k}}\Phi^{\mathrm{t}\mathrm{n}\mathrm{q}}$ podstawy wpisano kwadrat, którego jeden bok jest

zawarty $\mathrm{w}\mathrm{P}^{\mathrm{r}\mathrm{z}\mathrm{e}\mathrm{k}}\Phi^{\mathrm{t}\mathrm{n}\mathrm{e}\mathrm{j}}$ podstawy. Obliczyč stosunek pola tego kwadratu do pola podstawy

ostrosłupa. Sporz$\Phi$dzič staranny rysunek.

3. Wykonač działania $\mathrm{i}$ zapisač $\mathrm{w}$ najprostszej postaci wyrazenie

$s(a,b)= (\displaystyle \frac{a^{2}+b^{2}}{a^{2}-b^{2}}-\frac{a^{3}+b^{3}}{a^{3}-b^{3}})$ : $(\displaystyle \frac{a^{2}}{a^{3}-b^{3}}-\frac{a}{a^{2}+ab+b^{2}})$

Wyznaczyč wysokośč trójk$\Phi$ta prostokątnego wpisanego $\mathrm{w}\mathrm{o}\mathrm{k}\mathrm{r}\Phi \mathrm{g}\mathrm{o}$ promieniu 6 opusz-

$\mathrm{c}\mathrm{z}\mathrm{o}\mathrm{n}\Phi \mathrm{z}$ wierzchołka $\mathrm{k}_{\Phi^{\mathrm{t}\mathrm{a}}}$ prostego wiedząc, $\dot{\mathrm{z}}\mathrm{e}$ tangens jednego $\mathrm{z}$ k$\Phi$tów ostrych tego

trójk$\Phi$ta równy jest $s(\sqrt{5}+\sqrt{3},\sqrt{5}-\sqrt{3}).$

4. Wielomian $W(x) =x^{3}-x^{2}+bx+c$ jest podzielny przez $(x+3)$, a reszta $\mathrm{z}$ dzielenia

tego wielomianu przez $(x-3)$ równa jest 6. Wyznaczyč $b\mathrm{i} c$, a następnie rozwi$\Phi$zač

nierównośč $(x+1)W(x-1)-(x+2)W(x-2)\leq 0.$

5. $\mathrm{W}$ ramach przygotowań do EURO 2012 zap1anowano budowe komp1eksu sportowego zło-

$\dot{\mathrm{z}}$ onego $\mathrm{z}$ czterech jednakowych hal sportowych $\mathrm{w}$ kształcie pófkul $0$ środkach $\mathrm{w}$ rogach

kwadratu $0$ boku 100 $\mathrm{m}\mathrm{i}$ piątej hali $\mathrm{w}$ ksztafcie pófkuli stycznej do czterech pozosta-

fych. Jakie powinny byč wymiary tych hal, by koszt ich budowy był najmniejszy, $\mathrm{j}\mathrm{e}\dot{\mathrm{z}}$ eli

wiadomo, $\dot{\mathrm{z}}\mathrm{e}$ jest on proporcjonalny do pola powierzchni dachu hali?

6. $\mathrm{W}$ trójk$\Phi$cie prostokątnym $0$ kącie prostym przy wierzchoku $C$ na przedłuzeniu przeciw-

$\mathrm{p}\mathrm{r}\mathrm{o}\mathrm{s}\mathrm{t}\mathrm{o}\mathrm{k}_{\Phi^{\mathrm{t}}}\mathrm{n}\mathrm{e}\mathrm{j}$ AB odmierzono odcinek $BD\mathrm{t}\mathrm{a}\mathrm{k}, \dot{\mathrm{z}}\mathrm{e}|BD|=|BC|$. Wyznaczyč $|CD|$ oraz

obliczyč pole trójkta $\triangle ACD, \mathrm{j}\mathrm{e}\dot{\mathrm{z}}$ eli $|BC|=5, |AC|=12$. Sporz$\Phi$dzič staranny rysunek.
\end{document}
