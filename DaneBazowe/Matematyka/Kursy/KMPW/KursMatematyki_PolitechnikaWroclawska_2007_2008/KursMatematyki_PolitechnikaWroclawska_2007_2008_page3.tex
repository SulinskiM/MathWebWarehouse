\documentclass[a4paper,12pt]{article}
\usepackage{latexsym}
\usepackage{amsmath}
\usepackage{amssymb}
\usepackage{graphicx}
\usepackage{wrapfig}
\pagestyle{plain}
\usepackage{fancybox}
\usepackage{bm}

\begin{document}

PRACA KONTROLNA $\mathrm{n}\mathrm{r} 2-$ POZIOM ROZSZERZONY

l. Znalez/č wszystkie wartości parametru rzeczywistego $m$, dla których pierwiastki trójmia-

nu kwadratowego $f(x)=(m-2)x^{2}-(m+1)x-m \mathrm{s}\mathrm{p}\mathrm{e}\mathrm{f}\mathrm{n}\mathrm{i}\mathrm{a}\mathrm{j}_{\Phi}$ nierównośč $|x_{1}|+|x_{2}|\leq 1.$

2. Wyznaczyč dziedzinę funkcji

$f(x)=\displaystyle \frac{\sqrt{2^{4-x^{2}}-4^{x}}}{\log(2-x-x^{2}-\ldots)}.$

3. Grupa l75 robotników miala wykonač pewną pracę $\mathrm{w}$ określonym terminie. Po upływie

30 dni wspólnej pracy przesyłano codziennie po 3 robotników na inne stanowiska, wsku-

tek czego robota została wykonana $\mathrm{z}$ opóz/nieniem 21 $\mathrm{d}\mathrm{n}\mathrm{i}. \mathrm{W}$ ciągu ilu dni miała byč

wykonana praca według planu?

4. Wyznaczyč promień okręgu opisanego na czworokącie ABCD, $\mathrm{w}$ którym $\mathrm{k}_{\Phi^{\mathrm{t}}}$ przy wierz-

cholku $A$ ma miarę $\alpha, \mathrm{k}_{\Phi^{\mathrm{t}\mathrm{y}}}$ przy wierzchołkach $B,  D\mathrm{s}\Phi$ proste oraz $|BC|=a, |AD|=b.$

Sporz$\Phi$dzič staranny rysunek.

5. Narysowač staranny wykres funkcji $f(x)=\displaystyle \frac{\sin 2x-|\sin x|}{\sin x}.$

$\mathrm{W}$ przedziale $[0,\pi]$ wyznaczyč $\mathrm{r}\mathrm{o}\mathrm{z}\mathrm{w}\mathrm{i}_{\Phi}$zania nierówności $f(x)<2(\sqrt{2}-1)\cos^{2}x.$

6. Pole przekroju graniastosłupa prawidlowego $0$ podstawie kwadratowej paszczyz$\Phi$ prze-

$\mathrm{c}\mathrm{h}\mathrm{o}\mathrm{d}\mathrm{z}\Phi^{\mathrm{C}}\Phi$ przez $\mathrm{P}^{\mathrm{r}\mathrm{z}\mathrm{e}\mathrm{k}}\Phi^{\mathrm{t}\mathrm{n}}\Phi$ graniastosłupa $\mathrm{i}$ środek jednej $\mathrm{z}$ krawędzi podstawy jest 3 razy

większe $\mathrm{n}\mathrm{i}\dot{\mathrm{z}}$ pole podstawy. Wyznaczyč tangens $\mathrm{k}_{\Phi^{\mathrm{t}\mathrm{a}}}$ nachylenia $\mathrm{P}^{\mathrm{r}\mathrm{z}\mathrm{e}\mathrm{k}}\Phi^{\mathrm{t}\mathrm{n}\mathrm{e}\mathrm{j}}$ graniastosłu-

pa do podstawy. Obliczyč pole powierzchni całkowitej tego graniastosłupa $\mathrm{w}\mathrm{i}\mathrm{e}\mathrm{d}\mathrm{z}\Phi^{\mathrm{C}}, \dot{\mathrm{z}}\mathrm{e}$

pole rozwazanego przekroju równe jest 15. Sporządzič staranny rysunek.
\end{document}
