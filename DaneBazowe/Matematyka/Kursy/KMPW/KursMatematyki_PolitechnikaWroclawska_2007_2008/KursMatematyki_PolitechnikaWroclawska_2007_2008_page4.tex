\documentclass[a4paper,12pt]{article}
\usepackage{latexsym}
\usepackage{amsmath}
\usepackage{amssymb}
\usepackage{graphicx}
\usepackage{wrapfig}
\pagestyle{plain}
\usepackage{fancybox}
\usepackage{bm}

\begin{document}

PRACA KONTROLNA $\mathrm{n}\mathrm{r}3-$ POZIOM PODSTAWOWY

grudzień $2007\mathrm{r}.$

l. Rozwi$\Phi$zač równanie

$\sqrt{3-x}+\sqrt{3x-2}=2.$

2. Sześč kostek sześciennych $0$ objętościach 1, 2, 4, 8, 16 $\mathrm{i}32\mathrm{d}\mathrm{m}^{3}$ ustawiono $\mathrm{w}$ piramidę,

$\mathrm{u}\mathrm{k}l\mathrm{a}\mathrm{d}\mathrm{a}\mathrm{j}_{\Phi}\mathrm{c}\mathrm{j}\mathrm{e}\mathrm{d}\mathrm{n}\Phi$ kostkę na drugiej poczynając od największej. Czy wysokośč piramidy

przekroczy 120 cm? Odpowied $\acute{\mathrm{z}}$ uzasadnič bez prowadzenia obliczeń przyblizonych.

3. ()$\mathrm{P}\mathrm{a}\mathrm{n}\mathrm{W}$ wybrał się na spacer do parku mającego ksztalt $\mathrm{p}\mathrm{r}\mathrm{o}\mathrm{s}\mathrm{t}\mathrm{o}\mathrm{k}_{\Phi^{\mathrm{t}}}\mathrm{a}\mathrm{o}$ wymiarach 400 $\mathrm{m}$

na 300 $\mathrm{m}$, podzielonego alejkami na 12 kwadratów $0$ boku 100 $\mathrm{m}$, jak na rysunku ponizej.

Postanowił przejśč od punktu $A$ do $B, l_{\Phi}$cznie $700\mathrm{m}$, wybierajqc przypadkowo alejkę

na $\mathrm{k}\mathrm{a}\dot{\mathrm{z}}$ dym rozwidleniu. Jakie jest prawdopodobieństwo, $\dot{\mathrm{z}}\mathrm{e}$ Pan $\mathrm{W}$ przejdzie środkow$\Phi$

$\mathrm{a}\mathrm{l}\mathrm{e}\mathrm{j}\mathrm{k}_{\Phi^{\mathrm{O}\mathrm{Z}\mathrm{n}\mathrm{a}\mathrm{c}\mathrm{z}\mathrm{o}\mathrm{n}}\Phi}$ na rysunku $x$?
\begin{center}
\includegraphics[width=36.216mm,height=28.752mm]{./KursMatematyki_PolitechnikaWroclawska_2007_2008_page4_images/image001.eps}
\end{center}
$\sqrt{}^{B}$

4. $\mathrm{P}\mathrm{o}\mathrm{d}\mathrm{s}\mathrm{t}\mathrm{a}\mathrm{w}\Phi$ trójk$\Phi$ta równoramiennego jest odcinek AB $0$ końcach $A(-1,1), B(3,3), \mathrm{a}$

wierzchołek $C\mathrm{l}\mathrm{e}\dot{\mathrm{z}}\mathrm{y}$ na paraboli $0$ równaniu $y^{2}=x+1$. Wyznaczyč współrzędne wierz-

chołka $C$ oraz pole trójk$\Phi$ta $ABC$. Sporz$\Phi$dzič rysunek.

5. Na jednym rysunku sporz$\Phi$dzič dokładne wykresy funkcji $\sin x, \cos x$, tg $x$ oraz ctg $x$

$\mathrm{w}$ przedziale $(0,\displaystyle \frac{\pi}{2}) \mathrm{i}$ zaznaczyč na nich

ctg $(\displaystyle \cos\frac{\pi}{4}), \displaystyle \cos(\sin\frac{\pi}{3}), \displaystyle \sin(\cos\frac{\pi}{3})$, tg $(\displaystyle \sin\frac{\pi}{2})$

Uporz$\Phi$dkowač powyzsze liczby od najmniejszej do największej. Uzasadnič te relacje za

$\mathrm{P}^{\mathrm{o}\mathrm{m}\mathrm{o}\mathrm{c}}\Phi$ odpowiednich nierówności.

6. $\mathrm{W}$ ostrosłupie prawidłowym trójkątnym kąt między ścianami bocznymi ma miarę $\alpha, \mathrm{a}$

odległośč krawędzi podstawy od przeciwległej krawędzi bocznej jest równa $d$. Obliczyč

objętośč ostrosłupa.
\end{document}
