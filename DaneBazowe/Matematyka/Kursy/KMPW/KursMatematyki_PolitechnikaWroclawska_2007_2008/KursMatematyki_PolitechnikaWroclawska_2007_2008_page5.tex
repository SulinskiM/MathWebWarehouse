\documentclass[a4paper,12pt]{article}
\usepackage{latexsym}
\usepackage{amsmath}
\usepackage{amssymb}
\usepackage{graphicx}
\usepackage{wrapfig}
\pagestyle{plain}
\usepackage{fancybox}
\usepackage{bm}

\begin{document}

PRACA KONTROLNA nr 3 -POZIOM ROZSZERZONY

1. $\mathrm{S}\mathrm{t}\mathrm{o}\mathrm{s}\mathrm{u}\mathrm{j}_{\Phi}\mathrm{c}$ zasadę indukcji matematycznej, udowodnič prawdziwośč wzoru

$\left(\begin{array}{l}
3\\
2
\end{array}\right) + \left(\begin{array}{l}
5\\
2
\end{array}\right) +\ldots+\left(\begin{array}{ll}
2n+ & 1\\
2 & 
\end{array}\right) =\displaystyle \frac{n(n+1)(4n+5)}{6}$

dla $n\geq 1.$

2. Wojtuś wylosował $\mathrm{j}\mathrm{e}\mathrm{d}\mathrm{n}\Phi$ monetę ze skarbonki zawierającej 3 złotówki, 4 dwuzłotówki $\mathrm{i}3$

pięciozlotówki. Następnie, $\mathrm{w}$ zalezności od wyniku pierwszego losowania, wylosował jesz-

cze trzy monety, gdy za pierwszym razem otrzymał złotówkę, dwie monety, gdy pierwsza

była dwuzlotówk$\Phi$ oraz jedną monetę, gdy $\mathrm{w}$ pierwszym losowaniu dostał pięciozłotów-

kę. Obliczyč prawdopodobieństwo, $\dot{\mathrm{z}}\mathrm{e}$, postępuj$\Phi$c $\mathrm{w}$ ten sposób, zgromadził $\text{ł}_{\Phi}$cznie co

najmniej 8 złotych.

3. Jednym $\mathrm{z}$ wierzchołków kwadratujest punkt $A(2,2)$, a środkiemjednego $\mathrm{z}$ przeciwległych

boków jest punkt $M(-\displaystyle \frac{1}{2},-\frac{1}{2})$. Wyznaczyč współrzędne pozostałych wierzchołków oraz

równanie okręgu opisanego na tym kwadracie.

4. Rozwi$\Phi$zač nierównośč

$\displaystyle \frac{1}{\sqrt{3^{x+1}-2}}\geq\frac{1}{4-(\sqrt{3})^{x+2}}.$

5. $\mathrm{W}$ ostrosłup prawidłowy trójkątny wpisano walec, którego podstawa $\mathrm{l}\mathrm{e}\dot{\mathrm{z}}\mathrm{y}$ na podstawie

ostrosłupa. Srednica podstawy walcajest równajego wysokości. Znalez/č tangens $\mathrm{k}_{\Phi^{\mathrm{t}\mathrm{a}}}$ na-

chylenia krawędzi bocznej ostrosłupa do podstawy, dla którego stosunek objętości walca

do objętości ostrosłupa jest największy. Podač ten największy stosunek $\mathrm{w}$ procentach.

6. Długości boków trapezu opisanego na okręgu $0$ promieniu $R\mathrm{t}\mathrm{w}\mathrm{o}\mathrm{r}\mathrm{z}\Phi \mathrm{c}\mathrm{i}_{\Phi \mathrm{g}}$ arytmetyczny,

przy czym najkrótszy bok ma długośč $\displaystyle \frac{3}{4}R$. Obliczyč długości obu podstaw trapezu oraz

cosinus $\mathrm{k}_{\Phi^{\mathrm{t}\mathrm{a}}}$ pomiędzy $\mathrm{P}^{\mathrm{r}\mathrm{z}\mathrm{e}\mathrm{k}}\Phi^{\mathrm{t}\mathrm{n}\mathrm{y}\mathrm{m}\mathrm{i}}$. Sporządzič rysunek $\mathrm{p}\mathrm{r}\mathrm{z}\mathrm{y}\mathrm{j}\mathrm{m}\mathrm{u}\mathrm{j}_{\Phi}\mathrm{c}R=2$ cm.
\end{document}
