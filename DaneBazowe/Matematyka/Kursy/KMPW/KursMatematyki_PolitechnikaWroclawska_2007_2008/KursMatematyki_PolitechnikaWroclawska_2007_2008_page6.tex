\documentclass[a4paper,12pt]{article}
\usepackage{latexsym}
\usepackage{amsmath}
\usepackage{amssymb}
\usepackage{graphicx}
\usepackage{wrapfig}
\pagestyle{plain}
\usepackage{fancybox}
\usepackage{bm}

\begin{document}

PRACA KONTROLNA $\mathrm{n}\mathrm{r}4-$ POZIOM PODSTAWOWY

styczeń 2008r.

l. Ramka z drutu 0 długości 1 ma kształt kwadratu zakończonego

trójk$\Phi$tem równoramiennym, jak na rysunku. Bok kwadratu wynosi

a, natomiast ramię trójkąta równe jest b. Wyznaczyč a i b tak, by

pola kwadratu i trójkąta byly jednakowe.

2. Niech

$A=\{(x,y):x\in \mathbb{R},y\in \mathbb{R},y=-x+a,a\in\langle-2,2\rangle\},$

$B=\displaystyle \{(x,y):x\in \mathbb{R},y\in \mathbb{R},y=kx,k\in\langle\frac{1}{2},1\rangle\}.$

$\mathrm{W}\mathrm{P}^{\mathrm{r}\mathrm{o}\mathrm{s}\mathrm{t}\mathrm{o}\mathrm{k}}\Phi^{\mathrm{t}\mathrm{n}\mathrm{y}\mathrm{m}}$ układzie współrzędnych narysowač zbiór $A\cap B\mathrm{i}$ obliczyč jego pole.

Sprawdzič, czy punkt $(\displaystyle \frac{1}{2},\frac{3}{4})$ nalezy do zbioru $A\cap B.$

3. Dany jest stozek ścięty, $\mathrm{w}$ którym pole dolnej podstawy jest 4 razy większe od po1a

górnej. $\mathrm{W}$ stozek wpisano walec $\mathrm{t}\mathrm{a}\mathrm{k}, \dot{\mathrm{z}}\mathrm{e}$ dolna podstawa walca $\mathrm{l}\mathrm{e}\dot{\mathrm{z}}\mathrm{y}$ na dolnej podstawie

stozka, a brzeg górnej podstawy $\mathrm{l}\mathrm{e}\dot{\mathrm{z}}\mathrm{y}$ na jego powierzchni bocznej. $\mathrm{J}\mathrm{a}\mathrm{k}_{\Phi}$ częśč objętości

stozka ściętego stanowi objętośč walca, $\mathrm{j}\mathrm{e}\dot{\mathrm{z}}$ eli wysokośč walca jest 3 razy mniejsza od

wysokości stozka? Odpowied $\acute{\mathrm{z}}$ podač $\mathrm{w}$ procentach $\mathrm{z}$ dokładności$\Phi$ do jednego promila.

Sporz$\Phi$dzič staranny rysunek przekroju osiowego bryly.

4. Rozwi$\Phi$zač nierównośč $f(x)+3x>1$, gdzie $f(x)=\displaystyle \frac{1-3x}{\sqrt{2-\frac{3x+1}{x-2}}}.$

5. Dane $\mathrm{s}\Phi$ dwa $\displaystyle \mathrm{c}\mathrm{i}_{\Phi \mathrm{g}}\mathrm{i}a_{n}=\frac{1}{n}$ oraz $b_{n}=\displaystyle \frac{n-2}{(n+2)(n+4)}$. Zbadač monotonicznośč ciqgu

$c_{n}=(n-1)a_{n+1}+2b_{2n}.$

Czy $\mathrm{c}\mathrm{i}_{\Phi \mathrm{g}}c_{n}$ jest ograniczony? Dla jakich $n$ spefniona jest nierównośč $\displaystyle \frac{3}{4}<c_{n}<1$?

6. Okręgi $0$ promieniach $r\mathrm{i}2r\mathrm{p}\mathrm{r}\mathrm{z}\mathrm{e}\mathrm{c}\mathrm{i}\mathrm{n}\mathrm{a}\mathrm{j}_{\Phi}$ się $\mathrm{w}$ punktach A $\mathrm{i}B, \mathrm{b}\text{ę} \mathrm{d}_{\Phi}$cych wierzchołkami

trójk$\Phi$ta równobocznego $ABC$ wpisanego $\mathrm{w}$ jeden $\mathrm{z}$ okręgów. Obliczyč pole deltoidu

ADBC, którego wierzchołek $D\mathrm{l}\mathrm{e}\dot{\mathrm{z}}\mathrm{y}$ na drugim okręgu oraz wyznaczyč sinus kąta przy

wierzchołku $D.$
\end{document}
