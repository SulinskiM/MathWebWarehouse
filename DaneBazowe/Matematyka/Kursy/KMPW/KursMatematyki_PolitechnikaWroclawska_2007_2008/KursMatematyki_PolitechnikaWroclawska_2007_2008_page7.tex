\documentclass[a4paper,12pt]{article}
\usepackage{latexsym}
\usepackage{amsmath}
\usepackage{amssymb}
\usepackage{graphicx}
\usepackage{wrapfig}
\pagestyle{plain}
\usepackage{fancybox}
\usepackage{bm}

\begin{document}

PRACA KONTROLNA $\mathrm{n}\mathrm{r} 4-$ POZIOM ROZSZERZONY

l. Dany jest romb ABCD $0$ boku $a\mathrm{i}\mathrm{k}_{\Phi}\mathrm{c}\mathrm{i}\mathrm{e}$ ostrym $\alpha. \mathrm{Z}$ wierzcholka $A\mathrm{k}_{\Phi^{\mathrm{t}\mathrm{a}}}$ ostrego po-

prowadzono dwa jednakowej długości odcinki $0$ końcach zawartych $\mathrm{w}$ bokach $BC\mathrm{i}CD.$

Wyznaczyč długości tych odcinków oraz sinusy kątów, na jaki został podzielony $\mathrm{k}_{\Phi^{\mathrm{t}}}\alpha$

$\mathrm{w}\mathrm{i}\mathrm{e}\mathrm{d}\mathrm{z}\Phi^{\mathrm{C}}, \dot{\mathrm{z}}\mathrm{e}$ pole środkowego deltoidu jest równe połowie pola danego rombu.

2. Napisač równanie stycznej do krzywej $f(x)=\displaystyle \frac{x}{x^{2}-1} \mathrm{w}$ punkcie $x_{0} = 2$. Wykazač, $\dot{\mathrm{z}}\mathrm{e}$

obrazem tej stycznej $\mathrm{w}$ symetrii względem punktu $(0,0)$ jest prosta, która jest styczną

do tej samej krzywej. Wyznaczyč odległośč między tymi stycznymi.

3. Niech

$A=\{(x,y):x\in \mathbb{R},y\in \mathbb{R},|x-1|+x\geq y+|y-2|\},$

$B=\displaystyle \{(x,y):x\in \mathbb{R},y\in \mathbb{R},|x-1|+\frac{1}{4}|y|\leq 1\}.$

Na płaszczy $\acute{\mathrm{z}}\mathrm{n}\mathrm{i}\mathrm{e}OXY$ narysowač zbiory $A\cap B$ oraz $B'\backslash A.$

4. Dane jest równanie

8 $(\sin\alpha+4)x^{2}-8(\sin\alpha+1)x+1=0,$

gdzie $\alpha \in \langle 0,  2\pi\rangle$. Dla jakich wartości $\mathrm{k}_{\Phi^{\mathrm{t}\mathrm{a}}}\alpha$ suma odwrotności pierwiastków tego

równania jest równa co najmniej 8 $(\cos\alpha-(\cos\alpha)^{-1}+1)$ ?

5. Zbadač funkcję $f(m)=\displaystyle \frac{y}{x}$, gdzie para $x\mathrm{i}y$ jest $\mathrm{r}\mathrm{o}\mathrm{z}\mathrm{w}\mathrm{i}_{\Phi}$zaniem układu równań

$\left\{\begin{array}{l}
(m-2)x+(m+2)y=m^{2}-1\\
(m+2)x+(m-2)y=m^{2}+1,
\end{array}\right.$

$\mathrm{z}$ parametrem rzeczywistym $m$. Sporz$\Phi$dzič wykres funkcji $f(m).$

6. $\mathrm{W}$ stozek $0$ promieniu podstawy $r\mathrm{i}\mathrm{t}\mathrm{w}\mathrm{o}\mathrm{r}\mathrm{z}\Phi^{\mathrm{C}\mathrm{e}\mathrm{j}}l$ wpisano ostrosłup prawidłowy trójkątny

$\mathrm{t}\mathrm{a}\mathrm{k}, \dot{\mathrm{z}}\mathrm{e}$ wierzchołek tego ostrosłupa pokrywa się ze środkiem podstawy stozka, a pozo-

stałe wierzchołki $\mathrm{l}\mathrm{e}\mathrm{z}\Phi$ na ścianie bocznej stozka. Jaka jest maksymalna objętośč tego

ostrosłupa? Sporzqdzič staranny rysunek.
\end{document}
