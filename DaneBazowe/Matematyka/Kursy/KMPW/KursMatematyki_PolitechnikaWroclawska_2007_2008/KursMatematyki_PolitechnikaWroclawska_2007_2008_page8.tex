\documentclass[a4paper,12pt]{article}
\usepackage{latexsym}
\usepackage{amsmath}
\usepackage{amssymb}
\usepackage{graphicx}
\usepackage{wrapfig}
\pagestyle{plain}
\usepackage{fancybox}
\usepackage{bm}

\begin{document}

PRACA KONTROLNA $\mathrm{n}\mathrm{r}5-$ POZIOM PODSTAWOWY

luty $2008\mathrm{r}.$

l. Ile razy objętośč ostrosłupa trójkątnego prawidfowego opisanego na stozku $0$ objętości $V$

jest większa od objętości ostrosłupa trójkątnego prawidlowego wpisanego $\mathrm{w}$ ten stozek?

2. Rozwi$\Phi$zač nierównośč

$|4x^{2}-4|+2x\geq|1-x|+2.$

3. Kamilek ma 2 latka $\mathrm{i}85$ cm wzrostu. Przez kolejne 31ata będzie rósł średnio 1cm mie-

sięcznie. Potem $\mathrm{w}\mathrm{c}\mathrm{i}_{\Phi \mathrm{g}}\mathrm{u}\mathrm{k}\mathrm{a}\dot{\mathrm{z}}$ dych 10 miesięcy będzie rósł $0$ 10\% wolniej $\mathrm{n}\mathrm{i}\dot{\mathrm{z}}\mathrm{w}$ poprzednim

okresie. Jaki wzrost będzie miał chłopczyk $\mathrm{w}$ dniu swoich 15-tych urodzin? Wynik podač

$\mathrm{z}$ dokładności$\Phi$ do 5 mm.

4. Uzasadnič, wykonuj $\Phi^{\mathrm{C}}$ odpowiednie obliczenia, $\dot{\mathrm{z}}\mathrm{e}\mathrm{z}$ kartki papieru $\mathrm{w}$ kształcie sześciok$\Phi$ta

foremnego $0$ boku $a= 2(1+\sqrt{3}) \mathrm{m}\mathrm{o}\dot{\mathrm{z}}$ na wyci$\Phi$č 19 kółek $0$ promieniu l. Czy istnieje

mniejszy sześciok$\Phi$t foremny, $\mathrm{z}$ którego $\mathrm{m}\mathrm{o}\dot{\mathrm{z}}$ na wyciąč taką $\mathrm{s}\mathrm{a}\mathrm{m}\Phi$ ilośč identycznych kółek?

5. Punkty (l, l) $\mathrm{i} (5,4) \mathrm{s}\Phi$ dwoma wierzchołkami rombu $0$ polu 15. Opisač konstrukcje

wszystkich rombów spełniaj $\Phi^{\mathrm{C}}\mathrm{y}\mathrm{c}\mathrm{h}$ podane warunki. Wyznaczyč wspófrzędne pozostałych

wierzcholków, przy załozeniu, $\dot{\mathrm{z}}\mathrm{e}$ nie wszystkie wierzchołki $\mathrm{l}\mathrm{e}\mathrm{z}\Phi \mathrm{w}$ I čwiartce układu

wspólrzędnych.

6. Wyznaczyč równanie krzywej będ$\Phi$cej zbiorem wszystkich środków cięciw paraboli

$y=(x-1)^{2}+1 \mathrm{p}\mathrm{r}\mathrm{z}\mathrm{e}\mathrm{c}\mathrm{h}\mathrm{o}\mathrm{d}_{\mathrm{Z}\Phi}$cych przez punkt $P(-1,2).$

(Wsk. Zauwazyč, $\dot{\mathrm{z}}\mathrm{e}\mathrm{j}\mathrm{e}\dot{\mathrm{z}}$ eli $x_{1}, x_{2}$ są pierwiastkami trójmianu kwadratowego $y=ax^{2}+bx+c,$

to prawdziwa jest równośč $x_{1}+x_{2}=\displaystyle \frac{-b}{a}.$)
\end{document}
