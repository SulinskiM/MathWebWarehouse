\documentclass[a4paper,12pt]{article}
\usepackage{latexsym}
\usepackage{amsmath}
\usepackage{amssymb}
\usepackage{graphicx}
\usepackage{wrapfig}
\pagestyle{plain}
\usepackage{fancybox}
\usepackage{bm}

\begin{document}

PRACA KONTROLNA $\mathrm{n}\mathrm{r} 5-$ POZIOM ROZSZERZONY

l. Rozwi$\Phi$zač równanie

$\displaystyle \mathrm{t}\mathrm{g}^{2}x+\mathrm{t}\mathrm{g}^{4}x+\cdots=\frac{1}{2},$

$\mathrm{w}$ którym lewa strona jest $\mathrm{s}\mathrm{u}\mathrm{m}\Phi$ wyrazów nieskończonego ciągu geometrycznego.

2. Pani Józefa wpłaciła do banku pewien kapitał $K_{0}$ na okres jednego roku na lokatę opro-

$\mathrm{c}\mathrm{e}\mathrm{n}\mathrm{t}\mathrm{o}\mathrm{w}\mathrm{a}\mathrm{n}\Phi$ {\it P}\% $\mathrm{w}$ skali roku, przy czym kapitalizacja odsetek następuje $N$ razy rocznie.

Uzasadnič indukcyjnie, $\dot{\mathrm{z}}\mathrm{e}$ wzór $K_{n}=K_{0}(1+\displaystyle \frac{P}{100N})^{n}$ okeśla stan konta pani Józefy po

$n$-tym okresie kapitalizacyjnym. Sprawdzič, jaki będzie stan konta pani Józefy po roku

przy załozeniu, $\dot{\mathrm{z}}\mathrm{e}$ wplaci ona 10.000, 00 zł na 6\%, a odsetki kapita1izowane będą co

$\mathrm{m}\mathrm{i}\mathrm{e}\mathrm{s}\mathrm{i}_{\Phi}\mathrm{c}.$

3. Zaznaczyč na płaszczy $\acute{\mathrm{z}}\mathrm{n}\mathrm{i}\mathrm{e}$ zbiór rozwi$\Phi$zań nierówności

$\log_{\frac{1}{2}}(3\log_{x}(2y))\geq 0.$

4. $\mathrm{W}$ koło $0$ promieniu $R$ wpisano trójkqt, którego pole stanowi czwartą częśč pola koła,

ajeden $\mathrm{z}$ k$\Phi$tów ma miarę $\alpha$. Obliczyč iloczyn oraz sumę kwadratów długości boków tego

trójk$\Phi$ta.

5. Wyznaczyč równanie krzywej będ$\Phi$cej zbiorem wszystkich środków okręgów stycznych do

prostej $y=2\mathrm{i}\mathrm{p}\mathrm{r}\mathrm{z}\mathrm{e}\mathrm{c}\mathrm{h}\mathrm{o}\mathrm{d}_{\mathrm{Z}\Phi}$cych przez $\mathrm{P}^{\mathrm{O}\mathrm{C}\mathrm{Z}}\Phi^{\mathrm{t}\mathrm{e}\mathrm{k}}$ układu współrzędnych. Spośród rozwaza-

nych okręgów narysowač wszystkie okręgi styczne do jednej $\mathrm{z}$ osi układu współrzędnych

$\mathrm{i}$ wyznaczyč równanie okręgu przechodzącego przez ich środki.

6. Na dnie naczynia $\mathrm{w}$ kształcie walca umieszczono 6 małych ku1ek $0$ promieniu $R\mathrm{w}$ taki

sposób, $\dot{\mathrm{z}}\mathrm{e}\mathrm{k}\mathrm{a}\dot{\mathrm{z}}$ da $\mathrm{z}$ nich jest styczna do dwu innych kulek $\mathrm{i}$ ściany bocznej naczynia.

Następnie umieszczono $\mathrm{w}$ nim kulę $0$ promieniu $ 2R\mathrm{s}\mathrm{t}\mathrm{y}\mathrm{c}\mathrm{z}\mathrm{n}\Phi$ do $\mathrm{k}\mathrm{a}\dot{\mathrm{z}}$ dej $\mathrm{z}$ małych kulek

oraz górnej podstawy walca. Sprawdzič, ile wody zmieści się $\mathrm{w}$ tak zapełnionym naczyniu.
\end{document}
