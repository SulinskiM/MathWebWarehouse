\documentclass[a4paper,12pt]{article}
\usepackage{latexsym}
\usepackage{amsmath}
\usepackage{amssymb}
\usepackage{graphicx}
\usepackage{wrapfig}
\pagestyle{plain}
\usepackage{fancybox}
\usepackage{bm}

\begin{document}

XXXVIII

KORESPONDENCYJNY KURS

Z MATEMATYKI

$\mathrm{p}\mathrm{a}\acute{\mathrm{z}}$dziernik 2008 $\mathrm{r}.$

PRACA KONTROLNA nr l- POZIOM PODSTAWOWY

l. Ile jest liczb pięciocyfrowych podzielnych przez 9, które $\mathrm{w}$ rozwinieciu dziesiętnym maja:

a) obie cyfry 1, 2 $\mathrm{i}$ tylko $\mathrm{t}\mathrm{e}$? b) obie cyfry 1, 3 $\mathrm{i}$ tylko $\mathrm{t}\mathrm{e}$? c) wszystkie cyfry 1, 2, 3

$\mathrm{i}$ tylko $\mathrm{t}\mathrm{e}$? Odpowiedz/uzasadnič. $\mathrm{W}$ przypadku b) wypisač otrzymane liczby.

2. Uprościč wyrazenie $w(x)=9x^{2}-\sqrt{(-9x^{2})^{2}}+3x-\sqrt{9x^{2}}$, a następnie:

a) obliczyč $w(\displaystyle \frac{\sqrt{2}-1}{\sqrt{2}+1})$ oraz $w(\displaystyle \frac{1}{1-\sqrt{3}})$

nowniku.

i wynik podač bez niewymierności w mia-

b) wyznaczyč liczbę $b\mathrm{t}\mathrm{a}\mathrm{k}$, by pole obszaru ograniczonego osiami układu współrzędnych

$\mathrm{i}$ wykresem funkcji $f(x)=w(x)+b$ byfo równe 3. Sporz$\Phi$dzič wykres funkcji $f(x).$

3. Sprawdzič, $\dot{\mathrm{z}}\mathrm{e}$ liczby: $k=\displaystyle \frac{(\sqrt{2})^{-4}(\frac{1}{4})^{-\frac{5}{2}}\sqrt[4]{3}}{(\sqrt[4]{16})^{3}\cdot 27^{-\frac{1}{4}}}, n=(\sqrt{3}-\sqrt{2})^{2}+(\sqrt{6}+1)^{2}$ są całkowite

$\mathrm{i}$ dodatnie. Wyznaczyč $m\mathrm{t}\mathrm{a}\mathrm{k}$, by liczby $k, m, n$ byfy odpowiednio: pierwszym, drugim

$\mathrm{i}$ trzecim wyrazem rosnącego ciqgu geometrycznego. Ile trzeba wziąč początkowych wy-

razów tego $\mathrm{c}\mathrm{i}_{\Phi \mathrm{g}}\mathrm{u}$, by ich suma przekroczyła 100?

4. Miejscowości $A(1,1) \mathrm{i}B(3,3) \mathrm{c}\mathrm{h}\mathrm{c}\Phi$ wspólnie wybudowač oczyszczalnię ścieków. Zazna-

czyč na płaszczy $\acute{\mathrm{z}}\mathrm{n}\mathrm{i}\mathrm{e}$ zbiór $\mathrm{m}\mathrm{o}\dot{\mathrm{z}}$ liwych punktów umiejscowienia oczyszczalni wiedząc, $\dot{\mathrm{z}}\mathrm{e}$

powinna ona byč jednakowo oddalona od $\mathrm{k}\mathrm{a}\dot{\mathrm{z}}$ dej $\mathrm{z}$ miejscowości $\mathrm{i}$ odlegfośč ta nie $\mathrm{m}\mathrm{o}\dot{\mathrm{z}}\mathrm{e}$

przekraczač 2. Ponadto od1egfośč oczyszcza1ni od prosto1iniowego odcinka rzeki fączącej

punkty $D(-2,-\displaystyle \frac{3}{2}) \mathrm{i}E(4,3)$ nie powinna byč mniejsza $\mathrm{n}\mathrm{i}\dot{\mathrm{z}} 1$. Rozwiązanie zilustrowač

rysunkiem.

5. Jaką bryłę otrzymujemy łqcząc środki ścian sześcianu? Obliczyč stosunek objętości tej

bryfy do objętości wyjściowego sześcianu.

6. Wysokośč opuszczona na ramię trójkąta równoramiennego dzieli jego pole $\mathrm{w}$ stosunku

1 : 3. Wyznaczyč tangens kata przy podstawie oraz stosunek długości promienia okręgu

wpisanego do dfugości promienia okręgu opisanego na tym trójkącie. Sporządzič odpo-

wiednie rysunki.




PRACA KONTROLNA nr l- POZIOM ROZSZERZONY

1. $\mathrm{Z}$ przystani A wyrusza $\mathrm{z}$ biegiem rzeki statek do przystani $\mathrm{B}$, odległej od A $0140$ km.

Po upfywie l godziny wyrusza za nim łódz$\acute{}$ motorowa, dopędza statek $\mathrm{w}$ pofowie drogi,

po czym wraca do przystani A $\mathrm{w}$ tym samym momencie, $\mathrm{w}$ którym statek przybija do

przystani B. Wyznaczyč predkośč statku $\mathrm{i}$ prędkośč lodzi $\mathrm{w}$ wodzie stojacej wiedzac, $\dot{\mathrm{z}}\mathrm{e}$

prędkośč biegu rzeki wynosi 4 $\mathrm{k}\mathrm{m}/$godz.

2. Niech $a(x)=\displaystyle \frac{\sqrt{x-1}+1}{x-2}$. Dla jakich liczb rzeczywistych $x$ zarówno wartośč $a(x)$ jak $\mathrm{i}$

jej odwrotnośč $\mathrm{s}\Phi$ mniejsze $\mathrm{n}\mathrm{i}\dot{\mathrm{z}}2$?

3. Wyznaczyč cosinus kata między ścianami ośmiościanu foremnego. Obliczyč stosunek dlu-

gości promienia kuli wpisanej do dfugości promienia kuli opisanej na tej bryle. Sporządzič

odpowiednie rysunki.

4. Liczby: $a = 4\displaystyle \cos^{2}\frac{\pi}{12}$ -tg $\displaystyle \frac{\pi}{3}, b = \displaystyle \frac{(\sqrt[3]{2})^{54}(\frac{1}{\sqrt{3}})^{-6}-(2\sqrt{2})^{12}(\sqrt[3]{3})^{6}}{2^{3}\cdot(\sqrt[3]{\frac{1}{32}})^{-12}+(4\sqrt{2})^{8}}$ są odpowied-

nio pierwszym $\mathrm{i}$ piątym wyrazem nieskończonego, malejącego ciągu geometrycznego.

Obliczyč wyraz piętnasty oraz sumę wszystkich wyrazów tego ciągu. Ile początkowych

wyrazów tego ciągu nalezy wziqč, by ich suma przekroczyła 85\% sumy wszystkich wy-

razów?

5. $K\mathrm{a}\dot{\mathrm{z}}$ da $\mathrm{z}$ przekqtnych trapezu ma długośč 5, jedna $\mathrm{z}$ podstaw ma długośč 2, a po1e równe

jest 12. Ob1iczyč promień okręgu opisanego na tym trapezie. Sporządzič rysunek.

6. Jednym $\mathrm{z}$ boków trójkąta $ABC$ jest odcinek $AB$, gdzie $A(1,2), B(3,1)$. Wyznaczyč

równanie zbioru wszystkich punktów $C$ takich, $\dot{\mathrm{z}}\mathrm{e}$ kąt $BCA$ ma miarę $45^{\mathrm{o}}$ oraz opisač

konstrukcję wszystkich trójk$\Phi$tów równoramiennych spelniających warunek ten warunek.

Sporządzič rysunek.





XXXVIII

KORESPONDENCYJNY KURS

Z MATEMATYKI

marzec 2009 r.

PRACA KONTROLNA nr 6- POZIOM PODSTAWOWY

l. Obliczyč wartośč wyrazenia

$\displaystyle \frac{b^{2}-1}{a^{3}+b^{3}}$ : $(\displaystyle \frac{a+b}{1+ab-a^{2}-a^{3}b}+\frac{1}{a+b}\frac{ab+1}{a^{2}-1})$

$\mathrm{d}\mathrm{l}\mathrm{a}a=\sqrt{2}+1, b=\sqrt{2}-1.$

2. Pole deltoidu wpisanego $\mathrm{w}$ okrąg $0$ promieniu $r$ równe jest $r^{2} \mathrm{W}$

deltoidu.

Wyznaczyč kąty tego

3. $\mathrm{Z}$ miast A $\mathrm{i}\mathrm{B}$ wyruszyly jednocześnie dwa samochody jadące ze stałymi prędkościami

naprzeciw siebie. Do chwili spotkania pierwszy $\mathrm{z}$ nich przebyl drogę $\mathrm{o}d$ km większą $\mathrm{n}\mathrm{i}\dot{\mathrm{z}}$

drugi. Jadqc dalej $\mathrm{z}$ tymi samymi prędkościami, pierwszy samochód przebyf drogę od $\mathrm{A}$

do $\mathrm{B}\mathrm{w}m$ godzin, drugi zaś $\mathrm{w}n$ godzin. Obliczyč odlegfośč między miastami A $\mathrm{i}$ B.

4. Wyznaczyč wszystkie trójkqty równoramienne $0$ wierzchołkach $A(1,0), B(4,1), \mathrm{w}$ któ-

rych $|AB| = |AC| \mathrm{i}$ środkowa $CD$ boku $AB$ jest zawarta $\mathrm{w}$ prostej $x+y=3$. Znalez/č

wspólrzędne środka cięzkości tego $\mathrm{z}$ trójkątow, który ma najmniejsze pole.

5. Sporządzič staranny wykres funkcji $f$ zadanej wzorem

$f(x)=$

gdy

gdy

$|x-\displaystyle \frac{3}{2}|\leq\frac{3}{2},$

$|x-\displaystyle \frac{3}{2}|>\frac{3}{2}.$

Posfugując się wykresem określič zbiór wartości funkcji $f$. Wyznaczyč najmniejszą $\mathrm{i}$

najwiekszą wartośč funkcji $\mathrm{w}$ przedziale $[2-\sqrt{2},2+\sqrt{2}].$

6. $\mathrm{W}$ stozek wpisano graniastoslup prosty $\mathrm{t}\mathrm{a}\mathrm{k}, \dot{\mathrm{z}}\mathrm{e}$ podstawa dolna graniastosfupa zawiera się

$\mathrm{w}$ podstawie stozka, a wierzchołki górnej podstawy nalezą do powierzchni bocznej stozka.

Podstawą graniastosłupajest trójkąt prostokatny, $\mathrm{w}$ którym stosunek przyprostokqtnych

wynosi 1 : 3, a po1e powierzchni największej ściany bocznej jest 2 razy mniejsze $\mathrm{n}\mathrm{i}\dot{\mathrm{z}}$

pole przekroju osiowego stozka. Obliczyč stosunek objętości graniastosłupa do objętości

stozka.





PRACA KONTROLNA nr 6- POZIOM ROZSZERZONY

l. Sporządzič staranny wykres funkcji

stępowania.

$f(x)= |2\displaystyle \frac{3-|x|}{2}-1|$. Opisač $\mathrm{i}$ uzasadnič sposób po-

2. Rozwiązač nierównośč

$\displaystyle \frac{\sqrt{x^{2}-1}}{x}\leq\frac{\sqrt{6x+36}}{8}.$

3. Punkty $K, L, M$ dzielq odpowiednio boki AB, $BC, CA$ trójkąta $\mathrm{w}$ stosunku 1 : 3 oraz

$\vec{AB} =$ [11, 2], $\vec{AC} =$ [2, 4]. Posługuj $\Phi^{\mathrm{C}}$ się rachunkiem wektorowym, obliczyč cosinus

$\mathrm{k}_{\Phi}\mathrm{t}\mathrm{a}\angle MKL.$

4. Wyznaczyč wszystkie wartości parametru całkowitego $m$, dla których para liczb $(x,y)$

spefniająca ukfad równań

$\left\{\begin{array}{l}
2x+y=4\\
4x+3y=m
\end{array}\right.$

jest rozwiązaniem nierówności $x-\sqrt{8}y\leq 4$ oraz $x\log_{3}2+y\log_{3}5\leq x\log_{3}7.$

5. Podstawą ostrosłupa czworokątnego jest prostokąt $0$ przekątnej długości $d$, a wszyst-

kie krawędzie boczne maja tę samq długośč. Większa ściana boczna jest nachylona do

podstawy pod kątem $\alpha$, a mniejsza pod kątem $\beta$. Obliczyč objętośč $\mathrm{i}$ pole powierzchni

bocznej ostrosłupa.

6. Dany jest ukfad równań

$\left\{\begin{array}{l}
x-3|y+1|=0\\
(x-p)^{2}+y^{2}=5,
\end{array}\right.$

gdzie $p$ jest parametrem rzeczywistym.

a) Rozwiązač algebraicznie powyzszy układ dla $p=2\mathrm{i}$ podač jego interpretację geo-

metryczną. Sporz$\Phi$dzič rysunek.

b) Korzystając $\mathrm{z}$ rysunku $\mathrm{i}$ odpowiednich rozwazań geometrycznych, określič liczbę

rozwiązań danego układu $\mathrm{w}$ zalezności od parametru $p.$





XXXVIII

KORESPONDENCYJNY KURS

Z MATEMATYKI

listopad 2008 r.

PRACA KONTROLNA nr 2- POZIOM PODSTAWOWY

l. Niech $A=\{(x,y):|x|+2y\leq 3\}, B=\{(x,y):|y|\geq x^{2}\}$. Zaznaczyč na p{\it l}aszczy $\acute{\mathrm{z}}\mathrm{n}\mathrm{i}\mathrm{e}$

zbiory $A\cap B, A\backslash B.$

2. Trapez $0$ kątach przy podstawie $30^{\mathrm{o}}$ oraz $45^{\mathrm{o}}$ jest opisany na okręgu $0$ promieniu $R.$

Obliczyč stosunek pola kola do pola trapezu.

3. Dla jakich wartości $\mathrm{k}_{\Phi}\mathrm{t}\mathrm{a}\alpha\in[0,2\pi]$ równanie kwadratowe

$\sin\alpha\cdot x^{2}-2x+2\sin\alpha-1=0$

ma dokfadnie jedno rozwiązanie?

4. Pole powierzchni bocznej ostrosłupa prawidlowego trójkatnego jest 6 razy większe $\mathrm{n}\mathrm{i}\dot{\mathrm{z}}$

pole jego podstawy. Obliczyč cosinus $\mathrm{k}_{\Phi^{\mathrm{t}\mathrm{a}}}$ nachylenia krawędzi bocznej ostrosfupa do

pfaszczyzny podstawy.

5. Iloczyn dwu liczb jest 20 razy wiekszy $\mathrm{n}\mathrm{i}\dot{\mathrm{z}}$ odwrotnośč ich sumy. Suma sześcianów tych

liczb stanowi 325\% i1oczynu tych 1iczb $\mathrm{i}$ ich sumy. Jakie to liczby?

6. Narysowač wykres funkcji

$f(x)=$

gdy

gdy

$|x|\leq 1,$

$|x|>1.$

a) Obliczyč $f(-\displaystyle \frac{1}{1+\sqrt{2}})$ oraz $f(\displaystyle \frac{1+\sqrt{2}}{2})$

mianowniku.

Wynik podač bez niewymierności w

b) Wykorzystując wykres rozwiązač nierównośč

rozwiqzań na osi 0x

$f(x) \geq -\displaystyle \frac{1}{2}$

i zaznaczyč zbiór jej

c) Odczytač z wykresu przedziafy, na których funkcja f jest malejąca.





PRACA KONTROLNA nr 2- POZIOM ROZSZERZONY

l. Zaznaczyč na płaszczy $\acute{\mathrm{z}}\mathrm{n}\mathrm{i}\mathrm{e}$ zbiór $\displaystyle \{(x,y):|x|\leq\frac{3}{2},\log_{\frac{2}{3}}|x|<y<\log_{\frac{3}{2}}|x|\}.$

2. Wykazač, $\dot{\mathrm{z}}\mathrm{e}$ róznica kwadratów dwu dowolnych liczb cafkowitych niepodzielnych przez

3 jest liczbą podzielną przez 3.

3. $\mathrm{W}$ trójkącie równoramiennym $ABC0$ podstawie $AB$ ramię ma długośč $b$, a kąt przy

wierzchofku C- miarę $\gamma. D$ jest takim punktem ramienia $BC, \dot{\mathrm{z}}\mathrm{e}$ odcinek $AD$ dzieli pole

trójkąta na polowę. Wyznaczyč promienie okregów wpisanych $\mathrm{w}$ trójkqty $ABD\mathrm{i}ADC.$

Dla jakiego kąta $\gamma$ promienie te $\mathrm{s}\Phi$ równe?

4. Niech $f(x)=3(x+2)^{4}+x^{2}+4x+p$, gdzie $p$ jest parametrem rzeczywistym.

a) Uzasadnič, $\dot{\mathrm{z}}\mathrm{e}$ wykres funkcji $f(x)$ jest symetryczny względem prostej $x=-2.$

b) Dla jakiego parametru rzeczywistego $p$ najmniejszą wartością funkcji $f(x)$ jest

$y=-2$ ? Odpowied $\acute{\mathrm{z}}$ uzasadnič, nie $\mathrm{s}\mathrm{t}\mathrm{o}\mathrm{s}\mathrm{u}\mathrm{j}_{\Phi}\mathrm{c}$ metod rachunku rózniczkowego.

c) Określič liczbę pierwiastków równania $f(x)=0\mathrm{w}$ zalezności od parametru $p.$

5. Rozwiązač nierównośč $|\sin x-\sqrt{3}\cos x|\geq 1.$

6. Rozwiązač równanie

$1-(\displaystyle \frac{2^{x}}{3^{x}-2^{x}})+(\frac{2^{x}}{3^{x}-2^{x}})^{2}-(\frac{2^{x}}{3^{x}-2^{x}})^{3}+\ldots=\frac{3^{x-2}}{2^{x-1}},$

którego lewa strona jest sumą wyrazów nieskończonego ciągu geometrycznego.





XXXVIII

KORESPONDENCYJNY KURS

Z MATEMATYKI

grudzień 2008 r.

PRACA KONTROLNA nr 3 -POZIOM PODSTAWOWY

l. Boki $a_{n}\mathrm{i}b_{n}$ prostokąta $P_{n}$ są wyrazami ciągów arytmetycznych, $\mathrm{w}$ których $a_{1}=b_{1}=100$

oraz $r_{1}=5\mathrm{i}r_{2}=-5$. Znalez/č wszystkie wartości $n$, dla których pole prostokąta $P_{n}$ jest

mniejsze $0$ co najmniej 40\% od po1a $\mathrm{P}^{\mathrm{r}\mathrm{o}\mathrm{s}\mathrm{t}\mathrm{o}\mathrm{k}}\Phi^{\mathrm{t}\mathrm{a}P_{1}}.$

2. Znalez$\acute{}$č równania dwusiecznych katów zawartych między prostymi $x-7y+6 = 0,$

$x+y-2=0$. Następnie wybrač tę $\mathrm{d}\mathrm{w}\mathrm{u}\mathrm{s}\mathrm{i}\mathrm{e}\mathrm{c}\mathrm{z}\mathrm{n}\Phi$, która tworzy $\mathrm{z}$ osią odciętych mniejszy

$\mathrm{k}_{\Phi^{\mathrm{t}}}$. Sporządzič rysunek.

3. Pudelko zawiera 2l klocków po 7 $\mathrm{w}$ kolorach zółtym, czerwonym $\mathrm{i}$ niebieskim.

Wojtuś ufoz $\mathrm{y}l$ wiez$\cdot$ę $\mathrm{z}8$ przypadkowo wybranych klocków. Jakie jest prawdopodobień-

stwo tego, $\dot{\mathrm{z}}\mathrm{e}\mathrm{w}$ wiezy znalazly się klocki wszystkich trzech kolorów?

4. Nie rozwiązując nierówności wykazač, $\dot{\mathrm{z}}\mathrm{e}$ relacja

$\sqrt{3x-3x^{2}+3}>1+\sqrt[5]{x^{2}+1}$

nie jest spefniona dla $\dot{\mathrm{z}}$ adnej liczby rzeczywistej $x.$

5. $\mathrm{W}$ momencie spostrzezenia samolotu nadlatującego ze stafą prędkością $\mathrm{i}$ na stafej wyso-

kości obserwator widziaf go pod kątem $35^{\mathrm{o}}$ do poziomu. Po jednej minucie kąt ten wzrósl

do $65^{\mathrm{o}}$

a) Po jakim czasie od momentu spostrzezenia samolotu przeleciał on nad głową ob-

serwatora?

b) Przyjmując, $\dot{\mathrm{z}}\mathrm{e}$ samolot leciał $\mathrm{z}$ prędkościq 500 $\mathrm{k}\mathrm{m}/\mathrm{h}$, obliczyč na jakiej wysokości

odbywaf się lot.

Wyniki podač $\mathrm{w}$ zaokrqgleniu do pełnych sekund $\mathrm{i}$ pełnych setek metrów.

6. $\mathrm{W}$ stozek $0$ objętości $V\mathrm{i}$ wysokości $\mathrm{s}\mathrm{t}\mathrm{a}\mathrm{n}\mathrm{o}\mathrm{w}\mathrm{i}_{\Phi}\mathrm{c}\mathrm{e}\mathrm{j}$ 75\% promienia podstawy wpisano walec

$\mathrm{t}\mathrm{a}\mathrm{k}, \dot{\mathrm{z}}\mathrm{e}$ podstawa walca $\mathrm{l}\mathrm{e}\dot{\mathrm{z}}\mathrm{y}$ na podstawie stozka, a wysokośč walca jest równa średnicy

jego podstawy. Obliczyč stosunek pola powierzchni całkowitej walca do pola powierzchni

cafkowitej stozka oraz objętośč kuli opisanej na walcu. Sporządzič odpowiedni rysunek.





PRACA KONTROLNA nr 3 -POZIOM ROZSZERZONY

l. Na diagramie skladającym się $\mathrm{z} 9$ kwadratowych pól $\mathrm{w}$ układzie 3 $\rangle\langle 3$ zaznaczono

$\mathrm{w}$ losowo wybranych polach kófko $\mathrm{i}$ krzyzyk. Jakie jest prawdopodobieństwo tego, $\dot{\mathrm{z}}\mathrm{e}$

oba znaki znalazły się na sąsiednich polach $\mathrm{t}\mathrm{z}\mathrm{n}$. stykających się jednym bokiem.

2. Kąty $\mathrm{c}\mathrm{z}\mathrm{w}\mathrm{o}\mathrm{r}\mathrm{o}\mathrm{k}_{\Phi}\mathrm{t}\mathrm{a}$ wpisanego $\mathrm{w}$ okrąg $0$ promieniu $R\mathrm{t}\mathrm{w}\mathrm{o}\mathrm{r}\mathrm{z}\Phi \mathrm{c}\mathrm{i}_{\Phi \mathrm{g}}$ arytmetyczny, którego

pierwszy wyraz wynosi $\displaystyle \frac{\pi}{4}$. Przekątna czworokąta leząca naprzeciw kąta $\displaystyle \frac{\pi}{4}$ jest prosto-

padla do jednego $\mathrm{z}$ boków. Wyznaczyč kąty, obwód oraz pole tego czworokąta.

3. Trójkąt równoramienny $0$ podstawie $a \mathrm{i}$ kącie przy wierzchołku $36^{\mathrm{o}}$ obraca się wo-

kóf dwusiecznej kąta przy podstawie. Obliczyč objętośč powstafej bryly. Skorzystač

$\mathrm{z}$ twierdzenia $0$ dwusiecznej kąta $\mathrm{w}$ trójkącie. Wynik podač bez $\mathrm{u}\dot{\mathrm{z}}$ ycia funkcji trygono-

metrycznych.

4. Odcinek $0$ końcach $A(1,1)\mathrm{i}B(3,2)$ jest bokiem prostokąta, którego jeden $\mathrm{z}$ wierzchof-

ków $\mathrm{l}\mathrm{e}\dot{\mathrm{z}}\mathrm{y}$ na prostej $l$ : $x-y+1=0$. Znalez/č współrzędne wierzcholków $C\mathrm{i}D$. Obliczyč

cosinus kąta miedzy przekątnymi tego $\mathrm{p}\mathrm{r}\mathrm{o}\mathrm{s}\mathrm{t}\mathrm{o}\mathrm{k}_{\Phi^{\mathrm{t}}}\mathrm{a}$. Sporządzič rysunek.

5. Liczba 2 jest pierwiastkiem podwójnym wielomianu $w(x)=x^{3}+ax^{2}+bx+c$, a funkcja

$f(x) =w(x+1)+p$ jest nieparzysta. Znalez/č ten wielomian $\mathrm{i}$ obliczyč $p$. Na jednym

rysunku sporz$\Phi$dzič wykresy funkcji $f(x)$ oraz $h(x)=|w(x)|.$

6. Wyznaczyč dziedzinę funkcji

$y=\displaystyle \frac{\mathrm{c}\mathrm{t}\mathrm{g}4x}{\cos 2x+\cos 6x}.$





XXXVIII

KORESPONDENCYJNY KURS

Z MATEMATYKI

styczeń 2009 r.

PRACA KONTROLNA nr 4- POZIOM PODSTAWOWY

l. Dane sa funkcje określone wzorami $f(x)=x-3$ oraz $g(x)=4-x, x\in R.$

Rozwiązač nierównośč

$|f(2x-5)+g(x+1)|\displaystyle \leq|f(\frac{x}{2}-1)+g(\frac{x}{2}-4)|-2|g(\frac{x}{2})|.$

2. Wartośč $\mathrm{u}\dot{\mathrm{z}}$ ytkowa pewnego $\mathrm{u}\mathrm{r}\mathrm{z}\Phi^{\mathrm{d}}$Zenia maleje $\mathrm{z}$ roku na rok $\mathrm{w}$ postępie arytmetycz-

nym. $\mathrm{W}$ jakim czasie maszyna będzie całkowicie bezuzyteczna, $\mathrm{j}\mathrm{e}\dot{\mathrm{z}}$ eli po 251atach pracy

jej wartośč byfa trzykrotnie mniejsza, $\mathrm{n}\mathrm{i}\dot{\mathrm{z}}$ jej wartośč po 151atach pracy? Po pewnych

udoskonaleniach wydluzono czas $\mathrm{u}\dot{\mathrm{z}}$ ytkowania takiego urządzenia $0$ pieč lat. $\mathrm{O}$ ile wol-

niej będzie teraz spadač jego wartośč $\mathrm{u}\dot{\mathrm{z}}$ ytkowa rocznie? Wynik podač $\mathrm{w}$ procentach $\mathrm{z}$

dokladności$\Phi$ do jednego miejsca po przecinku.

3. $\mathrm{W}$ okrąg wpisano cztery okręgi $\mathrm{w}$ sposób pokazany na rysunku.

Wyznaczyč stosunek pola rombu, którego wierzchołkami są środki

czterech wpisanych okręgów, do pola kola, $\mathrm{w}$ które wpisano te okręgi.

4. Wyznaczyč wartośč parametru $a$, dla którego funkcja kwadratowa $0$ równaniu

$f(x) = (a-1)x^{2}+(a-2)x+1$ osiqga najmniejszą wartośč równa l. Następnie zna-

lez$\acute{}$č równanie prostej $\mathrm{P}^{\mathrm{r}\mathrm{z}\mathrm{e}\mathrm{c}\mathrm{h}\mathrm{o}\mathrm{d}\mathrm{z}}\Phi^{\mathrm{c}\mathrm{e}\mathrm{j}}$ przez punkt $A(a,2a+1)$ prostopadfej do prostej $0$

równaniu $4y+x+8=0$. Jakie jest wzajemne polozenie otrzymanej prostej $\mathrm{i}$ wykresu

funkcji $f$? Wykonač staranny wykres funkcji $f$ oraz obu prostych.

5. Wyznaczyč dziedzinę funkcji danej wzorem

$f(x)=\displaystyle \frac{x-1}{\sqrt{1-\frac{2x}{x-1}}},$

a następnie rozwiązač równanie $f(x)-f(-x)=2.$

6. $\mathrm{W}$ prawidłowym ostrosłupie trójkątnym ściana boczna ma pole dwa razy większe od

pola podstawy. Promień kuli wpisanej $\mathrm{w}$ ten ostrosfup ma dfugośč $r=1$. Obliczyč sumę

wszystkich wysokości tego ostrosłupa oraz wyznaczyč tangens kąta nachylenia krawędzi

bocznej do pfaszczyzny podstawy.





PRACA KONTROLNA nr 4- POZIOM ROZSZERZONY

l. Janek oszczędza na komputer $\mathrm{i}\mathrm{w}$ tym celu włozyl $4000\mathrm{z}l$ na lokatę roczna. Oprocento-

wanie tej lokaty wynosi 12\% $\mathrm{w}$ skali roku, a odsetki kapitalizowane $\mathrm{s}\varpi$ co $\mathrm{m}\mathrm{i}\mathrm{e}\mathrm{s}\mathrm{i}_{\Phi}\mathrm{c}$. Jaki

dochód przyniesie Jankowi ta lokata? Czy więcej uzyskafby na lokacie 18\%, $\mathrm{w}$ której

odsetki kapitalizowane są co kwartał?

2. Zbadač monotonicznośč $\mathrm{c}\mathrm{i}_{\Phi \mathrm{g}}\mathrm{u}0$ wyrazach $a_{n}=\displaystyle \frac{1}{n+1}+\frac{1}{n+2}+\ldots+\frac{1}{n+n}$. Czy ten

ciag jest ograniczony? Wyznaczyč $a_{1}, a_{2}\mathrm{i}a_{3}.$

3. Udowodnič, stosując zasadę indukcji matematycznej, $\dot{\mathrm{z}}\mathrm{e}$ dla $\mathrm{k}\mathrm{a}\dot{\mathrm{z}}$ dej liczby naturalnej $n$

liczba $8^{n+1}+9^{2n-1}$ jest podzielna przez 73.

4. Obliczyč sumę wszystkich tych pierwiastków równania

$\displaystyle \sin^{2}(x+\frac{\pi}{3})+\cos^{2}(x-\frac{\pi}{3})=\frac{7}{4},$

które nalezą do przedziafu $(-10,10).$

5. $\mathrm{W}$ trójkat równoboczny $ABC$ wpisano trzy kwadraty $\mathrm{w}$ taki sposób, $\dot{\mathrm{z}}\mathrm{e}$ jeden $\mathrm{z}$ boków

$\mathrm{k}\mathrm{a}\dot{\mathrm{z}}$ dego kwadratu zawiera się wjednym $\mathrm{z}$ boków trójkata. Środki tych kwadratów $\mathrm{t}\mathrm{w}\mathrm{o}\mathrm{r}\mathrm{z}\Phi$

trójkąt równoboczny $PQR$. Obliczyč stosunek pola trójkąta $ABC$ do pola trójkąta $PQR.$

6. {\it K}rawęd $\acute{\mathrm{z}}$ kwadratowej podstawy prostopadfościanu ma dlugośč $a$. Prostopadfościan prze-

cięto pfaszczyzną przechodzącą przezjeden $\mathrm{z}$ wierzchołków prostopadfościanu oraz środki

dwóch sąsiednich krawedzi przeciwległej podstawy $\mathrm{t}\mathrm{a}\mathrm{k}, \dot{\mathrm{z}}\mathrm{e}$ otrzymany przekrój jest pię-

ciokątem. Obliczyč obwód oraz pole tego pięciok$\Phi$ta, $\mathrm{j}\mathrm{e}\dot{\mathrm{z}}$ eli pfaszczyzna przekroju jest

nachylona do płaszczyzny podstawy pod kątem $\alpha.$





XXXVIII

KORESPONDENCYJNY KURS

Z MATEMATYKI

luty 2009 r.

PRACA KONTROLNA nr 5- POZIOM PODSTAWOWY

l. Biegacz wyruszył na trase maratonu, pokonując $\mathrm{k}\mathrm{a}\dot{\mathrm{z}}$ de 300 $\mathrm{m} \mathrm{w}$ ciągu l minuty. Po

uplywie 20 minut wyruszy1 za nim rowerzysta $\mathrm{i}$ jadąc ze $\mathrm{s}\mathrm{t}\mathrm{a}l_{\Phi}$ prędkości$\Phi$, dogonif ma-

ratończyka dokładnie 195 $\mathrm{m}$ przed linia mety. Jaka była prędkośč rowerzysty? Po jakim

czasie powinien wyjechač rowerzysta, aby jadąc ze stałq prędkością 30 $\mathrm{k}\mathrm{m}/\mathrm{h}$, przekro-

czyč linię mety równocześnie $\mathrm{z}$ biegaczem? Wynik zaokrąglič $\mathrm{w}$ dóf $\mathrm{z}$ dokfadnością do l

sekundy.

2. Tangens kata ostrego $\alpha$ równy jest $\displaystyle \frac{a}{b}$, gdzie

$a=(\sqrt{2+\sqrt{3}}-\sqrt{2-\sqrt{3}})^{2}b=(\sqrt{\sqrt{2}+1}-\sqrt{\sqrt{2}-1})^{2}$

Wyznaczyč wartości pozostałych funkcji trygonometrycznych tego kąta. Wykorzystując

wzór $\sin 2\alpha=2\sin\alpha\cos\alpha$, obliczyč miarę $\mathrm{k}_{\Phi}\mathrm{t}\mathrm{a}\alpha.$

3. $\mathrm{W}$ walec wpisano trzy wzajemnie styczne kule $\mathrm{w}$ ten sposób, $\dot{\mathrm{z}}\mathrm{e}\mathrm{k}\mathrm{a}\dot{\mathrm{z}}\mathrm{d}\mathrm{a}\mathrm{z}$ nich jest styczna

do ściany bocznej $\mathrm{i}$ obu podstaw walca. Sprawdzič, jaką cześč objętości walca zajmujq

kule. Wynik wyrazony $\mathrm{w}$ procentach podač $\mathrm{z}$ dokfadnością do l promila.

4. Wskazač wszystkie $\mathrm{t}\mathrm{e}$ wyrazy ciągu $(a_{n})$, gdzie

$a_{n}=\displaystyle \frac{\log_{2}^{2}n+\log_{\frac{1}{2}}(n^{3})}{\log_{n}2}-2\log_{4}(\frac{1}{n^{2}}),$

które są równe zero.

5. Dwie klepsydry, mała $\mathrm{i}\mathrm{d}\mathrm{u}\dot{\mathrm{z}}\mathrm{a}$, odmierzają odpowiednio $m\mathrm{i}n, m<n$, pełnych minut. Po

raz pierwszy obrócono je równocześnie $\mathrm{w}$ samo pofudnie. $K\mathrm{a}\dot{\mathrm{z}}\mathrm{d}_{\Phi}\mathrm{z}$ nich obracano, gdy

tylko przesypał się $\mathrm{w}$ niej cały piasek. Czas mierzono do momentu, gdy obie klepsydry

równocześnie przestały działač. Określič, która była wtedy godzina, $\mathrm{j}\mathrm{e}\dot{\mathrm{z}}$ eli wiadomo, $\dot{\mathrm{z}}\mathrm{e}$

mafą obrócono $013$ razy więcej $\mathrm{n}\mathrm{i}\dot{\mathrm{z}}$ duzą, a gdy mafą obracano po raz jedenasty, $\mathrm{d}\mathrm{u}\dot{\mathrm{z}}\mathrm{a}$

wypełniona byla dokładnie $\mathrm{w}$ połowie.

6. $\mathrm{W}$ trójkąt równoboczny $0$ polu $P$ wpisano $\mathrm{o}\mathrm{k}\mathrm{r}\Phi \mathrm{g}$ oraz trzy ma-

fe okręgi - jak na rysunku. Następnie odcięto narozniki trójkąta

wzdłuz łuków małych okregów. Obliczyč pole koła opisanego na

tak powstalej figurze.





PRACA KONTROLNA nr 5- POZIOM ROZSZERZONY

l. Wśród prostokątów $0$ ustalonej dlugości przekątnej $p$ znalez$\acute{}$č ten, którego pole jest

największe. Nie stosowač metod rachunku rózniczkowego.

2. Znalez/č wszystkie liczby rzeczywiste $m\neq 0$, dla których równanie

$\displaystyle \frac{x}{m}+m=\frac{m}{x}+x+1$

ma dwa rózne pierwiastki $x_{1}, x_{2}$ spefniające warunek $|x_{1}-x_{2}|>x_{1}+x_{2}.$

3. Rozwiązač nierównośč

$2^{3x-1}-2^{2x-1}-2^{x+1}+2>0.$

4. Stosując wzór na zamianę podstawy logarytmu uzasadnič, $\dot{\mathrm{z}}\mathrm{e}$ liczba

$S_{n}=\log_{m^{2^{0}}}x+\log_{m^{2^{1}}}x+\log_{m^{2^{2}}}x+\cdots+\log_{m^{2^{n}}}x$, gdzie $x>0$ oraz $m\in \mathbb{N}, m>1,$

jest $\mathrm{s}\mathrm{u}\mathrm{m}\Phi$ częściową pewnego nieskończonego $\mathrm{c}\mathrm{i}_{\Phi \mathrm{g}}\mathrm{u}$ geometrycznego. Obliczyč sumę wszyst-

kich wyrazów tego ciągu $\mathrm{i}$ zbadač, dla jakiego $x$ suma ta wynosi $\displaystyle \frac{1}{2}.$

5. Określič dziedzinę funkcji $f(x)=\log_{x^{2}}$($1-\mathrm{t}\mathrm{g}x$ tg $2x$).

6. $\mathrm{W}$ kulę wpisano 4 identyczne mafe ku1e wzajemnie do siebie styczne. Ob1iczyč, jaką

częśč objętości $\mathrm{d}\mathrm{u}\dot{\mathrm{z}}$ ej kuli wypełniajq małe kule. Wynik wyrazony $\mathrm{w}$ procentach podač

$\mathrm{z}$ dokladności$\Phi$ do l promila.



\end{document}