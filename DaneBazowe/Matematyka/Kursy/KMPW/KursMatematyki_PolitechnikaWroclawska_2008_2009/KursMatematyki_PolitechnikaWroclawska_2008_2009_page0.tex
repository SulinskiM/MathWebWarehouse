\documentclass[a4paper,12pt]{article}
\usepackage{latexsym}
\usepackage{amsmath}
\usepackage{amssymb}
\usepackage{graphicx}
\usepackage{wrapfig}
\pagestyle{plain}
\usepackage{fancybox}
\usepackage{bm}

\begin{document}

XXXVIII

KORESPONDENCYJNY KURS

Z MATEMATYKI

$\mathrm{p}\mathrm{a}\acute{\mathrm{z}}$dziernik 2008 $\mathrm{r}.$

PRACA KONTROLNA nr l- POZIOM PODSTAWOWY

l. Ile jest liczb pięciocyfrowych podzielnych przez 9, które $\mathrm{w}$ rozwinieciu dziesiętnym maja:

a) obie cyfry 1, 2 $\mathrm{i}$ tylko $\mathrm{t}\mathrm{e}$? b) obie cyfry 1, 3 $\mathrm{i}$ tylko $\mathrm{t}\mathrm{e}$? c) wszystkie cyfry 1, 2, 3

$\mathrm{i}$ tylko $\mathrm{t}\mathrm{e}$? Odpowiedz/uzasadnič. $\mathrm{W}$ przypadku b) wypisač otrzymane liczby.

2. Uprościč wyrazenie $w(x)=9x^{2}-\sqrt{(-9x^{2})^{2}}+3x-\sqrt{9x^{2}}$, a następnie:

a) obliczyč $w(\displaystyle \frac{\sqrt{2}-1}{\sqrt{2}+1})$ oraz $w(\displaystyle \frac{1}{1-\sqrt{3}})$

nowniku.

i wynik podač bez niewymierności w mia-

b) wyznaczyč liczbę $b\mathrm{t}\mathrm{a}\mathrm{k}$, by pole obszaru ograniczonego osiami układu współrzędnych

$\mathrm{i}$ wykresem funkcji $f(x)=w(x)+b$ byfo równe 3. Sporz$\Phi$dzič wykres funkcji $f(x).$

3. Sprawdzič, $\dot{\mathrm{z}}\mathrm{e}$ liczby: $k=\displaystyle \frac{(\sqrt{2})^{-4}(\frac{1}{4})^{-\frac{5}{2}}\sqrt[4]{3}}{(\sqrt[4]{16})^{3}\cdot 27^{-\frac{1}{4}}}, n=(\sqrt{3}-\sqrt{2})^{2}+(\sqrt{6}+1)^{2}$ są całkowite

$\mathrm{i}$ dodatnie. Wyznaczyč $m\mathrm{t}\mathrm{a}\mathrm{k}$, by liczby $k, m, n$ byfy odpowiednio: pierwszym, drugim

$\mathrm{i}$ trzecim wyrazem rosnącego ciqgu geometrycznego. Ile trzeba wziąč początkowych wy-

razów tego $\mathrm{c}\mathrm{i}_{\Phi \mathrm{g}}\mathrm{u}$, by ich suma przekroczyła 100?

4. Miejscowości $A(1,1) \mathrm{i}B(3,3) \mathrm{c}\mathrm{h}\mathrm{c}\Phi$ wspólnie wybudowač oczyszczalnię ścieków. Zazna-

czyč na płaszczy $\acute{\mathrm{z}}\mathrm{n}\mathrm{i}\mathrm{e}$ zbiór $\mathrm{m}\mathrm{o}\dot{\mathrm{z}}$ liwych punktów umiejscowienia oczyszczalni wiedząc, $\dot{\mathrm{z}}\mathrm{e}$

powinna ona byč jednakowo oddalona od $\mathrm{k}\mathrm{a}\dot{\mathrm{z}}$ dej $\mathrm{z}$ miejscowości $\mathrm{i}$ odlegfośč ta nie $\mathrm{m}\mathrm{o}\dot{\mathrm{z}}\mathrm{e}$

przekraczač 2. Ponadto od1egfośč oczyszcza1ni od prosto1iniowego odcinka rzeki fączącej

punkty $D(-2,-\displaystyle \frac{3}{2}) \mathrm{i}E(4,3)$ nie powinna byč mniejsza $\mathrm{n}\mathrm{i}\dot{\mathrm{z}} 1$. Rozwiązanie zilustrowač

rysunkiem.

5. Jaką bryłę otrzymujemy łqcząc środki ścian sześcianu? Obliczyč stosunek objętości tej

bryfy do objętości wyjściowego sześcianu.

6. Wysokośč opuszczona na ramię trójkąta równoramiennego dzieli jego pole $\mathrm{w}$ stosunku

1 : 3. Wyznaczyč tangens kata przy podstawie oraz stosunek długości promienia okręgu

wpisanego do dfugości promienia okręgu opisanego na tym trójkącie. Sporządzič odpo-

wiednie rysunki.
\end{document}
