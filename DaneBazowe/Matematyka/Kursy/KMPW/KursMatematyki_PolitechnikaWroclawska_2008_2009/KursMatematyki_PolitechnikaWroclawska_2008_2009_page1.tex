\documentclass[a4paper,12pt]{article}
\usepackage{latexsym}
\usepackage{amsmath}
\usepackage{amssymb}
\usepackage{graphicx}
\usepackage{wrapfig}
\pagestyle{plain}
\usepackage{fancybox}
\usepackage{bm}

\begin{document}

PRACA KONTROLNA nr l- POZIOM ROZSZERZONY

1. $\mathrm{Z}$ przystani A wyrusza $\mathrm{z}$ biegiem rzeki statek do przystani $\mathrm{B}$, odległej od A $0140$ km.

Po upfywie l godziny wyrusza za nim łódz$\acute{}$ motorowa, dopędza statek $\mathrm{w}$ pofowie drogi,

po czym wraca do przystani A $\mathrm{w}$ tym samym momencie, $\mathrm{w}$ którym statek przybija do

przystani B. Wyznaczyč predkośč statku $\mathrm{i}$ prędkośč lodzi $\mathrm{w}$ wodzie stojacej wiedzac, $\dot{\mathrm{z}}\mathrm{e}$

prędkośč biegu rzeki wynosi 4 $\mathrm{k}\mathrm{m}/$godz.

2. Niech $a(x)=\displaystyle \frac{\sqrt{x-1}+1}{x-2}$. Dla jakich liczb rzeczywistych $x$ zarówno wartośč $a(x)$ jak $\mathrm{i}$

jej odwrotnośč $\mathrm{s}\Phi$ mniejsze $\mathrm{n}\mathrm{i}\dot{\mathrm{z}}2$?

3. Wyznaczyč cosinus kata między ścianami ośmiościanu foremnego. Obliczyč stosunek dlu-

gości promienia kuli wpisanej do dfugości promienia kuli opisanej na tej bryle. Sporządzič

odpowiednie rysunki.

4. Liczby: $a = 4\displaystyle \cos^{2}\frac{\pi}{12}$ -tg $\displaystyle \frac{\pi}{3}, b = \displaystyle \frac{(\sqrt[3]{2})^{54}(\frac{1}{\sqrt{3}})^{-6}-(2\sqrt{2})^{12}(\sqrt[3]{3})^{6}}{2^{3}\cdot(\sqrt[3]{\frac{1}{32}})^{-12}+(4\sqrt{2})^{8}}$ są odpowied-

nio pierwszym $\mathrm{i}$ piątym wyrazem nieskończonego, malejącego ciągu geometrycznego.

Obliczyč wyraz piętnasty oraz sumę wszystkich wyrazów tego ciągu. Ile początkowych

wyrazów tego ciągu nalezy wziqč, by ich suma przekroczyła 85\% sumy wszystkich wy-

razów?

5. $K\mathrm{a}\dot{\mathrm{z}}$ da $\mathrm{z}$ przekqtnych trapezu ma długośč 5, jedna $\mathrm{z}$ podstaw ma długośč 2, a po1e równe

jest 12. Ob1iczyč promień okręgu opisanego na tym trapezie. Sporządzič rysunek.

6. Jednym $\mathrm{z}$ boków trójkąta $ABC$ jest odcinek $AB$, gdzie $A(1,2), B(3,1)$. Wyznaczyč

równanie zbioru wszystkich punktów $C$ takich, $\dot{\mathrm{z}}\mathrm{e}$ kąt $BCA$ ma miarę $45^{\mathrm{o}}$ oraz opisač

konstrukcję wszystkich trójk$\Phi$tów równoramiennych spelniających warunek ten warunek.

Sporządzič rysunek.
\end{document}
