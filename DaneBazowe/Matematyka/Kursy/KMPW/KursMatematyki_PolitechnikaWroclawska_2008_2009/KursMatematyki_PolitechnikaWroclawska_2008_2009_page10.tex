\documentclass[a4paper,12pt]{article}
\usepackage{latexsym}
\usepackage{amsmath}
\usepackage{amssymb}
\usepackage{graphicx}
\usepackage{wrapfig}
\pagestyle{plain}
\usepackage{fancybox}
\usepackage{bm}

\begin{document}

XXXVIII

KORESPONDENCYJNY KURS

Z MATEMATYKI

marzec 2009 r.

PRACA KONTROLNA nr 6- POZIOM PODSTAWOWY

l. Obliczyč wartośč wyrazenia

$\displaystyle \frac{b^{2}-1}{a^{3}+b^{3}}$ : $(\displaystyle \frac{a+b}{1+ab-a^{2}-a^{3}b}+\frac{1}{a+b}\frac{ab+1}{a^{2}-1})$

$\mathrm{d}\mathrm{l}\mathrm{a}a=\sqrt{2}+1, b=\sqrt{2}-1.$

2. Pole deltoidu wpisanego $\mathrm{w}$ okrąg $0$ promieniu $r$ równe jest $r^{2} \mathrm{W}$

deltoidu.

Wyznaczyč kąty tego

3. $\mathrm{Z}$ miast A $\mathrm{i}\mathrm{B}$ wyruszyly jednocześnie dwa samochody jadące ze stałymi prędkościami

naprzeciw siebie. Do chwili spotkania pierwszy $\mathrm{z}$ nich przebyl drogę $\mathrm{o}d$ km większą $\mathrm{n}\mathrm{i}\dot{\mathrm{z}}$

drugi. Jadqc dalej $\mathrm{z}$ tymi samymi prędkościami, pierwszy samochód przebyf drogę od $\mathrm{A}$

do $\mathrm{B}\mathrm{w}m$ godzin, drugi zaś $\mathrm{w}n$ godzin. Obliczyč odlegfośč między miastami A $\mathrm{i}$ B.

4. Wyznaczyč wszystkie trójkqty równoramienne $0$ wierzchołkach $A(1,0), B(4,1), \mathrm{w}$ któ-

rych $|AB| = |AC| \mathrm{i}$ środkowa $CD$ boku $AB$ jest zawarta $\mathrm{w}$ prostej $x+y=3$. Znalez/č

wspólrzędne środka cięzkości tego $\mathrm{z}$ trójkątow, który ma najmniejsze pole.

5. Sporządzič staranny wykres funkcji $f$ zadanej wzorem

$f(x)=$

gdy

gdy

$|x-\displaystyle \frac{3}{2}|\leq\frac{3}{2},$

$|x-\displaystyle \frac{3}{2}|>\frac{3}{2}.$

Posfugując się wykresem określič zbiór wartości funkcji $f$. Wyznaczyč najmniejszą $\mathrm{i}$

najwiekszą wartośč funkcji $\mathrm{w}$ przedziale $[2-\sqrt{2},2+\sqrt{2}].$

6. $\mathrm{W}$ stozek wpisano graniastoslup prosty $\mathrm{t}\mathrm{a}\mathrm{k}, \dot{\mathrm{z}}\mathrm{e}$ podstawa dolna graniastosfupa zawiera się

$\mathrm{w}$ podstawie stozka, a wierzchołki górnej podstawy nalezą do powierzchni bocznej stozka.

Podstawą graniastosłupajest trójkąt prostokatny, $\mathrm{w}$ którym stosunek przyprostokqtnych

wynosi 1 : 3, a po1e powierzchni największej ściany bocznej jest 2 razy mniejsze $\mathrm{n}\mathrm{i}\dot{\mathrm{z}}$

pole przekroju osiowego stozka. Obliczyč stosunek objętości graniastosłupa do objętości

stozka.
\end{document}
