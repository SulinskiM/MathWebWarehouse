\documentclass[a4paper,12pt]{article}
\usepackage{latexsym}
\usepackage{amsmath}
\usepackage{amssymb}
\usepackage{graphicx}
\usepackage{wrapfig}
\pagestyle{plain}
\usepackage{fancybox}
\usepackage{bm}

\begin{document}

PRACA KONTROLNA nr 6- POZIOM ROZSZERZONY

l. Sporządzič staranny wykres funkcji

stępowania.

$f(x)= |2\displaystyle \frac{3-|x|}{2}-1|$. Opisač $\mathrm{i}$ uzasadnič sposób po-

2. Rozwiązač nierównośč

$\displaystyle \frac{\sqrt{x^{2}-1}}{x}\leq\frac{\sqrt{6x+36}}{8}.$

3. Punkty $K, L, M$ dzielq odpowiednio boki AB, $BC, CA$ trójkąta $\mathrm{w}$ stosunku 1 : 3 oraz

$\vec{AB} =$ [11, 2], $\vec{AC} =$ [2, 4]. Posługuj $\Phi^{\mathrm{C}}$ się rachunkiem wektorowym, obliczyč cosinus

$\mathrm{k}_{\Phi}\mathrm{t}\mathrm{a}\angle MKL.$

4. Wyznaczyč wszystkie wartości parametru całkowitego $m$, dla których para liczb $(x,y)$

spefniająca ukfad równań

$\left\{\begin{array}{l}
2x+y=4\\
4x+3y=m
\end{array}\right.$

jest rozwiązaniem nierówności $x-\sqrt{8}y\leq 4$ oraz $x\log_{3}2+y\log_{3}5\leq x\log_{3}7.$

5. Podstawą ostrosłupa czworokątnego jest prostokąt $0$ przekątnej długości $d$, a wszyst-

kie krawędzie boczne maja tę samq długośč. Większa ściana boczna jest nachylona do

podstawy pod kątem $\alpha$, a mniejsza pod kątem $\beta$. Obliczyč objętośč $\mathrm{i}$ pole powierzchni

bocznej ostrosłupa.

6. Dany jest ukfad równań

$\left\{\begin{array}{l}
x-3|y+1|=0\\
(x-p)^{2}+y^{2}=5,
\end{array}\right.$

gdzie $p$ jest parametrem rzeczywistym.

a) Rozwiązač algebraicznie powyzszy układ dla $p=2\mathrm{i}$ podač jego interpretację geo-

metryczną. Sporz$\Phi$dzič rysunek.

b) Korzystając $\mathrm{z}$ rysunku $\mathrm{i}$ odpowiednich rozwazań geometrycznych, określič liczbę

rozwiązań danego układu $\mathrm{w}$ zalezności od parametru $p.$
\end{document}
