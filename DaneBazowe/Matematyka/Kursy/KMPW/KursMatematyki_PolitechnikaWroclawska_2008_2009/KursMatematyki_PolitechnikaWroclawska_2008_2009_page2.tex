\documentclass[a4paper,12pt]{article}
\usepackage{latexsym}
\usepackage{amsmath}
\usepackage{amssymb}
\usepackage{graphicx}
\usepackage{wrapfig}
\pagestyle{plain}
\usepackage{fancybox}
\usepackage{bm}

\begin{document}

XXXVIII

KORESPONDENCYJNY KURS

Z MATEMATYKI

listopad 2008 r.

PRACA KONTROLNA nr 2- POZIOM PODSTAWOWY

l. Niech $A=\{(x,y):|x|+2y\leq 3\}, B=\{(x,y):|y|\geq x^{2}\}$. Zaznaczyč na p{\it l}aszczy $\acute{\mathrm{z}}\mathrm{n}\mathrm{i}\mathrm{e}$

zbiory $A\cap B, A\backslash B.$

2. Trapez $0$ kątach przy podstawie $30^{\mathrm{o}}$ oraz $45^{\mathrm{o}}$ jest opisany na okręgu $0$ promieniu $R.$

Obliczyč stosunek pola kola do pola trapezu.

3. Dla jakich wartości $\mathrm{k}_{\Phi}\mathrm{t}\mathrm{a}\alpha\in[0,2\pi]$ równanie kwadratowe

$\sin\alpha\cdot x^{2}-2x+2\sin\alpha-1=0$

ma dokfadnie jedno rozwiązanie?

4. Pole powierzchni bocznej ostrosłupa prawidlowego trójkatnego jest 6 razy większe $\mathrm{n}\mathrm{i}\dot{\mathrm{z}}$

pole jego podstawy. Obliczyč cosinus $\mathrm{k}_{\Phi^{\mathrm{t}\mathrm{a}}}$ nachylenia krawędzi bocznej ostrosfupa do

pfaszczyzny podstawy.

5. Iloczyn dwu liczb jest 20 razy wiekszy $\mathrm{n}\mathrm{i}\dot{\mathrm{z}}$ odwrotnośč ich sumy. Suma sześcianów tych

liczb stanowi 325\% i1oczynu tych 1iczb $\mathrm{i}$ ich sumy. Jakie to liczby?

6. Narysowač wykres funkcji

$f(x)=$

gdy

gdy

$|x|\leq 1,$

$|x|>1.$

a) Obliczyč $f(-\displaystyle \frac{1}{1+\sqrt{2}})$ oraz $f(\displaystyle \frac{1+\sqrt{2}}{2})$

mianowniku.

Wynik podač bez niewymierności w

b) Wykorzystując wykres rozwiązač nierównośč

rozwiqzań na osi 0x

$f(x) \geq -\displaystyle \frac{1}{2}$

i zaznaczyč zbiór jej

c) Odczytač z wykresu przedziafy, na których funkcja f jest malejąca.
\end{document}
