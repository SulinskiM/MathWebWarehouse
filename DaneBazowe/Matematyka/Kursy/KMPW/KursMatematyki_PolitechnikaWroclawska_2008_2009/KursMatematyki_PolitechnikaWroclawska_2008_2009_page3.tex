\documentclass[a4paper,12pt]{article}
\usepackage{latexsym}
\usepackage{amsmath}
\usepackage{amssymb}
\usepackage{graphicx}
\usepackage{wrapfig}
\pagestyle{plain}
\usepackage{fancybox}
\usepackage{bm}

\begin{document}

PRACA KONTROLNA nr 2- POZIOM ROZSZERZONY

l. Zaznaczyč na płaszczy $\acute{\mathrm{z}}\mathrm{n}\mathrm{i}\mathrm{e}$ zbiór $\displaystyle \{(x,y):|x|\leq\frac{3}{2},\log_{\frac{2}{3}}|x|<y<\log_{\frac{3}{2}}|x|\}.$

2. Wykazač, $\dot{\mathrm{z}}\mathrm{e}$ róznica kwadratów dwu dowolnych liczb cafkowitych niepodzielnych przez

3 jest liczbą podzielną przez 3.

3. $\mathrm{W}$ trójkącie równoramiennym $ABC0$ podstawie $AB$ ramię ma długośč $b$, a kąt przy

wierzchofku C- miarę $\gamma. D$ jest takim punktem ramienia $BC, \dot{\mathrm{z}}\mathrm{e}$ odcinek $AD$ dzieli pole

trójkąta na polowę. Wyznaczyč promienie okregów wpisanych $\mathrm{w}$ trójkqty $ABD\mathrm{i}ADC.$

Dla jakiego kąta $\gamma$ promienie te $\mathrm{s}\Phi$ równe?

4. Niech $f(x)=3(x+2)^{4}+x^{2}+4x+p$, gdzie $p$ jest parametrem rzeczywistym.

a) Uzasadnič, $\dot{\mathrm{z}}\mathrm{e}$ wykres funkcji $f(x)$ jest symetryczny względem prostej $x=-2.$

b) Dla jakiego parametru rzeczywistego $p$ najmniejszą wartością funkcji $f(x)$ jest

$y=-2$ ? Odpowied $\acute{\mathrm{z}}$ uzasadnič, nie $\mathrm{s}\mathrm{t}\mathrm{o}\mathrm{s}\mathrm{u}\mathrm{j}_{\Phi}\mathrm{c}$ metod rachunku rózniczkowego.

c) Określič liczbę pierwiastków równania $f(x)=0\mathrm{w}$ zalezności od parametru $p.$

5. Rozwiązač nierównośč $|\sin x-\sqrt{3}\cos x|\geq 1.$

6. Rozwiązač równanie

$1-(\displaystyle \frac{2^{x}}{3^{x}-2^{x}})+(\frac{2^{x}}{3^{x}-2^{x}})^{2}-(\frac{2^{x}}{3^{x}-2^{x}})^{3}+\ldots=\frac{3^{x-2}}{2^{x-1}},$

którego lewa strona jest sumą wyrazów nieskończonego ciągu geometrycznego.
\end{document}
