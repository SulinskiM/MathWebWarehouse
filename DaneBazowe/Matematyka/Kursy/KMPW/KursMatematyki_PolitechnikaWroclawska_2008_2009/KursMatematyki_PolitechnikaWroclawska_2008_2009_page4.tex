\documentclass[a4paper,12pt]{article}
\usepackage{latexsym}
\usepackage{amsmath}
\usepackage{amssymb}
\usepackage{graphicx}
\usepackage{wrapfig}
\pagestyle{plain}
\usepackage{fancybox}
\usepackage{bm}

\begin{document}

XXXVIII

KORESPONDENCYJNY KURS

Z MATEMATYKI

grudzień 2008 r.

PRACA KONTROLNA nr 3 -POZIOM PODSTAWOWY

l. Boki $a_{n}\mathrm{i}b_{n}$ prostokąta $P_{n}$ są wyrazami ciągów arytmetycznych, $\mathrm{w}$ których $a_{1}=b_{1}=100$

oraz $r_{1}=5\mathrm{i}r_{2}=-5$. Znalez/č wszystkie wartości $n$, dla których pole prostokąta $P_{n}$ jest

mniejsze $0$ co najmniej 40\% od po1a $\mathrm{P}^{\mathrm{r}\mathrm{o}\mathrm{s}\mathrm{t}\mathrm{o}\mathrm{k}}\Phi^{\mathrm{t}\mathrm{a}P_{1}}.$

2. Znalez$\acute{}$č równania dwusiecznych katów zawartych między prostymi $x-7y+6 = 0,$

$x+y-2=0$. Następnie wybrač tę $\mathrm{d}\mathrm{w}\mathrm{u}\mathrm{s}\mathrm{i}\mathrm{e}\mathrm{c}\mathrm{z}\mathrm{n}\Phi$, która tworzy $\mathrm{z}$ osią odciętych mniejszy

$\mathrm{k}_{\Phi^{\mathrm{t}}}$. Sporządzič rysunek.

3. Pudelko zawiera 2l klocków po 7 $\mathrm{w}$ kolorach zółtym, czerwonym $\mathrm{i}$ niebieskim.

Wojtuś ufoz $\mathrm{y}l$ wiez$\cdot$ę $\mathrm{z}8$ przypadkowo wybranych klocków. Jakie jest prawdopodobień-

stwo tego, $\dot{\mathrm{z}}\mathrm{e}\mathrm{w}$ wiezy znalazly się klocki wszystkich trzech kolorów?

4. Nie rozwiązując nierówności wykazač, $\dot{\mathrm{z}}\mathrm{e}$ relacja

$\sqrt{3x-3x^{2}+3}>1+\sqrt[5]{x^{2}+1}$

nie jest spefniona dla $\dot{\mathrm{z}}$ adnej liczby rzeczywistej $x.$

5. $\mathrm{W}$ momencie spostrzezenia samolotu nadlatującego ze stafą prędkością $\mathrm{i}$ na stafej wyso-

kości obserwator widziaf go pod kątem $35^{\mathrm{o}}$ do poziomu. Po jednej minucie kąt ten wzrósl

do $65^{\mathrm{o}}$

a) Po jakim czasie od momentu spostrzezenia samolotu przeleciał on nad głową ob-

serwatora?

b) Przyjmując, $\dot{\mathrm{z}}\mathrm{e}$ samolot leciał $\mathrm{z}$ prędkościq 500 $\mathrm{k}\mathrm{m}/\mathrm{h}$, obliczyč na jakiej wysokości

odbywaf się lot.

Wyniki podač $\mathrm{w}$ zaokrqgleniu do pełnych sekund $\mathrm{i}$ pełnych setek metrów.

6. $\mathrm{W}$ stozek $0$ objętości $V\mathrm{i}$ wysokości $\mathrm{s}\mathrm{t}\mathrm{a}\mathrm{n}\mathrm{o}\mathrm{w}\mathrm{i}_{\Phi}\mathrm{c}\mathrm{e}\mathrm{j}$ 75\% promienia podstawy wpisano walec

$\mathrm{t}\mathrm{a}\mathrm{k}, \dot{\mathrm{z}}\mathrm{e}$ podstawa walca $\mathrm{l}\mathrm{e}\dot{\mathrm{z}}\mathrm{y}$ na podstawie stozka, a wysokośč walca jest równa średnicy

jego podstawy. Obliczyč stosunek pola powierzchni całkowitej walca do pola powierzchni

cafkowitej stozka oraz objętośč kuli opisanej na walcu. Sporządzič odpowiedni rysunek.
\end{document}
