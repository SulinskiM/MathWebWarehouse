\documentclass[a4paper,12pt]{article}
\usepackage{latexsym}
\usepackage{amsmath}
\usepackage{amssymb}
\usepackage{graphicx}
\usepackage{wrapfig}
\pagestyle{plain}
\usepackage{fancybox}
\usepackage{bm}

\begin{document}

PRACA KONTROLNA nr 3 -POZIOM ROZSZERZONY

l. Na diagramie skladającym się $\mathrm{z} 9$ kwadratowych pól $\mathrm{w}$ układzie 3 $\rangle\langle 3$ zaznaczono

$\mathrm{w}$ losowo wybranych polach kófko $\mathrm{i}$ krzyzyk. Jakie jest prawdopodobieństwo tego, $\dot{\mathrm{z}}\mathrm{e}$

oba znaki znalazły się na sąsiednich polach $\mathrm{t}\mathrm{z}\mathrm{n}$. stykających się jednym bokiem.

2. Kąty $\mathrm{c}\mathrm{z}\mathrm{w}\mathrm{o}\mathrm{r}\mathrm{o}\mathrm{k}_{\Phi}\mathrm{t}\mathrm{a}$ wpisanego $\mathrm{w}$ okrąg $0$ promieniu $R\mathrm{t}\mathrm{w}\mathrm{o}\mathrm{r}\mathrm{z}\Phi \mathrm{c}\mathrm{i}_{\Phi \mathrm{g}}$ arytmetyczny, którego

pierwszy wyraz wynosi $\displaystyle \frac{\pi}{4}$. Przekątna czworokąta leząca naprzeciw kąta $\displaystyle \frac{\pi}{4}$ jest prosto-

padla do jednego $\mathrm{z}$ boków. Wyznaczyč kąty, obwód oraz pole tego czworokąta.

3. Trójkąt równoramienny $0$ podstawie $a \mathrm{i}$ kącie przy wierzchołku $36^{\mathrm{o}}$ obraca się wo-

kóf dwusiecznej kąta przy podstawie. Obliczyč objętośč powstafej bryly. Skorzystač

$\mathrm{z}$ twierdzenia $0$ dwusiecznej kąta $\mathrm{w}$ trójkącie. Wynik podač bez $\mathrm{u}\dot{\mathrm{z}}$ ycia funkcji trygono-

metrycznych.

4. Odcinek $0$ końcach $A(1,1)\mathrm{i}B(3,2)$ jest bokiem prostokąta, którego jeden $\mathrm{z}$ wierzchof-

ków $\mathrm{l}\mathrm{e}\dot{\mathrm{z}}\mathrm{y}$ na prostej $l$ : $x-y+1=0$. Znalez/č współrzędne wierzcholków $C\mathrm{i}D$. Obliczyč

cosinus kąta miedzy przekątnymi tego $\mathrm{p}\mathrm{r}\mathrm{o}\mathrm{s}\mathrm{t}\mathrm{o}\mathrm{k}_{\Phi^{\mathrm{t}}}\mathrm{a}$. Sporządzič rysunek.

5. Liczba 2 jest pierwiastkiem podwójnym wielomianu $w(x)=x^{3}+ax^{2}+bx+c$, a funkcja

$f(x) =w(x+1)+p$ jest nieparzysta. Znalez/č ten wielomian $\mathrm{i}$ obliczyč $p$. Na jednym

rysunku sporz$\Phi$dzič wykresy funkcji $f(x)$ oraz $h(x)=|w(x)|.$

6. Wyznaczyč dziedzinę funkcji

$y=\displaystyle \frac{\mathrm{c}\mathrm{t}\mathrm{g}4x}{\cos 2x+\cos 6x}.$
\end{document}
