\documentclass[a4paper,12pt]{article}
\usepackage{latexsym}
\usepackage{amsmath}
\usepackage{amssymb}
\usepackage{graphicx}
\usepackage{wrapfig}
\pagestyle{plain}
\usepackage{fancybox}
\usepackage{bm}

\begin{document}

XXXVIII

KORESPONDENCYJNY KURS

Z MATEMATYKI

styczeń 2009 r.

PRACA KONTROLNA nr 4- POZIOM PODSTAWOWY

l. Dane sa funkcje określone wzorami $f(x)=x-3$ oraz $g(x)=4-x, x\in R.$

Rozwiązač nierównośč

$|f(2x-5)+g(x+1)|\displaystyle \leq|f(\frac{x}{2}-1)+g(\frac{x}{2}-4)|-2|g(\frac{x}{2})|.$

2. Wartośč $\mathrm{u}\dot{\mathrm{z}}$ ytkowa pewnego $\mathrm{u}\mathrm{r}\mathrm{z}\Phi^{\mathrm{d}}$Zenia maleje $\mathrm{z}$ roku na rok $\mathrm{w}$ postępie arytmetycz-

nym. $\mathrm{W}$ jakim czasie maszyna będzie całkowicie bezuzyteczna, $\mathrm{j}\mathrm{e}\dot{\mathrm{z}}$ eli po 251atach pracy

jej wartośč byfa trzykrotnie mniejsza, $\mathrm{n}\mathrm{i}\dot{\mathrm{z}}$ jej wartośč po 151atach pracy? Po pewnych

udoskonaleniach wydluzono czas $\mathrm{u}\dot{\mathrm{z}}$ ytkowania takiego urządzenia $0$ pieč lat. $\mathrm{O}$ ile wol-

niej będzie teraz spadač jego wartośč $\mathrm{u}\dot{\mathrm{z}}$ ytkowa rocznie? Wynik podač $\mathrm{w}$ procentach $\mathrm{z}$

dokladności$\Phi$ do jednego miejsca po przecinku.

3. $\mathrm{W}$ okrąg wpisano cztery okręgi $\mathrm{w}$ sposób pokazany na rysunku.

Wyznaczyč stosunek pola rombu, którego wierzchołkami są środki

czterech wpisanych okręgów, do pola kola, $\mathrm{w}$ które wpisano te okręgi.

4. Wyznaczyč wartośč parametru $a$, dla którego funkcja kwadratowa $0$ równaniu

$f(x) = (a-1)x^{2}+(a-2)x+1$ osiqga najmniejszą wartośč równa l. Następnie zna-

lez$\acute{}$č równanie prostej $\mathrm{P}^{\mathrm{r}\mathrm{z}\mathrm{e}\mathrm{c}\mathrm{h}\mathrm{o}\mathrm{d}\mathrm{z}}\Phi^{\mathrm{c}\mathrm{e}\mathrm{j}}$ przez punkt $A(a,2a+1)$ prostopadfej do prostej $0$

równaniu $4y+x+8=0$. Jakie jest wzajemne polozenie otrzymanej prostej $\mathrm{i}$ wykresu

funkcji $f$? Wykonač staranny wykres funkcji $f$ oraz obu prostych.

5. Wyznaczyč dziedzinę funkcji danej wzorem

$f(x)=\displaystyle \frac{x-1}{\sqrt{1-\frac{2x}{x-1}}},$

a następnie rozwiązač równanie $f(x)-f(-x)=2.$

6. $\mathrm{W}$ prawidłowym ostrosłupie trójkątnym ściana boczna ma pole dwa razy większe od

pola podstawy. Promień kuli wpisanej $\mathrm{w}$ ten ostrosfup ma dfugośč $r=1$. Obliczyč sumę

wszystkich wysokości tego ostrosłupa oraz wyznaczyč tangens kąta nachylenia krawędzi

bocznej do pfaszczyzny podstawy.
\end{document}
