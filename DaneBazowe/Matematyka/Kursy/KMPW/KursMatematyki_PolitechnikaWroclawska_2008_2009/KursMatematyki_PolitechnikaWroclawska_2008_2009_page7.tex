\documentclass[a4paper,12pt]{article}
\usepackage{latexsym}
\usepackage{amsmath}
\usepackage{amssymb}
\usepackage{graphicx}
\usepackage{wrapfig}
\pagestyle{plain}
\usepackage{fancybox}
\usepackage{bm}

\begin{document}

PRACA KONTROLNA nr 4- POZIOM ROZSZERZONY

l. Janek oszczędza na komputer $\mathrm{i}\mathrm{w}$ tym celu włozyl $4000\mathrm{z}l$ na lokatę roczna. Oprocento-

wanie tej lokaty wynosi 12\% $\mathrm{w}$ skali roku, a odsetki kapitalizowane $\mathrm{s}\varpi$ co $\mathrm{m}\mathrm{i}\mathrm{e}\mathrm{s}\mathrm{i}_{\Phi}\mathrm{c}$. Jaki

dochód przyniesie Jankowi ta lokata? Czy więcej uzyskafby na lokacie 18\%, $\mathrm{w}$ której

odsetki kapitalizowane są co kwartał?

2. Zbadač monotonicznośč $\mathrm{c}\mathrm{i}_{\Phi \mathrm{g}}\mathrm{u}0$ wyrazach $a_{n}=\displaystyle \frac{1}{n+1}+\frac{1}{n+2}+\ldots+\frac{1}{n+n}$. Czy ten

ciag jest ograniczony? Wyznaczyč $a_{1}, a_{2}\mathrm{i}a_{3}.$

3. Udowodnič, stosując zasadę indukcji matematycznej, $\dot{\mathrm{z}}\mathrm{e}$ dla $\mathrm{k}\mathrm{a}\dot{\mathrm{z}}$ dej liczby naturalnej $n$

liczba $8^{n+1}+9^{2n-1}$ jest podzielna przez 73.

4. Obliczyč sumę wszystkich tych pierwiastków równania

$\displaystyle \sin^{2}(x+\frac{\pi}{3})+\cos^{2}(x-\frac{\pi}{3})=\frac{7}{4},$

które nalezą do przedziafu $(-10,10).$

5. $\mathrm{W}$ trójkat równoboczny $ABC$ wpisano trzy kwadraty $\mathrm{w}$ taki sposób, $\dot{\mathrm{z}}\mathrm{e}$ jeden $\mathrm{z}$ boków

$\mathrm{k}\mathrm{a}\dot{\mathrm{z}}$ dego kwadratu zawiera się wjednym $\mathrm{z}$ boków trójkata. Środki tych kwadratów $\mathrm{t}\mathrm{w}\mathrm{o}\mathrm{r}\mathrm{z}\Phi$

trójkąt równoboczny $PQR$. Obliczyč stosunek pola trójkąta $ABC$ do pola trójkąta $PQR.$

6. {\it K}rawęd $\acute{\mathrm{z}}$ kwadratowej podstawy prostopadfościanu ma dlugośč $a$. Prostopadfościan prze-

cięto pfaszczyzną przechodzącą przezjeden $\mathrm{z}$ wierzchołków prostopadfościanu oraz środki

dwóch sąsiednich krawedzi przeciwległej podstawy $\mathrm{t}\mathrm{a}\mathrm{k}, \dot{\mathrm{z}}\mathrm{e}$ otrzymany przekrój jest pię-

ciokątem. Obliczyč obwód oraz pole tego pięciok$\Phi$ta, $\mathrm{j}\mathrm{e}\dot{\mathrm{z}}$ eli pfaszczyzna przekroju jest

nachylona do płaszczyzny podstawy pod kątem $\alpha.$
\end{document}
