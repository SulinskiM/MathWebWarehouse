\documentclass[a4paper,12pt]{article}
\usepackage{latexsym}
\usepackage{amsmath}
\usepackage{amssymb}
\usepackage{graphicx}
\usepackage{wrapfig}
\pagestyle{plain}
\usepackage{fancybox}
\usepackage{bm}

\begin{document}

XXXVIII

KORESPONDENCYJNY KURS

Z MATEMATYKI

luty 2009 r.

PRACA KONTROLNA nr 5- POZIOM PODSTAWOWY

l. Biegacz wyruszył na trase maratonu, pokonując $\mathrm{k}\mathrm{a}\dot{\mathrm{z}}$ de 300 $\mathrm{m} \mathrm{w}$ ciągu l minuty. Po

uplywie 20 minut wyruszy1 za nim rowerzysta $\mathrm{i}$ jadąc ze $\mathrm{s}\mathrm{t}\mathrm{a}l_{\Phi}$ prędkości$\Phi$, dogonif ma-

ratończyka dokładnie 195 $\mathrm{m}$ przed linia mety. Jaka była prędkośč rowerzysty? Po jakim

czasie powinien wyjechač rowerzysta, aby jadąc ze stałq prędkością 30 $\mathrm{k}\mathrm{m}/\mathrm{h}$, przekro-

czyč linię mety równocześnie $\mathrm{z}$ biegaczem? Wynik zaokrąglič $\mathrm{w}$ dóf $\mathrm{z}$ dokfadnością do l

sekundy.

2. Tangens kata ostrego $\alpha$ równy jest $\displaystyle \frac{a}{b}$, gdzie

$a=(\sqrt{2+\sqrt{3}}-\sqrt{2-\sqrt{3}})^{2}b=(\sqrt{\sqrt{2}+1}-\sqrt{\sqrt{2}-1})^{2}$

Wyznaczyč wartości pozostałych funkcji trygonometrycznych tego kąta. Wykorzystując

wzór $\sin 2\alpha=2\sin\alpha\cos\alpha$, obliczyč miarę $\mathrm{k}_{\Phi}\mathrm{t}\mathrm{a}\alpha.$

3. $\mathrm{W}$ walec wpisano trzy wzajemnie styczne kule $\mathrm{w}$ ten sposób, $\dot{\mathrm{z}}\mathrm{e}\mathrm{k}\mathrm{a}\dot{\mathrm{z}}\mathrm{d}\mathrm{a}\mathrm{z}$ nich jest styczna

do ściany bocznej $\mathrm{i}$ obu podstaw walca. Sprawdzič, jaką cześč objętości walca zajmujq

kule. Wynik wyrazony $\mathrm{w}$ procentach podač $\mathrm{z}$ dokfadnością do l promila.

4. Wskazač wszystkie $\mathrm{t}\mathrm{e}$ wyrazy ciągu $(a_{n})$, gdzie

$a_{n}=\displaystyle \frac{\log_{2}^{2}n+\log_{\frac{1}{2}}(n^{3})}{\log_{n}2}-2\log_{4}(\frac{1}{n^{2}}),$

które są równe zero.

5. Dwie klepsydry, mała $\mathrm{i}\mathrm{d}\mathrm{u}\dot{\mathrm{z}}\mathrm{a}$, odmierzają odpowiednio $m\mathrm{i}n, m<n$, pełnych minut. Po

raz pierwszy obrócono je równocześnie $\mathrm{w}$ samo pofudnie. $K\mathrm{a}\dot{\mathrm{z}}\mathrm{d}_{\Phi}\mathrm{z}$ nich obracano, gdy

tylko przesypał się $\mathrm{w}$ niej cały piasek. Czas mierzono do momentu, gdy obie klepsydry

równocześnie przestały działač. Określič, która była wtedy godzina, $\mathrm{j}\mathrm{e}\dot{\mathrm{z}}$ eli wiadomo, $\dot{\mathrm{z}}\mathrm{e}$

mafą obrócono $013$ razy więcej $\mathrm{n}\mathrm{i}\dot{\mathrm{z}}$ duzą, a gdy mafą obracano po raz jedenasty, $\mathrm{d}\mathrm{u}\dot{\mathrm{z}}\mathrm{a}$

wypełniona byla dokładnie $\mathrm{w}$ połowie.

6. $\mathrm{W}$ trójkąt równoboczny $0$ polu $P$ wpisano $\mathrm{o}\mathrm{k}\mathrm{r}\Phi \mathrm{g}$ oraz trzy ma-

fe okręgi - jak na rysunku. Następnie odcięto narozniki trójkąta

wzdłuz łuków małych okregów. Obliczyč pole koła opisanego na

tak powstalej figurze.
\end{document}
