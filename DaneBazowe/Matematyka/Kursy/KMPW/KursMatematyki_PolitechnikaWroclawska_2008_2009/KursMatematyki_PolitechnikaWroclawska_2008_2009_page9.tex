\documentclass[a4paper,12pt]{article}
\usepackage{latexsym}
\usepackage{amsmath}
\usepackage{amssymb}
\usepackage{graphicx}
\usepackage{wrapfig}
\pagestyle{plain}
\usepackage{fancybox}
\usepackage{bm}

\begin{document}

PRACA KONTROLNA nr 5- POZIOM ROZSZERZONY

l. Wśród prostokątów $0$ ustalonej dlugości przekątnej $p$ znalez$\acute{}$č ten, którego pole jest

największe. Nie stosowač metod rachunku rózniczkowego.

2. Znalez/č wszystkie liczby rzeczywiste $m\neq 0$, dla których równanie

$\displaystyle \frac{x}{m}+m=\frac{m}{x}+x+1$

ma dwa rózne pierwiastki $x_{1}, x_{2}$ spefniające warunek $|x_{1}-x_{2}|>x_{1}+x_{2}.$

3. Rozwiązač nierównośč

$2^{3x-1}-2^{2x-1}-2^{x+1}+2>0.$

4. Stosując wzór na zamianę podstawy logarytmu uzasadnič, $\dot{\mathrm{z}}\mathrm{e}$ liczba

$S_{n}=\log_{m^{2^{0}}}x+\log_{m^{2^{1}}}x+\log_{m^{2^{2}}}x+\cdots+\log_{m^{2^{n}}}x$, gdzie $x>0$ oraz $m\in \mathbb{N}, m>1,$

jest $\mathrm{s}\mathrm{u}\mathrm{m}\Phi$ częściową pewnego nieskończonego $\mathrm{c}\mathrm{i}_{\Phi \mathrm{g}}\mathrm{u}$ geometrycznego. Obliczyč sumę wszyst-

kich wyrazów tego ciągu $\mathrm{i}$ zbadač, dla jakiego $x$ suma ta wynosi $\displaystyle \frac{1}{2}.$

5. Określič dziedzinę funkcji $f(x)=\log_{x^{2}}$($1-\mathrm{t}\mathrm{g}x$ tg $2x$).

6. $\mathrm{W}$ kulę wpisano 4 identyczne mafe ku1e wzajemnie do siebie styczne. Ob1iczyč, jaką

częśč objętości $\mathrm{d}\mathrm{u}\dot{\mathrm{z}}$ ej kuli wypełniajq małe kule. Wynik wyrazony $\mathrm{w}$ procentach podač

$\mathrm{z}$ dokladności$\Phi$ do l promila.
\end{document}
