\documentclass[a4paper,12pt]{article}
\usepackage{latexsym}
\usepackage{amsmath}
\usepackage{amssymb}
\usepackage{graphicx}
\usepackage{wrapfig}
\pagestyle{plain}
\usepackage{fancybox}
\usepackage{bm}

\begin{document}

XXXIX

KORESPONDENCYJNY KURS

Z MATEMATYKI

$\mathrm{p}\mathrm{a}\acute{\mathrm{z}}$dziernik 2009 $\mathrm{r}.$

PRACA KONTROLNA $\mathrm{n}\mathrm{r} 1-$ POZIOM PODSTAWOWY

l. Właściciel hurtowni sprzedał $\displaystyle \frac{1}{3}$ partii bananów po załozonej przez siebie cenie. Okazalo

się, $\dot{\mathrm{z}}\mathrm{e}$ owoce zbyt szybko dojrzewają, więc obnizyf cenę $0$ 30\% $\mathrm{i}$ wówczas sprzedaf 60\%

pozostałej ilości owoców. Resztę bananów udało mu się sprzedač dopiero, gdy ustalił

ich cenę na poziomie $\displaystyle \frac{1}{5}$ ceny początkowej. Ile procent zaplanowanego zysku stanowi

kwota uzyskana ze sprzedaz $\mathrm{y}$? Po ile powinien byf sprzedač pierwszą partię towaru, by

jednokrotna obnizka ich ceny $0$ 25\% pozwoliła na sprzedaz wszystkich owoców $\mathrm{i}$ uzyskanie

zaplanowanego początkowo zysku?

2. Przekątne trapezu $0$ podstawach 3 $\mathrm{i}4$ przecinają się pod kątem prostym. Na $\mathrm{k}\mathrm{a}\dot{\mathrm{z}}$ dym

$\mathrm{z}$ boków trapezu, jako na średnicy, oparto półokrąg. Obliczyč sume pól otrzymanych

czterech pólkoli. Sporządzič rysunek.

3. Uprościč wyrazenie $\displaystyle \frac{1}{\sqrt{a}-\sqrt{b}}(\sqrt[6]{a^{5}}-\frac{b}{\sqrt[6]{\alpha}}) -\displaystyle \frac{a-b}{\sqrt[3]{a^{2}}+\sqrt[6]{\alpha}\sqrt{b}} \mathrm{d}\mathrm{l}\mathrm{a}a, b, \mathrm{d}\mathrm{l}\mathrm{a}$ których ma

ono sens. Następnie obliczyč jego wartośč, przyjmując $a=(4-2\sqrt{3})^{3}\mathrm{i} b=3+2\sqrt{2}.$

4. Podstawą ostrosfupa prawidlowego jest sześciok$\Phi$t foremny $0$ boku $a$. Obliczyč objętośč,

wiedząc, $\dot{\mathrm{z}}\mathrm{e}$ najmniejszy ($\mathrm{w}$ sensie powierzchni) $\mathrm{z}$ przekrojów ostrosłupa płaszczyzną

zawierajqcą wysokośč jest trójkątem równobocznym. Wyznaczyč cosinus kąta między

ścianami bocznymi ostrosfupa. Sporz$\Phi$dzič rysunek.

5. Dana jest funkcja liniowa $f(x)=2x-6.$

a) Dlajakiego $a$ pole trójkąta ograniczonego osiami ukfadu wspófrzędnych $\mathrm{i}$ wykresem

funkcji $h(x)=f(x-a)$ równe jest 4? Sporządzič rysunek.

b) Narysowač zbiór $D=\{(x,y):f(x^{2}+2x)\leq y\leq f(x+2)\}.$

6. Sporządzič wykres funkcji $f(x)=$

dla

dla

$x<0,$

$x\geq 0.$

Posfugując się nim, wyznaczyč przedzialy monotoniczności tej funkcji. Narysowač wy-

kres funkcji $g(m)$ określajacej liczbę rozwiqzań równania $f(x)=|m| \mathrm{w}$ zalezności od

parametru rzeczywistego $m.$




PRACA KONTROLNA nr l- POZIOM ROZSZERZONY

l. Statek wyrusza ($\mathrm{z}$ biegiem rzeki) $\mathrm{z}$ przystani A do odległej $0 140$ km przystani B. Po

uplywie l godziny wyrusza za nim łódz/ motorowa, dopędza statek $\mathrm{w}$ pofowie drogi,

po czym wraca do przystani A $\mathrm{w}$ tym samym momencie, $\mathrm{w}$ którym statek przybija do

przystani B. Wyznaczyč prędkośč statku $\mathrm{i}$ prędkośč lodzi $\mathrm{w}$ wodzie stojącej, wiedzqc, $\dot{\mathrm{z}}\mathrm{e}$

prędkośč nurtu rzeki wynosi 4 $\mathrm{k}\mathrm{m}/$godz.

2. Uprościč wyrazenie (dla $a, b$, dla których ma ono sens)

$(\displaystyle \frac{\sqrt[6]{b}}{\sqrt{b}-\sqrt[6]{a^{3}b^{2}}}-\frac{a}{\sqrt{ab}-a\sqrt[3]{b}})[\frac{1}{\sqrt{a}-\sqrt{b}}(\sqrt[6]{a^{5}}-\frac{b}{\sqrt[6]{a}})-\frac{a-b}{\sqrt[3]{a^{2}}+\sqrt[6]{a}\sqrt{b}}],$

a nastepnie obliczyč jego wartośč dla $a=4\log_{4}81 \mathrm{i} b=(\log_{3}2)^{-1}$

3. Rozwiązač równanie $\sin 2x+\sin x=2+\cos x-2\cos^{2}x.$

4. Rozwiązač nierównośč $\displaystyle \frac{1}{\sqrt{4-x^{2}}}\geq\frac{1}{x-1} \mathrm{i}$ starannie zaznaczyč zbiór rozwi$\Phi$zań na osi

liczb owej.

5. $K\mathrm{a}\dot{\mathrm{z}}\mathrm{d}\mathrm{a}\mathrm{z}$ przekątnych trapezu ma dlugośč 5, jedna $\mathrm{z}$ podstaw ma długośč 2, a po1e równe

jest 12. Ob1iczyč promień okręgu opisanego $\mathrm{n}\mathrm{a}$ tym trapezie. Sporządzič rysunek.

6. $\mathrm{W}$ czworościanie ABCD jedna krawęd $\acute{\mathrm{z}}$ jest $0$ połowę krótsza od pozostałych, które sq

równe. Obliczyč objętośč oraz cosinusy kątów dwuściennych tego czworościanu. Sporzą-

dzič rysunek.





XXXX

KORESPONDENCYJNY KURS

Z MATEMATYKI

marzec 2010 r.

PRACA KONTROLNA nr 6- POZIOM PODSTAWOWY

l. Logarytmy (przy ustalonej podstawie) $\mathrm{z}$ liczb: $a_{1}=\displaystyle \frac{2}{5}x, a_{2}=x-1, a_{3}=x+3$ tworzą ciąg

arytmetyczny. Wyznaczyč $x$. Dla znalezionego $x$ obliczyč sumę początkowych dziesięciu

wyrazów ciągu geometrycznego, którego trzema pierwszymi wyrazami są liczby $a_{1}, a_{2}, a_{3}.$

2. Odcinek $0$ końcach $A(\displaystyle \frac{5}{2},\frac{\sqrt{3}}{2}), B(\displaystyle \frac{5}{2},\frac{3\sqrt{3}}{2})$ jest bokiem wielokąta foremnego wpisanego $\mathrm{w}$

okrąg styczny do osi $Ox$. Wyznaczyc równanie tego okręgu $\mathrm{i}$ wspófrzędne pozostafych

wierzchołków wielokąta. Ile rozwiązań ma to zadanie? Sporządzič rysunek.

3. Dany jest ostroslup prawidlowy trójk$\Phi$tny, $\mathrm{w}$ którym krawęd $\acute{\mathrm{z}}$ bocznajest dwa razy dfuz-

sza $\mathrm{n}\mathrm{i}\dot{\mathrm{z}}$krawed $\acute{\mathrm{z}}$ podstawy. Ostrosłup ten podzielono płaszczyzną przechodzącą przez kra-

$\mathrm{w}\mathrm{e}\mathrm{d}\acute{\mathrm{z}}$ podstawy na dwie bryły $0$ tej samej objętości. Wyznaczyč tangens kąta nachylenia

tej pfaszczyzny do pfaszczyzny podstawy. Sporz$\Phi$dzič rysunek.

4. $\mathrm{O}$ kącie $\alpha$ wiadomo, $\displaystyle \dot{\mathrm{z}}\mathrm{e}\sin\alpha-\cos\alpha=\frac{2}{\sqrt{3}}.$

a) Określič, $\mathrm{w}$ której čwiartce jest kąt $\alpha.$

b) Obliczyč tg $\alpha+$ ctg $\alpha$ oraz $\sin\alpha+\cos\alpha.$

c) Wyznaczyč tg $\alpha.$

5. Dfuzsza przyprostokątna $b$ trójkqta prostokątnego $0$ kącie ostrym $30^{\mathrm{o}}$ jest średnicą pól-

okręgu dzielącego ten trójkqt na dwa obszary. Wyznaczyč stosunek pól tych obszarów

oraz dfugośč promienia okręgu wpisanego $\mathrm{w}$ obszar $\mathrm{z}\mathrm{a}\mathrm{w}\mathrm{i}\mathrm{e}\mathrm{r}\mathrm{a}\mathrm{j}_{\Phi}\mathrm{c}\mathrm{y}$ wierzchofek kąta $60^{\mathrm{o}}$

Sporządzič rysunek.

6. Dwaj turyści wyruszyli jednocześnie: jeden $\mathrm{z}$ punktu $A$ do punktu $B$, drugi-z $B$ do $A.$

$K\mathrm{a}\dot{\mathrm{z}}\mathrm{d}\mathrm{y}\mathrm{z}$ nich szedf ze stafą prędkością $\mathrm{i}$ dotarfszy do mety, natychmiast ruszaf $\mathrm{w}$ drogę

powrotną. Pierwszy raz mineli się $\mathrm{w}$ odległości 12 km od punktu $B$, drugi- po upływie

6 godzin od momentu pierwszego spotkania-w odległości 6 km od punktu $A$. Obliczyč

odległośč punktów $A\mathrm{i}B\mathrm{i}$ prędkości, $\mathrm{z}$ jakimi poruszali się turyści.





PRACA KONTROLNA nr 6- POZIOM ROZSZERZONY

l. Rozwiązač równanie

$\sqrt{x^{2}-3}+\sqrt{5-2x}=4-x.$

2. $\mathrm{Z}$ urny zawierającej 2 ku1e białe, 4 czerwone $\mathrm{i}3$ czarne wylosowanojedną kulę. Następnie

wylosowano jeszcze trzy kule, gdy pierwsza okazała się biala, dwie kule, gdy pierwsza

była czerwona, lub jedną kulę, gdy $\mathrm{w}$ pierwszym losowaniu wypadła czarna. Obliczyč

prawdopodobieństwo, $\dot{\mathrm{z}}\mathrm{e}\mathrm{w}$ urnie nie pozostafa $\dot{\mathrm{z}}$ adna kula biafa.

3. Podstawą graniastosłupa prostego jest trójkąt $0$ bokach $a, b\mathrm{i}$ kącie między nimi $\alpha, \mathrm{a}$

przekątne ścian bocznych, wychodzące $\mathrm{z}$ wierzchołka kąta $\alpha, \mathrm{s}\Phi$ do siebie prostopadfe.

Obliczyč objętośč graniastoslupa.

4. Na jednym rysunku sporzadzič staranne wykresy funkcji

$f(x)=\sqrt{6x-x^{2}}$

oraz

$g(x)=|\displaystyle \frac{3}{2}-f(x+2)|.$

Obliczyč pole figury ograniczonej wykresem funkcji $g(x)\mathrm{i}\mathrm{o}\mathrm{s}\mathrm{i}_{\Phi}Ox.$

5. Podač dziedzinę $\mathrm{i}$ sprawdzič $\mathrm{t}\mathrm{o}\dot{\mathrm{z}}$ samośč

tg2 -$\alpha$2 $=$ -11 $+$-ccooss $\alpha\alpha$.

Cosinus kąta ostrego $\alpha$ wynosi $\displaystyle \frac{1}{8}$. Korzystając $\mathrm{z}$ powyzszej $\mathrm{t}\mathrm{o}\dot{\mathrm{z}}$ samości, obliczyč wartośč

sumy tg $\displaystyle \frac{\alpha}{4}+\mathrm{t}\mathrm{g}\frac{\alpha}{2}+\mathrm{t}\mathrm{g}\frac{3\alpha}{4}+\mathrm{t}\mathrm{g}\alpha$. Wynik podač $\mathrm{w}$ najprostszej postaci.

6. Punkt $C(-2,-1)$ jest wierzchofkiem trójkąta równoramiennego $ABC, \mathrm{w}$ którym $|AC|=$

$|BC|$. Środkowe trójkąta przecinają się $\mathrm{w}$ punkcie $M(1,2)$, a dwusieczne $\mathrm{w}$ punkcie

$S(\displaystyle \frac{1}{2},\frac{3}{2})$. Wyznaczyč wspófrzędne wierzcholków A $\mathrm{i}B.$





XXXIX

KORESPONDENCYJNY KURS

Z MATEMATYKI

listopad 2009 r.

PRACA KONTROLNA $\mathrm{n}\mathrm{r} 2-$ POZIOM PODSTAWOWY

l. Suma $n$ początkowych wyrazów ciagu $(a_{n})$ określona jest wzorem $S_{n} =2n^{2}+5n+c.$

Wyznaczyč stafą $c\mathrm{t}\mathrm{a}\mathrm{k}$, by $(a_{n})\mathrm{b}\mathrm{y}l$ ciągiem arytmetycznym. Obliczyč sumę dwudziestu

jeden pierwszych wyrazów tego ciągu $0$ numerach parzystych.

2. Narysowač zbiory: $A=\{(x,y):(x-1)^{2}\leq y\leq 2-|x-1|\}, B=\{(x,y):|x|+|x-2|\leq 2y\}$

oraz $(A\backslash B)\cup(B\backslash A)$. Ile wynosi pole figury $A\cap B$?

3. Przekrój graniastosłupa prawidlowego czworokątnego płaszczyzną zawierającą przekątną

podstawy ijedną $\mathrm{z}$ krawędzi bocznychjest kwadratem. Obliczyč stosunek pola przekroju

tego graniastosłupa plaszczyzną zawierającą przekątną podstawy dolnej $\mathrm{i}$ przeciwległy

wierzchołek podstawy górnej do pola przekroju płaszczyznq zawierającq przekatną gra-

niastosfupa $\mathrm{i}$ środki przeciwlegfych krawędzi bocznych. Sporz$\Phi$dzič rysunek.

4. Niech $f(x)=$

dla

dla

$x\leq 1,$

$x>1.$

a) Sporządzič wykres funkcji $f\mathrm{i}$ na jego podstawie wyznaczyč zbiór wartości tej funk-

cji.

b) Obliczyč $f(\sqrt{3}-1) \mathrm{i}$ korzystając $\mathrm{z}$ wykresu zaznaczyč na osi $0x$ zbiór rozwiązań

nierówności $f^{2}(x)\leq 4.$

5. Wiadomo, $\dot{\mathrm{z}}\mathrm{e}$ liczby $-1$, 3 są pierwiastkami wielomianu $W(x)=x^{4}-ax^{3}-4x^{2}+bx+3.$

Rozwiązač nierównośč $\sqrt{W(x)}\leq x^{2}-x.$

6. Punkt $A=(1,0)$ jest wierzchofkiem rombu $0$ kącie przy tym wierzcholku równym $60^{\mathrm{o}}$

Wyznaczyč współrzędne pozostałych wierzchołków rombu wiedząc, $\dot{\mathrm{z}}\mathrm{e}$ dwa $\mathrm{z}$ nich lezą

na prostej $l$ : $2x-y+3=0$. Ile rozwiqzań ma to zadanie?





PRACA KONTROLNA nr 2- POZIOM ROZSZERZONY

l. Dane są liczby $m=\displaystyle \frac{\left(\begin{array}{l}
6\\
4
\end{array}\right)\left(\begin{array}{l}
8\\
2
\end{array}\right)}{\left(\begin{array}{l}
7\\
3
\end{array}\right)},$

{\it n}$=$ -($\sqrt{}$($\sqrt{}$24)1-64)(3-41.)2-7-25-$\sqrt{}$4-413.

Wyznaczyč sume wszystkich wyrazów nieskończonego ciqgu geometrycznego, którego

pierwszym wyrazem jest $m$, a piątym $n$. Ile wyrazów tego ciągu nalez $\mathrm{y}$ wziqč, by ich

suma przekroczyla 99\% sumy wszystkich wyrazów?

2. Narysowač zbiory: $A=\{(x,y):x^{2}+2x+y^{2}\leq 3\}, B=\{(x,y):|y|\leq\sqrt{3}x+\sqrt{3}\}$

oraz $(A\backslash B)\cup(B\backslash A)$. Wyznaczyč równanie okręgu wpisanego $\mathrm{w}$ figurę $A\cap B.$

3. Liczby: $a_{1}=\log_{(3-2\sqrt{2})^{2}}(\sqrt{2}-1), \displaystyle \alpha_{2}=\frac{1}{2}\log_{\frac{1}{3}}\frac{\sqrt{3}}{6}, a_{3}=3^{\log_{\sqrt{3}^{\frac{\sqrt{6}}{2}}}}, a_{4}=\log_{(\sqrt{2}-1)}(\sqrt{2}+1),$

$a_{5}=(2^{\sqrt{2}+1})^{\sqrt{2}-1}, a_{6}=\log_{3}2$ są wszystkimi pierwiastkami wielomianu $W(x)$, którego

wyraz wolny jest dodatni.

a) Które $\mathrm{z}$ tych pierwiastków są niewymierne? Odpowiedz/uzasadnič.

b) Wyznaczyč dziedzinę funkcji

nych.

$f(x) = \sqrt{W(x)},$

nie wykonując obliczeń przyblizo-

4. Narysowač wykres funkcji $f$ zadanej wzorem $f(x)=$

Posfugując się wykresem $\mathrm{i}$ odpowiednimi obliczeniami rozwiązač nierównośč

$|f(x)-\displaystyle \frac{1}{2}|<\frac{1}{4}$

5. Na prostej $x+2y=5$ wyznaczyč punkty, $\mathrm{z}$ których okrqg $(x-1)^{2}+(y-1)^{2}=1$ jest

widoczny pod kątem $60^{\mathrm{o}}$. Obliczyč pole obszaru ograniczonego lukiem okręgu $\mathrm{i}$ stycznymi

do niego poprowadzonymi $\mathrm{w}$ znalezionych punktach. Sporządzič rysunek.

6. Na dnie naczynia $\mathrm{w}$ ksztalcie walca umieszczono cztery jednakowe metalowe kulki $0$

$\mathrm{m}\mathrm{o}\dot{\mathrm{z}}$ liwie największej objętości. Następnie do naczynia wrzucono jeszcze $\mathrm{j}\mathrm{e}\mathrm{d}\mathrm{n}\Phi$ kulkę $\mathrm{i}$

okazało się, $\dot{\mathrm{z}}\mathrm{e}$ jest ona styczna do płaskiej pokrywy naczynia. Wyznaczyč promienie

kulek wiedząc, $\dot{\mathrm{z}}\mathrm{e}$ przekrój osiowy walca jest kwadratem $0$ boku $d.$





XXXIX

KORESPONDENCYJNY KURS

Z MATEMATYKI

grudzień 2009 r.

PRACA KONTROLNA nr 3 -POZIOM PODSTAWOWY

l. Sześč kostek sześciennych $0$ objętościach 256, 128, 64, 32, 16 $\mathrm{i}8\mathrm{c}\mathrm{m}^{3}$ ustawiono $\mathrm{w}$ pirami-

dę. Czy $\mathrm{m}\mathrm{o}\dot{\mathrm{z}}$ na tę piramidę umieścič na pófce $0$ wysokości 24 cm? Odpowied $\acute{\mathrm{z}}$ uzasadnič

bez wykonywania obliczeń przyblizonych.

2. Wojtuś postawif przypadkowo cztery pionki na szachownicy $016$ polach. Jakiejest praw-

dopodobieństwo, $\dot{\mathrm{z}}\mathrm{e}$ co najwyzej dwa pionki będą staly $\mathrm{w}$ szeregu (poziomo lub pionowo)?

3. Rozwiązač nierównośč

$|\displaystyle \frac{x^{2}+3x+2}{2x^{2}+7x+6}|\leq 1.$

4. Lamana ABCD jest przedstawiona na rysunku ponizej. Niech E będzie punktem prze-

cięcia się prostych AB iCD. Obliczyč pole trójk$\Phi$ta CBE.

5. Obserwator, stojąc $\mathrm{w}$ pewnej odlegfości, widzi wiezę kościofa pod kątem $60^{\mathrm{o}}$ Po odda-

leniu się $050\mathrm{m}$ kąt widzenia zmniejszył się do $45^{\mathrm{o}}$ Obliczyč cosinus kąta, pod jakim

obserwator będzie widział wiezę kościofa, jeśli oddali się $0$ kolejne 50 $\mathrm{m}.$

6. Wycinek koła ma obwód $2s$, gdzie $s > 0$ jest ustaloną liczbą. Wyrazič pole $P$ tego

wycinka jako funkcję promienia $r$ kofa. Sporządzič wykres funkcji $P=P(r).$





PRACA KONTROLNA nr 3 -POZIOM ROZSZERZONY

l. Sporządzič wykres funkcji $f(m) = \displaystyle \frac{1}{x_{1}}+\frac{1}{x_{2}}$, gdzie $x_{1}, x_{2}$ sa pierwiastkami równania

$x^{2}-2mx+m+2=0$, a $m$ jest parametrem rzeczywistym.

2. Ala ulozyła $\mathrm{z}$ czterech klocków liczbę 2009. Nastepnie spośród tych k1ocków 1osowa-

fa ze zwracaniem cztery razy po jednym klocku. Jakie jest prawdopodobieństwo, $\dot{\mathrm{z}}\mathrm{e}\mathrm{z}$

otrzymanych $\mathrm{w}$ ten sposób cyfr $\mathrm{m}\mathrm{o}\dot{\mathrm{z}}$ na byłoby utworzyč liczbe:

a) podzielną przez 3?

b) podzielną przez 4?

3. Rozwazmy funkcje $ f(x)=4^{x+1}+4^{2x+1}+4^{3x+1}+\ldots$ oraz $g(x)=2^{x}+2^{x-1}+2^{x-2}+\ldots,$

gdzie prawe strony wzorów określających obie funkcje są sumami wyrazów nieskończo-

nych ciągów geometrycznych. Wykazač, $\dot{\mathrm{z}}\mathrm{e}$ funkcja $f(x)$ jest rosnąca. Znalez/č wszystkie

liczby $x$, dla których $f(x)=g(x).$

4. Rozwiązač nierównośč

$\displaystyle \frac{\mathrm{t}\mathrm{g}x+\sin x}{3\mathrm{t}\mathrm{g}x-2\sin x}\geq\cos^{2}\frac{x}{2}.$

5. Okrag styczny do ramion paraboli $y = x^{2}-2x$ jest styczny równocześnie do osi $Ox.$

Znalez/č równania stycznych do okręgu $\mathrm{w}$ punktach jego styczności $\mathrm{z}$ parabolą.

6. $\mathrm{Z}$ odcinków $0$ długościach równych czterem najmniejszym nieparzystym liczbom pierw-

szym zbudowano trapez, którego pole jest liczbą wymierną. Wyznaczyč tangens $\mathrm{k}_{\Phi^{\mathrm{t}\mathrm{a}}}$

między przekątnymi tego trapezu.





XXXIX

KORESPONDENCYJNY KURS

Z MATEMATYKI

styczeń 2010 r.

PRACA KONTROLNA $\mathrm{n}\mathrm{r} 4-$ POZIOM PODSTAWOWY

l. Mamy dwa termosy kawy $\mathrm{z}$ mlekiem. $\mathrm{W}$ pierwszym termosie stosunek objętości mleka

do objętości kawy wynosi 2:3, a $\mathrm{w}$ drugim 3:7. I1e 1itrów p1ynu na1ez $\mathrm{y}$ wziąč $\mathrm{z}\mathrm{k}\mathrm{a}\dot{\mathrm{z}}$ dego

termosu, aby otrzymač 2,41itra kawy $\mathrm{z}$ mlekiem, $\mathrm{w}$ której objętośč kawy bedzie dwa

razy większa $\mathrm{n}\mathrm{i}\dot{\mathrm{z}}$ objętośč mleka?

2. Kwotę l00000 zf wpfacono na lokatę roczną, $\mathrm{w}$ której odsetki doliczane są co kwartaf. Po

roku suma odsetek wyniosła dokładnie 4060,401 $\mathrm{z}l$. Znalez/č oprocentowanie tej lokaty.

Jakie powinno byč oprocentowanie lokaty, aby przy kapitalizacji dokonywanej raz na póf

roku osiągnąč ten sam zysk?

3. Dane są zbiory $A=\{(x,y):x,y,\in \mathbb{R},y^{2}-4x^{2}\geq 0\}\mathrm{i}B=\{(x,y):|x|+|y|\leq 2\}$. Nary-

sowač zbiór $A\cup B$. Znalez$\acute{}$č punkt ze zbioru $A\cup B$ pofozony najblizej puntu $C=(3,2).$

4. Narysowač wykres trójmianu kwadratowego $f(x) = x^{2}+4x-5$ oraz wykres funkcji

$g(x)=4-f(x-2).$

a) Rozwiązač nierównośč $f(x)>g(x).$

b) Znalez/č obraz wykresu funkcji $f(x) \mathrm{w}$ symetrii względem prostej $x=2 \mathrm{i}$ na tej

podstawie podač wzór tej funkcji.

5. $\mathrm{W}$ ostrosfupie prawidłowym trójk$\Phi$tnym $0$ krawędzi podstawy równej $a$ kąt pfaski ściany

bocznej przy wierzchołku jest równy $ 2\alpha$. Obliczyč objętośč tego ostrosłupa oraz sinus

kąta nachylenia ściany bocznej do podstawy.

6. $\mathrm{W}$ trapezie ABCD, $\mathrm{w}$ którym bok $AB$ jest równolegfy do boku $DC$, dane są: $\angle BAD=$

$\displaystyle \frac{\pi}{3}, |AB| =20, |DC| =8$ oraz $|AD| =5$. Obliczyč obwód tego trapezu, $\sin\angle ADB$ oraz

odlegfośč punktu przecięcia się przekątnych tego trapezu od jego podstaw.





PRACA KONTROLNA nr 4- POZIOM ROZSZERZONY

l. Dzieląc wielomian $W(x)$ przez dwumian $x-3$ otrzymujemy resztę równą 2, a dzie1ąc

ten wielomian przez $x-2$ otrzymujemy resztę równą l. Wyznaczyč resztę $\mathrm{z}$ dzielenia

$W(x)$ przez $(x-2)(x-3)$. Znalez/č wielomian trzeciego stopnia spełniający powyzsze

warunki wiedzqc, $\dot{\mathrm{z}}\mathrm{e}x=1$ jest pierwiastkiem tego wielomianu, a suma wyrazu wolnego

$\mathrm{i}$ wspólczynnika przy $x^{3}$ jest równa 0.

2. Znalez/č najmniejszą $\mathrm{i}$ największą wartośč funkcji $f(x)=\displaystyle \sin x-\frac{1}{2}\cos 2x$ na przedziale

$[-\displaystyle \frac{\pi}{2},\frac{\pi}{2}] \mathrm{i}$ rozwiązač nierównośč - $\displaystyle \frac{1}{2}\leq f(x)\leq\frac{1}{4}$. Zadanie rozwiązač bez $\mathrm{u}\dot{\mathrm{z}}$ ywania pojęcia

pochodnej.

3. Rozwiązač nierównośč

$\log_{\frac{1}{\sqrt{2}}}(2^{2x+1}-16^{x})\geq-12x.$

4. $\mathrm{W}$ stozek $\mathrm{o}\mathrm{k}_{\Phi}\mathrm{c}\mathrm{i}\mathrm{e}$ rozwarcia równym $ 2\alpha$ wpisano kulę $0$ promieniu $R$. Wewnątrz stozka

stawiamy na kuli sześcian $0$ maksymalnej objętości $\mathrm{i}$ podstawie równoleglej do podstawy

stozka. Wyznaczyč dlugośč krawędzi tego sześcianu.

5. Stosunek dlugości promienia okręgu wpisanego do dfugości promienia okręgu opisanego

na trójkącie prostokqtnym wynosi $\displaystyle \frac{1}{3+2\sqrt{3}}$. Obliczyč sinusy kątów ostrych tego trójkąta.

6. Ślimak ma do przejścia taśmę $0$ długości 3 metrów zamocowaną $\mathrm{w}$ punkcie startu A. $\mathrm{W}$

ciągu $\mathrm{k}\mathrm{a}\dot{\mathrm{z}}$ dego dnia udaje mu się przejśč l metr, a $\mathrm{k}\mathrm{a}\dot{\mathrm{z}}$ dej nocy gdy śpi, ktoś- ciągnąc

za drugi koniec taśmy- wydłuza $\mathrm{j}\mathrm{a}$ równomiernie $0 1$ metr. Niech $d_{n}$ oznacza długośč

taśmy $\mathrm{w}n$-tym dniu, a $a_{n}$- odległośč ślimaka od punktu A przy końcu $n$-tego dnia.

a) Uzasadnič, $\dot{\mathrm{z}}\mathrm{e}$ ciąg $(a_{n})$ zdefiniowany jest następującym wzorem rekurencyjnym:

$a_{1}=1$ oraz $a_{n+1}=\displaystyle \frac{3+n}{2+n}a_{n}+1$ dla $n\geq 1.$

b) Pokazač, $\displaystyle \dot{\mathrm{z}}\mathrm{e}a_{n}=(n+2)(\frac{1}{3}+\frac{1}{4}+\ldots+\frac{1}{n+2}), n\geq 1.$

c) Czy ślimak dojdzie do końca taśmy? $\mathrm{J}\mathrm{e}\dot{\mathrm{z}}$ eli tak, to $\mathrm{w}$ którym dniu, to znaczy, dla

jakich $n$ prawdziwa jest nierównośč $a_{n}>d_{n}$?





XXXIX

KORESPONDENCYJNY KURS

Z MATEMATYKI

luty 2010 r.

PRACA KONTROLNA $\mathrm{n}\mathrm{r} 5-$ POZIOM PODSTAWOWY

l. Dwie wiewiórki, Kasiai Basia, postanowiły wspólnie zbierač orzechy. $\mathrm{K}\mathrm{a}\dot{\mathrm{z}}$ dego dnia Basia

przynosifa do wspólnej spizarni $04$ orzechy więcej $\mathrm{n}\mathrm{i}\dot{\mathrm{z}}$ Kasia, codziennie tyle samo. Po

30 dniach współpracy wiewiórki pokłóciły się. Basia zostawiła Kasi wszystkie orzechy

$\mathrm{i}$ zafozyła wlasną spizarnię. Od tamtej pory $\mathrm{k}\mathrm{a}\dot{\mathrm{z}}$ da $\mathrm{z}$ wiewiórek przynosi do swojej spizarni

tę samą ilośč orzechów co przedtem, ale Basia codziennie dostaje 6 orzechów od Kasi. Po

50 dniach samodzielnej pracy Kasia ma jeszcze $0100$ orzechów więcej $\mathrm{n}\mathrm{i}\dot{\mathrm{z}}$ Basia. Ustalič,

po ile orzechów zbiera codziennie $\mathrm{k}\mathrm{a}\dot{\mathrm{z}}$ da $\mathrm{z}$ wiewiórek $\mathrm{i}$ oszacowač, po ilu dniach $\mathrm{w}$ spizarni

Basi będzie więcej orzechów $\mathrm{n}\mathrm{i}\dot{\mathrm{z}}\mathrm{u}$ kolezanki.

2. Określič dziedzinę $\mathrm{i}$ zbiór wartości funkcji $f(x)=\sin x\cdot\sin 2x$. (tg $x+$ ctg $x$). Wykonač

staranny wykres funkcji $g(x)=f(x-\displaystyle \frac{\pi}{4})+1\mathrm{i}$ rozwiązač równanie $g(x)=0$. Posfugując

się sporzqdzonym wykresem określič zbiór rozwiązań nierówności $g(x)\geq 0.$

3. Wyznaczyč równania wszystkich prostych, które są styczne jednocześnie do obu okręgów

$(x-1)^{2}+(y-1)^{2}=1$

oraz

$(x-5)^{2}+(y-1)^{2}=1.$

Obliczenia zilustrowač odpowiednim rysunkiem.

4. Rozwiązač nierównośč

$\displaystyle \frac{3\sqrt{4-x}+1}{1-\sqrt{4-x}}>1-2\sqrt{4-x}.$

5. Koszt budowy I kondygnacji biurowca wynosi l0 mln zł., a $\mathrm{k}\mathrm{a}\dot{\mathrm{z}}$ dej kolejnej jest $\mathrm{n}\mathrm{i}\dot{\mathrm{z}}$ szy

$0100\mathrm{t}\mathrm{y}\mathrm{s}. \mathrm{z}\mathrm{f}$. od poprzedniej. Planowany koszt wynajmu powierzchni biurowych $\mathrm{w}$ tym

budynku jest stały do XL kondygnacji $\mathrm{i}$ wynosi 200 $\mathrm{t}\mathrm{y}\mathrm{s}$. zł. za całą kondygnację, $\mathrm{a}$

potem podwaja się co 5 kondygnacji (na ko1ejnych 5 kondygnacjach jest stały). Roczny

koszt wynajmu ostatniej, najbardziej prestizowej $\mathrm{i}$ drozszej od pozostafych kondygnacji

jest równy kosztowi budowy całego XXXVII piętra. Oszacowač, po ilu latach zwróci się

inwestorom koszt budowy tego budynku.

6. $\mathrm{W}$ trapezie równoramiennym kąt przy podstawie ma miarę $\displaystyle \frac{\pi}{3}$, a róznica długości podstaw

wynosi 4. Usta1ič, i1e powinno wynosič po1e tego trapezu, aby $\mathrm{m}\mathrm{o}\dot{\mathrm{z}}$ na było wpisač $\mathrm{w}$ niego

kofo. $\mathrm{W}$ tym przypadku wyznaczyč stosunek pola kofa opisanego na tym trapezie do pola

koła wpisanego.





PRACA KONTROLNA nr 5- POZIOM ROZSZERZONY

l. Znalez$\acute{}$č wszystkie liczby rzeczywiste m, dla których równanie

$\displaystyle \frac{x}{m}+m=\frac{m}{x}+x+1$

ma dwa pierwiastki róznych znaków.

2. Rozwiązač nierównośč

$2^{x^{2}+4}+2^{x^{2}+3}+2^{x^{2}}>5^{x^{2}+1}-25\cdot 2^{x^{2}-2}$

3. Określič dziedzinę $\mathrm{i}$ zbiór wartości funkcji $f(x)=\displaystyle \mathrm{c}\mathrm{t}\mathrm{g}(\pi+x)\mathrm{c}\mathrm{t}\mathrm{g}(x-\frac{\pi}{2})\cos x$. Sporz$\Phi$dzič

staranny wykres funkcji $g(x) =2f(|x-\displaystyle \frac{\pi}{4}|)-1$. Na podstawie wykresu $\mathrm{i}$ niezbędnych

obliczeń rozwiązač nierównośč $g(x)\leq-2$, a zbiór jej rozwiązań zaznaczyč na osi OX.

4. Rozwiązač nierównośč

$\displaystyle \log_{x^{2}}(3x-1)-\log_{x^{2}}(x-1)^{2}+\log_{x^{2}}|x-1|\geq\frac{1}{2}.$

5. $\mathrm{W}$ ostroslupie sześciokątnym prawidfowym kąt dwuścienny utworzony przez pfaszczyzny

przeciwległych ścian bocznych ma miarę $\displaystyle \frac{\pi}{4}$. Wyznaczyč promień $R$ kuli opisanej na tym

ostroslupie jako funkcję dfugości $a$ boku jego podstawy.

6. $\mathrm{W}$ kolo wpisano ośmiokąt foremny, $\mathrm{w}$ ośmiokąt kofo, $\mathrm{w}$ kofo kolejny ośmiokąt foremny

itd. Wysunač hipotezę $0$ wartości pola $n$-tego koła $\mathrm{i}$ uzasadnič $\mathrm{j}\mathrm{a}$ indukcyjnie. Suma pól

nieskończonego $\mathrm{c}\mathrm{i}_{\Phi \mathrm{g}}\mathrm{u}$ kól otrzymanych $\mathrm{w}$ ten sposób jest ośmiokrotnością pola jednego

$\mathrm{z}$ nich. Ustalič którego, nie stosując obliczeń przyblizonych.



\end{document}