\documentclass[a4paper,12pt]{article}
\usepackage{latexsym}
\usepackage{amsmath}
\usepackage{amssymb}
\usepackage{graphicx}
\usepackage{wrapfig}
\pagestyle{plain}
\usepackage{fancybox}
\usepackage{bm}

\begin{document}

XXXIX

KORESPONDENCYJNY KURS

Z MATEMATYKI

$\mathrm{p}\mathrm{a}\acute{\mathrm{z}}$dziernik 2009 $\mathrm{r}.$

PRACA KONTROLNA $\mathrm{n}\mathrm{r} 1-$ POZIOM PODSTAWOWY

l. Właściciel hurtowni sprzedał $\displaystyle \frac{1}{3}$ partii bananów po załozonej przez siebie cenie. Okazalo

się, $\dot{\mathrm{z}}\mathrm{e}$ owoce zbyt szybko dojrzewają, więc obnizyf cenę $0$ 30\% $\mathrm{i}$ wówczas sprzedaf 60\%

pozostałej ilości owoców. Resztę bananów udało mu się sprzedač dopiero, gdy ustalił

ich cenę na poziomie $\displaystyle \frac{1}{5}$ ceny początkowej. Ile procent zaplanowanego zysku stanowi

kwota uzyskana ze sprzedaz $\mathrm{y}$? Po ile powinien byf sprzedač pierwszą partię towaru, by

jednokrotna obnizka ich ceny $0$ 25\% pozwoliła na sprzedaz wszystkich owoców $\mathrm{i}$ uzyskanie

zaplanowanego początkowo zysku?

2. Przekątne trapezu $0$ podstawach 3 $\mathrm{i}4$ przecinają się pod kątem prostym. Na $\mathrm{k}\mathrm{a}\dot{\mathrm{z}}$ dym

$\mathrm{z}$ boków trapezu, jako na średnicy, oparto półokrąg. Obliczyč sume pól otrzymanych

czterech pólkoli. Sporządzič rysunek.

3. Uprościč wyrazenie $\displaystyle \frac{1}{\sqrt{a}-\sqrt{b}}(\sqrt[6]{a^{5}}-\frac{b}{\sqrt[6]{\alpha}}) -\displaystyle \frac{a-b}{\sqrt[3]{a^{2}}+\sqrt[6]{\alpha}\sqrt{b}} \mathrm{d}\mathrm{l}\mathrm{a}a, b, \mathrm{d}\mathrm{l}\mathrm{a}$ których ma

ono sens. Następnie obliczyč jego wartośč, przyjmując $a=(4-2\sqrt{3})^{3}\mathrm{i} b=3+2\sqrt{2}.$

4. Podstawą ostrosfupa prawidlowego jest sześciok$\Phi$t foremny $0$ boku $a$. Obliczyč objętośč,

wiedząc, $\dot{\mathrm{z}}\mathrm{e}$ najmniejszy ($\mathrm{w}$ sensie powierzchni) $\mathrm{z}$ przekrojów ostrosłupa płaszczyzną

zawierajqcą wysokośč jest trójkątem równobocznym. Wyznaczyč cosinus kąta między

ścianami bocznymi ostrosfupa. Sporz$\Phi$dzič rysunek.

5. Dana jest funkcja liniowa $f(x)=2x-6.$

a) Dlajakiego $a$ pole trójkąta ograniczonego osiami ukfadu wspófrzędnych $\mathrm{i}$ wykresem

funkcji $h(x)=f(x-a)$ równe jest 4? Sporządzič rysunek.

b) Narysowač zbiór $D=\{(x,y):f(x^{2}+2x)\leq y\leq f(x+2)\}.$

6. Sporządzič wykres funkcji $f(x)=$

dla

dla

$x<0,$

$x\geq 0.$

Posfugując się nim, wyznaczyč przedzialy monotoniczności tej funkcji. Narysowač wy-

kres funkcji $g(m)$ określajacej liczbę rozwiqzań równania $f(x)=|m| \mathrm{w}$ zalezności od

parametru rzeczywistego $m.$
\end{document}
