\documentclass[a4paper,12pt]{article}
\usepackage{latexsym}
\usepackage{amsmath}
\usepackage{amssymb}
\usepackage{graphicx}
\usepackage{wrapfig}
\pagestyle{plain}
\usepackage{fancybox}
\usepackage{bm}

\begin{document}

PRACA KONTROLNA nr l- POZIOM ROZSZERZONY

l. Statek wyrusza ($\mathrm{z}$ biegiem rzeki) $\mathrm{z}$ przystani A do odległej $0 140$ km przystani B. Po

uplywie l godziny wyrusza za nim łódz/ motorowa, dopędza statek $\mathrm{w}$ pofowie drogi,

po czym wraca do przystani A $\mathrm{w}$ tym samym momencie, $\mathrm{w}$ którym statek przybija do

przystani B. Wyznaczyč prędkośč statku $\mathrm{i}$ prędkośč lodzi $\mathrm{w}$ wodzie stojącej, wiedzqc, $\dot{\mathrm{z}}\mathrm{e}$

prędkośč nurtu rzeki wynosi 4 $\mathrm{k}\mathrm{m}/$godz.

2. Uprościč wyrazenie (dla $a, b$, dla których ma ono sens)

$(\displaystyle \frac{\sqrt[6]{b}}{\sqrt{b}-\sqrt[6]{a^{3}b^{2}}}-\frac{a}{\sqrt{ab}-a\sqrt[3]{b}})[\frac{1}{\sqrt{a}-\sqrt{b}}(\sqrt[6]{a^{5}}-\frac{b}{\sqrt[6]{a}})-\frac{a-b}{\sqrt[3]{a^{2}}+\sqrt[6]{a}\sqrt{b}}],$

a nastepnie obliczyč jego wartośč dla $a=4\log_{4}81 \mathrm{i} b=(\log_{3}2)^{-1}$

3. Rozwiązač równanie $\sin 2x+\sin x=2+\cos x-2\cos^{2}x.$

4. Rozwiązač nierównośč $\displaystyle \frac{1}{\sqrt{4-x^{2}}}\geq\frac{1}{x-1} \mathrm{i}$ starannie zaznaczyč zbiór rozwi$\Phi$zań na osi

liczb owej.

5. $K\mathrm{a}\dot{\mathrm{z}}\mathrm{d}\mathrm{a}\mathrm{z}$ przekątnych trapezu ma dlugośč 5, jedna $\mathrm{z}$ podstaw ma długośč 2, a po1e równe

jest 12. Ob1iczyč promień okręgu opisanego $\mathrm{n}\mathrm{a}$ tym trapezie. Sporządzič rysunek.

6. $\mathrm{W}$ czworościanie ABCD jedna krawęd $\acute{\mathrm{z}}$ jest $0$ połowę krótsza od pozostałych, które sq

równe. Obliczyč objętośč oraz cosinusy kątów dwuściennych tego czworościanu. Sporzą-

dzič rysunek.
\end{document}
