\documentclass[a4paper,12pt]{article}
\usepackage{latexsym}
\usepackage{amsmath}
\usepackage{amssymb}
\usepackage{graphicx}
\usepackage{wrapfig}
\pagestyle{plain}
\usepackage{fancybox}
\usepackage{bm}

\begin{document}

XXXX

KORESPONDENCYJNY KURS

Z MATEMATYKI

marzec 2010 r.

PRACA KONTROLNA nr 6- POZIOM PODSTAWOWY

l. Logarytmy (przy ustalonej podstawie) $\mathrm{z}$ liczb: $a_{1}=\displaystyle \frac{2}{5}x, a_{2}=x-1, a_{3}=x+3$ tworzą ciąg

arytmetyczny. Wyznaczyč $x$. Dla znalezionego $x$ obliczyč sumę początkowych dziesięciu

wyrazów ciągu geometrycznego, którego trzema pierwszymi wyrazami są liczby $a_{1}, a_{2}, a_{3}.$

2. Odcinek $0$ końcach $A(\displaystyle \frac{5}{2},\frac{\sqrt{3}}{2}), B(\displaystyle \frac{5}{2},\frac{3\sqrt{3}}{2})$ jest bokiem wielokąta foremnego wpisanego $\mathrm{w}$

okrąg styczny do osi $Ox$. Wyznaczyc równanie tego okręgu $\mathrm{i}$ wspófrzędne pozostafych

wierzchołków wielokąta. Ile rozwiązań ma to zadanie? Sporządzič rysunek.

3. Dany jest ostroslup prawidlowy trójk$\Phi$tny, $\mathrm{w}$ którym krawęd $\acute{\mathrm{z}}$ bocznajest dwa razy dfuz-

sza $\mathrm{n}\mathrm{i}\dot{\mathrm{z}}$krawed $\acute{\mathrm{z}}$ podstawy. Ostrosłup ten podzielono płaszczyzną przechodzącą przez kra-

$\mathrm{w}\mathrm{e}\mathrm{d}\acute{\mathrm{z}}$ podstawy na dwie bryły $0$ tej samej objętości. Wyznaczyč tangens kąta nachylenia

tej pfaszczyzny do pfaszczyzny podstawy. Sporz$\Phi$dzič rysunek.

4. $\mathrm{O}$ kącie $\alpha$ wiadomo, $\displaystyle \dot{\mathrm{z}}\mathrm{e}\sin\alpha-\cos\alpha=\frac{2}{\sqrt{3}}.$

a) Określič, $\mathrm{w}$ której čwiartce jest kąt $\alpha.$

b) Obliczyč tg $\alpha+$ ctg $\alpha$ oraz $\sin\alpha+\cos\alpha.$

c) Wyznaczyč tg $\alpha.$

5. Dfuzsza przyprostokątna $b$ trójkqta prostokątnego $0$ kącie ostrym $30^{\mathrm{o}}$ jest średnicą pól-

okręgu dzielącego ten trójkqt na dwa obszary. Wyznaczyč stosunek pól tych obszarów

oraz dfugośč promienia okręgu wpisanego $\mathrm{w}$ obszar $\mathrm{z}\mathrm{a}\mathrm{w}\mathrm{i}\mathrm{e}\mathrm{r}\mathrm{a}\mathrm{j}_{\Phi}\mathrm{c}\mathrm{y}$ wierzchofek kąta $60^{\mathrm{o}}$

Sporządzič rysunek.

6. Dwaj turyści wyruszyli jednocześnie: jeden $\mathrm{z}$ punktu $A$ do punktu $B$, drugi-z $B$ do $A.$

$K\mathrm{a}\dot{\mathrm{z}}\mathrm{d}\mathrm{y}\mathrm{z}$ nich szedf ze stafą prędkością $\mathrm{i}$ dotarfszy do mety, natychmiast ruszaf $\mathrm{w}$ drogę

powrotną. Pierwszy raz mineli się $\mathrm{w}$ odległości 12 km od punktu $B$, drugi- po upływie

6 godzin od momentu pierwszego spotkania-w odległości 6 km od punktu $A$. Obliczyč

odległośč punktów $A\mathrm{i}B\mathrm{i}$ prędkości, $\mathrm{z}$ jakimi poruszali się turyści.
\end{document}
