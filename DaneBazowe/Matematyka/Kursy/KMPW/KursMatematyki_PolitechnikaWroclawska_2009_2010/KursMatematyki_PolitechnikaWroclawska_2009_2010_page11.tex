\documentclass[a4paper,12pt]{article}
\usepackage{latexsym}
\usepackage{amsmath}
\usepackage{amssymb}
\usepackage{graphicx}
\usepackage{wrapfig}
\pagestyle{plain}
\usepackage{fancybox}
\usepackage{bm}

\begin{document}

PRACA KONTROLNA nr 6- POZIOM ROZSZERZONY

l. Rozwiązač równanie

$\sqrt{x^{2}-3}+\sqrt{5-2x}=4-x.$

2. $\mathrm{Z}$ urny zawierającej 2 ku1e białe, 4 czerwone $\mathrm{i}3$ czarne wylosowanojedną kulę. Następnie

wylosowano jeszcze trzy kule, gdy pierwsza okazała się biala, dwie kule, gdy pierwsza

była czerwona, lub jedną kulę, gdy $\mathrm{w}$ pierwszym losowaniu wypadła czarna. Obliczyč

prawdopodobieństwo, $\dot{\mathrm{z}}\mathrm{e}\mathrm{w}$ urnie nie pozostafa $\dot{\mathrm{z}}$ adna kula biafa.

3. Podstawą graniastosłupa prostego jest trójkąt $0$ bokach $a, b\mathrm{i}$ kącie między nimi $\alpha, \mathrm{a}$

przekątne ścian bocznych, wychodzące $\mathrm{z}$ wierzchołka kąta $\alpha, \mathrm{s}\Phi$ do siebie prostopadfe.

Obliczyč objętośč graniastoslupa.

4. Na jednym rysunku sporzadzič staranne wykresy funkcji

$f(x)=\sqrt{6x-x^{2}}$

oraz

$g(x)=|\displaystyle \frac{3}{2}-f(x+2)|.$

Obliczyč pole figury ograniczonej wykresem funkcji $g(x)\mathrm{i}\mathrm{o}\mathrm{s}\mathrm{i}_{\Phi}Ox.$

5. Podač dziedzinę $\mathrm{i}$ sprawdzič $\mathrm{t}\mathrm{o}\dot{\mathrm{z}}$ samośč

tg2 -$\alpha$2 $=$ -11 $+$-ccooss $\alpha\alpha$.

Cosinus kąta ostrego $\alpha$ wynosi $\displaystyle \frac{1}{8}$. Korzystając $\mathrm{z}$ powyzszej $\mathrm{t}\mathrm{o}\dot{\mathrm{z}}$ samości, obliczyč wartośč

sumy tg $\displaystyle \frac{\alpha}{4}+\mathrm{t}\mathrm{g}\frac{\alpha}{2}+\mathrm{t}\mathrm{g}\frac{3\alpha}{4}+\mathrm{t}\mathrm{g}\alpha$. Wynik podač $\mathrm{w}$ najprostszej postaci.

6. Punkt $C(-2,-1)$ jest wierzchofkiem trójkąta równoramiennego $ABC, \mathrm{w}$ którym $|AC|=$

$|BC|$. Środkowe trójkąta przecinają się $\mathrm{w}$ punkcie $M(1,2)$, a dwusieczne $\mathrm{w}$ punkcie

$S(\displaystyle \frac{1}{2},\frac{3}{2})$. Wyznaczyč wspófrzędne wierzcholków A $\mathrm{i}B.$
\end{document}
