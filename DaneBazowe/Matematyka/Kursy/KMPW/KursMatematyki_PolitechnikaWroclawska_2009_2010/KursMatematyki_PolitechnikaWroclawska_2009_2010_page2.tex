\documentclass[a4paper,12pt]{article}
\usepackage{latexsym}
\usepackage{amsmath}
\usepackage{amssymb}
\usepackage{graphicx}
\usepackage{wrapfig}
\pagestyle{plain}
\usepackage{fancybox}
\usepackage{bm}

\begin{document}

XXXIX

KORESPONDENCYJNY KURS

Z MATEMATYKI

listopad 2009 r.

PRACA KONTROLNA $\mathrm{n}\mathrm{r} 2-$ POZIOM PODSTAWOWY

l. Suma $n$ początkowych wyrazów ciagu $(a_{n})$ określona jest wzorem $S_{n} =2n^{2}+5n+c.$

Wyznaczyč stafą $c\mathrm{t}\mathrm{a}\mathrm{k}$, by $(a_{n})\mathrm{b}\mathrm{y}l$ ciągiem arytmetycznym. Obliczyč sumę dwudziestu

jeden pierwszych wyrazów tego ciągu $0$ numerach parzystych.

2. Narysowač zbiory: $A=\{(x,y):(x-1)^{2}\leq y\leq 2-|x-1|\}, B=\{(x,y):|x|+|x-2|\leq 2y\}$

oraz $(A\backslash B)\cup(B\backslash A)$. Ile wynosi pole figury $A\cap B$?

3. Przekrój graniastosłupa prawidlowego czworokątnego płaszczyzną zawierającą przekątną

podstawy ijedną $\mathrm{z}$ krawędzi bocznychjest kwadratem. Obliczyč stosunek pola przekroju

tego graniastosłupa plaszczyzną zawierającą przekątną podstawy dolnej $\mathrm{i}$ przeciwległy

wierzchołek podstawy górnej do pola przekroju płaszczyznq zawierającq przekatną gra-

niastosfupa $\mathrm{i}$ środki przeciwlegfych krawędzi bocznych. Sporz$\Phi$dzič rysunek.

4. Niech $f(x)=$

dla

dla

$x\leq 1,$

$x>1.$

a) Sporządzič wykres funkcji $f\mathrm{i}$ na jego podstawie wyznaczyč zbiór wartości tej funk-

cji.

b) Obliczyč $f(\sqrt{3}-1) \mathrm{i}$ korzystając $\mathrm{z}$ wykresu zaznaczyč na osi $0x$ zbiór rozwiązań

nierówności $f^{2}(x)\leq 4.$

5. Wiadomo, $\dot{\mathrm{z}}\mathrm{e}$ liczby $-1$, 3 są pierwiastkami wielomianu $W(x)=x^{4}-ax^{3}-4x^{2}+bx+3.$

Rozwiązač nierównośč $\sqrt{W(x)}\leq x^{2}-x.$

6. Punkt $A=(1,0)$ jest wierzchofkiem rombu $0$ kącie przy tym wierzcholku równym $60^{\mathrm{o}}$

Wyznaczyč współrzędne pozostałych wierzchołków rombu wiedząc, $\dot{\mathrm{z}}\mathrm{e}$ dwa $\mathrm{z}$ nich lezą

na prostej $l$ : $2x-y+3=0$. Ile rozwiqzań ma to zadanie?
\end{document}
