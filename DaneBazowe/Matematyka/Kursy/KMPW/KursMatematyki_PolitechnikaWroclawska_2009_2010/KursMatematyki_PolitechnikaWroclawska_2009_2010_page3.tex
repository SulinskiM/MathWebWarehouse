\documentclass[a4paper,12pt]{article}
\usepackage{latexsym}
\usepackage{amsmath}
\usepackage{amssymb}
\usepackage{graphicx}
\usepackage{wrapfig}
\pagestyle{plain}
\usepackage{fancybox}
\usepackage{bm}

\begin{document}

PRACA KONTROLNA nr 2- POZIOM ROZSZERZONY

l. Dane są liczby $m=\displaystyle \frac{\left(\begin{array}{l}
6\\
4
\end{array}\right)\left(\begin{array}{l}
8\\
2
\end{array}\right)}{\left(\begin{array}{l}
7\\
3
\end{array}\right)},$

{\it n}$=$ -($\sqrt{}$($\sqrt{}$24)1-64)(3-41.)2-7-25-$\sqrt{}$4-413.

Wyznaczyč sume wszystkich wyrazów nieskończonego ciqgu geometrycznego, którego

pierwszym wyrazem jest $m$, a piątym $n$. Ile wyrazów tego ciągu nalez $\mathrm{y}$ wziqč, by ich

suma przekroczyla 99\% sumy wszystkich wyrazów?

2. Narysowač zbiory: $A=\{(x,y):x^{2}+2x+y^{2}\leq 3\}, B=\{(x,y):|y|\leq\sqrt{3}x+\sqrt{3}\}$

oraz $(A\backslash B)\cup(B\backslash A)$. Wyznaczyč równanie okręgu wpisanego $\mathrm{w}$ figurę $A\cap B.$

3. Liczby: $a_{1}=\log_{(3-2\sqrt{2})^{2}}(\sqrt{2}-1), \displaystyle \alpha_{2}=\frac{1}{2}\log_{\frac{1}{3}}\frac{\sqrt{3}}{6}, a_{3}=3^{\log_{\sqrt{3}^{\frac{\sqrt{6}}{2}}}}, a_{4}=\log_{(\sqrt{2}-1)}(\sqrt{2}+1),$

$a_{5}=(2^{\sqrt{2}+1})^{\sqrt{2}-1}, a_{6}=\log_{3}2$ są wszystkimi pierwiastkami wielomianu $W(x)$, którego

wyraz wolny jest dodatni.

a) Które $\mathrm{z}$ tych pierwiastków są niewymierne? Odpowiedz/uzasadnič.

b) Wyznaczyč dziedzinę funkcji

nych.

$f(x) = \sqrt{W(x)},$

nie wykonując obliczeń przyblizo-

4. Narysowač wykres funkcji $f$ zadanej wzorem $f(x)=$

Posfugując się wykresem $\mathrm{i}$ odpowiednimi obliczeniami rozwiązač nierównośč

$|f(x)-\displaystyle \frac{1}{2}|<\frac{1}{4}$

5. Na prostej $x+2y=5$ wyznaczyč punkty, $\mathrm{z}$ których okrqg $(x-1)^{2}+(y-1)^{2}=1$ jest

widoczny pod kątem $60^{\mathrm{o}}$. Obliczyč pole obszaru ograniczonego lukiem okręgu $\mathrm{i}$ stycznymi

do niego poprowadzonymi $\mathrm{w}$ znalezionych punktach. Sporządzič rysunek.

6. Na dnie naczynia $\mathrm{w}$ ksztalcie walca umieszczono cztery jednakowe metalowe kulki $0$

$\mathrm{m}\mathrm{o}\dot{\mathrm{z}}$ liwie największej objętości. Następnie do naczynia wrzucono jeszcze $\mathrm{j}\mathrm{e}\mathrm{d}\mathrm{n}\Phi$ kulkę $\mathrm{i}$

okazało się, $\dot{\mathrm{z}}\mathrm{e}$ jest ona styczna do płaskiej pokrywy naczynia. Wyznaczyč promienie

kulek wiedząc, $\dot{\mathrm{z}}\mathrm{e}$ przekrój osiowy walca jest kwadratem $0$ boku $d.$
\end{document}
