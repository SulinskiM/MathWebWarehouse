\documentclass[a4paper,12pt]{article}
\usepackage{latexsym}
\usepackage{amsmath}
\usepackage{amssymb}
\usepackage{graphicx}
\usepackage{wrapfig}
\pagestyle{plain}
\usepackage{fancybox}
\usepackage{bm}

\begin{document}

XXXIX

KORESPONDENCYJNY KURS

Z MATEMATYKI

grudzień 2009 r.

PRACA KONTROLNA nr 3 -POZIOM PODSTAWOWY

l. Sześč kostek sześciennych $0$ objętościach 256, 128, 64, 32, 16 $\mathrm{i}8\mathrm{c}\mathrm{m}^{3}$ ustawiono $\mathrm{w}$ pirami-

dę. Czy $\mathrm{m}\mathrm{o}\dot{\mathrm{z}}$ na tę piramidę umieścič na pófce $0$ wysokości 24 cm? Odpowied $\acute{\mathrm{z}}$ uzasadnič

bez wykonywania obliczeń przyblizonych.

2. Wojtuś postawif przypadkowo cztery pionki na szachownicy $016$ polach. Jakiejest praw-

dopodobieństwo, $\dot{\mathrm{z}}\mathrm{e}$ co najwyzej dwa pionki będą staly $\mathrm{w}$ szeregu (poziomo lub pionowo)?

3. Rozwiązač nierównośč

$|\displaystyle \frac{x^{2}+3x+2}{2x^{2}+7x+6}|\leq 1.$

4. Lamana ABCD jest przedstawiona na rysunku ponizej. Niech E będzie punktem prze-

cięcia się prostych AB iCD. Obliczyč pole trójk$\Phi$ta CBE.

5. Obserwator, stojąc $\mathrm{w}$ pewnej odlegfości, widzi wiezę kościofa pod kątem $60^{\mathrm{o}}$ Po odda-

leniu się $050\mathrm{m}$ kąt widzenia zmniejszył się do $45^{\mathrm{o}}$ Obliczyč cosinus kąta, pod jakim

obserwator będzie widział wiezę kościofa, jeśli oddali się $0$ kolejne 50 $\mathrm{m}.$

6. Wycinek koła ma obwód $2s$, gdzie $s > 0$ jest ustaloną liczbą. Wyrazič pole $P$ tego

wycinka jako funkcję promienia $r$ kofa. Sporządzič wykres funkcji $P=P(r).$
\end{document}
