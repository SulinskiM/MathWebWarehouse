\documentclass[a4paper,12pt]{article}
\usepackage{latexsym}
\usepackage{amsmath}
\usepackage{amssymb}
\usepackage{graphicx}
\usepackage{wrapfig}
\pagestyle{plain}
\usepackage{fancybox}
\usepackage{bm}

\begin{document}

PRACA KONTROLNA nr 3 -POZIOM ROZSZERZONY

l. Sporządzič wykres funkcji $f(m) = \displaystyle \frac{1}{x_{1}}+\frac{1}{x_{2}}$, gdzie $x_{1}, x_{2}$ sa pierwiastkami równania

$x^{2}-2mx+m+2=0$, a $m$ jest parametrem rzeczywistym.

2. Ala ulozyła $\mathrm{z}$ czterech klocków liczbę 2009. Nastepnie spośród tych k1ocków 1osowa-

fa ze zwracaniem cztery razy po jednym klocku. Jakie jest prawdopodobieństwo, $\dot{\mathrm{z}}\mathrm{e}\mathrm{z}$

otrzymanych $\mathrm{w}$ ten sposób cyfr $\mathrm{m}\mathrm{o}\dot{\mathrm{z}}$ na byłoby utworzyč liczbe:

a) podzielną przez 3?

b) podzielną przez 4?

3. Rozwazmy funkcje $ f(x)=4^{x+1}+4^{2x+1}+4^{3x+1}+\ldots$ oraz $g(x)=2^{x}+2^{x-1}+2^{x-2}+\ldots,$

gdzie prawe strony wzorów określających obie funkcje są sumami wyrazów nieskończo-

nych ciągów geometrycznych. Wykazač, $\dot{\mathrm{z}}\mathrm{e}$ funkcja $f(x)$ jest rosnąca. Znalez/č wszystkie

liczby $x$, dla których $f(x)=g(x).$

4. Rozwiązač nierównośč

$\displaystyle \frac{\mathrm{t}\mathrm{g}x+\sin x}{3\mathrm{t}\mathrm{g}x-2\sin x}\geq\cos^{2}\frac{x}{2}.$

5. Okrag styczny do ramion paraboli $y = x^{2}-2x$ jest styczny równocześnie do osi $Ox.$

Znalez/č równania stycznych do okręgu $\mathrm{w}$ punktach jego styczności $\mathrm{z}$ parabolą.

6. $\mathrm{Z}$ odcinków $0$ długościach równych czterem najmniejszym nieparzystym liczbom pierw-

szym zbudowano trapez, którego pole jest liczbą wymierną. Wyznaczyč tangens $\mathrm{k}_{\Phi^{\mathrm{t}\mathrm{a}}}$

między przekątnymi tego trapezu.
\end{document}
