\documentclass[a4paper,12pt]{article}
\usepackage{latexsym}
\usepackage{amsmath}
\usepackage{amssymb}
\usepackage{graphicx}
\usepackage{wrapfig}
\pagestyle{plain}
\usepackage{fancybox}
\usepackage{bm}

\begin{document}

XXXIX

KORESPONDENCYJNY KURS

Z MATEMATYKI

styczeń 2010 r.

PRACA KONTROLNA $\mathrm{n}\mathrm{r} 4-$ POZIOM PODSTAWOWY

l. Mamy dwa termosy kawy $\mathrm{z}$ mlekiem. $\mathrm{W}$ pierwszym termosie stosunek objętości mleka

do objętości kawy wynosi 2:3, a $\mathrm{w}$ drugim 3:7. I1e 1itrów p1ynu na1ez $\mathrm{y}$ wziąč $\mathrm{z}\mathrm{k}\mathrm{a}\dot{\mathrm{z}}$ dego

termosu, aby otrzymač 2,41itra kawy $\mathrm{z}$ mlekiem, $\mathrm{w}$ której objętośč kawy bedzie dwa

razy większa $\mathrm{n}\mathrm{i}\dot{\mathrm{z}}$ objętośč mleka?

2. Kwotę l00000 zf wpfacono na lokatę roczną, $\mathrm{w}$ której odsetki doliczane są co kwartaf. Po

roku suma odsetek wyniosła dokładnie 4060,401 $\mathrm{z}l$. Znalez/č oprocentowanie tej lokaty.

Jakie powinno byč oprocentowanie lokaty, aby przy kapitalizacji dokonywanej raz na póf

roku osiągnąč ten sam zysk?

3. Dane są zbiory $A=\{(x,y):x,y,\in \mathbb{R},y^{2}-4x^{2}\geq 0\}\mathrm{i}B=\{(x,y):|x|+|y|\leq 2\}$. Nary-

sowač zbiór $A\cup B$. Znalez$\acute{}$č punkt ze zbioru $A\cup B$ pofozony najblizej puntu $C=(3,2).$

4. Narysowač wykres trójmianu kwadratowego $f(x) = x^{2}+4x-5$ oraz wykres funkcji

$g(x)=4-f(x-2).$

a) Rozwiązač nierównośč $f(x)>g(x).$

b) Znalez/č obraz wykresu funkcji $f(x) \mathrm{w}$ symetrii względem prostej $x=2 \mathrm{i}$ na tej

podstawie podač wzór tej funkcji.

5. $\mathrm{W}$ ostrosfupie prawidłowym trójk$\Phi$tnym $0$ krawędzi podstawy równej $a$ kąt pfaski ściany

bocznej przy wierzchołku jest równy $ 2\alpha$. Obliczyč objętośč tego ostrosłupa oraz sinus

kąta nachylenia ściany bocznej do podstawy.

6. $\mathrm{W}$ trapezie ABCD, $\mathrm{w}$ którym bok $AB$ jest równolegfy do boku $DC$, dane są: $\angle BAD=$

$\displaystyle \frac{\pi}{3}, |AB| =20, |DC| =8$ oraz $|AD| =5$. Obliczyč obwód tego trapezu, $\sin\angle ADB$ oraz

odlegfośč punktu przecięcia się przekątnych tego trapezu od jego podstaw.
\end{document}
