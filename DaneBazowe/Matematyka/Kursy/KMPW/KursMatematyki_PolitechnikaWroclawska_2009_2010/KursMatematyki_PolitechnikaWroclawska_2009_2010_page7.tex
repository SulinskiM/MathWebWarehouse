\documentclass[a4paper,12pt]{article}
\usepackage{latexsym}
\usepackage{amsmath}
\usepackage{amssymb}
\usepackage{graphicx}
\usepackage{wrapfig}
\pagestyle{plain}
\usepackage{fancybox}
\usepackage{bm}

\begin{document}

PRACA KONTROLNA nr 4- POZIOM ROZSZERZONY

l. Dzieląc wielomian $W(x)$ przez dwumian $x-3$ otrzymujemy resztę równą 2, a dzie1ąc

ten wielomian przez $x-2$ otrzymujemy resztę równą l. Wyznaczyč resztę $\mathrm{z}$ dzielenia

$W(x)$ przez $(x-2)(x-3)$. Znalez/č wielomian trzeciego stopnia spełniający powyzsze

warunki wiedzqc, $\dot{\mathrm{z}}\mathrm{e}x=1$ jest pierwiastkiem tego wielomianu, a suma wyrazu wolnego

$\mathrm{i}$ wspólczynnika przy $x^{3}$ jest równa 0.

2. Znalez/č najmniejszą $\mathrm{i}$ największą wartośč funkcji $f(x)=\displaystyle \sin x-\frac{1}{2}\cos 2x$ na przedziale

$[-\displaystyle \frac{\pi}{2},\frac{\pi}{2}] \mathrm{i}$ rozwiązač nierównośč - $\displaystyle \frac{1}{2}\leq f(x)\leq\frac{1}{4}$. Zadanie rozwiązač bez $\mathrm{u}\dot{\mathrm{z}}$ ywania pojęcia

pochodnej.

3. Rozwiązač nierównośč

$\log_{\frac{1}{\sqrt{2}}}(2^{2x+1}-16^{x})\geq-12x.$

4. $\mathrm{W}$ stozek $\mathrm{o}\mathrm{k}_{\Phi}\mathrm{c}\mathrm{i}\mathrm{e}$ rozwarcia równym $ 2\alpha$ wpisano kulę $0$ promieniu $R$. Wewnątrz stozka

stawiamy na kuli sześcian $0$ maksymalnej objętości $\mathrm{i}$ podstawie równoleglej do podstawy

stozka. Wyznaczyč dlugośč krawędzi tego sześcianu.

5. Stosunek dlugości promienia okręgu wpisanego do dfugości promienia okręgu opisanego

na trójkącie prostokqtnym wynosi $\displaystyle \frac{1}{3+2\sqrt{3}}$. Obliczyč sinusy kątów ostrych tego trójkąta.

6. Ślimak ma do przejścia taśmę $0$ długości 3 metrów zamocowaną $\mathrm{w}$ punkcie startu A. $\mathrm{W}$

ciągu $\mathrm{k}\mathrm{a}\dot{\mathrm{z}}$ dego dnia udaje mu się przejśč l metr, a $\mathrm{k}\mathrm{a}\dot{\mathrm{z}}$ dej nocy gdy śpi, ktoś- ciągnąc

za drugi koniec taśmy- wydłuza $\mathrm{j}\mathrm{a}$ równomiernie $0 1$ metr. Niech $d_{n}$ oznacza długośč

taśmy $\mathrm{w}n$-tym dniu, a $a_{n}$- odległośč ślimaka od punktu A przy końcu $n$-tego dnia.

a) Uzasadnič, $\dot{\mathrm{z}}\mathrm{e}$ ciąg $(a_{n})$ zdefiniowany jest następującym wzorem rekurencyjnym:

$a_{1}=1$ oraz $a_{n+1}=\displaystyle \frac{3+n}{2+n}a_{n}+1$ dla $n\geq 1.$

b) Pokazač, $\displaystyle \dot{\mathrm{z}}\mathrm{e}a_{n}=(n+2)(\frac{1}{3}+\frac{1}{4}+\ldots+\frac{1}{n+2}), n\geq 1.$

c) Czy ślimak dojdzie do końca taśmy? $\mathrm{J}\mathrm{e}\dot{\mathrm{z}}$ eli tak, to $\mathrm{w}$ którym dniu, to znaczy, dla

jakich $n$ prawdziwa jest nierównośč $a_{n}>d_{n}$?
\end{document}
