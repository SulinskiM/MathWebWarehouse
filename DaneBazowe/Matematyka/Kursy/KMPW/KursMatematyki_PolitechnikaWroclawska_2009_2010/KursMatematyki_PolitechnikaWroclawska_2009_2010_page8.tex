\documentclass[a4paper,12pt]{article}
\usepackage{latexsym}
\usepackage{amsmath}
\usepackage{amssymb}
\usepackage{graphicx}
\usepackage{wrapfig}
\pagestyle{plain}
\usepackage{fancybox}
\usepackage{bm}

\begin{document}

XXXIX

KORESPONDENCYJNY KURS

Z MATEMATYKI

luty 2010 r.

PRACA KONTROLNA $\mathrm{n}\mathrm{r} 5-$ POZIOM PODSTAWOWY

l. Dwie wiewiórki, Kasiai Basia, postanowiły wspólnie zbierač orzechy. $\mathrm{K}\mathrm{a}\dot{\mathrm{z}}$ dego dnia Basia

przynosifa do wspólnej spizarni $04$ orzechy więcej $\mathrm{n}\mathrm{i}\dot{\mathrm{z}}$ Kasia, codziennie tyle samo. Po

30 dniach współpracy wiewiórki pokłóciły się. Basia zostawiła Kasi wszystkie orzechy

$\mathrm{i}$ zafozyła wlasną spizarnię. Od tamtej pory $\mathrm{k}\mathrm{a}\dot{\mathrm{z}}$ da $\mathrm{z}$ wiewiórek przynosi do swojej spizarni

tę samą ilośč orzechów co przedtem, ale Basia codziennie dostaje 6 orzechów od Kasi. Po

50 dniach samodzielnej pracy Kasia ma jeszcze $0100$ orzechów więcej $\mathrm{n}\mathrm{i}\dot{\mathrm{z}}$ Basia. Ustalič,

po ile orzechów zbiera codziennie $\mathrm{k}\mathrm{a}\dot{\mathrm{z}}$ da $\mathrm{z}$ wiewiórek $\mathrm{i}$ oszacowač, po ilu dniach $\mathrm{w}$ spizarni

Basi będzie więcej orzechów $\mathrm{n}\mathrm{i}\dot{\mathrm{z}}\mathrm{u}$ kolezanki.

2. Określič dziedzinę $\mathrm{i}$ zbiór wartości funkcji $f(x)=\sin x\cdot\sin 2x$. (tg $x+$ ctg $x$). Wykonač

staranny wykres funkcji $g(x)=f(x-\displaystyle \frac{\pi}{4})+1\mathrm{i}$ rozwiązač równanie $g(x)=0$. Posfugując

się sporzqdzonym wykresem określič zbiór rozwiązań nierówności $g(x)\geq 0.$

3. Wyznaczyč równania wszystkich prostych, które są styczne jednocześnie do obu okręgów

$(x-1)^{2}+(y-1)^{2}=1$

oraz

$(x-5)^{2}+(y-1)^{2}=1.$

Obliczenia zilustrowač odpowiednim rysunkiem.

4. Rozwiązač nierównośč

$\displaystyle \frac{3\sqrt{4-x}+1}{1-\sqrt{4-x}}>1-2\sqrt{4-x}.$

5. Koszt budowy I kondygnacji biurowca wynosi l0 mln zł., a $\mathrm{k}\mathrm{a}\dot{\mathrm{z}}$ dej kolejnej jest $\mathrm{n}\mathrm{i}\dot{\mathrm{z}}$ szy

$0100\mathrm{t}\mathrm{y}\mathrm{s}. \mathrm{z}\mathrm{f}$. od poprzedniej. Planowany koszt wynajmu powierzchni biurowych $\mathrm{w}$ tym

budynku jest stały do XL kondygnacji $\mathrm{i}$ wynosi 200 $\mathrm{t}\mathrm{y}\mathrm{s}$. zł. za całą kondygnację, $\mathrm{a}$

potem podwaja się co 5 kondygnacji (na ko1ejnych 5 kondygnacjach jest stały). Roczny

koszt wynajmu ostatniej, najbardziej prestizowej $\mathrm{i}$ drozszej od pozostafych kondygnacji

jest równy kosztowi budowy całego XXXVII piętra. Oszacowač, po ilu latach zwróci się

inwestorom koszt budowy tego budynku.

6. $\mathrm{W}$ trapezie równoramiennym kąt przy podstawie ma miarę $\displaystyle \frac{\pi}{3}$, a róznica długości podstaw

wynosi 4. Usta1ič, i1e powinno wynosič po1e tego trapezu, aby $\mathrm{m}\mathrm{o}\dot{\mathrm{z}}$ na było wpisač $\mathrm{w}$ niego

kofo. $\mathrm{W}$ tym przypadku wyznaczyč stosunek pola kofa opisanego na tym trapezie do pola

koła wpisanego.
\end{document}
