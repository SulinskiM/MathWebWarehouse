\documentclass[a4paper,12pt]{article}
\usepackage{latexsym}
\usepackage{amsmath}
\usepackage{amssymb}
\usepackage{graphicx}
\usepackage{wrapfig}
\pagestyle{plain}
\usepackage{fancybox}
\usepackage{bm}

\begin{document}

PRACA KONTROLNA nr 5- POZIOM ROZSZERZONY

l. Znalez$\acute{}$č wszystkie liczby rzeczywiste m, dla których równanie

$\displaystyle \frac{x}{m}+m=\frac{m}{x}+x+1$

ma dwa pierwiastki róznych znaków.

2. Rozwiązač nierównośč

$2^{x^{2}+4}+2^{x^{2}+3}+2^{x^{2}}>5^{x^{2}+1}-25\cdot 2^{x^{2}-2}$

3. Określič dziedzinę $\mathrm{i}$ zbiór wartości funkcji $f(x)=\displaystyle \mathrm{c}\mathrm{t}\mathrm{g}(\pi+x)\mathrm{c}\mathrm{t}\mathrm{g}(x-\frac{\pi}{2})\cos x$. Sporz$\Phi$dzič

staranny wykres funkcji $g(x) =2f(|x-\displaystyle \frac{\pi}{4}|)-1$. Na podstawie wykresu $\mathrm{i}$ niezbędnych

obliczeń rozwiązač nierównośč $g(x)\leq-2$, a zbiór jej rozwiązań zaznaczyč na osi OX.

4. Rozwiązač nierównośč

$\displaystyle \log_{x^{2}}(3x-1)-\log_{x^{2}}(x-1)^{2}+\log_{x^{2}}|x-1|\geq\frac{1}{2}.$

5. $\mathrm{W}$ ostroslupie sześciokątnym prawidfowym kąt dwuścienny utworzony przez pfaszczyzny

przeciwległych ścian bocznych ma miarę $\displaystyle \frac{\pi}{4}$. Wyznaczyč promień $R$ kuli opisanej na tym

ostroslupie jako funkcję dfugości $a$ boku jego podstawy.

6. $\mathrm{W}$ kolo wpisano ośmiokąt foremny, $\mathrm{w}$ ośmiokąt kofo, $\mathrm{w}$ kofo kolejny ośmiokąt foremny

itd. Wysunač hipotezę $0$ wartości pola $n$-tego koła $\mathrm{i}$ uzasadnič $\mathrm{j}\mathrm{a}$ indukcyjnie. Suma pól

nieskończonego $\mathrm{c}\mathrm{i}_{\Phi \mathrm{g}}\mathrm{u}$ kól otrzymanych $\mathrm{w}$ ten sposób jest ośmiokrotnością pola jednego

$\mathrm{z}$ nich. Ustalič którego, nie stosując obliczeń przyblizonych.
\end{document}
