\documentclass[a4paper,12pt]{article}
\usepackage{latexsym}
\usepackage{amsmath}
\usepackage{amssymb}
\usepackage{graphicx}
\usepackage{wrapfig}
\pagestyle{plain}
\usepackage{fancybox}
\usepackage{bm}

\begin{document}

XL

KORESPONDENCYJNY KURS

Z MATEMATYKI

wrzesień 2010 r.

PRACA KONTROLNA $\mathrm{n}\mathrm{r} 1-$ POZIOM PODSTAWOWY

l. Ile jest trzycyfrowych liczb naturalnych:

a) podzielnych przez 31ub przez 5?

b) podzielnych przez 31ub przez 6?

c) podzielnych przez 3 $\mathrm{i}$ niepodzielnych przez 5?

2. Renomowany dom mody sprzedaf 40\% kolekcji letniej po zafozonej cenie. Po obnizce

ceny $0$ 50\% udało $\mathrm{s}\mathrm{i}\mathrm{e}$ sprzedač połowę pozostałej części towaru $\mathrm{i}$ dopiero kolejna 50\%-

owa obnizka pozwolifa opróznič magazyny. Ile procent zaplanowanego przychodu stanowi

uzyskana ze sprzedaz $\mathrm{y}$ kwota? $\mathrm{O}$ ile procent wyjściowa cena towaru powinna była byč

$\mathrm{w}\mathrm{y}\dot{\mathrm{z}}$ sza, by sklep uzyskał zaplanowany początkowo przychód?

3. Określič dziedzinę wyrazenia $w(x,y)=\displaystyle \frac{2}{x-y}-\frac{3xy}{x^{3}-y^{3}}-\frac{x-y}{x^{2}+xy+y^{2}}.$

Sprowadzič je do najprostszej postaci $\mathrm{i}$ obliczyč $w(1+\sqrt{2},(1+\sqrt{2})^{-1})$

4. Obliczyč sumę wszystkich liczb pierwszych spelniających nierównośč

$(p-4)x^{2}-4(p-2)x-p\leq 0$, gdzie $p=\displaystyle \frac{64^{\frac{1}{3}}\sqrt{8}+8^{\frac{1}{3}}\sqrt{64}}{\sqrt[3]{64\sqrt{8}}}$

5. Dwa naczynia zawierają $\mathrm{w}$ sumie 401itrów wody. Po prze1aniu pewnej części wody pierw-

szego naczynia do drugiego, $\mathrm{w}$ pierwszym naczyniu zostało trzy razy mniej wody $\mathrm{n}\mathrm{i}\dot{\mathrm{z}}\mathrm{w}$

drugim. Gdy następnie przelano taką samą częśč wody drugiego naczynia do pierwszego,

okazało się, $\dot{\mathrm{z}}\mathrm{e}\mathrm{w}$ obu naczyniach jest tyle samo płynu. Obliczyč, ile wody było pierwotnie

$\mathrm{w}\mathrm{k}\mathrm{a}\dot{\mathrm{z}}$ dym naczyniu $\mathrm{i}\mathrm{j}\mathrm{a}\mathrm{k}_{\Phi}$ jej częśč przelewano.

6. Dwie $\mathrm{g}\mathrm{a}\acute{\mathrm{z}}$dziny, pracując razem, mogą wykonač zamówioną partię pisanek $\mathrm{w}$ ciągu 7

dni pod warunkiem, $\dot{\mathrm{z}}\mathrm{e}$ pierwsza $\mathrm{z}$ nich rozpocznie pracę $0$ póltora dnia wcześniej $\mathrm{n}\mathrm{i}\dot{\mathrm{z}}$

druga. Gdyby $\mathrm{k}\mathrm{a}\dot{\mathrm{z}}\mathrm{d}\mathrm{a}\mathrm{z}$ nich pracowała oddzielnie, to druga wykonalaby cafą pracę $03$

$\mathrm{d}\mathrm{n}\mathrm{i}$ wcześniej od pierwszej. Ile $\mathrm{d}\mathrm{n}\mathrm{i}$ potrzebuje $\mathrm{k}\mathrm{a}\dot{\mathrm{z}}$ da $\mathrm{z}$ kobiet na wykonanie calej pracy?




PRACA KONTROLNA nr l- POZIOM ROZSZERZONY

l. Ile jest liczb pięciocyfrowych podzielnych przez 9, które $\mathrm{w}$ rozwinięciu dziesiętnym mają:

a) obie cyfry 1, 2 $\mathrm{i}$ tylko $\mathrm{t}\mathrm{e}$? b) obie cyfry 1, 3 $\mathrm{i}$ tylko $\mathrm{t}\mathrm{e}$? c) wszystkie cyfry 1, 2, 3

$\mathrm{i}$ tylko $\mathrm{t}\mathrm{e}$? Odpowiedz/uzasadnič. $\mathrm{W}$ przypadku b) wypisač otrzymane liczby.

2. Pan Kowalski zaciqgnąl 3l grudnia $\mathrm{p}\mathrm{o}\dot{\mathrm{z}}$ yczkę 4000 złotych oprocentowaną $\mathrm{w}$ wysoko-

ści 18\% $\mathrm{w}$ skali roku. Zobowiązał się splacič $\mathrm{j}_{\Phi}\mathrm{w}\mathrm{c}\mathrm{i}_{\Phi \mathrm{g}}\mathrm{u}$ roku $\mathrm{w}$ trzech równych ratach

platnych 30 kwietnia, 30 sierpnia $\mathrm{i}30$ grudnia. Oprocentowanie $\mathrm{p}\mathrm{o}\dot{\mathrm{z}}$ yczki liczy się od

l stycznia, a odsetki od kredytu naliczane są $\mathrm{w}$ terminach płatności rat. Obliczyč wyso-

kośč tych rat $\mathrm{w}$ zaokrągleniu do pełnych groszy.

3. Określič dziedzinę wyrazenia

$w(x,y)=\displaystyle \frac{x}{x^{3}+x^{2}y+xy^{2}+y^{3}}+\frac{y}{x^{3}-x^{2}y+xy^{2}-y^{3}}+\frac{1}{x^{2}-y^{2}}-\frac{1}{x^{2}+y^{2}}-\frac{x^{2}+2y^{2}}{x^{4}-y^{4}}.$

Sprowadzič je do najprostszej postaci $\mathrm{i}$ obliczyč $w(\cos 15^{\mathrm{o}},\sin 15^{\mathrm{o}}).$

4. Liczba $p = \displaystyle \frac{(\sqrt[3]{54}-2)(9\sqrt[3]{4}+6\sqrt[3]{2}+4)-(2-\sqrt{3})^{3}}{\sqrt{3}+(1+\sqrt{3})^{2}}$ jest miejscem zerowym funkcji

kwadratowej $f(x)=ax^{2}+bx+c$. Wyznaczyč współczynniki $a, b, c$ oraz drugie miejsce

zerowe tej funkcji wiedząc, $\dot{\mathrm{z}}\mathrm{e}$ największ$\Phi$ wartością funkcji jest 4, a jej wykres jest

symetryczny wzgledem prostej $x=1.$

5. Do zbiornika poprowadzono trzy rury. Pierwsza rura potrzebuje do napefnienia zbiornika

$04$ godziny więcej $\mathrm{n}\mathrm{i}\dot{\mathrm{z}}$ druga, a trzecia napełnia caly zbiornik $\mathrm{w}$ czasie dwa razy krótszym

$\mathrm{n}\mathrm{i}\dot{\mathrm{z}}$ pierwsza. Wjakim czasie napelnia zbiornik $\mathrm{k}\mathrm{a}\dot{\mathrm{z}}$ da $\mathrm{z}\mathrm{r}\mathrm{u}\mathrm{r}, \mathrm{j}\mathrm{e}\dot{\mathrm{z}}$ eli wiadomo, $\dot{\mathrm{z}}\mathrm{e}$ wszystkie

trzy rury otwarte jednocześnie napefniają zbiornik $\mathrm{w}$ ciągu 2 godzin $\mathrm{i}40$ minut?

6. $\mathrm{Z}$ przystani A wyrusza $\mathrm{z}$ biegiem rzeki statek do przystani $\mathrm{B}$, odległej od A $0140$ km. Po

upływie l godziny wyrusza za nim łódz/ motorowa, dopędza statek, po czym wraca do

przystani A $\mathrm{w}$ tym samym momencie, $\mathrm{w}$ którym statek przybija do przystani B. Znalez/č

prędkośč biegu rzeki, $\mathrm{j}\mathrm{e}\dot{\mathrm{z}}$ eli wiadomo, $\dot{\mathrm{z}}\mathrm{e}\mathrm{w}$ stojącej wodzie prędkośč statku wynosi 16

$\mathrm{k}\mathrm{m}/$godz, a prędkośč łodzi 24 $\mathrm{k}\mathrm{m}/$godz.





XL

KORESPONDENCYJNY KURS

Z MATEMATYKI

luty 2011 r.

PRACA KONTROLNA $\mathrm{n}\mathrm{r} 6-$ POZIOM PODSTAWOWY

l. Losujemy liczbę ze zbioru \{1, 2, 3, $\ldots$, 100\}, a następnie liczbę ze zbioru \{2, 3, 4, 5\}. Obli-

czyč prawdopodobieństwo, $\dot{\mathrm{z}}\mathrm{e}$ pierwsza $\mathrm{z}$ wylosowanych liczb jest podzielna przez $\mathrm{d}\mathrm{r}\mathrm{u}\mathrm{g}\Phi.$

2. Liczba 2-elementowych podzbiorów zbioru $A$ jest 7 razy większa $\mathrm{n}\mathrm{i}\dot{\mathrm{z}}$ liczba 2-e1emento-

wych podzbiorów zbioru $B$. Liczba 2-e1ementowych podzbiorów zbioru $A$ nie zawierają-

cych ustalonego elementu $a\in A$ jest 5 razy większa $\mathrm{n}\mathrm{i}\dot{\mathrm{z}}$ liczba 2-e1ementowych podzbiorów

zbioru $B$. Ile elementów ma $\mathrm{k}\mathrm{a}\dot{\mathrm{z}}\mathrm{d}\mathrm{y}\mathrm{z}$ tych zbiorów? Ile $\mathrm{k}\mathrm{a}\dot{\mathrm{z}}\mathrm{d}\mathrm{y}\mathrm{z}$ tych zbiorów ma podzbio-

rów 3-e1ementowych?

3. $\mathrm{W}$ turnieju szachowym $\mathrm{k}\mathrm{a}\dot{\mathrm{z}}\mathrm{d}\mathrm{y}$ uczestnik miał rozegrač $\mathrm{z}$ pozostałymi po jednej partii. Po

rozegraniu trzech partii dwóch szachistów zrezygnowalo $\mathrm{z}$ dalszej $\mathrm{g}\mathrm{r}\mathrm{y}. \mathrm{W}$ sumie rozegra-

no 84 partie. I1u by1o uczestników na początku turnieju, $\mathrm{j}\mathrm{e}\dot{\mathrm{z}}$ eli dwaj zawodnicy, którzy

zrezygnowali, nie grali ze sobq?

4. Suma pierwszego $\mathrm{i}$ trzeciego wyrazu $\mathrm{c}\mathrm{i}_{\Phi \mathrm{g}}\mathrm{u}$ geometrycznego $(a_{n})$ wynosi 20. Znajd $\acute{\mathrm{z}}$ wzór

ogólny ciągu arytmetycznego $(b_{n})$ takiego, $\dot{\mathrm{z}}\mathrm{e}b_{1}=a_{1}, b_{2}=a_{2}, b_{5}=a_{3}.$

5. Rozkład ocen ze sprawdzianu $\mathrm{w}$ klasie IIIa jest opisany tabelką
\begin{center}
\begin{tabular}{l|l|l|l|l|l}
\multicolumn{1}{l|}{ocena}&	\multicolumn{1}{|l|}{$1$}&	\multicolumn{1}{|l|}{ $2$}&	\multicolumn{1}{|l|}{ $3$}&	\multicolumn{1}{|l|}{ $4$}&	\multicolumn{1}{|l}{ $5$}	\\
\hline
\multicolumn{1}{l|}{liczba osób}&	\multicolumn{1}{|l|}{$1$}&	\multicolumn{1}{|l|}{ $2$}&	\multicolumn{1}{|l|}{ $8$}&	\multicolumn{1}{|l|}{ $9$}&	\multicolumn{1}{|l}{ $6$}
\end{tabular}

\end{center}
Jaś otrzymał ocenę 4. Czy wypadł powyzej średniej $\mathrm{w}$ swojej klasie? $\mathrm{W}$ pozostałych kla-

sach średnie punktów wynosily: 3,875 $\mathrm{w}$ IIIb (24 osoby) $\mathrm{i}4,6\mathrm{w}$ IIIc (25 osób). Czy ocena

otrzymana przez Jasia znajduje się powyzej średniej liczonej łącznie wśród wszystkich

uczniów klas trzecich? Ile co najmniej, a ile co najwyzej, osób miało piqtki $\mathrm{w}$ klasie IIIc

(skala ocen to $1,2,\ldots,5$)?

6. Ile liczb czterocyfrowych $0$ wszystkich cyfrach róznych $\mathrm{m}\mathrm{o}\dot{\mathrm{z}}$ na utworzyč $\mathrm{z}$ cyfr 1,2,3,4,5, $\mathrm{a}$

ile $\mathrm{z}$ cyfr 0,1,2,3,4,5,6? $\mathrm{W}$ obu przypadkach obliczyč, ile $\mathrm{m}\mathrm{o}\dot{\mathrm{z}}$ na utworzyč czterocyfrowych

liczb podzielnych przez 5.





PRACA KONTROLNA nr 6- POZIOM ROZSZERZONY

l. Trzeci skladnik rozwinięcia dwumianu $(\displaystyle \sqrt[3]{x}+\frac{1}{\sqrt{x}})^{n}$ ma wspólczynnik równy 45. Wyzna-

czyč wszystkie składniki tego rozwinięcia, $\mathrm{w}$ których $x$ wystepuje $\mathrm{w}$ potędze $0$ wykfadniku

cafkowitym.

2. $\mathrm{W}$ turnieju szachowym rozgrywanym systemem,,kazdy $\mathrm{z}\mathrm{k}\mathrm{a}\dot{\mathrm{z}}$ dym'' dwóch uczestników

nie ukończyfo turnieju, przy czym jeden $\mathrm{z}$ nich rozegra110 partii, a drugi ty1ko jedną. I1u

było zawodników $\mathrm{i}$ czy wspomniani zawodnicy grali ze sobą, $\mathrm{j}\mathrm{e}\dot{\mathrm{z}}$ eli rozegrano 55 partii?

3. $\mathrm{W}$ pudełku jest 400 ku1 $\mathrm{w}$ tym $n$ czerwonych. Wybieramy losowo dwie kule. Prawdopo-

dobieństwo wylosowania dwóch kul czerwonych jest równe $\displaystyle \frac{1}{760}.$

a) Ile kul czerwonych jest $\mathrm{w}$ tym pudełku?

b) Obliczyč prawdopodobieństwo, $\dot{\mathrm{z}}\mathrm{e}\dot{\mathrm{z}}$ adna $\mathrm{z}$ wylosowanych kul nie jest czerwona.

4. Suma wyrazów nieskończonego ciagu geometrycznego zmniejszy $\mathrm{s}\mathrm{i}\mathrm{e}0$ 25\%, $\mathrm{j}\mathrm{e}\dot{\mathrm{z}}$ eli wy-

kreślimy $\mathrm{z}$ niej skfadniki $0$ numerach parzystych niepodzielnych przez 4. Ob1iczyč sumę

wszystkich wyrazów tego ciqgu wiedząc, $\dot{\mathrm{z}}\mathrm{e}$ jego drugi wyraz wynosi l.

5. Stosując zasadę indukcji matematycznej udowodnič prawdziwośč wzoru

$\left(\begin{array}{l}
2\\
2
\end{array}\right) - \left(\begin{array}{l}
3\\
2
\end{array}\right) + \left(\begin{array}{l}
4\\
2
\end{array}\right) - \left(\begin{array}{l}
5\\
2
\end{array}\right) +\ldots+\left(\begin{array}{l}
2n\\
2
\end{array}\right) =n^{2},$

$n\geq 1.$

6. Wśród wszystkich $\mathrm{b}\mathrm{l}\mathrm{i}\acute{\mathrm{z}}$niąt 64\% stanowią bliz$\acute{}$nięta tej samej plci. Prawdopodobieństwo

urodzenia chłopca wynosi 0,51. Ob1iczyč prawdopodobieństwo, $\dot{\mathrm{z}}\mathrm{e}$ drugie $\mathrm{z}\mathrm{b}\mathrm{l}\mathrm{i}\acute{\mathrm{z}}$niąt jest

$\mathrm{d}\mathrm{z}\mathrm{i}\mathrm{e}\mathrm{w}\mathrm{c}\mathrm{z}\mathrm{y}\mathrm{n}\mathrm{k}_{\Phi}$, pod warunkiem, $\dot{\mathrm{z}}\mathrm{e}$:

a) pierwsze jest dziewczynką,

b) pierwsze jest chlopcem.





XL

KORESPONDENCYJNY KURS

Z MATEMATYKI

marzec 2011 r.

PRACA KONTROLNA nr 7- POZIOM PODSTAWOWY

l. Rozwiązač równanie

$1-|x|=\sqrt{1+x} \mathrm{i}$ podač jego ilustrację graficzną.

2. Wyznaczyč wszystkie punkty $x\mathrm{z}$ przedziafu $[0,2\pi]$, dla których spefnionajest nierównośč

$\sin 2x-\mathrm{t}\mathrm{g}x\leq 0$. Podač ilustrację graficzną nierówności.

3. Określič liczbę rozwiązań ukfadu równań

$\left\{\begin{array}{l}
y\\
y
\end{array}\right.$

$|x-2|+1,$

$ax$

$\mathrm{w}$ zalezności od wartości współczynnika kierunkowego prostej $y=ax$. Znalez$\acute{}$č rozwiąza-

nia $\mathrm{w}$ przypadku, gdy jednym $\mathrm{z}$ nich jest para (4, 3). Sporzadzič staranny rysunek.

4. Danajest prosta $l:x+2y-4=0$. Przez punkt (l, l) poprowadzič prostą $k\mathrm{o}$ dodatnim

współczynniku kierunkowym $\mathrm{t}\mathrm{a}\mathrm{k}$, aby pole trójkqta ograniczonego prostymi $l,  k\mathrm{i}\mathrm{o}\mathrm{s}\mathrm{i}\otimes$

$0x$ byfo dwa razy większe $\mathrm{n}\mathrm{i}\dot{\mathrm{z}}$ pole trójk$\Phi$ta ograniczonego tymi prostymi $\mathrm{i}\mathrm{o}\mathrm{s}\mathrm{i}_{\Phi}0y.$

5. Trójkąt równoboczny $ABC0$ boku długości $a$ zgięto wzdluz wysokości $CD$ pod pew-

nym kątem, otrzymujqc $\mathrm{w}$ ten sposób czworościan ABCD. Obliczyč objętośč $\mathrm{i}$ pole

powierzchni calkowitej tego czworościanu wiedząc, $\dot{\mathrm{z}}\mathrm{e}$ tangens kąta nachylenia ściany

$ABC$ do podstawy czworościanu równy jest $\sqrt{6}.$

6. Punkt $(0,2)$ jest środkiem symetrii wykresu funkcji

$\mathrm{i}b$ wiedząc, $\dot{\mathrm{z}}\mathrm{e} f(a)=0.$

$f(x)=x(|x|-2a)+b$. Wyznaczyč $a$





PRACA KONTROLNA nr 7- POZIOM ROZSZERZONY

l. Rozwiązač równanie

$\sqrt{8+2x-x^{2}}=2x-5.$

Zilustrowač je odpowiednim wykresem.

2. Wyznaczyč wszystkie wartości parametru rzeczywistego $p$, dla których rozwiqzania

ukfadu równań

$\left\{\begin{array}{l}
px+2y=p- 2\\
2x+py=p-1
\end{array}\right.$

są zawarte $\mathrm{w}$ kwadracie $K=\{(x,y):|x|+|y|\leq 1\}.$

3. Bok $AB$ trójkąta równoramiennego $ABC\mathrm{l}\mathrm{e}\dot{\mathrm{z}}\mathrm{y}$ na prostej $l$ : $x-3y-4 = 0$. Punkt

$D(4,0)$ jest spodkiem wysokości tego trójkąta, a $S(2,1)$ środkiem boku $AC$. Wyznaczyč

wspólrzędne wierzcholka $B$. Sporządzič rysunek.

4. Podstawą ostrosfupa $0$ wysokości $h$ jest trójk$\Phi$t $\mathrm{P}^{\mathrm{r}\mathrm{o}\mathrm{s}\mathrm{t}\mathrm{o}\mathrm{k}}\Phi^{\mathrm{t}\mathrm{n}\mathrm{y}\mathrm{o}}$ kącie ostrym $\alpha$. Wszystkie

ściany boczne ostrosłupa są nachylone do podstawy pod kątem $\alpha$, a pole powierzchni

całkowitej jest czterokrotnie większe od pola podstawy. Obliczyč objętośč ostroslupa.

Wynik podač $\mathrm{w}$ najprostszej postaci.

5. Rozwiązač nierównośč

sin2 {\it x}$+$ -csoins42 {\it xx}$+$ -csoins64 {\it xx}$+$ -csoins86 {\it xx}$+$... $\geq$ -83,

$\mathrm{w}$ której lewa strona jest sumą nieskończonego ciągu geometrycznego.

6. Jednym $\mathrm{z}$ pierwiastków wielomianu $w(x)=ax^{3}+bx^{2}+cx+d$ jest liczba $-1$. Znalez$\acute{}$č

pozostafe pierwiastki wiedząc, $\dot{\mathrm{z}}\mathrm{e}w(1)=-2\mathrm{i}$ środkiem symetrii wykresu funkcji $w(x)$

jest punkt $S(\displaystyle \frac{1}{4},\frac{5}{2})$. Nie prowadząc dodatkowego badania, sporządzič wykres funkcji

$w(x)$. Dobrač odpowiednio jednostki na osiach ukfadu.





XL

KORESPONDENCYJNY KURS

Z MATEMATYKI

kwiecień 2011 r.

PRACA KONTROLNA $\mathrm{n}\mathrm{r} 8-$ POZIOM PODSTAWOWY

l. Uprościč wyrazenie

$a(x)= (\displaystyle \frac{x+1}{x-2}-\frac{x^{3}+8}{x^{3}-8}\frac{x^{2}+2x+4}{x^{2}-4})$ : $\displaystyle \frac{1}{x-2}$

$\mathrm{i}$ rozwiązač nierównośč $|\alpha(x)|<1.$

2. Trzech robotników ma wykonač $\mathrm{P}^{\mathrm{e}\mathrm{w}\mathrm{n}}\Phi$ pracę. Wiadomo, $\dot{\mathrm{z}}\mathrm{e}$ pierwszy $\mathrm{i}$ drugi robotnik,

pracując razem, wykonaliby calą pracę $\mathrm{w}$ czasie $n\mathrm{d}\mathrm{n}\mathrm{i}$, drugi $\mathrm{i}$ trzeci-w czasie $m\mathrm{d}\mathrm{n}\mathrm{i}, \mathrm{a}$

pierwszy $\mathrm{i}$ trzeci-w czasie $k\mathrm{d}\mathrm{n}\mathrm{i}$. Ile dni potrzebuje $\mathrm{k}\mathrm{a}\dot{\mathrm{z}}\mathrm{d}\mathrm{y}\mathrm{z}$ robotników na samodzielne

wykonanie cafej pracy?

3. Dla jakich $\alpha\in [0,2\pi$) równanie kwadratowe $\cos\alpha\cdot x^{2}-2x+2\cos\alpha-1=0$ ma dwa

rózne pierwiastki?

4. Wierzcholkami czworokąta są punkty, których współrzędne spelniają układ równań

$\left\{\begin{array}{l}
xy+x-y\\
x^{2}-xy+y^{2}
\end{array}\right.$

1,

1.

Obliczyč pole czworokąta oraz wyznaczyč równanie okręgu na nim opisanego.

5. Pole powierzchni bocznej ostrosłupa prawidlowego czworokątnego jest 2 razy większe $\mathrm{n}\mathrm{i}\dot{\mathrm{z}}$

pole podstawy. Wyznaczyč cosinusy kątów dwuściennych przy krawędzi podstawy oraz

krawędzi bocznej. Sporządzič staranny rysunek.

6. Dany jest stozek ściety, $\mathrm{w}$ którym pole dolnej podstawy jest 4 razy wieksze od po1a górnej.

$\mathrm{W}$ stozek wpisano walec $\mathrm{t}\mathrm{a}\mathrm{k}, \dot{\mathrm{z}}\mathrm{e}$ dolna podstawa walca $\mathrm{l}\mathrm{e}\dot{\mathrm{z}}\mathrm{y}$ na dolnej podstawie stozka,

a brzeg górnej podstawy walca $\mathrm{l}\mathrm{e}\dot{\mathrm{z}}\mathrm{y}$ na powierzchni bocznej stozka. Jaką częśč objetości

stozka ściętego stanowi objętośč walca, $\mathrm{j}\mathrm{e}\dot{\mathrm{z}}$ eli wysokośč walca jest 3 razy mniejsza od

wysokości stozka? Odpowied $\acute{\mathrm{z}}$ podač $\mathrm{w}$ procentach $\mathrm{z}$ dokladnością do jednego promila.

Sporządzič staranny rysunek przekroju osiowego bryły.





PRACA KONTROLNA nr 8- POZIOM ROZSZERZONY

l. Rozwiązač nierównośč

2. Rozwiązač ukfad równań

$\displaystyle \frac{1}{x^{2}-2x-3}\geq\frac{1}{|x-2|+3}.$

$\left\{\begin{array}{l}
x^{2}+y^{2}=8,\\
- x1+-y1=1.
\end{array}\right.$

Obliczyč pole wielokąta $0$ wierzchofkach, których wspólrzędne spefniają powyzszy ukfad.

Podač ilustrację graficzną tego układu.

3. Wyznaczyč wszystkie wartości parametru $\alpha\in[-\pi,\pi$), dla których równanie kwadratowe

$(\sin 4\alpha)x^{2}-2(\cos\alpha)x+\sin 2\alpha=0$

ma dwa rózne nieujemne pierwiastki rzeczywiste. Rozwiązania zaznaczyč na kole trygo-

nometrycznym.

4. Udowodnič, $\dot{\mathrm{z}}\mathrm{e}\mathrm{j}\mathrm{e}\dot{\mathrm{z}}$ eli liczby rzeczywiste $a, b, c$ spełniajq warunki $a^{2}+b^{2}= (a+b-c)^{2}$

oraz $b, c\neq 0$, to

--{\it ab}22$++$(({\it ba}--{\it cc}))22$=$--{\it ab}--{\it cc}.

5. Trójkąt równoboczny $ABC0$ boku $a$ wpisano $\mathrm{w}$ okrąg. Na fuku $BC$ wybrano punkt

$D\mathrm{t}\mathrm{a}\mathrm{k}, \dot{\mathrm{z}}\mathrm{e}$ proste AB $\mathrm{i}CD$ przecinają się $\mathrm{w}$ punkcie $E\mathrm{i} |BE| = 2a$. Obliczyč pole $S$

czworokąta ABCD $\mathrm{i}$ wykazač, $\displaystyle \dot{\mathrm{z}}\mathrm{e}S=\frac{1}{4}(|BD|+|CD|)^{2}\sqrt{3}.$

6. Rozwinięcie, powierzchni, bocznej, stozka, ściętego, opisanego na kuli jest przedstawione

na rysunku. Obliczyč objętośč tego
\begin{center}
\includegraphics[width=61.416mm,height=51.456mm]{./KursMatematyki_PolitechnikaWroclawska_2010_2011_page15_images/image001.eps}
\end{center}
$\alpha$

{\it b}

stozka sciętego $\mathrm{i}$ promien kuli opisa-

nej na nim. Podac wynik liczbowy dla

$\displaystyle \alpha=\frac{\pi}{4}, b=4$ cm.





XL

KORESPONDENCYJNY KURS

Z MATEMATYKI

$\mathrm{p}\mathrm{a}\acute{\mathrm{z}}$dziernik 2010 $\mathrm{r}.$

PRACA KONTROLNA $\mathrm{n}\mathrm{r} 2-$ POZIOM PODSTAWOWY

l. Niech $A=\displaystyle \{x\in \mathbb{R}:\frac{1}{x^{2}+23}\geq\frac{1}{10x}\}$ oraz $B=\displaystyle \{x\in \mathbb{R}:|x-2|<\frac{7}{2}\}.$

Zbiory $A, B, A\cup B, A\cap B, A\backslash B\mathrm{i}B\backslash A$ zapisač $\mathrm{w}$ postaci przedzialów liczbowych $\mathrm{i}$

zaznaczyč je na osi liczbowej.

2. Zaznaczyč na pfaszczy $\acute{\mathrm{z}}\mathrm{n}\mathrm{i}\mathrm{e}$ zbiory

$A=\{(x,y):|x|+|y|\leq 2\}$ oraz

$\mathrm{i}$ obliczyč pole zbioru $A\cap B.$

$B=\displaystyle \{(x,y):\frac{1}{|x-1|}\leq\frac{1}{|x+3|},\frac{2}{|y-1|}\geq 1\}$

3. Trójmian kwadratowy $f(x)=ax^{2}+bx+c$ przyjmuje najmniejszą wartośč równą $-1 \mathrm{w}$

punkcie $x=1$ a reszta $\mathrm{z}$ dzielenia tego trójmianu przez dwumian $(x-2)$ równa jest l.

Wyznaczyč wspólczynniki $a, b, c$. Narysowač staranny wykres funkcji $g(x) = f(|x|) \mathrm{i}$

wyznaczyč najmniejszą $\mathrm{i}$ największą wartośč tej funkcji na przedziale [$-1,3].$

4. Tangens kąta ostrego $\alpha$ równy jest $\displaystyle \frac{a}{b}$, gdzie

$\alpha=(\sqrt{2+\sqrt{3}}-\sqrt{2-\sqrt{3}})^{2}b=(\sqrt{\sqrt{2}+1}-\sqrt{\sqrt{2}-1})^{2}$

Wyznaczyč wartości pozostalych funkcji trygonometrycznych tego $\mathrm{k}_{\Phi^{\mathrm{t}\mathrm{a}}}$. Wykorzystując

wzór $\sin 2\alpha=2\sin\alpha\cos\alpha$, obliczyč miarę kąta $\alpha.$

5. Narysowač wykres funkcji $f(x)=\sqrt{4x^{2}-4x+1}-x \mathrm{i}$ rozwiązač nierównośč $f(x)<0.$

$\mathrm{W}$ zalezności od parametru $m$ określič liczbę rozwiązań równania $|f(x)| = m$. Dla

jakiego $a$ pole trójkqta ograniczonego osia $Ox\mathrm{i}$ wykresem funkcji $g(x)=f(x)-a$ równe

jest 6?

6. Niech $f(x)=$

dla

dla

$x\leq 1,$

$x>1.$

a) Narysowač wykres funkcji $f\mathrm{i}$ na jego podstawie wyznaczyč zbiór wartości funkcji.

b) Obliczyč $f(\sqrt{3}-1)$ oraz $f(3-\sqrt{3}).$

c) Rozwiązač nierównośč $2\sqrt{f(x)}\leq 3\mathrm{i}$ zbiór jej rozwiązań zaznaczyč na osi $0x.$





PRACA KONTROLNA nr 2- POZIOM ROZSZERZONY

l. Rozwiązač nierównośč $\displaystyle \frac{1}{\sqrt{5+4x-x^{2}}}\geq\frac{1}{x-2} \mathrm{i}$ zbiór rozwiązań zaznaczyč $\mathrm{n}\mathrm{a}$ prostej.

2. Niech $A=\{(x,y):y\geq||x-2|-1|\}, B=\{(x,y):y+\sqrt{4x-x^{2}-3}\leq 2\}.$

Narysowač $\mathrm{n}\mathrm{a}\mathrm{p}${\it l}aszczy $\acute{\mathrm{z}}\mathrm{n}\mathrm{i}\mathrm{e}$ zbiór $A\cap B\mathrm{i}$ obliczyč jego pole.

3. Dla jakich wartości rzeczywistego parametru $p$ równanie $(p-1)x^{4}+(p-2)x^{2}+p=0$

ma dokladnie $\mathrm{d}\mathrm{w}\mathrm{a}$ rózne pierwiastki?

4. Znalez/č wszystkie wartości parametru rzeczywistego $m, \mathrm{d}\mathrm{l}\mathrm{a}$ których pierwiastki trójmia-

nu kwadratowego $f(x)=(m-2)x^{2}-(m+1)x-m$ spełniają nierównośč $|x_{1}|+|x_{2}|\leq 1.$

5. Narysowač staranny wykres funkcji

$f(x)=\{$

$\sqrt{x^{2}-4x+4}-1$

$-\sqrt{4x-x^{2}-3}$

, gdy

, gdy

$|x-2|\geq 1,$

$|x-2|\leq 1.$

$\mathrm{i}$ rozwiązač nierównośč $|f(x)| > \displaystyle \frac{1}{2}. \mathrm{W}$ zalezności od parametru $m$ określič liczbę roz-

wiązań równania $|f(x)| =m$. Obliczyč pole obszaru ograniczonego wykresem funkcji

$g(x)=|f(x)|\mathrm{i}$ prostą $y=\displaystyle \frac{1}{2}.$

6. Niech

$f(x)=$

gdy

gdy

$|x-1|\geq 1,$

$|x-1|<1.$

a) Obliczyč $f(-\displaystyle \frac{2}{3}), f(\displaystyle \frac{1+\sqrt{3}}{2})$ oraz $f(\pi-1).$

b) Narysowač wykres funkcji $f\mathrm{i}$ na jego podstawie podač zbiór wartości funkcji.

c) Rozwi$\mathfrak{B}$ač nierównośč $f(x)\displaystyle \geq-\frac{1}{2}\mathrm{i}$ zaznaczyč na osi $0x$ zbiór jej rozwi$\Phi$zań.





XL

KORESPONDENCYJNY KURS

Z MATEMATYKI

listopad 2010 r.

PRACA KONTROLNA $\mathrm{n}\mathrm{r} 3-$ POZIOM PODSTAWOWY

1. $\mathrm{W}$ trójkącie prostokątnym $0$ kącie prostym przy wierzchoku $C$ na przedluzeniu przeciw-

prostokqtnej $AB$ odmierzono odcinek $BD\mathrm{t}\mathrm{a}\mathrm{k}, \dot{\mathrm{z}}\mathrm{e}|BD|=|BC|$. Wyznaczyč $|CD|$ oraz

obliczyč pole trójkta $\triangle ACD, \mathrm{j}\mathrm{e}\dot{\mathrm{z}}$ eli $|BC|=5, |AC|=12.$

2. Harcerze rozbili 2 namioty, jeden $\mathrm{w}$ odległości 5 $\mathrm{m}$, drugi - 17 $\mathrm{m}$ od prostoliniowego

brzegu rzeki. Odlegfośč między namiotami równajest 13 $\mathrm{m}. \mathrm{W}$ którym miejscu na samym

brzegu rzeki (liczqc od punktu brzegu będacego rzutem prostopadłym punktu połozenia

pierwszego namiotu) powinni umieścič maszt $\mathrm{z}\mathrm{f}\mathrm{l}\mathrm{a}\mathrm{g}\Phi$ zastępu, by odlegfośč od masztu do

$\mathrm{k}\mathrm{a}\dot{\mathrm{z}}$ dego $\mathrm{z}$ namiotów była taka sama?

3. Na kole $0$ promieniu $r$ opisano trapez równoramienny, $\mathrm{w}$ którym stosunek dlugości pod-

staw wynosi 4: 3. Ob1iczyč stosunek po1a ko1a do po1a trapezu oraz cosinus kąta ostrego

$\mathrm{w}$ tym trapezie.

4. Wielomian $W(x) =x^{3}-x^{2}+bx+c$ jest podzielny przez $(x+3)$, a reszta $\mathrm{z}$ dzielenia

tego wielomianu przez $(x-3)$ równa jest 6. Wyznaczyč $b\mathrm{i} c$, a następnie rozwiązač

nierównośč $(x+1)W(x-1)-(x+2)W(x-2)\leq 0.$

5. Wykonač dziafania $\mathrm{i}$ zapisač $\mathrm{w}$ najprostszej postaci wyrazenie

$s(a,b)= (\displaystyle \frac{a^{2}+b^{2}}{a^{2}-b^{2}}-\frac{a^{3}+b^{3}}{a^{3}-b^{3}})$ : $(\displaystyle \frac{a^{2}}{a^{3}-b^{3}}-\frac{a}{a^{2}+ab+b^{2}})$

Wyznaczyč wysokośč trójkąta prostokątnego wpisanego $\mathrm{w}$ okrąg $0$ promieniu 6 opusz-

czoną $\mathrm{z}$ wierzcholka kąta prostego wiedząc, $\dot{\mathrm{z}}\mathrm{e}$ tangens jednego $\mathrm{z}$ kątów ostrych tego

trójkąta równy jest $s(\sqrt{5}+\sqrt{3},\sqrt{5}-\sqrt{3}).$

6. $\mathrm{W}$ trójkącie $ABC$ dane są $\angle CAB= \displaystyle \frac{\pi}{3}$, wysokośč $|CD| =h=5$ oraz $BD=d=\sqrt{2}.$

Obliczyč odległośč środków okręgów wpisanych $\mathrm{w}$ trójkąty ADC $\mathrm{i}\mathrm{D}\mathrm{B}\mathrm{C}.$





PRACA KONTROLNA nr 3- POZIOM ROZSZERZONY

l. Dany jest wielomian $W(x) = x^{3}+ax+b$, gdzie $b \neq 0$. Wykazač, $\dot{\mathrm{z}}\mathrm{e} W(x)$ posiada

pierwiastek podwójny wtedy $\mathrm{i}$ tylko wtedy, gdy spelniony jest warunek $4a^{3}+27b^{2}=0.$

Wyrazič pierwiastki za pomocą współczynnika $b.$

2. Wyznaczyč promień okręgu opisanego na czworokqcie ABCD, $\mathrm{w}$ którym $\mathrm{k}\mathrm{a}\mathrm{t}$ przy wierz-

chofku $A$ ma miarę $\alpha$, kąty przy wierzchofkach $B, D$ są proste oraz $|BC|=a, |AD|=b.$

Sporządzič staranny rysunek.

3. Narysowač staranny wykres funkcji $f(x)=\displaystyle \frac{\sin 2x-|\sin x|}{\sin x}.$

$\mathrm{W}$ przedziale $[0,\pi]$ wyznaczyč $\mathrm{r}\mathrm{o}\mathrm{z}\mathrm{w}\mathrm{i}_{\Phi}$zania nierówności $f(x)<2(\sqrt{2}-1)\cos^{2}x.$

4. $\mathrm{Z}$ wierzchołka $A$ kwadratu ABCD $0$ boku $a$ poprowadzono dwie proste, które dzielą kąt

przy tym wierzchołku na trzy równe części $\mathrm{i}$ przecinają boki kwadratu $\mathrm{w}$ punktach $K\mathrm{i}$

$L$. Wyznaczyč dfugości odcinków, na jakie te proste dzielą $\mathrm{P}^{\mathrm{r}\mathrm{z}\mathrm{e}\mathrm{k}}\Phi^{\mathrm{t}\mathrm{n}}\Phi$ kwadratu. Znalez/č

promień okręgu wpisanego $\mathrm{w}$ deltoid AKCL.

5. Czworokąt wypukły ABCD, $\mathrm{w}$ którym $AB=1, BC=2, CD=4, DA=3$ jest wpisany

$\mathrm{w}$ okrąg. Obliczyč promień $R$ tego okręgu. Sprawdzič, czy $\mathrm{w}$ czworokąt ten $\mathrm{m}\mathrm{o}\dot{\mathrm{z}}$ na wpisač

okrąg. $\mathrm{J}\mathrm{e}\dot{\mathrm{z}}$ eli $\mathrm{t}\mathrm{a}\mathrm{k}$, to obliczyč promień $r$ tego okręgu.

6. Na boku $BC$ trójkąta równobocznego obrano punkt $D\mathrm{t}\mathrm{a}\mathrm{k}, \dot{\mathrm{z}}\mathrm{e}$ promień okręgu wpisanego

$\mathrm{w}$ trójkąt $ADC$ jest dwa razy mniejszy $\mathrm{n}\mathrm{i}\dot{\mathrm{z}}$ promień okręgu wpisanego $\mathrm{w}$ trójkąt $ABD.$

$\mathrm{W}$ jakim stosunku punkt $D$ dzieli bok $BC$?





XL

KORESPONDENCYJNY KURS

Z MATEMATYKI

grudzień 2010 r.

PRACA KONTROLNA $\mathrm{n}\mathrm{r} 4-$ POZIOM PODSTAWOWY

l. Rozwiązač równanie $\displaystyle \frac{1}{\cos x}+\mathrm{t}\mathrm{g}x-\sin(\frac{\pi}{2}-x)=0$ dla $x\in[-2\pi,2\pi].$

2. Na p{\it l}aszczy $\acute{\mathrm{z}}\mathrm{n}\mathrm{i}\mathrm{e}$ dane są cztery punkty: $A(1,-1), B(5,7), C(4,-4), D(2,4)$. Obliczyč od-

ległośč punktu przecięcia prostych AB $\mathrm{i}CD$ od symetralnej odcinka $BC$. Sporz$\Phi$dzič

rysunek.

3. Rozwiązač uklad równań

$\left\{\begin{array}{l}
y+x^{2}=4\\
4x^{2}-y^{2}+2y=1
\end{array}\right.$

Podač interpretację geometryczną tego ukladu $\mathrm{i}$ wykazač, $\dot{\mathrm{z}}\mathrm{e}$ cztery punkty, które $\mathrm{s}\Phi$

jego rozwiązaniem, wyznaczają na płaszczy $\acute{\mathrm{z}}\mathrm{n}\mathrm{i}\mathrm{e}$ trapez równoramienny. Znalez$\acute{}$č równanie

okręgu opisanego na tym trapezie.

4. $\mathrm{W}$ ostroslupie prawidfowym trójkątnym dlugośč krawędzi podstawy jest równa $a$. Kąt

między krawędzią podstawy, a krawędzią boczną jest równy $\displaystyle \frac{\pi}{4}$. Obliczyč pole przekro-

ju ostrosłupa $\mathrm{p}\mathrm{a}$szczyz $\Phi$ przechodzącą przez krawędz/ podstawy $\mathrm{i}$ środek przeciwleglej

krawędzi bocznej. Sporządzič staranny rysunek.

5. Dane sa dwa okręgi: $K_{1}0$ środku $\mathrm{w}$ punkcie $(0,0)\mathrm{i}$ promieniu 5 $\mathrm{i}K_{2}\mathrm{o}$ równaniu

$x^{2}+6x+y^{2}-12y+5=0$. Obliczyč pole czworokąta wyznaczonego przez środki okręgów

oraz punkty, $\mathrm{w}$ których te okręgi się przecinają. Sporządzič staranny rysunek.

6. Podstawą graniastosłupa jest równolegfobok $0$ bokach dfugości $a\mathrm{i}2a$ oraz kącie ostrym

$\displaystyle \frac{\pi}{3}$. Krótsza przekątna graniastoslupa tworzy $\mathrm{w}$ pfaszczyzną podstawy kąt $\displaystyle \frac{\pi}{6}$. Obliczyč

długośč dłuzszej przekatnej oraz pole powierzchni całkowitej tego graniastoslupa.





PRACA KONTROLNA nr 4- POZIOM ROZSZERZONY

l. Rozwiązač równanie 2 $\sin^{2}x-2\sin x\cos 2x=1.$

2. Dane są dwa wektory $\vec{a}= [2,-3]$ oraz $\vec{b}= [-1,4]$. Pokazač, $\dot{\mathrm{z}}\mathrm{e}$ wektor $\vec{AB}=3\text{{\it ã}}+2\vec{b}$

jest prostopadły do wektora $\vec{BC}=8\text{{\it ã}}+11\vec{b}$. Obliczyč dlugośč środkowej trójkąta $ABC$

rozpiętego na wektorach $\vec{AB}\mathrm{i}\vec{BC}$, poprowadzonej $\mathrm{z}$ wierzchołka $B.$

3. Niech $K$ będzie wierzchofkiem paraboli $f(x)=-\displaystyle \frac{4}{9}x^{2}-\frac{8}{3}x$, a L- wierzcholkiem paraboli

$g(x) = -f(x-7)+7$. Na paraboli $g(x)$ znalez/č taki punkt $N$, aby wektor $\vec{NL}$ był

równolegfy do wektora $\vec{MK}$, gdzie $M=(0,f(0))$. Obliczyč pole czworokąta KMLN.

4. Przekrój sześcianu płaszczyznq jest sześciokątem foremnym. Wyznaczyč kąt nachylenia

tej pfaszczyzny do pfaszczyzny podstawy sześcianu oraz obliczyč pole tego przekroju.

Wykonač odpowiedni rysunek.

5. Dane są dwa okręgi: $K_{1} 0$ środku $\mathrm{w}$ punkcie $P(1,1) \mathrm{i}$ promieniu l oraz $K_{2} 0$ środku

$Q(9,5) \mathrm{i}$ promieniu 3. Zna1ez/č punkt $S$ na odcinku $\overline{PQ}$ oraz dobrač skalę $k\mathrm{t}\mathrm{a}\mathrm{k}$, aby

okrąg $K_{2}$ był obrazem okręgu $K_{1} \mathrm{w}$ jednokładności $0$ środku $S\mathrm{i}$ skali $k$. Wyznaczyč

równania prostych, które są styczne jednocześnie do obu okręgów $\mathrm{i}\mathrm{p}\mathrm{r}\mathrm{z}\mathrm{e}\mathrm{c}\mathrm{h}\mathrm{o}\mathrm{d}\mathrm{z}\Phi$ przez

punkt $S.$

6. $\mathrm{W}$ ostrosłupie prawidłowym czworokatnym pole $\mathrm{k}\mathrm{a}\dot{\mathrm{z}}$ dej $\mathrm{z}$ pięciu ścian jest równe l. Ostro-

slup ten ścięto $\mathrm{w}$ polowie wysokości $\mathrm{p}^{\mathrm{f}\mathrm{n}}$aszczyz $\Phi$ równolegfą do podstawy. Obliczyč ob-

jętośč oraz pole powierzchni całkowitej otrzymanego ostrosłupa ściętego. Wykonač od-

powiedni rysunek.





XL

KORESPONDENCYJNY KURS

Z MATEMATYKI

styczeń 2011 r.

PRACA KONTROLNA $\mathrm{n}\mathrm{r} 5-$ POZIOM PODSTAWOWY

1. $\mathrm{W}$ ciagu arytmetycznym suma poczatkowych dwudziestu jeden wyrazów wynosi $21\sqrt{2},$

a jego $\mathrm{d}\mathrm{z}\mathrm{i}\mathrm{e}\mathrm{s}\mathrm{i}_{\Phi}\mathrm{t}\mathrm{y}$ wyraz równy jest $-2-2\sqrt{2}$. Wyznaczyč najmniejszy dodatni wyraz

tego ciągu.

2. Rozwiazač nierównośč

$-2<\log_{\frac{1}{2}}(5x+2)\leq 2.$

3. Firmy X $\mathrm{i}\mathrm{Y}$ jednocześnie rozpoczęly dzialalnośč. $\mathrm{W}$ pierwszym miesiącu $\mathrm{k}\mathrm{a}\dot{\mathrm{z}}$ da $\mathrm{z}$ nich

miała dochód równy 50000 zf. Po pięciu miesiącach okazało sie, $\dot{\mathrm{z}}\mathrm{e}$ dochód firmy X

rósł $\mathrm{z}$ miesiąca na miesiąc $0$ tę samą kwote, a dochód firmy $\mathrm{Y}$ wzrastał co miesiąc

geometrycznie. $\mathrm{W}$ drugim $\mathrm{i}$ trzecim miesiącu działalnosci firma X miała dochód wiekszy

od dochodu firmy $\mathrm{Y} 0$ 2000 $\mathrm{z}l$. Ustalič, która $\mathrm{z}$ firm miala wiekszą sumę dochodów

$\mathrm{w}$ pierwszych pięciu miesiącach swojej dzialalności.

4. Sporządzič staranny wykres funkcji (za jednostkę przyjąč 2 cm)

$f(x)=(-2x^{2}+3x\displaystyle \frac{|x|}{1-x}$

dla

dla

$|x-1|\geq 1,$

$|x-1|<1.$

Korzystajqc $\mathrm{z}$ niego, określič ilośč rozwiazań równania $f(x) =m \mathrm{w}$ zalezności od rze-

czywistego parametru $m.$

5. Stosując wzór na sinus podwojonego kata oraz wzory redukcyjne, obliczyč wartośč wy-

$\mathrm{r}\mathrm{a}\dot{\mathrm{z}}$ enia

$\displaystyle \cos\frac{\pi}{5}\cdot\cos\frac{2\pi}{5}\cdot\cos\frac{3\pi}{5}\cdot\cos\frac{4\pi}{5}.$

6. Wiedząc, $\dot{\mathrm{z}}\mathrm{e} \displaystyle \sin\frac{\pi}{10} = \displaystyle \frac{1}{4}(\sqrt{5}-1)$, wyznaczyč wszystkie kąty $\alpha \in [0,\pi]$, dla których

spefnione jest równanie

$2^{2+\sin\alpha}=\sqrt{2}\cdot 4^{\cos^{2}\alpha}$





PRACA KONTROLNA nr 5- POZIOM ROZSZERZONY

l. Zaznaczyč na osi liczbowej zbiór rozwiązań nierówności

$\displaystyle \frac{2x-\sqrt{2-x}}{x}\geq x.$

2. Wyznaczyč wszystkie liczby rzeczywiste x, dla których funkcja

$f(x)=2^{x^{2}+2}-2^{x^{2}-1}-2\cdot 7^{x^{2}-1}$

przyjmuje wartości dodatnie.

3. Określič dziedzinę $\mathrm{i}$ sporządzič staranny wykres funkcji $f(x) = 1-\log_{3}(1-x)$. Za

jednostkę przyj$\Phi$č 2 cm. Zna1ez/č obraz tego wykresu $\mathrm{w}$ symetrii osiowej względem prostej

$x=y\mathrm{i}$ podač wzór funkcji, której wykresem jest nowo powstala krzywa.

4. Rozwiązač nierównośč

$\sqrt{\log_{2}(x^{2}-1)}>\log_{2}\sqrt{x^{2}-1}.$

5. Niech $c>0\mathrm{i}c\neq 1$. Znalez/č liczbę naturalną $m$, dla ktorej suma $m$ poczatkowych wyra-

zów ciągu arytmetycznego $a_{n}=\log_{2}(c^{n})$, jest 10100 razy większa od sumy wszystkich

wyrazów ciągu geometrycznego $b_{n}=\log_{2^{3^{n}}}(c).$

6. Korzystajqc ze wzoru

$\sin 5\alpha=5\sin\alpha-20\sin^{3}\alpha+16\sin^{5}\alpha,$

obliczyč wartośč $\displaystyle \sin\frac{\pi}{5}$. Podač wartości wyrazeń $\displaystyle \cos\frac{\pi}{5}, \displaystyle \sin\frac{\pi}{10}$ oraz $\displaystyle \cos\frac{\pi}{10}$. Wyprowa-

dzič wzór na pole dwudziestokąta foremnego wpisanego $\mathrm{w}$ okrąg $0$ promieniu $r.$



\end{document}