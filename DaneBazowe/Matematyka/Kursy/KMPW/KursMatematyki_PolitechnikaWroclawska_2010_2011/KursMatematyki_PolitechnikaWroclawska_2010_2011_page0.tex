\documentclass[a4paper,12pt]{article}
\usepackage{latexsym}
\usepackage{amsmath}
\usepackage{amssymb}
\usepackage{graphicx}
\usepackage{wrapfig}
\pagestyle{plain}
\usepackage{fancybox}
\usepackage{bm}

\begin{document}

XL

KORESPONDENCYJNY KURS

Z MATEMATYKI

wrzesień 2010 r.

PRACA KONTROLNA $\mathrm{n}\mathrm{r} 1-$ POZIOM PODSTAWOWY

l. Ile jest trzycyfrowych liczb naturalnych:

a) podzielnych przez 31ub przez 5?

b) podzielnych przez 31ub przez 6?

c) podzielnych przez 3 $\mathrm{i}$ niepodzielnych przez 5?

2. Renomowany dom mody sprzedaf 40\% kolekcji letniej po zafozonej cenie. Po obnizce

ceny $0$ 50\% udało $\mathrm{s}\mathrm{i}\mathrm{e}$ sprzedač połowę pozostałej części towaru $\mathrm{i}$ dopiero kolejna 50\%-

owa obnizka pozwolifa opróznič magazyny. Ile procent zaplanowanego przychodu stanowi

uzyskana ze sprzedaz $\mathrm{y}$ kwota? $\mathrm{O}$ ile procent wyjściowa cena towaru powinna była byč

$\mathrm{w}\mathrm{y}\dot{\mathrm{z}}$ sza, by sklep uzyskał zaplanowany początkowo przychód?

3. Określič dziedzinę wyrazenia $w(x,y)=\displaystyle \frac{2}{x-y}-\frac{3xy}{x^{3}-y^{3}}-\frac{x-y}{x^{2}+xy+y^{2}}.$

Sprowadzič je do najprostszej postaci $\mathrm{i}$ obliczyč $w(1+\sqrt{2},(1+\sqrt{2})^{-1})$

4. Obliczyč sumę wszystkich liczb pierwszych spelniających nierównośč

$(p-4)x^{2}-4(p-2)x-p\leq 0$, gdzie $p=\displaystyle \frac{64^{\frac{1}{3}}\sqrt{8}+8^{\frac{1}{3}}\sqrt{64}}{\sqrt[3]{64\sqrt{8}}}$

5. Dwa naczynia zawierają $\mathrm{w}$ sumie 401itrów wody. Po prze1aniu pewnej części wody pierw-

szego naczynia do drugiego, $\mathrm{w}$ pierwszym naczyniu zostało trzy razy mniej wody $\mathrm{n}\mathrm{i}\dot{\mathrm{z}}\mathrm{w}$

drugim. Gdy następnie przelano taką samą częśč wody drugiego naczynia do pierwszego,

okazało się, $\dot{\mathrm{z}}\mathrm{e}\mathrm{w}$ obu naczyniach jest tyle samo płynu. Obliczyč, ile wody było pierwotnie

$\mathrm{w}\mathrm{k}\mathrm{a}\dot{\mathrm{z}}$ dym naczyniu $\mathrm{i}\mathrm{j}\mathrm{a}\mathrm{k}_{\Phi}$ jej częśč przelewano.

6. Dwie $\mathrm{g}\mathrm{a}\acute{\mathrm{z}}$dziny, pracując razem, mogą wykonač zamówioną partię pisanek $\mathrm{w}$ ciągu 7

dni pod warunkiem, $\dot{\mathrm{z}}\mathrm{e}$ pierwsza $\mathrm{z}$ nich rozpocznie pracę $0$ póltora dnia wcześniej $\mathrm{n}\mathrm{i}\dot{\mathrm{z}}$

druga. Gdyby $\mathrm{k}\mathrm{a}\dot{\mathrm{z}}\mathrm{d}\mathrm{a}\mathrm{z}$ nich pracowała oddzielnie, to druga wykonalaby cafą pracę $03$

$\mathrm{d}\mathrm{n}\mathrm{i}$ wcześniej od pierwszej. Ile $\mathrm{d}\mathrm{n}\mathrm{i}$ potrzebuje $\mathrm{k}\mathrm{a}\dot{\mathrm{z}}$ da $\mathrm{z}$ kobiet na wykonanie calej pracy?
\end{document}
