\documentclass[a4paper,12pt]{article}
\usepackage{latexsym}
\usepackage{amsmath}
\usepackage{amssymb}
\usepackage{graphicx}
\usepackage{wrapfig}
\pagestyle{plain}
\usepackage{fancybox}
\usepackage{bm}

\begin{document}

PRACA KONTROLNA nr l- POZIOM ROZSZERZONY

l. Ile jest liczb pięciocyfrowych podzielnych przez 9, które $\mathrm{w}$ rozwinięciu dziesiętnym mają:

a) obie cyfry 1, 2 $\mathrm{i}$ tylko $\mathrm{t}\mathrm{e}$? b) obie cyfry 1, 3 $\mathrm{i}$ tylko $\mathrm{t}\mathrm{e}$? c) wszystkie cyfry 1, 2, 3

$\mathrm{i}$ tylko $\mathrm{t}\mathrm{e}$? Odpowiedz/uzasadnič. $\mathrm{W}$ przypadku b) wypisač otrzymane liczby.

2. Pan Kowalski zaciqgnąl 3l grudnia $\mathrm{p}\mathrm{o}\dot{\mathrm{z}}$ yczkę 4000 złotych oprocentowaną $\mathrm{w}$ wysoko-

ści 18\% $\mathrm{w}$ skali roku. Zobowiązał się splacič $\mathrm{j}_{\Phi}\mathrm{w}\mathrm{c}\mathrm{i}_{\Phi \mathrm{g}}\mathrm{u}$ roku $\mathrm{w}$ trzech równych ratach

platnych 30 kwietnia, 30 sierpnia $\mathrm{i}30$ grudnia. Oprocentowanie $\mathrm{p}\mathrm{o}\dot{\mathrm{z}}$ yczki liczy się od

l stycznia, a odsetki od kredytu naliczane są $\mathrm{w}$ terminach płatności rat. Obliczyč wyso-

kośč tych rat $\mathrm{w}$ zaokrągleniu do pełnych groszy.

3. Określič dziedzinę wyrazenia

$w(x,y)=\displaystyle \frac{x}{x^{3}+x^{2}y+xy^{2}+y^{3}}+\frac{y}{x^{3}-x^{2}y+xy^{2}-y^{3}}+\frac{1}{x^{2}-y^{2}}-\frac{1}{x^{2}+y^{2}}-\frac{x^{2}+2y^{2}}{x^{4}-y^{4}}.$

Sprowadzič je do najprostszej postaci $\mathrm{i}$ obliczyč $w(\cos 15^{\mathrm{o}},\sin 15^{\mathrm{o}}).$

4. Liczba $p = \displaystyle \frac{(\sqrt[3]{54}-2)(9\sqrt[3]{4}+6\sqrt[3]{2}+4)-(2-\sqrt{3})^{3}}{\sqrt{3}+(1+\sqrt{3})^{2}}$ jest miejscem zerowym funkcji

kwadratowej $f(x)=ax^{2}+bx+c$. Wyznaczyč współczynniki $a, b, c$ oraz drugie miejsce

zerowe tej funkcji wiedząc, $\dot{\mathrm{z}}\mathrm{e}$ największ$\Phi$ wartością funkcji jest 4, a jej wykres jest

symetryczny wzgledem prostej $x=1.$

5. Do zbiornika poprowadzono trzy rury. Pierwsza rura potrzebuje do napefnienia zbiornika

$04$ godziny więcej $\mathrm{n}\mathrm{i}\dot{\mathrm{z}}$ druga, a trzecia napełnia caly zbiornik $\mathrm{w}$ czasie dwa razy krótszym

$\mathrm{n}\mathrm{i}\dot{\mathrm{z}}$ pierwsza. Wjakim czasie napelnia zbiornik $\mathrm{k}\mathrm{a}\dot{\mathrm{z}}$ da $\mathrm{z}\mathrm{r}\mathrm{u}\mathrm{r}, \mathrm{j}\mathrm{e}\dot{\mathrm{z}}$ eli wiadomo, $\dot{\mathrm{z}}\mathrm{e}$ wszystkie

trzy rury otwarte jednocześnie napefniają zbiornik $\mathrm{w}$ ciągu 2 godzin $\mathrm{i}40$ minut?

6. $\mathrm{Z}$ przystani A wyrusza $\mathrm{z}$ biegiem rzeki statek do przystani $\mathrm{B}$, odległej od A $0140$ km. Po

upływie l godziny wyrusza za nim łódz/ motorowa, dopędza statek, po czym wraca do

przystani A $\mathrm{w}$ tym samym momencie, $\mathrm{w}$ którym statek przybija do przystani B. Znalez/č

prędkośč biegu rzeki, $\mathrm{j}\mathrm{e}\dot{\mathrm{z}}$ eli wiadomo, $\dot{\mathrm{z}}\mathrm{e}\mathrm{w}$ stojącej wodzie prędkośč statku wynosi 16

$\mathrm{k}\mathrm{m}/$godz, a prędkośč łodzi 24 $\mathrm{k}\mathrm{m}/$godz.
\end{document}
