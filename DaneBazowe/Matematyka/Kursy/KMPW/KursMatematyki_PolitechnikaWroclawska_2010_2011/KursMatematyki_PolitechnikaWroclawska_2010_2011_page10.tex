\documentclass[a4paper,12pt]{article}
\usepackage{latexsym}
\usepackage{amsmath}
\usepackage{amssymb}
\usepackage{graphicx}
\usepackage{wrapfig}
\pagestyle{plain}
\usepackage{fancybox}
\usepackage{bm}

\begin{document}

XL

KORESPONDENCYJNY KURS

Z MATEMATYKI

luty 2011 r.

PRACA KONTROLNA $\mathrm{n}\mathrm{r} 6-$ POZIOM PODSTAWOWY

l. Losujemy liczbę ze zbioru \{1, 2, 3, $\ldots$, 100\}, a następnie liczbę ze zbioru \{2, 3, 4, 5\}. Obli-

czyč prawdopodobieństwo, $\dot{\mathrm{z}}\mathrm{e}$ pierwsza $\mathrm{z}$ wylosowanych liczb jest podzielna przez $\mathrm{d}\mathrm{r}\mathrm{u}\mathrm{g}\Phi.$

2. Liczba 2-elementowych podzbiorów zbioru $A$ jest 7 razy większa $\mathrm{n}\mathrm{i}\dot{\mathrm{z}}$ liczba 2-e1emento-

wych podzbiorów zbioru $B$. Liczba 2-e1ementowych podzbiorów zbioru $A$ nie zawierają-

cych ustalonego elementu $a\in A$ jest 5 razy większa $\mathrm{n}\mathrm{i}\dot{\mathrm{z}}$ liczba 2-e1ementowych podzbiorów

zbioru $B$. Ile elementów ma $\mathrm{k}\mathrm{a}\dot{\mathrm{z}}\mathrm{d}\mathrm{y}\mathrm{z}$ tych zbiorów? Ile $\mathrm{k}\mathrm{a}\dot{\mathrm{z}}\mathrm{d}\mathrm{y}\mathrm{z}$ tych zbiorów ma podzbio-

rów 3-e1ementowych?

3. $\mathrm{W}$ turnieju szachowym $\mathrm{k}\mathrm{a}\dot{\mathrm{z}}\mathrm{d}\mathrm{y}$ uczestnik miał rozegrač $\mathrm{z}$ pozostałymi po jednej partii. Po

rozegraniu trzech partii dwóch szachistów zrezygnowalo $\mathrm{z}$ dalszej $\mathrm{g}\mathrm{r}\mathrm{y}. \mathrm{W}$ sumie rozegra-

no 84 partie. I1u by1o uczestników na początku turnieju, $\mathrm{j}\mathrm{e}\dot{\mathrm{z}}$ eli dwaj zawodnicy, którzy

zrezygnowali, nie grali ze sobq?

4. Suma pierwszego $\mathrm{i}$ trzeciego wyrazu $\mathrm{c}\mathrm{i}_{\Phi \mathrm{g}}\mathrm{u}$ geometrycznego $(a_{n})$ wynosi 20. Znajd $\acute{\mathrm{z}}$ wzór

ogólny ciągu arytmetycznego $(b_{n})$ takiego, $\dot{\mathrm{z}}\mathrm{e}b_{1}=a_{1}, b_{2}=a_{2}, b_{5}=a_{3}.$

5. Rozkład ocen ze sprawdzianu $\mathrm{w}$ klasie IIIa jest opisany tabelką
\begin{center}
\begin{tabular}{l|l|l|l|l|l}
\multicolumn{1}{l|}{ocena}&	\multicolumn{1}{|l|}{$1$}&	\multicolumn{1}{|l|}{ $2$}&	\multicolumn{1}{|l|}{ $3$}&	\multicolumn{1}{|l|}{ $4$}&	\multicolumn{1}{|l}{ $5$}	\\
\hline
\multicolumn{1}{l|}{liczba osób}&	\multicolumn{1}{|l|}{$1$}&	\multicolumn{1}{|l|}{ $2$}&	\multicolumn{1}{|l|}{ $8$}&	\multicolumn{1}{|l|}{ $9$}&	\multicolumn{1}{|l}{ $6$}
\end{tabular}

\end{center}
Jaś otrzymał ocenę 4. Czy wypadł powyzej średniej $\mathrm{w}$ swojej klasie? $\mathrm{W}$ pozostałych kla-

sach średnie punktów wynosily: 3,875 $\mathrm{w}$ IIIb (24 osoby) $\mathrm{i}4,6\mathrm{w}$ IIIc (25 osób). Czy ocena

otrzymana przez Jasia znajduje się powyzej średniej liczonej łącznie wśród wszystkich

uczniów klas trzecich? Ile co najmniej, a ile co najwyzej, osób miało piqtki $\mathrm{w}$ klasie IIIc

(skala ocen to $1,2,\ldots,5$)?

6. Ile liczb czterocyfrowych $0$ wszystkich cyfrach róznych $\mathrm{m}\mathrm{o}\dot{\mathrm{z}}$ na utworzyč $\mathrm{z}$ cyfr 1,2,3,4,5, $\mathrm{a}$

ile $\mathrm{z}$ cyfr 0,1,2,3,4,5,6? $\mathrm{W}$ obu przypadkach obliczyč, ile $\mathrm{m}\mathrm{o}\dot{\mathrm{z}}$ na utworzyč czterocyfrowych

liczb podzielnych przez 5.
\end{document}
