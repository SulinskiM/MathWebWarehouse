\documentclass[a4paper,12pt]{article}
\usepackage{latexsym}
\usepackage{amsmath}
\usepackage{amssymb}
\usepackage{graphicx}
\usepackage{wrapfig}
\pagestyle{plain}
\usepackage{fancybox}
\usepackage{bm}

\begin{document}

PRACA KONTROLNA nr 6- POZIOM ROZSZERZONY

l. Trzeci skladnik rozwinięcia dwumianu $(\displaystyle \sqrt[3]{x}+\frac{1}{\sqrt{x}})^{n}$ ma wspólczynnik równy 45. Wyzna-

czyč wszystkie składniki tego rozwinięcia, $\mathrm{w}$ których $x$ wystepuje $\mathrm{w}$ potędze $0$ wykfadniku

cafkowitym.

2. $\mathrm{W}$ turnieju szachowym rozgrywanym systemem,,kazdy $\mathrm{z}\mathrm{k}\mathrm{a}\dot{\mathrm{z}}$ dym'' dwóch uczestników

nie ukończyfo turnieju, przy czym jeden $\mathrm{z}$ nich rozegra110 partii, a drugi ty1ko jedną. I1u

było zawodników $\mathrm{i}$ czy wspomniani zawodnicy grali ze sobą, $\mathrm{j}\mathrm{e}\dot{\mathrm{z}}$ eli rozegrano 55 partii?

3. $\mathrm{W}$ pudełku jest 400 ku1 $\mathrm{w}$ tym $n$ czerwonych. Wybieramy losowo dwie kule. Prawdopo-

dobieństwo wylosowania dwóch kul czerwonych jest równe $\displaystyle \frac{1}{760}.$

a) Ile kul czerwonych jest $\mathrm{w}$ tym pudełku?

b) Obliczyč prawdopodobieństwo, $\dot{\mathrm{z}}\mathrm{e}\dot{\mathrm{z}}$ adna $\mathrm{z}$ wylosowanych kul nie jest czerwona.

4. Suma wyrazów nieskończonego ciagu geometrycznego zmniejszy $\mathrm{s}\mathrm{i}\mathrm{e}0$ 25\%, $\mathrm{j}\mathrm{e}\dot{\mathrm{z}}$ eli wy-

kreślimy $\mathrm{z}$ niej skfadniki $0$ numerach parzystych niepodzielnych przez 4. Ob1iczyč sumę

wszystkich wyrazów tego ciqgu wiedząc, $\dot{\mathrm{z}}\mathrm{e}$ jego drugi wyraz wynosi l.

5. Stosując zasadę indukcji matematycznej udowodnič prawdziwośč wzoru

$\left(\begin{array}{l}
2\\
2
\end{array}\right) - \left(\begin{array}{l}
3\\
2
\end{array}\right) + \left(\begin{array}{l}
4\\
2
\end{array}\right) - \left(\begin{array}{l}
5\\
2
\end{array}\right) +\ldots+\left(\begin{array}{l}
2n\\
2
\end{array}\right) =n^{2},$

$n\geq 1.$

6. Wśród wszystkich $\mathrm{b}\mathrm{l}\mathrm{i}\acute{\mathrm{z}}$niąt 64\% stanowią bliz$\acute{}$nięta tej samej plci. Prawdopodobieństwo

urodzenia chłopca wynosi 0,51. Ob1iczyč prawdopodobieństwo, $\dot{\mathrm{z}}\mathrm{e}$ drugie $\mathrm{z}\mathrm{b}\mathrm{l}\mathrm{i}\acute{\mathrm{z}}$niąt jest

$\mathrm{d}\mathrm{z}\mathrm{i}\mathrm{e}\mathrm{w}\mathrm{c}\mathrm{z}\mathrm{y}\mathrm{n}\mathrm{k}_{\Phi}$, pod warunkiem, $\dot{\mathrm{z}}\mathrm{e}$:

a) pierwsze jest dziewczynką,

b) pierwsze jest chlopcem.
\end{document}
