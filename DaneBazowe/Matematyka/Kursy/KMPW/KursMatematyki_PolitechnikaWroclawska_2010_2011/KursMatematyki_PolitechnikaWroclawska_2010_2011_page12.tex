\documentclass[a4paper,12pt]{article}
\usepackage{latexsym}
\usepackage{amsmath}
\usepackage{amssymb}
\usepackage{graphicx}
\usepackage{wrapfig}
\pagestyle{plain}
\usepackage{fancybox}
\usepackage{bm}

\begin{document}

XL

KORESPONDENCYJNY KURS

Z MATEMATYKI

marzec 2011 r.

PRACA KONTROLNA nr 7- POZIOM PODSTAWOWY

l. Rozwiązač równanie

$1-|x|=\sqrt{1+x} \mathrm{i}$ podač jego ilustrację graficzną.

2. Wyznaczyč wszystkie punkty $x\mathrm{z}$ przedziafu $[0,2\pi]$, dla których spefnionajest nierównośč

$\sin 2x-\mathrm{t}\mathrm{g}x\leq 0$. Podač ilustrację graficzną nierówności.

3. Określič liczbę rozwiązań ukfadu równań

$\left\{\begin{array}{l}
y\\
y
\end{array}\right.$

$|x-2|+1,$

$ax$

$\mathrm{w}$ zalezności od wartości współczynnika kierunkowego prostej $y=ax$. Znalez$\acute{}$č rozwiąza-

nia $\mathrm{w}$ przypadku, gdy jednym $\mathrm{z}$ nich jest para (4, 3). Sporzadzič staranny rysunek.

4. Danajest prosta $l:x+2y-4=0$. Przez punkt (l, l) poprowadzič prostą $k\mathrm{o}$ dodatnim

współczynniku kierunkowym $\mathrm{t}\mathrm{a}\mathrm{k}$, aby pole trójkqta ograniczonego prostymi $l,  k\mathrm{i}\mathrm{o}\mathrm{s}\mathrm{i}\otimes$

$0x$ byfo dwa razy większe $\mathrm{n}\mathrm{i}\dot{\mathrm{z}}$ pole trójk$\Phi$ta ograniczonego tymi prostymi $\mathrm{i}\mathrm{o}\mathrm{s}\mathrm{i}_{\Phi}0y.$

5. Trójkąt równoboczny $ABC0$ boku długości $a$ zgięto wzdluz wysokości $CD$ pod pew-

nym kątem, otrzymujqc $\mathrm{w}$ ten sposób czworościan ABCD. Obliczyč objętośč $\mathrm{i}$ pole

powierzchni calkowitej tego czworościanu wiedząc, $\dot{\mathrm{z}}\mathrm{e}$ tangens kąta nachylenia ściany

$ABC$ do podstawy czworościanu równy jest $\sqrt{6}.$

6. Punkt $(0,2)$ jest środkiem symetrii wykresu funkcji

$\mathrm{i}b$ wiedząc, $\dot{\mathrm{z}}\mathrm{e} f(a)=0.$

$f(x)=x(|x|-2a)+b$. Wyznaczyč $a$
\end{document}
