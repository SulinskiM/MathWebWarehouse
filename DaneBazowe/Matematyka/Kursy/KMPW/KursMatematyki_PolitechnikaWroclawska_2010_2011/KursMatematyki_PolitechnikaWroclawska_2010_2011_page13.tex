\documentclass[a4paper,12pt]{article}
\usepackage{latexsym}
\usepackage{amsmath}
\usepackage{amssymb}
\usepackage{graphicx}
\usepackage{wrapfig}
\pagestyle{plain}
\usepackage{fancybox}
\usepackage{bm}

\begin{document}

PRACA KONTROLNA nr 7- POZIOM ROZSZERZONY

l. Rozwiązač równanie

$\sqrt{8+2x-x^{2}}=2x-5.$

Zilustrowač je odpowiednim wykresem.

2. Wyznaczyč wszystkie wartości parametru rzeczywistego $p$, dla których rozwiqzania

ukfadu równań

$\left\{\begin{array}{l}
px+2y=p- 2\\
2x+py=p-1
\end{array}\right.$

są zawarte $\mathrm{w}$ kwadracie $K=\{(x,y):|x|+|y|\leq 1\}.$

3. Bok $AB$ trójkąta równoramiennego $ABC\mathrm{l}\mathrm{e}\dot{\mathrm{z}}\mathrm{y}$ na prostej $l$ : $x-3y-4 = 0$. Punkt

$D(4,0)$ jest spodkiem wysokości tego trójkąta, a $S(2,1)$ środkiem boku $AC$. Wyznaczyč

wspólrzędne wierzcholka $B$. Sporządzič rysunek.

4. Podstawą ostrosfupa $0$ wysokości $h$ jest trójk$\Phi$t $\mathrm{P}^{\mathrm{r}\mathrm{o}\mathrm{s}\mathrm{t}\mathrm{o}\mathrm{k}}\Phi^{\mathrm{t}\mathrm{n}\mathrm{y}\mathrm{o}}$ kącie ostrym $\alpha$. Wszystkie

ściany boczne ostrosłupa są nachylone do podstawy pod kątem $\alpha$, a pole powierzchni

całkowitej jest czterokrotnie większe od pola podstawy. Obliczyč objętośč ostroslupa.

Wynik podač $\mathrm{w}$ najprostszej postaci.

5. Rozwiązač nierównośč

sin2 {\it x}$+$ -csoins42 {\it xx}$+$ -csoins64 {\it xx}$+$ -csoins86 {\it xx}$+$... $\geq$ -83,

$\mathrm{w}$ której lewa strona jest sumą nieskończonego ciągu geometrycznego.

6. Jednym $\mathrm{z}$ pierwiastków wielomianu $w(x)=ax^{3}+bx^{2}+cx+d$ jest liczba $-1$. Znalez$\acute{}$č

pozostafe pierwiastki wiedząc, $\dot{\mathrm{z}}\mathrm{e}w(1)=-2\mathrm{i}$ środkiem symetrii wykresu funkcji $w(x)$

jest punkt $S(\displaystyle \frac{1}{4},\frac{5}{2})$. Nie prowadząc dodatkowego badania, sporządzič wykres funkcji

$w(x)$. Dobrač odpowiednio jednostki na osiach ukfadu.
\end{document}
