\documentclass[a4paper,12pt]{article}
\usepackage{latexsym}
\usepackage{amsmath}
\usepackage{amssymb}
\usepackage{graphicx}
\usepackage{wrapfig}
\pagestyle{plain}
\usepackage{fancybox}
\usepackage{bm}

\begin{document}

XL

KORESPONDENCYJNY KURS

Z MATEMATYKI

kwiecień 2011 r.

PRACA KONTROLNA $\mathrm{n}\mathrm{r} 8-$ POZIOM PODSTAWOWY

l. Uprościč wyrazenie

$a(x)= (\displaystyle \frac{x+1}{x-2}-\frac{x^{3}+8}{x^{3}-8}\frac{x^{2}+2x+4}{x^{2}-4})$ : $\displaystyle \frac{1}{x-2}$

$\mathrm{i}$ rozwiązač nierównośč $|\alpha(x)|<1.$

2. Trzech robotników ma wykonač $\mathrm{P}^{\mathrm{e}\mathrm{w}\mathrm{n}}\Phi$ pracę. Wiadomo, $\dot{\mathrm{z}}\mathrm{e}$ pierwszy $\mathrm{i}$ drugi robotnik,

pracując razem, wykonaliby calą pracę $\mathrm{w}$ czasie $n\mathrm{d}\mathrm{n}\mathrm{i}$, drugi $\mathrm{i}$ trzeci-w czasie $m\mathrm{d}\mathrm{n}\mathrm{i}, \mathrm{a}$

pierwszy $\mathrm{i}$ trzeci-w czasie $k\mathrm{d}\mathrm{n}\mathrm{i}$. Ile dni potrzebuje $\mathrm{k}\mathrm{a}\dot{\mathrm{z}}\mathrm{d}\mathrm{y}\mathrm{z}$ robotników na samodzielne

wykonanie cafej pracy?

3. Dla jakich $\alpha\in [0,2\pi$) równanie kwadratowe $\cos\alpha\cdot x^{2}-2x+2\cos\alpha-1=0$ ma dwa

rózne pierwiastki?

4. Wierzcholkami czworokąta są punkty, których współrzędne spelniają układ równań

$\left\{\begin{array}{l}
xy+x-y\\
x^{2}-xy+y^{2}
\end{array}\right.$

1,

1.

Obliczyč pole czworokąta oraz wyznaczyč równanie okręgu na nim opisanego.

5. Pole powierzchni bocznej ostrosłupa prawidlowego czworokątnego jest 2 razy większe $\mathrm{n}\mathrm{i}\dot{\mathrm{z}}$

pole podstawy. Wyznaczyč cosinusy kątów dwuściennych przy krawędzi podstawy oraz

krawędzi bocznej. Sporządzič staranny rysunek.

6. Dany jest stozek ściety, $\mathrm{w}$ którym pole dolnej podstawy jest 4 razy wieksze od po1a górnej.

$\mathrm{W}$ stozek wpisano walec $\mathrm{t}\mathrm{a}\mathrm{k}, \dot{\mathrm{z}}\mathrm{e}$ dolna podstawa walca $\mathrm{l}\mathrm{e}\dot{\mathrm{z}}\mathrm{y}$ na dolnej podstawie stozka,

a brzeg górnej podstawy walca $\mathrm{l}\mathrm{e}\dot{\mathrm{z}}\mathrm{y}$ na powierzchni bocznej stozka. Jaką częśč objetości

stozka ściętego stanowi objętośč walca, $\mathrm{j}\mathrm{e}\dot{\mathrm{z}}$ eli wysokośč walca jest 3 razy mniejsza od

wysokości stozka? Odpowied $\acute{\mathrm{z}}$ podač $\mathrm{w}$ procentach $\mathrm{z}$ dokladnością do jednego promila.

Sporządzič staranny rysunek przekroju osiowego bryły.
\end{document}
