\documentclass[a4paper,12pt]{article}
\usepackage{latexsym}
\usepackage{amsmath}
\usepackage{amssymb}
\usepackage{graphicx}
\usepackage{wrapfig}
\pagestyle{plain}
\usepackage{fancybox}
\usepackage{bm}

\begin{document}

PRACA KONTROLNA nr 8- POZIOM ROZSZERZONY

l. Rozwiązač nierównośč

2. Rozwiązač ukfad równań

$\displaystyle \frac{1}{x^{2}-2x-3}\geq\frac{1}{|x-2|+3}.$

$\left\{\begin{array}{l}
x^{2}+y^{2}=8,\\
- x1+-y1=1.
\end{array}\right.$

Obliczyč pole wielokąta $0$ wierzchofkach, których wspólrzędne spefniają powyzszy ukfad.

Podač ilustrację graficzną tego układu.

3. Wyznaczyč wszystkie wartości parametru $\alpha\in[-\pi,\pi$), dla których równanie kwadratowe

$(\sin 4\alpha)x^{2}-2(\cos\alpha)x+\sin 2\alpha=0$

ma dwa rózne nieujemne pierwiastki rzeczywiste. Rozwiązania zaznaczyč na kole trygo-

nometrycznym.

4. Udowodnič, $\dot{\mathrm{z}}\mathrm{e}\mathrm{j}\mathrm{e}\dot{\mathrm{z}}$ eli liczby rzeczywiste $a, b, c$ spełniajq warunki $a^{2}+b^{2}= (a+b-c)^{2}$

oraz $b, c\neq 0$, to

--{\it ab}22$++$(({\it ba}--{\it cc}))22$=$--{\it ab}--{\it cc}.

5. Trójkąt równoboczny $ABC0$ boku $a$ wpisano $\mathrm{w}$ okrąg. Na fuku $BC$ wybrano punkt

$D\mathrm{t}\mathrm{a}\mathrm{k}, \dot{\mathrm{z}}\mathrm{e}$ proste AB $\mathrm{i}CD$ przecinają się $\mathrm{w}$ punkcie $E\mathrm{i} |BE| = 2a$. Obliczyč pole $S$

czworokąta ABCD $\mathrm{i}$ wykazač, $\displaystyle \dot{\mathrm{z}}\mathrm{e}S=\frac{1}{4}(|BD|+|CD|)^{2}\sqrt{3}.$

6. Rozwinięcie, powierzchni, bocznej, stozka, ściętego, opisanego na kuli jest przedstawione

na rysunku. Obliczyč objętośč tego
\begin{center}
\includegraphics[width=61.416mm,height=51.456mm]{./KursMatematyki_PolitechnikaWroclawska_2010_2011_page15_images/image001.eps}
\end{center}
$\alpha$

{\it b}

stozka sciętego $\mathrm{i}$ promien kuli opisa-

nej na nim. Podac wynik liczbowy dla

$\displaystyle \alpha=\frac{\pi}{4}, b=4$ cm.
\end{document}
