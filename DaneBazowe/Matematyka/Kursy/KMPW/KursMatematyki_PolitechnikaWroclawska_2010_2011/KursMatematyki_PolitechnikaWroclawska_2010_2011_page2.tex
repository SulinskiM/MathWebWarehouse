\documentclass[a4paper,12pt]{article}
\usepackage{latexsym}
\usepackage{amsmath}
\usepackage{amssymb}
\usepackage{graphicx}
\usepackage{wrapfig}
\pagestyle{plain}
\usepackage{fancybox}
\usepackage{bm}

\begin{document}

XL

KORESPONDENCYJNY KURS

Z MATEMATYKI

$\mathrm{p}\mathrm{a}\acute{\mathrm{z}}$dziernik 2010 $\mathrm{r}.$

PRACA KONTROLNA $\mathrm{n}\mathrm{r} 2-$ POZIOM PODSTAWOWY

l. Niech $A=\displaystyle \{x\in \mathbb{R}:\frac{1}{x^{2}+23}\geq\frac{1}{10x}\}$ oraz $B=\displaystyle \{x\in \mathbb{R}:|x-2|<\frac{7}{2}\}.$

Zbiory $A, B, A\cup B, A\cap B, A\backslash B\mathrm{i}B\backslash A$ zapisač $\mathrm{w}$ postaci przedzialów liczbowych $\mathrm{i}$

zaznaczyč je na osi liczbowej.

2. Zaznaczyč na pfaszczy $\acute{\mathrm{z}}\mathrm{n}\mathrm{i}\mathrm{e}$ zbiory

$A=\{(x,y):|x|+|y|\leq 2\}$ oraz

$\mathrm{i}$ obliczyč pole zbioru $A\cap B.$

$B=\displaystyle \{(x,y):\frac{1}{|x-1|}\leq\frac{1}{|x+3|},\frac{2}{|y-1|}\geq 1\}$

3. Trójmian kwadratowy $f(x)=ax^{2}+bx+c$ przyjmuje najmniejszą wartośč równą $-1 \mathrm{w}$

punkcie $x=1$ a reszta $\mathrm{z}$ dzielenia tego trójmianu przez dwumian $(x-2)$ równa jest l.

Wyznaczyč wspólczynniki $a, b, c$. Narysowač staranny wykres funkcji $g(x) = f(|x|) \mathrm{i}$

wyznaczyč najmniejszą $\mathrm{i}$ największą wartośč tej funkcji na przedziale [$-1,3].$

4. Tangens kąta ostrego $\alpha$ równy jest $\displaystyle \frac{a}{b}$, gdzie

$\alpha=(\sqrt{2+\sqrt{3}}-\sqrt{2-\sqrt{3}})^{2}b=(\sqrt{\sqrt{2}+1}-\sqrt{\sqrt{2}-1})^{2}$

Wyznaczyč wartości pozostalych funkcji trygonometrycznych tego $\mathrm{k}_{\Phi^{\mathrm{t}\mathrm{a}}}$. Wykorzystując

wzór $\sin 2\alpha=2\sin\alpha\cos\alpha$, obliczyč miarę kąta $\alpha.$

5. Narysowač wykres funkcji $f(x)=\sqrt{4x^{2}-4x+1}-x \mathrm{i}$ rozwiązač nierównośč $f(x)<0.$

$\mathrm{W}$ zalezności od parametru $m$ określič liczbę rozwiązań równania $|f(x)| = m$. Dla

jakiego $a$ pole trójkqta ograniczonego osia $Ox\mathrm{i}$ wykresem funkcji $g(x)=f(x)-a$ równe

jest 6?

6. Niech $f(x)=$

dla

dla

$x\leq 1,$

$x>1.$

a) Narysowač wykres funkcji $f\mathrm{i}$ na jego podstawie wyznaczyč zbiór wartości funkcji.

b) Obliczyč $f(\sqrt{3}-1)$ oraz $f(3-\sqrt{3}).$

c) Rozwiązač nierównośč $2\sqrt{f(x)}\leq 3\mathrm{i}$ zbiór jej rozwiązań zaznaczyč na osi $0x.$
\end{document}
