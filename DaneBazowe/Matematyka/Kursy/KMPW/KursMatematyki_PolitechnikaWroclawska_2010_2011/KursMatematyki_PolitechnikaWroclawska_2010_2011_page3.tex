\documentclass[a4paper,12pt]{article}
\usepackage{latexsym}
\usepackage{amsmath}
\usepackage{amssymb}
\usepackage{graphicx}
\usepackage{wrapfig}
\pagestyle{plain}
\usepackage{fancybox}
\usepackage{bm}

\begin{document}

PRACA KONTROLNA nr 2- POZIOM ROZSZERZONY

l. Rozwiązač nierównośč $\displaystyle \frac{1}{\sqrt{5+4x-x^{2}}}\geq\frac{1}{x-2} \mathrm{i}$ zbiór rozwiązań zaznaczyč $\mathrm{n}\mathrm{a}$ prostej.

2. Niech $A=\{(x,y):y\geq||x-2|-1|\}, B=\{(x,y):y+\sqrt{4x-x^{2}-3}\leq 2\}.$

Narysowač $\mathrm{n}\mathrm{a}\mathrm{p}${\it l}aszczy $\acute{\mathrm{z}}\mathrm{n}\mathrm{i}\mathrm{e}$ zbiór $A\cap B\mathrm{i}$ obliczyč jego pole.

3. Dla jakich wartości rzeczywistego parametru $p$ równanie $(p-1)x^{4}+(p-2)x^{2}+p=0$

ma dokladnie $\mathrm{d}\mathrm{w}\mathrm{a}$ rózne pierwiastki?

4. Znalez/č wszystkie wartości parametru rzeczywistego $m, \mathrm{d}\mathrm{l}\mathrm{a}$ których pierwiastki trójmia-

nu kwadratowego $f(x)=(m-2)x^{2}-(m+1)x-m$ spełniają nierównośč $|x_{1}|+|x_{2}|\leq 1.$

5. Narysowač staranny wykres funkcji

$f(x)=\{$

$\sqrt{x^{2}-4x+4}-1$

$-\sqrt{4x-x^{2}-3}$

, gdy

, gdy

$|x-2|\geq 1,$

$|x-2|\leq 1.$

$\mathrm{i}$ rozwiązač nierównośč $|f(x)| > \displaystyle \frac{1}{2}. \mathrm{W}$ zalezności od parametru $m$ określič liczbę roz-

wiązań równania $|f(x)| =m$. Obliczyč pole obszaru ograniczonego wykresem funkcji

$g(x)=|f(x)|\mathrm{i}$ prostą $y=\displaystyle \frac{1}{2}.$

6. Niech

$f(x)=$

gdy

gdy

$|x-1|\geq 1,$

$|x-1|<1.$

a) Obliczyč $f(-\displaystyle \frac{2}{3}), f(\displaystyle \frac{1+\sqrt{3}}{2})$ oraz $f(\pi-1).$

b) Narysowač wykres funkcji $f\mathrm{i}$ na jego podstawie podač zbiór wartości funkcji.

c) Rozwi$\mathfrak{B}$ač nierównośč $f(x)\displaystyle \geq-\frac{1}{2}\mathrm{i}$ zaznaczyč na osi $0x$ zbiór jej rozwi$\Phi$zań.
\end{document}
