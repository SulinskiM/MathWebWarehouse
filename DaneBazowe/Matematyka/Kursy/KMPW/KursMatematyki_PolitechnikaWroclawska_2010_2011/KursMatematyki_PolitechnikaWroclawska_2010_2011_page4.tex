\documentclass[a4paper,12pt]{article}
\usepackage{latexsym}
\usepackage{amsmath}
\usepackage{amssymb}
\usepackage{graphicx}
\usepackage{wrapfig}
\pagestyle{plain}
\usepackage{fancybox}
\usepackage{bm}

\begin{document}

XL

KORESPONDENCYJNY KURS

Z MATEMATYKI

listopad 2010 r.

PRACA KONTROLNA $\mathrm{n}\mathrm{r} 3-$ POZIOM PODSTAWOWY

1. $\mathrm{W}$ trójkącie prostokątnym $0$ kącie prostym przy wierzchoku $C$ na przedluzeniu przeciw-

prostokqtnej $AB$ odmierzono odcinek $BD\mathrm{t}\mathrm{a}\mathrm{k}, \dot{\mathrm{z}}\mathrm{e}|BD|=|BC|$. Wyznaczyč $|CD|$ oraz

obliczyč pole trójkta $\triangle ACD, \mathrm{j}\mathrm{e}\dot{\mathrm{z}}$ eli $|BC|=5, |AC|=12.$

2. Harcerze rozbili 2 namioty, jeden $\mathrm{w}$ odległości 5 $\mathrm{m}$, drugi - 17 $\mathrm{m}$ od prostoliniowego

brzegu rzeki. Odlegfośč między namiotami równajest 13 $\mathrm{m}. \mathrm{W}$ którym miejscu na samym

brzegu rzeki (liczqc od punktu brzegu będacego rzutem prostopadłym punktu połozenia

pierwszego namiotu) powinni umieścič maszt $\mathrm{z}\mathrm{f}\mathrm{l}\mathrm{a}\mathrm{g}\Phi$ zastępu, by odlegfośč od masztu do

$\mathrm{k}\mathrm{a}\dot{\mathrm{z}}$ dego $\mathrm{z}$ namiotów była taka sama?

3. Na kole $0$ promieniu $r$ opisano trapez równoramienny, $\mathrm{w}$ którym stosunek dlugości pod-

staw wynosi 4: 3. Ob1iczyč stosunek po1a ko1a do po1a trapezu oraz cosinus kąta ostrego

$\mathrm{w}$ tym trapezie.

4. Wielomian $W(x) =x^{3}-x^{2}+bx+c$ jest podzielny przez $(x+3)$, a reszta $\mathrm{z}$ dzielenia

tego wielomianu przez $(x-3)$ równa jest 6. Wyznaczyč $b\mathrm{i} c$, a następnie rozwiązač

nierównośč $(x+1)W(x-1)-(x+2)W(x-2)\leq 0.$

5. Wykonač dziafania $\mathrm{i}$ zapisač $\mathrm{w}$ najprostszej postaci wyrazenie

$s(a,b)= (\displaystyle \frac{a^{2}+b^{2}}{a^{2}-b^{2}}-\frac{a^{3}+b^{3}}{a^{3}-b^{3}})$ : $(\displaystyle \frac{a^{2}}{a^{3}-b^{3}}-\frac{a}{a^{2}+ab+b^{2}})$

Wyznaczyč wysokośč trójkąta prostokątnego wpisanego $\mathrm{w}$ okrąg $0$ promieniu 6 opusz-

czoną $\mathrm{z}$ wierzcholka kąta prostego wiedząc, $\dot{\mathrm{z}}\mathrm{e}$ tangens jednego $\mathrm{z}$ kątów ostrych tego

trójkąta równy jest $s(\sqrt{5}+\sqrt{3},\sqrt{5}-\sqrt{3}).$

6. $\mathrm{W}$ trójkącie $ABC$ dane są $\angle CAB= \displaystyle \frac{\pi}{3}$, wysokośč $|CD| =h=5$ oraz $BD=d=\sqrt{2}.$

Obliczyč odległośč środków okręgów wpisanych $\mathrm{w}$ trójkąty ADC $\mathrm{i}\mathrm{D}\mathrm{B}\mathrm{C}.$
\end{document}
