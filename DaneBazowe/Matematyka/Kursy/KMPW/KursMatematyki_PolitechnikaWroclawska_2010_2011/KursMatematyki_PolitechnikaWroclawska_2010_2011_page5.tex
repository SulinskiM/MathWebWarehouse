\documentclass[a4paper,12pt]{article}
\usepackage{latexsym}
\usepackage{amsmath}
\usepackage{amssymb}
\usepackage{graphicx}
\usepackage{wrapfig}
\pagestyle{plain}
\usepackage{fancybox}
\usepackage{bm}

\begin{document}

PRACA KONTROLNA nr 3- POZIOM ROZSZERZONY

l. Dany jest wielomian $W(x) = x^{3}+ax+b$, gdzie $b \neq 0$. Wykazač, $\dot{\mathrm{z}}\mathrm{e} W(x)$ posiada

pierwiastek podwójny wtedy $\mathrm{i}$ tylko wtedy, gdy spelniony jest warunek $4a^{3}+27b^{2}=0.$

Wyrazič pierwiastki za pomocą współczynnika $b.$

2. Wyznaczyč promień okręgu opisanego na czworokqcie ABCD, $\mathrm{w}$ którym $\mathrm{k}\mathrm{a}\mathrm{t}$ przy wierz-

chofku $A$ ma miarę $\alpha$, kąty przy wierzchofkach $B, D$ są proste oraz $|BC|=a, |AD|=b.$

Sporządzič staranny rysunek.

3. Narysowač staranny wykres funkcji $f(x)=\displaystyle \frac{\sin 2x-|\sin x|}{\sin x}.$

$\mathrm{W}$ przedziale $[0,\pi]$ wyznaczyč $\mathrm{r}\mathrm{o}\mathrm{z}\mathrm{w}\mathrm{i}_{\Phi}$zania nierówności $f(x)<2(\sqrt{2}-1)\cos^{2}x.$

4. $\mathrm{Z}$ wierzchołka $A$ kwadratu ABCD $0$ boku $a$ poprowadzono dwie proste, które dzielą kąt

przy tym wierzchołku na trzy równe części $\mathrm{i}$ przecinają boki kwadratu $\mathrm{w}$ punktach $K\mathrm{i}$

$L$. Wyznaczyč dfugości odcinków, na jakie te proste dzielą $\mathrm{P}^{\mathrm{r}\mathrm{z}\mathrm{e}\mathrm{k}}\Phi^{\mathrm{t}\mathrm{n}}\Phi$ kwadratu. Znalez/č

promień okręgu wpisanego $\mathrm{w}$ deltoid AKCL.

5. Czworokąt wypukły ABCD, $\mathrm{w}$ którym $AB=1, BC=2, CD=4, DA=3$ jest wpisany

$\mathrm{w}$ okrąg. Obliczyč promień $R$ tego okręgu. Sprawdzič, czy $\mathrm{w}$ czworokąt ten $\mathrm{m}\mathrm{o}\dot{\mathrm{z}}$ na wpisač

okrąg. $\mathrm{J}\mathrm{e}\dot{\mathrm{z}}$ eli $\mathrm{t}\mathrm{a}\mathrm{k}$, to obliczyč promień $r$ tego okręgu.

6. Na boku $BC$ trójkąta równobocznego obrano punkt $D\mathrm{t}\mathrm{a}\mathrm{k}, \dot{\mathrm{z}}\mathrm{e}$ promień okręgu wpisanego

$\mathrm{w}$ trójkąt $ADC$ jest dwa razy mniejszy $\mathrm{n}\mathrm{i}\dot{\mathrm{z}}$ promień okręgu wpisanego $\mathrm{w}$ trójkąt $ABD.$

$\mathrm{W}$ jakim stosunku punkt $D$ dzieli bok $BC$?
\end{document}
