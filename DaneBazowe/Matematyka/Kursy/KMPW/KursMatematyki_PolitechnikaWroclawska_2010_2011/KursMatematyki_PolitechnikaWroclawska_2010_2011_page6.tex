\documentclass[a4paper,12pt]{article}
\usepackage{latexsym}
\usepackage{amsmath}
\usepackage{amssymb}
\usepackage{graphicx}
\usepackage{wrapfig}
\pagestyle{plain}
\usepackage{fancybox}
\usepackage{bm}

\begin{document}

XL

KORESPONDENCYJNY KURS

Z MATEMATYKI

grudzień 2010 r.

PRACA KONTROLNA $\mathrm{n}\mathrm{r} 4-$ POZIOM PODSTAWOWY

l. Rozwiązač równanie $\displaystyle \frac{1}{\cos x}+\mathrm{t}\mathrm{g}x-\sin(\frac{\pi}{2}-x)=0$ dla $x\in[-2\pi,2\pi].$

2. Na p{\it l}aszczy $\acute{\mathrm{z}}\mathrm{n}\mathrm{i}\mathrm{e}$ dane są cztery punkty: $A(1,-1), B(5,7), C(4,-4), D(2,4)$. Obliczyč od-

ległośč punktu przecięcia prostych AB $\mathrm{i}CD$ od symetralnej odcinka $BC$. Sporz$\Phi$dzič

rysunek.

3. Rozwiązač uklad równań

$\left\{\begin{array}{l}
y+x^{2}=4\\
4x^{2}-y^{2}+2y=1
\end{array}\right.$

Podač interpretację geometryczną tego ukladu $\mathrm{i}$ wykazač, $\dot{\mathrm{z}}\mathrm{e}$ cztery punkty, które $\mathrm{s}\Phi$

jego rozwiązaniem, wyznaczają na płaszczy $\acute{\mathrm{z}}\mathrm{n}\mathrm{i}\mathrm{e}$ trapez równoramienny. Znalez$\acute{}$č równanie

okręgu opisanego na tym trapezie.

4. $\mathrm{W}$ ostroslupie prawidfowym trójkątnym dlugośč krawędzi podstawy jest równa $a$. Kąt

między krawędzią podstawy, a krawędzią boczną jest równy $\displaystyle \frac{\pi}{4}$. Obliczyč pole przekro-

ju ostrosłupa $\mathrm{p}\mathrm{a}$szczyz $\Phi$ przechodzącą przez krawędz/ podstawy $\mathrm{i}$ środek przeciwleglej

krawędzi bocznej. Sporządzič staranny rysunek.

5. Dane sa dwa okręgi: $K_{1}0$ środku $\mathrm{w}$ punkcie $(0,0)\mathrm{i}$ promieniu 5 $\mathrm{i}K_{2}\mathrm{o}$ równaniu

$x^{2}+6x+y^{2}-12y+5=0$. Obliczyč pole czworokąta wyznaczonego przez środki okręgów

oraz punkty, $\mathrm{w}$ których te okręgi się przecinają. Sporządzič staranny rysunek.

6. Podstawą graniastosłupa jest równolegfobok $0$ bokach dfugości $a\mathrm{i}2a$ oraz kącie ostrym

$\displaystyle \frac{\pi}{3}$. Krótsza przekątna graniastoslupa tworzy $\mathrm{w}$ pfaszczyzną podstawy kąt $\displaystyle \frac{\pi}{6}$. Obliczyč

długośč dłuzszej przekatnej oraz pole powierzchni całkowitej tego graniastoslupa.
\end{document}
