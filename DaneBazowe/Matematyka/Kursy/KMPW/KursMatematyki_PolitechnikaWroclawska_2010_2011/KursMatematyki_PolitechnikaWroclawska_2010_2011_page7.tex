\documentclass[a4paper,12pt]{article}
\usepackage{latexsym}
\usepackage{amsmath}
\usepackage{amssymb}
\usepackage{graphicx}
\usepackage{wrapfig}
\pagestyle{plain}
\usepackage{fancybox}
\usepackage{bm}

\begin{document}

PRACA KONTROLNA nr 4- POZIOM ROZSZERZONY

l. Rozwiązač równanie 2 $\sin^{2}x-2\sin x\cos 2x=1.$

2. Dane są dwa wektory $\vec{a}= [2,-3]$ oraz $\vec{b}= [-1,4]$. Pokazač, $\dot{\mathrm{z}}\mathrm{e}$ wektor $\vec{AB}=3\text{{\it ã}}+2\vec{b}$

jest prostopadły do wektora $\vec{BC}=8\text{{\it ã}}+11\vec{b}$. Obliczyč dlugośč środkowej trójkąta $ABC$

rozpiętego na wektorach $\vec{AB}\mathrm{i}\vec{BC}$, poprowadzonej $\mathrm{z}$ wierzchołka $B.$

3. Niech $K$ będzie wierzchofkiem paraboli $f(x)=-\displaystyle \frac{4}{9}x^{2}-\frac{8}{3}x$, a L- wierzcholkiem paraboli

$g(x) = -f(x-7)+7$. Na paraboli $g(x)$ znalez/č taki punkt $N$, aby wektor $\vec{NL}$ był

równolegfy do wektora $\vec{MK}$, gdzie $M=(0,f(0))$. Obliczyč pole czworokąta KMLN.

4. Przekrój sześcianu płaszczyznq jest sześciokątem foremnym. Wyznaczyč kąt nachylenia

tej pfaszczyzny do pfaszczyzny podstawy sześcianu oraz obliczyč pole tego przekroju.

Wykonač odpowiedni rysunek.

5. Dane są dwa okręgi: $K_{1} 0$ środku $\mathrm{w}$ punkcie $P(1,1) \mathrm{i}$ promieniu l oraz $K_{2} 0$ środku

$Q(9,5) \mathrm{i}$ promieniu 3. Zna1ez/č punkt $S$ na odcinku $\overline{PQ}$ oraz dobrač skalę $k\mathrm{t}\mathrm{a}\mathrm{k}$, aby

okrąg $K_{2}$ był obrazem okręgu $K_{1} \mathrm{w}$ jednokładności $0$ środku $S\mathrm{i}$ skali $k$. Wyznaczyč

równania prostych, które są styczne jednocześnie do obu okręgów $\mathrm{i}\mathrm{p}\mathrm{r}\mathrm{z}\mathrm{e}\mathrm{c}\mathrm{h}\mathrm{o}\mathrm{d}\mathrm{z}\Phi$ przez

punkt $S.$

6. $\mathrm{W}$ ostrosłupie prawidłowym czworokatnym pole $\mathrm{k}\mathrm{a}\dot{\mathrm{z}}$ dej $\mathrm{z}$ pięciu ścian jest równe l. Ostro-

slup ten ścięto $\mathrm{w}$ polowie wysokości $\mathrm{p}^{\mathrm{f}\mathrm{n}}$aszczyz $\Phi$ równolegfą do podstawy. Obliczyč ob-

jętośč oraz pole powierzchni całkowitej otrzymanego ostrosłupa ściętego. Wykonač od-

powiedni rysunek.
\end{document}
