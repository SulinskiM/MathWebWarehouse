\documentclass[a4paper,12pt]{article}
\usepackage{latexsym}
\usepackage{amsmath}
\usepackage{amssymb}
\usepackage{graphicx}
\usepackage{wrapfig}
\pagestyle{plain}
\usepackage{fancybox}
\usepackage{bm}

\begin{document}

XL

KORESPONDENCYJNY KURS

Z MATEMATYKI

styczeń 2011 r.

PRACA KONTROLNA $\mathrm{n}\mathrm{r} 5-$ POZIOM PODSTAWOWY

1. $\mathrm{W}$ ciagu arytmetycznym suma poczatkowych dwudziestu jeden wyrazów wynosi $21\sqrt{2},$

a jego $\mathrm{d}\mathrm{z}\mathrm{i}\mathrm{e}\mathrm{s}\mathrm{i}_{\Phi}\mathrm{t}\mathrm{y}$ wyraz równy jest $-2-2\sqrt{2}$. Wyznaczyč najmniejszy dodatni wyraz

tego ciągu.

2. Rozwiazač nierównośč

$-2<\log_{\frac{1}{2}}(5x+2)\leq 2.$

3. Firmy X $\mathrm{i}\mathrm{Y}$ jednocześnie rozpoczęly dzialalnośč. $\mathrm{W}$ pierwszym miesiącu $\mathrm{k}\mathrm{a}\dot{\mathrm{z}}$ da $\mathrm{z}$ nich

miała dochód równy 50000 zf. Po pięciu miesiącach okazało sie, $\dot{\mathrm{z}}\mathrm{e}$ dochód firmy X

rósł $\mathrm{z}$ miesiąca na miesiąc $0$ tę samą kwote, a dochód firmy $\mathrm{Y}$ wzrastał co miesiąc

geometrycznie. $\mathrm{W}$ drugim $\mathrm{i}$ trzecim miesiącu działalnosci firma X miała dochód wiekszy

od dochodu firmy $\mathrm{Y} 0$ 2000 $\mathrm{z}l$. Ustalič, która $\mathrm{z}$ firm miala wiekszą sumę dochodów

$\mathrm{w}$ pierwszych pięciu miesiącach swojej dzialalności.

4. Sporządzič staranny wykres funkcji (za jednostkę przyjąč 2 cm)

$f(x)=(-2x^{2}+3x\displaystyle \frac{|x|}{1-x}$

dla

dla

$|x-1|\geq 1,$

$|x-1|<1.$

Korzystajqc $\mathrm{z}$ niego, określič ilośč rozwiazań równania $f(x) =m \mathrm{w}$ zalezności od rze-

czywistego parametru $m.$

5. Stosując wzór na sinus podwojonego kata oraz wzory redukcyjne, obliczyč wartośč wy-

$\mathrm{r}\mathrm{a}\dot{\mathrm{z}}$ enia

$\displaystyle \cos\frac{\pi}{5}\cdot\cos\frac{2\pi}{5}\cdot\cos\frac{3\pi}{5}\cdot\cos\frac{4\pi}{5}.$

6. Wiedząc, $\dot{\mathrm{z}}\mathrm{e} \displaystyle \sin\frac{\pi}{10} = \displaystyle \frac{1}{4}(\sqrt{5}-1)$, wyznaczyč wszystkie kąty $\alpha \in [0,\pi]$, dla których

spefnione jest równanie

$2^{2+\sin\alpha}=\sqrt{2}\cdot 4^{\cos^{2}\alpha}$
\end{document}
