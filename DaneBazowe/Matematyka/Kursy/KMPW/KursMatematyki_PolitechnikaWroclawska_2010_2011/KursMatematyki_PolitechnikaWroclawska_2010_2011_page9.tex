\documentclass[a4paper,12pt]{article}
\usepackage{latexsym}
\usepackage{amsmath}
\usepackage{amssymb}
\usepackage{graphicx}
\usepackage{wrapfig}
\pagestyle{plain}
\usepackage{fancybox}
\usepackage{bm}

\begin{document}

PRACA KONTROLNA nr 5- POZIOM ROZSZERZONY

l. Zaznaczyč na osi liczbowej zbiór rozwiązań nierówności

$\displaystyle \frac{2x-\sqrt{2-x}}{x}\geq x.$

2. Wyznaczyč wszystkie liczby rzeczywiste x, dla których funkcja

$f(x)=2^{x^{2}+2}-2^{x^{2}-1}-2\cdot 7^{x^{2}-1}$

przyjmuje wartości dodatnie.

3. Określič dziedzinę $\mathrm{i}$ sporządzič staranny wykres funkcji $f(x) = 1-\log_{3}(1-x)$. Za

jednostkę przyj$\Phi$č 2 cm. Zna1ez/č obraz tego wykresu $\mathrm{w}$ symetrii osiowej względem prostej

$x=y\mathrm{i}$ podač wzór funkcji, której wykresem jest nowo powstala krzywa.

4. Rozwiązač nierównośč

$\sqrt{\log_{2}(x^{2}-1)}>\log_{2}\sqrt{x^{2}-1}.$

5. Niech $c>0\mathrm{i}c\neq 1$. Znalez/č liczbę naturalną $m$, dla ktorej suma $m$ poczatkowych wyra-

zów ciągu arytmetycznego $a_{n}=\log_{2}(c^{n})$, jest 10100 razy większa od sumy wszystkich

wyrazów ciągu geometrycznego $b_{n}=\log_{2^{3^{n}}}(c).$

6. Korzystajqc ze wzoru

$\sin 5\alpha=5\sin\alpha-20\sin^{3}\alpha+16\sin^{5}\alpha,$

obliczyč wartośč $\displaystyle \sin\frac{\pi}{5}$. Podač wartości wyrazeń $\displaystyle \cos\frac{\pi}{5}, \displaystyle \sin\frac{\pi}{10}$ oraz $\displaystyle \cos\frac{\pi}{10}$. Wyprowa-

dzič wzór na pole dwudziestokąta foremnego wpisanego $\mathrm{w}$ okrąg $0$ promieniu $r.$
\end{document}
