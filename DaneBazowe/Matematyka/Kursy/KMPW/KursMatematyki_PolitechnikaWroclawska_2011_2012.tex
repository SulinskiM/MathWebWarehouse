\documentclass[a4paper,12pt]{article}
\usepackage{latexsym}
\usepackage{amsmath}
\usepackage{amssymb}
\usepackage{graphicx}
\usepackage{wrapfig}
\pagestyle{plain}
\usepackage{fancybox}
\usepackage{bm}

\begin{document}

XLI

KORESPONDENCYJNY KURS

Z MATEMATYKI

wrzesień 2011 r.

PRACA KONTROLNA $\mathrm{n}\mathrm{r} 1 -$ POZIOM PODSTAWOWY

l. Średni czas przeznaczony na matematykę na dwunastu wydziałach pewnej uczelni wy-

nosi 240 godzin. Utworzono nowy wydziaf $\mathrm{i}$ wówczas średnia liczba godzin matematyki

wzrosła $0$ 5\%. Ile godzin przeznaczono na matematykę na nowym wydziale?

2. Droge $\mathrm{z}$ miasta $A$ do miasta $B$ rowerzysta pokonuje $\mathrm{w}$ czasie 3 godzin. Po dfugotrwa1ych

deszczach stan $\displaystyle \frac{\mathrm{s}}{5}$ drogi pogorszyl się na tyle, $\dot{\mathrm{z}}\mathrm{e}$ na tym odcinku rowerzysta $\mathrm{m}\mathrm{o}\dot{\mathrm{z}}\mathrm{e}$ jechač

$\mathrm{z}$ prędkością $04\mathrm{k}\mathrm{m}/\mathrm{h}$ mniejszą. By czas podrózy $\mathrm{z}A$ do $B$ nie uległ zmianie, zmuszony

jest na pozostafym odcinku zwiększyč prędkośč $012\mathrm{k}\mathrm{m}/\mathrm{h}$. Jaka jest odleglośč $\mathrm{z}A$ do

$B\mathrm{i}\mathrm{z}$ jaką prędkością jez/dził rowerzysta przed ulewami?

3. Trzy klasy pewnego gimnazjum wyjechafy na zieloną szkolę. $K\mathrm{a}\dot{\mathrm{z}}\mathrm{d}\mathrm{y}$ uczeń $\mathrm{z}$ klasy $\mathrm{A}$

wyslaf tę $\mathrm{s}\mathrm{a}\mathrm{m}\Phi$ liczbę SMS-ów. $\mathrm{W}$ klasie $\mathrm{B}$ wysfano taką samą liczbę SMS-ów, ale liczba

uczniów byla $01$ mniejsza, a $\mathrm{k}\mathrm{a}\dot{\mathrm{z}}\mathrm{d}\mathrm{y}\mathrm{z}$ nich wyslał $02$ SMS-y więcej. $\mathrm{Z}$ kolei klasa $\mathrm{C}, \mathrm{w}$

której było $0$ dwóch uczniów więcej $\mathrm{i}\mathrm{k}\mathrm{a}\dot{\mathrm{z}}\mathrm{d}\mathrm{y}$ wysłaf $05$ SMS-ów więcej, wysfała $\mathrm{w}$ sumie

dwa razy więcej wiadomości. Ilu uczniów bylo na zielonej szkole $\mathrm{i}$ ile SMS-ów wyslali?

4. Ile jest czterocyfrowych liczb naturalnych:

a) podzielnych przez 4 $\mathrm{i}$ przez 5?

b) podzielnych przez 41ub przez 5?

c) podzielnych przez 4 $\mathrm{i}$ niepodzielnych przez 5?

5. Umowa określa wynagrodzenie miesięczne pana Kowalskiego na kwotę 4000 $\mathrm{z}\mathrm{f}$. Skfadka

na ubezpieczenie społeczne wynosi 18, 7\% tej kwoty, a składka na ubezpieczenie zdrowot-

ne- 7, 75\% kwoty pozosta1ej po od1iczeniu skfadki na ubezpieczenie społeczne. $\mathrm{W}$ celu

obliczenia podatku nalez $\mathrm{y}$ od 80\% wyjściowej kwoty umowy odjąč składkę na ubezpie-

czenie społeczne $\mathrm{i}$ wyznaczyč 19\% pozostałej sumy. Podatek jest róznicą tak otrzymanej

kwoty $\mathrm{i}$ skfadki na ubezpieczenie zdrowotne. Ile zfotych miesięcznie otrzymuje pan Ko-

walski? Jakie powinno byč jego wynagrodzenie, by co miesiąc dostawal przynajmniej

3000 $\mathrm{z}l$?

6. Uprościč wyrazenie (dla $x, y$, dla których ma ono sens)

-{\it xx}--23{\it y}-21-{\it y}--3221--{\it xx}3231{\it yy}--2331

( -{\it xy})- -32

$\mathrm{i}$ następnie obliczyč jego wartośč dla $x=1+\sqrt{2}, y=7+5\sqrt{2}.$




PRACA KONTROLNA nr l- POZIOM ROZSZERZONY

l. Wiek ojca jest $05$ lat większy $\mathrm{n}\mathrm{i}\dot{\mathrm{z}}$ suma lat trzech jego synów. Za 101at ojciec będzie

2 razy starszy od swego najstarszego syna, za 20 lat będzie 2 razy starszy od swego

średniego syna, a za 301at będzie 2 razy starszy od swego najmłodszego syna. Kiedy

ojciec byf 3 razy starszy od swego najstarszego syna, a kiedy będzie 3 razy starszy od

swego najmfodszego syna?

2. Dwaj rowerzyści wyruszyli jednocześnie $\mathrm{w}$ drogę, jeden $\mathrm{z}$ A do $\mathrm{B}$, drugi $\mathrm{z}\mathrm{B}$ do A $\mathrm{i}$ minęli

się po godzinie. Pierwszy jechał $\mathrm{z}$ prędkości$\Phi 03$ km większ$\Phi \mathrm{n}\mathrm{i}\dot{\mathrm{z}}$ drugi $\mathrm{i}$ przyjechał do

celu $027$ minut wcześniej. Jakie były prędkości obu rowerzystów $\mathrm{i}$ jaka jest odlegfośč od

A do $\mathrm{B}$ ?

3. Pierwszy $\mathrm{i}$ drugi pracownik $\mathrm{w}\mathrm{y}\mathrm{k}\mathrm{o}\mathrm{n}\mathrm{a}\mathrm{j}_{\Phi}$ wspólnie pewną pracę $\mathrm{w}$ czasie $\mathrm{c} \mathrm{d}\mathrm{n}\mathrm{i}$, drugi $\mathrm{i}$

trzeci-w czasie a $\mathrm{d}\mathrm{n}\mathrm{i}$, zaś pierwszy $\mathrm{i}$ trzeci-w czasie $b\mathrm{d}\mathrm{n}\mathrm{i}$? Ile dni potrzebuje $\mathrm{k}\mathrm{a}\dot{\mathrm{z}}\mathrm{d}\mathrm{y}\mathrm{z}$

pracowników na wykonanie tej pracy samodzielnie?

4. Ile jest liczb pięciocyfrowych podzielnych przez 6, które $\mathrm{w}$ zapisie dziesiętnym mają:

a) obie cyfry 1, 2 $\mathrm{i}$ tylko $\mathrm{t}\mathrm{e}$? b) obie cyfry 2, 3 $\mathrm{i}$ tylko $\mathrm{t}\mathrm{e}$? c) wszystkie cyfry 1, 2, 3

$\mathrm{i}$ tylko $\mathrm{t}\mathrm{e}$? Odpowied $\acute{\mathrm{z}}$ uzasadnič.

5. $\mathrm{W}$ hurtowni znajduje się towar, którego a\% sprzedano $\mathrm{z}$ zyskiem p\%, a b\% pozostałej

części sprzedano $\mathrm{z}$ zyskiem q\%. $\mathrm{Z}$ jakim zyskiem nalezy sprzedač resztę towaru, by

cafkowity zysk wyniósl r\%?

6. Uprościč wyrazenie (dla $x, y$, dla których ma ono sens)

( -{\it y} -21 -{\it y} -{\it x}61 -21 {\it y} -31 - -{\it x} -21 {\it y} -21 {\it x-xy} -31)

[ -{\it x} -21 -1 {\it y} -21 ({\it x} -65 - -{\it xy}-61) - -{\it x} -23 {\it x}$+$-{\it xy}-61 {\it y} -21]

$\mathrm{i}$ następnie obliczyč jego wartośč dla $x=5\sqrt{2}-7, y=7+5\sqrt{2}.$





XLI

KORESPONDENCYJNY KURS

Z MATEMATYKI

luty 2012 r.

PRACA KONTROLNA nr 6- POZIOM PODSTAWOWY

l. Obliczyč, ile jest wszystkich liczb czterocyfrowych, których suma cyfr wynosi 20 $\mathrm{i}$ które

$\mathrm{m}\mathrm{a}\mathrm{j}_{\Phi}$ dokfadnie jedno zero wśród swoich cyfr:

a) $\mathrm{j}\mathrm{e}\dot{\mathrm{z}}$ eli wszystkie cyfry muszą byč rózne,

b) $\mathrm{j}\mathrm{e}\dot{\mathrm{z}}$ eli cyfry mogą powtarzač się.

2. Do ponumerowania wszystkich stron grubej ksiązki zecer $\mathrm{z}\mathrm{u}\dot{\mathrm{z}}$ ył 2989 cyfr. I1e stron ma

ta ksiązka?

3. Zbiory $A, B, C$ są skończone, przy czym

$|A|=10,$

$|B|=9,$

$|A\cap B|=3, |A\cap C|=1,$

$|B\cap C|=1$ oraz

$|A\cup B\cup C|=18.$

Wyznaczyč liczbę elementów zbiorów $A\cap B\cap C$ oraz $C.$

4. Na egzamin $\mathrm{z}$ matematyki przygotowano $\mathrm{i}$ ogloszono 45 zadań. Student nauczył się

rozwiązywač tylko $\displaystyle \frac{2}{3}$ spośród nich. Na egzaminie student losuje trzy zadania. Otrzymuje

ocenę bardzo dobrq za poprawne rozwiqzanie trzech zadań, dobrą za rozwiązanie dwóch,

dostateczną za rozwiązanie jednego $\mathrm{i}$ niedostateczną, gdy nie rozwiąze $\dot{\mathrm{z}}$ adnego zadania.

Jakiejest prawdopodobieństwo, $\dot{\mathrm{z}}\mathrm{e}$ uzyska ocenę co najmniej dostateczną, ajakie- bardzo

dobrq?

5. Udowodnič, $\dot{\mathrm{z}}\mathrm{e}$ dla dowolnej liczby naturalnej $n$ liczba

$\displaystyle \frac{1}{25}\cdot 100^{n}+\frac{2}{5}\cdot 10^{n}+1$

jest kwadratem liczby naturalnej $\mathrm{i}$ jest liczbą podzielną przez 9.

6. $\mathrm{W}$ urnie I są dwie kule biafe $\mathrm{i}$ dwie czarne. $\mathrm{W}$ urnie II jest pięč kul bialych $\mathrm{i}$ trzy

czarne. Rzucamy dwiema kostkami do gry. $\mathrm{J}\mathrm{e}\dot{\mathrm{z}}$ eli iloczyn otrzymanych oczek jest liczbq

$\mathrm{n}\mathrm{i}\mathrm{e}\mathrm{p}\mathrm{a}\mathrm{r}\mathrm{z}\mathrm{y}\mathrm{s}\mathrm{t}_{\Phi}$, to losujemy kulę $\mathrm{z}$ urny I, $\mathrm{w}$ przeciwnym przypadku losujemy kulę $\mathrm{z}$ urny II.

a) Obliczyč prawdopodobieństwo wylosowania kuli czarnej?

b) Ile co najmniej razy nalez $\mathrm{y}$ powtórzyč opisane doświadczenie, aby $\mathrm{z}$ prawdopodo-

bieństwem nie mniejszym $\displaystyle \mathrm{n}\mathrm{i}\dot{\mathrm{z}}\frac{5}{7}$, co najmniej raz wyciągnač kulę białą?





PRACA KONTROLNA nr 6- POZIOM ROZSZERZONY

l. Jest pięč biletów po l zloty, trzy bilety po 3 złote $\mathrm{i}$ dwa bilety po 5 złotych. Wybrano

losowo trzy bilety. Obliczyč prawdopodobieństwo, $\dot{\mathrm{z}}\mathrm{e}:\mathrm{a}$) przynajmniej dwa $\mathrm{z}$ tych biletów

mają jednakową wartośč; b) wybrane bilety mają lączną wartośč 7 złotych.

2. Korzystajqc $\mathrm{z}$ zasady indukcji matematycznej udowodnič, $\dot{\mathrm{z}}\mathrm{e}$ nierównośč

-21

-43

$\displaystyle \frac{2n-1}{2n}<\frac{1}{\sqrt{2n+1}}$

jest prawdziwa dla dowolnej liczby naturalnej $n.$

3. Dwie osoby rzucaj $\Phi$ na przemian $\mathrm{m}\mathrm{o}\mathrm{n}\mathrm{e}\mathrm{t}_{\Phi}$. Wygrywa ta osoba, która pierwsza wyrzuci or-

ła. Obliczyč, ile wynoszą prawdopodobieństwa wygranej dla $\mathrm{k}\mathrm{a}\dot{\mathrm{z}}$ dego $\mathrm{z}$ graczy. Następnie

obliczyč prawdopodobieństwa wygranej obu graczy, gdy rozgrywka została zmieniona

$\mathrm{w}$ następujący sposób: pierwszy gracz rzuca jeden raz $\mathrm{m}\mathrm{o}\mathrm{n}\mathrm{e}\mathrm{t}_{\Phi}$, a potem gracze rzucają

monetą po dwa razy (zaczynając od drugiego gracza), $\mathrm{a}\dot{\mathrm{z}}$ do pierwszego wyrzucenia orla.

4. Ze zbioru liczb naturalnych $n$ spefniających warunek $\displaystyle \frac{1}{\log n}+\frac{\mathrm{l}}{1-\log n}>$ llosujemy kolejno

bez zwracania dwie liczby $\mathrm{i}$ tworzymy $\mathrm{z}$ nich liczbę dwucyfrową, $\mathrm{w}$ której cyfrą dziesiątek

jest pierwsza $\mathrm{z}$ wylosowanych liczb. Sprawdzič niezaleznośč zdarzeń: A- utworzona liczba

jest parzysta, B- utworzona liczba jest podzielna przez 3.

5. Obliczyč, ile liczb mniejszych od l00 nie jest podzielnych przez 2, 3, 5 ani przez 7. Udo-

wodnič, $\dot{\mathrm{z}}\mathrm{e}$ wszystkie te liczby oprócz l są pierwsze. Ile jest liczb pierwszych mniejszych

od 100?

6. Dla $\mathrm{k}\mathrm{a}\dot{\mathrm{z}}$ dej druzyny pilkarskiej biorącej udział $\mathrm{w}$ Euro 2012 eksperci wyznaczy1i współ-

czynnik $p$ oznaczaj $\Phi^{\mathrm{c}\mathrm{y}}$ prawdopodobieństwo, $\dot{\mathrm{z}}\mathrm{e}$ Polska pokona tę druzynę. Druzyny po-

dzielono na cztery koszyki. $\mathrm{Z} \mathrm{k}\mathrm{a}\dot{\mathrm{z}}$ dego koszyka do $\mathrm{k}\mathrm{a}\dot{\mathrm{z}}$ dej grupy zostanie wylosowana

jedna druzyna, tak $\dot{\mathrm{z}}\mathrm{e}$ po zakończeniu losowania powstaną cztery grupy po cztery dru-

$\dot{\mathrm{z}}$ yny. Polska znajduje się $\mathrm{w}$ koszyku A. Pozostale koszyki to:

$\mathrm{B}$: Niemcy $(p=0,2)$, Wlochy $(p=0,2)$, Anglia $(p=0,4)$, Rosja $(p=0,5)$ ;

$\mathrm{C}$: Chorwacja $(p=0,6)$, Grecja $(p=0,6)$, Portugalia $(p=0,4)$, Szwecja $(p=0,6)$ ;

$\mathrm{D}$: Dania $(p=0,4)$, Francja $(p=0,4)$, Czechy $(p=0,6)$, Irlandia $(p=0,5).$

a) Jakie jest prawdopodobieństwo, $\dot{\mathrm{z}}\mathrm{e}$ do grupy $\mathrm{z}$ Polskq trafią przynajmniej dwie

druzyny, których $p$ jest większe lub równe 0, 5?

b) Gospodarz Euro 2012, Po1ska, ma prawo do następuj $\Phi^{\mathrm{c}\mathrm{e}\mathrm{j}}$ modyfikacji: $\mathrm{z}$ losowo wy-

branego koszyka zostaną wylosowane do grupy $\mathrm{z}$ nią dwie druzyny, a $\mathrm{z}$ innego losowo

wybranego koszyka nie będzie losowana $\dot{\mathrm{z}}$ adna. Czy Polsce opłaca się skorzystač $\mathrm{z}$

tego prawa, jeśli chce powiększyč prawdopodobieństwo zdarzenia $\mathrm{z}$ punktu a)?





XLI

KORESPONDENCYJNY KURS

Z MATEMATYKI

marzec 2012 r.

PRACA KONTROLNA nr 7- POZIOM PODSTAWOWY

l. Narysowač wykres funkcji $f(x) = |2x-4|-\sqrt{x^{2}+4x+4}$. Określič liczbe rozwiazań

równania $|f(x)| = m \mathrm{w}$ zalezności od parametru $m$. Dla jakiego $m$ pole trójk$\Phi$ta

ograniczonego wykresem funkcji $f$ oraz prostą $y=m$ równe jest 6?

2. Wśród prostokątów 0 ustalonej dfugości przekątnej p wskazač ten, którego pole jest

największe. Nie stosowač metod rachunku rózniczkowego.

3. Wyznaczyč wszystkie liczby rzeczywiste $x$, dla których funkcja $f(x)=x-1-\log_{\frac{1}{3}}(4-$

$3^{x})$ przyjmuje wartości nieujemne.

4. Stosując wzór na cosinus podwojonego kąta, rozwiazač $\mathrm{w}$ przedziale $[0,2\pi]$ nierównośč

$\displaystyle \cos 2x\leq\frac{\cos 2x+\sin x-\cos^{2}x}{1-\sin x}.$

5. Niech $f(x)=$

dla

dla

$x\leq 1,$

$x>1.$

a) Sporządzič wykres funkcji $f\mathrm{i}$ na jego podstawie wyznaczyč zbiór wartości tej funk-

cji.

b) Obliczyč $f(\sqrt{3}-1) \mathrm{i}$ korzystając $\mathrm{z}$ wykresu zaznaczyč na osi $0x$ zbiór rozwiązań

nierówności $f^{2}(x)\leq 4.$

6. $\mathrm{W}$ kulę $0$ promieniu $R$ wpisano stozek $0$ kacie rozwarcia $\displaystyle \frac{\pi}{3}$ oraz walec $0$ tej samej podsta-

wie, co stozek. Obliczyč stosunek pola powierzchni bocznej stozka do pola powierzchni

bocznej walca.





PRACA KONTROLNA nr 7- POZIOM ROZSZERZONY

l. Uzasadnič, $\dot{\mathrm{z}}\mathrm{e}$ punkty przecięcia dwusiecznych kątów wewnętrznych dowolnego równo-

ległoboku są wierzchofkami prostokąta, którego przekątna ma dlugośč równą róznicy

długości sąsiednich boków równoległoboku.

2. Wśród walców wpisanych $\mathrm{w}$ kulę $0$ promieniu $R$ wskazač ten, którego pole powierzchni

bocznej jest największe. Nie stosowač metod rachunku rózniczkowego.

3. Dane są punkty $A(-1,2), B(1,-4)$ oraz $P(2m,4m^{3}-1)$. Wyznaczyč wszystkie wartości

parametru $m$, dla których $\triangle ABP$ jest prostokątny. $\mathrm{R}\mathrm{o}\mathrm{z}\mathrm{w}\mathrm{i}_{\Phi}$zanie zilustrowač starannym

rysunkiem.

4. Rozwiązač układ równań

$\left\{\begin{array}{l}
x^{2}+y^{2}-8=0\\
xy+x-y=0
\end{array}\right.$

$\mathrm{i}$ podač jego interpretację graficzną.

5. $\mathrm{W}$ przedziale $[-\displaystyle \frac{\pi}{2},\frac{3\pi}{2}]$ rozwiązač nierównośč

$1-\displaystyle \mathrm{t}\mathrm{g}x+\mathrm{t}\mathrm{g}^{2}x-\mathrm{t}\mathrm{g}^{3}x+\cdots>\frac{\sqrt{3}}{2}$ ($1-$ ctg $x$),

której lewa strona jest $\mathrm{s}\mathrm{u}\mathrm{m}\Phi$ nieskończonego ciągu geometrycznego.

6. Wyznaczyč wszystkie wartości rzeczywistego parametru $m$, dla których równanie

$(m^{2}-2)4^{x}+2^{x+1}+m=0$

ma dwa rózne rozwiazania.





XLI

KORESPONDENCYJNY KURS

Z MATEMATYKI

kwiecień 2012 r.

PRACA KONTROLNA $\mathrm{n}\mathrm{r} 6-$ POZIOM PODSTAWOWY

l. Wyznaczyč równanie paraboli, której wykres jest symetryczny względem punktu $(-\displaystyle \frac{3}{2},\frac{5}{2})$

do wykresu paraboli $y = (x+2)^{2}$ Wykazač, $\dot{\mathrm{z}}\mathrm{e}$ punkty przecięcia $\mathrm{i}$ wierzchofki obu

parabol są wierzchołkami równoległoboku $\mathrm{i}$ obliczyč jego pole.

2. $\mathrm{W}$ graniastoslup prawidlowy trójkątny $\mathrm{m}\mathrm{o}\dot{\mathrm{z}}$ na wpisač kulę. Wyznaczyč stosunek pola

powierzchni bocznej do sumy pól obu podstaw oraz cosinus kąta nachylenia przekątnej

ściany bocznej do sąsiedniej ściany bocznej.

3. Uzasadnič, $\dot{\mathrm{z}}\mathrm{e}$ dla $\alpha\in\langle 0,  2\pi\rangle$ równanie

$2x^{2}-2(2\cos\alpha-1)x+2\cos^{2}\alpha-5\cos\alpha+2=0$

nie ma pierwiastków tego samego znaku.

4. Punkty przecięcia prostych $x-y=0, x+y-4=0, x-3y=0$ są wierzchołkami trójkąta.

Obliczyč objętośč bryfy powstałej $\mathrm{z}$ obrotu tego trójkąta dookoła najdłuzszego boku.

5. Trzech pracowników ma wykonač pewnq pracę. Aby wykonač tę pracę samodzielnie,

pierwszy $\mathrm{z}$ nich pracowałby $07$ dni dluzej, drugi - $015$ dni dluzej, a trzeci - trzy razy

dłuzej, $\mathrm{n}\mathrm{i}\dot{\mathrm{z}}$ gdyby pracowali razem. $\mathrm{W}$ jakim czasie wykonają tę pracę wspólnie?

6. Wyznaczyč promień kuli stycznej do wszystkich krawędzi czworościanu foremnego $0$

krawędzi $\alpha.$





PRACA KONTROLNA nr 6- POZIOM ROZSZERZONY

l. Rozwiązač nierównośč $\displaystyle \frac{x}{\sqrt{x^{3}-2x+1}}\geq\frac{1}{\sqrt{x+3}}.$

2. Narysowač staranny wykres funkcji

$f(x)=\displaystyle \frac{\sin 2x-|\sin x|}{\sin x}.$

Następnie $\mathrm{w}$ przedziale $[0,\pi]$ wyznaczyč rozwiqzania nierówności

$f(x)<2(\sqrt{2}-1)\cos^{2}x$

3. Rozwiązač nierównośč

$1+\displaystyle \frac{\log_{2}x}{1+\log_{2}x}+(\frac{\log_{2}x}{1+\log_{2}x})^{2}+\cdots\geq 2\log_{2}x,$

której lewa strona jest $\mathrm{s}\mathrm{u}\mathrm{m}\Phi$ nieskończonego szeregu geometrycznego.

4. Objętośč stozka jest 4 razy miejsza $\mathrm{n}\mathrm{i}\dot{\mathrm{z}}$ objętośč opisanej na nim kuli. Wyznaczyč sto-

sunek pola powierzchni całkowitej stozka do pola powierzchni kuli oraz kąt, pod jakim

$\mathrm{t}\mathrm{w}\mathrm{o}\mathrm{r}\mathrm{z}\Phi^{\mathrm{C}\mathrm{a}}$ stozka jest widoczna ze środka kuli.

5. Promień światla przechodzi przez punkt $A(1,1)$, odbija się od prostej $0$ równaniu

$y = x-2$ (zgodnie $\mathrm{z}$ zasadq mówiąca, $\dot{\mathrm{z}}\mathrm{e}$ kąt padania jest równy kątowi odbicia) $\mathrm{i}$

przechodzi przez punkt $B(4,6)$. Wyznaczyč wspófrzędne punktu odbicia $P$ oraz równania

prostych, po których biegnie promień przed $\mathrm{i}$ po odbiciu.

6. Na boku $BC$ trójkąta równobocznego obrano punkt $D\mathrm{t}\mathrm{a}\mathrm{k}, \dot{\mathrm{z}}\mathrm{e}$ promień okręgu wpisanego

$\mathrm{w}$ trójkqt $ADC$ jest dwa razy mniejszy $\mathrm{n}\mathrm{i}\dot{\mathrm{z}}$ promień okręgu wpisanego $\mathrm{w}$ trójkąt $ABD.$

$\mathrm{W}$ jakim stosunku punkt $D$ dzieli bok $BC$?





XLI

KORESPONDENCYJNY KURS

Z MATEMATYKI

$\mathrm{p}\mathrm{a}\acute{\mathrm{z}}$dziernik 2011 $\mathrm{r}.$

PRACA KONTROLNA $\mathrm{n}\mathrm{r} 2-$ POZIOM PODSTAWOWY

l. Niech $A=\displaystyle \{x\in \mathbb{R}:\frac{x}{x^{2}-1}\geq\frac{1}{x}\}$ oraz $B=\{x\in \mathbb{R}:|x+2|<4\}$. Zbiory $A, B, A\cup B,$

$A\cap B, A\backslash B\mathrm{i}B\backslash A$ zapisač $\mathrm{w}$ postaci przedziałów liczbowych $\mathrm{i}$ zaznaczyč je na osi

liczbowej.

2. Zaznaczyč na płaszczy $\acute{\mathrm{z}}\mathrm{n}\mathrm{i}\mathrm{e}$ zbiory $A\cap B, A\backslash B,$

$B=\{(x,y):|y|>x^{2}\}.$

gdzie $A = \{(x,y):|x|+2y\leq 3\},$

3. Suma wysokości $h$ ostrosłupa prawidłowego czworokątnego $\mathrm{i}$ jego krawędzi bocznej $b$

równa jest 12. D1a jakiej wartości $h$ objętośč tego ostroslupa jest największa? Obliczyč

pole powierzchni cafkowitej ostrosfupa dla znalezionej wartości $h.$

4. Wykres trójmianu kwadratowego $f(x)=ax^{2}+bx+c$ jest symetryczny względem prostej

$x=2$, a największ$\Phi$ wartości$\Phi$ tej funkcjijest l. Wyznaczyč wspólczynniki $a, b, c$, wiedząc,

$\dot{\mathrm{z}}\mathrm{e}$ reszta $\mathrm{z}$ dzielenia tego trójmianu przez dwumian $(x+1)$ równa jest $-8$. Narysowač

staranny wykres funkcji $g(x) = f(|x|) \mathrm{i}$ wyznaczyč najmniejszą $\mathrm{i}$ największą wartośč

funkcji $g$ na przedziale [-1, 3].

5. Liczba $p=\displaystyle \frac{(2\sqrt{3}+\sqrt{2})^{3}+(2\sqrt{3}-\sqrt{2})^{3}}{(\sqrt{3}+2)^{2}-(\sqrt{3}-2)^{2}}$ jest kwadratem promienia okręgu opisanego

na trójkqcie prostokqtnym $0$ polu 7,2. Ob1iczyč wysokośč $\mathrm{i}$ tangens mniejszego $\mathrm{z}$ kątów

ostrych tego trójkąta.

6. Narysowač wykres funkcji $f(x)=\sqrt{x^{2}+2x+1}-|2x-4|$. Obliczyč pole obszaru ograni-

czonego wykresem funkcji $f(x)$ oraz wykresem funkcji $g(x)=-f(x)$. Narysowač wykresy

funkcji $f_{1}(x)=|f(x)|$ oraz $f_{2}(x)=f(|x|).$





PRACA KONTROLNA nr 2- POZIOM ROZSZERZONY

l. Dlajakich wartości rzeczywistego parametru $p$ równanie $(p-1)x^{2}-(p+1)x-1=0$ ma

dwa rózne pierwiastki ujemne?

2. Narysowač na płaszczyz/nie zbiór $\{(x,y):\sqrt{x-1}+x\leq 2,0\leq y^{3}\leq\sqrt{5}-2\}$

jego pole. Wsk. Obliczyč $a=(\displaystyle \frac{\sqrt{5}-1}{2})^{3}$

i obliczyč

3. Obliczyč $a=\mathrm{t}\mathrm{g}\alpha, \mathrm{j}\mathrm{e}\dot{\mathrm{z}}$ eli $\displaystyle \sin\alpha-\cos\alpha=\frac{1}{5}\mathrm{i}\mathrm{k}\mathrm{a}\mathrm{t}\alpha$ spełnia nierównośč $\displaystyle \frac{\pi}{4}<\alpha<\frac{\pi}{2}$. Znalez/č

promień kofa wpisanego $\mathrm{w}$ trójkąt $\mathrm{p}\mathrm{r}\mathrm{o}\mathrm{s}\mathrm{t}\mathrm{o}\mathrm{k}_{\Phi^{\mathrm{t}}}\mathrm{n}\mathrm{y}\mathrm{o}$ polu $ 25\pi$, wiedząc, $\dot{\mathrm{z}}\mathrm{e}$ tangens jednego

$\mathrm{z}$ kątów ostrych tego trójkąta jest równy $a.$

4. Narysowač wykres funkcji $f(x) =2|x-1|-\sqrt{x^{2}+2x+1}$. Dla jakiego $m$ pole figury

ograniczonej wykresem funkcji $f$ oraz prostą $y=m$ równe jest 32?

5. Wiadomo, $\dot{\mathrm{z}}\mathrm{e}$ liczby $-1$, 3 sq pierwiastkami wielomianu $W(x)=x^{4}-ax^{3}-4x^{2}+bx+3.$

Wyznaczyč $a, b\mathrm{i}$ rozwiązač nierównośč $\sqrt{W(x)}\leq x^{2}-x.$

6. Narysowač wykres funkcji $f(x)=$

$\mathrm{i}$ na jego podstawie wyznaczyč:

gdy

gdy

$|x-2|\leq 1,$

$|x-2|>1$

a) przedziafy, na których funkcja $f$ jest malejąca,

b) zbiór wartości funkcji $f(x),$

c) zbiór rozwiązań nierówności $|f(x)|\displaystyle \leq\frac{1}{2}.$





XLI

KORESPONDENCYJNY KURS

Z MATEMATYKI

listopad 2011 r.

PRACA KONTROLNA $\mathrm{n}\mathrm{r} 3-$ POZIOM PODSTAWOWY

1. $\mathrm{W}$ trapez równoramienny $0$ obwodzie 20 $\mathrm{i}$ kacie ostrym $\displaystyle \frac{\pi}{6}\mathrm{m}\mathrm{o}\dot{\mathrm{z}}$ na wpisač okrqg. Obliczyč

promień okręgu oraz dfugości boków tego trapezu.

2. Wielomian $W(x)=x^{3}+ax^{2}+bx-64$ ma trzy pierwiastki rzeczywiste, których średnia

arytmetyczna jest równa $\displaystyle \frac{14}{\mathrm{s}}$, a jeden $\mathrm{z}$ pierwiastków jest równy średniej geometrycznej

dwóch pozostafych. Wyznaczyč $a\mathrm{i}b$ oraz pierwiastki tego wielomianu.

3. Na okręgu $0$ promieniu $r$ opisano romb, którego dłuzsza przekątna ma długośč $4r$. Wy-

znaczyč pola wszystkich czterech figur ograniczonych bokami rombu $\mathrm{i}$ odpowiednimi

łukami okręgu.

4. Przez punkt $(-1,1)$ poprowadzono prostq $\mathrm{t}\mathrm{a}\mathrm{k}$, aby środek jej odcinka zawartego między

prostymi $x+2y= 1\mathrm{i}x+2y=3$ nalezaf do prostej $x-y= 1$. Wyznaczyč równanie

symetralnej odcinka.

5. $\mathrm{W}$ okręgu $0$ środku $\mathrm{w}$ punkcie $O\mathrm{i}$ promieniu $r$ poprowadzono dwie wzajemnie prosto-

padłe średnice AB $\mathrm{i}CD$ oraz cięciwe $AE$, która przecina średnicę $CD\mathrm{w}$ punkcie $F.$

Dla jakiego kąta $\angle BAE$, czworokąt OBEF ma dwa razy większe pole od pola trójkąta

$AFO$?

6. Na przeciwprostokątnej $AB$ trójkąta prostokqtnego $ABC$ zbudowano trójkqt równobocz-

ny $ADB$, którego pole jest dwa razy większe od pola trójkąta $ABC$. Wyznaczyč kąty

trójkąta $ABC$ oraz stosunek $|BK|$ : $|KA|$ dfugości odcinków, na jakie punkt styczności

$K$ okregu wpisanego $\mathrm{w}$ trójkąt $ABC$ dzieli przeciwprostokatną.





PRACA KONTROLNA nr 3- POZIOM ROZSZERZONY

l. Napisač równanie okręgu przechodzącego przez punkt (1, 2) stycznego do prostych

$y=-2x\mathrm{i}y=-2x+20.$

2. Na bokach $AC \mathrm{i} BC$ trójkąta $ABC$ zaznaczono odpowiednio punkty $E \mathrm{i} D \mathrm{t}\mathrm{a}\mathrm{k}, \dot{\mathrm{z}}\mathrm{e}$

$\displaystyle \frac{|EC|}{|AE|}=\frac{|DC|}{|BD|}=2$. Wyznaczyč stosunek pola trójkąta $ABC$ do pola trójkąta $ABF$, gdzie

$F$ jest punktem przecięcia odcinków $AD\mathrm{i}$ {\it BE}.

3. Kąt przy wierzchołku $C$ trójkąta $ABC$ jest równy $\displaystyle \frac{\pi}{3}$, a długości boków $AC\mathrm{i}BC$ wyno-

$\mathrm{s}\mathrm{z}\Phi$ odpowiednio 15 cm $\mathrm{i}10$ cm. Na bokach trójk$\Phi$ta zbudowano trójkąty równoboczne

$\mathrm{i}$ otrzymano $\mathrm{w}$ ten sposób wielokąt $0$ dodatkowych wierzcholkach $D, E, F$. Obliczyč

odległośč między wierzchołkami $C\mathrm{i}D, B\mathrm{i}F$ oraz A $\mathrm{i}D$?

4. Wielomian $W(x)=x^{4}-3x^{3}+ax^{2}+bx+c$ ma pierwiastek równy l. Reszta $\mathrm{z}$ dzielenia tego

wielomianu przez $x^{2}-x-2$ równa jest $4x-12$. Wyznaczyč $a, b, c\mathrm{i}$ pozostałe pierwiastki.

Rozwiązač nierównośč $W(x+1)\geq W(x-1).$

5. Dane jest równanie

$(2\sin\alpha-1)x^{2}-2x+\sin\alpha=0,$

$\mathrm{z}$ niewiadomą $x\mathrm{i}$ parametrem $\alpha\in [-\displaystyle \frac{\pi}{2},\frac{\pi}{2}]$. Dlajakich wartości $\alpha$ suma odwrotności pier-

wiastków równania jest większa od 8 $\sin\alpha$, a dla jakich- suma kwadratów odwrotności

pierwiastków jest równa 2 $\sin\alpha$?

6. $\mathrm{W}$ trójkąt równoramienny wpisano okrąg $0$ promieniu $r$. Wyznaczyč pole trójkąta, $\mathrm{j}\mathrm{e}\dot{\mathrm{z}}$ eli

środek okręgu opisanego na tym trójkącie $\mathrm{l}\mathrm{e}\dot{\mathrm{z}}\mathrm{y}$ na okręgu wpisanym $\mathrm{w}$ ten trójkąt.





XLI

KORESPONDENCYJNY KURS

Z MATEMATYKI

grudzień 2011 r.

PRACA KONTROLNA $\mathrm{n}\mathrm{r} 4-$ POZIOM PODSTAWOWY

l. Dane są punkty $A(1,2)$ oraz $B(-1,3)$. Znalez/č współrzędne wierzchołków $C\mathrm{i}D$, jeśli

ABCD jest równoleglobokiem, $\mathrm{w}$ którym $\displaystyle \not\simeq DAB=\frac{\pi}{4}, \displaystyle \mathrm{a}\not\in ADB=\frac{\pi}{2}.$

2. Zaznaczyč na płaszczyz/nie zbiór punktów określony przez uklad nierówności

$\left\{\begin{array}{l}
x^{2}+y^{2}-2|x|>0,\\
|y|\leq 2-x^{2}
\end{array}\right.$

3. $\mathrm{W}$ przedziale $[0,\pi]$ rozwiązač równanie

$\displaystyle \frac{6-12\sin^{2}x}{\mathrm{t}\mathrm{g}^{2}x-1}=8\sin^{4}x-5.$

4. $\mathrm{W}$ sześcian $0$ krawędzi dlugości $a$ wpisano walec, którego przekrój osiowy jest kwadra-

tem, a osią jest przekątna sześcianu. Obliczyč objetośč $V$ walca. Nie wykonując obliczeń

przyblizonych, uzasadnič, $\dot{\mathrm{z}}\mathrm{e}V$ stanowi ponad 25\% objętości sześcianu.

5. Znalez$\acute{}$č równania prostych prostopadłych do prostej $x+2y+4 = 0$ odcinających na

okręgu $(x-2)^{2}+(y-4)^{2} =24$ cięciwy $0$ dfugości 4. Zna1ez$\acute{}$č równanie tej przekątnej

czworokąta wyznaczonego przez otrzymane cięciwy, która tworzy $\mathrm{z}$ osią $Ox$ większy kąt.

6. Wysokośč ostrosfupa prawidłowego sześciokątnego wynosi $H$, a $\mathrm{k}\mathrm{a}\mathrm{t}$ między sqsiednimi

ścianami bocznymi ma miarę $\displaystyle \frac{3}{4}\pi$. Obliczyč objętośč tego ostroslupa oraz tangens $\mathrm{k}_{\Phi^{\mathrm{t}\mathrm{a}}}$

nachylenia ściany bocznej do podstawy.





PRACA KONTROLNA nr 4- POZIOM ROZSZERZONY

l. Znalez$\acute{}$č równania okręgów $0$ promieniu 2 przecinających okrąg $(x+2)^{2}+(y+1)^{2}=25$

$\mathrm{w}$ punkcie $P(1,3)$ pod $\mathrm{k}_{\Phi}\mathrm{t}\mathrm{e}\mathrm{m}$ prostym. Korzystač $\mathrm{z}$ metod rachunku wektorowego.

2. Rozwiązač graficznie układ równań

$\left\{\begin{array}{l}
x^{2}+y^{2}=3+|4x+2|,\\
y^{2}=5-|x|,
\end{array}\right.$

wykonując staranne wykresy krzywych danych powyzszymi równaniami oraz niezbędne

obliczenia.

3. Rozwiązač równanie

$\displaystyle \frac{\cos 6x}{\sin^{4}x-\cos^{4}x}=2\cos 4x+1.$

4. $\mathrm{W}$ trójkącie $ABC$ dany jest wierzchofek $B(-1,3)$. Prosta $y=x+1$ jest symetralnq boku

$BC$, a prosta $9x-3y-2=0$ symetralną boku $AB$. Obliczyč pole trójkąta $ABC$ oraz

tangens $\mathrm{k}_{\Phi}\mathrm{t}\mathrm{a}\alpha$ przy wierzcholku $A$. Uzasadnič, $\displaystyle \dot{\mathrm{z}}\mathrm{e}\frac{5\pi}{12}<\alpha<\frac{\pi}{2}$, nie wykonując obliczeń

przyblizonych.

5. $\mathrm{W}$ walec $0$ promieniu podstawy $R\mathrm{i}$ wysokości $tR$, gdzie $t$ jest parametrem dodatnim,

wpisano mniejszy walec $\mathrm{t}\mathrm{a}\mathrm{k}$, aby byf styczny do powierzchni bocznej $\mathrm{i}$ obu podstaw

większego walca, a jego oś była prostopadla do osi większego walca. Wyrazič stosunek

objętości mniejszego walca do objętości większego jako funkcję parametru $t$. Wyznaczyč

największą wartośč tego stosunku $\mathrm{i}$ odpowiadające mu wymiary obu walców. Podač

warunki rozwiqzalności zadania. Sporządzič odpowiednie rysunki.

6. Promień kuli opisanej na ostroslupie prawidlowym trójkątnym wynosi $R$. Wiadomo, $\dot{\mathrm{z}}\mathrm{e}$

$\mathrm{k}\mathrm{a}\mathrm{t}$ płaski przy wierzcholkujest dwa razy większy $\mathrm{n}\mathrm{i}\dot{\mathrm{z}}\mathrm{k}\mathrm{a}\mathrm{t}$ nachylenia krawędzi bocznej do

podstawy. Obliczyč objętośč ostroslupa $\mathrm{i}$ określič miarę kąta nachylenia ściany bocznej

do podstawy.





XLI

KORESPONDENCYJNY KURS

Z MATEMATYKI

styczeń 2012 r.

PRACA KONTROLNA $\mathrm{n}\mathrm{r} 5-$ POZIOM PODSTAWOWY

l. Wykazač, $\dot{\mathrm{z}}\mathrm{e}$ dla dowolnej liczby naturalnej $n$ liczba

przez 6.

$\displaystyle \frac{1}{4}n^{4}+\frac{1}{2}n^{3}-\frac{1}{4}n^{2}-\frac{1}{2}n$ jest podzielna

2. Niech $a=\log_{\frac{2}{5}}16+\log_{\frac{5}{2}}100$. Rozwiązač nierównośč $\log_{2}(x^{2}+x)+\log_{\frac{1}{2}}a\leq 0.$

3. Rozwiązač równanie $\displaystyle \frac{\sin 4x}{\cos 2x}=-1.$

4. Obliczyč $x, \mathrm{w}\mathrm{i}\mathrm{e}\mathrm{d}\mathrm{z}\Phi^{\mathrm{C}}, \dot{\mathrm{z}}\mathrm{e}\mathrm{t}\mathrm{g}\alpha = 2^{x}, \mathrm{t}\mathrm{g}\beta= 2^{-x}$ oraz $\alpha-\beta= \displaystyle \frac{\pi}{6}$. Wyznaczyč $n\mathrm{t}\mathrm{a}\mathrm{k}$, by

$1+4^{x}+4^{2x}+\cdots+4^{(n-1)x}=121.$

5. Logarytmy $\mathrm{z}$ trzech liczb dodatnich tworzą ciąg arytmetyczny. Suma tych liczb równa

jest 26, a suma ich odwrotności wynosi 0.7(2). Zna1ez$\acute{}$č $\mathrm{t}\mathrm{e}$ liczby.

6. $\mathrm{O}$ kącie $\alpha$ wiadomo, $\displaystyle \dot{\mathrm{z}}\mathrm{e}\sin\alpha+\cos\alpha=\frac{2}{\sqrt{3}}.$

a) Określič, $\mathrm{w}$ której čwiartce jest kąt $\alpha.$

b) Obliczyč $\mathrm{t}\mathrm{g}\alpha+$ ctg $\alpha$ oraz $\sin\alpha-\cos\alpha.$

c) Wyznaczyč $\mathrm{t}\mathrm{g}\alpha.$





PRACA KONTROLNA nr 5- POZIOM ROZSZERZONY

l. Wykorzystując zasade indukcji matematycznej udowodnič, $\dot{\mathrm{z}}\mathrm{e}$ dla $\mathrm{k}\mathrm{a}\dot{\mathrm{z}}$ dej liczby natural-

nej $n$ zachodzi równośč

$\left(\begin{array}{l}
2\\
2
\end{array}\right) + \left(\begin{array}{l}
3\\
2
\end{array}\right) + \left(\begin{array}{l}
4\\
2
\end{array}\right) +\cdots\left(\begin{array}{l}
2n\\
2
\end{array}\right) =\displaystyle \frac{(2n-1)n(2n+1)}{3}.$

2. Dla jakiego parametru $m$ równanie $x^{3}+(m-1)x^{2}-(2m^{2}+m)x+2m^{2}=0$ ma trzy

pierwiastki tworzące ciąg arytmetyczny?

3. Rozwiązač nierównośč $\log(1-2^{x})+x\log 5\leq x(1+\log 2)+\log 6.$

4. Rozwiązač równanie

$\displaystyle \frac{\sin x}{1+\cos x}=2-$ ctg $x.$

Podač interpretację geometryczna, sporządzając wykresy odpowiednich funkcji.

5. Dane są liczby: $m=\displaystyle \frac{\left(\begin{array}{l}
6\\
4
\end{array}\right)\left(\begin{array}{l}
8\\
2
\end{array}\right)}{\left(\begin{array}{l}
7\\
3
\end{array}\right)},$

{\it n}$=$ -($\sqrt{}$($\sqrt{}$24)1-64)(3-41.)2-7-25-$\sqrt{}$4-413.

a) Sprawdzič, wykonując odpowiednie obliczenia, $\dot{\mathrm{z}}\mathrm{e}m, n$ są liczbami naturalnymi.

b) Wyznaczyč $k\mathrm{t}\mathrm{a}\mathrm{k}$, by liczby $m, k, n$ były odpowiednio: pierwszym, drugim $\mathrm{i}$ trzecim

wyrazem $\mathrm{c}\mathrm{i}_{\Phi \mathrm{g}}\mathrm{u}$ geometrycznego.

c) Wyznaczyč sumę wszystkich wyrazów nieskończonego ciągu geometrycznego, któ-

rego pierwszymi trzema wyrazami są $m, k, n$. Ile wyrazów tego ciągu nalez $\mathrm{y}$ wziąč,

by ich suma przekroczyła 95\% sumy wszystkich wyrazów?

6. Rozwiązač równanie

$1-(\displaystyle \frac{2^{x}}{3^{x}-2^{x}})+(\frac{2^{x}}{3^{x}-2^{x}})^{2}-(\frac{2^{x}}{3^{x}-2^{x}})^{3}+\ldots=\frac{3^{x-2}}{2^{x-1}},$

którego lewa strona jest sumą wyrazów nieskończonego ciągu geometrycznego.



\end{document}