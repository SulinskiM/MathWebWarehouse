\documentclass[a4paper,12pt]{article}
\usepackage{latexsym}
\usepackage{amsmath}
\usepackage{amssymb}
\usepackage{graphicx}
\usepackage{wrapfig}
\pagestyle{plain}
\usepackage{fancybox}
\usepackage{bm}

\begin{document}

XLI

KORESPONDENCYJNY KURS

Z MATEMATYKI

wrzesień 2011 r.

PRACA KONTROLNA $\mathrm{n}\mathrm{r} 1 -$ POZIOM PODSTAWOWY

l. Średni czas przeznaczony na matematykę na dwunastu wydziałach pewnej uczelni wy-

nosi 240 godzin. Utworzono nowy wydziaf $\mathrm{i}$ wówczas średnia liczba godzin matematyki

wzrosła $0$ 5\%. Ile godzin przeznaczono na matematykę na nowym wydziale?

2. Droge $\mathrm{z}$ miasta $A$ do miasta $B$ rowerzysta pokonuje $\mathrm{w}$ czasie 3 godzin. Po dfugotrwa1ych

deszczach stan $\displaystyle \frac{\mathrm{s}}{5}$ drogi pogorszyl się na tyle, $\dot{\mathrm{z}}\mathrm{e}$ na tym odcinku rowerzysta $\mathrm{m}\mathrm{o}\dot{\mathrm{z}}\mathrm{e}$ jechač

$\mathrm{z}$ prędkością $04\mathrm{k}\mathrm{m}/\mathrm{h}$ mniejszą. By czas podrózy $\mathrm{z}A$ do $B$ nie uległ zmianie, zmuszony

jest na pozostafym odcinku zwiększyč prędkośč $012\mathrm{k}\mathrm{m}/\mathrm{h}$. Jaka jest odleglośč $\mathrm{z}A$ do

$B\mathrm{i}\mathrm{z}$ jaką prędkością jez/dził rowerzysta przed ulewami?

3. Trzy klasy pewnego gimnazjum wyjechafy na zieloną szkolę. $K\mathrm{a}\dot{\mathrm{z}}\mathrm{d}\mathrm{y}$ uczeń $\mathrm{z}$ klasy $\mathrm{A}$

wyslaf tę $\mathrm{s}\mathrm{a}\mathrm{m}\Phi$ liczbę SMS-ów. $\mathrm{W}$ klasie $\mathrm{B}$ wysfano taką samą liczbę SMS-ów, ale liczba

uczniów byla $01$ mniejsza, a $\mathrm{k}\mathrm{a}\dot{\mathrm{z}}\mathrm{d}\mathrm{y}\mathrm{z}$ nich wyslał $02$ SMS-y więcej. $\mathrm{Z}$ kolei klasa $\mathrm{C}, \mathrm{w}$

której było $0$ dwóch uczniów więcej $\mathrm{i}\mathrm{k}\mathrm{a}\dot{\mathrm{z}}\mathrm{d}\mathrm{y}$ wysłaf $05$ SMS-ów więcej, wysfała $\mathrm{w}$ sumie

dwa razy więcej wiadomości. Ilu uczniów bylo na zielonej szkole $\mathrm{i}$ ile SMS-ów wyslali?

4. Ile jest czterocyfrowych liczb naturalnych:

a) podzielnych przez 4 $\mathrm{i}$ przez 5?

b) podzielnych przez 41ub przez 5?

c) podzielnych przez 4 $\mathrm{i}$ niepodzielnych przez 5?

5. Umowa określa wynagrodzenie miesięczne pana Kowalskiego na kwotę 4000 $\mathrm{z}\mathrm{f}$. Skfadka

na ubezpieczenie społeczne wynosi 18, 7\% tej kwoty, a składka na ubezpieczenie zdrowot-

ne- 7, 75\% kwoty pozosta1ej po od1iczeniu skfadki na ubezpieczenie społeczne. $\mathrm{W}$ celu

obliczenia podatku nalez $\mathrm{y}$ od 80\% wyjściowej kwoty umowy odjąč składkę na ubezpie-

czenie społeczne $\mathrm{i}$ wyznaczyč 19\% pozostałej sumy. Podatek jest róznicą tak otrzymanej

kwoty $\mathrm{i}$ skfadki na ubezpieczenie zdrowotne. Ile zfotych miesięcznie otrzymuje pan Ko-

walski? Jakie powinno byč jego wynagrodzenie, by co miesiąc dostawal przynajmniej

3000 $\mathrm{z}l$?

6. Uprościč wyrazenie (dla $x, y$, dla których ma ono sens)

-{\it xx}--23{\it y}-21-{\it y}--3221--{\it xx}3231{\it yy}--2331

( -{\it xy})- -32

$\mathrm{i}$ następnie obliczyč jego wartośč dla $x=1+\sqrt{2}, y=7+5\sqrt{2}.$
\end{document}
