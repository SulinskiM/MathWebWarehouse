\documentclass[a4paper,12pt]{article}
\usepackage{latexsym}
\usepackage{amsmath}
\usepackage{amssymb}
\usepackage{graphicx}
\usepackage{wrapfig}
\pagestyle{plain}
\usepackage{fancybox}
\usepackage{bm}

\begin{document}

PRACA KONTROLNA nr l- POZIOM ROZSZERZONY

l. Wiek ojca jest $05$ lat większy $\mathrm{n}\mathrm{i}\dot{\mathrm{z}}$ suma lat trzech jego synów. Za 101at ojciec będzie

2 razy starszy od swego najstarszego syna, za 20 lat będzie 2 razy starszy od swego

średniego syna, a za 301at będzie 2 razy starszy od swego najmłodszego syna. Kiedy

ojciec byf 3 razy starszy od swego najstarszego syna, a kiedy będzie 3 razy starszy od

swego najmfodszego syna?

2. Dwaj rowerzyści wyruszyli jednocześnie $\mathrm{w}$ drogę, jeden $\mathrm{z}$ A do $\mathrm{B}$, drugi $\mathrm{z}\mathrm{B}$ do A $\mathrm{i}$ minęli

się po godzinie. Pierwszy jechał $\mathrm{z}$ prędkości$\Phi 03$ km większ$\Phi \mathrm{n}\mathrm{i}\dot{\mathrm{z}}$ drugi $\mathrm{i}$ przyjechał do

celu $027$ minut wcześniej. Jakie były prędkości obu rowerzystów $\mathrm{i}$ jaka jest odlegfośč od

A do $\mathrm{B}$ ?

3. Pierwszy $\mathrm{i}$ drugi pracownik $\mathrm{w}\mathrm{y}\mathrm{k}\mathrm{o}\mathrm{n}\mathrm{a}\mathrm{j}_{\Phi}$ wspólnie pewną pracę $\mathrm{w}$ czasie $\mathrm{c} \mathrm{d}\mathrm{n}\mathrm{i}$, drugi $\mathrm{i}$

trzeci-w czasie a $\mathrm{d}\mathrm{n}\mathrm{i}$, zaś pierwszy $\mathrm{i}$ trzeci-w czasie $b\mathrm{d}\mathrm{n}\mathrm{i}$? Ile dni potrzebuje $\mathrm{k}\mathrm{a}\dot{\mathrm{z}}\mathrm{d}\mathrm{y}\mathrm{z}$

pracowników na wykonanie tej pracy samodzielnie?

4. Ile jest liczb pięciocyfrowych podzielnych przez 6, które $\mathrm{w}$ zapisie dziesiętnym mają:

a) obie cyfry 1, 2 $\mathrm{i}$ tylko $\mathrm{t}\mathrm{e}$? b) obie cyfry 2, 3 $\mathrm{i}$ tylko $\mathrm{t}\mathrm{e}$? c) wszystkie cyfry 1, 2, 3

$\mathrm{i}$ tylko $\mathrm{t}\mathrm{e}$? Odpowied $\acute{\mathrm{z}}$ uzasadnič.

5. $\mathrm{W}$ hurtowni znajduje się towar, którego a\% sprzedano $\mathrm{z}$ zyskiem p\%, a b\% pozostałej

części sprzedano $\mathrm{z}$ zyskiem q\%. $\mathrm{Z}$ jakim zyskiem nalezy sprzedač resztę towaru, by

cafkowity zysk wyniósl r\%?

6. Uprościč wyrazenie (dla $x, y$, dla których ma ono sens)

( -{\it y} -21 -{\it y} -{\it x}61 -21 {\it y} -31 - -{\it x} -21 {\it y} -21 {\it x-xy} -31)

[ -{\it x} -21 -1 {\it y} -21 ({\it x} -65 - -{\it xy}-61) - -{\it x} -23 {\it x}$+$-{\it xy}-61 {\it y} -21]

$\mathrm{i}$ następnie obliczyč jego wartośč dla $x=5\sqrt{2}-7, y=7+5\sqrt{2}.$
\end{document}
