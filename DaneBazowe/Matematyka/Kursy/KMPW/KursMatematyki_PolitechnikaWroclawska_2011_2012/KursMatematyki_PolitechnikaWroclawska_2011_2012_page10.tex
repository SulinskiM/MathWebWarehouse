\documentclass[a4paper,12pt]{article}
\usepackage{latexsym}
\usepackage{amsmath}
\usepackage{amssymb}
\usepackage{graphicx}
\usepackage{wrapfig}
\pagestyle{plain}
\usepackage{fancybox}
\usepackage{bm}

\begin{document}

XLI

KORESPONDENCYJNY KURS

Z MATEMATYKI

luty 2012 r.

PRACA KONTROLNA nr 6- POZIOM PODSTAWOWY

l. Obliczyč, ile jest wszystkich liczb czterocyfrowych, których suma cyfr wynosi 20 $\mathrm{i}$ które

$\mathrm{m}\mathrm{a}\mathrm{j}_{\Phi}$ dokfadnie jedno zero wśród swoich cyfr:

a) $\mathrm{j}\mathrm{e}\dot{\mathrm{z}}$ eli wszystkie cyfry muszą byč rózne,

b) $\mathrm{j}\mathrm{e}\dot{\mathrm{z}}$ eli cyfry mogą powtarzač się.

2. Do ponumerowania wszystkich stron grubej ksiązki zecer $\mathrm{z}\mathrm{u}\dot{\mathrm{z}}$ ył 2989 cyfr. I1e stron ma

ta ksiązka?

3. Zbiory $A, B, C$ są skończone, przy czym

$|A|=10,$

$|B|=9,$

$|A\cap B|=3, |A\cap C|=1,$

$|B\cap C|=1$ oraz

$|A\cup B\cup C|=18.$

Wyznaczyč liczbę elementów zbiorów $A\cap B\cap C$ oraz $C.$

4. Na egzamin $\mathrm{z}$ matematyki przygotowano $\mathrm{i}$ ogloszono 45 zadań. Student nauczył się

rozwiązywač tylko $\displaystyle \frac{2}{3}$ spośród nich. Na egzaminie student losuje trzy zadania. Otrzymuje

ocenę bardzo dobrq za poprawne rozwiqzanie trzech zadań, dobrą za rozwiązanie dwóch,

dostateczną za rozwiązanie jednego $\mathrm{i}$ niedostateczną, gdy nie rozwiąze $\dot{\mathrm{z}}$ adnego zadania.

Jakiejest prawdopodobieństwo, $\dot{\mathrm{z}}\mathrm{e}$ uzyska ocenę co najmniej dostateczną, ajakie- bardzo

dobrq?

5. Udowodnič, $\dot{\mathrm{z}}\mathrm{e}$ dla dowolnej liczby naturalnej $n$ liczba

$\displaystyle \frac{1}{25}\cdot 100^{n}+\frac{2}{5}\cdot 10^{n}+1$

jest kwadratem liczby naturalnej $\mathrm{i}$ jest liczbą podzielną przez 9.

6. $\mathrm{W}$ urnie I są dwie kule biafe $\mathrm{i}$ dwie czarne. $\mathrm{W}$ urnie II jest pięč kul bialych $\mathrm{i}$ trzy

czarne. Rzucamy dwiema kostkami do gry. $\mathrm{J}\mathrm{e}\dot{\mathrm{z}}$ eli iloczyn otrzymanych oczek jest liczbq

$\mathrm{n}\mathrm{i}\mathrm{e}\mathrm{p}\mathrm{a}\mathrm{r}\mathrm{z}\mathrm{y}\mathrm{s}\mathrm{t}_{\Phi}$, to losujemy kulę $\mathrm{z}$ urny I, $\mathrm{w}$ przeciwnym przypadku losujemy kulę $\mathrm{z}$ urny II.

a) Obliczyč prawdopodobieństwo wylosowania kuli czarnej?

b) Ile co najmniej razy nalez $\mathrm{y}$ powtórzyč opisane doświadczenie, aby $\mathrm{z}$ prawdopodo-

bieństwem nie mniejszym $\displaystyle \mathrm{n}\mathrm{i}\dot{\mathrm{z}}\frac{5}{7}$, co najmniej raz wyciągnač kulę białą?
\end{document}
