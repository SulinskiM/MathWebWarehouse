\documentclass[a4paper,12pt]{article}
\usepackage{latexsym}
\usepackage{amsmath}
\usepackage{amssymb}
\usepackage{graphicx}
\usepackage{wrapfig}
\pagestyle{plain}
\usepackage{fancybox}
\usepackage{bm}

\begin{document}

PRACA KONTROLNA nr 6- POZIOM ROZSZERZONY

l. Jest pięč biletów po l zloty, trzy bilety po 3 złote $\mathrm{i}$ dwa bilety po 5 złotych. Wybrano

losowo trzy bilety. Obliczyč prawdopodobieństwo, $\dot{\mathrm{z}}\mathrm{e}:\mathrm{a}$) przynajmniej dwa $\mathrm{z}$ tych biletów

mają jednakową wartośč; b) wybrane bilety mają lączną wartośč 7 złotych.

2. Korzystajqc $\mathrm{z}$ zasady indukcji matematycznej udowodnič, $\dot{\mathrm{z}}\mathrm{e}$ nierównośč

-21

-43

$\displaystyle \frac{2n-1}{2n}<\frac{1}{\sqrt{2n+1}}$

jest prawdziwa dla dowolnej liczby naturalnej $n.$

3. Dwie osoby rzucaj $\Phi$ na przemian $\mathrm{m}\mathrm{o}\mathrm{n}\mathrm{e}\mathrm{t}_{\Phi}$. Wygrywa ta osoba, która pierwsza wyrzuci or-

ła. Obliczyč, ile wynoszą prawdopodobieństwa wygranej dla $\mathrm{k}\mathrm{a}\dot{\mathrm{z}}$ dego $\mathrm{z}$ graczy. Następnie

obliczyč prawdopodobieństwa wygranej obu graczy, gdy rozgrywka została zmieniona

$\mathrm{w}$ następujący sposób: pierwszy gracz rzuca jeden raz $\mathrm{m}\mathrm{o}\mathrm{n}\mathrm{e}\mathrm{t}_{\Phi}$, a potem gracze rzucają

monetą po dwa razy (zaczynając od drugiego gracza), $\mathrm{a}\dot{\mathrm{z}}$ do pierwszego wyrzucenia orla.

4. Ze zbioru liczb naturalnych $n$ spefniających warunek $\displaystyle \frac{1}{\log n}+\frac{\mathrm{l}}{1-\log n}>$ llosujemy kolejno

bez zwracania dwie liczby $\mathrm{i}$ tworzymy $\mathrm{z}$ nich liczbę dwucyfrową, $\mathrm{w}$ której cyfrą dziesiątek

jest pierwsza $\mathrm{z}$ wylosowanych liczb. Sprawdzič niezaleznośč zdarzeń: A- utworzona liczba

jest parzysta, B- utworzona liczba jest podzielna przez 3.

5. Obliczyč, ile liczb mniejszych od l00 nie jest podzielnych przez 2, 3, 5 ani przez 7. Udo-

wodnič, $\dot{\mathrm{z}}\mathrm{e}$ wszystkie te liczby oprócz l są pierwsze. Ile jest liczb pierwszych mniejszych

od 100?

6. Dla $\mathrm{k}\mathrm{a}\dot{\mathrm{z}}$ dej druzyny pilkarskiej biorącej udział $\mathrm{w}$ Euro 2012 eksperci wyznaczy1i współ-

czynnik $p$ oznaczaj $\Phi^{\mathrm{c}\mathrm{y}}$ prawdopodobieństwo, $\dot{\mathrm{z}}\mathrm{e}$ Polska pokona tę druzynę. Druzyny po-

dzielono na cztery koszyki. $\mathrm{Z} \mathrm{k}\mathrm{a}\dot{\mathrm{z}}$ dego koszyka do $\mathrm{k}\mathrm{a}\dot{\mathrm{z}}$ dej grupy zostanie wylosowana

jedna druzyna, tak $\dot{\mathrm{z}}\mathrm{e}$ po zakończeniu losowania powstaną cztery grupy po cztery dru-

$\dot{\mathrm{z}}$ yny. Polska znajduje się $\mathrm{w}$ koszyku A. Pozostale koszyki to:

$\mathrm{B}$: Niemcy $(p=0,2)$, Wlochy $(p=0,2)$, Anglia $(p=0,4)$, Rosja $(p=0,5)$ ;

$\mathrm{C}$: Chorwacja $(p=0,6)$, Grecja $(p=0,6)$, Portugalia $(p=0,4)$, Szwecja $(p=0,6)$ ;

$\mathrm{D}$: Dania $(p=0,4)$, Francja $(p=0,4)$, Czechy $(p=0,6)$, Irlandia $(p=0,5).$

a) Jakie jest prawdopodobieństwo, $\dot{\mathrm{z}}\mathrm{e}$ do grupy $\mathrm{z}$ Polskq trafią przynajmniej dwie

druzyny, których $p$ jest większe lub równe 0, 5?

b) Gospodarz Euro 2012, Po1ska, ma prawo do następuj $\Phi^{\mathrm{c}\mathrm{e}\mathrm{j}}$ modyfikacji: $\mathrm{z}$ losowo wy-

branego koszyka zostaną wylosowane do grupy $\mathrm{z}$ nią dwie druzyny, a $\mathrm{z}$ innego losowo

wybranego koszyka nie będzie losowana $\dot{\mathrm{z}}$ adna. Czy Polsce opłaca się skorzystač $\mathrm{z}$

tego prawa, jeśli chce powiększyč prawdopodobieństwo zdarzenia $\mathrm{z}$ punktu a)?
\end{document}
