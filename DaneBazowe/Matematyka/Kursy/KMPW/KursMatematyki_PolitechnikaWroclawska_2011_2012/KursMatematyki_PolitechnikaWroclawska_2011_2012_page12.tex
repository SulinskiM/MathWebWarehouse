\documentclass[a4paper,12pt]{article}
\usepackage{latexsym}
\usepackage{amsmath}
\usepackage{amssymb}
\usepackage{graphicx}
\usepackage{wrapfig}
\pagestyle{plain}
\usepackage{fancybox}
\usepackage{bm}

\begin{document}

XLI

KORESPONDENCYJNY KURS

Z MATEMATYKI

marzec 2012 r.

PRACA KONTROLNA nr 7- POZIOM PODSTAWOWY

l. Narysowač wykres funkcji $f(x) = |2x-4|-\sqrt{x^{2}+4x+4}$. Określič liczbe rozwiazań

równania $|f(x)| = m \mathrm{w}$ zalezności od parametru $m$. Dla jakiego $m$ pole trójk$\Phi$ta

ograniczonego wykresem funkcji $f$ oraz prostą $y=m$ równe jest 6?

2. Wśród prostokątów 0 ustalonej dfugości przekątnej p wskazač ten, którego pole jest

największe. Nie stosowač metod rachunku rózniczkowego.

3. Wyznaczyč wszystkie liczby rzeczywiste $x$, dla których funkcja $f(x)=x-1-\log_{\frac{1}{3}}(4-$

$3^{x})$ przyjmuje wartości nieujemne.

4. Stosując wzór na cosinus podwojonego kąta, rozwiazač $\mathrm{w}$ przedziale $[0,2\pi]$ nierównośč

$\displaystyle \cos 2x\leq\frac{\cos 2x+\sin x-\cos^{2}x}{1-\sin x}.$

5. Niech $f(x)=$

dla

dla

$x\leq 1,$

$x>1.$

a) Sporządzič wykres funkcji $f\mathrm{i}$ na jego podstawie wyznaczyč zbiór wartości tej funk-

cji.

b) Obliczyč $f(\sqrt{3}-1) \mathrm{i}$ korzystając $\mathrm{z}$ wykresu zaznaczyč na osi $0x$ zbiór rozwiązań

nierówności $f^{2}(x)\leq 4.$

6. $\mathrm{W}$ kulę $0$ promieniu $R$ wpisano stozek $0$ kacie rozwarcia $\displaystyle \frac{\pi}{3}$ oraz walec $0$ tej samej podsta-

wie, co stozek. Obliczyč stosunek pola powierzchni bocznej stozka do pola powierzchni

bocznej walca.
\end{document}
