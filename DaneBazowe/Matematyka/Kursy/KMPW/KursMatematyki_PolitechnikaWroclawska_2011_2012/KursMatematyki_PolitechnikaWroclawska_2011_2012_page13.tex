\documentclass[a4paper,12pt]{article}
\usepackage{latexsym}
\usepackage{amsmath}
\usepackage{amssymb}
\usepackage{graphicx}
\usepackage{wrapfig}
\pagestyle{plain}
\usepackage{fancybox}
\usepackage{bm}

\begin{document}

PRACA KONTROLNA nr 7- POZIOM ROZSZERZONY

l. Uzasadnič, $\dot{\mathrm{z}}\mathrm{e}$ punkty przecięcia dwusiecznych kątów wewnętrznych dowolnego równo-

ległoboku są wierzchofkami prostokąta, którego przekątna ma dlugośč równą róznicy

długości sąsiednich boków równoległoboku.

2. Wśród walców wpisanych $\mathrm{w}$ kulę $0$ promieniu $R$ wskazač ten, którego pole powierzchni

bocznej jest największe. Nie stosowač metod rachunku rózniczkowego.

3. Dane są punkty $A(-1,2), B(1,-4)$ oraz $P(2m,4m^{3}-1)$. Wyznaczyč wszystkie wartości

parametru $m$, dla których $\triangle ABP$ jest prostokątny. $\mathrm{R}\mathrm{o}\mathrm{z}\mathrm{w}\mathrm{i}_{\Phi}$zanie zilustrowač starannym

rysunkiem.

4. Rozwiązač układ równań

$\left\{\begin{array}{l}
x^{2}+y^{2}-8=0\\
xy+x-y=0
\end{array}\right.$

$\mathrm{i}$ podač jego interpretację graficzną.

5. $\mathrm{W}$ przedziale $[-\displaystyle \frac{\pi}{2},\frac{3\pi}{2}]$ rozwiązač nierównośč

$1-\displaystyle \mathrm{t}\mathrm{g}x+\mathrm{t}\mathrm{g}^{2}x-\mathrm{t}\mathrm{g}^{3}x+\cdots>\frac{\sqrt{3}}{2}$ ($1-$ ctg $x$),

której lewa strona jest $\mathrm{s}\mathrm{u}\mathrm{m}\Phi$ nieskończonego ciągu geometrycznego.

6. Wyznaczyč wszystkie wartości rzeczywistego parametru $m$, dla których równanie

$(m^{2}-2)4^{x}+2^{x+1}+m=0$

ma dwa rózne rozwiazania.
\end{document}
