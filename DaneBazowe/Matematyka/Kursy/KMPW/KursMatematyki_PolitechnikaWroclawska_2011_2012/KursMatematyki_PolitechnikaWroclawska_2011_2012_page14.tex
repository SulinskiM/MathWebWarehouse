\documentclass[a4paper,12pt]{article}
\usepackage{latexsym}
\usepackage{amsmath}
\usepackage{amssymb}
\usepackage{graphicx}
\usepackage{wrapfig}
\pagestyle{plain}
\usepackage{fancybox}
\usepackage{bm}

\begin{document}

XLI

KORESPONDENCYJNY KURS

Z MATEMATYKI

kwiecień 2012 r.

PRACA KONTROLNA $\mathrm{n}\mathrm{r} 6-$ POZIOM PODSTAWOWY

l. Wyznaczyč równanie paraboli, której wykres jest symetryczny względem punktu $(-\displaystyle \frac{3}{2},\frac{5}{2})$

do wykresu paraboli $y = (x+2)^{2}$ Wykazač, $\dot{\mathrm{z}}\mathrm{e}$ punkty przecięcia $\mathrm{i}$ wierzchofki obu

parabol są wierzchołkami równoległoboku $\mathrm{i}$ obliczyč jego pole.

2. $\mathrm{W}$ graniastoslup prawidlowy trójkątny $\mathrm{m}\mathrm{o}\dot{\mathrm{z}}$ na wpisač kulę. Wyznaczyč stosunek pola

powierzchni bocznej do sumy pól obu podstaw oraz cosinus kąta nachylenia przekątnej

ściany bocznej do sąsiedniej ściany bocznej.

3. Uzasadnič, $\dot{\mathrm{z}}\mathrm{e}$ dla $\alpha\in\langle 0,  2\pi\rangle$ równanie

$2x^{2}-2(2\cos\alpha-1)x+2\cos^{2}\alpha-5\cos\alpha+2=0$

nie ma pierwiastków tego samego znaku.

4. Punkty przecięcia prostych $x-y=0, x+y-4=0, x-3y=0$ są wierzchołkami trójkąta.

Obliczyč objętośč bryfy powstałej $\mathrm{z}$ obrotu tego trójkąta dookoła najdłuzszego boku.

5. Trzech pracowników ma wykonač pewnq pracę. Aby wykonač tę pracę samodzielnie,

pierwszy $\mathrm{z}$ nich pracowałby $07$ dni dluzej, drugi - $015$ dni dluzej, a trzeci - trzy razy

dłuzej, $\mathrm{n}\mathrm{i}\dot{\mathrm{z}}$ gdyby pracowali razem. $\mathrm{W}$ jakim czasie wykonają tę pracę wspólnie?

6. Wyznaczyč promień kuli stycznej do wszystkich krawędzi czworościanu foremnego $0$

krawędzi $\alpha.$
\end{document}
