\documentclass[a4paper,12pt]{article}
\usepackage{latexsym}
\usepackage{amsmath}
\usepackage{amssymb}
\usepackage{graphicx}
\usepackage{wrapfig}
\pagestyle{plain}
\usepackage{fancybox}
\usepackage{bm}

\begin{document}

PRACA KONTROLNA nr 6- POZIOM ROZSZERZONY

l. Rozwiązač nierównośč $\displaystyle \frac{x}{\sqrt{x^{3}-2x+1}}\geq\frac{1}{\sqrt{x+3}}.$

2. Narysowač staranny wykres funkcji

$f(x)=\displaystyle \frac{\sin 2x-|\sin x|}{\sin x}.$

Następnie $\mathrm{w}$ przedziale $[0,\pi]$ wyznaczyč rozwiqzania nierówności

$f(x)<2(\sqrt{2}-1)\cos^{2}x$

3. Rozwiązač nierównośč

$1+\displaystyle \frac{\log_{2}x}{1+\log_{2}x}+(\frac{\log_{2}x}{1+\log_{2}x})^{2}+\cdots\geq 2\log_{2}x,$

której lewa strona jest $\mathrm{s}\mathrm{u}\mathrm{m}\Phi$ nieskończonego szeregu geometrycznego.

4. Objętośč stozka jest 4 razy miejsza $\mathrm{n}\mathrm{i}\dot{\mathrm{z}}$ objętośč opisanej na nim kuli. Wyznaczyč sto-

sunek pola powierzchni całkowitej stozka do pola powierzchni kuli oraz kąt, pod jakim

$\mathrm{t}\mathrm{w}\mathrm{o}\mathrm{r}\mathrm{z}\Phi^{\mathrm{C}\mathrm{a}}$ stozka jest widoczna ze środka kuli.

5. Promień światla przechodzi przez punkt $A(1,1)$, odbija się od prostej $0$ równaniu

$y = x-2$ (zgodnie $\mathrm{z}$ zasadq mówiąca, $\dot{\mathrm{z}}\mathrm{e}$ kąt padania jest równy kątowi odbicia) $\mathrm{i}$

przechodzi przez punkt $B(4,6)$. Wyznaczyč wspófrzędne punktu odbicia $P$ oraz równania

prostych, po których biegnie promień przed $\mathrm{i}$ po odbiciu.

6. Na boku $BC$ trójkąta równobocznego obrano punkt $D\mathrm{t}\mathrm{a}\mathrm{k}, \dot{\mathrm{z}}\mathrm{e}$ promień okręgu wpisanego

$\mathrm{w}$ trójkqt $ADC$ jest dwa razy mniejszy $\mathrm{n}\mathrm{i}\dot{\mathrm{z}}$ promień okręgu wpisanego $\mathrm{w}$ trójkąt $ABD.$

$\mathrm{W}$ jakim stosunku punkt $D$ dzieli bok $BC$?
\end{document}
