\documentclass[a4paper,12pt]{article}
\usepackage{latexsym}
\usepackage{amsmath}
\usepackage{amssymb}
\usepackage{graphicx}
\usepackage{wrapfig}
\pagestyle{plain}
\usepackage{fancybox}
\usepackage{bm}

\begin{document}

XLI

KORESPONDENCYJNY KURS

Z MATEMATYKI

$\mathrm{p}\mathrm{a}\acute{\mathrm{z}}$dziernik 2011 $\mathrm{r}.$

PRACA KONTROLNA $\mathrm{n}\mathrm{r} 2-$ POZIOM PODSTAWOWY

l. Niech $A=\displaystyle \{x\in \mathbb{R}:\frac{x}{x^{2}-1}\geq\frac{1}{x}\}$ oraz $B=\{x\in \mathbb{R}:|x+2|<4\}$. Zbiory $A, B, A\cup B,$

$A\cap B, A\backslash B\mathrm{i}B\backslash A$ zapisač $\mathrm{w}$ postaci przedziałów liczbowych $\mathrm{i}$ zaznaczyč je na osi

liczbowej.

2. Zaznaczyč na płaszczy $\acute{\mathrm{z}}\mathrm{n}\mathrm{i}\mathrm{e}$ zbiory $A\cap B, A\backslash B,$

$B=\{(x,y):|y|>x^{2}\}.$

gdzie $A = \{(x,y):|x|+2y\leq 3\},$

3. Suma wysokości $h$ ostrosłupa prawidłowego czworokątnego $\mathrm{i}$ jego krawędzi bocznej $b$

równa jest 12. D1a jakiej wartości $h$ objętośč tego ostroslupa jest największa? Obliczyč

pole powierzchni cafkowitej ostrosfupa dla znalezionej wartości $h.$

4. Wykres trójmianu kwadratowego $f(x)=ax^{2}+bx+c$ jest symetryczny względem prostej

$x=2$, a największ$\Phi$ wartości$\Phi$ tej funkcjijest l. Wyznaczyč wspólczynniki $a, b, c$, wiedząc,

$\dot{\mathrm{z}}\mathrm{e}$ reszta $\mathrm{z}$ dzielenia tego trójmianu przez dwumian $(x+1)$ równa jest $-8$. Narysowač

staranny wykres funkcji $g(x) = f(|x|) \mathrm{i}$ wyznaczyč najmniejszą $\mathrm{i}$ największą wartośč

funkcji $g$ na przedziale [-1, 3].

5. Liczba $p=\displaystyle \frac{(2\sqrt{3}+\sqrt{2})^{3}+(2\sqrt{3}-\sqrt{2})^{3}}{(\sqrt{3}+2)^{2}-(\sqrt{3}-2)^{2}}$ jest kwadratem promienia okręgu opisanego

na trójkqcie prostokqtnym $0$ polu 7,2. Ob1iczyč wysokośč $\mathrm{i}$ tangens mniejszego $\mathrm{z}$ kątów

ostrych tego trójkąta.

6. Narysowač wykres funkcji $f(x)=\sqrt{x^{2}+2x+1}-|2x-4|$. Obliczyč pole obszaru ograni-

czonego wykresem funkcji $f(x)$ oraz wykresem funkcji $g(x)=-f(x)$. Narysowač wykresy

funkcji $f_{1}(x)=|f(x)|$ oraz $f_{2}(x)=f(|x|).$
\end{document}
