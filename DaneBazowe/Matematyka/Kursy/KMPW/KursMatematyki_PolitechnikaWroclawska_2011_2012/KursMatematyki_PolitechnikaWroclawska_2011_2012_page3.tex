\documentclass[a4paper,12pt]{article}
\usepackage{latexsym}
\usepackage{amsmath}
\usepackage{amssymb}
\usepackage{graphicx}
\usepackage{wrapfig}
\pagestyle{plain}
\usepackage{fancybox}
\usepackage{bm}

\begin{document}

PRACA KONTROLNA nr 2- POZIOM ROZSZERZONY

l. Dlajakich wartości rzeczywistego parametru $p$ równanie $(p-1)x^{2}-(p+1)x-1=0$ ma

dwa rózne pierwiastki ujemne?

2. Narysowač na płaszczyz/nie zbiór $\{(x,y):\sqrt{x-1}+x\leq 2,0\leq y^{3}\leq\sqrt{5}-2\}$

jego pole. Wsk. Obliczyč $a=(\displaystyle \frac{\sqrt{5}-1}{2})^{3}$

i obliczyč

3. Obliczyč $a=\mathrm{t}\mathrm{g}\alpha, \mathrm{j}\mathrm{e}\dot{\mathrm{z}}$ eli $\displaystyle \sin\alpha-\cos\alpha=\frac{1}{5}\mathrm{i}\mathrm{k}\mathrm{a}\mathrm{t}\alpha$ spełnia nierównośč $\displaystyle \frac{\pi}{4}<\alpha<\frac{\pi}{2}$. Znalez/č

promień kofa wpisanego $\mathrm{w}$ trójkąt $\mathrm{p}\mathrm{r}\mathrm{o}\mathrm{s}\mathrm{t}\mathrm{o}\mathrm{k}_{\Phi^{\mathrm{t}}}\mathrm{n}\mathrm{y}\mathrm{o}$ polu $ 25\pi$, wiedząc, $\dot{\mathrm{z}}\mathrm{e}$ tangens jednego

$\mathrm{z}$ kątów ostrych tego trójkąta jest równy $a.$

4. Narysowač wykres funkcji $f(x) =2|x-1|-\sqrt{x^{2}+2x+1}$. Dla jakiego $m$ pole figury

ograniczonej wykresem funkcji $f$ oraz prostą $y=m$ równe jest 32?

5. Wiadomo, $\dot{\mathrm{z}}\mathrm{e}$ liczby $-1$, 3 sq pierwiastkami wielomianu $W(x)=x^{4}-ax^{3}-4x^{2}+bx+3.$

Wyznaczyč $a, b\mathrm{i}$ rozwiązač nierównośč $\sqrt{W(x)}\leq x^{2}-x.$

6. Narysowač wykres funkcji $f(x)=$

$\mathrm{i}$ na jego podstawie wyznaczyč:

gdy

gdy

$|x-2|\leq 1,$

$|x-2|>1$

a) przedziafy, na których funkcja $f$ jest malejąca,

b) zbiór wartości funkcji $f(x),$

c) zbiór rozwiązań nierówności $|f(x)|\displaystyle \leq\frac{1}{2}.$
\end{document}
