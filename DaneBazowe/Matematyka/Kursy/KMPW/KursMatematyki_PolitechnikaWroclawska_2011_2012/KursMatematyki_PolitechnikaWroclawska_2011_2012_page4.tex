\documentclass[a4paper,12pt]{article}
\usepackage{latexsym}
\usepackage{amsmath}
\usepackage{amssymb}
\usepackage{graphicx}
\usepackage{wrapfig}
\pagestyle{plain}
\usepackage{fancybox}
\usepackage{bm}

\begin{document}

XLI

KORESPONDENCYJNY KURS

Z MATEMATYKI

listopad 2011 r.

PRACA KONTROLNA $\mathrm{n}\mathrm{r} 3-$ POZIOM PODSTAWOWY

1. $\mathrm{W}$ trapez równoramienny $0$ obwodzie 20 $\mathrm{i}$ kacie ostrym $\displaystyle \frac{\pi}{6}\mathrm{m}\mathrm{o}\dot{\mathrm{z}}$ na wpisač okrqg. Obliczyč

promień okręgu oraz dfugości boków tego trapezu.

2. Wielomian $W(x)=x^{3}+ax^{2}+bx-64$ ma trzy pierwiastki rzeczywiste, których średnia

arytmetyczna jest równa $\displaystyle \frac{14}{\mathrm{s}}$, a jeden $\mathrm{z}$ pierwiastków jest równy średniej geometrycznej

dwóch pozostafych. Wyznaczyč $a\mathrm{i}b$ oraz pierwiastki tego wielomianu.

3. Na okręgu $0$ promieniu $r$ opisano romb, którego dłuzsza przekątna ma długośč $4r$. Wy-

znaczyč pola wszystkich czterech figur ograniczonych bokami rombu $\mathrm{i}$ odpowiednimi

łukami okręgu.

4. Przez punkt $(-1,1)$ poprowadzono prostq $\mathrm{t}\mathrm{a}\mathrm{k}$, aby środek jej odcinka zawartego między

prostymi $x+2y= 1\mathrm{i}x+2y=3$ nalezaf do prostej $x-y= 1$. Wyznaczyč równanie

symetralnej odcinka.

5. $\mathrm{W}$ okręgu $0$ środku $\mathrm{w}$ punkcie $O\mathrm{i}$ promieniu $r$ poprowadzono dwie wzajemnie prosto-

padłe średnice AB $\mathrm{i}CD$ oraz cięciwe $AE$, która przecina średnicę $CD\mathrm{w}$ punkcie $F.$

Dla jakiego kąta $\angle BAE$, czworokąt OBEF ma dwa razy większe pole od pola trójkąta

$AFO$?

6. Na przeciwprostokątnej $AB$ trójkąta prostokqtnego $ABC$ zbudowano trójkqt równobocz-

ny $ADB$, którego pole jest dwa razy większe od pola trójkąta $ABC$. Wyznaczyč kąty

trójkąta $ABC$ oraz stosunek $|BK|$ : $|KA|$ dfugości odcinków, na jakie punkt styczności

$K$ okregu wpisanego $\mathrm{w}$ trójkąt $ABC$ dzieli przeciwprostokatną.
\end{document}
