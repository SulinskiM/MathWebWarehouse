\documentclass[a4paper,12pt]{article}
\usepackage{latexsym}
\usepackage{amsmath}
\usepackage{amssymb}
\usepackage{graphicx}
\usepackage{wrapfig}
\pagestyle{plain}
\usepackage{fancybox}
\usepackage{bm}

\begin{document}

PRACA KONTROLNA nr 3- POZIOM ROZSZERZONY

l. Napisač równanie okręgu przechodzącego przez punkt (1, 2) stycznego do prostych

$y=-2x\mathrm{i}y=-2x+20.$

2. Na bokach $AC \mathrm{i} BC$ trójkąta $ABC$ zaznaczono odpowiednio punkty $E \mathrm{i} D \mathrm{t}\mathrm{a}\mathrm{k}, \dot{\mathrm{z}}\mathrm{e}$

$\displaystyle \frac{|EC|}{|AE|}=\frac{|DC|}{|BD|}=2$. Wyznaczyč stosunek pola trójkąta $ABC$ do pola trójkąta $ABF$, gdzie

$F$ jest punktem przecięcia odcinków $AD\mathrm{i}$ {\it BE}.

3. Kąt przy wierzchołku $C$ trójkąta $ABC$ jest równy $\displaystyle \frac{\pi}{3}$, a długości boków $AC\mathrm{i}BC$ wyno-

$\mathrm{s}\mathrm{z}\Phi$ odpowiednio 15 cm $\mathrm{i}10$ cm. Na bokach trójk$\Phi$ta zbudowano trójkąty równoboczne

$\mathrm{i}$ otrzymano $\mathrm{w}$ ten sposób wielokąt $0$ dodatkowych wierzcholkach $D, E, F$. Obliczyč

odległośč między wierzchołkami $C\mathrm{i}D, B\mathrm{i}F$ oraz A $\mathrm{i}D$?

4. Wielomian $W(x)=x^{4}-3x^{3}+ax^{2}+bx+c$ ma pierwiastek równy l. Reszta $\mathrm{z}$ dzielenia tego

wielomianu przez $x^{2}-x-2$ równa jest $4x-12$. Wyznaczyč $a, b, c\mathrm{i}$ pozostałe pierwiastki.

Rozwiązač nierównośč $W(x+1)\geq W(x-1).$

5. Dane jest równanie

$(2\sin\alpha-1)x^{2}-2x+\sin\alpha=0,$

$\mathrm{z}$ niewiadomą $x\mathrm{i}$ parametrem $\alpha\in [-\displaystyle \frac{\pi}{2},\frac{\pi}{2}]$. Dlajakich wartości $\alpha$ suma odwrotności pier-

wiastków równania jest większa od 8 $\sin\alpha$, a dla jakich- suma kwadratów odwrotności

pierwiastków jest równa 2 $\sin\alpha$?

6. $\mathrm{W}$ trójkąt równoramienny wpisano okrąg $0$ promieniu $r$. Wyznaczyč pole trójkąta, $\mathrm{j}\mathrm{e}\dot{\mathrm{z}}$ eli

środek okręgu opisanego na tym trójkącie $\mathrm{l}\mathrm{e}\dot{\mathrm{z}}\mathrm{y}$ na okręgu wpisanym $\mathrm{w}$ ten trójkąt.
\end{document}
