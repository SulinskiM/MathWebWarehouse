\documentclass[a4paper,12pt]{article}
\usepackage{latexsym}
\usepackage{amsmath}
\usepackage{amssymb}
\usepackage{graphicx}
\usepackage{wrapfig}
\pagestyle{plain}
\usepackage{fancybox}
\usepackage{bm}

\begin{document}

XLI

KORESPONDENCYJNY KURS

Z MATEMATYKI

grudzień 2011 r.

PRACA KONTROLNA $\mathrm{n}\mathrm{r} 4-$ POZIOM PODSTAWOWY

l. Dane są punkty $A(1,2)$ oraz $B(-1,3)$. Znalez/č współrzędne wierzchołków $C\mathrm{i}D$, jeśli

ABCD jest równoleglobokiem, $\mathrm{w}$ którym $\displaystyle \not\simeq DAB=\frac{\pi}{4}, \displaystyle \mathrm{a}\not\in ADB=\frac{\pi}{2}.$

2. Zaznaczyč na płaszczyz/nie zbiór punktów określony przez uklad nierówności

$\left\{\begin{array}{l}
x^{2}+y^{2}-2|x|>0,\\
|y|\leq 2-x^{2}
\end{array}\right.$

3. $\mathrm{W}$ przedziale $[0,\pi]$ rozwiązač równanie

$\displaystyle \frac{6-12\sin^{2}x}{\mathrm{t}\mathrm{g}^{2}x-1}=8\sin^{4}x-5.$

4. $\mathrm{W}$ sześcian $0$ krawędzi dlugości $a$ wpisano walec, którego przekrój osiowy jest kwadra-

tem, a osią jest przekątna sześcianu. Obliczyč objetośč $V$ walca. Nie wykonując obliczeń

przyblizonych, uzasadnič, $\dot{\mathrm{z}}\mathrm{e}V$ stanowi ponad 25\% objętości sześcianu.

5. Znalez$\acute{}$č równania prostych prostopadłych do prostej $x+2y+4 = 0$ odcinających na

okręgu $(x-2)^{2}+(y-4)^{2} =24$ cięciwy $0$ dfugości 4. Zna1ez$\acute{}$č równanie tej przekątnej

czworokąta wyznaczonego przez otrzymane cięciwy, która tworzy $\mathrm{z}$ osią $Ox$ większy kąt.

6. Wysokośč ostrosfupa prawidłowego sześciokątnego wynosi $H$, a $\mathrm{k}\mathrm{a}\mathrm{t}$ między sqsiednimi

ścianami bocznymi ma miarę $\displaystyle \frac{3}{4}\pi$. Obliczyč objętośč tego ostroslupa oraz tangens $\mathrm{k}_{\Phi^{\mathrm{t}\mathrm{a}}}$

nachylenia ściany bocznej do podstawy.
\end{document}
