\documentclass[a4paper,12pt]{article}
\usepackage{latexsym}
\usepackage{amsmath}
\usepackage{amssymb}
\usepackage{graphicx}
\usepackage{wrapfig}
\pagestyle{plain}
\usepackage{fancybox}
\usepackage{bm}

\begin{document}

PRACA KONTROLNA nr 4- POZIOM ROZSZERZONY

l. Znalez$\acute{}$č równania okręgów $0$ promieniu 2 przecinających okrąg $(x+2)^{2}+(y+1)^{2}=25$

$\mathrm{w}$ punkcie $P(1,3)$ pod $\mathrm{k}_{\Phi}\mathrm{t}\mathrm{e}\mathrm{m}$ prostym. Korzystač $\mathrm{z}$ metod rachunku wektorowego.

2. Rozwiązač graficznie układ równań

$\left\{\begin{array}{l}
x^{2}+y^{2}=3+|4x+2|,\\
y^{2}=5-|x|,
\end{array}\right.$

wykonując staranne wykresy krzywych danych powyzszymi równaniami oraz niezbędne

obliczenia.

3. Rozwiązač równanie

$\displaystyle \frac{\cos 6x}{\sin^{4}x-\cos^{4}x}=2\cos 4x+1.$

4. $\mathrm{W}$ trójkącie $ABC$ dany jest wierzchofek $B(-1,3)$. Prosta $y=x+1$ jest symetralnq boku

$BC$, a prosta $9x-3y-2=0$ symetralną boku $AB$. Obliczyč pole trójkąta $ABC$ oraz

tangens $\mathrm{k}_{\Phi}\mathrm{t}\mathrm{a}\alpha$ przy wierzcholku $A$. Uzasadnič, $\displaystyle \dot{\mathrm{z}}\mathrm{e}\frac{5\pi}{12}<\alpha<\frac{\pi}{2}$, nie wykonując obliczeń

przyblizonych.

5. $\mathrm{W}$ walec $0$ promieniu podstawy $R\mathrm{i}$ wysokości $tR$, gdzie $t$ jest parametrem dodatnim,

wpisano mniejszy walec $\mathrm{t}\mathrm{a}\mathrm{k}$, aby byf styczny do powierzchni bocznej $\mathrm{i}$ obu podstaw

większego walca, a jego oś była prostopadla do osi większego walca. Wyrazič stosunek

objętości mniejszego walca do objętości większego jako funkcję parametru $t$. Wyznaczyč

największą wartośč tego stosunku $\mathrm{i}$ odpowiadające mu wymiary obu walców. Podač

warunki rozwiqzalności zadania. Sporządzič odpowiednie rysunki.

6. Promień kuli opisanej na ostroslupie prawidlowym trójkątnym wynosi $R$. Wiadomo, $\dot{\mathrm{z}}\mathrm{e}$

$\mathrm{k}\mathrm{a}\mathrm{t}$ płaski przy wierzcholkujest dwa razy większy $\mathrm{n}\mathrm{i}\dot{\mathrm{z}}\mathrm{k}\mathrm{a}\mathrm{t}$ nachylenia krawędzi bocznej do

podstawy. Obliczyč objętośč ostroslupa $\mathrm{i}$ określič miarę kąta nachylenia ściany bocznej

do podstawy.
\end{document}
