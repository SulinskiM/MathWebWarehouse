\documentclass[a4paper,12pt]{article}
\usepackage{latexsym}
\usepackage{amsmath}
\usepackage{amssymb}
\usepackage{graphicx}
\usepackage{wrapfig}
\pagestyle{plain}
\usepackage{fancybox}
\usepackage{bm}

\begin{document}

XLI

KORESPONDENCYJNY KURS

Z MATEMATYKI

styczeń 2012 r.

PRACA KONTROLNA $\mathrm{n}\mathrm{r} 5-$ POZIOM PODSTAWOWY

l. Wykazač, $\dot{\mathrm{z}}\mathrm{e}$ dla dowolnej liczby naturalnej $n$ liczba

przez 6.

$\displaystyle \frac{1}{4}n^{4}+\frac{1}{2}n^{3}-\frac{1}{4}n^{2}-\frac{1}{2}n$ jest podzielna

2. Niech $a=\log_{\frac{2}{5}}16+\log_{\frac{5}{2}}100$. Rozwiązač nierównośč $\log_{2}(x^{2}+x)+\log_{\frac{1}{2}}a\leq 0.$

3. Rozwiązač równanie $\displaystyle \frac{\sin 4x}{\cos 2x}=-1.$

4. Obliczyč $x, \mathrm{w}\mathrm{i}\mathrm{e}\mathrm{d}\mathrm{z}\Phi^{\mathrm{C}}, \dot{\mathrm{z}}\mathrm{e}\mathrm{t}\mathrm{g}\alpha = 2^{x}, \mathrm{t}\mathrm{g}\beta= 2^{-x}$ oraz $\alpha-\beta= \displaystyle \frac{\pi}{6}$. Wyznaczyč $n\mathrm{t}\mathrm{a}\mathrm{k}$, by

$1+4^{x}+4^{2x}+\cdots+4^{(n-1)x}=121.$

5. Logarytmy $\mathrm{z}$ trzech liczb dodatnich tworzą ciąg arytmetyczny. Suma tych liczb równa

jest 26, a suma ich odwrotności wynosi 0.7(2). Zna1ez$\acute{}$č $\mathrm{t}\mathrm{e}$ liczby.

6. $\mathrm{O}$ kącie $\alpha$ wiadomo, $\displaystyle \dot{\mathrm{z}}\mathrm{e}\sin\alpha+\cos\alpha=\frac{2}{\sqrt{3}}.$

a) Określič, $\mathrm{w}$ której čwiartce jest kąt $\alpha.$

b) Obliczyč $\mathrm{t}\mathrm{g}\alpha+$ ctg $\alpha$ oraz $\sin\alpha-\cos\alpha.$

c) Wyznaczyč $\mathrm{t}\mathrm{g}\alpha.$
\end{document}
