\documentclass[a4paper,12pt]{article}
\usepackage{latexsym}
\usepackage{amsmath}
\usepackage{amssymb}
\usepackage{graphicx}
\usepackage{wrapfig}
\pagestyle{plain}
\usepackage{fancybox}
\usepackage{bm}

\begin{document}

PRACA KONTROLNA nr 5- POZIOM ROZSZERZONY

l. Wykorzystując zasade indukcji matematycznej udowodnič, $\dot{\mathrm{z}}\mathrm{e}$ dla $\mathrm{k}\mathrm{a}\dot{\mathrm{z}}$ dej liczby natural-

nej $n$ zachodzi równośč

$\left(\begin{array}{l}
2\\
2
\end{array}\right) + \left(\begin{array}{l}
3\\
2
\end{array}\right) + \left(\begin{array}{l}
4\\
2
\end{array}\right) +\cdots\left(\begin{array}{l}
2n\\
2
\end{array}\right) =\displaystyle \frac{(2n-1)n(2n+1)}{3}.$

2. Dla jakiego parametru $m$ równanie $x^{3}+(m-1)x^{2}-(2m^{2}+m)x+2m^{2}=0$ ma trzy

pierwiastki tworzące ciąg arytmetyczny?

3. Rozwiązač nierównośč $\log(1-2^{x})+x\log 5\leq x(1+\log 2)+\log 6.$

4. Rozwiązač równanie

$\displaystyle \frac{\sin x}{1+\cos x}=2-$ ctg $x.$

Podač interpretację geometryczna, sporządzając wykresy odpowiednich funkcji.

5. Dane są liczby: $m=\displaystyle \frac{\left(\begin{array}{l}
6\\
4
\end{array}\right)\left(\begin{array}{l}
8\\
2
\end{array}\right)}{\left(\begin{array}{l}
7\\
3
\end{array}\right)},$

{\it n}$=$ -($\sqrt{}$($\sqrt{}$24)1-64)(3-41.)2-7-25-$\sqrt{}$4-413.

a) Sprawdzič, wykonując odpowiednie obliczenia, $\dot{\mathrm{z}}\mathrm{e}m, n$ są liczbami naturalnymi.

b) Wyznaczyč $k\mathrm{t}\mathrm{a}\mathrm{k}$, by liczby $m, k, n$ były odpowiednio: pierwszym, drugim $\mathrm{i}$ trzecim

wyrazem $\mathrm{c}\mathrm{i}_{\Phi \mathrm{g}}\mathrm{u}$ geometrycznego.

c) Wyznaczyč sumę wszystkich wyrazów nieskończonego ciągu geometrycznego, któ-

rego pierwszymi trzema wyrazami są $m, k, n$. Ile wyrazów tego ciągu nalez $\mathrm{y}$ wziąč,

by ich suma przekroczyła 95\% sumy wszystkich wyrazów?

6. Rozwiązač równanie

$1-(\displaystyle \frac{2^{x}}{3^{x}-2^{x}})+(\frac{2^{x}}{3^{x}-2^{x}})^{2}-(\frac{2^{x}}{3^{x}-2^{x}})^{3}+\ldots=\frac{3^{x-2}}{2^{x-1}},$

którego lewa strona jest sumą wyrazów nieskończonego ciągu geometrycznego.
\end{document}
