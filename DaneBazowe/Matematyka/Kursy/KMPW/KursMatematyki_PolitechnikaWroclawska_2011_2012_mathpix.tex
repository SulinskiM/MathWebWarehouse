\documentclass[10pt]{article}
\usepackage[polish]{babel}
\usepackage[utf8]{inputenc}
\usepackage[T1]{fontenc}
\usepackage{amsmath}
\usepackage{amsfonts}
\usepackage{amssymb}
\usepackage[version=4]{mhchem}
\usepackage{stmaryrd}
\usepackage{bbold}

\title{PRACA KONTROLNA nr 1 - POZIOM PODSTAWOWY }

\author{}
\date{}


\newcommand\Varangle{\mathop{{<\!\!\!\!\!\text{\small)}}\:}\nolimits}

\begin{document}
\maketitle
\begin{enumerate}
  \item Średni czas przeznaczony na matematykę na dwunastu wydziałach pewnej uczelni wynosi 240 godzin. Utworzono nowy wydział i wówczas średnia liczba godzin matematyki wzrosła o $5 \%$. Ile godzin przeznaczono na matematykę na nowym wydziale?
  \item Droge z miasta $A$ do miasta $B$ rowerzysta pokonuje w czasie 3 godzin. Po długotrwałych deszczach stan $\frac{3}{5}$ drogi pogorszył się na tyle, że na tym odcinku rowerzysta może jechać z prędkością o $4 \mathrm{~km} / \mathrm{h}$ mniejszą. By czas podróży z $A$ do $B$ nie uległ zmianie, zmuszony jest na pozostałym odcinku zwiększyć prędkość o $12 \mathrm{~km} / \mathrm{h}$. Jaka jest odległość z $A$ do $B$ i z jaką prędkością jeździł rowerzysta przed ulewami?
  \item Trzy klasy pewnego gimnazjum wyjechały na zieloną szkołę. Każdy uczeń z klasy A wysłał tę samą liczbę SMS-ów. W klasie B wysłano taką samą liczbę SMS-ów, ale liczba uczniów była o 1 mniejsza, a każdy z nich wysłał o 2 SMS-y więcej. Z kolei klasa C, w której było o dwóch uczniów więcej i każdy wysłał o 5 SMS-ów więcej, wysłała w sumie dwa razy więcej wiadomości. Ilu uczniów było na zielonej szkole i ile SMS-ów wysłali?
  \item Ile jest czterocyfrowych liczb naturalnych:\\
a) podzielnych przez 4 i przez 5 ?\\
b) podzielnych przez 4 lub przez 5?\\
c) podzielnych przez 4 i niepodzielnych przez 5?
  \item Umowa określa wynagrodzenie miesięczne pana Kowalskiego na kwotę 4000 zł. Składka na ubezpieczenie społeczne wynosi $18,7 \%$ tej kwoty, a składka na ubezpieczenie zdrowotne - $7,75 \%$ kwoty pozostałej po odliczeniu składki na ubezpieczenie społeczne. W celu obliczenia podatku należy od $80 \%$ wyjściowej kwoty umowy odjąć składkę na ubezpieczenie społeczne i wyznaczyć $19 \%$ pozostałej sumy. Podatek jest różnicą tak otrzymanej kwoty i składki na ubezpieczenie zdrowotne. Ile złotych miesięcznie otrzymuje pan Kowalski? Jakie powinno być jego wynagrodzenie, by co miesiąc dostawał przynajmniej $3000 \mathrm{zł}$ ?
  \item Uprościć wyrażenie (dla $x, y$, dla których ma ono sens)
\end{enumerate}

$$
\frac{x^{-\frac{1}{2}} y^{\frac{1}{2}}-x^{-\frac{3}{2}} y^{\frac{3}{2}}}{x^{\frac{2}{3}} y^{-\frac{2}{3}}-x^{-\frac{1}{3}} y^{\frac{1}{3}}} \cdot\left(\frac{y}{x}\right)^{-\frac{2}{3}}
$$

i następnie obliczyć jego wartość dla $x=1+\sqrt{2}, y=7+5 \sqrt{2}$.

\section*{PRACA KONTROLNA nr 1 - POZIOM ROZSZERZONY}
\begin{enumerate}
  \item Wiek ojca jest o 5 lat większy niż suma lat trzech jego synów. Za 10 lat ojciec będzie 2 razy starszy od swego najstarszego syna, za 20 lat będzie 2 razy starszy od swego średniego syna, a za 30 lat będzie 2 razy starszy od swego najmłodszego syna. Kiedy ojciec był 3 razy starszy od swego najstarszego syna, a kiedy będzie 3 razy starszy od swego najmłodszego syna?
  \item Dwaj rowerzyści wyruszyli jednocześnie w drogę, jeden z A do B, drugi z B do A i minęli się po godzinie. Pierwszy jechał z prędkością o 3 km większą niż drugi i przyjechał do celu o 27 minut wcześniej. Jakie były prędkości obu rowerzystów i jaka jest odległość od A do B ?
  \item Pierwszy i drugi pracownik wykonają wspólnie pewną pracę w czasie $c$ dni, drugi i trzeci - w czasie $a$ dni, zaś pierwszy i trzeci - w czasie $b$ dni? Ile dni potrzebuje każdy z pracowników na wykonanie tej pracy samodzielnie?
  \item Ile jest liczb pięciocyfrowych podzielnych przez 6 , które w zapisie dziesiętnym mają:\\
a) obie cyfry 1,2 i tylko te? b) obie cyfry 2,3 i tylko te? c) wszystkie cyfry 1, 2, 3 i tylko te? Odpowiedź uzasadnić.
  \item W hurtowni znajduje się towar, którego $a \%$ sprzedano z zyskiem $p \%$, a $b \%$ pozostałej części sprzedano z zyskiem $q \%$. Z jakim zyskiem należy sprzedać resztę towaru, by całkowity zysk wyniósł $r \%$ ?
  \item Uprościć wyrażenie (dla $x, y$, dla których ma ono sens)
\end{enumerate}

$$
\left(\frac{y^{\frac{1}{6}}}{y^{\frac{1}{2}}-x^{\frac{1}{2}} y^{\frac{1}{3}}}-\frac{x}{x^{\frac{1}{2}} y^{\frac{1}{2}}-x y^{\frac{1}{3}}}\right) \cdot\left[\frac{1}{x^{\frac{1}{2}}-y^{\frac{1}{2}}}\left(x^{\frac{5}{6}}-\frac{y}{x^{\frac{1}{6}}}\right)-\frac{x-y}{x^{\frac{2}{3}}+x^{\frac{1}{6}} y^{\frac{1}{2}}}\right]
$$

i następnie obliczyć jego wartość dla $x=5 \sqrt{2}-7, y=7+5 \sqrt{2}$.

\section*{PRACA KONTROLNA nr 2 - POZIOM PODSTAWOWY}
\begin{enumerate}
  \item Niech $A=\left\{x \in \mathbb{R}: \frac{x}{x^{2}-1} \geqslant \frac{1}{x}\right\}$ oraz $B=\{x \in \mathbb{R}:|x+2|<4\}$. Zbiory $A, B, A \cup B$, $A \cap B, A \backslash B$ i $B \backslash A$ zapisać w postaci przedziałów liczbowych i zaznaczyć je na osi liczbowej.
  \item Zaznaczyć na płaszczyźnie zbiory $A \cap B, A \backslash B$, gdzie $A=\{(x, y):|x|+2 y \leqslant 3\}$, $B=\left\{(x, y):|y|>x^{2}\right\}$.
  \item Suma wysokości $h$ ostrosłupa prawidłowego czworokątnego i jego krawędzi bocznej $b$ równa jest 12. Dla jakiej wartości $h$ objętość tego ostrosłupa jest największa? Obliczyć pole powierzchni całkowitej ostrosłupa dla znalezionej wartości $h$.
  \item Wykres trójmianu kwadratowego $f(x)=a x^{2}+b x+c$ jest symetryczny względem prostej $x=2$, a największą wartością tej funkcji jest 1 . Wyznaczyć współczynniki $a, b, c$, wiedząc, że reszta z dzielenia tego trójmianu przez dwumian $(x+1)$ równa jest -8 . Narysować staranny wykres funkcji $g(x)=f(|x|)$ i wyznaczyć najmniejszą i największą wartość funkcji $g$ na przedziale $[-1,3]$.
  \item Liczba $p=\frac{(2 \sqrt{3}+\sqrt{2})^{3}+(2 \sqrt{3}-\sqrt{2})^{3}}{(\sqrt{3}+2)^{2}-(\sqrt{3}-2)^{2}}$ jest kwadratem promienia okregu opisanego na trójkącie prostokątnym o polu 7,2. Obliczyć wysokość i tangens mniejszego z kątów ostrych tego trójkąta.
  \item Narysować wykres funkcji $f(x)=\sqrt{x^{2}+2 x+1}-|2 x-4|$. Obliczyć pole obszaru ograniczonego wykresem funkcji $f(x)$ oraz wykresem funkcji $g(x)=-f(x)$. Narysować wykresy funkcji $f_{1}(x)=|f(x)|$ oraz $f_{2}(x)=f(|x|)$.
\end{enumerate}

\section*{PRACA KONTROLNA nr 2 - POZIOM ROZSZERZONY}
\begin{enumerate}
  \item Dla jakich wartości rzeczywistego parametru $p$ równanie $(p-1) x^{2}-(p+1) x-1=0$ ma dwa różne pierwiastki ujemne?
  \item Narysować na płaszczyźnie zbiór $\left\{(x, y): \sqrt{x-1}+x \leqslant 2,0 \leqslant y^{3} \leqslant \sqrt{5}-2\right\}$ i obliczyć jego pole. Wsk. Obliczyć $a=\left(\frac{\sqrt{5}-1}{2}\right)^{3}$.
  \item Obliczyć $a=\operatorname{tg} \alpha$, jeżeli $\sin \alpha-\cos \alpha=\frac{1}{5}$ i kąt $\alpha$ spełnia nierówność $\frac{\pi}{4}<\alpha<\frac{\pi}{2}$. Znaleźć promień koła wpisanego w trójkąt prostokątny o polu $25 \pi$, wiedząc, że tangens jednego z kątów ostrych tego trójkąta jest równy $a$.
  \item Narysować wykres funkcji $f(x)=2|x-1|-\sqrt{x^{2}+2 x+1}$. Dla jakiego $m$ pole figury ograniczonej wykresem funkcji $f$ oraz prostą $y=m$ równe jest 32 ?
  \item Wiadomo, że liczby $-1,3$ są pierwiastkami wielomianu $W(x)=x^{4}-a x^{3}-4 x^{2}+b x+3$. Wyznaczyć $a, b$ i rozwiązać nierówność $\sqrt{W(x)} \leqslant x^{2}-x$.
  \item Narysować wykres funkcji $f(x)=\left\{\begin{array}{lll}\frac{x-2}{x}, & \text { gdy } & |x-2| \leqslant 1, \\ \frac{x}{x-2}, & \text { gdy } & |x-2|>1\end{array}\right.$ i na jego podstawie wyznaczyć:\\
a) przedziały, na których funkcja $f$ jest malejąca,\\
b) zbiór wartości funkcji $f(x)$,\\
c) zbiór rozwiązań nierówności $|f(x)| \leqslant \frac{1}{2}$.
\end{enumerate}

\section*{PRACA KONTROLNA nr 3 - POZIOM PODSTAWOWY}
\begin{enumerate}
  \item W trapez równoramienny o obwodzie 20 i kącie ostrym $\frac{\pi}{6}$ można wpisać okrąg. Obliczyć promień okręgu oraz długości boków tego trapezu.
  \item Wielomian $W(x)=x^{3}+a x^{2}+b x-64$ ma trzy pierwiastki rzeczywiste, których średnia arytmetyczna jest równa $\frac{14}{3}$, a jeden z pierwiastków jest równy średniej geometrycznej dwóch pozostałych. Wyznaczyć $a$ i $b$ oraz pierwiastki tego wielomianu.
  \item Na okręgu o promieniu $r$ opisano romb, którego dłuższa przekątna ma długość $4 r$. Wyznaczyć pola wszystkich czterech figur ograniczonych bokami rombu i odpowiednimi łukami okręgu.
  \item Przez punkt $(-1,1)$ poprowadzono prostą tak, aby środek jej odcinka zawartego między prostymi $x+2 y=1$ i $x+2 y=3$ należał do prostej $x-y=1$. Wyznaczyć równanie symetralnej odcinka.
  \item W okręgu o środku w punkcie $O$ i promieniu $r$ poprowadzono dwie wzajemnie prostopadłe średnice $A B$ i $C D$ oraz cięciwę $A E$, która przecina średnicę $C D$ w punkcie $F$. Dla jakiego kąta $\angle B A E$, czworokąt $O B E F$ ma dwa razy większe pole od pola trójkąta AFO?
  \item Na przeciwprostokątnej $A B$ trójkąta prostokątnego $A B C$ zbudowano trójkąt równoboczny $A D B$, którego pole jest dwa razy większe od pola trójkąta $A B C$. Wyznaczyć kąty trójkąta $A B C$ oraz stosunek $|B K|:|K A|$ długości odcinków, na jakie punkt styczności $K$ okręgu wpisanego w trójkąt $A B C$ dzieli przeciwprostokątną.
\end{enumerate}

\section*{PRACA KONTROLNA nr 3 - POZIOM RoZsZERzony}
\begin{enumerate}
  \item Napisać równanie okręgu przechodzącego przez punkt $(1,2)$ stycznego do prostych $y=-2 x$ i $y=-2 x+20$.
  \item Na bokach $A C$ i $B C$ trójkąta $A B C$ zaznaczono odpowiednio punkty $E$ i $D$ tak, że $\frac{|E C|}{|A E|}=\frac{|D C|}{|B D|}=2$. Wyznaczyć stosunek pola trójkąta $A B C$ do pola trójkąta $A B F$, gdzie $F$ jest punktem przecięcia odcinków $A D$ i $B E$.
  \item Kąt przy wierzchołku $C$ trójkąta $A B C$ jest równy $\frac{\pi}{3}$, a długości boków $A C$ i $B C$ wynoszą odpowiednio 15 cm i 10 cm . Na bokach trójkąta zbudowano trójkąty równoboczne i otrzymano w ten sposób wielokąt o dodatkowych wierzchołkach $D, E, F$. Obliczyć odległość między wierzchołkami $C$ i $D, B$ i $F$ oraz $A$ i $D$ ?
  \item Wielomian $W(x)=x^{4}-3 x^{3}+a x^{2}+b x+c$ ma pierwiastek równy 1. Reszta z dzielenia tego wielomianu przez $x^{2}-x-2$ równa jest $4 x-12$. Wyznaczyć $a, b, c$ i pozostałe pierwiastki. Rozwiązać nierówność $W(x+1) \geqslant W(x-1)$.
  \item Dane jest równanie
\end{enumerate}

$$
(2 \sin \alpha-1) x^{2}-2 x+\sin \alpha=0
$$

z niewiadomą $x$ i parametrem $\alpha \in\left[-\frac{\pi}{2}, \frac{\pi}{2}\right]$. Dla jakich wartości $\alpha$ suma odwrotności pierwiastków równania jest większa od $8 \sin \alpha$, a dla jakich - suma kwadratów odwrotności pierwiastków jest równa $2 \sin \alpha$ ?\\
6. W trójkąt równoramienny wpisano okrąg o promieniu $r$. Wyznaczyć pole trójkąta, jeżeli środek okręgu opisanego na tym trójkącie leży na okręgu wpisanym w ten trójkąt.

\section*{PRACA KONTROLNA nr 4 - POZIOM PODSTAWOWY}
\begin{enumerate}
  \item Dane są punkty $A(1,2)$ oraz $B(-1,3)$. Znaleźć współrzędne wierzchołków $C$ i $D$, jeśli $A B C D$ jest równoległobokiem, w którym $\Varangle D A B=\frac{\pi}{4}$, a $\Varangle A D B=\frac{\pi}{2}$.
  \item Zaznaczyć na płaszczyźnie zbiór punktów określony przez układ nierówności
\end{enumerate}

$$
\left\{\begin{array}{l}
x^{2}+y^{2}-2|x|>0 \\
|y| \leqslant 2-x^{2}
\end{array}\right.
$$

\begin{enumerate}
  \setcounter{enumi}{2}
  \item W przedziale $[0, \pi]$ rozwiązać równanie
\end{enumerate}

$$
\frac{6-12 \sin ^{2} x}{\operatorname{tg}^{2} x-1}=8 \sin ^{4} x-5
$$

\begin{enumerate}
  \setcounter{enumi}{3}
  \item W sześcian o krawędzi długości $a$ wpisano walec, którego przekrój osiowy jest kwadratem, a osią jest przekątna sześcianu. Obliczyć objętość $V$ walca. Nie wykonując obliczeń przybliżonych, uzasadnić, że $V$ stanowi ponad $25 \%$ objętości sześcianu.
  \item Znaleźć równania prostych prostopadłych do prostej $x+2 y+4=0$ odcinających na okręgu $(x-2)^{2}+(y-4)^{2}=24$ cięciwy o długości 4. Znaleźć równanie tej przekątnej czworokąta wyznaczonego przez otrzymane cięciwy, która tworzy z osią $O x$ większy kąt.
  \item Wysokość ostrosłupa prawidłowego sześciokątnego wynosi $H$, a kąt między sąsiednimi ścianami bocznymi ma miarę $\frac{3}{4} \pi$. Obliczyć objętość tego ostrosłupa oraz tangens kąta nachylenia ściany bocznej do podstawy.
\end{enumerate}

\section*{PRACA KONTROLNA nr 4 - POZIOM ROZSZERZONY}
\begin{enumerate}
  \item Znaleźć równania okręgów o promieniu 2 przecinających okrąg $(x+2)^{2}+(y+1)^{2}=25$ w punkcie $P(1,3)$ pod kątem prostym. Korzystać z metod rachunku wektorowego.
  \item Rozwiązać graficznie układ równań
\end{enumerate}

$$
\left\{\begin{array}{l}
x^{2}+y^{2}=3+|4 x+2| \\
y^{2}=5-|x|
\end{array}\right.
$$

wykonując staranne wykresy krzywych danych powyższymi równaniami oraz niezbędne obliczenia.\\
3. Rozwiązać równanie

$$
\frac{\cos 6 x}{\sin ^{4} x-\cos ^{4} x}=2 \cos 4 x+1
$$

\begin{enumerate}
  \setcounter{enumi}{3}
  \item W trójkącie $A B C$ dany jest wierzchołek $B(-1,3)$. Prosta $y=x+1$ jest symetralną boku $B C$, a prosta $9 x-3 y-2=0$ symetralną boku $A B$. Obliczyć pole trójkąta $A B C$ oraz tangens kąta $\alpha$ przy wierzchołku $A$. Uzasadnić, że $\frac{5 \pi}{12}<\alpha<\frac{\pi}{2}$, nie wykonując obliczeń przybliżonych.
  \item W walec o promieniu podstawy $R$ i wysokości $t R$, gdzie $t$ jest parametrem dodatnim, wpisano mniejszy walec tak, aby był styczny do powierzchni bocznej i obu podstaw większego walca, a jego oś była prostopadła do osi większego walca. Wyrazić stosunek objętości mniejszego walca do objętości większego jako funkcję parametru $t$. Wyznaczyć największą wartość tego stosunku i odpowiadające mu wymiary obu walców. Podać warunki rozwiązalności zadania. Sporządzić odpowiednie rysunki.
  \item Promień kuli opisanej na ostrosłupie prawidłowym trójkątnym wynosi $R$. Wiadomo, że kąt płaski przy wierzchołku jest dwa razy większy niż kąt nachylenia krawędzi bocznej do podstawy. Obliczyć objętość ostrosłupa i określić miarę kąta nachylenia ściany bocznej do podstawy.
\end{enumerate}

\section*{PRACA KONTROLNA nr 5 - POZIOM PODSTAWOWY}
\begin{enumerate}
  \item Wykazać, że dla dowolnej liczby naturalnej $n$ liczba $\frac{1}{4} n^{4}+\frac{1}{2} n^{3}-\frac{1}{4} n^{2}-\frac{1}{2} n$ jest podzielna przez 6.
  \item Niech $a=\log _{\frac{2}{5}} 16+\log _{\frac{5}{2}} 100$. Rozwiązać nierówność $\log _{2}\left(x^{2}+x\right)+\log _{\frac{1}{2}} a \leqslant 0$.
  \item Rozwiązać równanie $\frac{\sin 4 x}{\cos 2 x}=-1$.
  \item Obliczyć $x$, wiedzac, że $\operatorname{tg} \alpha=2^{x}$, $\operatorname{tg} \beta=2^{-x}$ oraz $\alpha-\beta=\frac{\pi}{6}$. Wyznaczyć $n$ tak, by $1+4^{x}+4^{2 x}+\cdots+4^{(n-1) x}=121$.
  \item Logarytmy z trzech liczb dodatnich tworzą ciąg arytmetyczny. Suma tych liczb równa jest 26, a suma ich odwrotności wynosi $0.7(2)$. Znaleźć te liczby.
  \item O kącie $\alpha$ wiadomo, że $\sin \alpha+\cos \alpha=\frac{2}{\sqrt{3}}$.\\
a) Określić, w której ćwiartce jest kąt $\alpha$.\\
b) Obliczyć $\operatorname{tg} \alpha+\operatorname{ctg} \alpha$ oraz $\sin \alpha-\cos \alpha$.\\
c) Wyznaczyć $\operatorname{tg} \alpha$.
\end{enumerate}

\section*{PRACA KONTROLNA nr 5 - POZIOM ROZSZERZONY}
\begin{enumerate}
  \item Wykorzystując zasadę indukcji matematycznej udowodnić, że dla każdej liczby naturalnej $n$ zachodzi równość
\end{enumerate}

$$
\binom{2}{2}+\binom{3}{2}+\binom{4}{2}+\cdots\binom{2 n}{2}=\frac{(2 n-1) n(2 n+1)}{3} .
$$

\begin{enumerate}
  \setcounter{enumi}{1}
  \item Dla jakiego parametru $m$ równanie $x^{3}+(m-1) x^{2}-\left(2 m^{2}+m\right) x+2 m^{2}=0$ ma trzy pierwiastki tworzące ciąg arytmetyczny?
  \item Rozwiązać nierówność $\log \left(1-2^{x}\right)+x \log 5 \leqslant x(1+\log 2)+\log 6$.
  \item Rozwiązać równanie
\end{enumerate}

$$
\frac{\sin x}{1+\cos x}=2-\operatorname{ctg} x
$$

Podać interpretację geometryczną, sporządzając wykresy odpowiednich funkcji.\\
5. Dane są liczby: $m=\frac{\binom{6}{4} \cdot\binom{8}{2}}{\binom{7}{3}}, \quad n=\frac{(\sqrt{2})^{-4}\left(\frac{1}{4}\right)^{-\frac{5}{2}} \sqrt[4]{3}}{(\sqrt[4]{16})^{3} \cdot 27^{-\frac{1}{4}}}$.\\
a) Sprawdzić, wykonując odpowiednie obliczenia, że $m, n$ są liczbami naturalnymi.\\
b) Wyznaczyć $k$ tak, by liczby $m, k, n$ były odpowiednio: pierwszym, drugim i trzecim wyrazem ciągu geometrycznego.\\
c) Wyznaczyć sumę wszystkich wyrazów nieskończonego ciągu geometrycznego, którego pierwszymi trzema wyrazami są $m, k$, $n$. Ile wyrazów tego ciągu należy wziąć, by ich suma przekroczyła $95 \%$ sumy wszystkich wyrazów?\\
6. Rozwiązać równanie

$$
1-\left(\frac{2^{x}}{3^{x}-2^{x}}\right)+\left(\frac{2^{x}}{3^{x}-2^{x}}\right)^{2}-\left(\frac{2^{x}}{3^{x}-2^{x}}\right)^{3}+\ldots=\frac{3^{x-2}}{2^{x-1}}
$$

którego lewa strona jest sumą wyrazów nieskończonego ciągu geometrycznego.

\section*{PRACA KONTROLNA nr 6 - POZIOM PODSTAWOWY}
\begin{enumerate}
  \item Obliczyć, ile jest wszystkich liczb czterocyfrowych, których suma cyfr wynosi 20 i które mają dokładnie jedno zero wśród swoich cyfr:\\
a) jeżeli wszystkie cyfry muszą być różne,\\
b) jeżeli cyfry mogą powtarzać się.
  \item Do ponumerowania wszystkich stron grubej książki zecer zużył 2989 cyfr. Ile stron ma ta książka?
  \item Zbiory $A, B, C$ są skończone, przy czym\\
$|A|=10, \quad|B|=9, \quad|A \cap B|=3, \quad|A \cap C|=1, \quad|B \cap C|=1$ oraz $\quad|A \cup B \cup C|=18$.\\
Wyznaczyć liczbę elementów zbiorów $A \cap B \cap C$ oraz $C$.
  \item Na egzamin z matematyki przygotowano i ogłoszono 45 zadań. Student nauczył się rozwiązywać tylko $\frac{2}{3}$ spośród nich. Na egzaminie student losuje trzy zadania. Otrzymuje ocenę bardzo dobrą za poprawne rozwiązanie trzech zadań, dobrą za rozwiązanie dwóch, dostateczną za rozwiązanie jednego i niedostateczną, gdy nie rozwiąże żadnego zadania. Jakie jest prawdopodobieństwo, że uzyska ocenę co najmniej dostateczną, a jakie - bardzo dobra?
  \item Udowodnić, że dla dowolnej liczby naturalnej $n$ liczba
\end{enumerate}

$$
\frac{1}{25} \cdot 100^{n}+\frac{2}{5} \cdot 10^{n}+1
$$

jest kwadratem liczby naturalnej i jest liczbą podzielną przez 9.\\
6. W urnie I są dwie kule białe i dwie czarne. W urnie II jest pięć kul białych i trzy czarne. Rzucamy dwiema kostkami do gry. Jeżeli iloczyn otrzymanych oczek jest liczbą nieparzystą, to losujemy kulę z urny I, w przeciwnym przypadku losujemy kulę z urny II.\\
a) Obliczyć prawdopodobieństwo wylosowania kuli czarnej?\\
b) Ile co najmniej razy należy powtórzyć opisane doświadczenie, aby z prawdopodobieństwem nie mniejszym niż $\frac{5}{7}$, co najmniej raz wyciągnacć kulę białą?

\section*{PRACA KONTROLNA nr 6 - POZIOM ROZSZERZONY}
\begin{enumerate}
  \item Jest pięć biletów po 1 złoty, trzy bilety po 3 złote i dwa bilety po 5 złotych. Wybrano losowo trzy bilety. Obliczyć prawdopodobieństwo, że: a) przynajmniej dwa z tych biletów mają jednakową wartość; b) wybrane bilety mają łączną wartość 7 złotych.
  \item Korzystając z zasady indukcji matematycznej udowodnić, że nierówność
\end{enumerate}

$$
\frac{1}{2} \cdot \frac{3}{4} \cdot \ldots \cdot \frac{2 n-1}{2 n}<\frac{1}{\sqrt{2 n+1}}
$$

jest prawdziwa dla dowolnej liczby naturalnej $n$.\\
3. Dwie osoby rzucają na przemian monetą. Wygrywa ta osoba, która pierwsza wyrzuci orła. Obliczyć, ile wynoszą prawdopodobieństwa wygranej dla każdego z graczy. Następnie obliczyć prawdopodobieństwa wygranej obu graczy, gdy rozgrywka została zmieniona w następujący sposób: pierwszy gracz rzuca jeden raz monetą, a potem gracze rzucają monetą po dwa razy (zaczynając od drugiego gracza), aż do pierwszego wyrzucenia orła.\\
4. Ze zbioru liczb naturalnych $n$ spełniających warunek $\frac{1}{\log n}+\frac{1}{1-\log n}>1$ losujemy kolejno bez zwracania dwie liczby i tworzymy z nich liczbę dwucyfrową, w której cyfrą dziesiątek jest pierwsza z wylosowanych liczb. Sprawdzić niezależność zdarzeń: A - utworzona liczba jest parzysta, B - utworzona liczba jest podzielna przez 3.\\
5. Obliczyć, ile liczb mniejszych od 100 nie jest podzielnych przez 2,3,5 ani przez 7. Udowodnić, że wszystkie te liczby oprócz 1 są pierwsze. Ile jest liczb pierwszych mniejszych od 100 ?\\
6. Dla każdej drużyny piłkarskiej biorącej udział w Euro 2012 eksperci wyznaczyli współczynnik $p$ oznaczający prawdopodobieństwo, że Polska pokona tę drużynę. Drużyny podzielono na cztery koszyki. Z każdego koszyka do każdej grupy zostanie wylosowana jedna drużyna, tak że po zakończeniu losowania powstaną cztery grupy po cztery drużyny. Polska znajduje się w koszyku A. Pozostałe koszyki to:\\
B: Niemcy $(p=0,2)$, Włochy $(p=0,2)$, Anglia $(p=0,4)$, Rosja ( $p=0,5)$;\\
C: Chorwacja $(p=0,6)$, Grecja $(p=0,6)$, Portugalia $(p=0,4)$, Szwecja $(p=0,6)$;\\
D: Dania $(p=0,4)$, Francja $(p=0,4)$, Czechy $(p=0,6)$, Irlandia $(p=0,5)$.\\
a) Jakie jest prawdopodobieństwo, że do grupy z Polską trafią przynajmniej dwie drużyny, których $p$ jest większe lub równe 0,5 ?\\
b) Gospodarz Euro 2012, Polska, ma prawo do następującej modyfikacji: z losowo wybranego koszyka zostaną wylosowane do grupy z nią dwie drużyny, a z innego losowo wybranego koszyka nie będzie losowana żadna. Czy Polsce opłaca się skorzystać z tego prawa, jeśli chce powiększyć prawdopodobieństwo zdarzenia z punktu a)?

\section*{PRACA KONTROLNA nr 7 - POZIOM PODSTAWOWY}
\begin{enumerate}
  \item Narysować wykres funkcji $f(x)=|2 x-4|-\sqrt{x^{2}+4 x+4}$. Określić liczbę rozwiązań równania $|f(x)|=m$ w zależności od parametru $m$. Dla jakiego $m$ pole trójkąta ograniczonego wykresem funkcji $f$ oraz prostą $y=m$ równe jest 6 ?
  \item Wśród prostokątów o ustalonej długości przekątnej $p$ wskazać ten, którego pole jest największe. Nie stosować metod rachunku różniczkowego.
  \item Wyznaczyć wszystkie liczby rzeczywiste $x$, dla których funkcja $f(x)=x-1-\log _{\frac{1}{3}}(4-$ $3^{x}$ ) przyjmuje wartości nieujemne.
  \item Stosując wzór na cosinus podwojonego kąta, rozwiązać w przedziale $[0,2 \pi]$ nierówność
\end{enumerate}

$$
\cos 2 x \leqslant \frac{\cos 2 x+\sin x-\cos ^{2} x}{1-\sin x}
$$

\begin{enumerate}
  \setcounter{enumi}{4}
  \item Niech $f(x)=\left\{\begin{array}{lll}x^{2}+2 x & \text { dla } & x \leqslant 1, \\ 1+\frac{2}{x} & \text { dla } & x>1 .\end{array}\right.$\\
a) Sporządzić wykres funkcji $f$ i na jego podstawie wyznaczyć zbiór wartości tej funkcji.\\
b) Obliczyć $f(\sqrt{3}-1)$ i korzystając z wykresu zaznaczyć na osi $0 x$ zbiór rozwiązań nierówności $f^{2}(x) \leqslant 4$.
  \item W kulę o promieniu $R$ wpisano stożek o kącie rozwarcia $\frac{\pi}{3}$ oraz walec o tej samej podstawie, co stożek. Obliczyć stosunek pola powierzchni bocznej stożka do pola powierzchni bocznej walca.
\end{enumerate}

\section*{PRACA KONTROLNA nr 7 - POZIOM ROZSZERZONY}
\begin{enumerate}
  \item Uzasadnić, że punkty przecięcia dwusiecznych kątów wewnętrznych dowolnego równoległoboku są wierzchołkami prostokąta, którego przekątna ma długość równą różnicy długości sąsiednich boków równoległoboku.
  \item Wśród walców wpisanych w kulę o promieniu $R$ wskazać ten, którego pole powierzchni bocznej jest największe. Nie stosować metod rachunku różniczkowego.
  \item Dane są punkty $A(-1,2), B(1,-4)$ oraz $P\left(2 m, 4 m^{3}-1\right)$. Wyznaczyć wszystkie wartości parametru $m$, dla których $\triangle A B P$ jest prostokątny. Rozwiązanie zilustrować starannym rysunkiem.
  \item Rozwiązać układ równań
\end{enumerate}

$$
\left\{\begin{array}{l}
x^{2}+y^{2}-8=0 \\
x y+x-y=0
\end{array}\right.
$$

i podać jego interpretację graficzną.\\
5. W przedziale $\left[-\frac{\pi}{2}, \frac{3 \pi}{2}\right]$ rozwiązać nierówność

$$
1-\operatorname{tg} x+\operatorname{tg}^{2} x-\operatorname{tg}^{3} x+\cdots>\frac{\sqrt{3}}{2}(1-\operatorname{ctg} x)
$$

której lewa strona jest sumą nieskończonego ciągu geometrycznego.\\
6. Wyznaczyć wszystkie wartości rzeczywistego parametru $m$, dla których równanie

$$
\left(m^{2}-2\right) 4^{x}+2^{x+1}+m=0
$$

ma dwa różne rozwiazania.

\section*{PRACA KONTROLNA nr 6 - POZIOM PODSTAWOWY}
\begin{enumerate}
  \item Wyznaczyć równanie paraboli, której wykres jest symetryczny względem punktu $\left(-\frac{3}{2}, \frac{5}{2}\right)$ do wykresu paraboli $y=(x+2)^{2}$. Wykazać, że punkty przecięcia i wierzchołki obu parabol są wierzchołkami równoległoboku i obliczyć jego pole.
  \item W graniastosłup prawidłowy trójkątny można wpisać kulę. Wyznaczyć stosunek pola powierzchni bocznej do sumy pól obu podstaw oraz cosinus kąta nachylenia przekątnej ściany bocznej do sąsiedniej ściany bocznej.
  \item Uzasadnić, że dla $\alpha \in\langle 0,2 \pi\rangle$ równanie
\end{enumerate}

$$
2 x^{2}-2(2 \cos \alpha-1) x+2 \cos ^{2} \alpha-5 \cos \alpha+2=0
$$

nie ma pierwiastków tego samego znaku.\\
4. Punkty przecięcia prostych $x-y=0, x+y-4=0, x-3 y=0$ są wierzchołkami trójkąta. Obliczyć objętość bryły powstałej z obrotu tego trójkąta dookoła najdłuższego boku.\\
5. Trzech pracowników ma wykonać pewną pracę. Aby wykonać tę pracę samodzielnie, pierwszy z nich pracowałby o 7 dni dłużej, drugi - o 15 dni dłużej, a trzeci - trzy razy dłużej, niż gdyby pracowali razem. W jakim czasie wykonają tę pracę wspólnie?\\
6. Wyznaczyć promień kuli stycznej do wszystkich krawędzi czworościanu foremnego o krawędzi $a$.

\section*{PRACA KONTROLNA nr 6 - POZIOM RoZsZERzony}
\begin{enumerate}
  \item Rozwiązać nierówność $\frac{x}{\sqrt{x^{3}-2 x+1}} \geqslant \frac{1}{\sqrt{x+3}}$.
  \item Narysować staranny wykres funkcji
\end{enumerate}

$$
f(x)=\frac{\sin 2 x-|\sin x|}{\sin x}
$$

Następnie w przedziale $[0, \pi]$ wyznaczyć rozwiązania nierówności

$$
f(x)<2(\sqrt{2}-1) \cos ^{2} x
$$

\begin{enumerate}
  \setcounter{enumi}{2}
  \item Rozwiązać nierówność
\end{enumerate}

$$
1+\frac{\log _{2} x}{1+\log _{2} x}+\left(\frac{\log _{2} x}{1+\log _{2} x}\right)^{2}+\cdots \geqslant 2 \log _{2} x
$$

której lewa strona jest sumą nieskończonego szeregu geometrycznego.\\
4. Objętość stożka jest 4 razy miejsza niż objętość opisanej na nim kuli. Wyznaczyć stosunek pola powierzchni całkowitej stożka do pola powierzchni kuli oraz kąt, pod jakim tworząca stożka jest widoczna ze środka kuli.\\
5. Promień światła przechodzi przez punkt $A(1,1)$, odbija się od prostej o równaniu $y=x-2$ (zgodnie z zasadą mówiącą, że kąt padania jest równy kątowi odbicia) i przechodzi przez punkt $B(4,6)$. Wyznaczyć współrzędne punktu odbicia $P$ oraz równania prostych, po których biegnie promień przed i po odbiciu.\\
6. Na boku $B C$ trójkąta równobocznego obrano punkt $D$ tak, że promień okręgu wpisanego w trójkąt $A D C$ jest dwa razy mniejszy niż promień okręgu wpisanego w trójkąt $A B D$. W jakim stosunku punkt $D$ dzieli bok $B C$ ?


\end{document}