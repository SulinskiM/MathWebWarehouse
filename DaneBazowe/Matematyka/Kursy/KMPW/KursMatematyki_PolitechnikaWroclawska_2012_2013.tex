\documentclass[a4paper,12pt]{article}
\usepackage{latexsym}
\usepackage{amsmath}
\usepackage{amssymb}
\usepackage{graphicx}
\usepackage{wrapfig}
\pagestyle{plain}
\usepackage{fancybox}
\usepackage{bm}

\begin{document}

XLII

KORESPONDENCYJNY KURS

Z MATEMATYKI

wrzesień 2012 r.

PRACA KONTROLNA $\mathrm{n}\mathrm{r} 1 -$ POZIOM PODSTAWOWY

l. Niech $A=\displaystyle \{x\in \mathbb{R}:\frac{1}{x^{2}+1}\geq\frac{1}{7-x}\}$ oraz $B=\{x\in \mathbb{R}:|x-2|+|x-7|<7\}$. Znalez/č

$\mathrm{i}$ zaznaczyč na osi liczbowej zbiory $A, B$ oraz $(A\backslash B)\cup(B\backslash A).$

2. Liczba $p=\displaystyle \frac{(\sqrt[3]{54}-2)(9\sqrt[3]{4}+6\sqrt[3]{2}+4)-(2-\sqrt{3})^{3}}{\sqrt{3}+(1+\sqrt{3})^{2}}$ jest miejscem zerowym funkcji

$f(x)=ax^{2}+bx+c$. Pole trójkąta, którego wierzcholkami są punkty przecięcia wykresu

$\mathrm{z}$ osiami układu współrzędnych równe jest 20. Wyznaczyč współczynnik $b$ oraz drugie

miejsce zerowe tej funkcji $\mathrm{w}\mathrm{i}\mathrm{e}\mathrm{d}\mathrm{z}\Phi^{\mathrm{C}}, \dot{\mathrm{z}}\mathrm{e}$ wykres funkcji jest symetryczny względem prostej

$x=3.$

3. Trapez $0$ kqtach przy podstawie $30^{\mathrm{o}}$ oraz $45^{\mathrm{o}}$ jest opisany na okręgu $0$ promieniu $R.$

Obliczyč stosunek pola kola do pola trapezu.

4. Niech $f(x)=$

Obliczyč $f(\displaystyle \frac{1+\sqrt{3}}{2})$ oraz $f(\displaystyle \frac{\pi+1}{\pi-2}).$

Narysowač wykres funkcji $f\mathrm{i}$ na jego podstawie podač zbiór wartości funkcji oraz roz-

wiqzač nierównośč $f(x)\displaystyle \geq-\frac{1}{2}.$

5. Tangens kąta ostrego $\alpha$ równy jest $\displaystyle \frac{a}{7b}$, gdzie

$a=(\sqrt{2}+1)^{3}-(\sqrt{2}-1)^{3},b=(\sqrt{\sqrt{2}+1}-\sqrt{\sqrt{2}-1})^{2}$

Wyznaczyč wartości pozostałych funkcji trygonometrycznych tego kąta oraz kąta $2\alpha.$

6. $\mathrm{W}$ trójk$\Phi$t otrzymany $\mathrm{w}$ przekroju ostrosłupa prawidłowego czworokątnego pfaszczyzną

przechodzącą przez wysokośč ostrosłupa $\mathrm{i}$ przekątną jego podstawy wpisano kwadrat,

którego jeden bok jest zawarty $\mathrm{w}$ przekatnej podstawy. Pole kwadratu jest dwa ra-

zy mniejsze $\mathrm{n}\mathrm{i}\dot{\mathrm{z}}$ pole podstawy ostrosfupa. Obliczyč stosunek pola powierzchni bocznej

ostrosłupa do pola jego podstawy oraz cosinus kąta między ścianami bocznymi.




PRACA KONTROLNA nr l- POZIOM ROZSZERZONY

l. Niech $A=\{(x,y):y\geq||x-2|-1|\}, B=\{(x,y):y+\sqrt{4x-x^{2}-3}\leq 2\}$. Narysowač

na pfaszczy $\acute{\mathrm{z}}\mathrm{n}\mathrm{i}\mathrm{e}$ zbiór $A\cap B\mathrm{i}$ obliczyč jego pole.

2. Pole powierzchni bocznej ostrosłupa prawidłowego trójkątnego jest $\mathrm{k}$ razy większe $\mathrm{n}\mathrm{i}\dot{\mathrm{z}}$

pole jego podstawy. Obliczyč cosinus kata nachylenia krawędzi bocznej ostroslupa do

pfaszczyzny podstawy.

3. Dane są liczby: $m = \displaystyle \frac{\left(\begin{array}{l}
6\\
4
\end{array}\right)\left(\begin{array}{l}
8\\
2
\end{array}\right)}{\left(\begin{array}{l}
7\\
3
\end{array}\right)}, n = \displaystyle \frac{(\sqrt{2})^{-4}(\frac{1}{4})^{-\frac{5}{2}}\sqrt[4]{3}}{(\sqrt[4]{16})^{3}\cdot 27^{-\frac{1}{4}}}$. Wyznaczyč $k \mathrm{t}\mathrm{a}\mathrm{k}$, by liczby

$m, k, n$ byly odpowiednio: pierwszym, drugim $\mathrm{i}$ trzecim wyrazem ciqgu geometrycznego,

a nstępnie wyznaczyč sumę wszystkich wyrazów nieskończonego ciągu geometrycznego,

którego pierwszymi trzema wyrazami są $m, k, n$. Ile wyrazów tego ciągu nalezy wziąč,

by ich suma przekroczyła 95\% sumy wszystkich wyrazów?

4. Narysowač wykres funkcji $f(x)=$ 

Poslugujqc się nim podač

wzór $\mathrm{i}$ narysowač wykres funkcji $g(m)$ określającej liczbę rozwiązań równania $f(x)=m,$

gdzie $m$ jest parametrem rzeczywistym.

5. Obliczyč tangens kąta wypukfego $\alpha$ spefniaj $\Phi^{\mathrm{c}\mathrm{e}\mathrm{g}\mathrm{o}}$ warunek $\sin\alpha-\cos\alpha=2\sqrt{6}\sin\alpha\cos\alpha.$

6. $\mathrm{W}$ trójkącie równoramiennym $ABC0$ podstawie $AB$ ramie ma dlugośč $b$, a kąt przy

wierzchofku C- miarę $\gamma. D$ jest takim punktem ramienia $BC, \dot{\mathrm{z}}\mathrm{e}$ odcinek $AD$ dzieli pole

trójkqta na polowę. Wyznaczyč promienie $\rho_{1}, \rho_{2}$ okręgów wpisanych $\mathrm{w}$ trójkąty $ABD\mathrm{i}$

$ADC$. Dla jakiego kąta $\gamma$ promienie te są równe, a dla jakiego $\rho_{1}=2\rho_{2}$?





XLII

KORESPONDENCYJNY KURS

Z MATEMATYKI

luty 2013 r.

PRACA KONTROLNA $\mathrm{n}\mathrm{r} 6-$ POZIOM PODSTAWOWY

l. Rozwiazač równanie

$\sqrt{2^{2x+1}-52^{x}+4}=2^{x+2}-5.$

2. Spośród cyfr liczby 2ll52ll25ll2 wylosowano trzy (bez zwracania). Obliczyč prawdo-

podobieństwo tego, $\dot{\mathrm{z}}\mathrm{e}$ liczba utworzona $\mathrm{z}$ wylosowanych cyfr nie jest podzielna przez

trzy.

3. Wyznaczyč dziedzinę funkcji

$f(x)=\sqrt{-\log_{2}\frac{3x}{x^{2}-4}}.$

4. 20 uczniów posadzono losowo $\mathrm{w}$ sali zawierającej 4 rzędy po 5 krzese1 $\mathrm{w}\mathrm{k}\mathrm{a}\dot{\mathrm{z}}$ dym. Obliczyč

prawdopodobieństwo tego, $\dot{\mathrm{z}}\mathrm{e}$ Bolek będzie siedział przy Lolku, $\mathrm{t}\mathrm{z}\mathrm{n}. \mathrm{z}$ przodu, $\mathrm{z}$ tylu, $\mathrm{z}$

prawej albo $\mathrm{z}$ lewej jego strony.

5. Uzasadnič, $\dot{\mathrm{z}}\mathrm{e}$ dla dowolnego $p$ oraz $x>-1$ prawdziwa jest nierównośč

$p^{2}+(1-p)^{2}x\displaystyle \geq\frac{x}{1+x}.$

Znalez/č $\mathrm{i}$ narysowač na pfaszczy $\acute{\mathrm{z}}\mathrm{n}\mathrm{i}\mathrm{e}$ zbiorów wszystkich par $(p,x)$, dla których $\mathrm{w}$ po-

$\mathrm{w}\mathrm{y}\dot{\mathrm{z}}$ szej nierówności ma miejsce równośč.

6. Trapez równoramienny ABCD $0$ polu $P$, ramieniu $c\mathrm{i}$ kącie ostrym przy podstawie $\alpha$

zgięto wzdluz jego osi symetrii $EF\mathrm{t}\mathrm{a}\mathrm{k}, \dot{\mathrm{z}}\mathrm{e}$ obie pofowy utworzyfy kąt $\alpha$. Obliczyč obję-

tośč powstałego $\mathrm{w}$ ten sposób wielościanu ABCDEF. Obliczyč tangens kąta nachylenia

do podstawy tej ściany bocznej, która nie jest prostopadła do podstawy. Sporz$\Phi$dzič

odpowiednie rysunki. Podač warunki istnienia rozwiązania.





PRACA KONTROLNA nr 6- POZIOM ROZSZERZONY

l. Rozwiązač równanie

$\sqrt{x^{2}-3}+2\sqrt{5-2x}=5-x.$

2. Wybrano losowo trzy krawędzie sześcianu. Obliczyč prawdopodobieństwo tego, $\dot{\mathrm{z}}\mathrm{e}\dot{\mathrm{z}}$ adne

dwie nie mają punktów wspólnych.

3. Gra $\mathrm{w}$ pary. $\mathrm{W}$ skarbonce znajduje się $\mathrm{d}\mathrm{u}\dot{\mathrm{z}}$ a liczba monet $0$ nominalach l $\mathrm{z}l$, 2 zł $\mathrm{i}$

5 $\mathrm{z}l. \mathrm{W}$ pierwszym kroku Jaś losuje trzy monety. Jesli wśród nich są dwie jednakowe,

to wrzuca je do skarbonki. $\mathrm{W}$ kolejnych krokach losuje ze skarbonki $\mathrm{k}\mathrm{a}\dot{\mathrm{z}}$ dorazowo tyle

monet, ile trzyma $\mathrm{w}$ rece, a nastepnie paryjednakowych monet wrzuca do skarbonki. Gra

kończy się, gdy wrzuci do skarbonki wszystkie monety. Obliczyč prawdopodobieństwo

tego, $\dot{\mathrm{z}}\mathrm{e}$ Jaś skończy grę: a) $\mathrm{w}$ drugim kroku; b) $\mathrm{w}$ drugim lub trzecim kroku.

4. Dane są wierzchołki $A(-3,2), C(4,2), D(0,4)$ trapezu równoramiennego ABCD, $\mathrm{w}$ któ-

rym $AB||CD$. Wyznaczyč współrzędne wierzcholka $B$ oraz równanie okręgu opisanego

na trapezie.

5. Udowodnič, $\dot{\mathrm{z}}\mathrm{e}$ dla $x>-1$ prawdziwa jest nierównośč podwójna

$1+\displaystyle \frac{x}{2}-\frac{x^{2}}{2}\leq\sqrt{1+x}\leq 1+\frac{x}{2}.$

Zilustrowač tę nierównośč odpowiednim rysunkiem.

6. $\mathrm{Z}$ dwóch przeciwlegfych wierzchołków prostokąta $0$ polu $P$, bdcego podstawą prosto-

padfościanu $0$ wysokości l, wystawiono po dwie przekątne sąsiednich ścian bocznych.

Wyrazič cosinus kąta pomiędzy plaszczyznami utworzonymi przez te pary przekątnych

jako funkcję sinusa kąta między nimi. Sporządzič rysunki.





XLII

KORESPONDENCYJNY KURS

Z MATEMATYKI

marzec 2013 r.

PRACA KONTROLNA nr 7- POZIOM PODSTAWOWY

l. Wyznaczyč rozwiazanie ogólne równania

$\displaystyle \sin(2x+\frac{\pi}{3})=\cos(x-\frac{\pi}{6}),$

a następnie podač rozwiązania $\mathrm{w}$ przedziale $[-2\pi,2\pi].$

2. Wyrazenie

$(\displaystyle \frac{a-2b}{\sqrt[3]{a^{2}}-\sqrt[3]{4b^{2}}}+\frac{\sqrt[3]{2a^{2}b}+\sqrt[3]{4ab^{2}}}{\sqrt[3]{a^{2}}+\sqrt[3]{4b^{2}}+\sqrt[3]{16ab}})$ : $\displaystyle \frac{a\sqrt[3]{a}+b\sqrt[3]{2b}+b\sqrt[3]{a}+a\sqrt[3]{2b}}{a+b}$

sprowadzič do najprostszej postaci. Przy jakich zalozeniach ma ono sens?

3. Narysowač wykres funkcji $f(x) =2|x|-\sqrt{x^{2}+4x+4}$ oraz wyznaczyč najmniejszą $\mathrm{i}$

największą wartośč funkcji $|f(x)|\mathrm{w}$ przedziale [-1, 2]. D1a jakiego $m$ pole figury ograni-

czonej wykresem funkcji $|f(x)|\mathrm{i}\mathrm{p}\mathrm{r}\mathrm{o}\mathrm{s}\mathrm{t}_{\Phi}y=m$ równe jest 16?

4. Rozwiązač układ równań

$\left\{\begin{array}{l}
x^{2}-4y^{2}+8y=4\\
x^{2}+y^{2}-2y=4
\end{array}\right.$

Podač interpretację geometryczną tego ukladu $\mathrm{i}$ obliczyč pole czworokata, którego wierz-

chofkami są cztery punkty będące jego rozwiązaniem.

5. $\mathrm{W}$ trapezie równoramiennym ABCD, $\mathrm{w}$ którym $BC||AD$ dane są $\vec{AB} = [1,-2]$ oraz

$\vec{AD}=[1$, 1$]$. Obliczyč pole trapezu $\mathrm{i}$ wyznaczyč $\mathrm{k}_{\Phi^{\mathrm{t}}}$ między jego przekątnymi.

6. $\mathrm{W}$ ostrosłupie prawidłowym trójkątnym cosinus kata nachylenia ściany bocznej do pod-

stawy równy jest $\displaystyle \frac{1}{9}$. Obliczyč stosunek pola powierzchni cafkowitej do pola podstawy.

Wykorzystując wzór $\sin 2\alpha=2\sin\alpha\cos\alpha$, wyznaczyč sinus kąta między ścianami bocz-

nymi tego ostrosfupa. Sporządzič rysunki.





PRACA KONTROLNA nr 7- POZIOM ROZSZERZONY

l. Rozwiązač równanie

$\displaystyle \sin x+\cos x=\frac{\cos 2x}{\sin 2x-1}$

2. Wyrazenie

$w(x,y)=\displaystyle \frac{x}{x^{3}+x^{2}y+xy^{2}+y^{3}}+\frac{y}{x^{3}-x^{2}y+xy^{2}-y^{3}}+\frac{1}{x^{2}-y^{2}}-\frac{1}{x^{2}+y^{2}}-\frac{x^{2}+2y^{2}}{x^{4}-y^{4}}$

doprowadzič do najprostszej postaci. Przy jakich zalozeniach ma ono sens? Obliczyč

$w(\cos 15^{\mathrm{o}},\sin 15^{\mathrm{o}}).$

3. Narysowač wykres funkcji

$f(x)=$

dla

dla

$x\leq 1,$

$x>1$

$\mathrm{i}$ posfugujac $\mathrm{s}\mathrm{i}\mathrm{e}$ nim wyznaczyč zbiór wartości funkcji $|f(x)|\mathrm{w}$ przedziale $[-\displaystyle \frac{1}{2},\frac{3}{2}].$

4. Rozwiązač ukfad równań

$\left\{\begin{array}{l}
y+x^{2}=4\\
4x^{2}-y^{2}+2y=1
\end{array}\right.$

Podač interpretację geometryczną tego ukladu $\mathrm{i}$ wykazač, $\dot{\mathrm{z}}\mathrm{e}$ cztery punkty, które są

jego rozwiązaniem, wyznaczaj $\Phi$ na płaszczy $\acute{\mathrm{z}}\mathrm{n}\mathrm{i}\mathrm{e}$ trapez równoramienny. Znalez/č równanie

okręgu opisanego na tym trapezie.

5. Odcinek $0$ końcach $A(0,7)\mathrm{i}B(5,2)$ jest przeciwprostokatna trójkąta prostokqtnego, któ-

rego wierzcholek $C\mathrm{l}\mathrm{e}\dot{\mathrm{z}}\mathrm{y}$ na prostej $x=3$. Posfugując się rachunkiem wektorowym ob-

liczyč cosinus kąta między dwusieczną kąta prostego a wysokością opuszczoną $\mathrm{z}$ wierz-

chofka $C.$

6. Pole powierzchni calkowitej ostrosfupa prawidfowego trójkątnego jest dziesięč razy więk-

sze $\mathrm{n}\mathrm{i}\dot{\mathrm{z}}$ pole jego podstawy. Wyznaczyč cosinus kąta między ścianami bocznymi oraz

stosunek objętości ostroslupa do objętości wpisanej $\mathrm{w}$ niego kuli.





XLII

KORESPONDENCYJNY KURS

Z MATEMATYKI

kwiecień 2013 r.

PRACA KONTROLNA $\mathrm{n}\mathrm{r} 8-$ POZIOM PODSTAWOWY

l. Cztery kolejne współczynniki wielomianu $f(x)=x^{3}+ax^{2}+bx+c$ tworzą $\mathrm{c}\mathrm{i}\otimes \mathrm{g}$ geome-

tryczny. Wiadomo, $\dot{\mathrm{z}}\mathrm{e}-3$ jest pierwiastkiem tego wielomianu. Wyznaczyč wspófczynniki

$a, b, c.$

2. Kolo $x^{2}+y^{2}+4x-2y-1\leq 0$ zostalo przesunięte $0$ wektor $\vec{w}=[3$, 3$]$. Znalez$\acute{}$č równanie

osi symetrii figury, która jest sumą kola $\mathrm{i}$ jego obrazu oraz obliczyč jej pole.

3. Podstawą ostroslupajest trójkąt $0$ bokach $\alpha, b, c$. Wszystkie kąty płaskie przy wierzchołku

ostroslupa są proste. Obliczyč objętośč ostroslupa.

4. Dane są punkty $A(0,2), B(4,4), C(3,6)$. Na prostej przechodzącej przez punkt $C$ rów-

noległej do prostej $AB$ znalez$\acute{}$č punkt $D$, który jest równo odlegly od punktów A $\mathrm{i}B.$

Wykazač, $\dot{\mathrm{z}}\mathrm{e}$ trójkąt $ABD$ jest prostokątny $\mathrm{i}$ napisač równanie okręgu opisanego na nim.

5. Wyznaczyč wartośč parametru $m$, dla którego równanie

$4x^{2}-2x\log_{2}m+1=0$

ma dwa rózne pierwiastki rzeczywiste $x_{1}, x_{2}$ spełniające warunek $x_{1}^{2}+x_{2}^{2}=1.$

6. Dane $\mathrm{s}\Phi$ funkcje $f(x)=4^{x-2}-7\cdot 3^{x-3}, g(x)=3^{3x+2}-5\cdot 4^{3x}$

Rozwiązač nierównośč

$f(x+3)>g(\displaystyle \frac{x}{3})$





PRACA KONTROLNA nr 8- POZIOM ROZSZERZONY

l. Niech $A$ bedzie wierzchołkiem kwadratu, a $M$ środkiem przeciwległego boku. Na prze-

$\mathrm{k}_{\Phi}$tnej kwadratu wychodzącej $\mathrm{z}$ wierzchofka $A$ wybrano punkt $P\mathrm{t}\mathrm{a}\mathrm{k}$, aby $|AP|=|MP|.$

Obliczyč, $\mathrm{w}$ jakim stosunku punkt $P$ dzieli przekatną kwadratu.

2. Stosując zasade indukcji matematycznej udowodnič nierównośč

$\left(\begin{array}{l}
2n\\
n
\end{array}\right) \displaystyle \leq\frac{4^{n}}{\sqrt{2n+2}},$

$n\geq 1.$

3. Wyznaczyč równanie okręgu $0$ środku lezącym na prostej $y-x=0$ oraz stycznego do

prostej $y-3=0\mathrm{i}$ do okręgu $x^{2}+y^{2}-4x+3=0$. Sporządzič rysunek.

4. Liczba $-2$ jest pierwiastkiem dwukrotnym wielomianu $w(x) = \displaystyle \frac{1}{2}x^{3}+ax^{2}+bx+c,$

a punkt $\mathrm{S}(-1,y_{0})$ jest środkiem symetrii wykresu $w(x)$. Wyznaczyč $a, b, c, y_{0}$ oraz trzeci

pierwiastek. Sporzqdzič wykres $w(x)\mathrm{w}$ przedziale $[-3,\displaystyle \frac{3}{2}].$

5. Wycinek kofa $0$ promieniu $3R\mathrm{i}\mathrm{k}_{\Phi}\mathrm{c}\mathrm{i}\mathrm{e}$ środkowym $\alpha$ zwinięto $\mathrm{w}$ powierzchnię boczną

stozka $S_{1}$. Podobnie, wycinek koła $0$ promieniu $R\mathrm{i}$ kqcie środkowym $ 3\alpha$ zwinieto $\mathrm{w}$ po-

wierzchnię boczną stozka $S_{2}$. Następnie obydwa stozki $\mathrm{z}l_{\Phi}$czono podstawami $\mathrm{t}\mathrm{a}\mathrm{k}$, aby

miafy wspólną oś obrotu, a ich wierzchofki byly skierowane $\mathrm{w}$ przeciwnych kierunkach.

Obliczyč promień kuli wpisanej $\mathrm{w}$ otrzymaną brylę. Sporzqdzič rysunek.

6. Podač interpretację geometryczną równania $\sqrt{2x+4}=mx+m+1\mathrm{z}$ parametrem $m.$

Graficznie $\mathrm{i}$ analitycznie określič, dla jakich wartości $m$ równanie ma dwa pierwiastki

$x_{1}=x_{1}(m), x_{2}=x_{2}(m)$. Nie korzystając $\mathrm{z}$ metod rachunku rózniczkowego, wykazač, $\dot{\mathrm{z}}\mathrm{e}$

funkcja $f(m)=x_{1}(m)+x_{2}(m)$ jest malejąca oraz sporządzič jej wykres.





XLII

KORESPONDENCYJNY KURS

Z MATEMATYKI

$\mathrm{p}\mathrm{a}\acute{\mathrm{z}}$dziernik 2012 $\mathrm{r}.$

PRACA KONTROLNA $\mathrm{n}\mathrm{r} 2-$ POZIOM PODSTAWOWY

l. Firma budowlana podpisała umowe na modernizację odcinka autostrady $0$ długości 21

km $\mathrm{w}$ określonym terminie. Ze względu na zblizające się mistrzostwa świata $\mathrm{w}$ rzu-

cie telefonem komórkowym postanowiono zrealizowač zamówienie 10 dni wcześniej, co

oznaczafo koniecznośč zwiększenia średniej normy dziennej $0$ 5\%. $\mathrm{W}$ jakim czasie firma

zamierzafa pierwotnie zrealizowač to zamówienie?

2. Pan Kowalski zaciągnął $\mathrm{w}$ banku kredyt $\mathrm{w}$ wysokości 4000 zł oprocentowany na 16\% $\mathrm{w}$

skali roku. Zgodnie $\mathrm{z}$ umową będzie go spłacaf $\mathrm{w}$ czterech ratach co 3 miesiące, spfacając

za $\mathrm{k}\mathrm{a}\dot{\mathrm{z}}$ dym razem 1000zł oraz 4\% pozosta1ego zadłuzenia. I1e złotych ostatecznie zwróci

bankowi pan Kowalski?

3. Ile jest czterocyfrowych liczb naturalnych:

a) podzielnych przez 2, 31ub przez 5?

b) podzielnych przez dokfadnie dwie spośród powyzszych liczb?

4. Na paraboli $y=x^{2}-6x+11$ znalez/č taki punkt $C, \dot{\mathrm{z}}\mathrm{e}$ pole trójkąta $0$ wierzchołkach

$A=(0,3), B=(4,0), C$ jest najmniejsze.

5. Przy prostoliniowej ulicy (oś Ox) $\mathrm{w}$ punkcie $x=0$ zainstalowano parkomat. $\mathrm{W}$ punkcie

$x=1 \mathrm{m}\mathrm{o}\dot{\mathrm{z}}$ na korzystač $\mathrm{z}$ bankomatu, a $\mathrm{w}$ punkcie $x=-2$ jest wejście do galerii han-

dlowej. $\mathrm{W}$ którym punkcie $x$ ulicy nalezy zaparkowač samochód, aby droga przebyta

od samochodu do parkomatu $\mathrm{i}\mathrm{z}$ powrotem (bilet parkingowy nalez $\mathrm{y}$ pofozyč za szybą

pojazdu), następnie do bankomatu po pieniądze, stąd do galerii $\mathrm{i}$ na końcu $\mathrm{z}$ zakupami

do samochodu, byfa najkrótsza? Jaka będzie odpowied $\acute{\mathrm{z}}$, gdy wejście do galerii będzie $\mathrm{w}$

punkcie $x=2$? $\mathrm{W}$ obu przypadkach podač wzór $\mathrm{i}$ narysowač wykres funkcji określającej

droge przebytq przez klienta domu handlowego $\mathrm{w}$ zalezności od punktu zaparkowania

samochodu.

6. Wykonač działania $\mathrm{i}$ zapisač $\mathrm{w}$ najprostszej postaci wyrazenie

$w(a,b)= (\displaystyle \frac{a}{a^{2}-ab+b^{2}}-\frac{a^{2}}{a^{3}+b^{3}})$ : $(\displaystyle \frac{a^{3}-b^{3}}{\alpha^{3}+b^{3}}-\frac{\alpha^{2}+b^{2}}{a^{2}-b^{2}})$

Wykazač, $\dot{\mathrm{z}}\mathrm{e}$ dla dowolnych $a<0$ zachodzi nierównośč $w(-a,a^{-1})\geq 1$, a dla dowolnych

$a>0$ prawdziwa jest nierównośč $w(-a,a^{-1})\leq 1.$





PRACA KONTROLNA nr 2- POZIOM ROZSZERZONY

l. Rozwiązač nierównośč $\displaystyle \frac{1}{\sqrt{5+4x-x^{2}}}\geq\frac{1}{|x|-2}\mathrm{i}$ zbiór rozwiązań zaznaczyč na osi liczbo-

wej.

2. Dwaj rowerzyści wyjechali jednocześnie naprzeciw siebie $\mathrm{z}$ miast A $\mathrm{i}\mathrm{B}$ odlegfych $030$

kilometrów. Minęli się po godzinie $\mathrm{i}$ nie zatrzymując się podqzyli $\mathrm{z}$ tymi samymi pręd-

kościami $\mathrm{k}\mathrm{a}\dot{\mathrm{z}}\mathrm{d}\mathrm{y}\mathrm{w}$ swoim kierunku. Rowerzysta, który wyjechal $\mathrm{z}$ A dotarf do $\mathrm{B}$ póftorej

godziny wcześniej $\mathrm{n}\mathrm{i}\dot{\mathrm{z}}$ jego kolegajadący $\mathrm{z}\mathrm{B}$ dotarł do A. $\mathrm{Z}$ jakimi prędkościami jechali

rowerzyści?

3. Pan Kowalski zaciągnąf 3l grudnia $\mathrm{p}\mathrm{o}\dot{\mathrm{z}}$ yczkę 4000 z1otych $\mathrm{o}\mathrm{P}^{\mathrm{r}\mathrm{o}\mathrm{c}\mathrm{e}\mathrm{n}\mathrm{t}\mathrm{o}\mathrm{w}\mathrm{a}\mathrm{n}}\Phi^{\mathrm{W}}$ wysokości

16\% $\mathrm{w}$ skali roku. Zobowiqzał się splacič ją $\mathrm{w}$ ciągu roku $\mathrm{w}$ czterech równych ratach

platnych 30. marca, 30. czerwca, 30. września $\mathrm{i}30$. grudnia. Oprocentowanie $\mathrm{p}\mathrm{o}\dot{\mathrm{z}}$ yczki

liczy się od l stycznia, a odsetki od kredytu naliczane są $\mathrm{w}$ terminach pfatności rat.

Obliczyč wysokośč tych rat $\mathrm{w}$ zaokrągleniu do pełnych groszy.

4. Dla jakiego parametru $m$ równanie

$2x^{2}-(2m+1)x+m^{2}-9m+39=0$

ma dwa pierwiastki, $\mathrm{z}$ których jeden jest dwa razy większy $\mathrm{n}\mathrm{i}\dot{\mathrm{z}}$ drugi?

5. Ile jest liczb pięciocyfrowych podzielnych przez 9, które $\mathrm{w}$ rozwinięciu dziesiętnym mają:

a) obie cyfry 1, 2 $\mathrm{i}$ tylko $\mathrm{t}\mathrm{e}$? b) obie cyfry 1, 3 $\mathrm{i}$ tylko $\mathrm{t}\mathrm{e}$? c) wszystkie cyfry 1, 2, 3

$\mathrm{i}$ tylko $\mathrm{t}\mathrm{e}$? Odpowiedz/uzasadnič. $\mathrm{W}$ przypadku b) wypisač otrzymane liczby.

6. $\mathrm{Z}$ przystani A wyrusza $\mathrm{z}$ biegiem rzeki statek do przystani $\mathrm{B}$, odleglej od A $0140$ km. Po

uplywie l godziny wyrusza za nim fódz/ motorowa, dopędza statek, po czym wraca do

przystani A $\mathrm{w}$ tym samym momencie, $\mathrm{w}$ którym statek przybija do przystani B. Predkośč

łodzi $\mathrm{w}$ wodzie stojącej jest póltora raza większa $\mathrm{n}\mathrm{i}\dot{\mathrm{z}}$ prędkośč statku $\mathrm{w}$ wodzie stojącej.

Wyznaczyč te prędkości $\mathrm{w}\mathrm{i}\mathrm{e}\mathrm{d}\mathrm{z}\Phi^{\mathrm{C}}, \dot{\mathrm{z}}\mathrm{e}$ rzeka plynie $\mathrm{z}$ prędkością 4 $\mathrm{k}\mathrm{m}/$godz.





XLII

KORESPONDENCYJNY KURS

Z MATEMATYKI

listopad 2012 r.

PRACA KONTROLNA $\mathrm{n}\mathrm{r} 3-$ POZIOM PODSTAWOWY

1. $\mathrm{Z}$ danych Głównego Urzędu Statystycznego wynika, $\dot{\mathrm{z}}\mathrm{e}$ wzrost Produktu Krajowego Brut-

to (PKB) $\mathrm{w}$ Polsce $\mathrm{w}$ roku 2010 wyniósf 3,7\%, a $\mathrm{w}$ roku 2011- 4,3\%. Jaki powinien byč

wzrost PKB $\mathrm{w}$ roku 2012, by średni roczny wzrost PKB $\mathrm{w}$ tych trzech latach wyniósł

4\%? Podač wynik $\mathrm{z}$ dokładnością do 0, 001\%.

2. Czy liczby $\sqrt{2}$, 2, $2\sqrt{2}$ mogą byč wyrazami (niekoniecznie kolejnymi) ciągu arytme-

tycznego? Odpowiedz/uzasadnič.

3. Wielomian $W(x) = x^{5}+ax^{4}+bx^{3}+4x$ jest podzielny przez $(x^{2}-1)$. Wyznaczyč

wspólczynniki $a, b\mathrm{i}$ rozwiązač nierównośč $W(x-1)\leq W(x)\leq W(x+1).$

4. Niech $f(x)=\sqrt{x}, g(x)=x-2, h(x)=|x|$. Narysowač wykresy funkcji zlozonych: $f0h\circ g,$

$f$ o{\it g}o $h$, {\it go} $f\mathrm{o}h$, {\it goho} $f, h\mathrm{o}f\mathrm{o}g$ {\it oraz hogo} $f.$

5. Przyprostokątną trójkata prostokątnego $ABC$ jest odcinek AB $0$ końcach $A(-2,2) \mathrm{i}$

$B(1,-1)$, a wierzchofek $C$ trójk$\Phi$ta $\mathrm{l}\mathrm{e}\dot{\mathrm{z}}\mathrm{y}$ na prostej $3x-y= 14$. Wyznaczyč równanie

okręgu opisanego na tym trójkącie. Ile rozwiązań ma to zadanie? Sporządzič rysunek.

6. Na prostej $x+2y=5$ wyznaczyč punkty, $\mathrm{z}$ których okrag $(x-1)^{2}+(y-1)^{2}=1$ jest

widoczny pod kątem $60^{\mathrm{o}}$. Obliczyč pole obszaru ograniczonego fukiem okręgu $\mathrm{i}$ stycznymi

do niego poprowadzonymi $\mathrm{w}$ znalezionych punktach. Sporządzič rysunek.





PRACA KONTROLNA nr 3- POZIOM ROZSZERZONY

l. Pan Kowalski umieścił swoje oszczędności na dwu róznych lokatach. Pieniądze, otrzy-

mane jako honorarium za podręcznik, zfozyl na lokacie oprocentowanej $\mathrm{w}$ wysokości 7\%

$\mathrm{w}$ skali roku, a wynagrodzenie za cykl wykładów- na lokacie 9\%. Po roku jego dochód

był $030$ zlotych, a po dwu latach- $070$ złotych $\mathrm{w}\mathrm{y}\dot{\mathrm{z}}$ szy od dochodu, który uzyskałby

skfadając cafą sumę na lokacie 8\%. I1e pieniędzy otrzyma1 pan Kowa1ski za podręcznik,

a ile za wykłady?

2. Czy liczby $\sqrt{2}, \sqrt{3}$, 2 mogą byč wyrazami (niekoniecznie kolejnymi) $\mathrm{c}\mathrm{i}_{\Phi \mathrm{g}}\mathrm{u}$ arytmetycz-

nego? Odpowiedz/uzasadnič.

3. Niech $f(x)=2^{x}, g(x)=2-x, h(x)=|x|$. Narysowač wykresy funkcji złozonych $f\mathrm{o}g\mathrm{o}h$

oraz $g\mathrm{o}f\mathrm{o}h\mathrm{i}$ rozwiązač nierównośč $(f\mathrm{o}g\mathrm{o}h)(x)<6+(g\mathrm{o}f\mathrm{o}h)(x).$

4. Dane są punkty $A(1,2), B(3,1).$

takich, $\dot{\mathrm{z}}\mathrm{e}\mathrm{k}\mathrm{a}\mathrm{t}BCA$ ma miarę $45^{\mathrm{o}}$

Wyznaczyč równanie zbioru wszystkich punktów C

5. Liczby: $a_{1}=\log_{(3-2\sqrt{2})^{2}}(\sqrt{2}-1), a_{2}=\displaystyle \frac{1}{2}\log_{\frac{1}{3}}\frac{\sqrt{3}}{6}, \alpha_{3}=3^{\log_{\sqrt{3}}\frac{\sqrt{6}}{2}}, a_{4}=\log_{(\sqrt{2}-1)}(\sqrt{2}+1),$

$a_{5}=(2^{\sqrt{2}+1})^{\sqrt{2}-1}, a_{6}=\log_{3}2$ są jedynymi pierwiastkami wielomianu $W(x)$, którego

wyraz wolny jest dodatni.

a) Które $\mathrm{z}$ tych pierwiastków są niewymierne? Odpowied $\acute{\mathrm{z}}$ uzasadnič.

b) Wyznaczyč dziedzinę funkcji $f(x)=\sqrt{W(x)}$, nie wykonując obliczeń przyblizonych.

6. Niech $f(x)=3(x+2)^{4}+x^{2}+4x+p$, gdzie $p$ jest parametrem rzeczywistym.

a) Uzasadnič, $\dot{\mathrm{z}}\mathrm{e}$ wykres funkcji $f(x)$ jest symetryczny względem prostej $x=-2.$

b) Dla jakiego parametru $p$ najmniejszą wartością funkcji $f(x)$ jest $y=-2$ ?

Odpowiedz/uzasadnič, nie stosując metod rachunku rózniczkowego.

c) Określič liczbę rozwiqzań równania $f(x)=0\mathrm{w}$ zalezności od parametru $p.$





XLII

KORESPONDENCYJNY KURS

Z MATEMATYKI

grudzień 2012 r.

PRACA KONTROLNA $\mathrm{n}\mathrm{r} 4-$ POZIOM PODSTAWOWY

l. Wyznaczyč wszystkie kąty $\alpha \mathrm{z}$ przedziału $[0,2\pi]$, dla których suma kwadratów pierwiast-

ków rzeczywistych równania $x^{2}+2x\sin\alpha-\cos^{2}\alpha=0$ jest równa co najwyzej 3.

2. Uzasadnič, $\dot{\mathrm{z}}\mathrm{e}$ suma średnic okręgu opisanego na trójkącie prostokqtnym $\mathrm{i}$ okręgu wpisa-

nego $\mathrm{w}$ ten trójkąt jest równa sumie długości przyprostokątnych. Znalez$\acute{}$č dlugości boków

trójkąta, $\mathrm{j}\mathrm{e}\dot{\mathrm{z}}$ eli promienie tych okręgów są równe $R=5\mathrm{i}r=2.$

3. Narysowač wykres funkcji $f(x)=\cos^{2}x+|\sin x|\sin x \mathrm{w}$ przedziale $[-2\pi,2\pi].$

a) Podač zbiór wartości $\mathrm{i}$ miejsca zerowe.

b) Wyznaczyč przedziafy monotoniczności.

c) Rozwiązač nierównośč $|f(x)|\displaystyle \geq\frac{1}{2}.$

4. $\mathrm{W}$ kwadracie $0$ boku dfugości $a$ narysowano cztery pólkola, których średnicami są boki

kwadratu. Pólkola przecinają $\mathrm{s}\mathrm{i}\mathrm{e}$ parami tworząc czterolistną rozetę. Obliczyč pole $\mathrm{i}$

obwód rozety.

5. Dach wiez $\mathrm{y}$ kościola ma kształt ostrosłupa, którego podstawą jest sześciokąt foremny $0$

boku 4 $\mathrm{m}$ a największy $\mathrm{z}$ przekrojów płaszczyzną zawierajqcq wysokośč jest trójkątem

równobocznym. Obliczyč kubaturę dachu wiez $\mathrm{y}$ kościofa. Ile 2-1itrowych puszek farby

antykorozyjnej trzeba kupič do pomalowania blachy, którą pokryty jest dach, $\mathrm{j}\mathrm{e}\dot{\mathrm{z}}$ eli wia-

domo, $\dot{\mathrm{z}}\mathrm{e}$ llitr farby wystarcza do pomalowania 6 $\mathrm{m}^{2}$ blachy $\mathrm{i}$ trzeba uwzględnič 8\%

farby $\mathrm{n}\mathrm{a}$ ewentualne straty.

6. Promień kuli opisanej na ostrosłupie prawidlowym czworokątnym wynosi $R$. Prosto-

padfa wyprowadzona ze środka kuli do ściany bocznej ostroslupa tworzy $\mathrm{z}$ wysokością

ostrosłupa kąt $\alpha$. Wyznaczyč wysokośč ostroslupa.





PRACA KONTROLNA nr 4- POZIOM ROZSZERZONY

l. Dla jakich katów $\alpha \mathrm{z}$ przedziału $[0,\displaystyle \frac{\pi}{2}]$ równanie $x^{2}\sin\alpha+x+\cos\alpha=0$ ma dwa róz-

ne pierwiastki rzeczywiste? Czy iloczyn pierwiastków równania $\mathrm{m}\mathrm{o}\dot{\mathrm{z}}\mathrm{e}$ byč równy $\sqrt{3}$?

Wyznaczyč wszystkie kąty $\alpha$, dla których suma pierwiastków jest większa od $-2.$

2. Przekrój ostroslupa prawidłowego czworokątnego plaszczyzną przechodzącą przez prze-

kątnq podstawy $\mathrm{i}$ wierzcholek ostrosłupajest trójkątem równobocznym. Wyznaczyč sto-

sunek promienia kuli wpisanej $\mathrm{w}$ ostrosfup do promienia kuli opisanej na ostrosfupie.

3. Narysowač wykres funkcji $f(x)=\displaystyle \frac{\sin 2x-|\sin x|}{\sin x}.$

równośč $f(x)<2(\sqrt{2}-1)\cos^{2}x.$

$\mathrm{W}$ przedziale $[0,2\pi]$ rozwiązač nie-

4. $\mathrm{C}\mathrm{z}\mathrm{w}\mathrm{o}\mathrm{r}\mathrm{o}\mathrm{k}_{\Phi^{\mathrm{t}}}$ wypukfy ABCD, $\mathrm{w}$ którym $AB=1, BC=2, CD=4, DA=3$ jest wpisany

$\mathrm{w}$ okrąg. Obliczyč promień $R$ tego okręgu. Sprawdzič, czy $\mathrm{w}$ ten czworokąt $\mathrm{m}\mathrm{o}\dot{\mathrm{z}}$ na wpisač

okrqg. $\mathrm{J}\mathrm{e}\dot{\mathrm{z}}$ eli $\mathrm{t}\mathrm{a}\mathrm{k}$, to obliczyč jego promień.

5. $\mathrm{W}$ kole $K\mathrm{o}$ promieniu 4 cm narysowano 6 kó1 $0$ promieniu 2 cm $\mathrm{P}^{\mathrm{r}\mathrm{z}\mathrm{e}\mathrm{c}\mathrm{h}\mathrm{o}\mathrm{d}\mathrm{z}}\Phi^{\mathrm{c}\mathrm{y}\mathrm{c}\mathrm{h}}$ przez

środek kola $K\mathrm{i}$ stycznych do niego $\mathrm{t}\mathrm{a}\mathrm{k}$, aby środki tych sześciu kół były wierzchołkami

sześciok$\Phi$ta foremnego. Obliczyč pole $\mathrm{i}$ obwód figury, która jest $\mathrm{s}\mathrm{u}\mathrm{m}\Phi$ tych sześciu kóf.

6. Stosunek pola powierzchni bocznej stozka ściętego do pola powierzchni wpisanej $\mathrm{w}$ ten

stozek kuli wyrazič jako funkcję kata nachylenia tworzącej stozka do podstawy.





XLII

KORESPONDENCYJNY KURS

Z MATEMATYKI

styczeń 2013 r.

PRACA KONTROLNA $\mathrm{n}\mathrm{r} 5-$ POZIOM PODSTAWOWY

l. Między $\mathrm{k}\mathrm{a}\dot{\mathrm{z}}$ de dwa kolejne wyrazy pięcioelementowego ciągu arytmetycznego wstawiono

$m$ liczb, otrzymując ciąg arytmetyczny, którego sumajest 13 razy większa $\mathrm{n}\mathrm{i}\dot{\mathrm{z}}$ suma wyj-

ściowego ciągu. Obliczyč $m$. Jaką jednakową ilośč liczb nalez $\mathrm{y}$ wstawič miedzy $\mathrm{k}\mathrm{a}\dot{\mathrm{z}}$ de dwa

kolejne wyrazy $n$ elementowego ciągu arytmetycznego, aby otrzymač ciąg arytmetyczny

$0$ sumie $n$ razy większej $\mathrm{n}\mathrm{i}\dot{\mathrm{z}}$ suma wyjściowego ciągu?

2. Linie kolejowe malują wagony klasy standard na niebiesko, klasy komfort na rózowo,

a klasy biznes na szaro. Na ile sposobów $\mathrm{m}\mathrm{o}\dot{\mathrm{z}}$ na zestawič skfad pięciowagonowy, który

zawiera co najmniej jeden wagon $\mathrm{k}\mathrm{a}\dot{\mathrm{z}}$ dej klasy, a kolejnośč wagonów jest istotna?

3. Niech $n$ będzie liczbą naturalną. $\mathrm{W}$ przedziale $[0,2\pi]$ rozwiqzač równanie

$1+\cos^{2}x+\cos^{4}x+\cdots+\cos^{2n}x=2-\cos^{2n}x.$

4. Zawodnik przebiegf równym tempem pierwsze l0 km biegu maratońskiego $(42\mathrm{k}\mathrm{m})\mathrm{w}$ cza-

sie 45 minut, a $\mathrm{k}\mathrm{a}\dot{\mathrm{z}}\mathrm{d}\mathrm{y}$ kolejny kilometr pokonywał $\mathrm{w}$ czasie $0$ 5\% dluzszym $\mathrm{n}\mathrm{i}\dot{\mathrm{z}}$ poprzedni.

Sprawdzič, czy zawodnik zmieścil się $\mathrm{w}$ sześciogodzinnym limicie czasowym.

5. Rozwiązač nierównośč

$\log_{2}(x+2)-\log_{4}(4-x^{2})\geq 0.$

6. Niech $A=\{(x,y):|x|+2|y|\leq 2\}$. Zbiór $B$ powstaje przez obrót figury $A0$ kąt $\displaystyle \frac{\pi}{2} (\mathrm{w}$

kierunku przeciwnym do ruchu wskazówek zegara) wokół początku układu współrzęd-

nych. Starannie narysowač zbiory $A\cup B$ oraz $A\triangle B=(A\backslash B)\cup(B\backslash A)\mathrm{i}$ obliczyč ich

pola.





PRACA KONTROLNA nr 5- POZIOM ROZSZERZONY

l. Zbadač, dla jakich argumentów funkcja

przyjmuje wartości ujemne.

$g(x)=2^{x^{3}-5} 3^{7x^{2}}\cdot 4^{7x-1}-2^{7x^{2}+1}$

$3^{x^{3}-2} 9^{7x-3}$

2. Rozwiązač nierównośč

$2^{-\sin x}+2^{-2\sin x}+2^{-3\sin x}+\ldots\leq\sqrt{2}+1,$

której lewa strona jest sumą nieskończonego ciągu geometrycznego.

3. Podač dziedzinę i wyznaczyč wszystkie miejsca zerowe funkcji

$f(x)=\displaystyle \log_{x+1}(x-1)-\log_{x+1}(2x-\frac{2}{x})+1.$

4. Dany jest ciąg liczbowy $(a_{n}), \mathrm{w}$ którym $\mathrm{k}\mathrm{a}\dot{\mathrm{z}}\mathrm{d}\mathrm{y}$ wyraz jest sumą podwojonego wyrazu

poprzedniego $\mathrm{i}4$, a jego czwarty wyraz wynosi 36. Podač wzór na n-ty wyraz $\mathrm{c}\mathrm{i}_{\Phi \mathrm{g}}\mathrm{u}\mathrm{i}$

udowodnič go, wykorzystując zasadę indukcji matematycznej.

5. Niech $A=\{(x,y):|x|+2|y|\leq 2\}$. Zbiór $B$ otrzymano przez obrót $A0$ kąt $\displaystyle \frac{\pi}{2}(\mathrm{w}$ kierunku

przeciwnym do ruchu wskazówek zegara) wokóf $\mathrm{P}^{\mathrm{o}\mathrm{c}\mathrm{z}}\Phi^{\mathrm{t}\mathrm{k}\mathrm{u}}$ ukladu wspólrzędnych, a zbiór

C- przez obrót zbioru $A\cup B0$ kąt $\displaystyle \frac{\pi}{4}$ wokól początku układu współrzędnych. Wykonač

staranny rysunek zbioru $A\cup B\cup C$ oraz obliczyč jego pole.

6. Boki $\triangle ABC$ zawarte są $\mathrm{w}$ prostych $y=2x+m, y=mx+1$ oraz $2y=2-x$. Podač wartośč

rzeczywistego parametru $m\displaystyle \in(-\frac{1}{2},2)$, dla której pole rozwazanego trójkąta wynosi $\displaystyle \frac{1}{5}.$

Dla wyznaczonego $m$ wykonač staranny rysunek (przyjąč jednostkę równą 3 cm).



\end{document}