\documentclass[a4paper,12pt]{article}
\usepackage{latexsym}
\usepackage{amsmath}
\usepackage{amssymb}
\usepackage{graphicx}
\usepackage{wrapfig}
\pagestyle{plain}
\usepackage{fancybox}
\usepackage{bm}

\begin{document}

XLII

KORESPONDENCYJNY KURS

Z MATEMATYKI

wrzesień 2012 r.

PRACA KONTROLNA $\mathrm{n}\mathrm{r} 1 -$ POZIOM PODSTAWOWY

l. Niech $A=\displaystyle \{x\in \mathbb{R}:\frac{1}{x^{2}+1}\geq\frac{1}{7-x}\}$ oraz $B=\{x\in \mathbb{R}:|x-2|+|x-7|<7\}$. Znalez/č

$\mathrm{i}$ zaznaczyč na osi liczbowej zbiory $A, B$ oraz $(A\backslash B)\cup(B\backslash A).$

2. Liczba $p=\displaystyle \frac{(\sqrt[3]{54}-2)(9\sqrt[3]{4}+6\sqrt[3]{2}+4)-(2-\sqrt{3})^{3}}{\sqrt{3}+(1+\sqrt{3})^{2}}$ jest miejscem zerowym funkcji

$f(x)=ax^{2}+bx+c$. Pole trójkąta, którego wierzcholkami są punkty przecięcia wykresu

$\mathrm{z}$ osiami układu współrzędnych równe jest 20. Wyznaczyč współczynnik $b$ oraz drugie

miejsce zerowe tej funkcji $\mathrm{w}\mathrm{i}\mathrm{e}\mathrm{d}\mathrm{z}\Phi^{\mathrm{C}}, \dot{\mathrm{z}}\mathrm{e}$ wykres funkcji jest symetryczny względem prostej

$x=3.$

3. Trapez $0$ kqtach przy podstawie $30^{\mathrm{o}}$ oraz $45^{\mathrm{o}}$ jest opisany na okręgu $0$ promieniu $R.$

Obliczyč stosunek pola kola do pola trapezu.

4. Niech $f(x)=$

Obliczyč $f(\displaystyle \frac{1+\sqrt{3}}{2})$ oraz $f(\displaystyle \frac{\pi+1}{\pi-2}).$

Narysowač wykres funkcji $f\mathrm{i}$ na jego podstawie podač zbiór wartości funkcji oraz roz-

wiqzač nierównośč $f(x)\displaystyle \geq-\frac{1}{2}.$

5. Tangens kąta ostrego $\alpha$ równy jest $\displaystyle \frac{a}{7b}$, gdzie

$a=(\sqrt{2}+1)^{3}-(\sqrt{2}-1)^{3},b=(\sqrt{\sqrt{2}+1}-\sqrt{\sqrt{2}-1})^{2}$

Wyznaczyč wartości pozostałych funkcji trygonometrycznych tego kąta oraz kąta $2\alpha.$

6. $\mathrm{W}$ trójk$\Phi$t otrzymany $\mathrm{w}$ przekroju ostrosłupa prawidłowego czworokątnego pfaszczyzną

przechodzącą przez wysokośč ostrosłupa $\mathrm{i}$ przekątną jego podstawy wpisano kwadrat,

którego jeden bok jest zawarty $\mathrm{w}$ przekatnej podstawy. Pole kwadratu jest dwa ra-

zy mniejsze $\mathrm{n}\mathrm{i}\dot{\mathrm{z}}$ pole podstawy ostrosfupa. Obliczyč stosunek pola powierzchni bocznej

ostrosłupa do pola jego podstawy oraz cosinus kąta między ścianami bocznymi.
\end{document}
