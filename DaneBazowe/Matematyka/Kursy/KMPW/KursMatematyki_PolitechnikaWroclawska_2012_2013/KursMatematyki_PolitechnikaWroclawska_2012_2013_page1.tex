\documentclass[a4paper,12pt]{article}
\usepackage{latexsym}
\usepackage{amsmath}
\usepackage{amssymb}
\usepackage{graphicx}
\usepackage{wrapfig}
\pagestyle{plain}
\usepackage{fancybox}
\usepackage{bm}

\begin{document}

PRACA KONTROLNA nr l- POZIOM ROZSZERZONY

l. Niech $A=\{(x,y):y\geq||x-2|-1|\}, B=\{(x,y):y+\sqrt{4x-x^{2}-3}\leq 2\}$. Narysowač

na pfaszczy $\acute{\mathrm{z}}\mathrm{n}\mathrm{i}\mathrm{e}$ zbiór $A\cap B\mathrm{i}$ obliczyč jego pole.

2. Pole powierzchni bocznej ostrosłupa prawidłowego trójkątnego jest $\mathrm{k}$ razy większe $\mathrm{n}\mathrm{i}\dot{\mathrm{z}}$

pole jego podstawy. Obliczyč cosinus kata nachylenia krawędzi bocznej ostroslupa do

pfaszczyzny podstawy.

3. Dane są liczby: $m = \displaystyle \frac{\left(\begin{array}{l}
6\\
4
\end{array}\right)\left(\begin{array}{l}
8\\
2
\end{array}\right)}{\left(\begin{array}{l}
7\\
3
\end{array}\right)}, n = \displaystyle \frac{(\sqrt{2})^{-4}(\frac{1}{4})^{-\frac{5}{2}}\sqrt[4]{3}}{(\sqrt[4]{16})^{3}\cdot 27^{-\frac{1}{4}}}$. Wyznaczyč $k \mathrm{t}\mathrm{a}\mathrm{k}$, by liczby

$m, k, n$ byly odpowiednio: pierwszym, drugim $\mathrm{i}$ trzecim wyrazem ciqgu geometrycznego,

a nstępnie wyznaczyč sumę wszystkich wyrazów nieskończonego ciągu geometrycznego,

którego pierwszymi trzema wyrazami są $m, k, n$. Ile wyrazów tego ciągu nalezy wziąč,

by ich suma przekroczyła 95\% sumy wszystkich wyrazów?

4. Narysowač wykres funkcji $f(x)=$ 

Poslugujqc się nim podač

wzór $\mathrm{i}$ narysowač wykres funkcji $g(m)$ określającej liczbę rozwiązań równania $f(x)=m,$

gdzie $m$ jest parametrem rzeczywistym.

5. Obliczyč tangens kąta wypukfego $\alpha$ spefniaj $\Phi^{\mathrm{c}\mathrm{e}\mathrm{g}\mathrm{o}}$ warunek $\sin\alpha-\cos\alpha=2\sqrt{6}\sin\alpha\cos\alpha.$

6. $\mathrm{W}$ trójkącie równoramiennym $ABC0$ podstawie $AB$ ramie ma dlugośč $b$, a kąt przy

wierzchofku C- miarę $\gamma. D$ jest takim punktem ramienia $BC, \dot{\mathrm{z}}\mathrm{e}$ odcinek $AD$ dzieli pole

trójkqta na polowę. Wyznaczyč promienie $\rho_{1}, \rho_{2}$ okręgów wpisanych $\mathrm{w}$ trójkąty $ABD\mathrm{i}$

$ADC$. Dla jakiego kąta $\gamma$ promienie te są równe, a dla jakiego $\rho_{1}=2\rho_{2}$?
\end{document}
