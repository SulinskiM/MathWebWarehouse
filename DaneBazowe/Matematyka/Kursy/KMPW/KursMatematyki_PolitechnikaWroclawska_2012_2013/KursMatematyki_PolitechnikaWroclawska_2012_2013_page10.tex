\documentclass[a4paper,12pt]{article}
\usepackage{latexsym}
\usepackage{amsmath}
\usepackage{amssymb}
\usepackage{graphicx}
\usepackage{wrapfig}
\pagestyle{plain}
\usepackage{fancybox}
\usepackage{bm}

\begin{document}

XLII

KORESPONDENCYJNY KURS

Z MATEMATYKI

luty 2013 r.

PRACA KONTROLNA $\mathrm{n}\mathrm{r} 6-$ POZIOM PODSTAWOWY

l. Rozwiazač równanie

$\sqrt{2^{2x+1}-52^{x}+4}=2^{x+2}-5.$

2. Spośród cyfr liczby 2ll52ll25ll2 wylosowano trzy (bez zwracania). Obliczyč prawdo-

podobieństwo tego, $\dot{\mathrm{z}}\mathrm{e}$ liczba utworzona $\mathrm{z}$ wylosowanych cyfr nie jest podzielna przez

trzy.

3. Wyznaczyč dziedzinę funkcji

$f(x)=\sqrt{-\log_{2}\frac{3x}{x^{2}-4}}.$

4. 20 uczniów posadzono losowo $\mathrm{w}$ sali zawierającej 4 rzędy po 5 krzese1 $\mathrm{w}\mathrm{k}\mathrm{a}\dot{\mathrm{z}}$ dym. Obliczyč

prawdopodobieństwo tego, $\dot{\mathrm{z}}\mathrm{e}$ Bolek będzie siedział przy Lolku, $\mathrm{t}\mathrm{z}\mathrm{n}. \mathrm{z}$ przodu, $\mathrm{z}$ tylu, $\mathrm{z}$

prawej albo $\mathrm{z}$ lewej jego strony.

5. Uzasadnič, $\dot{\mathrm{z}}\mathrm{e}$ dla dowolnego $p$ oraz $x>-1$ prawdziwa jest nierównośč

$p^{2}+(1-p)^{2}x\displaystyle \geq\frac{x}{1+x}.$

Znalez/č $\mathrm{i}$ narysowač na pfaszczy $\acute{\mathrm{z}}\mathrm{n}\mathrm{i}\mathrm{e}$ zbiorów wszystkich par $(p,x)$, dla których $\mathrm{w}$ po-

$\mathrm{w}\mathrm{y}\dot{\mathrm{z}}$ szej nierówności ma miejsce równośč.

6. Trapez równoramienny ABCD $0$ polu $P$, ramieniu $c\mathrm{i}$ kącie ostrym przy podstawie $\alpha$

zgięto wzdluz jego osi symetrii $EF\mathrm{t}\mathrm{a}\mathrm{k}, \dot{\mathrm{z}}\mathrm{e}$ obie pofowy utworzyfy kąt $\alpha$. Obliczyč obję-

tośč powstałego $\mathrm{w}$ ten sposób wielościanu ABCDEF. Obliczyč tangens kąta nachylenia

do podstawy tej ściany bocznej, która nie jest prostopadła do podstawy. Sporz$\Phi$dzič

odpowiednie rysunki. Podač warunki istnienia rozwiązania.
\end{document}
