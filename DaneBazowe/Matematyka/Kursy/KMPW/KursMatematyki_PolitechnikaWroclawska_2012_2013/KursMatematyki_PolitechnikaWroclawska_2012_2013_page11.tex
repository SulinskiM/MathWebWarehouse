\documentclass[a4paper,12pt]{article}
\usepackage{latexsym}
\usepackage{amsmath}
\usepackage{amssymb}
\usepackage{graphicx}
\usepackage{wrapfig}
\pagestyle{plain}
\usepackage{fancybox}
\usepackage{bm}

\begin{document}

PRACA KONTROLNA nr 6- POZIOM ROZSZERZONY

l. Rozwiązač równanie

$\sqrt{x^{2}-3}+2\sqrt{5-2x}=5-x.$

2. Wybrano losowo trzy krawędzie sześcianu. Obliczyč prawdopodobieństwo tego, $\dot{\mathrm{z}}\mathrm{e}\dot{\mathrm{z}}$ adne

dwie nie mają punktów wspólnych.

3. Gra $\mathrm{w}$ pary. $\mathrm{W}$ skarbonce znajduje się $\mathrm{d}\mathrm{u}\dot{\mathrm{z}}$ a liczba monet $0$ nominalach l $\mathrm{z}l$, 2 zł $\mathrm{i}$

5 $\mathrm{z}l. \mathrm{W}$ pierwszym kroku Jaś losuje trzy monety. Jesli wśród nich są dwie jednakowe,

to wrzuca je do skarbonki. $\mathrm{W}$ kolejnych krokach losuje ze skarbonki $\mathrm{k}\mathrm{a}\dot{\mathrm{z}}$ dorazowo tyle

monet, ile trzyma $\mathrm{w}$ rece, a nastepnie paryjednakowych monet wrzuca do skarbonki. Gra

kończy się, gdy wrzuci do skarbonki wszystkie monety. Obliczyč prawdopodobieństwo

tego, $\dot{\mathrm{z}}\mathrm{e}$ Jaś skończy grę: a) $\mathrm{w}$ drugim kroku; b) $\mathrm{w}$ drugim lub trzecim kroku.

4. Dane są wierzchołki $A(-3,2), C(4,2), D(0,4)$ trapezu równoramiennego ABCD, $\mathrm{w}$ któ-

rym $AB||CD$. Wyznaczyč współrzędne wierzcholka $B$ oraz równanie okręgu opisanego

na trapezie.

5. Udowodnič, $\dot{\mathrm{z}}\mathrm{e}$ dla $x>-1$ prawdziwa jest nierównośč podwójna

$1+\displaystyle \frac{x}{2}-\frac{x^{2}}{2}\leq\sqrt{1+x}\leq 1+\frac{x}{2}.$

Zilustrowač tę nierównośč odpowiednim rysunkiem.

6. $\mathrm{Z}$ dwóch przeciwlegfych wierzchołków prostokąta $0$ polu $P$, bdcego podstawą prosto-

padfościanu $0$ wysokości l, wystawiono po dwie przekątne sąsiednich ścian bocznych.

Wyrazič cosinus kąta pomiędzy plaszczyznami utworzonymi przez te pary przekątnych

jako funkcję sinusa kąta między nimi. Sporządzič rysunki.
\end{document}
