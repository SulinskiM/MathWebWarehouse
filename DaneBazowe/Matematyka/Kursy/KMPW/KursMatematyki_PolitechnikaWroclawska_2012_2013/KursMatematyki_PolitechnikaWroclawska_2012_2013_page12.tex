\documentclass[a4paper,12pt]{article}
\usepackage{latexsym}
\usepackage{amsmath}
\usepackage{amssymb}
\usepackage{graphicx}
\usepackage{wrapfig}
\pagestyle{plain}
\usepackage{fancybox}
\usepackage{bm}

\begin{document}

XLII

KORESPONDENCYJNY KURS

Z MATEMATYKI

marzec 2013 r.

PRACA KONTROLNA nr 7- POZIOM PODSTAWOWY

l. Wyznaczyč rozwiazanie ogólne równania

$\displaystyle \sin(2x+\frac{\pi}{3})=\cos(x-\frac{\pi}{6}),$

a następnie podač rozwiązania $\mathrm{w}$ przedziale $[-2\pi,2\pi].$

2. Wyrazenie

$(\displaystyle \frac{a-2b}{\sqrt[3]{a^{2}}-\sqrt[3]{4b^{2}}}+\frac{\sqrt[3]{2a^{2}b}+\sqrt[3]{4ab^{2}}}{\sqrt[3]{a^{2}}+\sqrt[3]{4b^{2}}+\sqrt[3]{16ab}})$ : $\displaystyle \frac{a\sqrt[3]{a}+b\sqrt[3]{2b}+b\sqrt[3]{a}+a\sqrt[3]{2b}}{a+b}$

sprowadzič do najprostszej postaci. Przy jakich zalozeniach ma ono sens?

3. Narysowač wykres funkcji $f(x) =2|x|-\sqrt{x^{2}+4x+4}$ oraz wyznaczyč najmniejszą $\mathrm{i}$

największą wartośč funkcji $|f(x)|\mathrm{w}$ przedziale [-1, 2]. D1a jakiego $m$ pole figury ograni-

czonej wykresem funkcji $|f(x)|\mathrm{i}\mathrm{p}\mathrm{r}\mathrm{o}\mathrm{s}\mathrm{t}_{\Phi}y=m$ równe jest 16?

4. Rozwiązač układ równań

$\left\{\begin{array}{l}
x^{2}-4y^{2}+8y=4\\
x^{2}+y^{2}-2y=4
\end{array}\right.$

Podač interpretację geometryczną tego ukladu $\mathrm{i}$ obliczyč pole czworokata, którego wierz-

chofkami są cztery punkty będące jego rozwiązaniem.

5. $\mathrm{W}$ trapezie równoramiennym ABCD, $\mathrm{w}$ którym $BC||AD$ dane są $\vec{AB} = [1,-2]$ oraz

$\vec{AD}=[1$, 1$]$. Obliczyč pole trapezu $\mathrm{i}$ wyznaczyč $\mathrm{k}_{\Phi^{\mathrm{t}}}$ między jego przekątnymi.

6. $\mathrm{W}$ ostrosłupie prawidłowym trójkątnym cosinus kata nachylenia ściany bocznej do pod-

stawy równy jest $\displaystyle \frac{1}{9}$. Obliczyč stosunek pola powierzchni cafkowitej do pola podstawy.

Wykorzystując wzór $\sin 2\alpha=2\sin\alpha\cos\alpha$, wyznaczyč sinus kąta między ścianami bocz-

nymi tego ostrosfupa. Sporządzič rysunki.
\end{document}
