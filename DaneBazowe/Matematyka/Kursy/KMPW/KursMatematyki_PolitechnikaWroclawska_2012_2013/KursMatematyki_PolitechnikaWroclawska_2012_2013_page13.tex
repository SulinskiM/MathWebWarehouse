\documentclass[a4paper,12pt]{article}
\usepackage{latexsym}
\usepackage{amsmath}
\usepackage{amssymb}
\usepackage{graphicx}
\usepackage{wrapfig}
\pagestyle{plain}
\usepackage{fancybox}
\usepackage{bm}

\begin{document}

PRACA KONTROLNA nr 7- POZIOM ROZSZERZONY

l. Rozwiązač równanie

$\displaystyle \sin x+\cos x=\frac{\cos 2x}{\sin 2x-1}$

2. Wyrazenie

$w(x,y)=\displaystyle \frac{x}{x^{3}+x^{2}y+xy^{2}+y^{3}}+\frac{y}{x^{3}-x^{2}y+xy^{2}-y^{3}}+\frac{1}{x^{2}-y^{2}}-\frac{1}{x^{2}+y^{2}}-\frac{x^{2}+2y^{2}}{x^{4}-y^{4}}$

doprowadzič do najprostszej postaci. Przy jakich zalozeniach ma ono sens? Obliczyč

$w(\cos 15^{\mathrm{o}},\sin 15^{\mathrm{o}}).$

3. Narysowač wykres funkcji

$f(x)=$

dla

dla

$x\leq 1,$

$x>1$

$\mathrm{i}$ posfugujac $\mathrm{s}\mathrm{i}\mathrm{e}$ nim wyznaczyč zbiór wartości funkcji $|f(x)|\mathrm{w}$ przedziale $[-\displaystyle \frac{1}{2},\frac{3}{2}].$

4. Rozwiązač ukfad równań

$\left\{\begin{array}{l}
y+x^{2}=4\\
4x^{2}-y^{2}+2y=1
\end{array}\right.$

Podač interpretację geometryczną tego ukladu $\mathrm{i}$ wykazač, $\dot{\mathrm{z}}\mathrm{e}$ cztery punkty, które są

jego rozwiązaniem, wyznaczaj $\Phi$ na płaszczy $\acute{\mathrm{z}}\mathrm{n}\mathrm{i}\mathrm{e}$ trapez równoramienny. Znalez/č równanie

okręgu opisanego na tym trapezie.

5. Odcinek $0$ końcach $A(0,7)\mathrm{i}B(5,2)$ jest przeciwprostokatna trójkąta prostokqtnego, któ-

rego wierzcholek $C\mathrm{l}\mathrm{e}\dot{\mathrm{z}}\mathrm{y}$ na prostej $x=3$. Posfugując się rachunkiem wektorowym ob-

liczyč cosinus kąta między dwusieczną kąta prostego a wysokością opuszczoną $\mathrm{z}$ wierz-

chofka $C.$

6. Pole powierzchni calkowitej ostrosfupa prawidfowego trójkątnego jest dziesięč razy więk-

sze $\mathrm{n}\mathrm{i}\dot{\mathrm{z}}$ pole jego podstawy. Wyznaczyč cosinus kąta między ścianami bocznymi oraz

stosunek objętości ostroslupa do objętości wpisanej $\mathrm{w}$ niego kuli.
\end{document}
