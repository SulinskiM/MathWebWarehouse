\documentclass[a4paper,12pt]{article}
\usepackage{latexsym}
\usepackage{amsmath}
\usepackage{amssymb}
\usepackage{graphicx}
\usepackage{wrapfig}
\pagestyle{plain}
\usepackage{fancybox}
\usepackage{bm}

\begin{document}

XLII

KORESPONDENCYJNY KURS

Z MATEMATYKI

kwiecień 2013 r.

PRACA KONTROLNA $\mathrm{n}\mathrm{r} 8-$ POZIOM PODSTAWOWY

l. Cztery kolejne współczynniki wielomianu $f(x)=x^{3}+ax^{2}+bx+c$ tworzą $\mathrm{c}\mathrm{i}\otimes \mathrm{g}$ geome-

tryczny. Wiadomo, $\dot{\mathrm{z}}\mathrm{e}-3$ jest pierwiastkiem tego wielomianu. Wyznaczyč wspófczynniki

$a, b, c.$

2. Kolo $x^{2}+y^{2}+4x-2y-1\leq 0$ zostalo przesunięte $0$ wektor $\vec{w}=[3$, 3$]$. Znalez$\acute{}$č równanie

osi symetrii figury, która jest sumą kola $\mathrm{i}$ jego obrazu oraz obliczyč jej pole.

3. Podstawą ostroslupajest trójkąt $0$ bokach $\alpha, b, c$. Wszystkie kąty płaskie przy wierzchołku

ostroslupa są proste. Obliczyč objętośč ostroslupa.

4. Dane są punkty $A(0,2), B(4,4), C(3,6)$. Na prostej przechodzącej przez punkt $C$ rów-

noległej do prostej $AB$ znalez$\acute{}$č punkt $D$, który jest równo odlegly od punktów A $\mathrm{i}B.$

Wykazač, $\dot{\mathrm{z}}\mathrm{e}$ trójkąt $ABD$ jest prostokątny $\mathrm{i}$ napisač równanie okręgu opisanego na nim.

5. Wyznaczyč wartośč parametru $m$, dla którego równanie

$4x^{2}-2x\log_{2}m+1=0$

ma dwa rózne pierwiastki rzeczywiste $x_{1}, x_{2}$ spełniające warunek $x_{1}^{2}+x_{2}^{2}=1.$

6. Dane $\mathrm{s}\Phi$ funkcje $f(x)=4^{x-2}-7\cdot 3^{x-3}, g(x)=3^{3x+2}-5\cdot 4^{3x}$

Rozwiązač nierównośč

$f(x+3)>g(\displaystyle \frac{x}{3})$
\end{document}
