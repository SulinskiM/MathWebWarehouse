\documentclass[a4paper,12pt]{article}
\usepackage{latexsym}
\usepackage{amsmath}
\usepackage{amssymb}
\usepackage{graphicx}
\usepackage{wrapfig}
\pagestyle{plain}
\usepackage{fancybox}
\usepackage{bm}

\begin{document}

PRACA KONTROLNA nr 8- POZIOM ROZSZERZONY

l. Niech $A$ bedzie wierzchołkiem kwadratu, a $M$ środkiem przeciwległego boku. Na prze-

$\mathrm{k}_{\Phi}$tnej kwadratu wychodzącej $\mathrm{z}$ wierzchofka $A$ wybrano punkt $P\mathrm{t}\mathrm{a}\mathrm{k}$, aby $|AP|=|MP|.$

Obliczyč, $\mathrm{w}$ jakim stosunku punkt $P$ dzieli przekatną kwadratu.

2. Stosując zasade indukcji matematycznej udowodnič nierównośč

$\left(\begin{array}{l}
2n\\
n
\end{array}\right) \displaystyle \leq\frac{4^{n}}{\sqrt{2n+2}},$

$n\geq 1.$

3. Wyznaczyč równanie okręgu $0$ środku lezącym na prostej $y-x=0$ oraz stycznego do

prostej $y-3=0\mathrm{i}$ do okręgu $x^{2}+y^{2}-4x+3=0$. Sporządzič rysunek.

4. Liczba $-2$ jest pierwiastkiem dwukrotnym wielomianu $w(x) = \displaystyle \frac{1}{2}x^{3}+ax^{2}+bx+c,$

a punkt $\mathrm{S}(-1,y_{0})$ jest środkiem symetrii wykresu $w(x)$. Wyznaczyč $a, b, c, y_{0}$ oraz trzeci

pierwiastek. Sporzqdzič wykres $w(x)\mathrm{w}$ przedziale $[-3,\displaystyle \frac{3}{2}].$

5. Wycinek kofa $0$ promieniu $3R\mathrm{i}\mathrm{k}_{\Phi}\mathrm{c}\mathrm{i}\mathrm{e}$ środkowym $\alpha$ zwinięto $\mathrm{w}$ powierzchnię boczną

stozka $S_{1}$. Podobnie, wycinek koła $0$ promieniu $R\mathrm{i}$ kqcie środkowym $ 3\alpha$ zwinieto $\mathrm{w}$ po-

wierzchnię boczną stozka $S_{2}$. Następnie obydwa stozki $\mathrm{z}l_{\Phi}$czono podstawami $\mathrm{t}\mathrm{a}\mathrm{k}$, aby

miafy wspólną oś obrotu, a ich wierzchofki byly skierowane $\mathrm{w}$ przeciwnych kierunkach.

Obliczyč promień kuli wpisanej $\mathrm{w}$ otrzymaną brylę. Sporzqdzič rysunek.

6. Podač interpretację geometryczną równania $\sqrt{2x+4}=mx+m+1\mathrm{z}$ parametrem $m.$

Graficznie $\mathrm{i}$ analitycznie określič, dla jakich wartości $m$ równanie ma dwa pierwiastki

$x_{1}=x_{1}(m), x_{2}=x_{2}(m)$. Nie korzystając $\mathrm{z}$ metod rachunku rózniczkowego, wykazač, $\dot{\mathrm{z}}\mathrm{e}$

funkcja $f(m)=x_{1}(m)+x_{2}(m)$ jest malejąca oraz sporządzič jej wykres.
\end{document}
