\documentclass[a4paper,12pt]{article}
\usepackage{latexsym}
\usepackage{amsmath}
\usepackage{amssymb}
\usepackage{graphicx}
\usepackage{wrapfig}
\pagestyle{plain}
\usepackage{fancybox}
\usepackage{bm}

\begin{document}

XLII

KORESPONDENCYJNY KURS

Z MATEMATYKI

$\mathrm{p}\mathrm{a}\acute{\mathrm{z}}$dziernik 2012 $\mathrm{r}.$

PRACA KONTROLNA $\mathrm{n}\mathrm{r} 2-$ POZIOM PODSTAWOWY

l. Firma budowlana podpisała umowe na modernizację odcinka autostrady $0$ długości 21

km $\mathrm{w}$ określonym terminie. Ze względu na zblizające się mistrzostwa świata $\mathrm{w}$ rzu-

cie telefonem komórkowym postanowiono zrealizowač zamówienie 10 dni wcześniej, co

oznaczafo koniecznośč zwiększenia średniej normy dziennej $0$ 5\%. $\mathrm{W}$ jakim czasie firma

zamierzafa pierwotnie zrealizowač to zamówienie?

2. Pan Kowalski zaciągnął $\mathrm{w}$ banku kredyt $\mathrm{w}$ wysokości 4000 zł oprocentowany na 16\% $\mathrm{w}$

skali roku. Zgodnie $\mathrm{z}$ umową będzie go spłacaf $\mathrm{w}$ czterech ratach co 3 miesiące, spfacając

za $\mathrm{k}\mathrm{a}\dot{\mathrm{z}}$ dym razem 1000zł oraz 4\% pozosta1ego zadłuzenia. I1e złotych ostatecznie zwróci

bankowi pan Kowalski?

3. Ile jest czterocyfrowych liczb naturalnych:

a) podzielnych przez 2, 31ub przez 5?

b) podzielnych przez dokfadnie dwie spośród powyzszych liczb?

4. Na paraboli $y=x^{2}-6x+11$ znalez/č taki punkt $C, \dot{\mathrm{z}}\mathrm{e}$ pole trójkąta $0$ wierzchołkach

$A=(0,3), B=(4,0), C$ jest najmniejsze.

5. Przy prostoliniowej ulicy (oś Ox) $\mathrm{w}$ punkcie $x=0$ zainstalowano parkomat. $\mathrm{W}$ punkcie

$x=1 \mathrm{m}\mathrm{o}\dot{\mathrm{z}}$ na korzystač $\mathrm{z}$ bankomatu, a $\mathrm{w}$ punkcie $x=-2$ jest wejście do galerii han-

dlowej. $\mathrm{W}$ którym punkcie $x$ ulicy nalezy zaparkowač samochód, aby droga przebyta

od samochodu do parkomatu $\mathrm{i}\mathrm{z}$ powrotem (bilet parkingowy nalez $\mathrm{y}$ pofozyč za szybą

pojazdu), następnie do bankomatu po pieniądze, stąd do galerii $\mathrm{i}$ na końcu $\mathrm{z}$ zakupami

do samochodu, byfa najkrótsza? Jaka będzie odpowied $\acute{\mathrm{z}}$, gdy wejście do galerii będzie $\mathrm{w}$

punkcie $x=2$? $\mathrm{W}$ obu przypadkach podač wzór $\mathrm{i}$ narysowač wykres funkcji określającej

droge przebytq przez klienta domu handlowego $\mathrm{w}$ zalezności od punktu zaparkowania

samochodu.

6. Wykonač działania $\mathrm{i}$ zapisač $\mathrm{w}$ najprostszej postaci wyrazenie

$w(a,b)= (\displaystyle \frac{a}{a^{2}-ab+b^{2}}-\frac{a^{2}}{a^{3}+b^{3}})$ : $(\displaystyle \frac{a^{3}-b^{3}}{\alpha^{3}+b^{3}}-\frac{\alpha^{2}+b^{2}}{a^{2}-b^{2}})$

Wykazač, $\dot{\mathrm{z}}\mathrm{e}$ dla dowolnych $a<0$ zachodzi nierównośč $w(-a,a^{-1})\geq 1$, a dla dowolnych

$a>0$ prawdziwa jest nierównośč $w(-a,a^{-1})\leq 1.$
\end{document}
