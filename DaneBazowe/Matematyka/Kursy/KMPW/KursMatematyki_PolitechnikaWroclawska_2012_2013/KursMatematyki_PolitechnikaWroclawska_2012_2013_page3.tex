\documentclass[a4paper,12pt]{article}
\usepackage{latexsym}
\usepackage{amsmath}
\usepackage{amssymb}
\usepackage{graphicx}
\usepackage{wrapfig}
\pagestyle{plain}
\usepackage{fancybox}
\usepackage{bm}

\begin{document}

PRACA KONTROLNA nr 2- POZIOM ROZSZERZONY

l. Rozwiązač nierównośč $\displaystyle \frac{1}{\sqrt{5+4x-x^{2}}}\geq\frac{1}{|x|-2}\mathrm{i}$ zbiór rozwiązań zaznaczyč na osi liczbo-

wej.

2. Dwaj rowerzyści wyjechali jednocześnie naprzeciw siebie $\mathrm{z}$ miast A $\mathrm{i}\mathrm{B}$ odlegfych $030$

kilometrów. Minęli się po godzinie $\mathrm{i}$ nie zatrzymując się podqzyli $\mathrm{z}$ tymi samymi pręd-

kościami $\mathrm{k}\mathrm{a}\dot{\mathrm{z}}\mathrm{d}\mathrm{y}\mathrm{w}$ swoim kierunku. Rowerzysta, który wyjechal $\mathrm{z}$ A dotarf do $\mathrm{B}$ póftorej

godziny wcześniej $\mathrm{n}\mathrm{i}\dot{\mathrm{z}}$ jego kolegajadący $\mathrm{z}\mathrm{B}$ dotarł do A. $\mathrm{Z}$ jakimi prędkościami jechali

rowerzyści?

3. Pan Kowalski zaciągnąf 3l grudnia $\mathrm{p}\mathrm{o}\dot{\mathrm{z}}$ yczkę 4000 z1otych $\mathrm{o}\mathrm{P}^{\mathrm{r}\mathrm{o}\mathrm{c}\mathrm{e}\mathrm{n}\mathrm{t}\mathrm{o}\mathrm{w}\mathrm{a}\mathrm{n}}\Phi^{\mathrm{W}}$ wysokości

16\% $\mathrm{w}$ skali roku. Zobowiqzał się splacič ją $\mathrm{w}$ ciągu roku $\mathrm{w}$ czterech równych ratach

platnych 30. marca, 30. czerwca, 30. września $\mathrm{i}30$. grudnia. Oprocentowanie $\mathrm{p}\mathrm{o}\dot{\mathrm{z}}$ yczki

liczy się od l stycznia, a odsetki od kredytu naliczane są $\mathrm{w}$ terminach pfatności rat.

Obliczyč wysokośč tych rat $\mathrm{w}$ zaokrągleniu do pełnych groszy.

4. Dla jakiego parametru $m$ równanie

$2x^{2}-(2m+1)x+m^{2}-9m+39=0$

ma dwa pierwiastki, $\mathrm{z}$ których jeden jest dwa razy większy $\mathrm{n}\mathrm{i}\dot{\mathrm{z}}$ drugi?

5. Ile jest liczb pięciocyfrowych podzielnych przez 9, które $\mathrm{w}$ rozwinięciu dziesiętnym mają:

a) obie cyfry 1, 2 $\mathrm{i}$ tylko $\mathrm{t}\mathrm{e}$? b) obie cyfry 1, 3 $\mathrm{i}$ tylko $\mathrm{t}\mathrm{e}$? c) wszystkie cyfry 1, 2, 3

$\mathrm{i}$ tylko $\mathrm{t}\mathrm{e}$? Odpowiedz/uzasadnič. $\mathrm{W}$ przypadku b) wypisač otrzymane liczby.

6. $\mathrm{Z}$ przystani A wyrusza $\mathrm{z}$ biegiem rzeki statek do przystani $\mathrm{B}$, odleglej od A $0140$ km. Po

uplywie l godziny wyrusza za nim fódz/ motorowa, dopędza statek, po czym wraca do

przystani A $\mathrm{w}$ tym samym momencie, $\mathrm{w}$ którym statek przybija do przystani B. Predkośč

łodzi $\mathrm{w}$ wodzie stojącej jest póltora raza większa $\mathrm{n}\mathrm{i}\dot{\mathrm{z}}$ prędkośč statku $\mathrm{w}$ wodzie stojącej.

Wyznaczyč te prędkości $\mathrm{w}\mathrm{i}\mathrm{e}\mathrm{d}\mathrm{z}\Phi^{\mathrm{C}}, \dot{\mathrm{z}}\mathrm{e}$ rzeka plynie $\mathrm{z}$ prędkością 4 $\mathrm{k}\mathrm{m}/$godz.
\end{document}
