\documentclass[a4paper,12pt]{article}
\usepackage{latexsym}
\usepackage{amsmath}
\usepackage{amssymb}
\usepackage{graphicx}
\usepackage{wrapfig}
\pagestyle{plain}
\usepackage{fancybox}
\usepackage{bm}

\begin{document}

XLII

KORESPONDENCYJNY KURS

Z MATEMATYKI

listopad 2012 r.

PRACA KONTROLNA $\mathrm{n}\mathrm{r} 3-$ POZIOM PODSTAWOWY

1. $\mathrm{Z}$ danych Głównego Urzędu Statystycznego wynika, $\dot{\mathrm{z}}\mathrm{e}$ wzrost Produktu Krajowego Brut-

to (PKB) $\mathrm{w}$ Polsce $\mathrm{w}$ roku 2010 wyniósf 3,7\%, a $\mathrm{w}$ roku 2011- 4,3\%. Jaki powinien byč

wzrost PKB $\mathrm{w}$ roku 2012, by średni roczny wzrost PKB $\mathrm{w}$ tych trzech latach wyniósł

4\%? Podač wynik $\mathrm{z}$ dokładnością do 0, 001\%.

2. Czy liczby $\sqrt{2}$, 2, $2\sqrt{2}$ mogą byč wyrazami (niekoniecznie kolejnymi) ciągu arytme-

tycznego? Odpowiedz/uzasadnič.

3. Wielomian $W(x) = x^{5}+ax^{4}+bx^{3}+4x$ jest podzielny przez $(x^{2}-1)$. Wyznaczyč

wspólczynniki $a, b\mathrm{i}$ rozwiązač nierównośč $W(x-1)\leq W(x)\leq W(x+1).$

4. Niech $f(x)=\sqrt{x}, g(x)=x-2, h(x)=|x|$. Narysowač wykresy funkcji zlozonych: $f0h\circ g,$

$f$ o{\it g}o $h$, {\it go} $f\mathrm{o}h$, {\it goho} $f, h\mathrm{o}f\mathrm{o}g$ {\it oraz hogo} $f.$

5. Przyprostokątną trójkata prostokątnego $ABC$ jest odcinek AB $0$ końcach $A(-2,2) \mathrm{i}$

$B(1,-1)$, a wierzchofek $C$ trójk$\Phi$ta $\mathrm{l}\mathrm{e}\dot{\mathrm{z}}\mathrm{y}$ na prostej $3x-y= 14$. Wyznaczyč równanie

okręgu opisanego na tym trójkącie. Ile rozwiązań ma to zadanie? Sporządzič rysunek.

6. Na prostej $x+2y=5$ wyznaczyč punkty, $\mathrm{z}$ których okrag $(x-1)^{2}+(y-1)^{2}=1$ jest

widoczny pod kątem $60^{\mathrm{o}}$. Obliczyč pole obszaru ograniczonego fukiem okręgu $\mathrm{i}$ stycznymi

do niego poprowadzonymi $\mathrm{w}$ znalezionych punktach. Sporządzič rysunek.
\end{document}
