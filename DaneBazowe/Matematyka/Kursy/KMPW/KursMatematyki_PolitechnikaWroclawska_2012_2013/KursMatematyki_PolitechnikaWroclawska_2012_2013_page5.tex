\documentclass[a4paper,12pt]{article}
\usepackage{latexsym}
\usepackage{amsmath}
\usepackage{amssymb}
\usepackage{graphicx}
\usepackage{wrapfig}
\pagestyle{plain}
\usepackage{fancybox}
\usepackage{bm}

\begin{document}

PRACA KONTROLNA nr 3- POZIOM ROZSZERZONY

l. Pan Kowalski umieścił swoje oszczędności na dwu róznych lokatach. Pieniądze, otrzy-

mane jako honorarium za podręcznik, zfozyl na lokacie oprocentowanej $\mathrm{w}$ wysokości 7\%

$\mathrm{w}$ skali roku, a wynagrodzenie za cykl wykładów- na lokacie 9\%. Po roku jego dochód

był $030$ zlotych, a po dwu latach- $070$ złotych $\mathrm{w}\mathrm{y}\dot{\mathrm{z}}$ szy od dochodu, który uzyskałby

skfadając cafą sumę na lokacie 8\%. I1e pieniędzy otrzyma1 pan Kowa1ski za podręcznik,

a ile za wykłady?

2. Czy liczby $\sqrt{2}, \sqrt{3}$, 2 mogą byč wyrazami (niekoniecznie kolejnymi) $\mathrm{c}\mathrm{i}_{\Phi \mathrm{g}}\mathrm{u}$ arytmetycz-

nego? Odpowiedz/uzasadnič.

3. Niech $f(x)=2^{x}, g(x)=2-x, h(x)=|x|$. Narysowač wykresy funkcji złozonych $f\mathrm{o}g\mathrm{o}h$

oraz $g\mathrm{o}f\mathrm{o}h\mathrm{i}$ rozwiązač nierównośč $(f\mathrm{o}g\mathrm{o}h)(x)<6+(g\mathrm{o}f\mathrm{o}h)(x).$

4. Dane są punkty $A(1,2), B(3,1).$

takich, $\dot{\mathrm{z}}\mathrm{e}\mathrm{k}\mathrm{a}\mathrm{t}BCA$ ma miarę $45^{\mathrm{o}}$

Wyznaczyč równanie zbioru wszystkich punktów C

5. Liczby: $a_{1}=\log_{(3-2\sqrt{2})^{2}}(\sqrt{2}-1), a_{2}=\displaystyle \frac{1}{2}\log_{\frac{1}{3}}\frac{\sqrt{3}}{6}, \alpha_{3}=3^{\log_{\sqrt{3}}\frac{\sqrt{6}}{2}}, a_{4}=\log_{(\sqrt{2}-1)}(\sqrt{2}+1),$

$a_{5}=(2^{\sqrt{2}+1})^{\sqrt{2}-1}, a_{6}=\log_{3}2$ są jedynymi pierwiastkami wielomianu $W(x)$, którego

wyraz wolny jest dodatni.

a) Które $\mathrm{z}$ tych pierwiastków są niewymierne? Odpowied $\acute{\mathrm{z}}$ uzasadnič.

b) Wyznaczyč dziedzinę funkcji $f(x)=\sqrt{W(x)}$, nie wykonując obliczeń przyblizonych.

6. Niech $f(x)=3(x+2)^{4}+x^{2}+4x+p$, gdzie $p$ jest parametrem rzeczywistym.

a) Uzasadnič, $\dot{\mathrm{z}}\mathrm{e}$ wykres funkcji $f(x)$ jest symetryczny względem prostej $x=-2.$

b) Dla jakiego parametru $p$ najmniejszą wartością funkcji $f(x)$ jest $y=-2$ ?

Odpowiedz/uzasadnič, nie stosując metod rachunku rózniczkowego.

c) Określič liczbę rozwiqzań równania $f(x)=0\mathrm{w}$ zalezności od parametru $p.$
\end{document}
