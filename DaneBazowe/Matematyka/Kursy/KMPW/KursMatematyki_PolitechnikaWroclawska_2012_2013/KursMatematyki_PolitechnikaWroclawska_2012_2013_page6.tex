\documentclass[a4paper,12pt]{article}
\usepackage{latexsym}
\usepackage{amsmath}
\usepackage{amssymb}
\usepackage{graphicx}
\usepackage{wrapfig}
\pagestyle{plain}
\usepackage{fancybox}
\usepackage{bm}

\begin{document}

XLII

KORESPONDENCYJNY KURS

Z MATEMATYKI

grudzień 2012 r.

PRACA KONTROLNA $\mathrm{n}\mathrm{r} 4-$ POZIOM PODSTAWOWY

l. Wyznaczyč wszystkie kąty $\alpha \mathrm{z}$ przedziału $[0,2\pi]$, dla których suma kwadratów pierwiast-

ków rzeczywistych równania $x^{2}+2x\sin\alpha-\cos^{2}\alpha=0$ jest równa co najwyzej 3.

2. Uzasadnič, $\dot{\mathrm{z}}\mathrm{e}$ suma średnic okręgu opisanego na trójkącie prostokqtnym $\mathrm{i}$ okręgu wpisa-

nego $\mathrm{w}$ ten trójkąt jest równa sumie długości przyprostokątnych. Znalez$\acute{}$č dlugości boków

trójkąta, $\mathrm{j}\mathrm{e}\dot{\mathrm{z}}$ eli promienie tych okręgów są równe $R=5\mathrm{i}r=2.$

3. Narysowač wykres funkcji $f(x)=\cos^{2}x+|\sin x|\sin x \mathrm{w}$ przedziale $[-2\pi,2\pi].$

a) Podač zbiór wartości $\mathrm{i}$ miejsca zerowe.

b) Wyznaczyč przedziafy monotoniczności.

c) Rozwiązač nierównośč $|f(x)|\displaystyle \geq\frac{1}{2}.$

4. $\mathrm{W}$ kwadracie $0$ boku dfugości $a$ narysowano cztery pólkola, których średnicami są boki

kwadratu. Pólkola przecinają $\mathrm{s}\mathrm{i}\mathrm{e}$ parami tworząc czterolistną rozetę. Obliczyč pole $\mathrm{i}$

obwód rozety.

5. Dach wiez $\mathrm{y}$ kościola ma kształt ostrosłupa, którego podstawą jest sześciokąt foremny $0$

boku 4 $\mathrm{m}$ a największy $\mathrm{z}$ przekrojów płaszczyzną zawierajqcq wysokośč jest trójkątem

równobocznym. Obliczyč kubaturę dachu wiez $\mathrm{y}$ kościofa. Ile 2-1itrowych puszek farby

antykorozyjnej trzeba kupič do pomalowania blachy, którą pokryty jest dach, $\mathrm{j}\mathrm{e}\dot{\mathrm{z}}$ eli wia-

domo, $\dot{\mathrm{z}}\mathrm{e}$ llitr farby wystarcza do pomalowania 6 $\mathrm{m}^{2}$ blachy $\mathrm{i}$ trzeba uwzględnič 8\%

farby $\mathrm{n}\mathrm{a}$ ewentualne straty.

6. Promień kuli opisanej na ostrosłupie prawidlowym czworokątnym wynosi $R$. Prosto-

padfa wyprowadzona ze środka kuli do ściany bocznej ostroslupa tworzy $\mathrm{z}$ wysokością

ostrosłupa kąt $\alpha$. Wyznaczyč wysokośč ostroslupa.
\end{document}
