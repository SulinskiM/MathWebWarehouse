\documentclass[a4paper,12pt]{article}
\usepackage{latexsym}
\usepackage{amsmath}
\usepackage{amssymb}
\usepackage{graphicx}
\usepackage{wrapfig}
\pagestyle{plain}
\usepackage{fancybox}
\usepackage{bm}

\begin{document}

PRACA KONTROLNA nr 4- POZIOM ROZSZERZONY

l. Dla jakich katów $\alpha \mathrm{z}$ przedziału $[0,\displaystyle \frac{\pi}{2}]$ równanie $x^{2}\sin\alpha+x+\cos\alpha=0$ ma dwa róz-

ne pierwiastki rzeczywiste? Czy iloczyn pierwiastków równania $\mathrm{m}\mathrm{o}\dot{\mathrm{z}}\mathrm{e}$ byč równy $\sqrt{3}$?

Wyznaczyč wszystkie kąty $\alpha$, dla których suma pierwiastków jest większa od $-2.$

2. Przekrój ostroslupa prawidłowego czworokątnego plaszczyzną przechodzącą przez prze-

kątnq podstawy $\mathrm{i}$ wierzcholek ostrosłupajest trójkątem równobocznym. Wyznaczyč sto-

sunek promienia kuli wpisanej $\mathrm{w}$ ostrosfup do promienia kuli opisanej na ostrosfupie.

3. Narysowač wykres funkcji $f(x)=\displaystyle \frac{\sin 2x-|\sin x|}{\sin x}.$

równośč $f(x)<2(\sqrt{2}-1)\cos^{2}x.$

$\mathrm{W}$ przedziale $[0,2\pi]$ rozwiązač nie-

4. $\mathrm{C}\mathrm{z}\mathrm{w}\mathrm{o}\mathrm{r}\mathrm{o}\mathrm{k}_{\Phi^{\mathrm{t}}}$ wypukfy ABCD, $\mathrm{w}$ którym $AB=1, BC=2, CD=4, DA=3$ jest wpisany

$\mathrm{w}$ okrąg. Obliczyč promień $R$ tego okręgu. Sprawdzič, czy $\mathrm{w}$ ten czworokąt $\mathrm{m}\mathrm{o}\dot{\mathrm{z}}$ na wpisač

okrqg. $\mathrm{J}\mathrm{e}\dot{\mathrm{z}}$ eli $\mathrm{t}\mathrm{a}\mathrm{k}$, to obliczyč jego promień.

5. $\mathrm{W}$ kole $K\mathrm{o}$ promieniu 4 cm narysowano 6 kó1 $0$ promieniu 2 cm $\mathrm{P}^{\mathrm{r}\mathrm{z}\mathrm{e}\mathrm{c}\mathrm{h}\mathrm{o}\mathrm{d}\mathrm{z}}\Phi^{\mathrm{c}\mathrm{y}\mathrm{c}\mathrm{h}}$ przez

środek kola $K\mathrm{i}$ stycznych do niego $\mathrm{t}\mathrm{a}\mathrm{k}$, aby środki tych sześciu kół były wierzchołkami

sześciok$\Phi$ta foremnego. Obliczyč pole $\mathrm{i}$ obwód figury, która jest $\mathrm{s}\mathrm{u}\mathrm{m}\Phi$ tych sześciu kóf.

6. Stosunek pola powierzchni bocznej stozka ściętego do pola powierzchni wpisanej $\mathrm{w}$ ten

stozek kuli wyrazič jako funkcję kata nachylenia tworzącej stozka do podstawy.
\end{document}
