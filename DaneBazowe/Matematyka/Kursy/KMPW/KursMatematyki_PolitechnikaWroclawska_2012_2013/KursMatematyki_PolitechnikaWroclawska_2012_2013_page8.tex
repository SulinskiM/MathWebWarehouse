\documentclass[a4paper,12pt]{article}
\usepackage{latexsym}
\usepackage{amsmath}
\usepackage{amssymb}
\usepackage{graphicx}
\usepackage{wrapfig}
\pagestyle{plain}
\usepackage{fancybox}
\usepackage{bm}

\begin{document}

XLII

KORESPONDENCYJNY KURS

Z MATEMATYKI

styczeń 2013 r.

PRACA KONTROLNA $\mathrm{n}\mathrm{r} 5-$ POZIOM PODSTAWOWY

l. Między $\mathrm{k}\mathrm{a}\dot{\mathrm{z}}$ de dwa kolejne wyrazy pięcioelementowego ciągu arytmetycznego wstawiono

$m$ liczb, otrzymując ciąg arytmetyczny, którego sumajest 13 razy większa $\mathrm{n}\mathrm{i}\dot{\mathrm{z}}$ suma wyj-

ściowego ciągu. Obliczyč $m$. Jaką jednakową ilośč liczb nalez $\mathrm{y}$ wstawič miedzy $\mathrm{k}\mathrm{a}\dot{\mathrm{z}}$ de dwa

kolejne wyrazy $n$ elementowego ciągu arytmetycznego, aby otrzymač ciąg arytmetyczny

$0$ sumie $n$ razy większej $\mathrm{n}\mathrm{i}\dot{\mathrm{z}}$ suma wyjściowego ciągu?

2. Linie kolejowe malują wagony klasy standard na niebiesko, klasy komfort na rózowo,

a klasy biznes na szaro. Na ile sposobów $\mathrm{m}\mathrm{o}\dot{\mathrm{z}}$ na zestawič skfad pięciowagonowy, który

zawiera co najmniej jeden wagon $\mathrm{k}\mathrm{a}\dot{\mathrm{z}}$ dej klasy, a kolejnośč wagonów jest istotna?

3. Niech $n$ będzie liczbą naturalną. $\mathrm{W}$ przedziale $[0,2\pi]$ rozwiqzač równanie

$1+\cos^{2}x+\cos^{4}x+\cdots+\cos^{2n}x=2-\cos^{2n}x.$

4. Zawodnik przebiegf równym tempem pierwsze l0 km biegu maratońskiego $(42\mathrm{k}\mathrm{m})\mathrm{w}$ cza-

sie 45 minut, a $\mathrm{k}\mathrm{a}\dot{\mathrm{z}}\mathrm{d}\mathrm{y}$ kolejny kilometr pokonywał $\mathrm{w}$ czasie $0$ 5\% dluzszym $\mathrm{n}\mathrm{i}\dot{\mathrm{z}}$ poprzedni.

Sprawdzič, czy zawodnik zmieścil się $\mathrm{w}$ sześciogodzinnym limicie czasowym.

5. Rozwiązač nierównośč

$\log_{2}(x+2)-\log_{4}(4-x^{2})\geq 0.$

6. Niech $A=\{(x,y):|x|+2|y|\leq 2\}$. Zbiór $B$ powstaje przez obrót figury $A0$ kąt $\displaystyle \frac{\pi}{2} (\mathrm{w}$

kierunku przeciwnym do ruchu wskazówek zegara) wokół początku układu współrzęd-

nych. Starannie narysowač zbiory $A\cup B$ oraz $A\triangle B=(A\backslash B)\cup(B\backslash A)\mathrm{i}$ obliczyč ich

pola.
\end{document}
