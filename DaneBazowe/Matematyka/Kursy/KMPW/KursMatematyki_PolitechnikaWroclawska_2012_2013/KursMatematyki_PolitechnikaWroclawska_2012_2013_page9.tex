\documentclass[a4paper,12pt]{article}
\usepackage{latexsym}
\usepackage{amsmath}
\usepackage{amssymb}
\usepackage{graphicx}
\usepackage{wrapfig}
\pagestyle{plain}
\usepackage{fancybox}
\usepackage{bm}

\begin{document}

PRACA KONTROLNA nr 5- POZIOM ROZSZERZONY

l. Zbadač, dla jakich argumentów funkcja

przyjmuje wartości ujemne.

$g(x)=2^{x^{3}-5} 3^{7x^{2}}\cdot 4^{7x-1}-2^{7x^{2}+1}$

$3^{x^{3}-2} 9^{7x-3}$

2. Rozwiązač nierównośč

$2^{-\sin x}+2^{-2\sin x}+2^{-3\sin x}+\ldots\leq\sqrt{2}+1,$

której lewa strona jest sumą nieskończonego ciągu geometrycznego.

3. Podač dziedzinę i wyznaczyč wszystkie miejsca zerowe funkcji

$f(x)=\displaystyle \log_{x+1}(x-1)-\log_{x+1}(2x-\frac{2}{x})+1.$

4. Dany jest ciąg liczbowy $(a_{n}), \mathrm{w}$ którym $\mathrm{k}\mathrm{a}\dot{\mathrm{z}}\mathrm{d}\mathrm{y}$ wyraz jest sumą podwojonego wyrazu

poprzedniego $\mathrm{i}4$, a jego czwarty wyraz wynosi 36. Podač wzór na n-ty wyraz $\mathrm{c}\mathrm{i}_{\Phi \mathrm{g}}\mathrm{u}\mathrm{i}$

udowodnič go, wykorzystując zasadę indukcji matematycznej.

5. Niech $A=\{(x,y):|x|+2|y|\leq 2\}$. Zbiór $B$ otrzymano przez obrót $A0$ kąt $\displaystyle \frac{\pi}{2}(\mathrm{w}$ kierunku

przeciwnym do ruchu wskazówek zegara) wokóf $\mathrm{P}^{\mathrm{o}\mathrm{c}\mathrm{z}}\Phi^{\mathrm{t}\mathrm{k}\mathrm{u}}$ ukladu wspólrzędnych, a zbiór

C- przez obrót zbioru $A\cup B0$ kąt $\displaystyle \frac{\pi}{4}$ wokól początku układu współrzędnych. Wykonač

staranny rysunek zbioru $A\cup B\cup C$ oraz obliczyč jego pole.

6. Boki $\triangle ABC$ zawarte są $\mathrm{w}$ prostych $y=2x+m, y=mx+1$ oraz $2y=2-x$. Podač wartośč

rzeczywistego parametru $m\displaystyle \in(-\frac{1}{2},2)$, dla której pole rozwazanego trójkąta wynosi $\displaystyle \frac{1}{5}.$

Dla wyznaczonego $m$ wykonač staranny rysunek (przyjąč jednostkę równą 3 cm).
\end{document}
