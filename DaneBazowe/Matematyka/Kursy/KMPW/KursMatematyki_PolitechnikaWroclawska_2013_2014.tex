\documentclass[a4paper,12pt]{article}
\usepackage{latexsym}
\usepackage{amsmath}
\usepackage{amssymb}
\usepackage{graphicx}
\usepackage{wrapfig}
\pagestyle{plain}
\usepackage{fancybox}
\usepackage{bm}

\begin{document}

XLIII

KORESPONDENCYJNY KURS

Z MATEMATYKI

wrzesień 2013 r.

PRACA KONTROLNA $\mathrm{n}\mathrm{r} 1 -$ POZIOM PODSTAWOWY

l. Wzrost kursu Euro $\mathrm{w}$ stosunku do złotego spowodował podwyzkę ceny nowego modelu

Volvo $0$ 5\%. Poniewaz nie było popytu na te samochody, więc postanowiono ustalič cenę

promocyjną na poziomie odpowiadającym wzrostowi kursu Euro $0$ 2\%.

a) $\mathrm{O}$ ile procent cena promocyjna byfa $\mathrm{n}\mathrm{i}\dot{\mathrm{z}}$ sza od ceny wynikającej $\mathrm{z}$ faktycznego wzro-

stu kursu Euro $\mathrm{w}$ stosunku do zfotego? Wynik podač $\mathrm{z}$ dokladności$\Phi$ do l promila.

b) Ile pan Kowalski stracif na wzroście kursu Euro, a ile zyskal dzięki cenie promo-

cyjnej, $\mathrm{j}\mathrm{e}\dot{\mathrm{z}}$ eli kupił samochód za 56000? Rachunki prowadzič $\mathrm{z}$ dokładnością do

cafkowitych złotych.

2. $\mathrm{Z}$ obozu A do obozu $\mathrm{B}\mathrm{m}\mathrm{o}\dot{\mathrm{z}}$ na przejśč drogą $\dot{\mathrm{z}}$ wirową lub ściezką przez las, która jest $0$

sześč kilometrów krótsza $\mathrm{n}\mathrm{i}\dot{\mathrm{z}}$ droga $\dot{\mathrm{z}}$ wirowa. Bolek wyszedł $\mathrm{z}$ A $\mathrm{i}$ idąc ściezką $\mathrm{z}$ prędkościq

4 $\mathrm{k}\mathrm{m}/\mathrm{h}$ dotarf do $\mathrm{B} 1$ godzinę wcześniej $\mathrm{n}\mathrm{i}\dot{\mathrm{z}}$ Lolek, który $\mathrm{w}$ tym samym momencie

wyruszył drogą $\dot{\mathrm{z}}$ wirową. Znalez/č dlugośč ściezki, wiedząc, $\dot{\mathrm{z}}\mathrm{e}$ prędkośč, $\mathrm{z}$ jaką porusza

się Lolek wyraza się liczbą całkowitą.

3. Ile jest naturalnych liczb pięciocyfrowych, $\mathrm{w}$ których zapisie dziesiętnym występują do-

kładnie dwa 0 $\mathrm{i}$ dokladnie jedna cyfra l?

4. Niech $A=\displaystyle \{x\in \mathbb{R}:\frac{1}{x^{2}+2}\geq\frac{1}{6-3x}\}$ oraz $B=\{x\in \mathbb{R}:|x-2|+|x+2|<6\}.$

Znalez$\acute{}$č $\mathrm{i}$ zaznaczyč na osi liczbowej zbiory $A, B$ oraz $(A\backslash B)\cup(B\backslash A).$

5. Uprościč wyrazenie

-$\sqrt{}$6{\it a}5-1$\sqrt{}$6{\it a}2{\it b}3($\sqrt{}$6{\it a}5--$\sqrt{}$6{\it ba})--{\it aa}$+-\sqrt{}${\it bab}

dla $a, b$, dla których ma ono sens, a następnie obliczyč jego wartośč, przyjmując

$a=4-2\sqrt{3}\mathrm{i} b=3+2\sqrt{2}.$

6. Grupa l75 robotników firmy pana Kowalskiego miala wykonač pewien odcinek autostra-

dy A4 $\mathrm{w}$ określonym terminie. Po upływie 30 dni wspó1nej pracy okazało się, $\dot{\mathrm{z}}\mathrm{e}$ musi

$\mathrm{m}\mathrm{o}\dot{\mathrm{z}}$ liwie szybko dokonač naprawy oddanego wcześniej odcinka autostrady A2. $\mathrm{W}\mathrm{z}\mathrm{w}\mathrm{i}_{\Phi}\mathrm{z}$-

ku $\mathrm{z}$ tym codziennie odsyłano do tego zadania kolejnych 3 robotników, wskutek czego

prace przy budowie autostrady A4 zakończono $\mathrm{z} 21$-dniowym opóznieniem. $\mathrm{W}$ jakim

czasie planowano pierwotnie wybudowač dany odcinek autostrady A4?




PRACA KONTROLNA nr l- POZIOM ROZSZERZONY

l. Pan Kowalski zaciągną131 grudnia $\mathrm{p}\mathrm{o}\dot{\mathrm{z}}$ yczkę 4000 złotych oprocentowaną $\mathrm{w}$ wysokości

18\% $\mathrm{w}$ skali roku. Zobowiązaf się splacič ją $\mathrm{w}$ ciągu roku $\mathrm{w}$ trzech równych ratach

płatnych 30 kwietnia, 30 sierpnia $\mathrm{i}30$ grudnia. Oprocentowanie $\mathrm{p}\mathrm{o}\dot{\mathrm{z}}$ yczki liczy się od l

stycznia, a odsetki od kredytu naliczane są $\mathrm{w}$ terminach płatności rat. Obliczyč wysokośč

tych rat $\mathrm{w}$ zaokrągleniu do pefnych groszy.

2. $\mathrm{Z}$ dwu stacji wyjezdzają jednocześnie naprzeciw siebie dwa pociągi. Pierwszy jedzie $\mathrm{z}$

prędkości$\Phi$ 15 $\mathrm{k}\mathrm{m}/\mathrm{h}$ większą $\mathrm{n}\mathrm{i}\dot{\mathrm{z}}$ drugi $\mathrm{i}$ spotykają się po 40 minutach. Gdyby drugi

pociąg wyjechaf $09$ minut wcześniej, to, jadąc $\mathrm{z}$ tymi samymi prędkościami, spotkalyby

się $\mathrm{w}$ połowie drogi. Znalez$\acute{}$č odległośč między miejscowościami oraz prędkości $\mathrm{k}\mathrm{a}\dot{\mathrm{z}}$ dego $\mathrm{z}$

pociągów.

3. Ile jest liczb pięciocyfrowych podzielnych przez 9, które $\mathrm{w}$ rozwinięciu dziesiętnym mają:

a) obie cyfry 1, 2 $\mathrm{i}$ tylko $\mathrm{t}\mathrm{e}$? b) obie cyfry 2, 3 $\mathrm{i}$ tylko $\mathrm{t}\mathrm{e}$? c) wszystkie cyfry 1, 2, 3

$\mathrm{i}$ tylko $\mathrm{t}\mathrm{e}$? Odpowiedz/uzasadnič.

4. Narysowač na płaszczyz/nie zbiór $A=\{(x,y):\sqrt{-2x-x^{2}}\leq y\leq\sqrt{3}|x+1|\}$

jego pole.

i obliczyč

5. Uprościč wyrazenie (dla a, b, dla których ma ono sens)

$(\displaystyle \frac{\sqrt[6]{b}}{\sqrt{b}-\sqrt[6]{a^{3}b^{2}}}-\frac{a}{\sqrt{ab}-a\sqrt[3]{b}})[\frac{\sqrt[6]{a}}{\sqrt{b}(\sqrt[6]{a^{5}}-\sqrt[3]{a}\sqrt{b})}(\sqrt[6]{a^{5}}-\frac{b}{\sqrt[6]{a}})-\frac{\sqrt[6]{a}(\alpha-b)}{a\sqrt{b}+b\sqrt{a}}],$

a następnie obliczyč jego wartośč dla $a=6\sqrt{3}-10$

i

$b=10+6\sqrt{3}$

6. Dwaj turyści wyruszyli jednocześnie: jeden $\mathrm{z}$ punktu $A$ do punktu $B$, drugi-z $B$ do $A.$

$K\mathrm{a}\dot{\mathrm{z}}\mathrm{d}\mathrm{y}\mathrm{z}$ nich szedf ze stafą prędkością $\mathrm{i}$ dotarlszy do mety, natychmiast ruszaf $\mathrm{w}$ drogę

powrotną. Pierwszy raz minęli się $\mathrm{w}$ odległości 12 km od punktu $B$, drugi- po uplywie

6 godzin od momentu pierwszego spotkania-w odległości 6 km od punktu $A$. Obliczyč

odlegfośč punktów $A\mathrm{i}B\mathrm{i}$ prędkości, $\mathrm{z}$ jakimi poruszali się turyści.





XLIII

KORESPONDENCYJNY KURS

Z MATEMATYKI

luty 2014 r.

PRACA KONTROLNA $\mathrm{n}\mathrm{r} 6-$ POZIOM PODSTAWOWY

l. Rozwiąz równanie

2 $($log2 $(2-x))^{2}-3\log_{2}(2-x)-2=0.$

2. Rozwiąz nierównośč wykfadniczą

$4^{\frac{1}{2}x^{2}-x}\cdot 3^{x^{2}+7x-2}\leq 9^{x^{2}+2x}\cdot 2^{x-2}$

3. Określ dziedzinę funkcji $f(x)=\displaystyle \frac{-1}{1-\sqrt{5-x^{2}}}-1$. Dlajakich argumentów funkcja przyjmuje

wartości ujemne?

4. $\mathrm{W}$ przedziale $[0,2\pi]$ wyznacz wszystkie liczby spefniające równanie

$\mathrm{t}\mathrm{g}^{2}x=8|\cos x|-1.$

5. Oblicz pole ośmiokąta będącego wspólna cześcią kwadratu $0$ boku dfugości 4 oraz jego

obrazu $\mathrm{w}$ obrocie $0$ kąt $\displaystyle \frac{\pi}{4}$ względem środka kwadratu. Wyznacz promień okręgu opisanego

na tym ośmiokącie $\mathrm{i}\mathrm{s}$porząd $\acute{\mathrm{z}}$ rysunek.

6. Dane $\mathrm{s}\Phi$ punkty $A(0,-2)$ oraz $B(4,0)$. Wyznacz wszystkie punkty $P\mathrm{l}\mathrm{e}\dot{\mathrm{z}}$ ące na paraboli

$y=x^{2}$, dla których $\triangle ABP$ jest prostokątny. Sporząd $\acute{\mathrm{z}}$ rysunek.





PRACA KONTROLNA nr 6- POZIOM ROZSZERZONY

l. Liczby $a_{1}, a_{2}, \ldots, a_{n}$, gdzie $n$ jest pewną liczba parzystą, tworzą ciąg arytmetyczny

$0$ sumie 15. Suma wszystkich wyrazów $0$ numerach parzystych $\mathrm{w}$ tym $\mathrm{c}\mathrm{i}_{\Phi \mathrm{g}}\mathrm{u}$ wynosi 0,

a iloczyn $a_{1}a_{2}=150$. Jakie to liczby?

2. Rozwiąz nierównośč logarytmiczną

$\log_{3}(x^{3}-x^{2}-4x-2)\leq\log_{\sqrt{3}}\sqrt{x+1}.$

3. Rozwiąz nierównośč trygonometryczną

$1-2\displaystyle \sin^{2}2x+4\sin^{4}2x-8\sin^{6}2x+\cdots>\frac{1}{3-2\sin^{2}x},$

której lewa strona jest sumą nieskończonego ciągu geometrycznego. Zaznacz dziedzinę

$\mathrm{i}$ zbiór rozwiązań nierówności na kole trygonometrycznym.

4. Kwadrat $0$ boku długości 4 obrócono $0$ kąt $\displaystyle \frac{\pi}{6}$ względem środka kwadratu, $\mathrm{w}$ kierunku

przeciwnym do ruchu wskazówek zegara. Oblicz pole wspólnej części kwadratu wyjścio-

wego $\mathrm{i}$ jego obrazu $\mathrm{w}$ tym obrocie. Sporząd $\acute{\mathrm{z}}$ rysunek.

5. Wyznacz równania tych stycznych do okręgu $x^{2}+y^{2}=1$, które $\mathrm{w}$ przecięciu $\mathrm{z}$ okregiem

$x^{2}-16x+y^{2}+39=0$ tworzą cięciwy dfugości 8. $\mathrm{s}_{\mathrm{P}^{\mathrm{o}\mathrm{r}\mathrm{z}}\Phi^{\mathrm{d}\acute{\mathrm{z}}\mathrm{r}\mathrm{y}\mathrm{s}\mathrm{u}\mathrm{n}\mathrm{e}\mathrm{k}}}.$

6. Wyznacz $\mathrm{i}$ narysuj funkcję $g(m)$ określającą liczbę rozwiązań równania

$(m-1)\displaystyle \frac{1}{4^{x}}+(m+1)2^{1-x}=2-m$

$\mathrm{w}$ zalezności od rzeczywistego parametru $m.$





XLIII

KORESPONDENCYJNY KURS

Z MATEMATYKI

marzec 2014 r.

PRACA KONTROLNA nr 7- POZIOM PODSTAWOWY

l. Rozwiązač nierównośč $\displaystyle \frac{1}{|x-1|}\leq x+3\mathrm{i}$ podač jej interpretację graficzną.

2. $\mathrm{W}$ przedziale $[0,2\pi]$ rozwiązač nierównośč 2 $\sin^{2}x>1+\cos x$. Zbiór rozwiązań zaznaczyč

na kole trygonometrycznym.

3. Znalez/č równanie okręgu stycznego do obu osi ukfadu wspófrzędnych $\mathrm{i}$ do dodatniej

gałęzi hiperboli $y=\displaystyle \frac{1}{x}$. Sporzqdzič rysunek.

4. Zaznaczyč na płaszczy $\acute{\mathrm{z}}\mathrm{n}\mathrm{i}\mathrm{e}$ zbiory $A = \{(x,y):1-\sqrt{2|x|-x^{2}}\leq|y|\leq 1+\sqrt{2-|x|}\}$

oraz $B=\{(x,y):|x|\leq 1,|y|\leq 1\}\mathrm{i}$ obliczyč pole figury $B\backslash A.$

5. Trapez prostokątny, $\mathrm{w}$ którym stosunek dfugości podstaw wynosi 3 : 2, jest opisany na

okręgu $0$ promieniu $r$. Wyznaczyč stosunek pola koła do pola trapezu oraz cosinus kąta

ostrego $\mathrm{w}$ tym trapezie.

6. Plaszczyzna przechodząca przez krawędz/ podstawy graniastosfupa prawidfowego trój-

kątnego, $\mathrm{w}$ którym wszystkie krawędzie są równe, dzieli ten graniastosłup na dwie bryły

$0$ tej samej objętości. Znalez/č kąt nachylenia plaszczyzny do podstawy. Sporządzič ry-

sunek.





PRACA KONTROLNA nr 7- POZIOM ROZSZERZONY

l. Rozwiązač nierównośč $\displaystyle \frac{3}{x^{2}-2x}\leq\frac{1}{|x|}.$

2. $\mathrm{W}$ przedziale $[0,2\pi]$ rozwiązač nierównośč

zaznaczyč na kole trygonometrycznym.

$\sqrt{\sin^{2}x-\sin x} \geq \cos x$. Zbiór rozwiazań

3. Znalez/č $\mathrm{i}$ zaznaczyč na płaszczy $\acute{\mathrm{z}}\mathrm{n}\mathrm{i}\mathrm{e}$ zbiór punktów $\{(x,y):\log_{x^{2}+y^{2}}(x+2y)\geq 1\}.$

4. Znalez/č równanie okręgu stycznego do osi $Ox$ oraz do obu gałęzi krzywej $0$ równaniu

$y=\displaystyle \frac{1}{x^{2}}$. Sporządzič rysunek. Wskazówka: Skorzystač $\mathrm{z}$ algebraicznego warunku styczności.

5. $\mathrm{W}$ trapezie opisanym na okręgu $0$ promieniu $r$ kąt ostry przy podstawie $1\mathrm{e}\dot{\mathrm{Z}}\mathrm{a}\mathrm{c}\mathrm{y}$ naprzeciw

krótszej przekątnej ma miarę $30^{o}$, a krótsza przekątna tworzy $\mathrm{z}$ podstawą $\mathrm{k}_{\Phi}\mathrm{t}45^{o}$ Obli-

czyč obwód trapezu oraz tangens kąta pomiędzy jego przekątnymi. Sporządzič rysunek.

6. Przez wierzchofek $S$ stozka poprowadzono plaszczyznę przecinajqcąjego podstawę wzdfuz

cięciwy $AB$. Miara kąta $\angle ASB$ jest równa $\alpha$, a miara kąta $\angle AOB$ jest równa $\beta$, gdzie

$O$ jest środkiem podstawy. Obliczyč sinus kata rozwarcia stozka. Podač warunki rozwią-

zalności zadania oraz warunek, aby kąt rozwarcia stozka byf $\mathrm{k}_{\Phi}\mathrm{t}\mathrm{e}\mathrm{m}$ prostym.





KORESPONDENCYJNY KURS

Z MATEMATYKI

$\mathrm{p}\mathrm{a}\acute{\mathrm{z}}$dziernik 2013 $\mathrm{r}.$

PRACA KONTROLNA $\mathrm{n}\mathrm{r} 2-$ POZIOM PODSTAWOWY

l. Rozwiązač nierównośč $x^{3}+nx^{2}-m^{2}x-m^{2}n\leq 0$, gdzie

$m=\displaystyle \frac{64^{\frac{1}{3}}\sqrt{2}+8^{\frac{1}{3}}\sqrt{64}}{\sqrt[3]{64\sqrt{8}}}$

oraz

{\it n}$=$ -($\sqrt{}$($\sqrt{}$24)1-64)(3-41.)2-7-25-$\sqrt{}$-441 3

2. $\dot{\mathrm{D}}$ la jakich wartości $\alpha\in[0,2\pi]$ liczby $\sin\alpha,  6\cos\alpha$, 6 tg $\alpha$ tworzą ciąg geometryczny?

3. Suma pewnej ilości kolejnych liczb naturalnych równa jest 33, a róznica kwadratów

najwiekszej $\mathrm{i}$ najmniejszej wynosi 55. Wyznaczyč te 1iczby.

4. Narysowač wykres funkcji

$f(x)=$

gdy

gdy

$|x-2|\leq 3,$

$|x-2|>3$

$\mathrm{i}$ wyznaczyč zbiór jej wartości. Dla jakich argumentów $x$ wykres funkcji $f(x) \mathrm{l}\mathrm{e}\dot{\mathrm{z}}\mathrm{y}$ pod

prostą $x-2y+10=0$ ? Zilustrowač rozwiązanie graficznie.

5. Dlajakiego parametru $m$ równanie $x^{2}-mx+m^{2}-2m+1=0$ ma dwa rózne pierwiastki

$\mathrm{w}$ przedziale $(0,2)$ ?

6. Wierzchołek $A$ wykresu funkcji $f(x)=ax^{2}+bx+c\mathrm{l}\mathrm{e}\dot{\mathrm{z}}\mathrm{y}$ na prostej $x=3\mathrm{i}$ jest odległy

od początku ukladu współrzędnych $05$. Pole trójkąta, którego wierzchofkami $\mathrm{s}\Phi$ punkty

przecięcia wykresu $\mathrm{z}$ osią $Ox$ oraz punkt $A$ równe jest 8. Podač wzór funkcji, której

wykres jest obrazem paraboli $f(x)\mathrm{w}$ symetrii względem punktu $(1,f(1)).$





PRACA KONTROLNA nr 2- POZIOM ROZSZERZONY

l. Obliczyč $a$ wiedząc, $\dot{\mathrm{z}}\mathrm{e}$ liczba $[\displaystyle \frac{2+9\sqrt{2}}{2\sqrt{2}-2}-\frac{1}{2}(2+\sqrt{2})^{2}]-(\frac{\sqrt[6]{32}}{2\sqrt{2}-2})^{3}$ jest miejscem zero-

wym funkcji $f(x)=2^{x}-a^{3}x.$

2. Dziesiąty wyraz rozwinięcia $(\displaystyle \frac{1}{\sqrt{x}}-\sqrt[3]{x})^{n}$ nie zawiera $x$. Wyznaczyč współczynniki przy

najnizszej $\mathrm{i}$ najwyzszej potędze $x.$

3. Wyznaczyč zbiór wartości funkcji $f(x)=(\displaystyle \log_{2}x)^{3}+\log_{2}\frac{x^{2}}{4}-1$ na przedziale (1, 2).

4. Tangens kąta ostrego $\alpha$ równy jest $\displaystyle \frac{a}{7b}$, gdzie

$a=(\sqrt{2}+1)^{3}-(\sqrt{2}-1)^{3}b=(\sqrt{\sqrt{2}+1}-\sqrt{\sqrt{2}-1})^{2}$

Wyznaczyč wartości pozostalych funkcji trygonometrycznych tego kąta oraz $\mathrm{k}_{\Phi^{\mathrm{t}\mathrm{a}}}2\alpha.$

Jaka jest miara kąta $\alpha$?

5. Trzy liczby $x<y<z$, których suma jest równa 93 tworza ciąg geometryczny. Te same

liczby $\mathrm{m}\mathrm{o}\dot{\mathrm{z}}$ na uwazač za pierwszy, drugi $\mathrm{i}$ siódmy wyraz ciqgu arytmetycznego. Jakie to

liczby?

6. Określič liczbę pierwiastków równania $(2m-3)x^{2}-4m|x|+m-1=0\mathrm{w}$ zalezności od

parametru $m.$





XLIII

KORESPONDENCYJNY KURS

Z MATEMATYKI

listopad 2013 r.

PRACA KONTROLNA $\mathrm{n}\mathrm{r} 3-$ POZIOM PODSTAWOWY

l. Wektory $\vec{AB} = [2$, 2$], \vec{BC} = [-2,3], \vec{CD} = [-2,-4]$ są bokami czworokąta ABCD.

Punkty $K\mathrm{i}M$ są środkami boków $CD$ oraz $AD$. Obliczyč pole trójkąta $KMB$ oraz jego

stosunek do pola całego czworokąta. Sporządzič rysunek.

2. Narysowač wykres funkcji

$f(x)=\displaystyle \frac{1}{\sqrt{1+\mathrm{t}\mathrm{g}^{2}x}}-\frac{1}{2},$

a nastepnie rozwiązač graficznie nierównośč $f(x)<0.$

3. Rozwiązač nierównośč $w(x-2)>w(x-1)$, gdzie

$w(x)=x^{4}-4x^{3}+5x^{2}-2x.$

4. Tangens kąta ostrego $\alpha$ równy jest

$\sqrt{7-4\sqrt{3}}.$

Wyznaczyč wartości pozostałych funkcji trygonometrycznych tego kąta. Wykorzystując

wzór $\sin 2\alpha=2\sin\alpha\cos\alpha$ wyznaczyč miarę kąta $\alpha.$

5. Punkt $B(2,6)$ jest wierzchołkiem trójkąta prostokątnego $0$ polu 25, którego przeciwpro-

stokątna zawarta jest $\mathrm{w}$ prostej $x-2y=0$. Obliczyč wysokośč opuszczoną na przeciw-

$\mathrm{P}^{\mathrm{r}\mathrm{o}\mathrm{s}\mathrm{t}\mathrm{o}\mathrm{k}}\Phi^{\mathrm{t}\mathrm{n}}\Phi^{\mathrm{i}}$ wyznaczyč wspófrzędne pozostafych wierzchofków trójkąta.

6. Dane są punkty $A(-1,-3) \mathrm{i}B(2,-2)$. Na paraboli $y=x^{2}-1$ znalez/č taki punkt $C,$

aby pole trójkąta $ABC$ byfo najmniejsze.





PRACA KONTROLNA nr 3- POZIOM ROZSZERZONY

l. Dla jakich wartości parametru $\alpha\in(0,2\pi)$ funkcja

$ f(x)=\sin\alpha\cdot x^{2}-x+\cos\alpha$

posiada minimum lokalne $\mathrm{i}$ wartośč najmniejsza funkcji jest ujemna?

2. Rozwiązač równanie

$\sqrt{3}+\mathrm{t}\mathrm{g}x=4\sin x.$

3. Wielomian $w(x)=x^{4}+3x^{3}+px^{2}+qx+r$ dzieli się przez $x-2$, a resztą $\mathrm{z}$ jego dzielenia

przez $x^{2}+x-2$ jest $-4x-12$. Wyznaczyč współczynniki $p, q, r\mathrm{i}$ rozwiązač nierównośč

$w(x)\geq 0.$

4. $\mathrm{W}$ czworokącie ABCD dane są $AD=a$ oraz $AB=2a$. Wiadomo, $\dot{\mathrm{z}}\mathrm{e}\vec{AC}=2\vec{AB}+3\vec{AD}$

oraz $\angle BAD=60^{\mathrm{o}}$. Stosując rachunek wektorowy obliczyč cosinus kąta $ABC$ oraz obwód

czworokąta. Rozwiązanie zilustrowač rysunkiem.

5. Punkt $P(-\displaystyle \sqrt{3},\frac{\sqrt{3}}{2})$ jest środkiem boku trójkąta równobocznego. Drugi bok trójkąta $\mathrm{l}\mathrm{e}\dot{\mathrm{z}}\mathrm{y}$

na prostej $y=2x$. Wyznaczyč współrzędne wszystkich wierzchołków trójkąta $\mathrm{i}$ obliczyč

jego pole. Sporzqdzič rysunek.

6. Wyznaczyč zbiór punktów płaszczyzny utworzonych przez środki wszystkich okręgów

stycznych jednocześnie do prostej $y=0$ oraz do okregu $x^{2}+y^{2}-4y+3=0$. Sporzqdzič

rysunek.





XLIII

KORESPONDENCYJNY KURS

Z MATEMATYKI

grudzień 2013 r.

PRACA KONTROLNA $\mathrm{n}\mathrm{r} 4-$ POZIOM PODSTAWOWY

l. Na półkuli $0$ promieniu $r$ opisano stozek $0$ kacie rozwarcia $2\alpha \mathrm{w}$ taki sposób, $\dot{\mathrm{z}}\mathrm{e}$ środek

podstawy stozka znajduje się $\mathrm{w}$ środku pófkuli. Oblicz objętośč $\mathrm{i}$ pole powierzchni stozka.

Jaki jest stosunek objetości stozka do objętości półkuli dla kąta rozwarcia $\pi/3$?

2. Kula jest styczna do wszystkich krawędzi czworościanu foremnego 0 krawędzi a. Oblicz

promień tej kuli.

3. $\mathrm{W}$ kwadrat ABCD wpisano kwadrat EFGH, który zajmuje 3/4 jego powierzchni. $\mathrm{W}$

jakim stosunku wierzchofki kwadratu EFGH dzielą boki kwadratu ABCD?

4. Niech $f(x)=4^{x+4}-7\cdot 3^{x+3}\mathrm{i}g(x)=6\cdot 4^{4x}-3^{4x+2}$

Rozwiąz nierównośč $f(x-3)\displaystyle \leq g(\frac{x}{4}).$

5. Znajd $\acute{\mathrm{z}}$ wymiary trapezu równoramiennego $0$ obwodzie $d\mathrm{i}$ kącie ostrym przy podstawie

$\alpha 0$ największym polu.

6. $\mathrm{W}$ trójkąt równoboczny $0$ boku $a$ wpisujemy trójkąt, którego wierzchołkami są środki

boków naszego trójkąta. Wpisany trójkat kolorujemy na niebiesko. Następnie $\mathrm{w}\mathrm{k}\mathrm{a}\dot{\mathrm{z}}\mathrm{d}\mathrm{y}\mathrm{z}$

niepokolorowanych trójkątów wpisujemy $\mathrm{w}$ ten sam sposób kolejne niebieskie trójk$\Phi$ty,

itd. Znajd $\acute{\mathrm{z}}$ sumę pól niebieskich trójkątów po $n$ krokach. Po ilu krokach niebieskie

trójkąty zajmą co najmniej 50\%, a po i1u- 75\% powierzchni wyjściowego trójkąta?





PRACA KONTROLNA nr 4- POZIOM ROZSZERZONY

1. $\mathrm{W}$ trójkącie prostokqtnym $ABC$ dane sq przyprostokątne $|AC| = 3$ oraz $|CB| = 4.$

Punkt $D$ jest spodkiem wysokości opuszczonej $\mathrm{z}$ wierzcholka kąta prostego, a $E\mathrm{i}$ {\it F}-

punktami przeciecia przeciwprostokątnej $\mathrm{z}$ dwusiecznymi kątów $ACD \mathrm{i} DCB$. Oblicz

długośč odcinka $EF$

2. Sześcian przecinamy pfaszczyzną, która przechodzi przez $\mathrm{P}^{\mathrm{r}\mathrm{z}\mathrm{e}\mathrm{k}}\Phi^{\mathrm{t}\mathrm{n}}\Phi$ jednej ze ścian oraz

środek krawędzi przeciwleglej ściany. Pod jakim kątem przecinają się przekątne otrzy-

manego przekroju?

3. Dane jest równanie kwadratowe $x^{2}+x(1-2^{m})+3(2^{m-2}-4^{m-1})=0$. Dla jakiego pa-

rametru $m$:

a) równanie ma pierwiastki róznych znaków?

b) suma kwadratów pierwiastków równania jest równa co najmniej l?

4. Pole powierzchni bocznej ostrosłupa prawidłowego $0$ podstawie trójkątnej wynosi $\sqrt{39}/4,$

a krawędz/ podstawy ma dlugośč l. Oblicz kąt nachylenia krawędzi bocznej do podstawy.

5. $\mathrm{W}$ trójkącie równoramiennym $ABC\mathrm{o}$ podstawie AB środkowe poprowadzone $\mathrm{z}$ wierz-

cholków $A\mathrm{i}B$ przecinajq się pod kątem prostym. Wyznacz sinus kąta $ACB.$

6. $\mathrm{W}$ trójkąt równoboczny $0$ boku $a$ wpisujemy okrąg. Następnie $\mathrm{w}\mathrm{k}\mathrm{a}\dot{\mathrm{z}}$ dym $\mathrm{z}$ trzech rogów

wpisujemy kolejny okrąg styczny do wpisanego okręgu oraz do dwóch boków trójkqta.

Postepujemy tak nieskończenie wiele razy. Oblicz sumę obwodów wpisanych okręgów.

Jaką powierzchnię trójkąta zajmują wpisane kola?





XLIII

KORESPONDENCYJNY KURS

Z MATEMATYKI

styczeń 2014 r.

PRACA KONTROLNA nr $5-$ POZIOM PODSTAWOWY

l. Na ile sposobów $\mathrm{z}$ grupy 10 ch1opców $\mathrm{i}8$ dziewcząt $\mathrm{m}\mathrm{o}\dot{\mathrm{z}}$ na wybrač dwie sześcioosobowe

druzyny do siatkówki $\mathrm{t}\mathrm{a}\mathrm{k}$, aby $\mathrm{w}\mathrm{k}\mathrm{a}\dot{\mathrm{z}}$ dej druzynie było po trzech chłopców?

2. Rzucamy pięcioma kostkami do gry. Co jest bardziej prawdopodobne: wyrzucenie tej

samej liczby oczek na co najmniej czterech kostkach, czy otrzymaniejednej $\mathrm{z}$ konfiguracji

1, 2, 3, 4, $5\mathrm{l}\mathrm{u}\mathrm{b}2$, 3, 4, 5, 6?

3. Wyznaczyč wszystkie wartości parametru $m$, dla których uklad równań

$\left\{\begin{array}{l}
x^{2}+y^{2}=2\\
4x^{2}-4y+m=0
\end{array}\right.$

ma dokładnie: a) jedno; b) $\mathrm{d}\mathrm{w}\mathrm{a};\mathrm{c}$) trzy rozwiązania. Uzasadnič odpowied $\acute{\mathrm{z}}$. Rozwiązanie

zilustrowač rysunkiem.

4. Obliczyč prawdopodobieństwo, $\dot{\mathrm{z}}\mathrm{e}$ dwie losowo wybrane rózne przekątne ośmiokąta fo-

remnego przecinają się.

5. Dany jest punkt $C(3,3)$. Na prostych $l$ : $x-y+1 = 0$ oraz $k$ : $x+2y-5 = 0,$

przecinających się $\mathrm{w}$ punkcie $M$, znalez/č odpowiednio punkty A $\mathrm{i}B\mathrm{t}\mathrm{a}\mathrm{k}$, aby kąt $\angle ACB$

był prosty, a czworokąt ABCM był trapezem. Sporzadzič rysunek.

6. $\mathrm{W}$ ostrosfupie prawidfowym trójk$\Phi$tnym dane są kąt pfaski $ 2\gamma$ przy wierzchofku oraz

odległośč $d$ krawędzi bocznej od przeciwleglej krawędzi podstawy. Obliczyč objetośč

ostroslupa. Następnie podstawič $2\displaystyle \gamma=\frac{\pi}{6}, d=\sqrt[4]{3} \mathrm{i}$ wynik podač $\mathrm{w}$ najprostszej postaci.





PRACA KONTROLNA nr $5 -$ POZIOM ROZSZERZONY

l. Na ile sposobów $\mathrm{m}\mathrm{o}\dot{\mathrm{z}}$ na ustawič $\mathrm{w}$ rzedzie trzy rózne pary butów $\mathrm{t}\mathrm{a}\mathrm{k}$, aby buty co naj-

mniej jednej pary stafy obok siebie, przy czym but lewy $\mathrm{z}$ lewej strony.

2. Stosując zasadę indukcji matematycznej, udowodnič nierównośč

$1+\displaystyle \sqrt{2}+\sqrt{3}+\ldots+\sqrt{n}\geq\frac{2}{3}n\sqrt{n+1},$

$n\geq 1.$

3. Pan Kowalski wyrusza $\mathrm{z}$ punktu $S$ na spacer po parku, którego plan jest przedstawiony
\begin{center}
\includegraphics[width=34.644mm,height=28.800mm]{./KursMatematyki_PolitechnikaWroclawska_2013_2014_page9_images/image001.eps}
\end{center}
na rysunku. Postanawia isc $\mathrm{k}\mathrm{a}\dot{\mathrm{z}}\mathrm{d}$ alejk co najwyzej jeden $\mathrm{r}\mathrm{a}\mathrm{z}.$

Obliczyc prawdopodobieństwo, $\dot{\mathrm{z}}\mathrm{e}$ przejdzie przez punkt $M,$

$\mathrm{j}\mathrm{e}\dot{\mathrm{z}}$ eli na $\mathrm{k}\mathrm{a}\dot{\mathrm{z}}$ dym skrzyzowaniu alejek wybiera kolejn (jeszcze

nie przebyt) alejk $\mathrm{z}$ tym samym prawdopodobienstwem lub

konczy spacer, gdy nie ma takiej alejki.

4. Uczeń zna odpowiedzi na 20 spośród 30 pytań egzaminacyjnych. Na egzaminie losuje dwa

pytania. $\mathrm{J}\mathrm{e}\dot{\mathrm{z}}$ eli odpowie poprawnie na oba, to egzamin $\mathrm{z}\mathrm{d}\mathrm{a}, \mathrm{j}\mathrm{e}\dot{\mathrm{z}}$ eli na $\dot{\mathrm{z}}$ adne, to nie $\mathrm{z}\mathrm{d}\mathrm{a},$

a $\mathrm{j}\mathrm{e}\dot{\mathrm{z}}$ eli na jedno, to wynik egzaminu rozstrzyga odpowied $\acute{\mathrm{z}}$ na dodatkowe wylosowane

pytanie. Obliczyč prawdopodobieństwo, $\dot{\mathrm{z}}\mathrm{e}$ uczeń zda egzamin.

5. $\mathrm{W}$ trójkąt $0$ wierzchofkach $A(-1,-1), B(3,1), C(1,3)$ wpisano kwadrat $\mathrm{t}\mathrm{a}\mathrm{k}, \dot{\mathrm{z}}\mathrm{e}$ dwa jego

wierzchołki lezą na boku $AB$ trójkąta. Wyznaczyč współrzędne wierzcholków kwadratu

oraz stosunek pola kwadratu do pola trójkąta. Sporz$\Phi$dzič rysunek.

6. Ostrosłup prawidlowy czworokątny ABCDS $0$ krawędzi podstawy $a$ ma pole powierzchni

całkowitej $5\alpha^{2}$ Środkiem krawedzi bocznej $AS$ jest punkt $M$. Obliczyč promień kuli

opisanej na ostroslupie ABCDM oraz cosinus kąta pomiędzy ścianami bocznymi $CDM$

oraz $BCM.$



\end{document}