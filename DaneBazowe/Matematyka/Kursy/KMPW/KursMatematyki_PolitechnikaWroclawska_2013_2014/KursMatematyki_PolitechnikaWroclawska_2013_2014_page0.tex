\documentclass[a4paper,12pt]{article}
\usepackage{latexsym}
\usepackage{amsmath}
\usepackage{amssymb}
\usepackage{graphicx}
\usepackage{wrapfig}
\pagestyle{plain}
\usepackage{fancybox}
\usepackage{bm}

\begin{document}

XLIII

KORESPONDENCYJNY KURS

Z MATEMATYKI

wrzesień 2013 r.

PRACA KONTROLNA $\mathrm{n}\mathrm{r} 1 -$ POZIOM PODSTAWOWY

l. Wzrost kursu Euro $\mathrm{w}$ stosunku do złotego spowodował podwyzkę ceny nowego modelu

Volvo $0$ 5\%. Poniewaz nie było popytu na te samochody, więc postanowiono ustalič cenę

promocyjną na poziomie odpowiadającym wzrostowi kursu Euro $0$ 2\%.

a) $\mathrm{O}$ ile procent cena promocyjna byfa $\mathrm{n}\mathrm{i}\dot{\mathrm{z}}$ sza od ceny wynikającej $\mathrm{z}$ faktycznego wzro-

stu kursu Euro $\mathrm{w}$ stosunku do zfotego? Wynik podač $\mathrm{z}$ dokladności$\Phi$ do l promila.

b) Ile pan Kowalski stracif na wzroście kursu Euro, a ile zyskal dzięki cenie promo-

cyjnej, $\mathrm{j}\mathrm{e}\dot{\mathrm{z}}$ eli kupił samochód za 56000? Rachunki prowadzič $\mathrm{z}$ dokładnością do

cafkowitych złotych.

2. $\mathrm{Z}$ obozu A do obozu $\mathrm{B}\mathrm{m}\mathrm{o}\dot{\mathrm{z}}$ na przejśč drogą $\dot{\mathrm{z}}$ wirową lub ściezką przez las, która jest $0$

sześč kilometrów krótsza $\mathrm{n}\mathrm{i}\dot{\mathrm{z}}$ droga $\dot{\mathrm{z}}$ wirowa. Bolek wyszedł $\mathrm{z}$ A $\mathrm{i}$ idąc ściezką $\mathrm{z}$ prędkościq

4 $\mathrm{k}\mathrm{m}/\mathrm{h}$ dotarf do $\mathrm{B} 1$ godzinę wcześniej $\mathrm{n}\mathrm{i}\dot{\mathrm{z}}$ Lolek, który $\mathrm{w}$ tym samym momencie

wyruszył drogą $\dot{\mathrm{z}}$ wirową. Znalez/č dlugośč ściezki, wiedząc, $\dot{\mathrm{z}}\mathrm{e}$ prędkośč, $\mathrm{z}$ jaką porusza

się Lolek wyraza się liczbą całkowitą.

3. Ile jest naturalnych liczb pięciocyfrowych, $\mathrm{w}$ których zapisie dziesiętnym występują do-

kładnie dwa 0 $\mathrm{i}$ dokladnie jedna cyfra l?

4. Niech $A=\displaystyle \{x\in \mathbb{R}:\frac{1}{x^{2}+2}\geq\frac{1}{6-3x}\}$ oraz $B=\{x\in \mathbb{R}:|x-2|+|x+2|<6\}.$

Znalez$\acute{}$č $\mathrm{i}$ zaznaczyč na osi liczbowej zbiory $A, B$ oraz $(A\backslash B)\cup(B\backslash A).$

5. Uprościč wyrazenie

-$\sqrt{}$6{\it a}5-1$\sqrt{}$6{\it a}2{\it b}3($\sqrt{}$6{\it a}5--$\sqrt{}$6{\it ba})--{\it aa}$+-\sqrt{}${\it bab}

dla $a, b$, dla których ma ono sens, a następnie obliczyč jego wartośč, przyjmując

$a=4-2\sqrt{3}\mathrm{i} b=3+2\sqrt{2}.$

6. Grupa l75 robotników firmy pana Kowalskiego miala wykonač pewien odcinek autostra-

dy A4 $\mathrm{w}$ określonym terminie. Po upływie 30 dni wspó1nej pracy okazało się, $\dot{\mathrm{z}}\mathrm{e}$ musi

$\mathrm{m}\mathrm{o}\dot{\mathrm{z}}$ liwie szybko dokonač naprawy oddanego wcześniej odcinka autostrady A2. $\mathrm{W}\mathrm{z}\mathrm{w}\mathrm{i}_{\Phi}\mathrm{z}$-

ku $\mathrm{z}$ tym codziennie odsyłano do tego zadania kolejnych 3 robotników, wskutek czego

prace przy budowie autostrady A4 zakończono $\mathrm{z} 21$-dniowym opóznieniem. $\mathrm{W}$ jakim

czasie planowano pierwotnie wybudowač dany odcinek autostrady A4?
\end{document}
