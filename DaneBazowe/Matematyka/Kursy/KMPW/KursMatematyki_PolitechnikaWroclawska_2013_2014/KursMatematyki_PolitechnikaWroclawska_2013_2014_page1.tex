\documentclass[a4paper,12pt]{article}
\usepackage{latexsym}
\usepackage{amsmath}
\usepackage{amssymb}
\usepackage{graphicx}
\usepackage{wrapfig}
\pagestyle{plain}
\usepackage{fancybox}
\usepackage{bm}

\begin{document}

PRACA KONTROLNA nr l- POZIOM ROZSZERZONY

l. Pan Kowalski zaciągną131 grudnia $\mathrm{p}\mathrm{o}\dot{\mathrm{z}}$ yczkę 4000 złotych oprocentowaną $\mathrm{w}$ wysokości

18\% $\mathrm{w}$ skali roku. Zobowiązaf się splacič ją $\mathrm{w}$ ciągu roku $\mathrm{w}$ trzech równych ratach

płatnych 30 kwietnia, 30 sierpnia $\mathrm{i}30$ grudnia. Oprocentowanie $\mathrm{p}\mathrm{o}\dot{\mathrm{z}}$ yczki liczy się od l

stycznia, a odsetki od kredytu naliczane są $\mathrm{w}$ terminach płatności rat. Obliczyč wysokośč

tych rat $\mathrm{w}$ zaokrągleniu do pefnych groszy.

2. $\mathrm{Z}$ dwu stacji wyjezdzają jednocześnie naprzeciw siebie dwa pociągi. Pierwszy jedzie $\mathrm{z}$

prędkości$\Phi$ 15 $\mathrm{k}\mathrm{m}/\mathrm{h}$ większą $\mathrm{n}\mathrm{i}\dot{\mathrm{z}}$ drugi $\mathrm{i}$ spotykają się po 40 minutach. Gdyby drugi

pociąg wyjechaf $09$ minut wcześniej, to, jadąc $\mathrm{z}$ tymi samymi prędkościami, spotkalyby

się $\mathrm{w}$ połowie drogi. Znalez$\acute{}$č odległośč między miejscowościami oraz prędkości $\mathrm{k}\mathrm{a}\dot{\mathrm{z}}$ dego $\mathrm{z}$

pociągów.

3. Ile jest liczb pięciocyfrowych podzielnych przez 9, które $\mathrm{w}$ rozwinięciu dziesiętnym mają:

a) obie cyfry 1, 2 $\mathrm{i}$ tylko $\mathrm{t}\mathrm{e}$? b) obie cyfry 2, 3 $\mathrm{i}$ tylko $\mathrm{t}\mathrm{e}$? c) wszystkie cyfry 1, 2, 3

$\mathrm{i}$ tylko $\mathrm{t}\mathrm{e}$? Odpowiedz/uzasadnič.

4. Narysowač na płaszczyz/nie zbiór $A=\{(x,y):\sqrt{-2x-x^{2}}\leq y\leq\sqrt{3}|x+1|\}$

jego pole.

i obliczyč

5. Uprościč wyrazenie (dla a, b, dla których ma ono sens)

$(\displaystyle \frac{\sqrt[6]{b}}{\sqrt{b}-\sqrt[6]{a^{3}b^{2}}}-\frac{a}{\sqrt{ab}-a\sqrt[3]{b}})[\frac{\sqrt[6]{a}}{\sqrt{b}(\sqrt[6]{a^{5}}-\sqrt[3]{a}\sqrt{b})}(\sqrt[6]{a^{5}}-\frac{b}{\sqrt[6]{a}})-\frac{\sqrt[6]{a}(\alpha-b)}{a\sqrt{b}+b\sqrt{a}}],$

a następnie obliczyč jego wartośč dla $a=6\sqrt{3}-10$

i

$b=10+6\sqrt{3}$

6. Dwaj turyści wyruszyli jednocześnie: jeden $\mathrm{z}$ punktu $A$ do punktu $B$, drugi-z $B$ do $A.$

$K\mathrm{a}\dot{\mathrm{z}}\mathrm{d}\mathrm{y}\mathrm{z}$ nich szedf ze stafą prędkością $\mathrm{i}$ dotarlszy do mety, natychmiast ruszaf $\mathrm{w}$ drogę

powrotną. Pierwszy raz minęli się $\mathrm{w}$ odległości 12 km od punktu $B$, drugi- po uplywie

6 godzin od momentu pierwszego spotkania-w odległości 6 km od punktu $A$. Obliczyč

odlegfośč punktów $A\mathrm{i}B\mathrm{i}$ prędkości, $\mathrm{z}$ jakimi poruszali się turyści.
\end{document}
