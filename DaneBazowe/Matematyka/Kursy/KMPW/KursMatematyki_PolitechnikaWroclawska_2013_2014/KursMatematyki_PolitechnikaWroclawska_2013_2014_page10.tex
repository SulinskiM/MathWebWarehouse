\documentclass[a4paper,12pt]{article}
\usepackage{latexsym}
\usepackage{amsmath}
\usepackage{amssymb}
\usepackage{graphicx}
\usepackage{wrapfig}
\pagestyle{plain}
\usepackage{fancybox}
\usepackage{bm}

\begin{document}

XLIII

KORESPONDENCYJNY KURS

Z MATEMATYKI

luty 2014 r.

PRACA KONTROLNA $\mathrm{n}\mathrm{r} 6-$ POZIOM PODSTAWOWY

l. Rozwiąz równanie

2 $($log2 $(2-x))^{2}-3\log_{2}(2-x)-2=0.$

2. Rozwiąz nierównośč wykfadniczą

$4^{\frac{1}{2}x^{2}-x}\cdot 3^{x^{2}+7x-2}\leq 9^{x^{2}+2x}\cdot 2^{x-2}$

3. Określ dziedzinę funkcji $f(x)=\displaystyle \frac{-1}{1-\sqrt{5-x^{2}}}-1$. Dlajakich argumentów funkcja przyjmuje

wartości ujemne?

4. $\mathrm{W}$ przedziale $[0,2\pi]$ wyznacz wszystkie liczby spefniające równanie

$\mathrm{t}\mathrm{g}^{2}x=8|\cos x|-1.$

5. Oblicz pole ośmiokąta będącego wspólna cześcią kwadratu $0$ boku dfugości 4 oraz jego

obrazu $\mathrm{w}$ obrocie $0$ kąt $\displaystyle \frac{\pi}{4}$ względem środka kwadratu. Wyznacz promień okręgu opisanego

na tym ośmiokącie $\mathrm{i}\mathrm{s}$porząd $\acute{\mathrm{z}}$ rysunek.

6. Dane $\mathrm{s}\Phi$ punkty $A(0,-2)$ oraz $B(4,0)$. Wyznacz wszystkie punkty $P\mathrm{l}\mathrm{e}\dot{\mathrm{z}}$ ące na paraboli

$y=x^{2}$, dla których $\triangle ABP$ jest prostokątny. Sporząd $\acute{\mathrm{z}}$ rysunek.
\end{document}
