\documentclass[a4paper,12pt]{article}
\usepackage{latexsym}
\usepackage{amsmath}
\usepackage{amssymb}
\usepackage{graphicx}
\usepackage{wrapfig}
\pagestyle{plain}
\usepackage{fancybox}
\usepackage{bm}

\begin{document}

PRACA KONTROLNA nr 6- POZIOM ROZSZERZONY

l. Liczby $a_{1}, a_{2}, \ldots, a_{n}$, gdzie $n$ jest pewną liczba parzystą, tworzą ciąg arytmetyczny

$0$ sumie 15. Suma wszystkich wyrazów $0$ numerach parzystych $\mathrm{w}$ tym $\mathrm{c}\mathrm{i}_{\Phi \mathrm{g}}\mathrm{u}$ wynosi 0,

a iloczyn $a_{1}a_{2}=150$. Jakie to liczby?

2. Rozwiąz nierównośč logarytmiczną

$\log_{3}(x^{3}-x^{2}-4x-2)\leq\log_{\sqrt{3}}\sqrt{x+1}.$

3. Rozwiąz nierównośč trygonometryczną

$1-2\displaystyle \sin^{2}2x+4\sin^{4}2x-8\sin^{6}2x+\cdots>\frac{1}{3-2\sin^{2}x},$

której lewa strona jest sumą nieskończonego ciągu geometrycznego. Zaznacz dziedzinę

$\mathrm{i}$ zbiór rozwiązań nierówności na kole trygonometrycznym.

4. Kwadrat $0$ boku długości 4 obrócono $0$ kąt $\displaystyle \frac{\pi}{6}$ względem środka kwadratu, $\mathrm{w}$ kierunku

przeciwnym do ruchu wskazówek zegara. Oblicz pole wspólnej części kwadratu wyjścio-

wego $\mathrm{i}$ jego obrazu $\mathrm{w}$ tym obrocie. Sporząd $\acute{\mathrm{z}}$ rysunek.

5. Wyznacz równania tych stycznych do okręgu $x^{2}+y^{2}=1$, które $\mathrm{w}$ przecięciu $\mathrm{z}$ okregiem

$x^{2}-16x+y^{2}+39=0$ tworzą cięciwy dfugości 8. $\mathrm{s}_{\mathrm{P}^{\mathrm{o}\mathrm{r}\mathrm{z}}\Phi^{\mathrm{d}\acute{\mathrm{z}}\mathrm{r}\mathrm{y}\mathrm{s}\mathrm{u}\mathrm{n}\mathrm{e}\mathrm{k}}}.$

6. Wyznacz $\mathrm{i}$ narysuj funkcję $g(m)$ określającą liczbę rozwiązań równania

$(m-1)\displaystyle \frac{1}{4^{x}}+(m+1)2^{1-x}=2-m$

$\mathrm{w}$ zalezności od rzeczywistego parametru $m.$
\end{document}
