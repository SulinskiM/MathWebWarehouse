\documentclass[a4paper,12pt]{article}
\usepackage{latexsym}
\usepackage{amsmath}
\usepackage{amssymb}
\usepackage{graphicx}
\usepackage{wrapfig}
\pagestyle{plain}
\usepackage{fancybox}
\usepackage{bm}

\begin{document}

XLIII

KORESPONDENCYJNY KURS

Z MATEMATYKI

marzec 2014 r.

PRACA KONTROLNA nr 7- POZIOM PODSTAWOWY

l. Rozwiązač nierównośč $\displaystyle \frac{1}{|x-1|}\leq x+3\mathrm{i}$ podač jej interpretację graficzną.

2. $\mathrm{W}$ przedziale $[0,2\pi]$ rozwiązač nierównośč 2 $\sin^{2}x>1+\cos x$. Zbiór rozwiązań zaznaczyč

na kole trygonometrycznym.

3. Znalez/č równanie okręgu stycznego do obu osi ukfadu wspófrzędnych $\mathrm{i}$ do dodatniej

gałęzi hiperboli $y=\displaystyle \frac{1}{x}$. Sporzqdzič rysunek.

4. Zaznaczyč na płaszczy $\acute{\mathrm{z}}\mathrm{n}\mathrm{i}\mathrm{e}$ zbiory $A = \{(x,y):1-\sqrt{2|x|-x^{2}}\leq|y|\leq 1+\sqrt{2-|x|}\}$

oraz $B=\{(x,y):|x|\leq 1,|y|\leq 1\}\mathrm{i}$ obliczyč pole figury $B\backslash A.$

5. Trapez prostokątny, $\mathrm{w}$ którym stosunek dfugości podstaw wynosi 3 : 2, jest opisany na

okręgu $0$ promieniu $r$. Wyznaczyč stosunek pola koła do pola trapezu oraz cosinus kąta

ostrego $\mathrm{w}$ tym trapezie.

6. Plaszczyzna przechodząca przez krawędz/ podstawy graniastosfupa prawidfowego trój-

kątnego, $\mathrm{w}$ którym wszystkie krawędzie są równe, dzieli ten graniastosłup na dwie bryły

$0$ tej samej objętości. Znalez/č kąt nachylenia plaszczyzny do podstawy. Sporządzič ry-

sunek.
\end{document}
