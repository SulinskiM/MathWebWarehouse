\documentclass[a4paper,12pt]{article}
\usepackage{latexsym}
\usepackage{amsmath}
\usepackage{amssymb}
\usepackage{graphicx}
\usepackage{wrapfig}
\pagestyle{plain}
\usepackage{fancybox}
\usepackage{bm}

\begin{document}

PRACA KONTROLNA nr 7- POZIOM ROZSZERZONY

l. Rozwiązač nierównośč $\displaystyle \frac{3}{x^{2}-2x}\leq\frac{1}{|x|}.$

2. $\mathrm{W}$ przedziale $[0,2\pi]$ rozwiązač nierównośč

zaznaczyč na kole trygonometrycznym.

$\sqrt{\sin^{2}x-\sin x} \geq \cos x$. Zbiór rozwiazań

3. Znalez/č $\mathrm{i}$ zaznaczyč na płaszczy $\acute{\mathrm{z}}\mathrm{n}\mathrm{i}\mathrm{e}$ zbiór punktów $\{(x,y):\log_{x^{2}+y^{2}}(x+2y)\geq 1\}.$

4. Znalez/č równanie okręgu stycznego do osi $Ox$ oraz do obu gałęzi krzywej $0$ równaniu

$y=\displaystyle \frac{1}{x^{2}}$. Sporządzič rysunek. Wskazówka: Skorzystač $\mathrm{z}$ algebraicznego warunku styczności.

5. $\mathrm{W}$ trapezie opisanym na okręgu $0$ promieniu $r$ kąt ostry przy podstawie $1\mathrm{e}\dot{\mathrm{Z}}\mathrm{a}\mathrm{c}\mathrm{y}$ naprzeciw

krótszej przekątnej ma miarę $30^{o}$, a krótsza przekątna tworzy $\mathrm{z}$ podstawą $\mathrm{k}_{\Phi}\mathrm{t}45^{o}$ Obli-

czyč obwód trapezu oraz tangens kąta pomiędzy jego przekątnymi. Sporządzič rysunek.

6. Przez wierzchofek $S$ stozka poprowadzono plaszczyznę przecinajqcąjego podstawę wzdfuz

cięciwy $AB$. Miara kąta $\angle ASB$ jest równa $\alpha$, a miara kąta $\angle AOB$ jest równa $\beta$, gdzie

$O$ jest środkiem podstawy. Obliczyč sinus kata rozwarcia stozka. Podač warunki rozwią-

zalności zadania oraz warunek, aby kąt rozwarcia stozka byf $\mathrm{k}_{\Phi}\mathrm{t}\mathrm{e}\mathrm{m}$ prostym.
\end{document}
