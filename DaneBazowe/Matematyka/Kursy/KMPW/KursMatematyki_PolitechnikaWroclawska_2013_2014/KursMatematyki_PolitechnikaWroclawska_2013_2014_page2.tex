\documentclass[a4paper,12pt]{article}
\usepackage{latexsym}
\usepackage{amsmath}
\usepackage{amssymb}
\usepackage{graphicx}
\usepackage{wrapfig}
\pagestyle{plain}
\usepackage{fancybox}
\usepackage{bm}

\begin{document}

KORESPONDENCYJNY KURS

Z MATEMATYKI

$\mathrm{p}\mathrm{a}\acute{\mathrm{z}}$dziernik 2013 $\mathrm{r}.$

PRACA KONTROLNA $\mathrm{n}\mathrm{r} 2-$ POZIOM PODSTAWOWY

l. Rozwiązač nierównośč $x^{3}+nx^{2}-m^{2}x-m^{2}n\leq 0$, gdzie

$m=\displaystyle \frac{64^{\frac{1}{3}}\sqrt{2}+8^{\frac{1}{3}}\sqrt{64}}{\sqrt[3]{64\sqrt{8}}}$

oraz

{\it n}$=$ -($\sqrt{}$($\sqrt{}$24)1-64)(3-41.)2-7-25-$\sqrt{}$-441 3

2. $\dot{\mathrm{D}}$ la jakich wartości $\alpha\in[0,2\pi]$ liczby $\sin\alpha,  6\cos\alpha$, 6 tg $\alpha$ tworzą ciąg geometryczny?

3. Suma pewnej ilości kolejnych liczb naturalnych równa jest 33, a róznica kwadratów

najwiekszej $\mathrm{i}$ najmniejszej wynosi 55. Wyznaczyč te 1iczby.

4. Narysowač wykres funkcji

$f(x)=$

gdy

gdy

$|x-2|\leq 3,$

$|x-2|>3$

$\mathrm{i}$ wyznaczyč zbiór jej wartości. Dla jakich argumentów $x$ wykres funkcji $f(x) \mathrm{l}\mathrm{e}\dot{\mathrm{z}}\mathrm{y}$ pod

prostą $x-2y+10=0$ ? Zilustrowač rozwiązanie graficznie.

5. Dlajakiego parametru $m$ równanie $x^{2}-mx+m^{2}-2m+1=0$ ma dwa rózne pierwiastki

$\mathrm{w}$ przedziale $(0,2)$ ?

6. Wierzchołek $A$ wykresu funkcji $f(x)=ax^{2}+bx+c\mathrm{l}\mathrm{e}\dot{\mathrm{z}}\mathrm{y}$ na prostej $x=3\mathrm{i}$ jest odległy

od początku ukladu współrzędnych $05$. Pole trójkąta, którego wierzchofkami $\mathrm{s}\Phi$ punkty

przecięcia wykresu $\mathrm{z}$ osią $Ox$ oraz punkt $A$ równe jest 8. Podač wzór funkcji, której

wykres jest obrazem paraboli $f(x)\mathrm{w}$ symetrii względem punktu $(1,f(1)).$
\end{document}
