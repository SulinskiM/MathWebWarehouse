\documentclass[a4paper,12pt]{article}
\usepackage{latexsym}
\usepackage{amsmath}
\usepackage{amssymb}
\usepackage{graphicx}
\usepackage{wrapfig}
\pagestyle{plain}
\usepackage{fancybox}
\usepackage{bm}

\begin{document}

PRACA KONTROLNA nr 2- POZIOM ROZSZERZONY

l. Obliczyč $a$ wiedząc, $\dot{\mathrm{z}}\mathrm{e}$ liczba $[\displaystyle \frac{2+9\sqrt{2}}{2\sqrt{2}-2}-\frac{1}{2}(2+\sqrt{2})^{2}]-(\frac{\sqrt[6]{32}}{2\sqrt{2}-2})^{3}$ jest miejscem zero-

wym funkcji $f(x)=2^{x}-a^{3}x.$

2. Dziesiąty wyraz rozwinięcia $(\displaystyle \frac{1}{\sqrt{x}}-\sqrt[3]{x})^{n}$ nie zawiera $x$. Wyznaczyč współczynniki przy

najnizszej $\mathrm{i}$ najwyzszej potędze $x.$

3. Wyznaczyč zbiór wartości funkcji $f(x)=(\displaystyle \log_{2}x)^{3}+\log_{2}\frac{x^{2}}{4}-1$ na przedziale (1, 2).

4. Tangens kąta ostrego $\alpha$ równy jest $\displaystyle \frac{a}{7b}$, gdzie

$a=(\sqrt{2}+1)^{3}-(\sqrt{2}-1)^{3}b=(\sqrt{\sqrt{2}+1}-\sqrt{\sqrt{2}-1})^{2}$

Wyznaczyč wartości pozostalych funkcji trygonometrycznych tego kąta oraz $\mathrm{k}_{\Phi^{\mathrm{t}\mathrm{a}}}2\alpha.$

Jaka jest miara kąta $\alpha$?

5. Trzy liczby $x<y<z$, których suma jest równa 93 tworza ciąg geometryczny. Te same

liczby $\mathrm{m}\mathrm{o}\dot{\mathrm{z}}$ na uwazač za pierwszy, drugi $\mathrm{i}$ siódmy wyraz ciqgu arytmetycznego. Jakie to

liczby?

6. Określič liczbę pierwiastków równania $(2m-3)x^{2}-4m|x|+m-1=0\mathrm{w}$ zalezności od

parametru $m.$
\end{document}
