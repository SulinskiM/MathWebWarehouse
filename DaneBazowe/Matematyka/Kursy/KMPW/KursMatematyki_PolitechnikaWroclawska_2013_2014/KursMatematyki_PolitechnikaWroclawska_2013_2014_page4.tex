\documentclass[a4paper,12pt]{article}
\usepackage{latexsym}
\usepackage{amsmath}
\usepackage{amssymb}
\usepackage{graphicx}
\usepackage{wrapfig}
\pagestyle{plain}
\usepackage{fancybox}
\usepackage{bm}

\begin{document}

XLIII

KORESPONDENCYJNY KURS

Z MATEMATYKI

listopad 2013 r.

PRACA KONTROLNA $\mathrm{n}\mathrm{r} 3-$ POZIOM PODSTAWOWY

l. Wektory $\vec{AB} = [2$, 2$], \vec{BC} = [-2,3], \vec{CD} = [-2,-4]$ są bokami czworokąta ABCD.

Punkty $K\mathrm{i}M$ są środkami boków $CD$ oraz $AD$. Obliczyč pole trójkąta $KMB$ oraz jego

stosunek do pola całego czworokąta. Sporządzič rysunek.

2. Narysowač wykres funkcji

$f(x)=\displaystyle \frac{1}{\sqrt{1+\mathrm{t}\mathrm{g}^{2}x}}-\frac{1}{2},$

a nastepnie rozwiązač graficznie nierównośč $f(x)<0.$

3. Rozwiązač nierównośč $w(x-2)>w(x-1)$, gdzie

$w(x)=x^{4}-4x^{3}+5x^{2}-2x.$

4. Tangens kąta ostrego $\alpha$ równy jest

$\sqrt{7-4\sqrt{3}}.$

Wyznaczyč wartości pozostałych funkcji trygonometrycznych tego kąta. Wykorzystując

wzór $\sin 2\alpha=2\sin\alpha\cos\alpha$ wyznaczyč miarę kąta $\alpha.$

5. Punkt $B(2,6)$ jest wierzchołkiem trójkąta prostokątnego $0$ polu 25, którego przeciwpro-

stokątna zawarta jest $\mathrm{w}$ prostej $x-2y=0$. Obliczyč wysokośč opuszczoną na przeciw-

$\mathrm{P}^{\mathrm{r}\mathrm{o}\mathrm{s}\mathrm{t}\mathrm{o}\mathrm{k}}\Phi^{\mathrm{t}\mathrm{n}}\Phi^{\mathrm{i}}$ wyznaczyč wspófrzędne pozostafych wierzchofków trójkąta.

6. Dane są punkty $A(-1,-3) \mathrm{i}B(2,-2)$. Na paraboli $y=x^{2}-1$ znalez/č taki punkt $C,$

aby pole trójkąta $ABC$ byfo najmniejsze.
\end{document}
