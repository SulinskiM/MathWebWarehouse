\documentclass[a4paper,12pt]{article}
\usepackage{latexsym}
\usepackage{amsmath}
\usepackage{amssymb}
\usepackage{graphicx}
\usepackage{wrapfig}
\pagestyle{plain}
\usepackage{fancybox}
\usepackage{bm}

\begin{document}

PRACA KONTROLNA nr 3- POZIOM ROZSZERZONY

l. Dla jakich wartości parametru $\alpha\in(0,2\pi)$ funkcja

$ f(x)=\sin\alpha\cdot x^{2}-x+\cos\alpha$

posiada minimum lokalne $\mathrm{i}$ wartośč najmniejsza funkcji jest ujemna?

2. Rozwiązač równanie

$\sqrt{3}+\mathrm{t}\mathrm{g}x=4\sin x.$

3. Wielomian $w(x)=x^{4}+3x^{3}+px^{2}+qx+r$ dzieli się przez $x-2$, a resztą $\mathrm{z}$ jego dzielenia

przez $x^{2}+x-2$ jest $-4x-12$. Wyznaczyč współczynniki $p, q, r\mathrm{i}$ rozwiązač nierównośč

$w(x)\geq 0.$

4. $\mathrm{W}$ czworokącie ABCD dane są $AD=a$ oraz $AB=2a$. Wiadomo, $\dot{\mathrm{z}}\mathrm{e}\vec{AC}=2\vec{AB}+3\vec{AD}$

oraz $\angle BAD=60^{\mathrm{o}}$. Stosując rachunek wektorowy obliczyč cosinus kąta $ABC$ oraz obwód

czworokąta. Rozwiązanie zilustrowač rysunkiem.

5. Punkt $P(-\displaystyle \sqrt{3},\frac{\sqrt{3}}{2})$ jest środkiem boku trójkąta równobocznego. Drugi bok trójkąta $\mathrm{l}\mathrm{e}\dot{\mathrm{z}}\mathrm{y}$

na prostej $y=2x$. Wyznaczyč współrzędne wszystkich wierzchołków trójkąta $\mathrm{i}$ obliczyč

jego pole. Sporzqdzič rysunek.

6. Wyznaczyč zbiór punktów płaszczyzny utworzonych przez środki wszystkich okręgów

stycznych jednocześnie do prostej $y=0$ oraz do okregu $x^{2}+y^{2}-4y+3=0$. Sporzqdzič

rysunek.
\end{document}
