\documentclass[a4paper,12pt]{article}
\usepackage{latexsym}
\usepackage{amsmath}
\usepackage{amssymb}
\usepackage{graphicx}
\usepackage{wrapfig}
\pagestyle{plain}
\usepackage{fancybox}
\usepackage{bm}

\begin{document}

XLIII

KORESPONDENCYJNY KURS

Z MATEMATYKI

grudzień 2013 r.

PRACA KONTROLNA $\mathrm{n}\mathrm{r} 4-$ POZIOM PODSTAWOWY

l. Na półkuli $0$ promieniu $r$ opisano stozek $0$ kacie rozwarcia $2\alpha \mathrm{w}$ taki sposób, $\dot{\mathrm{z}}\mathrm{e}$ środek

podstawy stozka znajduje się $\mathrm{w}$ środku pófkuli. Oblicz objętośč $\mathrm{i}$ pole powierzchni stozka.

Jaki jest stosunek objetości stozka do objętości półkuli dla kąta rozwarcia $\pi/3$?

2. Kula jest styczna do wszystkich krawędzi czworościanu foremnego 0 krawędzi a. Oblicz

promień tej kuli.

3. $\mathrm{W}$ kwadrat ABCD wpisano kwadrat EFGH, który zajmuje 3/4 jego powierzchni. $\mathrm{W}$

jakim stosunku wierzchofki kwadratu EFGH dzielą boki kwadratu ABCD?

4. Niech $f(x)=4^{x+4}-7\cdot 3^{x+3}\mathrm{i}g(x)=6\cdot 4^{4x}-3^{4x+2}$

Rozwiąz nierównośč $f(x-3)\displaystyle \leq g(\frac{x}{4}).$

5. Znajd $\acute{\mathrm{z}}$ wymiary trapezu równoramiennego $0$ obwodzie $d\mathrm{i}$ kącie ostrym przy podstawie

$\alpha 0$ największym polu.

6. $\mathrm{W}$ trójkąt równoboczny $0$ boku $a$ wpisujemy trójkąt, którego wierzchołkami są środki

boków naszego trójkąta. Wpisany trójkat kolorujemy na niebiesko. Następnie $\mathrm{w}\mathrm{k}\mathrm{a}\dot{\mathrm{z}}\mathrm{d}\mathrm{y}\mathrm{z}$

niepokolorowanych trójkątów wpisujemy $\mathrm{w}$ ten sam sposób kolejne niebieskie trójk$\Phi$ty,

itd. Znajd $\acute{\mathrm{z}}$ sumę pól niebieskich trójkątów po $n$ krokach. Po ilu krokach niebieskie

trójkąty zajmą co najmniej 50\%, a po i1u- 75\% powierzchni wyjściowego trójkąta?
\end{document}
