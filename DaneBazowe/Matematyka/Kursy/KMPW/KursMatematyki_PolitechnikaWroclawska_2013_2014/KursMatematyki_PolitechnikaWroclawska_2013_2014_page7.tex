\documentclass[a4paper,12pt]{article}
\usepackage{latexsym}
\usepackage{amsmath}
\usepackage{amssymb}
\usepackage{graphicx}
\usepackage{wrapfig}
\pagestyle{plain}
\usepackage{fancybox}
\usepackage{bm}

\begin{document}

PRACA KONTROLNA nr 4- POZIOM ROZSZERZONY

1. $\mathrm{W}$ trójkącie prostokqtnym $ABC$ dane sq przyprostokątne $|AC| = 3$ oraz $|CB| = 4.$

Punkt $D$ jest spodkiem wysokości opuszczonej $\mathrm{z}$ wierzcholka kąta prostego, a $E\mathrm{i}$ {\it F}-

punktami przeciecia przeciwprostokątnej $\mathrm{z}$ dwusiecznymi kątów $ACD \mathrm{i} DCB$. Oblicz

długośč odcinka $EF$

2. Sześcian przecinamy pfaszczyzną, która przechodzi przez $\mathrm{P}^{\mathrm{r}\mathrm{z}\mathrm{e}\mathrm{k}}\Phi^{\mathrm{t}\mathrm{n}}\Phi$ jednej ze ścian oraz

środek krawędzi przeciwleglej ściany. Pod jakim kątem przecinają się przekątne otrzy-

manego przekroju?

3. Dane jest równanie kwadratowe $x^{2}+x(1-2^{m})+3(2^{m-2}-4^{m-1})=0$. Dla jakiego pa-

rametru $m$:

a) równanie ma pierwiastki róznych znaków?

b) suma kwadratów pierwiastków równania jest równa co najmniej l?

4. Pole powierzchni bocznej ostrosłupa prawidłowego $0$ podstawie trójkątnej wynosi $\sqrt{39}/4,$

a krawędz/ podstawy ma dlugośč l. Oblicz kąt nachylenia krawędzi bocznej do podstawy.

5. $\mathrm{W}$ trójkącie równoramiennym $ABC\mathrm{o}$ podstawie AB środkowe poprowadzone $\mathrm{z}$ wierz-

cholków $A\mathrm{i}B$ przecinajq się pod kątem prostym. Wyznacz sinus kąta $ACB.$

6. $\mathrm{W}$ trójkąt równoboczny $0$ boku $a$ wpisujemy okrąg. Następnie $\mathrm{w}\mathrm{k}\mathrm{a}\dot{\mathrm{z}}$ dym $\mathrm{z}$ trzech rogów

wpisujemy kolejny okrąg styczny do wpisanego okręgu oraz do dwóch boków trójkqta.

Postepujemy tak nieskończenie wiele razy. Oblicz sumę obwodów wpisanych okręgów.

Jaką powierzchnię trójkąta zajmują wpisane kola?
\end{document}
