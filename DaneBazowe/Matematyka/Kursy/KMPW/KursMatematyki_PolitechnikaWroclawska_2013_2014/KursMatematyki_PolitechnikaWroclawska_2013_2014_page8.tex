\documentclass[a4paper,12pt]{article}
\usepackage{latexsym}
\usepackage{amsmath}
\usepackage{amssymb}
\usepackage{graphicx}
\usepackage{wrapfig}
\pagestyle{plain}
\usepackage{fancybox}
\usepackage{bm}

\begin{document}

XLIII

KORESPONDENCYJNY KURS

Z MATEMATYKI

styczeń 2014 r.

PRACA KONTROLNA nr $5-$ POZIOM PODSTAWOWY

l. Na ile sposobów $\mathrm{z}$ grupy 10 ch1opców $\mathrm{i}8$ dziewcząt $\mathrm{m}\mathrm{o}\dot{\mathrm{z}}$ na wybrač dwie sześcioosobowe

druzyny do siatkówki $\mathrm{t}\mathrm{a}\mathrm{k}$, aby $\mathrm{w}\mathrm{k}\mathrm{a}\dot{\mathrm{z}}$ dej druzynie było po trzech chłopców?

2. Rzucamy pięcioma kostkami do gry. Co jest bardziej prawdopodobne: wyrzucenie tej

samej liczby oczek na co najmniej czterech kostkach, czy otrzymaniejednej $\mathrm{z}$ konfiguracji

1, 2, 3, 4, $5\mathrm{l}\mathrm{u}\mathrm{b}2$, 3, 4, 5, 6?

3. Wyznaczyč wszystkie wartości parametru $m$, dla których uklad równań

$\left\{\begin{array}{l}
x^{2}+y^{2}=2\\
4x^{2}-4y+m=0
\end{array}\right.$

ma dokładnie: a) jedno; b) $\mathrm{d}\mathrm{w}\mathrm{a};\mathrm{c}$) trzy rozwiązania. Uzasadnič odpowied $\acute{\mathrm{z}}$. Rozwiązanie

zilustrowač rysunkiem.

4. Obliczyč prawdopodobieństwo, $\dot{\mathrm{z}}\mathrm{e}$ dwie losowo wybrane rózne przekątne ośmiokąta fo-

remnego przecinają się.

5. Dany jest punkt $C(3,3)$. Na prostych $l$ : $x-y+1 = 0$ oraz $k$ : $x+2y-5 = 0,$

przecinających się $\mathrm{w}$ punkcie $M$, znalez/č odpowiednio punkty A $\mathrm{i}B\mathrm{t}\mathrm{a}\mathrm{k}$, aby kąt $\angle ACB$

był prosty, a czworokąt ABCM był trapezem. Sporzadzič rysunek.

6. $\mathrm{W}$ ostrosfupie prawidfowym trójk$\Phi$tnym dane są kąt pfaski $ 2\gamma$ przy wierzchofku oraz

odległośč $d$ krawędzi bocznej od przeciwleglej krawędzi podstawy. Obliczyč objetośč

ostroslupa. Następnie podstawič $2\displaystyle \gamma=\frac{\pi}{6}, d=\sqrt[4]{3} \mathrm{i}$ wynik podač $\mathrm{w}$ najprostszej postaci.
\end{document}
