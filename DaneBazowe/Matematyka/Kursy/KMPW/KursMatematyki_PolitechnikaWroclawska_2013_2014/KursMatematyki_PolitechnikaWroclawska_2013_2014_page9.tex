\documentclass[a4paper,12pt]{article}
\usepackage{latexsym}
\usepackage{amsmath}
\usepackage{amssymb}
\usepackage{graphicx}
\usepackage{wrapfig}
\pagestyle{plain}
\usepackage{fancybox}
\usepackage{bm}

\begin{document}

PRACA KONTROLNA nr $5 -$ POZIOM ROZSZERZONY

l. Na ile sposobów $\mathrm{m}\mathrm{o}\dot{\mathrm{z}}$ na ustawič $\mathrm{w}$ rzedzie trzy rózne pary butów $\mathrm{t}\mathrm{a}\mathrm{k}$, aby buty co naj-

mniej jednej pary stafy obok siebie, przy czym but lewy $\mathrm{z}$ lewej strony.

2. Stosując zasadę indukcji matematycznej, udowodnič nierównośč

$1+\displaystyle \sqrt{2}+\sqrt{3}+\ldots+\sqrt{n}\geq\frac{2}{3}n\sqrt{n+1},$

$n\geq 1.$

3. Pan Kowalski wyrusza $\mathrm{z}$ punktu $S$ na spacer po parku, którego plan jest przedstawiony
\begin{center}
\includegraphics[width=34.644mm,height=28.800mm]{./KursMatematyki_PolitechnikaWroclawska_2013_2014_page9_images/image001.eps}
\end{center}
na rysunku. Postanawia isc $\mathrm{k}\mathrm{a}\dot{\mathrm{z}}\mathrm{d}$ alejk co najwyzej jeden $\mathrm{r}\mathrm{a}\mathrm{z}.$

Obliczyc prawdopodobieństwo, $\dot{\mathrm{z}}\mathrm{e}$ przejdzie przez punkt $M,$

$\mathrm{j}\mathrm{e}\dot{\mathrm{z}}$ eli na $\mathrm{k}\mathrm{a}\dot{\mathrm{z}}$ dym skrzyzowaniu alejek wybiera kolejn (jeszcze

nie przebyt) alejk $\mathrm{z}$ tym samym prawdopodobienstwem lub

konczy spacer, gdy nie ma takiej alejki.

4. Uczeń zna odpowiedzi na 20 spośród 30 pytań egzaminacyjnych. Na egzaminie losuje dwa

pytania. $\mathrm{J}\mathrm{e}\dot{\mathrm{z}}$ eli odpowie poprawnie na oba, to egzamin $\mathrm{z}\mathrm{d}\mathrm{a}, \mathrm{j}\mathrm{e}\dot{\mathrm{z}}$ eli na $\dot{\mathrm{z}}$ adne, to nie $\mathrm{z}\mathrm{d}\mathrm{a},$

a $\mathrm{j}\mathrm{e}\dot{\mathrm{z}}$ eli na jedno, to wynik egzaminu rozstrzyga odpowied $\acute{\mathrm{z}}$ na dodatkowe wylosowane

pytanie. Obliczyč prawdopodobieństwo, $\dot{\mathrm{z}}\mathrm{e}$ uczeń zda egzamin.

5. $\mathrm{W}$ trójkąt $0$ wierzchofkach $A(-1,-1), B(3,1), C(1,3)$ wpisano kwadrat $\mathrm{t}\mathrm{a}\mathrm{k}, \dot{\mathrm{z}}\mathrm{e}$ dwa jego

wierzchołki lezą na boku $AB$ trójkąta. Wyznaczyč współrzędne wierzcholków kwadratu

oraz stosunek pola kwadratu do pola trójkąta. Sporz$\Phi$dzič rysunek.

6. Ostrosłup prawidlowy czworokątny ABCDS $0$ krawędzi podstawy $a$ ma pole powierzchni

całkowitej $5\alpha^{2}$ Środkiem krawedzi bocznej $AS$ jest punkt $M$. Obliczyč promień kuli

opisanej na ostroslupie ABCDM oraz cosinus kąta pomiędzy ścianami bocznymi $CDM$

oraz $BCM.$
\end{document}
