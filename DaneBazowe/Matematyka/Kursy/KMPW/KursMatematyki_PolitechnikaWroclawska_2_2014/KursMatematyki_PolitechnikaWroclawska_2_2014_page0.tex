\documentclass[a4paper,12pt]{article}
\usepackage{latexsym}
\usepackage{amsmath}
\usepackage{amssymb}
\usepackage{graphicx}
\usepackage{wrapfig}
\pagestyle{plain}
\usepackage{fancybox}
\usepackage{bm}

\begin{document}

XLV

KORESPONDENCYJNY KURS

Z MATEMATYKI

$\mathrm{p}\mathrm{a}\acute{\mathrm{z}}$dziernik 2015 $\mathrm{r}.$

PRACA KONTROLNA $\mathrm{n}\mathrm{r} 2-$ POZIOM PODSTAWOWY

l. Czy suma długości przekatnych kwadratów $0$ polach 10 $\mathrm{i} \displaystyle \frac{21}{2}$ jest większa od długości

$\mathrm{P}^{\mathrm{r}\mathrm{z}\mathrm{e}\mathrm{k}}\Phi^{\mathrm{t}\mathrm{n}\mathrm{e}\mathrm{j}}$ kwadratu $0$ polu $\displaystyle \frac{81}{2}$? Odpowied $\acute{\mathrm{z}}$ uzasadnič nie $\mathrm{u}\dot{\mathrm{z}}$ ywając kalkulatora.

2. Grupa sluchaczy wykladu $\mathrm{z}$ algebry liczy 261 osób. Egzamin podstawowy zdała pewna

(dodatnia) ilośč osób. Po egzaminie poprawkowym liczba osób, które zdały, powiekszyla

się $0 5$, 6\%. Ile osób zdafo egzamin podstawowy (wskazówka: pamiętaj, $\dot{\mathrm{z}}\mathrm{e}$ ilośč osób,

które zdały egzamin jest liczbą calkowitą)?

3. Haslo do pewnego systemu komputerowego ma skfadač się $\mathrm{z}$ dokladnie 21iter (do wyboru

$\mathrm{z}26$ małych $\mathrm{i}26\mathrm{d}\mathrm{u}\dot{\mathrm{z}}$ ych liter alfabetu) oraz $\mathrm{z}$ przynajmniej 2 $\mathrm{i}$ co najwyzej 4 cyfr (od 0

do 9). Zarówno 1iteryjak $\mathrm{i}$ liczby mogą się powtarzač. Ilejest róznych hasel spelniajqcych

te warunki?

4. Rozwiązač nierównośč

$x+1\geq\sqrt{5-x}.$

5. Suma 2l pierwszych wyrazów pewnego ciqgu arytmetycznego wynosi zero a iloczyn

dwunastego $\mathrm{i}$ trzynastego wyrazu równy jest 8. D1a jakich 1iczb $n$ suma $n$ pierwszych

wyrazów tego ciagu jest mniejsza od 9?

6. Marcin stoi nad brzegiem morza $\mathrm{i}$ obserwuje $\mathrm{o}\mathrm{d}\mathrm{p}\mathrm{f}\mathrm{y}\mathrm{w}\mathrm{a}\mathrm{j}_{\Phi}\mathrm{c}\mathrm{y}$ statek.

a) Jak daleko będzie statek od (oczu) Marcina $\mathrm{w}$ momencie, $\mathrm{w}$ którym zniknie on za

horyzontem (Marcin przestanie go widzieč)?

b) Najak wysoką wiezę musi on wejśč, $\dot{\mathrm{z}}$ eby jeszcze widzieč statek bedący $\mathrm{w}$ odleglości

10 km od niego?

Przyjąč, $\dot{\mathrm{z}}\mathrm{e}$ Ziemia jest kulą $0$ promieniu 6371 km a oczy Marcina znajdują się na wyso-

kości 170 cm.
\end{document}
