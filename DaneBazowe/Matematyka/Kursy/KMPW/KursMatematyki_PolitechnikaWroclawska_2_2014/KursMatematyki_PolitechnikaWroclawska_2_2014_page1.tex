\documentclass[a4paper,12pt]{article}
\usepackage{latexsym}
\usepackage{amsmath}
\usepackage{amssymb}
\usepackage{graphicx}
\usepackage{wrapfig}
\pagestyle{plain}
\usepackage{fancybox}
\usepackage{bm}

\begin{document}

PRACA KONTROLNA nr 2- POZ1OM ROZSZERZONY

l. Ulozono dwie wieze $\mathrm{z}$ sześciennych klocków. Pierwszq $\mathrm{z}$ trzech klocków $0$ objętości 72,

8 oraz 3 $cm^{3}$, a drugą $\mathrm{z}$ czterech jednakowych klocków $0$ objętości 8 $cm^{3}$ Która $\mathrm{z}$ nich

jest $\mathrm{w}\mathrm{y}\dot{\mathrm{z}}$ sza? Odpowied $\acute{\mathrm{z}}$ uzasadnič nie $\mathrm{u}\dot{\mathrm{z}}$ ywając kalkulatora.

2. Kod do sejfu $\mathrm{w}$ willi pana Bogackiego jest pięciocyfrowy. Jego córka, korzystając $\mathrm{z}$ chwi-

lowej nieobecności taty, próbuje go otworzyč. Wie jednak tylko, $\dot{\mathrm{z}}\mathrm{e}$ kod ulozony jest

$\mathrm{z}$ dokładnie trzech róznych cyfr $\mathrm{i}$ nie występują $\mathrm{w}$ nim cyfry 1,4 $\mathrm{i}9$. Ile jest róznych

kodów spelniających te warunki?

3. Rozwiązač nierównośč

$x-1>\sqrt{4-\frac{6}{x}}.$

4. Wjednej szklance znajduje się woda, a $\mathrm{w}$ drugiej dokładnie taka sama ilośč wina. $\mathrm{Z}$ pierw-

szej szklanki przelano jedną $\text{ł} \mathrm{y}\dot{\mathrm{z}}$ kę wody do szklanki $\mathrm{z}$ winem $\mathrm{i}$ dokladnie wymieszano.

Następnie przelano jedną $l\mathrm{y}\dot{\mathrm{z}}$ kę powstałej mieszaniny $\mathrm{z}$ powrotem do pierwszej szklanki.

Sprawdzič czy po tych zabiegach jest więcej wody $\mathrm{w}$ winie czy wina $\mathrm{w}$ wodzie.

5. Trzy liczby tworzą $\mathrm{c}\mathrm{i}_{\Phi \mathrm{g}}$ geometryczny. Ich suma równa jest 13, a suma ich odwrotności

wynosi $\displaystyle \frac{13}{9}$. Znalez/č te liczby.

6. Bocian stoi na słupie $0$ wysokości 5 metrów. Magda, której oczy znajdują się na wysokości

160 cm nad $\mathrm{z}\mathrm{i}\mathrm{e}\mathrm{m}\mathrm{i}_{\Phi}$, stoi 10,2 metra od tego s1upai widzi bociana pod kątem 6 stopni. Jak

wysoki jest bocian? Podač wynik $\mathrm{z}$ dokładnością do l cm. $\mathrm{W}$ razie potrzeby odpowiednią

funkcję trygonometryczną kąta $6^{\mathrm{o}}$ przyblizyč za pomocq tablic matematycznych lub

kalkulatora.

Rozwiązania (rękopis) zadań z wybranego poziomu prosimy nadsyłač do

2015r. na adres:

19 $\mathrm{p}\mathrm{a}\acute{\mathrm{z}}$ dziernika

Wydziaf Matematyki

Politechnika Wrocfawska

Wybrzez $\mathrm{e}$ Wyspiańskiego 27

$50-370$ WROCLAW.

Na kopercie prosimy $\underline{\mathrm{k}\mathrm{o}\mathrm{n}\mathrm{i}\mathrm{e}\mathrm{c}\mathrm{z}\mathrm{n}\mathrm{i}\mathrm{e}}$ zaznaczyč wybrany poziom! (np. poziom podsta-

wowy lub rozszerzony). Do rozwiązań nalez $\mathrm{y}$ dolączyč zaadresowaną do siebie koperte

zwrotną $\mathrm{z}$ naklejonym znaczkiem, odpowiednim do wagi listu. Prace niespelniające po-

danych warunków nie będą poprawiane ani odsyłane.

Adres internetowy Kursu: http://www.im.pwr.wroc.pl/kurs
\end{document}
