\documentclass[a4paper,12pt]{article}
\usepackage{latexsym}
\usepackage{amsmath}
\usepackage{amssymb}
\usepackage{graphicx}
\usepackage{wrapfig}
\pagestyle{plain}
\usepackage{fancybox}
\usepackage{bm}

\begin{document}

XLVII

KORESPONDENCYJNY KURS

Z MATEMATYKI

$\mathrm{p}\mathrm{a}\acute{\mathrm{z}}$dziernik 2017 $\mathrm{r}.$

PRACA KONTROLNA $\mathrm{n}\mathrm{r} 2-$ POZIOM PODSTAWOWY

l. Rozwiązač nierównośč

$2x-2>\sqrt{7-4x}.$

2. Dla jakich wartości parametru $m$ pierwiastkiem wielomianu

$w(x)=2x^{3}-x^{2}-(m^{2}-2)x+m-1$

jest $x=2$? Dla znalezionych wartości $m$ wyznaczyč pozostafe pierwiastki $w(x).$

3. Narysowač staranny wykres funkcji $f(x)=|\displaystyle \sin x|\cos x-\frac{1}{4}\mathrm{i}$ rozwiązač nierównośč

$f(x)\displaystyle \leq-\frac{1}{2}.$

4. Rozwiązač równanie

$4^{x+\sqrt{x^{2}-2}}-5\cdot 2^{x-1+\sqrt{x^{2}-2}}=6.$

5. $\mathrm{W}$ trójkącie równoramiennym $ABC0$ podstawie $AB$ dane $\mathrm{s}\Phi A(2,-1)$ oraz $B(-1,3).$

Środkowe poprowadzone $\mathrm{z}A\mathrm{i}\mathrm{z}B$ są prostopadłe. $\mathrm{Z}\mathrm{n}\mathrm{a}\mathrm{l}\mathrm{e}\mathrm{z}^{J}\text{č}$ współrzędne punktu $C$ oraz

obliczyč pole $\mathrm{i}$ obwód tego trójkqta.

6. $\mathrm{W}$ okrąg $0$ promieniu $R$ wpisano trzy jednakowe okręgi wzajemnie styczne $\mathrm{w}$ punktach

$A, B, C\mathrm{i}$ styczne do danego okręgu. Obliczyč pole obszaru ograniczonego mniejszymi

fukami AB, $BC\mathrm{i}CA.$
\end{document}
