\documentclass[a4paper,12pt]{article}
\usepackage{latexsym}
\usepackage{amsmath}
\usepackage{amssymb}
\usepackage{graphicx}
\usepackage{wrapfig}
\pagestyle{plain}
\usepackage{fancybox}
\usepackage{bm}

\begin{document}

PRACA KONTROLNA nr 2- POZ1OM ROZSZERZONY

l. Rozwiązač nierównośč

$\sqrt{2x^{2}-x}<5-4x.$

2. Rozwiazač układ równań

$\left\{\begin{array}{l}
xy\\
x^{\log y}
\end{array}\right.$

400,

16.

3. Narysowač staranny wykres funkcji $f(x)=|\sin x|-\cos x$, wyznaczyč jej zbior wartości

oraz rozwiązač nierównośč

$\displaystyle \frac{1}{f(x)}\geq 1.$

4. Reszta $\mathrm{z}$ dzielenia wielomianu $w(x)=x^{4}+ax^{3}-bx^{2}+bx$ przez trójmian $x^{2}-9$ wynosi

$-5x+45$. Wyznaczyč wartości parametrów $a\mathrm{i}b$ oraz rozwiązač nierównośč

$w(x-1)\geq w(x+1).$

5. Dany jest punkt $A(2,1)$. Wyznaczyč $\mathrm{i}$ narysowač zbiór tych wszystkich punktów $C$, dla

których czworokąt ABCD jest prostokqtem takim, $\dot{\mathrm{z}}\mathrm{e}$ punkty $B\mathrm{i}D\mathrm{l}\mathrm{e}\dot{\mathrm{z}}$ ą na osiach układu

wspófrzędnych $\mathrm{i}$ nie $\mathrm{n}\mathrm{a}\mathrm{l}\mathrm{e}\mathrm{z}\Phi$ do tego samego boku $\mathrm{p}\mathrm{r}\mathrm{o}\mathrm{s}\mathrm{t}\mathrm{o}\mathrm{k}_{\Phi^{\mathrm{t}}}\mathrm{a}$. Wykonač rysunek.

6. Nad sześcianem $0$ krawędzi $a$ stojącym na pfaszczy $\acute{\mathrm{z}}\mathrm{n}\mathrm{i}\mathrm{e}$ umieszczono punktowe z/ródfo

światła na wysokości $b>a$ (rzut prostopadły punktu, $\mathrm{w}$ którym jest z/ródło światła na

tę pfaszczyznę, zawiera się $\mathrm{w}$ podstawie sześcianu). Obliczyč pole obszaru jaki zajmuje

cień sześcianu lącznie $\mathrm{z}$ jego podstawą na tej płaszczyz/nie.

Rozwiązania (rękopis) zadań z wybranego poziomu prosimy nadsyfač do

2017r. na adres:

18 $\mathrm{p}\mathrm{a}\acute{\mathrm{z}}$dziernika

Wydziaf Matematyki

Politechnika Wrocfawska

Wybrzez $\mathrm{e}$ Wyspiańskiego 27

$50-370$ WROCLAW.

Na kopercie prosimy $\underline{\mathrm{k}\mathrm{o}\mathrm{n}\mathrm{i}\mathrm{e}\mathrm{c}\mathrm{z}\mathrm{n}\mathrm{i}\mathrm{e}}$ zaznaczyč wybrany poziom! (np. poziom podsta-

wowy lub rozszerzony). Do rozwiązań nalez $\mathrm{y}$ dołączyč zaadresowaną do siebie koperte

zwrotną $\mathrm{z}$ naklejonym znaczkiem, odpowiednim do wagi listu. Prace niespełniające po-

danych warunków nie będą poprawiane ani odsyłane.

Adres internetowy Kursu: http://www.im.pwr.wroc.pl/kurs
\end{document}
