\documentclass[a4paper,12pt]{article}
\usepackage{latexsym}
\usepackage{amsmath}
\usepackage{amssymb}
\usepackage{graphicx}
\usepackage{wrapfig}
\pagestyle{plain}
\usepackage{fancybox}
\usepackage{bm}

\begin{document}

XLVIII

KORESPONDENCYJNY KURS

Z MATEMATYKI

$\mathrm{p}\mathrm{a}\acute{\mathrm{z}}$dziernik 2018 $\mathrm{r}.$

PRACA KONTROLNA $\mathrm{n}\mathrm{r} 2-$ POZIOM PODSTAWOWY

l. Rozwiązač nierównośč $x-1>\sqrt{x^{2}-3}.$

2. Rozwiązač równanie $\displaystyle \frac{1}{\sin 2x}+\frac{1}{\sin x}=0.$

3. Narysowač zbiór $\{(x,y):-1\leq\log_{\frac{1}{2}}|x|+\log_{2}|y|\leq 1,|x|+|y|\leq 3\}\mathrm{i}$ obliczyč jego pole.

4. Na prostej $l$ : $2x-y-1=0$ wyznaczyč punkty, $\mathrm{z}$ których odcinek $0$ końcach $A(0,1)$ oraz

$B(0,3)$ jest widoczny pod kątem $45^{\mathrm{o}}$

5. $\mathrm{W}$ obszar ograniczony wykresem funkcji kwadratowej $\mathrm{i}$ osią $Ox$ wpisano prostokąt $0$ polu

24, którego jeden $\mathrm{z}$ boków zawarty jest $\mathrm{w}$ osi $Ox$, a dwa wierzchofki lezą na paraboli.

Odległośč między miejscami zerowymi funkcji wynosi 10. Wyznaczyč wzór tej funkcji,

wiedzac, $\dot{\mathrm{z}}\mathrm{e}$ wierzchofek paraboli $\mathrm{l}\mathrm{e}\dot{\mathrm{z}}\mathrm{y}$ na osi $Oy$ ijeden $\mathrm{z}$ boków prostokąta ma dfugośč 6.

Rozwiązanie zilustrowač odpowiednim rysunkiem.

6. $\mathrm{W}$ ostrosłupie, którego podstawą jest romb $0$ boku $\alpha$, jedna $\mathrm{z}$ krawędzi bocznych równiez

ma dfugośč $a\mathrm{i}$ jest prostopadfa do podstawy. Wszystkie pozostałe krawędzie boczne są

równe. Obliczyč objętośč, pole powierzchni całkowitej ostrosłupa oraz sinus kąta nachy-

lenia do podstawy jego pochyłych ścian bocznych.




PRACA KONTROLNA nr $2$ - PozioM R0ZSZERZ0NY

l. Wyznaczyč dziedzinę funkcji $f(x)=\log_{2}(\sqrt{x-1-\sqrt{x^{2}-3x-4}}-1).$

2. Rozwiązač równanie $\sin^{4}x+\cos^{4}x=\sin x\cos x.$

3. Narysowač zbiór $\{(x,y):|x|+|y|\leq 6,|y|\leq 2^{|x|},|y|\geq\log_{2}|x|\}\mathrm{i}$ napisač równaniajego

osi symetrii. Podač odpowiednie uzasadnienie.

4. Niech $f(x) = \displaystyle \frac{2x-1}{x-2}, g(x) = (\sqrt{2})^{\log_{2}(2x-1)^{2}+4\log_{\frac{1}{2}}\sqrt{2-x}}$ Narysowač wykres funkcji

$h(x) = \displaystyle \max\{f(x),g(x)\}$. Czy $\mathrm{m}\mathrm{o}\dot{\mathrm{z}}$ na podač wzór funkcji $h(x)$, wykorzystujac jedynie

$f(x)$ ?

5. Punkt $A(1,1)$ jest wierzchołkiem rombu $0$ polu 10. Przekątna $AC$ rombu jest równo-

legła do wektora $\vec{v}=[1$, 2$]$. Wyznaczyč współrzędne pozostałych wierzchołków rombu,

$\mathrm{w}\mathrm{i}\mathrm{e}\mathrm{d}_{\mathrm{Z}\otimes}\mathrm{c}, \dot{\mathrm{z}}\mathrm{e}$ jeden $\mathrm{z}$ nich $\mathrm{l}\mathrm{e}\dot{\mathrm{z}}\mathrm{y}$ na prostej $y=x-2.$

6. $\mathrm{W}$ ostrosłupie, którego podstawq jest romb $0$ boku $\alpha$, jedna $\mathrm{z}$ krawędzi bocznych równiez

ma dlugośč $a\mathrm{i}$ jest prostopadła do podstawy. Wszystkie pozostafe krawędzie boczne

są równe. Wyznaczyč cosinusy kątów płaskich przy wierzchofku $\mathrm{k}\mathrm{a}\dot{\mathrm{z}}$ dej ściany bocznej

ostrosłupa oraz cosinusy katów między jego ścianami bocznymi

Rozwiązania (rękopis) zadań z wybranego poziomu prosimy nadsyłač do

2018r. na adres:

18 $\mathrm{p}\mathrm{a}\acute{\mathrm{z}}$ dziernika

Wydziaf Matematyki

Politechnika Wrocfawska

Wybrzez $\mathrm{e}$ Wyspiańskiego 27

$50-370$ WROCLAW.

Na kopercie prosimy $\underline{\mathrm{k}\mathrm{o}\mathrm{n}\mathrm{i}\mathrm{e}\mathrm{c}\mathrm{z}\mathrm{n}\mathrm{i}\mathrm{e}}$ zaznaczyč wybrany poziom! (np. poziom podsta-

wowy lub rozszerzony). Do rozwiązań nalez $\mathrm{y}$ dołączyč zaadresowaną do siebie kopertę

zwrotną $\mathrm{z}$ naklejonym znaczkiem, odpowiednim do wagi listu. Prace niespelniające po-

danych warunków nie będą poprawiane ani odsylane.

Uwaga. Wysyłając nam rozwiazania zadań uczestnik Kursu udostępnia Politechnice Wrocławskiej

swoje dane osobowe, które przetwarzamy wyłącznie $\mathrm{w}$ zakresie niezbędnym do jego prowadzenia

(odesłanie zadań, prowadzenie statystyki). Szczególowe informacje $0$ przetwarzaniu przez nas danych

osobowych są dostępne na stronie internetowej Kursu.

Adres internetowy Kursu: http: //www. im. pwr. edu. pl/kurs



\end{document}