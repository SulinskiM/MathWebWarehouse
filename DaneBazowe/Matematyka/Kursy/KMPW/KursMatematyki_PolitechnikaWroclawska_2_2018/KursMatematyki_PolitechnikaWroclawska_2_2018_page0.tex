\documentclass[a4paper,12pt]{article}
\usepackage{latexsym}
\usepackage{amsmath}
\usepackage{amssymb}
\usepackage{graphicx}
\usepackage{wrapfig}
\pagestyle{plain}
\usepackage{fancybox}
\usepackage{bm}

\begin{document}

XLVIII

KORESPONDENCYJNY KURS

Z MATEMATYKI

$\mathrm{p}\mathrm{a}\acute{\mathrm{z}}$dziernik 2018 $\mathrm{r}.$

PRACA KONTROLNA $\mathrm{n}\mathrm{r} 2-$ POZIOM PODSTAWOWY

l. Rozwiązač nierównośč $x-1>\sqrt{x^{2}-3}.$

2. Rozwiązač równanie $\displaystyle \frac{1}{\sin 2x}+\frac{1}{\sin x}=0.$

3. Narysowač zbiór $\{(x,y):-1\leq\log_{\frac{1}{2}}|x|+\log_{2}|y|\leq 1,|x|+|y|\leq 3\}\mathrm{i}$ obliczyč jego pole.

4. Na prostej $l$ : $2x-y-1=0$ wyznaczyč punkty, $\mathrm{z}$ których odcinek $0$ końcach $A(0,1)$ oraz

$B(0,3)$ jest widoczny pod kątem $45^{\mathrm{o}}$

5. $\mathrm{W}$ obszar ograniczony wykresem funkcji kwadratowej $\mathrm{i}$ osią $Ox$ wpisano prostokąt $0$ polu

24, którego jeden $\mathrm{z}$ boków zawarty jest $\mathrm{w}$ osi $Ox$, a dwa wierzchofki lezą na paraboli.

Odległośč między miejscami zerowymi funkcji wynosi 10. Wyznaczyč wzór tej funkcji,

wiedzac, $\dot{\mathrm{z}}\mathrm{e}$ wierzchofek paraboli $\mathrm{l}\mathrm{e}\dot{\mathrm{z}}\mathrm{y}$ na osi $Oy$ ijeden $\mathrm{z}$ boków prostokąta ma dfugośč 6.

Rozwiązanie zilustrowač odpowiednim rysunkiem.

6. $\mathrm{W}$ ostrosłupie, którego podstawą jest romb $0$ boku $\alpha$, jedna $\mathrm{z}$ krawędzi bocznych równiez

ma dfugośč $a\mathrm{i}$ jest prostopadfa do podstawy. Wszystkie pozostałe krawędzie boczne są

równe. Obliczyč objętośč, pole powierzchni całkowitej ostrosłupa oraz sinus kąta nachy-

lenia do podstawy jego pochyłych ścian bocznych.
\end{document}
