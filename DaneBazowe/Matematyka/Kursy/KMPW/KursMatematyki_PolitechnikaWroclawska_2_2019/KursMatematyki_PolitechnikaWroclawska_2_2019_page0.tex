\documentclass[a4paper,12pt]{article}
\usepackage{latexsym}
\usepackage{amsmath}
\usepackage{amssymb}
\usepackage{graphicx}
\usepackage{wrapfig}
\pagestyle{plain}
\usepackage{fancybox}
\usepackage{bm}

\begin{document}

XLIX

KORESPONDENCYJNY KURS

Z MATEMATYKI

$\mathrm{p}\mathrm{a}\acute{\mathrm{z}}$dziernik 2019 $\mathrm{r}.$

PRACA KONTROLNA $\mathrm{n}\mathrm{r} 2-$ POZIOM PODSTAWOWY

l. Niech $\alpha$ będzie kątem ostrym takim, $\dot{\mathrm{z}}\mathrm{e}\sin\alpha=\sqrt{15}\cos\alpha$. Wyznaczyč wszystkie wartości

funkcji trygonometrycznych kątów $\alpha$ oraz $2\alpha.$

2. Rozwiązač nierównośč

$x\geq 2+\sqrt{10-3x}.$

3. Wykres trójmianu kwadratowego $f(x)=ax^{2}+bx+c$ jest symetryczny wzgledem prostej

$x=3$, a $\mathrm{r}\mathrm{e}\mathrm{s}\mathrm{z}\mathrm{t}_{\Phi}\mathrm{z}$ jego dzielenia przez wielomian $x-2$ jest -$1$. Wiadomo tez$\cdot, \dot{\mathrm{z}}\mathrm{e}f(0)=3.$

Znalez/č wartości wspólczynników $a, b, c\mathrm{i}$ rozwiązač nierównośč

$\displaystyle \frac{1}{f(x)}\geq\frac{1}{3}.$

4. $\mathrm{W}$ ciqgu arytmetycznym, $\mathrm{w}$ którym trzeci wyraz jest odwrotnością pierwszego, suma

pierwszych ośmiu wyrazów wynosi 25. Ob1iczyč sumę pierwszych 10 wyrazów $0$ numerach

nieparzystych.

5. Pole trapezu równoramiennego, opisanego na okregu $0$ promieniu l, wynosi 5. Ob1iczyč

pole czworokąta, którego wierzchofkami są punkty styczności okręgu $\mathrm{i}$ trapezu.

6. Na szczycie góry, na którą wchodzi Agata po stoku $0$ kacie nachylenia $\beta$, stoi krowa

$0$ wysokości 150 cm. Dziewczynka widzi ją pod kątem $\alpha$, przy czym przyjmujemy tutaj

dla uproszczenia, $\dot{\mathrm{z}}\mathrm{e}$ punkt obserwacji znajduje się na poziomie drogi. Najakiej wysokości

nad poziomem morza stoi Agata, $\mathrm{j}\mathrm{e}\dot{\mathrm{z}}$ eli szczyt jest na wysokości 1520 $\mathrm{m}$ n.p.m.? Podač

wzór $\mathrm{i}$ następnie wykonač obliczenia dla $\beta=43^{\mathrm{o}}, \alpha=2^{\mathrm{o}}$
\end{document}
