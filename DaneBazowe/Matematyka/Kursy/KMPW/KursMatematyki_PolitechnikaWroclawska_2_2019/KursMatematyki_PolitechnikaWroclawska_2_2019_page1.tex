\documentclass[a4paper,12pt]{article}
\usepackage{latexsym}
\usepackage{amsmath}
\usepackage{amssymb}
\usepackage{graphicx}
\usepackage{wrapfig}
\pagestyle{plain}
\usepackage{fancybox}
\usepackage{bm}

\begin{document}

PRACA KONTROLNA nr 2- POZ1OM ROZSZERZONY

l. W nieskończonym ciągu geometrycznym, którego suma równa jest 4, trzeci wyraz jest

odwrotnością pierwszego. Wyznaczyč pierwszy wyraz i sumę wszystkich wyrazów 0 nu-

merach parzystych.

2. Narysowač wykres funkcji

$f(x)=\displaystyle \frac{\sin x}{\sqrt{1+\mathrm{t}\mathrm{g}^{2}x}}$

$\mathrm{i}$ rozwiązač nierównośč $f(x)\displaystyle \geq\frac{1}{4}.$

3. Rozwiązač nierównośč

$\displaystyle \sqrt{\frac{4x-2}{x+4}}\leq\frac{2}{x-1}-1.$

4. Reszta $\mathrm{z}$ dzielenia wielomianu $w(x)=ax^{5}+bx^{2}+c$ przez $p(x)=x^{3}-x^{2}-2x$ jest wielo-

mian $r(x)=11x^{2}+12x+1$. Wyznaczyč wartości współczynników $a, b, c$ oraz rozwiązač

nierównośč $w(x)\geq(x+1)^{2}$

5. Wyznaczyč pole trójkąta równobocznego, którego wierzcholki lezą na trzech róznych

prostych równolegfych, $\mathrm{z}$ których środkowa jest oddalona od skrajnych $0$ {\it a} $\mathrm{i}b.$

6. $\mathrm{W}$ punktach $A(0,0), B(4,0) \mathrm{i}C(0,4)$ ustawione są trzy znaczniki. Sensory robota po-

zwalają ustalič, $\dot{\mathrm{z}}\mathrm{e}\mathrm{z}$ miejsca, $\mathrm{w}$ którym znajduje się on obecnie odcinek $AB$ widač pod

kątem $\alpha=90^{\mathrm{o}}$, a odcinek $AC$ pod kątem $\beta=60^{\mathrm{o}}$ Ustalič pofozenie robota $\mathrm{w}$ ukfadzie

wspólrzędnych.

Rozwiązania (rękopis) zadań z wybranego poziomu prosimy nadsyłač do

2019r. na adres:

18 $\mathrm{p}\mathrm{a}\acute{\mathrm{z}}$ dziernika

Wydziaf Matematyki

Politechnika Wrocfawska

Wybrzez $\mathrm{e}$ Wyspiańskiego 27

$50-370$ WROCLAW.

Na kopercie prosimy $\underline{\mathrm{k}\mathrm{o}\mathrm{n}\mathrm{i}\mathrm{e}\mathrm{c}\mathrm{z}\mathrm{n}\mathrm{i}\mathrm{e}}$ zaznaczyč wybrany poziom! (np. poziom podsta-

wowy lub rozszerzony). Do rozwiązań nalez $\mathrm{y}$ dołączyč zaadresowana do siebie koperte

zwrotną $\mathrm{z}$ naklejonym znaczkiem, odpowiednim do wagi listu. Prace niespelniające po-

danych warunków nie będą poprawiane ani odsylane.

Uwaga. Wysyłając nam rozwi\S zania zadań uczestnik Kursu udostępnia Politechnice Wrocławskiej

swoje dane osobowe, które przetwarzamy wyłącznie $\mathrm{w}$ zakresie niezbędnym do jego prowadzenia

(odesłanie zadań, prowadzenie statystyki). Szczególowe informacje $0$ przetwarzaniu przez nas danych

osobowych są dostępne na stronie internetowej Kursu.

Adres internetowy Kursu: http: //www. im. pwr. edu. pl/kurs
\end{document}
