\documentclass[a4paper,12pt]{article}
\usepackage{latexsym}
\usepackage{amsmath}
\usepackage{amssymb}
\usepackage{graphicx}
\usepackage{wrapfig}
\pagestyle{plain}
\usepackage{fancybox}
\usepackage{bm}

\begin{document}

L

KORESPONDENCYJNY KURS

Z MATEMATYKI

$\mathrm{p}\mathrm{a}\acute{\mathrm{z}}$dziernik 2020 $\mathrm{r}.$

PRACA KONTROLNA $\mathrm{n}\mathrm{r} 2-$ POZIOM PODSTAWOWY

l. Niemieckie przepisy drogowe wymagaja zachowania bezpiecznego odstępu między po-

ruszającymi się $\mathrm{w}$ tym samym kierunku pojazdami. Zalecane jest przy tym zachowanie

zasady,,połowa licznika $\mathrm{j}\mathrm{e}\dot{\mathrm{z}}$ eli dwa pojazdyjadą $\mathrm{z}$ prędkością $x\mathrm{k}\mathrm{m}/\mathrm{h}$, to odstęp między

nimi powinien wynosič przynajmniej $x/2$ metrów. Jaki odstep czasowy powinien zatem

dzielič te dwa pojazdy? Przyjmując, $\dot{\mathrm{z}}\mathrm{e}$ dla samochodujadącego $\mathrm{z}$ prędkości$\Phi v\mathrm{m}/\mathrm{s}$ droga

hamowania wynosi $s_{h}=\displaystyle \frac{v^{2}}{2a}$ metrów (gdzie $a$ jest stałym współczynnikiem hamowania),

sprawd $\acute{\mathrm{z}}$ przy jakiej prędkości $x\mathrm{k}\mathrm{m}/\mathrm{h}$ dojdzie do wypadku, $\mathrm{j}\mathrm{e}\dot{\mathrm{z}}$ eli oba pojazdy jechały

$\mathrm{z}$ minimalnym zalecanym odstępem, pierwszy zatrzymał się nagle (przyjmij $a=10$), $\mathrm{a}$

drugi zaczał hamowač jednq sekundę póz/niej $\mathrm{i}\mathrm{z}$ sila taka, $\dot{\mathrm{z}}\mathrm{e}a=7.$

2. $\displaystyle \frac{\mathrm{J}\mathrm{a}\sqrt{6}}{2}?\mathrm{k}\mathrm{i}\mathrm{m}\mathrm{k}_{\Phi}$tami mogą byč $\alpha \mathrm{i}2\alpha, \mathrm{j}\mathrm{e}\dot{\mathrm{z}}$ eli wiadomo, $\dot{\mathrm{z}}\mathrm{e}\alpha$ jest $\mathrm{k}_{\Phi}\mathrm{t}\mathrm{e}\mathrm{m}$ ostrym oraz $\sin\alpha+\cos\alpha=$

$3$. Rozwazmy funkcję $f(x)=x^{2}-(a+2)x+3(a-1)$. Dla jakich wartości paramertu $a$:

(i) cafy wykres $f(x)\mathrm{l}\mathrm{e}\dot{\mathrm{z}}\mathrm{y}$ ponad prostą $y=-1$?

(ii) oba miejsca zerowe funkcji $f(x)$ sq wieksze od 2?

4. Rozwiąz nierównośč

$x\leq 1+\sqrt{2+x}.$

5. Narysuj starannie zbiór $A\cap B$, gdzie

$A=\{(x,y):2|x|+|y|\leq 2\},$

$B=\{(x,y):y^{2}-y<2\}$

$\mathrm{i}$ oblicz jego pole.

6. Jednym $\mathrm{z}$ wierzchofków kwadratu jest $A(1,-3)$, a jedna $\mathrm{z}$ jego przekątnych zawiera się

$\mathrm{w}$ prostej $y = -2x+2$. Wyznaczyč współrzędne pozostalych wierzchołków kwadratu

$\mathrm{i}$ równanie okręgu wpisanego $\mathrm{w}$ ten kwadrat.




PRACA KONTROLNA nr 2- POZ1OM ROZSZERZONY

l. Wyznacz kąty $\alpha \mathrm{i}2\alpha$ wiedzac, $\mathrm{i}\dot{\mathrm{z}}\alpha$ jest kątem rozwartym takim, $\dot{\mathrm{z}}\mathrm{e}$ tg $\alpha+\mathrm{c}\mathrm{t}\mathrm{g}\alpha=-2\sqrt{2}.$

2. Rozwiąz równanie

$x=\sqrt{5+\sqrt{3+x^{2}}}.$

Nie $\mathrm{u}\dot{\mathrm{z}}$ ywając kalkulatora zbadaj, czy jego rozwiązanie jest liczbą większą $\mathrm{n}\mathrm{i}\dot{\mathrm{z}}3.$

3. Udowodnij, $\dot{\mathrm{z}}\mathrm{e}\mathrm{j}\mathrm{e}\dot{\mathrm{z}}$ eli dwa trójkąty prostokątne mają równe obwody $\mathrm{i}$ dlugości przeciw-

$\mathrm{P}^{\mathrm{r}\mathrm{o}\mathrm{s}\mathrm{t}\mathrm{o}\mathrm{k}}\Phi^{\mathrm{t}\mathrm{n}\mathrm{y}\mathrm{c}\mathrm{h}}$, to $\mathrm{s}\Phi$ przystające.

4. Narysuj starannie zbiór $A\cap B$, gdzie

$A=\{(x,y):x^{2}-8|x|+y^{2}-8|y|+16\geq 0,|x|\leq 4,|y|\leq 4\},$

$B=\{(x,y):x^{2}+y^{2}>16(3-2\sqrt{2})\}$

$\mathrm{i}$ oblicz jego pole.

5. Dla jakich wartości parametrów $p\mathrm{i}q$ do zbioru rozwiązań równania

$x^{3}-3px^{2}+(q+4)x=0,$

nalezą zarówno $p$ jak $\mathrm{i}q$?

6. Napisz równanie prostej $k$ stycznej do okregu $x^{2}-4x+y^{2}+2y=0\mathrm{w}$ punkcie $P(3,1).$

Następnie wyznacz równania wszystkich prostych stycznych do tego okręgu, które tworzą

$\mathrm{z}$ prostą $k$ kąt $45^{\mathrm{o}}$

Rozwiązania (rękopis) zadań z wybranego poziomu prosimy nadsylač do

2020r. na adres:

20 $\mathrm{p}\mathrm{a}\acute{\mathrm{z}}$dziernika

Wydziaf Matematyki

Politechnika Wrocfawska

Wybrzez $\mathrm{e}$ Wyspiańskiego 27

$50-370$ WROCLAW.

Na kopercie prosimy $\underline{\mathrm{k}\mathrm{o}\mathrm{n}\mathrm{i}\mathrm{e}\mathrm{c}\mathrm{z}\mathrm{n}\mathrm{i}\mathrm{e}}$ zaznaczyč wybrany poziom! (np. poziom podsta-

wowy lub rozszerzony). Do rozwiązań nalez $\mathrm{y}$ dołączyč zaadresowaną do siebie kopertę

zwrotną $\mathrm{z}$ naklejonym znaczkiem, odpowiednim do formatu listu. Polecamy stosowanie

kopert formatu C5 $(160\mathrm{x}230\mathrm{m}\mathrm{m})$ ze znaczkiem $0$ wartości 3,30 zł. Na $\mathrm{k}\mathrm{a}\dot{\mathrm{z}}$ dą większą

koperte nalez $\mathrm{y}$ nakleič drozszy znaczek. Prace niespełniające podanych warunków nie

będą poprawiane ani odsyłane.

Uwaga. Wysylając nam rozwi\S zania zadań uczestnik Kursu udostępnia Politechnice Wroclawskiej

swoje dane osobowe, które przetwarzamy wyłącznie $\mathrm{w}$ zakresie niezbędnym do jego prowadzenia

(odesfanie zadań, prowadzenie statystyki). Szczegófowe informacje $0$ przetwarzaniu przez nas danych

osobowych są dostępne na stronie internetowej Kursu.

Adres internetowy Kursu: http://www.im.pwr.edu.pl/kurs



\end{document}