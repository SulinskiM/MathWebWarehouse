\documentclass[a4paper,12pt]{article}
\usepackage{latexsym}
\usepackage{amsmath}
\usepackage{amssymb}
\usepackage{graphicx}
\usepackage{wrapfig}
\pagestyle{plain}
\usepackage{fancybox}
\usepackage{bm}

\begin{document}

L

KORESPONDENCYJNY KURS

Z MATEMATYKI

$\mathrm{p}\mathrm{a}\acute{\mathrm{z}}$dziernik 2020 $\mathrm{r}.$

PRACA KONTROLNA $\mathrm{n}\mathrm{r} 2-$ POZIOM PODSTAWOWY

l. Niemieckie przepisy drogowe wymagaja zachowania bezpiecznego odstępu między po-

ruszającymi się $\mathrm{w}$ tym samym kierunku pojazdami. Zalecane jest przy tym zachowanie

zasady,,połowa licznika $\mathrm{j}\mathrm{e}\dot{\mathrm{z}}$ eli dwa pojazdyjadą $\mathrm{z}$ prędkością $x\mathrm{k}\mathrm{m}/\mathrm{h}$, to odstęp między

nimi powinien wynosič przynajmniej $x/2$ metrów. Jaki odstep czasowy powinien zatem

dzielič te dwa pojazdy? Przyjmując, $\dot{\mathrm{z}}\mathrm{e}$ dla samochodujadącego $\mathrm{z}$ prędkości$\Phi v\mathrm{m}/\mathrm{s}$ droga

hamowania wynosi $s_{h}=\displaystyle \frac{v^{2}}{2a}$ metrów (gdzie $a$ jest stałym współczynnikiem hamowania),

sprawd $\acute{\mathrm{z}}$ przy jakiej prędkości $x\mathrm{k}\mathrm{m}/\mathrm{h}$ dojdzie do wypadku, $\mathrm{j}\mathrm{e}\dot{\mathrm{z}}$ eli oba pojazdy jechały

$\mathrm{z}$ minimalnym zalecanym odstępem, pierwszy zatrzymał się nagle (przyjmij $a=10$), $\mathrm{a}$

drugi zaczał hamowač jednq sekundę póz/niej $\mathrm{i}\mathrm{z}$ sila taka, $\dot{\mathrm{z}}\mathrm{e}a=7.$

2. $\displaystyle \frac{\mathrm{J}\mathrm{a}\sqrt{6}}{2}?\mathrm{k}\mathrm{i}\mathrm{m}\mathrm{k}_{\Phi}$tami mogą byč $\alpha \mathrm{i}2\alpha, \mathrm{j}\mathrm{e}\dot{\mathrm{z}}$ eli wiadomo, $\dot{\mathrm{z}}\mathrm{e}\alpha$ jest $\mathrm{k}_{\Phi}\mathrm{t}\mathrm{e}\mathrm{m}$ ostrym oraz $\sin\alpha+\cos\alpha=$

$3$. Rozwazmy funkcję $f(x)=x^{2}-(a+2)x+3(a-1)$. Dla jakich wartości paramertu $a$:

(i) cafy wykres $f(x)\mathrm{l}\mathrm{e}\dot{\mathrm{z}}\mathrm{y}$ ponad prostą $y=-1$?

(ii) oba miejsca zerowe funkcji $f(x)$ sq wieksze od 2?

4. Rozwiąz nierównośč

$x\leq 1+\sqrt{2+x}.$

5. Narysuj starannie zbiór $A\cap B$, gdzie

$A=\{(x,y):2|x|+|y|\leq 2\},$

$B=\{(x,y):y^{2}-y<2\}$

$\mathrm{i}$ oblicz jego pole.

6. Jednym $\mathrm{z}$ wierzchofków kwadratu jest $A(1,-3)$, a jedna $\mathrm{z}$ jego przekątnych zawiera się

$\mathrm{w}$ prostej $y = -2x+2$. Wyznaczyč współrzędne pozostalych wierzchołków kwadratu

$\mathrm{i}$ równanie okręgu wpisanego $\mathrm{w}$ ten kwadrat.
\end{document}
