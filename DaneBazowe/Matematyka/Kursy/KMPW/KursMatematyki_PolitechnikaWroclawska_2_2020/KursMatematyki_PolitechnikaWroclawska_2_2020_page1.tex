\documentclass[a4paper,12pt]{article}
\usepackage{latexsym}
\usepackage{amsmath}
\usepackage{amssymb}
\usepackage{graphicx}
\usepackage{wrapfig}
\pagestyle{plain}
\usepackage{fancybox}
\usepackage{bm}

\begin{document}

PRACA KONTROLNA nr 2- POZ1OM ROZSZERZONY

l. Wyznacz kąty $\alpha \mathrm{i}2\alpha$ wiedzac, $\mathrm{i}\dot{\mathrm{z}}\alpha$ jest kątem rozwartym takim, $\dot{\mathrm{z}}\mathrm{e}$ tg $\alpha+\mathrm{c}\mathrm{t}\mathrm{g}\alpha=-2\sqrt{2}.$

2. Rozwiąz równanie

$x=\sqrt{5+\sqrt{3+x^{2}}}.$

Nie $\mathrm{u}\dot{\mathrm{z}}$ ywając kalkulatora zbadaj, czy jego rozwiązanie jest liczbą większą $\mathrm{n}\mathrm{i}\dot{\mathrm{z}}3.$

3. Udowodnij, $\dot{\mathrm{z}}\mathrm{e}\mathrm{j}\mathrm{e}\dot{\mathrm{z}}$ eli dwa trójkąty prostokątne mają równe obwody $\mathrm{i}$ dlugości przeciw-

$\mathrm{P}^{\mathrm{r}\mathrm{o}\mathrm{s}\mathrm{t}\mathrm{o}\mathrm{k}}\Phi^{\mathrm{t}\mathrm{n}\mathrm{y}\mathrm{c}\mathrm{h}}$, to $\mathrm{s}\Phi$ przystające.

4. Narysuj starannie zbiór $A\cap B$, gdzie

$A=\{(x,y):x^{2}-8|x|+y^{2}-8|y|+16\geq 0,|x|\leq 4,|y|\leq 4\},$

$B=\{(x,y):x^{2}+y^{2}>16(3-2\sqrt{2})\}$

$\mathrm{i}$ oblicz jego pole.

5. Dla jakich wartości parametrów $p\mathrm{i}q$ do zbioru rozwiązań równania

$x^{3}-3px^{2}+(q+4)x=0,$

nalezą zarówno $p$ jak $\mathrm{i}q$?

6. Napisz równanie prostej $k$ stycznej do okregu $x^{2}-4x+y^{2}+2y=0\mathrm{w}$ punkcie $P(3,1).$

Następnie wyznacz równania wszystkich prostych stycznych do tego okręgu, które tworzą

$\mathrm{z}$ prostą $k$ kąt $45^{\mathrm{o}}$

Rozwiązania (rękopis) zadań z wybranego poziomu prosimy nadsylač do

2020r. na adres:

20 $\mathrm{p}\mathrm{a}\acute{\mathrm{z}}$dziernika

Wydziaf Matematyki

Politechnika Wrocfawska

Wybrzez $\mathrm{e}$ Wyspiańskiego 27

$50-370$ WROCLAW.

Na kopercie prosimy $\underline{\mathrm{k}\mathrm{o}\mathrm{n}\mathrm{i}\mathrm{e}\mathrm{c}\mathrm{z}\mathrm{n}\mathrm{i}\mathrm{e}}$ zaznaczyč wybrany poziom! (np. poziom podsta-

wowy lub rozszerzony). Do rozwiązań nalez $\mathrm{y}$ dołączyč zaadresowaną do siebie kopertę

zwrotną $\mathrm{z}$ naklejonym znaczkiem, odpowiednim do formatu listu. Polecamy stosowanie

kopert formatu C5 $(160\mathrm{x}230\mathrm{m}\mathrm{m})$ ze znaczkiem $0$ wartości 3,30 zł. Na $\mathrm{k}\mathrm{a}\dot{\mathrm{z}}$ dą większą

koperte nalez $\mathrm{y}$ nakleič drozszy znaczek. Prace niespełniające podanych warunków nie

będą poprawiane ani odsyłane.

Uwaga. Wysylając nam rozwi\S zania zadań uczestnik Kursu udostępnia Politechnice Wroclawskiej

swoje dane osobowe, które przetwarzamy wyłącznie $\mathrm{w}$ zakresie niezbędnym do jego prowadzenia

(odesfanie zadań, prowadzenie statystyki). Szczegófowe informacje $0$ przetwarzaniu przez nas danych

osobowych są dostępne na stronie internetowej Kursu.

Adres internetowy Kursu: http://www.im.pwr.edu.pl/kurs
\end{document}
