\documentclass[a4paper,12pt]{article}
\usepackage{latexsym}
\usepackage{amsmath}
\usepackage{amssymb}
\usepackage{graphicx}
\usepackage{wrapfig}
\pagestyle{plain}
\usepackage{fancybox}
\usepackage{bm}

\begin{document}

LI KORESPONDENCYJNY KURS

Z MATEMATYKI

$\mathrm{p}\mathrm{a}\acute{\mathrm{z}}$dziernik 2021 $\mathrm{r}.$

PRACA KONTROLNA nr 2- POZIOM PODSTAWOWY

l. Rozwiąz równanie

$\displaystyle \sin 2x=\cos^{4}\frac{x}{2}$ -sin4 $\displaystyle \frac{x}{2}.$

2. Rozwiąz nierównośč

$\sqrt{4-x}\leq x+8.$

3. $\mathrm{W}$ ciągu geometrycznym $(a_{n})\mathrm{z}\mathrm{a}\mathrm{c}\mathrm{h}\mathrm{o}\mathrm{d}\mathrm{z}\Phi$ równości: $a_{4}-a_{2}=18$ oraz $a_{5}-a_{3}=36$. Wyznacz

$a_{3}.$

4. Dla jakich wartości parametru $m$ rozwiązaniem ukladu

$\left\{\begin{array}{l}
2x+3y=4\\
4x+my=2m
\end{array}\right.$

jest para liczb dodatnich?

5. Przekrój poprzeczny dwuspadowego dachu pewnego budynku jest czworokątem ABCD,

$\mathrm{w}$ którym kąt $DAB$ jest kątem prostym, $|AB| =9m$, a obie (nierówne) połacie dachu,

czyli odcinki $BC\mathrm{i}CD$, są nachylone pod kqtem $40^{\mathrm{o}}$ do poziomu (odcinka AB). Oblicz

fączną dfugośč (tzn. $|BC|+|CD|$) obu polaci dachu.

6. Wykaz, $\dot{\mathrm{z}}\mathrm{e}$ miara kąta ostrego $\mathrm{w}$ rombie wynosi $30^{\mathrm{o}}$ wtedy $\mathrm{i}$ tylko wtdy, gdy długośč jego

boku jest równa średniej geometrycznej jego przekątnych.




PRACA KONTROLNA $\mathrm{n}\mathrm{r} 2-$ POZIOM ROZSZERZONY

l. Rozwiąz równanie

tg $x\cdot \mathrm{t}\mathrm{g}(x+1)=1.$

2. Rozwiąz nierównośč

$2-3x>\sqrt{\frac{x+4}{1-x}}.$

3. Huragan znad Oceanu Atlantyckiego zbliza się do wybrzeza Florydy. $\mathrm{J}\mathrm{e}\dot{\mathrm{z}}$ eli jego cen-

trum znajdzie się $\mathrm{w}$ odległości mniejszej $\mathrm{n}\mathrm{i}\dot{\mathrm{z}}60$ km od centrum Miami, to miasto dozna

powaznych zniszczeń. Meteorolog modeluje centrum miastajako ustalony punkt $0$ współ-

rzędnych (240, 200), gdzie $\mathrm{j}\mathrm{e}\mathrm{d}\mathrm{n}\mathrm{o}\mathrm{s}\mathrm{t}\mathrm{k}_{\Phi}$ ukladu wspófrzędnych jest kilometr. Przyjmuje na-

tomiast, $\dot{\mathrm{z}}\mathrm{e}$ centrum huraganu porusza się po prostej $0$ równaniu $y=kx+20$. Dlajakich

wartości parametru $k$ miasto nie dozna powaznych zniszczeń?

4. Zbadaj liczbę rozwiązań równania

-{\it a}2{\it xx}2 $+$-21$\alpha$ - -2 -1{\it ax} $=$ -{\it xa}'

$\mathrm{w}$ zalezności od parametru $a\neq 0.$

5. Pole rombu jest równe $S$, a suma długości jego przekątnych wynosi $m$. Wyznacz długośč

jego boku oraz cosinus kąta ostrego. Jakie warunki $\mathrm{m}\mathrm{u}\mathrm{s}\mathrm{z}\Phi$ spełniač parametry $m\mathrm{i}S\dot{\mathrm{z}}$ eby

zadanie miało rozwiązanie?

6. Dany jest niestały ciag arytmetyczny $(a_{n})$ taki, $\dot{\mathrm{z}}\mathrm{e}$ iloraz

$\displaystyle \frac{\alpha_{1}+a_{2}+\ldots+\alpha_{n}}{a_{n+1}+a_{n+2}+\ldots+a_{2n}}$

jest liczbą stafą $C$. Wyznacz wartośč tej stalej oraz róznicę tego ciqgu, jeśli wiadomo,

$\dot{\mathrm{z}}\mathrm{e}$ jego pierwszym wyrazem jest $\alpha_{1}=p.$

Rozwiązania (rękopis) zadań z wybranego poziomu prosimy nadsyfač do

nika 2021r. na adres:

20 $\mathrm{p}\mathrm{a}\acute{\mathrm{z}}$ dzier-

Wydziaf Matematyki

Politechnika Wrocfawska

Wybrzez $\mathrm{e}$ Wyspiańskiego 27

$50-370$ WROCLAW.

Na kopercie prosimy $\underline{\mathrm{k}\mathrm{o}\mathrm{n}\mathrm{i}\mathrm{e}\mathrm{c}\mathrm{z}\mathrm{n}\mathrm{i}\mathrm{e}}$ zaznaczyč wybrany poziom! (np. poziom podsta-

wowy lub rozszerzony). Do rozwiązań nalez $\mathrm{y}$ dołączyč zaadresowaną do siebie kopertę

zwrotną $\mathrm{z}$ naklejonym znaczkiem, odpowiednim do formatu listu. Polecamy stosowanie

kopert formatu C5 $(160\mathrm{x}230\mathrm{m}\mathrm{m})$ ze znaczkiem $0$ wartości 3,30 zł. Na $\mathrm{k}\mathrm{a}\dot{\mathrm{z}}$ dą większą

kopertę nalez $\mathrm{y}$ nakleič $\mathrm{d}\mathrm{r}\mathrm{o}\dot{\mathrm{z}}$ szy znaczek. Prace niespełniające podanych warunków nie

bedą poprawiane ani odsyłane.

Uwaga. Wysyfajac nam rozwiazania zadań uczestnik Kursu udostępnia Politechnice Wroclawskiej

swoje dane osobowe, które przetwarzamy wyłącznie $\mathrm{w}$ zakresie niezbednym do jego prowadzenia

(odesfanie zadań, prowadzenie statystyki). Szczegófowe informacje $0$ przetwarzaniu przez nas danych

osobowych s\S dostępne na stronie internetowej Kursu.

Adres internetowy Kursu: http: //www. im. pwr. edu. pl/kurs



\end{document}