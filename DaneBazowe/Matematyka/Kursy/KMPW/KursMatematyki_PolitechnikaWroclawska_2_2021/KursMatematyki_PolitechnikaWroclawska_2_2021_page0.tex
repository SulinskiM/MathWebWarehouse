\documentclass[a4paper,12pt]{article}
\usepackage{latexsym}
\usepackage{amsmath}
\usepackage{amssymb}
\usepackage{graphicx}
\usepackage{wrapfig}
\pagestyle{plain}
\usepackage{fancybox}
\usepackage{bm}

\begin{document}

LI KORESPONDENCYJNY KURS

Z MATEMATYKI

$\mathrm{p}\mathrm{a}\acute{\mathrm{z}}$dziernik 2021 $\mathrm{r}.$

PRACA KONTROLNA nr 2- POZIOM PODSTAWOWY

l. Rozwiąz równanie

$\displaystyle \sin 2x=\cos^{4}\frac{x}{2}$ -sin4 $\displaystyle \frac{x}{2}.$

2. Rozwiąz nierównośč

$\sqrt{4-x}\leq x+8.$

3. $\mathrm{W}$ ciągu geometrycznym $(a_{n})\mathrm{z}\mathrm{a}\mathrm{c}\mathrm{h}\mathrm{o}\mathrm{d}\mathrm{z}\Phi$ równości: $a_{4}-a_{2}=18$ oraz $a_{5}-a_{3}=36$. Wyznacz

$a_{3}.$

4. Dla jakich wartości parametru $m$ rozwiązaniem ukladu

$\left\{\begin{array}{l}
2x+3y=4\\
4x+my=2m
\end{array}\right.$

jest para liczb dodatnich?

5. Przekrój poprzeczny dwuspadowego dachu pewnego budynku jest czworokątem ABCD,

$\mathrm{w}$ którym kąt $DAB$ jest kątem prostym, $|AB| =9m$, a obie (nierówne) połacie dachu,

czyli odcinki $BC\mathrm{i}CD$, są nachylone pod kqtem $40^{\mathrm{o}}$ do poziomu (odcinka AB). Oblicz

fączną dfugośč (tzn. $|BC|+|CD|$) obu polaci dachu.

6. Wykaz, $\dot{\mathrm{z}}\mathrm{e}$ miara kąta ostrego $\mathrm{w}$ rombie wynosi $30^{\mathrm{o}}$ wtedy $\mathrm{i}$ tylko wtdy, gdy długośč jego

boku jest równa średniej geometrycznej jego przekątnych.
\end{document}
