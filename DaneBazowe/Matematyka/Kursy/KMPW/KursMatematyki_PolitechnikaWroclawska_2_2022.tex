\documentclass[a4paper,12pt]{article}
\usepackage{latexsym}
\usepackage{amsmath}
\usepackage{amssymb}
\usepackage{graphicx}
\usepackage{wrapfig}
\pagestyle{plain}
\usepackage{fancybox}
\usepackage{bm}

\begin{document}

LII

KORESPONDENCYJNY KURS

Z MATEMATYKI

$\mathrm{p}\mathrm{a}\acute{\mathrm{z}}$dziernik 2022 $\mathrm{r}.$

PRACA KONTROLNA $\mathrm{n}\mathrm{r} 2-$ POZIOM PODSTAWOWY

l. Rozwiąz nierównośč

$1\displaystyle \leq\log_{\frac{1}{3}}\frac{1}{2x-1}<2.$

2. Średnia arytmetyczna czwartego, szóstego $\mathrm{i}$ dziesiątego wyrazu ciągu arytmetycznego

$(a_{n})$, gdzie $ n\geq 1$, wynosi 14, a ciąg $(a_{3},a_{5},a_{11})$ jest geometryczny. Uzasadnij, $\dot{\mathrm{z}}\mathrm{e}$ ciąg

$(a_{4},\alpha_{6},\alpha_{10})$ równiez jest geometryczny.

3. $\mathrm{W}\mathrm{c}\mathrm{i}_{\Phi \mathrm{g}}\mathrm{u}$ arytmetycznym $(a_{n})$, określonym dla $\mathrm{k}\mathrm{a}\dot{\mathrm{z}}$ dej liczby naturalnej $n\geq 1$, mamy

$a_{3}=0$ oraz

$a_{6}=7\sin^{2}\alpha,$

gdzie $\mathrm{t}\mathrm{g}\alpha=3$. Oblicz sumę 50 początkowych wyrazów tego ciągu, których indeksy są

liczbami parzystymi.

4. Bank oferuje kredyt, który nalezy spłacič jednorazowo wraz $\mathrm{z}$ odsetkami po roku. Jaki

jest calkowity koszt tego kredytu, jeśli co miesiąc bank nalicza odsetki $\mathrm{w}$ wysokości

2\% aktualnego zadłuzenia, a dodatkowo $\mathrm{w}$ chwili przyznania kredytu dolicza prowizję

$\mathrm{w}$ wysokości 3\% $\mathrm{p}\mathrm{o}\dot{\mathrm{z}}$ yczanej kwoty? $\mathrm{J}\mathrm{a}\mathrm{k}_{\Phi}$ kwotę trzeba będzie spfacič, jeśli $\mathrm{p}\mathrm{o}\dot{\mathrm{z}}$ yczymy

20000 zł? Prowizja naliczana jest jednorazowo $\mathrm{i}$ powiększa kwotę, którą nalez $\mathrm{y}$ spłacič.

5. Zaznacz na osi liczbowej zbiór wszystkich wartości parametru $t$, dla których funkcja

$f(x)=(\displaystyle \frac{2-t^{2}}{t-3})^{t-x}+1-t$

jest malejąca. Naszkicuj wykres funkcji $f$ dla największej cafkowitej wartości $t\mathrm{z}$ wyzna-

czonego zbioru.

6. Niech $c > 0 \mathrm{i} c \neq 1$. Wyznacz najmniejszą liczbę naturalną $m$, dla której suma $m$

$\mathrm{P}^{\mathrm{O}\mathrm{C}\mathrm{Z}}\varpi$tkowych wyrazów ciągu $(a_{n}), a_{n}=\log_{2}c^{n}$, przekracza liczbę

$\log_{2^{m}}c^{m^{2}}+16\log_{4}c^{2}$




PRACA KONTROLNA $\mathrm{n}\mathrm{r} 2-$ POZIOM ROZSZERZONY

l. Uzasadnij, $\dot{\mathrm{z}}\mathrm{e}\mathrm{c}\mathrm{i}_{\Phi \mathrm{g}}(a_{n})$, którego n-ty wyraz dany jest wzorem

$a_{n}=\displaystyle \frac{1}{2^{1}+3^{1}}+\frac{1}{2^{2}+3^{2}}+\frac{1}{2^{3}+3^{3}}+\cdots+\frac{1}{2^{n}+3^{n}},$

jest ograniczony.

2. Wyznacz dziedzinę $D_{f}$ funkcji

$f(x)=\displaystyle \log_{10+3x-x^{2}}(8-\frac{7}{1-x})$

3. Niech $c>0$. Zbadaj monotonicznośč oraz oblicz sumę wszystkich wyrazów nieskończo-

nego ciągu $(a_{n})$, gdzie

$a_{n}=\log_{3^{3^{n}}}c$ dla $\mathrm{k}\mathrm{a}\dot{\mathrm{z}}$ dego $n\geq 1.$

Ustal, dla jakiej wartości parametru $c$ suma ta jest nie mniejsza od liczby $\log_{9}(c^{2}-2).$

4. Rozwiąz nierównośč

$\displaystyle \sqrt{\log_{\sqrt{x}}(x+2)}>\frac{1}{\log_{\sqrt{x+2}}\sqrt{x}}.$

5. Określ ilośč rozwiązań równania

$|2^{x-1}-1|=m\cdot 2^{x+1}$

$\mathrm{w}$ zalezności od wartości parametru $m.$

6. Opisz metodę konstrukcji $\mathrm{i}$ starannie narysuj wykres funkcji

$f(x)=2+\displaystyle \log_{2}\frac{1}{2-x}.$

Następnie narysuj obraz tej krzywej $\mathrm{w}$ symetrii względem prostej $x =y.$ Wyprowad $\acute{\mathrm{z}}$

wzór funkcji, której wykresem jest powstafa $\mathrm{w}$ ten sposób krzywa.

Rozwiązania (rękopis) zadań $\mathrm{z}$ wybranego poziomu prosimy nadsyfač do $20.10.2022\mathrm{r}$. na

adres:

Wydziaf Matematyki

Politechnika Wrocfawska

Wybrzez $\mathrm{e}$ Wyspiańskiego 27

$50-370$ WROCLAW,

lub elektronicznie, za pośrednictwem portalu talent. $\mathrm{p}\mathrm{w}\mathrm{r}$. edu. pl

Na kopercie prosimy $\underline{\mathrm{k}\mathrm{o}\mathrm{n}\mathrm{i}\mathrm{e}\mathrm{c}\mathrm{z}\mathrm{n}\mathrm{i}\mathrm{e}}$ zaznaczyč wybrany poziom! (np. poziom podsta-

wowy lub rozszerzony). Do rozwiązań nalez $\mathrm{y}$ dołączyč zaadresowaną do siebie koperte

zwrotną $\mathrm{z}$ naklejonym znaczkiem, odpowiednim do formatu listu. Prace niespełniające

podanych warunków nie bedą poprawiane ani odsyłane.

Uwaga. Wysyfajac nam rozwiazania zadań uczestnik Kursu udostępnia Politechnice Wrocfawskiej

swoje dane osobowe, które przetwarzamy wyłącznie $\mathrm{w}$ zakresie niezbędnym do jego prowadzenia

(odeslanie zadań, prowadzenie statystyki). Szczegófowe informacje $0$ przetwarzaniu przez nas danych

osobowych $\mathrm{S}\otimes$ dostępne na stronie internetowej Kursu.

Adres internetowy Kursu: http: //www. im. pwr. edu. pl/kurs



\end{document}