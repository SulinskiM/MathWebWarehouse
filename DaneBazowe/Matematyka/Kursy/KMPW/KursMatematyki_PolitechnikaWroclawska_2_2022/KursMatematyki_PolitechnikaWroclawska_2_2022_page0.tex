\documentclass[a4paper,12pt]{article}
\usepackage{latexsym}
\usepackage{amsmath}
\usepackage{amssymb}
\usepackage{graphicx}
\usepackage{wrapfig}
\pagestyle{plain}
\usepackage{fancybox}
\usepackage{bm}

\begin{document}

LII

KORESPONDENCYJNY KURS

Z MATEMATYKI

$\mathrm{p}\mathrm{a}\acute{\mathrm{z}}$dziernik 2022 $\mathrm{r}.$

PRACA KONTROLNA $\mathrm{n}\mathrm{r} 2-$ POZIOM PODSTAWOWY

l. Rozwiąz nierównośč

$1\displaystyle \leq\log_{\frac{1}{3}}\frac{1}{2x-1}<2.$

2. Średnia arytmetyczna czwartego, szóstego $\mathrm{i}$ dziesiątego wyrazu ciągu arytmetycznego

$(a_{n})$, gdzie $ n\geq 1$, wynosi 14, a ciąg $(a_{3},a_{5},a_{11})$ jest geometryczny. Uzasadnij, $\dot{\mathrm{z}}\mathrm{e}$ ciąg

$(a_{4},\alpha_{6},\alpha_{10})$ równiez jest geometryczny.

3. $\mathrm{W}\mathrm{c}\mathrm{i}_{\Phi \mathrm{g}}\mathrm{u}$ arytmetycznym $(a_{n})$, określonym dla $\mathrm{k}\mathrm{a}\dot{\mathrm{z}}$ dej liczby naturalnej $n\geq 1$, mamy

$a_{3}=0$ oraz

$a_{6}=7\sin^{2}\alpha,$

gdzie $\mathrm{t}\mathrm{g}\alpha=3$. Oblicz sumę 50 początkowych wyrazów tego ciągu, których indeksy są

liczbami parzystymi.

4. Bank oferuje kredyt, który nalezy spłacič jednorazowo wraz $\mathrm{z}$ odsetkami po roku. Jaki

jest calkowity koszt tego kredytu, jeśli co miesiąc bank nalicza odsetki $\mathrm{w}$ wysokości

2\% aktualnego zadłuzenia, a dodatkowo $\mathrm{w}$ chwili przyznania kredytu dolicza prowizję

$\mathrm{w}$ wysokości 3\% $\mathrm{p}\mathrm{o}\dot{\mathrm{z}}$ yczanej kwoty? $\mathrm{J}\mathrm{a}\mathrm{k}_{\Phi}$ kwotę trzeba będzie spfacič, jeśli $\mathrm{p}\mathrm{o}\dot{\mathrm{z}}$ yczymy

20000 zł? Prowizja naliczana jest jednorazowo $\mathrm{i}$ powiększa kwotę, którą nalez $\mathrm{y}$ spłacič.

5. Zaznacz na osi liczbowej zbiór wszystkich wartości parametru $t$, dla których funkcja

$f(x)=(\displaystyle \frac{2-t^{2}}{t-3})^{t-x}+1-t$

jest malejąca. Naszkicuj wykres funkcji $f$ dla największej cafkowitej wartości $t\mathrm{z}$ wyzna-

czonego zbioru.

6. Niech $c > 0 \mathrm{i} c \neq 1$. Wyznacz najmniejszą liczbę naturalną $m$, dla której suma $m$

$\mathrm{P}^{\mathrm{O}\mathrm{C}\mathrm{Z}}\varpi$tkowych wyrazów ciągu $(a_{n}), a_{n}=\log_{2}c^{n}$, przekracza liczbę

$\log_{2^{m}}c^{m^{2}}+16\log_{4}c^{2}$
\end{document}
