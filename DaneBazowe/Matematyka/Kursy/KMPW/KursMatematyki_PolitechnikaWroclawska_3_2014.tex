\documentclass[a4paper,12pt]{article}
\usepackage{latexsym}
\usepackage{amsmath}
\usepackage{amssymb}
\usepackage{graphicx}
\usepackage{wrapfig}
\pagestyle{plain}
\usepackage{fancybox}
\usepackage{bm}

\begin{document}

XLIV

KORESPONDENCYJNY KURS

Z MATEMATYKI

listopad 2014 r.

PRACA KONTROLNA nr 3- POZIOM PODSTAWOWY

l. Rozwiąz nierównośč

$x^{5}+x^{4}-8x^{2}+16\geq 8x^{3}-16x.$

2. $\mathrm{W}$ przedziale $[\pi,2\pi]$ rozwiqz równanie

-cs  oins 36{\it xx} $=$ 1.

3. Dane $\mathrm{s}\Phi$ trzy wektory ã$= (1,$ 1$), \vec{b}= (2,-1), \vec{c}= (5,2)$. Dobierz takie liczby $p, q$, aby

$\mathrm{z}$ wektorów $p\vec{a}, q\vec{b}, \vec{c}\mathrm{m}\mathrm{o}\dot{\mathrm{z}}$ na było zbudowač trójkqt.

4. $\mathrm{W}$ przedziale $[0,\pi]$ narysuj wykres funkcji

$f(x)=\displaystyle \frac{1}{|\mathrm{t}\mathrm{g}x+\mathrm{c}\mathrm{t}\mathrm{g}x|}+\sin 2x,$

$\mathrm{i}$ rozwiąz nierównośč $f(x)<\displaystyle \frac{3}{4}.$

5. Na okręgu $x^{2}-2x+y^{2}+4y-4 = 0$ wyznacz punkt, którego odleglośč od prostej

$x-3y+6=0$ jest najmniejsza.

6. $\mathrm{P}\mathrm{r}\mathrm{z}\mathrm{e}\mathrm{k}_{\Phi}\mathrm{t}\mathrm{n}\mathrm{a}$ rombu $0$ polu 9 zawarta jest $\mathrm{w}$ prostej $x-2y+3 = 0$, a jednym $\mathrm{z}$ jego

wierzchofków jest punkt $A(2,-2)$. Wyznacz współrzędne pozostafych wierzchołków tego

rombu.




PRACA KONTROLNA nr 3- POZIOM ROZSZERZONY

l. Resztą $\mathrm{z}$ dzielenia wielomianu $w(x) =x^{4}+px^{3}-3x^{2}+qx-14$ przez $x^{2}-x-2$ jest

$4x-28$. Wyznacz współczynniki $p, q\mathrm{i}$rozwia $\dot{\mathrm{Z}}$ nierównośč $w(x)\geq 0.$

2. Wyznacz $\mathrm{n}\mathrm{a}\mathrm{j}\mathrm{m}\mathrm{n}\mathrm{i}\mathrm{e}\mathrm{j}\mathrm{s}\mathrm{z}\Phi$ wartośč funkcji

$f(x)=($tg $x+$ ctg $x)^{2}$

oraz $\mathrm{r}\mathrm{o}\mathrm{z}\mathrm{w}\mathrm{i}_{\Phi}\dot{\mathrm{z}}$ nierównośč $f(x)\leq f(2x).$

3. Rozwiąz równanie

$\cos x+\cos 2x+2\cos 3x+\cos 4x+\cos 5x=0.$

4. Znajd $\acute{\mathrm{z}}$ kąt między wektorami $\vec{a}\mathrm{i}\vec{b}\mathrm{w}\mathrm{i}\mathrm{e}\mathrm{d}\mathrm{z}\text{ą} \mathrm{c}, \dot{\mathrm{z}}\mathrm{e}$ wektor $5\text{{\it ã}}-4\vec{b}\mathrm{j}\mathrm{e}\mathrm{s}\mathrm{t}$ prostopadły do wektora

$2\vec{\alpha}+4\vec{b}$, a wektor $\text{{\it ã}}-5\vec{b}$ jest prostopadły do wektora $6\vec{\alpha}-2\vec{b}.$

5. $\mathrm{Z}$ wierzchofka $O$ paraboli $y^{2}=2x$ poprowadzono dwie proste wzajemnie prostopadłe $\mathrm{i}$

przecinające parabolę $\mathrm{w}$ punktach $P\mathrm{i}Q$. Wyznacz zbiór punktów pfaszczyzny utworzony

przez środki cięzkości trójkątów $OPQ.$ Sporząd $\acute{\mathrm{z}}$ rysunek.

6. $\mathrm{W}$ trójk$\Phi$cie $0$ wierzchofkach $A(-6,-7), B(8,-9), C(0,10)$ punkt $P$ jest środkiem boku

$BC$, a punkt $S$ jest punktem przecięcia środkowej poprowadzonej $\mathrm{z}$ wierzcholka $A$ oraz

wysokości opuszczonej na bok $AB$. Oblicz pole trójk$\Phi$ta $CSP$ oraz znajd $\acute{\mathrm{z}}$ równanie

okręgu opisanego na nim.

Rozwiązania (rękopis) zadań z wybranego poziomu prosimy nadsyfač do

2014r. na adres:

18 1istopada

Instytut Matematyki $\mathrm{i}$ Informatyki

Politechniki Wrocfawskiej

Wybrzez $\mathrm{e}$ Wyspiańskiego 27

$50-370$ WROCLAW.

Na kopercie prosimy $\underline{\mathrm{k}\mathrm{o}\mathrm{n}\mathrm{i}\mathrm{e}\mathrm{c}\mathrm{z}\mathrm{n}\mathrm{i}\mathrm{e}}$ zaznaczyč wybrany poziom! (np. poziom podsta-

wowy lub rozszerzony). Do rozwiązań nalez $\mathrm{y}$ dołączyč zaadresowaną do siebie kopertę

zwrotną $\mathrm{z}$ naklejonym znaczkiem, odpowiednim do wagi listu. Prace niespełniające po-

danych warunków nie będą poprawiane ani odsyłane.

Adres internetowy Kursu: http://www.im.pwr.wroc.pl/kurs



\end{document}