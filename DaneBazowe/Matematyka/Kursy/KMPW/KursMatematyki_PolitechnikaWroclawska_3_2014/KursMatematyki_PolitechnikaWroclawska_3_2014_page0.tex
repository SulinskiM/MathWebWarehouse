\documentclass[a4paper,12pt]{article}
\usepackage{latexsym}
\usepackage{amsmath}
\usepackage{amssymb}
\usepackage{graphicx}
\usepackage{wrapfig}
\pagestyle{plain}
\usepackage{fancybox}
\usepackage{bm}

\begin{document}

XLIV

KORESPONDENCYJNY KURS

Z MATEMATYKI

listopad 2014 r.

PRACA KONTROLNA nr 3- POZIOM PODSTAWOWY

l. Rozwiąz nierównośč

$x^{5}+x^{4}-8x^{2}+16\geq 8x^{3}-16x.$

2. $\mathrm{W}$ przedziale $[\pi,2\pi]$ rozwiqz równanie

-cs  oins 36{\it xx} $=$ 1.

3. Dane $\mathrm{s}\Phi$ trzy wektory ã$= (1,$ 1$), \vec{b}= (2,-1), \vec{c}= (5,2)$. Dobierz takie liczby $p, q$, aby

$\mathrm{z}$ wektorów $p\vec{a}, q\vec{b}, \vec{c}\mathrm{m}\mathrm{o}\dot{\mathrm{z}}$ na było zbudowač trójkqt.

4. $\mathrm{W}$ przedziale $[0,\pi]$ narysuj wykres funkcji

$f(x)=\displaystyle \frac{1}{|\mathrm{t}\mathrm{g}x+\mathrm{c}\mathrm{t}\mathrm{g}x|}+\sin 2x,$

$\mathrm{i}$ rozwiąz nierównośč $f(x)<\displaystyle \frac{3}{4}.$

5. Na okręgu $x^{2}-2x+y^{2}+4y-4 = 0$ wyznacz punkt, którego odleglośč od prostej

$x-3y+6=0$ jest najmniejsza.

6. $\mathrm{P}\mathrm{r}\mathrm{z}\mathrm{e}\mathrm{k}_{\Phi}\mathrm{t}\mathrm{n}\mathrm{a}$ rombu $0$ polu 9 zawarta jest $\mathrm{w}$ prostej $x-2y+3 = 0$, a jednym $\mathrm{z}$ jego

wierzchofków jest punkt $A(2,-2)$. Wyznacz współrzędne pozostafych wierzchołków tego

rombu.
\end{document}
