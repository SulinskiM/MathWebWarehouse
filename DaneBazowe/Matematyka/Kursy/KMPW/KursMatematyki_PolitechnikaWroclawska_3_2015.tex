\documentclass[a4paper,12pt]{article}
\usepackage{latexsym}
\usepackage{amsmath}
\usepackage{amssymb}
\usepackage{graphicx}
\usepackage{wrapfig}
\pagestyle{plain}
\usepackage{fancybox}
\usepackage{bm}

\begin{document}

XLV

KORESPONDENCYJNY KURS

Z MATEMATYKI

listopad 2015 r.

PRACA KONTROLNA nr 3- POZIOM PODSTAWOWY

l. Rozwiązač równanie tg $x-\displaystyle \sin x=\frac{1-\cos x}{2\cos x}.$

2. Narysowač wykres funkcji $f(x)=2\sin x+|\sin x|\mathrm{i}$ rozwiązač nierównośč $|f(x)|\displaystyle \leq\frac{3\sqrt{3}}{2}.$

3. Odcinek $CD$ jest obrazem odcinka $0$ końcach $A(1,1)\mathrm{i}B(2,0)$ wjednokładności $0$ środku

$S(1,-1)\mathrm{i}$ skali $k=-2$. Obliczyč pole czworokąta ABCD. Sporządzič rysunek.

4. Wielomian $W(x)=x^{3}+ax^{2}+bx+c$ jest podzielny przez dwumian $x+1$, ajego wykres

jest symetryczny względem punktu $(0,0)$. Wyznaczyč $a, b, c\mathrm{i}$ rozwiązač nierównośč

$(x-1)W(x+2)-(x-2)W(x+1)\leq 0.$

5. Punkty $\mathrm{A}(1,1), \mathrm{B}(0,3)$ są kolejnymi wierzchofkami rombu ABCD. Wyznaczyč pozostafe

wierzchołki, wiedząc, $\dot{\mathrm{z}}\mathrm{e}$ jeden $\mathrm{z}$ nich $\mathrm{l}\mathrm{e}\dot{\mathrm{z}}\mathrm{y}$ na prostej $x-y-2=0$. Sporzqdzič rysunek.

6. $\mathrm{W}$ trójkąt równoramienny wpisano $\mathrm{o}\mathrm{k}\mathrm{r}\varpi \mathrm{g}\mathrm{o}$ promieniu $r$. Wyznaczyč pole trójk$\Phi$ta, $\mathrm{j}\mathrm{e}\dot{\mathrm{z}}$ eli

środek okręgu opisanego na tym trójkącie $\mathrm{l}\mathrm{e}\dot{\mathrm{z}}\mathrm{y}$ na okręgu wpisanym $\mathrm{w}$ ten trójkąt. Ile

rozwiązań ma to zadanie? Sporządzič rysunek.




PRACA KONTROLNA nr 3- POZIOM ROZSZERZONY

l. Narysowač wykres funkcji $f(x)=\cos 2x-\sin^{2}x \mathrm{i}$ rozwiązač nierównośč $f(x)\displaystyle \geq\frac{1}{4}.$

2. Obliczyč pole trójkąta $ABC 0$ wierzcholkach $A(3,6), B(1,0)$, wiedząc, $\dot{\mathrm{z}}\mathrm{e}$ wysokości

przecinają się $\mathrm{w}$ punkcie (4, 4). Sporządzič rysunek.

3. Dla jakiego kąta ostrego $\alpha$ zachodzi równośč

$\log_{\sin\alpha}(2\cos^{2}\alpha+\sin\alpha\cos\alpha-1)=2$?

4. Dla jakiego parametru $p$ wielomian $W(x) = x^{3}+px^{2}+11x-6$ ma trzy pierwiastki,

$\mathrm{z}$ których jeden jest średnią arytmetyczną pozostafych? Znalez/č wielomian $0$ powyzszej

własności, którego wszystkie pierwiastki są wymierne.

5. Wyznaczyč równania wszystkich prostych stycznych do $\mathrm{k}\mathrm{a}\dot{\mathrm{z}}$ dej $\mathrm{z}$ parabol $y=(x+1)^{2}$

oraz $y=-(x-3)^{2}-2$. Sporz$\Phi$dzič rysunek.

6. $\mathrm{W}$ trójkącie równoramiennym $ABC$ sinus kąta przy wierzcholku $C$ jest równy 3/5. Pod

jakim $\mathrm{k}_{\Phi}\mathrm{t}\mathrm{e}\mathrm{m}$ przecinają się środkowe poprowadzone $\mathrm{z}$ wierzcholków podstawy AB?

Rozwiazania (rękopis) zadań z wybranego poziomu prosimy nadsyłač do

2015r. na adres:

18 1istopada

$\mathrm{W}\mathrm{y}\mathrm{d}\mathrm{z}\mathrm{i}\mathrm{a}\not\subset$ Matematyki

Politechnika Wrocfawska

Wybrzez $\mathrm{e}$ Wyspiańskiego 27

$50-370$ WROCLAW.

Na kopercie prosimy $\underline{\mathrm{k}\mathrm{o}\mathrm{n}\mathrm{i}\mathrm{e}\mathrm{c}\mathrm{z}\mathrm{n}\mathrm{i}\mathrm{e}}$ zaznaczyč wybrany poziom! (np. poziom podsta-

wowy lub rozszerzony). Do rozwiązań nalez $\mathrm{y}$ dołączyč zaadresowaną do siebie koperte

zwrotną $\mathrm{z}$ naklejonym znaczkiem, odpowiednim do wagi listu. Prace niespelniające po-

danych warunków nie będą poprawiane ani odsyłane.

Adres internetowy Kursu: http: //www. im. pwr. edu. pl/kurs



\end{document}