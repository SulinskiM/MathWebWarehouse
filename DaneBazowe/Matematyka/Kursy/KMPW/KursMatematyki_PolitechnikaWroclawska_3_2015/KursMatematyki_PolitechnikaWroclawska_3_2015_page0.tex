\documentclass[a4paper,12pt]{article}
\usepackage{latexsym}
\usepackage{amsmath}
\usepackage{amssymb}
\usepackage{graphicx}
\usepackage{wrapfig}
\pagestyle{plain}
\usepackage{fancybox}
\usepackage{bm}

\begin{document}

XLV

KORESPONDENCYJNY KURS

Z MATEMATYKI

listopad 2015 r.

PRACA KONTROLNA nr 3- POZIOM PODSTAWOWY

l. Rozwiązač równanie tg $x-\displaystyle \sin x=\frac{1-\cos x}{2\cos x}.$

2. Narysowač wykres funkcji $f(x)=2\sin x+|\sin x|\mathrm{i}$ rozwiązač nierównośč $|f(x)|\displaystyle \leq\frac{3\sqrt{3}}{2}.$

3. Odcinek $CD$ jest obrazem odcinka $0$ końcach $A(1,1)\mathrm{i}B(2,0)$ wjednokładności $0$ środku

$S(1,-1)\mathrm{i}$ skali $k=-2$. Obliczyč pole czworokąta ABCD. Sporządzič rysunek.

4. Wielomian $W(x)=x^{3}+ax^{2}+bx+c$ jest podzielny przez dwumian $x+1$, ajego wykres

jest symetryczny względem punktu $(0,0)$. Wyznaczyč $a, b, c\mathrm{i}$ rozwiązač nierównośč

$(x-1)W(x+2)-(x-2)W(x+1)\leq 0.$

5. Punkty $\mathrm{A}(1,1), \mathrm{B}(0,3)$ są kolejnymi wierzchofkami rombu ABCD. Wyznaczyč pozostafe

wierzchołki, wiedząc, $\dot{\mathrm{z}}\mathrm{e}$ jeden $\mathrm{z}$ nich $\mathrm{l}\mathrm{e}\dot{\mathrm{z}}\mathrm{y}$ na prostej $x-y-2=0$. Sporzqdzič rysunek.

6. $\mathrm{W}$ trójkąt równoramienny wpisano $\mathrm{o}\mathrm{k}\mathrm{r}\varpi \mathrm{g}\mathrm{o}$ promieniu $r$. Wyznaczyč pole trójk$\Phi$ta, $\mathrm{j}\mathrm{e}\dot{\mathrm{z}}$ eli

środek okręgu opisanego na tym trójkącie $\mathrm{l}\mathrm{e}\dot{\mathrm{z}}\mathrm{y}$ na okręgu wpisanym $\mathrm{w}$ ten trójkąt. Ile

rozwiązań ma to zadanie? Sporządzič rysunek.
\end{document}
