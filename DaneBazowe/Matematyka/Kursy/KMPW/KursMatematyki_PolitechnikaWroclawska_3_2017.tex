\documentclass[a4paper,12pt]{article}
\usepackage{latexsym}
\usepackage{amsmath}
\usepackage{amssymb}
\usepackage{graphicx}
\usepackage{wrapfig}
\pagestyle{plain}
\usepackage{fancybox}
\usepackage{bm}

\begin{document}

XLVII

KORESPONDENCYJNY KURS

Z MATEMATYKI

listopad 2017 r.

PRACA KONTROLNA $\mathrm{n}\mathrm{r} 3-$ POZIOM PODSTAWOWY

l. Dwaj kolarze jezdzą po torze $\mathrm{w}$ kształcie okręgu ze stałymi prędkościami. $\mathrm{J}\mathrm{e}\dot{\mathrm{z}}$ eli startujq

$\mathrm{z}$ tego samego punktu $\mathrm{i}\mathrm{j}\mathrm{a}\mathrm{d}_{\Phi}\mathrm{w}$ tę $\mathrm{s}\mathrm{a}\mathrm{m}\Phi$ stronę, to szybszy $\mathrm{z}$ nich pierwszy raz ponownie

zrówna się $\mathrm{z}$ wolniejszym, wyprzedzając go ojedno okrązenie, po przejechaniu dokładnie

7 okrązeń. Ilu okrązeń potrzebuje szybszy kolarz $\dot{\mathrm{z}}$ eby dogonič kolegę, $\mathrm{j}\mathrm{e}\dot{\mathrm{z}}$ eli startują $\mathrm{z}$

przeciwlegfych stron toru (tzn. odcinek lączący punkty ich startu jest średnic$\Phi$ kofa)?

2. Liczby $0$ 16\% mniejsza $\mathrm{i}\mathrm{o}$ 43\% większa od ułamka okresowego 0, (75) są pierwiastkami

trójmianu kwadratowego $0$ wspófczynnikach całkowitych względnie pierwszych. Obliczyč

resztę $\mathrm{z}$ dzielenia tego trójmianu przez dwumian $(x-1).$

3. Rozwiązač równanie

$\displaystyle \sin x+\cos x=\frac{1}{\sin x}.$

4. Rozwiązač nierównośč

$\displaystyle \frac{\log_{2}(10-x^{2})}{\log_{2}(4-x)}>2.$

5. Dwa okręgi $0$ promieniach $r\mathrm{i}R$ styczne zewnętrznie $\mathrm{w}$ punkcie $C$, są styczne do prostej

$k\mathrm{w}$ punktach A $\mathrm{i}B$. Wyznaczyč kąt $\angle ACB\mathrm{i}$ promień okręgu opisanego na trójk$\Phi$cie

$ABC.$

6. Dane są punkty $A(2,-2)\mathrm{i}B(8,1)$. Na paraboli $y=x^{2}-x$ znalez/č taki punkt $C, \dot{\mathrm{z}}$ eby

pole trójkąta $ABC$ bylo najmniejsze. Wykonač rysunek.




PRACA KONTROLNA nr 3- POZ1OM ROZSZERZONY

l. Czy wieza zbudowana $\mathrm{z}$ sześciennych klocków $0$ objętościach 1, 3, 9, 27, zmieści $\mathrm{s}\mathrm{i}\mathrm{e}$ na

pófce $0$ wysokości $\displaystyle \frac{15}{2}?$Odpowied $\acute{\mathrm{z}}$ uzasadnič nie $\mathrm{s}\mathrm{t}\mathrm{o}\mathrm{s}\mathrm{u}\mathrm{j}_{\Phi}\mathrm{c}$ obliczeń przyblizonych.

2. Rozwiązač równanie

$\cos 2x=(\sqrt{3}-1)\sin x(\cos x+\sin x).$

3. Sporz$\Phi$dzič staranny wykres funkcji $f(x)=|2^{-|x|+1}-1|-\displaystyle \frac{1}{2}$. Opisač sposób postępowania.

Rozwiązač nierównośč $f(x)>0.$

4. Rozwiązač nierównośč

$\displaystyle \log_{2}x+\log_{2}^{3}x+\log_{2}^{5}x+<\frac{20}{9}.$

5. Podjakim kątem przecinajq się okręgi $0$ równaniach $(x-6)^{2}+y^{2}=9, x^{2}+(y+4)^{2}=25$

(kątem miedzy dwoma okręgami nazywamy kąt między stycznymi $\mathrm{w}$ punkcie przecięcia)?

Znalez/č równanie okręgu, którego środek $\mathrm{l}\mathrm{e}\dot{\mathrm{z}}\mathrm{y}$ na prostej $2x-y=0, \mathrm{i}$ który przecina

$\mathrm{k}\mathrm{a}\dot{\mathrm{z}}\mathrm{d}\mathrm{y}\mathrm{z}$ danych okręgów pod $\mathrm{k}_{\Phi^{\mathrm{t}\mathrm{e}\mathrm{m}}}$ prostym.

6. Boisko do gry $\mathrm{w}$ football amerykański ma kształt prostokąta $0$ dlugości $a\mathrm{i}$ szerokości

$b<a$. Na środku krótszych boków stoją bramki $0$ szerokości $d<b. \mathrm{Z}$ którego miejsca

linii bocznej boiska (czyli dfuzszego boku $\mathrm{P}^{\mathrm{r}\mathrm{o}\mathrm{s}\mathrm{t}\mathrm{o}\mathrm{k}}\Phi^{\mathrm{t}\mathrm{a})}$ widač bramkę pod największym

$\mathrm{m}\mathrm{o}\dot{\mathrm{z}}$ liwym kątem? Wyrazič odpowied $\acute{\mathrm{z}}\mathrm{z}\mathrm{a}$ pomocą wzoru zawierającego symbole $a, b, d,$

a następnie wykonač obliczenia dla wartości $\alpha=110m, b=49m, d=5m.$

Rozwiązania (rękopis) zadań z wybranego poziomu prosimy nadsylač do

2016r. na adres:

18 1istopada

Wydziaf Matematyki

Politechnika Wrocfawska

Wybrzez $\mathrm{e}$ Wyspiańskiego 27

$50-370$ WROCLAW.

Na kopercie prosimy $\underline{\mathrm{k}\mathrm{o}\mathrm{n}\mathrm{i}\mathrm{e}\mathrm{c}\mathrm{z}\mathrm{n}\mathrm{i}\mathrm{e}}$ zaznaczyč wybrany poziom! (np. poziom podsta-

wowy lub rozszerzony). Do rozwiązań nalez $\mathrm{y}$ dolączyč zaadresowaną do siebie koperte

zwrotną $\mathrm{z}$ naklejonym znaczkiem, odpowiednim do wagi listu. Prace niespelniające po-

danych warunków nie będą poprawiane ani odsyłane.

Adres internetowy Kursu: http://www.im.pwr.wroc.pl/kurs



\end{document}