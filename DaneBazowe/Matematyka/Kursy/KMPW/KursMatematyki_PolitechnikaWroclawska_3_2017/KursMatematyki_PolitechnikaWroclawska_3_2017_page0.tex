\documentclass[a4paper,12pt]{article}
\usepackage{latexsym}
\usepackage{amsmath}
\usepackage{amssymb}
\usepackage{graphicx}
\usepackage{wrapfig}
\pagestyle{plain}
\usepackage{fancybox}
\usepackage{bm}

\begin{document}

XLVII

KORESPONDENCYJNY KURS

Z MATEMATYKI

listopad 2017 r.

PRACA KONTROLNA $\mathrm{n}\mathrm{r} 3-$ POZIOM PODSTAWOWY

l. Dwaj kolarze jezdzą po torze $\mathrm{w}$ kształcie okręgu ze stałymi prędkościami. $\mathrm{J}\mathrm{e}\dot{\mathrm{z}}$ eli startujq

$\mathrm{z}$ tego samego punktu $\mathrm{i}\mathrm{j}\mathrm{a}\mathrm{d}_{\Phi}\mathrm{w}$ tę $\mathrm{s}\mathrm{a}\mathrm{m}\Phi$ stronę, to szybszy $\mathrm{z}$ nich pierwszy raz ponownie

zrówna się $\mathrm{z}$ wolniejszym, wyprzedzając go ojedno okrązenie, po przejechaniu dokładnie

7 okrązeń. Ilu okrązeń potrzebuje szybszy kolarz $\dot{\mathrm{z}}$ eby dogonič kolegę, $\mathrm{j}\mathrm{e}\dot{\mathrm{z}}$ eli startują $\mathrm{z}$

przeciwlegfych stron toru (tzn. odcinek lączący punkty ich startu jest średnic$\Phi$ kofa)?

2. Liczby $0$ 16\% mniejsza $\mathrm{i}\mathrm{o}$ 43\% większa od ułamka okresowego 0, (75) są pierwiastkami

trójmianu kwadratowego $0$ wspófczynnikach całkowitych względnie pierwszych. Obliczyč

resztę $\mathrm{z}$ dzielenia tego trójmianu przez dwumian $(x-1).$

3. Rozwiązač równanie

$\displaystyle \sin x+\cos x=\frac{1}{\sin x}.$

4. Rozwiązač nierównośč

$\displaystyle \frac{\log_{2}(10-x^{2})}{\log_{2}(4-x)}>2.$

5. Dwa okręgi $0$ promieniach $r\mathrm{i}R$ styczne zewnętrznie $\mathrm{w}$ punkcie $C$, są styczne do prostej

$k\mathrm{w}$ punktach A $\mathrm{i}B$. Wyznaczyč kąt $\angle ACB\mathrm{i}$ promień okręgu opisanego na trójk$\Phi$cie

$ABC.$

6. Dane są punkty $A(2,-2)\mathrm{i}B(8,1)$. Na paraboli $y=x^{2}-x$ znalez/č taki punkt $C, \dot{\mathrm{z}}$ eby

pole trójkąta $ABC$ bylo najmniejsze. Wykonač rysunek.
\end{document}
