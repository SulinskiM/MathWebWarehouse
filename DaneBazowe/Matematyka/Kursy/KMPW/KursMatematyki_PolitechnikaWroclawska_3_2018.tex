\documentclass[a4paper,12pt]{article}
\usepackage{latexsym}
\usepackage{amsmath}
\usepackage{amssymb}
\usepackage{graphicx}
\usepackage{wrapfig}
\pagestyle{plain}
\usepackage{fancybox}
\usepackage{bm}

\begin{document}

XLVIII

KORESPONDENCYJNY KURS

Z MATEMATYKI

listopad 2018 r.

PRACA KONTROLNA $\mathrm{n}\mathrm{r} 3-$ POZIOM PODSTAWOWY

l. Narysowač wykres funkcji $f(x)=2\cos x-|\cos x|\mathrm{i}$ rozwiazač nierównośč $f(x)<-\displaystyle \frac{3}{2}.$

2. Znalez/č punkt nalezący do paraboli $y^{2}=4x$, którego odleglośč od punktu $A(3,0)$ jest

najmniejsza.

3. Dany jest punkt $A(2,1)$ oraz dwie proste:

$p$: $x+y+2=0, q$: $x-2y-4=0.$

Znalez/č taki punkt $B$ na prostej $q, \dot{\mathrm{z}}\mathrm{e}\mathrm{b}\mathrm{y}$ środek odcinka AB $\mathrm{l}\mathrm{e}\dot{\mathrm{z}}$ af $\mathrm{n}\mathrm{a}$ prostej $p$. Sporządzič

rysunek.

4. Logarytmy liczb l, $3^{x}-2, 3^{x}+4$ tworzą ciąg arytmetyczny ($\mathrm{w}$ podanej kolejności). Ob-

liczyč $x.$

5. Kolejne środki boków czworokąta wypuklego ABCD polączono odcinkami otrzymując

czworokqt EFGH. Jaka figurą jest czworokąt EFGH? Odpowied $\acute{\mathrm{z}}$ uzasadnič. Obliczyč

pole czworokąta ABCD, wiedząc, $\dot{\mathrm{z}}\mathrm{e}$ pole czworokąta EFGH jest równe 5.

6. Rozwiązač nierównośč

$f(x)\displaystyle \leq\frac{4}{f(x)},$

gdzie $f(x)=-\displaystyle \frac{4}{3}x^{2}+2x+\frac{4}{3}.$




PRACA KONTROLNA nr 3- POZ1OM ROZSZERZONY

l. Narysowač wykres funkcji $f(x)=2\displaystyle \cos^{2}x-\sin(2x-\frac{\pi}{2})\mathrm{i}$ rozwiazač nierównośč $|f(x)|<2.$

2. Znalez/č punkt nalezący do paraboli $y^{2}=2x$, którego odlegfośč od prostej $x-2y+6=0$

jest najmniejsza.

3. Wielomian $w(x)=x^{4}+\alpha x^{3}+bx^{2}+cx+d$ jest podzielny przez trójmian $x^{2}-x-2, \mathrm{a}$

jego wykres jest symetryczny względem osi $0y$. Wyznaczyč wartości parametrów $a, b, c, d$

$\mathrm{i}$ rozwiqzač nierównośč $w(x+1)\leq w(x-2).$

4. Rozwiązač nierównośč

$\log x+\log^{3}x+\log^{5}x+\leq 2\sqrt{5}.$

5. Punkt $S$ jest środkiem boku AB $\mathrm{w}$ trójkącie $ABC$. Ponadto $AC\neq BC$ oraz $\angle BAC+$

$\angle SCB=90^{\mathrm{o}}$ Niech $D$ bedzie punktem przeciecia symetralnej AB $\mathrm{z}$ prostą $AC$. Udo-

wodnič, $\dot{\mathrm{z}}\mathrm{e}$ na czworokącie SBDC $\mathrm{m}\mathrm{o}\dot{\mathrm{z}}$ na opisač okrąg. Dlaczego musimy zalozyč, $\dot{\mathrm{z}}\mathrm{e}$

$AC\neq BC$?

6. Wyznaczyč równanie zbioru wszystkich środków tych cięciw paraboli $y = x^{2}$, które

przechodzą przez punkt $A(0,2).$

Rozwiązania (rękopis) zadań z wybranego poziomu prosimy nadsyłač do

2018r. na adres:

18 1istopada

Wydziaf Matematyki

Politechnika Wrocfawska

Wybrzez $\mathrm{e}$ Wyspiańskiego 27

$50-370$ WROCLAW.

Na kopercie prosimy $\underline{\mathrm{k}\mathrm{o}\mathrm{n}\mathrm{i}\mathrm{e}\mathrm{c}\mathrm{z}\mathrm{n}\mathrm{i}\mathrm{e}}$ zaznaczyč wybrany poziom! (np. poziom podsta-

wowy lub rozszerzony). Do rozwiązań nalez $\mathrm{y}$ dołączyč zaadresowaną do siebie kopertę

zwrotną $\mathrm{z}$ naklejonym znaczkiem, odpowiednim do wagi listu. Prace niespełniające po-

danych warunków nie będą poprawiane ani odsyłane.

Uwaga. Wysylając nam rozwiazania zadań uczestnik Kursu udostępnia Politechnice Wroclawskiej

swoje dane osobowe, które przetwarzamy wyłącznie $\mathrm{w}$ zakresie niezbędnym do jego prowadzenia

(odesłanie zadań, prowadzenie statystyki). Szczegółowe informacje $0$ przetwarzaniu przez nas danych

osobowych są dostępne na stronie internetowej Kursu.

Adres internetowy Kursu: http: //www. im. pwr. edu. pl/kurs



\end{document}