\documentclass[a4paper,12pt]{article}
\usepackage{latexsym}
\usepackage{amsmath}
\usepackage{amssymb}
\usepackage{graphicx}
\usepackage{wrapfig}
\pagestyle{plain}
\usepackage{fancybox}
\usepackage{bm}

\begin{document}

XLVIII

KORESPONDENCYJNY KURS

Z MATEMATYKI

listopad 2018 r.

PRACA KONTROLNA $\mathrm{n}\mathrm{r} 3-$ POZIOM PODSTAWOWY

l. Narysowač wykres funkcji $f(x)=2\cos x-|\cos x|\mathrm{i}$ rozwiazač nierównośč $f(x)<-\displaystyle \frac{3}{2}.$

2. Znalez/č punkt nalezący do paraboli $y^{2}=4x$, którego odleglośč od punktu $A(3,0)$ jest

najmniejsza.

3. Dany jest punkt $A(2,1)$ oraz dwie proste:

$p$: $x+y+2=0, q$: $x-2y-4=0.$

Znalez/č taki punkt $B$ na prostej $q, \dot{\mathrm{z}}\mathrm{e}\mathrm{b}\mathrm{y}$ środek odcinka AB $\mathrm{l}\mathrm{e}\dot{\mathrm{z}}$ af $\mathrm{n}\mathrm{a}$ prostej $p$. Sporządzič

rysunek.

4. Logarytmy liczb l, $3^{x}-2, 3^{x}+4$ tworzą ciąg arytmetyczny ($\mathrm{w}$ podanej kolejności). Ob-

liczyč $x.$

5. Kolejne środki boków czworokąta wypuklego ABCD polączono odcinkami otrzymując

czworokqt EFGH. Jaka figurą jest czworokąt EFGH? Odpowied $\acute{\mathrm{z}}$ uzasadnič. Obliczyč

pole czworokąta ABCD, wiedząc, $\dot{\mathrm{z}}\mathrm{e}$ pole czworokąta EFGH jest równe 5.

6. Rozwiązač nierównośč

$f(x)\displaystyle \leq\frac{4}{f(x)},$

gdzie $f(x)=-\displaystyle \frac{4}{3}x^{2}+2x+\frac{4}{3}.$
\end{document}
