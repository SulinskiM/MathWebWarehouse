\documentclass[a4paper,12pt]{article}
\usepackage{latexsym}
\usepackage{amsmath}
\usepackage{amssymb}
\usepackage{graphicx}
\usepackage{wrapfig}
\pagestyle{plain}
\usepackage{fancybox}
\usepackage{bm}

\begin{document}

XLIX

KORESPONDENCYJNY KURS

Z MATEMATYKI

listopad 2019 r.

PRACA KONTROLNA $\mathrm{n}\mathrm{r} 3-$ POZIOM PODSTAWOWY

l. Znalez$\acute{}$č największą wartośč funkcji

$f(x)=\displaystyle \frac{2}{\sqrt{4x^{2}-12x+11}}$

$\mathrm{i}$ rozwiqzač nierównośč $f(x)\geq 1.$

2. Rozwiązač równanie

$(1+\cos 4x)\sin 2x=\cos^{2}2x.$

3. Rozwiązač równanie

$\log_{\sqrt{5}}(4^{x}-6)-\log_{\sqrt{5}}(2^{x}-2)=2.$

4. Stosunek długości przekątnych rombu jest równy 5:l2. Obliczyč stosunek pola rombu do

do pola koła wpisanego $\mathrm{w}$ ten romb.

5. Dane są punkty $A(1,1)\mathrm{i}B(7,4)$. Na paraboli $y=x^{2}+x+3$ znalez/č taki punkt $C, \dot{\mathrm{z}}$ eby

pole trójkąta $ABC$ było najmniejsze. Wykonač rysunek.

6. Ramiona trójk$\Phi$ta równoramiennego zawarte $\mathrm{s}\Phi^{\mathrm{W}}$ prostych $0$ równaniach $8x-y+17=0$

oraz $4x+7y-59 = 0$, a jego podstawa przechodzi przez punkt $P(0,2)$. Wyznaczyč

równanie prostej zawierajacej podstawę $\mathrm{i}$ obliczyč pole tego trójkqta.




PRACA KONTROLNA nr 3- POZIOM ROZSZERZONY

l. Dla jakich wartości parametru $m$ równanie

$x^{2}-2(m-4)x+m^{2}+5m+6=0$

ma dwa rózne pierwiastki rzeczywiste, których suma odwrotności jest dodatnia?

2. Rozwiązač równanie

$\displaystyle \frac{1}{\sin^{2}2x}+\mathrm{t}\mathrm{g}x-$ ctg $x=2.$

3. Rozwiązač układ równań

$\left\{\begin{array}{l}
- 2\mathrm{l}\mathrm{o}\mathrm{l}\mathrm{o}\mathrm{g}\mathrm{g}2(- x- \mathrm{l}\mathrm{o}y\mathrm{g})(- x+1y)\\
- \mathrm{l}\mathrm{l}\mathrm{o}\mathrm{o}\mathrm{g}\mathrm{g}xy-- \mathrm{l}\mathrm{l}\mathrm{o}\mathrm{o}\mathrm{g}\mathrm{g}73
\end{array}\right.$

$=1$

$=-1.$

4. Dany jest trójkąt $ABC, \mathrm{w}$ którym $\displaystyle \angle ACB=\frac{2\pi}{3}$. Dwusieczna kąta $ACB$ przecina prostą

przechodzqca przez punkt $A\mathrm{i}$ równoległq do boku $BC\mathrm{w}$ punkcie $P$, a prostq przecho-

dzącą przez punkt $B\mathrm{i}$ równolegl$\Phi$ do boku $AC\mathrm{w}$ punkcie $Q$. Udowodnič, $\dot{\mathrm{z}}\mathrm{e}AQ=BP.$

5. Wyznaczyč stosunek promienia okręgu wpisanego $\mathrm{w}$ {\it romb ABCD} $0$ kącie ostrym $\alpha=$

$\angle DAB$ do promienia okregu opisanego na trójkącie $ABD$. Sprawdzič dlajakiego kata $\alpha$

stosunek ten jest najwięszy.

6. Wyznaczyč równanie zbioru wszystkich środków tych cięciw paraboli $y = x^{2}$, które

zaczynają się $\mathrm{w}$ punkcie $A(1,1)$. Rozwiązanie zilustrowač rysunkiem.

Rozwiązania (rękopis) zadań z wybranego poziomu prosimy nadsyfač do

2019r. na adres:

18 1istopada

Wydziaf Matematyki

Politechnika Wrocfawska

Wybrzez $\mathrm{e}$ Wyspiańskiego 27

$50-370$ WROCLAW.

Na kopercie prosimy $\underline{\mathrm{k}\mathrm{o}\mathrm{n}\mathrm{i}\mathrm{e}\mathrm{c}\mathrm{z}\mathrm{n}\mathrm{i}\mathrm{e}}$ zaznaczyč wybrany poziom! (np. poziom podsta-

wowy lub rozszerzony). Do rozwiązań nalez $\mathrm{y}$ dołączyč zaadresowaną do siebie koperte

zwrotną $\mathrm{z}$ naklejonym znaczkiem, odpowiednim do wagi listu. Prace niespelniające po-

danych warunków nie bedą poprawiane ani odsyłane.

Uwaga. Wysylając nam rozwi\S zania zadań uczestnik Kursu udostępnia Politechnice Wroclawskiej

swoje dane osobowe, które przetwarzamy wyłącznie $\mathrm{w}$ zakresie niezbednym do jego prowadzenia

(odesfanie zadań, prowadzenie statystyki). Szczegófowe informacje $0$ przetwarzaniu przez nas danych

osobowych są dostępne na stronie internetowej Kursu.

Adres internetowy Kursu: http: //www. im. pwr. edu. pl/kurs



\end{document}