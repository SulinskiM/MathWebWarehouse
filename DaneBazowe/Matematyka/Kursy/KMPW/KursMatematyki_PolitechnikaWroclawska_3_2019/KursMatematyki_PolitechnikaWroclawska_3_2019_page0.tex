\documentclass[a4paper,12pt]{article}
\usepackage{latexsym}
\usepackage{amsmath}
\usepackage{amssymb}
\usepackage{graphicx}
\usepackage{wrapfig}
\pagestyle{plain}
\usepackage{fancybox}
\usepackage{bm}

\begin{document}

XLIX

KORESPONDENCYJNY KURS

Z MATEMATYKI

listopad 2019 r.

PRACA KONTROLNA $\mathrm{n}\mathrm{r} 3-$ POZIOM PODSTAWOWY

l. Znalez$\acute{}$č największą wartośč funkcji

$f(x)=\displaystyle \frac{2}{\sqrt{4x^{2}-12x+11}}$

$\mathrm{i}$ rozwiqzač nierównośč $f(x)\geq 1.$

2. Rozwiązač równanie

$(1+\cos 4x)\sin 2x=\cos^{2}2x.$

3. Rozwiązač równanie

$\log_{\sqrt{5}}(4^{x}-6)-\log_{\sqrt{5}}(2^{x}-2)=2.$

4. Stosunek długości przekątnych rombu jest równy 5:l2. Obliczyč stosunek pola rombu do

do pola koła wpisanego $\mathrm{w}$ ten romb.

5. Dane są punkty $A(1,1)\mathrm{i}B(7,4)$. Na paraboli $y=x^{2}+x+3$ znalez/č taki punkt $C, \dot{\mathrm{z}}$ eby

pole trójkąta $ABC$ było najmniejsze. Wykonač rysunek.

6. Ramiona trójk$\Phi$ta równoramiennego zawarte $\mathrm{s}\Phi^{\mathrm{W}}$ prostych $0$ równaniach $8x-y+17=0$

oraz $4x+7y-59 = 0$, a jego podstawa przechodzi przez punkt $P(0,2)$. Wyznaczyč

równanie prostej zawierajacej podstawę $\mathrm{i}$ obliczyč pole tego trójkqta.
\end{document}
