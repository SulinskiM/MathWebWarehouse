\documentclass[a4paper,12pt]{article}
\usepackage{latexsym}
\usepackage{amsmath}
\usepackage{amssymb}
\usepackage{graphicx}
\usepackage{wrapfig}
\pagestyle{plain}
\usepackage{fancybox}
\usepackage{bm}

\begin{document}

L

KORESPONDENCYJNY KURS

Z MATEMATYKI

listopad 2020 r.

PRACA KONTROLNA nr 3- POZIOM PODSTAWOWY

l. Punkty $K\mathrm{i}L$ sa środkami boków AB $\mathrm{i}CD$ czworokąta ABCD. Wykaz, $\dot{\mathrm{z}}\mathrm{e}$

$\displaystyle \vec{KL}=\frac{1}{2}(\vec{AD}+\vec{BC}).$

Wykonaj rysunek.

2. $\mathrm{W}$ pewnym ciągu geometrycznym $\mathrm{k}\mathrm{a}\dot{\mathrm{z}}\mathrm{d}\mathrm{y}$ ($\mathrm{z}$ wyjątkiem pierwszego) wyraz jest róznicą

wyrazu następnego $\mathrm{i}$ poprzedniego. Znajd $\acute{\mathrm{z}}$ iloraz tego $\mathrm{c}\mathrm{i}_{\Phi \mathrm{g}}\mathrm{u}.$

3. Rozwiąz nierównośč

$[\log_{0,2}(x-1)]^{2}>4.$

4. Rozwiąz równanie

$\displaystyle \sin^{2}x+\frac{1}{2}\sin 2x=1.$

5. Statek plynie prosto $\mathrm{w}$ kierunku klifu. $K_{\Phi^{\mathrm{t}}}$ elewacji (kąt utworzony przez linię $\mathrm{P}^{\mathrm{o}\mathrm{z}\mathrm{i}\mathrm{o}\mathrm{m}}\Phi$

$\mathrm{i}$ odcinek fączący obserwatora na statku ze szczytem klifu) wynosi początkowo $\alpha$, ale po

przepłynięciu przez statek $d$ metrów wzrasta do $\beta$. Wyznacz wysokośč klifu. Wykonaj

obliczenia dla wartości $\alpha=10^{\mathrm{o}}, \beta=15^{\mathrm{o}}, d=50.$

6. Obliczyč pole cześci wspólnej trzech kól $0$ promieniach $r \mathrm{i}$ środkach $\mathrm{w}$ wierzchołkach

trójk$\Phi$ta równobocznego $0$ boku $r\sqrt{2}.$




PRACA KONTROLNA nr 3- POZIOM ROZSZERZONY

1. Znajd $\acute{\mathrm{z}}$ taki ciąg arytmetyczny, $\mathrm{w}$ którym suma pierwszych $n$ wyrazów równajest $n^{2}$ dla

wszystkich $n\in \mathbb{N}.$

2. $\mathrm{W}$ sześciok$\Phi$cie foremnym ABCDEF punkty $M \mathrm{i} N$ są środkami boków $CD \mathrm{i}$ {\it DE}.

Wyznacz $\mathrm{k}\mathrm{a}\mathrm{t}$ między wektorami $\vec{AM}\mathrm{i}\vec{BN}.$

3. Rozwiąz nierównośč

$\log_{2x}(x^{2}-5x+6)<1.$

4. Rozwiąz równanie

$\displaystyle \cos 2x-3\cos x=4\cos^{2}\frac{x}{2}.$

5. Znajd $\acute{\mathrm{z}}$ najmniejszą wartośč ilorazu pola powierzchni bocznej stozka $\mathrm{i}$ pola powierzchni

kuli wpisanej $\mathrm{w}$ ten stozek oraz kąt rozwarcia stozka realizujący tę wartośč najmniejszq.

6. Na dachu budynku stoi antena, której wysokośč chcemy wyznaczyč nie wchodząc na

górę. Urządzenie pomiarowe ustawione $\mathrm{w}$ pewnej odległości od budynku zmierzyło kąty

między pionem a odcinkiem $l_{\Phi}$czącym punkt pomiaru ze szczytem anteny oraz między

pionem a odcinkiem łączącym punkt pomiaru $\mathrm{z}$ podstawą anteny. Otrzymano kąty $\alpha_{1}$

$\mathrm{i}\beta_{1}$ odpowiednio. Nastepnie przesunięto urządzenie $0d$ metrów $\mathrm{w}$ kierunku budynku

bez zmiany wysokości punktu pomiarowego $\mathrm{i}$ ponowiono pomiary, otrzymując kąty $\alpha_{2}$

$\mathrm{i}\beta_{2}$. Podaj wzór na wysokośč anteny $\mathrm{i}$ wykonaj obliczenia dla kątów $\alpha_{1}=53^{\mathrm{o}}, \beta_{1}=55^{\mathrm{o}},$

$\alpha_{2}=51^{\mathrm{o}}, \beta_{2}=53.04^{\mathrm{o}}$, oraz $d=5m.$

Rozwiązania (rękopis) zadań z wybranego poziomu prosimy nadsyfač do

2020r. na adres:

20 1istopada

Wydziaf Matematyki

Politechnika Wrocfawska

Wybrzez $\mathrm{e}$ Wyspiańskiego 27

$50-370$ WROCLAW.

Na kopercie prosimy $\underline{\mathrm{k}\mathrm{o}\mathrm{n}\mathrm{i}\mathrm{e}\mathrm{c}\mathrm{z}\mathrm{n}\mathrm{i}\mathrm{e}}$ zaznaczyč wybrany poziom! (np. poziom podsta-

wowy lub rozszerzony). Do rozwiązań nalez $\mathrm{y}$ dołączyč zaadresowaną do siebie kopertę

zwrotną $\mathrm{z}$ naklejonym znaczkiem, odpowiednim do formatu listu. Polecamy stosowanie

kopert formatu C5 $(160\mathrm{x}230\mathrm{m}\mathrm{m})$ ze znaczkiem $0$ wartości 3,30 zł. Na $\mathrm{k}\mathrm{a}\dot{\mathrm{z}}$ dą wiekszą

kopertę nalez $\mathrm{y}$ nakleič $\mathrm{d}\mathrm{r}\mathrm{o}\dot{\mathrm{z}}$ szy znaczek. Prace niespełniające podanych warunków nie

będą poprawiane ani odsyłane.

Uwaga. Wysylając nam rozwiazania zadań uczestnik Kursu udostępnia Politechnice Wroclawskiej

swoje dane osobowe, które przetwarzamy wylącznie $\mathrm{w}$ zakresie niezbędnym do jego prowadzenia

(odesfanie zadań, prowadzenie statystyki). Szczegófowe informacje $0$ przetwarzaniu przez nas danych

osobowych są dostępne na stronie internetowej Kursu.

Adres internetowy Kursu: http://www. im.pwr.edu.pl/kurs



\end{document}