\documentclass[a4paper,12pt]{article}
\usepackage{latexsym}
\usepackage{amsmath}
\usepackage{amssymb}
\usepackage{graphicx}
\usepackage{wrapfig}
\pagestyle{plain}
\usepackage{fancybox}
\usepackage{bm}

\begin{document}

L

KORESPONDENCYJNY KURS

Z MATEMATYKI

listopad 2020 r.

PRACA KONTROLNA nr 3- POZIOM PODSTAWOWY

l. Punkty $K\mathrm{i}L$ sa środkami boków AB $\mathrm{i}CD$ czworokąta ABCD. Wykaz, $\dot{\mathrm{z}}\mathrm{e}$

$\displaystyle \vec{KL}=\frac{1}{2}(\vec{AD}+\vec{BC}).$

Wykonaj rysunek.

2. $\mathrm{W}$ pewnym ciągu geometrycznym $\mathrm{k}\mathrm{a}\dot{\mathrm{z}}\mathrm{d}\mathrm{y}$ ($\mathrm{z}$ wyjątkiem pierwszego) wyraz jest róznicą

wyrazu następnego $\mathrm{i}$ poprzedniego. Znajd $\acute{\mathrm{z}}$ iloraz tego $\mathrm{c}\mathrm{i}_{\Phi \mathrm{g}}\mathrm{u}.$

3. Rozwiąz nierównośč

$[\log_{0,2}(x-1)]^{2}>4.$

4. Rozwiąz równanie

$\displaystyle \sin^{2}x+\frac{1}{2}\sin 2x=1.$

5. Statek plynie prosto $\mathrm{w}$ kierunku klifu. $K_{\Phi^{\mathrm{t}}}$ elewacji (kąt utworzony przez linię $\mathrm{P}^{\mathrm{o}\mathrm{z}\mathrm{i}\mathrm{o}\mathrm{m}}\Phi$

$\mathrm{i}$ odcinek fączący obserwatora na statku ze szczytem klifu) wynosi początkowo $\alpha$, ale po

przepłynięciu przez statek $d$ metrów wzrasta do $\beta$. Wyznacz wysokośč klifu. Wykonaj

obliczenia dla wartości $\alpha=10^{\mathrm{o}}, \beta=15^{\mathrm{o}}, d=50.$

6. Obliczyč pole cześci wspólnej trzech kól $0$ promieniach $r \mathrm{i}$ środkach $\mathrm{w}$ wierzchołkach

trójk$\Phi$ta równobocznego $0$ boku $r\sqrt{2}.$
\end{document}
