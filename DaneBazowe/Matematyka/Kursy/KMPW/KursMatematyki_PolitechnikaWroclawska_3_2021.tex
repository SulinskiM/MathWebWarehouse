\documentclass[a4paper,12pt]{article}
\usepackage{latexsym}
\usepackage{amsmath}
\usepackage{amssymb}
\usepackage{graphicx}
\usepackage{wrapfig}
\pagestyle{plain}
\usepackage{fancybox}
\usepackage{bm}

\begin{document}

LI KORESPONDENCYJNY KURS

Z MATEMATYKI

listopad 2021 r.

PRACA KONTROLNA nr $3$- POZIOM PODSTAWOWY

l. Narysuj staranny wykres funkcji $f(x)=|\sin x|\cos x\mathrm{i}$ rozwiąz nierównośč $|f(x)|\displaystyle \leq\frac{1}{4}.$

2. Wyznacz dziedzinę funkcji

$f(x)=\displaystyle \log_{2}(\frac{3x-5}{x-2}+1)$

$\mathrm{i}$ sprawd $\acute{\mathrm{z}}$ dla jakich argumentów funkcja ta przyjmuje wartości dodatnie.

3. $\mathrm{W}$ trójkącie dane są dlugości dwóch boków a $\mathrm{i}b$. Oblicz długośč trzeciego boku, wiedząc,

$\dot{\mathrm{z}}\mathrm{e}$ suma wysokości poprowadzonych do boków $a\mathrm{i}b$ jest równa trzeciej wysokości.

4. Niech ABCDEF będzie sześciokątem foremnym. Wykaz$\cdot, \dot{\mathrm{z}}\mathrm{e}$

$\vec{AB}+\vec{AC}+\vec{AD}+\vec{AE}+\vec{AF}=3\vec{AD}.$

5. Na krzywej $0$ równaniu $y= \sqrt{2x}$znajd $\acute{\mathrm{z}}$ miejsce, które polozone jest najblizej punktu

$P(3,0).$ Sporząd $\acute{\mathrm{z}}$ rysunek.

6. Dla jakich wartości parametru $m$ pierwiastkiem wielomianu

$w(x)=2x^{3}-7x^{2}-(m^{2}-12)x+m^{2}+m-6$

jest $x=3$? Dla znalezionych wartości $m$ wyznacz pozostałe pierwiastki $w(x).$




PRACA KONTROLNA $\mathrm{n}\mathrm{r} 3-$ POZIOM ROZSZERZONY

l. Dany jest trójkąt $0$ wierzchofkach $A(-1,3), B(-4,-1), \mathrm{i}C(3,0).$ Znajd $\acute{\mathrm{z}}$ kąt pomiędzy

wysokością tego trójkąta poprowadzonq $\mathrm{z}$ wierzcholka $A\mathrm{i}$ bokiem $AC$. Oblicz pole tego

trójkąta.

2. Narysuj wykres funkcji $f(x)=\sin^{2}x-\cos 2x\mathrm{i}\mathrm{r}\mathrm{o}\mathrm{z}\mathrm{w}\mathrm{i}_{\Phi}\dot{\mathrm{z}}$ nierównośč $f(x)\displaystyle \geq-\frac{1}{4}.$

3. Zaznacz na płaszczyz$\acute{}$nie zbiór punktów, których współrzędne spełniajq nierównośč

$\log_{y}(\log_{x}y)>0.$

4. Reszta $\mathrm{z}$ dzielenia wielomianu $w(x)=x^{4}+ax^{3}+(b+2)x^{2}+bx+a-3$ przez trójmian

$x^{2}+2x-8$ wynosi $-5x+40$. Wyznacz wartośč parametrów $a\mathrm{i}b$ oraz rozwiąz nierównośč

$w(x-1)\geq w(x+1).$

5. Dany jest trapez ABCD $0$ podstawach AB $\mathrm{i}CD, \mathrm{w}$ którym $\angle ABC=90^{\mathrm{o}}$ Dwusieczna

kąta BAD przecina odcinek $BC\mathrm{w}$ punkcie $P$. Niech $Q$ będzie rzutem prostopadłym

punktu $P$ na prostą $AD$. Wykaz, $\dot{\mathrm{z}}\mathrm{e}\mathrm{j}\mathrm{e}\dot{\mathrm{z}}$ eli pole czworokqta APCD jest równe polu

trójk$\Phi$ta $ABP$, to $|PC|=|DQ|.$

6. Boisko do gry $\mathrm{w}$ piłkę recznq jest prostokątem $0$ długości $40\mathrm{m}\mathrm{i}$ szerokości $20\mathrm{m}$. Bramki

$\mathrm{m}\mathrm{a}\mathrm{j}_{\Phi}$ szerokośč $3\mathrm{m}\mathrm{i}$ stoją dokfadnie na środku linii bramkowej (krótszego boku pro-

stokąta). $\mathrm{Z}$ jakiego punktu linii bocznej (dłuzszego boku prostokąta) widač bramkę pod

najwiekszym $\mathrm{m}\mathrm{o}\dot{\mathrm{z}}$ liwym kątem?

Rozwiązania (rękopis) zadań z wybranego poziomu prosimy nadsyfač do

2021r. na adres:

201istopada

Wydziaf Matematyki

Politechnika Wrocfawska

Wybrzez $\mathrm{e}$ Wyspiańskiego 27

$50-370$ WROCLAW,

lub elektronicznie, za pośrednictwem portalu talent. $\mathrm{p}\mathrm{w}\mathrm{r}$. edu. pl

Na kopercie prosimy $\underline{\mathrm{k}\mathrm{o}\mathrm{n}\mathrm{i}\mathrm{e}\mathrm{c}\mathrm{z}\mathrm{n}\mathrm{i}\mathrm{e}}$ zaznaczyč wybrany poziom! (np. poziom podsta-

wowy lub rozszerzony). Do rozwiązań nalez $\mathrm{y}$ dołączyč zaadresowaną do siebie koperte

zwrotną $\mathrm{z}$ naklejonym znaczkiem, odpowiednim do formatu listu. Polecamy stosowanie

kopert formatu C5 $(160\mathrm{x}230\mathrm{m}\mathrm{m})$ ze znaczkiem $0$ wartości 3,30 zł. Na $\mathrm{k}\mathrm{a}\dot{\mathrm{z}}$ dą wiekszą

kopertę nalez $\mathrm{y}$ nakleič drozszy znaczek. Prace niespelniające podanych warunków nie

bedą poprawiane ani odsyłane.

Uwaga. Wysyłając nam rozwi\S zania zadań uczestnik Kursu udostępnia Politechnice Wroclawskiej

swoje dane osobowe, które przetwarzamy wyłącznie $\mathrm{w}$ zakresie niezbednym do jego prowadzenia

(odesfanie zadań, prowadzenie statystyki). Szczególowe informacje $0$ przetwarzaniu przez nas danych

osobowych sa dostępne na stronie internetowej Kursu.

Adres internetowy Kursu: http://www.im.pwr.edu.pl/kurs



\end{document}