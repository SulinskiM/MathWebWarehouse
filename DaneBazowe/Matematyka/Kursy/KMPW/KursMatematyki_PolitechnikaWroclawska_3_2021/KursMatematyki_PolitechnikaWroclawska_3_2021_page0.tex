\documentclass[a4paper,12pt]{article}
\usepackage{latexsym}
\usepackage{amsmath}
\usepackage{amssymb}
\usepackage{graphicx}
\usepackage{wrapfig}
\pagestyle{plain}
\usepackage{fancybox}
\usepackage{bm}

\begin{document}

LI KORESPONDENCYJNY KURS

Z MATEMATYKI

listopad 2021 r.

PRACA KONTROLNA nr $3$- POZIOM PODSTAWOWY

l. Narysuj staranny wykres funkcji $f(x)=|\sin x|\cos x\mathrm{i}$ rozwiąz nierównośč $|f(x)|\displaystyle \leq\frac{1}{4}.$

2. Wyznacz dziedzinę funkcji

$f(x)=\displaystyle \log_{2}(\frac{3x-5}{x-2}+1)$

$\mathrm{i}$ sprawd $\acute{\mathrm{z}}$ dla jakich argumentów funkcja ta przyjmuje wartości dodatnie.

3. $\mathrm{W}$ trójkącie dane są dlugości dwóch boków a $\mathrm{i}b$. Oblicz długośč trzeciego boku, wiedząc,

$\dot{\mathrm{z}}\mathrm{e}$ suma wysokości poprowadzonych do boków $a\mathrm{i}b$ jest równa trzeciej wysokości.

4. Niech ABCDEF będzie sześciokątem foremnym. Wykaz$\cdot, \dot{\mathrm{z}}\mathrm{e}$

$\vec{AB}+\vec{AC}+\vec{AD}+\vec{AE}+\vec{AF}=3\vec{AD}.$

5. Na krzywej $0$ równaniu $y= \sqrt{2x}$znajd $\acute{\mathrm{z}}$ miejsce, które polozone jest najblizej punktu

$P(3,0).$ Sporząd $\acute{\mathrm{z}}$ rysunek.

6. Dla jakich wartości parametru $m$ pierwiastkiem wielomianu

$w(x)=2x^{3}-7x^{2}-(m^{2}-12)x+m^{2}+m-6$

jest $x=3$? Dla znalezionych wartości $m$ wyznacz pozostałe pierwiastki $w(x).$
\end{document}
