\documentclass[a4paper,12pt]{article}
\usepackage{latexsym}
\usepackage{amsmath}
\usepackage{amssymb}
\usepackage{graphicx}
\usepackage{wrapfig}
\pagestyle{plain}
\usepackage{fancybox}
\usepackage{bm}

\begin{document}

LII

KORESPONDENCYJNY KURS

Z MATEMATYKI

listopad 2022 r.

PRACA KONTROLNA $\mathrm{n}\mathrm{r} 3-$ POZIOM PODSTAWOWY

1. $\mathrm{W}$ trójk$\Phi$cie $ABC$ wpisanym $\mathrm{w}$ okrąg $0$ środku $S\mathrm{i}$ promieniu $r$ dany jest kąt $\alpha=\angle ABC.$

Oblicz pole trójkqta $ASC.$

2. Rozwiąz równanie

$|\displaystyle \sin x|+|\cos x|=\frac{\sqrt{6}}{2}.$

3. Dana jest funkcja

$f(x)=\displaystyle \cos(2x-\frac{\pi}{6})$

Narysuj starannie wykres funkcji $f(x)$. Rozwiqz nierównośč $(f(x))^{2}\displaystyle \geq\frac{1}{2}.$

4. Niech $\alpha, \beta \mathrm{i}\gamma$ oznaczają kąty pewnego trójkąta. Wykaz, $\dot{\mathrm{z}}\mathrm{e}\mathrm{j}\mathrm{e}\dot{\mathrm{z}}$ eli

-ssiinn $\beta\alpha =$2 cos $\gamma$,

to ten trójkąt jest równoramienny.

5. Na okręgu $0$ promieniu $r$ opisano trapez prostokątny, którego najkrótszy bok jest równy

$\displaystyle \frac{4}{3}r$. Oblicz pole tego trapezu.

6. Pewną górę widač najpierw pod kątem $\alpha$ (jest to kąt między linią poziomą, a odcinkiem

lączącym szczyt $\mathrm{z}$ obserwatorem), a po przyblizeniu się do niej $\mathrm{o}d$ metrów widač $\mathrm{j}\mathrm{a}$ pod

nieco większym kątem $\beta$. Wyznaczyč względną wysokośč tej góry. Wykonač obliczenia

dla wartości $\alpha=41^{\mathrm{o}}, \beta=45^{\mathrm{o}}, d=90\mathrm{m}.$




PRACA KONTROLNA $\mathrm{n}\mathrm{r} 3-$ POZIOM ROZSZERZONY

l. Udowodnij, $\dot{\mathrm{z}}\mathrm{e}$

$\cos 4x=1-8\cos^{2}x+8\cos^{4}x.$

Wykorzystując ten wzór, znajd $\acute{\mathrm{z}}$ wartośč $\displaystyle \cos\frac{\pi}{24}.$

2. Wykaz$\cdot, \dot{\mathrm{z}}\mathrm{e}$ dla $\mathrm{k}\mathrm{a}\dot{\mathrm{z}}$ dego trójkata zachodzi nierównośč

-21  {\it r} $<$ -{\it h}1{\it a} $+$ -{\it h}1{\it b} $<$ -{\it r}1,

gdzie $h_{a}, h_{b}$ sq wysokościami, a $r$ promieniem okregu wpisanego $\mathrm{w}$ ten trójkąt.

3. Dana jest funkcja $f(x) = \sin 4x$ ctg $2x-\displaystyle \frac{1}{2}$. Rozwiąz nierównośč $f(x) \geq$

staranny wykres $f(x).$

l i narysuj

4. Przekątne trapezu dzielą ten trapez na cztery trójkąty. Pola tych dwóch trójkątów, któ-

rych bokami są podstawy trapezu równe są $S_{a}\mathrm{i}S_{b}$. Oblicz pole tego trapezu.

5. Manipulator robota składa się $\mathrm{z}$ dwóch ramion $0$ długościach $l_{1}\mathrm{i}l_{2}$, połączonych prze-

gubem. Pierwsze ramię umieszczono $\mathrm{w}$ poczatku układu wspófrzednych.

Niech $\alpha$ oznacza kąt miedzy pierwszym ramieniem $\mathrm{i}$ osią

$Ox$, a $\beta$ - kąt mi dzy drugim ramieniem $\mathrm{i}$ kierunkiem

pierwszego ramienia (patrz rysunek). Wyznacz wspól-

rzędne końca drugiego ramienia (punktu $P$) $\mathrm{w}$ zalezno-

sci od $\mathrm{k}$ tow $\alpha \mathrm{i}\beta$. Sprawdz, czy punkt $P\mathrm{m}\mathrm{o}\dot{\mathrm{z}}\mathrm{e}$ przesu-

$\mathrm{n} \mathrm{c}$ si do punktow $S(2,1)$ oraz $Q(3,-1)\mathrm{j}\mathrm{e}\dot{\mathrm{z}}$ eli $l_{1} =3,$

$l_{2} = 2$ oraz ruchy manipulatora ograniczone są $\mathrm{t}\mathrm{a}\mathrm{k}, \dot{\mathrm{z}}\mathrm{e}$

$\alpha, \beta\in -\displaystyle \frac{2\pi}{3}, \displaystyle \frac{2\pi}{3} \mathrm{J}\mathrm{e}\dot{\mathrm{z}}$ eli $\mathrm{t}\mathrm{a}\mathrm{k}$, to wskaz konkretne $\mathrm{k}$ ty $\alpha \mathrm{i}$

$\beta$ (podaj przyblizenia, jesli nie $\mathrm{m}\mathrm{o}\dot{\mathrm{z}}$ na okreslic dokladnej

ich wartości), a jeśli $\mathrm{n}\mathrm{i}\mathrm{e}$, to uzasadnij dlaczego.
\begin{center}
\includegraphics[width=60.048mm,height=45.720mm]{./KursMatematyki_PolitechnikaWroclawska_3_2022_page1_images/image001.eps}
\end{center}
$y$

{\it P}

2

$l_{2}$

$\prime\prime\beta_{\rightarrow}$

1

$l_{1}$

$\alpha$

1 $2 3 4 5 x$

$-1$

$-2$

6. Okrąg $0$ promieniu $r$ toczy się wewnętrznie bez poślizgu po okręgu $0$ promieniu $2r$. Ja-

ką linię zakreśla ustalony (dowolnie wybrany) punkt $P$ ruchomego okręgu? Wskazówka:

rozwaz dwa rózne pofozenia mniejszego okręgu $\mathrm{i}\mathrm{s}$prawd $\acute{\mathrm{z}}$ gdzie przesuwa się punkt stycz-

ności, skorzystaj ze związku między dlugością łuku, kqtem środkowym opartym na tym

fuku $\mathrm{i}$ promieniem okręgu.

Rozwiązania (rękopis) zadań $\mathrm{z}$ wybranego poziomu prosimy nadsylač do $20.11.2022\mathrm{r}$. na

adres:

Wydziaf Matematyki

Politechnika Wrocfawska

Wybrzez $\mathrm{e}$ Wyspiańskiego 27

$50-370$ WROCLAW,

lub elektronicznie, za pośrednictwem portalu talent. $\mathrm{p}\mathrm{w}\mathrm{r}$. edu. pl

Na kopercie prosimy $\underline{\mathrm{k}\mathrm{o}\mathrm{n}\mathrm{i}\mathrm{e}\mathrm{c}\mathrm{z}\mathrm{n}\mathrm{i}\mathrm{e}}$ zaznaczyč wybrany poziom! (np. poziom podsta-

wowy $\mathrm{l}\mathrm{u}\mathrm{b}$ rozszerzony). Do rozwiązań nalez $\mathrm{y}$ dolączyč zaadresowaną do siebie kopertę

zwrotną $\mathrm{z}$ naklejonym znaczkiem, odpowiednim do formatu listu. Prace niespełniające

podanych warunków nie będą poprawiane ani odsyłane.

Uwaga. Wysyfaj\S c nam rozwi\S zania zadań uczestnik Kursu udostępnia Politechnice Wrocfawskiej

swoje dane osobowe, które przetwarzamy wyłącznie $\mathrm{w}$ zakresie niezbędnym do jego prowadzenia

(odeslanie zadań, prowadzenie statystyki). Szczególowe informacje $0$ przetwarzaniu przez nas danych

osobowych s\S dostępne na stronie internetowej Kursu.

Adres internetowy Kursu: http: //www. im. pwr. edu. pl/kurs



\end{document}