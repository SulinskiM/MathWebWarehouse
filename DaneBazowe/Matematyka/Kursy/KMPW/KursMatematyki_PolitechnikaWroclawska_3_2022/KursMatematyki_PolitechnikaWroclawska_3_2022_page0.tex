\documentclass[a4paper,12pt]{article}
\usepackage{latexsym}
\usepackage{amsmath}
\usepackage{amssymb}
\usepackage{graphicx}
\usepackage{wrapfig}
\pagestyle{plain}
\usepackage{fancybox}
\usepackage{bm}

\begin{document}

LII

KORESPONDENCYJNY KURS

Z MATEMATYKI

listopad 2022 r.

PRACA KONTROLNA $\mathrm{n}\mathrm{r} 3-$ POZIOM PODSTAWOWY

1. $\mathrm{W}$ trójk$\Phi$cie $ABC$ wpisanym $\mathrm{w}$ okrąg $0$ środku $S\mathrm{i}$ promieniu $r$ dany jest kąt $\alpha=\angle ABC.$

Oblicz pole trójkqta $ASC.$

2. Rozwiąz równanie

$|\displaystyle \sin x|+|\cos x|=\frac{\sqrt{6}}{2}.$

3. Dana jest funkcja

$f(x)=\displaystyle \cos(2x-\frac{\pi}{6})$

Narysuj starannie wykres funkcji $f(x)$. Rozwiqz nierównośč $(f(x))^{2}\displaystyle \geq\frac{1}{2}.$

4. Niech $\alpha, \beta \mathrm{i}\gamma$ oznaczają kąty pewnego trójkąta. Wykaz, $\dot{\mathrm{z}}\mathrm{e}\mathrm{j}\mathrm{e}\dot{\mathrm{z}}$ eli

-ssiinn $\beta\alpha =$2 cos $\gamma$,

to ten trójkąt jest równoramienny.

5. Na okręgu $0$ promieniu $r$ opisano trapez prostokątny, którego najkrótszy bok jest równy

$\displaystyle \frac{4}{3}r$. Oblicz pole tego trapezu.

6. Pewną górę widač najpierw pod kątem $\alpha$ (jest to kąt między linią poziomą, a odcinkiem

lączącym szczyt $\mathrm{z}$ obserwatorem), a po przyblizeniu się do niej $\mathrm{o}d$ metrów widač $\mathrm{j}\mathrm{a}$ pod

nieco większym kątem $\beta$. Wyznaczyč względną wysokośč tej góry. Wykonač obliczenia

dla wartości $\alpha=41^{\mathrm{o}}, \beta=45^{\mathrm{o}}, d=90\mathrm{m}.$
\end{document}
