\documentclass[a4paper,12pt]{article}
\usepackage{latexsym}
\usepackage{amsmath}
\usepackage{amssymb}
\usepackage{graphicx}
\usepackage{wrapfig}
\pagestyle{plain}
\usepackage{fancybox}
\usepackage{bm}

\begin{document}

XLIV

KORESPONDENCYJNY KURS

Z MATEMATYKI

grudzień 2014 r.

PRACA KONTROLNA $\mathrm{n}\mathrm{r} 4-$ POZIOM PODSTAWOWY

l. Dla jakich kątów $\alpha\in\langle 0,  2\pi\rangle$ równanie $2x^{2}-2(2\cos\alpha-1)x+2\cos^{2}\alpha-5\cos\alpha+2=0$

ma dwa rózne pierwiastki rzeczywiste?

2. Dane są punkty $A(-2,0), B(2,4)$ oraz $C(1,5)$. Oblicz pole trapezu ABCD, wiedząc, $\dot{\mathrm{z}}\mathrm{e}$

punkt $D$ jest jednakowo odległy od punktów A $\mathrm{i}B.$

3. $\mathrm{W}$ trójkącie równoramiennym kąt przy podstawie ma miarę $30^{\mathrm{o}}$ Oblicz stosunek dfu-

gości promienia okręgu opisanego na trójkącie do długości promienia okręgu wpisanego

$\mathrm{w}$ trójkąt.

4. Płaszczyzna przechodząca przez środek dolnej podstawy walca jest nachylona do pod-

stawy pod kątem $\alpha \mathrm{i}$ przecina górną podstawe walca wzdłuz cięciwy dlugości $a$. Cięciwa

ta odcina $\mathrm{f}\mathrm{u}\mathrm{k}$, na którym oparty jest $\mathrm{k}_{\Phi^{\mathrm{t}}}$ środkowy $0$ mierze $120^{\mathrm{o}}$ Oblicz objętośč walca.

5. Niech $x_{1}\mathrm{i}x_{2}$ będą pierwiastkami wielomianu $p(x)=x^{2}-x+a$, a $x_{3}\mathrm{i}x_{4}-$ pierwiastkami

wielomianu $q(x)=x^{2}-4x+b$. Dlajakich $a\mathrm{i}b$ liczby $x_{1}, x_{2}, x_{3}, x_{4}$ są kolejnymi wyrazami

ciągu geometrycznego?

6. Na dwóch zewnętrznie stycznych kulach opisano stozek $\mathrm{t}\mathrm{a}\mathrm{k}, \dot{\mathrm{z}}\mathrm{e}$ środki tych kul lezą na

wysokości stozka. Promień mniejszej kulijest równy $r$, a stosunek objętości kul wynosi 8.

Oblicz pole powierzchni bocznej stozka.
\end{document}
