\documentclass[a4paper,12pt]{article}
\usepackage{latexsym}
\usepackage{amsmath}
\usepackage{amssymb}
\usepackage{graphicx}
\usepackage{wrapfig}
\pagestyle{plain}
\usepackage{fancybox}
\usepackage{bm}

\begin{document}

PRACA KONTROLNA nr 4- POZ1OM ROZSZERZONY

l. Dane są proste $y = 4x \mathrm{i} y = x-2$ oraz punkt $M = (1,2)$. Wyznacz współrzędne

punktów $A\mathrm{i}B\mathrm{l}\mathrm{e}\dot{\mathrm{z}}$ ących odpowiednio na danych prostych takich, $\dot{\mathrm{z}}\mathrm{e}$ punkty $A, B,  M\mathrm{s}\Phi$

współliniowe oraz $\displaystyle \frac{|AM|}{|BM|}=\frac{2}{3}.$

2. $\mathrm{W}$ równoległoboku $0$ kącie ostrym $60^{\mathrm{o}}$ stosunek kwadratów długości przekątnych wynosi

1:3. Oblicz stosunek dlugości dwóch sąsiednich boków.

3. Niech $a, b, c, d$ będą kolejnymi liczbami naturalnymi. Pokaz, $\dot{\mathrm{z}}\mathrm{e}$ wielomian $w(x)=ax^{3}-$

$bx^{2}-cx+d$ ma trzy pierwiastki rzeczywiste, wśród których co najmniej jeden jest liczbą

cafkowitą. Dla jakich parametrów $a, b, c, d$ suma tych pierwiastków jest największa?

4. Dla jakich kątów $\alpha\in\langle 0,  2\pi\rangle$ spełniona jest nierównośč

$2^{\sin^{2}x}+\sqrt[4]{2}\cdot 2^{\cos^{2}x}\leq\sqrt{2}+\sqrt[4]{8}$?

5. $\mathrm{W}$ ostrosfupie prawidfowym czworokątnym $0$ krawędzi podstawy $a$ stosunek dlugości

krawędzi podstawy do wysokości wynosi 2:3. Ostrosłup przecięto p1aszczyzna przecho-

dzącą przez krawędz/ podstawy $\mathrm{i}$ prostopadlą do przeciwleglej ściany bocznej. Oblicz pole

otrzymanego przekroju.

6. Wierzchołek stozka jest środkiem kuli a brzeg podstawy stozka zawiera $\mathrm{s}\mathrm{i}\mathrm{e}\mathrm{w}$ powierzchni

kuli. Pole powierzchni calkowitej stozka stanowi $\displaystyle \frac{1}{4}$ pola powierzchni kuli. Oblicz stosunek

objętości stozka do objętości kuli.

Rozwiązania (rękopis) zadań z wybranego poziomu prosimy nadsylač do

na adres:

18 grudnia 20l4r.

Instytut Matematyki $\mathrm{i}$ Informatyki

Politechniki Wrocfawskiej

Wybrzez $\mathrm{e}$ Wyspiańskiego 27

$50-370$ WROCLAW.

Na kopercie prosimy $\underline{\mathrm{k}\mathrm{o}\mathrm{n}\mathrm{i}\mathrm{e}\mathrm{c}\mathrm{z}\mathrm{n}\mathrm{i}\mathrm{e}}$ zaznaczyč wybrany poziom! (np. poziom podsta-

wowy lub rozszerzony). Do rozwiązań nalez $\mathrm{y}$ dołączyč zaadresowaną do siebie kopertę

zwrotną $\mathrm{z}$ naklejonym znaczkiem, odpowiednim do wagi listu. Prace niespełniające po-

danych warunków nie będą poprawiane ani odsyłane.

Adres internetowy Kursu: http://www.im.pwr.wroc.pl/kurs
\end{document}
