\documentclass[a4paper,12pt]{article}
\usepackage{latexsym}
\usepackage{amsmath}
\usepackage{amssymb}
\usepackage{graphicx}
\usepackage{wrapfig}
\pagestyle{plain}
\usepackage{fancybox}
\usepackage{bm}

\begin{document}

XLV

KORESPONDENCYJNY KURS

Z MATEMATYKI

grudzień 2015 r.

PRACA KONTROLNA $\mathrm{n}\mathrm{r} 4-$ POZIOM PODSTAWOWY

l. Znalez$\acute{}$č miejsca zerowe $\mathrm{i}$ naszkicowač wykres funkcji $f(x)=x^{2}-x-5|x|+5$. Wyznaczyč

najmniejszą $\mathrm{i}$ największ$\Phi$ wartośč tej funkcji na przedziale [-5, 5].

2. Romb $0$ boku $a\mathrm{i}$ kącie ostrym $\alpha$ podzielono na trzy części $0$ równych polach odcinka-

mi majacymi wspólny początek $\mathrm{w}$ wierzchofku kąta ostrego $\mathrm{i}$ końce na bokach rombu.

Obliczyč dlugości tych odcinków. Wykonač odpowiedni rysunek.

3. Odcinek $0$ końcach $A(-1,-1) \mathrm{i} B(3,2)$ jest podstawą trapezu. Druga podstawa jest

trzy razy dluzsza $\mathrm{i}$ ma środek $\mathrm{w}$ punkcie $P(1,5)$. Wyznaczyč wspófrzędne pozostafych

wierzchołków trapezu $\mathrm{i}$ obliczyč jego pole.

4. $\mathrm{W}$ okrqg $0$ promieniu l wpisujemy trójkąt równoboczny $\mathrm{i}$ zakreślamy odcinki kofa, które

$ 1\mathrm{e}\mathrm{Z}\otimes$ na zewnatrz trójkąta. $\mathrm{W}$ otrzymany trójkąt wpisujemy okrąg $\mathrm{i}$ powtarzamy proce-

durę, zaznaczając za $\mathrm{k}\mathrm{a}\dot{\mathrm{z}}$ dym razem odcinki kolejnych kół znajdujące się poza kolejnym

trójk$\Phi$tem. Obliczyč pole zaznaczonego obszaru po sześciu krokach, czyli po narysowaniu

sześciu trójkątów.

5. Sześcian podzielono na dwie bryły plaszczyznq przechodzącą przez krawęd $\acute{\mathrm{z}}$ podsta-

wy. Jedna częśč ma 5, a druga 6 ścian. Po1e powierzchni ca1kowitej bry1y, która ma 5

ścianjest równa połowie pola powierzchni sześcianu. Wyznaczyč tangens kąta nachylenia

płaszczyzny dzielącej sześcian do pfaszczyzny podstawy.

6. Rozwazamy zbiór liczb cafkowitych dodatnich równych co najwyzej l800, które nie dzielą

się ani przez 5 ani przez 6. Ob1iczyč sumę 1iczb $\mathrm{z}$ tego zbioru. Ile $\mathrm{w}$ tym zbiorze jest liczb

parzystych, a ile nieparzystych?




PRACA KONTROLNA nr 4- POZ1OM ROZSZERZONY

l. Punkty $A(2,0)\mathrm{i}B(0,2)$ są wierzchołkami podstawy trójkąta równoramiennego. Znalez/č

wspófrzędne wierzchołka $C, \mathrm{w}\mathrm{i}\mathrm{e}\mathrm{d}\mathrm{z}\Phi^{\mathrm{C}}, \dot{\mathrm{z}}\mathrm{e}$ środkowe $AD\mathrm{i}$ {\it BE} $\mathrm{s}\Phi$ prostopadle.

2. Trzy pierwiastki wielomianu $0$ współczynnikach całkowitych tworzą ciag arytmetyczny.

Suma tych pierwiastków jest równa 21, a i1oczyn 315. Pokazač, $\dot{\mathrm{z}}\mathrm{e}$ wartośč wielomianu

$\mathrm{w}$ dowolnym punkcie, który jest liczbą nieparzystą, jest podzielna przez 48.

3. $\mathrm{W}$ trójkącie równobocznym $0$ boku długości $a$ przeprowadzamy prostą przechodząca

przez środek wysokości $\mathrm{n}\mathrm{a}\mathrm{c}\mathrm{h}\mathrm{y}\mathrm{l}\mathrm{o}\mathrm{n}\Phi$ do niej pod $\mathrm{k}_{\Phi}\mathrm{t}\mathrm{e}\mathrm{m}30^{\mathrm{o}}$ Odcina ona od trójk$\Phi$ta trapez.

Obliczyč pole $\mathrm{i}$ obwód tego trapezu oraz objetośč $\mathrm{i}$ pole powierzchni bryfy powstafej $\mathrm{z}$

jego obrotu dookoła dłuzszej podstawy.

4. $\mathrm{W}$ trójk$\Phi$t równoboczny $0$ boku dlugości l wpisano kwadrat. Następnie $\mathrm{w}$ pozostalą częśč

(nad kwadratem) znów wpisano kwadrat, itd. Jaką dlugośč ma bok kwadratu $\mathrm{w}n$-tym

kroku? Podač wzór ciągu $P_{n}$ określającego sumę pól wpisanych kwadratów po $n$ krokach,

a następnie obliczyč jego granicę.

5. $\mathrm{W}$ okrąg $0$ promieniu $r$ wpisano trapez, którego podstawą jest średnica okręgu. Dla

jakiego kąta przy podstawie pole trapezu jest największe?

6. Znalez/č dziedzinę oraz przedzialy monotoniczności funkcji

$f(x)=1+\displaystyle \frac{2x}{x^{2}-3}+(\frac{2x}{x^{2}-3})^{2}+\ldots.$

Naszkicowač wykres tej funkcji oraz zbadač liczbę rozwi$\Phi$zań równania $f(x) = m \mathrm{w}$

zalezności od parametru $m.$

Rozwiązania (rękopis) zadań z wybranego poziomu prosimy nadsylač do

na adres:

18 grudnia 20l5r.

Wydziaf Matematyki

Politechnika Wrocfawska

Wybrzez $\mathrm{e}$ Wyspiańskiego 27

$50-370$ WROCLAW.

Na kopercie prosimy $\underline{\mathrm{k}\mathrm{o}\mathrm{n}\mathrm{i}\mathrm{e}\mathrm{c}\mathrm{z}\mathrm{n}\mathrm{i}\mathrm{e}}$ zaznaczyč wybrany poziom! (np. poziom podsta-

wowy lub rozszerzony). Do rozwiązań nalez $\mathrm{y}$ dołączyč zaadresowana do siebie kopertę

zwrotną $\mathrm{z}$ naklejonym znaczkiem, odpowiednim do wagi listu. Prace niespełniające po-

danych warunków nie będą poprawiane ani odsyłane.

Adres internetowy Kursu: http://www.im.pwr.wroc.pl/kurs



\end{document}