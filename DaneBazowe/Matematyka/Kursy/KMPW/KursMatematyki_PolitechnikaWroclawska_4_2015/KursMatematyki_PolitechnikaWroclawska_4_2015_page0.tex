\documentclass[a4paper,12pt]{article}
\usepackage{latexsym}
\usepackage{amsmath}
\usepackage{amssymb}
\usepackage{graphicx}
\usepackage{wrapfig}
\pagestyle{plain}
\usepackage{fancybox}
\usepackage{bm}

\begin{document}

XLV

KORESPONDENCYJNY KURS

Z MATEMATYKI

grudzień 2015 r.

PRACA KONTROLNA $\mathrm{n}\mathrm{r} 4-$ POZIOM PODSTAWOWY

l. Znalez$\acute{}$č miejsca zerowe $\mathrm{i}$ naszkicowač wykres funkcji $f(x)=x^{2}-x-5|x|+5$. Wyznaczyč

najmniejszą $\mathrm{i}$ największ$\Phi$ wartośč tej funkcji na przedziale [-5, 5].

2. Romb $0$ boku $a\mathrm{i}$ kącie ostrym $\alpha$ podzielono na trzy części $0$ równych polach odcinka-

mi majacymi wspólny początek $\mathrm{w}$ wierzchofku kąta ostrego $\mathrm{i}$ końce na bokach rombu.

Obliczyč dlugości tych odcinków. Wykonač odpowiedni rysunek.

3. Odcinek $0$ końcach $A(-1,-1) \mathrm{i} B(3,2)$ jest podstawą trapezu. Druga podstawa jest

trzy razy dluzsza $\mathrm{i}$ ma środek $\mathrm{w}$ punkcie $P(1,5)$. Wyznaczyč wspófrzędne pozostafych

wierzchołków trapezu $\mathrm{i}$ obliczyč jego pole.

4. $\mathrm{W}$ okrqg $0$ promieniu l wpisujemy trójkąt równoboczny $\mathrm{i}$ zakreślamy odcinki kofa, które

$ 1\mathrm{e}\mathrm{Z}\otimes$ na zewnatrz trójkąta. $\mathrm{W}$ otrzymany trójkąt wpisujemy okrąg $\mathrm{i}$ powtarzamy proce-

durę, zaznaczając za $\mathrm{k}\mathrm{a}\dot{\mathrm{z}}$ dym razem odcinki kolejnych kół znajdujące się poza kolejnym

trójk$\Phi$tem. Obliczyč pole zaznaczonego obszaru po sześciu krokach, czyli po narysowaniu

sześciu trójkątów.

5. Sześcian podzielono na dwie bryły plaszczyznq przechodzącą przez krawęd $\acute{\mathrm{z}}$ podsta-

wy. Jedna częśč ma 5, a druga 6 ścian. Po1e powierzchni ca1kowitej bry1y, która ma 5

ścianjest równa połowie pola powierzchni sześcianu. Wyznaczyč tangens kąta nachylenia

płaszczyzny dzielącej sześcian do pfaszczyzny podstawy.

6. Rozwazamy zbiór liczb cafkowitych dodatnich równych co najwyzej l800, które nie dzielą

się ani przez 5 ani przez 6. Ob1iczyč sumę 1iczb $\mathrm{z}$ tego zbioru. Ile $\mathrm{w}$ tym zbiorze jest liczb

parzystych, a ile nieparzystych?
\end{document}
