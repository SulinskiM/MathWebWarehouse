\documentclass[a4paper,12pt]{article}
\usepackage{latexsym}
\usepackage{amsmath}
\usepackage{amssymb}
\usepackage{graphicx}
\usepackage{wrapfig}
\pagestyle{plain}
\usepackage{fancybox}
\usepackage{bm}

\begin{document}

XLVI

KORESPONDENCYJNY KURS

Z MATEMATYKI

grudzień 2016 r.

PRACA KONTROLNA $\mathrm{n}\mathrm{r} 4-$ POZIOM PODSTAWOWY

l. Dwa samochody wyjechałyjednocześnie zjednego miejsca ijada $\mathrm{w}$ tym samym kierunku.

Pierwszyjedzie $\mathrm{z}$ prędkością 50 $\mathrm{k}\mathrm{m}/\mathrm{h}$, a drugi $\mathrm{z}$ prędkością 40 $\mathrm{k}\mathrm{m}/\mathrm{h}$. Pół godziny póz$\acute{}$niej

$\mathrm{z}$ tego samego miejsca $\mathrm{i}\mathrm{w}$ tym samym kierunku wyruszyl trzeci samochód, który dopędził

pierwszy samochód $0 1$ godzinę $\mathrm{i}30$ minut póz/niej $\mathrm{n}\mathrm{i}\dot{\mathrm{z}}$ drugi. $\mathrm{Z}$ jaka prędkościq jechaf

trzeci samochód?

2. Proste $y = 2, y = 2x+10$ oraz $4x+3y = 0$ wyznaczają trójkąt $ABC$. Otrzymany

trójk$\Phi$t przeksztafcono $\mathrm{u}\dot{\mathrm{z}}$ ywaj $\Phi^{\mathrm{C}}$ najpierw jednokładności $0$ środku $O(0,0)\mathrm{i}$ skali $k=3,$

a następnie symetrii względem osi $OX$. Wyznaczyč współrzędne trójkąta $ABC$ oraz

wspófrzędne obrazów jego wierzcholków. Obliczyč pole trójkqta $ABC\mathrm{i}$ jego obrazu $\mathrm{w}$

tym przeksztalceniu.

3. Rozwazmy zbiór wszystkich prostokątów wpisanych $\mathrm{w}$ kwadrat $0$ boku dlugości $a\mathrm{w}$ taki

sposób, $\dot{\mathrm{z}}\mathrm{e}$ boki tego prostokąta $\mathrm{s}\Phi$ parami równolegfe do przekątnych danego kwadratu.

Obliczyč długości boków tego prostokąta, który ma największe pole.

4. Podstawa trójkqta równobocznego jest średnica koła $0$ promieniu $r$. Obliczyč stosunek

pola powierzchni części trójk$\Phi$ta lezącej na $\mathrm{z}\mathrm{e}\mathrm{w}\mathrm{n}\Phi \mathrm{t}\mathrm{r}\mathrm{z}$ kofa do pola powierzchni części

trójkąta lezącej wewnątrz kola.

5. $\mathrm{W}$ stozku pole podstawy, pole powierzchni kuli wpisanej $\mathrm{w}$ ten stozek $\mathrm{i}$ pole powierzchni

bocznej stozka, tworzą ciag arytmetyczny. Znalez/č cosinus kąta nachylenia tworzqcej

stozka do plaszczyzny jego podstawy.

6. $\mathrm{O}\mathrm{k}\mathrm{r}\Phi \mathrm{g}O_{1}\mathrm{o}$ promieniu l jest styczny do ramion kąta $0$ mierze $\displaystyle \frac{\pi}{3}$. Mniejszy od niego okrąg

$O_{2}$ jest styczny zewnętrznie do niego $\mathrm{i}$ obu ramion tego kąta. Procedurę kontynuujemy.

Znalez$\acute{}$č sumę obwodów pieciu otrzymanych kolejno $\mathrm{w}$ ten sposób okręgów. Dla jakiego

$n$ suma obwodów $\mathrm{c}\mathrm{i}_{\Phi \mathrm{g}}\mathrm{u}$ tych okręgów jest większa od $\displaystyle \frac{299}{100}\pi$?




PRACA KONTROLNA nr 4- POZ1OM ROZSZERZONY

l. Do punktu $A$ po dwóch prostoliniowych drogach jada ze stałymi prędkościami samochód

$\mathrm{i}$ rower. $\mathrm{W}$ chwili początkowej samochód, rower $\mathrm{i}$ punkt $ A\mathrm{t}\mathrm{w}\mathrm{o}\mathrm{r}\mathrm{z}\Phi$ trójk$\Phi$t $\mathrm{p}\mathrm{r}\mathrm{o}\mathrm{s}\mathrm{t}\mathrm{o}\mathrm{k}_{\Phi^{\mathrm{t}}}\mathrm{n}\mathrm{y}.$

Gdy samochód przejechał 25 km trójkąt, którego dwa wierzchołki przesunęły się, stał

się trójkatem równobocznym. Znalez$\acute{}$č odległośč między samochodem a rowerem $\mathrm{w}$ chwili

$\mathrm{P}^{\mathrm{O}\mathrm{C}\mathrm{Z}}\Phi^{\mathrm{t}\mathrm{k}\mathrm{o}\mathrm{w}\mathrm{e}\mathrm{j}}$, jeśli $\mathrm{w}$ momencie dotarcia samochodu do punktu $A$ rower miaf jeszcze do

przejechania 12 km.

2. Na pfaszczy $\acute{\mathrm{z}}\mathrm{n}\mathrm{i}\mathrm{e}$ dane $\mathrm{s}\Phi$ punkty A $\mathrm{i}B$. Udowodnij, $\dot{\mathrm{z}}\mathrm{e}$ złozenie symetrii środkowej wzglę-

dem punktu $A\mathrm{z}$ przesunięciem $0$ wektor $\overline{A}B$ jest symetrią środkową względem środka

odcinka $\overline{AB}.$

3. Wyznaczyč największą wartośč pola $\mathrm{P}^{\mathrm{r}\mathrm{o}\mathrm{s}\mathrm{t}\mathrm{o}\mathrm{k}}\Phi^{\mathrm{t}\mathrm{a}}$, którego dwa wierzchofki $\mathrm{l}\mathrm{e}\dot{\mathrm{z}}$ ą na paraboli

$y=x^{2}-4x+4$, a dwa pozostale na cięciwie paraboli wyznaczonej przez prostą $y=3.$

4. Suma trzech początkowych wyrazów nieskończonego ciągu geometrycznego wynosi 6, $\mathrm{a}$

suma $S$ wszystkich wyrazów tego ciągu równa się $\displaystyle \frac{16}{\mathrm{s}}$. Dlajakich $n$ naturalnych spefniona

jest nierównośč $|S-S_{n}|<\displaystyle \frac{1}{96}$?

5. Dwa jednakowe stozki zfozono podstawami. Obliczyč objętośč powstafej bryfy, jeśli pro-

mień kuli wpisanej $\mathrm{w}$ tę bryłę wynosi $R$, a punkt styczności kuli $\mathrm{i}$ stozka dzieli tworzącą

stozka $\mathrm{w}$ stosunku $m$ do $n$?

6. $\mathrm{W}$ czworościan foremny ABCD $0$ krawędzi dlugości $d$ wpisano kulę. Prowadzimy pfasz-

czyzny równoległe do ścian czworościanu $\mathrm{i}$ styczne do wpisanej kuli odcinając $\mathrm{w}$ ten

sposób cztery $\mathrm{p}\mathrm{r}\mathrm{z}\mathrm{y}\mathrm{s}\mathrm{t}\mathrm{a}\mathrm{j}_{\Phi}\mathrm{c}\mathrm{e}$ czworościany foremne. $\mathrm{W}\mathrm{k}\mathrm{a}\dot{\mathrm{z}}\mathrm{d}\mathrm{y}\mathrm{z}$ nich wpisujemy kulę $\mathrm{i}$ po-

stępujemy analogiczniejak $\mathrm{z}$ kulą wpisaną $\mathrm{w}$ czworościan ABCD. Obliczyč sumę objęto-

ści wszystkich kul wpisanych $\mathrm{w}$ otrzymane czworościany, jeśli proces ten kontynuujemy

nieskończenie wiele razy.

Rozwiązania (rekopis) zadań z wybranego poziomu prosimy nadsyłač do

na adres:

18 grudnia 20l6r.

Wydziaf Matematyki

Politechnika Wrocfawska

Wybrzez $\mathrm{e}$ Wyspiańskiego 27

$50-370$ WROCLAW.

Na kopercie prosimy $\underline{\mathrm{k}\mathrm{o}\mathrm{n}\mathrm{i}\mathrm{e}\mathrm{c}\mathrm{z}\mathrm{n}\mathrm{i}\mathrm{e}}$ zaznaczyč wybrany poziom! (np. poziom podsta-

wowy lub rozszerzony). Do rozwiązań nalez $\mathrm{y}$ dołączyč zaadresowana do siebie kopertę

zwrotną $\mathrm{z}$ naklejonym znaczkiem, odpowiednim do wagi listu. Prace niespelniające po-

danych warunków nie będą poprawiane ani odsyłane.

Adres internetowy Kursu: http://www.wmat.pwr.wroc.pl/kurs



\end{document}