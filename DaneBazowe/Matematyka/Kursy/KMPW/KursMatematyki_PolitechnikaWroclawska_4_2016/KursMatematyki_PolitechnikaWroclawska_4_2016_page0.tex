\documentclass[a4paper,12pt]{article}
\usepackage{latexsym}
\usepackage{amsmath}
\usepackage{amssymb}
\usepackage{graphicx}
\usepackage{wrapfig}
\pagestyle{plain}
\usepackage{fancybox}
\usepackage{bm}

\begin{document}

XLVI

KORESPONDENCYJNY KURS

Z MATEMATYKI

grudzień 2016 r.

PRACA KONTROLNA $\mathrm{n}\mathrm{r} 4-$ POZIOM PODSTAWOWY

l. Dwa samochody wyjechałyjednocześnie zjednego miejsca ijada $\mathrm{w}$ tym samym kierunku.

Pierwszyjedzie $\mathrm{z}$ prędkością 50 $\mathrm{k}\mathrm{m}/\mathrm{h}$, a drugi $\mathrm{z}$ prędkością 40 $\mathrm{k}\mathrm{m}/\mathrm{h}$. Pół godziny póz$\acute{}$niej

$\mathrm{z}$ tego samego miejsca $\mathrm{i}\mathrm{w}$ tym samym kierunku wyruszyl trzeci samochód, który dopędził

pierwszy samochód $0 1$ godzinę $\mathrm{i}30$ minut póz/niej $\mathrm{n}\mathrm{i}\dot{\mathrm{z}}$ drugi. $\mathrm{Z}$ jaka prędkościq jechaf

trzeci samochód?

2. Proste $y = 2, y = 2x+10$ oraz $4x+3y = 0$ wyznaczają trójkąt $ABC$. Otrzymany

trójk$\Phi$t przeksztafcono $\mathrm{u}\dot{\mathrm{z}}$ ywaj $\Phi^{\mathrm{C}}$ najpierw jednokładności $0$ środku $O(0,0)\mathrm{i}$ skali $k=3,$

a następnie symetrii względem osi $OX$. Wyznaczyč współrzędne trójkąta $ABC$ oraz

wspófrzędne obrazów jego wierzcholków. Obliczyč pole trójkqta $ABC\mathrm{i}$ jego obrazu $\mathrm{w}$

tym przeksztalceniu.

3. Rozwazmy zbiór wszystkich prostokątów wpisanych $\mathrm{w}$ kwadrat $0$ boku dlugości $a\mathrm{w}$ taki

sposób, $\dot{\mathrm{z}}\mathrm{e}$ boki tego prostokąta $\mathrm{s}\Phi$ parami równolegfe do przekątnych danego kwadratu.

Obliczyč długości boków tego prostokąta, który ma największe pole.

4. Podstawa trójkqta równobocznego jest średnica koła $0$ promieniu $r$. Obliczyč stosunek

pola powierzchni części trójk$\Phi$ta lezącej na $\mathrm{z}\mathrm{e}\mathrm{w}\mathrm{n}\Phi \mathrm{t}\mathrm{r}\mathrm{z}$ kofa do pola powierzchni części

trójkąta lezącej wewnątrz kola.

5. $\mathrm{W}$ stozku pole podstawy, pole powierzchni kuli wpisanej $\mathrm{w}$ ten stozek $\mathrm{i}$ pole powierzchni

bocznej stozka, tworzą ciag arytmetyczny. Znalez/č cosinus kąta nachylenia tworzqcej

stozka do plaszczyzny jego podstawy.

6. $\mathrm{O}\mathrm{k}\mathrm{r}\Phi \mathrm{g}O_{1}\mathrm{o}$ promieniu l jest styczny do ramion kąta $0$ mierze $\displaystyle \frac{\pi}{3}$. Mniejszy od niego okrąg

$O_{2}$ jest styczny zewnętrznie do niego $\mathrm{i}$ obu ramion tego kąta. Procedurę kontynuujemy.

Znalez$\acute{}$č sumę obwodów pieciu otrzymanych kolejno $\mathrm{w}$ ten sposób okręgów. Dla jakiego

$n$ suma obwodów $\mathrm{c}\mathrm{i}_{\Phi \mathrm{g}}\mathrm{u}$ tych okręgów jest większa od $\displaystyle \frac{299}{100}\pi$?
\end{document}
