\documentclass[a4paper,12pt]{article}
\usepackage{latexsym}
\usepackage{amsmath}
\usepackage{amssymb}
\usepackage{graphicx}
\usepackage{wrapfig}
\pagestyle{plain}
\usepackage{fancybox}
\usepackage{bm}

\begin{document}

XLVII

KORESPONDENCYJNY KURS

Z MATEMATYKI

grudzień 2017 r.

PRACA KONTROLNA $\mathrm{n}\mathrm{r} 4-$ POZIOM PODSTAWOWY

l. Rodzina składa się $\mathrm{z}$ pięciorga dzieci $\mathrm{i}$ dwojga rodziców. Załózmy, $\dot{\mathrm{z}}\mathrm{e}$ dzieci nie moga

wyjśč na spacer ani nie $\mathrm{m}\mathrm{o}\mathrm{g}\Phi$ zostač $\mathrm{w}$ domu bez opieki któregokolwiek $\mathrm{z}$ rodziców. $\mathrm{W}$ ilu

$\mathrm{m}\mathrm{o}\dot{\mathrm{z}}$ liwych kombinacjach dzieci mogą wyjśč na spacer zakladajqc, $\dot{\mathrm{z}}\mathrm{e}$ przynajmniej jedno

dziecko idzie na spacer?

2. Na bokach prostokąta $0$ stałym obwodzie $4p$ opisano na średnicach pólokręgi $1\mathrm{e}\mathrm{z}\Phi^{\mathrm{C}\mathrm{e}}$

na zewnqtrz prostokąta. Dla jakich wartości boków prostokąta pole figury ograniczonej

$\mathrm{k}\mathrm{r}\mathrm{z}\mathrm{y}\mathrm{w}\Phi \mathrm{z}1_{\mathrm{o}\mathrm{Z}\mathrm{o}\mathrm{n}}\varpi \mathrm{z}$ tych czterech pólokręgów jest najmniejsze? Wykonač staranny rysunek.

3. Punkty $A(1,3), B(5,1), C(4,4)$ są wierzchołkami trójkąta. Obliczyč stosunek pola koła

opisanego na tym trójkqcie do pola kola wpisanego $\mathrm{w}$ ten trójkąt.

4. Liczby $x_{1},  x_{2}\mathrm{s}\Phi$ pierwiastkami równania $x^{2}-3x+A=0$, a liczby $x_{3}, x_{4}$ pierwiastkami

równania $x^{2}-12x+B=0$. Wiadomo, $\dot{\mathrm{z}}\mathrm{e}$ liczby $x_{1}, x_{2}, x_{3}, x_{4}$ tworza ciąg geometryczny.

Znalez/č ten ciąg oraz liczby A $\mathrm{i}B.$

5. Rozwiązač układ równań:

$\{$

a następnie obliczyč pole obszaru, który jest rozwiązaniem ukladu nierówności:

$\left\{\begin{array}{l}
x^{2}+y^{2}-2x-4y+1=0,\\
|x-1|-y=0,\\
x^{2}+y^{2}-2x-4y+1\leq 0,\\
|x-1|-y\leq 0.
\end{array}\right.$

Sporządzič staranny rysunek.

6. $\mathrm{W}$ graniastoslupie prawidfowym czworokqtnym okrąg styczny do dwóch boków podsta-

wy $\mathrm{i}$ przechodzqcy przez jej wierzcholek nielezący na $\dot{\mathrm{z}}$ adnym $\mathrm{z}$ tych boków ma promień

$r=2$. Płaszczyzna $\mathrm{P}^{\mathrm{r}\mathrm{z}\mathrm{e}\mathrm{c}\mathrm{h}\mathrm{o}\mathrm{d}\mathrm{z}}\Phi^{\mathrm{c}\mathrm{a}}$ przez środki krawędzi wychodzących $\mathrm{z}$ jednego wierz-

chołka graniastosłupa tworzy $\mathrm{z}$ płaszczyzną jego podstawy kąt $45^{\mathrm{o}}$ Obliczyč objetośč

graniastosłupa.
\end{document}
