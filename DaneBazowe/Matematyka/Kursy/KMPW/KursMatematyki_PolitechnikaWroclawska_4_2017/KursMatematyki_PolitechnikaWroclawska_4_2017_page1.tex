\documentclass[a4paper,12pt]{article}
\usepackage{latexsym}
\usepackage{amsmath}
\usepackage{amssymb}
\usepackage{graphicx}
\usepackage{wrapfig}
\pagestyle{plain}
\usepackage{fancybox}
\usepackage{bm}

\begin{document}

PRACA KONTROLNA nr 4- POZ1OM ROZSZERZONY

l. Na ile sposbów $\mathrm{m}\mathrm{o}\dot{\mathrm{z}}$ na umieścič 6 osób $\mathrm{w}$ pokojach dwuosobowych przy zalozeniu, $\dot{\mathrm{z}}\mathrm{e}$

pewne dwie ustalone osoby nie chcą mieszkač razem oraz $\dot{\mathrm{z}}\mathrm{e} \mathrm{a}$) pokoje są jednakowe,

a wiec $\mathrm{w}\mathrm{a}\dot{\mathrm{z}}$ ne jest kto mieszka $\mathrm{z} \mathrm{k}\mathrm{i}\mathrm{m}$, ale niewazne $\mathrm{w}$ którym pokoju; b) pokoje są

istotnie rózne, a więc $\mathrm{w}\mathrm{a}\dot{\mathrm{z}}$ ne jest kto mieszka $\mathrm{w}$ którym pokoju?

2. Rozwiązač $\mathrm{n}\mathrm{a}\mathrm{s}\mathrm{t}\text{ę} \mathrm{p}\mathrm{u}\mathrm{j}_{\Phi^{\mathrm{C}\oplus}}$ nierównośč

$\cos^{2}x+\cos^{3}x+\cos^{4}x+\ldots<\cos x+1$

dla $x\in[0,2\pi].$

3. Pokazač, $\dot{\mathrm{z}}\mathrm{e}$ dla $\mathrm{k}\mathrm{a}\dot{\mathrm{z}}$ dej wartości parametru $m$ wielomian

$w(x)=x^{3}+(2m-1)x^{2}-(3+2m)x+3$

ma pierwiastek całkowity. Dlajakich wartości parametru $m$ pierwiastki tego wielomianu

$\mathrm{t}\mathrm{w}\mathrm{o}\mathrm{r}\mathrm{z}\Phi$ ciąg arytmetyczny?

4. Punkt $A$ nalez $\mathrm{y}$ do obszaru kąta $0$ mierze stopniowej 60. Od1egłości tego punktu od ra-

mion kata są równe a $\mathrm{i}b$. Wyznaczyč odleglośč punktu $A$ od wierzchołka kąta. Następnie

obliczyč tę odlegfośč dla $a=2\mathrm{i}b=\sqrt{3}-1.$

5. $\mathrm{Z}$ punktu $A(1,1)$ wychodzą dwie półproste prostopadłe przecinające oś $OX$ układu

wspólrzędnych. Niech $F$ będzie obszarem $\mathrm{k}_{\Phi^{\mathrm{t}\mathrm{a}}}$ prostego wyznaczonego przez te pól-

proste, $G$ zaś zbiorem wszystkich punktów $0$ obydwóch wspólrzędnych nieujemnych.

Wyznaczyč połozenie półprostych, dla których pole figury $F\cap G$ jest najmniejsze.

6. Znalez/č $\mathrm{n}\mathrm{a}\mathrm{j}\mathrm{m}\mathrm{n}\mathrm{i}\mathrm{e}\mathrm{j}\mathrm{s}\mathrm{z}\Phi \mathrm{m}\mathrm{o}\dot{\mathrm{z}}\mathrm{l}\mathrm{i}\mathrm{w}\Phi$ objętośč stozka opisanego na walcu, którego przekrojem

osiowym jest kwadrat $0$ boku 2.

Rozwiązania (rękopis) zadań z wybranego poziomu prosimy nadsylač do

na adres:

18 grudnia 20l7r.

Wydziaf Matematyki

Politechniki Wrocfawskiej

Wybrzez $\mathrm{e}$ Wyspiańskiego 27

$50-370$ WROCLAW.

Na kopercie prosimy $\underline{\mathrm{k}\mathrm{o}\mathrm{n}\mathrm{i}\mathrm{e}\mathrm{c}\mathrm{z}\mathrm{n}\mathrm{i}\mathrm{e}}$ zaznaczyč wybrany poziom! (np. poziom podsta-

wowy lub rozszerzony). Do rozwiązań nalez $\mathrm{y}$ dołączyč zaadresowaną do siebie kopertę

zwrotną $\mathrm{z}$ naklejonym znaczkiem, odpowiednim do wagi listu. Prace niespelniające po-

danych warunków nie bedą poprawiane ani odsyłane.

Adres internetowy Kursu: http: //www. im. pwr. edu. pl/kurs
\end{document}
