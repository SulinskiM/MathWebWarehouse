\documentclass[a4paper,12pt]{article}
\usepackage{latexsym}
\usepackage{amsmath}
\usepackage{amssymb}
\usepackage{graphicx}
\usepackage{wrapfig}
\pagestyle{plain}
\usepackage{fancybox}
\usepackage{bm}

\begin{document}

XLVIII

KORESPONDENCYJNY KURS

Z MATEMATYKI

grudzień 2018 r.

PRACA KONTROLNA $\mathrm{n}\mathrm{r} 4-$ POZIOM PODSTAWOWY

1. $\mathrm{W}$ zawodach szachowych bierze udział pewna ilośč zawodników, przy czym $\mathrm{k}\mathrm{a}\dot{\mathrm{z}}\mathrm{d}\mathrm{y}$ zawod-

nik gra $\mathrm{z}\mathrm{k}\mathrm{a}\dot{\mathrm{z}}$ dym innym dokladnie $\mathrm{r}\mathrm{a}\mathrm{z}$. Ilu bylo zawodników, jeśli wiadomo, $\dot{\mathrm{z}}\mathrm{e}$ rozegrano

55 partii? Ile rozegranoby partii $\mathrm{w}$ tych zawodach, gdyby jeden $\mathrm{z}$ zawodników zrezygno-

waf $\mathrm{z}$ zawodów rozegrawszy cztery partie?

2. Dane są trzy wektory: ã $= [1,-2], \vec{b}= [-2,-1], \vec{c}= [3$, 4$]$. Dla jakich rzeczywistych

parametrów $t\mathrm{i}s$, {\it wektory Afi}$=${\it tã}, $\overline{B}7=s\vec{b}$ oraz $c\infty=\vec{c}$ tworza trójkąt $ABC$? Zna-

lez/č wspófrzędne środka cięzkości otrzymanego trójk$\Phi$ta, przyjmując $A(0,0)$. Sporządzič

staranny rysunek.

3. Wartośč $\mathrm{u}\dot{\mathrm{z}}$ ytkowa pewnej maszyny maleje $\mathrm{z}$ roku na rok $0$ tę samą wielkośč. Obliczyč

czas, wjakim maszyna straci cafkowitą wartośč $\mathrm{u}\dot{\mathrm{z}}$ ytkową, $\mathrm{j}\mathrm{e}\dot{\mathrm{z}}$ eli wiadomo, $\dot{\mathrm{z}}$ ejej wartośč

po 251atach pracy była trzy razy mniejsza $\mathrm{n}\mathrm{i}\dot{\mathrm{z}}$ jej wartośč po 151atach.

4. Na okręgu $0$ promieniu dfugości $r$ opisano trapez prostokątny, którego najdfuzszy bok

ma długośč $3r$. Obliczyč pole tego trapezu. Sporządzič staranny rysunek.

5. Obliczyč pierwiastek równania

--4{\it x}--6{\it mx}---22{\it xx}$++${\it m}1$=$--26{\it x}-2{\it m-x}--7{\it x}22

wiedząc, $\dot{\mathrm{z}}\mathrm{e}$ jest on $02$ większy od wartości parametru $m.$

6. $\mathrm{Z}$ czworościanu foremnego odcinamy cztery naroz $\mathrm{a}$, których krawędziami bocznymi są

połówki krawędzi czworościanu. Jaki wielościan otrzymujemy? Obliczyč stosunek jego

objętości $\mathrm{i}$ pola powierzchni do objetości $\mathrm{i}$ pola powierzchni czworościanu. Sporządzič

staranny rysunek.
\end{document}
