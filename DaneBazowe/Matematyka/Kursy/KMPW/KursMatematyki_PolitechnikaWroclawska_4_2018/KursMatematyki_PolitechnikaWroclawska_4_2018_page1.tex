\documentclass[a4paper,12pt]{article}
\usepackage{latexsym}
\usepackage{amsmath}
\usepackage{amssymb}
\usepackage{graphicx}
\usepackage{wrapfig}
\pagestyle{plain}
\usepackage{fancybox}
\usepackage{bm}

\begin{document}

PRACA KONTROLNA nr 4- POZIOM ROZSZERZONY

1. $\mathrm{W}$ zawodach szachowych bierze udział pewna ilośč zawodników, przy czym $\mathrm{k}\mathrm{a}\dot{\mathrm{z}}\mathrm{d}\mathrm{y}$ zawod-

nik gra $\mathrm{z}\mathrm{k}\mathrm{a}\dot{\mathrm{z}}$ dym innym zawodnikiem dokładnie $\mathrm{r}\mathrm{a}\mathrm{z}$. Ilu bylo zawodników tych zawodów,

jeśli rozegrano 84 partie, a dwóch zawodników wycofało się $\mathrm{z}$ zawodów po rozegraniu

przez $\mathrm{k}\mathrm{a}\dot{\mathrm{z}}$ dego trzech partii?

2. Przez środek boku trójkąta równobocznego poprowadzono prostą tworzącą $\mathrm{z}$ tym bokiem

$\mathrm{k}\mathrm{a}\mathrm{t}45^{\mathrm{o}}\mathrm{i}$ dzielącą ten trójkqt na dwie figury. Obliczyč stosunek pól tych figur (większej

do mniejszej). Wynik przedstawič $\mathrm{w}$ najprostszej postaci.

3. Dla jakich wartości parametru $m$, punkty $A(m,-\displaystyle \frac{3}{2}), B(2,0)$ oraz $C(4,-m)$ są wierz-

cholkami trójkąta $ABC$? Zbadač jak zmienia $\mathrm{s}\mathrm{i}\mathrm{e}$ pole tego trójkata $\mathrm{w}$ zalezności od $m.$

Znalez/č, $0$ ile istnieją, najmniejszą $\mathrm{i}$ największ$\Phi$ wartośč tego pola dla $m\in[-2,2].$

4. $\mathrm{Z}$ miast $A\mathrm{i}B$ odległych $0119$ km wyruszają naprzeciw siebie dwaj rowerzyści, przy czym

drugi rowerzysta startuje dwie godziny po wyje $\acute{\mathrm{z}}\mathrm{d}\mathrm{z}\mathrm{i}\mathrm{e}$ pierwszego. Pierwszy rowerzysta,

ruszający $\mathrm{z}$ miasta $A, \mathrm{w}$ ciągu pierwszej godziny przejez $\mathrm{d}\dot{\mathrm{z}}$ a 20 km $\mathrm{i}\mathrm{w}\mathrm{k}\mathrm{a}\dot{\mathrm{z}}$ dej następnej

godzinie przejezdza $02$ km mniej $\mathrm{n}\mathrm{i}\dot{\mathrm{z}}\mathrm{w}$ poprzedniej. Natomiast drugi rowerzysta $\mathrm{w}$ ciągu

pierwszej godziny przejez $\mathrm{d}\dot{\mathrm{z}}$ a 10 km $\mathrm{i}\mathrm{w}\mathrm{k}\mathrm{a}\dot{\mathrm{z}}$ dej następnej godzinie przejezdza $03$ km

więcej $\mathrm{n}\mathrm{i}\dot{\mathrm{z}}\mathrm{w}$ poprzedniej. Po ilu godzinach jazdy się spotkają $\mathrm{i}\mathrm{w}$ jakiej odległości będą

wtedy od obu miast?

5. Wyznaczyč sumę pierwiastków równania

$2^{(m+1)x^{2}-4mx+m+\frac{3}{2}}=\sqrt{2}$

jako funkcję parametru $m$. Wyznaczyč przedziafy, na których funkcja ta jest $\mathrm{r}\mathrm{o}\mathrm{s}\mathrm{n}\Phi^{\mathrm{C}\mathrm{a}}.$

6. $\mathrm{Z}$ sześcianu odcinamy osiem narozy (małych czworościanów), których wierzchołkami

są wierzchołki sześcianu, a bocznymi krawędziami - połówki krawędzi sześcianu. Jaki

wielościan otrzymujemy? Obliczyč stosunekjego objętości $\mathrm{i}$ pola powierzchni do objętości

$\mathrm{i}$ pola powierzchni sześcianu. Znalez/č odległośč między dwoma najbardziej odległymi

wierzchofkami tego wielościanu. Sporządzič staranny rysunek.

Rozwiązania (rękopis) zadań z wybranego poziomu prosimy nadsyłač do

na adres:

18 grudnia 20l8r.

Wydziaf Matematyki

Politechnika Wrocfawska

Wybrzez $\mathrm{e}$ Wyspiańskiego 27

$50-370$ WROCLAW.

Na kopercie prosimy $\underline{\mathrm{k}\mathrm{o}\mathrm{n}\mathrm{i}\mathrm{e}\mathrm{c}\mathrm{z}\mathrm{n}\mathrm{i}\mathrm{e}}$ zaznaczyč wybrany poziom! (np. poziom podsta-

wowy lub rozszerzony). Do rozwiązań nalez $\mathrm{y}$ dołączyč zaadresowaną do siebie koperte

zwrotną $\mathrm{z}$ naklejonym znaczkiem, odpowiednim do wagi listu. Prace niespelniające po-

danych warunków nie będą poprawiane ani odsyłane.

Uwaga. Wysylajac nam rozwiazania zadań uczestnik Kursu udostępnia Politechnice Wroclawskiej

swoje dane osobowe, które przetwarzamy wyłącznie $\mathrm{w}$ zakresie niezbednym do jego prowadzenia

(odesfanie zadań, prowadzenie statystyki). Szczegófowe informacje $0$ przetwarzaniu przez nas danych

osobowych $\mathrm{S}\otimes$ dostępne na stronie internetowej Kursu.

Adres internetowy Kursu: http: //www. im. pwr. edu. pl/kurs
\end{document}
