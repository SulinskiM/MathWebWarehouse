\documentclass[10pt]{article}
\usepackage[polish]{babel}
\usepackage[utf8]{inputenc}
\usepackage[T1]{fontenc}
\usepackage{amsmath}
\usepackage{amsfonts}
\usepackage{amssymb}
\usepackage[version=4]{mhchem}
\usepackage{stmaryrd}
\usepackage{hyperref}
\hypersetup{colorlinks=true, linkcolor=blue, filecolor=magenta, urlcolor=cyan,}
\urlstyle{same}

\title{PRACA KONTROLNA nr 4 - POZIOM PODSTAWOWY }

\author{}
\date{}


\begin{document}
\maketitle
\begin{enumerate}
  \item W zawodach szachowych bierze udział pewna ilość zawodników, przy czym każdy zawodnik gra z każdym innym dokładnie raz. Ilu było zawodników, jeśli wiadomo, że rozegrano 55 partii? Ile rozegranoby partii w tych zawodach, gdyby jeden z zawodników zrezygnował z zawodów rozegrawszy cztery partie?
  \item Dane są trzy wektory: $\vec{a}=[1,-2], \vec{b}=[-2,-1], \vec{c}=[3,4]$. Dla jakich rzeczywistych parametrów $t$ i $s$, wektory $\overrightarrow{A B}=t \vec{a}, \overrightarrow{B C}=s \vec{b}$ oraz $\overrightarrow{C A}=\vec{c}$ tworzą trójkąt $A B C$ ? Znaleźć współrzędne środka ciężkości otrzymanego trójkąta, przyjmując $A(0,0)$. Sporządzić staranny rysunek.
  \item Wartość użytkowa pewnej maszyny maleje z roku na rok o tę samą wielkość. Obliczyć czas, w jakim maszyna straci całkowitą wartość użytkową, jeżeli wiadomo, że jej wartość po 25 latach pracy była trzy razy mniejsza niż jej wartość po 15 latach.
  \item Na okręgu o promieniu długości $r$ opisano trapez prostokątny, którego najdłuższy bok ma długość $3 r$. Obliczyć pole tego trapezu. Sporządzić staranny rysunek.
  \item Obliczyć pierwiastek równania
\end{enumerate}

$$
\frac{x-m}{4-6 x}-\frac{2 x+m}{2 x+1}=\frac{2-m-7 x^{2}}{6 x^{2}-x-2}
$$

wiedząc, że jest on o 2 większy od wartości parametru $m$.\\
6. Z czworościanu foremnego odcinamy cztery naroża, których krawędziami bocznymi są połówki krawędzi czworościanu. Jaki wielościan otrzymujemy? Obliczyć stosunek jego objętości i pola powierzchni do objętości i pola powierzchni czworościanu. Sporządzić staranny rysunek.

\section*{PRACA KONTROLNA nr 4 - POZIOM ROZSZERZONY}
\begin{enumerate}
  \item W zawodach szachowych bierze udział pewna ilość zawodników, przy czym każdy zawodnik gra z każdym innym zawodnikiem dokładnie raz. Ilu było zawodników tych zawodów, jeśli rozegrano 84 partie, a dwóch zawodników wycofało się z zawodów po rozegraniu przez każdego trzech partii?
  \item Przez środek boku trójkąta równobocznego poprowadzono prostą tworzącą z tym bokiem kąt $45^{\circ}$ i dzielącą ten trójkąt na dwie figury. Obliczyć stosunek pól tych figur (większej do mniejszej). Wynik przedstawić w najprostszej postaci.
  \item Dla jakich wartości parametru $m$, punkty $A\left(m,-\frac{3}{2}\right), B(2,0)$ oraz $C(4,-m)$ są wierzchołkami trójkąta $A B C$ ? Zbadać jak zmienia się pole tego trójkąta w zależności od $m$. Znaleźć, o ile istnieją, najmniejszą i największą wartość tego pola dla $m \in[-2,2]$.
  \item Z miast $A$ i $B$ odległych o 119 km wyruszają naprzeciw siebie dwaj rowerzyści, przy czym drugi rowerzysta startuje dwie godziny po wyjeździe pierwszego. Pierwszy rowerzysta, ruszający z miasta $A$, w ciągu pierwszej godziny przejeżdża 20 km i w każdej następnej godzinie przejeżdża o 2 km mniej niż w poprzedniej. Natomiast drugi rowerzysta w ciągu pierwszej godziny przejeżdża 10 km i w każdej następnej godzinie przejeżdża o 3 km więcej niż w poprzedniej. Po ilu godzinach jazdy się spotkają i w jakiej odległości będą wtedy od obu miast?
  \item Wyznaczyć sumę pierwiastków równania
\end{enumerate}

$$
2^{(m+1) x^{2}-4 m x+m+\frac{3}{2}}=\sqrt{2}
$$

jako funkcję parametru $m$. Wyznaczyć przedziały, na których funkcja ta jest rosnąca.\\
6. Z sześcianu odcinamy osiem naroży (małych czworościanów), których wierzchołkami są wierzchołki sześcianu, a bocznymi krawędziami - połówki krawędzi sześcianu. Jaki wielościan otrzymujemy? Obliczyć stosunek jego objętości i pola powierzchni do objętości i pola powierzchni sześcianu. Znaleźć odległość między dwoma najbardziej odległymi wierzchołkami tego wielościanu. Sporządzić staranny rysunek.

Rozwiązania (rękopis) zadań z wybranego poziomu prosimy nadsyłać do 18 grudnia 2018r. na adres:

Wydział Matematyki\\
Politechnika Wrocławska\\
Wybrzeże Wyspiańskiego 27\\
50-370 WROCEAW.\\
Na kopercie prosimy koniecznie zaznaczyć wybrany poziom! (np. poziom podstawowy lub rozszerzony). Do rozwiązań należy dołączyć zaadresowaną do siebie kopertę zwrotną z naklejonym znaczkiem, odpowiednim do wagi listu. Prace niespełniające podanych warunków nie będą poprawiane ani odsyłane.

Uwaga. Wysyłając nam rozwiązania zadań uczestnik Kursu udostępnia Politechnice Wrocławskiej swoje dane osobowe, które przetwarzamy wyłącznie w zakresie niezbędnym do jego prowadzenia (odesłanie zadań, prowadzenie statystyki). Szczegółowe informacje o przetwarzaniu przez nas danych osobowych są dostępne na stronie internetowej Kursu.\\
Adres internetowy Kursu: \href{http://www.im.pwr.edu.pl/kurs}{http://www.im.pwr.edu.pl/kurs}


\end{document}