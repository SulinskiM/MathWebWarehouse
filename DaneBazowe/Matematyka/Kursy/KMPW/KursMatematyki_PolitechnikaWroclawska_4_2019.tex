\documentclass[a4paper,12pt]{article}
\usepackage{latexsym}
\usepackage{amsmath}
\usepackage{amssymb}
\usepackage{graphicx}
\usepackage{wrapfig}
\pagestyle{plain}
\usepackage{fancybox}
\usepackage{bm}

\begin{document}

XLIX

KORESPONDENCYJNY KURS

Z MATEMATYKI

grudzień 2019 r.

PRACA KONTROLNA $\mathrm{n}\mathrm{r} 4-$ POZIOM PODSTAWOWY

l. Rozwiązač nierównośč $\sqrt{2^{x}-1}\leq 2^{x}-3.$

2. Trójkąt prostokątny $0$ przyprostokątnych $a, b$ obracamy wokóf $\mathrm{k}\mathrm{a}\dot{\mathrm{z}}$ dej $\mathrm{z}$ przyprostokąt-

nych. Obliczyč stosunek sumy objętości tych stozków do objętości bryly otrzymanej

przez obrót trójkąta wokóf przeciwprostokątnej $\mathrm{i}$ wyrazič go jako funkcję zmiennej $\displaystyle \frac{a}{b}.$

3. Punkty $(-1,1), (0,0), (\sqrt{2},0)$ są trzema kolejnymi wierzcholkami wielokąta foremnego.

Wyznaczyč wspófrzędne pozostalych wierzchofków wielokąta oraz jego pole. Podač rów-

nania okręgów wpisanego $\mathrm{i}$ opisanego na tym wielokącie oraz wyznaczyč stosunek ich

promieni.

4. Niech $f(x)=\{$

$\displaystyle \frac{2-|x|}{|x|-1}$

$\displaystyle \frac{8}{9}x^{2}-1$

gdy

gdy

$|x|>\displaystyle \frac{3}{2}.$

$|x|\displaystyle \leq\frac{3}{2}.$

a) Narysowač wykres funkcji $f\mathrm{i}$ na jego podstawie wyznaczyč zbiór wartości funkcji.

b) Obliczyč $f(\sqrt{2})$ oraz $f(\sqrt{3}).$

c) Rozwiązač nierównośč $f(x)\displaystyle \leq-\frac{1}{2}\mathrm{i}$ zaznaczyč na osi $0x$ zbiór rozwiazań.

5. Punkty $A(0,1), B(4,3) \mathrm{s}\Phi$ dwoma kolejnymi wierzcholkami równolegfoboku ABCD,

a $S(2,3)$ punktem przecięcia przekqtnych. Posługujac się rachunkiem wektorowym, wy-

znaczyč pozostafe wierzchofki równolegfoboku oraz wierzchofki równolegfoboku otrzy-

manego przez obrót ABCD wokól punktu $A090^{\mathrm{o}} \mathrm{w}$ kierunku przeciwnym do ruchu

wskazówek zegara.

6. Ostroslup prawidlowy trójkątny, $\mathrm{w}$ którym bok podstawy $\mathrm{i}$ wysokośč są równe $a$ przecięto

plaszczyzną przechodzącq przez jedną $\mathrm{z}$ krawędzi podstawy na dwie bryły $0$ tej samej

objętości. Wyznaczyč tangens kąta nachylenia tej pfaszczyzny do pfaszczyzny podstawy.

Sporządzič rysunek.




PRACA KONTROLNA nr 4- POZ1OM ROZSZERZONY

l. Punkty $A(0,1), B(4,3)$ są dwoma kolejnymi wierzchołkami równolegloboku ABCD, $\mathrm{a}$

$S(2,3)$ punktem przecięcia przekątnych. Posfugując się rachunkiem wektorowym, wyzna-

czyč pozostałe wierzchołki równoległoboku oraz wierzchołki równoległoboku $A'B'C'D'$

otrzymanego przez obrót ABCD $0$ kąt $90^{\mathrm{o}}$ wokól punktu $(0,0)\mathrm{w}$ kierunku przeciwnym

do ruchu wskazówek zegara. Sprawdzič, $\dot{\mathrm{z}}\mathrm{e}A'B'C'D'$ jest obrazem ABCD $\mathrm{w}$ przeksztal-

ceniu $T_{2}\mathrm{o}O\mathrm{o}T_{1}$, gdzie $T_{1}$ jest przesunięciem $0$ wektor $[0$, 1$]$, O- obrotem $0$ kąt $90^{o}$ wokół

punktu $(0,0)\mathrm{w}$ kierunku przeciwnym do ruchu wskazówek zegara, a $T_{2}$- przesunięciem

$0$ wektor [1, 0].

2. Narysowač wykres funkcji

$f(x)=1-\displaystyle \frac{2^{x}}{3^{x}-2^{x}}+(\frac{2^{x}}{3^{x}-2^{x}})^{2}$

$\mathrm{i}$ uzasadnič, $\dot{\mathrm{z}}\mathrm{e}$ przyjmuje ona $\mathrm{w}\mathrm{y}l_{\Phi}$cznie wartości większe $\displaystyle \mathrm{n}\mathrm{i}\dot{\mathrm{z}}\frac{1}{2}.$

3. Niech $f(x)=$

dla

dla

$x\leq 1,$

$x>1.$

a) Narysowač wykres funkcji $f\mathrm{i}$ na jego podstawie wyznaczyč zbiór wartości funkcji.

b) Obliczyč $f(\displaystyle \log_{\frac{1}{2}}(\sqrt{2}-\frac{1}{2}))$ oraz $f(2^{\sqrt{2}}+\displaystyle \frac{1}{2}).$

c) Rozwiązač nierównośč $f(x)\displaystyle \leq\frac{1}{2}\mathrm{i}$ zaznaczyč na osi $0x$ zbiór rozwiązań.

4. Punkt $C(0,0)$ jest wierzcholkiem trójkąta równoramiennego, $\mathrm{w}$ którym środkowa podsta-

{\it wy AB} $\mathrm{i}$ wysokośč poprowadzona zjednego $\mathrm{z}$ wierzcholków $A, B$ przecinają się $\mathrm{w}$ punkcie

$S(2,1)$. Pole trójkąta $ABS$ jest dwa razy mniejsze $\mathrm{n}\mathrm{i}\dot{\mathrm{z}}$ pole trójkąta $ABC$. Wyznaczyč

wspófrzędne wierzchołków $A, B$ oraz równanie okręgu opisanego na trójkącie $ABC.$

5. $\mathrm{W}$ ośmiościan foremny wpisano dwa sześciany. Wierzchofki pierwszego $\mathrm{z}$ nich lezą na

krawędziach ośmiościanu, a wierzchołki drugiego- na wysokościach ścian bocznych. Ob-

liczyč stosunek objętości tych sześcianów.

6. Prostokąt $0$ bokach $a\mathrm{i}2a$ obraca się wokół przekątnej. Obliczyč pole powierzchni całko-

witej $\mathrm{i}$ objętośč otrzymanej bryły.

Rozwiązania (rękopis) zadań z wybranego poziomu prosimy nadsyłač do

na adres:

18 grudnia 20l9r.

Wydziaf Matematyki

Politechnika Wrocfawska

Wybrzez $\mathrm{e}$ Wyspiańskiego 27

$50-370$ WROCLAW.

Na kopercie prosimy $\underline{\mathrm{k}\mathrm{o}\mathrm{n}\mathrm{i}\mathrm{e}\mathrm{c}\mathrm{z}\mathrm{n}\mathrm{i}\mathrm{e}}$ zaznaczyč wybrany poziom! (np. poziom podsta-

wowy lub rozszerzony). Do rozwiązań nalez $\mathrm{y}$ dołączyč zaadresowaną do siebie kopertę

zwrotną $\mathrm{z}$ naklejonym znaczkiem, odpowiednim do wagi listu. Prace niespelniające po-

danych warunków nie będą poprawiane ani odsylane.

Uwaga. Wysyłając nam rozwiazania zadań uczestnik Kursu udostępnia Politechnice Wrocławskiej

swoje dane osobowe, które przetwarzamy wyłącznie $\mathrm{w}$ zakresie niezbędnym do jego prowadzenia

(odesłanie zadań, prowadzenie statystyki). Szczególowe informacje $0$ przetwarzaniu przez nas danych

osobowych są dostępne na stronie internetowej Kursu.

Adres internetowy Kursu: http: //www. im. pwr. edu. pl/kurs



\end{document}