\documentclass[a4paper,12pt]{article}
\usepackage{latexsym}
\usepackage{amsmath}
\usepackage{amssymb}
\usepackage{graphicx}
\usepackage{wrapfig}
\pagestyle{plain}
\usepackage{fancybox}
\usepackage{bm}

\begin{document}

XLIX

KORESPONDENCYJNY KURS

Z MATEMATYKI

grudzień 2019 r.

PRACA KONTROLNA $\mathrm{n}\mathrm{r} 4-$ POZIOM PODSTAWOWY

l. Rozwiązač nierównośč $\sqrt{2^{x}-1}\leq 2^{x}-3.$

2. Trójkąt prostokątny $0$ przyprostokątnych $a, b$ obracamy wokóf $\mathrm{k}\mathrm{a}\dot{\mathrm{z}}$ dej $\mathrm{z}$ przyprostokąt-

nych. Obliczyč stosunek sumy objętości tych stozków do objętości bryly otrzymanej

przez obrót trójkąta wokóf przeciwprostokątnej $\mathrm{i}$ wyrazič go jako funkcję zmiennej $\displaystyle \frac{a}{b}.$

3. Punkty $(-1,1), (0,0), (\sqrt{2},0)$ są trzema kolejnymi wierzcholkami wielokąta foremnego.

Wyznaczyč wspófrzędne pozostalych wierzchofków wielokąta oraz jego pole. Podač rów-

nania okręgów wpisanego $\mathrm{i}$ opisanego na tym wielokącie oraz wyznaczyč stosunek ich

promieni.

4. Niech $f(x)=\{$

$\displaystyle \frac{2-|x|}{|x|-1}$

$\displaystyle \frac{8}{9}x^{2}-1$

gdy

gdy

$|x|>\displaystyle \frac{3}{2}.$

$|x|\displaystyle \leq\frac{3}{2}.$

a) Narysowač wykres funkcji $f\mathrm{i}$ na jego podstawie wyznaczyč zbiór wartości funkcji.

b) Obliczyč $f(\sqrt{2})$ oraz $f(\sqrt{3}).$

c) Rozwiązač nierównośč $f(x)\displaystyle \leq-\frac{1}{2}\mathrm{i}$ zaznaczyč na osi $0x$ zbiór rozwiazań.

5. Punkty $A(0,1), B(4,3) \mathrm{s}\Phi$ dwoma kolejnymi wierzcholkami równolegfoboku ABCD,

a $S(2,3)$ punktem przecięcia przekqtnych. Posługujac się rachunkiem wektorowym, wy-

znaczyč pozostafe wierzchofki równolegfoboku oraz wierzchofki równolegfoboku otrzy-

manego przez obrót ABCD wokól punktu $A090^{\mathrm{o}} \mathrm{w}$ kierunku przeciwnym do ruchu

wskazówek zegara.

6. Ostroslup prawidlowy trójkątny, $\mathrm{w}$ którym bok podstawy $\mathrm{i}$ wysokośč są równe $a$ przecięto

plaszczyzną przechodzącq przez jedną $\mathrm{z}$ krawędzi podstawy na dwie bryły $0$ tej samej

objętości. Wyznaczyč tangens kąta nachylenia tej pfaszczyzny do pfaszczyzny podstawy.

Sporządzič rysunek.
\end{document}
