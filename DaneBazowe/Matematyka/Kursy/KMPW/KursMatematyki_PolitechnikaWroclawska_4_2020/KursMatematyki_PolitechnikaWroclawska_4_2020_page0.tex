\documentclass[a4paper,12pt]{article}
\usepackage{latexsym}
\usepackage{amsmath}
\usepackage{amssymb}
\usepackage{graphicx}
\usepackage{wrapfig}
\pagestyle{plain}
\usepackage{fancybox}
\usepackage{bm}

\begin{document}

L KORESPONDENCYJNY KURS

Z MATEMATYKI

grudzień 2020 r.

PRACA KONTROLNA $\mathrm{n}\mathrm{r} 4-$ POZIOM PODSTAWOWY

l. Wykaz$\cdot, \mathrm{z}\mathrm{e}$ dla dowolnej liczby naturalnej $n$ liczba $\displaystyle \vec{3}^{n^{4}}1-\vec{3}^{n^{3}}2-\frac{1}{3}n^{2}+\frac{2}{3}n$ jest podzielna

przez 8.

2. Podaj wzór funkcji kwadratowej, której wykres jest obrazem paraboli $f(x)=-4x(x-1)$

$\mathrm{w}$ symetrii względem punktu $(0,2)$. Uzasadnij poprawnośč znalezionego wzoru $\mathrm{i}\mathrm{s}$porząd $\acute{\mathrm{z}}$

wykresy obu funkcji $\mathrm{w}$ jednym ukladzie wspólrzednych.

3. Wyznacz wielomian $f(x)=x^{3}+ax^{2}+bx+c$ wiedząc, $\dot{\mathrm{z}}\mathrm{e}$ jego pierwiastki są całkowite $\mathrm{i}$

tworzą ciąg geometryczny, a wykres przecina oś $Oy\mathrm{w}$ punkcie $0$ wspólrzędnej $-8.$

4. Narysuj wykres funkcji $f(x)=\displaystyle \frac{|x-1|}{|x|-1}$. Wyznacz zbiór jej wartości $\mathrm{i}$ rozwiąz nierównośč

$|f(x)|\leq 2.$

5. $\mathrm{W}$ zalezności od parametru $a$ określ liczbe rozwiązań układu

Podaj interpretację graficzną dla $a=\sqrt{5}, a=1$ oraz $a=3.$

$\left\{\begin{array}{l}
x^{2}+y^{2}=1\\
|2x-y|=a.
\end{array}\right.$

6. Ostroslup prawidłowy czworokątny, $\mathrm{w}$ którym najmniejszy przekrój płaszczyzną zawie-

rającą wysokośč, prostopadła do płaszczyzny podstawy, jest trójkątem równobocznym,

przecięto płaszczyzną przechodzącą przez jedną $\mathrm{z}$ krawędzi podstawy prostopadlą do

przeciwległej ściany bocznej. Wyznacz stosunek objetości brył, na jakie plaszczyzna ta

$\mathrm{p}\mathrm{o}\mathrm{d}\mathrm{z}\mathrm{i}\mathrm{e}\mathrm{l}\mathrm{i}\ddagger \mathrm{a}$ ostrosłup.
\end{document}
