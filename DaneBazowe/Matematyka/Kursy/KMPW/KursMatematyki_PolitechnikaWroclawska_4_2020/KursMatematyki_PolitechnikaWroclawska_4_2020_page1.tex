\documentclass[a4paper,12pt]{article}
\usepackage{latexsym}
\usepackage{amsmath}
\usepackage{amssymb}
\usepackage{graphicx}
\usepackage{wrapfig}
\pagestyle{plain}
\usepackage{fancybox}
\usepackage{bm}

\begin{document}

PRACA KONTROLNA nr 4- POZ1OM ROZSZERZONY

l. Trzeci składnik rozwinięcia dwumianu $(\displaystyle \sqrt[3]{x^{2}}+\frac{1}{\sqrt{x}})^{n}$ ma wspófczynnik równy 45. Wy-

znacz wszystkie skladniki tego rozwinięcia, $\mathrm{w}$ których $x$ wystepuje $\mathrm{w}$ potedze $0$ wykład-

niku cafkowitym.

2. Wykres wielomianu $w(x) =x^{3}+ax^{2}+bx+c$ przecina oś $Oy\mathrm{w}$ punkcie $(0,-6) \mathrm{i}$ jest

symetryczny względem punktu $(-1,-2)$. Wyznacz wspófczynniki $a, b, c$ oraz pierwiastki

tego wielomianu. Sporząd $\acute{\mathrm{z}}$ wykres.

3. $\mathrm{W}$ zalezności od parametru $m$ określ liczbę rozwiqzań równania

$4^{x-1}-2^{x+1}\log_{2}m+1=0$

4. Narysuj wykres funkcji

$f(x)=1-\displaystyle \frac{\log_{2}|x-1|}{1-\log_{2}|x-1|}+(\frac{\log_{2}|x-1|}{1-\log_{2}|x-1|})^{2}-(\frac{\log_{2}|x-1|}{1-\log_{2}|x-1|})^{3}+$

gdzie prawa strona jest sumą nieskończonego ciągu geometrycznego.

5. $\mathrm{W}$ zalezności od parametru $a$ określ liczbe rozwiqzań układu

Podaj interpretację graficzną dla $a=0, a=-1$ oraz $a=7.$

$\left\{\begin{array}{l}
xy-y=1\\
x^{2}+y^{2}-2x=a+1.
\end{array}\right.$

6. Dany jest ostroslup prawidłowy trójkątny, $\mathrm{w}$ którym krawęd $\acute{\mathrm{z}}$ boczna jest dwa razy

dłuzsza $\mathrm{n}\mathrm{i}\dot{\mathrm{z}}$ krawędz$\acute{}$ podstawy. Ostrosfup ten podzielono pfaszczyzną przechodzącą przez

krawędz/ podstawy na dwie bryły $0$ tej samej objętości. Wyznacz stosunek objętości kul

wpisanych $\mathrm{w}\mathrm{k}\mathrm{a}\dot{\mathrm{z}}$ dą $\mathrm{z}$ tych brył. Sporzqd $\acute{\mathrm{z}}$ rysunek.

Rozwiązania (rekopis) zadań z wybranego poziomu prosimy nadsyłač do

adres:

31.12.2020r. na

Wydziaf Matematyki

Politechnika Wrocfawska

Wybrzez $\mathrm{e}$ Wyspiańskiego 27

$50-370$ WROCLAW.

Na kopercie prosimy $\underline{\mathrm{k}\mathrm{o}\mathrm{n}\mathrm{i}\mathrm{e}\mathrm{c}\mathrm{z}\mathrm{n}\mathrm{i}\mathrm{e}}$ zaznaczyč wybrany poziom! (np. poziom podsta-

wowy lub rozszerzony). Do rozwiązań nalez $\mathrm{y}$ dołączyč zaadresowaną do siebie kopertę

zwrotną $\mathrm{z}$ naklejonym znaczkiem, odpowiednim do formatu listu. Polecamy stosowanie

kopert formatu C5 $(160\mathrm{x}230\mathrm{m}\mathrm{m})$ ze znaczkiem $0$ wartości 3,30 $\mathrm{z}1$. Na $\mathrm{k}\mathrm{a}\dot{\mathrm{z}}$ dą wiekszą

kopertę nalez $\mathrm{y}$ nakleič drozszy znaczek. Prace niespełniające podanych warunków nie

będą poprawiane ani odsyłane.

Uwaga. Wysyfaj\S c nam rozwiązania zadań uczestnik Kursu udostępnia Politechnice Wrocfawskiej

swoje dane osobowe, które przetwarzamy wylącznie $\mathrm{w}$ zakresie niezbędnym do jego prowadzenia

(odeslanie zadań, prowadzenie statystyki). Szczegółowe informacje $0$ przetwarzaniu przez nas danych

osobowych są dostępne na stronie internetowej Kursu.

Adres internetowy Kursu: http: //www. im. pwr. edu. pl/kurs
\end{document}
