\documentclass[a4paper,12pt]{article}
\usepackage{latexsym}
\usepackage{amsmath}
\usepackage{amssymb}
\usepackage{graphicx}
\usepackage{wrapfig}
\pagestyle{plain}
\usepackage{fancybox}
\usepackage{bm}

\begin{document}

LI KORESPONDENCYJNY KURS

Z MATEMATYKI

grudzień 2021 r.

PRACA KONTROLNA nr 4- POZIOM PODSTAWOWY

l. Trzy liczby naturalne $0$ iloczynie 80 tworzą ciąg arytmetyczny. $\mathrm{J}\mathrm{e}\dot{\mathrm{z}}$ eli drugi wyraz tego

ciągu zmniejszymy $0 1$, to liczby te (rozwazane $\mathrm{w}$ tej samej kolejności) utworzą ciąg

geometryczny. Jakie to liczby?

2. Liczby dodatnie $a, b$ spełniają warunek $\alpha^{2}+b^{2}=7ab$. Wykaz, $\dot{\mathrm{z}}\mathrm{e}$

$\log_{3}a+\log_{3}b+2=2\log_{3}(a+b).$

3. Rozwiąz równanie

tg2 {\it x}$=$ -11 $+$-csoins {\it xx}.

4. Narysuj wykres funkcji

$f(x)=\{$

$\displaystyle \frac{2}{3}x^{2}-\frac{8}{3}x+2,$

$|4-2|x-3||,$

gdy

gdy

$|2x-5|\leq 3,$

$|2x-5|>3.$

Na jego podstawie wyznacz: zbiór wartości funkcji $f(x)$ oraz liczbę rozwiqzań równania

$f(x)=m \mathrm{w}$ zalezności od parametru $m.$

5. Punkt $A(0,0)$ jest wierzchołkiem ośmiokąta foremnego wpisanego $\mathrm{w}$ okrąg $x^{2}-2x+y^{2}=0.$

Wyznacz współrzedne pozostafych wierzchołków.

6. Przekrój ostrosfupa prawidfowego $\mathrm{c}\mathrm{z}\mathrm{w}\mathrm{o}\mathrm{r}\mathrm{o}\mathrm{k}_{\Phi^{\mathrm{t}}}$nego plaszczyzną $\mathrm{p}\mathrm{r}\mathrm{z}\mathrm{e}\mathrm{c}\mathrm{h}\mathrm{o}\mathrm{d}\mathrm{z}\text{ą}_{\mathrm{C}\Phi}$ przez wierz-

chołek $\mathrm{i}$ przekątną jego podstawy jest trójkqtem równobocznym. $\mathrm{W}$ ostrosłup wpisano

sześcian, którego dolna podstawa jest zawarta $\mathrm{w}$ podstawie ostrosfupa, a wierzchołki

górnej podstawy sześcianu lezą na krawędziach ostrosłupa. Oblicz stosunek objętości

sześcianu do objetości ostroslupa.




PRACA KONTROLNA $\mathrm{n}\mathrm{r} 4-$ POZIOM ROZSZERZONY

l. Liczby dodatnie $a, b, c$ spełniają warunki: $c>b, a\neq 1, c-b\neq 1, c+b\neq 1$. Wykaz$\cdot, \dot{\mathrm{z}}\mathrm{e}$

równośč

$\displaystyle \log_{c+b}a\cdot\log_{c-b}a=\frac{\log_{c+b}a+\log_{c-b}a}{2}$

zachodzi wtedy $\mathrm{i}$ tylko wtedy, gdy $a^{2}+b^{2}=c^{2}$

2. Rozwiąz nierównośč $\displaystyle \sin^{4}x+\cos^{4}x\leq\frac{3}{4}.$

3. Oblicz sumę wyrazów nieskończonego ciągu geometrycznego, $\mathrm{w}$ którym $\alpha_{1}=1$, a $\mathrm{k}\mathrm{a}\dot{\mathrm{z}}\mathrm{d}\mathrm{y}$

kolejny wyraz jest pofowq róznicy wyrazu następnego $\mathrm{i}$ poprzedniego..

4. Narysuj wykres funkcji $f(x)=\{$

$2^{-x},$

$\log_{2}|x\sqrt{2}|,$

gdy

gdy

$|x+1|\leq 2,$

$|x+1|>2.$

Na podstawie wykresu wyznacz zbiór wartości funkcji $f(x)\mathrm{i}\mathrm{s}$prawd $\acute{\mathrm{z}}$, wjakich punktach

jest ona ciągła. Odpowied $\acute{\mathrm{z}}$ poprzyj odpowiednim rachunkiem.

5. Okręgi $0$ promieniach $r<R$ sq styczne zewnętrznie $\mathrm{w}$ punkcie $M\mathrm{i}$ styczne do prostej $l$

$\mathrm{w}$ punktach A $\mathrm{i}B$. Wyznacz pole trójk$\Phi$ta $ABM\mathrm{w}$ zalezności od $r\mathrm{i}R.$

6. $\mathrm{W}$ ostrosłupie prawidfowym trójkątnym krawędz/ boczna jest nachylona do podstawy

pod kątem $60^{\mathrm{o}}$ Oblicz stosunek objętości kuli wpisanej do objętości kuli opisanej na

ostroslupie.

$\mathrm{R}\mathrm{o}\mathrm{z}\mathrm{w}\mathrm{i}_{\Phi}$zania (rękopis) zadań $\mathrm{z}$ wybranego poziomu prosimy nadsylač do

$2021\mathrm{r}$. na adres:

31 grudnia

Wydziaf Matematyki

Politechnika Wrocfawska

Wybrzez $\mathrm{e}$ Wyspiańskiego 27

$50-370$ WROCLAW,

lub elektronicznie, za pośrednictwem portalu talent. pwr. edu. pl

Na kopercie prosimy $\underline{\mathrm{k}\mathrm{o}\mathrm{n}\mathrm{i}\mathrm{e}\mathrm{c}\mathrm{z}\mathrm{n}\mathrm{i}\mathrm{e}}$ zaznaczyč wybrany poziom! (np. poziom podsta-

wowy lub rozszerzony). Do rozwiązań nalez $\mathrm{y}$ dołączyč zaadresowaną do siebie koperte

zwrotną $\mathrm{z}$ naklejonym znaczkiem, odpowiednim do formatu listu. Polecamy stosowanie

kopert formatu C5 $(160\mathrm{x}230\mathrm{m}\mathrm{m})$ ze znaczkiem $0$ wartości 3,30 zł. Na $\mathrm{k}\mathrm{a}\dot{\mathrm{z}}$ dą wiekszą

kopertę nalez $\mathrm{y}$ nakleič drozszy znaczek. Prace niespełniające podanych warunków nie

będą poprawiane ani odsyłane.

Uwaga. Wysyłając nam rozwiązania zadań uczestnik Kursu udostępnia Politechnice Wrocfawskiej

swoje dane osobowe, które przetwarzamy wyłącznie $\mathrm{w}$ zakresie niezbędnym do jego prowadzenia

(odesłanie zadań, prowadzenie statystyki). Szczególowe informacje $0$ przetwarzaniu przez nas danych

osobowych są dostępne na stronie internetowej Kursu.

Adres internetowy Kursu: http: //www. im. pwr. edu. pl/kurs



\end{document}