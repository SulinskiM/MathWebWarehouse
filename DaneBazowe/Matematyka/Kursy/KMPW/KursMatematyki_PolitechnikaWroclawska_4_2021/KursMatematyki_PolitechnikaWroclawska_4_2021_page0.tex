\documentclass[a4paper,12pt]{article}
\usepackage{latexsym}
\usepackage{amsmath}
\usepackage{amssymb}
\usepackage{graphicx}
\usepackage{wrapfig}
\pagestyle{plain}
\usepackage{fancybox}
\usepackage{bm}

\begin{document}

LI KORESPONDENCYJNY KURS

Z MATEMATYKI

grudzień 2021 r.

PRACA KONTROLNA nr 4- POZIOM PODSTAWOWY

l. Trzy liczby naturalne $0$ iloczynie 80 tworzą ciąg arytmetyczny. $\mathrm{J}\mathrm{e}\dot{\mathrm{z}}$ eli drugi wyraz tego

ciągu zmniejszymy $0 1$, to liczby te (rozwazane $\mathrm{w}$ tej samej kolejności) utworzą ciąg

geometryczny. Jakie to liczby?

2. Liczby dodatnie $a, b$ spełniają warunek $\alpha^{2}+b^{2}=7ab$. Wykaz, $\dot{\mathrm{z}}\mathrm{e}$

$\log_{3}a+\log_{3}b+2=2\log_{3}(a+b).$

3. Rozwiąz równanie

tg2 {\it x}$=$ -11 $+$-csoins {\it xx}.

4. Narysuj wykres funkcji

$f(x)=\{$

$\displaystyle \frac{2}{3}x^{2}-\frac{8}{3}x+2,$

$|4-2|x-3||,$

gdy

gdy

$|2x-5|\leq 3,$

$|2x-5|>3.$

Na jego podstawie wyznacz: zbiór wartości funkcji $f(x)$ oraz liczbę rozwiqzań równania

$f(x)=m \mathrm{w}$ zalezności od parametru $m.$

5. Punkt $A(0,0)$ jest wierzchołkiem ośmiokąta foremnego wpisanego $\mathrm{w}$ okrąg $x^{2}-2x+y^{2}=0.$

Wyznacz współrzedne pozostafych wierzchołków.

6. Przekrój ostrosfupa prawidfowego $\mathrm{c}\mathrm{z}\mathrm{w}\mathrm{o}\mathrm{r}\mathrm{o}\mathrm{k}_{\Phi^{\mathrm{t}}}$nego plaszczyzną $\mathrm{p}\mathrm{r}\mathrm{z}\mathrm{e}\mathrm{c}\mathrm{h}\mathrm{o}\mathrm{d}\mathrm{z}\text{ą}_{\mathrm{C}\Phi}$ przez wierz-

chołek $\mathrm{i}$ przekątną jego podstawy jest trójkqtem równobocznym. $\mathrm{W}$ ostrosłup wpisano

sześcian, którego dolna podstawa jest zawarta $\mathrm{w}$ podstawie ostrosfupa, a wierzchołki

górnej podstawy sześcianu lezą na krawędziach ostrosłupa. Oblicz stosunek objętości

sześcianu do objetości ostroslupa.
\end{document}
