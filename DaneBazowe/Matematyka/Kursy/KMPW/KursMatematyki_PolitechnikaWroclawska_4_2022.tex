\documentclass[a4paper,12pt]{article}
\usepackage{latexsym}
\usepackage{amsmath}
\usepackage{amssymb}
\usepackage{graphicx}
\usepackage{wrapfig}
\pagestyle{plain}
\usepackage{fancybox}
\usepackage{bm}

\begin{document}

LII

KORESPONDENCYJNY KURS

Z MATEMATYKI

grudzień 2022 r.

PRACA KONTROLNA $\mathrm{n}\mathrm{r} 4-$ POZIOM PODSTAWOWY

l. Wyznacz miarę kąta ostrego $\alpha$, wiedząc, $\dot{\mathrm{z}}\mathrm{e} \displaystyle \cos\alpha+\sin\alpha=\frac{1}{\sin\alpha}.$

2. Dane są wierzchofki $A(-1,-2)\mathrm{i}B(6,-1)$ równolegfoboku, którego $\mathrm{P}^{\mathrm{r}\mathrm{z}\mathrm{e}\mathrm{k}}\Phi^{\mathrm{t}\mathrm{n}\mathrm{e}}$ przecinają

się $\mathrm{w}$ punkcie $S(4,0)$. Wyznacz współrzędne pozostałych wierzchołków $\mathrm{i}$ oblicz pole

równolegfoboku.

3. Trójkqt prostokątny $0$ polu 30 jest opisany na okręgu $0$ promieniu 2. Wyznacz dfugości

jego boków.

4. Cięciwy AB $\mathrm{i}CD$ (punkt $C\mathrm{l}\mathrm{e}\dot{\mathrm{z}}\mathrm{y}$ na łuku AB) przecinaj $\Phi$ się pod $\mathrm{k}_{\Phi^{\mathrm{t}}}\mathrm{e}\mathrm{m}$ prostym $\mathrm{w}$ punk-

cie $S$. Pole trójkąta $BSD$ jest równe 4, a po1e trójkąta $ASC$ wynosi 9. Ob1icz po1e

czworokąta ADBC, $\mathrm{j}\mathrm{e}\dot{\mathrm{z}}$ eli suma długości tych cięciw jest równa 15.

5. Dane $\mathrm{s}\Phi$ punkty $A(8,2)\mathrm{i}B(1,6)$. Punkt $C\mathrm{l}\mathrm{e}\dot{\mathrm{z}}\mathrm{y}$ najednej $\mathrm{z}$ osi ukfadu ijest wierzchofkiem

kata prostego $\mathrm{w}$ trójkącie $ABC$. Wyznacz współrzedne punku $C.$

6. $\mathrm{W}$ ostrosłupie prawidlowym trójk$\Phi$tnym zachodzi równośč $\cos\alpha=\sqrt{3}\cos\beta$, gdzie $\alpha$ jest

kątem nachylenia krawędzi bocznej, a $\beta$- kątem nachylenia ściany bocznej do podstawy.

Wykaz, $\dot{\mathrm{z}}\mathrm{e}$ ten ostrosłup jest czworościanem foremnym.




PRACA KONTROLNA $\mathrm{n}\mathrm{r} 4-$ POZIOM ROZSZERZONY

l. Wiedzqc, $\displaystyle \dot{\mathrm{z}}\mathrm{e}\sin 2x=-\frac{3}{4} \mathrm{i}  x\in (\displaystyle \frac{\pi}{2},\pi)$, oblicz wartośč wyrazenia

$\displaystyle \frac{\sin(3x+30^{\mathrm{o}})-\sin(x-30^{\mathrm{o}})}{4\cos^{2}x-2}.$

2. Wektory $\vec{u}, \vec{v}$ mają długośč l $\mathrm{i}$ tworzą kąt $60^{\mathrm{o}}$ Oblicz dlugości przekątnych równoległo-

boku rozpiętego na wektorach $(2\vec{u}-\vec{v})\mathrm{i}(\vec{u}-2\vec{v})$. Wyznacz jego kąt ostry $\mathrm{i}\mathrm{s}$prawd $\acute{\mathrm{z}},$

czy $\mathrm{m}\mathrm{o}\dot{\mathrm{z}}$ na $\mathrm{w}$ ten równoleglobok wpisač okrąg. $\mathrm{J}\mathrm{e}\dot{\mathrm{z}}$ eli $\mathrm{t}\mathrm{a}\mathrm{k}$, to oblicz jego promień.

3. Przekątne trapezu ABCD przecinają się $\mathrm{w}$ takim punkcie $P, \dot{\mathrm{z}}\mathrm{e}$

$|AP|^{2}+|BP|^{2}-|AB|^{2}=\displaystyle \frac{2\sqrt{5}}{3}|AP||BP|.$

$\mathrm{O}$ ile dluzszy jest promień okręgu opisanego na trójkącie $ABP$ od promienia okręgu opi-

sanego na trójkącie $PCD, \mathrm{j}\mathrm{e}\dot{\mathrm{z}}$ eli $|AB|-|CD|=4$?

4. Na okręgu $x^{2}+y^{2}-2x-2y=0$, opisany jest trapez prostokątny ABCD $0$ polu 12.

Wyznacz współrzedne wierzchołków trapezu, wiedzqc, $\dot{\mathrm{z}}\mathrm{e}$ wieksza $\mathrm{z}$ jego podstaw $AB$

jest zawarta jest $\mathrm{w}$ prostej $x+y=0$, a kąt przy wierzchofku $A$ jest prosty.

5. $\mathrm{W}$ trójkącie równoramiennym $ABC$ kąt przy wierzchołku $C$ ma miarę $20^{\mathrm{o}} \mathrm{Z}$ wierzchoł-

ków $A\mathrm{i}B$ poprowadzono półproste pod kqtami $50^{\mathrm{o}}\mathrm{i}60^{\mathrm{o}}$ wzgledem podstawy, przecina-

jące ramiona $AC\mathrm{i}BC\mathrm{w}$ punktach $D\mathrm{i}E$ odpowiednio. Wyznacz miarę $\mathrm{k}_{\Phi}\mathrm{t}\mathrm{a}BDE.$

$\mathrm{W}\mathrm{S}K.$ Poprowad $\acute{\mathrm{z}}$ półprosta $\mathrm{z}$ punktu $A$ przecinająca odcinek $BD\mathrm{w}$ punkcie $G$, a bok

$BC\mathrm{w}$ takim punkcie $F, \dot{\mathrm{z}}\mathrm{e}\angle BAF=60^{\mathrm{o}}\mathrm{i}$ przyjrzyj się czworokątowi DGEF.

6. $\mathrm{W}$ ostrosłupie prawidłowym trójkątnym krawęd $\acute{\mathrm{z}}$ boczna jest dwa razy dluzsza $\mathrm{n}\mathrm{i}\dot{\mathrm{z}}$ kra-

wedz' podstawy. Wyznacz cosinus kata między ścianami bocznymi ostrosłupa oraz sto-

sunek promienia kuli opisanej na ostrosfupie do promienia kuli wpisanej $\mathrm{w}$ ostroslup.

Rozwiązania (rękopis) zadań z wybranego poziomu prosimy nadsyfač do 31.12.2022r.

adres:

na

Wydziaf Matematyki

Politechnika Wrocfawska

Wybrzez $\mathrm{e}$ Wyspiańskiego 27

$50-370$ WROCLAW,

lub elektronicznie, za pośrednictwem portalu talent. $\mathrm{p}\mathrm{w}\mathrm{r}$. edu. pl

Na kopercie prosimy $\underline{\mathrm{k}\mathrm{o}\mathrm{n}\mathrm{i}\mathrm{e}\mathrm{c}\mathrm{z}\mathrm{n}\mathrm{i}\mathrm{e}}$ zaznaczyč wybrany poziom! (np. poziom podsta-

wowy lub rozszerzony). Do rozwiązań nalez $\mathrm{y}$ dołączyč zaadresowana do siebie koperte

zwrotną $\mathrm{z}$ naklejonym znaczkiem, odpowiednim do formatu listu. Prace niespełniające

podanych warunków nie będą poprawiane ani odsyłane.

Uwaga. Wysylajac nam rozwiazania zadań uczestnik Kursu udostępnia Politechnice Wroclawskiej

swoje dane osobowe, które przetwarzamy wyłącznie $\mathrm{w}$ zakresie niezbędnym do jego prowadzenia

(odesfanie zadań, prowadzenie statystyki). Szczegófowe informacje $0$ przetwarzaniu przez nas danych

osobowych $\mathrm{S}\otimes$ dostępne na stronie internetowej Kursu.

Adres internetowy Kursu: http://www.im.pwr.edu.pl/kurs



\end{document}