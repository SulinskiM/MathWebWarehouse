\documentclass[a4paper,12pt]{article}
\usepackage{latexsym}
\usepackage{amsmath}
\usepackage{amssymb}
\usepackage{graphicx}
\usepackage{wrapfig}
\pagestyle{plain}
\usepackage{fancybox}
\usepackage{bm}

\begin{document}

LII

KORESPONDENCYJNY KURS

Z MATEMATYKI

grudzień 2022 r.

PRACA KONTROLNA $\mathrm{n}\mathrm{r} 4-$ POZIOM PODSTAWOWY

l. Wyznacz miarę kąta ostrego $\alpha$, wiedząc, $\dot{\mathrm{z}}\mathrm{e} \displaystyle \cos\alpha+\sin\alpha=\frac{1}{\sin\alpha}.$

2. Dane są wierzchofki $A(-1,-2)\mathrm{i}B(6,-1)$ równolegfoboku, którego $\mathrm{P}^{\mathrm{r}\mathrm{z}\mathrm{e}\mathrm{k}}\Phi^{\mathrm{t}\mathrm{n}\mathrm{e}}$ przecinają

się $\mathrm{w}$ punkcie $S(4,0)$. Wyznacz współrzędne pozostałych wierzchołków $\mathrm{i}$ oblicz pole

równolegfoboku.

3. Trójkqt prostokątny $0$ polu 30 jest opisany na okręgu $0$ promieniu 2. Wyznacz dfugości

jego boków.

4. Cięciwy AB $\mathrm{i}CD$ (punkt $C\mathrm{l}\mathrm{e}\dot{\mathrm{z}}\mathrm{y}$ na łuku AB) przecinaj $\Phi$ się pod $\mathrm{k}_{\Phi^{\mathrm{t}}}\mathrm{e}\mathrm{m}$ prostym $\mathrm{w}$ punk-

cie $S$. Pole trójkąta $BSD$ jest równe 4, a po1e trójkąta $ASC$ wynosi 9. Ob1icz po1e

czworokąta ADBC, $\mathrm{j}\mathrm{e}\dot{\mathrm{z}}$ eli suma długości tych cięciw jest równa 15.

5. Dane $\mathrm{s}\Phi$ punkty $A(8,2)\mathrm{i}B(1,6)$. Punkt $C\mathrm{l}\mathrm{e}\dot{\mathrm{z}}\mathrm{y}$ najednej $\mathrm{z}$ osi ukfadu ijest wierzchofkiem

kata prostego $\mathrm{w}$ trójkącie $ABC$. Wyznacz współrzedne punku $C.$

6. $\mathrm{W}$ ostrosłupie prawidlowym trójk$\Phi$tnym zachodzi równośč $\cos\alpha=\sqrt{3}\cos\beta$, gdzie $\alpha$ jest

kątem nachylenia krawędzi bocznej, a $\beta$- kątem nachylenia ściany bocznej do podstawy.

Wykaz, $\dot{\mathrm{z}}\mathrm{e}$ ten ostrosłup jest czworościanem foremnym.
\end{document}
