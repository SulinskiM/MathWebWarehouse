\documentclass[a4paper,12pt]{article}
\usepackage{latexsym}
\usepackage{amsmath}
\usepackage{amssymb}
\usepackage{graphicx}
\usepackage{wrapfig}
\pagestyle{plain}
\usepackage{fancybox}
\usepackage{bm}

\begin{document}

XLIV

KORESPONDENCYJNY KURS

Z MATEMATYKI

styczeń 2015 r.

PRACA KONTROLNA nr 5- POZIOM PODSTAWOWY

1. $\mathrm{W}$ ciągu arytmetycznym suma wyrazów od drugiego do piqtego wynosi 50 $\mathrm{i}$ jest ona

równa iloczynowi wyrazu czwartego $\mathrm{i}\mathrm{p}\mathrm{i}_{\Phi}$tego. Znajd $\acute{\mathrm{z}}$ pierwszy wyraz $\mathrm{i}$ róznicę $\mathrm{c}\mathrm{i}_{\Phi \mathrm{g}}\mathrm{u}.$

2. Punkt A(l, l) jest wierzchołkiem trójkąta równobocznego wpisanego w okrag 0 środku

w punkcie (2, 0). Wyznacz współrzędne pozostałych wierzchofków trójkąta. Rozwiązanie

zilustruj starannym rysunkiem.

3. $\mathrm{W}$ konkursie matematycznym trzy $\mathrm{P}^{\mathrm{o}\mathrm{c}\mathrm{z}}\Phi^{\mathrm{t}\mathrm{k}\mathrm{o}\mathrm{w}\mathrm{e}}$ miejsca zostafy przyznane Asi, Basi, Kasi,

Kamilowi $\mathrm{i}$ Rafałowi. Ilejest $\mathrm{m}\mathrm{o}\dot{\mathrm{z}}$ liwych rozstrzygnięč konkursu, $\mathrm{j}\mathrm{e}\dot{\mathrm{z}}$ eli wiadomo, $\dot{\mathrm{z}}\mathrm{e}\mathrm{k}\mathrm{a}\dot{\mathrm{z}}$ de

$\mathrm{z}$ miejsc I- III zostało przyznane?

4. Opisz równaniem $\mathrm{i}$ narysuj $\mathrm{w}$ układzie wspólrzędnych zbiór punktów płaszczyzny, któ-

rych odległośč od punktu $A(-2,-1)$ jest dwa razy większa od odleglości od punktu

$B(1,2).$

5. Rozwiqz nierównośč

$5^{1-x^{4}}\cdot 2^{x^{2}(x^{2}-1)}>16^{x^{2}-1}\cdot 5^{5-5x^{2}}$

6. Wyznacz wszystkie liczby $x\mathrm{z}$ przedziału $[0,2\pi]$ spelniajqce równanie

1$+$2 $\displaystyle \sin x+2^{2}\sin^{2}x+\cdots+2^{n-1}\sin^{n-1}x=\frac{1-2^{n}\sin^{n}x}{1-\sqrt{2}\sin 2x}$

dla $\mathrm{k}\mathrm{a}\dot{\mathrm{z}}$ dej liczby naturalnej $n\geq 1.$
\end{document}
