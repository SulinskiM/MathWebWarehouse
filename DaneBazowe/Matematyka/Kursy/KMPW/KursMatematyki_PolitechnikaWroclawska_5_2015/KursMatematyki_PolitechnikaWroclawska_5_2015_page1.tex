\documentclass[a4paper,12pt]{article}
\usepackage{latexsym}
\usepackage{amsmath}
\usepackage{amssymb}
\usepackage{graphicx}
\usepackage{wrapfig}
\pagestyle{plain}
\usepackage{fancybox}
\usepackage{bm}

\begin{document}

PRACA KONTROLNA nr 5- POZIOM ROZSZERZONY

l. Wyznacz wszystkie liczby rzeczywiste x, dla których funkcja

$f(x)=\displaystyle \frac{x^{2}-\sqrt{2-x}}{x-1}-x$

przyjmuje wartości nieujemne.

2. Rozwiąz równanie

$1+3^{-3\sin^{2}x}+3^{-6\sin^{2}x}+3^{-9\sin^{2}x}+\displaystyle \cdots=\frac{3}{3-3^{\sin^{2}x}},$

którego lewa strona jest sumą nieskończonego ciqgu geometrycznego.

3. Danajest liczba $\alpha\in(0,1)\cup(1,\infty)$ oraz ciąg liczbowy $(a_{n})$ taki, $\dot{\mathrm{z}}\mathrm{e}a=2^{a_{1}}$ oraz $a= \sqrt[n]{2^{a_{n}}}$

dla $\mathrm{k}\mathrm{a}\dot{\mathrm{z}}$ dego naturalnego $n$. Wyznacz liczbę naturalną $m$, dla której suma $m\mathrm{P}^{\mathrm{o}\mathrm{c}\mathrm{z}}\Phi^{\mathrm{t}\mathrm{k}\mathrm{o}-}$

wych wyrazów ciągu $(a_{n})$ jest 5050 razy większa od pierwszego wyrazu.

4. Drzewa rosnące przed galerią handlową $\mathrm{z}\mathrm{o}\mathrm{s}\tan\Phi$ przed świętami ozdobione jednobarwny-

mi diodami LED. Na ile sposobów $\mathrm{m}\mathrm{o}\dot{\mathrm{z}}$ na wykonač iluminację świątecznq, jeśli wiadomo,

$\dot{\mathrm{z}}\mathrm{e}$ drzew jest 6, $\mathrm{k}\mathrm{a}\dot{\mathrm{z}}$ de drzewo zostanie podświetlone na jeden $\mathrm{z}3$ kolorów, a $\mathrm{k}\mathrm{a}\dot{\mathrm{z}}\mathrm{d}\mathrm{y}$ kolor

zostanie wykorzystany co najmniej $\mathrm{r}\mathrm{a}\mathrm{z}$?

5. Krzywa $\Gamma$ jest zbiorem punktów pfaszczyzny, których odleglośč od punktu $A(-\displaystyle \frac{2}{3},0)$

jest trzy razy mniejsza od odległości od punktu $B(2,-8)$. Opisz krzywą równaniem

$\mathrm{i}$ zbadaj, dla jakich wartości rzeczywistego parametru $m$ prosta

$mx-y-3m-1=0$

ma dokładnie 2 punkty wspó1ne $\mathrm{z}$ krzywą $\Gamma. \mathrm{R}\mathrm{o}\mathrm{z}\mathrm{w}\mathrm{i}_{\Phi}$zanie zilustruj rysunkiem.

6. Rozwiąz nierównośč

$\sqrt{\frac{1}{2}\log_{2}(x^{4}-2x^{3}+x^{2})}\geq 4\log_{4}\sqrt{x^{2}-x}.$

Rozwiqzania prosimy nadsyłač do dnia

18 stycznia 20l5 na adres:

Katedra Matematyki WPPT

Politechniki Wrocfawskiej

Wybrzez $\mathrm{e}$ Wyspiańskiego 27

$50-370$ Wrocfaw.

Na kopercie prosimy koniecznie zaznaczyč wybrany poziom. Do rozwiązań nalez$\mathrm{y}$ do-

laczyč zaadresowan\S do siebie kopertę zwrotn\S z naklejonym znaczkiem, odpowiednim do wagi listu.

Prace $\mathrm{n}\mathrm{i}\mathrm{e}\mathrm{s}\mathrm{p}\mathrm{e}l\mathrm{n}\mathrm{i}\mathrm{a}\mathrm{j}_{\Phi}\mathrm{c}\mathrm{e}$ podanych warunków nie będą poprawiane ani odsyłane.

Adres internetowy Kursu:

http://www. im.pwr.edu.pl/kurs
\end{document}
