\documentclass[10pt]{article}
\usepackage[polish]{babel}
\usepackage[utf8]{inputenc}
\usepackage[T1]{fontenc}
\usepackage{amsmath}
\usepackage{amsfonts}
\usepackage{amssymb}
\usepackage[version=4]{mhchem}
\usepackage{stmaryrd}
\usepackage{hyperref}
\hypersetup{colorlinks=true, linkcolor=blue, filecolor=magenta, urlcolor=cyan,}
\urlstyle{same}

\title{PRACA KONTROLNA nr 5 - POZIOM PODSTAWOWY }

\author{}
\date{}


\begin{document}
\maketitle
\begin{enumerate}
  \item W ciągu arytmetycznym suma wyrazów od drugiego do piątego wynosi 50 i jest ona równa iloczynowi wyrazu czwartego i piątego. Znajdź pierwszy wyraz i różnicę ciągu.
  \item Punkt $A(1,1)$ jest wierzchołkiem trójkąta równobocznego wpisanego w okrąg o środku w punkcie $(2,0)$. Wyznacz współrzędne pozostałych wierzchołków trójkąta. Rozwiązanie zilustruj starannym rysunkiem.
  \item W konkursie matematycznym trzy początkowe miejsca zostały przyznane Asi, Basi, Kasi, Kamilowi i Rafałowi. Ile jest możliwych rozstrzygnięć konkursu, jeżeli wiadomo, że każde z miejsc I - III zostało przyznane?
  \item Opisz równaniem i narysuj w układzie współrzędnych zbiór punktów płaszczyzny, których odległość od punktu $A(-2,-1)$ jest dwa razy większa od odległości od punktu $B(1,2)$.
  \item Rozwiąż nierówność
\end{enumerate}

$$
5^{1-x^{4}} \cdot 2^{x^{2}\left(x^{2}-1\right)}>16^{x^{2}-1} \cdot 5^{5-5 x^{2}}
$$

\begin{enumerate}
  \setcounter{enumi}{5}
  \item Wyznacz wszystkie liczby $x$ z przedziału $[0,2 \pi]$ spełniające równanie
\end{enumerate}

$$
1+2 \sin x+2^{2} \sin ^{2} x+\cdots+2^{n-1} \sin ^{n-1} x=\frac{1-2^{n} \sin ^{n} x}{1-\sqrt{2} \sin 2 x}
$$

dla każdej liczby naturalnej $n \geqslant 1$.

\section*{PRACA KONTROLNA nr 5 - POZIOM ROZSZERZONY}
\begin{enumerate}
  \item Wyznacz wszystkie liczby rzeczywiste $x$, dla których funkcja
\end{enumerate}

$$
f(x)=\frac{x^{2}-\sqrt{2-x}}{x-1}-x
$$

przyjmuje wartości nieujemne.\\
2. Rozwiąż równanie

$$
1+3^{-3 \sin ^{2} x}+3^{-6 \sin ^{2} x}+3^{-9 \sin ^{2} x}+\cdots=\frac{3}{3-3^{\sin ^{2} x}}
$$

którego lewa strona jest sumą nieskończonego ciągu geometrycznego.\\
3. Dana jest liczba $a \in(0,1) \cup(1, \infty)$ oraz ciąg liczbowy $\left(a_{n}\right)$ taki, że $a=2^{a_{1}}$ oraz $a=\sqrt[n]{2^{a_{n}}}$ dla każdego naturalnego $n$. Wyznacz liczbę naturalną $m$, dla której suma $m$ początkowych wyrazów ciągu $\left(a_{n}\right)$ jest 5050 razy większa od pierwszego wyrazu.\\
4. Drzewa rosnące przed galerią handlową zostaną przed świętami ozdobione jednobarwnymi diodami LED. Na ile sposobów można wykonać iluminację świąteczną, jeśli wiadomo, że drzew jest 6 , każde drzewo zostanie podświetlone na jeden z 3 kolorów, a każdy kolor zostanie wykorzystany co najmniej raz?\\
5. Krzywa $\Gamma$ jest zbiorem punktów płaszczyzny, których odległość od punktu $A\left(-\frac{2}{3}, 0\right)$ jest trzy razy mniejsza od odległości od punktu $B(2,-8)$. Opisz krzywą równaniem i zbadaj, dla jakich wartości rzeczywistego parametru $m$ prosta

$$
m x-y-3 m-1=0
$$

ma dokładnie 2 punkty wspólne z krzywą $\Gamma$. Rozwiązanie zilustruj rysunkiem.\\
6. Rozwiąż nierówność

$$
\sqrt{\frac{1}{2} \log _{2}\left(x^{4}-2 x^{3}+x^{2}\right)} \geqslant 4 \log _{4} \sqrt{x^{2}-x}
$$

Rozwiązania prosimy nadsyłać do dnia 18 stycznia 2015 na adres:

\begin{verbatim}
Katedra Matematyki WPPT
Politechniki Wrocławskiej
Wybrzeże Wyspiańskiego 27
50-370 Wrocław.
\end{verbatim}

Na kopercie prosimy koniecznie zaznaczyć wybrany poziom. Do rozwiązań należy dołączyć zaadresowaną do siebie kopertę zwrotną z naklejonym znaczkiem, odpowiednim do wagi listu. Prace niespełniające podanych warunków nie będą poprawiane ani odsyłane.

Adres internetowy Kursu: \href{http://www.im.pwr.edu.pl/kurs}{http://www.im.pwr.edu.pl/kurs}


\end{document}