\documentclass[a4paper,12pt]{article}
\usepackage{latexsym}
\usepackage{amsmath}
\usepackage{amssymb}
\usepackage{graphicx}
\usepackage{wrapfig}
\pagestyle{plain}
\usepackage{fancybox}
\usepackage{bm}

\begin{document}

XLV

KORESPONDENCYJNY KURS

Z MATEMATYKI

styczeń 2016 r.

PRACA KONTROLNA nr 5- POZIOM PODSTAWOWY

l. Udowodnič, $\dot{\mathrm{z}}\mathrm{e}$ róznica kwadratów dwu dowolnych liczb całkowitych niepodzielnych przez

3 jest podzielna przez 3.

2. Rozwiązač równanie

$\mathrm{w}$ przedziale $[0,2\pi].$

sin2(-$\pi+$2{\it x})-sin(-$\pi$-2{\it x})$+$sin2(-$\pi$-2{\it x})$=$1

3. Dla jakiego parametru $m$ równanie

$(\log_{2}^{2}m-1)\cdot x^{2}+2$ (log2 $m-1$)$\cdot x+2=0$

ma tylko jedno $\mathrm{r}\mathrm{o}\mathrm{z}\mathrm{w}\mathrm{i}_{\Phi}\mathrm{z}\mathrm{a}\mathrm{n}\mathrm{i}\mathrm{e}$?

4. Jedna $\mathrm{z}$ krawędzi bocznych ostrosfupa, którego podstawą jest kwadrat $0$ boku $a$, jest

prostopadła do podstawy. Najdluzsza krawędz/ boczna jest nachylona do podstawy pod

$\mathrm{k}_{\Phi}\mathrm{t}\mathrm{e}\mathrm{m}60^{\mathrm{o}}$. Obliczyč pole powierzchni calkowitej oraz sumę dfugości krawędzi ostrosfupa.

Sporządzič rysunek.

5. $\mathrm{J}\mathrm{a}\mathrm{k}_{\Phi}\mathrm{k}\mathrm{r}\mathrm{z}\mathrm{y}\mathrm{w}\Phi^{\mathrm{t}\mathrm{w}\mathrm{o}\mathrm{r}\mathrm{Z}}\Phi$ punkty plaszczyzny, $\mathrm{z}$ których odcinek $0$ końcach $A(1,0)\mathrm{i}B(0,1)$

jest widoczny pod kątem $30^{\mathrm{o}}$

6. Narysowač wykres funkcji $f(x)=\displaystyle \frac{|x+1|-1}{|x-1|}\mathrm{i}$ na jego podstawie wyznaczyč przedzialy

jej monotoniczności oraz najmniejszą wartośč $\mathrm{w}$ przedziale $[-2,\displaystyle \frac{1}{2}]$




PRACA KONTROLNA nr 4- POZ1OM ROZSZERZONY

l. Udowodnič $\mathrm{t}\mathrm{o}\dot{\mathrm{z}}$ samośč

$x^{3}+y^{3}+z^{3}-3xyz=(x+y+z)(x^{2}+y^{2}+z^{2}-xy-xz-yz)$

$\mathrm{i}$ wykorzystujqc ją, usunąč niewymiernośč $\mathrm{z}$ mianownika ułamka $\displaystyle \frac{1}{1+\sqrt[3]{2}+\sqrt[3]{4}}.$

2. Jaką krzywą tworzą środki wszystkich okręgów stycznych równocześnie do osi $Ox\mathrm{i}$ do

okręgu $x^{2}+(y-1)^{2}=1$ ?

3. Wyznaczyč wszystkie styczne do wykresu funkcji $f(x)=\displaystyle \frac{x-3}{x-2}$ prostopadłe do prostej

$4x+y+1=0$. Sporządzič staranne wykresy wszystkich rozwazanych krzywych.

4. Narysowač wykres funkcji

$ f(x)=1-\displaystyle \frac{2^{x}}{1-2^{x}}+(\frac{2^{x}}{1-2^{x}})^{2}-(\frac{2^{x}}{1-2^{x}})^{3}+(\frac{2^{x}}{1-2^{x}})^{4}-\ldots$

bdcej sumą nieskończonego szeregu geometrycznego. Rozwiązač nierównośč

$f(x)>4^{x}-21\cdot 2^{x-2}+2.$

5. Dla jakiego parametru $m$ równanie

(log4 $m+1$)$\cdot x^{2}+3\log_{4}m\cdot x+2\log_{4}m-1=0$

ma dwa pierwiastki $x_{1}, x_{2}$ spefniające warunki: $x_{1}<x_{2}$, oraz $2(x_{2}^{2}-x_{1}^{2})>x_{2}^{4}-x_{1}^{4}$?

6. $\mathrm{W}$ ostrosłupie prawidfowym trójkątnym tangens kąta nachylenia ściany bocznej do pod-

stawy jest równy $2\sqrt{3}$. Obliczyč stosunek objętości kuli wpisanej do objętości kuli opi-

sanej na ostrosfupie.

Rozwiązania (rękopis) zadań z wybranego poziomu prosimy nadsylač do 18 stycznia 2016 r.

na adres:

Wydziaf Matematyki

Politechnika Wrocfawska

Wybrzez $\mathrm{e}$ Wyspiańskiego 27

$50-370$ WROCLAW.

Na kopercie prosimy $\underline{\mathrm{k}\mathrm{o}\mathrm{n}\mathrm{i}\mathrm{e}\mathrm{c}\mathrm{z}\mathrm{n}\mathrm{i}\mathrm{e}}$ zaznaczyč wybrany poziom! (np. poziom podsta-

wowy lub rozszerzony). Do rozwiązań nalez $\mathrm{y}$ dołączyč zaadresowaną do siebie kopertę

zwrotną $\mathrm{z}$ naklejonym znaczkiem, odpowiednim do wagi listu. Prace niespełniajace po-

danych warunków nie będą poprawiane ani odsyłane.

Adres internetowy Kursu: http://www.im.pwr.wroc.pl/kurs



\end{document}