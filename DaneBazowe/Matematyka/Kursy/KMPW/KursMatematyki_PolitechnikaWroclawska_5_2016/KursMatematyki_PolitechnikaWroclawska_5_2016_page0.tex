\documentclass[a4paper,12pt]{article}
\usepackage{latexsym}
\usepackage{amsmath}
\usepackage{amssymb}
\usepackage{graphicx}
\usepackage{wrapfig}
\pagestyle{plain}
\usepackage{fancybox}
\usepackage{bm}

\begin{document}

XLV

KORESPONDENCYJNY KURS

Z MATEMATYKI

styczeń 2016 r.

PRACA KONTROLNA nr 5- POZIOM PODSTAWOWY

l. Udowodnič, $\dot{\mathrm{z}}\mathrm{e}$ róznica kwadratów dwu dowolnych liczb całkowitych niepodzielnych przez

3 jest podzielna przez 3.

2. Rozwiązač równanie

$\mathrm{w}$ przedziale $[0,2\pi].$

sin2(-$\pi+$2{\it x})-sin(-$\pi$-2{\it x})$+$sin2(-$\pi$-2{\it x})$=$1

3. Dla jakiego parametru $m$ równanie

$(\log_{2}^{2}m-1)\cdot x^{2}+2$ (log2 $m-1$)$\cdot x+2=0$

ma tylko jedno $\mathrm{r}\mathrm{o}\mathrm{z}\mathrm{w}\mathrm{i}_{\Phi}\mathrm{z}\mathrm{a}\mathrm{n}\mathrm{i}\mathrm{e}$?

4. Jedna $\mathrm{z}$ krawędzi bocznych ostrosfupa, którego podstawą jest kwadrat $0$ boku $a$, jest

prostopadła do podstawy. Najdluzsza krawędz/ boczna jest nachylona do podstawy pod

$\mathrm{k}_{\Phi}\mathrm{t}\mathrm{e}\mathrm{m}60^{\mathrm{o}}$. Obliczyč pole powierzchni calkowitej oraz sumę dfugości krawędzi ostrosfupa.

Sporządzič rysunek.

5. $\mathrm{J}\mathrm{a}\mathrm{k}_{\Phi}\mathrm{k}\mathrm{r}\mathrm{z}\mathrm{y}\mathrm{w}\Phi^{\mathrm{t}\mathrm{w}\mathrm{o}\mathrm{r}\mathrm{Z}}\Phi$ punkty plaszczyzny, $\mathrm{z}$ których odcinek $0$ końcach $A(1,0)\mathrm{i}B(0,1)$

jest widoczny pod kątem $30^{\mathrm{o}}$

6. Narysowač wykres funkcji $f(x)=\displaystyle \frac{|x+1|-1}{|x-1|}\mathrm{i}$ na jego podstawie wyznaczyč przedzialy

jej monotoniczności oraz najmniejszą wartośč $\mathrm{w}$ przedziale $[-2,\displaystyle \frac{1}{2}]$
\end{document}
