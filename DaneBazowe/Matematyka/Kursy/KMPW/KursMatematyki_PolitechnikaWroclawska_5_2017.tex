\documentclass[a4paper,12pt]{article}
\usepackage{latexsym}
\usepackage{amsmath}
\usepackage{amssymb}
\usepackage{graphicx}
\usepackage{wrapfig}
\pagestyle{plain}
\usepackage{fancybox}
\usepackage{bm}

\begin{document}

XLVI

KORESPONDENCYJNY KURS

Z MATEMATYKI

styczeń 2017 r.

PRACA KONTROLNA nr 5- POZIOM PODSTAWOWY

1. $\mathrm{W}$ urnie znajduje się 9 ku1 ponumerowanych od 1 do 9. Losujemy bez zwracania 4

kule $\mathrm{i}$ dodajemy ich numery. Ile jest $\mathrm{m}\mathrm{o}\dot{\mathrm{z}}$ liwych wyników losowania, $\mathrm{w}$ których suma

wylosowanych numerów jest parzysta, a ile wyników losowania prowadzi do uzyskania

liczby nieparzystej?

2. Narysuj na płaszczy $\acute{\mathrm{z}}\mathrm{n}\mathrm{i}\mathrm{e}$ krzywą

$y=|2^{|x-1|}-2|$

i starannie opisz metodę jej konstrukcji.

3. Wyznacz dziedzinę funkcji

$f(x)=\sqrt{\log_{\frac{1}{2}}(2x-1)-2\log_{2}\frac{1}{x-2}}.$

4. Rozwiąz równanie

$(\displaystyle \frac{9}{4})^{x}(\frac{8}{27})^{x-2}\log(27-x)-3\log_{\frac{1}{10}}\frac{1}{\sqrt{27-x}}=0$

5. Narysuj w układzie współrzędnych zbiór

$A=\{(x,y)\in \mathbb{R}^{2}:\sqrt{(x^{2}-y)^{2}}+1<(|x|+1)^{2}\}.$

6. Wśród walców wpisanych w kulę 0 promieniu R wskaz ten 0 największym polu po-

wierzchni bocznej. Podaj jego wymiary oraz stosunek pola jego powierzchni cafkowitej

do pola powierzchni kuli.




PRACA KONTROLNA nr 5- POZIOM ROZSZERZONY

1. $\mathrm{W}$ finale pewnego konkursu bierze udział 10 osób. Prowadzący wybiera 1osowo jedną

$\mathrm{z}$ nich $\mathrm{i}$ zadaje jej pytanie finafowe. Obliczyč prawdopodobieństwo, $\dot{\mathrm{z}}\mathrm{e}$ zapytana osoba

udzieli poprawnej odpowiedzi, jeśli wiadomo, $\dot{\mathrm{z}}\mathrm{e}$ k-ty finalista odpowie poprawnie na

pytanie finałowe $\mathrm{z}$ prawdopodobieństwem $\displaystyle \frac{1}{2^{k}}$, gdzie $k\in\{1, \ldots$, 10$\}.$

2. Rozwiąz równanie

$\mathrm{x}^{\log_{3}x-1}=9.$

3. Zbadaj, dla jakich argumentów x funkcja

$f(x)=(2-x)^{\frac{3x-4}{2-x}}-1$

przyjmuje wartości ujemne.

4. Podaj dziedzinę $\mathrm{i}$ narysuj wykres funkcji

$f(x)=2|\log_{2}\sqrt{|x-1|-1}|.$

Starannie opisz metodę jego konstrukcji. Rozwiąz równanie $f(x)=2.$

5. Narysuj na płaszczyz/nie zbiór

$A=\{(x,y)\in \mathbb{R}^{2}$:

$\log_{|x|}(\log_{y+1}(|x|+1)) \leq 0\}.$

6. Wśród prostopadłościanów wpisanych $\mathrm{w}$ kulę $0$ promieniu $R$, których przekątna tworzy

kąt $\alpha \mathrm{z}$ jednq ze ścian, wskaz ten $0$ największej objętości. Podaj jego wymiary oraz

stosunek jego objętości do objętości kuli. Jaki procent objętości kuli stanowi objętośč

prostopadfościanu dla kąta $\alpha=45^{\mathrm{o}}$? Wynik podač $\mathrm{z}$ dokfadnością do jednego promila.

Rozwiązania prosimy nadsyłač do dnia

18 stycznia 20l7 na adres:

Wydziaf Matematyki

Politechniki Wrocfawskiej

Wybrzez $\mathrm{e}$ Wyspiańskiego 27

$50\rightarrow 370$ Wroclaw.

Na kopercie prosimy koniecznie zaznaczyč wybrany poziom. Do rozwiązań nalez$\mathrm{y}$ do-

f\S czyč zaadresowaną do siebie kopertę zwrotn\S z naklejonym znaczkiem, odpowiednim do wagi listu.

Prace niespelniające podanych warunków nie będą poprawiane ani odsyłane.

Adres internetowy Kursu:

http://www. im.pwr.edu.pl/kur s



\end{document}