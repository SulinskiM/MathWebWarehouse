\documentclass[a4paper,12pt]{article}
\usepackage{latexsym}
\usepackage{amsmath}
\usepackage{amssymb}
\usepackage{graphicx}
\usepackage{wrapfig}
\pagestyle{plain}
\usepackage{fancybox}
\usepackage{bm}

\begin{document}

XLVI

KORESPONDENCYJNY KURS

Z MATEMATYKI

styczeń 2017 r.

PRACA KONTROLNA nr 5- POZIOM PODSTAWOWY

1. $\mathrm{W}$ urnie znajduje się 9 ku1 ponumerowanych od 1 do 9. Losujemy bez zwracania 4

kule $\mathrm{i}$ dodajemy ich numery. Ile jest $\mathrm{m}\mathrm{o}\dot{\mathrm{z}}$ liwych wyników losowania, $\mathrm{w}$ których suma

wylosowanych numerów jest parzysta, a ile wyników losowania prowadzi do uzyskania

liczby nieparzystej?

2. Narysuj na płaszczy $\acute{\mathrm{z}}\mathrm{n}\mathrm{i}\mathrm{e}$ krzywą

$y=|2^{|x-1|}-2|$

i starannie opisz metodę jej konstrukcji.

3. Wyznacz dziedzinę funkcji

$f(x)=\sqrt{\log_{\frac{1}{2}}(2x-1)-2\log_{2}\frac{1}{x-2}}.$

4. Rozwiąz równanie

$(\displaystyle \frac{9}{4})^{x}(\frac{8}{27})^{x-2}\log(27-x)-3\log_{\frac{1}{10}}\frac{1}{\sqrt{27-x}}=0$

5. Narysuj w układzie współrzędnych zbiór

$A=\{(x,y)\in \mathbb{R}^{2}:\sqrt{(x^{2}-y)^{2}}+1<(|x|+1)^{2}\}.$

6. Wśród walców wpisanych w kulę 0 promieniu R wskaz ten 0 największym polu po-

wierzchni bocznej. Podaj jego wymiary oraz stosunek pola jego powierzchni cafkowitej

do pola powierzchni kuli.
\end{document}
