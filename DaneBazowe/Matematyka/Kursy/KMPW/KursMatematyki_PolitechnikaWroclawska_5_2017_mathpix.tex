\documentclass[10pt]{article}
\usepackage[polish]{babel}
\usepackage[utf8]{inputenc}
\usepackage[T1]{fontenc}
\usepackage{amsmath}
\usepackage{amsfonts}
\usepackage{amssymb}
\usepackage[version=4]{mhchem}
\usepackage{stmaryrd}
\usepackage{bbold}
\usepackage{hyperref}
\hypersetup{colorlinks=true, linkcolor=blue, filecolor=magenta, urlcolor=cyan,}
\urlstyle{same}

\title{PRACA KONTROLNA nr 5 - POZIOM PODSTAWOWY }

\author{}
\date{}


\begin{document}
\maketitle
\begin{enumerate}
  \item W urnie znajduje się 9 kul ponumerowanych od 1 do 9 . Losujemy bez zwracania 4 kule i dodajemy ich numery. Ile jest możliwych wyników losowania, w których suma wylosowanych numerów jest parzysta, a ile wyników losowania prowadzi do uzyskania liczby nieparzystej?
  \item Narysuj na płaszczyźnie krzywą
\end{enumerate}

$$
y=\left|2^{|x-1|}-2\right|
$$

i starannie opisz metodę jej konstrukcji.\\
3. Wyznacz dziedzinę funkcji

$$
f(x)=\sqrt{\log _{\frac{1}{2}}(2 x-1)-2 \log _{2} \frac{1}{x-2}}
$$

\begin{enumerate}
  \setcounter{enumi}{3}
  \item Rozwiąż równanie
\end{enumerate}

$$
\left(\frac{9}{4}\right)^{x}\left(\frac{8}{27}\right)^{x-2} \log (27-x)-3 \log _{\frac{1}{10}} \frac{1}{\sqrt{27-x}}=0
$$

\begin{enumerate}
  \setcounter{enumi}{4}
  \item Narysuj w układzie współrzędnych zbiór
\end{enumerate}

$$
A=\left\{(x, y) \in \mathbb{R}^{2}: \sqrt{\left(x^{2}-y\right)^{2}}+1<(|x|+1)^{2}\right\}
$$

\begin{enumerate}
  \setcounter{enumi}{5}
  \item Wśród walców wpisanych w kulę o promieniu $R$ wskaż ten o największym polu powierzchni bocznej. Podaj jego wymiary oraz stosunek pola jego powierzchni całkowitej do pola powierzchni kuli.
\end{enumerate}

\section*{PRACA KONTROLNA nr 5 - POZIOM ROZSZERZONY}
\begin{enumerate}
  \item W finale pewnego konkursu bierze udział 10 osób. Prowadzący wybiera losowo jedną z nich i zadaje jej pytanie finałowe. Obliczyć prawdopodobieństwo, że zapytana osoba udzieli poprawnej odpowiedzi, jeśli wiadomo, że $k$-ty finalista odpowie poprawnie na pytanie finałowe z prawdopodobieństwem $\frac{1}{2^{k}}$, gdzie $k \in\{1, \ldots, 10\}$.
  \item Rozwiąż równanie
\end{enumerate}

$$
x^{\log _{3} x-1}=9 .
$$

\begin{enumerate}
  \setcounter{enumi}{2}
  \item Zbadaj, dla jakich argumentów $x$ funkcja
\end{enumerate}

$$
f(x)=(2-x)^{\frac{3 x-4}{2-x}}-1
$$

przyjmuje wartości ujemne.\\
4. Podaj dziedzinę i narysuj wykres funkcji

$$
f(x)=2\left|\log _{2} \sqrt{|x-1|-1}\right| .
$$

Starannie opisz metodę jego konstrukcji. Rozwią̇̇ równanie $f(x)=2$.\\
5. Narysuj na płaszczyźnie zbiór

$$
A=\left\{(x, y) \in \mathbb{R}^{2}: \log _{|x|}\left(\log _{y+1}(|x|+1)\right) \leqslant 0\right\}
$$

\begin{enumerate}
  \setcounter{enumi}{5}
  \item Wśród prostopadłościanów wpisanych w kulę o promieniu $R$, których przekątna tworzy kąt $\alpha$ z jedną ze ścian, wskaż ten o największej objętości. Podaj jego wymiary oraz stosunek jego objętości do objętości kuli. Jaki procent objętości kuli stanowi objętość prostopadłościanu dla kąta $\alpha=45^{\circ}$ ? Wynik podać z dokładnością do jednego promila.
\end{enumerate}

Rozwiązania prosimy nadsyłać do dnia 18 stycznia 2017 na adres:

\begin{verbatim}
Wydział Matematyki
Politechniki Wrocławskiej
Wybrzeże Wyspiańskiego 27
50-370 Wrocław.
\end{verbatim}

Na kopercie prosimy koniecznie zaznaczyć wybrany poziom. Do rozwiązań należy dołączyć zaadresowaną do siebie kopertę zwrotną z naklejonym znaczkiem, odpowiednim do wagi listu. Prace niespełniające podanych warunków nie będą poprawiane ani odsyłane.

Adres internetowy Kursu: \href{http://www.im.pwr.edu.pl/kurs}{http://www.im.pwr.edu.pl/kurs}


\end{document}