\documentclass[a4paper,12pt]{article}
\usepackage{latexsym}
\usepackage{amsmath}
\usepackage{amssymb}
\usepackage{graphicx}
\usepackage{wrapfig}
\pagestyle{plain}
\usepackage{fancybox}
\usepackage{bm}

\begin{document}

XLVII

KORESPONDENCYJNY KURS

Z MATEMATYKI

styczeń 2018 r.

PRACA KONTROLNA $\mathrm{n}\mathrm{r} 5-$ POZIOM PODSTAWOWY

l. Rozwiązač równanie $3^{\log_{\sqrt{3}}(2^{x}-1)}=2^{x+1}+1.$

2. Jaki zbiór tworzą środki wszystkich cięciw $\mathrm{P}^{\mathrm{r}\mathrm{z}\mathrm{e}\mathrm{c}\mathrm{h}\mathrm{o}\mathrm{d}\mathrm{z}}\Phi^{\mathrm{c}\mathrm{y}\mathrm{c}\mathrm{h}}$ przez ustalony punkt zada-

nego okręgu?

3. Narysowač wykres funkcji $f(x) = \displaystyle \frac{|x+2|-1}{x-1}$. Wyznaczyč zbiór jej wartości oraz naj-

mniejszą $\mathrm{i}$ największą wartośč na przedziale $[$-3, $0].$

4. Niech $T$ będzie przeksztalceniem płaszczyzny polegającym na przesunięciu $0$ wektor

[1, 2], a $S-$ symetrią względem prostej $y=x$. Wyznaczyč (analitycznie) obrazy kwadratu

$0$ wierzchofkach $(0,1)$, (1, 1), (1, 2) $\mathrm{i}(0,2)\mathrm{w}$ przeksztafceniach $S0T\mathrm{i}T0S$. Sporz$\Phi$dzič

staranne rysunki.

5. Wspólne styczne do stycznych zewnętrznie okręgów $0$ promieniach $r<R$ przecinają się

pod kątem $ 2\alpha$. Wyznaczyč stosunek pól tych okręgów. Dla jakiego kąta $\alpha \mathrm{d}\mathrm{u}\dot{\mathrm{z}}\mathrm{e}$ kofo ma

9 razy większe pole $\mathrm{n}\mathrm{i}\dot{\mathrm{z}}$ małe?

6. Pole powierzchni cafkowitej ostroslupa prawidlowego trójk$\Phi$tnego jest 4 razy większe od

pola jego podstawy. Obliczyč sinus kąta między ścianami ostroslupa.




PRACA KONTROLNA nr 5- POZIOM ROZSZERZONY

1. $\mathrm{W}$ rozwinięciu $(a+b)^{n} = \displaystyle \sum_{k=0}^{n}\left(\begin{array}{l}
n\\
k
\end{array}\right)a^{n-k}b^{k}$ dla $a = \sqrt{x}, b = \displaystyle \frac{1}{2\sqrt[4]{x}}$ trzy pierwsze

wspólczynniki przy potęgach $x$ tworza ciag arytmetyczny. Znalez$\acute{}$č wszystkie składniki

rozwinięcia, $\mathrm{w}$ którym $x$ występuje $\mathrm{w}$ potędze $0$ wykladniku cafkowitym.

2. Punkty $K, L, M$ dzielą boki AB, $BC, CA$ trójkąta $ABC$ (odpowiednio) $\mathrm{w}$ tym samym

stosunku, $\mathrm{t}\mathrm{z}\mathrm{n}.$

$\displaystyle \frac{|KB|}{|AB|}=\frac{|LC|}{|BC|}=\frac{|MA|}{|CA|}=s$

Wykazač, $\dot{\mathrm{z}}\mathrm{e}$ dla dowolnego punktu $P$ znajdujacego się wewnqtrz trójkąta zachodzi rów-

nośč

$\vec{PK}+\vec{PL}+\vec{PM}=\vec{PA}+\vec{PB}+\vec{PC}.$

3. Narysowač wykres funkcji $f(x)=\displaystyle \frac{(x+1)^{2}-1}{x|x-1|}$. Wyznaczyč styczną do wykresu $\mathrm{w}$ punk-

cie $(-2,f(-2))$ oraz stycznq do niej prostopadła.

4. Końce odcinka AB $0$ długości $l\mathrm{p}\mathrm{o}\mathrm{r}\mathrm{u}\mathrm{s}\mathrm{z}\mathrm{a}\mathrm{j}_{\Phi}$ się po okręgu $0$ promieniu $R (l<2R)$. Na

odcinku obrano punkt $P\mathrm{t}\mathrm{a}\mathrm{k}, \dot{\mathrm{z}}\mathrm{e} \displaystyle \frac{|AP|}{|PB|} = \displaystyle \frac{1}{3}$. Uzasadnič, $\dot{\mathrm{z}}\mathrm{e}$ poruszający $\mathrm{s}\mathrm{i}\mathrm{e}$ punkt $P$

zakreśla okrąg $0$ tym samym środku. Dla jakiego $l$ wycięte $\mathrm{w}$ ten sposób kolo ma pole

dwa razy mniejsze od pola $\mathrm{d}\mathrm{u}\dot{\mathrm{z}}$ ego koła.

5. Rozwazamy zbiór wszystkich trójkątów $0$ polu 10, których jednym $\mathrm{z}$ wierzcholków jest

$A(5,0)$ a pozostale dwa $\mathrm{l}\mathrm{e}\dot{\mathrm{z}}\Phi$ na osi $Oy$. Wyznaczyč zbiór wszystkich punktów pfaszczy-

zny, które są środkami okręgów opisanych na tych trójkątach.

6. $\mathrm{W}$ przeciwlegle narozniki sześcianu $0$ boku l wpisano dwie kule $0$ takich samych pro-

mieniach $\mathrm{t}\mathrm{a}\mathrm{k}, \dot{\mathrm{z}}\mathrm{e}\mathrm{k}\mathrm{a}\dot{\mathrm{z}}$ da $\mathrm{z}$ nich jest styczna do drugiej $\mathrm{i}$ do trzech ścian wychodzących $\mathrm{z}$

odpowiedniego wierzchołka. Jaka jest odleglośč ich środków?

Rozwiązania (rękopis) zadań z wybranego poziomu prosimy nadsylač do

na adres:

18 stycznia 20l8r.

Wydziaf Matematyki

Politechnika Wrocfawska

Wybrzez $\mathrm{e}$ Wyspiańskiego 27

$50-370$ WROCLAW.

Na kopercie prosimy $\underline{\mathrm{k}\mathrm{o}\mathrm{n}\mathrm{i}\mathrm{e}\mathrm{c}\mathrm{z}\mathrm{n}\mathrm{i}\mathrm{e}}$ zaznaczyč wybrany poziom! (np. poziom podsta-

wowy lub rozszerzony). Do rozwiązań nalez $\mathrm{y}$ dołączyč zaadresowaną do siebie koperte

zwrotną $\mathrm{z}$ naklejonym znaczkiem, odpowiednim do wagi listu. Prace niespelniające po-

danych warunków nie będą poprawiane ani odsyłane.

Adres internetowy Kursu: http://www.im.pwr.wroc.pl/kurs



\end{document}