\documentclass[a4paper,12pt]{article}
\usepackage{latexsym}
\usepackage{amsmath}
\usepackage{amssymb}
\usepackage{graphicx}
\usepackage{wrapfig}
\pagestyle{plain}
\usepackage{fancybox}
\usepackage{bm}

\begin{document}

XLVII

KORESPONDENCYJNY KURS

Z MATEMATYKI

styczeń 2018 r.

PRACA KONTROLNA $\mathrm{n}\mathrm{r} 5-$ POZIOM PODSTAWOWY

l. Rozwiązač równanie $3^{\log_{\sqrt{3}}(2^{x}-1)}=2^{x+1}+1.$

2. Jaki zbiór tworzą środki wszystkich cięciw $\mathrm{P}^{\mathrm{r}\mathrm{z}\mathrm{e}\mathrm{c}\mathrm{h}\mathrm{o}\mathrm{d}\mathrm{z}}\Phi^{\mathrm{c}\mathrm{y}\mathrm{c}\mathrm{h}}$ przez ustalony punkt zada-

nego okręgu?

3. Narysowač wykres funkcji $f(x) = \displaystyle \frac{|x+2|-1}{x-1}$. Wyznaczyč zbiór jej wartości oraz naj-

mniejszą $\mathrm{i}$ największą wartośč na przedziale $[$-3, $0].$

4. Niech $T$ będzie przeksztalceniem płaszczyzny polegającym na przesunięciu $0$ wektor

[1, 2], a $S-$ symetrią względem prostej $y=x$. Wyznaczyč (analitycznie) obrazy kwadratu

$0$ wierzchofkach $(0,1)$, (1, 1), (1, 2) $\mathrm{i}(0,2)\mathrm{w}$ przeksztafceniach $S0T\mathrm{i}T0S$. Sporz$\Phi$dzič

staranne rysunki.

5. Wspólne styczne do stycznych zewnętrznie okręgów $0$ promieniach $r<R$ przecinają się

pod kątem $ 2\alpha$. Wyznaczyč stosunek pól tych okręgów. Dla jakiego kąta $\alpha \mathrm{d}\mathrm{u}\dot{\mathrm{z}}\mathrm{e}$ kofo ma

9 razy większe pole $\mathrm{n}\mathrm{i}\dot{\mathrm{z}}$ małe?

6. Pole powierzchni cafkowitej ostroslupa prawidlowego trójk$\Phi$tnego jest 4 razy większe od

pola jego podstawy. Obliczyč sinus kąta między ścianami ostroslupa.
\end{document}
