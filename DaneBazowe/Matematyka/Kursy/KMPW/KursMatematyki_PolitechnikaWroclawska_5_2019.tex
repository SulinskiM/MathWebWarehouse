\documentclass[a4paper,12pt]{article}
\usepackage{latexsym}
\usepackage{amsmath}
\usepackage{amssymb}
\usepackage{graphicx}
\usepackage{wrapfig}
\pagestyle{plain}
\usepackage{fancybox}
\usepackage{bm}

\begin{document}

XLVIII

KORESPONDENCYJNY KURS

Z MATEMATYKI

styczeń 2019 r.

PRACA KONTROLNA $\mathrm{n}\mathrm{r} 5-$ POZIOM PODSTAWOWY

l. Znalez$\acute{}$č stuelementowy ciag arytmetyczny, w którym suma wyrazów 0 numerach niepa-

rzystych jest dwa razy większa od sumy wyrazów 0 numerach parzystych io50 mniejsza

od sumy wszystkich wyrazów.

2. Rozwiązač układ równań 

$2^{y-1},$

$\log_{2}(x+2).$

3. Narysowač wykres funkcji $f(x) =x|x|-4|x|+3\mathrm{i}$ określič liczbę rozwiązań równania

$f(x)=m\mathrm{w}$ zalezności od parametru $m.$

4. $\mathrm{W}$ {\it romb ABCD} $0$ kącie ostrym $\alpha$ wpisano czworokąt, którego boki są równoległe do

przekqtnych rombu. Jakie jest $\mathrm{m}\mathrm{o}\dot{\mathrm{z}}$ liwie największe pole takiego czworokąta?

5. Znalez/č równania wspólnych stycznych do wykresów funkcji

$f(x)=-x^{2}+2x\mathrm{i}g(x)=x^{2}+1.$

6. $\mathrm{W}$ stozek $0$ promieniu podstawy $R$ wpisano stozek $0$ osiem razy mniejszej objętości.

Wysokośč malego stozka jest zawarta $\mathrm{w}$ wysokości $\mathrm{d}\mathrm{u}\dot{\mathrm{z}}$ ego stozka, jego wierzchołek jest

$\mathrm{w}$ środku podstawy, a okrąg ograniczający podstawę malego stozka jest zawarty $\mathrm{w}$ po-

wierzchni bocznej $\mathrm{d}\mathrm{u}\dot{\mathrm{z}}$ ego stozka. Obliczyč $\displaystyle \frac{r}{R}$, gdzie $r$ oznacza promień podstawy stozka

wpisanego.




PRACA KONTROLNA nr5 -P0Zi0M R0ZSZERZ0NY

l. Rozwiązač ukfad równań 

16,

16

2. Wyznaczyč równania wszystkich stycznych do wykresu funkcji

są prostopadłe do prostej $x+3y+1=0.$

$f(x) = \displaystyle \frac{2x-1}{x+1},$

które

3. Granicą ciągu $0$ wyrazie ogólnym $a_{n}=n^{2}-\sqrt{n^{4}-an^{2}+bn}$ jest większy $\mathrm{z}$ pierwiastków

równania

$x^{\log_{2}x}-3=4x^{\log_{\frac{1}{2}}x}$

Wyznaczyč parametry a $\mathrm{i}b.$

4. Na boku $BC$ trójkąta równobocznego obrano punkt $D\mathrm{t}\mathrm{a}\mathrm{k}, \dot{\mathrm{z}}\mathrm{e}$ promień okręgu wpisanego

$\mathrm{w}$ trójkąt $ADC$ jest dwa razy mniejszy $\mathrm{n}\mathrm{i}\dot{\mathrm{z}}$ promień okręgu wpisanego $\mathrm{w}$ trójkąt $ABD.$

$\mathrm{W}$ jakim stosunku punkt $D$ dzieli bok $BC$?

5. Rozwiązač nierównośč

$1+\displaystyle \frac{\sin x}{\sqrt{3}+\sin x}+(\frac{\sin x}{\sqrt{3}+\sin x})^{2}+(\frac{\sin x}{\sqrt{3}+\sin x})^{3}+\cdots\leq\cos x,$

której lewa strona jest sumą wszystkich wyrazów nieskończonego ciągu geometrycznego.

6. Jakie wymiary ma walec $\mathrm{o}\mathrm{m}\mathrm{o}\dot{\mathrm{z}}$ liwie największej objętości wpisany $\mathrm{w}$ sześcian $0$ boku $a$

$\mathrm{w}$ taki sposób, $\dot{\mathrm{z}}\mathrm{e}$ jego oś jest zawarta $\mathrm{w}$ przekątnej sześcianu, a $\mathrm{k}\mathrm{a}\dot{\mathrm{z}}$ da $\mathrm{z}$ podstaw jest

styczna do trzech ścian wychodzących $\mathrm{z}$ jednego wierzchofka.

Rozwiązania (rękopis) zadań z wybranego poziomu prosimy nadsyfač do

na adres:

18 stycznia 20l9r.

Wydziaf Matematyki

Politechnika Wrocfawska

Wybrzeže Wyspiańskiego 27

$50-370$ WROCLAW.

Na kopercie prosimy $\underline{\mathrm{k}\mathrm{o}\mathrm{n}\mathrm{i}\mathrm{e}\mathrm{c}\mathrm{z}\mathrm{n}\mathrm{i}\mathrm{e}}$ zaznaczyč wybrany poziom! (np. poziom podsta-

wowy lub rozszerzony). Do rozwiązań nalez $\mathrm{y}$ dołączyč zaadresowaną do siebie koperte

zwrotną $\mathrm{z}$ naklejonym znaczkiem, odpowiednim do wagi listu. Prace niespełniające po-

danych warunków nie będą poprawiane ani odsyłane.

Uwaga. Wysylając nam rozwiązania zadań uczestnik Kursu udostępnia nam swoje dane osobo-

we, które przetwarzamy wyłącznie $\mathrm{w}$ zakresie niezbędnym do jego prowadzenia (odesłanie zadań,

prowadzenie statystyki). Szczególowe informacje $0$ przetwarzaniu przez nas danych osobowych sq

dostępne na stronie internetowej Kursu.

Adres internetowy Kursu: http: //www. im. pwr. edu. pl/kurs



\end{document}