\documentclass[a4paper,12pt]{article}
\usepackage{latexsym}
\usepackage{amsmath}
\usepackage{amssymb}
\usepackage{graphicx}
\usepackage{wrapfig}
\pagestyle{plain}
\usepackage{fancybox}
\usepackage{bm}

\begin{document}

XLVIII

KORESPONDENCYJNY KURS

Z MATEMATYKI

styczeń 2019 r.

PRACA KONTROLNA $\mathrm{n}\mathrm{r} 5-$ POZIOM PODSTAWOWY

l. Znalez$\acute{}$č stuelementowy ciag arytmetyczny, w którym suma wyrazów 0 numerach niepa-

rzystych jest dwa razy większa od sumy wyrazów 0 numerach parzystych io50 mniejsza

od sumy wszystkich wyrazów.

2. Rozwiązač układ równań 

$2^{y-1},$

$\log_{2}(x+2).$

3. Narysowač wykres funkcji $f(x) =x|x|-4|x|+3\mathrm{i}$ określič liczbę rozwiązań równania

$f(x)=m\mathrm{w}$ zalezności od parametru $m.$

4. $\mathrm{W}$ {\it romb ABCD} $0$ kącie ostrym $\alpha$ wpisano czworokąt, którego boki są równoległe do

przekqtnych rombu. Jakie jest $\mathrm{m}\mathrm{o}\dot{\mathrm{z}}$ liwie największe pole takiego czworokąta?

5. Znalez/č równania wspólnych stycznych do wykresów funkcji

$f(x)=-x^{2}+2x\mathrm{i}g(x)=x^{2}+1.$

6. $\mathrm{W}$ stozek $0$ promieniu podstawy $R$ wpisano stozek $0$ osiem razy mniejszej objętości.

Wysokośč malego stozka jest zawarta $\mathrm{w}$ wysokości $\mathrm{d}\mathrm{u}\dot{\mathrm{z}}$ ego stozka, jego wierzchołek jest

$\mathrm{w}$ środku podstawy, a okrąg ograniczający podstawę malego stozka jest zawarty $\mathrm{w}$ po-

wierzchni bocznej $\mathrm{d}\mathrm{u}\dot{\mathrm{z}}$ ego stozka. Obliczyč $\displaystyle \frac{r}{R}$, gdzie $r$ oznacza promień podstawy stozka

wpisanego.
\end{document}
