\documentclass[a4paper,12pt]{article}
\usepackage{latexsym}
\usepackage{amsmath}
\usepackage{amssymb}
\usepackage{graphicx}
\usepackage{wrapfig}
\pagestyle{plain}
\usepackage{fancybox}
\usepackage{bm}

\begin{document}

XLIX

KORESPONDENCYJNY KURS

Z MATEMATYKI

styczeń 2020 r.

PRACA KONTROLNA $\mathrm{n}\mathrm{r} 5-$ POZIOM PODSTAWOWY

l. Załózmy, $\dot{\mathrm{z}}\mathrm{e}$ mamy 12 ku1 białych $\mathrm{i}9$ kul czarnych. Na ile sposobów $\mathrm{m}\mathrm{o}\dot{\mathrm{z}}$ na ustawič te

kule $\mathrm{w}$ rzędzie $\mathrm{w}$ taki sposób, aby $\dot{\mathrm{z}}$ adna czarna kula nie sąsiadowala $\mathrm{z}$ czarną? Na ile

róznych sposobów $\mathrm{m}\mathrm{o}\dot{\mathrm{z}}$ na ustawič te kule $\mathrm{w}$ rzędzie $\mathrm{w}$ taki sposób, aby $\dot{\mathrm{z}}$ adna czarna kula

nie sąsiadowafa $\mathrm{z}$ czarną, jeśli kule białe ponumerujemy kolejnymi liczbami parzystymi,

a kule czarne- kolejnymi liczbami nieparzystymi?

2. Ścianki kostki do gry oznaczono liczbami: -$3,$ -$2,$ -$1$, 1, 2, 3. Jakie jest prawdopodobień-

stwo zdarzenia, $\dot{\mathrm{z}}\mathrm{e}$ przy dwóch rzutach tą kostką: a) otrzymana suma liczb wynosi 2; b)

wartośč bezwzględna sumy liczb jest równa co najwyzej 3?

3. Wyznaczyč ciag arytmetyczny $0$ pierwszym wyrazie równym 2, wiedząc, $\dot{\mathrm{z}}\mathrm{e}$ wyrazy:

pierwszy, trzeci $\mathrm{i}$ jedenasty $\mathrm{w}$ podanej kolejności tworzą ciąg geometryczny. Ile pierw-

szych kolejnych wyrazów tego ciqgu nalezy dodač, aby otrzymana suma była większa

$\mathrm{n}\mathrm{i}\dot{\mathrm{z}}$ 1000?

4. $\mathrm{W}$ zbiorze $[0,2\pi]$ rozwiązač nierównośč

$\sin x+\sin 3x\geq\cos x+\cos 3x.$

5. Znalez$\acute{}$č równania okręgów, które są styczne do obu osi układu współrzędnych oraz do

prostej $0$ równaniu $x+y=4$. Wykonač rysunek.

6. Pokazač, $\dot{\mathrm{z}}\mathrm{e}$ stosunek objetości stozka do objętości wpisanej $\mathrm{w}$ ten stozek kuli jest równy

stosunkowi pola powierzchni cafkowitej stozka do pola powierzchni kuli.




PRACA KONTROLNA nr $5$ - PozioM ROZSZERZONY

l. Na ile sposobów $\mathrm{m}\mathrm{o}\dot{\mathrm{z}}$ na wybrač 5 kart $\mathrm{z}$ talii 52 kart $\mathrm{t}\mathrm{a}\mathrm{k}$, aby mieč przynajmniej po

jednej karcie $\mathrm{w}\mathrm{k}\mathrm{a}\dot{\mathrm{z}}$ dym $\mathrm{z}$ czterech kolorów? A jaka jest odpowied $\acute{\mathrm{z}}$, gdy wybieramy 6

kart $\mathrm{z}$ talii?

2. Rozpatrujemy zbiór ciągów $n$-elementowych $0$ wyrazach -$1, 0$ lub l. Obliczyč prawdo-

podobieństwo tego, $\dot{\mathrm{z}}\mathrm{e}$ losowo wybrany ciąg ma co najwyzej jeden wyraz równy 0 $\mathrm{i}$ suma

jego wyrazów równa jest 0.

3. Suma wszystkich wspófczynników wielomianu $W_{n}(x)$ jest równa

$\displaystyle \lim_{n\rightarrow\infty}(1+\frac{1}{2}+\frac{1}{4}+\ldots+\frac{1}{2^{n}}),$

a suma współczynników przy nieparzystych potęgach zmiennej równa jest sumie współ-

czynników przy jej parzystych potęgach. Wyznaczyč resztę $R(x)\mathrm{z}$ dzielenia wielomianu

$W_{n}(x)$ przez dwumian $x^{2}-1.$

4. Rozwiązač nierównośč

$\sin x+\sin 2x+\sin 3x\geq\cos x+\cos 2x+\cos 3x.$

5. Zbadač przebieg zmienności funkcji $f(x) = \displaystyle \frac{4x^{2}-3x-1}{4x^{2}+1} \mathrm{i}$ naszkicowač jej wykres. Na

podstawie sporządzonego wykresu określič liczbę rozwi$\Phi$zań równania $f(x) =m\mathrm{w}$ za-

$\mathrm{l}\mathrm{e}\dot{\mathrm{z}}$ ności od parametru $m.$

6. $\mathrm{W}$ stozku pole podstawy, pole powierzchni kuli wpisanej $\mathrm{w}$ ten stozek $\mathrm{i}$ pole powierzchni

bocznej stozka tworzą ciąg arytmetyczny. Wyznaczyč kąt nachylenia tworzącej stozka

do plaszczyzny jego podstawy. Wykonač rysunek.

Rozwiązania (rękopis) zadań z wybranego poziomu prosimy nadsyfač do

na adres:

18 stycznia 2020r.

Wydziaf Matematyki

Politechnika Wrocfawska

Wybrzez $\mathrm{e}$ Wyspiańskiego 27

$50-370$ WROCLAW.

Na kopercie prosimy $\underline{\mathrm{k}\mathrm{o}\mathrm{n}\mathrm{i}\mathrm{e}\mathrm{c}\mathrm{z}\mathrm{n}\mathrm{i}\mathrm{e}}$ zaznaczyč wybrany poziom! (np. poziom podsta-

wowy lub rozszerzony). Do rozwiązań nalez $\mathrm{y}$ dołączyč zaadresowana do siebie koperte

zwrotną $\mathrm{z}$ naklejonym znaczkiem, odpowiednim do formatu listu. Polecamy stosowanie

kopert formatu C5 $(160\mathrm{x}230\mathrm{m}\mathrm{m})$ ze znaczkiem $0$ wartości 3,30 zł. Na $\mathrm{k}\mathrm{a}\dot{\mathrm{z}}$ dą większą

kopertę nalez $\mathrm{y}$ nakleič $\mathrm{d}\mathrm{r}\mathrm{o}\dot{\mathrm{z}}$ szy znaczek. Prace niespełniające podanych warunków nie

będą poprawiane ani odsyłane.

Uwaga. Wysylaj\S c nam rozwi\S zania zadań uczestnik Kursu udostępnia Politechnice Wrocfawskiej

swoje dane osobowe, które przetwarzamy wyłącznie $\mathrm{w}$ zakresie niezbędnym do jego prowadzenia

(odeslanie zadań, prowadzenie statystyki). Szczególowe informacje $0$ przetwarzaniu przez nas danych

osobowych są dostępne na stronie internetowej Kursu.

Adres internetowy Kursu: http: //www. im. pwr. edu. pl/kurs



\end{document}