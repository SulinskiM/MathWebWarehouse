\documentclass[a4paper,12pt]{article}
\usepackage{latexsym}
\usepackage{amsmath}
\usepackage{amssymb}
\usepackage{graphicx}
\usepackage{wrapfig}
\pagestyle{plain}
\usepackage{fancybox}
\usepackage{bm}

\begin{document}

XLIX

KORESPONDENCYJNY KURS

Z MATEMATYKI

styczeń 2020 r.

PRACA KONTROLNA $\mathrm{n}\mathrm{r} 5-$ POZIOM PODSTAWOWY

l. Załózmy, $\dot{\mathrm{z}}\mathrm{e}$ mamy 12 ku1 białych $\mathrm{i}9$ kul czarnych. Na ile sposobów $\mathrm{m}\mathrm{o}\dot{\mathrm{z}}$ na ustawič te

kule $\mathrm{w}$ rzędzie $\mathrm{w}$ taki sposób, aby $\dot{\mathrm{z}}$ adna czarna kula nie sąsiadowala $\mathrm{z}$ czarną? Na ile

róznych sposobów $\mathrm{m}\mathrm{o}\dot{\mathrm{z}}$ na ustawič te kule $\mathrm{w}$ rzędzie $\mathrm{w}$ taki sposób, aby $\dot{\mathrm{z}}$ adna czarna kula

nie sąsiadowafa $\mathrm{z}$ czarną, jeśli kule białe ponumerujemy kolejnymi liczbami parzystymi,

a kule czarne- kolejnymi liczbami nieparzystymi?

2. Ścianki kostki do gry oznaczono liczbami: -$3,$ -$2,$ -$1$, 1, 2, 3. Jakie jest prawdopodobień-

stwo zdarzenia, $\dot{\mathrm{z}}\mathrm{e}$ przy dwóch rzutach tą kostką: a) otrzymana suma liczb wynosi 2; b)

wartośč bezwzględna sumy liczb jest równa co najwyzej 3?

3. Wyznaczyč ciag arytmetyczny $0$ pierwszym wyrazie równym 2, wiedząc, $\dot{\mathrm{z}}\mathrm{e}$ wyrazy:

pierwszy, trzeci $\mathrm{i}$ jedenasty $\mathrm{w}$ podanej kolejności tworzą ciąg geometryczny. Ile pierw-

szych kolejnych wyrazów tego ciqgu nalezy dodač, aby otrzymana suma była większa

$\mathrm{n}\mathrm{i}\dot{\mathrm{z}}$ 1000?

4. $\mathrm{W}$ zbiorze $[0,2\pi]$ rozwiązač nierównośč

$\sin x+\sin 3x\geq\cos x+\cos 3x.$

5. Znalez$\acute{}$č równania okręgów, które są styczne do obu osi układu współrzędnych oraz do

prostej $0$ równaniu $x+y=4$. Wykonač rysunek.

6. Pokazač, $\dot{\mathrm{z}}\mathrm{e}$ stosunek objetości stozka do objętości wpisanej $\mathrm{w}$ ten stozek kuli jest równy

stosunkowi pola powierzchni cafkowitej stozka do pola powierzchni kuli.
\end{document}
