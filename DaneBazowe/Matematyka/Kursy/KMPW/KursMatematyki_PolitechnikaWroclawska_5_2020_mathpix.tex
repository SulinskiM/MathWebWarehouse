\documentclass[10pt]{article}
\usepackage[polish]{babel}
\usepackage[utf8]{inputenc}
\usepackage[T1]{fontenc}
\usepackage{amsmath}
\usepackage{amsfonts}
\usepackage{amssymb}
\usepackage[version=4]{mhchem}
\usepackage{stmaryrd}
\usepackage{hyperref}
\hypersetup{colorlinks=true, linkcolor=blue, filecolor=magenta, urlcolor=cyan,}
\urlstyle{same}

\title{PRACA KONTROLNA nr 5 - POZIOM PODSTAWOWY }

\author{}
\date{}


\begin{document}
\maketitle
\begin{enumerate}
  \item Załóżmy, że mamy 12 kul białych i 9 kul czarnych. Na ile sposobów można ustawić te kule w rzędzie w taki sposób, aby żadna czarna kula nie sąsiadowała z czarną? Na ile różnych sposobów można ustawić te kule w rzędzie w taki sposób, aby żadna czarna kula nie sąsiadowała z czarną, jeśli kule białe ponumerujemy kolejnymi liczbami parzystymi, a kule czarne - kolejnymi liczbami nieparzystymi?
  \item Ścianki kostki do gry oznaczono liczbami: $-3,-2,-1,1,2,3$. Jakie jest prawdopodobieństwo zdarzenia, że przy dwóch rzutach tą kostką: a) otrzymana suma liczb wynosi 2; b) wartość bezwzględna sumy liczb jest równa co najwyżej 3?
  \item Wyznaczyć ciąg arytmetyczny o pierwszym wyrazie równym 2, wiedząc, że wyrazy: pierwszy, trzeci i jedenasty w podanej kolejności tworzą ciąg geometryczny. Ile pierwszych kolejnych wyrazów tego ciągu należy dodać, aby otrzymana suma była większa niż 1000?
  \item W zbiorze $[0,2 \pi]$ rozwiązać nierówność
\end{enumerate}

$$
\sin x+\sin 3 x \geqslant \cos x+\cos 3 x
$$

\begin{enumerate}
  \setcounter{enumi}{4}
  \item Znaleźć równania okręgów, które są styczne do obu osi układu współrzędnych oraz do prostej o równaniu $x+y=4$. Wykonać rysunek.
  \item Pokazać, że stosunek objętości stożka do objętości wpisanej w ten stożek kuli jest równy stosunkowi pola powierzchni całkowitej stożka do pola powierzchni kuli.
\end{enumerate}

\section*{PRACA KONTROLNA nr 5 - POZIOM RoZsZERZoNY}
\begin{enumerate}
  \item Na ile sposobów można wybrać 5 kart z talii 52 kart tak, aby mieć przynajmniej po jednej karcie w każdym z czterech kolorów? A jaka jest odpowiedź, gdy wybieramy 6 kart z talii?
  \item Rozpatrujemy zbiór ciągów $n$-elementowych o wyrazach $-1,0$ lub 1 . Obliczyć prawdopodobieństwo tego, że losowo wybrany ciąg ma co najwyżej jeden wyraz równy 0 i suma jego wyrazów równa jest 0 .
  \item Suma wszystkich współczynników wielomianu $W_{n}(x)$ jest równa
\end{enumerate}

$$
\lim _{n \rightarrow \infty}\left(1+\frac{1}{2}+\frac{1}{4}+\ldots+\frac{1}{2^{n}}\right)
$$

a suma współczynników przy nieparzystych potęgach zmiennej równa jest sumie współczynników przy jej parzystych potęgach. Wyznaczyć resztę $R(x)$ z dzielenia wielomianu $W_{n}(x)$ przez dwumian $x^{2}-1$.\\
4. Rozwiązać nierówność

$$
\sin x+\sin 2 x+\sin 3 x \geqslant \cos x+\cos 2 x+\cos 3 x
$$

\begin{enumerate}
  \setcounter{enumi}{4}
  \item Zbadać przebieg zmienności funkcji $f(x)=\frac{4 x^{2}-3 x-1}{4 x^{2}+1}$ i naszkicować jej wykres. Na podstawie sporządzonego wykresu określić liczbę rozwiązań równania $f(x)=m \mathrm{w}$ zależności od parametru $m$.
  \item W stożku pole podstawy, pole powierzchni kuli wpisanej w ten stożek i pole powierzchni bocznej stożka tworzą ciąg arytmetyczny. Wyznaczyć kąt nachylenia tworzącej stożka do płaszczyzny jego podstawy. Wykonać rysunek.
\end{enumerate}

Rozwiązania (rękopis) zadań z wybranego poziomu prosimy nadsyłać do 18 stycznia 2020r. na adres:

\begin{verbatim}
Wydział Matematyki
Politechnika Wrocławska
Wybrzeże Wyspiańskiego 27
50-370 WROCEAW.
\end{verbatim}

Na kopercie prosimy koniecznie zaznaczyć wybrany poziom! (np. poziom podstawowy lub rozszerzony). Do rozwiązań należy dołączyć zaadresowaną do siebie kopertę zwrotną z naklejonym znaczkiem, odpowiednim do formatu listu. Polecamy stosowanie kopert formatu C5 ( $160 \times 230 \mathrm{~mm}$ ) ze znaczkiem o wartości $3,30 \mathrm{zl}$. Na każdą większą kopertę należy nakleić droższy znaczek. Prace niespełniające podanych warunków nie będą poprawiane ani odsyłane.

Uwaga. Wysyłając nam rozwiązania zadań uczestnik Kursu udostępnia Politechnice Wrocławskiej swoje dane osobowe, które przetwarzamy wyłącznie w zakresie niezbędnym do jego prowadzenia (odesłanie zadań, prowadzenie statystyki). Szczegółowe informacje o przetwarzaniu przez nas danych osobowych są dostępne na stronie internetowej Kursu.\\
Adres internetowy Kursu: \href{http://www.im.pwr.edu.pl/kurs}{http://www.im.pwr.edu.pl/kurs}


\end{document}