\documentclass[a4paper,12pt]{article}
\usepackage{latexsym}
\usepackage{amsmath}
\usepackage{amssymb}
\usepackage{graphicx}
\usepackage{wrapfig}
\pagestyle{plain}
\usepackage{fancybox}
\usepackage{bm}

\begin{document}

L KORESPONDENCYJNY KURS

Z MATEMATYKI

styczeń 2021 r.

PRACA KONTROLNA nr 5- POZIOM PODSTAWOWY

l. Jeden $\mathrm{z}$ wierzchołków trójkąta równobocznego wpisanego $\mathrm{w}$ okrąg $x^{2}+y^{2}=2$ znajduje się

$\mathrm{w}$ punkcie $P(1,1)$. Wyznacz polozenie pozostałych wierzchołków $\mathrm{i}$ sporząd $\acute{\mathrm{z}}$ odpowiedni

rysunek.

2. Zbadaj, dla jakiej wartości parametru $\alpha \in [0,2\pi]$ liczba 0 jest najwiekszą wartościa

funkcji

$f(x)=x^{2}\cos\alpha+x(1+\cos 2\alpha)-1$

$\mathrm{w}$ calej jej dziedzinie.

3. Wyznacz te argumenty funkcji

$g(x)=16\cdot 2^{x^{4}}\cdot 243^{x^{2}}-81\cdot 3^{x^{4}}\cdot 32^{x^{2}}$

dla których funkcja ta przyjmuje wartości nieujemne.

4. Zakładając, $\dot{\mathrm{z}}\mathrm{e}x\in[0,2\pi]$, rozwiąz nierównośč trygonometryczną

16 $\displaystyle \sin^{4}\frac{x}{2}-16\sin^{2}\frac{x}{2}+3\geq 0.$

5. Wyznacz wszystkie punkty wspólne krzywych

$y=\displaystyle \log_{\sqrt{2}}\sqrt{2x-1}+\log_{\frac{1}{2}}\frac{1}{3x+1}$

oraz

$y=1+2\log_{4}(x+1).$

6. Narysuj wykres funkcji

$f(x)=|2-|2-2^{|x|}||$

i precyzyjnie opisz zastosowaną metode jego konstrukcji. Na podstawie rysunku wskaz

lokalne ekstrema funkcji oraz określ jej najmniejszą i największą wartośč w cafej dzie-

dzinie, 0 i1e one istnieją.
\end{document}
