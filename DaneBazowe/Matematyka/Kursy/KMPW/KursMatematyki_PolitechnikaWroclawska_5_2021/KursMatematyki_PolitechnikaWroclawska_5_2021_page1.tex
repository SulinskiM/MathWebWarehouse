\documentclass[a4paper,12pt]{article}
\usepackage{latexsym}
\usepackage{amsmath}
\usepackage{amssymb}
\usepackage{graphicx}
\usepackage{wrapfig}
\pagestyle{plain}
\usepackage{fancybox}
\usepackage{bm}

\begin{document}

PRACA KONTROLNA $\mathrm{n}\mathrm{r} 5-$ POZIOM ROZSZERZONY

l. Jeden $\mathrm{z}$ wierzchołków sześciokąta foremnego wpisanego $\mathrm{w}$ okrąg $x^{2}+y^{2}=2$ znajduje się $\mathrm{w}$

punkcie $P(-1,-1)$. Wyznacz pofozenie pozostalych wierzchołków $\mathrm{i}\mathrm{s}$porząd $\acute{\mathrm{z}}$ odpowiedni

rysunek.

2. Rozwiąz nierównośč

$2^{3x^{3}+x^{2}-3x+1}\cdot 3^{6x^{4}-x^{2}}\geq 3^{x^{3}+6x^{2}-x-1}\cdot 4^{3x^{4}+x^{3}-3x^{2}-x+1}$

3. Określ dziedzinę $\mathrm{i}$ zbadaj, dla jakich argumentów funkcja

$f(x)=\displaystyle \log_{x-1}(x+2)+\log_{x+2}\frac{1}{x-1}$

przyjmuje wartości dodatnie.

4. Rozwiąz nierównośč

$3-3\displaystyle \sin^{2}x+3\sin^{4}x-3\sin^{6}x+\ldots\leq\frac{16\cos^{2}x-16\cos^{4}x}{2-\cos^{2}x},$

której lewa strona jest suma wszystkich wyrazów nieskończonego ciągu geometrycznego.

5. Na stozku $0$ promieniu podstawy $R$ opisano ostrosłup prawidłowy czworokątny, a $\mathrm{w}$ sto-

$\dot{\mathrm{z}}$ ek ten wpisano ostrosłup prawidlowy sześciok$\Phi$tny. Stosunek pól powierzchni bocznych

obu ostrosfupów wynosi $k$. Wyznacz zakres zmienności parametru $k$, a dla $k=\displaystyle \frac{11}{8}$ oblicz

wysokośč stozka $\mathrm{i}$ wykonač staranne rysunki rozwazanych brył.

6. Określ dziedzinę, wyznacz wszystkie asymptoty, przedziały monotoniczności oraz wszyst-

kie lokalne ekstrema funkcji

$f(x)=\displaystyle \frac{x^{3}+x^{2}-x+2}{x^{2}+x-2}.$

$\mathrm{s}_{\mathrm{P}^{\mathrm{o}\mathrm{r}\mathrm{z}}\Phi^{\mathrm{d}\acute{\mathrm{z}}}}$ staranny wykres.

Rozwiązania (rękopis) zadań z wybranego poziomu prosimy nadsyfač do

2021r. na adres:

20 stycznia

Wydziaf Matematyki

Politechnika Wrocfawska

Wybrzez $\mathrm{e}$ Wyspiańskiego 27

$50-370$ WROCLAW.

Na kopercie prosimy $\underline{\mathrm{k}\mathrm{o}\mathrm{n}\mathrm{i}\mathrm{e}\mathrm{c}\mathrm{z}\mathrm{n}\mathrm{i}\mathrm{e}}$ zaznaczyč wybrany poziom! (np. poziom podsta-

wowy lub rozszerzony). Do rozwiązań nalez $\mathrm{y}$ dołączyč zaadresowaną do siebie kopertę

zwrotną $\mathrm{z}$ naklejonym znaczkiem, odpowiednim do formatu listu. Polecamy stosowanie

kopert formatu C5 $(160\mathrm{x}230\mathrm{m}\mathrm{m})$ ze znaczkiem $0$ wartości 3,30 zł. Na $\mathrm{k}\mathrm{a}\dot{\mathrm{z}}$ dą większą

kopertę nalez $\mathrm{y}$ nakleič $\mathrm{d}\mathrm{r}\mathrm{o}\dot{\mathrm{z}}$ szy znaczek. Prace niespełniające podanych warunków nie

bedą poprawiane ani odsyłane.

Uwaga. Wysyfajac nam rozwiazania zadań uczestnik Kursu udostępnia Politechnice Wroclawskiej

swoje dane osobowe, które przetwarzamy wyłącznie $\mathrm{w}$ zakresie niezbednym do jego prowadzenia

(odesfanie zadań, prowadzenie statystyki). Szczegófowe informacje $0$ przetwarzaniu przez nas danych

osobowych s\S dostępne na stronie internetowej Kursu.

Adres internetowy Kursu: http: //www. im. pwr. edu. pl/kurs
\end{document}
