\documentclass[a4paper,12pt]{article}
\usepackage{latexsym}
\usepackage{amsmath}
\usepackage{amssymb}
\usepackage{graphicx}
\usepackage{wrapfig}
\pagestyle{plain}
\usepackage{fancybox}
\usepackage{bm}

\begin{document}

LI KORESPONDENCYJNY KURS

Z MATEMATYKI

styczeń 2022 r.

PRACA KONTROLNA nr 5- POZIOM PODSTAWOWY

l. Do sklepu dostarczono ziemniaki $\mathrm{w}$ dwóch gatunkach. II gatunekjest po $a\mathrm{z}1$ za kilogram,

a I gatunek jest $020$ \% drozszy. Lączna wartośč dostarczonych ziemniaków wyniosla $56a$

$\mathrm{z}l. \mathrm{W}$ ciągu dnia sprzedano 1/5 ziemniaków I gatunku $\mathrm{i} 1/4$ ziemniaków II gatunku,

$\mathrm{w}$ sumie za kwotę $12,2\alpha \mathrm{z}l$. Ile kilogramów ziemniaków $\mathrm{k}\mathrm{a}\dot{\mathrm{z}}$ dego gatunku dostarczono do

sklepu?

2. Na loteriijest l00 losów, $\mathrm{z}$ których $5$jest wygrywających. Jakiejest prawdopodobieństwo,

$\dot{\mathrm{z}}\mathrm{e}$ wśród trzech kupionych losów a) dokfadnie jeden wygrywa; b) przynajmniej jeden

wygrywa?

3. Dany jest kwadrat $0$ boku $a$. Do boków tego kwadratu dołączono jednakowe trójkqty

równoramienne $0$ podstawie boku kwadratu. Następnie zfączono wierzchofki trójkątów

$\mathrm{w}$ jeden wierzchołek tworząc ostrosłup $0$ objętości $V$. Wyznacz długośč ramienia dolą-

czonych trójkątów, a następnie wykonaj rachunki, przyjmując $a=3$ cm oraz $V= 18$

$\mathrm{c}\mathrm{m}^{3}$

4. Wysokośč rombu $0$ boku $\alpha$ dzieli jeden $\mathrm{z}$ jego boków na dwie części $\mathrm{w}$ stosunku 1 : 2.

Wyznacz dlugości przekątnych rombu oraz promień okręgu wpisanego $\mathrm{w}$ ten romb.

5. Znajd $\acute{\mathrm{z}}$ współrzędne wierzcholka $C$ trójkąta równoramiennego $ABC0$ podstawie $AB,$

gdzie $A(0,0) \mathrm{i} B(2,0)$, wiedząc, $\dot{\mathrm{z}}\mathrm{e}$ środkowe tego trójkąta $AD \mathrm{i}$ BE są prostopadłe

względem siebie.

6. Prosta $0$ równaniu $x-2y+10 = 0$ przecina parabolę $y = x^{2}-4x+5\mathrm{w}$ punktach

{\it A} $\mathrm{i}B$. Wykaz, $\dot{\mathrm{z}}\mathrm{e}$ trójkąt $ABC$, gdzie $C$ jest wierzchołkiem paraboli, jest prostokątny,

a następnie oblicz pole tego trójkata. Wykonaj staranny rysunek.
\end{document}
