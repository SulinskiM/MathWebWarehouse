\documentclass[a4paper,12pt]{article}
\usepackage{latexsym}
\usepackage{amsmath}
\usepackage{amssymb}
\usepackage{graphicx}
\usepackage{wrapfig}
\pagestyle{plain}
\usepackage{fancybox}
\usepackage{bm}

\begin{document}

PRACA KONTROLNA $\mathrm{n}\mathrm{r} 5-$ POZIOM ROZSZERZONY

1. $K\mathrm{a}\mathrm{t}$ ostry równolegloboku ma miarę $45^{\mathrm{o}}$ Punkt przeciecia przekątnych równoległoboku

jest oddalony od boków $0 1\mathrm{i}\sqrt{2}$. Oblicz pole tego równolegfoboku oraz dlugości jego

przekątnych.

2. Spośród 20 pytań egzaminacyjnych uczeń zna odpowied $\acute{\mathrm{z}}\mathrm{n}\mathrm{a}12$ pytań. Jakie jest prawdo-

podobieństwo, $\dot{\mathrm{z}}\mathrm{e}$ uczeń zda egzamin, jeśli przyjętajest następująca zasada: uczeń losuje

dwa pytania $\mathrm{i}$ jeśli na oba odpowie dobrze, to egzamin jest zdany, a jeśli tylko na jed-

no pytanie odpowie dobrze, to losuje jeszcze jedno pytanie $\mathrm{i}$ musi na nie odpowiedzieč

poprawnie, $\dot{\mathrm{z}}$ eby zdač egzamin?

3. Czworościan rozcięto wzdfuz trzech krawędzi wychodzących $\mathrm{z}$ tego samego wierzchofka

$\mathrm{i}$ po rozprostowaniu otrzymano kwadrat $0$ boku $a$. Oblicz objętośč czworościaniu oraz

wykonaj odpowiedni rysunek.

4. Przez punkt $(-1,2)$ przeprowad $\acute{\mathrm{z}} \mathrm{p}\mathrm{r}\mathrm{o}\mathrm{s}\mathrm{t}_{\Phi}\mathrm{t}\mathrm{a}\mathrm{k}$, aby środek jej odcinka zawartego między

prostymi $x+2y = 3\mathrm{i}x+2y = 5$ nalezał do prostej $x+y = 2$. Wyznacz równanie

symetralnej tego odcinka. Wykonaj staranny rysnuek.

5. Rozwiąz algebraicznie następujący ukfad równań

$\left\{\begin{array}{l}
y=|x^{2}-2x|+1\\
x^{2}+y^{2}+1=2x+2y
\end{array}\right.$

$\mathrm{i}$ podaj jego interpretację graficznq (wykonaj staranny rysunek).

6. Funkcja $f(x) = \displaystyle \frac{x^{2}-4x+4}{2x}$ ma $\mathrm{w}$ punktach $A\mathrm{i}B$ wartości ekstremalne. Znajd $\acute{\mathrm{z}}$ taki

punkt $C$ nalezący do osi odciętych, aby pole trójkąta $ABC$ było równe pierwiastkowi

równania $x^{1-\frac{1}{2}+\frac{1}{4}-\frac{1}{8}} =4$, gdzie $x>0$. Naszkicuj wykres funkcji $f(x)$ wraz $\mathrm{z}$ trójkątem

$ABC.$

Rozwiązania (rękopis) zadań z wybranego poziomu prosimy nadsyłač do

2022r. na adres:

20 stycznia

Wydziaf Matematyki

Politechnika Wrocfawska

Wybrzez $\mathrm{e}$ Wyspiańskiego 27

$50-370$ WROCLAW,

lub elektronicznie, za pośrednictwem portalu talent. $\mathrm{p}\mathrm{w}\mathrm{r}$. edu. pl

Na kopercie prosimy $\underline{\mathrm{k}\mathrm{o}\mathrm{n}\mathrm{i}\mathrm{e}\mathrm{c}\mathrm{z}\mathrm{n}\mathrm{i}\mathrm{e}}$ zaznaczyč wybrany poziom! (np. poziom podsta-

wowy lub rozszerzony). Do rozwiązań nalez $\mathrm{y}$ dołączyč zaadresowaną do siebie kopertę

zwrotną $\mathrm{z}$ naklejonym znaczkiem, odpowiednim do formatu listu. Polecamy stosowanie

kopert formatu C5 $(160\mathrm{x}230\mathrm{m}\mathrm{m})$ ze znaczkiem $0$ wartości 3,30 zł. Na $\mathrm{k}\mathrm{a}\dot{\mathrm{z}}$ dą większą

kopertę nalez $\mathrm{y}$ nakleič $\mathrm{d}\mathrm{r}\mathrm{o}\dot{\mathrm{z}}$ szy znaczek. Prace niespełniające podanych warunków nie

bedą poprawiane ani odsyłane.

Uwaga. Wysyfajac nam rozwiazania zadań uczestnik Kursu udostępnia Politechnice Wroclawskiej

swoje dane osobowe, które przetwarzamy wyłącznie $\mathrm{w}$ zakresie niezbednym do jego prowadzenia

(odesfanie zadań, prowadzenie statystyki). Szczegófowe informacje $0$ przetwarzaniu przez nas danych

osobowych s\S dostępne na stronie internetowej Kursu.

Adres internetowy Kursu: http: //www. im. pwr. edu. pl/kurs
\end{document}
