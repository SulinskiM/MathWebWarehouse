\documentclass[a4paper,12pt]{article}
\usepackage{latexsym}
\usepackage{amsmath}
\usepackage{amssymb}
\usepackage{graphicx}
\usepackage{wrapfig}
\pagestyle{plain}
\usepackage{fancybox}
\usepackage{bm}

\begin{document}

LII

KORESPONDENCYJNY KURS

Z MATEMATYKI

styczeń 2023 r.

PRACA KONTROLNA $\mathrm{n}\mathrm{r} 5-$ POZIOM PODSTAWOWY

l. Rozwiqz nierówność

$\displaystyle \frac{\sqrt{30+x-x^{2}}}{x}<\frac{\sqrt{10}}{5}.$

2. $\mathrm{Z}$ ilu domin składa się komplet klocków do gry $\mathrm{w}$ domino, zawierajqcy pojednym dominie

dla $\mathrm{k}\mathrm{a}\dot{\mathrm{z}}$ dej kombinacji oczek od 0 do 6? A jaka jest odpowiedz' d1a kombinacji oczek od

0 do $n$?

3. $\mathrm{W}$ prostokątnym ukladzie współrzędnych narysuj zbiór $A\cap B, \mathrm{j}\mathrm{e}\dot{\mathrm{z}}$ eli:

$A=\{(x,y):x\in \mathbb{R},y\in \mathbb{R},y=x+b,b\in[-2,2]\},$

$B=\displaystyle \{(x,y):x\in \mathbb{R},y\in \mathbb{R},y=ax,a\in[-3,-\frac{1}{3}]\}.$

Zbadaj, czy punkt $(1,-\displaystyle \frac{1}{2})$ nalezy do zbioru $A\cap B.$

4. Spośród trapezów równoramiennych $0$ danym obwodzie $p\mathrm{i}$ danym kącie $\alpha$ przy podstawie

wyznacz trapez $0$ największym polu.

5. Dane są trzy kolejne wierzcholki prostokąta ABCD: $A(-5,-3), B(-2,0), C(-7,5)$. Na-

pisz równanie okręgu opisanego na tym prostokącie oraz równanie prostej stycznej do

tego okręgu $\mathrm{w}$ punkcie $D.$

6. Kwadrat ABCD jest podstawą prostopadłościanu ABCDEFGH. $\acute{\mathrm{S}}$ rodek $M$ krawędzi

AB łączymy $\mathrm{z}$ wierzchołkiem $G$ otrzymując odcinek dlugości $d$ nachylony do ściany

DCGH pod kątem $\alpha$. Oblicz pole powierzchni bocznej tego prostopadłościanu.




PRACA KONTROLNA $\mathrm{n}\mathrm{r} 5-$ POZIOM ROZSZERZONY

l. Para $(x,y)$ jest rozwiązaniem układu:

$\left\{\begin{array}{l}
x-y=-1-m\\
2x-y=2m.
\end{array}\right.$

Dlajakich wartości $m$ punkt $P(x,y)$ nalezy do wnętrza koła $0$ promieniu długości $r=\sqrt{5}$

$\mathrm{i}$ środku $\mathrm{w}$ początku układu współrzędnych?

2. Na ile sposobów $\mathrm{m}\mathrm{o}\dot{\mathrm{z}}$ na ustawić 5 ksiqzek na trzech półkach, jeś1i $\mathrm{w}\mathrm{a}\dot{\mathrm{z}}$ na jest kolejność

ustawienia ksiązek oraz to, na której pólce stoją?

3. Wyznacz zbiór środków wszystkich cięciw okręgu $0$ równaniu $x^{2}+y^{2}=16$, które prze-

chodzą przez punkt $(0,4)$. Wykonaj staranny rysunek.

4. Wykres funkcji $f(x)=x^{3}-3x^{2}+bx+c$ przechodzi przez punkt $A(2,5)$. Wspólczynnik

kierunkowy stycznej do wykresu funkcju $\mathrm{w}$ punkcie $A$ jest rozwiązaniem równania

$(\displaystyle \frac{4}{9})^{x+1}=(\frac{81}{16})^{x+13}$

Wyznacz najmniejszą $\mathrm{i}$ największą wartość funkcji $\mathrm{w}$ przedziale [-2, 2].

5. Obwód trójkąta równoramiennego jest równy $a$. Przy jakich dlugošciach boków pole

trójkąta jest największe? Podaj największą wartość pola trójkąta dla $a=3+2\sqrt{3}.$

6. $\mathrm{W}$ kole $0$ šrodku $O$ poprowadzono dwie prostopadłe średnice $\overline{AB}\mathrm{i}\overline{CD}$ oraz cięciwę $\overline{AM}$

przecinającq średnicę $\overline{CD}\mathrm{w}$ punkcie $K$. Dlajakiego kąta między šrednicą $\overline{AB}$ a cięciwą

$\overline{AM}\mathrm{w}$ czworokąt OBMK $\mathrm{m}\mathrm{o}\dot{\mathrm{z}}$ na wpisać okrąg?

Rozwiązania (rękopis) zadań z wybranego poziomu prosimy nadsylać do 20.01.2023r.

adres:

na

Wydzial Matematyki

Politechnika $\mathrm{W}\mathrm{r}\mathrm{o}\mathrm{c}\not\subset$awska

Wybrzez $\mathrm{e}$ Wyspiańskiego 27

$50-370$ WROCLAW,

lub elektronicznie, za pośrednictwem portalu talent. $\mathrm{p}\mathrm{w}\mathrm{r}$. edu. pl

Na kopercie prosimy $\underline{\mathrm{k}\mathrm{o}\mathrm{n}\mathrm{i}\mathrm{e}\mathrm{c}\mathrm{z}\mathrm{n}\mathrm{i}\mathrm{e}}$ zaznaczyć wybrany poziom! (np. poziom podsta-

wowy lub rozszerzony). Do rozwiązań nalez $\mathrm{y}$ dołączyć zaadresowaną do siebie kopertę

zwrotną $\mathrm{z}$ naklejonym znaczkiem, odpowiednim do formatu listu. Prace niespełniające

podanych warunków nie będą poprawiane ani odsyłane.

Uwaga. Wysylając nam rozwiązania zadań uczestnik Kursu udostępnia Politechnice Wrocławskiej

swoje dane osobowe, które przetwarzamy wyłącznie $\mathrm{w}$ zakresie niezbędnym do jego prowadzenia

(odesłanie zadań, prowadzenie statystyki). Szczegófowe informacje $0$ przetwarzaniu przez nas danych

osobowych są dostępne na stronie internetowej Kursu.

Adres internetowy Kursu: http: //www. im. pwr. edu. pl/kurs



\end{document}