\documentclass[a4paper,12pt]{article}
\usepackage{latexsym}
\usepackage{amsmath}
\usepackage{amssymb}
\usepackage{graphicx}
\usepackage{wrapfig}
\pagestyle{plain}
\usepackage{fancybox}
\usepackage{bm}

\begin{document}

LII

KORESPONDENCYJNY KURS

Z MATEMATYKI

styczeń 2023 r.

PRACA KONTROLNA $\mathrm{n}\mathrm{r} 5-$ POZIOM PODSTAWOWY

l. Rozwiqz nierówność

$\displaystyle \frac{\sqrt{30+x-x^{2}}}{x}<\frac{\sqrt{10}}{5}.$

2. $\mathrm{Z}$ ilu domin składa się komplet klocków do gry $\mathrm{w}$ domino, zawierajqcy pojednym dominie

dla $\mathrm{k}\mathrm{a}\dot{\mathrm{z}}$ dej kombinacji oczek od 0 do 6? A jaka jest odpowiedz' d1a kombinacji oczek od

0 do $n$?

3. $\mathrm{W}$ prostokątnym ukladzie współrzędnych narysuj zbiór $A\cap B, \mathrm{j}\mathrm{e}\dot{\mathrm{z}}$ eli:

$A=\{(x,y):x\in \mathbb{R},y\in \mathbb{R},y=x+b,b\in[-2,2]\},$

$B=\displaystyle \{(x,y):x\in \mathbb{R},y\in \mathbb{R},y=ax,a\in[-3,-\frac{1}{3}]\}.$

Zbadaj, czy punkt $(1,-\displaystyle \frac{1}{2})$ nalezy do zbioru $A\cap B.$

4. Spośród trapezów równoramiennych $0$ danym obwodzie $p\mathrm{i}$ danym kącie $\alpha$ przy podstawie

wyznacz trapez $0$ największym polu.

5. Dane są trzy kolejne wierzcholki prostokąta ABCD: $A(-5,-3), B(-2,0), C(-7,5)$. Na-

pisz równanie okręgu opisanego na tym prostokącie oraz równanie prostej stycznej do

tego okręgu $\mathrm{w}$ punkcie $D.$

6. Kwadrat ABCD jest podstawą prostopadłościanu ABCDEFGH. $\acute{\mathrm{S}}$ rodek $M$ krawędzi

AB łączymy $\mathrm{z}$ wierzchołkiem $G$ otrzymując odcinek dlugości $d$ nachylony do ściany

DCGH pod kątem $\alpha$. Oblicz pole powierzchni bocznej tego prostopadłościanu.
\end{document}
