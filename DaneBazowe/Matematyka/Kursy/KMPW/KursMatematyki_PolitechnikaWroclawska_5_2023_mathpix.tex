\documentclass[10pt]{article}
\usepackage[polish]{babel}
\usepackage[utf8]{inputenc}
\usepackage[T1]{fontenc}
\usepackage{amsmath}
\usepackage{amsfonts}
\usepackage{amssymb}
\usepackage[version=4]{mhchem}
\usepackage{stmaryrd}
\usepackage{bbold}
\usepackage{hyperref}
\hypersetup{colorlinks=true, linkcolor=blue, filecolor=magenta, urlcolor=cyan,}
\urlstyle{same}

\title{PRACA KONTROLNA nr 5 - POZIOM PODSTAWOWY }

\author{}
\date{}


\begin{document}
\maketitle
\begin{enumerate}
  \item Rozwiąż nierówność
\end{enumerate}

$$
\frac{\sqrt{30+x-x^{2}}}{x}<\frac{\sqrt{10}}{5}
$$

\begin{enumerate}
  \setcounter{enumi}{1}
  \item Z ilu domin składa się komplet klocków do gry w domino, zawierający po jednym dominie dla każdej kombinacji oczek od 0 do 6? A jaka jest odpowiedź dla kombinacji oczek od 0 do $n$ ?
  \item W prostokątnym układzie współrzędnych narysuj zbiór $A \cap B$, jeżeli:
\end{enumerate}

$$
\begin{aligned}
A & =\{(x, y): x \in \mathbb{R}, y \in \mathbb{R}, y=x+b, b \in[-2,2]\} \\
B & =\left\{(x, y): x \in \mathbb{R}, y \in \mathbb{R}, y=a x, a \in\left[-3,-\frac{1}{3}\right]\right\}
\end{aligned}
$$

Zbadaj, czy punkt $\left(1,-\frac{1}{2}\right)$ należy do zbioru $A \cap B$.\\
4. Spośród trapezów równoramiennych o danym obwodzie $p$ i danym kącie $\alpha$ przy podstawie wyznacz trapez o największym polu.\\
5. Dane są trzy kolejne wierzchołki prostokąta $A B C D: A(-5,-3), B(-2,0), C(-7,5)$. Napisz równanie okręgu opisanego na tym prostokącie oraz równanie prostej stycznej do tego okręgu w punkcie $D$.\\
6. Kwadrat $A B C D$ jest podstawą prostopadłościanu $A B C D E F G H$. Środek $M$ krawędzi $A B$ łączymy z wierzchołkiem $G$ otrzymując odcinek długości $d$ nachylony do ściany $D C G H$ pod kątem $\alpha$. Oblicz pole powierzchni bocznej tego prostopadłościanu.

\section*{PRACA KONTROLNA nr 5 - POZIOM ROZSZERZONY}
\begin{enumerate}
  \item Para $(x, y)$ jest rozwiązaniem układu:
\end{enumerate}

$$
\left\{\begin{array}{l}
x-y=-1-m \\
2 x-y=2 m
\end{array}\right.
$$

Dla jakich wartości $m$ punkt $P(x, y)$ należy do wnętrza koła o promieniu długości $r=\sqrt{5}$ i środku w początku układu współrzędnych?\\
2. Na ile sposobów można ustawić 5 książek na trzech półkach, jeśli ważna jest kolejność ustawienia książek oraz to, na której półce stoją?\\
3. Wyznacz zbiór środków wszystkich cięciw okręgu o równaniu $x^{2}+y^{2}=16$, które przechodzą przez punkt $(0,4)$. Wykonaj staranny rysunek.\\
4. Wykres funkcji $f(x)=x^{3}-3 x^{2}+b x+c$ przechodzi przez punkt $A(2,5)$. Współczynnik kierunkowy stycznej do wykresu funkcju w punkcie $A$ jest rozwiązaniem równania

$$
\left(\frac{4}{9}\right)^{x+1}=\left(\frac{81}{16}\right)^{x+13}
$$

Wyznacz najmniejszą i największą wartość funkcji w przedziale [-2,2].\\
5. Obwód trójkąta równoramiennego jest równy $a$. Przy jakich długościach boków pole trójkąta jest największe? Podaj największą wartość pola trójkąta dla $a=3+2 \sqrt{3}$.\\
6. W kole o środku $O$ poprowadzono dwie prostopadłe średnice $\overline{A B}$ i $\overline{C D}$ oraz cięciwę $\overline{A M}$ przecinającą średnicę $\overline{C D}$ w punkcie $K$. Dla jakiego kąta między średnicą $\overline{A B}$ a cięciwą $\overline{A M}$ w czworokąt $O B M K$ można wpisać okrąg?

Rozwiązania (rękopis) zadań z wybranego poziomu prosimy nadsyłać do 20.01.2023r. na adres:

Wydział Matematyki\\
Politechnika Wrocławska\\
Wybrzeże Wyspiańskiego 27\\
50-370 WROCŁAW,\\
lub elektronicznie, za pośrednictwem portalu \href{http://talent.pwr.edu.pl}{talent.pwr.edu.pl}\\
Na kopercie prosimy koniecznie zaznaczyć wybrany poziom! (np. poziom podstawowy lub rozszerzony). Do rozwiązań należy dołączyć zaadresowaną do siebie kopertę zwrotną z naklejonym znaczkiem, odpowiednim do formatu listu. Prace niespełniające podanych warunków nie będą poprawiane ani odsyłane.\\
Uwaga. Wysyłając nam rozwiązania zadań uczestnik Kursu udostępnia Politechnice Wrocławskiej swoje dane osobowe, które przetwarzamy wyłącznie w zakresie niezbędnym do jego prowadzenia (odesłanie zadań, prowadzenie statystyki). Szczegółowe informacje o przetwarzaniu przez nas danych osobowych są dostępne na stronie internetowej Kursu.\\
Adres internetowy Kursu: \href{http://www.im.pwr.edu.pl/kurs}{http://www.im.pwr.edu.pl/kurs}


\end{document}