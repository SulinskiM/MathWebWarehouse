\documentclass[a4paper,12pt]{article}
\usepackage{latexsym}
\usepackage{amsmath}
\usepackage{amssymb}
\usepackage{graphicx}
\usepackage{wrapfig}
\pagestyle{plain}
\usepackage{fancybox}
\usepackage{bm}

\begin{document}

XLIV

KORESPONDENCYJNY KURS

Z MATEMATYKI

luty 2015 r.

PRACA KONTROLNA nr 6- POZIOM PODSTAWOWY

l. Wyznacz dziedzinę funkcji

$f(x)=\log_{4-x^{2}}(2^{x}+2^{1-x}-3).$

2. $\mathrm{W}$ przedziale $[0,2\pi]$ rozwiąz nierównośč

$\displaystyle \cos^{2}2x+\sin^{2}x\leq\frac{1}{2}.$

3. Obwód trójk$\Phi$ta równoramiennego jest równy 8. Jaka powinna byč dfugośč boków tego

trójkąta, by objętośč bryły powstałej $\mathrm{z}$ jego obrotu dokola podstawy byla największa?

4. Rozwiąz równanie

$\sqrt{1-23^{x}+9^{x}}=3^{2x-1}-7\cdot 3^{x-1}+2.$

5. Punkt $B(1,1)$ jest wierzchołkiem kąta prostego $\mathrm{w}$ trójkącie prostokątnym $0$ polu 2, wpi-

sanym $\mathrm{w}$ okrąg $x^{2}+y^{2}+2x-2y-2=0.$ Znajd $\acute{\mathrm{z}}$ współrzędne pozostałych wierzchołków

tego trójkąta. Rozwiązanie zilustruj starannym rysunkiem.

6. Sporzqd $\acute{\mathrm{z}}$ staranny wykres funkcji

$f(x)=$

dla

dla

$|2x-5|\geq 3,$

$|2x-5|<3,$

$\mathrm{i}$ na jego podstawie wyznacz zbiór wartości tej funkcji. Rozwiąz nierównośč $f^{2}(x) \leq 1$

$\mathrm{i}$ zaznacz zbiór jej rozwiązań na osi $0x.$
\end{document}
