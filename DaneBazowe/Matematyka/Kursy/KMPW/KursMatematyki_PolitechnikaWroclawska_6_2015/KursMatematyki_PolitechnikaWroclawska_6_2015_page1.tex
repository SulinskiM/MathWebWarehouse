\documentclass[a4paper,12pt]{article}
\usepackage{latexsym}
\usepackage{amsmath}
\usepackage{amssymb}
\usepackage{graphicx}
\usepackage{wrapfig}
\pagestyle{plain}
\usepackage{fancybox}
\usepackage{bm}

\begin{document}

PRACA KONTROLNA nr 6- POZ1OM ROZSZERZONY

l. Narysuj staranny wykres funkcji

$f(x)=|2^{|x-1|}-4|-2$

$\mathrm{i}$ opisz dokładnie sposób jego konstrukcji. Korzystając $\mathrm{z}$ rysunku, określ ilośč rozwiązań

równania $f(x)=m\mathrm{w}$ zalezności od parametru $m.$

2. Rozwiąz równanie

2 $\cos 2x+1=\sqrt{2\cos^{2}2x-6\sin^{2}x+5}.$

3. $\mathrm{W}$ trójkącie prostokątnym przeciwprostokątna ma długośč 3. Jakie powinny byč d1ugości

przyprostokątnych, aby objętośč bryły powstafej $\mathrm{z}$ jego obrotu dokołajednej $\mathrm{z}$ nich byla

największa?

4. Rozwiąz nierównośč

$2^{x}(1+\displaystyle \frac{\sqrt{3}}{2})^{\frac{1}{x}}-(2-\sqrt{3})^{-x}\geq 0.$

5. Znajd $\acute{\mathrm{z}}$ równania prostych stycznych do okręgu $x^{2}+y^{2}=25$ przechodzących przez punkt

$S(6,8)$. Wyznacz współrzędne punktów styczności $A, B\mathrm{i}$ oblicz pole obszaru ograniczo-

nego odcinkami AS, $BS$ oraz większym fukiem $AB$. Wykonaj staranny rysunek.

6. Zbadaj przebieg zmienności $\mathrm{i}$ narysuj staranny wykres funkcji

$f(x)=\displaystyle \frac{3x-2}{(x-1)^{2}}.$

Rozwiązania (rękopis) zadań z wybranego poziomu prosimy nadsyfač do

na adres:

181utego 20l5r.

Katedra Matematyki WPPT

Politechniki Wrocfawskiej

Wybrzez $\mathrm{e}$ Wyspiańskiego 27

$50-370$ WROCLAW.

Na kopercie prosimy $\underline{\mathrm{k}\mathrm{o}\mathrm{n}\mathrm{i}\mathrm{e}\mathrm{c}\mathrm{z}\mathrm{n}\mathrm{i}\mathrm{e}}$ zaznaczyč wybrany poziom! (np. poziom podsta-

wowy $\mathrm{l}\mathrm{u}\mathrm{b}$ rozszerzony). Do rozwiązań nalez $\mathrm{y}$ dołączyč zaadresowaną do siebie kopertę

zwrotną $\mathrm{z}$ naklejonym znaczkiem, odpowiednim do wagi listu. Prace niespelniające po-

danych warunków nie będą poprawiane ani odsyłane.

Adres internetowy Kursu: http://www.im.pwr.wroc.pl/kurs
\end{document}
