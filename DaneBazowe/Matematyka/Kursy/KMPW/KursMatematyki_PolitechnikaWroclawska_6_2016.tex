\documentclass[a4paper,12pt]{article}
\usepackage{latexsym}
\usepackage{amsmath}
\usepackage{amssymb}
\usepackage{graphicx}
\usepackage{wrapfig}
\pagestyle{plain}
\usepackage{fancybox}
\usepackage{bm}

\begin{document}

XLV

KORESPONDENCYJNY KURS

Z MATEMATYKI

luty 2016 r.

PRACA KONTROLNA nr 6- POZIOM PODSTAWOWY

l. Andrzej przebiegł maraton, pokonujac drugą połowę trasy 10\% wo1niej od pierwszej.

Bernard, biegnąc początkowo w tempie narzuconym przez Andrzeja, w połowie czasu

biegu zwolnił 010\%. Usta1, który z biegaczy pierwszy przekroczy11inię mety.

2. Niech $p$ będzie liczbą pierwszą, $p\geq 7$. Uzasadnij, $\dot{\mathrm{z}}\mathrm{e}$ liczba $p^{\mathrm{z}}-49$ jest podzielna przez

24.

3. Rozwia $\dot{\mathrm{z}}$ równanie

12 $\cos^{2}3x\cdot\sin^{2}2x+\sin^{2}3x=4\sin^{2}3x\cdot\sin^{2}2x+3\cos^{2}3x.$

4. Wyznacz wszystkie argumenty x, dla których funkcja

$f(x)=\log_{3}(x^{2}-x)$ -log9 $(x^{2}+x-2)$

przyjmuje wartości dodatnie.

5. Przekątna rombu $0$ obwodzie 12 jest zawarta $\mathrm{w}$ prostej $x-2y=0$, a punkt $A(1,3)$ jest

jednym $\mathrm{z}$ jego wierzchofków. Wyznaczyč wspófrzędne pozostalych wierzchofków tego

rombu $\mathrm{i}$ obliczyč jego pole. Wykonač staranny rysunek.

6. Narysuj wykres funkcji

$f(x)=\sin^{2}x+\cos^{2}x+\sin^{4}x+\cos^{4}x+\sin^{6}x+\cos^{6}x.$

Znajd $\acute{\mathrm{z}}$ wszystkie liczby $\mathrm{z}$ przedziafu $[0,2\pi]$ spelniające nierównośč $8f(x)>19$. Zastosuj

wzory $\sin 2\alpha=2\sin\alpha\cdot\cos\alpha$ oraz $\cos 2\alpha=\cos^{2}\alpha-\sin^{2}\alpha.$




PRACA KONTROLNA nr 6- POZIOM ROZSZERZONY

l. Na nowym osiedlu wybudowano sześč budynków. $K\mathrm{a}\dot{\mathrm{z}}\mathrm{d}\mathrm{y}$ zostanie pomalowany na jeden

$\mathrm{z}$ trzech kolorów, a $\mathrm{k}\mathrm{a}\dot{\mathrm{z}}\mathrm{d}\mathrm{y}$ kolor zostanie wykorzystany co najmniej $\mathrm{r}\mathrm{a}\mathrm{z}$. Ustal, na ile

sposobów $\mathrm{m}\mathrm{o}\dot{\mathrm{z}}$ na pomalowač te budynki.

2. Zbadaj, dla jakich argumentów $x$ funkcja

$f(x)=7^{x^{4}}\cdot 49^{x}\cdot 5^{2x^{3}+x^{2}}-5^{x^{4}-2}\cdot 25^{x+1}\cdot 49^{x^{3}+\frac{1}{2}x^{2}}$

przyjmuje wartości dodatnie.

3. Rozwiąz równanie

$\mathrm{t}\mathrm{g}^{2}x=$ ($4\mathrm{t}\mathrm{g}^{2}x+3$ tg $x-1$) ($1$ -tg $ x+\mathrm{t}\mathrm{g}^{2}x-\mathrm{t}\mathrm{g}^{3}x+\ldots$).

4. Wskaz wszystkie wartości $x$, dla których suma nieskończonego ciągu geometrycznego

$ S(x)=2^{-2\sin 3x}+2^{-4\sin 3x}+2^{-6\sin 3x}+\cdots+2^{-2n\sin 3x}+\ldots$

nie przekracza jedności.

5. Rozwiąz nierównośč $\mathrm{l}\mathrm{y}\mathrm{n}$

$\log_{x+1}(x^{3}-x)\geq\log_{x+1}(x+2)+1.$

6. Boki $\triangle ABC$ zawarte są $\mathrm{w}$ prostych $y=4, y= 1-mx$ oraz $y=2(x-m)$. Wyznacz

wszystkie wymierne wartości parametru $m$, dla których pole rozwazanego trójkąta wy-

nosi $|\triangle ABC|=12$. Dla $\mathrm{k}\mathrm{a}\dot{\mathrm{z}}$ dej wyznaczonej wartości $m$ wykonaj odpowiedni rysunek.

Rozwiązania prosimy nadsyłač do dnia

181utego 20l6 na adres:

Wydziaf Matematyki

Politechniki Wrocfawskiej

Wybrzez $\mathrm{e}$ Wyspiańskiego 27

$50\rightarrow 370$ Wroclaw.

Na kopercie prosimy koniecznie zaznaczyč wybrany poziom. Do rozwiązań nalezy do-

l\S czyč zaadresowaną do siebie kopertę zwrotn\S z naklejonym znaczkiem, odpowiednim do wagi listu.

Prace niespelniające podanych warunków nie będą poprawiane ani odsylane.

Adres internetowy Kursu:

http://www. im.pwr.edu.pl/kur s



\end{document}