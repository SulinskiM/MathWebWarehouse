\documentclass[a4paper,12pt]{article}
\usepackage{latexsym}
\usepackage{amsmath}
\usepackage{amssymb}
\usepackage{graphicx}
\usepackage{wrapfig}
\pagestyle{plain}
\usepackage{fancybox}
\usepackage{bm}

\begin{document}

XLV

KORESPONDENCYJNY KURS

Z MATEMATYKI

luty 2016 r.

PRACA KONTROLNA nr 6- POZIOM PODSTAWOWY

l. Andrzej przebiegł maraton, pokonujac drugą połowę trasy 10\% wo1niej od pierwszej.

Bernard, biegnąc początkowo w tempie narzuconym przez Andrzeja, w połowie czasu

biegu zwolnił 010\%. Usta1, który z biegaczy pierwszy przekroczy11inię mety.

2. Niech $p$ będzie liczbą pierwszą, $p\geq 7$. Uzasadnij, $\dot{\mathrm{z}}\mathrm{e}$ liczba $p^{\mathrm{z}}-49$ jest podzielna przez

24.

3. Rozwia $\dot{\mathrm{z}}$ równanie

12 $\cos^{2}3x\cdot\sin^{2}2x+\sin^{2}3x=4\sin^{2}3x\cdot\sin^{2}2x+3\cos^{2}3x.$

4. Wyznacz wszystkie argumenty x, dla których funkcja

$f(x)=\log_{3}(x^{2}-x)$ -log9 $(x^{2}+x-2)$

przyjmuje wartości dodatnie.

5. Przekątna rombu $0$ obwodzie 12 jest zawarta $\mathrm{w}$ prostej $x-2y=0$, a punkt $A(1,3)$ jest

jednym $\mathrm{z}$ jego wierzchofków. Wyznaczyč wspófrzędne pozostalych wierzchofków tego

rombu $\mathrm{i}$ obliczyč jego pole. Wykonač staranny rysunek.

6. Narysuj wykres funkcji

$f(x)=\sin^{2}x+\cos^{2}x+\sin^{4}x+\cos^{4}x+\sin^{6}x+\cos^{6}x.$

Znajd $\acute{\mathrm{z}}$ wszystkie liczby $\mathrm{z}$ przedziafu $[0,2\pi]$ spelniające nierównośč $8f(x)>19$. Zastosuj

wzory $\sin 2\alpha=2\sin\alpha\cdot\cos\alpha$ oraz $\cos 2\alpha=\cos^{2}\alpha-\sin^{2}\alpha.$
\end{document}
