\documentclass[a4paper,12pt]{article}
\usepackage{latexsym}
\usepackage{amsmath}
\usepackage{amssymb}
\usepackage{graphicx}
\usepackage{wrapfig}
\pagestyle{plain}
\usepackage{fancybox}
\usepackage{bm}

\begin{document}

PRACA KONTROLNA nr 6- POZIOM ROZSZERZONY

l. Na nowym osiedlu wybudowano sześč budynków. $K\mathrm{a}\dot{\mathrm{z}}\mathrm{d}\mathrm{y}$ zostanie pomalowany na jeden

$\mathrm{z}$ trzech kolorów, a $\mathrm{k}\mathrm{a}\dot{\mathrm{z}}\mathrm{d}\mathrm{y}$ kolor zostanie wykorzystany co najmniej $\mathrm{r}\mathrm{a}\mathrm{z}$. Ustal, na ile

sposobów $\mathrm{m}\mathrm{o}\dot{\mathrm{z}}$ na pomalowač te budynki.

2. Zbadaj, dla jakich argumentów $x$ funkcja

$f(x)=7^{x^{4}}\cdot 49^{x}\cdot 5^{2x^{3}+x^{2}}-5^{x^{4}-2}\cdot 25^{x+1}\cdot 49^{x^{3}+\frac{1}{2}x^{2}}$

przyjmuje wartości dodatnie.

3. Rozwiąz równanie

$\mathrm{t}\mathrm{g}^{2}x=$ ($4\mathrm{t}\mathrm{g}^{2}x+3$ tg $x-1$) ($1$ -tg $ x+\mathrm{t}\mathrm{g}^{2}x-\mathrm{t}\mathrm{g}^{3}x+\ldots$).

4. Wskaz wszystkie wartości $x$, dla których suma nieskończonego ciągu geometrycznego

$ S(x)=2^{-2\sin 3x}+2^{-4\sin 3x}+2^{-6\sin 3x}+\cdots+2^{-2n\sin 3x}+\ldots$

nie przekracza jedności.

5. Rozwiąz nierównośč $\mathrm{l}\mathrm{y}\mathrm{n}$

$\log_{x+1}(x^{3}-x)\geq\log_{x+1}(x+2)+1.$

6. Boki $\triangle ABC$ zawarte są $\mathrm{w}$ prostych $y=4, y= 1-mx$ oraz $y=2(x-m)$. Wyznacz

wszystkie wymierne wartości parametru $m$, dla których pole rozwazanego trójkąta wy-

nosi $|\triangle ABC|=12$. Dla $\mathrm{k}\mathrm{a}\dot{\mathrm{z}}$ dej wyznaczonej wartości $m$ wykonaj odpowiedni rysunek.

Rozwiązania prosimy nadsyłač do dnia

181utego 20l6 na adres:

Wydziaf Matematyki

Politechniki Wrocfawskiej

Wybrzez $\mathrm{e}$ Wyspiańskiego 27

$50\rightarrow 370$ Wroclaw.

Na kopercie prosimy koniecznie zaznaczyč wybrany poziom. Do rozwiązań nalezy do-

l\S czyč zaadresowaną do siebie kopertę zwrotn\S z naklejonym znaczkiem, odpowiednim do wagi listu.

Prace niespelniające podanych warunków nie będą poprawiane ani odsylane.

Adres internetowy Kursu:

http://www. im.pwr.edu.pl/kur s
\end{document}
