\documentclass[10pt]{article}
\usepackage[polish]{babel}
\usepackage[utf8]{inputenc}
\usepackage[T1]{fontenc}
\usepackage{amsmath}
\usepackage{amsfonts}
\usepackage{amssymb}
\usepackage[version=4]{mhchem}
\usepackage{stmaryrd}
\usepackage{hyperref}
\hypersetup{colorlinks=true, linkcolor=blue, filecolor=magenta, urlcolor=cyan,}
\urlstyle{same}

\title{PRACA KONTROLNA nr 6 - POZIOM PODSTAWOWY }

\author{}
\date{}


\begin{document}
\maketitle
\begin{enumerate}
  \item Andrzej przebiegł maraton, pokonując drugą połowę trasy $10 \%$ wolniej od pierwszej. Bernard, biegnąc początkowo w tempie narzuconym przez Andrzeja, w połowie czasu biegu zwolnił o $10 \%$. Ustal, który z biegaczy pierwszy przekroczył linię mety.
  \item Niech $p$ będzie liczbą pierwszą, $p \geqslant 7$. Uzasadnij, że liczba $p^{2}-49$ jest podzielna przez 24.
  \item Rozwiąż równanie
\end{enumerate}

$$
12 \cos ^{2} 3 x \cdot \sin ^{2} 2 x+\sin ^{2} 3 x=4 \sin ^{2} 3 x \cdot \sin ^{2} 2 x+3 \cos ^{2} 3 x
$$

\begin{enumerate}
  \setcounter{enumi}{3}
  \item Wyznacz wszystkie argumenty $x$, dla których funkcja
\end{enumerate}

$$
f(x)=\log _{3}\left(x^{2}-x\right)-\log _{9}\left(x^{2}+x-2\right)
$$

przyjmuje wartości dodatnie.\\
5. Przekątna rombu o obwodzie 12 jest zawarta w prostej $x-2 y=0$, a punkt $A(1,3)$ jest jednym z jego wierzchołków. Wyznaczyć współrzędne pozostałych wierzchołków tego rombu i obliczyć jego pole. Wykonać staranny rysunek.\\
6. Narysuj wykres funkcji

$$
f(x)=\sin ^{2} x+\cos ^{2} x+\sin ^{4} x+\cos ^{4} x+\sin ^{6} x+\cos ^{6} x
$$

Znajdź wszystkie liczby z przedziału $[0,2 \pi]$ spełniające nierówność $8 f(x)>19$. Zastosuj wzory $\sin 2 \alpha=2 \sin \alpha \cdot \cos \alpha$ oraz $\cos 2 \alpha=\cos ^{2} \alpha-\sin ^{2} \alpha$.

\section*{PRACA KONTROLNA nr 6 - POZIOM ROZSZERZONY}
\begin{enumerate}
  \item Na nowym osiedlu wybudowano sześć budynków. Każdy zostanie pomalowany na jeden z trzech kolorów, a każdy kolor zostanie wykorzystany co najmniej raz. Ustal, na ile sposobów można pomalować te budynki.
  \item Zbadaj, dla jakich argumentów $x$ funkcja
\end{enumerate}

$$
f(x)=7^{x^{4}} \cdot 49^{x} \cdot 5^{2 x^{3}+x^{2}}-5^{x^{4}-2} \cdot 25^{x+1} \cdot 49^{x^{3}+\frac{1}{2} x^{2}}
$$

przyjmuje wartości dodatnie.\\
3. Rozwiąż równanie

$$
\operatorname{tg}^{2} x=\left(4 \operatorname{tg}^{2} x+3 \operatorname{tg} x-1\right)\left(1-\operatorname{tg} x+\operatorname{tg}^{2} x-\operatorname{tg}^{3} x+\ldots\right)
$$

\begin{enumerate}
  \setcounter{enumi}{3}
  \item Wskaż wszystkie wartości $x$, dla których suma nieskończonego ciągu geometrycznego
\end{enumerate}

$$
S(x)=2^{-2 \sin 3 x}+2^{-4 \sin 3 x}+2^{-6 \sin 3 x}+\cdots+2^{-2 n \sin 3 x}+\ldots
$$

nie przekracza jedności.\\
5. Rozwiąż nierówność logarytmiczną

$$
\log _{x+1}\left(x^{3}-x\right) \geqslant \log _{x+1}(x+2)+1
$$

\begin{enumerate}
  \setcounter{enumi}{5}
  \item Boki $\triangle A B C$ zawarte są w prostych $y=4, y=1-m x$ oraz $y=2(x-m)$. Wyznacz wszystkie wymierne wartości parametru $m$, dla których pole rozważanego trójkąta wynosi $|\triangle A B C|=12$. Dla każdej wyznaczonej wartości $m$ wykonaj odpowiedni rysunek.
\end{enumerate}

Rozwiązania prosimy nadsyłać do dnia 18 lutego 2016 na adres:

\begin{verbatim}
Wydział Matematyki
Politechniki Wrocławskiej
Wybrzeże Wyspiańskiego 27
50-370 Wrocław.
\end{verbatim}

Na kopercie prosimy koniecznie zaznaczyć wybrany poziom. Do rozwiązań należy dołączyć zaadresowaną do siebie kopertę zwrotną z naklejonym znaczkiem, odpowiednim do wagi listu. Prace niespełniające podanych warunków nie będą poprawiane ani odsyłane.

Adres internetowy Kursu: \href{http://www.im.pwr.edu.pl/kurs}{http://www.im.pwr.edu.pl/kurs}


\end{document}