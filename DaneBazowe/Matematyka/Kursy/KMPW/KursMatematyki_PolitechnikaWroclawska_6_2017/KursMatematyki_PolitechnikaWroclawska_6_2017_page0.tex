\documentclass[a4paper,12pt]{article}
\usepackage{latexsym}
\usepackage{amsmath}
\usepackage{amssymb}
\usepackage{graphicx}
\usepackage{wrapfig}
\pagestyle{plain}
\usepackage{fancybox}
\usepackage{bm}

\begin{document}

XLVI

KORESPONDENCYJNY KURS

Z MATEMATYKI

luty 2017 r.

PRACA KONTROLNA nr 6- POZIOM PODSTAWOWY

l. Rozwiązač równanie

$\displaystyle \frac{\sin x}{2\cos^{2}2x-1}=1.$

2. Niech $f(x)=\sqrt{x}$. Podač wzór funkcji:

a) $g(x)$, której wykresjest symetrycznym obrazem wykresu $f(x)$ względem prostej $x=1.$

b) $h(x)$, której wykresjest symetrycznym obrazem wykresu $f(x)$ względem punktu $(0,-1).$

Narysowač wykresy wszystkich funkcji. Uzasadnič, wykonując odpowiednie obliczenia,

$\dot{\mathrm{z}}\mathrm{e}$ znalezione funkcje spełniają podane warunki.

3. Wykazač, $\dot{\mathrm{z}}\mathrm{e}$ dla dowolnego $n\geq 2$ liczba

naturalnej $\mathrm{i}$ jest podzielna przez 81.

$\displaystyle \frac{1}{4} 100^{n}+4\cdot 10^{n}+16$ jest kwadratem liczby

4. Narysowač wykres funkcji

$f(x)=$

, gdy

, gdy

$-1\leq x\leq 1,$

$|x|>1.$

Poslugując się wykresem, podač zbiór wartości funkcji $f$ oraz jej najmniejszą $\mathrm{i}$ największą

wartośč na przedziałach [-1, 2] oraz $[0$, 3$].$

5. Znalez$\acute{}$č równanie stycznej $l$ do paraboli $y=x^{2}$ równolegfej do prostej $y=2x-3.$

Wyznaczyč punkt, $\mathrm{w}$ którym styczna do tej paraboli jest prostopadła do znalezionej

prostej $l$. Sporządzič rysunek.

6. Rozwiązač uklad równań

$\left\{\begin{array}{l}
x^{2}+y^{2}\\
- x1+-y1
\end{array}\right.$

$\mathrm{i}$ podač jego interpretację geometryczną.

8,

1.
\end{document}
