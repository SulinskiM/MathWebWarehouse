\documentclass[a4paper,12pt]{article}
\usepackage{latexsym}
\usepackage{amsmath}
\usepackage{amssymb}
\usepackage{graphicx}
\usepackage{wrapfig}
\pagestyle{plain}
\usepackage{fancybox}
\usepackage{bm}

\begin{document}

PRACA KONTROLNA nr 5- POZIOM ROZSZERZONY

l. Niech $f(x)=\displaystyle \frac{x-1}{x+2}$. Podač $\mathrm{i}$ uzasadnič wzór funkcji, której wykres jest obrazem syme-

trycznym wykresu funkcji $f(x)$ względem prostej $x=2$. Sporządzič wykresy obu funkcji

$\mathrm{w}$ jednym ukladzie współrzędnych.

2. Stosujac zasadę indukcji matematycznej, udowodnič prawdziwośč wzoru

$\left(\begin{array}{l}
2\\
2
\end{array}\right) +\left(\begin{array}{l}
4\\
2
\end{array}\right)+\cdots+\left(\begin{array}{l}
2n\\
2
\end{array}\right) =\displaystyle \frac{n(n+1)(4n-1)}{6}$ dla $n\geq 1.$

3. $\mathrm{W}\mathrm{y}\mathrm{k}\mathrm{o}\mathrm{r}\mathrm{z}\mathrm{y}\mathrm{s}\mathrm{t}\mathrm{u}\mathrm{j}_{\Phi}\mathrm{c}$ metody rachunku rózniczkowego znalez/č zbiór wartości funkcji

$f(x)=x^{3}-3x^{2}-9x+3$

na przedziale [-1, 4]. Wyznaczyč przedzia1y $0$ dlugości l, $\mathrm{w}$ których znajdują się miejsca

zerowe tej funkcji $\mathrm{i}$ sporządzič jej wykres.

4. Znalez/č równanie stycznej $l$ do wykresu funkcji $f(x) = \displaystyle \frac{2}{x}+x^{2}\mathrm{w}$ punkcie przecięcia $\mathrm{z}$

prostą $y=x$. Wyznaczyč wszystkie styczne równolegle do znalezionej prostej $l.$

5. Narysowač wykres funkcji

$f(x)=1+\displaystyle \frac{\sin x}{1+\sin x}+(\frac{\sin x}{1+\sin x})^{2}+(\frac{\sin x}{1+\sin x})^{3}+(\frac{\sin x}{1+\sin x})^{4}+\ldots,$

gdzie prawa stronajest suma wszystkich wyrazów nieskończonego ciągu geometrycznego.

Rozwiazač nierównośč

$f(x)>\sqrt{3}\cos x.$

6. Wyznaczyč liczbę rozwiązań układu równań

$\left\{\begin{array}{l}
x^{2}+y^{2}=2y,\\
y=x^{2}-p.
\end{array}\right.$

$\mathrm{w}$ zalezności od parametru $p$. Podač interpretacje geometryczną układu.

$\mathrm{R}\mathrm{o}\mathrm{z}\mathrm{w}\mathrm{i}_{\Phi}$zania prosimy nadsyfač do dnia

181utego 20l7 na adres:

Wydziaf Matematyki

Politechniki Wrocfawskiej

Wybrzez $\mathrm{e}$ Wyspiańskiego 27

$50-370$ Wrocfaw.

Na kopercie prosimy koniecznie zaznaczyč wybrany poziom. Do rozwiązań nalez$\mathrm{y}$ do-

lączyč zaadresowana do siebie kopertę zwrotnq $\mathrm{z}$ naklejonym znaczkiem, odpowiednim do wagi listu.

Prace niespelniaj\S ce podanych warunków nie będą poprawiane ani odsyfane.

Adres internetowy Kursu: http: //www. im. pwr. edu. pl/kurs
\end{document}
