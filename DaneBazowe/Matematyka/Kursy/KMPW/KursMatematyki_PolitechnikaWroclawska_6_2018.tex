\documentclass[a4paper,12pt]{article}
\usepackage{latexsym}
\usepackage{amsmath}
\usepackage{amssymb}
\usepackage{graphicx}
\usepackage{wrapfig}
\pagestyle{plain}
\usepackage{fancybox}
\usepackage{bm}

\begin{document}

XLVII

KORESPONDENCYJNY KURS

Z MATEMATYKI

luty 2018 r.

PRACA KONTROLNA nr 6- POZIOM PODSTAWOWY

1. Rozwia $\dot{\mathrm{z}}$ nierównośč

$\displaystyle \frac{3x-1}{x}\geq 1+\frac{\sqrt{1-x}}{x}.$

2. $\mathrm{W}$ zagrodzie jest 10 zwierząt, po parze danego gatunku. Ob1icz prawdopodobieństwo,

$\dot{\mathrm{z}}\mathrm{e}\mathrm{w}$ zagrodzie zostanie choč jedno zwierzę $\mathrm{k}\mathrm{a}\dot{\mathrm{z}}$ dego gatunku, jeśli wypuścimy $\mathrm{z}$ niej 4

losowo wybrane zwierzęta.

3. Bez $\mathrm{u}\dot{\mathrm{z}}$ ycia kalkulatora porównaj liczby

$a=\sqrt{11-4\sqrt{7}}$

oraz

$b=\log^{2}2\cdot\log 250+\log^{2}5$. log40.

4. Wyznacz wszystkie argumenty $x$, dla których funkcja

$f(x)=27^{x^{2}}\cdot 4^{x^{2}(x-3)}\cdot 3^{x}-6\cdot 3^{x^{3}+2}\cdot 2^{2x-7}$

przyjmuje wartości dodatnie.

5. Wyznacz skalę podobieństwa trójkąta równobocznego opisanego na okregu do trójkąta

równobocznego wpisanego $\mathrm{w}$ ten okrąg. Jaki jest stosunek pól tych trójk$\Phi$tów, a jaki

stosunek objetości stozka $0$ kącie rozwarcia $60^{\mathrm{o}}$ opisanego na kuli do objętości podobnego

stozka wpisanęgo $\mathrm{w}$ tę kulę?

6. Wśród prostokątów 0 ustalonej długości przekątnej p wskaz ten 0 największym polu.




PRACA KONTROLNA nr 6- POZ1OM ROZSZERZONY

l. Rozwiąz nierównośč

$x+1+\displaystyle \frac{1}{x-1}\geq(1+\frac{1}{x-1})\sqrt{2-x}.$

2. Narysuj wykres funkcji $f(x) = |1+ \displaystyle \log_{2}\frac{1}{|1-|x||}|$, opisz sfownie metodę jego kon-

strukcji oraz zbadaj, dla jakich argumentów spelniona jest nierównośč $f(x)\leq 1.$

3. Rozwiąz równanie logarytmiczne

$\log_{(x+2)^{2}}|x-1|=\log_{|x-1|}\sqrt{x+2}.$

4. Trzech alpinistów atakuje szczyt, wchodząc jednocześnie, niezaleznie od siebie, $\mathrm{z}$ róz-

nych stron góry. Prawdopodobieństwo zdobycia szczytu szlakiem północnym wynosi $\displaystyle \frac{1}{3},$

szlakiem zachodnim- $\displaystyle \frac{1}{2}$, a południowym- $\displaystyle \frac{3}{7}$. Oblicz prawdopodobieństwo, $\dot{\mathrm{z}}\mathrm{e}$ atak się

powiedzie (tzn. przynajmniej jeden $\mathrm{z}$ alpinistów zdobędzie szczyt).

5. Oblicz tangens kąta rozwarcia stozka, dla którego kula wpisana w ten stozek zajmuje

dokfadnie polowę jego objętości.

6. Wyznacz równanie linii będącej zbiorem środków wszystkich okręgów stycznych do pro-

stej $y=0$ ijednocześnie stycznych do okręgu $x^{2}+y^{2}=2$. Wykonaj odpowiedni rysunek.

Rozwiqzania prosimy nadsyłač do dnia 181utego 2018 na adres:

Korespondencyjny Kurs $\mathrm{z}$ Matematyki

POZIOM$\ldots$ (wpisač wfaściwy)

Wydziaf Matematyki

Politechnika Wrocfawska

Wybrzez $\mathrm{e}$ Wyspiańskiego 27

$50-370$ Wrocfaw

Na kopercie prosimy koniecznie zaznaczyč wybrany poziom (podstawowy, rozsze-

rzony lub podstawowy $\mathrm{i}$ rozszerzony). Do rozwiązań nalez$\mathrm{y}$ dofączyč zaadresowaną do siebie

kopertę zwrotn\S z naklejonym znaczkiem, odpowiednim do wagi listu $\mathrm{i}$ rozmiaru koperty. Prace nie-

spełniające podanych warunków nie będa poprawiane ani odsyfane.

Adres internetowy Kursu:

http://www. im.pwr.edu.pl/kur s



\end{document}