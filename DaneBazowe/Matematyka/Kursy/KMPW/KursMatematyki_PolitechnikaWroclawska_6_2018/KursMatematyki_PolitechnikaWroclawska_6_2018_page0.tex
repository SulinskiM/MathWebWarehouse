\documentclass[a4paper,12pt]{article}
\usepackage{latexsym}
\usepackage{amsmath}
\usepackage{amssymb}
\usepackage{graphicx}
\usepackage{wrapfig}
\pagestyle{plain}
\usepackage{fancybox}
\usepackage{bm}

\begin{document}

XLVII

KORESPONDENCYJNY KURS

Z MATEMATYKI

luty 2018 r.

PRACA KONTROLNA nr 6- POZIOM PODSTAWOWY

1. Rozwia $\dot{\mathrm{z}}$ nierównośč

$\displaystyle \frac{3x-1}{x}\geq 1+\frac{\sqrt{1-x}}{x}.$

2. $\mathrm{W}$ zagrodzie jest 10 zwierząt, po parze danego gatunku. Ob1icz prawdopodobieństwo,

$\dot{\mathrm{z}}\mathrm{e}\mathrm{w}$ zagrodzie zostanie choč jedno zwierzę $\mathrm{k}\mathrm{a}\dot{\mathrm{z}}$ dego gatunku, jeśli wypuścimy $\mathrm{z}$ niej 4

losowo wybrane zwierzęta.

3. Bez $\mathrm{u}\dot{\mathrm{z}}$ ycia kalkulatora porównaj liczby

$a=\sqrt{11-4\sqrt{7}}$

oraz

$b=\log^{2}2\cdot\log 250+\log^{2}5$. log40.

4. Wyznacz wszystkie argumenty $x$, dla których funkcja

$f(x)=27^{x^{2}}\cdot 4^{x^{2}(x-3)}\cdot 3^{x}-6\cdot 3^{x^{3}+2}\cdot 2^{2x-7}$

przyjmuje wartości dodatnie.

5. Wyznacz skalę podobieństwa trójkąta równobocznego opisanego na okregu do trójkąta

równobocznego wpisanego $\mathrm{w}$ ten okrąg. Jaki jest stosunek pól tych trójk$\Phi$tów, a jaki

stosunek objetości stozka $0$ kącie rozwarcia $60^{\mathrm{o}}$ opisanego na kuli do objętości podobnego

stozka wpisanęgo $\mathrm{w}$ tę kulę?

6. Wśród prostokątów 0 ustalonej długości przekątnej p wskaz ten 0 największym polu.
\end{document}
