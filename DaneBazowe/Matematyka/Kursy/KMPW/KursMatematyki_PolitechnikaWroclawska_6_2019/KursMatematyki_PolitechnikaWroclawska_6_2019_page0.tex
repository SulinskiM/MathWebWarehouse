\documentclass[a4paper,12pt]{article}
\usepackage{latexsym}
\usepackage{amsmath}
\usepackage{amssymb}
\usepackage{graphicx}
\usepackage{wrapfig}
\pagestyle{plain}
\usepackage{fancybox}
\usepackage{bm}

\begin{document}

XLVIII

KORESPONDENCYJNY KURS

Z MATEMATYKI

luty 2019 r.

PRACA KONTROLNA nr 6- POZIOM PODSTAWOWY

l. Pewnej $\mathrm{m}\mathrm{r}\mathrm{o}\acute{\mathrm{z}}\mathrm{n}\mathrm{e}\mathrm{j}$ zimy trzy przeziębione krasnale kupowały $\mathrm{w}$ aptece lekarstwa. Pierw-

szego męczył straszny ból gardfa. Kupil więc trzy tabletki do ssania, tabletkę na kaszel

$\mathrm{i}$ kropelkę do nosa. Zapłacił za wszyskto 4 grosze. Drugiemu dokuczaf uporczywy kasze1,

za tę samą cenę kupił trzy tabletki na kaszel, tabletkę do ssania $\mathrm{i}$ kropelkę do nosa. Trze-

ci mial straszny katar. Poprosif więc $0$ trzy kropelki do nosa, $0$ tabletkę do ssania oraz

$0$ tabletkę na kaszel. A dowiedziawszy się, $\dot{\mathrm{z}}\mathrm{e}$ ma zaplacič 2 grosze, pomyś1ał chwi1kę,

kichnął $\mathrm{i}$ powiedziaf do aptekarza:,,Pomylił się Pan!'' Uzasadnič, $\dot{\mathrm{z}}\mathrm{e}$ krasnal miał rację.

2. Obliczyč, ile kolejnych dodatnich liczb naturalnych podzielnych przez 3 nalez $\mathrm{y}$ dodač do

siebie, aby otrzymana suma była równa liczbie $115a^{-1}$, gdzie

$a=\displaystyle \frac{1}{3\cdot 5}+\frac{1}{5\cdot 7}+\frac{1}{7\cdot 9}+\ldots+\frac{1}{691\cdot 693}.$

3. Rozwiązač równanie

$\sin^{3}x$ ($1+$ ctg $x$)$+\cos^{3}x(1+\mathrm{t}\mathrm{g}x)=\sin 2x+2\sin^{2}x.$

4. Rzucamy pięč razy jednorodną kostką do gry. Obliczyč prawdopodobieństwo wyrzuce-

nia sumy oczek większej od 20, jeś1i wiadomo, $\dot{\mathrm{z}}\mathrm{e}$ suma oczek wyrzuconych $\mathrm{w}$ trzech

pierwszych rzutach wynosi 10.

5. $\mathrm{W}$ trójkąt równoramienny, którego ramiona sa dwa razy dłuzsze od podstawy, wpisano

$\mathrm{p}\mathrm{r}\mathrm{o}\mathrm{s}\mathrm{t}\mathrm{o}\mathrm{k}_{\Phi}\mathrm{t}\mathrm{w}$ taki sposób, $\dot{\mathrm{z}}$ ejeden $\mathrm{z}$ boków $\mathrm{P}^{\mathrm{r}\mathrm{o}\mathrm{s}\mathrm{t}\mathrm{o}\mathrm{k}}\Phi^{\mathrm{t}\mathrm{a}}$ zawarty jest $\mathrm{w}$ podstawie trójk$\Phi$ta.

Jakie powinny byč wymiary tego prostokąta, aby jego pole było największe? Wyznaczyč

wartośč tego największego pola.

6. Narysowač $\mathrm{w}$ prostokqtnym układzie wspólrzędnych wykresy funkcji

$f(x)=-\displaystyle \frac{2}{x}$

oraz

$g(x)=f(|x|-1)+1.$

Rozwiązač nierównośč $g(x)\geq f(x)\mathrm{i}$ zaznaczyč zbiór jej rozwiązań na osi liczbowej.
\end{document}
