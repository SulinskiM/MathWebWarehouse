\documentclass[a4paper,12pt]{article}
\usepackage{latexsym}
\usepackage{amsmath}
\usepackage{amssymb}
\usepackage{graphicx}
\usepackage{wrapfig}
\pagestyle{plain}
\usepackage{fancybox}
\usepackage{bm}

\begin{document}

PRACA KONTROLNA nr 6- POZIOM ROZSZERZONY

l. Ojciec $\mathrm{i}$ syn obchodzą urodziny tego samego dnia. $\mathrm{W}$ roku 2019 podczas uroczystości

urodzin zapytano jubilatów, ile maja lat. Ojciec odpowiedziaf:,,Jeśli wiek mego syna

przemnozę przez swój wiek za $31\mathrm{l}\mathrm{a}\mathrm{t}$, to otrzymam rok swego urodzenia'' Syn dodaf:

,,Aja otrzymam rok swego urodzenia, jeśli wiek mego ojca sprzed 161at przemnozę przez

swój wiek za 331ata'' $\mathrm{W}$ którym roku urodzif się $\mathrm{k}\mathrm{a}\dot{\mathrm{z}}\mathrm{d}\mathrm{y}\mathrm{z}$ jubilatów?

2. Wyznaczyč dziedzinę naturalną funkcji

$f(x)=\log(3^{3x-1}-3^{2x-1}-3^{x+1}+3).$

3. Rozwiązač równanie

4 $\sin x\cdot\sin 2x\cdot\sin 3x=\sin 4x.$

4. $\mathrm{W}$ dwóch urnach znajdują się kule biafe $\mathrm{i}$ czarne, przy czym $\mathrm{w}$ pierwszej urnie $\mathrm{s}\Phi 4$ kule

białe $\mathrm{i}6$ czarnych, a $\mathrm{w}$ drugiej jest 7 ku1 bia1ych $\mathrm{i}3$ czarne. Rzucamy dwa razy jedno-

rodną $\mathrm{k}\mathrm{o}\mathrm{s}\mathrm{t}\mathrm{k}_{\Phi}$ do gry. Jeśli suma wyrzuconych oczek jest mniejsza $\mathrm{n}\mathrm{i}\dot{\mathrm{z}}6$, losujemy dwie

kule $\mathrm{z}$ pierwszej urny. Jeśli suma wyrzuconych oczek jest większa $\mathrm{n}\mathrm{i}\dot{\mathrm{z}}9$, losujemy dwie

kule $\mathrm{z}$ drugiej urny. $\mathrm{W}$ pozostałych przypadkach losujemy po jednej kuli $\mathrm{z}\mathrm{k}\mathrm{a}\dot{\mathrm{z}}$ dej urny.

Obliczyč prawdopodobieństwo wylosowania dwóch kul bialych.

5. Uzasadnič, $\dot{\mathrm{z}}\mathrm{e}$ dla $\mathrm{k}\mathrm{a}\dot{\mathrm{z}}$ dej liczby naturalnej $n$ liczba $n^{5}-n$ jest podzielna przez 5. Czy

prawdq jest, $\dot{\mathrm{z}}\mathrm{e}$ jest ona $\mathrm{t}\mathrm{e}\dot{\mathrm{z}}$ podzielna przez 30?

6. $\mathrm{W}$ trójkąt równoramienny, którego ramiona są dfugości $a$, a miara kąta zawartego po-

między nimi wynosi $\alpha$, wpisano prostokat $\mathrm{w}$ taki sposób, $\dot{\mathrm{z}}\mathrm{e}$ jeden $\mathrm{z}$ boków prostokąta

zawarty jest $\mathrm{w}$ jednym $\mathrm{z}$ ramion trójkąta. Jakie powinny byč wymiary tego $\mathrm{P}^{\mathrm{r}\mathrm{o}\mathrm{s}\mathrm{t}\mathrm{o}\mathrm{k}}\Phi^{\mathrm{t}\mathrm{a}},$

aby jego pole bylo największe? Wyznaczyč wartośč tego największego pola.

Rozwiązania (rękopis) zadań z wybranego poziomu prosimy nadsyfač do

na adres:

18 stycznia 20l9r.

Wydziaf Matematyki

Politechnika Wrocfawska

Wybrzez $\mathrm{e}$ Wyspiańskiego 27

$50-370$ WROCLAW.

Na kopercie prosimy $\underline{\mathrm{k}\mathrm{o}\mathrm{n}\mathrm{i}\mathrm{e}\mathrm{c}\mathrm{z}\mathrm{n}\mathrm{i}\mathrm{e}}$ zaznaczyč wybrany poziom! (np. poziom podsta-

wowy lub rozszerzony). Do rozwiązań nalez $\mathrm{y}$ dołaczyč zaadresowaną do siebie kopertę

zwrotną $\mathrm{z}$ naklejonym znaczkiem, odpowiednim do wagi listu. Prace niespełniające po-

danych warunków nie bedą poprawiane ani odsyłane.

Uwaga. Wysylaj\S c nam rozwiązania zadań uczestnik Kursu udostępnia nam swoje dane osobo-

we, które przetwarzamy wyłącznie $\mathrm{w}$ zakresie niezbędnym do jego prowadzenia (odeslanie zadań,

prowadzenie statystyki). Szczególowe informacje $0$ przetwarzaniu przez nas danych osobowych sa

dostepne na stronie internetowej Kursu.

Adres internetowy Kursu: http://www.im.pwr.edu.pl/kurs
\end{document}
