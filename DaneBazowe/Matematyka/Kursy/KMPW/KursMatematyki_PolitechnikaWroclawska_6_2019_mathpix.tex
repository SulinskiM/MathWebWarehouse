\documentclass[10pt]{article}
\usepackage[polish]{babel}
\usepackage[utf8]{inputenc}
\usepackage[T1]{fontenc}
\usepackage{amsmath}
\usepackage{amsfonts}
\usepackage{amssymb}
\usepackage[version=4]{mhchem}
\usepackage{stmaryrd}
\usepackage{hyperref}
\hypersetup{colorlinks=true, linkcolor=blue, filecolor=magenta, urlcolor=cyan,}
\urlstyle{same}

\title{PRACA KONTROLNA nr 6 - POZIOM PODSTAWOWY }

\author{}
\date{}


\begin{document}
\maketitle
\begin{enumerate}
  \item Pewnej mroźnej zimy trzy przeziębione krasnale kupowały w aptece lekarstwa. Pierwszego męczył straszny ból gardła. Kupił więc trzy tabletki do ssania, tabletkę na kaszel i kropelkę do nosa. Zapłacił za wszyskto 4 grosze. Drugiemu dokuczał uporczywy kaszel, za tę samą cenę kupił trzy tabletki na kaszel, tabletkę do ssania i kropelkę do nosa. Trzeci miał straszny katar. Poprosił więc o trzy kropelki do nosa, o tabletkę do ssania oraz o tabletkę na kaszel. A dowiedziawszy się, że ma zapłacić 2 grosze, pomyślał chwilkę, kichnął i powiedział do aptekarza: „Pomylił się Pan!" Uzasadnić, że krasnal miał rację.
  \item Obliczyć, ile kolejnych dodatnich liczb naturalnych podzielnych przez 3 należy dodać do siebie, aby otrzymana suma była równa liczbie $115 a^{-1}$, gdzie
\end{enumerate}

$$
a=\frac{1}{3 \cdot 5}+\frac{1}{5 \cdot 7}+\frac{1}{7 \cdot 9}+\ldots+\frac{1}{691 \cdot 693}
$$

\begin{enumerate}
  \setcounter{enumi}{2}
  \item Rozwiązać równanie
\end{enumerate}

$$
\sin ^{3} x(1+\operatorname{ctg} x)+\cos ^{3} x(1+\operatorname{tg} x)=\sin 2 x+2 \sin ^{2} x
$$

\begin{enumerate}
  \setcounter{enumi}{3}
  \item Rzucamy pięć razy jednorodną kostką do gry. Obliczyć prawdopodobieństwo wyrzucenia sumy oczek większej od 20, jeśli wiadomo, że suma oczek wyrzuconych w trzech pierwszych rzutach wynosi 10 .
  \item W trójkąt równoramienny, którego ramiona są dwa razy dłuższe od podstawy, wpisano prostokąt w taki sposób, że jeden z boków prostokąta zawarty jest w podstawie trójkąta. Jakie powinny być wymiary tego prostokąta, aby jego pole było największe? Wyznaczyć wartość tego największego pola.
  \item Narysować w prostokątnym układzie współrzędnych wykresy funkcji
\end{enumerate}

$$
f(x)=-\frac{2}{x} \quad \text { oraz } \quad g(x)=f(|x|-1)+1
$$

Rozwiązać nierówność $g(x) \geqslant f(x)$ i zaznaczyć zbiór jej rozwiązań na osi liczbowej.

\section*{PRACA KONTROLNA nr 6 - POZIOM ROZSZERZONY}
\begin{enumerate}
  \item Ojciec i syn obchodzą urodziny tego samego dnia. W roku 2019 podczas uroczystości urodzin zapytano jubilatów, ile mają lat. Ojciec odpowiedział: „Jeśli wiek mego syna przemnożę przez swój wiek za 31 lat, to otrzymam rok swego urodzenia". Syn dodał: „A ja otrzymam rok swego urodzenia, jeśli wiek mego ojca sprzed 16 lat przemnożę przez swój wiek za 33 lata". W którym roku urodził się każdy z jubilatów?
  \item Wyznaczyć dziedzinę naturalną funkcji
\end{enumerate}

$$
f(x)=\log \left(3^{3 x-1}-3^{2 x-1}-3^{x+1}+3\right)
$$

\begin{enumerate}
  \setcounter{enumi}{2}
  \item Rozwiązać równanie
\end{enumerate}

$$
4 \sin x \cdot \sin 2 x \cdot \sin 3 x=\sin 4 x
$$

\begin{enumerate}
  \setcounter{enumi}{3}
  \item W dwóch urnach znajdują się kule białe i czarne, przy czym w pierwszej urnie są 4 kule białe i 6 czarnych, a w drugiej jest 7 kul białych i 3 czarne. Rzucamy dwa razy jednorodną kostką do gry. Jeśli suma wyrzuconych oczek jest mniejsza niż 6, losujemy dwie kule z pierwszej urny. Jeśli suma wyrzuconych oczek jest większa niż 9, losujemy dwie kule z drugiej urny. W pozostałych przypadkach losujemy po jednej kuli z każdej urny. Obliczyć prawdopodobieństwo wylosowania dwóch kul białych.
  \item Uzasadnić, że dla każdej liczby naturalnej $n$ liczba $n^{5}-n$ jest podzielna przez 5. Czy prawdą jest, że jest ona też podzielna przez 30?
  \item W trójkąt równoramienny, którego ramiona są długości $a$, a miara kąta zawartego pomiędzy nimi wynosi $\alpha$, wpisano prostokąt w taki sposób, że jeden z boków prostokąta zawarty jest w jednym z ramion trójkąta. Jakie powinny być wymiary tego prostokąta, aby jego pole było największe? Wyznaczyć wartość tego największego pola.
\end{enumerate}

Rozwiązania (rękopis) zadań z wybranego poziomu prosimy nadsyłać do 18 stycznia 2019r. na adres:

\begin{verbatim}
Wydział Matematyki
Politechnika Wrocławska
Wybrzeże Wyspiańskiego 27
50-370 WROCEAW.
\end{verbatim}

Na kopercie prosimy koniecznie zaznaczyć wybrany poziom! (np. poziom podstawowy lub rozszerzony). Do rozwiązań należy dołączyć zaadresowaną do siebie kopertę zwrotną z naklejonym znaczkiem, odpowiednim do wagi listu. Prace niespełniające podanych warunków nie będą poprawiane ani odsyłane.

Uwaga. Wysyłając nam rozwiązania zadań uczestnik Kursu udostępnia nam swoje dane osobowe, które przetwarzamy wyłącznie w zakresie niezbędnym do jego prowadzenia (odesłanie zadań, prowadzenie statystyki). Szczegółowe informacje o przetwarzaniu przez nas danych osobowych są dosteqpne na stronie internetowej Kursu.\\
Adres internetowy Kursu: \href{http://www.im.pwr.edu.pl/kurs}{http://www.im.pwr.edu.pl/kurs}


\end{document}