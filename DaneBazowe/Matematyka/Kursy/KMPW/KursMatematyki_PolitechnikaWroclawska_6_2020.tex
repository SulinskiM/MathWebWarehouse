\documentclass[a4paper,12pt]{article}
\usepackage{latexsym}
\usepackage{amsmath}
\usepackage{amssymb}
\usepackage{graphicx}
\usepackage{wrapfig}
\pagestyle{plain}
\usepackage{fancybox}
\usepackage{bm}

\begin{document}

XLIX

KORESPONDENCYJNY KURS

Z MATEMATYKI

luty 2020 r.

PRACA KONTROLNA $\mathrm{n}\mathrm{r} 6-$ POZIOM PODSTAWOWY

1. $\mathrm{W}$ szufiadzie znajduje się 6 róznych par rękawiczek. Ob1icz prawdopodobieństwo, $\dot{\mathrm{z}}\mathrm{e}$

wśród 51osowo wybranych rękawic jest co najmniej jedna para.

2. Wyznacz dziedzinę $\mathrm{i}$ zbadaj, dla jakich argumentów funkcja

$f(x)=\log_{\sqrt{3}}(x+3)-\log_{3}(9-x^{2})$

przyjmuje wartości ujemne.

3. Wśród prostok$\Phi$tów wpisanych $\mathrm{w}\mathrm{o}\mathrm{k}\mathrm{r}\Phi \mathrm{g}\mathrm{o}$ promieniu $R$ bez $\mathrm{u}\dot{\mathrm{z}}$ ycia metod rachunku róz-

niczkowego wskaz ten, którego pole jest największe.

4. Rozwiąz nierównośč

$4^{x^{3}-x+2}\cdot 5^{2x-3x^{2}}-2^{4-3x^{2}}\cdot 25^{x^{3}}\geq 0.$

5. Powierzchnia boczna stozka po rozcięciu jest wycinkiem kofa $\mathrm{o}\mathrm{k}_{\Phi}\mathrm{c}\mathrm{i}\mathrm{e}216^{\mathrm{o}}$

stawy stozka wynosi $ 6\pi$. Oblicz objętośč kuli wpisanej $\mathrm{w}$ ten stozek.

Obwód pod-

6. Narysuj wykres funkcji

$f(x)=-1+2^{1-|1-|x||}$

i precyzyjnie opisz zastosowaną metodę jego konstrukcji. Na podstawie rysunku wskaz

przedziafy monotoniczności funkcji oraz zbiór jej wartości.




PRACA KONTROLNA $\mathrm{n}\mathrm{r} 6-$ POZIOM ROZSZERZONY

l. Developer chce pomalowač $\mathrm{k}\mathrm{a}\dot{\mathrm{z}}$ de $\mathrm{z} 11$ pięter nowo wybudowanego wiezowca na jeden

$\mathrm{z}3$ kolorów występujących $\mathrm{w}$ jego logo, przy czym $\mathrm{k}\mathrm{a}\dot{\mathrm{z}}\mathrm{d}\mathrm{y}$ kolor ma zostač wykorzystany

co najmniej jeden $\mathrm{r}\mathrm{a}\mathrm{z}$. Obliczyč prawdopodobieństwo, $\dot{\mathrm{z}}\mathrm{e}$ dwaj niezalezni graficy, którym

zlecono zaprojektowanie kolorystyki budynku, przedstawią ten sam projekt. Przyjąč, $\dot{\mathrm{z}}\mathrm{e}$

wybór przez nich $\mathrm{k}\mathrm{a}\dot{\mathrm{z}}$ dego takiego ukladu kolorów jest jednakowo prawdopodobny.

2. Rozwiąz równanie

$8x^{3}=1+6x,$

stosując podstawienie $x=\cos\alpha.$

3. Określ dziedzinę $\mathrm{i}$ zbadaj, dla jakich argumentów funkcja

$f(x)=\displaystyle \log_{x^{2}-1}(x^{2}-2x)-\log_{x^{2}-1}(2-\frac{4}{x})$

przyjmuje wartości nieujemne.

4. Rozwia $\dot{\mathrm{Z}}$ nierównośč

$1+\mathrm{t}\mathrm{g}^{2}2x-\mathrm{t}\mathrm{g}^{4}2x+\mathrm{t}\mathrm{g}^{6}2x-\ldots\leq 3\sin 2x-\sin^{2}2x.$

5. Wśród prostopadfościanów $0$ podstawie kwadratu wpisanych $\mathrm{w}$ kulę $0$ promieniu $R$ wskaz

ten, którego objętośč jest największa.

6. Określ dziedzinę, wyznacz przedzialy monotoniczności oraz wszystkie lokalne ekstrema

funkcji

$f(x)=\displaystyle \frac{(x+1)^{2}}{x(x-2)}.$

$\mathrm{s}_{\mathrm{P}^{\mathrm{o}\mathrm{r}\mathrm{z}}\mathrm{a}}\mathrm{d}\acute{\mathrm{z}}$ jej staranny wykres.

Rozwiqzania (rękopis) zadań z wybranego poziomu prosimy nadsyłač do

na adres:

181utego 2020r.

WydziaX Matematyki

Politechnika Wrocfawska

Wybrzez $\mathrm{e}$ Wyspiańskiego 27

$50-370$ WROCLAW.

Na kopercie prosimy $\underline{\mathrm{k}\mathrm{o}\mathrm{n}\mathrm{i}\mathrm{e}\mathrm{c}z\mathrm{n}\mathrm{i}\mathrm{e}}$ zaznaczyč wybrany poziom! (np. poziom podsta-

wowy lub rozszerzony). Do rozwiązań nalez $\mathrm{y}$ dołączyč zaadresowaną do siebie koperte

zwrotną $\mathrm{z}$ naklejonym znaczkiem, odpowiednim do formatu listu. Polecamy stosowanie

kopert formatu C5 $(160\mathrm{x}230\mathrm{m}\mathrm{m})$ ze znaczkiem $0$ wartości 3,30 zł. Na $\mathrm{k}\mathrm{a}\dot{\mathrm{z}}$ dą większą

koperte nale $\dot{\mathrm{z}}\mathrm{y}$ nakleič drozszy znaczek. Prace niespelniające podanych warunków nie

będą poprawiane ani odsyłane.

Uwaga. Wysylając nam rozwiązania zadań uczestnik Kursu udostępnia Politechnice Wroclawskiej swoje da-

ne osobowe, które przetwarzamy wyłącznie $\mathrm{w}$ zakresie niezbędnym do jego prowadzenia (odeslanie zadań,

prowadzenie statystyki). Szczególowe informacje $0$ przetwarzaniu przez nas danych osobowych są dostepne na

stronie internetowej Kursu.

Adres internetowy Kursu: http://www. im.pwr.edu.pl/kurs



\end{document}