\documentclass[a4paper,12pt]{article}
\usepackage{latexsym}
\usepackage{amsmath}
\usepackage{amssymb}
\usepackage{graphicx}
\usepackage{wrapfig}
\pagestyle{plain}
\usepackage{fancybox}
\usepackage{bm}

\begin{document}

XLIX

KORESPONDENCYJNY KURS

Z MATEMATYKI

luty 2020 r.

PRACA KONTROLNA $\mathrm{n}\mathrm{r} 6-$ POZIOM PODSTAWOWY

1. $\mathrm{W}$ szufiadzie znajduje się 6 róznych par rękawiczek. Ob1icz prawdopodobieństwo, $\dot{\mathrm{z}}\mathrm{e}$

wśród 51osowo wybranych rękawic jest co najmniej jedna para.

2. Wyznacz dziedzinę $\mathrm{i}$ zbadaj, dla jakich argumentów funkcja

$f(x)=\log_{\sqrt{3}}(x+3)-\log_{3}(9-x^{2})$

przyjmuje wartości ujemne.

3. Wśród prostok$\Phi$tów wpisanych $\mathrm{w}\mathrm{o}\mathrm{k}\mathrm{r}\Phi \mathrm{g}\mathrm{o}$ promieniu $R$ bez $\mathrm{u}\dot{\mathrm{z}}$ ycia metod rachunku róz-

niczkowego wskaz ten, którego pole jest największe.

4. Rozwiąz nierównośč

$4^{x^{3}-x+2}\cdot 5^{2x-3x^{2}}-2^{4-3x^{2}}\cdot 25^{x^{3}}\geq 0.$

5. Powierzchnia boczna stozka po rozcięciu jest wycinkiem kofa $\mathrm{o}\mathrm{k}_{\Phi}\mathrm{c}\mathrm{i}\mathrm{e}216^{\mathrm{o}}$

stawy stozka wynosi $ 6\pi$. Oblicz objętośč kuli wpisanej $\mathrm{w}$ ten stozek.

Obwód pod-

6. Narysuj wykres funkcji

$f(x)=-1+2^{1-|1-|x||}$

i precyzyjnie opisz zastosowaną metodę jego konstrukcji. Na podstawie rysunku wskaz

przedziafy monotoniczności funkcji oraz zbiór jej wartości.
\end{document}
