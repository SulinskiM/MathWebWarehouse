\documentclass[a4paper,12pt]{article}
\usepackage{latexsym}
\usepackage{amsmath}
\usepackage{amssymb}
\usepackage{graphicx}
\usepackage{wrapfig}
\pagestyle{plain}
\usepackage{fancybox}
\usepackage{bm}

\begin{document}

L KORESPONDENCYJNY KURS

Z MATEMATYKI

luty 2021 r.

PRACA KONTROLNA nr 6- POZIOM PODSTAWOWY

l. Suma wszystkich krawędzi prostopadłościanu $0$ podstawie kwadratowej wynosi 16 cm.

Jakie są wymiary tego prostopadfościanu, który ma najwieksze pole powierzchni cafko-

witej?

2. Sporząd $\acute{\mathrm{z}}$ wykres funkcji

$f(x)=|x^{2}-4|-2x$

oraz wyznacz liczbę pierwiastków równania

$f(x)=m$

$\mathrm{w}$ zalezności od parametru $m.$

3. Ze zbioru trzech elementów $\{a,b,c\}$ pobrano ze zwracaniem próbkę $0$ liczności 9 e1e-

mentów. Oblicz prawdopodobieństwo zdarzenia, $\dot{\mathrm{z}}\mathrm{e}\mathrm{w}$ tej próbie $\mathrm{k}\mathrm{a}\dot{\mathrm{z}}\mathrm{d}\mathrm{y}$ element wystąpi

dokładnie trzy razy.

4. Sześciu przyjaciól $A, B, C, D, E, F$ zajmuje sześč kolejnych miejsc $\mathrm{w}$ jednym rzędzie sali

kinowej. Na ile sposobów mogą usiąśč, aby: a) osoby $A, B, C$ siedziałyjedna obok drugiej

($\mathrm{w}$ dowolnej kolejności)? b) $\dot{\mathrm{z}}$ adne dwie $\mathrm{z}$ osób $A, B, C$ nie siedziały obok siebie?

5. Wyznacz wspólrzędne wierzcholków trójk$\Phi$ta $ABC$, którego boki zawieraja się $\mathrm{w}$ pro-

stych: $y=2, 2x-y+10=0, 4x+3y=0$. Następnie wyznacz współrzędne wierzchołków

trójk$\Phi$ta, który jest obrazem trójkąta $ABC$ wjednokfadności $0$ środku $O(0,0)\mathrm{i}$ skali $-2.$

Oblicz pole trójkąta $ABC\mathrm{i}$ jego obrazu $\mathrm{w}$ tym przeksztafceniu.

6. Trójkąt równoboczny $ABC 0$ boku l dzielimy na cztery przystajqce trójkąty, lqczqc

środki jego boków. Usuwamy środkowy trójkąt (krok l). To samo robimy $\mathrm{z} \mathrm{k}\mathrm{a}\dot{\mathrm{z}}$ dym

$\mathrm{z}$ trzech pozostałych trójkątów (krok 2). Proces ten wykonujemy $n$ razy. Jaka jest suma

pól usuniętych trójkątów po trzech krokach? Ile kroków wystarczy wykonač, aby suma

pól usuniętych trójkątów była większa $\mathrm{n}\mathrm{i}\dot{\mathrm{z}}3/4$ pola wyjściowego trójkąta?




PRACA KONTROLNA $\mathrm{n}\mathrm{r} 6-$ POZIOM ROZSZERZONY

l. Ile jest czterocyfrowych kodów PIN, $\mathrm{w}$ których: a) $\dot{\mathrm{z}}$ adna cyfra się nie powtarza? b)

któraś $\mathrm{z}$ cyfr się powtarza? Ile kodów jest więcej: tych, $\mathrm{w}$ których $\dot{\mathrm{z}}$ adna cyfra się nie

powtarza, czy tych, $\mathrm{w}$ których któraś $\mathrm{z}$ cyfr się powtarza?

2. Pięciu wioślarzy $A, B, C, D, E$ plynie lodzią, na której znajduje się pięč poprzecznych

lawek dwuosobowych. Wioslarze $A, B, C$ mogą usiąśč tylko przy prawej burcie, natomiast

wioślarze $D\mathrm{i}$ E- tylko przy lewej. Jakie jest prawdopodobieństwo zdarzenia, $\dot{\mathrm{z}}\mathrm{e}$ miejsca

obok wioślarzy $D\mathrm{i}E$ będą zajęte?

3. Znajd $\acute{\mathrm{z}}$ współrzędne wierzchołka $C$ trójkąta równoramiennego $ABC$, gdzie $A(2,0), B(0,2),$

wiedząc, $\dot{\mathrm{z}}\mathrm{e}$ środkowe $AD\mathrm{i}$ {\it BE} $\mathrm{p}\mathrm{r}\mathrm{z}\mathrm{e}\mathrm{c}\mathrm{i}\mathrm{n}\mathrm{a}\mathrm{j}_{\Phi}$ się pod $\mathrm{k}_{\Phi}\mathrm{t}\mathrm{e}\mathrm{m}$ prostym.

4. $\mathrm{W}$ prostokatnym ukladzie wspólrzędnych dane sq punkty $A(\alpha,0)\mathrm{i}B(b,0)$, gdzie

$0 < a < b.$ Znajd $\acute{\mathrm{z}}$ punkt $C(0,c)$, gdzie $c > 0$, dla którego miara kąta $\angle ACB$ jest

największa.

5. Wyznacz wszystkie styczne do wykresu funkcji $f(x)=\displaystyle \frac{x-1}{x+1}$ równolegle do prostej

$x-2y=0\mathrm{i}$ oblicz pole wielokąta, którego wierzchołkami są punkty przecięcia otrzyma-

nych prostych $\mathrm{z}$ osiami układu. Wykonaj staranny rysunek.

6. Kwadrat ABCD $0$ boku $a$ dzielimy na dziewięč przystających kwadratów, dzieląc $\mathrm{k}\mathrm{a}\dot{\mathrm{z}}\mathrm{d}\mathrm{y}$

$\mathrm{z}$ boków kwadratu na trzy równe części $\mathrm{i}$ usuwamy środkowy kwadrat (krok l). Nastepnie

to samo robimy $\mathrm{w}$ pozostafych ośmiu kwadratach (krok 2). Proces ten powtarzany jest

nieskończenie wiele razy. Jaka jest suma pól kwadratów usuniętych $\mathrm{w}n$ krokach? Ile

kroków wystarczy wykonač, aby suma pól usuniętych kwadratów byfa większa $\mathrm{n}\mathrm{i}\dot{\mathrm{z}}$ połowa

pola wyjściowego kwadratu? Jaka jest suma pól wszystkich usuniętych kwadratów (po

nieskończenie wielu krokach)?

Rozwiązania (rękopis) zadań z wybranego poziomu prosimy nadsyfač do 20.02.2021r. na

adres:

Wydziaf Matematyki

Politechnika Wrocfawska

Wybrzez $\mathrm{e}$ Wyspiańskiego 27

$50-370$ WROCLAW.

Na kopercie prosimy $\underline{\mathrm{k}\mathrm{o}\mathrm{n}\mathrm{i}\mathrm{e}\mathrm{c}\mathrm{z}\mathrm{n}\mathrm{i}\mathrm{e}}$ zaznaczyč wybrany poziom! (np. poziom podsta-

wowy lub rozszerzony). Do rozwiązań nalez $\mathrm{y}$ dołączyč zaadresowaną do siebie kopertę

zwrotną $\mathrm{z}$ naklejonym znaczkiem, odpowiednim do formatu listu. Polecamy stosowanie

kopert formatu C5 $(160\mathrm{x}230\mathrm{m}\mathrm{m})$ ze znaczkiem $0$ wartości 3,30 zł. Na $\mathrm{k}\mathrm{a}\dot{\mathrm{z}}$ dą większą

kopertę nalez $\mathrm{y}$ nakleič $\mathrm{d}\mathrm{r}\mathrm{o}\dot{\mathrm{z}}$ szy znaczek. Prace niespełniające podanych warunków nie

bedą poprawiane ani odsyłane.

Uwaga. Wysyfaj\S c nam rozwi\S zania zadań uczestnik Kursu udostępnia Politechnice Wrocfawskiej

swoje dane osobowe, które przetwarzamy wyłącznie $\mathrm{w}$ zakresie niezbędnym do jego prowadzenia

(odeslanie zadań, prowadzenie statystyki). Szczególowe informacje $0$ przetwarzaniu przez nas danych

osobowych są dostępne na stronie internetowej Kursu.

Adres internetowy Kursu: http: //www. im. pwr. edu. pl/kurs



\end{document}