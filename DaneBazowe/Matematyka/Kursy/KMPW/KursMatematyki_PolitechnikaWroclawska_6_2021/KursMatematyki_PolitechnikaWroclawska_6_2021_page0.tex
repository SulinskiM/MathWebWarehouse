\documentclass[a4paper,12pt]{article}
\usepackage{latexsym}
\usepackage{amsmath}
\usepackage{amssymb}
\usepackage{graphicx}
\usepackage{wrapfig}
\pagestyle{plain}
\usepackage{fancybox}
\usepackage{bm}

\begin{document}

L KORESPONDENCYJNY KURS

Z MATEMATYKI

luty 2021 r.

PRACA KONTROLNA nr 6- POZIOM PODSTAWOWY

l. Suma wszystkich krawędzi prostopadłościanu $0$ podstawie kwadratowej wynosi 16 cm.

Jakie są wymiary tego prostopadfościanu, który ma najwieksze pole powierzchni cafko-

witej?

2. Sporząd $\acute{\mathrm{z}}$ wykres funkcji

$f(x)=|x^{2}-4|-2x$

oraz wyznacz liczbę pierwiastków równania

$f(x)=m$

$\mathrm{w}$ zalezności od parametru $m.$

3. Ze zbioru trzech elementów $\{a,b,c\}$ pobrano ze zwracaniem próbkę $0$ liczności 9 e1e-

mentów. Oblicz prawdopodobieństwo zdarzenia, $\dot{\mathrm{z}}\mathrm{e}\mathrm{w}$ tej próbie $\mathrm{k}\mathrm{a}\dot{\mathrm{z}}\mathrm{d}\mathrm{y}$ element wystąpi

dokładnie trzy razy.

4. Sześciu przyjaciól $A, B, C, D, E, F$ zajmuje sześč kolejnych miejsc $\mathrm{w}$ jednym rzędzie sali

kinowej. Na ile sposobów mogą usiąśč, aby: a) osoby $A, B, C$ siedziałyjedna obok drugiej

($\mathrm{w}$ dowolnej kolejności)? b) $\dot{\mathrm{z}}$ adne dwie $\mathrm{z}$ osób $A, B, C$ nie siedziały obok siebie?

5. Wyznacz wspólrzędne wierzcholków trójk$\Phi$ta $ABC$, którego boki zawieraja się $\mathrm{w}$ pro-

stych: $y=2, 2x-y+10=0, 4x+3y=0$. Następnie wyznacz współrzędne wierzchołków

trójk$\Phi$ta, który jest obrazem trójkąta $ABC$ wjednokfadności $0$ środku $O(0,0)\mathrm{i}$ skali $-2.$

Oblicz pole trójkąta $ABC\mathrm{i}$ jego obrazu $\mathrm{w}$ tym przeksztafceniu.

6. Trójkąt równoboczny $ABC 0$ boku l dzielimy na cztery przystajqce trójkąty, lqczqc

środki jego boków. Usuwamy środkowy trójkąt (krok l). To samo robimy $\mathrm{z} \mathrm{k}\mathrm{a}\dot{\mathrm{z}}$ dym

$\mathrm{z}$ trzech pozostałych trójkątów (krok 2). Proces ten wykonujemy $n$ razy. Jaka jest suma

pól usuniętych trójkątów po trzech krokach? Ile kroków wystarczy wykonač, aby suma

pól usuniętych trójkątów była większa $\mathrm{n}\mathrm{i}\dot{\mathrm{z}}3/4$ pola wyjściowego trójkąta?
\end{document}
