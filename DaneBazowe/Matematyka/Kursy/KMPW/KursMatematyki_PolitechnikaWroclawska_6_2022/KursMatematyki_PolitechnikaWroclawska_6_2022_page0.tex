\documentclass[a4paper,12pt]{article}
\usepackage{latexsym}
\usepackage{amsmath}
\usepackage{amssymb}
\usepackage{graphicx}
\usepackage{wrapfig}
\pagestyle{plain}
\usepackage{fancybox}
\usepackage{bm}

\begin{document}

LI KORESPONDENCYJNY KURS

Z MATEMATYKI

luty 2022 r.

PRACA KONTROLNA nr 6- POZIOM PODSTAWOWY

l. Prawdopodobieństwo, $\dot{\mathrm{z}}\mathrm{e}\mathrm{w}$ dowolnie wybranym przedziale pociągu relacji Warszawa-

Wrocfaw podrózny znajdzie co najmniej jedno wolne miejsce wynosi $\displaystyle \frac{1}{2}$. Podrózny szuka

pierwszego wolnego miejsca, zaglądając do $\mathrm{k}\mathrm{a}\dot{\mathrm{z}}$ dego kolejnego przedziafu. Oblicz praw-

dopodobieństwo zdarzenia, $\dot{\mathrm{z}}\mathrm{e}$ liczba odwiedzonych przez niego przedziałów nie przekro-

czy 4.

2. Rozwiąz nierównośč wykładniczą

$2^{x^{3}}\cdot 9^{2x-1}<3^{x^{3}-2}\cdot 4^{2x}$

3. $\mathrm{W}$ trójkącie równoramiennym $\triangle ABC0$ ramionach $AC\mathrm{i}BC\mathrm{k}_{\Phi}\mathrm{t}$ przy podstawie $AB$ ma

miarę $\alpha$. Na boku $AC$ umieszczono punkt $D\mathrm{w}$ taki sposób, $\dot{\mathrm{z}}\mathrm{e}$ trójkąty $\triangle ABC\mathrm{i}\triangle ABD$

są podobne. Wyznacz skalę podobieństwa tych trójkątów oraz warunki rozwiązalności

zadania. Oblicz stosunek pól tych trójkątów oraz stosunek objętości stozków powstałych

przez obrót tych trójkątów wokół ich osi symetrii.

4. Wyznacz wszystkie $\mathrm{m}\mathrm{o}\dot{\mathrm{z}}$ liwe wartości kąta ostrego $\alpha \mathrm{j}\mathrm{e}\dot{\mathrm{z}}$ eli wiadomo, $\dot{\mathrm{z}}\mathrm{e}$

tg $2\alpha+$ ctg $2\displaystyle \alpha=-\frac{4\sqrt{3}}{3}.$

5. Niech $x\in[0,2\pi]$. Rozwiąz nierównośč

$\sin^{5}x+\cos^{5}x\geq\sin^{4}x\cdot\cos x+\cos^{4}x\cdot\sin x.$

6. Wyznacz wszystkie argumenty $x$, dla których funkcja

$f(x)=\log_{2}(x+2)-2\log_{4}\sqrt{x^{3}+8}$

przyjmuje wartości niedodatnie.
\end{document}
