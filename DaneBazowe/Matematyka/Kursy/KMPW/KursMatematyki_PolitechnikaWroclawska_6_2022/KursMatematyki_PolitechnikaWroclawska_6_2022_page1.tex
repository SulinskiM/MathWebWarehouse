\documentclass[a4paper,12pt]{article}
\usepackage{latexsym}
\usepackage{amsmath}
\usepackage{amssymb}
\usepackage{graphicx}
\usepackage{wrapfig}
\pagestyle{plain}
\usepackage{fancybox}
\usepackage{bm}

\begin{document}

PRACA KONTROLNA $\mathrm{n}\mathrm{r} 6-$ POZIOM ROZSZERZONY

l. Rzucamy cztery razy $\mathrm{j}\mathrm{e}\mathrm{d}\mathrm{d}\mathrm{n}\mathrm{k}\mathrm{o}\mathrm{s}\mathrm{t}\mathrm{k}_{\Phi}$ do gry. Oblicz prawdopodobieństwo, $\dot{\mathrm{z}}\mathrm{e}$ suma

wyrzuconych oczek przekroczy 12, jeś1i wiadomo, $\dot{\mathrm{z}}\mathrm{e}$ suma oczek wyrzuconych $\mathrm{w}$ dwóch

pierwszych rzutach wynosi 8.

2. Rozwiąz równanie trygonometryczne

-csions22{\it xx}.. ssiinn{\it xx} --csoins 22{\it xx}.. ccooss {\it xx} $=$1.

3. Rozwiąz równanie

$5^{\mathrm{t}\mathrm{g}^{2}x-1}+5^{3-\mathrm{t}\mathrm{g}^{2}x}=26.$

4. Rozwiąz nierównośč logarytmiczną

$1+\log_{x-1}x<\log_{x-1}(x+6).$

5. Wyznacz dziedzinę $\mathrm{i}$ miejsca zerowe funkcji

$f(x)=\log_{\sin(-x)}(4\sin x\cdot\cos x-1).$

6. $\mathrm{W}$ trójk$\Phi$cie równoramiennym $\triangle ABC$, którego podstawa $AB$ ma dlugośč 4, miara $\mathrm{k}_{\Phi^{\mathrm{t}\mathrm{a}}}$

pomiędzy ramionami $AC\mathrm{i}BC$ wynosi $30^{\mathrm{o}}$ Oblicz objętośč bryly powstałej przez obrót

tego trójkąta wzgledem jednego $\mathrm{z}$ jego ramion.

Rozwiqzania (rękopis) zadań z wybranego poziomu prosimy nadsylač do

2022r. na adres:

20 1utego

Wydziaf Matematyki

Politechnika Wrocfawska

Wybrzez $\mathrm{e}$ Wyspiańskiego 27

$50-370$ WROCLAW,

lub elektronicznie, za pośrednictwem portalu talent. $\mathrm{p}\mathrm{w}\mathrm{r}$. edu. pl

Na kopercie prosimy $\underline{\mathrm{k}\mathrm{o}\mathrm{n}\mathrm{i}\mathrm{e}\mathrm{c}\mathrm{z}\mathrm{n}\mathrm{i}\mathrm{e}}$ zaznaczyč wybrany poziom! (np. poziom podsta-

wowy lub rozszerzony). Do rozwiązań nalez $\mathrm{y}$ dołączyč zaadresowaną do siebie kopertę

zwrotną $\mathrm{z}$ naklejonym znaczkiem, odpowiednim do formatu listu. Polecamy stosowanie

kopert formatu C5 $(160\mathrm{x}230\mathrm{m}\mathrm{m})$ ze znaczkiem $0$ wartości 3,30 zł. Na $\mathrm{k}\mathrm{a}\dot{\mathrm{z}}$ dą większą

koperte nalez $\mathrm{y}$ nakleić drozszy znaczek. Prace niespelniające podanych warunków nie

będą poprawiane ani odsyłane.

Uwaga. Wysyłając nam rozwi\S zania zadań uczestnik Kursu udostępnia Politechnice Wroclawskiej

swoje dane osobowe, które przetwarzamy wyłącznie $\mathrm{w}$ zakresie niezbednym do jego prowadzenia

(odesfanie zadań, prowadzenie statystyki). Szczegófowe informacje $0$ przetwarzaniu przez nas danych

osobowych sa dostępne na stronie internetowej Kursu.

Adres internetowy Kursu: http: //www. im. pwr. edu. pl/kurs
\end{document}
