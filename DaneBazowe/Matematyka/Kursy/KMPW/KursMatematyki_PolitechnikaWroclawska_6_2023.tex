\documentclass[a4paper,12pt]{article}
\usepackage{latexsym}
\usepackage{amsmath}
\usepackage{amssymb}
\usepackage{graphicx}
\usepackage{wrapfig}
\pagestyle{plain}
\usepackage{fancybox}
\usepackage{bm}

\begin{document}

LII

KORESPONDENCYJNY KURS

Z MATEMATYKI

luty 2023 r.

PRACA KONTROLNA $\mathrm{n}\mathrm{r} 6-$ POZIOM PODSTAWOWY

l. Na ile róznych sposobów $\mathrm{m}\mathrm{o}\dot{\mathrm{z}}\mathrm{e}$ się ustawić do zdjęcia sześcioosobowa rodzina, $\mathrm{j}\mathrm{e}\dot{\mathrm{z}}$ eli wszy-

scy mają stać $\mathrm{w}$ jednym rzędzie, a najmłodsza córka musi stać obok mamy?

2. $\mathrm{J}\mathrm{e}\dot{\mathrm{z}}$ eli $\mathrm{w}$ dwóch rzutach sześcienną kostką do gry gracz otrzyma sumę oczek wynoszącą

przynajmniej 10, to wygrywa 100 $\mathrm{z}l.$, a $\mathrm{j}\mathrm{e}\dot{\mathrm{z}}$ eli otrzyma mniej $\mathrm{n}\mathrm{i}\dot{\mathrm{z}} 10\mathrm{i}$ więcej $\mathrm{n}\mathrm{i}\dot{\mathrm{z}}6$, to

wygrywa 50 $\mathrm{z}l. \mathrm{W}$ pozostalych przypadkach przegrywa $\mathrm{i}$ musi zapfacić $80\mathrm{z}1$. Wyznacz

wartość oczekiwaną wygranej gracza $\mathrm{w}$ tej grze. Jak organizator takiej gry powinien

zmienić oplatę za przegraną $\dot{\mathrm{z}}$ eby mógł liczyć na zarobek po wzięciu $\mathrm{w}$ niej udziału przez

wielu graczy?

3. Uzasadnij, $\dot{\mathrm{z}}\mathrm{e}$ dla $\mathrm{k}\mathrm{a}\dot{\mathrm{z}}$ dego $n$ naturalnego liczba $2n^{3}+3n^{2}+n$ jest podzielna przez 6.

4. Oblicz piąty wyraz ciągu arytmetycznego

$\log_{2}x_{1},\log_{2}x_{2},\log_{2}x_{3},$

wiedząc, $\displaystyle \dot{\mathrm{z}}\mathrm{e}x_{1}+x_{2}+x_{3}=\frac{7}{4}$ oraz $x_{2}=\displaystyle \frac{1}{2}.$

5. Oblicz prawdopodobieństwo, $\dot{\mathrm{z}}\mathrm{e}\mathrm{w} 8$ rzutach monetą pojawi się seria przynajmniej 5

reszek lub 5 orłów pod rząd.

6. Losujemy jedną liczbę spošród liczb l, 2, 2023. Znajd $\acute{\mathrm{z}}$ prawdopodobieństwo, $\dot{\mathrm{z}}\mathrm{e} \mathrm{a}$)

wybrana liczba będzie podzielna przez 5 $\mathrm{i}$ przez ll, b) wybrana liczba będzie podzielna

przez 51ub przez 11.




PRACA KONTROLNA $\mathrm{n}\mathrm{r} 6-$ POZIOM ROZSZERZONY

l. Jakiejest prawdopodobieństwo, $\dot{\mathrm{z}}\mathrm{e}\mathrm{w}$ sześciu rzutach standardową kostką do gry wypadną

wszystkie $\mathrm{m}\mathrm{o}\dot{\mathrm{z}}$ liwe liczby oczek?

2. Dla jakich wartości parametru $p$ równanie

$x^{2}-(2^{p}-1)x-3(4^{p-1}-2^{p-2})=0$

ma dwa pierwiastki rzeczywiste róznych znaków?

3. $\mathrm{Z}$ pierwszej urny zawierajqcej $n$ kul bialych $\mathrm{i}$ cztery czarne losujemy dwie kule $\mathrm{i}$ wrzucamy

je do drugiej urny, początkowo pustej. $\mathrm{Z}$ tej drugiej losujemy wtedy jedną kulę.

a) Dlajakich wartości $n$ prawdopodobieństwo wyciągnięcia bialej kuli $\mathrm{z}$ drugiej urny jest

większe od 3/4?

b) Przyjmując $n=6$ oblicz prawdopodobieństwo, $\dot{\mathrm{z}}\mathrm{e}\mathrm{z}$ pierwszej urny wylosowano dwie

białe kule, ješli wiadomo, $\dot{\mathrm{z}}\mathrm{e}\mathrm{z}$ drugiej urny wylosowano białą kulę.

4. $\mathrm{W}$ urnie jest 15 ku1 ponumerowanych 1iczbami od 1 do 15. Wyciągamy $\mathrm{z}$ niej kolejno pięć

kul bez zwracania. Obliczyć prawdopodobieństwo, $\dot{\mathrm{z}}\mathrm{e}$ numer na drugiej kuli jest liczbą

podzielną przez trzy ijednocześnie numer na piątej kuli jest liczbą podzielną przez pięć.

5. Znajdz' dziedzinę oraz wartości największą $\mathrm{i}$ najmniejszą (ješli istnieją) funkcji

$f(x)=\displaystyle \frac{2-x^{2}}{x^{2}}+(2-x^{2})+(2x^{2}-x^{4})+$

która jest sumą szeregu geometrycznego.

6. $\mathrm{W}$ urniejest 99 ku1 białych ijedna czarna. Agnieszka $\mathrm{i}$ Jacek losują $\mathrm{z}$ tej urny na przemian

po jednej kuli bez zwracania. Wygrywa ten, kto wylosuje czarną kulę. Pierwszą kulę

wyciqga Agnieszka. Jakie jest prawdopodobieństwo, $\dot{\mathrm{z}}\mathrm{e}$ to ona wygra?

Rozwiązania (rękopis) zadań z wybranego poziomu prosimy nadsylać do 20.02.2023r.

adres:

na

Wydzial Matematyki

Politechnika Wroclawska

Wybrzez $\mathrm{e}$ Wyspiańskiego 27

$50-370$ WROCLAW,

$\mathrm{l}\mathrm{u}\mathrm{b}$ elektronicznie, za pośrednictwem portalu talent. $\mathrm{p}\mathrm{w}\mathrm{r}$. edu. pl

Na kopercie prosimy $\underline{\mathrm{k}\mathrm{o}\mathrm{n}\mathrm{i}\mathrm{e}\mathrm{c}\mathrm{z}\mathrm{n}\mathrm{i}\mathrm{e}}$ zaznaczyć wybrany poziom! (rip. poziom podsta-

wowy lub rozszerzony). Do rozwiązań nalez $\mathrm{y}$ dołączyć zaadresowaną do siebie kopertę

zwrotną $\mathrm{z}$ naklejonym znaczkiem, odpowiednim do formatu listu. Prace niespełniające

podanych warunków nie będą poprawiane ani odsyłane.

Uwaga. Wysyłając nam rozwiązania zadań uczestnik Kursu udostępnia Politechnice Wrocławskiej

swoje dane osobowe, które przetwarzamy wyłącznie $\mathrm{w}$ zakresie niezbędnym do jego prowadzenia

(odeslanie zadań, prowadzenie statystyki). Szczególowe informacje $0$ przetwarzaniu przez nas danych

osobowych są dostępne na stronie internetowej Kursu.

Adres internetowy Kursu: http: //www. im. pwr. edu. pl/kurs



\end{document}