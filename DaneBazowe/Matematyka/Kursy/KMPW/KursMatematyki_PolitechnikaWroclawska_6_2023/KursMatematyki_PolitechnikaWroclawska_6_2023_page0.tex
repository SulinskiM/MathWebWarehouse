\documentclass[a4paper,12pt]{article}
\usepackage{latexsym}
\usepackage{amsmath}
\usepackage{amssymb}
\usepackage{graphicx}
\usepackage{wrapfig}
\pagestyle{plain}
\usepackage{fancybox}
\usepackage{bm}

\begin{document}

LII

KORESPONDENCYJNY KURS

Z MATEMATYKI

luty 2023 r.

PRACA KONTROLNA $\mathrm{n}\mathrm{r} 6-$ POZIOM PODSTAWOWY

l. Na ile róznych sposobów $\mathrm{m}\mathrm{o}\dot{\mathrm{z}}\mathrm{e}$ się ustawić do zdjęcia sześcioosobowa rodzina, $\mathrm{j}\mathrm{e}\dot{\mathrm{z}}$ eli wszy-

scy mają stać $\mathrm{w}$ jednym rzędzie, a najmłodsza córka musi stać obok mamy?

2. $\mathrm{J}\mathrm{e}\dot{\mathrm{z}}$ eli $\mathrm{w}$ dwóch rzutach sześcienną kostką do gry gracz otrzyma sumę oczek wynoszącą

przynajmniej 10, to wygrywa 100 $\mathrm{z}l.$, a $\mathrm{j}\mathrm{e}\dot{\mathrm{z}}$ eli otrzyma mniej $\mathrm{n}\mathrm{i}\dot{\mathrm{z}} 10\mathrm{i}$ więcej $\mathrm{n}\mathrm{i}\dot{\mathrm{z}}6$, to

wygrywa 50 $\mathrm{z}l. \mathrm{W}$ pozostalych przypadkach przegrywa $\mathrm{i}$ musi zapfacić $80\mathrm{z}1$. Wyznacz

wartość oczekiwaną wygranej gracza $\mathrm{w}$ tej grze. Jak organizator takiej gry powinien

zmienić oplatę za przegraną $\dot{\mathrm{z}}$ eby mógł liczyć na zarobek po wzięciu $\mathrm{w}$ niej udziału przez

wielu graczy?

3. Uzasadnij, $\dot{\mathrm{z}}\mathrm{e}$ dla $\mathrm{k}\mathrm{a}\dot{\mathrm{z}}$ dego $n$ naturalnego liczba $2n^{3}+3n^{2}+n$ jest podzielna przez 6.

4. Oblicz piąty wyraz ciągu arytmetycznego

$\log_{2}x_{1},\log_{2}x_{2},\log_{2}x_{3},$

wiedząc, $\displaystyle \dot{\mathrm{z}}\mathrm{e}x_{1}+x_{2}+x_{3}=\frac{7}{4}$ oraz $x_{2}=\displaystyle \frac{1}{2}.$

5. Oblicz prawdopodobieństwo, $\dot{\mathrm{z}}\mathrm{e}\mathrm{w} 8$ rzutach monetą pojawi się seria przynajmniej 5

reszek lub 5 orłów pod rząd.

6. Losujemy jedną liczbę spošród liczb l, 2, 2023. Znajd $\acute{\mathrm{z}}$ prawdopodobieństwo, $\dot{\mathrm{z}}\mathrm{e} \mathrm{a}$)

wybrana liczba będzie podzielna przez 5 $\mathrm{i}$ przez ll, b) wybrana liczba będzie podzielna

przez 51ub przez 11.
\end{document}
