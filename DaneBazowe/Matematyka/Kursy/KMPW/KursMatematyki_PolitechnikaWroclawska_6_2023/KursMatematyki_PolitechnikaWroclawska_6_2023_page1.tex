\documentclass[a4paper,12pt]{article}
\usepackage{latexsym}
\usepackage{amsmath}
\usepackage{amssymb}
\usepackage{graphicx}
\usepackage{wrapfig}
\pagestyle{plain}
\usepackage{fancybox}
\usepackage{bm}

\begin{document}

PRACA KONTROLNA $\mathrm{n}\mathrm{r} 6-$ POZIOM ROZSZERZONY

l. Jakiejest prawdopodobieństwo, $\dot{\mathrm{z}}\mathrm{e}\mathrm{w}$ sześciu rzutach standardową kostką do gry wypadną

wszystkie $\mathrm{m}\mathrm{o}\dot{\mathrm{z}}$ liwe liczby oczek?

2. Dla jakich wartości parametru $p$ równanie

$x^{2}-(2^{p}-1)x-3(4^{p-1}-2^{p-2})=0$

ma dwa pierwiastki rzeczywiste róznych znaków?

3. $\mathrm{Z}$ pierwszej urny zawierajqcej $n$ kul bialych $\mathrm{i}$ cztery czarne losujemy dwie kule $\mathrm{i}$ wrzucamy

je do drugiej urny, początkowo pustej. $\mathrm{Z}$ tej drugiej losujemy wtedy jedną kulę.

a) Dlajakich wartości $n$ prawdopodobieństwo wyciągnięcia bialej kuli $\mathrm{z}$ drugiej urny jest

większe od 3/4?

b) Przyjmując $n=6$ oblicz prawdopodobieństwo, $\dot{\mathrm{z}}\mathrm{e}\mathrm{z}$ pierwszej urny wylosowano dwie

białe kule, ješli wiadomo, $\dot{\mathrm{z}}\mathrm{e}\mathrm{z}$ drugiej urny wylosowano białą kulę.

4. $\mathrm{W}$ urnie jest 15 ku1 ponumerowanych 1iczbami od 1 do 15. Wyciągamy $\mathrm{z}$ niej kolejno pięć

kul bez zwracania. Obliczyć prawdopodobieństwo, $\dot{\mathrm{z}}\mathrm{e}$ numer na drugiej kuli jest liczbą

podzielną przez trzy ijednocześnie numer na piątej kuli jest liczbą podzielną przez pięć.

5. Znajdz' dziedzinę oraz wartości największą $\mathrm{i}$ najmniejszą (ješli istnieją) funkcji

$f(x)=\displaystyle \frac{2-x^{2}}{x^{2}}+(2-x^{2})+(2x^{2}-x^{4})+$

która jest sumą szeregu geometrycznego.

6. $\mathrm{W}$ urniejest 99 ku1 białych ijedna czarna. Agnieszka $\mathrm{i}$ Jacek losują $\mathrm{z}$ tej urny na przemian

po jednej kuli bez zwracania. Wygrywa ten, kto wylosuje czarną kulę. Pierwszą kulę

wyciqga Agnieszka. Jakie jest prawdopodobieństwo, $\dot{\mathrm{z}}\mathrm{e}$ to ona wygra?

Rozwiązania (rękopis) zadań z wybranego poziomu prosimy nadsylać do 20.02.2023r.

adres:

na

Wydzial Matematyki

Politechnika Wroclawska

Wybrzez $\mathrm{e}$ Wyspiańskiego 27

$50-370$ WROCLAW,

$\mathrm{l}\mathrm{u}\mathrm{b}$ elektronicznie, za pośrednictwem portalu talent. $\mathrm{p}\mathrm{w}\mathrm{r}$. edu. pl

Na kopercie prosimy $\underline{\mathrm{k}\mathrm{o}\mathrm{n}\mathrm{i}\mathrm{e}\mathrm{c}\mathrm{z}\mathrm{n}\mathrm{i}\mathrm{e}}$ zaznaczyć wybrany poziom! (rip. poziom podsta-

wowy lub rozszerzony). Do rozwiązań nalez $\mathrm{y}$ dołączyć zaadresowaną do siebie kopertę

zwrotną $\mathrm{z}$ naklejonym znaczkiem, odpowiednim do formatu listu. Prace niespełniające

podanych warunków nie będą poprawiane ani odsyłane.

Uwaga. Wysyłając nam rozwiązania zadań uczestnik Kursu udostępnia Politechnice Wrocławskiej

swoje dane osobowe, które przetwarzamy wyłącznie $\mathrm{w}$ zakresie niezbędnym do jego prowadzenia

(odeslanie zadań, prowadzenie statystyki). Szczególowe informacje $0$ przetwarzaniu przez nas danych

osobowych są dostępne na stronie internetowej Kursu.

Adres internetowy Kursu: http: //www. im. pwr. edu. pl/kurs
\end{document}
