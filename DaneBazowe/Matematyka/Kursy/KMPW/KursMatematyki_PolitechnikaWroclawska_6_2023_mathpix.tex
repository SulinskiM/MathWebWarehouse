\documentclass[10pt]{article}
\usepackage[polish]{babel}
\usepackage[utf8]{inputenc}
\usepackage[T1]{fontenc}
\usepackage{amsmath}
\usepackage{amsfonts}
\usepackage{amssymb}
\usepackage[version=4]{mhchem}
\usepackage{stmaryrd}
\usepackage{hyperref}
\hypersetup{colorlinks=true, linkcolor=blue, filecolor=magenta, urlcolor=cyan,}
\urlstyle{same}

\title{PRACA KONTROLNA nr 6 - POZIOM PODSTAWOWY }

\author{}
\date{}


\begin{document}
\maketitle
\begin{enumerate}
  \item Na ile różnych sposobów może się ustawić do zdjęcia sześcioosobowa rodzina, jeżeli wszyscy mają stać w jednym rzędzie, a najmłodsza córka musi stać obok mamy?
  \item Jeżeli w dwóch rzutach sześcienną kostką do gry gracz otrzyma sumę oczek wynoszącą przynajmniej 10 , to wygrywa 100 z ., a jeżeli otrzyma mniej niż 10 i więcej niż 6 , to wygrywa $50 \mathrm{zł}$. W pozostałych przypadkach przegrywa i musi zapłacić 80zł. Wyznacz wartość oczekiwaną wygranej gracza w tej grze. Jak organizator takiej gry powinien zmienić opłatę za przegraną żeby mógł liczyć na zarobek po wzięciu w niej udziału przez wielu graczy?
  \item Uzasadnij, że dla każdego $n$ naturalnego liczba $2 n^{3}+3 n^{2}+n$ jest podzielna przez 6 .
  \item Oblicz piąty wyraz ciągu arytmetycznego
\end{enumerate}

$$
\log _{2} x_{1}, \log _{2} x_{2}, \log _{2} x_{3}, \ldots
$$

wiedzac, że $x_{1}+x_{2}+x_{3}=\frac{7}{4}$ oraz $x_{2}=\frac{1}{2}$.\\
5. Oblicz prawdopodobieństwo, że w 8 rzutach monetą pojawi się seria przynajmniej 5 reszek lub 5 orłów pod rząd.\\
6. Losujemy jedną liczbę spośród liczb 1,2, .., 2023. Znajdź prawdopodobieństwo, że a) wybrana liczba będzie podzielna przez 5 i przez 11, b) wybrana liczba będzie podzielna przez 5 lub przez 11.

\section*{PRACA KONTROLNA nr 6 - POZIOM ROZSZERZONY}
\begin{enumerate}
  \item Jakie jest prawdopodobieństwo, że w sześciu rzutach standardową kostką do gry wypadną wszystkie możliwe liczby oczek?
  \item Dla jakich wartości parametru $p$ równanie
\end{enumerate}

$$
x^{2}-\left(2^{p}-1\right) x-3\left(4^{p-1}-2^{p-2}\right)=0
$$

ma dwa pierwiastki rzeczywiste różnych znaków?\\
3. Z pierwszej urny zawierającej $n$ kul białych i cztery czarne losujemy dwie kule i wrzucamy je do drugiej urny, początkowo pustej. Z tej drugiej losujemy wtedy jedną kulę.\\
a) Dla jakich wartości $n$ prawdopodobieństwo wyciągnięcia białej kuli z drugiej urny jest większe od $3 / 4$ ?\\
b) Przyjmując $n=6$ oblicz prawdopodobieństwo, że z pierwszej urny wylosowano dwie białe kule, jeśli wiadomo, że z drugiej urny wylosowano białą kulę.\\
4. W urnie jest 15 kul ponumerowanych liczbami od 1 do 15 . Wyciągamy z niej kolejno pięć kul bez zwracania. Obliczyć prawdopodobieństwo, że numer na drugiej kuli jest liczbą podzielną przez trzy i jednocześnie numer na piątej kuli jest liczbą podzielną przez pięć.\\
5. Znajdź dziedzinę oraz wartości największą i najmniejszą (jeśli istnieją) funkcji

$$
f(x)=\frac{2-x^{2}}{x^{2}}+\left(2-x^{2}\right)+\left(2 x^{2}-x^{4}\right)+\ldots
$$

która jest sumą szeregu geometrycznego.\\
6. W urnie jest 99 kul białych i jedna czarna. Agnieszka i Jacek losują z tej urny na przemian po jednej kuli bez zwracania. Wygrywa ten, kto wylosuje czarną kulę. Pierwszą kulę wyciąga Agnieszka. Jakie jest prawdopodobieństwo, że to ona wygra?

Rozwiązania (rękopis) zadań z wybranego poziomu prosimy nadsyłać do 20.02.2023r. na adres:

Wydział Matematyki\\
Politechnika Wrocławska\\
Wybrzeże Wyspiańskiego 27\\
50-370 WROCEAW,\\
lub elektronicznie, za pośrednictwem portalu \href{http://talent.pwr.edu.pl}{talent.pwr.edu.pl}\\
Na kopercie prosimy koniecznie zaznaczyć wybrany poziom! (np. poziom podstawowy lub rozszerzony). Do rozwiązań należy dołączyć zaadresowaną do siebie kopertę zwrotną z naklejonym znaczkiem, odpowiednim do formatu listu. Prace niespełniające podanych warunków nie będą poprawiane ani odsyłane.

Uwaga. Wysyłając nam rozwiązania zadań uczestnik Kursu udostępnia Politechnice Wrocławskiej swoje dane osobowe, które przetwarzamy wyłącznie w zakresie niezbędnym do jego prowadzenia (odesłanie zadań, prowadzenie statystyki). Szczegółowe informacje o przetwarzaniu przez nas danych osobowych są dostępne na stronie internetowej Kursu.\\
Adres internetowy Kursu: \href{http://www.im.pwr.edu.pl/kurs}{http://www.im.pwr.edu.pl/kurs}


\end{document}