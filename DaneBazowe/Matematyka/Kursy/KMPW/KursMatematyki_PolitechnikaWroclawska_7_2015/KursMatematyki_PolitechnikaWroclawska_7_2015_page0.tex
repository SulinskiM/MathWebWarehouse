\documentclass[a4paper,12pt]{article}
\usepackage{latexsym}
\usepackage{amsmath}
\usepackage{amssymb}
\usepackage{graphicx}
\usepackage{wrapfig}
\pagestyle{plain}
\usepackage{fancybox}
\usepackage{bm}

\begin{document}

XLIV

KORESPONDENCYJNY KURS

Z MATEMATYKI

marzec 2015 r.

PRACA KONTROLNA nr 7 -POZIOM PODSTAWOWY

l. Wspótczynniki $a, b$ trójmianu kwadratowego $x^{2} -2ax+b$ oraz pierwiastki tego

trójmianu, napisane $\mathrm{w}$ odpowiedniej kolejności, sa czterema poczatkowymi wyra-

zami pewnego ciagu arytmetycznego. Dla $a=2$ obliczyč róznice ciagu, wspólczynnik

$b$ oraz pierwiastki trójmianu.

2. Kwadrat $0$ boku $a$ zgieto wzdluz jednej $\mathrm{z}$ przekatnych $\mathrm{t}\mathrm{a}\mathrm{k}$, aby odleglośč pozostalych

wierzcholków byla równa potowie dlugości przekatnej kwadratu. $\mathrm{W}$ tak powstaly

czworościan wpisano dwie identyczne, wzajemnie styczne kule. Obliczyč promień

tych $\mathrm{k}\mathrm{u}\mathrm{l}.$

3. Trzy czerwone, trzy zóite $\mathrm{i}$ jedna zieloną kredke wlozono $\mathrm{w}$ przypadkowy sposób

do pudelka. Obliczyč prawdopodobieństwo tego, $\dot{\mathrm{z}}\mathrm{e}\dot{\mathrm{z}}$ adne dwie kredki tego samego

koloru nie beda $\mathrm{l}\mathrm{e}\dot{\mathrm{z}}$ aly obok siebie.

4. Wyznaczyč dziedzine funkcji $f(x)=\sqrt{\frac{\log_{2}x}{1-\log_{2}x}}$. Uzasadnič, $\dot{\mathrm{z}}\mathrm{e}f(x)$ jest rosnaca.

Korzystajac $\mathrm{z}$ tego faktu, określič zbiór wartości funkcji $f(x).$

5. $\mathrm{W}$ ostrostup prawidIowy czworokatny wpisano prostopadlościan prosty $0$ podstawie

kwadratowej $\mathrm{w}$ ten sposób, $\dot{\mathrm{z}}\mathrm{e}$ wierzcholki jego górnej podstawy $\mathrm{l}\mathrm{e}\dot{\mathrm{z}}$ a $\mathrm{w}$ środkach

$\mathrm{c}\mathrm{i}\mathrm{e}\dot{\mathrm{z}}$ kości ścian bocznych ostroslupa. Pole powierzchni calkowitej prostopadlościanu

stanowi trzecia cześč pola powierzchni calkowitej ostrostupa. Obliczyč tangens kata

nachylenia krawędzi bocznej ostroslupa do podstawy.

6. Rozwiazač uklad równań

$\left\{\begin{array}{l}
x^{2}+y^{2}=2\\
- x1+-y1=2
\end{array}\right.$

Podač interpretacje geometryczna tego ukladu i sporzadzič rysunek.
\end{document}
