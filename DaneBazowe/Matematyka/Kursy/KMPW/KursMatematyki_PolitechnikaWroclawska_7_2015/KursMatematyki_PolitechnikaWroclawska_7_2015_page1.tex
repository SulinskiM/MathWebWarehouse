\documentclass[a4paper,12pt]{article}
\usepackage{latexsym}
\usepackage{amsmath}
\usepackage{amssymb}
\usepackage{graphicx}
\usepackage{wrapfig}
\pagestyle{plain}
\usepackage{fancybox}
\usepackage{bm}

\begin{document}

PRACA KONTROLNA nr 7 -POZIOM ROZSZERZONY

1. Na $\mathrm{k}\mathrm{a}\dot{\mathrm{z}}$ dym $\mathrm{z}$ trzech drutów linii elektrycznej wysokiego napiecia siedzi po pieč

wróbli. $\mathrm{W}$ pewnej chwili odfrunelo przypadkowych sześč wróbli. Obliczyč prawdo-

podobieństwo tego, $\dot{\mathrm{z}}\mathrm{e}$ na co najmniej dwóch drutach pozostaIa taka sama liczba

ptaków.

2. Dolna cześč namiotu ma ksztaIt walca $0$ wysokości $h=2\mathrm{m}$, a górna jest stozkiem $0$

tworzacej $l=\sqrt{15}\mathrm{m}\mathrm{i}$ tym samym promieniu, co cześč dolna. Wyznaczyč pozostale

parametry namiotu $\mathrm{t}\mathrm{a}\mathrm{k}$, aby jego objetośč byla najwieksza. Sporzadzič rysunek.

3. $\mathrm{Z}$ pudeIka zawierającego 10 k1ocków ponumerowanych cyframi od 0 do 9 wy1osowano

dwa klocki $\mathrm{i}$ ustawiono obok siebie $\mathrm{w}$ przypadkowej kolejności, tworzac $\mathrm{w}$ ten sposób

liczbe $k$ (ustawienie 03 rozumiemy jako 1iczbe 3). Nastepnie wy1osowano trzeci

klocek $\mathrm{z}$ pozostaIych $\mathrm{i}$ ustawiono go za tamtymi, gdy suma cyfr liczby $k$ byla mniejsza

$\mathrm{n}\mathrm{i}\dot{\mathrm{z}}10$, lub przed tamtymi, $\mathrm{w}$ przeciwnym wypadku. Obliczyč prawdopodobieństwo

tego, $\dot{\mathrm{z}}\mathrm{e}$ otrzymana liczba jest wieksza od 500.

Wsk. $\mathrm{U}\dot{\mathrm{z}}$ yč wzoru na prawdopodobieństwo catkowite.

4. Stosujac zasade indukcji matematycznej, udowodnič $\mathrm{t}\mathrm{o}\dot{\mathrm{z}}$ samośč

$\sin^{2}\alpha+\sin^{2}3\alpha+ +\displaystyle \sin^{2}(2n-1)\alpha=\frac{n}{2}-\frac{\sin 4n\alpha}{4\sin 2\alpha},$

$n\geq 1,$

gdzie $\displaystyle \alpha\neq k\frac{\pi}{2}, k$ caIkowite.

5. Znalez/č równanie stycznej $l$ do wykresu funkcji $f(x)=\displaystyle \frac{1}{x}+x^{2}\mathrm{w}$ punkcie, $\mathrm{w}$ którym

przecina on oś $Ox$. Wyznaczyč wszystkie styczne, które sa równolegle do prostej

$l$. Znalez/č punkty wspólne tych stycznych $\mathrm{z}$ wykresem funkcji. Rozwiązanie zilu-

strowač odpowiednim rysunkiem.

6. {\it K}rawed $\acute{\mathrm{z}}$ podstawy graniastoslupa trójkatnego prawidIowego ma dIugośč $a$. Oznaczmy

przez $ 2\alpha \mathrm{k}\mathrm{a}\mathrm{t}$ miedzy przekatnymi ścian bocznych wychodzacymi $\mathrm{z}$ jednego wierz-

choIka. Graniastoslup $\mathrm{p}\mathrm{r}\mathrm{z}\mathrm{e}\mathrm{c}\mathrm{i}_{\xi}!\mathrm{t}\mathrm{o}$ na dwie cześci pIaszczyzna przechodzaca przez

krawed $\acute{\mathrm{z}}$ dolnej podstawy $\mathrm{i}$ przeciwlegly wierzchoiek górnej podstawy. Obliczyč tan-

gens kata $\alpha$, dla którego $\mathrm{w}$ wieksza cześč graniastoslupa $\mathrm{m}\mathrm{o}\dot{\mathrm{z}}$ na wpisač kule. Dla

znalezionego kata $\alpha$, obliczyč promień kuli wpisanej $\mathrm{w}$ mniejsza cześč graniastosIupa.
\end{document}
