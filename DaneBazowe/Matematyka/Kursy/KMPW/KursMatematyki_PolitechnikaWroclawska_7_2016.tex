\documentclass[a4paper,12pt]{article}
\usepackage{latexsym}
\usepackage{amsmath}
\usepackage{amssymb}
\usepackage{graphicx}
\usepackage{wrapfig}
\pagestyle{plain}
\usepackage{fancybox}
\usepackage{bm}

\begin{document}

XLV

KORESPONDENCYJNY KURS

Z MATEMATYKI

marzec 2016 r.

PRACA KONTROLNA nr 7 -POZIOM PODSTAWOWY

l. Cztery cyfry 0 $\mathrm{i}$ pięč cyfr l ustawiono $\mathrm{w}$ przypadkowej kolejności. Obliczyč praw-

dopodobieństwo tego, $\dot{\mathrm{z}}\mathrm{e}$ na obu końcach powstafego ciągu znalazfy się jednakowe

cyfry.

2. Drugi wyraz pewnego $\mathrm{c}\mathrm{i}_{\Phi \mathrm{g}}\mathrm{u}$ geometrycznego wynosi 8, a ósmy 2. Ob1iczyč siedemna-

sty wyraz tego ciągu oraz sumę pietnastu wyrazów, poczynając od wyrazu trzeciego.

Wynik zapisač $\mathrm{w}$ najprostszej postaci.

3. Rozwiązač nierównośč

$\sqrt{2^{x-2}-2}\leq 2^{x-1}-5.$

4. Dana jest funkcja $f(x)=\displaystyle \frac{\sqrt{2-x-x^{2}}}{\sqrt{1-x^{2}}}$. Znalez/č wszystkie wartości parametru rze-

czywistego $a$, dla których równanie $f(x)=2^{a}$ posiada rozwiązanie. Sporządzič wy-

kres funkcji $f(x).$

5. Romb $0$ boku $a\mathrm{i}$ kącie ostrym $\alpha$ zgięto wzdluz prostej $l_{\Phi}$czącej środki przeciwlegfych

boków, tak aby obie części rombu byly wzajemnie prostopadle. Obliczyč odległośč

wierzchołków katów ostrych oraz cosinus kąta pomiędzy polowami krótszej przekąt-

nej $\mathrm{w}$ zgiętym rombie.

6. Dlugości boków trapezu opisanego na okręgu są liczbami naturalnymi $\mathrm{i}$ są kolejny-

mi wyrazami ciągu arytmetycznego. Obwód trapezu wynosi 24. Ob1iczyč po1e oraz

dłuzszą przekątna trapezu.




PRACA KONTROLNA nr 7 -POZIOM ROZSZERZONY

l. Spośród 12 pączków, $1\mathrm{e}\mathrm{z}\Phi^{\mathrm{C}}\mathrm{y}\mathrm{c}\mathrm{h}$ na pófmisku, 6 byfo nadziewanych, 61ukrowanych,

a 4 nie miały nadzienia ani nie były 1ukrowane. Franek zjad1 dwa 1osowo wybrane

pączki. Obliczyč prawdopodobieństwo, $\dot{\mathrm{z}}\mathrm{e}$ jadf zarówno pqczka lukrowanego jak $\mathrm{i}$

pączka $\mathrm{z}$ nadzieniem.

2. Na krzywej $0$ równaniu $y = \sqrt{4-x}, x \geq 0$, znalez$\acute{}$č punkt $P$, tak aby odcinek

$\iota_{\Phi^{\mathrm{c}\mathrm{z}\text{ą} \mathrm{c}\mathrm{y}P\mathrm{z}\mathrm{p}^{\mathrm{O}\mathrm{C}\mathrm{Z}}\Phi^{\mathrm{t}\mathrm{k}\mathrm{i}\mathrm{e}\mathrm{m}}}}$ ukfadu wspófrzędnych, przy obrocie wokóf osi $Ox$, zakreślił

powierzchnię $0$ największym polu. Sporządzič rysunek.

3. Wyznaczyč punkty przecięcia się wykresu funkcji $f(x) = \displaystyle \frac{3x-7}{2x-2} \mathrm{z}$ wykresem jej

pochodnej $f'(x). K\mathrm{o}\mathrm{r}\mathrm{z}\mathrm{y}\mathrm{s}\mathrm{t}\mathrm{a}\mathrm{j}_{\Phi}\mathrm{c}$ ze wzoru tg $(\displaystyle \alpha-\beta)=\frac{\mathrm{t}\mathrm{g}\alpha-\mathrm{t}\mathrm{g}\beta}{1+\mathrm{t}\mathrm{g}\alpha \mathrm{t}\mathrm{g}\beta}$, obliczyč tangensy

katów, pod którymi przecinają się te wykresy. Rozwiazanie zilustrowač odpowiednim

rysunkiem.

4. Stosujqc zasadę indukcji matematycznej, udowodnič nierównośč

$2\displaystyle \sqrt{n}-\frac{3}{2}<1+\frac{1}{\sqrt{2}}+\frac{1}{\sqrt{3}}+ +\displaystyle \frac{1}{\sqrt{n}}\leq 2\sqrt{n}-1,$

$n\geq 1.$

Dla jakich $n$ nierównośč ta pozwala na oszacowanie występującej $\mathrm{w}$ niej sumy $\mathrm{z}$

blędem względnym mniejszym $\mathrm{n}\mathrm{i}\dot{\mathrm{z}}$ 1\%.

5. $\mathrm{Z}$ punktu $P$ widač okrąg $0$ środku $O \mathrm{i}$ promieniu $r$ pod kqtem $ 2\alpha$. Prosta $PO$

przecina okrąg $\mathrm{w}$ punktach A $\mathrm{i}C$, a styczne do okręgu, poprowadzone $\mathrm{z}$ punktu $P,$

przechodzą przez punkty $B\mathrm{i}D$ na okręgu. Obliczyč promień okręgu wpisanego $\mathrm{w}$

czworokąt ABCD oraz odległośč środków obu okręgów.

6. Podstawą ostrosfupajest romb $0$ boku 5. Spodek wysokości ostrosfupa $\mathrm{l}\mathrm{e}\dot{\mathrm{z}}\mathrm{y}\mathrm{w}$ środku

podstawy, a krawędzie boczne mają długości 6 $\mathrm{i}7$. Obliczyč objętośč ostroslupa oraz

cosinus kąta nachylenia ściany bocznej do podstawy.



\end{document}