\documentclass[a4paper,12pt]{article}
\usepackage{latexsym}
\usepackage{amsmath}
\usepackage{amssymb}
\usepackage{graphicx}
\usepackage{wrapfig}
\pagestyle{plain}
\usepackage{fancybox}
\usepackage{bm}

\begin{document}

XLV

KORESPONDENCYJNY KURS

Z MATEMATYKI

marzec 2016 r.

PRACA KONTROLNA nr 7 -POZIOM PODSTAWOWY

l. Cztery cyfry 0 $\mathrm{i}$ pięč cyfr l ustawiono $\mathrm{w}$ przypadkowej kolejności. Obliczyč praw-

dopodobieństwo tego, $\dot{\mathrm{z}}\mathrm{e}$ na obu końcach powstafego ciągu znalazfy się jednakowe

cyfry.

2. Drugi wyraz pewnego $\mathrm{c}\mathrm{i}_{\Phi \mathrm{g}}\mathrm{u}$ geometrycznego wynosi 8, a ósmy 2. Ob1iczyč siedemna-

sty wyraz tego ciągu oraz sumę pietnastu wyrazów, poczynając od wyrazu trzeciego.

Wynik zapisač $\mathrm{w}$ najprostszej postaci.

3. Rozwiązač nierównośč

$\sqrt{2^{x-2}-2}\leq 2^{x-1}-5.$

4. Dana jest funkcja $f(x)=\displaystyle \frac{\sqrt{2-x-x^{2}}}{\sqrt{1-x^{2}}}$. Znalez/č wszystkie wartości parametru rze-

czywistego $a$, dla których równanie $f(x)=2^{a}$ posiada rozwiązanie. Sporządzič wy-

kres funkcji $f(x).$

5. Romb $0$ boku $a\mathrm{i}$ kącie ostrym $\alpha$ zgięto wzdluz prostej $l_{\Phi}$czącej środki przeciwlegfych

boków, tak aby obie części rombu byly wzajemnie prostopadle. Obliczyč odległośč

wierzchołków katów ostrych oraz cosinus kąta pomiędzy polowami krótszej przekąt-

nej $\mathrm{w}$ zgiętym rombie.

6. Dlugości boków trapezu opisanego na okręgu są liczbami naturalnymi $\mathrm{i}$ są kolejny-

mi wyrazami ciągu arytmetycznego. Obwód trapezu wynosi 24. Ob1iczyč po1e oraz

dłuzszą przekątna trapezu.
\end{document}
