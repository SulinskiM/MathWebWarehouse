\documentclass[10pt]{article}
\usepackage[polish]{babel}
\usepackage[utf8]{inputenc}
\usepackage[T1]{fontenc}
\usepackage{amsmath}
\usepackage{amsfonts}
\usepackage{amssymb}
\usepackage[version=4]{mhchem}
\usepackage{stmaryrd}

\title{PRACA KONTROLNA nr 7 -POZIOM PODSTAWOWY }

\author{}
\date{}


\begin{document}
\maketitle
\begin{enumerate}
  \item Cztery cyfry 0 i pięć cyfr 1 ustawiono w przypadkowej kolejności. Obliczyć prawdopodobieństwo tego, że na obu końcach powstałego ciągu znalazły się jednakowe cyfry.
  \item Drugi wyraz pewnego ciągu geometrycznego wynosi 8 , a ósmy 2. Obliczyć siedemnasty wyraz tego ciągu oraz sumę piętnastu wyrazów, poczynając od wyrazu trzeciego. Wynik zapisać w najprostszej postaci.
  \item Rozwiązać nierówność
\end{enumerate}

$$
\sqrt{2^{x-2}-2} \leqslant 2^{x-1}-5
$$

\begin{enumerate}
  \setcounter{enumi}{3}
  \item Dana jest funkcja $f(x)=\frac{\sqrt{2-x-x^{2}}}{\sqrt{1-x^{2}}}$. Znaleźć wszystkie wartości parametru rzeczywistego $a$, dla których równanie $f(x)=2^{a}$ posiada rozwiązanie. Sporządzić wykres funkcji $f(x)$.
  \item Romb o boku $a$ i kącie ostrym $\alpha$ zgięto wzdłuż prostej łączącej środki przeciwległych boków, tak aby obie części rombu były wzajemnie prostopadłe. Obliczyć odległość wierzchołków kątów ostrych oraz cosinus kąta pomiędzy połowami krótszej przekątnej w zgiętym rombie.
  \item Długości boków trapezu opisanego na okręgu są liczbami naturalnymi i są kolejnymi wyrazami ciągu arytmetycznego. Obwód trapezu wynosi 24. Obliczyć pole oraz dłuższą przekątna trapezu.
\end{enumerate}

\section*{PRACA KONTROLNA nr 7 -POZIOM ROZSZERZONY}
\begin{enumerate}
  \item Spośród 12 pączków, leżących na półmisku, 6 było nadziewanych, 6 lukrowanych, a 4 nie miały nadzienia ani nie były lukrowane. Franek zjadł dwa losowo wybrane pączki. Obliczyć prawdopodobieństwo, że jadł zarówno pączka lukrowanego jak i pączka z nadzieniem.
  \item Na krzywej o równaniu $y=\sqrt{4-x}, x \geqslant 0$, znaleźć punkt $P$, tak aby odcinek łączący $P$ z początkiem układu współrzędnych, przy obrocie wokół osi $O x$, zakreślił powierzchnię o największym polu. Sporządzić rysunek.
  \item Wyznaczyć punkty przecięcia się wykresu funkcji $f(x)=\frac{3 x-7}{2 x-2}$ z wykresem jej pochodnej $f^{\prime}(x)$. Korzystając ze wzoru $\operatorname{tg}(\alpha-\beta)=\frac{\operatorname{tg} \alpha-\operatorname{tg} \beta}{1+\operatorname{tg} \alpha \operatorname{tg} \beta}$, obliczyć tangensy kątów, pod którymi przecinają się te wykresy. Rozwiązanie zilustrować odpowiednim rysunkiem.
  \item Stosując zasadę indukcji matematycznej, udowodnić nierówność
\end{enumerate}

$$
2 \sqrt{n}-\frac{3}{2}<1+\frac{1}{\sqrt{2}}+\frac{1}{\sqrt{3}}+\ldots+\frac{1}{\sqrt{n}} \leqslant 2 \sqrt{n}-1, \quad n \geqslant 1 .
$$

Dla jakich $n$ nierówność ta pozwala na oszacowanie występującej w niej sumy z błędem względnym mniejszym niż $1 \%$.\\
5. Z punktu $P$ widać okrąg o środku $O$ i promieniu $r$ pod kątem $2 \alpha$. Prosta $P O$ przecina okrąg w punktach $A$ i $C$, a styczne do okręgu, poprowadzone z punktu $P$, przechodzą przez punkty $B$ i $D$ na okręgu. Obliczyć promień okręgu wpisanego w czworokąt $A B C D$ oraz odległość środków obu okręgów.\\
6. Podstawą ostrosłupa jest romb o boku 5. Spodek wysokości ostrosłupa leży w środku podstawy, a krawędzie boczne mają długości 6 i 7 . Obliczyć objętość ostrosłupa oraz cosinus kąta nachylenia ściany bocznej do podstawy.


\end{document}