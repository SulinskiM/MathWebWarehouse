\documentclass[a4paper,12pt]{article}
\usepackage{latexsym}
\usepackage{amsmath}
\usepackage{amssymb}
\usepackage{graphicx}
\usepackage{wrapfig}
\pagestyle{plain}
\usepackage{fancybox}
\usepackage{bm}

\begin{document}

XLVI

KORESPONDENCYJNY KURS

Z MATEMATYKI

marzec 2017 r.

PRACA KONTROLNA nr 7- POZIOM PODSTAWOWY

l. Pierwszym wyrazem ciagu arytmetycznego jest $a_{1}=2017$, a jego róznica jest rozwiąza-

niem równania $\sqrt{2-x}-x = 10$. Obliczyč sume wszystkich dodatnich wyrazów tego

ciagu.

2. Spośród dwucyfrowych liczb nieparzystych mniejszych od 50 wylosowano bez zwracania

dwie. Obliczyč prawdopodobieństwo tego, $\dot{\mathrm{z}}\mathrm{e}$ obie wylosowane liczby sa pierwsze oraz

prawdopodobieństwo tego, $\dot{\mathrm{z}}\mathrm{e}$ iloczyn wylosowanych liczb nie jest podzielny przez 15.

3. Uzasadnič, $\dot{\mathrm{z}}\mathrm{e}$ ciag $0$ wyrazach $a_{n}=\displaystyle \frac{2^{n}+2^{n+1}+..+2^{2n}}{2^{2}+2^{4}+\ldots+2^{2n}}, n\geq 1$, nie jest rosnacy oraz,

$\dot{\mathrm{z}}\mathrm{e}$ jest rosnacy, poczynajac od $n=2.$

4. Znalez/č wszystkie wartości parametru rzeczywistego $m$, dla których proste $0$ równaniach

$x-my+2m=0, 2mx+4y+1=0, mx-y-3m-1=0$ sa parami rózne $\mathrm{i}$ przecinaja

$\mathrm{s}\mathrm{i}\mathrm{e}\mathrm{w}$ jednym punkcie. Sporzadzič odpowiedni rysunek dla najmniejszej ze znalezionych

wartości tego parametru.

5. $\mathrm{W}$ ostrostupie prawidtowym czworokatnym dana jest odleglośč $d$ środka podstawy od

krawedzi bocznej oraz $\mathrm{k}\mathrm{a}\mathrm{t}2\alpha$ miedzy sasiednimi ścianami bocznymi. Obliczyč objetośč

ostroslupa.

6. Podstawa $AB$ trójkata równoramiennego $ABC$ jest krótsza od ramion. Wysokości $AD$

$\mathrm{i}$ CE dziela trojkat na cztery cześci, $\mathrm{z}$ których dwie sa trójkatami prostokatnymi $0$ polach

równych 9 oraz 2. Ob1iczyč po1a pozostatych cz$\xi$!ści oraz obwód trójkata.
\end{document}
