\documentclass[a4paper,12pt]{article}
\usepackage{latexsym}
\usepackage{amsmath}
\usepackage{amssymb}
\usepackage{graphicx}
\usepackage{wrapfig}
\pagestyle{plain}
\usepackage{fancybox}
\usepackage{bm}

\begin{document}

XLVI

KORESPONDENCYJNY KURS

Z MATEMATYKI

marzec 2017 r.

PRACA KONTROLNA $\mathrm{n}\mathrm{r} 7$- POZIOM ROZSZERZONY

l. Turysta zabtądzil $\mathrm{w}$ lesie zajmujacym obszar ($\mathrm{w}$ km)

$D=\{(x,y):x^{2}+y^{2}\leq 2y+3,-2y\leq x\leq y\}.$

Wskazač mu najkrótsza droge wyjścia $\mathrm{z}$ lasu, jeśli znaduje $\mathrm{s}\mathrm{i}\mathrm{e}\mathrm{w}$ punkcie $P(-\displaystyle \frac{1}{4},\frac{3}{2})$. Ile

minut bedzie trwala wedrówka, jeśli idzie $\mathrm{z}$ predkościa 4 $\mathrm{k}\mathrm{m}/\mathrm{h}$?

2. Korzystajac $\mathrm{z}$ zasady indukcji matematycznej, udowodnič prawdziwośč nierówności

$1^{5}+2^{5}++n^{5}<\displaystyle \frac{n^{3}(n+1)^{3}}{6},n\geq 1.$

3. Kubuś zaobserwowal, $\dot{\mathrm{z}}\mathrm{e}\mathrm{w}$ pewnej chwili $\mathrm{w}$ trzypietrowej kamienicy po drugiej stronie

ulicy pali $\mathrm{s}\mathrm{i}\mathrm{e}$ światto $\mathrm{w}10$ oknach. Na $\mathrm{k}\mathrm{a}\dot{\mathrm{z}}$ dej kondygnacji kamienicy znajduja $\mathrm{s}\mathrm{i}\mathrm{e}4$ okna.

Zakladamy, $\dot{\mathrm{z}}\mathrm{e}$ okna zapalaja $\mathrm{s}\mathrm{i}\mathrm{e}\mathrm{i}$ gasna losowo. Obliczyč prawdopodobieństwo tego, $\dot{\mathrm{z}}\mathrm{e}$

zarówno na drugim jak $\mathrm{i}$ na trzecim pietrze kamienicy świeca $\mathrm{s}\mathrm{i}\mathrm{e}$ co najmniej dwa okna.

Wsk. Skorzystač ze wzoru $P(A\cup B)=P(A)+P(B)-P(A\cap B).$

4. Podstawa graniastosIupa prostego $0$ wysokości $h=2$jest trójkat, $\mathrm{w}$ którym tangens kata

przy wierzchoIku $A$ wynosi $-\sqrt{2}$. Przekatne $e, f$ sasiednich ścian bocznych, wychodzace

$\mathrm{z}$ wierzcholka $A$, sa do siebie prostopadle, a liczby $h, e, f$ sa kolejnymi wyrazami pewnego

ciągu geometrycznego. Obliczyč objetośč graniastosIupa.

5. Znalez/č dziedzine $\mathrm{i}$ zbiór wartości funkcji

$f(x)=\sqrt{\log_{2}\frac{1}{\cos x+\sqrt{3}\sin x}}.$

6. $\mathrm{K}\mathrm{a}\mathrm{t}$ ptaski przy wierzcholku $D$ ostroslupa prawidlowego trójkatnego $0$ podstawie $ABC$

jest równy $\alpha$. Na krawedzi $BD$ wybrano punkt $E \mathrm{t}\mathrm{a}\mathrm{k}, \dot{\mathrm{z}}\mathrm{e} \triangle ACE$ jest trójkatem

równobocznym. Znalez/č stosunek $k(\alpha)$ objetości ostroslupa ABCE do objetości ostro-

{\it sIupa ACED} $\mathrm{w}$ zalezności od kata $\alpha$. Sporzadzič wykres funkcji $k(\alpha).$

Rozwiazania (rekopis) zadań $\mathrm{z}$ wybranego poziomu prosimy nadsyIač do 18 marca 2017 $\mathrm{r}$. na

adres:

WydziaI Matematyki

Politechniki Wroclawskiej,

$\mathrm{u}1$. Wybrzez $\mathrm{e}$ Wyspiańskiego 27,

50-370 WROCLAW.
\begin{center}
\begin{tabular}{|l|l|l|}
\hline
\multicolumn{1}{|l|}{Na kopercie prosimy $\underline{\mathrm{k}\mathrm{o}\mathrm{n}\mathrm{i}\mathrm{e}\mathrm{c}\mathrm{z}\mathrm{n}\mathrm{i}\mathrm{e}}$ zaznaczyč wybrany poziom!}&	\multicolumn{1}{|l|}{(np.}&	\multicolumn{1}{|l|}{poziom podstawowy lub}	\\
\hline
\end{tabular}

\end{center}
Na kopercie prosimy$\underline{\mathrm{k}\mathrm{o}\mathrm{n}\mathrm{i}\mathrm{e}\mathrm{c}\mathrm{z}\mathrm{n}\mathrm{i}\mathrm{e}}$ zaznaczyč wybrany poziom! (np. poziom podstawowy lub

jonym znaczkiem, odpowiednim do wagi listu (od 1.01.2017nowe ceny znaczków!). Prace nie

spelniajace podanych warunków nie beda poprawiane ani odsylane.

ï{\it r}
\end{document}
