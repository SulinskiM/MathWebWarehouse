\documentclass[a4paper,12pt]{article}
\usepackage{latexsym}
\usepackage{amsmath}
\usepackage{amssymb}
\usepackage{graphicx}
\usepackage{wrapfig}
\pagestyle{plain}
\usepackage{fancybox}
\usepackage{bm}

\begin{document}

XLVI

KORESPONDENCYJNY KURS

Z MATEMATYKI

marzec 2018 r.

PRACA KONTROLNA nr 7- POZIOM PODSTAWOWY

l. Liczba l jest pierwiastkiem wielomianu trzeciego stopnia $w(x)$ oraz wielomianu $w(x+1).$

Środkiem symetrii wykresu $w(x)$ jest punkt $S(0,2)$. Narysowač staranny wykres funkcji

$f(x)= |w(x-1)|. (\acute{\mathrm{S}}$rodkiem symetrii krzywej $0$ równaniu $y(x) =ax^{3}+bx^{2}+cx+d$

jest punkt $S(\displaystyle \frac{-b}{3a},y(\frac{-b}{3a})).$)

2. Sala jest oświetlona 5 $\dot{\mathrm{z}}$ arówkami. Wkrecono losowo $\dot{\mathrm{z}}$ arówki z$\cdot$óIte, czerwone, zielone $\mathrm{i}$

niebieskie. Obliczyč prawdopodobieństwo, $\dot{\mathrm{z}}\mathrm{e}$ wkrecono co najmniej dwie $\dot{\mathrm{z}}$ arówki z$\cdot$ólte

$\mathrm{i}$ co najmniej dwie czerwone.

3. Rozwiazač równanie

$\displaystyle \frac{\cos 5x}{\cos 3x}+1=0.$

4. Wazon $\mathrm{w}$ ksztatcie walca, którego wysokośč jest wieksza od średnicy podstawy, ma

objetośč 1200 $\mathrm{c}\mathrm{m}^{3}$ Napelniony woda wazon przechylono $\mathrm{t}\mathrm{a}\mathrm{k}, \dot{\mathrm{z}}\mathrm{e}$ jego oś symetrii utwo-

rzyla $\mathrm{z}$ pionem $\mathrm{k}\mathrm{a}\mathrm{t}45^{o}$ Wylalo $\mathrm{s}\mathrm{i}\mathrm{e}200\mathrm{c}\mathrm{m}^{3}$ wody. Podač wymiary wazonu (pominač

grubośč ścianek).

5. Podstawa $AB$ trapezu równoramiennego jest średnicą okregu opisanego na nim. Za

pomoca rachunku wektorowego wyznaczyč wspólrzedne wierzchotków $B\mathrm{i}C$, wiedzac,

$\dot{\mathrm{z}}\mathrm{e}|AB|=5, A(1,1), D(3,2)$ oraz $\dot{\mathrm{z}}\mathrm{e}B\mathrm{l}\mathrm{e}\dot{\mathrm{z}}\mathrm{y}\mathrm{w}$ dolnej pólplaszczyz/nie.

6. Krzywa spiralna jest utworzona $\mathrm{z}$ čwiartek okregów, których promienie tworza ciag geo-

{\it O} $p_{0} Oy \mathrm{t}\mathrm{a}\mathrm{k}, \dot{\mathrm{z}}\mathrm{e}$ uki obu okr gów acza si $\mathrm{w}$ punk-

$P_{2} O_{3}$ cie $P_{1}$ (rysunek). Środki kolejnych okregów sa

$P_{3}$ Nast pnie wykonac obliczenia dla $q=\displaystyle \frac{3}{2}.$
\begin{center}
\includegraphics[width=68.940mm,height=69.852mm]{./KursMatematyki_PolitechnikaWroclawska_7_2018_page0_images/image001.eps}
\end{center}
$P_{1}$

$O_{2}$

metryczny $0$ ilorazie $q >$ l. Środek pierw-

szego okr gu znajduje si $\mathrm{w}$ poczatku uk adu

wspo rz dnych, a punkt $P_{0}(2,0)$ jest poczatkiem

krzywej. Srodek $O_{2}$ drugiego okr gu $\mathrm{l}\mathrm{e}\dot{\mathrm{z}}\mathrm{y}$ na osi

tak po $0\dot{\mathrm{z}}$ one, $\dot{\mathrm{z}}\mathrm{e}$ utworzona krzywa jest $\mathrm{g}$ adka $\mathrm{i}$

promien uku mniejszego okregu jest cześcia pro-

mienia uku wiekszego okr gu (rysunek). Znale $\acute{\mathrm{z}}\mathrm{c}$

wspo rz dne srodka $O_{6}$ oraz $\mathrm{d}$ ugośc uku spi-

rali $P_{0}P_{6}$. Wynik podač $\mathrm{w}$ najprostszej postaci.
\end{document}
