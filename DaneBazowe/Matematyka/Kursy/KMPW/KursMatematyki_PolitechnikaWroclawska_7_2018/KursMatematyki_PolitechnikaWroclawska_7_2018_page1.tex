\documentclass[a4paper,12pt]{article}
\usepackage{latexsym}
\usepackage{amsmath}
\usepackage{amssymb}
\usepackage{graphicx}
\usepackage{wrapfig}
\pagestyle{plain}
\usepackage{fancybox}
\usepackage{bm}

\begin{document}

XLVI

KORESPONDENCYJNY KURS

Z MATEMATYKI

marzec 2018 r.

PRACA KONTROLNA nr 7- POZIOM ROZSZERZONY

l. Stosujac zasade indukcji matematycznej, wykazač, $\dot{\mathrm{z}}\mathrm{e}$ dla wszystkich $n$

$10^{n}+18n-1$ jest podzielna przez 27.

$\geq 1$ liczba

2. Sprawdzič $\mathrm{t}\mathrm{o}\dot{\mathrm{z}}$ samośč

$\displaystyle \frac{\cos^{2}\alpha-\cos^{2}\beta}{\sin^{2}\alpha-\cos^{2}\beta}=\mathrm{t}\mathrm{g}(\alpha+\beta)\mathrm{t}\mathrm{g}(\alpha-\beta)$

$\mathrm{i}$ określič jej dziedzine.

3. Dwóch strzelców oddato $\mathrm{k}\mathrm{a}\dot{\mathrm{z}}\mathrm{d}\mathrm{y}$ po dwa strzaly $\mathrm{i}$ okazato $\mathrm{s}\mathrm{i}\mathrm{e}, \dot{\mathrm{z}}\mathrm{e}$ cel zostal trafiony dokIadnie

dwa razy. Obliczyč prawdopodobieństwo, $\dot{\mathrm{z}}\mathrm{e}$ dwukrotnie trafil pierwszy strzelec, jeśli

za $\mathrm{k}\mathrm{a}\dot{\mathrm{z}}$ dym razem pierwszy trafia $\mathrm{z}$ prawdopodobieństwem $\displaystyle \frac{4}{5}$, a drugi $\mathrm{z}$ prawdopodo-

bieństwem $\displaystyle \frac{3}{5}.$

4. Znalez/č wartośč parametru nieujemnego $p$, dla którego suma kwadratów odwrotności

pierwiastków równania

$x^{2}+(p+1)x-(p+3)=0$

jest najmniejsza.

5. Rozwiazač uktad równań

$\left\{\begin{array}{l}
x^{2}y^{2}=4\\
y^{4}-6y^{2}-x^{2}+9=0
\end{array}\right.$

Podač interpretacje geometryczna tego ukladu $\mathrm{i}$ obliczyč pole wielokata utworzonego

przez jego rozwiazania (interpretowane jako punkty na plaszczy $\acute{\mathrm{z}}\mathrm{n}\mathrm{i}\mathrm{e}$). Sporzadzič rysu-

nek.

6. Podstawa ostrosIupa ABCD jest trójkat równoramienny $0$ kacie przy wierzchoIku $2\alpha.$

Plaszczyzna przechodząca przez wierzcholek $D$ ostroslupa $\mathrm{i}$ wysokośč podstawy jest

pIaszczyzna symetrii ostrosIupa, a przekrój bryIy ta pIaszczyzna jest trójkatem równo-

bocznym $0$ boku $a$. Wykazač, $\dot{\mathrm{z}}\mathrm{e}$ ostrosIup ma jeszcze jedna plaszczyzne symetrii $\mathrm{i}$

obliczyč promień kuli opisanej na nim.

Rozwiązania (rekopis) zadań z wybranego poziomu prosimy nadsylač do 18 marca 2018 r. na

adres:

Wydziaf Matematyki

Politechniki Wrocfawskiej,

ul. Wybrzeze Wyspiańskiego 27,

50-370 WROCLAW.
\begin{center}
\begin{tabular}{|l|l|l|}
\hline
\multicolumn{1}{|l|}{Na kopercie prosimy $\underline{\mathrm{k}\mathrm{o}\mathrm{n}\mathrm{i}\mathrm{e}\mathrm{c}\mathrm{z}\mathrm{n}\mathrm{i}\mathrm{e}}$ zaznaczyč wybrany poziom!}&	\multicolumn{1}{|l|}{(np.}&	\multicolumn{1}{|l|}{poziom podstawowy lub}	\\
\hline
	\\
	\\
\multicolumn{1}{|l|}{beda poprawiane ani odsytane.}	\\
\cline{1-1}
\end{tabular}

\end{center}
Adres Internetowy Kursu: http://www.im.pwr.wroc.pl/kurs
\end{document}
