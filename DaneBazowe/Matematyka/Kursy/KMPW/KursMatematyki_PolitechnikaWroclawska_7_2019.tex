\documentclass[a4paper,12pt]{article}
\usepackage{latexsym}
\usepackage{amsmath}
\usepackage{amssymb}
\usepackage{graphicx}
\usepackage{wrapfig}
\pagestyle{plain}
\usepackage{fancybox}
\usepackage{bm}

\begin{document}

XLVIII

KORESPONDENCYJNY KURS

Z MATEMATYKI

marzec 2019 r.

PRACA KONTROLNA nr 7- POZIOM PODSTAWOWY

1. $\mathrm{W}$ pierwszej godzinie rowerzysta A jedzie $\mathrm{z}$ prędkościq 25 $\mathrm{k}\mathrm{m}/\mathrm{h}$, a $\mathrm{w}\mathrm{k}\mathrm{a}\dot{\mathrm{z}}$ dej kolejnej

godziniejedzie ze stafą prędkości$\Phi$ mniejszą $0$ 20\% $\mathrm{w}$ stosunku do prędkości $\mathrm{w}$ poprzedniej

godzinie. Natomiast rowerzysta $\mathrm{B}$ jedzie $\mathrm{w}$ pierwszej godzinie $\mathrm{z}$ prędkością 8 $\mathrm{k}\mathrm{m}/\mathrm{h},$

a $\mathrm{w}\mathrm{k}\mathrm{a}\dot{\mathrm{z}}$ dej kolejnej godzinie jedzie ze stałq prędkością większq $0$ 20\% $\mathrm{w}$ stosunku do

prędkości $\mathrm{w}$ poprzedniej godzinie. Obaj startują równocześnie $\mathrm{z}$ tego samego punktu.

Który $\mathrm{z}$ nich dotrze prędzej do celu lezącego $\mathrm{w}$ odleglości 100 km od punktu startu?

Po której godzinie jazdy odlegfośč między nimi $\mathrm{w}$ zaokrągleniu do pełnych kilometrów

będzie największa $\mathrm{i}$ ile będzie wynosič? Odpowiedzi uzasadnič bez stosowania obliczeń

przyblizonych.

2. $\mathrm{W}$ skarbonce jest 5 monet 5 zf $\mathrm{i}5$ monet 2 $\mathrm{z}\mathrm{f}$. Kuba wylosowaf ze skarbonki 6 monet.

Obliczyč prawdopodobieństwo tego, $\dot{\mathrm{z}}\mathrm{e}$ wystarczy mu pieniędzy na kupno ksiązki $\mathrm{w}$ cenie

20 $\mathrm{z}1.$

3. Rozwiązač nierównośč

$2\log_{2}(3-\sqrt{2^{x+1}-7})>x.$

4. Dla jakich wartości parametru $m$ liczby $x_{0}, y_{0}$, spełniające uklad równań

$\left\{\begin{array}{l}
x\\
3x
\end{array}\right.$

$+$

$+$

{\it my}

2{\it y}

$=2$

$=m$

są odpowiednio cosinusem $\mathrm{i}$ sinusem tego samego kąta $\alpha \in [0,\pi]$. Podač $x_{0} \mathrm{i} y_{0}$ dla

znalezionych wartości parametru $m.$

5. $\mathrm{W}$ ostrosfupie prawidfowym trójkątnym kąt pomiędzy ścianami bocznymi wynosi $2\alpha.$

Niech $P$ będzie spodkiem wysokości ściany bocznej opuszczonej na krawęd $\acute{\mathrm{z}}$ boczną.

Pfaszczyzna równolegfa do podstawy przechodząca przez $P$ dzieli ostrosfup na dwie czę-

ści, $\mathrm{z}$ których górna ma objetośč $V$. Obliczyč objętośč oraz krawędz/ podstawy ostrosłupa.

Podač dziedzinę kąta $\alpha.$

6. Kąty przy podstawie $AB$ trójkąta sq równe $\alpha$ oraz $2\alpha, \displaystyle \alpha<\frac{\pi}{4}$, a środkowa boku $AB$ ma

dlugośč $d$. Znalez/č dlugości boków trójk$\Phi$ta. Następnie podstawič do wyniku ogólnego

dane $d=\sqrt{11}$ oraz $\displaystyle \sin\alpha=\frac{\sqrt{2}}{4}\mathrm{i}$ wykonač obliczenia.




XLVIII

KORESPONDENCYJNY KURS

Z MATEMATYKI

marzec 2019 r.

PRACA KONTROLNA $\mathrm{n}\mathrm{r} 7$- POZIOM ROZSZERZONY

l. Rozwiqzač nierównośč

$\sqrt{\sin 2x-\cos 2x+1}\leq 2\sin x.$

2. Ze zbioru $\{$1, 2, $3n\}, n\geq 1$, wylosowano bez zwracania dwie liczby. Obliczyč prawdo-

podobieństwo tego, $\dot{\mathrm{z}}\mathrm{e}$ suma otrzymanych liczb jest mniejsza od $4n\mathrm{i}$ co najmniej jedna

$\mathrm{z}$ nich jest większa od $n.$

3. Stosując zasadę indukcji matematycznej, udowodnič prawdziwośč wzoru

$1^{4}+2^{4}++n^{4}+\displaystyle \frac{1^{2}+2^{2}++n^{2}}{5}=\frac{n^{2}(n+1)^{2}(2n+1)}{10},$

$n\geq 1.$

4. Dana jest funkcja $f(x)=\displaystyle \frac{1}{3}x^{3}-\frac{4}{3}x$. Styczna do wykresu tej funkcji $\mathrm{w}$ punkcie $A(1,-1)$

przecina wykres $\mathrm{w}$ punkcie $B(x_{1},f(x_{1}))$, a styczna do jej wykresu $\mathrm{w}$ punkcie $B$ przecina

wykres $\mathrm{w}$ punkcie $C(x_{2},f(x_{2}))$. Znalez/č punkty $B \mathrm{i} C$ oraz obliczyč tangensy katów

trójkąta $\triangle ABC$. Sporządzič rysunek, dobierając odpowiednie skale na obu osiach.

5. $\mathrm{W}$ czworokącie ABCD $0$ bokach $|AB|=a, |AD|=2a$ mamy $\displaystyle \vec{AC}=2\vec{AB}+\frac{1}{2}\vec{AD}$ oraz

$\displaystyle \cos\angle BCD=\frac{1}{4}$. Wykazač, $\dot{\mathrm{z}}\mathrm{e}$ na tym czworokącie $\mathrm{m}\mathrm{o}\dot{\mathrm{z}}$ na opisač okrąg. Obliczyč promień

tego okregu. Sporz$\Phi$dzič rysunek.

6. Podstawą ostroslupa jest trójkąt równoramienny $0$ kącie przy wierzchofku $2\alpha, \alpha<\pi/4,$

$\mathrm{i}$ podstawie $2a$. Dwie ściany boczne są przystajqcymi do siebie trójkątami podobny-

mi, ale nie przystającymi, do podstawy ostroslupa. Znalez/č cosinus kąta pfaskiego przy

wierzchołku trzeciej ściany bocznej oraz objętośč ostroslupa. Narysowač starannie siatkę

tego ostrosłupa dla $\displaystyle \alpha=\frac{\pi}{5}.$

Rozwiązania (rękopis) zadań $\mathrm{z}$ wybranego poziomu prosimy nadsyłaČ do 18 marca 2019 $\mathrm{r}.$

na adres:

Wydziaf Matematyki

Politechniki Wrocfawskiej,

Wybrzez $\mathrm{e}$ Wyspiańskiego 27,

$50-370$ WROCLAW.

Na kopercie prosimy $\underline{\mathrm{k}\mathrm{o}\mathrm{n}\mathrm{i}\mathrm{e}\mathrm{c}\mathrm{z}\mathrm{n}\mathrm{i}\mathrm{e}}$ zaznaczyč wybrany poziom! (np. poziom pod-

stawowy lub rozszerzony). Do rozwiazań nalez $\mathrm{y}$ dolączyč zaadresowaną do siebie

kopertę zwrotną $\mathrm{z}$ naklejonym znaczkiem, odpowiednim do wagi listu. Prace nie

spelniające podanych warunków nie będą poprawiane ani odsyłane.

Uwaga. Wysyłaj\S c nam rozwi\S zania zadań uczestnik Kursu udostępnia nam swoje dane osobowe,

które przetwarzamy wyłącznie $\mathrm{w}$ zakresie niezbednym do jego prowadzenia (odeslanie pracy, prowa-

dzenie statystyki). Szczególowe informacje $0$ przetwarzaniu przez nas danych osobowych są dostępne

na stronie internetowej Kursu.

Adres Internetowy Kursu: http://www.im.pwr.edu.pl/kurs



\end{document}