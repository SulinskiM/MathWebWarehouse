\documentclass[a4paper,12pt]{article}
\usepackage{latexsym}
\usepackage{amsmath}
\usepackage{amssymb}
\usepackage{graphicx}
\usepackage{wrapfig}
\pagestyle{plain}
\usepackage{fancybox}
\usepackage{bm}

\begin{document}

XLVIII

KORESPONDENCYJNY KURS

Z MATEMATYKI

marzec 2019 r.

PRACA KONTROLNA nr 7- POZIOM PODSTAWOWY

1. $\mathrm{W}$ pierwszej godzinie rowerzysta A jedzie $\mathrm{z}$ prędkościq 25 $\mathrm{k}\mathrm{m}/\mathrm{h}$, a $\mathrm{w}\mathrm{k}\mathrm{a}\dot{\mathrm{z}}$ dej kolejnej

godziniejedzie ze stafą prędkości$\Phi$ mniejszą $0$ 20\% $\mathrm{w}$ stosunku do prędkości $\mathrm{w}$ poprzedniej

godzinie. Natomiast rowerzysta $\mathrm{B}$ jedzie $\mathrm{w}$ pierwszej godzinie $\mathrm{z}$ prędkością 8 $\mathrm{k}\mathrm{m}/\mathrm{h},$

a $\mathrm{w}\mathrm{k}\mathrm{a}\dot{\mathrm{z}}$ dej kolejnej godzinie jedzie ze stałq prędkością większq $0$ 20\% $\mathrm{w}$ stosunku do

prędkości $\mathrm{w}$ poprzedniej godzinie. Obaj startują równocześnie $\mathrm{z}$ tego samego punktu.

Który $\mathrm{z}$ nich dotrze prędzej do celu lezącego $\mathrm{w}$ odleglości 100 km od punktu startu?

Po której godzinie jazdy odlegfośč między nimi $\mathrm{w}$ zaokrągleniu do pełnych kilometrów

będzie największa $\mathrm{i}$ ile będzie wynosič? Odpowiedzi uzasadnič bez stosowania obliczeń

przyblizonych.

2. $\mathrm{W}$ skarbonce jest 5 monet 5 zf $\mathrm{i}5$ monet 2 $\mathrm{z}\mathrm{f}$. Kuba wylosowaf ze skarbonki 6 monet.

Obliczyč prawdopodobieństwo tego, $\dot{\mathrm{z}}\mathrm{e}$ wystarczy mu pieniędzy na kupno ksiązki $\mathrm{w}$ cenie

20 $\mathrm{z}1.$

3. Rozwiązač nierównośč

$2\log_{2}(3-\sqrt{2^{x+1}-7})>x.$

4. Dla jakich wartości parametru $m$ liczby $x_{0}, y_{0}$, spełniające uklad równań

$\left\{\begin{array}{l}
x\\
3x
\end{array}\right.$

$+$

$+$

{\it my}

2{\it y}

$=2$

$=m$

są odpowiednio cosinusem $\mathrm{i}$ sinusem tego samego kąta $\alpha \in [0,\pi]$. Podač $x_{0} \mathrm{i} y_{0}$ dla

znalezionych wartości parametru $m.$

5. $\mathrm{W}$ ostrosfupie prawidfowym trójkątnym kąt pomiędzy ścianami bocznymi wynosi $2\alpha.$

Niech $P$ będzie spodkiem wysokości ściany bocznej opuszczonej na krawęd $\acute{\mathrm{z}}$ boczną.

Pfaszczyzna równolegfa do podstawy przechodząca przez $P$ dzieli ostrosfup na dwie czę-

ści, $\mathrm{z}$ których górna ma objetośč $V$. Obliczyč objętośč oraz krawędz/ podstawy ostrosłupa.

Podač dziedzinę kąta $\alpha.$

6. Kąty przy podstawie $AB$ trójkąta sq równe $\alpha$ oraz $2\alpha, \displaystyle \alpha<\frac{\pi}{4}$, a środkowa boku $AB$ ma

dlugośč $d$. Znalez/č dlugości boków trójk$\Phi$ta. Następnie podstawič do wyniku ogólnego

dane $d=\sqrt{11}$ oraz $\displaystyle \sin\alpha=\frac{\sqrt{2}}{4}\mathrm{i}$ wykonač obliczenia.
\end{document}
