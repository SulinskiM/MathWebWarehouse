\documentclass[a4paper,12pt]{article}
\usepackage{latexsym}
\usepackage{amsmath}
\usepackage{amssymb}
\usepackage{graphicx}
\usepackage{wrapfig}
\pagestyle{plain}
\usepackage{fancybox}
\usepackage{bm}

\begin{document}

XLVIII

KORESPONDENCYJNY KURS

Z MATEMATYKI

marzec 2019 r.

PRACA KONTROLNA $\mathrm{n}\mathrm{r} 7$- POZIOM ROZSZERZONY

l. Rozwiqzač nierównośč

$\sqrt{\sin 2x-\cos 2x+1}\leq 2\sin x.$

2. Ze zbioru $\{$1, 2, $3n\}, n\geq 1$, wylosowano bez zwracania dwie liczby. Obliczyč prawdo-

podobieństwo tego, $\dot{\mathrm{z}}\mathrm{e}$ suma otrzymanych liczb jest mniejsza od $4n\mathrm{i}$ co najmniej jedna

$\mathrm{z}$ nich jest większa od $n.$

3. Stosując zasadę indukcji matematycznej, udowodnič prawdziwośč wzoru

$1^{4}+2^{4}++n^{4}+\displaystyle \frac{1^{2}+2^{2}++n^{2}}{5}=\frac{n^{2}(n+1)^{2}(2n+1)}{10},$

$n\geq 1.$

4. Dana jest funkcja $f(x)=\displaystyle \frac{1}{3}x^{3}-\frac{4}{3}x$. Styczna do wykresu tej funkcji $\mathrm{w}$ punkcie $A(1,-1)$

przecina wykres $\mathrm{w}$ punkcie $B(x_{1},f(x_{1}))$, a styczna do jej wykresu $\mathrm{w}$ punkcie $B$ przecina

wykres $\mathrm{w}$ punkcie $C(x_{2},f(x_{2}))$. Znalez/č punkty $B \mathrm{i} C$ oraz obliczyč tangensy katów

trójkąta $\triangle ABC$. Sporządzič rysunek, dobierając odpowiednie skale na obu osiach.

5. $\mathrm{W}$ czworokącie ABCD $0$ bokach $|AB|=a, |AD|=2a$ mamy $\displaystyle \vec{AC}=2\vec{AB}+\frac{1}{2}\vec{AD}$ oraz

$\displaystyle \cos\angle BCD=\frac{1}{4}$. Wykazač, $\dot{\mathrm{z}}\mathrm{e}$ na tym czworokącie $\mathrm{m}\mathrm{o}\dot{\mathrm{z}}$ na opisač okrąg. Obliczyč promień

tego okregu. Sporz$\Phi$dzič rysunek.

6. Podstawą ostroslupa jest trójkąt równoramienny $0$ kącie przy wierzchofku $2\alpha, \alpha<\pi/4,$

$\mathrm{i}$ podstawie $2a$. Dwie ściany boczne są przystajqcymi do siebie trójkątami podobny-

mi, ale nie przystającymi, do podstawy ostroslupa. Znalez/č cosinus kąta pfaskiego przy

wierzchołku trzeciej ściany bocznej oraz objętośč ostroslupa. Narysowač starannie siatkę

tego ostrosłupa dla $\displaystyle \alpha=\frac{\pi}{5}.$

Rozwiązania (rękopis) zadań $\mathrm{z}$ wybranego poziomu prosimy nadsyłaČ do 18 marca 2019 $\mathrm{r}.$

na adres:

Wydziaf Matematyki

Politechniki Wrocfawskiej,

Wybrzez $\mathrm{e}$ Wyspiańskiego 27,

$50-370$ WROCLAW.

Na kopercie prosimy $\underline{\mathrm{k}\mathrm{o}\mathrm{n}\mathrm{i}\mathrm{e}\mathrm{c}\mathrm{z}\mathrm{n}\mathrm{i}\mathrm{e}}$ zaznaczyč wybrany poziom! (np. poziom pod-

stawowy lub rozszerzony). Do rozwiazań nalez $\mathrm{y}$ dolączyč zaadresowaną do siebie

kopertę zwrotną $\mathrm{z}$ naklejonym znaczkiem, odpowiednim do wagi listu. Prace nie

spelniające podanych warunków nie będą poprawiane ani odsyłane.

Uwaga. Wysyłaj\S c nam rozwi\S zania zadań uczestnik Kursu udostępnia nam swoje dane osobowe,

które przetwarzamy wyłącznie $\mathrm{w}$ zakresie niezbednym do jego prowadzenia (odeslanie pracy, prowa-

dzenie statystyki). Szczególowe informacje $0$ przetwarzaniu przez nas danych osobowych są dostępne

na stronie internetowej Kursu.

Adres Internetowy Kursu: http://www.im.pwr.edu.pl/kurs
\end{document}
