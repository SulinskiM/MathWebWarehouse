\documentclass[10pt]{article}
\usepackage[polish]{babel}
\usepackage[utf8]{inputenc}
\usepackage[T1]{fontenc}
\usepackage{amsmath}
\usepackage{amsfonts}
\usepackage{amssymb}
\usepackage[version=4]{mhchem}
\usepackage{stmaryrd}
\usepackage{hyperref}
\hypersetup{colorlinks=true, linkcolor=blue, filecolor=magenta, urlcolor=cyan,}
\urlstyle{same}

\title{PRACA KONTROLNA nr 7 - POZIOM PODSTAWOWY }

\author{}
\date{}


\begin{document}
\maketitle
\begin{enumerate}
  \item W pierwszej godzinie rowerzysta A jedzie z prędkością $25 \mathrm{~km} / \mathrm{h}$, a w każdej kolejnej godzinie jedzie ze stałą prędkością mniejszą o $20 \%$ w stosunku do prędkości w poprzedniej godzinie. Natomiast rowerzysta B jedzie w pierwszej godzinie z prędkością $8 \mathrm{~km} / \mathrm{h}$, a w każdej kolejnej godzinie jedzie ze stałą prędkością większą o $20 \%$ w stosunku do prędkości w poprzedniej godzinie. Obaj startują równocześnie z tego samego punktu. Który z nich dotrze prędzej do celu leżącego w odległości 100 km od punktu startu? Po której godzinie jazdy odległość między nimi w zaokrągleniu do pełnych kilometrów będzie największa i ile będzie wynosić? Odpowiedzi uzasadnić bez stosowania obliczeń przybliżonych.
  \item W skarbonce jest 5 monet $5 \mathrm{zł}$ i 5 monet 2 zł. Kuba wylosował ze skarbonki 6 monet. Obliczyć prawdopodobieństwo tego, że wystarczy mu pieniędzy na kupno książki w cenie $20 \mathrm{zł}$.
  \item Rozwiązać nierówność
\end{enumerate}

$$
2 \log _{2}\left(3-\sqrt{2^{x+1}-7}\right)>x
$$

\begin{enumerate}
  \setcounter{enumi}{3}
  \item Dla jakich wartości parametru $m$ liczby $x_{0}, y_{0}$, spełniające układ równań
\end{enumerate}

$$
\left\{\begin{array}{l}
x+m y=2 \\
3 x+2 y=m
\end{array}\right.
$$

są odpowiednio cosinusem i sinusem tego samego kąta $\alpha \in[0, \pi]$. Podać $x_{0}$ i $y_{0}$ dla znalezionych wartości parametru $m$.\\
5. W ostrosłupie prawidłowym trójkątnym kąt pomiędzy ścianami bocznymi wynosi $2 \alpha$. Niech $P$ będzie spodkiem wysokości ściany bocznej opuszczonej na krawędź boczną. Płaszczyzna równoległa do podstawy przechodząca przez $P$ dzieli ostrosłup na dwie części, z których górna ma objętość $V$. Obliczyć objętość oraz krawędź podstawy ostrosłupa. Podać dziedzinę kąta $\alpha$.\\
6. Kąty przy podstawie $A B$ trójkąta są równe $\alpha$ oraz $2 \alpha, \alpha<\frac{\pi}{4}$, a środkowa boku $A B$ ma długość $d$. Znaleźć długości boków trójkąta. Następnie podstawić do wyniku ogólnego dane $d=\sqrt{11}$ oraz $\sin \alpha=\frac{\sqrt{2}}{4}$ i wykonać obliczenia.

\section*{PRACA KONTROLNA nr 7 - POZIOM ROZSZERZONY}
\begin{enumerate}
  \item Rozwiązać nierówność
\end{enumerate}

$$
\sqrt{\sin 2 x-\cos 2 x+1} \leqslant 2 \sin x .
$$

\begin{enumerate}
  \setcounter{enumi}{1}
  \item Ze zbioru $\{1,2, \ldots, 3 n\}, n \geqslant 1$, wylosowano bez zwracania dwie liczby. Obliczyć prawdopodobieństwo tego, że suma otrzymanych liczb jest mniejsza od $4 n$ i co najmniej jedna z nich jest większa od $n$.
  \item Stosując zasadę indukcji matematycznej, udowodnić prawdziwość wzoru
\end{enumerate}

$$
1^{4}+2^{4}+\ldots+n^{4}+\frac{1^{2}+2^{2}+\ldots+n^{2}}{5}=\frac{n^{2}(n+1)^{2}(2 n+1)}{10}, n \geqslant 1
$$

\begin{enumerate}
  \setcounter{enumi}{3}
  \item Dana jest funkcja $f(x)=\frac{1}{3} x^{3}-\frac{4}{3} x$. Styczna do wykresu tej funkcji w punkcie $A(1,-1)$ przecina wykres w punkcie $B\left(x_{1}, f\left(x_{1}\right)\right.$ ), a styczna do jej wykresu w punkcie $B$ przecina wykres w punkcie $C\left(x_{2}, f\left(x_{2}\right)\right)$. Znaleźć punkty $B$ i $C$ oraz obliczyć tangensy kątów trójkąta $\triangle A B C$. Sporządzić rysunek, dobierając odpowiednie skale na obu osiach.
  \item W czworokącie $A B C D$ o bokach $|A B|=a,|A D|=2 a$ mamy $\overrightarrow{A C}=2 \overrightarrow{A B}+\frac{1}{2} \overrightarrow{A D}$ oraz $\cos \angle B C D=\frac{1}{4}$. Wykazać, że na tym czworokącie można opisać okrąg. Obliczyć promień tego okręgu. Sporządzić rysunek.
  \item Podstawą ostrosłupa jest trójkąt równoramienny o kącie przy wierzchołku $2 \alpha, \alpha<\pi / 4$, i podstawie $2 a$. Dwie ściany boczne są przystającymi do siebie trójkątami podobnymi, ale nie przystającymi, do podstawy ostrosłupa. Znaleźć cosinus kąta płaskiego przy wierzchołku trzeciej ściany bocznej oraz objętość ostrosłupa. Narysować starannie siatkę tego ostrosłupa dla $\alpha=\frac{\pi}{5}$.
\end{enumerate}

Rozwiązania (rękopis) zadań z wybranego poziomu prosimy nadsyłać do 18 marca 2019 r. na adres:

Wydział Matematyki\\
Politechniki Wrocławskiej,\\
Wybrzeże Wyspiańskiego 27,\\
50-370 WROCEAW.\\
Na kopercie prosimy koniecznie zaznaczyć wybrany poziom! (np. poziom podstawowy lub rozszerzony). Do rozwiązań należy dołączyć zaadresowaną do siebie kopertę zwrotną z naklejonym znaczkiem, odpowiednim do wagi listu. Prace nie spełniające podanych warunków nie będą poprawiane ani odsyłane.\\
Uwaga. Wysyłając nam rozwiązania zadań uczestnik Kursu udostępnia nam swoje dane osobowe, które przetwarzamy wyłącznie w zakresie niezbędnym do jego prowadzenia (odesłanie pracy, prowadzenie statystyki). Szczegółowe informacje o przetwarzaniu przez nas danych osobowych są dostępne na stronie internetowej Kursu.\\
Adres Internetowy Kursu: \href{http://www.im.pwr.edu.pl/kurs}{http://www.im.pwr.edu.pl/kurs}


\end{document}