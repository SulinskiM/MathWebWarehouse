\documentclass[a4paper,12pt]{article}
\usepackage{latexsym}
\usepackage{amsmath}
\usepackage{amssymb}
\usepackage{graphicx}
\usepackage{wrapfig}
\pagestyle{plain}
\usepackage{fancybox}
\usepackage{bm}

\begin{document}

XLIX

KORESPONDENCYJNY KURS

Z MATEMATYKI

marzec 2020 r.

PRACA KONTROLNA nr 7- POZIOM PODSTAWOWY

l. Wyznaczyč $z$ jako funkcję zmiennej $y$, wiedząc, $\displaystyle \dot{\mathrm{z}}\mathrm{e}x=2\frac{\mathrm{l}}{1-\log_{2}z}$ oraz $y=2\displaystyle \frac{\mathrm{l}}{1-\log_{2}x}$

2. Pokazač, $\dot{\mathrm{z}}\mathrm{e}$ dla $\mathrm{k}\mathrm{a}\dot{\mathrm{z}}$ dej wartości parametru $\alpha \in [0,2\pi]$, dla której istnieje rozwiązanie

równania $x^{2}-2\cos\alpha\cdot x+\sin^{2}\alpha= 0$ suma kwadratów jego pierwiastków jest równa

przynajmniej l.

3. $\mathrm{W}$ zalezności od parametru rzeczywistego $k$ przedyskutowač liczbę rozwiązań ukfadu

równań 

Sporządzič ilustrację graficzną układu dla kilku charakterystycznych $k.$

4. Przekątna $BD$ równoległoboku ABCD jest prostopadla do boku $AD$, a $\mathrm{k}\mathrm{a}\mathrm{t}$ ostry te-

go równolegloboku jest równy kątowi między jego przekątnymi. Wyznaczyč stosunek

dlugości przekqtnych. Sporządzič rysunek.

5. Wyznaczyč zbiór punktów, $\mathrm{z}$ których odcinek $0$ końcach $A(2,0)\mathrm{i}B(1,\sqrt{2})$ jest widoczny

pod kątem $30^{\mathrm{o}}$ Sporządzič rysunek.

6. Podstawą graniastosłupa prostego $0$ wszystkich krawędziach równych $a$, jest romb $0$ kącie

ostrym $\alpha$. Graniastoslup przecięto $\mathrm{p}\mathrm{f}\mathrm{z}\Phi \mathrm{P}^{\mathrm{r}\mathrm{z}\mathrm{e}\mathrm{c}\mathrm{h}\mathrm{o}\mathrm{d}_{\mathrm{Z}}}\text{ą}^{\mathrm{C}}\Phi$ przechodzącą przez dfuzszą

przekqtną $AC$ podstawy dolnej $\mathrm{i}$ przeciwległy wierzchołek podstawy górnej. Wyznaczyč

cosinus $\mathrm{k}_{\Phi^{\mathrm{t}\mathrm{a}}}$ nachylenia tej płaszczyzny do plaszczyzny podstawy $\mathrm{i}$ pole otrzymanego

przekroju. Sporządzič rysunek.
\end{document}
