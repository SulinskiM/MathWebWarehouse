\documentclass[a4paper,12pt]{article}
\usepackage{latexsym}
\usepackage{amsmath}
\usepackage{amssymb}
\usepackage{graphicx}
\usepackage{wrapfig}
\pagestyle{plain}
\usepackage{fancybox}
\usepackage{bm}

\begin{document}

PRACA KONTROLNA nr 3- POZIOM ROZSZERZONY

l. Rozwiązač równanie $(\displaystyle \frac{1}{x})^{2-3\log_{2}x}=\frac{1}{2}x^{1+\log_{x}2}$

2. Dlajakich wartości parametru $m$ równanie $x^{3}+(m-2)x^{2}+(2-m-m^{2})x-(1-m^{2})=0$

ma trzy rózne pierwiastki, których suma kwadratów nie przekracza 5?

3. Czworokąt wypukly ABCD, $\mathrm{w}$ którym $AB=1, BC=2, CD=4, DA=3$ jest wpisany

$\mathrm{w}$ okrąg. Obliczyč promień $R$ tego okręgu. Sprawdzič, czy $\mathrm{w}$ czworokąt ten $\mathrm{m}\mathrm{o}\dot{\mathrm{z}}$ na wpisač

okrąg. $\mathrm{J}\mathrm{e}\dot{\mathrm{z}}$ eli $\mathrm{t}\mathrm{a}\mathrm{k}$, to obliczyč promień $r$ tego okręgu. Sporządzič rysunek.

4. Podstawa graniastosłupa prostego $0$ wszystkich krawedziach równych jest romb $0$ kącie

ostrym $\displaystyle \frac{\pi}{3}$. Graniastosfup ten przecięto dwiema płaszczyznami: plaszczyzną przechodzącą

przez bok $AB$ podstawy dolnej $\mathrm{i}$ wierzchołek $C'$ oraz płaszczyzną przechodzącą przez bok

$AD$ podstawy dolnej $\mathrm{i}$ ten sam wierzchofek $C'$. Wyznaczyč kąt dwuścienny między tymi

plaszczyznami oraz stosunek objętości brył, na jakie zostal podzielony graniastoslup.

Sporzadzič rysunek.

5. $\mathrm{W}$ zalezności od parametru rzeczywistego $p$ przedyskutowač liczbę rozwiązań ukfadu

równań

$\left\{\begin{array}{ll}
x^{4}+y^{4}+2x^{2}y^{2}-4x^{2} & =0,\\
x^{2}+y^{2}-2\sqrt{3}y & =p
\end{array}\right.$

Sporządzič ilustrację graficzną układu dla kilku charakterystycznych $p.$

6. Wykorzystując wzór Newtona $\mathrm{i}\mathrm{o}\mathrm{b}\mathrm{l}\mathrm{i}\mathrm{c}\mathrm{z}\mathrm{a}\mathrm{j}_{\Phi}\mathrm{c}$ pochodną wielomianu $w(x)=(1-x)^{n}$, wy-

kazač, $\dot{\mathrm{z}}\mathrm{e}$ dla dowolnego $n\in 1\mathrm{N}, n\geq 2$ zachodzi równośč

$\left(\begin{array}{l}
n\\
1
\end{array}\right)-2\left(\begin{array}{l}
n\\
2
\end{array}\right)+3\left(\begin{array}{l}
n\\
3
\end{array}\right)-4\left(\begin{array}{l}
n\\
4
\end{array}\right)+\ldots+(-1)^{n-1}n\left(\begin{array}{l}
n\\
n
\end{array}\right)=0$

Wywnioskowač stąd, $\dot{\mathrm{z}}$ ejezeli liczby $a_{1}, a_{2}, \ldots, a_{n}, a_{n+1}$ tworzą ciąg arytmetyczny, to dla

dowolnego $n\in 1\mathrm{N}$ zachodzi równośč

$a_{1}-\left(\begin{array}{l}
n\\
1
\end{array}\right)a_{2}+\left(\begin{array}{l}
n\\
2
\end{array}\right)a_{3}-\left(\begin{array}{l}
n\\
3
\end{array}\right)a_{4}+\ldots+(-1)^{n}\left(\begin{array}{l}
n\\
n
\end{array}\right)a_{n+1}=0$

Rozwiązania (rękopis) zadań z wybranego poziomu prosimy nadsylač do

2019r. na adres:

18 1istopada

Wydziaf Matematyki

Politechnika Wrocfawska

Wybrzez $\mathrm{e}$ Wyspiańskiego 27

$50-370$ WROCLAW.

Na kopercie prosimy $\underline{\mathrm{k}\mathrm{o}\mathrm{n}\mathrm{i}\mathrm{e}\mathrm{c}\mathrm{z}\mathrm{n}\mathrm{i}\mathrm{e}}$ zaznaczyč wybrany poziom! (np. poziom podsta-

wowy lub rozszerzony). Do rozwiązań nalez $\mathrm{y}$ dołączyč zaadresowana do siebie koperte

zwrotną $\mathrm{z}$ naklejonym znaczkiem, odpowiednim do formatu listu. Polecamy stosowanie

kopert formatu C5 $(160\mathrm{x}230\mathrm{m}\mathrm{m})$ ze znaczkiem $0$ wartości 3,30 zł. Na $\mathrm{k}\mathrm{a}\dot{\mathrm{z}}$ dą wiekszą

kopertę nalez $\mathrm{y}$ nakleič drozszy znaczek. Prace niespełniające podanych warunków nie

będą poprawiane ani odsyłane.

Uwaga. Wysyłając nam rozwi\S zania zadań uczestnik Kursu udostępnia Politechnice Wroclawskiej

swoje dane osobowe, które przetwarzamy wyłącznie $\mathrm{w}$ zakresie niezbędnym do jego prowadzenia

(odesfanie zadań, prowadzenie statystyki). Szczególowe informacje $0$ przetwarzaniu przez nas danych

osobowych są dostępne na stronie internetowej Kursu.

Adres internetowy Kursu: http://www.im.pwr.edu.pl/kurs
\end{document}
