\documentclass[a4paper,12pt]{article}
\usepackage{latexsym}
\usepackage{amsmath}
\usepackage{amssymb}
\usepackage{graphicx}
\usepackage{wrapfig}
\pagestyle{plain}
\usepackage{fancybox}
\usepackage{bm}

\begin{document}

L KORESPONDENCYJNY KURS

Z MATEMATYKI

marzec 2021 r.

PRACA KONTROLNA nr 7- POZIOM PODSTAWOWY

l. Wykaz$\cdot, \dot{\mathrm{z}}\mathrm{e}$ dla dowolnych liczb $a, b$ róznych od zera, posiadających ten sam znak, praw-

dziwa jest nierównośč

-{\it ab}$+-\alpha${\it b} $> -85^{\cdot}$

2. Wyznacz $\mathrm{t}\mathrm{g}\alpha$, wiedząc, $\dot{\mathrm{z}}\mathrm{e}\alpha$ jest kqtem ostrym spełniającym równanie

$\displaystyle \frac{2\sin\alpha+3\cos\alpha}{\cos\alpha}=2$ ctg $\alpha.$

3. Spośród l0 biafych $\mathrm{i}2$ czarnych kul losujemy bez zwracania $m\mathrm{k}\mathrm{u}\mathrm{l}$. Jaka jest najmniejsza

liczba $m$, dla której prawdopodobieństwo, $\dot{\mathrm{z}}\mathrm{e}$ wśród wylosowanych kul jest przynajmniej

jedna czarna, przekracza $\displaystyle \frac{1}{2}$?

4. Wielomian $W(x)=2x^{3}+px^{2}+qx-2$ ma współczynniki całkowite $\mathrm{i}$ pierwiastek całkowity,

a reszta $\mathrm{z}$ jego dzielenia przez dwumian $x-2$ jest równa 10. D1a jakich $x$ przyjmuje on

wartości dodatnie?

5. Odcinek $0$ końcach $A(1,0) \mathrm{i}B(2,1)$ jest podstawą trójkąta równoramiennego, którego

trzeci wierzchofek $\mathrm{l}\mathrm{e}\dot{\mathrm{z}}\mathrm{y}$ na prostej $y=2x+1$. Podaj równania prostych $\mathrm{z}\mathrm{a}\mathrm{w}\mathrm{i}\mathrm{e}\mathrm{r}\mathrm{a}\mathrm{j}_{\Phi}$cych

ramiona tego trójkąta $\mathrm{i}$ oblicz jego pole.

6. Na bokach $AC\mathrm{i}BC$ trójkqta równoramiennego $ABC$ obrano punkty $M\mathrm{i}N$, których

rzutami prostokątnymi na podstawę AB $\mathrm{s}\Phi$ punkty $S, T$. Wykaz$\cdot, \dot{\mathrm{z}}\mathrm{e}|AB|=2|ST|$ wtedy

$\mathrm{i}$ tylko wtedy, gdy $|AM|=|CN|.$
\end{document}
