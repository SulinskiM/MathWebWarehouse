\documentclass[a4paper,12pt]{article}
\usepackage{latexsym}
\usepackage{amsmath}
\usepackage{amssymb}
\usepackage{graphicx}
\usepackage{wrapfig}
\pagestyle{plain}
\usepackage{fancybox}
\usepackage{bm}

\begin{document}

PRACA KONTROLNA $\mathrm{n}\mathrm{r} 7-$ POZIOM ROZSZERZONY

l. Wykaz, $\dot{\mathrm{z}}\mathrm{e}$ dla dowolnych liczb rzeczywistych $a, b$ równośč $a^{3}-2b^{3}=ab(a+b)$ zachodzi

wtedy $\mathrm{i}$ tylko wtedy, gdy $a=2b.$

2. Rozwiąz równanie $\displaystyle \cos x-\sin x=\frac{\cos 2x}{\sin 2x+1}$

3. Liczba dwuelementowych podzbiorów zbioru $A$ jest 3 razy większa $\mathrm{n}\mathrm{i}\dot{\mathrm{z}}$ liczba dwuele-

mentowych podzbiorów zbioru $B$. Liczba dwuelementowych podzbiorów zbioru $A$ nie

zawierających ustalonego elementu $a\in A$ jest sumą liczby dwuelementowych podzbio-

rów zbioru $B\mathrm{i}$ liczby dwuelementowych podzbiorów zbioru $B$, do którego dodano jeden

element. Ile elementów ma $\mathrm{k}\mathrm{a}\dot{\mathrm{z}}\mathrm{d}\mathrm{y}\mathrm{z}$ tych zbiorów? Ile $\mathrm{k}\mathrm{a}\dot{\mathrm{z}}\mathrm{d}\mathrm{y}\mathrm{z}$ tych zbiorów ma podzbiorów

trzyelementowych?

4. Reszta $\mathrm{z}$ dzielenia wielomianu $W(x)=x^{4}+x^{3}+px^{2}+qx+2$ przez $(x^{2}+1)$ jest równa

$(-2x+6)$. Rozwiąz nierównośč $W(x)>0.$

5. Dwa boki trójkąta zawierają $\mathrm{s}\mathrm{i}\mathrm{e}\mathrm{w}$ prostych $2x-y=0\mathrm{i}x-2y=0$, a proste zawierające

jego wysokości przecinają się $\mathrm{w}$ punkcie $S(5,-2)$. Wyznacz wierzcholki trójk$\Phi$ta $\mathrm{i}$ oblicz

jego pole.

6. Wyznacz równanie krzywej będącej zbiorem środków okręgów, które sq styczne do prostej

$x=2\mathrm{i}$ do okręgu $x^{2}+2x+y^{2}-2y+1=0.$

Rozwiązania (rękopis) zadań z wybranego poziomu prosimy nadsyłač do 20.03.2021r. na

adres:

Wydziaf Matematyki

Politechnika Wrocfawska

Wybrzez $\mathrm{e}$ Wyspiańskiego 27

$50-370$ WROCLAW.

Na kopercie prosimy $\underline{\mathrm{k}\mathrm{o}\mathrm{n}\mathrm{i}\mathrm{e}\mathrm{c}\mathrm{z}\mathrm{n}\mathrm{i}\mathrm{e}}$ zaznaczyč wybrany poziom! (np. poziom podsta-

wowy lub rozszerzony). Do rozwiązań nalez $\mathrm{y}$ dołączyč zaadresowana do siebie koperte

zwrotną $\mathrm{z}$ naklejonym znaczkiem, odpowiednim do formatu listu. Polecamy stosowanie

kopert formatu C5 $(160\mathrm{x}230\mathrm{m}\mathrm{m})$ ze znaczkiem $0$ wartości 3,30 zł. Na $\mathrm{k}\mathrm{a}\dot{\mathrm{z}}$ dą wiekszą

kopertę nalez $\mathrm{y}$ nakleič drozszy znaczek. Prace niespełniające podanych warunków nie

będą poprawiane ani odsyłane.

Uwaga. Wysyłając nam rozwi\S zania zadań uczestnik Kursu udostępnia Politechnice Wroclawskiej

swoje dane osobowe, które przetwarzamy wyłącznie $\mathrm{w}$ zakresie niezbednym do jego prowadzenia

(odesfanie zadań, prowadzenie statystyki). Szczegófowe informacje $0$ przetwarzaniu przez nas danych

osobowych sa dostępne na stronie internetowej Kursu.

Adres internetowy Kursu: http://www.im.pwr.edu.pl/kurs
\end{document}
