\documentclass[a4paper,12pt]{article}
\usepackage{latexsym}
\usepackage{amsmath}
\usepackage{amssymb}
\usepackage{graphicx}
\usepackage{wrapfig}
\pagestyle{plain}
\usepackage{fancybox}
\usepackage{bm}

\begin{document}

LI KORESPONDENCYJNY KURS

Z MATEMATYKI

marzec 2022 r.

PRACA KONTROLNA nr 7- POZIOM PODSTAWOWY

l. Grupa przyjaciól postanowiła kupič wspólnie ciekawą grę komputerową za 1920 z1otych.

Gdy zgfosifo sięjeszcze czterech chętnych do korzystania $\mathrm{z}$ tego oprogramowania, okazało

się, $\dot{\mathrm{z}}\mathrm{e}$, przy równym podziale kosztów, $\mathrm{k}\mathrm{a}\dot{\mathrm{z}}\mathrm{d}\mathrm{y}$ będzie mógf zaplacič 80 zfotych mniej. I1e

osób będzie korzystało $\mathrm{z}$ tej gry $\mathrm{i}$ ile $\mathrm{k}\mathrm{a}\dot{\mathrm{z}}\mathrm{d}\mathrm{y}\mathrm{z}$ nich musi za $\mathrm{n}\mathrm{i}\mathrm{a}$ zapłacič?

2. Liczby a, b, c dają przy dzieleniu przez 7 reszty (odpowiednio) - l, 2, 3.

suma kwadratów tych liczb jest podzielna przez 7.

Wykaz, $\dot{\mathrm{z}}\mathrm{e}$

3. Dla jakiego parametru $m$ pierwiastkiem równania

$x^{2}+(2m+1)x+m+4=0$

jest liczba $(-2)$ ? Dla znalezionego $m$ wyznacz drugi pierwiastek tego równania $\mathrm{i}\mathrm{s}$prawd $\acute{\mathrm{z}},$

dlajakich argumentów otrzymana funkcja kwadratowa $f(x)=x^{2}+(2m+1)x+m+4$

spełnia nierównośč

$2f(x)>1+\sqrt{2}.$

4. Oblicz wartośč wyrazeń

$\displaystyle \alpha=\frac{\sin 45^{\mathrm{o}}\cos 15^{\mathrm{o}}-\cos 45^{\mathrm{o}}\sin 15^{\mathrm{o}}}{\sin^{2}20^{\mathrm{o}}+\sin^{2}70^{\mathrm{o}}},$

{\it b}$=$ -ssiinn 7205oo ccooss 7150oo $+$-ccooss 2705oo ssiinn 7105oo.

Wyznacz stosunek promieni okregów wpisanego $\mathrm{i}$ opisanego na trójkącie prostokątnym,

którego przyprostokątne mają dlugości a $\mathrm{i}b.$

5. Punkty $A(1,0), B(5,2), C(3,3) \mathrm{s}\Phi$ trzema kolejnymi wierzchofkami trapezu prostokąt-

nego, $\mathrm{w}$ którym $AB||CD$. Wyznacz współrzędne wierzcholka $D$ oraz równania przekąt-

nych trapezu. $\mathrm{W}$ jakim stosunku $\mathrm{k}\mathrm{a}\dot{\mathrm{z}}$ da $\mathrm{z}$ tych przekątnych dzieli pole trapezu?

6. Krawędz/ boczna ostrosłupa prawidfowego trójkątnego jest dwa razy dłuzsza $\mathrm{n}\mathrm{i}\dot{\mathrm{z}}$ kra-

$\mathrm{w}\mathrm{e}\mathrm{d}\acute{\mathrm{z}}$ podstawy. Oblicz objetośč ostrosłupa $\mathrm{i}$ cosinus kąta nachylenia ściany bocznej do

podstawy, $\mathrm{w}\mathrm{i}\mathrm{e}\mathrm{d}\mathrm{z}\Phi^{\mathrm{C}}, \dot{\mathrm{z}}\mathrm{e}$ suma dlugości wszystkich jego krawędzi jest równa 18.
\end{document}
