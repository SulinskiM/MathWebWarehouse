\documentclass[a4paper,12pt]{article}
\usepackage{latexsym}
\usepackage{amsmath}
\usepackage{amssymb}
\usepackage{graphicx}
\usepackage{wrapfig}
\pagestyle{plain}
\usepackage{fancybox}
\usepackage{bm}

\begin{document}

PRACA KONTROLNA $\mathrm{n}\mathrm{r} 7-$ POZIOM ROZSZERZONY

l. Dlajakich wartości parametru $a$ równanie $4-|x-1|=(a+2)^{2}$ ma dwa rózne rozwiązania?

2. Wykorzystując dwumian Newtona, uzasadnij, $\dot{\mathrm{z}}\mathrm{e}$ liczba $11^{2k}-9^{2k}$ jest podzielna przez

100 dla dowolnej liczby naturalnej $k$ podzielnej przez 5.

3. Wykaz$\cdot, \dot{\mathrm{z}}\mathrm{e}\mathrm{w}$ dowolnym trójk$\Phi$cie prostokątnym wartośč bezwzględna róznicy tangensów

kątów ostrych jest dwa razy większa $\mathrm{n}\mathrm{i}\dot{\mathrm{z}}$ wartośč bezwzględna tangensa kąta, jaki tworzą

wysokośč $\mathrm{i}$ środkowa poprowadzone $\mathrm{z}$ wierzchołka kąta prostego.

4. Dany jest trapez prostokątny $0$ podstawach długości $a\mathrm{i}b$ oraz wysokości $2h$. Wykaz$\cdot,$

$\dot{\mathrm{z}}\mathrm{e}\mathrm{j}\mathrm{e}\dot{\mathrm{z}}$ eli $h^{2}=ab,$ to dłuzsze ramię trapezu jest równe $\alpha+b$, a okrąg, którego jest ono

średnicą, jest styczny do drugiego ramienia.

5. Narysuj wykres funkcji

$ f(x)=1-\displaystyle \frac{x}{x+2}+(\frac{x}{x+2})^{2}-(\frac{x}{x+2})^{3}+\ldots$

która jest sumą nieskończonego szeregu geometrycznego $\mathrm{i}$ wyznacz równanie prostej

stycznej do wykresu prostopadłej do prostej $2x-y=0. \mathrm{S}_{\mathrm{P}^{\mathrm{o}\mathrm{r}\mathrm{Z}\otimes}}\mathrm{d}\acute{\mathrm{z}}$ staranny rysunek.

6. Podstawą ostrosfupa jest trapez $0$ obwodzie 32, którego jedna podstawa jest trzy razy

dluzsza $\mathrm{n}\mathrm{i}\dot{\mathrm{z}}$ druga. Wszystkie krawedzie boczne ostroslupa są nachylone do podstawy

pod $\mathrm{k}_{\Phi}\mathrm{t}\mathrm{e}\mathrm{m}60^{\mathrm{o}}$ Oblicz objętośč ostrosfupa, wiedząc, $\dot{\mathrm{z}}\mathrm{e}\mathrm{w}$ jego podstawę $\mathrm{m}\mathrm{o}\dot{\mathrm{z}}$ na wpisač

okrąg.

Rozwiązania (rękopis) zadań z wybranego poziomu prosimy nadsyłač do

2022r. na adres:

20 marca

Wydziaf Matematyki

Politechnika Wrocfawska

Wybrzez $\mathrm{e}$ Wyspiańskiego 27

$50-370$ WROCLAW,

lub elektronicznie, za pośrednictwem portalu talent. $\mathrm{p}\mathrm{w}\mathrm{r}$. edu. pl

Na kopercie prosimy $\underline{\mathrm{k}\mathrm{o}\mathrm{n}\mathrm{i}\mathrm{e}\mathrm{c}\mathrm{z}\mathrm{n}\mathrm{i}\mathrm{e}}$ zaznaczyč wybrany poziom! (np. poziom podsta-

wowy lub rozszerzony). Do rozwiązań nalez $\mathrm{y}$ dołączyč zaadresowaną do siebie koperte

zwrotną $\mathrm{z}$ naklejonym znaczkiem, odpowiednim do formatu listu. Polecamy stosowanie

kopert formatu C5 $(160\mathrm{x}230\mathrm{m}\mathrm{m})$ ze znaczkiem $0$ wartości 3,30 zł. Na $\mathrm{k}\mathrm{a}\dot{\mathrm{z}}$ dą wiekszą

kopertę nalez $\mathrm{y}$ nakleič $\mathrm{d}\mathrm{r}\mathrm{o}\dot{\mathrm{z}}$ szy znaczek. Prace niespełniające podanych warunków nie

będą poprawiane ani odsyłane.

Uwaga. Wysylajqc nam rozwiazania zadań uczestnik Kursu udostępnia Politechnice Wroclawskiej

swoje dane osobowe, które przetwarzamy wyłącznie $\mathrm{w}$ zakresie niezbędnym do jego prowadzenia

(odesfanie zadań, prowadzenie statystyki). Szczegófowe informacje $0$ przetwarzaniu przez nas danych

osobowych sa dostępne na stronie internetowej Kursu.

Adres internetowy Kursu: http: //www. im. pwr. edu. pl/kurs
\end{document}
