\documentclass[a4paper,12pt]{article}
\usepackage{latexsym}
\usepackage{amsmath}
\usepackage{amssymb}
\usepackage{graphicx}
\usepackage{wrapfig}
\pagestyle{plain}
\usepackage{fancybox}
\usepackage{bm}

\begin{document}

LII

KORESPONDENCYJNY KURS

Z MATEMATYKI

marzec 2023 r.

PRACA KONTROLNA $\mathrm{n}\mathrm{r} 7-$ POZIOM PODSTAWOWY

l. Wielomian $W(x) =x^{3}-(k+m)x^{2}-(k-m)x+3$ jest podzielny przez dwumian $(x-1),$

a suma jego współczynników przy parzystych potęgach zmiennej $x$ jest równa sumie

wspófczynników przy nieparzystych potegach zmiennej. Rozwiqz nierównośč

$W(x)\leq x^{2}-1.$

2. Rozwia $\dot{\mathrm{z}}$ algebraicznie układ równań

tację geometryczną.

$\left\{\begin{array}{l}
|y|=2-x^{2},\\
x^{2}+y^{2}=2
\end{array}\right.$

a następnie podaj jego interpre-

3. $\mathrm{W}$ przedziale $[0,2\pi]$ określ liczbę rozwiązań równania

$\cos x$. ctg $x-\sin x=a\cos 2x,$

$\mathrm{w}$ zalezności od parametru $a.$

4. Niech $P(k)$ oznacza pole trójkąta ograniczonego prostą $y=kx\mathrm{i}$ wykresem funkcji

$f(x)=4-2|x|.$

Wyznacz $\mathrm{n}\mathrm{a}\mathrm{j}\mathrm{m}\mathrm{n}\mathrm{i}\mathrm{e}\mathrm{j}\mathrm{s}\mathrm{z}\Phi$ wartośč $P(k).$

5. Punkty $A(0,0)\mathrm{i}B(4,3)$ są wierzchołkami rombu $0$ kącie ostrym $45^{\mathrm{o}}$, który zawarty jest

$\mathrm{w}$ pierwszej čwiartce ukfadu wspólrzędnych. Wyznacz współrzędne jego wierzcholków.

Podaj równanie okręgu wpisanego $\mathrm{w}$ ten romb. Ile jest wszystkich rombów $0$ boku $AB$

$\mathrm{i}$ kącie ostrym $45^{\mathrm{o}}$? Oblicz objętośč bryły otrzymanej przez obrót rombu wokół jego

boku.

6. $\mathrm{W}$ ostroslupie prawidlowym czworokątnym środek podstawy jest odlegly $\mathrm{o}d$ od krawędzi

bocznej a kąt między sąsiednimi ścianami bocznymi ostroslupa jest równy $ 2\alpha$. Oblicz

objętośč ostrosfupa.




PRACA KONTROLNA $\mathrm{n}\mathrm{r} 7-$ POZIOM ROZSZERZONY

l. Dla jakiego parametru $m$ równanie

$mx^{3}-(2m+1)x^{2}+(2-3m)x+3=0$

ma trzy rózne pierwiastki, które są kolejnymi wyrazami $\mathrm{c}\mathrm{i}_{\Phi \mathrm{g}}\mathrm{u}$ arytmetycznego?

2. Rozwiąz równanie

$\displaystyle \frac{1+\mathrm{t}\mathrm{g}x+\mathrm{t}\mathrm{g}^{2}x+\mathrm{t}\mathrm{g}^{3}x+\ldots+\mathrm{t}\mathrm{g}^{n}x+}{1-\mathrm{t}\mathrm{g}x+\mathrm{t}\mathrm{g}^{2}x-\mathrm{t}\mathrm{g}^{3}x+\ldots+(-1)^{n}\mathrm{t}\mathrm{g}^{n}x+}=1+\sin 2x.$

3. Narysuj $\mathrm{w}$ prostokątnym ukfadzie wspófrzędnych zbiór punktów spefniających warunek

$\log_{(x-y)}(x+y)\leq 1.$

4. Podaj równanie prostej $l$ stycznej do wykresu funkcji $f(x)=\displaystyle \frac{3x-2}{(x-1)^{2}} \mathrm{w}$ punkcie jego

przecięcia $\mathrm{z}\mathrm{o}\mathrm{s}\mathrm{i}_{\Phi}Oy\mathrm{i}$ wyznacz równania wszystkich stycznych do wykresu równoleglych

do $l$. Oblicz odleglośč między otrzymanymi prostymi. Sporząd $\acute{\mathrm{z}}$ staranny wykres funkcji

wraz $\mathrm{z}$ otrzymanymi stycznymi.

5. Ostrosłup prawidłowy czworokątny przecięto plaszczyzną przechodzącą przez przekqtną

podstawy $\mathrm{i}$ środek przeciwleglej krawedzi bocznej. Płaszczyzna ta jest nachylona do

plaszczyzny podstawy pod kątem $\alpha$. Wyznacz kąt między ścianami bocznymi.

6. Odcinek $0$ końcach $A(0,0)\mathrm{i}B(8,6)$ jest dłuzszą podstawą trapezu prostokątnego opisa-

nego na okręgu. Wyznacz współrzędne pozostafych wierzcholków trapezu, wiedzqc, $\dot{\mathrm{z}}\mathrm{e}$

bok $CD$ jest dwa razy krótszy od boku $AB$. Podaj równanie okręgu wpisanego $\mathrm{w}$ ten

trapez. Oblicz objętośč bryly otrzymanej przez obrót trapezu wokół ramienia $BC.$

$\mathrm{R}\mathrm{o}\mathrm{z}\mathrm{w}\mathrm{i}_{\Phi}$zania (rękopis) zadań $\mathrm{z}$ wybranego poziomu prosimy nadsyfač do $20.03.2023\mathrm{r}$. na

adres:

Wydziaf Matematyki

Politechnika Wrocfawska

Wybrzez $\mathrm{e}$ Wyspiańskiego 27

$50-370$ WROCLAW,

$\mathrm{l}\mathrm{u}\mathrm{b}$ elektronicznie, za poŚrednictwem portalu talent. $\mathrm{p}\mathrm{w}\mathrm{r}$. edu. pl

Na kopercie prosimy $\underline{\mathrm{k}\mathrm{o}\mathrm{n}\mathrm{i}\mathrm{e}\mathrm{c}\mathrm{z}\mathrm{n}\mathrm{i}\mathrm{e}}$ zaznaczyč wybrany poziom! (np. poziom podsta-

wowy lub rozszerzony). Do rozwiązań nalez $\mathrm{y}$ dołączyč zaadresowana do siebie koperte

zwrotną $\mathrm{z}$ naklejonym znaczkiem, odpowiednim do formatu listu. Prace niespełniające

podanych warunków nie będą poprawiane ani odsyłane.

Uwaga. Wysylając nam rozwiazania zadań uczestnik Kursu udostępnia Politechnice Wroclawskiej

swoje dane osobowe, które przetwarzamy wyłącznie $\mathrm{w}$ zakresie niezbędnym do jego prowadzenia

(odesłanie zadań, prowadzenie statystyki). Szczegółowe informacje $0$ przetwarzaniu przez nas danych

osobowych są dostępne na stronie internetowej Kursu.

Adres internetowy Kursu: http: //www. im. pwr. edu. pl/kurs



\end{document}