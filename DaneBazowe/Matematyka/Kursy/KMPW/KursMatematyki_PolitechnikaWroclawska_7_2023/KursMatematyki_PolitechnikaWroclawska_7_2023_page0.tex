\documentclass[a4paper,12pt]{article}
\usepackage{latexsym}
\usepackage{amsmath}
\usepackage{amssymb}
\usepackage{graphicx}
\usepackage{wrapfig}
\pagestyle{plain}
\usepackage{fancybox}
\usepackage{bm}

\begin{document}

LII

KORESPONDENCYJNY KURS

Z MATEMATYKI

marzec 2023 r.

PRACA KONTROLNA $\mathrm{n}\mathrm{r} 7-$ POZIOM PODSTAWOWY

l. Wielomian $W(x) =x^{3}-(k+m)x^{2}-(k-m)x+3$ jest podzielny przez dwumian $(x-1),$

a suma jego współczynników przy parzystych potęgach zmiennej $x$ jest równa sumie

wspófczynników przy nieparzystych potegach zmiennej. Rozwiqz nierównośč

$W(x)\leq x^{2}-1.$

2. Rozwia $\dot{\mathrm{z}}$ algebraicznie układ równań

tację geometryczną.

$\left\{\begin{array}{l}
|y|=2-x^{2},\\
x^{2}+y^{2}=2
\end{array}\right.$

a następnie podaj jego interpre-

3. $\mathrm{W}$ przedziale $[0,2\pi]$ określ liczbę rozwiązań równania

$\cos x$. ctg $x-\sin x=a\cos 2x,$

$\mathrm{w}$ zalezności od parametru $a.$

4. Niech $P(k)$ oznacza pole trójkąta ograniczonego prostą $y=kx\mathrm{i}$ wykresem funkcji

$f(x)=4-2|x|.$

Wyznacz $\mathrm{n}\mathrm{a}\mathrm{j}\mathrm{m}\mathrm{n}\mathrm{i}\mathrm{e}\mathrm{j}\mathrm{s}\mathrm{z}\Phi$ wartośč $P(k).$

5. Punkty $A(0,0)\mathrm{i}B(4,3)$ są wierzchołkami rombu $0$ kącie ostrym $45^{\mathrm{o}}$, który zawarty jest

$\mathrm{w}$ pierwszej čwiartce ukfadu wspólrzędnych. Wyznacz współrzędne jego wierzcholków.

Podaj równanie okręgu wpisanego $\mathrm{w}$ ten romb. Ile jest wszystkich rombów $0$ boku $AB$

$\mathrm{i}$ kącie ostrym $45^{\mathrm{o}}$? Oblicz objętośč bryły otrzymanej przez obrót rombu wokół jego

boku.

6. $\mathrm{W}$ ostroslupie prawidlowym czworokątnym środek podstawy jest odlegly $\mathrm{o}d$ od krawędzi

bocznej a kąt między sąsiednimi ścianami bocznymi ostroslupa jest równy $ 2\alpha$. Oblicz

objętośč ostrosfupa.
\end{document}
