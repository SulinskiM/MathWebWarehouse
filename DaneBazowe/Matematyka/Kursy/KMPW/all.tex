\documentclass[a4paper,12pt]{article}
\usepackage{latexsym}
\usepackage{amsmath}
\usepackage{amssymb}
\usepackage{graphicx}
\usepackage{wrapfig}
\pagestyle{plain}
\usepackage{fancybox}
\usepackage{bm}

\begin{document}

KORESPONDENCYJNY KURS PRZYGOTOWAWCZY Z

MATEMATYKI

PRACA KONTROLNA nr l

$\mathrm{p}\mathrm{a}\acute{\mathrm{z}}$dziernik 1999 $\mathrm{r}$

l. Stop składa się $\mathrm{z}$ 40\% srebra próby 0,6, 30\% srebra próby 0,7 oraz l kg srebra próby 0,8.

Jaka jest waga $\mathrm{i}$ jaka jest próba tego stopu?

2. Rozwiązač równanie

$3^{x}+1+3^{-x}+\ldots=4,$

którego lewa strona jest $\mathrm{s}\mathrm{u}\mathrm{m}\Phi$ nieskończonego ciągu geometrycznego.

3. $\mathrm{W}$ trójkącie $ABC$ znane są wierzcholki $A(0,0)$ oraz $B(4,-1)$. Wiadomo, $\dot{\mathrm{z}}\mathrm{e}\mathrm{w}$ punkcie

$H(3,2)$ przecinają się proste zawierające wysokości tego trójkąta. Wyznaczyč wspólrzędne

wierzchofka $C$. Wykonač odpowiedni rysunek.

4. Rozwiqzač równanie

$\cos 4x=\sin 3x.$

5. Wykonač staranny wykres funkcji

$f(x)=|\log_{2}(x-2)^{2}|.$

6. Rozwiązač nierównośč

$\displaystyle \frac{1}{x^{2}}\geq\frac{1}{x+6}.$

7. $\mathrm{W}$ ostrosłupie prawidłowym sześciokatnym krawęd $\acute{\mathrm{z}}$ podstawy ma długośč $p$, a krawędz/

boczna dfugośč $2p$. Obliczyč cosinus $\mathrm{k}_{\Phi^{\mathrm{t}\mathrm{a}}}$ dwuściennego między sąsiednimi ścianami bocz-

nymi tego ostrosłupa.

8. Wyznaczyč równania wszystkich prostych stycznych do wykresu funkcji $y=\displaystyle \frac{2x+10}{x+4}$, które

są równolegfe do prostej stycznej do wykresu funkcji $y = \sqrt{1-x}\mathrm{w}$ punkcie $x = 0.$

Rozwiqzanie zilustrowač rysunkiem.

1




PRACA KONTROLNA nr 2

listopad $1999\mathrm{r}$

l. Udowodnič, $\dot{\mathrm{z}}\mathrm{e}$ dla $\mathrm{k}\mathrm{a}\dot{\mathrm{z}}$ dego $n$ naturalnego wielomian $x^{4n-2}+1$ jest podzielny przez trój-

mian kwadratowy $x^{2}+1.$

2. $\mathrm{W}$ równoramienny trójkąt prostokątny $0$ polu powierzchni $S=10\mathrm{c}\mathrm{m}^{2}$ wpisano prostokąty

$\mathrm{w}$ ten sposób, $\dot{\mathrm{z}}\mathrm{e}$ jeden $\mathrm{z}$ jego boków $\mathrm{l}\mathrm{e}\dot{\mathrm{z}}\mathrm{y}$ na przeciwprostokątnej, a pozostale wierzchof-

ki znajdują się na przyprostokątnych. Znalez/č ten $\mathrm{z}$ prostokątów, który ma najkrótszą

przekątną $\mathrm{i}$ obliczyč jej długośč.

3. Rozwiązač nierównośč

log125 $3\cdot\log_{x}5+\log_{9}8\cdot\log_{4}x>1.$

4. Znalez$\acute{}$č wszystkie wartości parametru $p$, dla których wykres funkcji $y=x^{2}+4x+3\mathrm{l}\mathrm{e}\dot{\mathrm{z}}\mathrm{y}$

nad prostą $y=px+1.$

5. Zbadač liczbę rozwiązań równania

$||x+5|-1|=m$

$\mathrm{w}$ zalezności od parametru $m.$

6. Rozwiazač układ równań

$\left\{\begin{array}{l}
x^{2}+y^{2}=50\\
(x-2)(y+2)=-9
\end{array}\right.$

Podač interpretację $\mathrm{g}\mathrm{e}\mathrm{o}\mathrm{m}\mathrm{e}\mathrm{t}\mathrm{r}\mathrm{y}\mathrm{c}\mathrm{z}\mathrm{n}\Phi$ tego ukladu $\mathrm{i}$ wykonač odpowiedni rysunek.

7. Wyznaczyč na osi x-ów punkty A $\mathrm{i}\mathrm{B}, \mathrm{z}$ których okrąg $x^{2}+y^{2}-4x+2y=20$ widač pod

kątem prostym $\mathrm{t}\mathrm{z}\mathrm{n}$. styczne do okręgu wychodzące $\mathrm{z}\mathrm{k}\mathrm{a}\dot{\mathrm{z}}$ dego $\mathrm{z}$ tych punktów są do siebie

prostopadle. Obliczyč pole figury ograniczonej stycznymi do okręgu przechodzącymi przez

punkty A $\mathrm{i}$ B. Wykonač staranny rysunek.

8. $\mathrm{W}$ przedziale $[0,2\pi]$ rozwiązač równanie

$1-\mathrm{t}\mathrm{g}^{2}x+\mathrm{t}\mathrm{g}^{4}x-\mathrm{t}\mathrm{g}^{6}x+\ldots=\sin^{2}3x.$

2





PRACA KONTROLNA nr 3

grudzień $1999\mathrm{r}$

l. Nie korzystając $\mathrm{z}$ metod rachunku rózniczkowego wyznaczyč dziedzinę $\mathrm{i}$ zbiór wartości

funkcji

$y=\sqrt{2+\sqrt{x}-x}.$

2. Jednym $\mathrm{z}$ wierzchofków rombu $0$ polu 20 $\mathrm{c}\mathrm{m}^{2}$ jest $A(6,3)$, ajedna $\mathrm{z}$ przekątnych zawiera

się $\mathrm{w}$ prostej $0$ równaniu $2x+y=5$. Wyznaczyč równania prostych, $\mathrm{w}$ których zawierają

się boki $\overline{AB} \mathrm{i} \overline{AD}.$

3. Stosując zasadę indukcji matematycznej udowodnič prawdziwośč wzoru

$3(1^{5}+2^{5}+\displaystyle \ldots+n^{5})+(1^{3}+2^{3}+\ldots+n^{3})=\frac{n^{3}(n+1)^{3}}{2}.$

4. Ostrosłup prawidłowy trójkątny ma pole powierzchni całkowitej $P = 12\sqrt{3}\mathrm{c}\mathrm{m}^{2}$, a kąt

nachylenia ściany bocznej do płaszczyzny podstawy $\alpha = 60^{0}$ Obliczyč objętośč tego

ostrosfupa.

5. Wśród trójkątów równoramiennych wpisanych $\mathrm{w}$ koło $0$ promieniu $R$ znalez/č ten, który

ma największe pole.

6. Przeprowadzič badanie przebiegu funkcji $y=\displaystyle \frac{1}{2}x^{2}\sqrt{5-2x}\mathrm{i}$ wykonač jej staranny wykres.

7. $\mathrm{W}$ trapezie równoramiennym dane $\mathrm{s}\Phi$ ramię $r$, kąt ostry przy podstawie $\alpha$ oraz suma

długości przekątnej $\mathrm{i}$ dluzszej podstawy wynosząca $d$. Obliczyč pole trapezu oraz pro-

mień okręgu opisanego na tym trapezie. Ustalič warunki istnienia rozwiązania. Następnie

podstawič $\alpha=30^{0}, r=\sqrt{3}$ cm $\mathrm{i} d=6$ cm.

8. Rozwiązač nierównośč

$|\cos x+\sqrt{3}\sin x|\leq\sqrt{2},x\in[0,3\pi].$

3





PRACA KONTROLNA nr 4

styczeń $2000\mathrm{r}$

l. Rozwiązač równanie $16+19+22+\cdots+x=2000$, którego lewa strona jest sumq pewnej

liczby kolejnych wyrazów ciqgu arytmetycznego.

2. Spośród cyfr $0,1,\cdots,9$ losujemy bez zwracania pięč cyfr. Obliczyč prawdopodobieństwo

tego, $\dot{\mathrm{z}}\mathrm{e}\mathrm{z}$ otrzymanych cyfr $\mathrm{m}\mathrm{o}\dot{\mathrm{z}}$ na utworzyč liczbę podzielną przez 5.

3. Zbadač, czy istnieje pochodna funkcji $f(x)=\sqrt{1-\cos x}\mathrm{w}$ punkcie $x=0$. Wynik zilu-

strowač na wykresie funkcji $f(x).$

4. Udowodnič, $\dot{\mathrm{z}}\mathrm{e}$ dwusieczne kątów wewnętrznych równolegfoboku tworzą prostokąt, którego

przekątna ma dlugośč równą róznicy długości sąsiednich boków równoległoboku.

5. Rozwiązač uklad nierówności

$\left\{\begin{array}{l}
x+y\leq 3\\
\log_{y}(2^{x+1}+32)\leq 2\log_{y}(8-2^{x})
\end{array}\right.$

$\mathrm{i}$ zaznaczyč zbiór jego rozwiązań na p{\it l}aszczy $\acute{\mathrm{z}}\mathrm{n}\mathrm{i}\mathrm{e}.$

6. Wyznaczyč równanie zbioru wszystkich punktów pfaszczyzny Oxy będących środkami

okręgów stycznych wewnętrznie do okręgu $x^{2} +y^{2} = 25 \mathrm{i}$ równocześnie stycznych

zewnetrznie do okręgu $(x+2)^{2}+y^{2}= 1$. Jaką linię przedstawia znalezione równanie?

Sporządzič staranny rysunek.

7. Zbadač iloczyn pierwiastków rzeczywistych równania

$m^{2}x^{2}+8mx+4m-4=0$

jako funkcję parametru $\mathrm{m}$. Sporządzič wykres tej funkcji.

8. Podstawą czworościanu ABCD jest trójk$\Phi$t równoboczny ABC $0$ boku $\mathrm{a}$, ściana bocz-

na BCD jest trójkątem równoramiennym prostopadfym do pfaszczyzny podstawy, a kąt

płaski ściany bocznej przy wierzchołku A jest równy $\alpha$. Obliczyč pole powierzchni kuli

opisanej na tym czworościanie.

4





PRACA KONTROLNA nr 5

luty 2000r

l. Narysowač na płaszczy $\acute{\mathrm{z}}\mathrm{n}\mathrm{i}\mathrm{e}$ zbiór $A$ wszystkich punktów $(x,y)$, których wspófrzędne spef-

niają warunki

$||x| -y| \leq 1,$

$-1\leq x\leq 2,$

$\mathrm{i}$ znalez/č punkt zbioru $A\mathrm{l}\mathrm{e}\dot{\mathrm{z}}$ ący najblizej punktu $P(0,4).$

2. Obliczyč $\sin^{3}\alpha+\cos^{3}\alpha$ wiedząc, $\displaystyle \dot{\mathrm{z}}\mathrm{e}\sin 2\alpha=\frac{1}{4}$ oraz $\alpha\in (0,2\pi).$

3. Rozwazmy rodzinę prostych przechodzących przez punkt $P(0,-1) \mathrm{i}$ przecinających pa-

rabolę $y = \displaystyle \frac{1}{4}x^{2} \mathrm{w}$ dwóch punktach. Wyznaczyč równanie środków powstalych $\mathrm{w}$ ten

sposób cięciw paraboli. Sporządzič rysunek $\mathrm{i}$ opisač otrzymaną krzywq.

4. Rozwiązač równanie

$\sqrt{x+\sqrt{x^{2}-x+2}}-\sqrt{x-\sqrt{x^{2}-x+2}}=4.$

5. Dwóch strzelców wykonuje strzelanie. Pierwszy trafia do celu $\mathrm{z}$ prawdopodobieństwem $\displaystyle \frac{2}{3}$

$\mathrm{w}\mathrm{k}\mathrm{a}\dot{\mathrm{z}}$ dym strzale $\mathrm{i}$ wykonuje 4 strza1y, a drugi trafia $\mathrm{z}$ prawdpodobieństwem $\displaystyle \frac{1}{3}\mathrm{i}$ wykonuje

8 strzałów. Który ze strzelców ma większe prawdopodobieństwo uzyskania co najmniej

trzech trafień do celu, jeśli wyniki kolejnych strzafów są wzajemnie niezalezne?

6. Do naczynia $\mathrm{w}$ ksztalcie walca $0$ promieniu podstawy $\mathrm{R}$ wrzucono trzy jednakowe kulki

$0$ promieniu $\mathrm{r}$, przy czym $R< 2r < 2R$. Okazafo się, $\dot{\mathrm{z}}\mathrm{e}$ płaska pokrywa naczynia jest

styczna do kulki znajdującej się najwyzej $\mathrm{w}$ naczyniu. Obliczyč wysokośč naczynia.

7. Dla jakich wartości parametru $m$ funkcja

$f(x)=\displaystyle \frac{x^{3}}{mx^{2}+6x+m}$

jest określona $\mathrm{i}$ rosnąca na całej prostej rzeczywistej.

8. Dany jest trójk$\Phi$t $0$ wierzchofkach $A(-2,1), B(-1,-6), C(2,5)$. Poslugując się rachun-

kiem wektorowym obliczyč cosinus kąta pomiędzy dwusieczną kąta $A\mathrm{i}$ środkową boku

$\overline{BC}$. Wykonač rysunek.

5





PRACA KONTROLNA nr 6

marzec 2000r

l. Rozwiązač równanie

xlog2 $(2x-1)+\log_{2}(x+2) =\underline{1}$

$X^{2}.$

2. Styczna do okręgu $x^{2}+y^{2}-4x$ -- $2y = 5\mathrm{w}$ punkcie $\mathrm{M}(-1,2)$, prosta $l0$ równaniu

$24x+5y$ -- $12 =0$ oraz oś Ox tworzą trójk$\Phi$t. Obliczyč pole tego trójkąta $\mathrm{i}$ wykonač

rysunek.

3. Udowodnič prawdziwośč $\mathrm{t}\mathrm{o}\dot{\mathrm{z}}$ samości

COS $\alpha+$ COS $\displaystyle \beta+\cos\gamma=4\cos\frac{\alpha+\beta}{2}\cos\frac{\beta+\gamma}{2}$ COS $\displaystyle \frac{\gamma+\alpha}{2}$)

gdzie $\alpha, \beta, \gamma \mathrm{s}\Phi$ kątami ostrymi, których suma wynosi $\displaystyle \frac{\pi}{2}$

4. Dfugości krawędzi prostopadfościanu $0$ objętości $V = 8$ tworzą ciąg geometryczny, $\mathrm{a}$

stosunek długości przekątnej prostopadłościanu do najdłuzszej $\mathrm{z}$ przekątnych ścian tej

bryły wynosi $\displaystyle \frac{3}{4}\sqrt{2}$. Obliczyč pole powierzchni cafkowitej prostopadfościanu.

5. $\mathrm{Z}$ urny zawierającej siedem kul czarnych $\mathrm{i}$ trzy biafe wybrano losowo trzy kule $\mathrm{i}$ przełozono

do drugiej, pustej urny. Jakie jest prawdopodobieństwo wylosowania kuli białej $\mathrm{z}$ drugiej

urny?

6. Prostokąt obraca się wokół swojej przekątnej. Obliczyč objętośč powstałej bryły, jeśli

przekątna ma długośč $d$, a $\mathrm{k}\mathrm{a}\mathrm{t}$ pomiędzy przekątną, a dfuzszym bokiem ma miarę $\alpha.$

Wykonač odpowiedni rysunek.

7. Wyznaczyč największq $\mathrm{i}$ najmniejszą wartośč funkcji

$f(x) =x^{5/2}$ -- $10x^{3/2}+40x^{1/2}$

w przedziale [1,5].

8. Stosunek promienia okręgu wpisanego do promienia okręgu opisanego na trójkącie prosto-

kątnym jest równy k. Obliczyč w jakim stosunku środek okręgu wpisanego w ten trójkąt

dzieli dwusieczną kata prostego. Określič dziedzine dla parametru k.

6





PRACA KONTROLNA nr 7

kwiecień 2000r

l. Rozwiązač nierównośč

$|9^{x}-2|<3^{x+1}-2.$

2. Wyznaczyč równanie krzywej $\mathrm{b}\text{ę} \mathrm{d}_{\Phi}\mathrm{c}\mathrm{e}\mathrm{j}$ obrazem okręgu $(x+1)^{2}+(y-6)^{2}=4\mathrm{w}$ po-

winowactwie prostokqtnym $0$ osi $\mathrm{O}\mathrm{x}\mathrm{i}$ stosunku $k=\displaystyle \frac{1}{2}$. Obliczyč pole figury ograniczonej

$\mathrm{t}_{\Phi}$ krzywą. Wykonač staranny rysunek.

3. Pewien zbiór zawiera dokładnie 67 podzbiorów $0$ co najwyzej dwóch elementach. Ile

podzbiorów siedmioelementowych zawiera ten zbiór?

4. Na kole $0$ promieniu $R$ opisano trapez $0$ kątach przy dfuzszej podstawie $15^{0} \mathrm{i} 45^{0}$

Obliczyč stosunek pola koła do pola tego trapezu.

5. Rozwiązač uklad równań

$\left\{\begin{array}{l}
mx\\
2x
\end{array}\right.$

$+$

$6y$

$(m-7)y$

$=3$

$=m-1$

w zalezności od parametru rzeczywistego m. Podač wszystkie rozwiązania

(i odpowiadające im wartości parametru m), dla których x jest równe y.

6. Rozwiązač nierównośč

$\sin 2x<\sin x$

$\mathrm{w}$ przedziale $[-\displaystyle \frac{\pi}{2},\frac{\pi}{2}]. \mathrm{R}\mathrm{o}\mathrm{z}\mathrm{w}\mathrm{i}_{\Phi}$zanie zilustrowač starannym wykresem.

7. Ostroslup przecięto na trzy części dwiema plaszczyznami równoległymi do jego podstawy.

Pierwsza pfaszczyznajest połozona $\mathrm{w}$ odlegfości $d_{1} =2$ cm, a druga $\mathrm{w}$ odlegfości $d_{2}=3$

cm od podstawy. Pola przekrojów ostroslupa tymi plaszczyznami równe są odpowiednio

$S_{1} = 25 \mathrm{c}\mathrm{m}^{2}$ oraz $S_{2} = 16 \mathrm{c}\mathrm{m}^{2}$ Obliczyč objętośč tego ostrosłupa oraz objętośč

najmniejszej części.

8. Trylogię skladającą się $\mathrm{z}$ dwóch powieści dwutomowych oraz jednej jednotomowej usta-

wiono przypadkowo na półce. Jakie jest prawdopodobieństwo tego, $\dot{\mathrm{z}}\mathrm{e}$ tomy

a) obydwu, b) co najmniej jednej $\mathrm{z}$ dwutomowych powieści znajdują się obok siebie $\mathrm{i}$ przy

tym tom I $\mathrm{z}$ lewej, a tom II $\mathrm{z}$ prawej strony.

7






KORESPONDENCYJNY KURS PRZYGOTOWAWCZY Z

MATEMATYKI

PRACA KONTROLNA nr l

$\mathrm{p}\mathrm{a}\acute{\mathrm{z}}$dziernik 1999 $\mathrm{r}$

l. Stop składa się $\mathrm{z}$ 40\% srebra próby 0,6, 30\% srebra próby 0,7 oraz l kg srebra próby 0,8.

Jaka jest waga $\mathrm{i}$ jaka jest próba tego stopu?

2. Rozwiązač równanie

$3^{x}+1+3^{-x}+\ldots=4,$

którego lewa strona jest $\mathrm{s}\mathrm{u}\mathrm{m}\Phi$ nieskończonego ciągu geometrycznego.

3. $\mathrm{W}$ trójkącie $ABC$ znane są wierzcholki $A(0,0)$ oraz $B(4,-1)$. Wiadomo, $\dot{\mathrm{z}}\mathrm{e}\mathrm{w}$ punkcie

$H(3,2)$ przecinają się proste zawierające wysokości tego trójkąta. Wyznaczyč wspólrzędne

wierzchofka $C$. Wykonač odpowiedni rysunek.

4. Rozwiqzač równanie

$\cos 4x=\sin 3x.$

5. Wykonač staranny wykres funkcji

$f(x)=|\log_{2}(x-2)^{2}|.$

6. Rozwiązač nierównośč

$\displaystyle \frac{1}{x^{2}}\geq\frac{1}{x+6}.$

7. $\mathrm{W}$ ostrosłupie prawidłowym sześciokatnym krawęd $\acute{\mathrm{z}}$ podstawy ma długośč $p$, a krawędz/

boczna dfugośč $2p$. Obliczyč cosinus $\mathrm{k}_{\Phi^{\mathrm{t}\mathrm{a}}}$ dwuściennego między sąsiednimi ścianami bocz-

nymi tego ostrosłupa.

8. Wyznaczyč równania wszystkich prostych stycznych do wykresu funkcji $y=\displaystyle \frac{2x+10}{x+4}$, które

są równolegfe do prostej stycznej do wykresu funkcji $y = \sqrt{1-x}\mathrm{w}$ punkcie $x = 0.$

Rozwiqzanie zilustrowač rysunkiem.

1





PRACA KONTROLNA nr 2

listopad $1999\mathrm{r}$

l. Udowodnič, $\dot{\mathrm{z}}\mathrm{e}$ dla $\mathrm{k}\mathrm{a}\dot{\mathrm{z}}$ dego $n$ naturalnego wielomian $x^{4n-2}+1$ jest podzielny przez trój-

mian kwadratowy $x^{2}+1.$

2. $\mathrm{W}$ równoramienny trójkąt prostokątny $0$ polu powierzchni $S=10\mathrm{c}\mathrm{m}^{2}$ wpisano prostokąty

$\mathrm{w}$ ten sposób, $\dot{\mathrm{z}}\mathrm{e}$ jeden $\mathrm{z}$ jego boków $\mathrm{l}\mathrm{e}\dot{\mathrm{z}}\mathrm{y}$ na przeciwprostokątnej, a pozostale wierzchof-

ki znajdują się na przyprostokątnych. Znalez/č ten $\mathrm{z}$ prostokątów, który ma najkrótszą

przekątną $\mathrm{i}$ obliczyč jej długośč.

3. Rozwiązač nierównośč

log125 $3\cdot\log_{x}5+\log_{9}8\cdot\log_{4}x>1.$

4. Znalez$\acute{}$č wszystkie wartości parametru $p$, dla których wykres funkcji $y=x^{2}+4x+3\mathrm{l}\mathrm{e}\dot{\mathrm{z}}\mathrm{y}$

nad prostą $y=px+1.$

5. Zbadač liczbę rozwiązań równania

$||x+5|-1|=m$

$\mathrm{w}$ zalezności od parametru $m.$

6. Rozwiazač układ równań

$\left\{\begin{array}{l}
x^{2}+y^{2}=50\\
(x-2)(y+2)=-9
\end{array}\right.$

Podač interpretację $\mathrm{g}\mathrm{e}\mathrm{o}\mathrm{m}\mathrm{e}\mathrm{t}\mathrm{r}\mathrm{y}\mathrm{c}\mathrm{z}\mathrm{n}\Phi$ tego ukladu $\mathrm{i}$ wykonač odpowiedni rysunek.

7. Wyznaczyč na osi x-ów punkty A $\mathrm{i}\mathrm{B}, \mathrm{z}$ których okrąg $x^{2}+y^{2}-4x+2y=20$ widač pod

kątem prostym $\mathrm{t}\mathrm{z}\mathrm{n}$. styczne do okręgu wychodzące $\mathrm{z}\mathrm{k}\mathrm{a}\dot{\mathrm{z}}$ dego $\mathrm{z}$ tych punktów są do siebie

prostopadle. Obliczyč pole figury ograniczonej stycznymi do okręgu przechodzącymi przez

punkty A $\mathrm{i}$ B. Wykonač staranny rysunek.

8. $\mathrm{W}$ przedziale $[0,2\pi]$ rozwiązač równanie

$1-\mathrm{t}\mathrm{g}^{2}x+\mathrm{t}\mathrm{g}^{4}x-\mathrm{t}\mathrm{g}^{6}x+\ldots=\sin^{2}3x.$

2





PRACA KONTROLNA nr 3

grudzień $1999\mathrm{r}$

l. Nie korzystając $\mathrm{z}$ metod rachunku rózniczkowego wyznaczyč dziedzinę $\mathrm{i}$ zbiór wartości

funkcji

$y=\sqrt{2+\sqrt{x}-x}.$

2. Jednym $\mathrm{z}$ wierzchofków rombu $0$ polu 20 $\mathrm{c}\mathrm{m}^{2}$ jest $A(6,3)$, ajedna $\mathrm{z}$ przekątnych zawiera

się $\mathrm{w}$ prostej $0$ równaniu $2x+y=5$. Wyznaczyč równania prostych, $\mathrm{w}$ których zawierają

się boki $\overline{AB} \mathrm{i} \overline{AD}.$

3. Stosując zasadę indukcji matematycznej udowodnič prawdziwośč wzoru

$3(1^{5}+2^{5}+\displaystyle \ldots+n^{5})+(1^{3}+2^{3}+\ldots+n^{3})=\frac{n^{3}(n+1)^{3}}{2}.$

4. Ostrosłup prawidłowy trójkątny ma pole powierzchni całkowitej $P = 12\sqrt{3}\mathrm{c}\mathrm{m}^{2}$, a kąt

nachylenia ściany bocznej do płaszczyzny podstawy $\alpha = 60^{0}$ Obliczyč objętośč tego

ostrosfupa.

5. Wśród trójkątów równoramiennych wpisanych $\mathrm{w}$ koło $0$ promieniu $R$ znalez/č ten, który

ma największe pole.

6. Przeprowadzič badanie przebiegu funkcji $y=\displaystyle \frac{1}{2}x^{2}\sqrt{5-2x}\mathrm{i}$ wykonač jej staranny wykres.

7. $\mathrm{W}$ trapezie równoramiennym dane $\mathrm{s}\Phi$ ramię $r$, kąt ostry przy podstawie $\alpha$ oraz suma

długości przekątnej $\mathrm{i}$ dluzszej podstawy wynosząca $d$. Obliczyč pole trapezu oraz pro-

mień okręgu opisanego na tym trapezie. Ustalič warunki istnienia rozwiązania. Następnie

podstawič $\alpha=30^{0}, r=\sqrt{3}$ cm $\mathrm{i} d=6$ cm.

8. Rozwiązač nierównośč

$|\cos x+\sqrt{3}\sin x|\leq\sqrt{2},x\in[0,3\pi].$

3





PRACA KONTROLNA nr 4

styczeń $2000\mathrm{r}$

l. Rozwiązač równanie $16+19+22+\cdots+x=2000$, którego lewa strona jest sumq pewnej

liczby kolejnych wyrazów ciqgu arytmetycznego.

2. Spośród cyfr $0,1,\cdots,9$ losujemy bez zwracania pięč cyfr. Obliczyč prawdopodobieństwo

tego, $\dot{\mathrm{z}}\mathrm{e}\mathrm{z}$ otrzymanych cyfr $\mathrm{m}\mathrm{o}\dot{\mathrm{z}}$ na utworzyč liczbę podzielną przez 5.

3. Zbadač, czy istnieje pochodna funkcji $f(x)=\sqrt{1-\cos x}\mathrm{w}$ punkcie $x=0$. Wynik zilu-

strowač na wykresie funkcji $f(x).$

4. Udowodnič, $\dot{\mathrm{z}}\mathrm{e}$ dwusieczne kątów wewnętrznych równolegfoboku tworzą prostokąt, którego

przekątna ma dlugośč równą róznicy długości sąsiednich boków równoległoboku.

5. Rozwiązač uklad nierówności

$\left\{\begin{array}{l}
x+y\leq 3\\
\log_{y}(2^{x+1}+32)\leq 2\log_{y}(8-2^{x})
\end{array}\right.$

$\mathrm{i}$ zaznaczyč zbiór jego rozwiązań na p{\it l}aszczy $\acute{\mathrm{z}}\mathrm{n}\mathrm{i}\mathrm{e}.$

6. Wyznaczyč równanie zbioru wszystkich punktów pfaszczyzny Oxy będących środkami

okręgów stycznych wewnętrznie do okręgu $x^{2} +y^{2} = 25 \mathrm{i}$ równocześnie stycznych

zewnetrznie do okręgu $(x+2)^{2}+y^{2}= 1$. Jaką linię przedstawia znalezione równanie?

Sporządzič staranny rysunek.

7. Zbadač iloczyn pierwiastków rzeczywistych równania

$m^{2}x^{2}+8mx+4m-4=0$

jako funkcję parametru $\mathrm{m}$. Sporządzič wykres tej funkcji.

8. Podstawą czworościanu ABCD jest trójk$\Phi$t równoboczny ABC $0$ boku $\mathrm{a}$, ściana bocz-

na BCD jest trójkątem równoramiennym prostopadfym do pfaszczyzny podstawy, a kąt

płaski ściany bocznej przy wierzchołku A jest równy $\alpha$. Obliczyč pole powierzchni kuli

opisanej na tym czworościanie.

4





PRACA KONTROLNA nr 5

luty 2000r

l. Narysowač na płaszczy $\acute{\mathrm{z}}\mathrm{n}\mathrm{i}\mathrm{e}$ zbiór $A$ wszystkich punktów $(x,y)$, których wspófrzędne spef-

niają warunki

$||x| -y| \leq 1,$

$-1\leq x\leq 2,$

$\mathrm{i}$ znalez/č punkt zbioru $A\mathrm{l}\mathrm{e}\dot{\mathrm{z}}$ ący najblizej punktu $P(0,4).$

2. Obliczyč $\sin^{3}\alpha+\cos^{3}\alpha$ wiedząc, $\displaystyle \dot{\mathrm{z}}\mathrm{e}\sin 2\alpha=\frac{1}{4}$ oraz $\alpha\in (0,2\pi).$

3. Rozwazmy rodzinę prostych przechodzących przez punkt $P(0,-1) \mathrm{i}$ przecinających pa-

rabolę $y = \displaystyle \frac{1}{4}x^{2} \mathrm{w}$ dwóch punktach. Wyznaczyč równanie środków powstalych $\mathrm{w}$ ten

sposób cięciw paraboli. Sporządzič rysunek $\mathrm{i}$ opisač otrzymaną krzywq.

4. Rozwiązač równanie

$\sqrt{x+\sqrt{x^{2}-x+2}}-\sqrt{x-\sqrt{x^{2}-x+2}}=4.$

5. Dwóch strzelców wykonuje strzelanie. Pierwszy trafia do celu $\mathrm{z}$ prawdopodobieństwem $\displaystyle \frac{2}{3}$

$\mathrm{w}\mathrm{k}\mathrm{a}\dot{\mathrm{z}}$ dym strzale $\mathrm{i}$ wykonuje 4 strza1y, a drugi trafia $\mathrm{z}$ prawdpodobieństwem $\displaystyle \frac{1}{3}\mathrm{i}$ wykonuje

8 strzałów. Który ze strzelców ma większe prawdopodobieństwo uzyskania co najmniej

trzech trafień do celu, jeśli wyniki kolejnych strzafów są wzajemnie niezalezne?

6. Do naczynia $\mathrm{w}$ ksztalcie walca $0$ promieniu podstawy $\mathrm{R}$ wrzucono trzy jednakowe kulki

$0$ promieniu $\mathrm{r}$, przy czym $R< 2r < 2R$. Okazafo się, $\dot{\mathrm{z}}\mathrm{e}$ płaska pokrywa naczynia jest

styczna do kulki znajdującej się najwyzej $\mathrm{w}$ naczyniu. Obliczyč wysokośč naczynia.

7. Dla jakich wartości parametru $m$ funkcja

$f(x)=\displaystyle \frac{x^{3}}{mx^{2}+6x+m}$

jest określona $\mathrm{i}$ rosnąca na całej prostej rzeczywistej.

8. Dany jest trójk$\Phi$t $0$ wierzchofkach $A(-2,1), B(-1,-6), C(2,5)$. Poslugując się rachun-

kiem wektorowym obliczyč cosinus kąta pomiędzy dwusieczną kąta $A\mathrm{i}$ środkową boku

$\overline{BC}$. Wykonač rysunek.

5





PRACA KONTROLNA nr 6

marzec 2000r

l. Rozwiązač równanie

xlog2 $(2x-1)+\log_{2}(x+2) =\underline{1}$

$X^{2}.$

2. Styczna do okręgu $x^{2}+y^{2}-4x$ -- $2y = 5\mathrm{w}$ punkcie $\mathrm{M}(-1,2)$, prosta $l0$ równaniu

$24x+5y$ -- $12 =0$ oraz oś Ox tworzą trójk$\Phi$t. Obliczyč pole tego trójkąta $\mathrm{i}$ wykonač

rysunek.

3. Udowodnič prawdziwośč $\mathrm{t}\mathrm{o}\dot{\mathrm{z}}$ samości

COS $\alpha+$ COS $\displaystyle \beta+\cos\gamma=4\cos\frac{\alpha+\beta}{2}\cos\frac{\beta+\gamma}{2}$ COS $\displaystyle \frac{\gamma+\alpha}{2}$)

gdzie $\alpha, \beta, \gamma \mathrm{s}\Phi$ kątami ostrymi, których suma wynosi $\displaystyle \frac{\pi}{2}$

4. Dfugości krawędzi prostopadfościanu $0$ objętości $V = 8$ tworzą ciąg geometryczny, $\mathrm{a}$

stosunek długości przekątnej prostopadłościanu do najdłuzszej $\mathrm{z}$ przekątnych ścian tej

bryły wynosi $\displaystyle \frac{3}{4}\sqrt{2}$. Obliczyč pole powierzchni cafkowitej prostopadfościanu.

5. $\mathrm{Z}$ urny zawierającej siedem kul czarnych $\mathrm{i}$ trzy biafe wybrano losowo trzy kule $\mathrm{i}$ przełozono

do drugiej, pustej urny. Jakie jest prawdopodobieństwo wylosowania kuli białej $\mathrm{z}$ drugiej

urny?

6. Prostokąt obraca się wokół swojej przekątnej. Obliczyč objętośč powstałej bryły, jeśli

przekątna ma długośč $d$, a $\mathrm{k}\mathrm{a}\mathrm{t}$ pomiędzy przekątną, a dfuzszym bokiem ma miarę $\alpha.$

Wykonač odpowiedni rysunek.

7. Wyznaczyč największq $\mathrm{i}$ najmniejszą wartośč funkcji

$f(x) =x^{5/2}$ -- $10x^{3/2}+40x^{1/2}$

w przedziale [1,5].

8. Stosunek promienia okręgu wpisanego do promienia okręgu opisanego na trójkącie prosto-

kątnym jest równy k. Obliczyč w jakim stosunku środek okręgu wpisanego w ten trójkąt

dzieli dwusieczną kata prostego. Określič dziedzine dla parametru k.

6





PRACA KONTROLNA nr 7

kwiecień 2000r

l. Rozwiązač nierównośč

$|9^{x}-2|<3^{x+1}-2.$

2. Wyznaczyč równanie krzywej $\mathrm{b}\text{ę} \mathrm{d}_{\Phi}\mathrm{c}\mathrm{e}\mathrm{j}$ obrazem okręgu $(x+1)^{2}+(y-6)^{2}=4\mathrm{w}$ po-

winowactwie prostokqtnym $0$ osi $\mathrm{O}\mathrm{x}\mathrm{i}$ stosunku $k=\displaystyle \frac{1}{2}$. Obliczyč pole figury ograniczonej

$\mathrm{t}_{\Phi}$ krzywą. Wykonač staranny rysunek.

3. Pewien zbiór zawiera dokładnie 67 podzbiorów $0$ co najwyzej dwóch elementach. Ile

podzbiorów siedmioelementowych zawiera ten zbiór?

4. Na kole $0$ promieniu $R$ opisano trapez $0$ kątach przy dfuzszej podstawie $15^{0} \mathrm{i} 45^{0}$

Obliczyč stosunek pola koła do pola tego trapezu.

5. Rozwiązač uklad równań

$\left\{\begin{array}{l}
mx\\
2x
\end{array}\right.$

$+$

$6y$

$(m-7)y$

$=3$

$=m-1$

w zalezności od parametru rzeczywistego m. Podač wszystkie rozwiązania

(i odpowiadające im wartości parametru m), dla których x jest równe y.

6. Rozwiązač nierównośč

$\sin 2x<\sin x$

$\mathrm{w}$ przedziale $[-\displaystyle \frac{\pi}{2},\frac{\pi}{2}]. \mathrm{R}\mathrm{o}\mathrm{z}\mathrm{w}\mathrm{i}_{\Phi}$zanie zilustrowač starannym wykresem.

7. Ostroslup przecięto na trzy części dwiema plaszczyznami równoległymi do jego podstawy.

Pierwsza pfaszczyznajest połozona $\mathrm{w}$ odlegfości $d_{1} =2$ cm, a druga $\mathrm{w}$ odlegfości $d_{2}=3$

cm od podstawy. Pola przekrojów ostroslupa tymi plaszczyznami równe są odpowiednio

$S_{1} = 25 \mathrm{c}\mathrm{m}^{2}$ oraz $S_{2} = 16 \mathrm{c}\mathrm{m}^{2}$ Obliczyč objętośč tego ostrosłupa oraz objętośč

najmniejszej części.

8. Trylogię skladającą się $\mathrm{z}$ dwóch powieści dwutomowych oraz jednej jednotomowej usta-

wiono przypadkowo na półce. Jakie jest prawdopodobieństwo tego, $\dot{\mathrm{z}}\mathrm{e}$ tomy

a) obydwu, b) co najmniej jednej $\mathrm{z}$ dwutomowych powieści znajdują się obok siebie $\mathrm{i}$ przy

tym tom I $\mathrm{z}$ lewej, a tom II $\mathrm{z}$ prawej strony.

7






Tadeusz Inglot

KURS

KORESPONDENCYJNY

MATEMATYKA

Zbiór $\mathrm{z}\mathrm{a}\mathrm{d}\mathrm{a}\acute{\mathrm{r}}\perp 1999-2004$




Recenzenci

Roicistaw RABCZUK

$\mathrm{Z}\mathrm{b}_{\mathrm{f}}$gniew ROMANOWICZ

Opracowanie redakcyjne

Atina KACZAK

Projekt okfadki

Zofia l- Danusz GODLEWSCY

\copyright Copyright by Oficyna Wydawnicza Politechniki Wroctawskiej, Wrociaw 2005

OFICYNA WYDAWNICZA POLITECHNIKI WROCLAWSKIEJ

Wybrzeze Wyspiańskiego 27, 50-370 Wroctaw

ISBN 83-7085-871-6

Diukarnia $\mathrm{O}\Gamma$icyny Wydawniczej Politechniki Wrociawskiej. $\mathrm{Z}\mathrm{a}\mathrm{m}_{\wedge}$ nr $403/200\overline{\supset}.$





14

Praca kontrolna nr 6

6.1. Rozwiazač równanie

$\displaystyle \mathrm{x}^{\log_{2}(2x-1)+\log_{2}(x+2)}=\frac{1}{x^{2}}.$

6.2. Styczna do okregu $x^{2}+y^{2}-4x-2y=5\mathrm{w}$ punkcie $M(-1,2)$, prosta

$l0$ równaniu $24x+5y-12=0$ oraz oś $Ox$ tworza trójkat. Obliczyč

pole tego trójkata. Sporzadzič rysunek.

6.3. Udowodnič prawdziwośč $\mathrm{t}\mathrm{o}\dot{\mathrm{z}}$ samości

$\displaystyle \cos\alpha+\cos\beta+\cos\gamma=4\cos\frac{\alpha+\beta}{2}\cos\frac{\beta+\gamma}{2}\cos\frac{\gamma+\alpha}{2},$

gdzie $\alpha, \beta, \gamma$ sa katami ostrymi, których suma wynosi $\displaystyle \frac{\pi}{2}.$

6.4. Dlugości krawedzi prostopadlościanu $0$ objetości $V = 8$ tworza ciag

geometryczny, a stosunek dlugości przekatnej prostopadlościanu do

najdluzszej $\mathrm{z}$ przekatnych jego ścian wynosi $\displaystyle \frac{3}{4}\sqrt{2}$. Obliczyč pole

powierzchni calkowitej prostopadlościanu.

6.5. $\mathrm{Z}$ urny zawierajacej siedem kul czarnych $\mathrm{i}$ trzy biale wybrano losowo

trzy kule $\mathrm{i}$ przelozono do drugiej, pustej urny. Jakie jest prawdopodo-

bieństwo wylosowania kuli bialej $\mathrm{z}$ drugiej urny?

6.6. Prostokat obraca $\mathrm{s}\mathrm{i}\mathrm{e}$ wokól swojej przekatnej. Obliczyč objetośč pow-

stalej bryly, jeśli przekatna ma dlugośč $d$, a $\mathrm{k}\mathrm{a}\mathrm{t}$ pomiedzy przekatna

$\mathrm{i}$ dluzszym bokiem ma miare $\alpha$. Sporzadzič odpowiedni rysunek.

6.7. Wyznaczyč najwieksza $\mathrm{i}$ najmniejsza wartośč funkcji

$f(x)=x^{5/2}-10x^{3/2}+40x^{1/2}$

w przedziale [1, 5].

6.8. Stosunek promienia okregu wpisanego w trójkat prostokatny do pro-

mienia okregu opisanego na tym trójkacie jest równy k. W jakim

stosunku środek okregu wpisanego w ten trójkat dzieli dwusieczna

kata prostego? Określič dziedzine dla parametru k.





118

20.4. Stosowač wzór na odleglośč punktu od prostej. Pamietač, $\dot{\mathrm{z}}\mathrm{e}$

rozwazamy tylko punkty wewnatrz danego trójkata. Nazwač wyznaczona

krzywa.

20.5. Rozwazyč przypadki $x > 1 \mathrm{i}x < 1 \mathrm{i}$ uprościč wzór określajacy

funkcje. Podczas rysowania wykresu pamietač $0$ dziedzinie funkcji.

20.6. Napisač $\displaystyle \frac{1}{x^{2}} = |x|^{-2}\mathrm{i}$ rozwazyč przypadki $|x| =1, |x| < 1$ oraz

$|x|>1$. Nie stosowač bezpośrednio definicji wartości bezwzglednej.

20.7. Warunek zadania oznacza, $\dot{\mathrm{z}}\mathrm{e}$ rozwazane styczne maja wspólczyn-

niki kierunkowe $+1\mathrm{l}\mathrm{u}\mathrm{b}-1$. Obliczyč pochodna funkcji $f$, przyrównač jej

wartośč bezwzgledna do l $\mathrm{i}$ rozwiazač otrzymane równanie niewymierne.

20.8. Oznaczyč $x=|AD|$ oraz $y=|AE|$. Ze stosunku pól obliczyč $xy,$

a $\mathrm{z}$ twierdzenia sinusów $\mathrm{w}$ trójkacie $ADE$ iloraz $\displaystyle \frac{x}{y}$. Nie wyznaczač jawnie

$x\mathrm{i}y$, lecz tylko sume $x+y$ (korzystač ze wzoru skróconego mnozenia).

21.1. Oznaczyč przez $x, y$ krawedzie mniejszych sześcianów. Napisač

uklad równań $\mathrm{z}$ niewiadomymi $x\mathrm{i}y\mathrm{i}$ nie wyznaczajac ich jawnie, obliczyč

tylko $x^{2} +y^{2}$ za pomoca wzorów skróconego mnozenia. Stad od razu

otrzymač odpowied $\acute{\mathrm{z}}.$

21.2. Wyznaczyč wektory $\vec{AC}\mathrm{i}\vec{BD}\mathrm{i}$ zastosowač iloczyn skalarny oraz

$\mathrm{t}\mathrm{o}\dot{\mathrm{z}}$ samośč podana we wskazówce do $\mathrm{z}\mathrm{a}\mathrm{d}$. 2.8.

21.3. Wyznaczyč skale podobieństwa trójkatów i wyrazič przeciwpros-

tokatna przez promień okregu r. Stad obliczyč sume przyprostokatnych

wyjściowego trójkata iw konsekwencji sume cosinusów katów ostrych trój-

kata. Podnoszac te równośč do kwadratu obliczyč oba cosinusy.

21.4. Przenieśč niewymiernośč do mianownika $\mathrm{i}$ podzielič licznik $\mathrm{i}$ mia-

nownik przez $n$. Korzystač $\mathrm{z}$ faktu, $\dot{\mathrm{z}}\mathrm{e}$ zlozenie funkcji malejacych jest

funkcja rosnaca.

21.5. Korzystač ze wzoru podanego we wskazówce do zadania 3.8.





119

21.6. Napisač warunki określajace dziedzine, ale nie wyznaczač dziedzi-

ny w sposóbjawny. Sprowadzič logarytmy do wspólnej podstawy 4 i przejśč

do równania algebraicznego trzeciego stopnia. Obliczyč jego pierwiastki

i wybrač te, które naleza do dziedziny.

21.7. Narysowač przekrój osiowy stozka. Objetośč wyrazičjako funkcje

wysokości stozka. Nie mylič tego zadania z zagadnieniem wyznaczania

ekstremów lokalnych.

21.8. Obie parabole lacznie ze stycznymi tworza figure majaca środek

symetrii S (dlaczego?). Wiec szukane styczne przechodza przez punkt S.

Wyznaczyč S. Napisač równanie peku prostych przechodzacych przez S

i z warunku styczności (wyróznik odpowiedniego równania kwadratowego

równy zeru) obliczyč wspólczynniki kierunkowe szukanych stycznych.

22.1. Wykorzystač parzystośč funkcji.

zwrócič uwage na otoczenie punktu $x=0.$

Podczas rysowania wykresu

22.2. Uzasadnič, $\dot{\mathrm{z}}\mathrm{e}$ liczby metrów sześciennych wody wplywajace do

basenu $\mathrm{w}$ kolejnych minutach tworza ciag arytmetyczny. We wszystkich

obliczeniach przyjač $minut_{G}$ jako jednostke czasu. Dane liczbowe podstawič

na końcu.

22.3. Oznaczyč średnice obu podstaw przez $x\mathrm{i}y$. Ulozyč uklad równań

$\mathrm{z}$ niewiadomymi $x, y \mathrm{i}$ przejśč od razu do alternatywy ukladów równań

liniowych.

22.4. $\mathrm{Z}$ twierdzenia sinusów wynika, $\dot{\mathrm{z}}\mathrm{e}$ znany jest takze bok $|BC|.$

$\mathrm{W}$ okregu $0$ promieniu $R$ zaznaczyč cieciwe $0$ dlugości $|BC| \mathrm{i}$ rozwazač

katy wpisane oparte na luku wyznaczonym przez $\mathrm{t}\mathrm{e}$ cieciwe. Wybrač takie

polozenie (polozenia) wierzcholka $A$, które daje $|AB|=\displaystyle \frac{8}{5}R. \mathrm{W}$ zalezności

od wielkości kata $\alpha$ (czyli dlugości cieciwy $|BC|$) mamy rózne przypadki,

które nalezy kolejno rozpatrzyč.

22.5. Od razu zlogarytmowač obie strony,

logarytmu liczbe 8.

przyjmujac za podstawe





120

22.6. Wyrazič wektory

iloczynu skalarnego.

$\vec{CB}$

i

$\vec{CD}$ przez $\vec{AB}=\vec{u} \mathrm{i} \vec{BD}=\vec{v}\mathrm{i}\mathrm{u}\dot{\mathrm{z}}$ yč

22.7. Wyznaczyč dziedzine równania. Pomnozyč obie strony przez

wyrazenie $(\sin x\cos x) \mathrm{i}$ doprowadzič do równania elementarnego postaci

$\sin(f(x))=\sin(g(x))$. Rozwiazania zapisač $\mathrm{w}$ postaci jednej serii.

22.8. Napisač równanie stycznej $\mathrm{w}$ punkcie $x_{0}$, wyznaczyč punkty prze-

cieč tej stycznej $\mathrm{z}$ osiami ukladu wspólrzednych $\mathrm{i}$ wyrazič kwadrat dlugości

odcinka stycznej jako funkcje $x_{0}$. Do rózniczkowania pozostawič $\mathrm{t}\mathrm{e}$ funkcje

$\mathrm{w}$ postaci sumy funkcji potegowych. Nie mylič postawionego pytania $\mathrm{z}$ za-

gadnieniem wyznaczania ekstremów lokalnych.

23.1. Liczba,,slów'' utworzonych $\mathrm{z}$ danych liter odpowiada liczbie per-

mutacji $\mathrm{z}$ powtórzeniami.

23.2. Zadanie rozwiazač bez dzielenia wielomianów. Zauwazyč, $\dot{\mathrm{z}}\mathrm{e}$ i10-

raz danych wielomianów ma postač $ x+\alpha \mathrm{i}$ wyznaczyč najpierw niewia-

doma $\alpha.$

23.3. Wykorzystač symetrie figury $\mathrm{i}$ twierdzenie $0$ okregach wzajemnie

stycznych.

23.4. Przez punkty $K \mathrm{i} L$ poprowadzič plaszczyzny prostopadle do

plaszczyzny podstawy $\mathrm{i}$ równolegle do $BC$. Obliczač oddzielnie objetości

$\mathrm{k}\mathrm{a}\dot{\mathrm{z}}$ dej $\mathrm{z}$ tak otrzymanych bryl (dwie $\mathrm{z}$ nich sa identyczne). Por. $\mathrm{z}\mathrm{a}\mathrm{d}$. 15.3.

23.5. Wyznaczyč dziedzine nierówności. Rozpatrzyč najpierw oczy-

wisty przypadek $x < 0$. Dla $x > 0$ podnieśč obie strony nierówności

do kwadratu $\mathrm{i}$ rozwiazač nierównośč dwukwadratowa. Wykresem funkcji

$\mathrm{z}$ prawej strony nierówności nie jest luk paraboli lecz inna dobrze znana

krzywa (por. wskazówka do $\mathrm{z}\mathrm{a}\mathrm{d}$. 13.7).

23.6. Dowód kroku indukcyjnego przeprowadzič wprost. Nie stosowač

niewygodnej metody redukcji. Dbač $0$ logiczna poprawnośč zapisu dowodu.





121

23.7. Punkt $M(y_{0}^{2},y_{0}), y_{0}>0, \mathrm{l}\mathrm{e}\dot{\mathrm{z}}\mathrm{y}$ najblizej $P$, gdy odcinek $PM$ jest

prostopadly do stycznej do danej krzywej $\mathrm{w}$ punkcie $M. \mathrm{U}\dot{\mathrm{z}}$ yč rachunku

wektorowego.

23.8. Poniewaz wspólczynnik przy $x^{2}$ jest dodatni, wiec pierwiastki

trójmianu kwadratowego beda $\mathrm{l}\mathrm{e}\dot{\mathrm{z}}$ eč $\mathrm{w}$ odcinku $(0,1)$, gdy odcieta wierz-

cholka paraboli bedacej jego wykresem znajdzie $\mathrm{s}\mathrm{i}\mathrm{e} \mathrm{w}$ tym przedziale,

a wartości trójmianu dla $x=0\mathrm{i}x=1$ beda dodatnie. Otrzymane nierów-

ności trygonometryczne rozwiazač analitycznie. Ewentualny rysunek sluzy

do ilustracji rozwiazania.

24.1. Pamietač 0 warunku istnienia sumy nieskończonego ciagu geo-

metrycznego.

24.2. Zaczač od określenia modelu probabilistycznego, $\mathrm{t}\mathrm{j}$. zbioru zdarzeń

elementarnych $\Omega$ oraz prawdopodobieństwa $P$. Oznaczyč przez $A$ zdarze-

nie polegajace na $\mathrm{t}\mathrm{y}\mathrm{m}, \dot{\mathrm{z}}\mathrm{e}$ kości pasuja do siebie, a przez $A_{i}$ zdarzenie, $\dot{\mathrm{z}}\mathrm{e}$

na jednym $\mathrm{z}$ pól obu kości jest $i$ oczek, a na pozostalych polach cokolwiek,

$i=0$, 6. Wtedy $ A=A_{0}\cup \cup A_{6} \mathrm{i}$ skladniki parami wykluczaja $\mathrm{s}\mathrm{i}\mathrm{e}$

(dlaczego?). Obliczyč $P(A_{i})\mathrm{i}$ skorzystač $\mathrm{z}$ wlasności prawdopodobieństwa.

24.3. Wykazač, $\dot{\mathrm{z}}\mathrm{e}$ dla $m=10$ uklad jest sprzeczny, a dla $m\neq 10$ ma

jedno rozwiazanie. Zauwazyč, $\dot{\mathrm{z}}\mathrm{e}$ dla $\dot{\mathrm{z}}$ adnego $m \in \mathrm{R}$ para (l, l) nie jest

rozwiazaniem ukladu.

24.4. Określič dziedzine dla kata $\alpha$ porównujac ten $\mathrm{k}\mathrm{a}\mathrm{t}\mathrm{z}$ jego rzutem

prostokatnym na podstawe. $\mathrm{Z}$ twierdzenia $0$ trzech prostopadlych uza-

sadnič, $\dot{\mathrm{z}}\mathrm{e}$ {\it AB} $\perp BD'$. Wywnioskowač stad, $\dot{\mathrm{z}}\mathrm{e} \mathrm{k}\mathrm{a}\mathrm{t} DBD'$ jest katem

plaskim kata dwuściennego miedzy plaszczyzna ABD'E' $\mathrm{i}$ podstawa gra-

niastoslupa.

24.5. Rozwazyč przypadki $x < 1$ oraz $x > 1 \mathrm{i}$ pomnozyč obie strony

przez mianownik (dodatni lub ujemny, odpowiednio). Jedna $\mathrm{z}$ nierówności

podwójnych jest automatycznie spelniona, a druga, przez podstawienie

$2^{x}=t$, sprowadza $\mathrm{s}\mathrm{i}\mathrm{e}$ do do nierówności kwadratowej. Nie potrzeba rozwazač

nierówności $\mathrm{w}\mathrm{y}\dot{\mathrm{z}}$ szego stopnia.





122

24.6. Zauwazyč, $\dot{\mathrm{z}}\mathrm{e}$ tg $82^{\circ}30' = \displaystyle \frac{1}{\mathrm{t}\mathrm{g}7^{\circ}30'}$ oraz $\dot{\mathrm{z}}\mathrm{e}82^{\circ}30'-7^{\circ}30' = 75^{\circ}$

$\mathrm{i}$ zastosowač wzór na tangens róznicy katów. Nastepnie korzystač $\mathrm{z}$ rów-

ności $75^{\circ}=45^{\circ}+30^{\circ}$

24.7. Skorzystač ze wskazówki do zadania 6.2, a w drugiej cześci rozwia-

zania ze wskazówki do zad. 5.8.

24.8. Przypadek $\alpha=1$ wymaga oddzielnego rozpatrzenia (dlaczego?).

Pochodna funkcji $\displaystyle \frac{b}{x^{2}-1}=b(x^{2}-1)^{-1}$ wygodniej jest obliczač za pomoca

reguly rózniczkowania funkcji zlozonej. Zauwazyč, $\dot{\mathrm{z}}\mathrm{e}$ dla $\alpha= 3, b= 32,$

gwarantujacych ciaglośči rózniczkowalnośč $f(x)$, punkt $P(3$, 4$)$ jestjej punk-

tem przegiecia.

25.1.

$t\neq 0.$

Najpierw rozpatrzyč oczywisty przypadek $t = 0$, a nastepnie

25.2. Korzystajac $\mathrm{z}$ twierdzenia Talesa wykazač, $\dot{\mathrm{z}}\mathrm{e}$ przekrój jest równo-

leglobokiem. Nastepnie prowadzič plaszczyzne symetrii czworościanu $\mathrm{i}$ sto-

sujac twierdzenie $0$ trzech prostopadlych, wykazač, $\dot{\mathrm{z}}\mathrm{e}$ przekrój jest pros-

tokatem.

25.3. Określič dziedzine nierówności. Zauwazyč, $\dot{\mathrm{z}}\mathrm{e}$ szukany zbiór jest

symetryczny wzgledem poczatku ukladu, co pozwala ograniczyč rozwazania

do I čwiartki ukladu. Rozpatrzyč przypadki $xy>1$ oraz $xy<1.$

25.4. Pólprosta wychodzaca ze środka okregu $\mathrm{i}$ zawierajaca dany punkt

$A$ przecina ten okrag $\mathrm{w}$ punkcie $A'\mathrm{l}\mathrm{e}\dot{\mathrm{z}}$ acym najblizej punktu $A$. Stad $|AA'|$

jest odleglościa punktu $A$ od danego okregu. Prowadzac rozwazania geo-

metryczne uzasadnič, $\dot{\mathrm{z}}\mathrm{e}$ dla punktów $\mathrm{l}\mathrm{e}\dot{\mathrm{z}}$ acych wewnatrz okregu zachodzi

relacja $OA+PA=10$, co oznacza, $\dot{\mathrm{z}}\mathrm{e}A\mathrm{l}\mathrm{e}\dot{\mathrm{z}}\mathrm{y}$ na elipsie $0$ ogniskach $O\mathrm{i}P$

(por. wskazówka do $\mathrm{z}\mathrm{a}\mathrm{d}$. 4.6). Inaczej jest, gdy $A\mathrm{l}\mathrm{e}\dot{\mathrm{z}}\mathrm{y}$ na zewnatrz danego

okregu.

25.5. Wszystkie przeprowadzane losowania sa wzajemnie niezalezne,

wiec ich kolejnośč nie ma wplywu na prawdopodobieństwo rozwazanego

zdarzenia. Oznaczyč przez $K, N$ zdarzenia polegajace na $\mathrm{t}\mathrm{y}\mathrm{m}, \dot{\mathrm{z}}\mathrm{e}$ dziecko,

odpowiednio, Kowalskich, Nowakowskich zostalo wybrane przedstawicielem.





123

Wówczas $ K\cap N=\emptyset$ oraz $ K\cup N=\Omega$. Nastepnie zastosowač wzór na praw-

dopodobieństwo calkowite.

25.6. Unikač niewygodnego dowodu redukcyjnego, ajeśli $\mathrm{s}\mathrm{i}\mathrm{e}$ go stosuje,

pamietač $0$ odpowiednim zakończeniu potrzebnym dla poprawności rozu-

mowania.

25.7. Nie tracič czasu na badanie wlasności, których ta funkcja nie $\mathrm{m}\mathrm{o}\dot{\mathrm{z}}\mathrm{e}$

mieč (np. asymptoty ukośne). Do obliczania pochodnej przedstawič funkcje

$\mathrm{w}$ postaci iloczynu funkcji potegowych, $\mathrm{t}\mathrm{j}. f(x)=\sqrt{3}(x-1)^{1/2}(5-x)^{-1/2}$

$\mathrm{i}$ zastosowač regule rózniczkowania iloczynu. Zauwazyč, a nastepnie wyka-

zač, $\dot{\mathrm{z}}\mathrm{e}$ prosta $x= 1$ jest styczna do wykresu $f(x) \mathrm{w}$ punkcie $x= 1$ (por.

wskazówka do $\mathrm{z}\mathrm{a}\mathrm{d}$. 3.6).

25.8. Wykazač, $\dot{\mathrm{z}}\mathrm{e}$ kolejne odcinki lamanej tworza ciag geometryczny

$0$ ilorazie mniejszym od l. Nastepnie zastosowač wzór na sume wyrazów

nieskończonego ciagu geometrycznego lub uzasadnič, $\dot{\mathrm{z}}\mathrm{e}$ suma tajest równa

obwodowi danego trójkata.

26.1. Odcinek pasa laczacy oba kola jest styczny do $\mathrm{k}\mathrm{a}\dot{\mathrm{z}}$ dego $\mathrm{z}$ nich,

wiec prostopadly do promieni poprowadzonych do punktów styczności. Nie

$\mathrm{u}\dot{\mathrm{z}}$ ywač zapisu postaci $ 26\displaystyle \frac{2}{3}\pi$ cm który jest niejednoznaczny.

26.2. Zachowač podana $\mathrm{w}$ zadaniu kolejnośč obliczeń.

26.3. Wygodnie jest posluzyč $\mathrm{s}\mathrm{i}\mathrm{e}$ rachunkiem wektorowym. Oznaczyč

przez $A, B$ punkty przeciecia $\mathrm{s}\mathrm{i}\mathrm{e}$ szukanej prostej $l$ odpowiednio $\mathrm{z}$ prosta $k$

$\mathrm{i}m$. Wówczas mamy $A(x,x+3)$. Wyrazič $\vec{AP}\mathrm{i}\vec{AB}=2\vec{AP}$ przy pomocy

niewiadomej $x \mathrm{i}$ korzystajac $\mathrm{z}$ faktu, $\dot{\mathrm{z}}\mathrm{e} B \mathrm{l}\mathrm{e}\dot{\mathrm{z}}\mathrm{y}$ na prostej $m$ obliczyč $x.$

$\rightarrow$

Wektor normalny prostej $l$ jest prostopadly do $AB.$

26.4. Wierzcholek $C$ kata prostego, spodek $O$ wysokości ostroslupa

ijego rzuty prostokatne $K, L$ na przyprostokatne podstawy tworza kwadrat

$0$ boku $r$. Stad wynika, $\dot{\mathrm{z}}\mathrm{e}$ rzuty prostokatne punktów $K\mathrm{i}L$ na krawed $\acute{\mathrm{z}}DC$

pokrywaja $\mathrm{s}\mathrm{i}\mathrm{e}$ ($\mathrm{w}$ punkcie $E$), zatem $\beta=\angle KEL$. Wyznaczyč dziedzine dla

$\beta$. Wysokośč czworościanu obliczyč $\mathrm{z}$ podobieństwa odpowiednich trójkatów

$\mathrm{w}$ przekroju plaszczyzna $ODC.$





124

26.5. Skorzystač $\mathrm{z}$ parzystości funkcji oraz ze wzoru $\log_{c}\alpha^{2}=2\log_{c}|\alpha|,$

$c>0, c\neq 1, \alpha\neq 0$. Wykres funkcji $f$ otrzymujemy $\mathrm{z}$ wykresu standardowej

krzywej $y=\log_{2}x$ przez translacje $\mathrm{i}$ odbicia symetryczne.

26.6. Zastosowač wzór $\displaystyle \cos 2x=\frac{1-\mathrm{t}\mathrm{g}^{2}x}{1+\mathrm{t}\mathrm{g}^{2}x}\mathrm{i}$ podstawič tg $x=t.$

26.7. Dwie funkcje $\mathrm{m}\mathrm{o}\dot{\mathrm{z}}$ na zlozyč wtedy $\mathrm{i}$ tylko wtedy, gdy zbiór wartości

funkcji wewnetrznej jest zawarty $\mathrm{w}$ dziedzinie funkcji zewnetrznej. Przy-

padek $\alpha=0$ rozpatrzyč oddzielnie.

26.8. Patrz wskazówka do zadania l2.8.

27.1. Wyznaczyč dziedzine równania. Aby istnialo rozwiazanie, prawa

strona musi byč nieujemna. Wtedy obie strony $\mathrm{m}\mathrm{o}\dot{\mathrm{z}}$ na podnieśč do kwadratu.

Przypadek $p=0$ rozpatrzyč oddzielnie.

27.2. Zauwazyč, $\dot{\mathrm{z}}\mathrm{e}$ środki okregów $K\mathrm{i}K_{1}$ oraz punkt $S\mathrm{l}\mathrm{e}\dot{\mathrm{z}}$ a na prostej

prostopadlej do danej prostej. Nastepnie korzystač $\mathrm{z}$ zalezności miedzy

promieniami rozwazanych okregów.

27.3. Dane określaja jednoznacznie przekatna $AC$ trapezu, na której,

jako na cieciwie okregu, jest oparty $\mathrm{k}\mathrm{a}\mathrm{t}$ ostry przy wierzcholku $B$ podstawy.

Przez zmiane polozenia punktu $B$ na okregu, poczynajac od punktu $C,$

otrzymujemy rózne trapezy (tj. $0$ róznych wartościach $d$). Minimalne $d$

odpowiada sytuacji, gdy krótsza podstawa trapezu jest równa zeru $\mathrm{i}$ trapez

staje $\mathrm{s}\mathrm{i}\mathrm{e}$ trójkatem, a maksymalne, gdy $B$ pokrywa $\mathrm{s}\mathrm{i}\mathrm{e} \mathrm{z} C$. Wysokośč

trapezu obliczyč $\mathrm{z}$ twierdzenia Pitagorasa $\mathrm{i}$ stad bezpośrednio ramie trapezu.

27.4. Najpierw ustalič dziedzine dla kata $\beta$ (porównujac go $\mathrm{z}$ rzutem

prostokatnym na podstawe). $\mathrm{Z}$ twierdzenia $0$ trzech prostopadlych wywnios-

kowač, $\dot{\mathrm{z}}\mathrm{e}$ przekrój ostroslupa jest deltoidem. $\mathrm{W}$ obliczeniach korzystač

$\mathrm{z}$ podobieństwa trójkatów $\mathrm{i}$ twierdzenia $0$ środkowych $\mathrm{w}$ trójkacie.

27.5. Przenieśč niewymiernośč do mianownika, stosujac wzór na róznice

sześcianów $\mathrm{i}$ podzielič licznik $\mathrm{i}$ mianownik przez $n^{13/5}$ Skorzystač $\mathrm{z}$ faktu,

$\dot{\mathrm{z}}\mathrm{e}\alpha<3.$





125

27.6. Stosujac definicje logarytmu sprowadzič dana nierównośč do pros-

tej nierówności trygonometrycznej. Od razu ograniczyč $\mathrm{s}\mathrm{i}\mathrm{e}$ do dziedziny

(I čwiartka, cosinus dodatni), co pozwala latwo rozwiazač $\mathrm{t}\mathrm{e}$ nierównośč.

27.7. Rozwazmy losowanie jednej liczby $\mathrm{i}$ odpowiadajacy mu model

probabilistyczny $\Omega_{0} \mathrm{i}P_{0}$. Niech $ A\subset \Omega_{0}$ oznacza zdarzenie, $\dot{\mathrm{z}}\mathrm{e}$ liczba czy-

tana od strony lewej do prawej jest podzielna przez 4, a $B$ zdarzenie, $\dot{\mathrm{z}}\mathrm{e}$

liczba czytana od strony prawej do lewej jest podzielna przez 4. Wówczas

zdarzenia $A, B$ sa niezalezne (dlaczego?). $P_{0}(A\cup B)$ obliczyč, znajac

$P_{0}(A)\mathrm{i}P_{0}(B)$. Zauwazyč, $\dot{\mathrm{z}}\mathrm{e}P_{0}(A\cup B)$ jest prawdopodobieństwem sukcesu

$\mathrm{w}$ schemacie czterech prób Bernoulliego.

27.8. Szukany zbiór jest przekrojem pasa pomiedzy dwiema prostymi

równoleglymi $\mathrm{i}$ zbioru punktów $\mathrm{l}\mathrm{e}\dot{\mathrm{z}}$ acych pod wykresem $\mathrm{i}$ na wykresie funkcji

$f(x) = \sqrt[3]{x}$. Zwrócič uwage na przebieg tej funkcji $\mathrm{w}$ otoczeniu punktu

$x = 0. \mathrm{W}$ dwóch punktach wykres funkcji $f(x)$ jest styczny do danych

prostych, a $\mathrm{w}$ dwóch innych przecina te proste pod tym samym katem

(dlaczego?). Do obliczenia tangensa tego kata $\mathrm{u}\dot{\mathrm{z}}$ yč pochodnej.

28.1. Nie wyznaczač predkości obu punktów, lecz od razu ich stosunek.

28.2. Aby nierównośč byla spelniona dla $\mathrm{k}\mathrm{a}\dot{\mathrm{z}}$ dego $x\in \mathrm{R}$, mianownik nie

$\mathrm{m}\mathrm{o}\dot{\mathrm{z}}\mathrm{e}$ mieč pierwiastków rzeczywistych, czyli jest dodatni na calej prostej.

Wtedy $\mathrm{m}\mathrm{o}\dot{\mathrm{z}}$ na obie strony pomnozyč przez ten mianownik, zachowujac znak

nierówności $\mathrm{i}$ badač nieujemnośč otrzymanego trójmianu kwadratowego.

Przypadek $p=1$ rozpatrzyč oddzielnie.

28.3. Zastosowač twierdzenie cosinusów. Nie wyznaczač dlugości boków,

lecz od razu ich iloczyn. Określič dziedzine dla $\alpha, r\mathrm{i}d.$

28.4. Przekrój plaszczyzna symetrii zawiera środek kuli, środek jed-

nej nózki oraz środek odcinka laczacego pozostale nózki. Wykonač rysunek

tego przekroju, przyjmujac $r$ bardzo male $\mathrm{w}$ porównaniu $\mathrm{z}R$. Korzystač

$\mathrm{z}$ twierdzenia $0$ okregach stycznych zewnetrznie.

28.5. Rozwiazanie $\mathrm{w}$ przedziale (-00, 0) wyznaczyč bezpośrednio, ko-

rzystajac ze wzoru na sześcian sumy. $\mathrm{W}(0,\infty)$ wyznaczyč przedzialy mono-





126

toniczności, ekstrema lokalne oraz wartośč wielomianu $\mathrm{w} x = 0$. Stad

$\mathrm{i}\mathrm{z}$ wlasności Darboux określič liczbe rozwiazań (nie wyznaczač ich jawnie).

28.6. Skorzystač ze wzoru $\alpha^{n}-b^{n}=(\alpha-b)(\alpha^{n-1}+\alpha^{n-2}b++b^{n-1})$

$\mathrm{i}$ rozwazyč oddzielnie $n$ parzyste $\mathrm{i}$ nieparzyste.

28.7. Napisač warunki określajace dziedzine (warunek istnienia sum

obu nieskończonych ciagów geometrycznych), nie wyznaczajac jej $\mathrm{w}$ sposób

jawny. Podstawič $\cos x = t \mathrm{i}$ wyeliminowač pierwiastki nie nalezace do

dziedziny.

28.8. Za pomoca pochodnej napisač równanie stycznej $\mathrm{w}$ punkcie

$P(x_{0},\displaystyle \frac{x_{0}^{2}}{2}) \mathrm{l}\mathrm{e}\dot{\mathrm{z}}$ acym na danej paraboli $\mathrm{i}$ bezpośrednio stad równanie prostej

prostopadlej do stycznej przechodzacej przez $P$. Wyznaczyč wspólrzedne

środka rozwazanego odcinka tej normalnej $\mathrm{i}$ po wyeliminowaniu parametru

$x_{0}$ otrzymač równanie krzywej. Zauwazyč, $\dot{\mathrm{z}}\mathrm{e}x_{0}$ nie $\mathrm{m}\mathrm{o}\dot{\mathrm{z}}\mathrm{e}$ byč równe zeru

(dlaczego?).

29.1. Oznaczyč $C(x,3x-14)$. Wyznaczyč środek $S$ odcinka AB. Ko-

$\rightarrow$

$\rightarrow$

rzystajac $\mathrm{z}$ prostopadlości wektorów AB $\mathrm{i}SC$ (iloczyn skalarny równy zeru)

wyznaczyč niewiadoma $x.$

29.2. Oznaczyč przez $x$ liczbe pieciocyfrowa powstala po skreśleniu pier-

wszej cyfry $\mathrm{i}$ ulozyč równanie liniowe $\mathrm{z}$ niewiadoma $x.$

29.3. Wyrazič promień okregu wpisanego za pomoca krótszego ramie-

nia $c$. Uzasadnič, $\dot{\mathrm{z}}\mathrm{e}$ środek $O$ okregu wpisanego $\mathrm{i}$ krótsza podstawa $CD$

wyznaczaja trójkat, $\mathrm{w}$ którym wysokośč do boku $CD$ tworzy $\mathrm{z}$ odcinkami

$OC\mathrm{i}OD$ katy $\displaystyle \alpha \mathrm{i}\frac{\alpha}{2}$. Stad wyznaczyč $|CD|.$

29.4. Najpierw rozpatrzyč przypadek oczywisty, gdy $x^{2}-x-2<0$. Po-

zostale przypadki, przez odwrócenie ulamków po obu stronach nierówności,

prowadza do nierówności kwadratowych (uwaga na znak nierówności).

29.5. Ustalič dziedzine nierówności $\mathrm{i}$ rozpatrzyč przypadki $x< 1$ oraz

$x>1$. Wykres funkcji $f(x)=1+\sqrt[3]{x-1}$ jest translacja standardowej krzy-





127

wej $y=\sqrt[3]{x}\mathrm{i}$ powinien byč sporzadzony dokladnie, szczególnie $\mathrm{w}$ otoczeniu

punktu $x=1$. Opisač, które cześci brzegu wyznaczonego zbioru naleza do

tego zbioru.

29.6. Wygodna metoda przeksztalcania obu stron jest przejście do cos-

inusów podwojonych katów $(2\sin^{2}\gamma = 1-\cos 2\gamma, \mathrm{p}\mathrm{o}\mathrm{r}$. wskazówka do

$\mathrm{z}\mathrm{a}\mathrm{d}$. 4.3$)$. Otrzymane serie rozwiazań polaczyč $\mathrm{w}$ dwie serie.

29.7. Uzasadnič, $\dot{\mathrm{z}}\mathrm{e}$ dziedzina szukanego kata jest przedzial $(\displaystyle \frac{\pi}{2},\pi).$

Poprowadzič przekrój plaszczyzna symetrii przechodzaca przez wierzcholek

ostroslupa $\mathrm{i}$ środki przeciwleglych krawedzi podstawy $\mathrm{i}$ korzystač $\mathrm{z}$ podobień-

stwa odpowiednich trójkatów. Cosinus szukanego kata wyznaczyč za po-

moca twierdzenia cosinusów.

29.8. Wyznaczyč dziedzine $D$ funkcji $S(x)$, pamietač $0x=-1$. Posluzyč

$\mathrm{s}\mathrm{i}\mathrm{e}$ pochodna funkcji, ale nie wyznaczač ekstremów lokalnych, lecz ograniczyč

$\mathrm{s}\mathrm{i}\mathrm{e}$ do podania wartości najwiekszej $\mathrm{i}$ najmniejszej funkcji $S(x)\mathrm{w}D.$

30.1. Objetośč rozwazanej bryly jest róznica objetości dwóch stozków

$0$ wspólnej podstawie. Oznaczyč dluzsza przyprostokatna przez $\alpha$, krótsza

przez $b$, a objetośč stozka powstalego $\mathrm{z}$ obrotu trójkata wokól krótszej

przyprostokatnej przez $V_{1}$. Wtedy $V_{1} \geq V_{2}$. Nie wyznaczač przypros-

tokatnych ani innych wielkości liniowych, lecz od razu objetośč $\mathrm{i}$ po wye-

liminowaniu $\alpha \mathrm{i}b$ wyrazič $\mathrm{j}\mathrm{a}$ przez $V_{1}\mathrm{i}V_{2}.$

30.2. Przyjač wysokośč najmniejszej nagrody, róznice ciagu oraz liczbe

nagród $n$ za niewiadome. Ulozyč uklad dwóch równań $\mathrm{i}$ wykazač, $\dot{\mathrm{z}}\mathrm{e}$

$4\leq n\leq 6$. Rozwiazania wyznaczamy przez bezpośrednie sprawdzenie.

30.3. Równania okregów, których środki $\mathrm{l}\mathrm{e}\dot{\mathrm{z}}$ a na prostej $y = 1$, wyz-

naczyč bezpośrednio $\mathrm{z}$ twierdzenia $0$ okregach stycznych zewnetrznie lub

wewnetrznie. Środki pozostalych okregów otrzymujemy po rozwiazaniu

odpowiedniego ukladu równań.

30.4. Korzystamy $\mathrm{z}$ twierdzenia cosinusów. Nie wyznaczamy boków

równolegloboku, lecz tylko ich iloczyn $\mathrm{i}$ przez porównanie dwóch wyrazeń

na pole równolegloboku otrzymujemy od razu tangens szukanego kata.





15

Praca kontrolna

nr 7

7.1. Rozwiazač nierównośč

$|9^{x}-2|<3^{x+1}-2.$

7.2. Wyznaczyč równanie krzywej bedacej obrazem okregu

$(x+1)^{2}+(y-6)^{2}=4\mathrm{w}$ powinowactwie prostokatnym $0$ osi $Ox\mathrm{i}$ sto-

sunku $k=\displaystyle \frac{1}{2}$. Obliczyč pole figury ograniczonej ta krzywa. Sporzadzič

staranny rysunek.

7.3. Pewien zbiór zawiera dokladnie 67 podzbiorów $0$ co najwyzej dwóch

elementach. Ile podzbiorów siedmioelementowych zawiera ten zbiór?

7.4. Trapez $0$ katach przy podstawie wynoszacych $15^{\circ}\mathrm{i}45^{\circ}$ opisano na kole

$0$ promieniu $R$. Obliczyč stosunek pola kola do pola tego trapezu.

7.5. Rozwiazač uklad równań

$\left\{\begin{array}{l}
mx-6y=3\\
2x+(m-7)y=m-1
\end{array}\right.$

$\mathrm{w}$ zalezności od parametru rzeczywistego $m$. Podač wszystkie rozwia-

zania ($\mathrm{i}$ odpowiadajace im wartości parametru $m$), dla których $x$ jest

równe $y.$

7.6. Rozwiazač nierównośč $\sin 2x<\sin x\mathrm{w}$ przedziale $[-\displaystyle \frac{\pi}{2},\frac{\pi}{2}]$.

zanie zilustrowač starannym wykresem.

Rozwia-

7.7. Ostroslup podzielono na trzy cześci dwiema plaszczyznami równolegly-

mi do jego podstawy. Pierwsza plaszczyznajest polozona $\mathrm{w}$ odleglości

$d_{1} = 2$ cm, a druga $\mathrm{w}$ odleglości $d_{2} = 3$ cm od podstawy. Pola

przekrojów ostroslupa tymi plaszczyznami równe sa odpowiednio

$S_{1} = 25 \mathrm{c}\mathrm{m}^{2}$ oraz $S_{2} = 16 \mathrm{c}\mathrm{m}^{2}$ Obliczyč objetośč tego ostroslupa

oraz objetośč najmniejszej cześci.

7.8. Trylogie skladajaca $\mathrm{s}\mathrm{i}\mathrm{e} \mathrm{z}$ dwóch powieści dwutomowych oraz jednej

jednotomowej ustawiono na pólce $\mathrm{w}$ przypadkowej kolejności. Jakie

jest prawdopodobieństwo tego, $\dot{\mathrm{z}}\mathrm{e}$ tomy a) obydwu, b) co najmniej

jednej $\mathrm{z}$ dwutomowych powieści znajduja $\mathrm{s}\mathrm{i}\mathrm{e}$ obok siebie $\mathrm{i}$ przy tym

tom I $\mathrm{z}$ lewej, a tom II $\mathrm{z}$ prawej strony.





128

30.5. Określič dziedzine równania. Poniewaz $\mathrm{w}$ dziedzinie obie strony

równania sa dodatnie $\mathrm{m}\mathrm{o}\dot{\mathrm{z}}$ na podnieśč je do kwadratu.

30.6. Wyznaczyč wszystkie (trzy) pierwiastki danego wielomianu. Je-

den $\mathrm{z}$ nich nie $\mathrm{m}\mathrm{o}\dot{\mathrm{z}}\mathrm{e}$ byč pierwiastkiem trójmianu $2x^{2}+\alpha x+b$ (dlaczego?).

Znajac dwa pierwiastki, napisač ten trójmian $\mathrm{i}$ odczytač niewiadome $\alpha \mathrm{i}b.$

30.7. Podstawič $2^{x}=t$. Korzystač $\mathrm{z}\mathrm{t}\mathrm{o}\dot{\mathrm{z}}$ samości podanej we wskazówce

do $\mathrm{z}\mathrm{a}\mathrm{d}$. 5.1. Wykresy obu stron otrzymač przez translacje $\mathrm{i}$ odbicia syme-

tryczne standardowej krzywej $y=2^{x}$

30.8. Wyznaczyč miejsca zerowe pochodnej $\mathrm{i}$ za pomoca wykresu rozwia-

zač odpowiednia nierównośč trygonometryczna.

31.1. Określič model probabilistyczny. Rozwazane zdarzenie przed-

stawič $\mathrm{w}$ postaci sumy rozlacznych zdarzeń $B_{1}\cup B_{2}\cup B_{3}\cup B_{4}$, gdzie $B_{i}$ jest

zdarzeniem polegajacym na otrzymaniu przez gracza 4 kart $\mathrm{w}i$-tym kolorze

$\mathrm{z}$ asem, królem $\mathrm{i}$ dama. $P(B_{i})$ obliczyč bezpośrednio, pamietajac, $\dot{\mathrm{z}}\mathrm{e}P$ jest

prawdopodobieństwem klasycznym $\mathrm{i}$ skorzystač $\mathrm{z}$ wlasności prawdopodo-

bieństwa.

31.2. Patrz wskazówka do zad. 7.7.

31.3. Określič dziedzine ukladu. Zwrócič uwage najej asymetrie wzgle-

dem zmiennych $x\mathrm{i}y$. Dodajac $\mathrm{i}$ odejmujac równania stronami przejśč do

ukladów równań liniowych.

31.4. Najpierw określič ($\mathrm{i}$ uzasadnič geometrycznie) dziedzine dla kata

$\alpha$ oraz wyznaczyč katy trójkata $ABC.$

31.5. $\mathrm{W}$ dowodzie kroku indukcyjnego przeksztalcač lewa strong $\mathrm{i}$ do-

prowadzič $\mathrm{j}\mathrm{a}$ do równości $\mathrm{z}$ prawa strong. Korzystač ze wzoru na róznice

sinusów. Nie prowadzič dowodu metoda redukcji.

31.6. Pomnozyč licznik $\mathrm{i}$ mianownik przez $\sqrt{n} + \sqrt{n+\sqrt[3]{4n^{2}}},$

a nastepnie podzielič je przez $n^{2/3}$, zamieniajac wcześniej pierwiastki na

potegi $0$ wykladnikach ulamkowych.





129

31.7. Korzystač $\mathrm{z}$ nastepujacej wlasności wektorów na plaszczy $\acute{\mathrm{z}}\mathrm{n}\mathrm{i}\mathrm{e}$

(uzasadnič $\mathrm{j}\mathrm{a}$):

{\it Jeśli} $\text{{\it ũ}}\perp\vec{v},\ ||${\it ũ}$|| =||\vec{v}||$ {\it oraz ũ}$= (\alpha,b)$, {\it to} $\vec{v}=(b,-\alpha) lub\vec{v}=(-b,\alpha).$

Sugeruje ona, $\dot{\mathrm{z}}\mathrm{e}$ zadanie ma dwa rozwiazania.

31.8. Zapisač funkcje $\mathrm{w}$ postaci $f(x) = x^{1/2} +x^{-1/2} \mathrm{i}$ obliczyč

pochodna ze wzoru na pochodna funkcji potegowej. Zauwazyč, $\dot{\mathrm{z}}\mathrm{e}$

$\displaystyle \lim_{x\rightarrow+\infty}(f(x)-\sqrt{x})=0$. Jaka wlasnośč geometryczna wykresu funkcji $f(x)$

opisuje ta równośč?

32.1. Oznaczyč przez $x$ predkośč statku, przez $y$ predkośč wody, a przez

$d$ odleglośč $\mathrm{z}$ Wroclawia do Szczecina. Zapisač odpowiednie równania $\mathrm{i}$ nie

wyznaczajac niewiadomych, odpowiedzieč tylko na postawione pytanie.

32.2. Sprowadzič wszystkie logarytmy do tej samej podstawy 2 lub 8

$\mathrm{i}$ skorzystač $\mathrm{z}$ definicji ciagu geometrycznego.

32.3. Narysowač przekrój pionowy wanny $\mathrm{z}\mathrm{l}\mathrm{e}\dot{\mathrm{z}}\mathrm{a}\mathrm{c}\mathrm{a}$ na dnie belka. Ponie-

$\mathrm{w}\mathrm{a}\dot{\mathrm{z}}$ średnica belki jest polowa promienia wanny, wjej przekroju pionowym

pojawiaja $\mathrm{s}\mathrm{i}\mathrm{e}$ trójkaty równoboczne.

32.4. Zarówno $v(x)$, jak $\mathrm{i} w(x)$ musza mieč dwa rózne pierwiastki

rzeczywiste. To daje dziedzine dla parametru $m$. Obliczyč pierwiastki $x_{1},$

$x_{2}$ wielomianu $w(x)$. Jeśli wierzcholek paraboli $0$ równaniu $y=v(x) \mathrm{l}\mathrm{e}\dot{\mathrm{z}}\mathrm{y}$

pomiedzy $x_{1}\mathrm{i}x_{2}$ oraz $v(x_{1})\mathrm{i}v(x_{2})$ sa dodatnie, to wymagany warunek jest

spelniony.

32.5. Rozwazyč nastepujace zdarzenia: $C -$ wylosowano co najmniej

dwie kule biale, $D \mathrm{z}$ urny $\mathrm{B}$ wylosowano kule biala, $E_{i} - \mathrm{z}$ urny $\mathrm{A}$

wylosowano $i$ kul bialych, $i=0$, 1, 2, 3, 4. Wówczas $C'=E_{0}\cup D'\cap E_{1}.$

Skorzystač $\mathrm{z}$ niezalezności zdarzeń $D, E_{i}$, rozlaczności zdarzeń $E_{0}, D'\cap E_{1}$

oraz ze schematu Bernoulliego.

32.6. Wyznaczyč dziedzine równania. Pomnozyč obie strony przez $\cos x$

$\mathrm{i}$ po zastosowaniu wzorów $\sin 2x = 2\sin x\cos x$ oraz $\cos 2x = 1-2\sin^{2}x$

rozlozyč wyrazenie na czynniki, wylaczajac przed nawias czynnik

$(\sin x-\cos x).$





130

32.7. Napisač równanie stycznej $\mathrm{w}$ punkcie $S(x_{0},x_{0}^{4}-2x_{0}^{2})$, gdzie

$x_{0} \in \mathrm{R}$, nastepnie wyznaczyč wszystkie $x_{0}$, dla których $P \mathrm{l}\mathrm{e}\dot{\mathrm{z}}\mathrm{y}$ na sty-

cznej (trzy punkty). Dwa $\mathrm{z}$ nich wyznaczaja $\mathrm{t}\mathrm{e}$ sama styczna, a trzeci inna.

Sporzadzič wykres funkcji $f(x)$, korzystajac zjej parzystości oraz informacji

zebranych przy wyznaczaniu stycznych bez dalszego badania jej przebiegu.

32.8. $\mathrm{Z}$ twierdzenia $0$ trzech prostopadlych wywnioskowač, $\dot{\mathrm{z}}\mathrm{e}$ plasz-

czyzna $SCD$ jest plaszczyzna symetrii ostroslupa, a wiec zawiera środek kuli

opisanej. $\mathrm{L}\mathrm{e}\dot{\mathrm{z}}\mathrm{y}$ on na prostej prostopadlej do podstawy ostroslupa wysta-

wionej $\mathrm{w}$ środku okregu opisanego na podstawie. Wykazač, $\dot{\mathrm{z}}\mathrm{e}\triangle SCD$ jest

równoboczny $\mathrm{i}$ stad określič polozenie środka kuli.

33.1. Zastosowač wzór Newtona. Liczba $x$ jest wieksza od $y$

$\mathrm{g}\mathrm{d}\mathrm{y}x= (1+\displaystyle \frac{p}{100})y.$

0

{\it p}\%,

33.2. Zastosowač wzór na odleglośč punktu od prostej. Nalezy za-

uwazyč, $\dot{\mathrm{z}}\mathrm{e}$ punkt przeciecia $\mathrm{s}\mathrm{i}\mathrm{e}$ prostych $k\mathrm{i}l$ nie spelnia $\dot{\mathrm{z}}$ adanego warunku.

33.3. Skorzystač $\mathrm{z}$ twierdzenia $0$ dwusiecznej kata $\mathrm{w}$ trójkacie oraz ze

wzoru Herona.

33.4. Iloraz $q$ ciagu $(\alpha_{n})$ jest mniejszy od l, wiec droga przebyta przez

czastke jest skończona $\mathrm{i}$ ruch czastki kończy $si_{G} \mathrm{w}$ punkcie $P$. Znajac

wspólrzedne tego punktu, ulozyč dwa równania $\mathrm{z}$ niewiadomymi $\alpha_{1}\mathrm{i}q.$

33.5. Nie $\mathrm{u}\dot{\mathrm{z}}$ ywač algorytmu dzielenia wielomianów, lecz umiejetnie

stosowač rozklad na czynniki np. $x^{4} +x^{2} + 1 = (x^{2}+1)^{2} - x^{2} =$

$=(x^{2}+x+1)(x^{2}-x+1)$. Podobnie postepowač $\mathrm{w}$ dowodzie kroku induk-

cyjnego.

33.6. Oddzielnie rozwazyč przedzialy $(0,\infty)$ oraz (-00, 0). Wykresy

$\mathrm{w}$ tych przedzialach sa istotnie róznymi krzywymi. Nazwač je. Dokladnie

stosowač definicje asymptoty ukośnej prawostronnej $\mathrm{i}$ lewostronnej.

33.7. Przypadek $|\cos x| =$ l jest oczywisty. Dla przypadku

$0< |\cos x| <1$ przejśč do porównania wykladników obu stron. Rozwiazač

odpowiednie równanie trygonometryczne $\mathrm{i}$ za pomoca wykresu wyznaczyč





131

zbiór rozwiazań nierówności.

metrycznym.

Wygodnie jest posluzyč sie kolem trygono-

33.8. Wykonač przekrój osiowy stozka przechodzacy przez jedna

$\mathrm{z}$ krawedzi graniastoslupa. Wyrazič stosunek objetości bryl jako funkcje

zmiennej $x =$ tg $\alpha \in (0,\infty)$. Nie mylič postawionego pytania $\mathrm{z}$ zagad-

nieniem wyznaczania ekstremów lokalnych.

34.1. Napisač uklad równań $\mathrm{z}$ niewiadomymi przyprostokatnymi $\alpha \mathrm{i}b.$

Nie wyznaczač ich oddzielnie, lecz tylko sume $\alpha+b$ potrzebna do obliczenia

obwodu.

34.2. Skorzystač ze wzoru na sume sześcianów oraz ze wzorów na $\sin 2\gamma$

$\mathrm{i}\cos 2\gamma.$

34.3. Warunkiem stycznościjest istnienie pierwiastka podwójnego odpo-

wiedniego trójmianu kwadratowego. Zadanie ma wiecej $\mathrm{n}\mathrm{i}\dot{\mathrm{z}}$ jedno rozwia-

zanie.

34.4. Wektory (swobodne) $\vec{u}\mathrm{i}\vec{v}$ sa równolegle, gdy $\vec{v}=c\vec{u}\mathrm{d}\mathrm{l}\mathrm{a}$ pewnego

skalara $c$. Prostopadlośč wektorów wyrazič za pomoca iloczynu skalarnego.

34.5. Oznaczyč przez $B_{i}$ zdarzenie polegajace na $\mathrm{t}\mathrm{y}\mathrm{m}, \dot{\mathrm{z}}\mathrm{e}$ za pierwszym

razem wylosowano monete $i \mathrm{z}l, i = 1$, 2, 5. Wtedy $B_{1}\cup B_{2}\cup B_{5} = \Omega$

$\mathrm{i}$ skladniki sa rozlaczne. Prawdopodobieństwo zdarzenia, $\dot{\mathrm{z}}\mathrm{e}$ Jaś wyciagnie

dokladnie dwie monety obliczyč ze wzoru na prawdopodobieństwo calkowite,

a prawdopodobieństwo, $\dot{\mathrm{z}}\mathrm{e}$ Jaś wyciagnie tylko jedna monete (czyli 5 $\mathrm{z}l$)

wynosi $\displaystyle \frac{1}{6}$. Stad otrzymač odpowied $\acute{\mathrm{z}}.$

34.6. Zastosowač wzór $\sqrt{\alpha^{2}}=|\alpha|$. Uzasadnič, $\dot{\mathrm{z}}\mathrm{e}$ krzywa $K\mathrm{o}$ równaniu

$y=\sqrt{4x-x^{2}}$ jest górna polowa okregu $0$ środku $S(2,0)\mathrm{i}$ promieniu 2. Przy

obliczaniu odleglości $P$ od brzegu $\mathcal{F}$ ograniczyč $\mathrm{s}\mathrm{i}\mathrm{e}$ do porównania odleglości

$P$ od krzywej $K$ oraz od odcinka prostej $y=1-x,  x\in (1,4)$. Pozostale

cześci brzegu $\mathcal{F}$ sa znacznie dalej polozone, co wystarczy uzasadnič przez

powolanie $\mathrm{s}\mathrm{i}\mathrm{e}$ na (staranny) rysunek.





132

34.7. Wzór określajacy $f(x)$ sprowadzič do najprostszej postaci $\mathrm{i}$ za-

uwazyč, $\dot{\mathrm{z}}\mathrm{e}$ jest ona zlozeniem dwóch funkcji rosnacych ($\mathrm{w}$ dziedzinie!).

Dziedzina $f^{-1}$ jest zbiór wartości $f\mathrm{i}$ na odwrót.

34.8. Do obliczenia krawedzi podstawy $\alpha$ wykorzystač wskazówke do

zadania 3.4. Poprowadzič przekrój ostros1upa p1aszczyzna symetrii prze-

chodzaca przez wierzcholek ostroslupa $\mathrm{i}$ odpowiednia przekatna podstawy

$\mathrm{i}$ korzystač wielokrotnie $\mathrm{z}$ podobieństwa trójkatów. Objetośč wyrazič naj-

pierw przez $\alpha \mathrm{i}$ dopiero na końcu podstawič $c$. Zadanie ma sens, gdy krawed $\acute{\mathrm{z}}$

bocznajest nachylona do podstawy pod katem co najmniej $45^{\circ}$ (dlaczego?).

Stad warunek na $\alpha.$

35.1. Wykluczyč $p = 0 \mathrm{i} \mathrm{z}$ warunku istnienia sumy nieskończonego

ciagu geometrycznego wyznaczyč $\alpha_{1}\mathrm{i}q.$

35.2. $K\mathrm{a}\mathrm{t}$ miedzy prostymi jest równy katowi miedzy ich wektorami

normalnymi (odpowiednio zorientowanymi). Napisač równania danych

prostych $\mathrm{w}$ postaci ogólnej $\mathrm{i}\mathrm{u}\dot{\mathrm{z}}$ yč iloczynu skalarnego.

35.3. Rozwazyč przekrój sześcianu plaszczyzna symetrii (zawierajacy

środek $\mathrm{i}$ kolo wielkie danej kuli oraz przekroje czterech narozników). Szu-

kana krawed $\acute{\mathrm{z}}$ obliczyč za pomoca twierdzenia Pitagorasa dla odpowiedniego

trójkata $\mathrm{w}$ tym przekroju.

35.4. Uzasadnič, $\dot{\mathrm{z}}\mathrm{e}\mathrm{w}$ przedziale [-l, l] obie strony nierówności sa nieu-

jemne $\mathrm{i}$ podnieśč je do kwadratu. Wykresy nalezy wykonač dokladnie (leza

blisko siebie), zwracajac uwage na otoczenia punktów $x=0\mathrm{i}x=-1.$

35.5. Wyznaczyč dziedzine równania. Pomnozyč obie strony przez

$\sin 2x$. Zastosowač wzór na iloczyn sinusów $\mathrm{i}\mathrm{z}$ równości dwóch cosinusów

przejśč od razu do porównywania katów.

35.6. Napisač wzór na styczna do okregu $\mathrm{w}$ punkcie $\mathrm{l}\mathrm{e}\dot{\mathrm{z}}$ acym na nim

(por. wskazówka do zadania 6.2) $\mathrm{i}$ po podstawieniu wspólrzednych punktu $P$

wyznaczyč punkt styczności, dla którego styczna ma dodatni wspólczynnik

kierunkowy.





133

35.7. Dane $r\mathrm{i}d$ jednoznacznie określaja $\mathrm{k}\mathrm{a}\mathrm{t}\alpha$ przy podstawie trapezu,

przy czym $\displaystyle \alpha>\frac{\pi}{3}$. Obwód wyrazič jako funkcje wysokości trapezu. Ustalič

dziedzine. Wartośč najwieksza funkcji wyznaczyč, badajac jej przedzialy

monotoniczności.

35.8. Wyznaczyč $y \mathrm{z}$ pierwszego równania $\mathrm{i}$ podstawič do drugiego.

Nastepnie skorzystač $\mathrm{z}\mathrm{t}\mathrm{o}\dot{\mathrm{z}}$ samości $(|\alpha|=b)\Leftrightarrow$($\alpha=b$ lub $\alpha=-b$) prawdzi-

wej dla $b\geq 0$. Otrzymane alternatywy prowadza do czterech przypadków

$m = -\displaystyle \frac{1}{2}, m = \displaystyle \frac{1}{2}, m = 1$ oraz pozostale $m$. Na rysunku zaznaczyč

odpowiednio wybrane proste $\mathrm{z}$ peku prostych (któremu odpowiada pierwsze

równanie ukladu) przechodzacych przez $P(0$, 2$).$





12  przykIadowych

rozwiązań





137

Rozwiazanie zadania 2.1

I sposób. Rozwazmy wielomian $p_{n}(y) =y^{2n-1}+1$. Poniewaz $2n-1$

jest liczba nieparzysta dla $\mathrm{k}\mathrm{a}\dot{\mathrm{z}}$ dego $n\in N$, wiec $p(-1)=(-1)^{2n-1}+1=0.$

$\mathrm{Z}$ twierdzenia Bézouta wynika wiec, $\dot{\mathrm{z}}\mathrm{e}p_{n}(y)$ jest podzielny przez dwumian

$y+1, \mathrm{t}\mathrm{z}\mathrm{n}$. istnieje wielomian $q_{n}(y)$ stopnia $2n-2$ taki, $\dot{\mathrm{z}}\mathrm{e}$

$p_{n}(y)=y^{2n-1}+1=(y+1)q_{n}(y).$

(1)

Zauwazmy, $\dot{\mathrm{z}}\mathrm{e} w_{n}(x) = x^{4n-2}+1 =p_{n}(x^{2})$. Stad $\mathrm{i} \mathrm{z}$ (l) wynika, $\dot{\mathrm{z}}\mathrm{e}$

$w_{n}(x)=(x^{2}+1)q_{n}(x^{2})$. Poniewaz $q_{n}(x^{2})$ jest wielomianem stopnia $4n-4,$

wiec równośč ta dowodzi prawdziwości tezy.

Uwaga. Stosujac wzór skróconego mnozenia

$\alpha^{2n-1}+b^{2n-1}=(\alpha+b)(\alpha^{2n-2}-\alpha^{2n-3}b+-\alpha b^{2n-3}+b^{2n-2})$

$\mathrm{m}\mathrm{o}\dot{\mathrm{z}}$ na wielomian $q_{n}(y)$ napisač $\mathrm{w}$ postacijawnej. Niejest tojednak koniecz-

ne dla poprawności dowodu.

II sposób. Dowód indukcyjny. Rozwazmy funkcje zdaniowa zmien-

nej naturalnej $n$

$T(n)$ : $w_{n}(x)=x^{4n-2}+1$ jest podzielny przez $x^{2}+1.$

Sprawdzimy teraz, $\dot{\mathrm{z}}\mathrm{e}$ dla $T(n)$ obydwa zalozenia zasady indukcji mate-

matycznej sa spelnione.

$1^{\mathrm{O}}$ Sprawdzenie prawdziwości zdania $T(1).$

Mamy $w_{1}(x)=x^{4\cdot 1-2}+1=x^{2}+1$, czyli oczywiście dzieli $\mathrm{s}\mathrm{i}\mathrm{e}$ przez $x^{2}+1,$

a wiec $T(1)$ jest prawdziwe.

$2^{\circ}$ Wykazemy, $\dot{\mathrm{z}}\mathrm{e}$ implikacja $(T(n)\Rightarrow T(n+1))$ jest prawdziwa dla

$\mathrm{k}\mathrm{a}\dot{\mathrm{z}}$ dego $n\in N.$

Dowód. Niech $n$ bedzie dowolna ustalona liczba naturalna. Zalózmy,

$\dot{\mathrm{z}}\mathrm{e}$ zdanie $T(n)$ jest prawdziwe $\mathrm{t}\mathrm{z}\mathrm{n}$. istnieje wielomian $v_{n}(x)$ taki, $\dot{\mathrm{z}}\mathrm{e}$

$w_{n}(x) =x^{4n-2}+1 = (x^{2}+1)v_{n}(x)$. Wówczas korzystajac $\mathrm{z}$ tej równości

mamy

$w_{n+1}(x)=x^{4(n+1)-2}+1=x^{4n+2}+1=(x^{4n+2}+x^{4})-(x^{4}-1)$





138

$=x^{4}(x^{4n-2}+1)-(x^{2}+1)(x^{2}-1)=x^{4}(x^{2}+1)v_{n}(x)-(x^{2}+1)(x^{2}-1)$

$=(x^{2}+1)(x^{4}v_{n}(x)-x^{2}+1).$

Poniewaz $x^{4}v_{n}(x)-x^{2}+1$ jest wielomianem, wiec powyzsza równośč

oznacza, $\dot{\mathrm{z}}\mathrm{e}w_{n+1}(x)$ dzieli $\mathrm{s}\mathrm{i}\mathrm{e}$ przez $x^{2}+1$. To kończy dowód $2^{\circ}$

$\mathrm{Z}$ wykazanej prawdziwości warunków $1^{\circ} \mathrm{i} 2^{\circ}$ oraz $\mathrm{z}$ zasady indukcji

matematycznej wynika, $\dot{\mathrm{z}}\mathrm{e} T(n)$ jest prawdziwe dla $\mathrm{k}\mathrm{a}\dot{\mathrm{z}}$ dej liczby

naturalnej $n.$

Rozwiazanie zadania 3.8

Dziedzina nierówności jest R.

cosinus róznicy katów mamy

Poniewaz $\sqrt{3}=$ tg $\displaystyle \frac{\pi}{3}$, wiec ze wzoru na

$\cos x+\sqrt{3}\sin x=\cos x +$ tg $\displaystyle \frac{\pi}{3}\sin x=$

$\displaystyle \frac{\cos x\cos\frac{\pi}{3}+\sin x\sin\frac{\pi}{3}}{\cos\frac{\pi}{3}}=2\cos(x-\frac{\pi}{3})$

Nierównośč przyjmuje zatem postač $|2\displaystyle \cos(x-\frac{\pi}{3})| \leq \sqrt{2}$. Obie strony

nierówności sa nieujemne, wiec po podniesieniu do kwadratu dostajemy

nierównośč równowazna 2$\displaystyle \cos^{2}(x-\frac{\pi}{3}) \leq 1$. Stosujemy wzór l$+ \cos 2\gamma=$

$2\cos^{2}\gamma \mathrm{i}$ przeksztalcamy $\mathrm{j}\mathrm{a}$ do prostszej postaci $\displaystyle \cos(2x-\frac{2\pi}{3}) \leq 0$. Wiemy,

$\dot{\mathrm{z}}\mathrm{e}$ cosinus jest ujemny $\mathrm{w}$ II $\mathrm{i}$ III čwiartce, otrzymujemy wiec

$\displaystyle \frac{\pi}{2}+2k\pi\leq 2x-\frac{2\pi}{3}\leq\frac{3\pi}{2}+2k\pi$, czyli

$\displaystyle \frac{7\pi}{12}+k\pi\leq x\leq\frac{13\pi}{12}+k\pi, k\in \mathrm{Z}.$

(2)

Wyznaczamy cześč wspólna zbioru rozwiazań (2) $\mathrm{i}$ przedzialu $[0,3\pi]$, dosta-

jemy (podstawiamy kolejno $k=-1, 0$, 1, 2) odpowied $\acute{\mathrm{z}}.$

Odp. $ x\in [0,\displaystyle \frac{\pi}{12}]\cup [\displaystyle \frac{7\pi}{12},\frac{13\pi}{12}]\cup [\displaystyle \frac{19\pi}{12},\frac{25\pi}{12}]\cup [\displaystyle \frac{31\pi}{12},3\pi].$





139

Rozwiazanie zadania 12.6

Oznaczmy przez $O$ środek okregu opisanego na trapezie, a przez $h$

wysokośč trapezu (rys. 25). Wówczas $P= \displaystyle \frac{1}{2}sh$, czyli $h= \displaystyle \frac{2P}{s}. \mathrm{Z}$ twier-

dzenia Pitagorasa $\mathrm{w} \triangle DEB$ otrzymujemy $d^{2}=h^{2}+\displaystyle \frac{s^{2}}{4}=\frac{16P^{2}+s^{4}}{4s^{2}}.$
\begin{center}
\includegraphics[width=78.492mm,height=74.472mm]{./KursMatematyki_PolitechnikaWroclawska_1999_2004_page119_images/image001.eps}
\end{center}
' $l D C$

{\it M}

$c r$

$c h$

$kA E s 2 B$

$\mathrm{R}\mathrm{y}\mathrm{s}$. 25

Z drugiej strony $\mathrm{z}$ twierdzenia

$0$ kacie wpisanym $\mathrm{w}$ okrag wynika

$\dot{\mathrm{z}}\mathrm{e} \angle DAE = \displaystyle \frac{1}{2}\angle DOB = \angle DOM,$

zatem trójkaty prostokatne $\triangle DAE$

$\mathrm{i} \triangle DOM$ maja identyczne katy,

czyli sa podobne. To pozwala napi-

sač proporcje $\displaystyle \frac{h}{c}= \displaystyle \frac{d}{2r}$, skad otrzy-

$2rh$

mujemy $c =$ Po podstawie-

$d$

niu obliczonej wartości $d$ mamy

$8Pr$

$c= \sqrt{16P^{2}+s^{4}}$. Ostatecznie ob-

wód wynosi $0=s+2c=s+\displaystyle \frac{16Pr}{\sqrt{16P^{2}+s^{4}}}$. Dane $P\mathrm{i}s$ wyznaczaja jedno-

znacznie $h\mathrm{i}d$. Zadanie ma zatem rozwiazanie, gdy promień $r$ jest wystar-

czajaco $\mathrm{d}\mathrm{u}\dot{\mathrm{z}}\mathrm{y}$, aby powstal trójkat $\triangle DOM, \mathrm{t}\mathrm{z}\mathrm{n}. r\displaystyle \geq\frac{1}{2}d=\frac{\sqrt{16P^{2}+s^{4}}}{4s}.$

Poprawnośč tego warunku, jak ijednoznacznośč rozwiazania, najlepiej widač

$\mathrm{z}$ opisu konstrukcji trapezu, który dla kompletności przedstawiamy ponizej.

Opis konstrukcji trapezu

1. $\mathrm{Z}$ odcinków $h \mathrm{i} \displaystyle \frac{s}{2}$, jako przyprostokatnych, konstruujemy trójkat

prostokatny $DEB$. Odcinek $BE$ przedluzamy $\mathrm{i}$ otrzymujemy prosta $k,$

a przez punkt $D$ prowadzimy prosta $l$ równolegla do $k.$

2. $\mathrm{Z}$ punktów $B\mathrm{i}D$ kreślimy luki okregów $0$ promieniu $r$, które przeci-

najac $\mathrm{s}\mathrm{i}\mathrm{e}$ daja środek okregu opisanego $O (\mathrm{z}$ dwóch punktów, $\mathrm{w}$ których

przecinaja $\mathrm{s}\mathrm{i}\mathrm{e}$ te luki, wybieramy $\mathrm{l}\mathrm{e}\dot{\mathrm{z}}\mathrm{a}\mathrm{c}\mathrm{y}$ blizej prostej $k$, która ma zawierač

dluzsza podstawe trapezu).





Edycja

XXX

2000/2001





140
\begin{center}
\includegraphics[width=73.512mm,height=64.968mm]{./KursMatematyki_PolitechnikaWroclawska_1999_2004_page120_images/image001.eps}
\end{center}
3. Ze środka $O$ kreślimy okrag

$0$ promieniu $r$. Okrag ten prze-

cinajac prosta $k$, wyznacza wierz-

cho ek $A$ (oraz przechodzi przez

$B)$. Podobnie, okrag ten przeci-

najac prosta $l$, wyznacza wierz-

cho ek $C (\mathrm{i}$ równocześnie prze-

chodzi przez $D$). Na rysunku 26

przedstawiono konstrukcje

trapezu dla danych liczbo-

wych zadania, $\mathrm{t}\mathrm{j}. P = 12 \mathrm{c}\mathrm{m}^{2},$

$h=3$ cmi $s=8$ cm.

$\mathrm{R}\mathrm{y}\mathrm{s}$. 26

$16Pr$

Odp. Obwód wynosi $s+$ a zadanie ma rozwiazanie, gdy

$\sqrt{16P^{2}+s^{4}}$'

{\it r}2$\geq$ --{\it Ps}22$+$--1{\it s}62.

Rozwiazanie zadania 21.7

Logarytm jest określony dla liczb dodatnich, wiec dziedzine równania

wyznaczaja warunki:

$D$: 

czyli $D$ : $(x\in(-4,1))\wedge(x^{3}-x^{2}-3x+5>0). \mathrm{W}$ celu rozwiazania równania

wszystkie skladniki zapiszemy jako logarytmy $0$ podstawie 4. D1a $x \in D$

jest $\displaystyle \log_{2}(1-x)=\frac{\log_{4}(1-x)}{\log_{4}2}=2\log_{4}(1-x)=\log_{4}(x-1)^{2}$ oraz $\displaystyle \frac{1}{2}=\log_{4}2$

$\mathrm{i}$ równanie przyjmuje postač

$\log_{4}(x-1)^{2}+\log_{4}(x+4)=\log_{4}(x^{3}-x^{2}-3x+5)+\log_{4}2.$

(3)

Korzystajac $\mathrm{z}$ wlasności logarytmu oraz $\mathrm{z}$ róznowartościowości funkcji

logarytmicznej, widzimy, $\dot{\mathrm{z}}\mathrm{e}$ równanie (3) jest równowazne ($\mathrm{w}$ dziedzinie)

równaniu algebraicznemu $(x-1)^{2}(x+4)=2(x^{3}-x^{2}-3x+5)$. Po wykonaniu

dzialań $\mathrm{i}$ przeniesieniu wszystkich skladników na jedna strong dostajemy

$x^{3}-4x^{2}+x+6=0.$

(4)





141

Pierwiastki calkowite równania (4) sa podzie1nikami wyrazu wo1nego, $\mathrm{t}\mathrm{j}.$

liczby 6. Przez podstawienie sprawdzamy bezpośrednio, $\dot{\mathrm{z}}\mathrm{e}$ liczby $-1, 2\mathrm{i}3$

spelniaja (4), czy1i sa wszystkimi pierwiastkami tego równania (majac dwa

pierwiastki, np. $-1\mathrm{i}2$, trzeci $\mathrm{m}\mathrm{o}\dot{\mathrm{z}}$ na znalez$\acute{}$č $\mathrm{z}$ relacji $x_{1}x_{2}x_{3}=-6$). Liczby

2 $\mathrm{i}3$ znajduja $\mathrm{s}\mathrm{i}\mathrm{e}$ poza przedzialem $(-4,1)$, czyli $\mathrm{l}\mathrm{e}\dot{\mathrm{z}}$ a poza $D$. Natomiast

$-1\in(-4,1)$ oraz $(-1)^{3}-(-1)^{2}-3(-1)+5=6>0$, czyli liczba $-1$ jest

jedynym pierwiastkiem danego równania.

Odp. Równanie ma tylko jeden pierwiastek $\mathrm{i}$ jest nim liczba $-1.$

Rozwiazanie zadania 22.7

Dziedzine równania określaja warunki

$D$: 

czyli warunki $ x\neq k\pi$ oraz $ x\displaystyle \neq\frac{\pi}{2}+k\pi$. To daje ostatecznie

$D:x\displaystyle \neq k\frac{\pi}{2},$

$ k\in$ Z.

Dla $x\in D$ mnozymy obie strony równania przez $(\sin x\cos x)\mathrm{i}$ otrzymujemy

równanie równowazne

$\sin x+\cos x=\sqrt{8}\sin x\cos x.$

(5)

Korzystajac ze wzoru redukcyjnego oraz wzoru na róznice cosinusów,

mamy $\sin x+\cos x = \displaystyle \cos x-\cos(x+\frac{\pi}{2}) = -2\displaystyle \sin(-\frac{\pi}{4})\sin(x+\frac{\pi}{4}).$

Ponadto $\sqrt{8}\sin x\cos x = \sqrt{2}\sin 2x$, zatem równanie (5), po podzie1eniu

obu stron przez $\sqrt{2}, \mathrm{m}\mathrm{o}\dot{\mathrm{z}}$ na zapisač $\mathrm{w}$ postaci

$\displaystyle \sin(x+\frac{\pi}{4})=\sin 2x.$





142

Rys. 27

Stad otrzymujemy alternatywe równań li-

niowych $x+\displaystyle \frac{\pi}{4} =  2x+2k\pi$ lub $x+\displaystyle \frac{\pi}{4} =$

$\pi-2x+2k\pi$, gdzie $k \in$ Z. Po standardo-

wych przeksztalceniach mamy $x= \displaystyle \frac{\pi}{4}+2k\pi$

lub $x = \displaystyle \frac{\pi}{4}+k\frac{2\pi}{3}$. Zauwazmy, $\dot{\mathrm{z}}\mathrm{e}$ pierwsza

seria zawiera $\mathrm{s}\mathrm{i}\mathrm{e} \mathrm{w}$ drugiej (rys. 27), a ta

$\mathrm{z}$ kolei jest zawarta $\mathrm{w}$ dziedzinie równania.

Odp. $x=\displaystyle \frac{\pi}{4}+k\frac{2\pi}{3},  k\in$ Z.

Rozwiazanie zadania 26.4

Oznaczmy przez $O$ spodek wysokości czworościanu, a przez $K, L$

jego rzuty prostokatne odpowiednio na przyprostokatne $BC\mathrm{i}AC$ podstawy

(rys. 28). Poniewaz $O$ jest środkiem okregu wpisanego $\mathrm{w} \triangle ABC$, wiec

{\it D} $|OK| = |OL| = r$, czyli punkty

$L |KC|=|LC|$.   (6)

$r$ Mamy $\triangle DOK \equiv \triangle DOL$, gdyz
\begin{center}
\includegraphics[width=66.240mm,height=81.792mm]{./KursMatematyki_PolitechnikaWroclawska_1999_2004_page122_images/image001.eps}
\end{center}
{\it E}

{\it A} $\cdot C$

$\alpha$

{\it S}

{\it O} $r K$

{\it B}

$O, K, L \mathrm{i}$ wierzcho ek kata

prostego $C$ tworza kwadrat $0$ bo-

ku $r$. Stad

oba sa prostokatne $\mathrm{i}$ maja takie

same przyprostokatne. Stad

$|DK| = |DL|$. Poniewaz wyso-

kośč DO jest prostopad a do pod-

stawy, wiec DO $\perp BC$. Ponad-

{\it to OK} $\perp BC. \mathrm{Z}$ twierdzenia

$0$ trzech prostopad ych wniosku-

$\mathrm{R}\mathrm{y}\mathrm{s}$. 28

jemy, $\dot{\mathrm{z}}\mathrm{e} DK \perp BC$. Analogi-

cznie stwierdzamy, $\dot{\mathrm{z}}\mathrm{e} DL\perp AC.$

Wynika stad, $\dot{\mathrm{z}}\mathrm{e}\triangle DKC\mathrm{i}\triangle DLC$ sa przystajacymi trójkatami prostokat-

nymi $\mathrm{i}\mathrm{w}$ konsekwencji

$\angle DCK=\angle DCL$.   (7)

Niech $E$ oznacza rzut prostokatny punktu $K$ na krawed $\acute{\mathrm{z}} DC$. Ze

wzorów (6) $\mathrm{i}$ (7) oraz $\mathrm{z}$ II cechy przystawania trójkatów (bkb) wynika, $\dot{\mathrm{z}}\mathrm{e}$

$\triangle KCE \equiv \triangle LCE$, a stad $ LE\perp DC$. To oznacza, $\dot{\mathrm{z}}\mathrm{e}$ krawed $\acute{\mathrm{z}} DC$ jest





143

prostopadla do plaszczyzny wyznaczonej przez punkty $K, L\mathrm{i}E\mathrm{i}\mathrm{w}$ kon-

sekwencji $\angle KEL=\beta. \mathrm{Z}$ porównania trójkatow równoramiennych $\triangle KCL$

$\mathrm{i} \triangle KEL$, majacych wspólna podstawe oraz $|KE| < |KC| (|KE|$ jest

przyprostokatna), wnioskujemy, $\dot{\mathrm{z}}\mathrm{e} \angle KEL = \beta > \angle KCL = \displaystyle \frac{\pi}{2}$, zatem

dziedzina dla kata $\beta$ jest przedzial $(\displaystyle \frac{\pi}{2},\pi).$
\begin{center}
\includegraphics[width=51.720mm,height=54.408mm]{./KursMatematyki_PolitechnikaWroclawska_1999_2004_page123_images/image001.eps}
\end{center}
{\it D}

$(\cdot E$

{\it O S}

Rys. 29

{\it C}
\begin{center}
\includegraphics[width=66.804mm,height=90.168mm]{./KursMatematyki_PolitechnikaWroclawska_1999_2004_page123_images/image002.eps}
\end{center}
{\it A  L C}

$\alpha$  {\it r}

{\it r}

{\it S}

{\it O K}

{\it r}

{\it B}

Rys. 30

$\mathrm{W}$ celu wyznaczenia wysokości czworościanu oznaczmy przez $S$ środek

kwadratu OKCL. Wówczas $|ES|= |SK|\displaystyle \mathrm{c}\mathrm{t}\mathrm{g}\frac{\beta}{2}=\frac{r}{\sqrt{2}}\mathrm{c}\mathrm{t}\mathrm{g}\frac{\beta}{2}.$ Poprowad $\acute{\mathrm{z}}\mathrm{m}\mathrm{y}$

plaszczyzne przechodzaca przez $DO$ oraz przez $C$. Przekrój czworościanu ta

plaszczyzna pokazano na rysunku 29. $\mathrm{Z}$ twierdzenia Pitagorasa $\mathrm{w}\triangle ESC$

mamy $|EC|^{2}=|SC|^{2}-|ES|^{2}=\displaystyle \frac{r^{2}}{2}-\frac{r^{2}}{2}\mathrm{c}\mathrm{t}\mathrm{g}^{2}\frac{\beta}{2}=r^{2}\frac{-\cos\beta}{2\sin^{2}\frac{\beta}{2}}. \mathrm{Z}$ podobieństwa

trójkatów $\triangle ESC\mathrm{i}\triangle DOC$ dostajemy proporcje $\displaystyle \frac{H}{|OC|} = \displaystyle \frac{|ES|}{|EC|}$. Stad

$H=\displaystyle \frac{|OC||ES|}{|EC|}=\frac{r^{2}\mathrm{c}\mathrm{t}\mathrm{g}\frac{\beta}{2}}{r\sqrt{\frac{-\cos\beta}{2\sin^{2}\frac{\beta}{2}}}}=r\sqrt{2}\frac{\cos\frac{\beta}{2}}{\sqrt{-\cos\beta}}.$

(8)





144

Dla obliczenia pola podstawy czworościanu (rys. 30) zauwazmy, $\dot{\mathrm{z}}\mathrm{e}$

$|AC|=|LC|+|AL|=r+r\displaystyle \mathrm{c}\mathrm{t}\mathrm{g}\frac{\alpha}{2}$ oraz $|BC|=|AC|$ tg $\alpha$. Stad mamy

$P_{p}=\displaystyle \frac{1}{2}|AC|$

$|BC|=\displaystyle \frac{1}{2}r^{2}(\mathrm{c}\mathrm{t}\mathrm{g}\frac{\alpha}{2}+1)^{2}$ tg $\displaystyle \alpha=\frac{1}{2}r^{2}\frac{(\sin\frac{\alpha}{2}+\cos\frac{\alpha}{2})^{2}}{\sin^{2}\frac{\alpha}{2}}$ tg $\alpha$

$\mathrm{i}$ ostatecznie

$P_{p}=r^{2}\displaystyle \frac{1+\sin\alpha}{\cos\alpha}$ ctg $\displaystyle \frac{\alpha}{2}.$

(9)

$\mathrm{Z}$ równości (8) $\mathrm{i}$ (9) otrzymujemy

$V=\displaystyle \frac{1}{3}P_{p}H=\frac{\sqrt{2}}{3}r^{3}\frac{1+\sin\alpha}{\cos\alpha\sqrt{-\cos\beta}}\cos\frac{\beta}{2}$ ctg $\displaystyle \frac{\alpha}{2}.$

Odp. Objetośč czworościanu wynosi $\displaystyle \frac{\sqrt{2}}{3}r^{3}\frac{1+\sin\alpha}{\cos\alpha\sqrt{-\cos\beta}}\cos\frac{\beta}{2}\mathrm{c}\mathrm{t}\mathrm{g}\frac{\alpha}{2}.$

Rozwiazanie zadania 28.2

Aby nierównośč

$\displaystyle \frac{2px^{2}+2px+1}{x^{2}+x+2-p^{2}}\geq 2$

(10)

byla spelniona dla $\mathrm{k}\mathrm{a}\dot{\mathrm{z}}$ dej liczby rzeczywistej, jej dziedzina musi byč $\mathrm{R}$, czyli

trójmian kwadratowy $\mathrm{w}$ mianowniku nie $\mathrm{m}\mathrm{o}\dot{\mathrm{z}}\mathrm{e}$ mieč pierwiastków rzeczy-

wistych. Stad otrzymujemy warunek $\triangle_{0} = 1-4(2-p^{2}) =4p^{2}-7< 0.$

Nierównośč ta jest spelniona dla

$p\displaystyle \in(-\frac{\sqrt{7}}{2},\frac{\sqrt{7}}{2})$

(11)

Dla parametru $p$ spelniajacego warunek (ll) mianownik lewej strony

(10) jest dodatni na calej prostej, wiec po pomnozeniu obu stron

(10) przez ten mianownik otrzymujemy nierównośč równowazna

$2px^{2}+2px+1\geq 2x^{2}+2x+4-2p^{2}$, a po uporzadkowaniu

$2(p-1)x^{2}+2(p-1)x+2p^{2}-3\geq 0.$

(12)





145

Dla $p=1$lewa strona (12) jest równa $-1\mathrm{i}$ nierównośč niejest spelniona dla

$\dot{\mathrm{z}}$ adnego $x$. Gdy $p<1\mathrm{t}\mathrm{z}\mathrm{n}$. wspólczynnik przy $x^{2}$ jest ujemny, nierównośč

(12) nie $\mathrm{m}\mathrm{o}\dot{\mathrm{z}}\mathrm{e}$ byč spelniona dla wszystkich $x$ (gdyz,,ramiona paraboli sa

skierowane $\mathrm{w}$ dól''). Natomiast dla $p>1$, nierównośč (12) bedzie spe1niona

dla wszystkich liczb rzeczywistych wtedy $\mathrm{i}$ tylko wtedy, gdy

$\triangle_{1} =4(p-1)^{2}-8(p-1)(2p^{2}-3) \leq 0$. Po podzieleniu obu stron przez

wyrazenie dodatnie $4(p-1)$ otrzymujemy $-4p^{2}+p+5\leq 0$, skad od razu

mamy $ p\leq$ -llub $ p\geq \displaystyle \frac{5}{4}$. Poniewaz $\displaystyle \frac{5}{4}< \displaystyle \frac{\sqrt{7}}{2}\mathrm{i}$ zalozyliśmy, $\dot{\mathrm{z}}\mathrm{e}p>1$, wiec

laczac wszystkie otrzymane warunki dostajemy ostatecznie $ p\in [\displaystyle \frac{5}{4},\frac{\sqrt{7}}{2}$).

Odp. Nierównośč jest spelniona dla $\mathrm{k}\mathrm{a}\dot{\mathrm{z}}$ dej liczby rzeczywistej, gdy

$ p\in [\displaystyle \frac{5}{4},\frac{\sqrt{7}}{2}).$

Rozwiazanie zadania 29.8

$\mathrm{Z}$ postaci ciagu odczytujemy wyraz poczatkowy $\alpha_{0}=x+1$ oraz iloraz

$q = -x^{2}$ Jeśli $x = -1$, to wszystkie wyrazy ciagu sa zerami $\mathrm{i}$ suma

$S(-1)=0$. Gdy $x\neq-1$, wówczas warunkiem istnienia sumy nieskończonego

ciagu geometrycznegojest $|q|<1$, czyli $|-x^{2}|=x^{2}<1$, skad od razu otrzy-

mujemy $x\in(-1,1)$. Ostatecznie dziedzina sumy $S(x)$ jest $D=[-1$, 1).

Korzystajac ze wzoru na sume nieskończonego ciagu geometrycznego,

dostajemy $S(x) =\displaystyle \frac{x+1}{1-(-x^{2})}=\frac{x+1}{x^{2}+1},  x\in (-1,1)$. Wzór ten pozostaje

prawdziwy takze dla $x=-1$. Dlatego $\mathrm{m}\mathrm{o}\dot{\mathrm{z}}$ na napisač

$S(x)=\displaystyle \frac{x+1}{x^{2}+1}, x\in D=[-1$, 1$)$.   (13)

Dalsze postepowanie sprowadza $\mathrm{s}\mathrm{i}\mathrm{e}$ do wyznaczenia wartości namniejszej

$\mathrm{i}$ najwiekszej funkcji wymiernej $S(x)$, danej wzorem (13). Zauwazmy, $\dot{\mathrm{z}}\mathrm{e}$

mianownik jest dodatni, a licznik nieujemny, zatem $S(x) \geq 0$ dla wszyst-

kich $x$. Stad wynika, $\dot{\mathrm{z}}\mathrm{e}$ najmniejsza wartościa tej funkcji jest 0 $\mathrm{i}$ jest ona

osiagana dla $x=-1.$

Dla znalezienia wartości najwiekszej wykorzystamy pochodna funkcji

$S(x).$

$S'(x)=\displaystyle \frac{1\cdot(x^{2}+1)-2x(x+1)}{(x^{2}+1)^{2}}=\frac{-x^{2}-2x+1}{(x^{2}+1)^{2}}.$





146

Miejsca zerowe pochodnej spelniaja równanie $-x^{2}-2x+1=0$. Stad dosta-

jemy $\triangle= 8$ oraz $x_{1} = -1-\sqrt{2}, x_{2} = -1+\sqrt{2}$. Tylko $x_{2} \in D$. Mamy

$S(x_{2}) = \displaystyle \frac{\sqrt{2}}{1+2-2\sqrt{2}+1} = \displaystyle \frac{1+\sqrt{2}}{2}$ oraz $\displaystyle \lim_{x\rightarrow 1-}S(x) = \displaystyle \lim_{x\rightarrow 1-}\frac{x+1}{x^{2}+1} = 1.$

Poniewaz $\displaystyle \frac{1+\sqrt{2}}{2}>1$, wiec najwieksza wartościa funkcji jest $\displaystyle \frac{1+\sqrt{2}}{2}.$

Odp. Wartośč najmniejsza sumy danego nieskończonego ciagu geo-

metrycznego wynosi 0 $\mathrm{i}$ jest osiagana dla $x = -1$, a wartośč najwieksza

tej sumy wynosi $\displaystyle \frac{1+\sqrt{2}}{2}\mathrm{i}$ jest osiagana dla $x=-1+\sqrt{2}.$

Rozwiazanie zadania 30.7

Dziedzina nierówności

$|2^{x}-3|\leq 2^{1-x}$

(14)

jest R. Nierównośč $\mathrm{t}\mathrm{e}\mathrm{r}$ozwia $\dot{\mathrm{z}}$ emy przez podstawienie $2^{x}=t, t>0$. Mamy

$2^{1-x}=22^{-x}=2\displaystyle \frac{1}{t},$ wiec po podstawieniu nierównośč (14) przyjmie postač

$|t-3|\displaystyle \leq\frac{2}{t}$. Stad od razu przechodzimy do nierówności podwójnej

$-\displaystyle \frac{2}{t}\leq t-3\leq\frac{2}{t},$

$t>0.$

(15)

Ze wzgledu na dodatni znak niewiadomej $t \mathrm{m}\mathrm{o}\dot{\mathrm{z}}$ emy $\mathrm{t}\mathrm{e}$ nierównośč pomnozyč

przez $t\mathrm{i}$ otrzymamy nastepujacy uklad nierówności kwadratowych

$\left\{\begin{array}{l}
t^{2}-3t+2\geq 0\\
t^{2}-3t-2\leq 0
\end{array}\right.$

$t>0.$

Pierwsza nierównośč powyzszego ukladu jest spelniona dla $t \leq 1 \mathrm{i}t \geq 2,$

czyli po uwzglednieniu warunku $t> 0$ dla $ t\in (0,1]\cup[2,\infty)$. Dla drugiej

nierówności mamy $\triangle_{2} = 17, t_{1}'' = \displaystyle \frac{3-\sqrt{17}}{2} < 0, t_{2}'' = \displaystyle \frac{3+\sqrt{17}}{2} \in (3,4).$

Druga nierównośč jest zatem spelniona dla $t \in (0,t_{2}''$]. Cześč wspólna

zbiorów rozwiazań obu nierówności ma postač $(0,1[\cup[2,t_{2}'']$. Poniewaz funkcja

$t = 2^{x}$ jest rosnaca, wiec zbiór rozwiazań nierówności (14) ma postač

$(-\infty,0]\cup[1,x_{0}]$, gdzie $x_{0}=\log_{2}t_{2}''\in(1,2).$





147

Wykresy funkcji wystepujacych po obu stronach nierówności (14) otrzy-

mujemy przez translacje $\mathrm{i}$ odbicia symetryczne standardowej krzywej

$\mathrm{I}^{\urcorner}$ : $y = 2^{x}$ Wykres krzywej $y = |2^{x}-3|$ dostajemy przez translacje $\mathrm{I}^{\urcorner}$

$0$ wektor $[0,-3]$, a nastepnie odbicie symetryczne cześci $\mathrm{l}\mathrm{e}\dot{\mathrm{z}}$ acej pod osia

odcietych wzgledem tej osi. Krzywa ta ma asymptote pozioma lewostronna

$y=3$. Natomiast krzywa $y=2^{1-x}$ dostajemy przez odbicie symetryczne $\mathrm{I}^{\urcorner}$

wzgledem osi rzednych, a nastepnie translacje $0$ wektor $($1, $0)$. Wykresy sa

przedstawione na rysunku 31.
\begin{center}
\includegraphics[width=110.484mm,height=74.472mm]{./KursMatematyki_PolitechnikaWroclawska_1999_2004_page127_images/image001.eps}
\end{center}
{\it y}

$y=2^{1-x}  \Gamma$

3  $y=|2^{x}-3|$

2

1

$-1$  0 1  $x_{0}$  3 4  {\it x}

Rys. 31

Odp. Zbiorem rozwiazań

(-00, $ 0]\cup [1,\displaystyle \log_{2}\frac{3+\sqrt{17}}{2}].$

nierówności jest

suma

przedzialów

Rozwiazanie zadania 31.7

Przy rozwiazywaniu zadania skorzystamy nastepujacej wlasności wek-

torów na plaszczy $\acute{\mathrm{z}}\mathrm{n}\mathrm{i}\mathrm{e}$:

$\mathrm{T}\mathrm{w}\mathrm{i}\mathrm{e}\mathrm{r}\mathrm{d}\mathrm{z}\mathrm{e}\mathrm{n}\mathrm{i}\mathrm{e}$. {\it Jeśli wektory} $\vec{u}i\vec{v}sq_{f}$ {\it prostopadte} $i$ {\it majq} $t_{G}$

$samq_{f}$ {\it dtugośč oraz} $\vec{u}=(\alpha,b)$, {\it to} $\vec{v}=(b,-\alpha) lub\vec{v}=(-b,\alpha).$

Przez $B$ oznaczmy wierzcholek kwadratu $\mathrm{l}\mathrm{e}\dot{\mathrm{z}}\mathrm{a}\mathrm{c}\mathrm{y}$ na prostej $l$, a przez

$D$ jego wierzcholek $\mathrm{l}\mathrm{e}\dot{\mathrm{z}}\mathrm{a}\mathrm{c}\mathrm{y}$ na prostej $k$. Korzystajac $\mathrm{z}$ równań prostych,

$\mathrm{m}\mathrm{o}\dot{\mathrm{z}}$ emy napisač $B(2y-1,y), D(4-3z,z)$, gdzie $y, z$ sa niezna-

$\rightarrow$

nymi rzednymi tych wierzcholków, zatem $AB= [2y-7,y-1]$ oraz





148

$\vec{AD}= [-3z-2,z-1]$. Poniewaz $\vec{AB}\perp\vec{AD}$ oraz oba wektory sa tej samej

dlugości, wiec $\mathrm{z}$ powyzszego twierdzenia otrzymujemy $\mathrm{z}$ porównania odpo-

wiednich wspólrzednych dwa uklady równań liniowych:

$\left\{\begin{array}{l}
2y-7=-z+1\\
-3z-2=y-1
\end{array}\right.$

oraz

$\left\{\begin{array}{l}
2y-7=z-1\\
3z+2=y-1
\end{array}\right.$

Po rozwiazaniu pierwszego ukladu dostajemy $y=3, z=0$, czyli $B_{1}(5,3),$

$\rightarrow$

$\rightarrow$

$\rightarrow$

$D_{1}(4,0)$ oraz $AB= [-1,2]$. Poniewaz $AB=DC$, wiec $C_{1}(3,2)$. Rozwiaza-

niem drugiego ukladu jest $y = 5, z = -2$, czyli $B_{2}(9,5), D_{2}(10,-2)$

$\rightarrow$

$\rightarrow$

$\mathrm{i}$ podobnie jak poprzednio $AB=DC=[4,-3]$, skad $C_{2}(13,2)$. Rozwiazanie

ilustruje rysunek 32.

Uwaga. Ze wzgledu na ogólnie przyjety sposób oznaczania wierzcholków

wielokatów na rysunku 32 przestawiono 1itery $B \mathrm{i} D$, oznaczajac

$B_{2}(10,-2)\mathrm{i}D_{2}(9,5).$
\begin{center}
\includegraphics[width=108.108mm,height=65.028mm]{./KursMatematyki_PolitechnikaWroclawska_1999_2004_page128_images/image001.eps}
\end{center}
{\it y}

5

$D_{2}$

$B_{1}$

3  $C_{1}$

$C_{2}$

1  {\it A}

0 1  $D_{1}$  7  10  13 x

$B_{2}$

Rys. 32

Odp. Istnieja dwa kwadraty spelniajace warunki zadania. Ich wierz-

cholkami, oprócz wierzcholka $A$, sa punkty $B_{1}(5,3), C_{1}(3,2), D_{1}(4,0)$ oraz

$B_{2}(10,-2), C_{2}(13,2), D_{2}(9,5).$

Rozwiazanie zadania 32.7

Równanie stycznej do wykresu funkcji $f(x)\mathrm{w}$ punkcie $S(x_{0},f(x_{0}))$ ma

postač ogólna $y-f(x_{0}) =f'(x_{0})(x-x_{0})$. Poniewaz $\mathrm{w}$ naszym przypadku





149

jest $f(x)=x^{4}-2x^{2}$ oraz $f'(x)=4x^{3}-4x$, wiec równanie stycznej przyjmie

postač

$y-(x_{0}^{4}-2x_{0}^{2})=(4x_{0}^{3}-4x_{0})(x-x_{0})$.   (16)

Punkt $P(1,-1)\mathrm{l}\mathrm{e}\dot{\mathrm{z}}\mathrm{y}$ na tej stycznej, wiec niewiadoma $x_{0}$ spelnia równanie

$-1+2x_{0}^{2}-x_{0}^{4}=4x_{0}(x_{0}^{2}-1)(1-x_{0})$. Po wylaczeniu wspólnych czynników

$\mathrm{i}$ uporzadkowaniu dostajemy $(x_{0}^{2}-1)(x_{0}-1)(3x_{0}-1)=0$. Równanie (16)

ma wiec trzy pierwiastki $-1$, l oraz $\displaystyle \frac{1}{3}.$

Po podstawieniu do równania (16) pierwiastków $x_{0} = -1 \mathrm{i} x_{0} = 1$

otrzymujemy $\mathrm{t}\mathrm{e}$ sama prosta $p$: $y+1=0$. Prosta ta jest wiec styczna do

wykresu $f$ równocześnie $\mathrm{w}$ punktach $P(-1,1)$ oraz $Q(-1,-1)$. Poniewaz

$f(x)=x^{4}-2x^{2}\geq-1$ (inaczej $(x^{2}-1)^{2}\geq 0$) dla wszystkich $x\mathrm{i}$ równośč ma

miejsce jedynie dla $x=-1\mathrm{i}x=1$, wiec styczna $p$ ma dwa punkty wspólne

$\mathrm{z}$ wykresem $f.$

Dla $x_{0}=\displaystyle \frac{1}{3}$ równanie (16) przyjmuje postač $l$ : $32x+27y-5=0. \mathrm{W}$ celu

określenia liczby punktów wspólnych stycznej $l\mathrm{z}$ wykresem $f$ nalezy określič

liczbe róznych pierwiastków równania $x^{4}-2x^{2} = \displaystyle \frac{5-32x}{27}, \mathrm{t}\mathrm{j}$. równania

$27x^{4}-54x^{2}+32x-5 = 0$. Ze wzgledu na stycznośč $\mathrm{w}$ punkcie $x_{0} = \displaystyle \frac{1}{3}$

równanie to ma podwójny pierwiastek $\displaystyle \frac{1}{3}$ oraz pierwiastek l (punkt $P\mathrm{l}\mathrm{e}\dot{\mathrm{z}}\mathrm{y}$

na wykresie $f$), zatem, jako równanie czwartego stopnia, ma takze czwarty

pierwiastek rzeczywisty, który obliczamy $\mathrm{z}$ równości $x_{1}x_{2}x_{3}x_{4}=\displaystyle \frac{-5}{27}$, czyli

$\mathrm{w}$ naszym przypadku $\displaystyle \frac{1}{9}x_{4} = \displaystyle \frac{-5}{27}$, skad $x_{4} = \displaystyle \frac{-5}{3}$. Styczna $l$ ma zatem

trzy punkty wspólne $\mathrm{z}$ wykresem $f$: $P, S(\displaystyle \frac{1}{3},-\frac{17}{81})$ oraz $A(-\displaystyle \frac{5}{3},\frac{175}{81}).$

$\mathrm{W}$ punkcie $A$ styczna $l$ przecina wykres $f$. Dla sporzadzenia rysunku

zauwazmy, $\dot{\mathrm{z}}\mathrm{e}f(x)$ jest funkcja parzysta. Liczba $x=0$ jest pierwiastkiem

podwójnym równania $x^{4}-2x^{2}=0$, co oznacza, $\dot{\mathrm{z}}\mathrm{e}$ wykres $f$ jest styczny do

osi odcietych $\mathrm{w}$ poczatku ukladu. Pozostale miejsca zerowe funkcji to $-\sqrt{2}$

$\mathrm{i}\sqrt{2}$. Kreślac styczne $y=0, l$ oraz $p\mathrm{i}$ zaznaczajac punkty styczności oraz

punkty $(\sqrt{2},0), B(\displaystyle \frac{5}{3},\frac{175}{81}), \mathrm{m}\mathrm{o}\dot{\mathrm{z}}$ emy narysowač wykres funkcji na $(0,\infty),$

a przez odbicie symetryczne takze $\mathrm{w}$ (-00, 0). Wykres przedstawiono na ry-

sunku 33.





19

Praca kontrolna

nr l

8.1. Suma wszystkich wyrazów nieskończonego ciagu geometrycznego

wynosi 2040. Jeś1i pierwszy wyraz tego ciagu zmniejszymy $0 172,$

a jego iloraz zwiekszymy 3-krotnie, to suma wszystkich wyrazów tak

otrzymanego ciagu wyniesie 2000. Wyznaczyč i1oraz $\mathrm{i}$ pierwszy wyraz

danego ciagu.

8.2. Obliczyč wszystkie te skladniki rozwiniecia dwumianu $(\sqrt{3}+\sqrt[3]{2})^{11}$,

które sa liczbami calkowitymi.

8.3. Narysowač staranny wykres funkcji $f(x) = |x^{2}-2|x|-3| \mathrm{i}$ na jego

podstawie podač ekstrema lokalne oraz przedzialy monotoniczności tej

funkcji.

8.4. Rozwiazač nierównośč

$x+1\geq\log_{2}(4^{x}-8).$

8.5. $\mathrm{W}$ ostroslupie prawidlowym trójkatnym krawed $\acute{\mathrm{z}}$ podstawy ma dlugośč

$\alpha$, a polowa kata plaskiego przy wierzcholkujest równa katowi nachyle-

nia ściany bocznej do podstawy. Obliczyč objetośč ostroslupa. Sporza-

dzič odpowiednie rysunki.

8.6. Znalez$\acute{}$č wszystkie wartości parametru $p$, dla których trójkat $KLM$

$0$ wierzcholkach $K(1,1), L(5,0)\mathrm{i}M(p,p-1)$ jest prostokatny. Roz-

wiazanie zilustrowač rysunkiem.

8.7. Rozwiazač równanie

--ssiinn35{\it xx}$=$--ssiinn46{\it xx}.

8.8. Przez punkt $P\mathrm{l}\mathrm{e}\dot{\mathrm{z}}\mathrm{a}\mathrm{c}\mathrm{y}$ wewnatrz trójkata $ABC$ poprowadzono proste

równolegle do wszystkich boków trójkata. Pola utworzonych $\mathrm{w}$ ten

sposób trzech mniejszych trójkatów $0$ wspólnym wierzcholku $P$ wyno-

$\mathrm{s}\mathrm{z}\mathrm{a}$ odpowiednio $S_{1}, S_{2}, S_{3}$. Obliczyč pole $S$ trójkata $ABC.$





150
\begin{center}
\includegraphics[width=90.012mm,height=83.916mm]{./KursMatematyki_PolitechnikaWroclawska_1999_2004_page130_images/image001.eps}
\end{center}
{\it y}

3

{\it l}

{\it A  B}

$1_{11}$  2  111

1

$-2  -1$  0 1  {\it x}

{\it S}  $\sqrt{2}2$

{\it p  Q  P}

Rys. 33

Odp. $\mathrm{s}_{\mathrm{a}}$ dwie takie styczne jedna $0$ równaniu $y= -1$, która ma dwa

punkty wspólne $\mathrm{z}$ wykresem funkcji $f(x)$, oraz druga $0$ równaniu

$32x+27y-5=0$ majaca trzy punkty wspólne $\mathrm{z}$ wykresem.

Rozwiazanie zadania 34.5

Wprowad $\acute{\mathrm{z}}\mathrm{m}\mathrm{y}$ nastepujace zdarzenia:

$A-$ Jaś wyciagnie co najmniej trzy monety;

$B_{i}-$ za pierwszym razem zostanie wylosowana moneta $0$ nominale $i \mathrm{z}l,$

$i=1$, 2, 5;

$C_{j}-$ dla uiszczenia zaplaty Jaś wyciagnie $j$ monet, $j=1$, 2, 3, 4.

Wówczas $A' = C_{1}\cup C_{2} \mathrm{i}$ oba skladniki sa rozlaczne. Zauwazmy, $\dot{\mathrm{z}}\mathrm{e}$

$C_{1} = B_{5}$ oraz $B_{1}\cup B_{2}\cup B_{5} = \Omega$. Ponadto $P(B_{1}) = \displaystyle \frac{1}{2}, P(B_{2}) = \displaystyle \frac{1}{3}$

$\displaystyle \mathrm{i}P(B_{5})=P(C_{1})=\frac{1}{6}$. Ze wzoru na prawdopodobieństwo calkowite mamy

$P(C_{2})=P(C_{2}|B_{1})P(B_{1})+P(C_{2}|B_{2})P(B_{2})+P(C_{2}|B_{5})P(B_{5})$.   (17)

Mamy $P(C_{2}|B_{1})=\displaystyle \frac{1}{5}$, gdyz za drugim razem Jaś musi wyciagnač mone-

$\mathrm{t}\mathrm{e} 5 \mathrm{z}l$ spośród 5 monet $\mathrm{w}$ portmonetce. Podobnie $P(C_{2}|B_{2}) = \displaystyle \frac{2}{5}$ (za





151

drugim razem musi byč wyciagnieta moneta 5 $\mathrm{z}l$ lub pozostala dostepna

moneta 2 $\mathrm{z}l$) oraz $P(C_{2}|B_{5}) =0$ (nie ma drugiego losowania, gdy $\mathrm{w}$ pier-

wszym byla moneta 5 $\mathrm{z}l$ lub inaczej $ C_{2}\cap B_{5}=\emptyset$). Po podstawieniu tych

wartości do wzoru (17) dostajemy $P(C_{2})=\displaystyle \frac{7}{30}\mathrm{i}$ ostatecznie

$P(A)=1-P(C_{1})-P(C_{2})=1-\displaystyle \frac{7}{30}-\frac{1}{6}=\frac{6}{10}.$

Odp. Prawdopodobieństwo tego, $\dot{\mathrm{z}}\mathrm{e}$ Jaś wyciagnie co najmniej trzy

monety wynosi $\displaystyle \frac{3}{5}.$





20

Praca kontrolna nr 2

9.1. Promień kuli powiekszono $\mathrm{t}\mathrm{a}\mathrm{k},\ \dot{\mathrm{z}}\mathrm{e}$ pole jej powierzchni wzroslo $0$ 44\%.

$\mathrm{O}$ ile procent wzrosla jej objetośč?

9.2. Wyznaczyč równanie krzywej utworzonej przez środki odcinków maja-

cych obydwa końce na osiach ukladu wspólrzednych $\mathrm{i}$ zawierajacych

punkt $P(2,1)$. Sporzadzič dokladny wykres $\mathrm{i}$ podač nazwe otrzymanej

krzywej.

9.3. Znalez$\acute{}$č wszystkie wartości parametru $m$, dla których równanie

$(m-1)9^{x}-4\cdot 3^{x}+m+2=0$

ma dwa rózne pierwiastki.

9.4. Róznica promienia kuli opisanej na czworościanie foremnym $\mathrm{i}$ promienia

kuli wpisanej $\mathrm{w}$ niegojest równa l. Obliczyč objetośč tego czworościanu.

9.5. Rozwiazač nierównośč

$\displaystyle \frac{2}{|x^{2}-9|}\geq\frac{1}{x+3}.$

9.6. Stosunek dlugości przyprostokatnych trójkata prostokatnego wynosi k.

Obliczyč stosunek dlugości dwusiecznych katów ostrych tego trójkata.

Zastosowač odpowiednie wzory trygonometryczne.

9.7. Zbadač przebieg zmienności i narysowač wykres funkcji

$f(x)=\displaystyle \frac{x^{2}+4}{(x-2)^{2}}.$

9.8. Znalez$\acute{}$č równania wszystkich prostych przechodzacych przez punkt

$A(\displaystyle \frac{7}{5},-2)\mathrm{i}$ stycznych do wykresu funkcji $f(x)=x^{3}-2x$. Rozwiazanie

zilustrowač rysunkiem.





21

Praca kontrolna nr 3

10.1. Stosujac zasade indukcji matematycznej, udowodnič, $\dot{\mathrm{z}}\mathrm{e}$ dla $\mathrm{k}\mathrm{a}\dot{\mathrm{z}}$ dej

liczby naturalnej $n$ suma $2^{n+1}+3^{2n-1}$ jest podzielna przez 7.

10.2. Tworzaca stozka ma dlugośč $l \mathrm{i}$ widač $\mathrm{j}\mathrm{a}$ ze środka kuli wpisanej

$\mathrm{w}$ ten stozek pod katem $\alpha$. Obliczyč objetośč $\mathrm{i}\mathrm{k}\mathrm{a}\mathrm{t}$ rozwarcia stozka.

Określič dziedzine dla kata $\alpha.$

10.3. Bez stosowania metod rachunku rózniczkowego wyznaczyč dziedzine

$\mathrm{i}$ zbiór wartości funkcji

$y=\sqrt{2+\sqrt{x}-x}.$

10.4. $\mathrm{Z}$ talii 24 kart wy1osowano (bez zwracania) cztery karty. Jakie jest

prawdopodobieństwo, $\dot{\mathrm{z}}\mathrm{e}$ otrzymano dokladnie trzy karty $\mathrm{z}$ jednego

koloru ($\mathrm{z}$ czterech $\mathrm{m}\mathrm{o}\dot{\mathrm{z}}$ liwych)?

10.5. Rozwiazač nierównośč

$\log_{1/3}$ (log2 $4x$) $\geq\log_{1/3}(2-\log_{2x}4)-1.$

10.6. $\mathrm{Z}$ punktu $C(1,0)$ poprowadzono styczne do okregu $x^{2}+y^{2} = r^{2},$

$ r\in (0,1)$. Punkty styczności oznaczono przez A $\mathrm{i}B$. Wyrazič pole

trójkata $ABC$ jako funkcje promienia $r\mathrm{i}$ znalez$\acute{}$č najwieksza wartośč

tego pola.

10.7. Rozwiazač uklad równań

$\left\{\begin{array}{l}
x^{2}+y^{2}=5|x|\\
|4y-3x+10|=10.
\end{array}\right.$

Podač interpretacje geometryczna $\mathrm{k}\mathrm{a}\dot{\mathrm{z}}$ dego $\mathrm{z}$ równań $\mathrm{i}$ sporzadzič sta-

ranny rysunek.

10.8. Rozwiazač $\mathrm{w}$ przedziale $[0,\pi]$ równanie

1$+ \sin 2x=2\sin^{2}x,$

a nastepnie nierównośč l$+ \sin 2x>2\sin^{2}x.$





22

Praca kontrolna nr 4

$\mathrm{W}$ celu przyblizenia sluchaczom, jakie wymagania byly stawiane ich

starszym kolegom przed ponad dwudziestu laty, niniejszy zestaw

zadań jest powtórzeniem pracy kontrolnej ze stycznia 1979 $\mathrm{r}.$

ll.l. Przez środek boku trójkata równobocznego przeprowadzono prosta,

tworzaca $\mathrm{z}$ tym bokiem $\mathrm{k}\mathrm{a}\mathrm{t}$ ostry $\alpha \mathrm{i}$ dzielaca ten trójkat na dwie

figury, których stosunek pól jest równy 1 : 7. Ob1iczyč miare kata $\alpha.$

11.2. $\mathrm{W}$ kule $0$ promieniu $R$ wpisano graniastoslup trójkatny prawidlowy

$0$ krawedzi podstawy równej promieniowi kuli. Obliczyč wysokośč

tego graniastoslupa.

11.3. Wyznaczyč wartości parametru $\alpha$, dla których funkcja

$f(x)=\displaystyle \frac{\alpha x}{1+x^{2}}$

osiaga maksimum równe 2.

11.4. Rozwiazač nierównośč

$\cos^{2}x+\cos^{3}x+\ldots+\cos^{n+1}x+\ldots<1+\cos x$

dla $x\in[0,2\pi].$

11.5. Wykazač, $\dot{\mathrm{z}}\mathrm{e}$ dla $\mathrm{k}\mathrm{a}\dot{\mathrm{z}}$ dej liczby naturalnej $n$

równośč

$\geq$

2 prawdziwa jest

$1^{2}+2^{2}++n^{2}=\left(\begin{array}{lll}
n & + & 1\\
 & 2 & 
\end{array}\right)+2[\left(\begin{array}{l}
n\\
2
\end{array}\right)+\left(\begin{array}{lll}
n & - & 1\\
 & 2 & 
\end{array}\right)+\ldots+\left(\begin{array}{l}
2\\
2
\end{array}\right)]$

11.6. Wyznaczyč równanie linii bedacej zbiorem środków wszystkich okre-

gów stycznych do prostej $y=0$ ijednocześnie stycznych zewnetrznie

do okregu $(x+2)^{2}+y^{2}=4$. Narysowač $\mathrm{t}\mathrm{e}$ linie.

11.7. Wyznaczyč wartości parametru $m$, dla których równanie

$9x^{2}-3x\log_{3}m+1=0$ ma dwa rózne pierwiastki rzeczywiste $x_{1}, x_{2}$

spelniajace warunek $x_{1}^{2}+x_{2}^{2}=1.$

11.8. Rozwiazač nierównośč

$\displaystyle \frac{\sqrt{30+x-x^{2}}}{x}<\frac{\sqrt{10}}{5}.$





23

Praca kontrolna nr 5

12.1. Za pomoca odpowiedniego wykresu wykazač, $\dot{\mathrm{z}}\mathrm{e}$ równanie

$\sqrt{x-3}+x = 4$ ma dokladnie jeden pierwiastek. Nastepnie wyz-

naczyč ten pierwiastek analitycznie.

12.2. Wiadomo, $\dot{\mathrm{z}}\mathrm{e}$ wielomian $w(x) = 3x^{3}-5x+1$ ma trzy pierwiastki

rzeczywiste $x_{1}, x_{2}, x_{3}$. Bez wyznaczania tych pierwiastków obliczyč

wartośč wyrazenia $(1+x_{1})(1+x_{2})(1+x_{3})$

12.3. Rzucono jeden raz kostka, a nastepnie moneta tyle razy, ile oczek

pokazala kostka. Obliczyč prawdopodobieństwo tego, $\dot{\mathrm{z}}\mathrm{e}$ rzuty mone-

ta daly co najmniej jednego orla.

12.4. Wyznaczyč równania wszystkich okregów stycznych do obu osi ukladu

wspólrzednych oraz do prostej $3x+4y=12.$

12.5. $\mathrm{W}$ ostroslupie prawidlowym czworokatnym dana jest odleglośč $d$

środka podstawy od krawedzi bocznej oraz $\mathrm{k}\mathrm{a}\mathrm{t}2\alpha$ miedzy sasiednimi

ścianami bocznymi. Obliczyč objetośč ostroslupa.

12.6. $\mathrm{W}$ trapezie równoramiennym $0$ polu $P$ dane sa promień okregu opisa-

nego $r$ oraz suma dlugości obu podstaw $s$. Obliczyč obwód tego tra-

pezu. Podač warunki rozwiazalności zadania. Sporzadzič rysunek dla

$P=12\mathrm{c}\mathrm{m}^{2}, r=3$ cmi $s=8$ cm.

12.7. Rozwiazač uklad równań

$\left\{\begin{array}{l}
px+y=3p^{2}-3p-2\\
(p+2)x+py=4p
\end{array}\right.$

$\mathrm{w}$ zalezności od parametru rzeczywistego $p$. Podač wszystkie rozwia-

zania ($\mathrm{i}$ odpowiadajace im wartości parametru $p$), dla których obie

niewiadome sa liczbami calkowitymi $0$ wartości bezwzglednej mniej-

szej od 3.

12.8. Odcinek AB $0$ końcach $A(0,\displaystyle \frac{3}{2}) \mathrm{i}B(1,y)$, gdzie $ y\in [0,\displaystyle \frac{3}{2}]$, obraca

$\mathrm{s}\mathrm{i}\mathrm{e}$ wokól osi $Ox$. Wyrazič pole powstalej powierzchni jako funkcje

zmiennej $y\mathrm{i}$ znalez$\acute{}$č najmniejsza wartośč tego pola. Sporzadzič ry-

sunek.





24

Praca kontrolna nr 6

13.1. Wykazač, $\dot{\mathrm{z}}\mathrm{e}$ dla $\mathrm{k}\mathrm{a}\dot{\mathrm{z}}$ dego kata $\alpha$ prawdziwa jest nierównośč

$\sqrt{3}\sin\alpha+\sqrt{6}\cos\alpha\leq 3.$

13.2. Dane sa punkty $A(2,2)\mathrm{i}B(-1,4)$. Wyznaczyč dlugośč rzutu prosto-

katnego odcinka $AB$ na prosta $0$ równaniu $12x+5y=30$. Sporzadzič

rysunek.

13.3. Niech $f(m)$ bedzie suma odwrotności pierwiatków rzeczywistych rów-

nania kwadratowego

$(2^{m}-7)x^{2}-|2^{m+1}-8|x+2^{m}=0,$

gdzie m jest parametrem rzeczywistym.

f(m) i narysowač wykres tej funkcji.

Napisač wzór określajacy

13.4. Dwóch strzelców strzela równocześnie do tego samego celu niezaleznie

od siebie. Pierwszy strzelec trafia za $\mathrm{k}\mathrm{a}\dot{\mathrm{z}}$ dym razem $\mathrm{z}$ prawdopodobień-

stwem $\displaystyle \frac{2}{3}\mathrm{i}$ oddaje 2 strza1y, a drugi trafia $\mathrm{z}$ prawdopodobieństwem $\displaystyle \frac{1}{2}$

$\mathrm{i}$ oddaje 5 strza1ów. Ob1iczyč prawdopodobieństwo, $\dot{\mathrm{z}}\mathrm{e}$ cel zostanie

trafiony dokladnie 3 razy.

13.5. Liczby $\alpha_{1}, \alpha_{2}, \alpha_{n}, n\geq 3$, tworza ciag arytmetyczny. Suma wyrazów

tego ciagu wynosi 28, suma wyrazów $0$ numerach nieparzystych wyno-

si 16, a i1oczyn $\alpha_{2}\cdot\alpha_{3}=48$. Wyznaczyč te liczby.

13.6. $\mathrm{W}$ trójkacie $ABC, \mathrm{w}$ którym $|AB| = 7$ oraz $|AC| =9$, a $\mathrm{k}\mathrm{a}\mathrm{t}$ przy

wierzcholku $A$ jest dwa razy wiekszy $\mathrm{n}\mathrm{i}\dot{\mathrm{z}} \mathrm{k}\mathrm{a}\mathrm{t}$ przy wierzcholku $B.$

Obliczyč stosunek promienia kola wpisanego $\mathrm{w}$ trójkat do promienia

kola opisanego na tym trójkacie. Rozwiazanie zilustrowač rysunkiem.

13.7. Zaznaczyč na plaszczy $\acute{\mathrm{z}}\mathrm{n}\mathrm{i}\mathrm{e}$ nastepujace zbiory punktów

$A=\{(x,y):x+y-2\geq|x-2|\},$

$B=\{(x,y):y\leq\sqrt{4x-x^{2}}\}.$

Nastepnie znalez$\acute{}$č na brzegu zbioru $A\cap B$ punkt $Q$, którego odleglośč

od punktu $P(\displaystyle \frac{5}{2},1)$ jest najmniejsza.

13.8. Zbadač przebieg zmienności $\mathrm{i}$ narysowač wykres funkcji

$f(x)=\displaystyle \frac{1}{2}x^{2}-4+\sqrt{8-x^{2}}.$





25

Praca kontrolna nr 7

14.1. Ile elementów ma zbiór $A$, jeśli liczba jego podzbiorów trójelemen-

towych jest wieksza $048$ od liczby podzbiorów dwuelementowych?

14.2. $\mathrm{W}$ sześciokat foremny $0$ boku l wpisano okrag. Nastepnie $\mathrm{w}$ otrzy-

many okrag wpisano sześciokat foremny, $\mathrm{w}$ który znów wpisano okrag

$\mathrm{i}\mathrm{t}\mathrm{d}$. Obliczyč sume obwodów wszystkich otrzymanych $\mathrm{w}$ taki sposób

okregów.

14.3. Dana jest rodzina prostych $0$ równaniach $2x+my-m-2 = 0,$

$m\in R$. Które $\mathrm{z}$ prostych tej rodziny sa:

a) prostopadle do prostej $x+4y+2=0,$

b) równolegle do prostej $3x+2y=0,$

c) tworza $\mathrm{z}$ prosta $x-\displaystyle \sqrt{3}y-1=0\mathrm{k}\mathrm{a}\mathrm{t}\frac{\pi}{3}.$

14.4. Sprawdzič $\mathrm{t}\mathrm{o}\dot{\mathrm{z}}$ samośč tg $(x-\displaystyle \frac{\pi}{4})-1=\frac{-2}{\mathrm{t}\mathrm{g}x+1}$. Korzystajac $\mathrm{z}$ niej,

sporzadzič wykres funkcji $f(x)=\displaystyle \frac{1}{\mathrm{t}\mathrm{g}x+1}\mathrm{w}$ przedziale $[0,\pi].$

14.5. Dany jest okrag $K\mathrm{o}$ równaniu $x^{2}+y^{2}-6y=27$. Wyznaczyč równanie

krzywej $\Gamma$ bedacej obrazem okregu $K\mathrm{w}$ powinowactwie prostokatnym

$0$ osi $ox \mathrm{i}$ skali $k = \displaystyle \frac{1}{3}$. Obliczyč pole figury ograniczonej lukiem

okregu $K\mathrm{i}$ krzywej $\Gamma, \mathrm{l}\mathrm{e}\dot{\mathrm{z}}$ acej pod osia odcietych. Wykonač rysunek.

14.6. Korzystajac $\mathrm{z}$ nierówności $2\sqrt{\alpha b} \leq \alpha+b, \alpha, b > 0$, obliczyč gra-

nice $\displaystyle \lim_{n\rightarrow\infty}(\frac{\log_{5}16}{\log_{2}3})^{n}$

14.7. Trylogie skladajaca $\mathrm{s}\mathrm{i}\mathrm{e}\mathrm{z}$ dwóch powieści dwutomowych oraz jednej

jednotomowej ustawiono na pólce $\mathrm{w}$ przypadkowej kolejności. Jakie

jest prawdopodobieństwo tego, $\dot{\mathrm{z}}\mathrm{e}$ tomy a) obydwu, b) co najmniej

jednej $\mathrm{z}$ dwutomowych powieści znajduja $\mathrm{s}\mathrm{i}\mathrm{e}$ obok siebie $\mathrm{i}$ przy tym

tom I $\mathrm{z}$ lewej, a tom II $\mathrm{z}$ prawej strony.

14.8. $\mathrm{W}$ ostroslupie prawidlowym czworokatnym krawed $\acute{\mathrm{z}}$ boczna jest na-

chylona do plaszczyzny podstawy pod katem $\alpha$, a krawed $\acute{\mathrm{z}}$ podstawy

ma dlugośč $\alpha$. Obliczyč promień kuli stycznej do wszystkich krawedzi

tego ostroslupa. Sporzadzič odpowiednie rysunki.





SPIS TREŚCI

l. Przedmowa $\ldots\ldots\ldots\ldots\ldots\ldots\ldots\ldots\ldots\ldots\ldots\ldots\ldots\ldots\ldots\ldots$.. 5

2. Zadania $\ldots\ldots\ldots\ldots\ldots\ldots\ldots\ldots\ldots\ldots\ldots\ldots\ldots\ldots\ldots\ldots\ldots$.. 7

7

3. Indeks tematyczny $\ldots\ldots\ldots\ldots\ldots\ldots\ldots\ldots\ldots\ldots\ldots\ldots\ldots\ldots$. 57

4. Odpowiedzi do zadań $\ldots\ldots\ldots\ldots\ldots\ldots\ldots\ldots\ldots\ldots\ldots\ldots\ldots$. 65

5. Wskazówki do zadań $\ldots\ldots\ldots\ldots\ldots\ldots\ldots\ldots\ldots\ldots\ldots\ldots\ldots$.. 97

6. 12 przykladowych rozwiazań

135





Edycja

XXXI

2001/2002





29

Praca kontrolna nr l

15.1. Dwaj rowerzyści wyruszyli jednocześnie $\mathrm{w}$ droge, jeden $\mathrm{z}$ A do $\mathrm{B},$

drugi $\mathrm{z} \mathrm{B}$ do A $\mathrm{i}$ spotkali $\mathrm{s}\mathrm{i}\mathrm{e}$ po jednej godzinie. Pierwszy $\mathrm{z}$ nich

przebywal $\mathrm{w}$ ciagu godziny $03$ km wiecej $\mathrm{n}\mathrm{i}\dot{\mathrm{z}}$ drugi $\mathrm{i}$ przyjechal do celu

$027$ minut wcześniej $\mathrm{n}\mathrm{i}\dot{\mathrm{z}}$ drugi. Jakie byly predkości obu rowerzystów

$\mathrm{i}$ jaka jest odleglośč AB?

15.2. Rozwiazač nierównośč $\displaystyle \sqrt{x^{2}-3}>\frac{2}{x}.$

15.3. Rysunek przedstawia dach budynku $\mathrm{w}$ rzucie poziomym.

$\mathrm{z}\mathrm{p}$ aszczyznjest nachylona do $\mathrm{p}$ aszczyzny

poziomej pod katem $30^{\mathrm{O}} \mathrm{D}$ ugośč dachu

wynosi 18 $\mathrm{m}$, a szerokośč 9 $\mathrm{m}$. Obliczyč

pole powierzchni dachu oraz ca kowita ku-

bature strychu $\mathrm{w}$ tym budynku.

$K\mathrm{a}\dot{\mathrm{z}}$ da
\begin{center}
\includegraphics[width=48.204mm,height=24.228mm]{./KursMatematyki_PolitechnikaWroclawska_1999_2004_page21_images/image001.eps}
\end{center}
15.4. Pewna firma przeprowadza co kwartal regulacje plac dla swoich pra-

cowników, waloryzujac je zgodnie ze wska $\acute{\mathrm{z}}\mathrm{n}\mathrm{i}\mathrm{k}\mathrm{i}\mathrm{e}\mathrm{m}$ inflacji, który jest

staly $\mathrm{i}$ wynosi 1,5\% kwarta1nie, oraz do1iczajac sta1a kwote podwyzki

16 $\mathrm{z}l. \mathrm{W}$ styczniu 2001 pan Kowa1ski otrzyma1 wynagrodzenie 1600

$\mathrm{z}l$. Jaka pensje otrzyma $\mathrm{w}$ kwietniu 2002? Wyznaczyč wzór ogó1ny

na pensje $w_{n}$ pana Kowalskiego $\mathrm{w}n$-tym kwartale, przyjmujac, $\dot{\mathrm{z}}\mathrm{e}$

$w_{1}=1600$ jest placa $\mathrm{w}$ pierwszym kwartale 2001. Ob1iczyč średnia

miesieczna place pana Kowalskiego $\mathrm{w}$ 2002 roku.

15.5. Wyznaczyč funkcje odwrotna do $f(x) = x^{3}, x \in R$. Nastepnie

narysowač wykres funkcji $h(x) = \sqrt[3]{(|x|-1)}+1$, wyrazajac $\mathrm{j}\mathrm{a}$

za pomoca $f^{-1}$

15.6. Rozwiazač równanie $\displaystyle \frac{\sin 2x}{\cos 4x}=1.$

15.7. Dany jest trójkat $0$ wierzcholkach $A(-2,1), B(-1,-6), C(2,5).$

Za pomoca rachunku wektorowego obliczyč cosinus kata miedzy dwu-

sieczna kata $A\mathrm{i}$ środkowa boku $BC$. Sporzadzič rysunek.

15.8. Zbadač przebieg zmienności $\mathrm{i}$ narysowač wykres funkcji

$f(x)=x+\displaystyle \frac{x}{x-1}+\frac{x}{(x-1)^{2}}+\frac{x}{(x-1)^{3}}+$





30

Praca kontrolna nr 2

16.1. Cena llitra paliwa zostala obnizona $0$ 15\%. Po dwóch tygodniach

dokonano kolejnej zmiany ceny llitra paliwa, podwyzszajacja $0$ 15\%.

$\mathrm{O}$ ile procent końcowa cena paliwa rózni $\mathrm{s}\mathrm{i}\mathrm{e}$ od poczatkowej?

16.2. Wyznaczyč $\mathrm{i}$ narysowač zbiór zlozony $\mathrm{z}$ punktów $(x,y)$ plaszczyzny

spelniajacych warunek

$x^{2}+y^{2}=8|x|+6|y|.$

16.3. Wysokośč ostroslupa trójkatnego prawidlowego wynosi $h$, a $\mathrm{k}\mathrm{a}\mathrm{t}$ mie-

dzy wysokościami ścian bocznych poprowadzonymi $\mathrm{z}$ wierzcholka

ostroslupa jest równy $ 2\alpha$. Obliczyč pole powierzchni bocznej tego

ostroslupa. Sporzadzič odpowiednie rysunki.

16.4. $\mathrm{Z}$ arkusza blachy $\mathrm{w}$ ksztalcie równolegloboku $0$ bokach 30 cm $\mathrm{i}60$ cm

$\mathrm{i}$ kacie ostrym $60^{\mathrm{o}}$ nalezy odciač dwa przeciwlegle trójkatne narozniki

$\mathrm{t}\mathrm{a}\mathrm{k}$, aby powstal romb $\mathrm{o}\mathrm{m}\mathrm{o}\dot{\mathrm{z}}$ liwie najwiekszym polu. Określič przez

który punkt na dluzszym boku równolegloboku nalezy przeprowadzič

ciecie oraz obliczyč $\mathrm{k}\mathrm{a}\mathrm{t}$ ostry otrzymanego rombu. Wynik zaokraglič

do jednej minuty katowej.

16.5. Rozwiazač równanie

$2^{\log_{\sqrt{2}^{X}}}=(\sqrt{2})^{\log_{x}2}$

16.6. Wyznaczyč dziedzine $\mathrm{i}$ zbiór wartości funkcji

$f(x)=\displaystyle \frac{4}{\sin x+2\cos x+3}.$

16.7. Znalez$\acute{}$č wszystkie wartości parametru $p$, dla których równanie

$px^{4}-4x^{2}+p+1=0$

ma dwa rózne pierwiastki.

16.8. Wyznaczyč tangens kata, pod którym styczna do wykresu funkcji

$f(x)=\displaystyle \frac{8}{x^{2}+3}$

$\mathrm{w}$ punkcie $A($3, $\displaystyle \frac{2}{3})$ przecina ten wykres.





31

Praca kontrolna nr 3

17.1. Dla jakich wartości $\sin x$ liczby $\sin x, \cos x, \sin 2x (\mathrm{w}$ podanym

porzadku) sa kolejnymi wyrazami ciagu geometrycznego? Wyznaczyč

czwarty wyraz tego ciagu dla $\mathrm{k}\mathrm{a}\dot{\mathrm{z}}$ dego $\mathrm{z}$ rozwiazań.

17.2. $\mathrm{W}$ pewnych zawodach sportowych startuje 16 druzyn. $\mathrm{W}$ elimina-

cjach sa one losowo dzielone na 4 grupy po 4 druzyny $\mathrm{w}\mathrm{k}\mathrm{a}\dot{\mathrm{z}}$ dej grupie.

Obliczyč prawdopodobieństwo tego, $\dot{\mathrm{z}}\mathrm{e}$ trzy zwycieskie druzyny $\mathrm{z}$ po-

przednich zawodów znajda $\mathrm{s}\mathrm{i}\mathrm{e}\mathrm{w}$ trzech róznych grupach.

17.3. Nie wykonujac dzielenia, udowodnič, $\dot{\mathrm{z}}\mathrm{e}$ wielomian

$(x^{2}+x+1)^{3}-x^{6}-x^{3}-1$

jest podzielny przez trójmian $(x+1)^{2}$

17.4. Wyznaczyč równanie okregu $0$ promieniu $r$ stycznego do paraboli

$y=x^{2}\mathrm{w}$ dwóch punktach. Dla jakiego $r$ zadanie ma rozwiazanie?

Sporzadzič rysunek, przyjmujac $r=3/2.$

17.5. Stosujac zasade indukcji matematycznej, udowodnič prawdziwośč

wzoru

$\left(\begin{array}{l}
2\\
2
\end{array}\right) - \left(\begin{array}{l}
3\\
2
\end{array}\right)+\left(\begin{array}{l}
4\\
2
\end{array}\right) - \left(\begin{array}{l}
5\\
2
\end{array}\right)+\ldots+\left(\begin{array}{l}
2n\\
2
\end{array}\right) =n^{2},$

$n\geq 1.$

17.6. Rozwiazač nierównośč

$\log_{x}(1-6x^{2})\geq 1.$

17.7. $\mathrm{W}$ trapezie ABCD opisanym na okregu $0$ środku $S$ dane sa ramie

$|AD| =c$ oraz $|AS| =d$. Punkt styczności okregu $\mathrm{z}$ podstawa $AB$

dzieli $\mathrm{j}\mathrm{a}\mathrm{w}$ stosunku 1 : 2. Ob1iczyč po1e tego trapezu. Sporzadzič

rysunek dla $c=5\mathrm{i}d=4.$

17.8. Wszystkie ściany równoleglościanu sa rombami $0$ boku $\alpha \mathrm{i}$ kacie

ostrym $\beta$. Obliczyč objetośč tego równoleglościanu. Sporzadzič

rysunek. Obliczenia odpowiednio uzasadnič.





32

Praca kontrolna nr 4

18.1. Obliczyč granice ciagu 0 wyrazie ogólnym

$\displaystyle \alpha_{n}=\frac{2^{n}+2^{n+1}+..+2^{2n}}{2^{2}+2^{4}+\ldots+2^{2n}}.$

18.2. Wyznaczyč równanie prostej prostopadlej do prostej $0$ równaniu

$2x+3y+3 = 0 \mathrm{i} \mathrm{l}\mathrm{e}\dot{\mathrm{z}}$ acej $\mathrm{w}$ równej odleglości od dwóch danych

punktów $A(-1,1)\mathrm{i}B(3,3)$. Sporzadzič rysunek.

18.3. Tworzaca stozka ma dlugośč $l \mathrm{i}$ widač $\mathrm{j}\mathrm{a}$ ze środka kuli wpisanej

$\mathrm{w}$ ten stozek pod katem $\alpha$. Obliczyč objetośč $\mathrm{i}\mathrm{k}\mathrm{a}\mathrm{t}$ rozwarcia stozka.

Określič dziedzine kata $\alpha.$

18.4. Bolek kupil jeden dlugopis $\mathrm{i}k$ zeszytów, zaplacil $k\mathrm{z}l\mathrm{i}50$ gr, a Lolek

kupil $k$ dlugopisów $\mathrm{i}4$ zeszyty, $\mathrm{i}$ zaplaci12, 5 $k\mathrm{z}l$. Wyznaczyč cene

dlugopisu $\mathrm{i}$ zeszytu $\mathrm{w}$ zalezności od parametru $k$. Znalez$\acute{}$č wszystkie

$\mathrm{m}\mathrm{o}\dot{\mathrm{z}}$ liwe wartości tych cen wiedzac, $\dot{\mathrm{z}}\mathrm{e}$ zeszyt kosztuje nie mniej $\mathrm{n}\mathrm{i}\dot{\mathrm{z}}$

50 gr, dlugopis jest drozszy od zeszytu, a ceny obydwu artykulów

wyrazaja $\mathrm{s}\mathrm{i}\mathrm{e}\mathrm{w}$ pelnych zlotych $\mathrm{i}$ dziesiatkach groszy.

18.5. Rozwiazač nierównośč $\mathrm{t}\mathrm{g}^{3}x\geq\sin 2x.$

18.6. $\dot{\mathrm{Z}}$ arówki sa sprzedawane $\mathrm{w}$ opakowaniach po 6 sztuk. Prawdopodo-

bieństwo, $\dot{\mathrm{z}}\mathrm{e}$ pojedyncza $\dot{\mathrm{z}}$ arówka jest dobra wynosi $\displaystyle \frac{2}{3}$. Jakie jest

prawdopodobieństwo tego, $\dot{\mathrm{z}}\mathrm{e} \mathrm{w}$ jednym opakowaniu znajda $\mathrm{s}\mathrm{i}\mathrm{e}$ co

najmniej 4 dobre $\dot{\mathrm{z}}$ arówki. $\mathrm{O}$ ile zwiekszy $\mathrm{s}\mathrm{i}\mathrm{e}$ prawdopodobieństwo

tego zdarzenia, jeśli jedna, wylosowana $\mathrm{z}$ opakowania $\dot{\mathrm{z}}$ arówka, oka-

zala $\mathrm{s}\mathrm{i}\mathrm{e}$ dobra.

18.7. Prosta styczna $\mathrm{w}$ punkcie $P$ do okregu $0$ promieniu 2 $\mathrm{i}$ pólprosta

wychodzaca ze środka okregu majaca $\mathrm{z}$ okregiem punkt wspólny $S$

przecinaja $\mathrm{s}\mathrm{i}\mathrm{e}\mathrm{w}$ punkcie $A$ pod katem $60^{\circ}$ Znalez$\acute{}$č promień okregu

stycznego do odcinków $AP$, {\it AS} $\mathrm{i}$ luku $PS$. Sporzadzič rysunek.

18.8. $\mathrm{W}$ ostroslupie prawidlowym, którego podstawa jest kwadrat, pole

$\mathrm{k}\mathrm{a}\dot{\mathrm{z}}$ dej $\mathrm{z}$ pieciu ścian wynosi l. Ostroslup ten ścieto plaszczyzna

równolegla do podstawy $\mathrm{t}\mathrm{a}\mathrm{k}$, aby uzyskač maksymalny stosunek obje-

tości do pola powierzchni calkowitej. Obliczyč pole powierzchni calko-

witej otrzymanego ostroslupa ścietego. Sporzadzič rysunek.





33

Praca kontrolna nr 5

19.1. $\mathrm{W}$ czworokacie ABCD dane sa wktory $\vec{AB}= [2,-1], \vec{BC}= [3$, 3$],$

$\vec{CD}=[-4,1]$. Punkty $K\mathrm{i}M$ sa środkami boków $CD$ oraz $AD$. Za

pomoca rachunku wektorowego obliczyč pole trójkata $KMB.$

Sporzadzič rysunek.

19.2. Trzy rózne krawedzie oraz przekatna prostopadlościanu tworza cztery

kolejne wyrazy ciagu arytmetycznego. Wyznaczyč sume dlugości

wszystkich krawedzi tego prostopadlościanu, jeśli przekatna ma dlu-

gośč 7 cm.

19.3. Na plaszczy $\acute{\mathrm{z}}\mathrm{n}\mathrm{i}\mathrm{e}Oxy$ dane sa zbiory $A = \{(x,y):y\leq\sqrt{5x-x^{2}}\}$

oraz $B_{s} = \{(x,y):3x+4y=s\}$. Dla jakich wartości parametru $s$

zbiór $A\cap B_{s}$ nie jest pusty? Sporzadzič rysunek.

19.4. Dzialka gruntu ma ksztalt trapezu $0$ bokach 20 $\mathrm{m}, 30\mathrm{m}, 40\mathrm{m}\mathrm{i}60$

$\mathrm{m}$. Wlaściciel dzialki twierdzi, $\dot{\mathrm{z}}\mathrm{e}$ powierzchnia jego dzialki wynosi

ponad ll arów. Czy wlaściciel ma racje? Jeśli $\mathrm{t}\mathrm{a}\mathrm{k}$, to narysowač plan

dzialki $\mathrm{w}$ skali 1:1000 $\mathrm{i}$ podač jej dokladna powierzchnie.

19.5. Dane jest równanie kwadratowe $\mathrm{z}$ parametrem $m$

$(m+2)x^{2}+4\sqrt{m}x+(m-3)=0.$

Dlajakiej wartości parametru $m$ kwadrat róznicy pierwiastków rzeczy-

wistych tego równaniajest najwiekszy. Podač $\mathrm{t}\mathrm{e}$ najwieksza wartośč.

19.6. Stosujac zasade indukcji matematycznej, udowodnič, $\dot{\mathrm{z}}\mathrm{e}$ dla $\mathrm{k}\mathrm{a}\dot{\mathrm{z}}$ dego

$n\geq 2$ liczba $2^{2^{n}}-6$ jest podzielna przez 10.

19.7. Rozwiazač uklad równań

$\left\{\begin{array}{l}
\mathrm{t}\mathrm{g}x+\mathrm{t}\mathrm{g}y=4\\
\cos(x+y)+\cos(x-y)=\frac{1}{2}
\end{array}\right.$

dla $x, y\in[-\pi,\pi].$

19.8. Równoramienny trójkat prostokatny $ABC$ zgieto wzdluz środkowej

$CD$ wychodzacej $\mathrm{z}$ wierzcholka kata prostego $C\mathrm{t}\mathrm{a}\mathrm{k}, \dot{\mathrm{z}}\mathrm{e}$ obie polowy

tego trójkata utworzyly $\mathrm{k}\mathrm{a}\mathrm{t}60^{\circ}$ Obliczyč sinusy wszystkich katów

dwuściennych otrzymanego czworościanu ABCD. Rozwiazanie zilus-

trowač odpowiednimi rysunkami, a obliczenia uzasadnič.





34

Praca kontrolna nr 6

20.1. Wyznaczyč wszystkie wartości parametru rzeczywistego

których prosta $x = m$ jest osia symetrii wykresu

$p(x)=(m^{2}-2m)x^{2}-(2m-4)x+3$. Sporzadzič rysunek.

m, dla

funkcji

20.2. $\mathrm{Z}$ kuli $0$ promieniu $R$ wycieto ósma cześč trzema wzajemnie prosto-

padlymi plaszczyznami przechodzacymi przez środek kuli. $\mathrm{W}$ tak

otrzymana bryle wpisano inna kule. Obliczyč stosunek pola powierz-

chni tej kuli do pola powierzchni bryly.

20.3. $\mathrm{W}$ trzech pustych urnach $K, L, M$ rozmieszczamy losowo 4 rózne

kule. Obliczyč prawdopodobieństwo tego, $\dot{\mathrm{z}}\mathrm{e}\dot{\mathrm{z}}$ adna $\mathrm{z}$ urn $K\mathrm{i}L$ nie

pozostanie pusta.

20.4. Dane sa punkty $A(2,6), B(-2,6) \mathrm{i} C(0,0)$. Wyznaczyč równanie

linii zawierajacej wszystkie punkty trójkata $ABC$, dla których suma

kwadratów ich odleglości od trzech boków jest stala $\mathrm{i}$ wynosi 9. Spo-

rzadzič rysunek.

20.5. Narysowač dokladny wykres $\mathrm{i}$ napisač równania asymptot funkcji

$f(x)=\displaystyle \frac{(x+1)^{2}-1}{x|x-1|}$

nie badajac jej przebiegu.

20.6. Rozwiazač nierównośč

$|x|^{2x-1}\displaystyle \leq\frac{1}{x^{2}}.$

20.7. Styczna do wykresu funkcji $f(x) = \sqrt{3+x}+\sqrt{3-x} \mathrm{w}$ punkcie

$A(x_{0},f(x_{0}))$ przecina oś $Ox \mathrm{w}$ punkcie $P$, a oś $Oy \mathrm{w}$ punkcie $Q$

$\mathrm{t}\mathrm{a}\mathrm{k}, \dot{\mathrm{z}}\mathrm{e} |OP|=|OQ|$. Wyznaczyč $x_{0}.$

20.8. Trójkat równoboczny $0$ boku $\alpha$ podzielono prosta $l$ na dwie figury,

których stosunek pól jest równy 1 : 5. Prosta ta przecina bok $AC$

$\mathrm{w}$ punkcie $D$ pod katem $15^{\circ}$, a bok AB $\mathrm{w}$ punkcie $E$. Wykazač, $\dot{\mathrm{z}}\mathrm{e}$

$|AD|+|AE|=\alpha.$





35

Praca kontrolna nr 7

21.1. Sześcian $0$ krawedzi 3 cm ma objetośč taka sama jak dwa sześciany,

których suma obydwu krawedzi wynosi 4 cm. $\mathrm{O}$ ile $\mathrm{c}\mathrm{m}^{2}$ pole powierz-

chni wiekszego sześcianujest mniejsze od sumy pól powierzchni dwóch

mniejszych sześcianów.

21.2. Obliczyč tangens kata utworzonego przez przekatne czworokata

$0$ wierzcholkach $A(1,1), B(2,0), C(2,4), D(0,6)$. Rozwiazanie zilu-

strowač rysunkiem.

21.3. $\mathrm{W}$ trójkat prostokatny wpisano okrag, a $\mathrm{w}$ okrag ten wpisano podobny

trójkat prostokatny. Wyznaczyč cosinusy katów ostrych trójkata,

jeśli wiadomo, $\dot{\mathrm{z}}\mathrm{e}$ stosunek pól obu trójkatów wynosi 9.

21.4. Wykazač, $\dot{\mathrm{z}}\mathrm{e}$ ciag

granice.

$\alpha_{n}=\sqrt{n(n+1)}-n$ jest rosnacy. Obliczyč jego

21.5. Rozwiazač nierównośč

$2\displaystyle \cos^{2}\frac{x}{4}>1.$

21.6. Rozwiazač równanie

$\displaystyle \log_{2}(1-x)+\log_{4}(x+4)=\log_{4}(x^{3}-x^{2}-3x+5)+\frac{1}{2}.$

Nie wyznaczač dziedziny równania $\mathrm{w}$ sposób jawny.

21.7. $\mathrm{W}$ kule $0$ promieniu $R$ wpisano stozek $0$ najwiekszej objetości. Wyz-

naczyč promień podstawy $r \mathrm{i}$ wysokośč $h$ tego stozka. Sporzadzič

rysunek.

21.8. Znalez$\acute{}$č równania wszystkich prostych, które sa styczne jednocześnie

do krzywych

$y=-x^{2},y=x^{2}-8x+18.$

Sporzadzič rysunek.





Edycja

XXXII

2002/2003





39

Praca kontrolna nr l

22.1. Narysowač wykres funkcji $y = 4+2|x| -x^{2}$ Na podstawie tego

wykresu określič liczbe rozwiazań równania $4 + 2|x| - x^{2} = p$

$\mathrm{w}$ zalezności od parametru rzeczywistego $p.$

22.2. Pompa napelniajaca pusty basen $\mathrm{w}$ pierwszej minucie pracy miala

wydajnośč 0,2 $\mathrm{m}^{3}/\mathrm{s}$, a $\mathrm{w}\mathrm{k}\mathrm{a}\dot{\mathrm{z}}$ dej kolejnej minuciejej wydajnośč zwiek-

szano $0 0,01 \mathrm{m}^{3}/\mathrm{s}$. Polowa basenu zostala napelniona po $2n$ mi-

nutach, a caly basen po kolejnych $n$ minutach, gdzie $n$ jest liczba

naturalna. Wyznaczyč czas napelniania basenu oraz jego pojemnośč.

22.3. Stozek ścietyjest opisany na kuli $0$ promieniu $r=2$ cm. Objetośč kuli

stanowi 25\% objetości stozka. Wyznaczyč średnice podstaw $\mathrm{i}$ dlu-

gośč tworzacej tego stozka.

22.4. $\mathrm{W}$ trójkacie $ABC$ dane sa promień okregu opisanego $R, \mathrm{k}\mathrm{a}\mathrm{t}\angle A=\alpha$

oraz $|AB|=\displaystyle \frac{8}{5}R$. Obliczyč pole tego trójkata.

22.5. Rozwiazač nierównośč

$(\sqrt{x})^{\log_{8}x}\geq\sqrt[3]{16x}.$

22.6. $\mathrm{W}$ czworokacie ABCD odcinki AB $\mathrm{i} BD$ sa prostopadle,

$|AD| = 2|AB| = \alpha$ oraz $\vec{AC}= \displaystyle \frac{5}{3} \vec{AB} +\displaystyle \frac{1}{3} \vec{AD}$. Wyznaczyč cosi-

nus kata $\angle BCD = \alpha$ oraz obwód czworokata ABCD. Sporzadzič

rysunek.

22.7. Rozwiazač równanie

$\displaystyle \frac{1}{\sin x}+\frac{1}{\cos x}=\sqrt{8}.$

22.8. Wyznaczyč równanie prostej stycznej do wykresu funkcji $y = \displaystyle \frac{1}{x^{2}}$

$\mathrm{w}$ punkcie $P(x_{0},y_{0}), x_{0}>0$, takim, $\dot{\mathrm{z}}$ eby odcinek tej stycznej zawarty

$\mathrm{w}$ pierwszej čwiartce ukladu wspólrzednych byl najkrótszy. Rozwia-

zanie zilustrowač odpowiednim wykresem.





Przedmowa

{\it Zbiór obejmuje zadania Korespondencyjnego Kursu} $\mathrm{z}$ {\it Matematyki} $\mathrm{z}$ lat 1999-

2004. Kurs tenjest prowadzony przez Instytut Matematyki Politechniki Wroclaw-

{\it skiej. Jest kontynuacja Korespondencyjnego Kursu Przygotowawczego} $\mathrm{z}$ {\it Mate}-

{\it matyki, który} $\mathrm{w}$ latach $1972-1999$ byl prowadzony wspólnie $\mathrm{z}$ Instytutem Mate-

matyki Politechniki Warszawskiej.

Opracowujac niniejszy zbiór, autor pragnal ulatwič szerokiemu gronu matu-

rzystów $\mathrm{i}$ kandydatów na studia dostęp do materialów kursu ujętych $\mathrm{w}$ wygodna,

zwarta formę $\mathrm{i}\mathrm{w}$ ten sposób pomóc im $\mathrm{w}$ lepszym opanowaniu wiadomości $\mathrm{z}$

matematyki $\mathrm{w}$ zakresie szkoly średniej oraz dač jeszcze jedna okazję do powtó-

rzenia materialu.

Większośč zadań jest oryginalna, częśč pochodzi $\mathrm{z}$ egzaminów wstępnych na

Politechnikę Wroclawska $\mathrm{z}$ ostatnich 201at, a ty1ko niewie1ka 1iczba $\mathrm{z}$ innych

z$\acute{}$ródel ($\mathrm{w}$ tym powtórzenia zadań $\mathrm{z}$ lat ubieglych). Dla udogodnienia samodzielnej

pracy $\mathrm{i}$ zachęcenia do korzystania $\mathrm{z}$ tego zbioru, podano odpowiedzi, a $\mathrm{w}$ oddziel-

nym rozdziale takze wskazówki do wszystkich zadań. $\mathrm{W}$ końcowej części zbioru

przedstawiono 12 przyk1adowych rozwiazań róznorodnych zadań wybranych ze

wszystkich dzialów. Celem ich zamieszczenia jest pokazanie najwazniejszych

metod $\mathrm{i}$ narzędzi $\mathrm{u}\dot{\mathrm{z}}$ ywanych do rozwiazywania zadań. Moga więc sluzyč jako

wzorzec do przygotowania innych rozwiazań. Zastosowano dwuczlonowa nume-

rację zadań uwzględniajaca chronologię kursu. Pierwsza liczba jest kolejnym

numerem pracy kontrolnej ($\mathrm{z}$ okresu $1999-2004$), a druga podaje numer tematu

$\mathrm{w}$ danym zestawie. Dolaczony indeks tematyczny pozwala na szybkie wyszukanie

zadań $\mathrm{z}$ dowolnie wybranego dzialu matematyki.

Ze zbioru moga korzystač zarówno osoby zdajace maturę na poziomie podsta-

wowym, jak $\mathrm{i}$ rozszerzonym. Poczynajac od XXXI edycji, $\mathrm{t}\mathrm{j}$. od pracy kontrolnej

$0$ numerze 15, pierwsze cztery zadania $\mathrm{w}\mathrm{k}\mathrm{a}\dot{\mathrm{z}}$ dym zestawie odpowiadaja zakresowi

podstawowemu, a cztery następne zakresowi rozszerzonemu. Podzial ten dotyczy

$\mathrm{w}$ przyblizeniu takze wcześniejszych prac kontrolnych.

Kurs ma swoja strong internetowa, na której $\mathrm{m}\mathrm{o}\dot{\mathrm{z}}$ na znalez$\acute{}$č zarówno mate-

rialy biezace, jak $\mathrm{i}$ archiwum zawierajace tematy $\mathrm{z}$ lat ubieglych. Dostęp do niej

$\mathrm{m}\mathrm{o}\dot{\mathrm{z}}$ na uzyskač przez strong glówna Politechniki Wroclawskiej: $\mathrm{w}\mathrm{w}\mathrm{w}$. pwr. wroc. pl.

Następnie nalezy wybrač dzial Rekrutacja $\mathrm{i}\mathrm{w}$ nim wyszukač pozycję Korespon-

{\it dencyjny Kurs} $\mathrm{z}$ {\it Matematyki}.

Serdecznie dziękuję Recenzentom Docentowi Zbigniewowi Romanowiczowi

oraz Doktorowi Rościslawowi Rabczukowi za cenne uwagi, które pozwolily usunač

usterki $\mathrm{i}$ ulepszyč pierwotna wersję ksia $\dot{\mathrm{Z}}$ ki.

Wroclaw, marzec 2005

{\it Tadeusz Inglot}





40

Praca kontrolna nr 2

23.1. Czy liczby róznych,,slów'', jakie $\mathrm{m}\mathrm{o}\dot{\mathrm{z}}$ na utworzyč zmieniajac kole-

jnośč liter $\mathrm{w},$,slowach'' TANATAN $\mathrm{i}$ AKABARA, sa takie same?

Uzasadnič odpowied $\acute{\mathrm{z}}$. Przez,,slowo'' rozumiemy tutaj dowolny ciag

liter.

23.2. Reszta $\mathrm{z}$ dzielenia wielomianu $x^{3}+px^{2}-x+q$ przez trójmian $(x+2)^{2}$

wynosi $(-x+1)$. Wyznaczyč pierwiastki tego wielomianu.
\begin{center}
\includegraphics[width=33.324mm,height=23.724mm]{./KursMatematyki_PolitechnikaWroclawska_1999_2004_page30_images/image001.eps}
\end{center}
A $\mathrm{B}$

$A, B$, oraz $\mathrm{z}$ odcinka AB $0$ dugości $\alpha.$

Obliczyč promień okregu stycznego do obu

uków oraz do odcinka $AB.$

23.3. Figura na rysunku sklada $\mathrm{s}\mathrm{i}\mathrm{e}\mathrm{Z}$ luków $BC, CA$ okregów $0$ promie-

C niu $\alpha \mathrm{i}$ środkach odpowiednio $\mathrm{w}$ punktach

23.4. Podstawa pryzmy przedstawionej na rysunku jest prostokat

ABCD, którego bok $AB$ ma

D $\mathrm{C}$ gdzie $\alpha > b$. Wszystkie ściany

$b$ boczne pryzmy sa nachylone pod
\begin{center}
\includegraphics[width=60.552mm,height=30.480mm]{./KursMatematyki_PolitechnikaWroclawska_1999_2004_page30_images/image002.eps}
\end{center}
{\it K} $\mathrm{L}$

A $\alpha \mathrm{B}$

$\mathrm{d}$ ugośč $\alpha$, a bok $BC \mathrm{d}$ ugośč $b,$

katem $\alpha$ do $\mathrm{p}$ aszczyzny podstawy.

Obliczyč objetośč tej pryzmy.

23.5. Rozwiazač nierównośč

$- x2<\sqrt{}$5-{\it x}2.

Rozwiazanie zilustrowač wykresami funkcji wystepujacych po obu

stronach nierówności. Zaznaczyč na rysunku otrzymany zbiór rozwia-

zań.

23.6. Ciag $(\alpha_{n})$ jest określony rekurencyjnie warunkami $\alpha_{1} =$ 4,

$\alpha_{n+1} = 1+2\sqrt{\alpha_{n}}, n \geq 1$. Stosujac zasade indukcji matematycznej,

wykazač, $\dot{\mathrm{z}}\mathrm{e}$ ciag $(\alpha_{n})$ jest rosnacy oraz $4\leq\alpha_{n}<6$ dla $n\geq 1.$

23.7. Na krzywej $0$ równaniu $y=\sqrt{x}$ znalez$\acute{}$č punkt $\mathrm{l}\mathrm{e}\dot{\mathrm{z}}\mathrm{a}\mathrm{c}\mathrm{y}$ najblizej punktu

$P(0,3)$. Sporzadzič rysunek.

23.8. Wykazač, $\dot{\mathrm{z}}\mathrm{e}$ dla $\mathrm{k}\mathrm{a}\dot{\mathrm{z}}$ dej wartości parametru $\alpha \in \mathrm{R}$ równanie

kwadratowe $3x^{2}+4x\sin\alpha-\cos 2\alpha=0$ ma dwa rózne pierwiastki

rzeczywiste. Wyznaczyč te wartości parametru $\alpha$, dla których oba

pierwiastki $\mathrm{l}\mathrm{e}\dot{\mathrm{z}}$ a $\mathrm{w}$ przedziale $(0,1).$





41

Praca kontrolna nr 3

24.1. Suma wyrazów nieskończonego ciagu geometrycznego zmniejszy $\mathrm{s}\mathrm{i}\mathrm{e}$

$0$ 25\%, jeśli wykreślimy $\mathrm{z}$ niej skladniki $0$ numerach parzystych niepo-

dzielnych przez 4. Ob1iczyč sume wszystkich wyrazów tego ciagu,

wiedzac, $\dot{\mathrm{z}}\mathrm{e}$ jego drugi wyraz wynosi l.

24.2. $\mathrm{Z}$ kompletu 28 kości do gry $\mathrm{w}$ domino wylosowano dwie kości (bez

zwracania). Obliczyč prawdopodobieństwo tego, $\dot{\mathrm{z}}\mathrm{e}$ kości $pasujq_{f}$ do

siebie, $\mathrm{t}\mathrm{z}\mathrm{n}$. najednym $\mathrm{z}$ pól obu kości wystepuje ta sama liczba oczek.

24.3. Rozwiazač uklad równań

$\left\{\begin{array}{l}
x+2y=3\\
5x+my=m
\end{array}\right.$

$\mathrm{w}$ zalezności od parametru rzeczywistego $m$. Wyznaczyč $\mathrm{i}$ narysowač

zbiór, jaki tworza rozwiazania $(x(m),y(m))$ tego ukladu, gdy $m$

przebiega zbiór liczb rzeczywistych.

24.4. $\mathrm{W}$ graniastoslupie prawidlowym sześciokatnym krawed $\acute{\mathrm{z}}$ dolnej pod-

stawy $AB$ widač ze środka górnej podstawy $P$ pod katem $\alpha$. Wyz-

naczyč cosinus kata utworzonego przez plaszczyzne podstawy $\mathrm{i}$ plasz-

czyzne zawierajaca krawed $\acute{\mathrm{z}}$ AB oraz przeciwlegla do niej $\mathrm{k}\mathrm{r}\mathrm{a}\mathrm{w}\mathrm{e}\mathrm{d}\acute{\mathrm{z}}$

$D'E'$ górnej podstawy. Uzasadnič odpowiednio obliczenia.

24.5. Rozwiazač nierównośč $-1\displaystyle \leq\frac{2^{x+1/2}}{4^{x}-4}\leq 1.$

24.6. Bez $\mathrm{u}\dot{\mathrm{z}}$ ycia tablic wykazač, $\dot{\mathrm{z}}\mathrm{e}$ tg $82^{\circ}30'$ -tg $7^{\circ}30'=4+2\sqrt{3}.$

24.7. Napisač równanie prostej $k$ stycznej $\mathrm{w}$ punkcie $P(2,3)$ do okregu

$x^{2}+y^{2}-2x-2y-3=0$ Nastepnie wyznaczyč równania wszystkich

prostych stycznych do tego okregu, które tworza $\mathrm{z}$ prosta $k\mathrm{k}\mathrm{a}\mathrm{t}45^{\circ}$

24.8. Dobrač parametry $\alpha>0 \mathrm{i} b\in R \mathrm{t}\mathrm{a}\mathrm{k}$, aby funkcja

$f(x)=$

dla $x\leq\alpha,$

dla $x>\alpha,$

byla ciagla $\mathrm{i}$ miala pochodna $\mathrm{w}$ punkcie $x =\alpha$. Sporzadzič wykres

funkcji $f(x)$ oraz stycznej do wykresu $\mathrm{w}$ punkcie $P(\alpha,f(\alpha))$ bez

szczególowego badania jej przebiegu.





42

Praca kontrolna nr 4

25.1. Dla jakich wartości parametru rzeczywistego

$x+3=-(tx+1)^{2}$ ma dokladnie jeden pierwiastek.

t równanie

25.2. Czworościan foremny $0$ krawedzi $\alpha$ przecieto plaszczyzna równolegla

do dwóch przeciwleglych krawedzi. Wyrazič pole otrzymanego prze-

krojujako funkcje dlugości odcinka wyznaczonego przez ten przekrój

najednej $\mathrm{z}$ pozostalych krawedzi. Uzasadnič postepowanie. Przedsta-

wič znaleziona funkcje na wykresie $\mathrm{i}$ podač jej najwieksza wartośč.

25.3. Zaznaczyč na wykresie zbiór punktów $(x,y)$ plaszczyzny spelniaja-

cych warunek $\log_{xy}|y|\geq 1.$

25.4. Wyznaczyč równanie linii utworzonej przez wszystkie punkty plasz-

czyzny, których odleglośč od okregu $x^{2}+y^{2}=81$ jest $01$ mniejsza

$\mathrm{n}\mathrm{i}\dot{\mathrm{z}}$ od punktu $P(8,0)$. Sporzadzič rysunek.

25.5. Na dziesiatym pietrze pewnego bloku mieszkaja Kowalscy $\mathrm{i}$ Nowa-

kowie. Kowalscy maja dwóch synów $\mathrm{i}$ dwie córki, a Nowakowie jed-

nego syna $\mathrm{i}$ dwie córki. Postanowili oni wybrač mlodziezowego przed-

stawiciela swojego pietra. $\mathrm{W}$ tym celu Kowalscy wybrali losowo jedno

ze swoich dzieci $\mathrm{i}$ Nowakowie jedno ze swoich. Nastepnie spośród tej

dwójki wylosowano jedna osobe. Obliczyč prawdopodobieństwo, $\dot{\mathrm{z}}\mathrm{e}$

przedstawicielem zostal chlopiec.

25.6. Uzasadnič prawdziwośč nierówności $n+\displaystyle \frac{1}{2}\geq\sqrt{n(n+1)}, n\geq 1.$ {\it Ko}-

rzystajac $\mathrm{z}$ niej oraz $\mathrm{z}$ zasady indukcji matematycznej, udowodnič,

$\dot{\mathrm{z}}\mathrm{e}$

$\displaystyle \left(\begin{array}{l}
2n\\
n
\end{array}\right)\geq\frac{4^{n}}{2\sqrt{n}}$

dla $\mathrm{k}\mathrm{a}\dot{\mathrm{z}}$ dej liczby naturalnej $n.$

25.7. Zbadač przebieg zmienności $\mathrm{i}$ narysowač wykres funkcji

$f(x)=\sqrt{\frac{3x-3}{5-x}}.$

25.8. $\mathrm{W}$ trójkacie $ABC \mathrm{k}\mathrm{a}\mathrm{t} A$ ma miare $\alpha, \mathrm{k}\mathrm{a}\mathrm{t} B$ miare $2\alpha,$

$\mathrm{a}|BC|=\alpha$. Oznaczmy kolejno przez $A_{1}$ punkt na boku $AC$ taki, $\dot{\mathrm{z}}\mathrm{e}$

BAl jest dwusieczna kata $B;B_{1}$ punkt na boku $BC$ taki, $\dot{\mathrm{z}}\mathrm{e}$ AlBl jest

dwusieczna kata $A_{1}, \mathrm{i}\mathrm{t}\mathrm{d}$. Wyznaczyč dlugośč nieskończonej lamanej

$ABA_{1}B_{1}A_{2}$





43

Praca kontrolna nr 5

26.1. Jakiej dlugości powinien byč pas transmisyjny, aby $\mathrm{m}\mathrm{o}\dot{\mathrm{z}}$ na go bylo

$\mathrm{u}\dot{\mathrm{z}}$ yč do polaczenia dwóch kól $0$ promieniach 20 cm $\mathrm{i}5$ cm, jeśli od-

leglośč środków tych kól wynosi 30 cm?

26.2. Umowa określa wynagrodzenie na kwote 4000 $\mathrm{z}l$. Skladka na ubez-

pieczenie spoleczne wynosi 18,7\% tej kwoty, a sk1adka na Kase Cho-

rych 7,75\% kwoty pozosta1ej po od1iczeniu sk1adki na ubezpiecze-

nie spoleczne. $\mathrm{W}$ celu obliczenia podatku nalezy od 80\% wyjściowej

kwoty umowy odjač skladke na ubezpieczenie spoleczne $\mathrm{i}$ wyznaczyč

19\% pozostalej sumy. Podatek jest róznica tak otrzymanej liczby

$\mathrm{i}$ skladki na Kase Chorych. Ile wynosi podatek?.

26.3. Przez punkt $P(1,3)$ poprowadzič prosta $l$, tak aby odcinek tej prostej

zawarty miedzy prostymi $x-y+3=0\mathrm{i}x+2y-12=0$ dzielil $\mathrm{s}\mathrm{i}\mathrm{e}$

$\mathrm{w}$ punkcie $P$ na polowy. Wyznaczyč równanie ogólne prostej $l\mathrm{i}$ obli-

czyč pole trójkata, jaki prosta $l$ tworzy $\mathrm{z}$ danymi prostymi.

26.4. Podstawa czworościanu ABCD jest trójkat prostokatny $ABC\mathrm{o}$ kacie

ostrym $\alpha \mathrm{i}$ promieniu okregu wpisanego $r$. Spodek wysokości opusz-

czonej $\mathrm{z}$ wierzcholka $D \mathrm{l}\mathrm{e}\dot{\mathrm{z}}\mathrm{y}\mathrm{w}$ punkcie przeciecia $\mathrm{s}\mathrm{i}\mathrm{e}$ dwusiecznych

trójkata $ABC$, a ściany boczne wychodzace $\mathrm{z}$ wierzcholka kata pros-

tego podstawy tworza $\mathrm{k}\mathrm{a}\mathrm{t}\beta$. Obliczyč objetośč tego ostroslupa.

26.5. Sporzadzič wykres funkcji $f(x)=\log_{4}(2|x|-4)^{2}$ Odczytač $\mathrm{z}$ wykre-

su wszystkie ekstrema lokalne tej funkcji.

26.6. Rozwiazač równanie $\displaystyle \cos 2x+\frac{\mathrm{t}\mathrm{g}x}{\sqrt{3}+\mathrm{t}\mathrm{g}x}=0.$

26.7. Dla jakich wartości parametru $\alpha \in \mathrm{R} \mathrm{m}\mathrm{o}\dot{\mathrm{z}}$ na określič funkcje

$g(x) = f(f(x))$, gdzie $f(x) = \displaystyle \frac{x^{2}}{\alpha x-1}$. Napisač wzór funkcji $g(x).$

Wyznaczyč asymptoty funkcji $g(x)$ dla najwiekszej $\mathrm{m}\mathrm{o}\dot{\mathrm{z}}$ liwej calko-

witej wartości parametru $\alpha.$

26.8. Odcinek $0$ końcach $A(0,3), B(2,y), y \in [0$, 3$]$, obraca $\mathrm{s}\mathrm{i}\mathrm{e}$ wokól

osi $Ox$. Wyznaczyč pole powierzchni bocznej powstalej bryly jako

funkcje $y \mathrm{i}$ znalez$\acute{}$č najmniejsza wartośč tego pola. Sporzadzič ry-

sunek.





44

Praca kontrolna nr 6

27.1. Znalez$\acute{}$č wszystkie wartości parametru rzeczywistego $p$, dla których

równanie $\sqrt{x+8p}=\sqrt{x}+2p$ ma rozwiazanie.

27.2. Obrazem okregu $K\mathrm{w}$ jednokladności $0$ środku $S(0,1)\mathrm{i}$ skali $k=-3$

jest okrag $K_{1}$, natomiast obrazem $K_{1} \mathrm{w}$ symetrii wzgledem prostej

$0$ równaniu $2x+y+3 = 0$ jest okrag $0$ tym samym środku co

okrag $K$. Wyznaczyč równanie okregu $K$, jeśli wiadomo, $\dot{\mathrm{z}}\mathrm{e}$ okregi

$K\mathrm{i}K_{1}$ sa styczne zewnetrznie.

27.3. $\mathrm{W}$ trapezie równoramiennym dane sa promień okregu opisanego $r,$

$\mathrm{k}\mathrm{a}\mathrm{t}$ ostry przy podstawie $\alpha$ oraz suma dlugości obu podstaw $d$. Obli-

czyč dlugośč ramienia tego trapezu. Zbadač warunki rozwiazalności

zadania. Sporzadzič rysunek dla $\alpha=60^{\circ}, d=\displaystyle \frac{5}{2}r.$

27.4. $\mathrm{W}$ ostroslupie prawidlowym czworokatnym $\mathrm{k}\mathrm{a}\mathrm{t}$ plaski ściany bocznej

przy wierzcholku wynosi $ 2\beta$. Przez wierzcholek $A$ podstawy oraz

środek przeciwleglej krawedzi bocznej poprowadzono plaszczyzne

równolegla do przekatnej podstawy wyznaczajaca przekrój plaski

ostroslupa. Obliczyč objetośč ostroslupa, wiedzac, $\dot{\mathrm{z}}\mathrm{e}$ pole przekroju

wynosi $S.$

27.5. Obliczyč granice

$\displaystyle \lim_{n\rightarrow\infty}\frac{n-\sqrt[3]{n^{3}+n^{\alpha}}}{\sqrt[5]{n^{3}}},$

jeśli $\alpha$ jest najmniejszym

$2\cos\alpha=-\sqrt{3}.$

dodatnim pierwiastkiem

równania

27.6. Rozwiazač nierównośč

$2^{1+2\log_{2}\cos x}-\displaystyle \frac{3}{4}\geq 9^{0}$' $5+\log_{3}\sin x$

27.7. Wylosowano, ze zwracaniem, 4 liczby czterocyfrowe (cyfra tysiecy

nie $\mathrm{m}\mathrm{o}\dot{\mathrm{z}}\mathrm{e}$ byč zerem!). Obliczyč prawdopodobieństwo tego, $\dot{\mathrm{z}}\mathrm{e}$ co

najmniej dwie $\mathrm{z}$ tych liczb czytane od strony lewej do prawej lub od

strony prawej do lewej beda podzielne przez 4.

27.8. Zaznaczyč na rysunku zbiór punktów $(x,y)$ plaszczyzny określony

warunkami $|x-3y| < 2$ oraz $y^{3} \leq x$. Obliczyč tangens kata, pod

którym przecinaja $\mathrm{s}\mathrm{i}\mathrm{e}$ linie tworzace brzeg tego zbioru.





45

Praca kontrolna nr 7

28.1. Dwa punkty poruszaja $\mathrm{s}\mathrm{i}\mathrm{e}$ ruchem jednostajnym po okregu $\mathrm{w}$ tym

samym kierunku, przy czym jeden $\mathrm{z}$ nich wyprzedza drugi co 44

sekundy. $\mathrm{J}\mathrm{e}\dot{\mathrm{z}}$ eli zmienič kierunek ruchu jednego $\mathrm{z}$ tych punktów na

przeciwny, to beda $\mathrm{s}\mathrm{i}\mathrm{e}$ one spotykač co 8 sekund. Ob1iczyč stosunek

predkości tych punktów.

28.2. Dlajakich wartości parametru $p$ nierównośč

$\displaystyle \frac{2px^{2}+2px+1}{x^{2}+x+2-p^{2}}\geq 2$

jest spelniona dla $\mathrm{k}\mathrm{a}\dot{\mathrm{z}}$ dej liczby rzeczywistej $x$?

28.3. $\mathrm{W}$ równolegloboku dane sa $\mathrm{k}\mathrm{a}\mathrm{t}$ ostry $\alpha$, dluzsza przekatna $d$ oraz

róznica boków $r$. Obliczyč pole równolegloboku.

28.4. Naczynie $\mathrm{w}$ ksztalcie pólkuli $0$ promieniu $R$ ma trzy nózki $\mathrm{w}$ ksztalcie

kulek $0$ promieniu $r, 4r < R$, przymocowanych do naczynia $\mathrm{w}$ ten

sposób, $\dot{\mathrm{z}}\mathrm{e}$ ich środki tworza trójkat równoboczny, a naczynie posta-

wione na plaskiej powierzchni dotyka $\mathrm{j}\mathrm{a}$ wjednym punkcie. Obliczyč

wzajemna odleglośč punktów przymocowania kulek. Sporzadzič od-

powiednie rysunki.

28.5. Za pomoca metod rachunku rózniczkowego określič liczbe rozwiazań

równania $2x^{3}+1=6|x|-6x^{2}$

28.6. Bez stosowania zasady indukcji matematycznej wykazač, $\dot{\mathrm{z}}\mathrm{e} \displaystyle \frac{n^{n}-1}{n-1}$

jest nieparzysta liczba naturalna dla wszystkich $n\geq 2.$

28.7. Rozwiazač równanie

$\displaystyle \frac{8}{3}(\sin^{2}x+\sin^{4}x+\ldots)=4-2\cos x+3\cos^{2}x-\frac{9}{2}\cos^{3}x+$

28.8. Rozwazmy rodzine prostych normalnych do paraboli $0$ równaniu

$2y = x^{2}$ (tj. prostopadlych do stycznych $\mathrm{w}$ punktach styczności).

Znalez$\acute{}$č równanie krzywej utworzonej ze środków odcinków tych nor-

malnych zawartych miedzy osia rzednych $\mathrm{i}$ wyznaczajacymi je punk-

tami paraboli. Sporzadzič rysunek.





Edycja

XXXIII

2003/2004





49

Praca kontrolna nr l

29.1. Podstawa trójkata równoramiennego jest odcinek AB $0$ końcach

$A(-1,3), B(1,-1)$, a wierzcholek $C$ tego trójkata $\mathrm{l}\mathrm{e}\dot{\mathrm{z}}\mathrm{y}$ na prostej

$l\mathrm{o}$ równaniu $3x-y-14=0$. Obliczyč pole trójkata $ABC.$

29.2. Pewna liczba sześciocyfrowa zaczyna $\mathrm{s}\mathrm{i}\mathrm{e}$ ($\mathrm{z}$ lewej strony) cyfra 3. Jeś1i

cyfre $\mathrm{t}\mathrm{e}$ przestawimy $\mathrm{z}$ pierwszej pozycji na ostatnia, to otrzymamy

liczbe stanowiaca 25\% 1iczby pierwotnej. Zna1ez$\acute{}$č $\mathrm{t}\mathrm{e}$ liczbe.

29.3. $\mathrm{W}$ trapezie opisanym na okregu katy ostre przy podstawie maja

miary $\alpha \mathrm{i}2\alpha$, a dlugośč krótszego ramienia wynosi $c$. Obliczyč dlugośč

krótszej podstawy tego trapezu. Wynik przedstawič $\mathrm{w}$ najprostszej

postaci.

29.4. Rozwiazač nierównośč

$\displaystyle \frac{1}{x^{2}-x-2}\leq\frac{1}{|x|}.$

29.5. Zaznaczyč na plaszczy $\acute{\mathrm{z}}\mathrm{n}\mathrm{i}\mathrm{e}$ zbiór wszystkich punktów $(x,y)$ spelnia-

jacych nierównośč $\log_{x}(1+(y-1)^{3})\leq 1.$

29.6. Rozwiazač równanie $\sin^{2}3x-\sin^{2}2x=\sin^{2}x.$

29.7. Wysokośč ostroslupa prawidlowego czworokatnegojest trzy razy dluz-

sza od promienia kuli wpisanej $\mathrm{w}$ ten ostroslup. Obliczyč cosinus kata

miedzy sasiednimi ścianami bocznymi tego ostroslupa.

29.8. Dany jest nieskończony ciag geometryczny

$x+1,-x^{2}(x+1),x^{4}(x+1),$

Wyznaczyč najmniejsza $\mathrm{i}$ najwieksza wartośč funkcji $S(x)$ bedacej

suma wszystkich wyrazów tego ciagu.





50

Praca kontrolna nr 2

30.1. Trójkat prostokatny, obracajac $\mathrm{s}\mathrm{i}\mathrm{e}$ wokól jednej $\mathrm{i}$ drugiej przypros-

tokatnej tworzy bryly $0$ objetościach odpowiednio $V_{1}\mathrm{i}V_{2}$. Obliczyč

objetośč bryly powstalej $\mathrm{z}$ obrotu tego trójkata wokól dwusiecznej

kata prostego.

30.2. Czy $\mathrm{m}\mathrm{o}\dot{\mathrm{z}}$ na sume $42000$ zlotych podzielič na pewna liczbe nagród,

tak aby kwoty tych nagród wyrazaly $\mathrm{s}\mathrm{i}\mathrm{e}\mathrm{w}$ pelnych setkach zlotych,

tworzyly ciag arytmetyczny oraz $\dot{\mathrm{z}}$ eby najwyzsza nagroda wynosila

13000 $\mathrm{z}l$? Jeśli $\mathrm{t}\mathrm{a}\mathrm{k}$, to podač liczbe $\mathrm{i}$ wysokości tych nagród.

30.3 Dane sa okregi $0$ równaniach $(x-1)^{2} + (y-1)^{2} =$ l oraz

$(x-2)^{2}+(y-1)^{2}=16$. Wyznaczyč równania wszystkich okregów

stycznych równocześnie do obu danych okregów oraz do osi $Oy.$

Sporzadzič rysunek.

30.4. $\mathrm{W}$ równolegloboku $\mathrm{k}\mathrm{a}\mathrm{t}$ ostry miedzy przekatnymi ma miare $\beta$, a sto-

sunek dlugości dluzszej przekatnej do krótszej przekatnej wynosi $k.$

Obliczyč tangens kata ostrego tego równolegloboku.

30.5. Rozwiazač równanie $\sqrt{4x-3}-3=\sqrt{2x-10}.$

30.6. Dobrač liczby calkowite $\alpha, b, \mathrm{t}\mathrm{a}\mathrm{k}$ aby wielomian $6x^{3}-7x^{2}+1$ dzielil

siebez reszty przez trójmian kwadratowy $2x^{2}+\alpha x+b.$

30.7. Rozwiazač nierównośč $|2^{x}-3| \leq 2^{1-x}$ Sporzadzič wykresy funkcji

wystepujacych po obu stronach tej nierówności oraz zaznaczyč na

rysunku zbiór rozwiazań.

30.8. Wyznaczyč przedzialy monotoniczności funkcji

$f(x)=\displaystyle \sin^{2}x+\frac{\sqrt{3}}{2}x,$

$x\in[-\pi,\pi].$





51

Praca kontrolna nr 3

31.1. $\mathrm{Z}$ talii 24 kart do gry wy1osowano 7 kart. Jakie jest prawdopodobień-

stwo otrzymania dokladnie czterech kart wjednym $\mathrm{z}$ czterech kolorów,

$\mathrm{w}$ tym asa, króla $\mathrm{i}$ dame.

31.2. Pewien ostroslup podzielono na trzy cześci dwiema plaszczyznami

równoleglymi do jego podstawy. Pierwsza plaszczyzna $\mathrm{l}\mathrm{e}\dot{\mathrm{z}}\mathrm{y} \mathrm{w}$ od-

leglości $d_{1} = 2$ cm, a druga $\mathrm{w}$ odleglości $d_{2} = 3$ cm od podstawy.

Pola przekrojów ostroslupa tymi plaszczyznami równe sa odpowied-

nio $S_{1}=25\mathrm{c}\mathrm{m}^{2}$ oraz $S_{2}=16\mathrm{c}\mathrm{m}^{2}$ Obliczyč objetośč tego ostroslupa

oraz objetośč najmniejszej cześci.

31.3. Rozwiazač uklad równań

$\left\{\begin{array}{l}
x^{2}+y^{2}=24\\
\frac{2\log x+\log y^{2}}{\log(x+y)}=2.
\end{array}\right.$

31.4. $\mathrm{W}$ trójkacie równoramiennym $ABC$ odleglośč środka okregu wpisane-

go od wierzcholka $C$ wynosi $d$, a podstawe $AB$ widač ze środka okregu

wpisanego pod katem $\alpha$. Obliczyč pole tego trójkata.

31.5. Stosujac zasade indukcji matematycznej, udowodnič prawdziwośč dla

$n\geq 1$ wzoru

$\displaystyle \cos x+\cos 3x++\cos(2n-1)x=\frac{\sin 2nx}{2\sin x},\sin x\neq 0.$

31.6. Wyznaczyč granice ciagu $0$ wyrazie ogólnym

$\displaystyle \alpha_{n}=\frac{\sqrt[6]{4n}}{\sqrt{n}-\sqrt{n+\sqrt[3]{4n^{2}}}},$

$n\geq 1.$

31.7. Dany jest wierzcholek $A(6,1)$ kwadratu. Wyznaczyč pozostale wierz-

cholki tego kwadratu, gdy wierzcholki sasiadujace $\mathrm{z}A\mathrm{l}\mathrm{e}\dot{\mathrm{z}}$ a jeden na

prostej $l$ : $x-2y+1 = 0$, a drugi na prostej $k$ : $x+3y-4 = 0.$

Sporzadzič rysunek.

31.8. Zbadač przebieg zmienności $\mathrm{i}$ narysowač wykres funkcji

$f(x)=\displaystyle \frac{x+1}{\sqrt{x}}.$





Edycja

XXIX

1999/2000





52

Praca kontrolna nr 4

32.1. Statek $\mathrm{z}$ Wroclawia do Szczecina plynie 3 $\mathrm{d}\mathrm{n}\mathrm{i}$, a ze Szczecina do

Wroclawia 5 $\mathrm{d}\mathrm{n}\mathrm{i}$. Jak dlugo $\mathrm{z}$ Wroclawia do Szczecina plynie woda?

32.2. Dla jakich wartości rzeczywistych $x$ liczby $1 +$ log23, $\log_{x}36,$

$\displaystyle \frac{4}{3}$ log86 sa trzema ko1ejnymi wyrazami ciagu geometrycznego.

32.3. Wanna $0$ pojemności 2001 majaca kszta1t po1owy wa1ca (rozcietego

wzdluz osi) $\mathrm{l}\mathrm{e}\dot{\mathrm{z}}\mathrm{y}$ poziomo na ziemi $\mathrm{i}$ zawiera pewna ilośč wody. Do

wanny wlozono belke ($\mathrm{c}\mathrm{i}\dot{\mathrm{z}}\mathrm{s}\mathrm{z}$ od wody) $\mathrm{w}$ ksztalcie walca $0$ średnicy

cztery razy mniejszej $\mathrm{n}\mathrm{i}\dot{\mathrm{z}}$ średnica wanny $\mathrm{i}$ dlugości równej polowie

dlugości wanny. Okazalo $\mathrm{s}\mathrm{i}\mathrm{e}, \dot{\mathrm{z}}\mathrm{e}$ lustro wody styka $\mathrm{s}\mathrm{i}\mathrm{e}\mathrm{Z}$ powierzchnia

belki zanurzonej $\mathrm{w}$ wodzie. Podač, $\mathrm{z}$ dokladnościa do 0,11, i1e wody

znajduje $\mathrm{s}\mathrm{i}\mathrm{e}\mathrm{w}$ wannie?

32.4. Wyznaczyč wszystkie wartości parametru $m$, dla których obydwa

pierwiastki trójmianu kwadratowego $v(x) = x^{2}+mx-m^{2} \mathrm{l}\mathrm{e}\dot{\mathrm{z}}\mathrm{a}$

miedzy pierwiastkami trójmianu $w(x)=x^{2}-(m-1)x-m.$

32.5. Urna A zawiera trzy kule biale $\mathrm{i}$ dwie czarne, a urna $\mathrm{B}$ dwie kule biale

$\mathrm{i}$ trzy czarne. Wylosowano cztery razy jedna kule (ze zwracaniem)

$\mathrm{z}$ urny A oraz jedna kule $\mathrm{z}$ urny B. Obliczyč prawdopodobieństwo

tego, $\dot{\mathrm{z}}\mathrm{e}$ wśród pieciu wylosowanych kul sa co najmniej dwie kule

biale.

32.6. Rozwiazač równanie 2 $\sin 2x+2\cos 2x+\mathrm{t}\mathrm{g}x=3.$

32.7. Danajest funkcja $f(x)=x^{4}-2x^{2}$ Wyznaczyč wszystkie proste sty-

czne do wykresu $\mathrm{t}\mathrm{e}\mathrm{j}$ funkcji zawierajace punkt $P(1,-1)$. Ile punktów

wspólnych $\mathrm{z}$ wykresem $\mathrm{t}\mathrm{e}\mathrm{j}$ funkcji maja wyznaczone styczne? Rozwia-

zanie zilustrowač rysunkiem.

32.8. Podstawa ostroslupa ABCS jest trójkat równoramienny, którego $\mathrm{k}\mathrm{a}\mathrm{t}$

przy wierzcholku $C$ ma miare $\alpha$, a ramie $BC$ ma dlugośč $b$. Spodek

wysokości ostroslupa $\mathrm{l}\mathrm{e}\dot{\mathrm{z}}\mathrm{y}\mathrm{w}$ środku wysokości $CD$ podstawy, a $\mathrm{k}\mathrm{a}\mathrm{t}$

plaski ściany bocznej $ABS$ przy wierzcholku ma miare $\alpha$. Obliczyč

promień kuli opisanej $\mathrm{n}\mathrm{a}\mathrm{t}\mathrm{y}\mathrm{m}$ ostroslupie oraz cosinusy katów nachyle-

nia ścian bocznych do podstawy.





53

Praca kontrolna nr 5

33.1. Piaty wyraz rozwiniecia dwumianu $(\alpha+b)^{18}$, gdzie $\alpha, b > 0,$

jest $0$ 180\% wiekszy od wyrazu trzeciego. $\mathrm{O}$ ile procent wyraz ósmy

tego rozwiniecia jest mniejszy $\mathrm{b}\mathrm{a}\mathrm{d}\acute{\mathrm{z}}$ wiekszy od wyrazu czwartego?

33.2. Wyznaczyč równanie linii utworzonej przez wszystkie punkty

plaszczyzny, dla których stosunek kwadratu odleglości od prostej

$k:x-2y+3=0$ do kwadratu odleglości od prostej $l:3x+y+2=0$

wynosi 2. Sporzadzič rysunek.

33.3. Obwód trójkata $ABC$ wynosi 15 cm, a dwusieczna kata $A$ dzieli bok

przeciwlegly na odcinki dlugości 3 cm oraz 2 cm. Ob1iczyč po1e ko1a

wpisanego $\mathrm{w}$ ten trójkat.

33.4. Czastka startuje $\mathrm{z}$ poczatku ukladu wspólrzednych $\mathrm{i}$ porusza $\mathrm{s}\mathrm{i}\mathrm{e}$ ze

sta a predkościa po nieskończo-

nej amanej jak na rysunku, któ- 

rej $\mathrm{d}$ ugości kolejnych odcinków
\begin{center}
\includegraphics[width=44.352mm,height=37.488mm]{./KursMatematyki_PolitechnikaWroclawska_1999_2004_page41_images/image001.eps}
\end{center}
$\alpha_{2}$

$\alpha_{3}$

$\alpha_{1}$

$\alpha_{4}$

{\it O}

{\it P}

tworza ciag geometryczny maleja-

cy. Po pewnym czasie czastka za-

trzyma a $\mathrm{s}\mathrm{i}\mathrm{e} \mathrm{w}$ punkcie $P(10,3).$

Jaka droge przeby a czastka?

33.5. Stosujac zasade indukcji matematycznej, udowodnič, $\dot{\mathrm{z}}\mathrm{e}$ dla wszyst-

kich $n \geq 1$ wielomian $x^{3n+1}+x^{3n-1}+1$ dzieli $\mathrm{s}\mathrm{i}\mathrm{e}$ bez reszty przez

wielomian $x^{2}+x+1.$

33.6. Narysowač wykres funkcji $f(x) = \displaystyle \frac{|x-2|}{x-|x|+2}$ bez badania jej prze-

biegu. Podač równania asymptot $\mathrm{i}$ ekstrema lokalne tej funkcji.

33.7. Rozwiazač nierównośč

$|\cos x|^{1+\sqrt{2}\sin x+\sqrt{2}\cos x}\leq 1,$

$x\in[-\pi,\pi].$

33.8. $\mathrm{W}$ stozek wpisano graniastoslup trójkatny prawidlowy $0$ wszystkich

krawedziach tej samej dlugości, tak $\dot{\mathrm{z}}\mathrm{e}$ dolna podstawa $\mathrm{l}\mathrm{e}\dot{\mathrm{z}}\mathrm{y}$ na pod-

stawie stozka. Przy jakim kacie rozwarcia stozka stosunek objetości

graniastoslupa do objetości stozka jest najwiekszy?





54

Praca kontrolna nr 6

34.1. $\mathrm{W}$ kolo $0$ polu $\displaystyle \frac{5}{4}\pi$ wpisano trójkat prostokatny $0$ polu l.

obwód tego trójkata.

Obliczyč

34.2. Sprowadzič do najprostszej postaci wyrazenie

2(sin6 $\alpha+\cos^{6}\alpha$)$-7(\sin^{4}\alpha+\cos^{4}\alpha)+\cos 4\alpha.$

34.3. Wyznaczyč trójmian kwadratowy, którego wykresem jest parabola

styczna do prostej $y=x+2$, przechodzaca przez punkt $P(-2,-2)$

oraz symetryczna wzgledem prostej $x=1$. Sporzadzič rysunek.

34.4. $\mathrm{W}$ trapezie ABCD, $\mathrm{w}$ którym AB $\Vert CD$, dane sa $\vec{AC}= [4$, 7$]$ oraz

$\vec{BD}=[-6,2]$. Za pomoca rachunku wektorowego wyznaczyč wektory

$\vec{AB}\mathrm{i}\vec{CD}$, jeśli wiadomo, $\dot{\mathrm{z}}\mathrm{e} \vec{AD}\perp\vec{BD}.$

34.5. Jaś ma $\mathrm{w}$ portmonetce 3 monetyjednoz1otowe, 2 monety dwuz1otowe

$\mathrm{i}$ jedna pieciozlotowa. Kupujac zeszyt $\mathrm{w}$ cenie 4 $\mathrm{z}l$, wyciaga losowo

$\mathrm{z}$ portmonetki po jednej monecie tak dlugo, $\mathrm{a}\dot{\mathrm{z}}$ uzbiera $\mathrm{s}\mathrm{i}\mathrm{e}$ suma wys-

tarczajaca na kupno zeszytu. Obliczyč prawdopodobieństwo, $\dot{\mathrm{z}}\mathrm{e}$ Jaś

wyciagnie co najmniej trzy monety. Podač odpowiednie uzasadnienie

(nie jest nim $\mathrm{t}\mathrm{z}\mathrm{w}$. drzewko).

34.6. Narysowač na plaszczy $\acute{\mathrm{z}}\mathrm{n}\mathrm{i}\mathrm{e}$ zbiór punktów określony nastepujaco

$\mathcal{F}=\{(x,y):\sqrt{4x-x^{2}}\leq y\leq 4-\sqrt{1-2x+x^{2}}\}.$

Wjakiej odleglości od brzegu figury $\mathcal{F}$ znajduje $\mathrm{s}\mathrm{i}\mathrm{e}$ punkt $P(\displaystyle \frac{3}{2},\frac{5}{2})$ ?

34.7. Danajest funkcja $f(x)=\log_{2}(1-x^{2})-\log_{2}(x^{2}-x)$. Bez stosowania

metod rachunku rózniczkowego wykazač, $\dot{\mathrm{z}}\mathrm{e}f$ jest rosnaca $\mathrm{w}$ swojej

dziedzinie oraz, $\dot{\mathrm{z}}\mathrm{e}g(x) = f(x-\displaystyle \frac{1}{2})$ jest nieparzysta. Wyznaczyč

funkcje odwrotna $f^{-1}$, jej dziedzine $\mathrm{i}$ zbiór wartości.

34.8. Pole powierzchni bocznej ostroslupa prawidlowego czworokatnego wy-

nosi $c^{2}$, a $\mathrm{k}\mathrm{a}\mathrm{t}$ nachylenia ściany bocznej do podstawy ma miare $\alpha.$

Ostroslup przecieto na dwie cześci plaszczyzna przechodzaca przez

jeden $\mathrm{z}$ wierzcholków podstawy $\mathrm{i}$ prostopadla do przeciwleglej krawe-

dzi bocznej. Obliczyč objetośč cześci zawierajacej wierzcholek ostro-

slupa. Podač warunek rozwiazalności zadania.





55

Praca kontrolna nr 7

35.1. Dwa pierwsze wyrazy nieskończonego ciagu geometrycznego sa pier-

wiastkami równania $4x^{2}-4px-3p^{2}=0$, gdzie $p$ jest nieznana liczba.

Wyznaczyč ten ciag, jeśli suma wszystkich jego wyrazów wynosi 3.

35.2. Wiedzac, $\dot{\mathrm{z}}\mathrm{e} \cos\varphi=\sqrt{\frac{2}{3}}$ oraz $\varphi\in (\displaystyle \frac{3}{2}\pi,2\pi)$, obliczyč cosinus kata

pomiedzy prostymi $y= (\displaystyle \sin\frac{\varphi}{2})x, y= (\displaystyle \cos\frac{\varphi}{2})x.$

35.3. Kostka sześcienna ma krawed $\acute{\mathrm{z}}  2\alpha$. Aby zmieścič $\mathrm{j}\mathrm{a}\mathrm{w}$ pojemniku

$\mathrm{w}$ ksztalcie kuli $0$ średnicy $ 3\alpha$, ze wszystkich narozników odcieto

$\mathrm{w}$ minimalny sposób jednakowe ostroslupy prawidlowe trójkatne.

Obliczyč dlugośč krawedzi bocznych odcietych czworościanów?

35.4. Udowodnič prawdziwośč nierówności

$1+\displaystyle \frac{x}{2}\geq\sqrt{1+x}\geq 1+\frac{x}{2}-\frac{x^{2}}{2}$ dla $x\in[-1,1].$

Zilustrowač $\mathrm{j}\mathrm{a}$ na odpowiednim wykresie.

35.5. Rozwiazač równanie

--csions25{\it xx}$=$-sin3{\it x}.

35.6. Dany jest okrag $\mathcal{K}$ : $x^{2}-4x+y^{2}+6y = 0$. Znalez$\acute{}$č równanie

okregu symetrycznego do $\mathcal{K}$ wzgledem stycznej do $\mathcal{K}$ poprowadzonej

$\mathrm{z}$ punktu $P(3,5)\mathrm{i}$ majacej dodatni wspólczynnik kierunkowy.

35.7. W okrag 0 promieniu r wpisano trapez 0 przekatnej d,

i najwiekszym obwodzie. Obliczyč pole tego trapezu.

$ d\geq r\sqrt{3},$

35.8. Metoda analityczna określič dla jakich wartości parametru $m$ uklad

równań

$\left\{\begin{array}{l}
mx-y+2=0\\
x-2|y|+2=0
\end{array}\right.$

ma dokladnie jedno rozwiazanie. Wyznaczyč to rozwiazanie $\mathrm{w}$ zalez-

ności od $m$. Sporzadzič rysunek.





Indeks

tematyczny





59

l. Liczby rzeczywiste

$\bullet$ Obliczenia procentowe: 1.1, 9.1, 16.1, 26.2, 33.1.

$\bullet$ Zasada indukcji matematycznej:

23.6, 25.6, 31.5, 33.5.

2.1, 3.3, 10.1, 11.5, 17.5, 19.6,

$\bullet$ Inne: 8.2, 28.6, 33.1.

2. Funkcja liniowa

$\bullet$ Uklady równań liniowych $\mathrm{z}$ parametrem: 7.5, 12.7, 18.4, 24.3, 35.8.

$\bullet$ Wartośč bezwzgledna: 2.5, 5.1, 13.7, 27.8, 35.8.

$\bullet$ Inne: 15.1, 28.1, 29.2, 32.1.

3. Funkcja kwadratowa

$\bullet$ Równania kwadratowe $\mathrm{z}$ parametrem: 2.4, 4.7, 5.7, 9.3, 11.7, 13.3,

16.7, 19.5, 20.1, 23.8, 25.1, 28.2, 32.4.

$\bullet$ Uklady równań drugiego stopnia: 2.6, 10.7, 21.1, 31.3.

$\bullet$ Inne: 2.2, 3.1 (10.3), 8.3, 21.8, 22.1, 34.3, 35.1, 35.4.

4. Wielomiany

$\bullet$ 2.1, 12.2, 17.3, 21.1, 23.2, 28.5, 30.6, 33.5.





60

5. Funkcje wymierne i niewymierne

$\bullet$ Wykresy funkcji: 12.1, 13.7, 15.5, 19.3, 20.5, 26.7, 33.6.

$\bullet$ Równania $\mathrm{i}$ nierówności: 1.6, 5.4, 9.5, 11.8, 15.2, 23.5, 27.1, 29.4,

30.5.

6.

Funkcje potegowe, wykIadnicze

i logarytmiczne

$\bullet$ Wykresy funkcji: 1.5, 13.3, 26.5.

$\bullet$ Przeksztalcanie wyrazeń: 14.6, 27.5, 31.6, 32.2.

$\bullet$ Równania: 1.2, 6.1, 9.3, 16.5, 21.6, 31.3.

$\bullet$ Nierówności: 2.3, 4.5, 7.1, 8.4, 10.5, 11.7, 17.6, 20.6, 22.5, 24.5,

25.3, 27.6, 29.5, 30.7, 33.7.

7. Funkcje trygonometryczne

$\bullet \mathrm{T}\mathrm{o}\dot{\mathrm{z}}$ samości, przeksztalcanie wyrazeń: 4.3, 5.2, 6.3, 13.1, 14.4, 16.6,

17.1, 24.6, 31.5, 34.2, 35.2.

$\bullet$ Równania: 1.4, 2.8, 8.7, 10.8, 15.6, 19.7, 22.7, 23.8, 26.6, 27.5,

28.7, 29.6, 32.6, 35.5.

$\bullet$ Nierówności: 3.8, 7.6, 10.8, 11.4, 18.5, 21.5, 23.8, 27.6, 33.7.





61

8. WIasności funkcji

$\bullet$ 3.1 (10.3), 15.5, 16.6, 26.7, 34.7.

9. Ciagi liczbowe

$\bullet$ Ciag arytmetyczny: 4.1, 13.5, 19.2, 22.2, 30.2.

$\bullet$ Ciag geometryczny: 6.4, 15.4, 17.1, 18.1, 32.2.

$\bullet$ Ciag geometryczny nieskończony:

24.1, 25.8, 28.7, 29.8, 33.4, 35.1.

1.2, 2.8, 8.1, 11.4, 14.2, 15.8,

$\bullet$ Granica ciagu: 14.6, 18.1, 21.4, 27.5, 31.6.

$\bullet$ Wlasności ciagu: 21.4, 23.6.

10.

Granica i ciagIośč funkcji.

Pochodna, styczna

$\bullet$ 1.8, 4.3, 9.8, 16.8, 20.7, 21.8, 22.8, 23.7, 24.8, 27.8, 32.7.

ll. Zastosowania pochodnej

$\bullet$ Badanie przebiegu funkcji: 3.6, 4.7, 9.7, 13.8, 15.8, 25.7, 31.8.

$\bullet$ Ekstrema lokalne: 8.3, 11.3, 26.5, 28.5, 33.6.

$\bullet$ Wartośč najmniejsza $\mathrm{i}$ najwieksza funkcji $\mathrm{w}$ zbiorze: 2.2, 3.1 (10.3),

3.5, 6.7, 10.6, 12.8 (26.8), 18.8, 19.5, 21.7, 22.8, 25.2, 29.8, 33.8, 35.7.

$\bullet$ Inne: 5.7, 24.8, 30.8.





62

12. Geometria analityczna

$\bullet$ Rachunek wektorowy, iloczyn skalarny: 1.3, 5.8 (15.7), 8.6, 13.2,

14.3, 19.1, 21.2, 22.6, 23.7, 29.1, 31.7, 34.4.

$\bullet$ Prosta: 3.2, 14.3, 18.2, 26.3, 35.2.

$\bullet$ Okrag: 2.7, 6.2, 10.6, 12.4, 16.2, 24.7, 27.2, 30.3, 35.6.

$\bullet$ Zbiory punktów $0$ danej wlasności (miejsca geometryczne punktów):

4.6, 5.3, 9.2, 11.6, 20.4, 25.4, 28.8, 33.2.

$\bullet$ Geometryczna interpretacja ukladów równań $\mathrm{i}$ nierówności: 2.6, 4.5,

5.1, 10.7, 13.7, 19.3, 23.5, 25.3, 27.8, 29.5, 34.6.

$\bullet$ Inne: 7.2, 14.5, 17.4, 21.8, 34.3.

13. Planimetria

$\bullet$ Trójkat: 3.5, 6.8, 8.8, 9.6, 11.1, 13.6, 20.8, 21.3, 22.4, 23.3, 25.8,

31.4, 33.3, 34.1.

$\bullet$ Trapez: 3.7, 7.4, 12.6, 17.7, 19.4, 27.3, 29.3, 35.7.

$\bullet$ Inne figury: 4.4, 14.2, 16.4, 18.7, 23.3, 26.1, 28.3, 30.4.





63

14. Stereometria

$\bullet$ Graniastoslupy: 6.4, 11.2, 19.2, 21.1, 24.4, 35.3.

$\bullet$ Ostroslupy: 1.7, 3.4, 4.8, 7.7 (31.2), 8.5, 9.4, 12.5, 14.8, 16.3,

18.8, 19.8, 25.2, 26.4, 27.4, 29.7, 32.8, 34.8.

$\bullet$ Bryly obrotowe: 5.6, 6.6, 10.2 (18.3), 12.8 (26.8), 20.2, 21.7, 22.3,

28.4, 30.1, 32.3, 33.8.

$\bullet$ Inne bryly: 15.3, 17.8, 23.4.

15.

Rachunek prawdopodobieństwa

i kombinatoryka

$\bullet$ Prawdopodobieństwo klasyczne:

24.2, 31.1.

4.2, 7.8 (14.7), 10.4, 17.2, 20.3,

$\bullet$ Wzór na prawdopodobieństwo calkowite: 6.5, 12.3, 25.5, 34.5.

$\bullet$ Niezaleznośč $\mathrm{i}$ schemat Bernoulliego: 5.5, 13.4, 18.6, 27.7, 32.5.

$\bullet$ Kombinatoryka: 7.3, 14.1, 23.1.





9

Praca kontrolna nr l

l.l. Stop sklada $\mathrm{s}\mathrm{i}\mathrm{e} \mathrm{z}$ 40\% srebra próby 0,6, 30\% srebra próby 0,7 oraz

l kg srebra próby 0,8. Jaka jest masa $\mathrm{i}$ jaka jest próba tego stopu?

1.2. Rozwiazač równanie

$3^{x}+1+3^{-x}+=4,$

którego lewa strona jest suma nieskończonego ciagu geometrycznego.

1.3. $\mathrm{W}$ trójkacie $ABC$ znane sa wierzcholki $A(0,0)$ oraz $B(4,-1)$. Wiado-

mo, $\dot{\mathrm{z}}\mathrm{e}\mathrm{w}$ punkcie $H(3,2)$ przecinaja $\mathrm{s}\mathrm{i}\mathrm{e}$ proste zawierajace wysokości

tego trójkata. Wyznaczyč wspólrzedne wierzcholka $C$. Sporzadzič

rysunek.

1.4. Rozwiazač równanie

$\cos 4x=\sin 3x.$

1.5. Narysowač staranny wykres funkcji

$f(x)=|\log_{2}(x-2)^{2}|.$

1.6. Rozwiazač nierównośč

$\displaystyle \frac{1}{x^{2}}\geq\frac{1}{x+6}.$

1.7. $\mathrm{W}$ ostroslupie prawidlowym sześciokatnym krawed $\acute{\mathrm{z}}$ podstawy ma dlu-

gośč $p$, a krawed $\acute{\mathrm{z}}$ boczna dlugośč $2p$. Obliczyč cosinus kata dwuścien-

nego miedzy sasiednimi ścianami bocznymi tego ostroslupa.

1.8. Wyznaczyč równania wszystkich prostych stycznych do wykresu funkcji

$f(x)=\displaystyle \frac{2x+10}{x+4},$

które sa równolegle do prostej stycznej do wykresu funkcji

$g(x)=\sqrt{1-x}\mathrm{w}$ punkcie $x=0$. Rozwiazanie zilustrowač rysunkiem.





Odpowiedzi

do

zadań





67

l.l. Masa stopu 2,3 kg, próba stopu 0,690.

1.2. 1og32.

1.3. {\it C}(--3101'--1101).

1.4. $\displaystyle \frac{\pi}{14}+k\frac{2\pi}{7},  k\in$ Z.

1.5. Rysunek l.
\begin{center}
\includegraphics[width=108.000mm,height=84.228mm]{./KursMatematyki_PolitechnikaWroclawska_1999_2004_page51_images/image001.eps}
\end{center}
{\it y}  í í

4

2

$-2$  0 1  3 4  6 {\it x}

Rys. l

1.6. $(- 00,-6)\cup[-2,0)\cup(0$, 3$].$

1.7. $-\displaystyle \frac{3}{5}.$

1.8. $\mathrm{s}_{\mathrm{a}}$ dwie takie styczne $\mathrm{i}$ maja równania $y = -\displaystyle \frac{1}{2}x +2$ oraz

$y=-\displaystyle \frac{1}{2}x-2.$

2.2. Podstawa prostokata $\alpha = \displaystyle \frac{2}{5}\sqrt{10}$ cm, wysokośč $b = \displaystyle \frac{4}{5}\sqrt{10}$ cm,

przekatna $p=2\sqrt{2}$ cm.





68

2.3. $(1,\sqrt[3]{9})\cup(\sqrt[3]{9},\infty).$

2.4. $4-2\sqrt{2}<p<4+2\sqrt{2}.$

2.5. Dla $m<0$ brak rozwiazań,

dla $m=0$ lub $m>1$ sa dwa rozwiazania,

dla $m=1$ sa trzy rozwiazania,

dla $0<m<1$ sa cztery rozwiazania.

2.6. Uklad ma trzy rozwiazania:

$\left\{\begin{array}{l}
x_{1}=-7\\
y_{1}=-1,
\end{array}\right.$

2.7. $S=\displaystyle \frac{1225}{12}.$

$\left\{\begin{array}{l}
x_{2}=1\\
y_{2}=7,
\end{array}\right.$

$\left\{\begin{array}{l}
x_{3}=5\\
y_{3}=-5.
\end{array}\right.$

2.8. -$\pi$8'--78$\pi$, --98$\pi$, --158$\pi$.

3.1. Dziedzina jest przedzial $[0$, 4$]$, a zbiorem wartości przedzial $[0,\displaystyle \frac{3}{2}]$.

3.2. Prosta $AB$ ma równanie $y=3$, a prosta $AD$ równanie $4x-3y=15$.

3.4. $\displaystyle \frac{8}{3}\sqrt{3}\mathrm{c}\mathrm{m}^{3}$

3.5. Trójkat równoboczny $0$ boku $R\sqrt{3}\mathrm{i}$ polu $\displaystyle \frac{3\sqrt{3}}{4}R^{2}$

3.6. $D = (-\displaystyle \infty,\frac{5}{2}]$; miejsca zerowe $0, \displaystyle \frac{5}{2}$; minimum lokalne 0

dla $x = 0$; maksimum lokalne 2 d1a $x = 2$; funkcja rosnaca $\mathrm{w} (0,2)$ ;

malejaca $\mathrm{w} (-\infty,0)$ oraz $\mathrm{w} (2,\displaystyle \frac{5}{2})$ ; wypukla $\mathrm{w} (-\displaystyle \infty,2-\frac{\sqrt{6}}{3})$ ;

wklesla $\mathrm{w} (2-\displaystyle \frac{\sqrt{6}}{3},\frac{5}{2})$ ; punkt przegiecia $P(2-\displaystyle \frac{\sqrt{6}}{3},\sqrt{\frac{62\sqrt{6}-117}{27}})$ ;

prosta $x= \displaystyle \frac{5}{2}$ jest styczna do wykresu funkcji $\mathrm{w}$ punkcie $(\displaystyle \frac{5}{2},0)$. Wykres

funkcji przedstawiono na rysunku 2.





69
\begin{center}
\includegraphics[width=66.036mm,height=56.544mm]{./KursMatematyki_PolitechnikaWroclawska_1999_2004_page53_images/image001.eps}
\end{center}
{\it y}

2

1  {\it P}

$-1$  0 1  2  3 {\it x}

Rys. 2

3.7.

$S = \displaystyle \frac{d^{2}-2dr\cos\alpha+r^{2}\cos 2\alpha}{2(d-r\cos\alpha)}r\sin\alpha$; $R = \displaystyle \frac{d^{2}-2dr\cos\alpha+r^{2}}{4(d-r\cos\alpha)\sin\alpha}$;

rozwiazanie istnieje, gdy $d\geq r(1+2\cos\alpha)$. Wynik liczbowy $S=\displaystyle \frac{13}{12}\sqrt{3}\mathrm{c}\mathrm{m}^{2},$

$R=\displaystyle \frac{7}{3}$ cm.

3.8.[0,-1$\pi$2]$\cup$[-71$\pi$2'-1132$\pi$]$\cup$[-1192$\pi$,-2152$\pi$]$\cup$[-3112$\pi$,3$\pi$].

4.1.109.

4.2.-97.

lub $\{$

4.3. Pochodna nie istnieje.

4.5. $\{x\leq 11<y\leq 3-x$

$0<y<1$

$1\leq x\leq 3-y.$

4.6. Elipsa $0$ równaniu $\displaystyle \frac{(x+4)^{2}}{36}+\frac{y^{2}}{20}=1$, środku $M(-4,0)\mathrm{i}$ pólosiach

$\alpha=6, b=2\sqrt{5}.$





70

4.7. $y = f(m) = \displaystyle \frac{4}{m}-\frac{4}{m^{2}}. D = ($-00, $0)\cup(0,5]$; miejsce zerowe l;

asymptota pionowa obustronna $m = 0$; asymptota pozioma lewostronna

$y=0$; maksimum lokalne l dla $m=2$; funkcja rosnaca $\mathrm{w} (0,2)$ ; malejaca

$\mathrm{w} (-\infty,0)$ oraz $\mathrm{w} (2,5)$ ; wypukla $\mathrm{w} (3,5)$ ; wklesla $\mathrm{w} (-\infty,0)$ oraz

$\mathrm{w}(0,3)$ ; punkt przegiecia $P(3,\displaystyle \frac{8}{9})$. Wykres przedstawiono na rysunku 3.
\begin{center}
\includegraphics[width=135.588mm,height=70.608mm]{./KursMatematyki_PolitechnikaWroclawska_1999_2004_page54_images/image001.eps}
\end{center}
$y$

1

{\it P}

$-4  -2$  0 2 3  {\it 5 m}

$-1$

$-3$

Rys. 3

4.8. $S=\displaystyle \frac{\pi}{16}\alpha^{2_{\cos^{2}\alpha(3-4\cos^{2}\alpha)}}27-32\cos^{2}\alpha, \alpha\in \displaystyle \frac{\pi}{6}, \displaystyle \frac{\pi}{2}$

5.1. Zbiór $A$ przedstawiono na rysunku 4.

najblizej punktu P.

Punkt $Q \displaystyle \frac{3}{2}, \displaystyle \frac{5}{2}$

$\mathrm{l}\mathrm{e}\dot{\mathrm{z}}\mathrm{y}$
\begin{center}
\includegraphics[width=60.192mm,height=72.084mm]{./KursMatematyki_PolitechnikaWroclawska_1999_2004_page54_images/image002.eps}
\end{center}
{\it y}

{\it P}

{\it Q}

2

$-1$  2  {\it x}

Rys. 4





71

5.2. $\displaystyle \frac{7}{16}\sqrt{5}$ lub - $\displaystyle \frac{7}{16}\sqrt{5}.$

5.3. Szukana krzywa stanowia dwie galezie paraboli $y= \displaystyle \frac{1}{2}x^{2}-1$ dla

$x\geq 2$ oraz dla $x\leq 2.$

5.4. 11.

5.5. Pierwszy.

5.6. $2r+4\sqrt{2Rr-R^{2}}.$

5.7. Dla $ m\in [2\sqrt{3},\infty$).

5.8. $\displaystyle \frac{9}{85}\sqrt{85}.$

6.1. $\displaystyle \frac{1}{4}(-3+3\sqrt{3}).$

6.2. $\displaystyle \frac{13}{3}.$

6.4. $8+(1+\sqrt{33})^{3/2}$

6.5. $\displaystyle \frac{3}{10}.$

6.6. $\displaystyle \frac{\pi}{12}d^{3}\mathrm{t}\mathrm{g}^{2}\alpha(8\cos^{4}\alpha-1).$

6.7. Wartośč najmniejsza 3l, a najwieksza $24\sqrt{2}$.

6.8. Stosunek wynosi $1+k$, a dziedzina $k$ jest przedzial $(0,\sqrt{2}-1$].

7.1. $(0,1).$

7.2.

Elipsa $0$ równaniu $\displaystyle \frac{(x+1)^{2}}{4} + \displaystyle \frac{(y-3)^{2}}{1} =$

l, środku $S(-1,3)$

$\mathrm{i}$ pólosiach $\alpha=2, b=1$. Pole figury wynosi $2\pi.$





72

7.3. 330.

7.4. $\displaystyle \frac{\pi}{4}(\sqrt{8}-\sqrt{6}).$

7.5. Dla $m$ róznych od 3 $\mathrm{i}4$ jedno rozwiazanie $x=\displaystyle \frac{9}{m-4},\ y=\displaystyle \frac{m+2}{m-4}$.

Dla $m = 4$ uklad sprzeczny. Dla $m = 3$ nieskończenie wiele rozwiazań

spelniajacych warunek $x-2y = 1$, gdzie $x$ dowolne rzeczywiste. $\mathrm{s}_{\mathrm{a}}$ dwa

rozwiazania spelniajace warunek $x=y$: dla $m=7 (x=y=3)$ oraz dla

$m=3 (x=y=-1).$

7.6. $(-\displaystyle \frac{\pi}{3},0)\cup(\frac{\pi}{3},\frac{\pi}{2}].$

7.7. Objetośč ostroslupa wynosi $\displaystyle \frac{343}{3}\mathrm{c}\mathrm{m}^{3}$, a objetośč najmniejszej cześci

$\displaystyle \frac{61}{3}\mathrm{c}\mathrm{m}^{3}$

7.8. a) $\displaystyle \frac{1}{20};\mathrm{b}) \displaystyle \frac{7}{20}.$

8.1. $q=\displaystyle \frac{1}{30}, \alpha_{1}=1972.$

8.2. $\mathrm{s}_{\mathrm{a}}$ dwa takie skladniki 26730 oraz 1320.

8.3. Wykres funkcji przedstawiono na rysunku 5.

Rys. 5





73

Maksima lokalne 4 d1a $x = 1 \mathrm{i}x = -1$; minima lokalne 0 d1a $x = 3$

$\mathrm{i} x = -3$ oraz 3 d1a $x = 0$. Funkcja rosnaca $\mathrm{w}$ przedzialach $(-3,-1),$

$(0,1), ($3, $\infty)$ ; malejaca $\mathrm{w}$ przedzialach $(-\infty,-3), (-1,0)$, (1, 3).

8.4.

(-23, 2].

8.5. $\displaystyle \frac{1}{48}\alpha^{3}\sqrt{\sqrt{52}-2}.$

8.6. -32, 1, -27, --139.

8.7. $\displaystyle \frac{\pi}{9}+k\frac{\pi}{3}$ lub $\displaystyle \frac{2\pi}{9}+k\frac{\pi}{3},  k\in$ Z.

8.8. $(\sqrt{S_{1}}+\sqrt{S_{2}}+\sqrt{S_{3}})^{2}$

9.1. $\mathrm{O}$ 72,8\%.

9.2. Prawa gala $\acute{\mathrm{z}}$ hiperboli $0$ równaniu $y=\displaystyle \frac{1}{2}+\frac{1}{2(x-1)}$,

$x>1.$

9.3. $m\in(1,2).$

9.4. $\sqrt{3}.$

9.5. $(- 00,-3)\cup[1,3)\cup(3$, 5$].$

9.6. $\displaystyle \frac{\sqrt{1+k^{2}}-1+k}{k^{2}\sqrt{2}}.$

9.7. $D=\mathrm{R}\backslash \{2\}$; asymptota pionowa obustronna $x= 2$; asymptota

pozioma obustronna $y=1$; minimum lokalne $\displaystyle \frac{1}{2}$ dla $x=-2$; funkcja rosnaca

$\mathrm{w} (-2,2)$ ; malejaca $\mathrm{w} (- 00,-2)$ oraz $\mathrm{w} (2,\infty)$ ; wypukla $\mathrm{w} (-4,2)$ oraz

$\mathrm{w}(2,\infty)$ ; wklesla $\mathrm{w}(-\infty,-4)$ ; punkt przegiecia $P(-4,\displaystyle \frac{5}{9})$. Wykres funkcji

przedstawiono na rysunku 6.





74
\begin{center}
\includegraphics[width=144.324mm,height=79.656mm]{./KursMatematyki_PolitechnikaWroclawska_1999_2004_page58_images/image001.eps}
\end{center}
{\it y}

5

1

$-4$

{\it P}

$-2$

2 4  6 {\it x}

Rys. 6

9.8. $y=10x-16, y=-\displaystyle \frac{5}{4}x-\frac{1}{4}, y=-\displaystyle \frac{38}{25}x+\frac{16}{125}.$

10.2. $V=-\displaystyle \frac{\pi}{6}l^{3}\sin 4\alpha\cos 2\alpha, \varphi=3\pi-4\alpha, \alpha\in (\displaystyle \frac{\pi}{2},\frac{3\pi}{4}).$

10.3. Dziedzinajest przedzial $[0$, 4$]$, a zbiorem wartości przedzial $[0,\displaystyle \frac{3}{2}].$

10.4. $\displaystyle \frac{240}{1771}\approx 0$, 136.

10.5. $(\displaystyle \frac{1}{4},\frac{1}{2})\cup[2$, 4$].$

10.6.

$r=\displaystyle \frac{1}{2}.$

$S(r) = r(1-r^{2})^{3/2}, r \in (0,1)$. Wartośč najwieksza $\displaystyle \frac{3\sqrt{3}}{16}$ dla

10.7. Uklad ma cztery rozwiazania:

$\left\{\begin{array}{l}
x_{1}=0\\
y_{1}=0,
\end{array}\right.$

$\left\{\begin{array}{l}
x_{2}=\frac{16}{5}\\
y_{2}=\frac{12}{5},
\end{array}\right.$

$\left\{\begin{array}{l}
x_{3}=-\frac{16}{5}\\
y_{3}=-\frac{12}{5},
\end{array}\right.$

$\left\{\begin{array}{l}
x_{4}=4\\
y_{4}=-2.
\end{array}\right.$





75

Pierwsze równanie przedstawia dwa okregi styczne do osi $Oy 0$ środkach

$S_{1}(\displaystyle \frac{5}{2},0), S_{2}(-\displaystyle \frac{5}{2},0) \mathrm{i}$ promieniu $\displaystyle \frac{5}{2}$. Drugie równanie przedstawia dwie

proste równolegle.

10.8. Rozwiazania równania: $\displaystyle \frac{3\pi}{8}, \displaystyle \frac{7\pi}{8}.$

$[0,\displaystyle \frac{3\pi}{8})\cup(\frac{7\pi}{8},\pi].$

11.1. $\displaystyle \frac{\pi}{6}.$

Zbiór rozwiazań nierówności:

11.2. $\displaystyle \frac{2\sqrt{6}}{3}R.$

11.3. 4, $-4.$

11.4. $(\displaystyle \frac{\pi}{4},\frac{3\pi}{4})\cup(\frac{5\pi}{4},\frac{7\pi}{4}).$

11.6. $|y|=\displaystyle \frac{(x+2)^{2}}{4}-1, x\in(-\infty,-4)\cup(0,\infty).$

11.7. $3^{\sqrt{11}}, 3^{-\sqrt{11}}.$

11.8. $[-5,0)\cup(5$, 6$].$

12.1. $\displaystyle \frac{9-\sqrt{5}}{2}.$

12.2. $-1.$

12.3. $\displaystyle \frac{107}{128}\approx 0$, 836.

12.4. $(x-1)^{2}+(y-1)^{2}=1, (x-6)^{2}+(y-6)^{2}=36, (x+2)^{2}+(y-2)^{2}=4,$

$(x-3)^{2}+(y+3)^{2}=9.$

12.5. -32{\it d}3--$\sqrt{}$tgtg2$\alpha$3$\alpha$-1' $\alpha\in$ (-$\pi$4'-$\pi$2).

12.6. $s+\displaystyle \frac{16Pr}{\sqrt{16P^{2}+s^{4}}}$. Warunek rozwiazalności $r\displaystyle \geq\frac{\sqrt{16P^{2}+s^{4}}}{4s}.$





10

Praca kontrolna nr 2

2.1. Udowodnič, $\dot{\mathrm{z}}\mathrm{e}$ dla $\mathrm{k}\mathrm{a}\dot{\mathrm{z}}$ dego $n$ naturalnego wielomian $x^{4n-2}+1$ jest

podzielny przez trójmian kwadratowy $x^{2}+1.$

2.2. $\mathrm{W}$ równoramienny trójkat prostokatny $0$ polu $S = 10\mathrm{c}\mathrm{m}^{2}$ wpisano

prostokat $\mathrm{w}$ taki sposób, aby jeden $\mathrm{z}$ jego boków $\mathrm{l}\mathrm{e}\dot{\mathrm{z}}\mathrm{a}l$ na przeci-

wprostokatnej trójkata, a pozostale dwa wierzcholki znalazly $\mathrm{s}\mathrm{i}\mathrm{e}$ na

przyprostokatnych $\mathrm{i}$ równocześnie $\mathrm{t}\mathrm{a}\mathrm{k}$, aby mial on najkrótsza prze-

katna. Obliczyč dlugośč przekatnej tego prostokata.

2.3. Rozwiazač nierównośč

log1253 $\log_{x}5+\log_{9}8\log_{4}x>1.$

2.4. Znalez$\acute{}$č wszystkie wartości parametru $p$, dla których wykres funkcji

$y=x^{2}+4x+3\mathrm{l}\mathrm{e}\dot{\mathrm{z}}\mathrm{y}$ nad prosta $y=px+1.$

2.5. Zbadač liczbe rozwiazań równania

$||x+5|-1|=m$

$\mathrm{w}$ zalezności od parametru $m.$

2.6. Rozwiazač uklad równań

$\left\{\begin{array}{l}
x^{2}+y^{2}=50\\
(x-2)(y+2)=-9.
\end{array}\right.$

Podač interpretacje geometryczna tego ukladu $\mathrm{i}$ sporzadzič odpowiedni

rysunek.

2.7. Wyznaczyč na osi odcietych punkty A $\mathrm{i} B,\ \mathrm{z}$ których okrag

$x^{2}+y^{2}-4x+2y= 20$ widač pod katem prostym, $\mathrm{t}\mathrm{z}\mathrm{n}$. styczne do

okregu wychodzace $\mathrm{z}\mathrm{k}\mathrm{a}\dot{\mathrm{z}}$ dego $\mathrm{z}$ tych punktów sa do siebie prostopadle.

Obliczyč pole figury ograniczonej stycznymi do okregu przechodzacymi

przez punkty A $\mathrm{i}B$. Rozwiazanie zilustrowač rysunkiem.

2.8. $\mathrm{W}$ przedziale $[0,2\pi]$ rozwiazač równanie

$1-\mathrm{t}\mathrm{g}^{2}x+\mathrm{t}\mathrm{g}^{4}x-\mathrm{t}\mathrm{g}^{6}x+\ldots=\sin^{2}3x.$





76

12.7. Dla parametrów $p$ róznych od 2 $\mathrm{i}-1$ jedno rozwiazanie $x=3p,$

$y=-3p-2$. Dla $p=2$ nieskończenie wiele rozwiazań spelniajacych waru-

nek $2x+y-4=0$, gdzie $x$ dowolne rzeczywiste. Dla $p=-1$ nieskończenie

wiele rozwiazań spelniajacych warunek $x-y+4 = 0$, gdzie $x$ dowolne

rzeczywiste. Rozwiazania $0$ wspólrzednych calkowitych:

$\left\{\begin{array}{l}
x=-2\\
y=2
\end{array}\right.$

, $p=-1$;

$\left\{\begin{array}{l}
x=-2\\
y=0
\end{array}\right.$

, {\it p}$=$ - -32;

$\left\{\begin{array}{l}
x=-1\\
y=-1
\end{array}\right.$

, {\it p}$=$ - -31;

$\left\{\begin{array}{l}
x=0\\
y=-2
\end{array}\right.$

, $p=0$;

$\left\{\begin{array}{l}
x=1\\
y=2
\end{array}\right.$

, $p=2$;

$\left\{\begin{array}{l}
x=2\\
y=0
\end{array}\right.$

, $p=2.$

12.8. $S(y) = \displaystyle \pi(y+\frac{3}{2})\sqrt{1+(y-\frac{3}{2})^{2}},$

mniejsza $\displaystyle \frac{3\sqrt{13}}{4}\pi$ dla $y=0.$

$y \in$

[0, -23].

Wartośč naj-

13.2. 3.

13.3. $f(m)=|2^{-(m-3)}-2|,$

przedstawiono na rysunku 7.

$ m\in$ (-00, 4$]$ \{log27\}. Wykres funkcji $f$
\begin{center}
\includegraphics[width=60.252mm,height=108.048mm]{./KursMatematyki_PolitechnikaWroclawska_1999_2004_page60_images/image001.eps}
\end{center}
$y$

6

$m_{0}=\log_{2}7$

4

2

0 1  2  $m_{0}$  4{\it m}

Rys. 7





77

13.4. $\displaystyle \frac{35}{144}\approx 0$, 243.

13.5.

20, 28.

10, 8, 6, 4 1ub

10, 8, 6, 4, 2, $0, -2$ lub $-20, -12, -4$, 4, 12,

13.6. $\displaystyle \frac{10}{27}.$

13.7. $Q(2+\displaystyle \frac{2}{5}\sqrt{5},\frac{4}{5}\sqrt{5})$

na rysunku 8.

Zbiory A, B oraz

$A\cap B$ przedstawiono

Rys. 8
\begin{center}
\includegraphics[width=161.640mm,height=77.520mm]{./KursMatematyki_PolitechnikaWroclawska_1999_2004_page61_images/image001.eps}
\end{center}
13.8. Funkcja jest parzysta. $D = [-\sqrt{8},\sqrt{8}]$; miejsca zerowe $-2\mathrm{i}2$;

maksima lokalne $\displaystyle \frac{1}{2}$ dla $x = -\sqrt{7}$ oraz dla $x = \sqrt{7}$; minimum lokalne

$y$

1

0,5

$-3  -2  -1$  1 2  3  {\it x}

$-1$

Rys. 9





78

$-4+\sqrt{8}$ dla $x=0$; funkcja rosnaca $\mathrm{w}(-\sqrt{8},-\sqrt{7})$ oraz $\mathrm{w}(0,\sqrt{7})$ ; malejaca

$\mathrm{w}(-\sqrt{7},0)$ oraz $\mathrm{w}(\sqrt{7},\sqrt{8})$ ; wypukla $\mathrm{w}(-2,2)$ ; wklesla $\mathrm{w}(-\sqrt{8},-2)$ oraz

$\mathrm{w}(2,\sqrt{8})$ ; punkty przegiecia $(-2,0), ($2, $0)$, proste $x=-\sqrt{8}$ oraz $x=\sqrt{8}$

styczne do wykresu funkcji. Wykres funkcji przedstawiono na rysunku 9.

14.1. 9.

14.2. $2\pi(3+2\sqrt{3}).$

14.3. a) $m=-\displaystyle \frac{1}{2}$; b$)m=\displaystyle \frac{4}{3}$; c$)m=01\mathrm{u}\mathrm{b}m=2\sqrt{3}.$

14.5. Elipsa $0$ równaniu $\displaystyle \frac{x^{2}}{36}+\frac{(y-1)^{2}}{4}=1$, środku $S(0,1)\mathrm{i}$ pólosiach

$\alpha=6, b=2$. Pole figury wynosi $8\pi-6\sqrt{3}.$

14.6. $+\infty.$

14.7. a) $\displaystyle \frac{1}{20};\mathrm{b}) \displaystyle \frac{7}{20}.$

14.8. $\displaystyle \frac{\sqrt{2}-\cos\alpha}{2\sin\alpha}\alpha,$

$\alpha\in (0,\displaystyle \frac{\pi}{2}).$

15.1. 12 $\mathrm{k}\mathrm{m}/\mathrm{h}, 15\mathrm{k}\mathrm{m}/\mathrm{h}, AB=27$ km.

15.2. (-00, $-\sqrt{3}]\cup(2,\infty).$

15.3. $108\sqrt{3}\mathrm{m}^{2}, \displaystyle \frac{405}{4}\sqrt{3}\mathrm{m}^{3}$

15.4. $w_{n}=1600+\displaystyle \frac{8000}{3}((\frac{203}{200})^{n-1}-1)$, pensja $\mathrm{w}$ kwietniu 2002 roku

wynosi 1806,09 $\mathrm{z}l$, średnia pensja $\mathrm{w}$ 2002 roku wynosi l827,96 $\mathrm{z}l.$

15.5. $f^{-1}(x) = \sqrt[3]{x},$

sunku 10.

$x \in$ R. Wykres funkcji $h$ przedstawiono na ry-

15.6. $\displaystyle \frac{\pi}{12}+k\frac{\pi}{3},$

$k \in$ Z.





79
\begin{center}
\includegraphics[width=130.044mm,height=59.484mm]{./KursMatematyki_PolitechnikaWroclawska_1999_2004_page63_images/image001.eps}
\end{center}
{\it y}

2

1

$-2  -1$  1 2  {\it x}

Rys. 10

15.7. $\displaystyle \frac{9}{85}\sqrt{85}.$

15.8. $f(x) = \displaystyle \frac{x^{2}-x}{x-2} = x+ 1 + \displaystyle \frac{2}{x-2}$; $D = (-\infty,0]\cup(2,\infty)$ ;

asymptota pionowa prawostronna $x = 2$; asymptota ukośna obustronna

$y = x+1$; minimum lokalne $3+2\sqrt{2}$ dla $x_{0} = 2+\sqrt{2}$; funkcja rosnaca

$\mathrm{w}$ (-00, 0) oraz $\mathrm{w}(2+\sqrt{2},\infty)$ ; malejaca $\mathrm{w}(2,2+\sqrt{2})$ ; wypukla $\mathrm{w}(2,\infty)$ ;

wklesla $\mathrm{w}(-\infty,0)$. Wykres funkcji przedstawiono na rysunku ll.
\begin{center}
\includegraphics[width=93.012mm,height=100.992mm]{./KursMatematyki_PolitechnikaWroclawska_1999_2004_page63_images/image002.eps}
\end{center}
{\it y}

6

4

{\it S}

2

$-2$  2 $x_{0}4$ 6  {\it x}

$-2$

Rys. ll





80

16.1. Cena mniejsza od poczatkowej 02,25\%.

16.2. Zbiór sklada sieZ luków czterech okregów oraz punktu (0,0) ijest

przedstawiony na rysunku 12.
\begin{center}
\includegraphics[width=91.740mm,height=77.220mm]{./KursMatematyki_PolitechnikaWroclawska_1999_2004_page64_images/image001.eps}
\end{center}
$K_{2}$  {\it y}  $K_{1}$

6

$S_{2}  S_{1}$

2

2 4  8  {\it x}

$S_{3}  S_{4}$

$K_{3}  K_{4}$

16.3. $18h^{2}\displaystyle \frac{\sin^{2}\alpha}{\sin 3\alpha},$

Rys. 12

$\alpha\in (0,\displaystyle \frac{\pi}{3}).$

16.4. 18 cm od wierzcholka kata rozwartego, $\alpha=38^{\circ}13'.$

16.5. $\sqrt{2},$

$\displaystyle \frac{\sqrt{2}}{2}.$

16.6. Dziedzina jest $\mathrm{R}$, a zbiorem wartości przedzial $[3-\sqrt{5},3+\sqrt{5}].$

16.7. $(-1,0]\displaystyle \cup\{\frac{\sqrt{17}-1}{2}\}.$

16.8. 2.

17.1. 1, $-1, \sqrt{\frac{\sqrt{17}-1}{8}}, -\sqrt{\frac{\sqrt{17}-1}{8}}$; wyraz czwarty 0 1ub $\displaystyle \frac{9-\sqrt{17}}{4}.$

17.2. $\displaystyle \frac{16}{35}\approx 0$, 457.





81

17.4. $x^{2}+(y-r^{2}-\displaystyle \frac{1}{4})^{2}=r^{2}$ Rozwiazanie istnieje dla $r>\displaystyle \frac{1}{2}.$

17.6. $[\displaystyle \frac{1}{3},\frac{\sqrt{6}}{6}).$

17.7. $\displaystyle \frac{3d(c^{2}+d^{2})}{2c^{2}}\sqrt{c^{2}-d^{2}}$

lub $\displaystyle \frac{3d(2c^{2}-d^{2})}{2c^{2}}\sqrt{c^{2}-d^{2}},$

$c>d.$

17.8. Gdy $\mathrm{w}$ równoleglościanie sa dwa wierzcholki trójścienne $0$ trzech

katach plaskich $\beta$, to objetośč wynosi $2\alpha^{3}\sqrt{\sin\frac{3}{2}\beta\sin^{3}\frac{1}{2}\beta}$. Gdy $\beta\in (\displaystyle \frac{\pi}{3},\frac{\pi}{2})$

$\mathrm{i}\mathrm{w}$ równoleglościanie sa dwa wierzcholki trójścienne $0$ trzech katach plaskich

$\pi-\beta$, to objetośč wynosi $2\alpha^{3}\sqrt{-\cos\frac{3}{2}\beta\cos^{3}\frac{1}{2}\beta},$

18.1. $\displaystyle \frac{3}{2}.$

18.2. $3x-2y+1=0.$

18.3. $V=-\displaystyle \frac{\pi}{6}l^{3}\sin 4\alpha\cos 2\alpha, \varphi=3\pi-4\alpha,$

$\alpha\in (\displaystyle \frac{\pi}{2},\frac{3\pi}{4}).$

18.4. Niech $x$ oznacza cene dlugopisu, a $y$ cene zeszytu. Dla $k\neq 2$ jest

$x=\displaystyle \frac{5k+2}{2k+2}, y=\displaystyle \frac{k}{k+2}$. Dla $k=2$ spelnionajest relacja $2x+4y=5$. Ceny

dlugopisu $\mathrm{i}$ zeszytu moga byč nastepujace:

$\left\{\begin{array}{l}
x=2,3\\
y=0,9;
\end{array}\right.$

$\left\{\begin{array}{l}
x=2,1\\
y=0,8;
\end{array}\right.$

$\left\{\begin{array}{l}
x=1,7\\
y=0,6;
\end{array}\right.$

$\left\{\begin{array}{l}
x=1,5\\
y=0,5;
\end{array}\right.$

$\left\{\begin{array}{l}
x=1,3\\
y=0,6;
\end{array}\right.$

$\left\{\begin{array}{l}
x=1,1\\
y=0,7;
\end{array}\right.$

$\left\{\begin{array}{l}
x=0,9\\
y=0,8.
\end{array}\right.$

18.5.

$[-\displaystyle \frac{\pi}{4}+k\pi,k\pi]\cup[\frac{\pi}{4}+k\pi,\frac{\pi}{2}+k\pi),$

$ k\in$ Z.

18.6. $\displaystyle \frac{496}{729}\approx 0$, 680; $0 \displaystyle \frac{496}{728\cdot 729}\approx 0$, 001.

18.7. $2-\displaystyle \frac{4}{3}\sqrt{2}.$





82

18.8. $6\sqrt{2}-4.$

19.1. $\displaystyle \frac{31}{8}.$

19.2. $12+24\sqrt{2}$ cm.

19.3. $s\leq 20.$

19.4. $\displaystyle \frac{3}{2}\sqrt{55}$ arów. Plan dzialki $\mathrm{w}$ skali 1:1000 przedstawia rysunek 13.
\begin{center}
\includegraphics[width=72.444mm,height=42.828mm]{./KursMatematyki_PolitechnikaWroclawska_1999_2004_page66_images/image001.eps}
\end{center}
20

40  30

60

Rys. 13

19.5. Wartośč najwieksza 6 dla $m=0.$

19.7.

$\left\{\begin{array}{l}
x1=-- 51\pi 2\\
y_{1}=\frac{\pi}{12},
\end{array}\right.$

$\left\{\begin{array}{l}
x_{2}=\frac{\pi}{12}\\
y2=-- 51\pi 2,
\end{array}\right.$

$\left\{\begin{array}{l}
x3=--- 71\pi 2\\
y_{3}=-\frac{11\pi}{12},
\end{array}\right.$

$\left\{\begin{array}{l}
x_{4}=-\frac{11\pi}{12}\\
y4=--- 71\pi 2^{\cdot}
\end{array}\right.$

19.8. 1, 1, $\displaystyle \frac{\sqrt{3}}{2}, \displaystyle \frac{2\sqrt{7}}{7}, \displaystyle \frac{\sqrt{42}}{7}, \displaystyle \frac{\sqrt{42}}{7}.$

20.1. $-1$, 1, 2.

20.2. $\displaystyle \frac{8}{5}(2-\sqrt{3}).$

20.3. $\displaystyle \frac{50}{81}\approx 0$, 617.





83

20.4. Cześč elipsy $0$ równaniu $\displaystyle \frac{x^{2}}{\frac{5}{3}}+\frac{(y-5)^{2}}{\frac{5}{2}}=1$ dla $y\leq 6.$

20.5. Asymptota pionowa obustronna $x=1$; asymptota pozioma lewo-

stronna $y=-1$; asymptota pozioma prawostronna $y=1$. Wykres funkcji

przedstawiono na rysunku 14.
\begin{center}
\includegraphics[width=120.348mm,height=78.840mm]{./KursMatematyki_PolitechnikaWroclawska_1999_2004_page67_images/image001.eps}
\end{center}
$y$

4

2

$-2$  0 2  4  {\it x}

Rys. 14

20.6. $(-\displaystyle \infty,-1]\cup[-\frac{1}{2},0)\cup(0,1].$

20.7. $-\sqrt{8}$ lub $\sqrt{8}.$

21.1. $\mathrm{O} 5\mathrm{c}\mathrm{m}^{2}$

21.2. $\displaystyle \frac{3}{4}.$

21.3. $\displaystyle \frac{4-\sqrt{2}}{6}, \displaystyle \frac{4+\sqrt{2}}{6}.$

21.4. Granica ciagu wynosi $\displaystyle \frac{1}{2}.$

21.5. $(-\pi+4k\pi,\pi+4k\pi),  k\in$ Z.

21.6. $-1.$





84

21.7. $r=\displaystyle \frac{2\sqrt{2}}{3}R, h=\displaystyle \frac{4}{3}R.$

21.8. $y=1-(4+2\sqrt{5})(x-2),$

$y=1-(4-2\sqrt{5})(x-2).$
\begin{center}
\includegraphics[width=116.892mm,height=81.276mm]{./KursMatematyki_PolitechnikaWroclawska_1999_2004_page68_images/image001.eps}
\end{center}
{\it y}

5

$y=p_{1}$

3

$y=p_{2}$

1

$-4  -x_{0}  -1$  1 2  $x_{0}$  4 {\it x}

Rys. 15

22.1. Wykres funkcji przedstawiono na rysunku l5, gdzie $x_{0}=1+\sqrt{5}.$

Niech $f(p)$ oznacza liczbe rozwiazań równania $4+2|x|-x^{2}=p$. Wtedy

$f(p)=$

dla

dla

dla

dla

$p>5,$

$p<4$ lub $p=5,$

$p=4,$

$4<p<5.$

22.2. 117 minut; 5475,6 $\mathrm{m}^{3}$

22.3. Średnice podstaw $6+2\sqrt{5}$ cm oraz $6-2\sqrt{5}$ cm; tworzaca 6 cm.

22.4. Gdy $\mathrm{k}\mathrm{a}\mathrm{t}\alpha$ jest ostry $\mathrm{i}\sin\alpha < \displaystyle \frac{4}{5}$, wówczas sa dwa rozwiazania:

$ P_{1}=\displaystyle \frac{8}{25}R^{2}(4\cos\alpha-3\sin\alpha)\sin\alpha$ oraz $P_{2}=\displaystyle \frac{8}{25}R^{2}(4\cos\alpha+3\sin\alpha)\sin\alpha.$





85

Jeśli $\displaystyle \sin\alpha\geq\frac{4}{5}$, to jest jedno rozwiazanie $P_{2}$, a jeśli $\alpha$ rozwarty $\displaystyle \mathrm{i}\sin\alpha<\frac{4}{5},$

to brak rozwiazań.

22.5. $(0,\displaystyle \frac{1}{4}]\cup[16,\infty).$

22.6. $\displaystyle \cos\alpha=\frac{\sqrt{7}}{14}$, obwód $\displaystyle \frac{1}{6}(9+\sqrt{12}+\sqrt{21})\alpha.$

22.7. $\displaystyle \frac{\pi}{4}+k\frac{2\pi}{3},  k\in$ Z.

22.8. $\sqrt{2}x+2y-3=0.$

23.1. Tak. $\mathrm{W}$ obu przypadkach liczba,,slów'' wynosi 210.

23.2. $-3, -1$, 1.

23.3. $\displaystyle \frac{3}{8}\alpha.$

23.4. $\displaystyle \frac{1}{12}b^{2}(3\alpha-b)\mathrm{t}\mathrm{g}\alpha.$

23.5. $[-\sqrt{5},0)\cup(1$, 2$).$

23.7. Punkt $Q(1,1).$

23.8. $(\displaystyle \frac{5\pi}{4}+2k\pi,\frac{3\pi}{2}+2k\pi)\cup(\frac{3\pi}{2}+2k\pi,\frac{7\pi}{4}+2k\pi),$

24.1. $2+\displaystyle \frac{3}{2}\sqrt{2}.$

$ k\in$ Z.

24.2. $\displaystyle \frac{7}{18}\approx 0$, 389.

24.3. Dla $ m\neq 10$ jedno rozwiazanie $x= \displaystyle \frac{m}{m-10}, y= \displaystyle \frac{m-15}{m-10}$. Dla

$m= 10$ uklad sprzeczny. Rozwiazania tworza prosta $x+2y-3=0$ bez

punktu $P(1,1).$

24.4. $\sqrt{\frac{6-6\cos\alpha}{5-4\cos\alpha}},$

$\alpha\in (0,\displaystyle \frac{\pi}{3}).$





11

Praca kontrolna

nr 3

3.1. Bez stosowania metod rachunku rózniczkowego wyznaczyč dziedzine

i zbiór wartości funkcji

$f(x)=\sqrt{2+\sqrt{x}-x}.$

3.2. Jednym $\mathrm{z}$ wierzcholków rombu $0$ polu 20 $\mathrm{c}\mathrm{m}^{2}$ jest punkt $A(6,3)$,

a jedna $\mathrm{z}$ przekatnych zawiera $\mathrm{s}\mathrm{i}\mathrm{e}\mathrm{w}$ prostej $0$ równaniu $2x+y=5.$

Wyznaczyč równania prostych, $\mathrm{w}$ których zawieraja $\mathrm{s}\mathrm{i}\mathrm{e}$ boki AB $\mathrm{i}AD.$

3.3. Stosujac zasade indukcji matematycznej, wykazač prawdziwośč wzoru

3 $(1^{5}+2^{5}+\displaystyle \ldots+n^{5})+(1^{3}+2^{3}+\ldots+n^{3})=\frac{n^{3}(n+1)^{3}}{2},$

$n\geq 1.$

3.4. Ostroslup prawidlowy trójkatny ma pole powierzchni calkowitej

$P=12\sqrt{3}\mathrm{c}\mathrm{m}^{2}$, a $\mathrm{k}\mathrm{a}\mathrm{t}$ nachylenia ściany bocznej do plaszczyzny pod-

stawy $\alpha=60^{\circ}$ Obliczyč objetośč tego ostroslupa.

3.5. Wśród trójkatów równoramiennych wpisanych $\mathrm{w}$ kolo $0$ promieniu $R$

znalez$\acute{}$č ten, który ma najwieksze pole.

3.6. Zbadač przebieg zmienności $\mathrm{i}$ narysowač wykres funkcji

$f(x)=\displaystyle \frac{1}{2}x^{2}\sqrt{5-2x}.$

3.7. $\mathrm{W}$ trapezie równoramiennym dane sa ramie $r, \mathrm{k}\mathrm{a}\mathrm{t}$ ostry przy pod-

stawie $\alpha$ oraz suma $d$ dlugości przekatnej $\mathrm{i}$ dluzszej podstawy. Wyz-

naczyč pole trapezu oraz promień okregu opisanego na tym trapezie.

Podač warunki istnienia rozwiazania. Nastepnie przeprowadzič obli-

czenia dla $\alpha=30^{\circ}, r=\sqrt{3}$ cm $\mathrm{i} d=6$ cm.

3.8. Rozwiazač nierównośč

$|\cos x+\sqrt{3}\sin x|\leq\sqrt{2},x\in[0,3\pi].$





86

24.5. (-00, $\displaystyle \frac{1}{2}]\cup[\frac{3}{2},\infty).$

24.7. Równanie prostej $k$: $x+2y-8 = 0$. Równania stycznych

tworzacych $\mathrm{z} k \mathrm{k}\mathrm{a}\mathrm{t} 45^{\circ}$: $x-3y+2+5\sqrt{2}= 0, x-3y+2-5\sqrt{2}= 0,$

$3x+y-4+5\sqrt{2}=0, 3x+y-4-5\sqrt{2}=0.$

24.8. $\alpha=3, b=32$. Styczna $y= -3x+13.$

stawiono na rysunku 16.

Wykres funkcji przed-
\begin{center}
\includegraphics[width=109.224mm,height=96.768mm]{./KursMatematyki_PolitechnikaWroclawska_1999_2004_page70_images/image001.eps}
\end{center}
{\it y}

6

4

2

0 1  3 5  7  {\it x}

Rys. 16

25.1. -$\displaystyle \frac{1}{6}, 0, \displaystyle \frac{1}{2}.$

25.2. $S(x)=x(\alpha-x), x\in(0,\alpha)$. Wartośč najwieksza $\displaystyle \frac{\alpha^{2}}{4}$ dla $x=\displaystyle \frac{\alpha}{2}.$

25.3. Rysunek l7.

25.4. Elipsa $0$ równaniu $\displaystyle \frac{(x-4)^{2}}{25}+\frac{y^{2}}{9}=1$ oraz cześč prostej $y=0$ dla

$x>9.$





87

Rys. 17

25.5. $\displaystyle \frac{5}{12}\approx 0$, 417.

25.7. $D = [1$, 5$)$ ; asymptota pionowa lewostronna $x = 5$; funkcja

rosnaca $\mathrm{w}(1,5)$ ; wypukla $\mathrm{w}(2,5)$ ; wklesla $\mathrm{w}(1,2)$ ; punkt przegiecia $P(2,1)$ ;

prosta $x=1$ styczna do wykresu funkcji. Wykres funkcji przedstawiono na

rysunku 18.
\begin{center}
\includegraphics[width=72.036mm,height=72.084mm]{./KursMatematyki_PolitechnikaWroclawska_1999_2004_page71_images/image001.eps}
\end{center}
{\it y}

3

1  {\it P}

0 1  2  {\it 5 x}

Rys. 18





88

25.8. $2\alpha\cos\alpha(1+2\cos\alpha), \alpha\in (0,\displaystyle \frac{\pi}{3}).$

26.1. 30 $(\pi+\sqrt{3})$ cm.

26.2. 213 $\mathrm{z}l\mathrm{i}85$ gr.

26.3. $4x-7y+17=0$; pole $\displaystyle \frac{10}{3}.$

26.4. $\displaystyle \frac{\sqrt{2}}{3}r^{3}\frac{(1+\sin\alpha)\cos\frac{\beta}{2}\mathrm{c}\mathrm{t}\mathrm{g}\frac{\alpha}{2}}{\cos\alpha\sqrt{-\cos\beta}},$

$\beta\in (\displaystyle \frac{\pi}{2},\pi).$

26.5. Maksimum lokalne 2 dla $x=0$. Wykres funkcji przedstawiono na

rysunku 19.
\begin{center}
\includegraphics[width=156.612mm,height=66.240mm]{./KursMatematyki_PolitechnikaWroclawska_1999_2004_page72_images/image001.eps}
\end{center}
{\it y}

3

1

$-6  -3  -1$  1 3 4  6 {\it x}

Rys. 19

26.6. $\displaystyle \frac{\pi}{3}+k\frac{\pi}{2},$

$ k\in$ Z.

26.7. Dla $\alpha\in[0$, 4). Wtedy

$g(x)=$

dla

dla

$\alpha=0,$

$0<\alpha<4.$

Dla $\alpha=3$ asymptota pionowa obustronna $x= \displaystyle \frac{1}{3}$, asymptota ukośna obu-

stronna $3x-27y+4=0.$





89

26.8. $S(y)=\pi(y+3)\sqrt{4+(y-3)^{2}},$

dla $y=0$ wynoszaca $3\pi\sqrt{13}.$

$ y\in [0$, 3$]$. Wartośč najmniejsza

27.1. $p\in[-2,2].$

27.2. $(x-\displaystyle \frac{8}{5})^{2}+(y-\frac{9}{5})^{2}=\frac{16}{5}.$

27.3. $\displaystyle \frac{\sqrt{16r^{2}\sin^{2}\alpha-d^{2}}}{2\sin\alpha},$

$4r\sin\alpha\cos\alpha<d<4r\sin\alpha.$
\begin{center}
\includegraphics[width=36.216mm,height=15.192mm]{./KursMatematyki_PolitechnikaWroclawska_1999_2004_page73_images/image001.eps}
\end{center}
$(17+\mathrm{c}\mathrm{t}\mathrm{g}^{2}\beta)^{3}$

27.4. $2S^{3}24 18(\mathrm{c}\mathrm{t}\mathrm{g}^{2}\beta-1)^{2}$

27.5. $-\infty.$

27.6. $(2k\displaystyle \pi,\frac{\pi}{6}+2k\pi],$

27.7. $\displaystyle \frac{425}{768}\approx 0$, 553.

$ k\in$ Z.

27.8. Tangens kata przeciecia linii wynosi $\displaystyle \frac{9}{37}$. Szukany zbiór pokazano

na rysunku 20.
\begin{center}
\includegraphics[width=151.380mm,height=57.300mm]{./KursMatematyki_PolitechnikaWroclawska_1999_2004_page73_images/image002.eps}
\end{center}
{\it y}

2

1

$-8  -1$  2  8  {\it x}

$-1$

$-2$

Rys. 20

28.1. $\displaystyle \frac{13}{9}.$





90

28.2. $ p\in [\displaystyle \frac{5}{4},\frac{\sqrt{7}}{2}).$

28.3. $\displaystyle \frac{d^{2}-r^{2}}{2}\mathrm{t}\mathrm{g}\frac{\alpha}{2}, r<d.$

28.4. $\displaystyle \frac{2R}{R+r}\sqrt{3Rr}.$

28.5. Trzy pierwiastki, $\mathrm{w}$ tym jeden ujemny $\mathrm{i}$ dwa dodatnie.

28.7. $\displaystyle \frac{2\pi}{3}+2k\pi$ lub $\displaystyle \frac{4\pi}{3}+2k\pi,  k\in$ Z.

28.8. Szukana krzywa jest parabola $0$ równaniu $y = 2x^{2}+ \displaystyle \frac{1}{2}$ bez

punktu $W(0,\displaystyle \frac{1}{2}).$

29.1. 15.

29.2. 307692.

29.3. $c(\cos\alpha-\cos 2\alpha), \alpha\in (0,\displaystyle \frac{\pi}{4}).$

29.4. (-00, $-\sqrt{2}]\cup(-1,0)\cup(0,2)\cup[1+\sqrt{3},\infty).$

29.5. Rysunek 2l.
\begin{center}
\includegraphics[width=91.032mm,height=66.648mm]{./KursMatematyki_PolitechnikaWroclawska_1999_2004_page74_images/image001.eps}
\end{center}
Rys. 21





91

29.6. $\displaystyle \frac{\pi}{6}+k\frac{\pi}{3}$

29.7. $-\displaystyle \frac{1}{4}.$

lub

$k\pi,$

$ k\in$ Z.

29.8. Wartośč najmniejsza 0 dla $x =$

$\displaystyle \frac{1+\sqrt{2}}{2}$ dla $x=-1+\sqrt{2}.$

30.1. $\displaystyle \frac{\sqrt{2}}{2}\frac{V_{1}^{2}}{V_{1}+V_{2}}$, gdzie $V_{1}\geq V_{2}.$

$-1$, a wartośč najwieksza

30.2. Tak, na dwa sposoby: 3800, 6l00, 8400, l0700 i l3000 zl

1000, 3400, 5800, 8200, 10600 i l3000zl.

lub

30.3. $\mathrm{s}_{\mathrm{a}}$ cztery takie okregi $\mathrm{i}$ maja równania:

$(x-\displaystyle \frac{3}{2})^{2} + (y-1-\sqrt{6})^{2} = \displaystyle \frac{9}{4}, (x-\displaystyle \frac{3}{2})^{2} + (y-1+\sqrt{6})^{2}$

-49,

$(x+1)^{2}+(y-1)^{2}=1, (x-3)^{2}+(y-1)^{2}=9.$

30.4. $\displaystyle \frac{2k}{k^{2}-1}\sin\beta, k>1.$

30.5. 7, 13.

30.6. $\alpha=-3, b=1.$

30.7. $(-\infty,0]\cup [1,\displaystyle \log_{2}\frac{3+\sqrt{17}}{2}].$

30.8. $(-\displaystyle \pi,-\frac{2\pi}{6}), (-\displaystyle \frac{\pi}{6},\frac{4\pi}{6}), (\displaystyle \frac{5\pi}{6},\pi).$

31.1. $\displaystyle \frac{136}{4807}\approx 0$, 028.

31.2. Objestośč ostroslupa wynosi $\displaystyle \frac{343}{3} \mathrm{c}\mathrm{m}^{3}$, a objetośč najmniejszej

cześci $\displaystyle \frac{61}{3}\mathrm{c}\mathrm{m}^{3}$

31.3. Uklad ma trzy rozwiazania:

$\left\{\begin{array}{l}
x_{1}=3+\sqrt{3}\\
y_{1}=3-\sqrt{3},
\end{array}\right.$

$\left\{\begin{array}{l}
x_{1}=3-\sqrt{3}\\
y_{1}=3+\sqrt{3},
\end{array}\right.$

$\left\{\begin{array}{l}
x_{1}=2+2\sqrt{2}\\
y_{1}=2-2\sqrt{2}.
\end{array}\right.$





92

31.4. - $\displaystyle \frac{1}{2}d^{2}\sin 2\alpha \mathrm{t}\mathrm{g}^{2}\frac{\alpha}{2}, \alpha\in (\displaystyle \frac{\pi}{2},\pi).$

31.6. $-\sqrt[3]{4}.$

31.7. $B_{1}(5,3), C_{1}(3,2), D_{1}(4,0)$ lub $B_{2}(10,-2), C_{2}(13,2), D_{2}(9,5).$

31.8. $D= (0,\infty)$ ; asymptota pionowa prawostronna $x=0$; minimum

lokalne 2 $\mathrm{d}\mathrm{l}\mathrm{a}x=1$; funkcja rosnaca $\mathrm{w}(1,\infty)$ ; malejaca $\mathrm{w} (0,1)$, wypukla

$\mathrm{w}(0,3)$ ; wklesla $\mathrm{w}(3,\infty)$ ; punkt przegiecia $P(3,\displaystyle \frac{4}{3}\sqrt{3})$ ; krzywa asympto-

tyczna $(\mathrm{w}+\infty) y=\sqrt{x}$. Wykres funkcji przedstawiono na rysunku 22.
\begin{center}
\includegraphics[width=116.436mm,height=54.864mm]{./KursMatematyki_PolitechnikaWroclawska_1999_2004_page76_images/image001.eps}
\end{center}
{\it y}

3

{\it P}

2

0 1  3  8  {\it x}

Rys. 22

32.1. 15 $\mathrm{d}\mathrm{n}\mathrm{i}.$

32.2. 8, $\displaystyle \frac{1}{8}.$

32.3. 65,71itra.

32.4. $ m\in (0,\displaystyle \frac{\sqrt{5}-1}{2}).$

32.5. $\displaystyle \frac{2757}{3125}\approx 0$, 882.





93

32.6. $\displaystyle \frac{\pi}{4}+k\pi$ lub $\displaystyle \frac{\pi}{12}+k\pi$

lub $\displaystyle \frac{5\pi}{12}+k\pi,  k\in$ Z.

32.7. $y=-1$ (dwa punkty wspólne), $32x+27y-5=0$ (trzy punkty

wspólne).

32.8. $R = \displaystyle \frac{1}{3}b \sqrt{\frac{9+3\cos^{2}\alpha}{2+2\cos\alpha}}$. Cosinusy katów nachylenia ścian

bocznych wynosza $\displaystyle \frac{1}{2}$

oraz $\sqrt{\frac{1-\cos\alpha}{7-\cos\alpha}}.$

33.1. Mniejszy 023,56\%.

33.2. Szukana linie stanowia dwie proste $0$ równaniach $2x+3y-1=0$

oraz $4x-y+5=0$ bez punktu ich przeciecia $P(-1,1).$

33.3. $\displaystyle \frac{7\pi}{4}.$

33.4. 2 $(7+\sqrt{19}).$
\begin{center}
\includegraphics[width=120.192mm,height=90.420mm]{./KursMatematyki_PolitechnikaWroclawska_1999_2004_page77_images/image001.eps}
\end{center}
{\it y}

4

2

$-4  -1$  0 2  4  {\it x}

Rys. 23





94

33.6. Asymptota pionowa obustronna $x=1$; asymptota pozioma lewo-

stronna $y= -\displaystyle \frac{1}{2}$; asymptota ukośna prawostronna $y= \displaystyle \frac{1}{2}x-1$; minimum

lokalne 0 d1a $x=2$. Wykres funkcji przedstawiono na rysunku 23.

33.7. $[-\displaystyle \frac{5\pi}{12},\frac{\pi}{2})\cup(\frac{\pi}{2},\frac{11\pi}{12}]\cup\{-\pi,\pi\}.$

33.8. Cosinus kata rozwarcia wynosi $\displaystyle \frac{11}{13}.$

34.1. $3+\sqrt{5}.$

34.2. $-4.$

34.3. - $\displaystyle \frac{1}{2}x^{2}+x+2$ lub - $\displaystyle \frac{1}{18}x^{2}+\frac{1}{9}x-\frac{14}{9}.$

34.4. $\vec{AB}=[8$, 4$], \vec{CD}=[-2,-1].$

34.5. $\displaystyle \frac{6}{10}.$

34.6. Odleglośč $P$ od brzegu $\mathcal{F}$ wynosi $\displaystyle \frac{\sqrt{26}}{2}-2$. Zbiór $\mathcal{F}$ przedstawiono

na rysunku 24.
\begin{center}
\includegraphics[width=66.036mm,height=60.096mm]{./KursMatematyki_PolitechnikaWroclawska_1999_2004_page78_images/image001.eps}
\end{center}
{\it y}

4

2

0 2  4  {\it x}

Rys. 24

34.7. $f^{-1}(y)=-\displaystyle \frac{1}{1+2^{y}}, D_{f^{-1}}=\mathrm{R}, W_{f^{-1}}=(-1,0).$





95

34.8. $\displaystyle \frac{1}{6}c^{3}\frac{1-3\cos^{2}\alpha}{(1+\cos^{2}\alpha)\sin\alpha}\sqrt{\cos\alpha}$. Zadanie ma sens, gdy $\displaystyle \cos\alpha<\frac{\sqrt{3}}{3}.$

35.1. 4, $-\displaystyle \frac{4}{3}, \displaystyle \frac{4}{9}, -\displaystyle \frac{4}{27},$

35.2. $\displaystyle \frac{2\sqrt{3}-1}{5}.$

35.3. $\displaystyle \frac{\alpha}{2}.$

35.5. $\displaystyle \frac{\pi}{4}+k\frac{\pi}{2}$

lub

$\displaystyle \frac{\pi}{6}+k\pi$

lub $\displaystyle \frac{5\pi}{6}+k\pi,  k\in$ Z.

35.6. $(x+4)^{2}+(y-1)^{2}=13.$

35.7. $\displaystyle \frac{2rd^{3}}{4r^{2}+d^{2}}.$

35.8. $ m\in [-\displaystyle \frac{1}{2},\frac{1}{2})\cup\{1\}.$





12

Praca kontrolna nr 4

4.1. Rozwiazač równanie $16+19+22+\ldots+x=2000$, którego lewa strona

jest suma pewnej liczby kolejnych wyrazów ciagu arytmetycznego.

4.2. Ze zbioru $\{0$, 1, $\ldots$, 9$\}$ losujemy bez zwracania pieč cyfr. Obliczyč

prawdopodobieństwo tego, $\dot{\mathrm{z}}\mathrm{e}\mathrm{m}\mathrm{o}\dot{\mathrm{z}}$ na $\mathrm{z}$ nich utworzyč liczbe podzielna

przez 5.

4.3. Zbadač, czy istnieje pochodna funkcji $f(x) = \sqrt{1-\cos x}\mathrm{w}$ punkcie

$x=0$. Wynik zilustrowač na wykresie funkcji $f(x).$

4.4. Udowodnič, $\dot{\mathrm{z}}\mathrm{e}$ dwusieczne katów wewnetrznych równolegloboku two-

$\mathrm{r}\mathrm{z}\mathrm{a}$ prostokat, którego przekatna ma dlugośč równa róznicy dlugości

sasiednich boków równolegloboku.

4.5. Rozwiazač uklad nierówności

$\left\{\begin{array}{l}
x+y\leq 3\\
\log_{y}(2^{x+1}+32)\leq 2\log_{y}(8-2^{x})
\end{array}\right.$

$\mathrm{i}$ zaznaczyč zbiór jego rozwiazań na plaszczy $\acute{\mathrm{z}}\mathrm{n}\mathrm{i}\mathrm{e}.$

4.6. Znalez$\acute{}$č równanie zbioru wszystkich punktów plaszczyzny $Oxy$, które

sa środkami okregów stycznych wewnetrznie do okregu $x^{2}+y^{2}=121$

$\mathrm{i}$ równocześnie stycznych zewnetrznie do okregu $(x+8)^{2}+y^{2}=1$. Jaka

linie przedstawia znalezione równanie? Sporzadzič staranny rysunek.

4.7. Zbadač iloczyn pierwiastków rzeczywistych równania

$m^{2}x^{2}+8mx+4m-4=0$

jako funkcje parametru $m$. Sporzadzič wykres tej funkcji.

4.8. Podstawa czworościanu ABCD jest trójkat równoboczny $ABC\mathrm{o}$ boku

$\alpha$, ściana boczna $BCD$ jest trójkatem równoramiennym prostopadlym

do plaszczyzny podstawy, a $\mathrm{k}\mathrm{a}\mathrm{t}$ plaski ściany bocznej przy wierzcholku

$A$ jest równy $\alpha$. Obliczyč pole powierzchni kuli opisanej na tym

czworościanie.





Wskazówki

do

zadań





99

Uwaga. Podano wskazówki do wszystkich zadań. Zaproponowano

pewna metodę rozwiazania $\mathrm{k}\mathrm{a}\dot{\mathrm{z}}$ dego $\mathrm{z}$ zadań, najczęściej nie jedyna

$\mathrm{i}\mathrm{z}$ pewnościa nie zawsze najprostsza.

l.l. Najpierw obliczyč oddzielnie mase stopu $\mathrm{i}$ mase srebra $\mathrm{w}$ stopie.

1.2. Pamietač $0$ wyznaczeniu dziedziny równania.

1.3. Oznaczyč nieznane wspólrzedne punktu $C$ przez $x \mathrm{i} y$, zapisač

wektory $\vec{AC}\mathrm{i}\vec{BC}$ za pomoca $x\mathrm{i}y\mathrm{i}$ korzystač $\mathrm{z}$ prostopadlości $\vec{AC}\perp\vec{BH}$

oraz $\vec{BC}\perp\vec{AH}. \mathrm{U}\dot{\mathrm{z}}$ yč iloczynu skalarnego.

1.4. Zamienič sinus na cosinus, stosujac odpowiedni wzór redukcyjny

$\mathrm{i}$ od razu przejśč do porównywania katów. Odpowied $\acute{\mathrm{z}}$ zapisač $\mathrm{w}$ postaci

jednej serii rozwiazań.

1.5. Pamietač, $\dot{\mathrm{z}}\mathrm{e} \log_{2}\alpha^{2} = 2\log_{2}|\alpha| \mathrm{i}$ skorzystač $\mathrm{z}$ symetrii wykresu

wzgledem prostej $x = 2$. Wykres otrzymač przez odbicia symetryczne

$\mathrm{i}$ translacje standardowej krzywej $y=\log_{2}x.$

1.6. Najpierw rozwazyč przypadek oczywisty $x < -6$. Dla $x > -6$

porównač odwrotności obu stron $\mathrm{i}$ przejśč do nierówności kwadratowej.

Pamietač $0$ dziedzinie nierówności.

1.7. Zastosowač twierdzenie cosinusów. Podczas wykonywania rysunku

pamietač, $\dot{\mathrm{z}}\mathrm{e}\mathrm{w}$ rzucie równoleglym zachowuje $\mathrm{s}\mathrm{i}\mathrm{e}$ równoleglośč oraz propor-

cje odcinków równoleglych.

1.8. Proste równolegle maja takie same wspólczynniki kierunkowe.

Wspólczynniki te wyznaczyč za pomoca pochodnych obu funkcji. Przy

kreśleniu wykresu krzywej $y=\sqrt{1-x}$ zwrócič uwage na lewostronne otocze-

nie punktu $x=1.$

2.1. Wystarczy pokazač, $\dot{\mathrm{z}}\mathrm{e}$ dla $\mathrm{k}\mathrm{a}\dot{\mathrm{z}}$ dego $n$ naturalnego wielomian

$y^{2n-1}+1$ jest podzielny przez dwumian $y+1.$

2.2. Kwadrat dlugości przekatnej wyrazič jako funkcje wysokości pros-

tokata wpisanego $\mathrm{w}$ trójkat. Jest to funkcja kwadratowa $\mathrm{i}$ do jej badania

nie jest potrzebna pochodna.





100

2.3. Przypadek $0 < x < 1$ jest oczywisty. Dla $x >$

logarytmy do wspólnej podstawy 3 $\mathrm{i}$ przyjač $\log_{3}x=t.$

l sprowadzič

2.4. Warunek geometryczny zapisač $\mathrm{w}$ jezyku nierówności kwadratowej

$\mathrm{z}$ parametrem.

2.5. Podstawič $x+5 =t\mathrm{i}$ badač równanie $||t|-1| =m$. Przypadki

$m<0\mathrm{i}m=0$ rozpatrzeč bezpośrednio, a dla $m>0$ korzystač $\mathrm{z}\mathrm{t}\mathrm{o}\dot{\mathrm{z}}$ samości

$(|\alpha|=b)\Leftrightarrow$($\alpha=b$ lub $\alpha=-b$) prawdziwej dla $b\geq 0.$

2.6. Pomnozyč drugie równanie przez 2 $\mathrm{i}$ nastepnie odjač oba równania

stronami. Podstawienie $x-y = t$ prowadzi do równania kwadratowego

$\mathrm{z}$ niewiadoma $t.$

2.7. Uzasadnič, $\dot{\mathrm{z}}\mathrm{e}$ szukane punkty $A\mathrm{i}B\mathrm{l}\mathrm{e}\dot{\mathrm{z}}$ a na osi $Ox\mathrm{w}$ odleglości

$5\sqrt{2}$ od środka danego okregu. Przy obliczaniu pola figury (która jest

deltoid), najprościej jest korzystač $\mathrm{z}$ podobieństwa odpowiednich trzech

trójkatów prostokatnych.

2.8. Dziedzine równania określaja warunek istnienia tangensa $\mathrm{i}$ warunek

istnienia sumy nieskończonego ciagu geometrycznego. Korzystajac ze wzoru

$1+\mathrm{t}\mathrm{g}^{2}\gamma= \displaystyle \frac{1}{\cos^{2}\gamma}$ oraz ze wzorów podanych we wskazówkach do $\mathrm{z}\mathrm{a}\mathrm{d}$. 3.8

$\mathrm{i}4.3$, przeksztalcič obie strony do równości dwóch cosinusów lub sinusów

$\mathrm{i}$ przejśč od razu do porównywania katów.

3.1. Podstawič$\sqrt{}$x$=t\mathrm{i}$ korzystač $\mathrm{z}$ wlasności funkcji kwdratowej oraz

$\mathrm{z}$ monotoniczności pierwiastka kwadratowego.

3.2. Wyznaczyč środek $S$ rombu korzystajac $\mathrm{z}$ relacji $S\in l$ oraz $\vec{AS}\perp l$

$\mathrm{t}\mathrm{z}\mathrm{n}. \vec{AS}= \alpha$ñ, gdzie ñ $= [2$, 1$]$ jest wektorem prostopadlym do prostej $l.$

$\mathrm{Z}$ warunku $\vec{AS}\perp\vec{SB}$ wynika, $\dot{\mathrm{z}}\mathrm{e}\vec{SB}= - \vec{SD}=c[1,-2]$. Dane pole rombu

pozwala wyznaczyč skalar $c \mathrm{i}$ stad od razu otrzymujemy wspól-

rzedne wierzcholków $B\mathrm{i}D.$

3.3. $\mathrm{W}$ dowodzie kroku indukcyjnego przeksztalcajac lewa strong do-

prowadzič do równości $\mathrm{z}$ prawa. Unikač dowodu metoda redukcji.





101

3.4. Pole podstawy obliczyč korzystajac $\mathrm{z}$ nastepujacego twierdzenia

$0$ zmianie pola figury plaskiej $\mathrm{w}$ rzucie prostokatnym:

{\it Pole rzutu} $prostokq_{f}tnego$ {\it figury ptaskiej jest równe polu} $tej$ {\it figury po}-

{\it mnozonemu przez cosinus} $kq_{f}tami_{G}dzy$ {\it ptaszczyznami figury} $ijej$ {\it rzutu}.

3.5. Kwadrat pola trójkata wyrazič jako funkcje wysokości trójkata.

Funkcja ta jest wielomianem. Nie mylič tego zadania $\mathrm{z}$ zagadnieniem wy-

znaczania ekstremów lokalnych.

3.6. Zauwazyč, $\dot{\mathrm{z}}\mathrm{e}$ granica lewostronna pochodnej $y'(x) \mathrm{w}$ punkcie

$x = \displaystyle \frac{5}{2}$ jest równa $-\mathrm{o}\mathrm{o}$ co oznacza, $\dot{\mathrm{z}}\mathrm{e}$ wykres jest $\mathrm{w}$ punkcie $(\displaystyle \frac{5}{2},0)$

styczny (lewostronnie) do prostej $x=\displaystyle \frac{5}{2}.$

3.7. Dla danych $ r\mathrm{i}\alpha$ najmniejsze $d$ jest wtedy, gdy krótsza podstawa

trapezu ma dlugośč 0, $\mathrm{t}\mathrm{z}\mathrm{n}$. trapez staje $\mathrm{s}\mathrm{i}\mathrm{e}$ trójkatem. Stad otrzymač

dziedzine dla $d$. Analiza otrzymanych wzorów na pole $\mathrm{i}$ promień okregu

opisanego na trapezie prowadzi do blednej dziedziny. $\mathrm{W}$ obliczeniach przyjač

jako niewiadoma polowe sumy obu podstaw $\mathrm{i}$ wyznaczyč $\mathrm{j}\mathrm{a}\mathrm{z}$ twierdzenia

Pitagorasa $\mathrm{w}$ trójkacie zawierajacym przekatna $\mathrm{i}$ wysokośč trapezu. Promień

okregu opisanego wyznaczyč stosujac twierdzenie sinusów.

3.8. Wyrazenie znajdujace $\mathrm{s}\mathrm{i}\mathrm{e}$ pod wartościa bezwzgledna przedstawič

jako $\alpha\cos(x-\alpha)$ dla odpowiedniego $\alpha \mathrm{i}\alpha$, podnieśč obie strony do kwadratu

$\mathrm{i}$ skorzystač ze wzoru 2 $\cos^{2}\gamma=1+\cos 2\gamma.$

4.1. Wyrazič $x$ przez niewiadoma liczbe skladników $n \mathrm{i}$ rozwiazač

równanie kwadratowe $\mathrm{z}$ ta niewiadoma.

4.2. Zbudowač model probabilistyczny doświadczenia, $\mathrm{t}\mathrm{j}$. określič zbiór

$\Omega \mathrm{i}$ prawdopodobieństwo $P$. Wygodniej jest obliczač prawdopodobieństwo

zdarzenia przeciwnego, $\mathrm{t}\mathrm{j}. \dot{\mathrm{z}}\mathrm{e} \mathrm{z}$ wylosowanych cyfr nie $\mathrm{m}\mathrm{o}\dot{\mathrm{z}}$ na utworzyč

liczby podzielnej przez 5.

4.3. Korzystač ze wzorów l- $\cos 2\gamma=2\sin^{2}\gamma$ oraz $\sqrt{\alpha^{2}}=|\alpha|$. Obliczyč

pochodnejednostronne bezpośrednio $\mathrm{z}$ definicji. Podczas rysowania wykresu

zwrócič uwage na otoczenie punktu $x=0.$





102

4.4. Korzystač $\mathrm{z}$ wlasności katów $\mathrm{w}$ równolegloboku. Nastepnie $\mathrm{z}$ przy-

stawania odpowiednich (trzech) trójkatów wywnioskowač, $\dot{\mathrm{z}}\mathrm{e}$ przekatna

utworzonego prostokata jest równolegla do dluzszych boków równoleglo-

boku.

4.5. Podczas rozwiazywania drugiej nierówności rozpatrzyč przypadki

$ y\in (0,1)$ oraz $ y\in (1,\infty)$. Po podstawieniu $2^{x}=t$ przejśč do nierówności

kwadratowych zmiennej $t$. Nie zapomnieč $0$ dziedzinie ukladu $\mathrm{i}$ szczególo-

wym ustaleniu, które punkty brzegu naleza do rozwazanego zbioru.

4.6. Korzystajac $\mathrm{z}$ wlasności okregów stycznych zewnetrznie $\mathrm{i}$ wewne-

trznie, wykazač, $\dot{\mathrm{z}}\mathrm{e}$ suma odleglości rozwazanych punktów od środków obu

danych okregów jest stala $\mathrm{i}$ wynosi 12. Nastepnie zastosowač geometryczna

definicje elipsy.

4.7. Dziedzina funkcji jest określona przez warunki istnienia dwóch

pierwiastków rzeczywistych równania (ale niekoniecznie róznych). $\mathrm{U}\dot{\mathrm{z}}$ yč

wzorów Viète'a. Do rózniczkowania przedstawič otrzymana funkcje jako

sume funkcji potegowych. Ze wzgledu na postač dziedziny nie $\mathrm{m}\mathrm{o}\dot{\mathrm{z}}$ na mówič

$0$ asymptocie ukośnej prawostronnej. Pamietač, $\dot{\mathrm{z}}\mathrm{e},$,przyleganie'' wykresu

funkcji do asymptoty pionowej $\mathrm{m}\mathrm{o}\dot{\mathrm{z}}\mathrm{e}$ byč inne $\mathrm{z}\mathrm{k}\mathrm{a}\dot{\mathrm{z}}$ dej strony tej asymptoty.

4.8. Zauwazyč, $\dot{\mathrm{z}}\mathrm{e}$ czworościan ma plaszczyzne symetrii, która prze-

chodzi przez wierzcholki $A, D\mathrm{i}$ środek krawedzi $BC$. Środek kuli opisanej

$\mathrm{l}\mathrm{e}\dot{\mathrm{z}}\mathrm{y}$ na tej plaszczy $\acute{\mathrm{z}}\mathrm{n}\mathrm{i}\mathrm{e}\mathrm{w}$ punkcie przeciecia $\mathrm{s}\mathrm{i}\mathrm{e}$ prostej prostopadlej do pod-

stawy wystawionej $\mathrm{w}$ środku okregu opisanego na podstawie $\mathrm{z}$ symetralna

krawedzi $AD$. Dziedzine kata $\alpha$ ustalič poprzez rozwazania geometryczne

($\mathrm{k}\mathrm{a}\mathrm{t}\alpha$ musi byč wiekszy od jego rzutu prostokatnego na podstawe).

5.1. Korzystač $\mathrm{z} \mathrm{t}\mathrm{o}\dot{\mathrm{z}}$ samości $(|\alpha|\leq b) \Leftrightarrow (-b\leq\alpha\leq b)$.

narysowač za pomoca translacji standardowej krzywej $y=|x|.$

Zbiór A

5.2. Wyznaczyč najpierw $\sin\alpha+\cos\alpha \mathrm{i}$ stosowač wzór na sume sześcia-

nów.

5.3. Rozwazyč rodzine prostych przechodzacych przez punkt P. Proste

te przecinajac dana parabole, wyznaczaja cieciwy. Napisač uklad rów-

nań, który spelniaja końce cieciw i nie rozwiazujac go, wyznaczyč środki





103

tych cieciw ze wzorów Viète'a. Zwrócič uwage na dziedzine (szukana krzywa

nie jest cala parabola!).

5.4. Wyznaczyč dziedzine $\mathrm{i}$ podnieśč obie strony równania do kwadratu,

otrzymujac proste równanie równowazne wyjściowemu.

5.5. Korzystajac ze schematu Bernoulliego, obliczyč odpowiednie praw-

dopodobieństwa dla obu strzelców. Dla drugiego strzelca najpierw obliczyč

prawdopodobieństwo zdarzenia przeciwnego.

5.6. Jeśli $R$ jest nieduzo wieksze $\mathrm{n}\mathrm{i}\dot{\mathrm{z}}r$, to środki kulek $\mathrm{l}\mathrm{e}\dot{\mathrm{z}}$ a na przekroju

osiowym walca, gdyz kulki zajmuja $\mathrm{m}\mathrm{o}\dot{\mathrm{z}}$ liwie najnizsze polozenie. Najwiek-

sze $R$ (przy ustalonym $r$), przy którym kulki przyjmuja takie polozenie jest

wtedy, gdy trzecia kulka (tj. $\mathrm{l}\mathrm{e}\dot{\mathrm{z}}\mathrm{a}\mathrm{c}\mathrm{a}$ najwyzej) bedzie styczna $\mathrm{z}$ pierwsza

(tj. $\mathrm{l}\mathrm{e}\dot{\mathrm{z}}\mathrm{a}\mathrm{c}\mathrm{a}$ na dnie naczynia). To odpowiada warunkowi $r<R\displaystyle \leq r+\frac{r\sqrt{3}}{2}.$

Narysowač przekrój osiowy walca, zaznaczajac na nim przekroje kulek.

Korzystač $\mathrm{z}$ twierdzenia $0$ okregach stycznych zewnetrznie $\mathrm{i}\mathrm{z}$ twierdzenia

Pitagorasa.

5.7. Przypadek $m=0$ rozpatrzeč oddzielnie. Dla $m\neq 0$ badač mono-

tonicznośč rozwazajac znak pochodnej. Prowadzi to do warunków, przy

których odpowiedni trójmian kwadratowy $\mathrm{w}$ liczniku pochodnej jest nieu-

jemny na R. Pamietač, $\dot{\mathrm{z}}\mathrm{e}$ funkcja jest rosnaca $\mathrm{w}$ pewnym przedziale takze

wtedy, gdy jej pochodna jest nieujemna $\mathrm{i}$ zeruje $\mathrm{s}\mathrm{i}\mathrm{e}\mathrm{w}$ skończonej liczbie

punktów.

5.8. Przekatne $\mathrm{w}$ rombie sa równocześnie dwusiecznymi jego katów.

Jeśli wiec dwa wektory sa równej dlugości, to ich suma wyznacza kierunek

dwusiecznej kata miedzy tymi wektorami.

6.1. Zauwazyč, $\dot{\mathrm{z}}\mathrm{e} x = 1$ spelnia równanie, a dla $x \neq 1$ przejśč do

porównania wykladników. Pamietač $0$ wyznaczeniu dziedziny równania.

6.2. Równanie stycznej do okregu $(x-x_{0})^{2}+(y-y_{0})^{2}=r^{2}\mathrm{w}$ punkcie

$A(x_{1},y_{1})\mathrm{l}\mathrm{e}\dot{\mathrm{z}}$ acym na tym okregu ma postač

$(x_{1}-x_{0})(x-x_{0})+(y_{1}-y_{0})(y-y_{0})=r^{2}$





104

6.3. Korzystač ze wzoru na sume cosinusów oraz ze wzorów reduk-

cyjnych. Przeksztalcač tylko lewa strong $\mathrm{i}$ doprowadzič do równości $\mathrm{z}$ prawa.

6.4. Przyjač, $\dot{\mathrm{z}}\mathrm{e}$ iloraz $q$ ciagujest wiekszy od l. Zauwazyč, $\dot{\mathrm{z}}\mathrm{e}$ środkowy

wyraz ciagu jest równy 2 $\mathrm{i}$ ulozyč równanie $\mathrm{z}$ niewiadoma $q.$

6.5. Oznaczyč przez $A_{i}$ zdarzenie polegajace na wylosowaniu $\mathrm{z}$ pierwszej

urny $i$ kul bialych, $i=0$, 1, 2, 3, $\mathrm{i}$ zastosowač wzór na prawdopodobieństwo

calkowite.

6.6. Zauwazyč, $\dot{\mathrm{z}}\mathrm{e}$ bryle $\mathrm{m}\mathrm{o}\dot{\mathrm{z}}$ na podzielič na dwie (identyczne) polowy

odpowiednia plaszczyzna prostopadla do osi obrotu, a $\mathrm{k}\mathrm{a}\dot{\mathrm{z}}$ da polowa sklada

$\mathrm{s}\mathrm{i}\mathrm{e}$ ze stozka oraz stozka ścietego $0$ wspólnej podstawie.

6.7. Wyznaczyč tylko miejsca zerowe pochodnej $\mathrm{i}$ porównač wartości

funkcji $\mathrm{w}$ tych punktach zjej wartościami na końcach przedzialu. Nie tracič

czasu na wyznaczanie ekstremów lokalnych.

6.8. Maksymalna wartośč $k$ jest osiagana wtedy, gdy trójkat jest równo-

ramienny. Stad ustalič dziedzine $k$. Korzystač $\mathrm{z}$ podobieństwa odpowied-

nich trójkatów $\mathrm{i}\mathrm{z}$ nastepujacej wlasności trójkata prostokatnego:

{\it Suma} $przyprostokq_{f}tnych$ {\it jest równa sumie średnic okrGgów}

{\it wpisanego} $i$ {\it opisanego}.

7.1. Podstawič $3^{x}=t\mathrm{i}$ korzystač $\mathrm{z}\mathrm{t}\mathrm{o}\dot{\mathrm{z}}$ samości podanej we wskazówce

do zadania 5.1.

7.2. Wykorzystač zwiazek wspólrzednych punktu ijego obrazu w powino-

wactwie prostokatnym oraz zwiazek pól figury i jej obrazu w tym przek-

sztalceniu.

7.3. Liczba $k$-elementowych podzbiorów zbioru $n$-elementowego wynosi

$\left(\begin{array}{l}
n\\
k
\end{array}\right)$. Nie pominač zbioru pustego, który jest podzbiorem $\mathrm{k}\mathrm{a}\dot{\mathrm{z}}$ dego zbioru.

7.4. Korzystač $\mathrm{z}$ twierdzenia $0$ czworokacie opisanym na okregu. Do

wyznaczenia $\sin 15^{\mathrm{O}}$ oraz $\cos 15^{\circ}$ nie korzystač $\mathrm{z}$ tablic, lecz przeksztal-

cič wyrazenie $\mathrm{t}\mathrm{a}\mathrm{k}$, aby otrzymač funkcje kata $30^{\circ}$ (por. wskazówka do

$\mathrm{z}\mathrm{a}\mathrm{d}$. 3.8$).$





105

7.5. Rozwiazań, dla których $x=y$, szukač takze wśród nieskończenie

wielu rozwiazań ukladu dla przypadku $m=3.$

7.6. Rozwazyč oddzielnie przedzialy $[-\displaystyle \frac{\pi}{2},0$) oraz $[0,\displaystyle \frac{\pi}{2}],\ \mathrm{w}$ których

$\sin x$ ma staly znak, a funkcja cosinus jest monotoniczna. Zbiór rozwiazań

zaznaczyč na wykresie jako podzbiór osi odcietych.

7.7. Korzystač $\mathrm{z}$ zalezności miedzy polami $\mathrm{i}$ objetościami figur $\mathrm{i}$ bryl

podobnych.

7.8. Skonstruowač model probabilistyczny, czyli określič zbiór $\Omega$ oraz

prawdopodobieństwo $P$. Oznaczyč przez $A_{\mathrm{I}}, A_{\mathrm{I}\mathrm{I}}$ zdarzenia polegajace na

$\mathrm{t}\mathrm{y}\mathrm{m}, \dot{\mathrm{z}}\mathrm{e}$ oba tomy odpowiednio I, II powieści znajduja $\mathrm{s}\mathrm{i}\mathrm{e}$ obok siebie

$\mathrm{i}$ we wlaściwej kolejności. Interesuja nas zdarzenia $A_{\mathrm{I}} \cap A_{\mathrm{I}\mathrm{I}}$ oraz

$A_{\mathrm{I}}\cup A_{\mathrm{I}\mathrm{I}}$. Prawdopodobieństwo tego drugiego obliczyč, stosujac wzór na

prawdopodobieństwo sumy dwóch dowolnych zdarzeń.

8.1. Pamietač $0$ warunku istnienia sumy nieskończonego ciagu geome-

trycznego.

8.2. Skladnik $\left(\begin{array}{l}
11\\
i
\end{array}\right)3^{i/3}2^{(11-i)/2}$ bedzie liczba calkowita wtedy $\mathrm{i}$ tylko

wtedy, gdy $i$ bedzie podzielne przez 3, a $11-i$ bedzie parzyste.

8.3. Korzystač $\mathrm{z}$ parzystości funkcji. Narysowač $\mathrm{w}$ przedziale $[0,\infty$)

wykres funkcji $g(x)=x^{2}-2x-3\mathrm{i}$ zastosowač geometryczna interpretacje

nalozenia na $\mathrm{n}\mathrm{i}\mathrm{a}$ wartości bezwzglednej.

8.4. Najpierw określič dziedzine nierówności. Napisač $x+1=\log_{2}2^{x+1}$,

podstawič $2^{x}=t\mathrm{i}$ przejśč do nierówności kwadratowej.

8.5. Do obliczenia objetości potrzebny jest tylko tangens kata nachyle-

nia ściany bocznej do podstawy $ t=\mathrm{t}\mathrm{g}\alpha$. Warunek podany $\mathrm{w}$ zadaniu zapi-

sač $\mathrm{w}$ postaci równania $\mathrm{z}$ niewiadoma $t. \mathrm{U}\dot{\mathrm{z}}$ yč $\mathrm{t}\mathrm{o}\dot{\mathrm{z}}$ samości $\displaystyle \frac{1}{\cos^{2}\alpha}=1+\mathrm{t}\mathrm{g}^{2}\alpha.$

8.6. $K\mathrm{a}\mathrm{t}$ prosty $\mathrm{m}\mathrm{o}\dot{\mathrm{z}}\mathrm{e}\mathrm{s}\mathrm{i}\mathrm{e}$ znajdowač wjednym $\mathrm{z}$ trzech podanych wierz-

cholków trójkata. Zastosowač iloczyn skalarny.





106

8.7. Po wymnozeniu $\mathrm{n}\mathrm{a}$ krzyz'' skorzystač ze wzoru na iloczyn si-

nusów, doprowadzič do równości dwóch cosinusów $\mathrm{i}$ stad od razu przejśč do

porównania katów. Nie zapomnieč $0$ uwzglednieniu dziedziny.

8.8. Korzystač $\mathrm{z}$ twierdzenia $0$ stosunku pól figur podobnych. Zauwazyč

$\mathrm{i}$ uzasadnič, $\dot{\mathrm{z}}\mathrm{e}$ suma skal podobieństwa trzech mniejszych trójkatów jest

równa l.

9.1. Pole powierzchni powiekszonej kuli jest l,44 razy wieksze od pola

kuli wyjściowej.

9.2. Napisač równanie peku prostych przechodzacych przez punkt $P$

$\mathrm{i}$ majacych ujemny wspólczynnik kierunkowy $m$ (dlaczego?). Wyznaczyč

wspólrzedne punktów $A, B$ przeciecia $\mathrm{s}\mathrm{i}\mathrm{e}$ tych prostych $\mathrm{z}$ osiami ukladu

oraz środków odcinków AB $\mathrm{w}$ zalezności od $m$. Eliminujac parametr $m$

zapisač równanie krzywej $\mathrm{w}$ postaci $y=f(x).$

9.3. Po podstawieniu $3^{x} = t$ zadanie sprowadza $\mathrm{s}\mathrm{i}\mathrm{e}$ do znalezienia

warunków, przy których równanie kwadratowe $\mathrm{z}$ niewiadoma $t$ ma dwa rózne

pierwiastki dodatnie.

9.4. Rozwazyč przekrój czworościanu plaszczyzna symetrii. Korzystajac

$\mathrm{z}$ podobieństwa odpowiednich dwóch trójkatów $\mathrm{w}$ tym przekroju, wykazač,

$\dot{\mathrm{z}}\mathrm{e}$ stosunek promieni kuli opisanej do wpisanej wynosi 3. Stad ob1iczyč

wysokośč czworościanu, a nastepnie kolejno krawed $\acute{\mathrm{z}} \mathrm{i}$ objetośč.

9.5. Dla $x<-3$ lewa stronajest dodatnia, a prawa ujemna $\mathrm{i}$ nierównośč

jest oczywiście spelniona. Dla $x > -3, x \neq 3$, obie strony sa dodat-

nie. Pomnozyč je przez $(x+3)|x-3|$. Po uproszczeniu dostajemy prosta

nierównośč, do której zastosowač $\mathrm{t}\mathrm{o}\dot{\mathrm{z}}$ samośč $(|\alpha|\leq b)\Leftrightarrow(-b\leq\alpha\leq b).$

9.6. Przyjač $k \geq 1$ oraz oznaczyč przez $\alpha$ polowe wiekszego $\mathrm{z}$ katów

ostrych trójkata. Stosunek dwusiecznych wyrazič za pomoca $k$ oraz funkcji

trygonometrycznych kata $\alpha \mathrm{i}$ przeksztalcič $\mathrm{t}\mathrm{a}\mathrm{k}$, aby wystapil tylko tg $\alpha.$

Wartośč tg $\alpha$ obliczyč, wiedzac, $\dot{\mathrm{z}}\mathrm{e}$ tg $2\alpha=k.$





107

9.7. Przedstawič funkcje $\mathrm{w}$ postaci $f(x)=1+\displaystyle \frac{4}{x-2}+\frac{8}{(x-2)^{2}}\mathrm{i}\mathrm{w}$ tej

postaci $\mathrm{j}\mathrm{a}$ rózniczkowač. Zauwazyč, $\dot{\mathrm{z}}\mathrm{e}$ wykres jest wyra $\acute{\mathrm{z}}\mathrm{n}\mathrm{i}\mathrm{e}$ asymetryczny

wzgledem asymptoty $x=2.$

9.8. Napisač równanie stycznej $\mathrm{w}$ punkcie $(x_{0},f(x_{0}))$. Po podstawieniu

do niego wspólrzednych punktu $A$ otrzymujemy równanie trzeciego stop-

nia $\mathrm{z}$ niewiadoma $x_{0}$. Równanie to ma trzy pierwiastki wymierne. Przez

bezpośrednie sprawdzenie wystarczy znalez$\acute{}$č $\mathrm{d}\mathrm{w}\mathrm{a}$. Trzeci $\mathrm{m}\mathrm{o}\dot{\mathrm{z}}$ na obliczyč,

wiedzac, $\dot{\mathrm{z}}\mathrm{e}$ iloczyn pierwiastków wyraza $\mathrm{s}\mathrm{i}\mathrm{e}$ przez wyraz wolny $\mathrm{i}$ wspól-

czynnik przy najwyzszej potedze $x_{0}$. Podczas rysowania wykresu korzystač

$\mathrm{z}$ nieparzystości funkcji $f \mathrm{i}\mathrm{j}\mathrm{u}\dot{\mathrm{z}}$ wyznaczonych stycznych. Dodatkowe ba-

danie nie jest potrzebne.

10.1. Patrz wskazówka do zadania 3.3.

10.2. $K\mathrm{a}\mathrm{t}$ widzenia odcinka AB $\mathrm{z}$ punktu $C$ niewspólliniowego $\mathrm{z}A\mathrm{i}B$

to $\mathrm{k}\mathrm{a}\mathrm{t}\angle ACB$. Dany $\mathrm{w}$ zadaniu $\mathrm{k}\mathrm{a}\mathrm{t}$ zaznaczyč na przekroju osiowym stozka.

Objetośč wyrazič przez $l$ oraz funkcje trygonometryczne wielokrotności kata

$\alpha$. Uwaznie stosowač wzory redukcyjne $\mathrm{i}$ nie bač $\mathrm{s}\mathrm{i}\mathrm{e}$ napisač znaku minus

we wzorze na objetośč.

10.3. Patrz wskazówka do $\mathrm{z}\mathrm{a}\mathrm{d}$. 3.1.

10.4. Najpierw określič model probabilistyczny $\mathrm{t}\mathrm{j}. \Omega \mathrm{i} P$. Zdarze-

nie określone $\mathrm{w}$ treści zadania jest suma czterech rozlacznych (dlaczego?)

zdarzeń $A_{i}, i=1$, 2, 3, 4, gdzie $A_{i}$ oznacza otrzymanie trzech kart $\mathrm{w}i$-tym

kolorze $\mathrm{i}$ jednej $\mathrm{z}$ innego koloru. $P(A_{i})$ obliczyč bezpośrednio, korzystajac

$\mathrm{z}$ tego, $\dot{\mathrm{z}}\mathrm{e}P$ jest prawdopodobieństwem klasycznym.

10.5. Wyznaczyč dziedzine nierówności. Podstawič $\log_{2}x=t \mathrm{i}$ korzy-

stajac $\mathrm{z}$ monotoniczności funkcji logarytmicznej $0$ podstawie $\displaystyle \frac{1}{3}$, przejśč do

nierówności wymiernej.

10.6. Skorzystač ze wskazówki do zadania 6.2 $\mathrm{i}$ wyrazič wspólrzedne

punktów stycznościjako funkcje zmiennej $r$. Wygodniej jest szukač wartości

najwiekszej kwadratu pola, który jest funkcja wymierna.





13

Praca kontrolna

nr 5

5.1. Narysowač na plaszczy $\acute{\mathrm{z}}\mathrm{n}\mathrm{i}\mathrm{e}$ zbiór

$A=\{(x,y)$ : $||x|-y|\leq 1,$

$-1\leq x\leq 2\}$

$\mathrm{i}$ znalez$\acute{}$č punkt zbioru $A\mathrm{l}\mathrm{e}\dot{\mathrm{z}}\mathrm{a}\mathrm{c}\mathrm{y}$ najblizej punktu $P(0,4).$

5.2. Obliczyč $\sin^{3}\alpha+\cos^{3}\alpha,\ \mathrm{m}\mathrm{a}\mathrm{j}_{s}\mathrm{a}\mathrm{c}$ dane $\displaystyle \sin 2\alpha=\frac{1}{4},\ \alpha\in(0,2\pi)$.

5.3. Rozwazmy rodzine prostych przechodzacych przez punkt $P(0,-1)$

$\mathrm{i}$ przecinajacych parabole $y = \displaystyle \frac{1}{4}x^{2} \mathrm{w}$ dwóch punktach. Wyznaczyč

równanie środków powstalych $\mathrm{w}$ ten sposób cieciw paraboli. Sporza-

dzič rysunek $\mathrm{i}$ opisač otrzymana krzywa.

5.4. Rozwiazač równanie

$\sqrt{x+\sqrt{x^{2}-x+2}}-\sqrt{x-\sqrt{x^{2}-x+2}}=4.$

5.5. Dwaj strzelcy strzelaja do tarczy. Pierwszy trafia $\mathrm{z}$ prawdopodo-

bieństwem $\displaystyle \frac{2}{3} \mathrm{w} \mathrm{k}\mathrm{a}\dot{\mathrm{z}}$ dym strzale $\mathrm{i}$ wykonuje 4 strza1y, a drugi trafia

$\mathrm{z}$ prawdopodobieństwem $\displaystyle \frac{1}{3} \mathrm{i}$ oddaje 8 strza1ów. Który ze strze1ców

ma wieksze prawdopodobieństwo uzyskania co najmniej trzech trafień,

jeśli wyniki kolejnych strzalów sa wzajemnie niezalezne?

5.6. Do naczynia $\mathrm{w}$ ksztalcie walca $0$ promieniu podstawy $R$ wrzucono trzy

jednakowe kulki $0$ promieniu $r$, gdzie $2r<2R\leq r(2+\sqrt{3})$. Okazalo

$\mathrm{s}\mathrm{i}\mathrm{e}, \dot{\mathrm{z}}\mathrm{e}$ plaska pokrywa naczynia jest styczna do kulki znajdujacej $\mathrm{s}\mathrm{i}\mathrm{e}$

najwyzej $\mathrm{w}$ naczyniu. Obliczyč wysokośč naczynia.

5.7. Dlajakich wartości parametru $m$ funkcja

$f(x)=\displaystyle \frac{x^{3}}{mx^{2}+6x+m}$

jest określona $\mathrm{i}$ rosnaca na calej prostej rzeczywistej.

5.8. Dany jest trójkat $0$ wierzcholkach $A(-2,1),\ B(-1,-6),\ C(2,5)$.

Za pomoca rachunku wektorowego obliczyč cosinus kata miedzy dwu-

sieczna kata $A\mathrm{i}$ środkowa boku $BC$. Sporzadzič rysunek.





108

10.7. Korzystajac $\mathrm{z}\mathrm{t}\mathrm{o}\dot{\mathrm{z}}$ samości podanej we wskazówce do zadania 2.5,

wyznaczyč $y\mathrm{z}$ drugiego równania, otrzymujac dwa przypadki $y=\displaystyle \frac{3}{4}x$ oraz

$y= \displaystyle \frac{3}{4}x-5\mathrm{i}$ podstawič kolejno do pierwszego równania. Krzywa opisana

pierwszym równaniem jest symetryczna wzgledem osi rzednych, a drugie

równanie przedstawia dwie proste równolegle.

10.8. Przenieśč wszystkie wyrazy na lewa strong, $\mathrm{u}\dot{\mathrm{z}}$ yč wzoru podanego

we wskazówce do zadania 4.3, a nastepnie wzoru na sume sinusów.

ll.l. Jedna $\mathrm{z}$ figur jest trójkat, którego pole stanowi ósma cześč pola

calego trójkata (dlaczego?). Stad wywnioskowač, $\displaystyle \dot{\mathrm{z}}\mathrm{e}\alpha=\frac{\pi}{6}.$

11.2. Plaszczyzna przechodzaca przez jedna $\mathrm{z}$ krawedzi bocznych

$\mathrm{i}$ środek kuli jest plaszczyzna symetrii $\mathrm{i}$ przecina podstawy graniastoslupa

wzdluz ich wysokości. Wybierajac odpowiedni trójkat, obliczyč szukana

wysokośč. (Mozna $\mathrm{t}\mathrm{e}\dot{\mathrm{z}}$ argumentowač inaczej zauwazajac, $\dot{\mathrm{z}}\mathrm{e}$ środek kuli

opisanej oraz wierzcholki podstawy tworza czworościan foremny, którego

wysokośč stanowi polowe szukanej wysokości graniastoslupa.)

11.3. Najpierw wyznaczyč ekstrema lokalne funkcji $g(x) = \displaystyle \frac{x}{1+x^{2}}.$

Poniewaz $f(x) = \alpha g(x)$, wiec dobór $\alpha$ jest natychmiastowy. Trzeba tylko

pamietač, $\dot{\mathrm{z}}\mathrm{e}\alpha \mathrm{m}\mathrm{o}\dot{\mathrm{z}}\mathrm{e}$ byč takze ujemne $\mathrm{i}$ wtedy maksimum $f$ jest osiagane

tam, gdzie $g$ ma minimum.

11.4. Wyznaczyč dziedzine (warunek istnienia sumy nieskończonego

ciagu geometrycznego) $\mathrm{i}$ pamietač, $\dot{\mathrm{z}}\mathrm{e} \mathrm{w}$ niej mianownik sumy po lewej

stronie jest dodatni. Pomnozyč obie strony przez ten mianownik $\mathrm{i}$ sko-

rzystač ze wzoru podanego we wskazówce do zadania 3.8.

11.5. $\mathrm{U}\dot{\mathrm{z}}$ ycie indukcji matematycznej nie jest potrzebne. Przeksztalcič

prawa strong piszac 2 $\left(\begin{array}{l}
i\\
2
\end{array}\right) = i(i-1) = i^{2}-i \mathrm{i}$ pogrupowač skladniki

kwadratowe oddzielnie, a liniowe zsumowač jako kolejne liczby naturalne.

11.6. Oznaczyč środek jednego $\mathrm{z}$ rozwazanych okregów przez $A(x,y).$

Stycznośč do osi $Ox$ oznacza, $\dot{\mathrm{z}}\mathrm{e}$ promień tego okregu wynosi $|y|$, czyli





109

odleglośč $A$ od środka $S$ danego okregu wynosi $|AS| = 2+ |y|$ (stycz-

nośč zewnetrzna!). Odleglośč $|AS|$ wyrazič bezpośrednio za pomoca $x\mathrm{i}y$

$\mathrm{i}$ tak otrzymač szukane równanie. Nazwač otrzymana krzywa. Pamietač,

$\dot{\mathrm{z}}\mathrm{e}$ środki okregów $\mathrm{l}\mathrm{e}\dot{\mathrm{z}}$ a na zewnatrz danego okregu.

11.7. Przyjač $\log_{3}m=t\mathrm{i}$ korzystač ze wzorów Viète'a.

11.8. Najpierw wyznaczyč dziedzine nierówności. Przypadek $x<0$jest

oczywisty, a dla $x>0\mathrm{m}\mathrm{o}\dot{\mathrm{z}}$ na podnieśč obie strony do kwadratu, nastepnie

pomnozyč przez $x^{2}$, otrzymujac nierównośč kwadratowa.

12.1. Narysowač krzywe $y = \sqrt{x-3}$ oraz $y = 4-x \mathrm{i}$ za pomoca

rysunku uzasadnič, $\dot{\mathrm{z}}\mathrm{e}$ równanie to ma tylko jeden pierwiastek oraz $\dot{\mathrm{z}}\mathrm{e}\mathrm{l}\mathrm{e}\dot{\mathrm{z}}\mathrm{y}$

$\mathrm{w}$ przedziale (3, 4). Ob1iczyč go przez podniesienie obu stron równania do

kwadratu.

12.2. Napisač rozklad $w(x)$ na czynniki $\mathrm{i}$ podstawič do obu stron

równości $x=-1.$

12.3. Niech $A_{i}$ oznacza zdarzenie polegajace na wypadnieciu $i$ oczek na

kostce. Wówczas $\Omega =  A_{1}\cup \cup A_{6} \mathrm{i}$ skladniki sa parami rozlaczne. Za-

stosowač wzór na prawdopodobieństwo calkowite. Dla wygody obliczyč

najpierw prawdopodobieństwo zdarzenia przeciwnego do określonego

$\mathrm{w}$ zadaniu, polegajacego na $\mathrm{t}\mathrm{y}\mathrm{m}, \dot{\mathrm{z}}\mathrm{e}$ rzuty moneta nie daly $\dot{\mathrm{z}}$ adnego orla.

12.4. Zauwazyč, $\dot{\mathrm{z}}\mathrm{e}$ sa cztery takie okregi; dwa $\mathrm{w}$ I čwiartce $\mathrm{i}$ pojednym

$\mathrm{w}$ II $\mathrm{i}$ IV čwiartce. Środek szukanego okregu ma $\mathrm{w}$ I čwiartce postač $S(r,r),$

$\mathrm{w}$ II čwiartce $S(-r,r)$, a $\mathrm{w}$ IV $S(r,-r)$, gdzie $r>0$jest nieznanym promie-

niem rozwazanego okregu. $\mathrm{W} \mathrm{k}\mathrm{a}\dot{\mathrm{z}}$ dym przypadku niewiadoma $r$ wyzna-

czyč ze wzoru na odleglośč punktu od danej prostej, $\mathrm{t}\mathrm{j}. 3x+4y=12.$

12.5. Poprowadzič wysokości sasiednich ścian bocznych do ich wspólnej

krawedzi. Tworza one wraz $\mathrm{z}$ przekatna podstawy trójkat równoramienny,

którego $\mathrm{k}\mathrm{a}\mathrm{t}$ przy wierzcholku wynosi $ 2\alpha (\mathrm{z}$ twierdzenia $0$ trzech prostopa-

dlych), a wysokośč jest równa $d.$

12.6. Znajac $P \mathrm{i} s$, obliczamy wysokośč trapezu, a nastepnie jego

przekatna $\mathrm{z}$ twierdzenia Pitagorasa, gdyz rzut prostokatny przekatnej na





110

podstawe ma dlugośč $\displaystyle \frac{s}{2}$. Ramie trapezu wyznaczamy $\mathrm{z}$ podobieństwa odpo-

wiednich trójkatów. Przekatna trapezu nie $\mathrm{m}\mathrm{o}\dot{\mathrm{z}}\mathrm{e}$ przekroczyč średnicy okre-

gu. Stad wynika warunek rozwiazalności zadania.

12.7. Dla $p=-1\mathrm{i}p=2$ ukladjest nieoznaczony $\mathrm{t}\mathrm{z}\mathrm{n}$. ma nieskończenie

wiele rozwiazań. Rozwiazania te tworza dwie proste. Dla $\mathrm{k}\mathrm{a}\dot{\mathrm{z}}$ dego $\mathrm{z}$ po-

zostalych $p$ uklad ma jedno rozwiazanie, które przy zmieniajacym $\mathrm{s}\mathrm{i}\mathrm{e}p$

przebiega trzecia prosta. Na tych trzech prostych znalez$\acute{}$č punkty $0$ podanej

wlasności.

12.8. Badač kwadrat pola powierzchni jako funkcje $y$. Jest ona wielo-

mianem. Nie mylič postawionego pytania $\mathrm{z}$ zagadnieniem wyznaczania

ekstremów lokalnych. Wartośč najmniejsza jest osiagana $\mathrm{w}$ punkcie $y=0,$

a nie $\mathrm{w}$ minimum lokalnym. (Wynik ten klóci $\mathrm{s}\mathrm{i}\mathrm{e}\mathrm{z}$ intuicja, gdyz $\mathrm{w}$ tym

przypadku tworzaca stozka jest najdluzsza.)

13.1. Korzystajac ze wzoru na cosinus róznicy katów przedstawič lewa

strong $\mathrm{w}$ postaci $\alpha\cos(x-\varphi)$ dla odpowiednio dobranego kata $\varphi.$

13.2. Wektor [l2, 5] jest wektorem normalnym prostej $l$, czyli wektor

$\vec{v}= [5,-12]$ jest do niej równolegly (por. wskazówka do zadania 31.7.).

$\mathrm{Z}$ definicji iloczynu skalarnego wynika, $\dot{\mathrm{z}}\mathrm{e}$ liczba $\displaystyle \frac{|\vec{AB}0\vec{v}|}{|\vec{v}|}$ jest dlugościa

rzutu prostokatnego odcinka $AB$ na prosta $l.$

13.3. Wyznaczyč dziedzine (nie zapomnieč $0$ warunku $2^{m}\neq 7$) $\mathrm{i}\mathrm{u}\dot{\mathrm{z}}$ yč

wzorów Viète'a. Wykres $f$ otrzymač ze standardowej krzywej $y=2^{m}$ przez

translacje $\mathrm{i}$ odbicie symetryczne.

13.4. Oznaczyč przez $B_{i}$ zdarzenie polegajace na $\mathrm{t}\mathrm{y}\mathrm{m}, \dot{\mathrm{z}}\mathrm{e}$ pierwszy

strzelec trafil $i$ razy, $i=0$, 1, 2, a przez $C_{j}$ zdarzenie, $\dot{\mathrm{z}}\mathrm{e}$ drugi strzelec

trafil $j$ razy, $j = 0$, 1, 5. Wtedy rozwazane zdarzenie ma postač

$(B_{0}\cap C_{3})\cup(B_{1}\cap C_{2})\cup(B_{2}\cap C_{1})$. Korzystač ze schematu Bernoulliego

$\mathrm{i}$ niezalezności par zdarzeń $B_{i}, C_{j}.$

13.5. Oddzielnie rozwazyč $n$ parzyste $\mathrm{i}$ nieparzyste. Zapisač warunki

na sumy wyrazów tego ciagu $\mathrm{i}$ eliminujac niewiadome wyrazič $\alpha_{2}$ oraz $\alpha_{3}$





111

tylko przez róznice tego ciagu.

$\alpha_{2}\alpha_{3}=48.$

Nastepnie obliczyč te róznice z równania

13.6. Poprowadzič dwusieczna $AD \mathrm{i}$ wyznaczyč $|BC|$, korzystajac

$\mathrm{z}$ podobieństwa trójkatów $ABC \mathrm{i} ADC$. Dalej korzystač $\mathrm{z}$ twierdzenia

sinusów oraz ze wzoru na promień okregu wpisanego $\mathrm{w}$ trójkat $r=\displaystyle \frac{S}{p}.$

13.7. Zbiór $A$ wyznaczyč korzystajac ze wskazówki do zadania 5.1.

Uzasadnič (podnoszac obie strony do kwadratu), $\dot{\mathrm{z}}\mathrm{e}$ krzywa $0$ równaniu

$y=\sqrt{4x-x^{2}}$ nie jest lukiem paraboli lecz pólokregiem. Obliczyč odleglośč

punktu $P$ od $\mathrm{k}\mathrm{a}\dot{\mathrm{z}}$ dej $\mathrm{z}$ trzech cześci brzegu zbioru $A\cap B\mathrm{i}$ porównač je.

13.8. Korzystač $\mathrm{z}$ parzystości funkcji. $\mathrm{Z}$ postaci dziedziny wynika, $\dot{\mathrm{z}}\mathrm{e}$

funkcja nie $\mathrm{m}\mathrm{o}\dot{\mathrm{z}}\mathrm{e}$ mieč asymptot (dlaczego?). Granica lewostronna pochod-

nej $\mathrm{w}$ punkcie $x = \sqrt{8}$ wynosi $-\infty$, wiec prosta $x = \sqrt{8}$ jest styczna do

wykresu funkcji $f(x)$. Dla sporzadzenia wykresu dobrač odpowiednia jed-

nostke na obu osiach ukladu wspólrzednych.

14.1. Korzystač ze wskazówki do $\mathrm{z}\mathrm{a}\mathrm{d}$. 7.3. Otrzymane wyrazenie jest

ciagiem rosnacym $\mathrm{i}$ zadanie $\mathrm{m}\mathrm{o}\dot{\mathrm{z}}\mathrm{e}$ mieč co najwyzej jedno rozwiazanie.

14.2. Uzasadnič, $\dot{\mathrm{z}}\mathrm{e}$ promienie kolejnych okregów tworza ciag geome-

tryczny, którego iloraz jest równy pierwszemu wyrazowi ciagu.

14.3. Korzystač $\mathrm{z}$ rachunku wektorowego $\mathrm{i}$ iloczynu skalarnego. Za-

uwazyč, $\dot{\mathrm{z}}\mathrm{e}$ wszystkie proste $\mathrm{z}$ danej rodziny przechodza przez punkt $P(1,1).$

14.4. Stosowač wzór na tangens róznicy katów.

tej $\mathrm{t}\mathrm{o}\dot{\mathrm{z}}$ samości $\mathrm{i}$ funkcji $f(x).$

Wyznaczyč dziedzine

14.5. Skorzystač ze wskazówki do zadania 7.2. Rozwazana figura jest

róznica odcinka danego kola, wyznaczonego przez oś odcietych, oraz jego

obrazu $\mathrm{w}$ powinowactwie określonym $\mathrm{w}$ zadaniu.

14.6. Zastosowač podana nierównośč $\mathrm{i}$ sprowadzič logarytmy do pod-

stawy 2. Nastepnie wykazač, $\dot{\mathrm{z}}\mathrm{e}$ iloraz rozwazanego ciagu geometrycznego

jest wiekszy od l $\mathrm{i}$ stad od razu otrzymač odpowied $\acute{\mathrm{z}}.$

14.7. Patrz wskazówka do zadania 7.8.





112

14.8. Rozwazyč przekrój plaszczyzna przechodzaca przez przekatna

podstawy $\mathrm{i}$ wierzcholek ostroslupa. $\mathrm{Z}$ twierdzenia $0$ odcinkach stycznych

do kuli poprowadzonych $\mathrm{z}$ ustalonego punktu wynika, $\dot{\mathrm{z}}\mathrm{e}$ punkt styczności

kuli $\mathrm{z}$ krawedzia boczna $\mathrm{l}\mathrm{e}\dot{\mathrm{z}}\mathrm{y}\mathrm{w}$ odleglości $\displaystyle \frac{\alpha}{2}$ od wierzcholka podstawy. Ko-

rzystajac $\mathrm{z}$ tej obserwacji obliczyč krawed $\acute{\mathrm{z}}$ boczna na dwa sposoby $\mathrm{i}$ stad

wyznaczyč promień kuli.

15.1. Oznaczyč przez x odleglośč miejscowości, a przez y predkośč

drugiego rowerzysty. Ulozyč uklad dwóch równań z tymi niewiadomymi,

zapisujac informacje podane w treści zadania.

15.2. Określič dziedzine nierówności. Przypadek $x<0$ jest oczywisty.

Dla $x>0$ podnieśč obie strony do kwadratu, po pomnozeniu przez $x^{2}$ otrzy-

mujemy nierównośč dwukwadratowa.

15.3. Pole powierzchni dachu obliczyč z twierdzenia podanego we wska-

zówce do zadania 3.4. Objetośč dachu ob1iczyč, dzie1ac bry1e p1aszczyznami

pionowymi na dwa ostroslupy i graniastoslup.

15.4. Wyrazič $w_{n+1}$ przez $w_{n}$, korzystajac $\mathrm{z}$ danych zadania. Uzasadnič,

$\dot{\mathrm{z}}\mathrm{e}$ ciag $\triangle_{n}=w_{n+1}-w_{n}$ jest ciagiem geometrycznym $0$ ilorazie 1,015 oraz

$\dot{\mathrm{z}}\mathrm{e}w_{n}=w_{1}+\triangle_{1}+ +\triangle_{n-1}$. Pensja $\mathrm{w}$ kwietniu 2002 roku jest równa $w_{6}.$

15.5. Funkcja $f(x)$ jest rosnaca $\mathrm{i}$ zbiorem jej wartości jest R. Stad

$f^{-1}(x)=\sqrt[3]{x}$ jest określona na $\mathrm{R}$, ajej wykres jest odbiciem symetrycznym

wykresu $f(x)$ wzgledem prostej $0$ równaniu $y = x$. Wykres funkcji $h(x)$

$\mathrm{w}$ przedziale $(0,\infty)$ otrzymač przez translacje cześci wykresu funkcji $f^{-1}(x)$

$\mathrm{i}$ korzystajac $\mathrm{z}$ parzystości funkcji $h(x).$

15.6. Wyznaczyč dziedzine, pomnozyč obie strony przez mianownik,

przejśč za pomoca wzoru redukcyjnego do równości dwóch cosinusów i stad

od razu do porównania katów. Wynik zapisač w postaci jednej serii.

15.7. Patrz wskazówka do zadania 5.8.





113

15.8. Wyznaczyč dziedzine funkcji; nie pominač punktu $x=0$. Sume

wyrazów nieskończonego ciagu geometrycznego zapisač $\mathrm{w}$ postaci

$f(x)=x+1+\displaystyle \frac{2}{x-2}, \mathrm{z}$ której od razu odczytač równania asymptot (uwazač

na dziedzine). Ta postač jest takze wygodna do rózniczkowania (nie jest

celowe stosowanie wzoru na pochodna ilorazu). Podczas rysowania wykresu

jeszcze raz uwazač na dziedzine.

16.1. Oznaczyč przez $x, y, z$ ceny odpowiednio poczatkowa, po obnizce

$\mathrm{i}$ po podwyzce. Wyrazič $y$ przez $x$ oraz $z$ przez $y\mathrm{i}\mathrm{w}$ konsekwencji $z$ przez $x.$

16.2. Punkt $(0,0)$ rozpatrzyč oddzielnie. Zauwazyč, $\dot{\mathrm{z}}\mathrm{e}$ zbiór jest syme-

tryczny wzgledem obu osi ukladu wspólrzednych $\mathrm{i}$ wyznaczyč (oraz opisač)

najpierw $\mathrm{t}\mathrm{e}$ cześč, która $\mathrm{l}\mathrm{e}\dot{\mathrm{z}}\mathrm{y}\mathrm{w}$ I čwiartce.

16.3. Wysokości ścian bocznych oraz odcinek laczacy środki dwóch

odpowiadajacych im krawedzi podstawy tworza trójkat równoramienny

$0$ kacie przy wierzcholku $ 2\alpha \mathrm{i}$ podstawie $\displaystyle \frac{\alpha}{2}$ (dlaczego?). Podstawa tego

trójkata nie przechodzi przez spodek wysokości ostroslupa. Przez porów-

nanie tego trójkata $\mathrm{z}$ jego rzutem prostokatnym na podstawe ostroslupa,

określič dziedzine dla kata $\alpha.$

16.4. $\mathrm{M}\mathrm{o}\dot{\mathrm{z}}$ na odciač narozniki zawierajace wierzcholki katów ostrych

równolegloboku lub zawierajace wierzcholki katów rozwartych. Nalezy wy-

brač ($\mathrm{i}$ uzasadnič odpowiednim rachunkiem) to ciecie, które daje romb

$0$ wiekszym polu, $\mathrm{t}\mathrm{j}$. odciač narozniki zawierajace katy rozwarte. Punkt,

przez który nalezy poprowadzič ciecie wyznaczyč $\mathrm{z}$ twierdzenia cosinusów.

16.5. $\sqrt{2}$ zamienič na potege $0$ podstawie 2 $\mathrm{i}$ wykladniku ulamkowym,

skorzystač $\mathrm{z}$ regul dzialań na potegach, przejśč do porównania wykladników

$\mathrm{i}$ podstawič $\log_{2}x=t.$

16.6. Wyrazenie $\mathrm{w}$ mianowniku zapisač $\mathrm{w}$ postaci $3+\alpha\cos(x-\alpha)$ (por.

wskazówka do zadania 3.8). Wykazač, $\dot{\mathrm{z}}\mathrm{e} |\alpha| <3$, co oznacza, $\dot{\mathrm{z}}\mathrm{e}$ dziedzina

funkcji $f(x)$ jest $\mathrm{R}$, a mianownik jest dodatni $\mathrm{w}$ R. Wartośč najmniejsza

funkcji $f(x)$ jest osiagana $\mathrm{w}$ tym punkcie, $\mathrm{w}$ którym mianownik jest naj-

wiekszy $\mathrm{i}$ na odwrót. $\mathrm{U}\dot{\mathrm{z}}$ ycie pochodnej jest zbedne.





114

16.7. Oddzielnie rozpatrzyč przypadek $p = 0$. Dla $p \neq 0$ równanie

dwukwadratowe ma dokladnie dwa rózne pierwiastki, gdy odpowiadajace

mu równanie kwadratowe ma wyróznik równy zeru $\mathrm{b}\mathrm{a}\mathrm{d}\acute{\mathrm{z}}$ ma wyróznik do-

datni $\mathrm{i}$ jednocześnie jeden $\mathrm{z}$ pierwiastków ujemny.

16.8. Napisač równanie stycznej $\mathrm{w}$ punkcie $A$, korzystajac $\mathrm{z}$ pochod-

nej funkcji. Styczna ta przecina wykres funkcji $\mathrm{w}$ innym punkcie $B$. Przy

wyznaczaniu tego punktu otrzymujemy równanie trzeciego stopnia, które

ze wzgledu na stycznośč $\mathrm{w}$ punkcie $A$ ma pierwiastek podwójny 3 $\mathrm{i}$ tylko

trzeba znalez$\acute{}$č trzeci pierwiastek (por. wskazówka do $\mathrm{z}\mathrm{a}\mathrm{d}$. 9.8).

17.1. Najpierw rozwazyč przypadek, gdy iloraz równy zeru, $\mathrm{t}\mathrm{z}\mathrm{n}.$

$\cos x = 0$. Wtedy wszystkie dalsze wyrazy ciagu sa równe zeru. Jeśli

$\cos x \neq 0$, to liczby $\sin x, \cos x, \sin 2x$ tworza ciag geometryczny wtedy

$\mathrm{i}$ tylko wtedy, gdy kwadrat liczby środkowej jest iloczynem liczb skrajnych,

$\mathrm{t}\mathrm{z}\mathrm{n}$. gdy $\cos x=2\sin^{2}x$. Podstawič $\cos x=t.$

17.2. Losowe dzielenie druzyn na grupy interpretowač jako permuta-

cje numerów wszystkich druzyn, $\mathrm{t}\mathrm{j}$. liczb 1, 2, 16, gdzie ko1ejne czwórki

wyrazów permutacji wyznaczaja sklad kolejnych grup. Pamietač $0$ określeniu

na poczatku modelu probabilistycznego, $\mathrm{t}\mathrm{j}. \Omega \mathrm{i}P.$

17.3. Zauwazyč, $\dot{\mathrm{z}}\mathrm{e}$ dane wyrazenie $\mathrm{m}\mathrm{o}\dot{\mathrm{z}}$ na zapisač $\mathrm{w}$ postaci

$[(x^{2}+x+1)^{3}+x^{3}] - [x^{6}+2x^{3}+1] \mathrm{i}$ stosujac wzór na sume sześcianów,

wykazač, $\dot{\mathrm{z}}\mathrm{e}$ oba skladniki tej sumy dziela $\mathrm{s}\mathrm{i}\mathrm{e}$ przez $(x+1)^{2}$

17.4. $\mathrm{Z}$ symetrii figury wynika, $\dot{\mathrm{z}}\mathrm{e}$ środek $S$ okregu stycznego $\mathrm{w}$ dwóch

punktach do danej paraboli $\mathrm{l}\mathrm{e}\dot{\mathrm{z}}\mathrm{y}$ na osi rzednych, $\mathrm{t}\mathrm{z}\mathrm{n}$. mamy $S(0,y_{0})$, przy

czym $y_{0} > r$. Stycznośč oznacza, $\dot{\mathrm{z}}\mathrm{e}$ równanie kwadratowe $\mathrm{z}$ niewiadoma

rzedna $y$ punktu styczności powinno mieč dodatni pierwiastek podwójny, co

jest spelnione, gdy wyróznik tego równaniajest równy zeru, a wspólczynnik

przy niewiadomej $y$ jest ujemny.

17.5. Dbač $0$ logiczna poprawnośč zapisu calego dowodu indukcyjnego.

$\mathrm{W}$ dowodzie kroku indukcyjnego przeksztalcač tylko lewa strong, pamietajac,

$\dot{\mathrm{z}}\mathrm{e}$ zwiekszenie $n\mathrm{o}1$ powoduje pojawienie $\mathrm{s}\mathrm{i}\mathrm{e}$ dwóch dodatkowych skladników.





115

17.6. Ustalič dziedzine nierówności $\mathrm{i}$ korzystač $\mathrm{z}$ wlasności logarytmu

$0$ podstawie mniejszej od l (dziedzina jest zawarta $\mathrm{w}$ odcinku $(0,1)$).

17.7. Uzasadnič, $\dot{\mathrm{z}}\mathrm{e} \angle ASD$ jest prosty. To oznacza, $\dot{\mathrm{z}}\mathrm{e}$ dane $c= |AD|$

oraz $d = |AS|, d\sqrt{2} \geq c > d$, jednoznacznie określaja trójkat $ASD$

oraz promień okregu $r \mathrm{i}\mathrm{k}\mathrm{a}\mathrm{t} \angle DAB$ trapezu. Wyznaczyč $r$ oraz dlugości

odcinków, na które punkt styczności dzieli $AD. \mathrm{M}\mathrm{o}\dot{\mathrm{z}}$ liwe sa dwa przy-

padki: albo podzial $AB$, liczac od wierzcholka $A$, jest $\mathrm{w}$ stosunku 2:1, a1bo

$\mathrm{w}$ stosunku 1:2. $\mathrm{W}$ drugim przypadku $\mathrm{m}\mathrm{o}\dot{\mathrm{z}}\mathrm{e}\mathrm{s}\mathrm{i}\mathrm{e}$ zdarzyč, $\dot{\mathrm{z}}\mathrm{e}\mathrm{k}\mathrm{a}\mathrm{t}$ przy wierz-

cholku $B$ jest rozwarty.

17.8. $\mathrm{M}\mathrm{o}\dot{\mathrm{z}}$ liwe sa dwa przypadki: albo $\mathrm{w}$ jednym $\mathrm{z}$ wierzcholków pod-

stawy wszystkie katy plaskie kata trójściennego wychodzacego $\mathrm{z}$ tego wierz-

cholka sa ostre, albo wszystkie sa rozwarte. $\mathrm{W}$ obu przypadkach poprowa-

dzič plaszczyzne symetrii przez ten wierzcholek $\mathrm{i}$ przeciwlegly wierzcholek

drugiej podstawy oraz przez odpowiednie przekatne obu podstaw. Niezna-

$\mathrm{n}\mathrm{a}$ wysokośč równoleglościanu obliczamy $\mathrm{z}$ twierdzenia $0$ trzech prostopa-

dlych. Obliczamy najpierw wysokośč rombu tworzacego $\mathrm{k}\mathrm{a}\dot{\mathrm{z}}\mathrm{d}\mathrm{a}$ ściane równo-

leglościanu, nastepnie odleglośč spodka wysokości równoleglościanu od kra-

wedzi podstawy $\mathrm{i}$ wreszcie $\mathrm{z}$ twierdzenia Pitagorasa wysokośč równoleglo-

ścianu. $\mathrm{W}$ obu przypadkach obliczenia sa analogiczne.

18.1. Zarówno licznik jak $\mathrm{i}$ mianownik sa sumami skończenie wielu

(ustalič $\mathrm{i}\mathrm{l}\mathrm{u}$) wyrazów dwóch ciagów geometrycznych. Obliczyč te sumy

$\mathrm{i}$ podzielič licznik $\mathrm{i}$ mianownik przez $2^{2n}$

18.2. Szukana prosta przechodzi przez środek odcinka $AB$ ijest prosto-

padla do danej prostej. Stad od razu $\mathrm{m}\mathrm{o}\dot{\mathrm{z}}$ na napisač równanie tej prostej.

18.3. Patrz wskazówka do zadania l0.2.

18.4. Oznaczyč przez $x, y$ ceny dlugopisu $\mathrm{i}$ zeszytu. Wtedy

$x >y\geq 0$, 50. Ulozyč uklad dwóch równań $\mathrm{z}$ niewiadomymi $x, y\mathrm{i}$ para-

metrem $k$. Oddzielnie rozwazyč przypadek $k = 2$, dla którego uklad jest

nieoznaczony, oraz $k\neq 2$, gdy uklad ma jedno rozwiazanie. $\mathrm{W}$ pierwszym

przypadku wybrač wszystkie $k$, dla których $x\mathrm{i}y$ wyrazaja $\mathrm{s}\mathrm{i}\mathrm{e}\mathrm{w}$ pelnych

dziesiatkach groszy $\mathrm{i}$ spelniaja warunek $x>y\geq 0$, 50. Odpowiedni rysunek

ulatwia znalezienie wszystkich rozwiazań.





116

18.5. Korzystač ze wzoru $\displaystyle \sin 2\gamma=\frac{2\mathrm{t}\mathrm{g}\gamma}{1+\mathrm{t}\mathrm{g}^{2}\gamma} \mathrm{i}$ podstawič $\mathrm{t}\mathrm{g}\gamma=t.$

18.6. Stosowač schemat Bernoulliego. Drugie pytanie dotyczy praw-

dopodobieństwa warunkowego rozwazanego zdarzenia przy warunku, $\dot{\mathrm{z}}\mathrm{e}$ co

najmniej jedna $\dot{\mathrm{z}}$ arówka jest dobra.

18.7. Poniewaz promień szukanego okregu jest bardzo many, nalezy

przyjač $\mathrm{n}\mathrm{a}$ rysunku $\mathrm{d}\mathrm{u}\dot{\mathrm{z}}$ ajednostke $\mathrm{i}$ narysowač tylko odpowiedni $\mathrm{l}\mathrm{u}\mathrm{k}$ danego

okregu. $\mathrm{W}$ obliczeniach korzystač $\mathrm{z}$ twierdzenia $0$ okregach stycznych ze-

wnetrznie oraz $\mathrm{z}$ twierdzenia Pitagorasa $\mathrm{w}$ trójkacie, którego wierzcholkami

sa środki obu okregów oraz rzut prostokatny środka malego okregu na od-

cinek $AS.$

18.8. Pole $\mathrm{i}$ objetośč ostroslupa ścietego wyrazičjako funkcje dlugości $x$

krawedzi górnej podstawy tego ostroslupa, $0 < x <$ l. Korzystač

$\mathrm{z}$ twierdzenia $0$ stosunku pól $\mathrm{i}$ objetości figur $\mathrm{i}$ bryl podobnych. Wyznaczyč

miejsce zerowe pochodnej znalezionej funkcji, zbadač znak pochodnej $\mathrm{i}$ uza-

sadnič, $\dot{\mathrm{z}}\mathrm{e}\mathrm{w}$ tym punkcie funkcja osiaga nie tylko ekstremum lokalne, ale

takze wartośč najwieksza.

19.1. Wektory $\vec{BM}$ oraz $\vec{BK}$ wyrazič za pomoca wektorów $\vec{AB}, \vec{BC}$

oraz $\vec{CD}$. Majac wspólrzedne tych wektorów, od razu obliczyč pole $\triangle KMB.$

19.2. Napisač zwiazek przekatnej prostopadlościanu $\mathrm{z}$ dlugościami jego

krawedzi $\mathrm{i}$ stad obliczyč nieznana róznice ciagu. Odrzucič to rozwiazanie,

które prowadzi do ujemnych dlugości krawedzi.

19.3. Zbiór $A$ wyznaczyč korzystajac ze wskazówki do zadania 13.7

($\mathrm{w}$ cześci dotyczacej zbioru $B \mathrm{w}$ tamtej wskazówce). Dobrač $s \mathrm{t}\mathrm{a}\mathrm{k}$, aby

prosta $B_{s}$ miala jeden punkt wspólny ze zbiorem $A$ (co to znaczy geome-

trycznie?) $\mathrm{i}$ stad od razu podač odpowied $\acute{\mathrm{z}}.$

19.4. Korzystajac $\mathrm{z}$ nierówności trójkata, ustalič, które pary odcinków

moga byč podstawami trapezu. $\mathrm{s}_{\mathrm{a}}$ trzy takie $\mathrm{m}\mathrm{o}\dot{\mathrm{z}}$ liwości (spośród sześciu).

$\mathrm{W}$ dwóch przypadkach pole trapezu jest mniejsze od ll arów. Wykazač to,

zauwazajac, $\dot{\mathrm{z}}\mathrm{e}$ wysokośč trapezu jest mniejsza od $\mathrm{k}\mathrm{a}\dot{\mathrm{z}}$ dego $\mathrm{z}$ jego ramion.

$\mathrm{W}$ trzecim przypadku nalezy obliczyč pole $\mathrm{i}$ wykazač, $\dot{\mathrm{z}}\mathrm{e}$ przekracza ono

ll arów.





117

19.5. Ustalič dziedzine dla parametru $m\mathrm{i}$ stosowač wzory Viète'a. Za

pomoca pochodnej wykazač, $\dot{\mathrm{z}}\mathrm{e}$ kwadrat róznicy pierwiastków, jako funkcja

zmiennej $m$, jest malejacy $\mathrm{w}$ wyznaczonej dziedzinie.

19.6. $\mathrm{W}$ dowodzie kroku indukcyjnego uwaznie stosowač reguly dzialań

na potegach.

19.7. Sumy $\mathrm{z}$ lewych stron przeksztalcič na iloczyny. Stad wywniosko-

wač, $\dot{\mathrm{z}}\mathrm{e}\sin(x+y)=1$, czyli $\mathrm{z}$ drugiego równania $\displaystyle \cos(x-y)=\frac{1}{2}\mathrm{i}$ od razu

przejśč do ukladów równań liniowych $\mathrm{z}$ niewiadomymi $x\mathrm{i}y.$

19.8. Oznaczyč $|CD| = |CA| = |CB| = \alpha$. Poniewaz $CD \perp AD$

oraz $CD \perp BD$, wiec dwie ściany boczne sa prostopadle do podstawy

$ABD$ (bedacej trójkatem równobocznym) $\mathrm{i}$ tworza ze soba $\mathrm{k}\mathrm{a}\mathrm{t} 60^{\circ} K\mathrm{a}\mathrm{t}$

miedzy podstawa $\mathrm{i}$ plaszczyzna $ABC$ wyznaczamy, przecinajac czworościan

plaszczyzna symetrii $CDE$, gdzie $E$ jest środkiem $AB$. Dla wyznaczenia

kata miedzy plaszczyzna $ABC\mathrm{i}$ plaszczyzna $BCD$ ($\mathrm{i}$ równocześnie $ACD$)

nalezy $\mathrm{z}$ wierzcholka $D$ poprowadzič wysokośč czworościanu na ściane $ABC.$

Poniewaz $\triangle BCD$ jest prostokatny $\mathrm{i}$ równoramienny, wiec $\mathrm{z}$ twierdzenia

$0$ trzech prostopadlych wynika, $\dot{\mathrm{z}}\mathrm{e}$ spodek tej wysokości $\mathrm{l}\mathrm{e}\dot{\mathrm{z}}\mathrm{y} \mathrm{w}$ środku

okregu opisanego na trójkacie $ABC$. Wyrazič $\mathrm{t}\mathrm{e}$ wysokośč przez $\alpha,$

obliczajac objetośč czworościanu na dwa sposoby $\mathrm{i}$ stad od razu wyznaczyč

sinus rozwazanego kata.

20.1. Oddzielnie rozpatrzeč $m=0\mathrm{i}m=2$. Dla pozostalych parametrów

$m$ korzystač $\mathrm{z}$ faktu, $\dot{\mathrm{z}}\mathrm{e}$ oś symetrii paraboli przechodzi przezjej wierzcholek.

20.2. Uzasadnič, $\dot{\mathrm{z}}\mathrm{e}$ środek danej kuli $\mathrm{i}$ środek kuli wpisanej $\mathrm{w}$ dana

bryle tworza przekatna sześcianu $0$ krawedzi równej promieniowi kuli wpisa-

nej. Rozwazyč przekrój plaszczyzna symetrii zawierajaca środki obu $\mathrm{k}\mathrm{u}\mathrm{l}.$

20.3. Określič model probabilistyczny, $\mathrm{t}\mathrm{j}. \Omega \mathrm{i} P$. Obliczyč prawdo-

podobieństwo zdarzenia przeciwnego, korzystajac ze wzoru na prawdopodo-

bieństwo sumy dwóch dowolnych zdarzeń.







XLIV

KORESPONDENCYJNY KURS

Z MATEMATYKI

wrzesień 2014 r.

PRACA KONTROLNA nr l- POZIOM PODSTAWOWY

l. Wykaz, $\dot{\mathrm{z}}\mathrm{e}$ róznica kwadratów dwóch liczb nieparzystych jest podzielna przez 8.

2. Wlaściciel hurtowni sprzedaf $\displaystyle \frac{1}{3}$ partii bananów po zalozonej przez siebie cenie. Poniewaz

pozostałe owoce zaczęły zbyt szybko dojrzewač, więc obnizył ich cenę $0$ 30\%. Dzięki te-

mu sprzedaf 60\% aktua1nego stanu. Resztę bananów udało mu się sprzedač dopiero, gdy

ustalif ich cenę na poziomie $\displaystyle \frac{1}{5}$ ceny $\mathrm{P}^{\mathrm{O}\mathrm{C}\mathrm{Z}}\Phi^{\mathrm{t}\mathrm{k}\mathrm{o}\mathrm{w}\mathrm{e}\mathrm{j}}$. Ile procent zaplanowanego zysku sta-

nowi kwota uzyskana ze sprzedazy? Wjakiej cenie ($\mathrm{w}$ porównaniu $\mathrm{z}$ zalozonq) powinien

sprzedač pierwszą partię towaru, $\dot{\mathrm{z}}$ eby jednokrotna obnizka ich ceny $0$ 25\% pozwolifa na

sprzedanie wszystkich owoców $\mathrm{i}$ uzyskanie zaplanowanego początkowo zysku?

3. Narysuj wykres funkcji

$f(x)=\displaystyle \frac{|x-1|+x}{|x+1|}.$

Następnie rozwiąz nierównośč $f(x)\geq 1 \mathrm{i}$, korzystając $\mathrm{z}$ wykresu, podaj jej interpretację

graficzną.

4. Wykresem funkcji $f(x) =x^{2}+bx+c$ jest parabola $0$ wierzcholku $\mathrm{w}$ punkcie $(3,-1).$

Podaj wzór funkcji, której wykres jest obrazem symetrycznym tej paraboli:

a) względem prostej $x=1,$

b) względem punktu $($1, $0).$

$\mathrm{s}_{\mathrm{P}^{\mathrm{o}\mathrm{r}\mathrm{z}}\Phi^{\mathrm{d}\acute{\mathrm{z}}}}$ staranne wykresy wszystkich funkcji.

5. Oblicz

2$\sin^{3}\alpha+3\sin\alpha\cos^{2}\alpha$

$\sin\alpha\sqrt{\cos\alpha}+\cos\alpha\sqrt{\sin\alpha}'$

wiedząc, $\dot{\mathrm{z}}\mathrm{e}$ tg $\displaystyle \alpha=\frac{1}{2}$. Wynik podaj bez niewymierności $\mathrm{w}$ mianowniku.

6. $\mathrm{Z}$ miejscowości $A\mathrm{i}B$ odległych $090$ kilometrów wyruszyli dwaj rowerzyści. Adam wy-

jechaf $\mathrm{z} A0$ godzinę wcześniej $\mathrm{n}\mathrm{i}\dot{\mathrm{z}}$ Bartek $\mathrm{z} B$. Od momentu spotkania Adam jechaf

do $B90$ minut, a Bartek dotarl do $A$ po 4 godzinach. $\mathrm{Z}$ jaką prędkością jechał $\mathrm{k}\mathrm{a}\dot{\mathrm{z}}\mathrm{d}\mathrm{y}\mathrm{z}$

rowerzystów?




PRACA KONTROLNA nr l- POZ1OM ROZSZERZONY

l. Wykaz, $\dot{\mathrm{z}}\mathrm{e}$ róznica czwartych potęg dwóch liczb nieparzystych jest podzielna przez 16.

2. 31 grudnia Kowalski zaciągnąf $\mathrm{p}\mathrm{o}\dot{\mathrm{z}}$ yczkę 4000 zfotych oprocentowaną $\mathrm{w}$ wysokości 16\%

$\mathrm{w}$ skali roku. Zobowiązaf się spfacič ją $\mathrm{w}$ ciągu roku $\mathrm{w}$ czterech równych ratach platnych

31 marca, 30 czerwca, 30 września $\mathrm{i}31$ grudnia. Oprocentowanie $\mathrm{p}\mathrm{o}\dot{\mathrm{z}}$ yczki liczy się od

l stycznia, a odsetki od kredytu naliczane $\mathrm{s}\Phi^{\mathrm{W}}$ terminach pfatności rat. Oblicz wysokośč

tych rat $\mathrm{w}$ zaokrągleniu do pelnych groszy.

3. Narysuj wykres funkcji

$f(x)=\displaystyle \frac{|x+1|+x}{|x-1|}$

$\mathrm{i}$ wyznacz zbiór jej wartości. Następnie rozwiąz nierównośč $f(x-1) < x \mathrm{i}$ podaj jej

interpretację graficzną.

4. Dla jakich wartości parametru rzeczywistego $m$ równanie kwadratowe

$2x^{2}-mx+m+2=0$

ma dwa pierwiastki rzeczywiste $x_{1}, x_{2}$, których suma odwrotności jest nieujemna? Spo-

rząd $\acute{\mathrm{z}}$ wykres funkcji $f(m)=\displaystyle \frac{1}{x_{1}}+\frac{1}{x_{2}}.$

5. Odcinek $0$ końcach $A(\displaystyle \frac{5}{2},\frac{\sqrt{3}}{2}), B(\displaystyle \frac{5}{2},\frac{3\sqrt{3}}{2})$ jest bokiem $\mathrm{w}\mathrm{i}\mathrm{e}\mathrm{l}\mathrm{o}\mathrm{k}_{\Phi}\mathrm{t}\mathrm{a}$ foremnego wpi-

sanego $\mathrm{w}$ okrąg styczny do osi $Ox$. Wyznacz równanie tego okręgu $\mathrm{i}$ wspólrzędne pozo-

stałych wierzchofków $\mathrm{w}\mathrm{i}\mathrm{e}\mathrm{l}\mathrm{o}\mathrm{k}_{\Phi}\mathrm{t}\mathrm{a}$. Ile rozwi$\mathfrak{B}$ań ma to zadanie? Sporząd $\acute{\mathrm{z}}$ rysunek.

6. $\mathrm{Z}$ wierzcholków podstawy $AB$ trójkąta równobocznego $0$ boku $\alpha$ rozpoczęfy ruch dwa

punkty. Poruszajq się one wzdłuz boków $AC\mathrm{i}BC\mathrm{z}$ prędkościami odpowiednio $v_{1}\mathrm{i}v_{2}.$

Po jakim czasie odlegfośč między nimi będzie równa wysokości trójkąta?

Rozwiązania (rękopis) zadań z wybranego poziomu prosimy nadsylač do

na adres:

28 września 20l4r.

Instytut Matematyki $\mathrm{i}$ Informatyki

Politechniki Wrocfawskiej

Wybrzez $\mathrm{e}$ Wyspiańskiego 27

$50-370$ WROCLAW.

Na kopercie prosimy koniecznie zaznaczyč wybrany poziom! (np. poziom podsta-

wowy lub rozszerzony). Do rozwiązań nalez $\mathrm{y}$ dołączyč zaadresowaną do siebie koperte

zwrotną $\mathrm{z}$ naklejonym znaczkiem, odpowiednim do wagi listu. Prace niespelniające po-

danych warunków nie bedą poprawiane ani odsyłane.

Adres internetowy Kursu: http://www. im.pwr.edu.pl/kurs







XLV

KORESPONDENCYJNY KURS

Z MATEMATYKI

wrzesień 2015 r.

PRACA KONTROLNA $\mathrm{n}\mathrm{r} 1 -$ POZIOM PODSTAWOWY

l. Dla pewnego kąta ostrego $\alpha$ zachodzi równośč $\cos\alpha =  2\sin\alpha$. Wyznaczyč wartości

wszystkich funkcji trygonometrycznych tego $\mathrm{k}_{\Phi^{\mathrm{t}\mathrm{a}}}.$

2. Po modernizacji linii kolejowej łączącej Walbrzych $\mathrm{z}$ Wrocławiem średnia prędkośč po-

ciągu wzrosla $014\mathrm{k}\mathrm{m}/\mathrm{h}$, a czas przejazdu 70 km skrócił się $025$ minut. $\mathrm{Z}$ jaką średnią

prędkością jedzie teraz pociąg na tej linii?

3. Wyznaczyč dziedzinę oraz najmniejszą wartośč funkcji

$f(x)=\displaystyle \frac{1}{\sqrt{10+8x^{2}-x^{4}}}.$

4. Wyznaczyč wzory tych funkcji kwadratowych $f(x)=ax^{2}+bx+c$, dla których najmniej-

szą wartości$\Phi$ jest - $\displaystyle \frac{9}{2}, f(0) =-4$, a jednym $\mathrm{z}$ miejsc zerowych jest $x=4$. Narysowač

wykresy tych funkcji.

5. Uprościč wyrazenie (dla tych $a, b$, dla których ma ono sens)

$(\displaystyle \frac{1}{b}+\frac{2}{\sqrt[6]{a^{2}b^{3}}}+\frac{1}{\sqrt[3]{a^{2}}})(\sqrt[3]{a^{2}}(\sqrt[3]{a}+\sqrt{b})-\frac{a(2\sqrt{b}+\sqrt[3]{a})}{\sqrt[3]{a}+\sqrt{b}})$

Następnie obliczyč jego wartośč dla $a=5\sqrt{5}\mathrm{i}b=14-6\sqrt{5}.$

6. Dane są zbiory $A=\{(x,y):4|x|-4\leq 2|y|\leq|x|+2\}$ oraz $B=\displaystyle \{(x,y):|x|+|y|\leq\frac{5}{2}\}.$

Obliczyč pole zbioru $A\cap B$. Wykonač staranny rysunek.




PRACA KONTROLNA nr l- POZ1OM ROZSZERZONY

1. $\mathrm{W}\mathrm{i}\mathrm{e}\mathrm{d}\mathrm{z}\Phi^{\mathrm{C}}, \dot{\mathrm{z}}\mathrm{e}$ dla wypukłego $\mathrm{k}_{\Phi^{\mathrm{t}\mathrm{a}}} \alpha$ zachodzi równośč $\cos\alpha-\sin\alpha = \displaystyle \frac{1}{3}$, wyznaczyč

wszystkie funkcje trygonometryczne tego kąta.

2. Dlajakich wartości parametru $p$ suma kwadratów pierwiastków trójmianu $px^{2}-2px+2$

jest większa od 3?

3. Cięzarówka $0$ długości $16\mathrm{m}$ jedzie ze stałą prędkością $70\mathrm{k}\mathrm{m}/\mathrm{h}$. Wyprzedza ją samo-

chód osobowy $0$ dfugości $4\mathrm{m}$ jadąc ze stafą prędkości$\Phi 100\mathrm{k}\mathrm{m}/\mathrm{h}$. Manewr wyprzedzania

rozpoczyna od zjazdu na lewy pas dokładnie $20\mathrm{m}$ za $\mathrm{c}\mathrm{i}\mathrm{e}\dot{\mathrm{z}}$ arówką, a kończy, powraca-

jqc na prawy pas jezdni dokładnie $20\mathrm{m}$ przed $\mathrm{n}\mathrm{i}\mathrm{a}$ (odstęp między pojazdami wynosi $\mathrm{w}$

tych momentach $20\mathrm{m}$). $\mathrm{Z}$ naprzeciwka nadjez $\mathrm{d}\dot{\mathrm{z}}$ a inny samochód osobowy $\mathrm{z}$ prędkością

$105\mathrm{k}\mathrm{m}/\mathrm{h}$. Jaka powinna byč odległośč między oboma samochodami osobowymi na po-

czątku manewru wyprzedzania, $\dot{\mathrm{z}}$ eby zakończyf się on bezpiecznie (bez zmiany prędkości

obu samochodów)?

4. Narysowač wykres funkcji

$f(x)=$

dla

dla

$x\leq 1,$

$x>1.$

Posfugując się nim, podač wzór funkcji $g(m)$ określającej liczbe rozwiązań równania

$f(x)=m$, gdzie $m$ jest parametrem rzeczywistym.

5. Uprościč wyrazenie (dla tych $a, b$, dla których ma ono sens)

$(\displaystyle \frac{\sqrt[4]{a}}{\sqrt{b}}-\frac{b}{\sqrt{a}}+\frac{3\sqrt{b}}{\sqrt[4]{a}}-3)(\sqrt[4]{ab^{2}}-b+\frac{2b\sqrt[4]{\alpha}-\sqrt{b^{3}}}{\sqrt[4]{a}-\sqrt{b}})$

Następnie obliczyč jego wartośč dla $a=28-16\sqrt{3}\mathrm{i}b=3.$

6. Dane są zbiory $A=\{(x,y):x^{2}+y^{2}<16\}$ oraz $B=\{(x,y):x^{2}+y^{2}<4||x|-|y||\}.$

Narysowač zbiór $A\backslash B$ oraz obliczyč jego pole.

Rozwiązania (rękopis) zadań z wybranego poziomu prosimy nadsyfač do

na adres:

28 września 20l5r.

Katedra Matematyki WPPT

Politechniki Wrocfawskiej

Wybrzez $\mathrm{e}$ Wyspiańskiego 27

$50-370$ WROCLAW.

Na kopercie prosimy $\underline{\mathrm{k}\mathrm{o}\mathrm{n}\mathrm{i}\mathrm{e}\mathrm{c}\mathrm{z}\mathrm{n}\mathrm{i}\mathrm{e}}$ zaznaczyč wybrany poziom! (np. poziom podsta-

wowy lub rozszerzony). Do rozwiązań nalez $\mathrm{y}$ dołaczyč zaadresowana do siebie kopertę

zwrotną $\mathrm{z}$ naklejonym znaczkiem, odpowiednim do wagi listu. Prace niespelniające po-

danych warunków nie będą poprawiane ani odsyłane.

Adres internetowy Kursu: http://www.im.pwr.wroc.pl/kurs







XLVI

KORESPONDENCYJNY KURS

Z MATEMATYKI

wrzesień 2016 r.

PRACA KONTROLNA $\mathrm{n}\mathrm{r} 1-$ POZIOM PODSTAWOWY

1. $\mathrm{Z}$ miast A $\mathrm{i}\mathrm{B}$ odległych $0700$ km $0$ tej samej godzinie wyruszajq naprzeciw siebie (po

dwu równoleglych torach) dwa pociągi. Pociąg pospieszny, który wyjezdza $\mathrm{z}\mathrm{B}$, jedzie

$\mathrm{z}$ prędkością $035\mathrm{k}\mathrm{m}/\mathrm{h}$ większą $\mathrm{n}\mathrm{i}\dot{\mathrm{z}}$ wyjez $\mathrm{d}\dot{\mathrm{z}}$ ający $\mathrm{z}$ A pociąg osobowy $\mathrm{i}$ przyjez $\mathrm{d}\dot{\mathrm{z}}$ a do

A godzinę wcześniej $\mathrm{n}\mathrm{i}\dot{\mathrm{z}}$ pociag osobowy osiąga B. $\mathrm{Z}$ jakimi prędkościami poruszają się

pociągi $\mathrm{i}\mathrm{w}$ jakiej odlegfości od A się minęly.

2. Wyznaczyč dziedziny funkcji $f(x)=\sqrt{\frac{|x-1|-4}{x+2}}$ oraz $g(x)=f(x+1) \mathrm{i}h(x)=f(|x|).$

3. Liczby

{\it p}$=$--($\sqrt{}$354-2)(9$\sqrt{}\sqrt{}$334$++$(61$\sqrt{}$3$+$2$\sqrt{}+$34))2-(2-$\sqrt{}$3)3 i {\it q}$=$--$\sqrt{}$6346-314$\sqrt{}\sqrt{}$88$+$(18$+$-31$\sqrt{}\sqrt{}$624)

są miejscami zerowym trójmianu kwadratowego $f(x)=x^{2}+ax+b$. Znalez/č najmniejszą

$\mathrm{i}$ największq wartośč $f(x)$ na przedziale $[0$, 5$].$

4. Niech $f(x) = x^{2}$ Narysowač wykres funkcji $g(x) = |f(x-1) -4| \mathrm{i}$ określič liczbę

rozwiązań równania $g(x)=m\mathrm{w}$ zalezności $0$ parametru $m.$

5. Wykresy funkcji $f(x) = \displaystyle \frac{m-1}{m+2}x+1\mathrm{i}g(x) =\displaystyle \frac{m+2}{m-3}x+1$ są prostymi prostopadlymi.

Obliczyč pole trójkata ograniczonego wykresami tych funkcji $\mathrm{i}$ osią $Ox$. Podač równanie

okręgu opisanego na tym trójkqcie. Sporządzič rysunek.

6. $\mathrm{W}$ kwadrat ABCD wpisano kwadrat EFGH, ktory zajmuje
\begin{center}
\includegraphics[width=29.772mm,height=29.364mm]{./KursMatematyki_PolitechnikaWroclawska_1_2016_page0_images/image001.eps}
\end{center}
{\it D}

{\it G} $C$

{\it H}

{\it F}

{\it A E  B}

34 jego powierzchni. Wyznaczyc wartosci wszystkich funkcji

trygonometrycznych mniejszego $\mathrm{z}\mathrm{k}$ tów trójk ta $EBF.$




PRACA KONTROLNA nr l- POZ1OM ROZSZERZONY

l. Statek wyrusza ($\mathrm{z}$ biegiem rzeki) $\mathrm{z}$ przystani A do odległej $0 140$ km przystani B. Po

uplywie l godziny wyrusza za nim łódz/ motorowa, dopędza statek $\mathrm{w}$ pofowie drogi,

po czym wraca do przystani A $\mathrm{w}$ tym samym momencie, $\mathrm{w}$ którym statek przybija do

przystani B. Wyznaczyč prędkośč statku $\mathrm{i}$ prędkośč lodzi $\mathrm{w}$ wodzie stojącej, wiedzqc, $\dot{\mathrm{z}}\mathrm{e}$

prędkośč nurtu rzeki wynosi 4 $\mathrm{k}\mathrm{m}/$godz.

2. Narysowač wykres funkcji $f(x)=\displaystyle \min\{x^{3},\frac{1}{x}\}\mathrm{i}$ wyznaczyčjej dziedzinę oraz zbiór warto-

ści. Podač wzór funkcji $h(x)$, której wykres jest symetryczny do wykresu $f(x)$ względem

punktu $(0,0)$. Określič liczbę rozwiązań równania $f(x)=m\mathrm{w}$ zalezności $0$ parametru $m.$

3. Dla jakich wartości rzeczywistego parametru $p$ równanie $(p-1)x^{2}-(p+1)x-1=0$

ma dwa pierwiastki tego samego znaku odległe co najwyzej $01$?

4. Wykresy funkcji $f(x)=(m-1)x+1\displaystyle \mathrm{i}g(x)=\frac{m}{m-1}x+b$ są prostymi prostopadłymi,

a pole trójkata ograniczonego wykresami tych funkcji $\mathrm{i}$ osią $Ox$ jest równe polu trójkąta

ograniczonego tymi wykresami $\mathrm{i}$ osia $Oy$. Wyznaczyč wzory funkcji $f\mathrm{i}g\mathrm{i}$ obliczyč pole

rozwazanych trójkątów. Sporządzič rysunek.

5. Obliczyč wartości

$p=\displaystyle \sqrt{19-8\sqrt{3}}-\sqrt[3]{26-15\sqrt{3}}\mathrm{i}q=\frac{14\log_{9}\frac{1}{2}-\log_{\sqrt[3]{3}}\frac{1}{4}}{\log_{9}8+\log_{\sqrt{3}}\frac{1}{2}}.$

Następnie wyznaczyč wzór $\mathrm{i}$ narysowač wykres funkcji $f(x) =\displaystyle \frac{ax+b}{cx+d}$, wiedząc, $\dot{\mathrm{z}}\mathrm{e}$ jest

on symetryczny względem punktu $(p,q)\mathrm{i}$ przechodzi przez punkt $(0,0).$

6. Punkt $D$ dzieli bok $AB$ trójkąta równobocznego $ABC\mathrm{w}$ stosunku 2:1. Wyznaczyč stosu-

nek dlugości promienia okręgu wpisanego $\mathrm{w}$ trójkąt $ADC$ do dfugości promienia okręgu

wpisanego $\mathrm{w}$ trójkąt $DBC.$

Rozwiązania (rękopis) zadań z wybranego poziomu prosimy nadsyfač do

na adres:

28 września 20l6r.

Wydziaf Matematyki

Politechnika Wrocfawska

Wybrzez $\mathrm{e}$ Wyspiańskiego 27

$50-370$ WROCLAW.

Na kopercie prosimy $\underline{\mathrm{k}\mathrm{o}\mathrm{n}\mathrm{i}\mathrm{e}\mathrm{c}\mathrm{z}\mathrm{n}\mathrm{i}\mathrm{e}}$ zaznaczyč wybrany poziom! (np. poziom podsta-

wowy lub rozszerzony). Do rozwiązań nalez $\mathrm{y}$ dołączyč zaadresowana do siebie kopertę

zwrotną $\mathrm{z}$ naklejonym znaczkiem, odpowiednim do wagi listu. Prace niespelniające po-

danych warunków nie będą poprawiane ani odsyłane.

Adres internetowy Kursu: http: //www. im. pwr. edu. pl/kurs







XLVII

KORESPONDENCYJNY KURS

Z MATEMATYKI

wrzesień 2017 r.

PRACA KONTROLNA $\mathrm{n}\mathrm{r} 1 -$ POZIOM PODSTAWOWY

l. Uprościč następujace wyrazenie, określiwszy uprzednio jego dziedzine:

$\displaystyle \frac{1}{\sqrt[6]{a^{3}b^{2}}-\sqrt[6]{b^{5}}}(\sqrt[3]{a^{2}}-\frac{b}{\sqrt[3]{a}})+\frac{1}{\sqrt{a}+\sqrt{b}}$ : $\displaystyle \frac{\sqrt[3]{ab}}{a-b}$

Obliczyč wartośč tego wyrazenia, przyjmując $a=3+2\sqrt{2} \mathrm{i} b=1+\sqrt{2}.$

2. Niech $B$ oznacza dziedzinę funkcji $f(x)=\displaystyle \frac{1}{\sqrt{3+2x-x^{2}}}$, a $A=\displaystyle \{x\in 1\mathrm{R}:\frac{1}{|x^{2}-1|}\geq 4\}.$

Wyznaczyč $\mathrm{i}$ zaznaczyč na osi liczbowej zbiory $A, B, A\cap B, A\cup B$ oraz $(A\backslash B)\cup(B\backslash A).$

3. Podač wzór funkcji kwadratowej, której wykres jest symetrycznym odbiciem wykresu

funkcji $f(x)=x^{2}+2x$ względem: a) prostej $x=1$, b) punktu $(0,0)$, c) punktu $($1, $0).$

Odpowiedz/uzasadnič, przeprowadzając odpowiednie obliczenia. Sporzqdzič staranne wy-

kresy wszystkich rozwazanych funkcji.

4. $\mathrm{W}$ pewnym ciągu arytmetycznym róznica piętnastego $\mathrm{i}$ drugiego wyrazu jest równa 13.

Oblicz $\alpha_{30}-a_{4}$ oraz sumę pierwszych dziesięciu wyrazów $0$ numerach nieparzystych,

wiedząc, $\dot{\mathrm{z}}\mathrm{e}$ suma pierwszych dziesięciu wyrazów $0$ numerach parzystych jest równa 125.

5. Przekqtne trapezu prostokatnego $0$ podstawach 3 $\mathrm{i}4$ przecinają się pod kątem prostym.

Obliczyč obwód $\mathrm{i}$ pole trapezu. Sporz$\Phi$dzič rysunek.

6. Ostrosłup prawidłowy, którego podstawą jest kwadrat $0$ boku $a$, przecięto płaszczyzną

przechodzącą przez wysokośč ostrosfupa $\mathrm{i}$ przekatną podstawy. Pole otrzymanego prze-

kroju jest równe polu podstawy. Wyznaczyč pole powierzchni cafkowitej ostrosfupa oraz

cosinus kąta nachylenia ściany bocznej do podstawy.




PRACA KONTROLNA nr l- POZ1OM ROZSZERZONY

l. Uprościč następujące wyrazenie, określiwszy uprzednio jego dziedzinę:

$\displaystyle \frac{\sqrt[3]{a}-\sqrt[3]{b}}{\sqrt[3]{a^{2}}+\sqrt[3]{ab}+\sqrt[3]{b^{2}}} \displaystyle \frac{a-b}{\sqrt[3]{a^{2}}-\sqrt[3]{b^{2}}} (1+\displaystyle \frac{\sqrt[3]{b}}{\sqrt[3]{a}-\sqrt[3]{b}}-\frac{1+\sqrt[3]{b}}{\sqrt[3]{b}})$ : $\displaystyle \frac{\sqrt[3]{b}(1+\sqrt[3]{b})-\sqrt[3]{a}}{\sqrt[3]{b}}$

Obliczyč wartośč tego wyrazenia dla $a=7+5\sqrt{2} \mathrm{i} b=7-5\sqrt{2}.$

2. Dla jakiego rzeczywistego parametru $m$ równanie

--{\it mm}$+${\it x}1--{\it mx}$=$1$+$-{\it mx}

ma dwa pierwiastki będqce sinusem $\mathrm{i}$ cosinusem kąta $\mathrm{z}$ przedziału $(\displaystyle \frac{\pi}{2},\pi)$ ?

3. Dane są liczby: $m= \displaystyle \frac{\left(\begin{array}{l}
6\\
4
\end{array}\right)\left(\begin{array}{l}
8\\
2
\end{array}\right)}{\left(\begin{array}{l}
7\\
3
\end{array}\right)}, n= \displaystyle \frac{(\sqrt{2})^{-4}(\frac{1}{4})^{-\frac{5}{2}}\sqrt[4]{3}}{(\sqrt[4]{4})^{5}\cdot\sqrt{32}\cdot 27^{-\frac{1}{4}}}$. Wyznaczyč $k$ tak, by liczby

$m, k, n$ byly odpowiednio: pierwszym, drugim $\mathrm{i}$ trzecim wyrazem ciqgu geometrycznego,

a następnie wyznaczyč sumę wszystkich wyrazów nieskończonego ciągu geometrycznego,

którego pierwszymi trzema wyrazami są $m, k, n$. Ile wyrazów tego ciągu nalezy wziąč,

by ich suma przekroczyła 95\% sumy wszystkich wyrazów?

4. Podač wzór funkcji homograficznej, której wykres jest symetrycznym odbiciem wykresu

funkcji $f(x)=\displaystyle \frac{x-1}{x+1}$ względem: a) prostej $x=1$, b) punktu $(0,0)$, c) punktu $($1, $0).$

Odpowiedz/uzasadnič, przeprowadzaj $\Phi^{\mathrm{C}}$ odpowiednie obliczenia. Sporządzič staranne wy-

kresy wszystkich rozwazanych funkcji.

5. $\mathrm{W}$ czworokącie wypukfym ABCD, $\mathrm{w}$ którym $AB= 1, BC = 2, CD = 4, DA = 3,$

cosinus kąta przy wierzchofku $B$ jest równy - $\displaystyle \frac{5}{7}$. Wykazač, $\dot{\mathrm{z}}\mathrm{e}$ czworokąt ten $\mathrm{m}\mathrm{o}\dot{\mathrm{z}}$ na

wpisač $\mathrm{w}$ okrąg $\mathrm{i}$ obliczyč promień $R$ tego okręgu. Sprawdzič, czy $\mathrm{w}$ rozwazany czworokąt

$\mathrm{m}\mathrm{o}\dot{\mathrm{z}}$ na wpisač okrqg. $\mathrm{J}\mathrm{e}\dot{\mathrm{z}}$ eli $\mathrm{t}\mathrm{a}\mathrm{k}$, to obliczyč jego promień.

6. $\mathrm{W}$ ostrosłupie prawidfowym czworokątnym, $\mathrm{w}$ którym wszystkie krawędzie $\mathrm{s}\Phi$ równe,

poprowadzono płaszczyznę przechodzącą przez jedną $\mathrm{z}$ krawędzi podstawy oraz środ-

ki dwu przeciwleglych do niej krawędzi bocznych. Obliczyč stosunek pola powierzchni

przekroju do pola podstawy oraz stosunek objętości brył, najakie płaszczyzna podzieliła

ostrosłup.

Rozwiązania (rękopis) zadań z wybranego poziomu prosimy nadsyłač do

na adres:

28 września 20l7r.

Wydziaf Matematyki

Politechnika Wrocfawska

Wybrzez $\mathrm{e}$ Wyspiańskiego 27

$50-370$ WROCLAW.

Na kopercie prosimy $\underline{\mathrm{k}\mathrm{o}\mathrm{n}\mathrm{i}\mathrm{e}\mathrm{c}\mathrm{z}\mathrm{n}\mathrm{i}\mathrm{e}}$ zaznaczyč wybrany poziom! (np. poziom podsta-

wowy lub rozszerzony). Do rozwiązań nalez $\mathrm{y}$ dołączyč zaadresowaną do siebie kopertę

zwrotną $\mathrm{z}$ naklejonym znaczkiem, odpowiednim do wagi listu. Prace niespelniajace po-

danych warunków nie będą poprawiane ani odsylane.

Adres internetowy Kursu: http: //www. im. pwr. edu. pl/kurs







XLVIII

KORESPONDENCYJNY KURS

Z MATEMATYKI

wrzesień 2018 r.

PRACA KONTROLNA $\mathrm{n}\mathrm{r} 1 -$ POZIOM PODSTAWOWY

l. Promień podstawy stozka obrotowego zmniejszono $0$ 20\%. $\mathrm{O}$ ile procent trzeba zwiększyč

wysokośč tego stozka, $\dot{\mathrm{z}}$ eby jego objętośč nie ulegfa zmianie?

2. Dla jakich wartości parametru $m$ nierównośč

$mx^{2}+(m+1)x+2m<0$

jest spelniona dla wszystkich $x\in \mathbb{R}$?

3. Określič dziedzinę $\mathrm{i}$ uprościč nastepujące wyrazenie:

-(($\sqrt{}\sqrt{}$5{\it aa}$\sqrt{}$3-43{\it a})2{\it b}-)-234:[-($\sqrt{}$4$\sqrt{}$5{\it aa}$\sqrt{}$-{\it b}4)2]3

Następnie obliczyč wartośč tego wyrazenia dla $a=\sqrt{3}+\sqrt{2} \mathrm{i} b=5-2\sqrt{6}.$

4. Niech $f(x) = x^{2}$ Narysowač wykres funkcji $g(x) = |f(x+1) -4| \mathrm{i}$ określič liczbę

rozwiązań równania $g(x)=m\mathrm{w}$ zalezności od parametru $m.$

5. Obliczyč pole koła wpisanego $\mathrm{w}$ romb $0$ polu 10 $\mathrm{i}$ kacie ostrym $30^{\mathrm{o}}$

6. Niech $A = \displaystyle \{x\in \mathbb{R}:\frac{3}{2x^{2}+x-6}\geq\frac{1}{2x-3}\}$ oraz $B = \{x\in \mathbb{R}:\sqrt{x^{2}-4x+4}<x\}.$

Wyznaczyč $\mathrm{i}$ narysowač na osi liczbowej zbiory $A, B$ oraz $A\backslash B, B\backslash A.$




PRACA KONTROLNA nr l- POZ1OM ROZSZERZONY

l. Pewna liczba pieciocyfrowa zaczyna się ($\mathrm{z}$ lewej strony) cyfrą 8. Jeś1i cyfrę tę przestawimy

$\mathrm{z}$ pierwszej pozycji na ostatnią, to otrzymamy liczbę stanowiąca 16\% 1iczby pierwotnej.

Znalez/č tę liczbę.

2. Określič dziedzinę $\mathrm{i}$ uprościč następujące wyrazenie:

$\displaystyle \frac{(\sqrt{a}+\sqrt{b})^{2}-4b}{(a-b)(\sqrt{\frac{1}{b}}+3\sqrt{\frac{1}{a}})^{-1}}$ : $\displaystyle \frac{a+9b+6\sqrt{ab}}{\frac{1}{\sqrt{b}}+\frac{1}{\sqrt{a}}}.$

Następnie wyznaczyč jego wartośč dla $a=\sqrt{4-2\sqrt{3}} \mathrm{i} b=\sqrt{3}+1.$

3. Narysowač wykres funkcji $f(x) = \displaystyle \min\{\frac{2x}{x-1},x^{2}\}$. Podač wzór funkcji, której wykres

jest symetryczny do wykresu funkcji $f(x)$ względem początku ukladu wspófrzędnych.

Określič liczbę rozwiązań równania $f(x)=m\mathrm{w}$ zalezności od parametru $m.$

4. Dfugości boków trójkąta prostokątnego tworzą ciąg arytmetyczny $0$ róznicy $p>0$. Ob-

liczyč stosunek promienia okręgu opisanego na $\mathrm{t}\mathrm{y}\mathrm{m}$ trójkącie do promienia okręgu wpi-

sanego $\mathrm{w}$ ten trójkąt.

5. Dla jakich wartości parametru $m$ suma sześcianów pierwiastków równania

$x^{2}+(m-1)x+m=\displaystyle \frac{7}{4}$

nalez $\mathrm{y}$ do przedzialu $[-\displaystyle \frac{1}{2},0$)?

6. Dane sa zbiory

$A=\{(x,y)\in \mathbb{R}^{2}:9-4\sqrt{2}\leq x^{2}+y^{2}<9+4\sqrt{2}\}$

oraz

$B=\{(x,y)\in \mathbb{R}^{2}:x^{2}+y^{2}<4|x|+4|y|-7\}.$

Narysowač starannie zbiór $A\backslash B \mathrm{i}$ wyznaczyč jego pole. Zadbač $0$ odpowiednią skalę

$\mathrm{i}$ czytelnośč rysunku.

Rozwiązania (rękopis) zadań $\mathrm{z}$ wybranego poziomu prosimy nadsyłač do 28 września $2018\mathrm{r}.$

na adres:

Wydziaf Matematyki

Politechnika Wrocfawska

Wybrzez $\mathrm{e}$ Wyspiańskiego 27

$50-370$ WROCLAW.

Na kopercie prosimy $\underline{\mathrm{k}\mathrm{o}\mathrm{n}\mathrm{i}\mathrm{e}\mathrm{c}\mathrm{z}\mathrm{n}\mathrm{i}\mathrm{e}}$ zaznaczyč wybrany poziom! (np. poziom podsta-

wowy lub rozszerzony). Do rozwiązań nalez $\mathrm{y}$ dołączyč zaadresowaną do siebie koperte

zwrotną $\mathrm{z}$ naklejonym znaczkiem, odpowiednim do wagi listu. Prace niespelniające po-

danych warunków nie będą poprawiane ani odsylane.

Uwaga. Wysylajac nam rozwiązania zadań uczestnik Kursu udostępnia nam swoje dane osobo-

we, które przetwarzamy wyłącznie $\mathrm{w}$ zakresie niezbędnym do jego prowadzenia (odesfanie zadań,

prowadzenie statystyki). Szczególowe informacje $0$ przetwarzaniu przez nas danych osobowych są

dostępne na stronie internetowej Kursu.

Adres internetowy Kursu: http: //www. im. pwr. edu. pl/kurs







XLIX

KORESPONDENCYJNY KURS

Z MATEMATYKI

wrzesień 2019 r.

PRACA KONTROLNA $\mathrm{n}\mathrm{r} 1 -$ POZIOM PODSTAWOWY

l. Pan Kowalski załozył dwie lokaty, wplacając do banku $\mathrm{w}$ sumie 10120 $\mathrm{z}1$. Pierwsza $\mathrm{z}$ nich

ma oprocentowanie 12\% $\mathrm{w}$ skali roku $\mathrm{z}$ pófroczn$\Phi$ kapitalizacją odsetek, a druga daje 18\%

zysku, przy czym odsetki są naliczane dopiero po roku. Okazało się, $\dot{\mathrm{z}}\mathrm{e}$ na obu kontach

przybyła mu taka sama kwota. Jakie sumy wplacif na $\mathrm{k}\mathrm{a}\dot{\mathrm{z}}$ dą $\mathrm{z}$ lokat ijaki osiagnąf zysk?

Jaki byfby zysk pana Kowalskiego, gdyby na $\mathrm{k}\mathrm{a}\dot{\mathrm{z}}$ dą $\mathrm{z}$ lokat wpfacif tę $\mathrm{s}\mathrm{a}\mathrm{m}\Phi$ sumę 5060

$\mathrm{z}l.$?

2. Niech $A=\displaystyle \{x\in 1\mathrm{R}:\frac{1}{\sqrt{5-x}}\geq\frac{2}{\sqrt{x+1}}\}$ oraz $B=\{x\in 1\mathrm{R}:|x|+|x-1|\geq 3\}.$

Znalez/č $\mathrm{i}$ zaznaczyč na osi liczbowej zbiory $A, B$ oraz $(A\backslash B)\cup(B\backslash A).$

3. Uprościč wyrazenie (dla tych $a, b$, dla których ma ono sens)

( -{\it b}1 $+$ -$\sqrt{}$6 {\it a}22{\it b}3 $+$ -$\sqrt{}$31{\it a}2) : -$\sqrt{}$3 {\it ba}$\sqrt{}$3$+${\it a}2$\sqrt{}${\it b}.

Następnie obliczyč jego wartośč dla $a=5\sqrt{5}\mathrm{i}b=14-6\sqrt{5}.$

4. Odcinek $AB$ jest średnic$\Phi$ okręgu. Styczna $\mathrm{w}$ punkcie $A\mathrm{i}$ prosta, na której $\mathrm{l}\mathrm{e}\dot{\mathrm{z}}\mathrm{y}$ cięciwa

$BC$ przecinają się $\mathrm{w}$ punkcie $P$ odległym od A $04\sqrt{3}$. Wyznaczyč promień okręgu oraz

długośč cięciwy $BC$, wiedzqc, $\dot{\mathrm{z}}\mathrm{e}$ pole trójkata $ABP$ jest równe $8\sqrt{3}.$

5. Pole trójk$\Phi$ta równobocznego $ABX$ zbudowanego na przeciwprostokątnej $AB$ trójk$\Phi$ta

prostokqtnego $ABC$ jest dwa razy większe od pola wyjściowego trójkąta. Niech $D$ będzie

środkiem boku $AB$. Wykazač, $\dot{\mathrm{z}}\mathrm{e}$ trójkąty $ABC\mathrm{i}ADX$ są $\mathrm{p}\mathrm{r}\mathrm{z}\mathrm{y}\mathrm{s}\mathrm{t}\mathrm{a}\mathrm{j}_{\Phi}\mathrm{c}\mathrm{e}.$

6. Pole powierzchni bocznej stozka jest trzy razy większe $\mathrm{n}\mathrm{i}\dot{\mathrm{z}}$ pole jego podstawy. $\mathrm{W}$ stozek

wpisano walec, którego dolna podstawa jest zawarta $\mathrm{w}$ podstawie stozka, a przekrój

plaszczyzną zawierającą oś stozka jest kwadratem. Wyznaczyč stosunek objętości walca

do objętości stozka.




PRACA KONTROLNA nr l- POZ1OM ROZSZERZONY

l. Określič dziedzinę i uprościč nastepujące wyrazenie

$[\displaystyle \frac{y\sqrt[3]{x}}{\sqrt[3]{x}+\sqrt{y}}-\frac{x-y\sqrt{y}}{x+y\sqrt{y}}\frac{y\sqrt[3]{x^{2}}-y\sqrt{y}\sqrt[3]{x}+y^{2}}{\sqrt[3]{x^{2}}-y}]$ : $\displaystyle \frac{y^{2}}{\sqrt[3]{x}+\sqrt{y}}$

Następnie wyznaczyč jego wartośč dla $x=6\sqrt{3}-10 \mathrm{i} y=12-6\sqrt{3}.$

2. Wyznaczyč sinus kąta przy wierzchofku $C\mathrm{w}$ trójkącie równoramiennym, $\mathrm{w}$ którym środ-

kowe ramion $AC\mathrm{i}BC$ przecinają się pod kqtem prostym.

3. Narysowač obszar $D = \{(x,y):|y|\leq x\leq 4-y^{2}\}$. Obliczyč pole kwadratu, którego

boki są równolegfe do osi ukfadu wspófrzędnych, a wszystkie wierzchofki $\mathrm{l}\mathrm{e}\dot{\mathrm{z}}$ ą $\mathrm{n}\mathrm{a}$ krzywej

ograniczającej obszar $D.$

4. $\mathrm{W}$ trójkącie $ABC$ dane są: $|BC|=a, |AB|=c, \angle ABC=\beta$. Okrąg $\mathrm{P}^{\mathrm{r}\mathrm{z}\mathrm{e}\mathrm{c}\mathrm{h}\mathrm{o}\mathrm{d}\mathrm{z}}\Phi^{\mathrm{c}\mathrm{y}}$ przez

punkty $B\mathrm{i}C$ przecina boki AB $\mathrm{i}AC\mathrm{w}$ takich punktach $D\mathrm{i}E, \dot{\mathrm{z}}\mathrm{e}$ pole czworokąta

BCDE stanowi 75\% po1a trójkąta $ABC$. Wyznaczyč obwód $\mathrm{i}$ pole czworokąta.

5. Basen $\mathrm{m}\mathrm{o}\dot{\mathrm{z}}$ na napefnič, $\mathrm{o}\mathrm{t}\mathrm{w}\mathrm{i}\mathrm{e}\mathrm{r}\mathrm{a}\mathrm{j}_{\Phi}\mathrm{c}$ którykolwiek $\mathrm{z}$ trzech zaworów. Otwarcie pierwszych

$\mathrm{d}\mathrm{w}\mathrm{u}$ pozwala napelnič basen $\mathrm{w}$ czasie $02$ godziny dłuzszym $\mathrm{n}\mathrm{i}\dot{\mathrm{z}}$ otwarcie drugiego $\mathrm{i}$ trze-

ciego zaworu, natomiast otwarcie zaworów pierwszego $\mathrm{i}$ trzeciego pozwala napefnič basen

$\mathrm{w}$ czasie $\mathrm{d}\mathrm{w}\mathrm{a}$ razy krótszym $\mathrm{n}\mathrm{i}\dot{\mathrm{z}}$ otwarcie $\mathrm{d}\mathrm{w}\mathrm{u}$ pierwszych. Napełnienie basenu, gdy otwar-

te są wszystkie trzy zawory, trwa 2 godziny 40 minut. I1e trwa napełnienie basenu, gdy

otwarty jest tylko jeden zawór?

6. $\mathrm{W}$ ostrosłupie prawidlowym czworokątnym przekrój plaszczyzną przechodzącą przez

wierzchofek ostrosfupa $\mathrm{i}$ środki dwu przeciwlegfych krawędzi podstawy jest trójkątem

równobocznym. Ostrosfup przecięto pfaszczyzną przechodzącą przez jedną $\mathrm{z}$ krawędzi

podstawy prostopadła do przeciwleglej ściany bocznej. Obliczyč stosunek objętości brył,

$\mathrm{n}\mathrm{a}$ jakie pfaszczyzna ta dzieli ostroslup.

Rozwiązania (rękopis) zadań z wybranego poziomu prosimy nadsyłač do

na adres:

28 września 20l9r.

Wydziaf Matematyki

Politechnika Wrocfawska

Wybrzez $\mathrm{e}$ Wyspiańskiego 27

$50-370$ WROCLAW.

Na kopercie prosimy $\underline{\mathrm{k}\mathrm{o}\mathrm{n}\mathrm{i}\mathrm{e}\mathrm{c}\mathrm{z}\mathrm{n}\mathrm{i}\mathrm{e}}$ zaznaczyč wybrany poziom! (np. poziom podsta-

wowy lub rozszerzony). Do rozwiązań nalez $\mathrm{y}$ dołączyč zaadresowaną do siebie koperte

zwrotną $\mathrm{z}$ naklejonym znaczkiem, odpowiednim do wagi listu. Prace niespełniające po-

danych warunków nie będą poprawiane ani odsyłane.

Uwaga. Wysylając nam rozwiazania zadań uczestnik Kursu udostępnia Politechnice Wrocfawskiej

swoje dane osobowe, które przetwarzamy wyłącznie $\mathrm{w}$ zakresie niezbędnym do jego prowadzenia

(odesłanie zadań, prowadzenie statystyki). Szczegófowe informacje $0$ przetwarzaniu przez nas danych

osobowych są dostępne na stronie internetowej Kursu.

Adres internetowy Kursu: http: //www. im. pwr. edu. pl/kurs







L

KORESPONDENCYJNY KURS

Z MATEMATYKI

wrzesień 2020 r.

PRACA KONTROLNA $\mathrm{n}\mathrm{r} 1 -$ POZIOM PODSTAWOWY

1. $\mathrm{W}$ pierwszym naczyniu było $\alpha$ litrów $p$-procentowego kwasu siarkowego, $\mathrm{w}$ drugim na-

tomiast $b$ litrów $q$-procentowego kwasu siarkowego. $\mathrm{Z}\mathrm{k}\mathrm{a}\dot{\mathrm{z}}$ dego $\mathrm{z}$ naczyń odlano czwartą

częśč objętości roztworu, a następnie roztwór odlany $\mathrm{z}$ drugiego naczynia wlano do pierw-

szego, a odlany $\mathrm{z}$ pierwszego wlano do drugiego naczynia. Okazafo się, $\dot{\mathrm{z}}\mathrm{e}$ po wymieszaniu

stęzenia roztworów $\mathrm{w}$ obu naczyniach byly równe. Wyznacz stosunek stęzeń wyjściowych

roztworów.

2. Uprośč następujące wyrazenie, określiwszy uprzednio jego dziedzinę:

$\displaystyle \frac{1}{\sqrt[6]{x^{3}y^{2}}-\sqrt[6]{y^{5}}}(\sqrt[3]{x^{2}}-\frac{y}{\sqrt[3]{x}})+\frac{1}{\sqrt{x}+\sqrt{y}}$ : $\displaystyle \frac{\sqrt[3]{xy}}{x-y}$

Oblicz wartośč tego wyrazenia, przyjmując $x=3+2\sqrt{2} \mathrm{i} y=1+\sqrt{2}.$

3. Narysuj wykres funkcji $f(x)=(\displaystyle \sin x+\frac{1}{2}\cos x)^{2}+(\frac{1}{2}\sin x+\cos x)^{2}$

wartości $\mathrm{i}$ rozwiqz nierównośč $f(x)\displaystyle \geq\frac{5}{4}.$

Wyznacz zbiór jej

4. Niech $A=\{(x,y)\in \mathbb{R}^{2}:|x|\leq 2,|y|\leq 2\}$ oraz $B=\{(x,y)\in \mathbb{R}^{2}$ :

Zaznacz na płaszczyz$\acute{}$nie zbiory $A\backslash B$ oraz $A\backslash (A\backslash B).$

$|x-y|\leq|x|+1\}.$

5. $\mathrm{W}$ kwadrat wpisano trójkąt równoboczny $\mathrm{w}$ taki sposób, $\dot{\mathrm{z}}\mathrm{e}$ jeden $\mathrm{z}$ jego wierzchofków

jest $\mathrm{w}$ wierzchołku kwadratu, a dwa pozostałe lezą na przeciwległych bokach kwadratu.

Wyznacz stosunek pola trójkąta do pola kwadratu.

6. $\mathrm{W}$ ostrosfupie prawidlowym trójkątnym podstawa ma dfugośč $a$, a krawęd $\acute{\mathrm{z}}$ boczna jest

do niej nachylona pod kątem $\alpha$. Oblicz objętośč $\mathrm{i}$ pole powierzchni bocznej bryły.




PRACA KONTROLNA nrl -P0Zi0M R0ZSZERZ0NY

1. $\mathrm{W}$ pierwszym naczyniu było $a$ litrów $p$-procentowego kwasu siarkowego, $\mathrm{w}$ drugim nato-

miast $b$ litrów $q$-procentowego kwasu siarkowego. $\mathrm{Z}$ obu naczyń odlano równe objętości

roztworów, a następnie roztwór odlany $\mathrm{z}$ drugiego naczynia wlano do pierwszego, a od-

lany $\mathrm{z}$ pierwszego wlano do drugiego naczynia. Okazało się, $\dot{\mathrm{z}}\mathrm{e}$ po wymieszaniu stęzenia

roztworów $\mathrm{w}$ obu naczyniach byfy równe. Jakie ilości roztworów odlano $\mathrm{z}\mathrm{k}\mathrm{a}\dot{\mathrm{z}}$ dego $\mathrm{z}$ na-

czyń?

2. Uprośč wyrazenie (dla tych $x, y$, dla których ma ono sens)

$(\displaystyle \frac{1}{\sqrt[3]{x}-\sqrt[3]{y}}-\frac{3\sqrt[3]{xy}}{x-y}-\frac{\sqrt[3]{y}-\sqrt[3]{x}}{\sqrt[3]{x^{2}}+\sqrt[3]{xy}+\sqrt[3]{y^{2}}})\frac{x-y}{4\sqrt[3]{xy}}.$

Następnie oblicz jego wartośč dla $x=5\sqrt{2}-7\mathrm{i}y=5\sqrt{2}+7.$

3. Narysuj wykres funkcji $f(x)=\sin^{2}x+\sin x\cos x$. Wyznacz zbiór jej wartości $\mathrm{i}$ rozwiąz

nierównośč $f(x)\geq 1.$

4. Niech $A=\{(x,y)\in \mathbb{R}^{2}:|x-1|+|y-1|\leq 3\}$ oraz $B=\{(x,y)\in \mathbb{R}^{2}$ :

Zaznacz na pfaszczy $\acute{\mathrm{z}}\mathrm{n}\mathrm{i}\mathrm{e}$ zbiór $A\cap B\mathrm{i}$ oblicz jego pole.

$|x-y|\leq|x+y|\}.$

5. $\mathrm{W}$ {\it romb ABCD} $0$ boku $a\mathrm{i}$ kącie ostrym $\alpha$ wpisano trójkąt $APQ\mathrm{t}\mathrm{a}\mathrm{k}, \dot{\mathrm{z}}\mathrm{e}$ punkt $P\mathrm{l}\mathrm{e}\dot{\mathrm{z}}\mathrm{y}$

na boku $BC$ a punkt $Q$ na boku $DC$, przy czym $|PC|=|DQ|=x$. Dla jakiego $x$ pole

trójkąta jest najmniejsze?

6. $\mathrm{W}$ ostrosłupie prawidłowym trójkątnym ściana boczna jest nachylona do podstawy pod

$\mathrm{k}_{\Phi}\mathrm{t}\mathrm{e}\mathrm{m}\alpha$. Wyznacz kąt między ścianami bocznymi.

Rozwiązania (rękopis) zadań z wybranego poziomu prosimy nadsyfač do

na adres:

28 września 2020r.

Wydziaf Matematyki

Politechnika Wrocfawska

Wybrzez $\mathrm{e}$ Wyspiańskiego 27

$50-370$ WROCLAW.

Na kopercie prosimy $\underline{\mathrm{k}\mathrm{o}\mathrm{n}\mathrm{i}\mathrm{e}\mathrm{c}\mathrm{z}\mathrm{n}\mathrm{i}\mathrm{e}}$ zaznaczyč wybrany poziom! (np. poziom podsta-

wowy lub rozszerzony). Do rozwiązań nalez $\mathrm{y}$ dołaczyč zaadresowaną do siebie kopertę

zwrotną $\mathrm{z}$ naklejonym znaczkiem, odpowiednim do formatu listu. Polecamy stosowanie

kopert formatu C5 $(160\mathrm{x}230\mathrm{m}\mathrm{m})$ ze znaczkiem $0$ wartości 3,30 zł. Na $\mathrm{k}\mathrm{a}\dot{\mathrm{z}}$ dą większą

kopertę nalez $\mathrm{y}$ nakleič $\mathrm{d}\mathrm{r}\mathrm{o}\dot{\mathrm{z}}$ szy znaczek. Prace niespełniające podanych warunków nie

bedą poprawiane ani odsyłane.

Uwaga. Wysyfajac nam rozwiazania zadań uczestnik Kursu udostępnia Politechnice Wrocfawskiej

swoje dane osobowe, które przetwarzamy wyłącznie $\mathrm{w}$ zakresie niezbędnym do jego prowadzenia

(odeslanie zadań, prowadzenie statystyki). Szczegófowe informacje $0$ przetwarzaniu przez nas danych

osobowych $\mathrm{S}\otimes$ dostępne na stronie internetowej Kursu.

Adres internetowy Kursu: http: //www. im. pwr. edu. pl/kurs







LI KORESPONDENCYJNY KURS

Z MATEMATYKI

wrzesień 2021 r.

PRACA KONTROLNA nr l- POZIOM PODSTAWOWY

l. Wykaz$\cdot, \dot{\mathrm{z}}\mathrm{e}$ róznica kwadratów dwóch liczb nieparzystych jest podzielna przez 8.

2. Określ dziedzinę wyrazenia $w(x,y)= [\displaystyle \frac{\sqrt{x}+\sqrt{y}}{\sqrt{x}-\sqrt{y}}-\frac{4\sqrt{x}\sqrt{y}}{x-y}]$ : $[\displaystyle \frac{1}{\sqrt{x}+\sqrt{y}}-\frac{1}{x-y}]$

Sprowad $\acute{\mathrm{z}}$ je do najprostszej postaci $\mathrm{i}$ oblicz $w(3+2\sqrt{2},3-2\sqrt{2}).$

3. Dwie druzyny harcerskie postanowiły zebrač dla ogrodu zoologicznego określoną ilośč

$\dot{\mathrm{z}}$ ofędzi. Pierwsza $\mathrm{z}$ nich rozpoczęfa pracę póltora dnia wcześniej. $\mathrm{W}\mathrm{c}\mathrm{i}_{\Phi \mathrm{g}}\mathrm{u}$ siedmiu na-

stępnych dni pracowały razem $\mathrm{i}$ zebrały zaplanowaną ilośč $\dot{\mathrm{z}}$ olędzi. Gdyby $\mathrm{k}\mathrm{a}\dot{\mathrm{z}}$ da $\mathrm{z}$ druzyn

pracowafa oddzielnie, to druga wykonalaby calą pracę $03$ dni wcześniej od pierwszej.

Ile dni potrzebuje $\mathrm{k}\mathrm{a}\dot{\mathrm{z}}$ da $\mathrm{z}$ druzyn na zebranie tej ilości $\dot{\mathrm{z}}$ ołędzi?

4. Wyznacz wartości wszystkich funkcji trygonometrycznych kata ostrego $\alpha$, wiedząc, $\dot{\mathrm{z}}\mathrm{e}$

spefnione jest równanie

$\displaystyle \frac{2\sin\alpha+3\cos\alpha}{\cos\alpha}=2$ ctg $\alpha.$

5. Funkcja liniowa $f(x)=ax+b$ spelnia warunek $f(5)-f(3)=4$. Wyznaczjej wzór, wiedząc,

$\dot{\mathrm{z}}\mathrm{e}$ pole obszaru ograniczonego wykresami funkcji $g(x)=a|x|-b$ oraz $h(x)=-a|x|+b$

jest równe 16. $\mathrm{s}_{\mathrm{P}^{\mathrm{o}\mathrm{r}\mathrm{z}}\Phi^{\mathrm{d}\acute{\mathrm{z}}\mathrm{r}\mathrm{y}\mathrm{s}\mathrm{u}\mathrm{n}\mathrm{e}\mathrm{k}}}.$

6. Niech $A= \{(x,y):|x|\leq 2,|y|\leq 2\}$ oraz $B_{p}= \{(x,y):|x|+|y|\leq p\}$ dla $p> 2.$

Narysuj $\mathrm{w}$ jednym układzie współrzędnych zbiory A $\mathrm{i}B_{3}$. Oblicz pole zbiorów $A\cap B_{3}$

$\mathrm{i} A\cup B_{3}$. Dla jakiego $p$ zbiór $A\cap B_{p}$ jest wielokątem foremnym?




PRACA KONTROLNA $\mathrm{n}\mathrm{r} 1 -$ POZIOM ROZSZERZONY

l. Wykaz, $\dot{\mathrm{z}}$ ejezeli $p$ jest liczbą pierwszą większ$\Phi \mathrm{n}\mathrm{i}\dot{\mathrm{z}}3$, to jej czwarta potęga pomniejszona

$01$ jest wielokrotnością 48.

2. Określ dziedzinę wyrazenia:

$w(x,y)=(\displaystyle \frac{\sqrt[6]{y}}{\sqrt{y}-\sqrt[6]{x^{3}y^{2}}}-\frac{x}{\sqrt{xy}-x\sqrt[3]{y}})[\frac{1}{\sqrt{x}-\sqrt{y}}(\sqrt[6]{x^{5}}-\frac{y}{\sqrt[6]{x}})-\frac{x-y}{\sqrt[3]{x^{2}}+\sqrt[6]{x}\sqrt{y}}]$

$\mathrm{i}$ sprowad $\acute{\mathrm{z}}$ je do najprostszej postaci. Oblicz $w(7+5\sqrt{2},-7+5\sqrt{2}).$

Wskazówka: Oblicz najpierw $(\sqrt{2}+1)^{3}$

3. Trzech informatyków podjęfo się naprawy awarii $\mathrm{d}\mathrm{u}\dot{\mathrm{z}}$ ego systemu komputerowego. $\mathrm{Z}$ wcze-

śniejszych doświadczeń wiadomo, $\dot{\mathrm{z}}\mathrm{e}$ pierwszy $\mathrm{z}$ nich potrzebowałby na realizację tego

zlecenia 4 godziny więcej $\mathrm{n}\mathrm{i}\dot{\mathrm{z}}$ drugi, a trzeci pracowafby nad nim dwa razy krócej $\mathrm{n}\mathrm{i}\dot{\mathrm{z}}$

pierwszy. $\mathrm{W}$ jakim czasie wykonałby to zadanie $\mathrm{k}\mathrm{a}\dot{\mathrm{z}}\mathrm{d}\mathrm{y}\mathrm{z}$ informatyków, $\mathrm{j}\mathrm{e}\dot{\mathrm{z}}$ eli wiadomo,

$\dot{\mathrm{z}}\mathrm{e}$, pracując razem, naprawili awarię $\mathrm{w}$ ciągu 2 godzin $\mathrm{i}40$ minut?

4. Wyznacz wartości wszystkich funkcji trygonometrycznych $\mathrm{k}_{\Phi^{\mathrm{t}\mathrm{a}}}\alpha \in (\displaystyle \frac{\pi}{2},\pi)$, wiedząc,

$\dot{\mathrm{z}}\mathrm{e}$ spełnione jest równanie

3 $\displaystyle \cos\alpha-\frac{1}{\cos\alpha}=\sin\alpha.$

5. Dla jakich wartości parametru rzeczywistego $m$ wielomian

$w(x)=2x^{3}-(2+m)x^{2}+(2m+2)x-m-2$

ma trzy parami rózne pierwiastki rzeczywiste $x_{1}, x_{2}, x_{3}$, których suma odwrotności jest

nieujemna? $\mathrm{s}_{\mathrm{P}^{\mathrm{o}\mathrm{r}\mathrm{z}}\Phi^{\mathrm{d}\acute{\mathrm{z}}}}$ wykres funkcji $f(m)=\displaystyle \frac{1}{x_{1}}+\frac{1}{x_{2}}+\frac{1}{x_{3}}.$

6. Niech $A = \{(x,y):\sqrt{3}|x|+|y|\leq\sqrt{3}\}, B = \{(x,y):(|x|-1)^{2}+y^{2}\leq 1\}$ oraz

$C=\{(x,y):x^{2}+(|y|-\sqrt{3})^{2}\leq 1\}$. Narysuj $\mathrm{w}$ jednym ukfadzie wspófrzędnych zbiory

$A, B\mathrm{i}C$. Oblicz pole zbioru $A\backslash (B\cup C).$

$\mathrm{R}\mathrm{o}\mathrm{z}\mathrm{w}\mathrm{i}_{\Phi}$zania (rękopis) zadań $\mathrm{z}$ wybranego poziomu prosimy nadsyfač do 28.$09.2021\mathrm{r}.$

adres:

na

Wydziaf Matematyki

Politechnika Wrocfawska

Wybrzeže Wyspiańskiego 27

$50-370$ WROCLAW.

Na kopercie prosimy $\underline{\mathrm{k}\mathrm{o}\mathrm{n}\mathrm{i}\mathrm{e}\mathrm{c}\mathrm{z}\mathrm{n}\mathrm{i}\mathrm{e}}$ zaznaczyč wybrany poziom! (np. poziom podsta-

wowy lub rozszerzony). Do rozwiązań nalez $\mathrm{y}$ dołączyč zaadresowaną do siebie koperte

zwrotną $\mathrm{z}$ naklejonym znaczkiem, odpowiednim do formatu listu. Polecamy stosowanie

kopert formatu C5 $(160\mathrm{x}230\mathrm{m}\mathrm{m})$ ze znaczkiem $0$ wartości 3,30 zł. Na $\mathrm{k}\mathrm{a}\dot{\mathrm{z}}$ dą wiekszą

kopertę nalez $\mathrm{y}$ nakleič $\mathrm{d}\mathrm{r}\mathrm{o}\dot{\mathrm{z}}$ szy znaczek. Prace niespełniające podanych warunków nie

będą poprawiane ani odsyłane.

Uwaga. Wysylajac nam rozwiazania zadań uczestnik Kursu udostępnia Politechnice Wroclawskiej

swoje dane osobowe, które przetwarzamy wyłącznie $\mathrm{w}$ zakresie niezbednym do jego prowadzenia

(odesfanie zadań, prowadzenie statystyki). Szczegófowe informacje $0$ przetwarzaniu przez nas danych

osobowych są dostępne na stronie internetowej Kursu.

Adres internetowy Kursu: http: //www. im. pwr. edu. pl/kurs







LII

KORESPONDENCYJNY KURS

Z MATEMATYKI

wrzesień 2022 r.

PRACA KONTROLNA $\mathrm{n}\mathrm{r} 1-$ POZIOM PODSTAWOWY

l. Uprośč wyrazenie

$\displaystyle \frac{x^{-1}-a^{-1}}{a^{-1}-b(ax)^{-1}},$

jeśli

$x=\displaystyle \frac{1}{(\alpha+b)^{-1}}- (\displaystyle \frac{a+b}{a^{2}+b^{2}})^{-1}$

2. $\mathrm{W}$ jakim stosunku nalez $\mathrm{y}$ zmieszač dwa roztwory cukru $0$ stęzeniach 5\% oraz 23\%, aby

otrzymač roztwór 17\%?

3. Rozwiąz nierównośč

$x-|5x-2|<0.$

4. Dla jakich wartości parametru $a$ nierównośč

$(a^{2}-1)x^{2}+2(a-1)x+2>0$

jest spełniona dla $\mathrm{k}\mathrm{a}\dot{\mathrm{z}}$ dego $x\in \mathbb{R}$?

5. $\mathrm{W}\mathrm{i}\mathrm{e}\mathrm{d}\mathrm{z}\Phi^{\mathrm{C}}, \dot{\mathrm{z}}\mathrm{e}1\mathrm{i}3$ są pierwiastkami równania

$x^{3}+mx^{2}+23x+n=0,$

oblicz $m, n\mathrm{i}$ wyznacz trzeci pierwiastek równania.

6. Narysuj wykres funkcji

$f(x)=$

dla

dla

$|2x-2|\leq 4,$

$|2x-2|>4.$

Wykorzystuj $\otimes \mathrm{C}$ wykres, wyznacz zbiór wartości funkcji $f(x)$ oraz $\mathrm{n}\mathrm{a}\mathrm{j}\mathrm{m}\mathrm{n}\mathrm{i}\mathrm{e}\mathrm{j}_{\mathrm{S}\mathrm{Z}\Phi}\mathrm{i}$ najwięk-

szą wartośč funkcji $\mathrm{w}$ przedziale $[0$, 4$].$




PRACA KONTROLNA $\mathrm{n}\mathrm{r} 1 -$ POZIOM ROZSZERZONY

l. Dla jakich wartości parametru $a$ równanie

$2x^{2}-ax+a+2=0$

ma pierwiastki spefniające warunek $|x_{2}-x_{1}|=1$?

2. $\mathrm{W}$ sali ustawiono krzesla $\mathrm{i}$ trzyosobowe ławki, $\mathrm{w}$ lącznej liczbie 268. Do sa1i weszło 480

osób. Po zajeciu wszystkich miejsc siedzących proporcja osób stojących do siedzących

okazafa się większa $\displaystyle \mathrm{n}\mathrm{i}\dot{\mathrm{z}}\frac{39}{160}$, ale mniejsza $\displaystyle \mathrm{n}\mathrm{i}\dot{\mathrm{z}}\frac{41}{160}$. Ile fawek $\mathrm{i}$ ile krzesel bylo $\mathrm{w}$ sali?

3. Rozwiąz nierównośč

$|||||x|-1|-2|-1|-2|\leq 3.$

4. Oblicz

$x^{4}+y^{4}+z^{4},$

jeśli $x+y+z=0$

oraz

$x^{2}+y^{2}+z^{2}=3.$

5. Rozwiąz układ równań

$\left\{\begin{array}{l}
x-|y+1|=1,\\
x^{2}+y=10.
\end{array}\right.$

Podaj jego interpretację geometryczną (narysuj starannie obie dane powyzszymi równa-

niami krzywe).

6. Wyznacz wartości parametru $p$, dla których równanie

$(p-1)x^{4}-2(p+4)x^{2}+p=0$

ma cztery pierwiastki rózne od 0.

$\mathrm{R}\mathrm{o}\mathrm{z}\mathrm{w}\mathrm{i}_{\Phi}$zania (rękopis) zadań $\mathrm{z}$ wybranego poziomu prosimy nadsyfač do $28.09.2022\mathrm{r}$. na

adres:

Wydziaf Matematyki

Politechnika Wrocfawska

Wybrzez $\mathrm{e}$ Wyspiańskiego 27

$50-370$ WROCLAW.

lub elektronicznie, za pośrednictwem portalu talent. $\mathrm{p}\mathrm{w}\mathrm{r}$. edu. pl

Na kopercie prosimy $\underline{\mathrm{k}\mathrm{o}\mathrm{n}\mathrm{i}\mathrm{e}\mathrm{c}\mathrm{z}\mathrm{n}\mathrm{i}\mathrm{e}}$ zaznaczyč wybrany poziom! (np. poziom podsta-

wowy lub rozszerzony). Do rozwiązań nalez $\mathrm{y}$ dołaczyč zaadresowaną do siebie koperte

zwrotną $\mathrm{z}$ naklejonym znaczkiem, odpowiednim do formatu listu. Prace niespełniające

podanych warunków nie będą poprawiane ani odsyłane.

Uwaga. Wysyfajac nam rozwiazania zadań uczestnik Kursu udostępnia Politechnice Wroclawskiej

swoje dane osobowe, które przetwarzamy wyłącznie $\mathrm{w}$ zakresie niezbędnym do jego prowadzenia

(odeslanie zadań, prowadzenie statystyki). Szczegófowe informacje $0$ przetwarzaniu przez nas danych

osobowych $\mathrm{S}\otimes$ dostępne na stronie internetowej Kursu.

Adres internetowy Kursu: http: //www. im. pwr. edu. pl/kurs







KORESPONDENCYJNY KURS Z MATEMATYKI

PRACA KONTROLNA nr l

$\mathrm{p}\mathrm{a}\acute{\mathrm{z}}$dziernik 2$000\mathrm{r}$

l. Suma wszystkich wyrazów nieskończonego ciągu geometrycznego wynosi 2040. Jeś1i

pierwszy wyraz tego ciągu zmniejszymy $0172$, a jego iloraz zwiększymy 3-krotnie,

to suma wszystkich wyrazów tak otrzymanego ciągu wyniesie 2000. Wyznaczyč

iloraz $\mathrm{i}$ pierwszy wyraz danego ciągu.

2. Obliczyč wszystkie te skfadniki rozwinięcia dwumianu $(\sqrt{3}+\sqrt[3]{2})^{11}$, które są

liczbami całkowitymi.

3. Wykonač staranny wykres funkcji

$f(x)=|x^{2}-2|x|-3|$

i na jego podstawie podač ekstrema lokalne oraz przedzialy monotoniczności tej

funkcji.

4. Rozwiązač nierównośč

$x+1\geq\log_{2}(4^{x}-8).$

5. $\mathrm{W}$ ostrosfupie prawidfowym trójkątnym krawędz/ podstawy ma dfugośč $a$, a polowa

kąta płaskiego przy wierzchołku jest równa kątowi nachylenia ściany bocznej do

podstawy. Obliczyč objętośč ostroslupa. Sporządzič odpowiednie rysunki.

6. Znalez/č wszystkie wartości parametru $p$, dla których trójk$\Phi$t KLM $0$ wierzchofkach

$\mathrm{K}(1,1), \mathrm{L}(5,0)\mathrm{i}\mathrm{M}(\mathrm{p},\mathrm{p}-1)$ jest prostokątny. Rozwiązanie zilustrowač rysunkiem.

7. Rozwiązač równanie

$\sin 5x\sin 4x$

$\overline{\sin 3x}^{=}\overline{\sin 6x}.$

8. Przez punkt $P$ lezący wewnątrz trójkąta $ABC$ poprowadzono proste równolegle

do wszystkich boków trójkąta. Pola utworzonych $\mathrm{w}$ ten sposób trzech mniejszych

trójkatów $0$ wspólnym wierzchołku $P$ wynosza $S_{1}, S_{2}, S_{3}$. Obliczyč pole $S$

trójkąta $ABC.$

1




PRACA KONTROLNA nr 2

listopad 2000 $\mathrm{r}$

l. Promień kuli zwiększono $\mathrm{t}\mathrm{a}\mathrm{k}, \dot{\mathrm{z}}\mathrm{e}$ pole jej powierzchni wzrosło $0$ 44\%. $\mathrm{O}$ ile procent

wzrosfa jej objętośč?

2. Wyznaczyč równanie krzywej utworzonej przez środki odcinków majqcych obydwa

końce na osiach ukfadu wspófrzędnych $\mathrm{i}$ zawierających punkt $\mathrm{P}(2,1)$. Sporządzič

dokładny wykres $\mathrm{i}$ podač nazwę otrzymanej krzywej.

3. Znalez$\acute{}$č wszystkie wartości parametru $m$, dla których równanie

$(m-1)9^{x}-4\cdot 3^{x}+m+2=0$

ma dwa rózne rozwiazania.

4. Róznica promienia kuli opisanej na czworościanie foremnym $\mathrm{i}$ promienia kuli wpi-

sanej $\mathrm{w}$ niego jest równa l. Obliczyč objętośč tego czworościanu.

5. Rozwiązač nierównośč

$\displaystyle \frac{2}{|x^{2}-9|}\geq\frac{1}{x+3}$

6. Stosunek dfugości $\mathrm{p}\mathrm{r}\mathrm{z}\mathrm{y}\mathrm{p}\mathrm{r}\mathrm{o}\mathrm{s}\mathrm{t}\mathrm{o}\mathrm{k}_{\Phi^{\mathrm{t}}}$nych trójkąta $\mathrm{P}^{\mathrm{r}\mathrm{o}\mathrm{s}\mathrm{t}\mathrm{o}\mathrm{k}}\Phi^{\mathrm{t}\mathrm{n}\mathrm{e}\mathrm{g}\mathrm{o}}$ wynosi $k$. Obliczyč sto-

sunek dlugości dwusiecznych kątów ostrych tego trójkąta. $\mathrm{U}\dot{\mathrm{z}}$ yč odpowiednich wzo-

rów trygonometrycznych.

7. Zbadač przebieg zmienności funkcji

$f(x)=\displaystyle \frac{x^{2}+4}{(x-2)^{2}}$

$\mathrm{i}$ wykonač jej staranny wykres.

8. Wyznaczyč równania wszystkich prostych stycznych do wykresu funkcji $f(x) =$

$x^{3}-2x\mathrm{i}$ przechodzących przez punkt $A(\displaystyle \frac{7}{5},-2)$. Wykonač odpowiedni rysunek.

2





PRACA KONTROLNA nr 3

grudzień 2000 $\mathrm{r}$

l. Stosując zasadę indukcji matematycznej udowodnič, $\dot{\mathrm{z}}\mathrm{e}$ dla $\mathrm{k}\mathrm{a}\dot{\mathrm{z}}$ dej liczby naturalnej

$n$ suma $2^{n+1}+3^{2n-1}$ jest podzielna przez 7.

2. Tworząca stozka ma długośč $l\mathrm{i}$ widač ją ze środka kuli wpisanej $\mathrm{w}$ ten stozek pod

$\mathrm{k}_{\Phi}\mathrm{t}\mathrm{e}\mathrm{m}\alpha$. Obliczyč objętośč $\mathrm{i}$ kąt rozwarcia stozka. Określič dziedzinę dla kąta $\alpha.$

3. Nie korzystajqc $\mathrm{z}$ metod rachunku rózniczkowego wyznaczyč dziedzinę $\mathrm{i}$ zbiór war-

tości funkcji

$y=\sqrt{2+\sqrt{x}-x}.$

4. $\mathrm{Z}$ talii 24 kart wy1osowano (bez zwracania) cztery karty. Jakie jest prawdopodobień-

stwo, $\dot{\mathrm{z}}\mathrm{e}$ otrzymano dokładnie trzy karty $\mathrm{z}$ jednego koloru ($\mathrm{z}$ czterech $\mathrm{m}\mathrm{o}\dot{\mathrm{z}}$ liwych)?

5. Rozwiązač nierównośč

$\log_{1/3}$ (log2 $4x$)$\geq\log_{1/3}(2-\log_{2x}4)-1.$

6. $\mathrm{Z}$ punktu $C(1,0)$ poprowadzono styczne do okręgu $x^{2}+y^{2} =r^{2}, r \in (0,1).$

Punkty styczności oznaczono przez A $\mathrm{i}B$. Wyrazič pole trójkąta ABC jako funkcję

promienia $r\mathrm{i}$ znalez/č największą wartośč tego pola.

7. Rozwiązač ukfad równań

$\left\{\begin{array}{l}
x^{2}+y^{2}\\
|4y-3x+10|
\end{array}\right.$

$=5|x|$

$=10$

Podač interpretację geometryczną $\mathrm{k}\mathrm{a}\dot{\mathrm{z}}$ dego $\mathrm{z}$ równań $\mathrm{i}$ wykonač staranny rysunek.

8. Rozwiązač $\mathrm{w}$ przedziale $[0,\pi]$ równanie

1$+ \sin 2x=2\sin^{2}x,$

a następnie nierównośč $1+\sin 2x>2\sin^{2}x.$

3





PRACA KONTROLNA nr 4

styczeń 2001 $\mathrm{r}$

$\mathrm{W}$ celu przyblizenia sfuchaczom Kursu, jakie wymagania były stawiane ich starszym

kolegom przed ponad dwudziestu laty, niniejszy zestaw zadań jest dokladnym powtórze-

niem pracy kontrolnej ze stycznia 1979 $\mathrm{r}.$

l. Przez środek boku trójk$\Phi$ta równobocznego przeprowadzono prostą, $\mathrm{t}\mathrm{w}\mathrm{o}\mathrm{r}\mathrm{z}\text{ą}^{\mathrm{C}}\Phi \mathrm{z}$ tym

bokiem kąt ostry $\alpha \mathrm{i}$ dzielącą ten trójkąt na dwie figury, których stosunek pól jest

równy 1 : 7. Ob1iczyč miarę kata $\alpha.$

2. $\mathrm{W}$ kulę $0$ promieniu $R$ wpisano graniastosłup trójkątny prawidfowy $0$ krawędzi pod-

stawy równej $R$. Obliczyč wysokośč tego graniastoslupa.

3. Wyznaczyč wartości parametru $a$, dla których funkcja $f(x) = \displaystyle \frac{ax}{1+x^{2}}$ osiąga maksi-

mum równe 2.

4. Rozwiązač nierównośč

$\cos^{2}x+\cos^{3}x+\ldots+\cos^{n+1}x+\ldots<1+\cos x$

dla $x\in[0,2\pi].$

5. Wykazač, $\dot{\mathrm{z}}\mathrm{e}$ dla $\mathrm{k}\mathrm{a}\dot{\mathrm{z}}$ dej liczby naturalnej $n\geq 2$ prawdziwa jest równośč

$1^{2}+2^{2}+\ldots+n^{2}= \left(\begin{array}{lll}
n & + & 1\\
 & 2 & 
\end{array}\right)+2[\left(\begin{array}{l}
n\\
2
\end{array}\right)+\left(\begin{array}{ll}
n & -1\\
 & 2
\end{array}\right)+\ldots+ \left(\begin{array}{l}
2\\
2
\end{array}\right)]$

6. Wyznaczyč równanie linii bedącej zbiorem środków wszystkich okręgów stycznych

do prostej $y=0\mathrm{i}$ jednocześnie stycznych zewnętrznie do okręgu $(x+2)^{2}+y^{2}=4.$

Narysowač tę linię.

7. Wyznaczyč wartości parametru $m$, dla których równanie $9x^{2}-3x\log_{3}m+1 =0$

ma dwa rózne pierwiastki rzeczywiste $x_{1}, x_{2}$ spelniające warunek $x_{1}^{2}+x_{2}^{2}=1.$

8. Rozwiązač nierównośč

$\displaystyle \frac{\sqrt{30+x-x^{2}}}{x}<\frac{\sqrt{10}}{5}.$

4





PRACA KONTROLNA nr 5

luty 2001 $\mathrm{r}$

l. Posfugując się odpowiednim wykresem wykazač, $\dot{\mathrm{z}}\mathrm{e}$ równanie

$\sqrt{x-3}+x=4$

posiada dokfadnie jedno $\mathrm{r}\mathrm{o}\mathrm{z}\mathrm{w}\mathrm{i}_{\Phi}$zanie. Następnie wyznaczyč to rozwiązanie anali-

tycznie.

2. Wiadomo, $\dot{\mathrm{z}}\mathrm{e}$ wielomian $w(x) = 3x^{3}-5x+1$ ma trzy pierwiastki rzeczywiste

$x_{1}, x_{2}, x_{3}$. Nie wyznaczając tych pierwiastków obliczyč wartośč wyrazenia

$(1+x_{1})(1+x_{2})(1+x_{3}).$

3. Rzucamy jeden raz kostką, a nastepnie monetą tyle razy, ile oczek pokazała kostka.

Obliczyč prawdopodobieństwo tego, $\dot{\mathrm{z}}\mathrm{e}$ rzuty monetą dały co najmniej jednego orfa.

4. Wyznaczyč równania wszystkich okręgów stycznych do obu osi układu współrzęd-

nych oraz do prostej $3x+4y=12.$

5. $\mathrm{W}$ ostrosfupie prawidlowym czworokątnym dana jest odlegfośč $d$ środka podstawy

od krawędzi bocznej oraz $\mathrm{k}\mathrm{a}\mathrm{t}  2\alpha$ miedzy sąsiednimi ścianami bocznymi. Obliczyč

objętośč ostrosfupa.

6. $\mathrm{W}$ trapezie równoramiennym $0$ polu $P$ dane są promień okręgu opisanego $r$ oraz

suma długości obu podstaw $s$. Obliczyč obwód tego trapezu. Podač warunki roz-

wiązalności zadania. Wykonač rysunek dla $P=12\mathrm{c}\mathrm{m}^{2}, r=3$ cm $\mathrm{i}s=8$ cm.

7. Rozwiązač uklad równań

$\left\{\begin{array}{l}
px\\
(p+2)x
\end{array}\right.$

$+$

$+$

{\it y}

{\it py}

$3p^{2}-3p-2$

$4p$

$\mathrm{w}$ zalezności od parametru rzeczywistego $p$. Podač wszystkie rozwiązania $(\mathrm{i}$ od-

powiadające im wartości parametru $p$), dla których obie niewiadome są liczbami

całkowitymi $0$ wartości bezwzględnej mniejszej od 3.

8. Odcinek $\overline{AB}\mathrm{o}$ końcach $A(0,\displaystyle \frac{3}{2}) \mathrm{i} B(1,y), y \in [0,\displaystyle \frac{3}{2}]$, obraca się wokól osi Ox.

Wyrazič pole powstałej powierzchni jako funkcje $y\mathrm{i}$ znalez/č najmniejszą wartośč

tego pola. Sporządzič rysunek.

5





PRACA KONTROLNA nr 6

marzec 2001 r

l. Wykazač, $\dot{\mathrm{z}}\mathrm{e}$ dla $\mathrm{k}\mathrm{a}\dot{\mathrm{z}}$ dego $\mathrm{k}_{\Phi^{\mathrm{t}\mathrm{a}}} \alpha$ prawdziwa jest nierównośč

$\sqrt{3}\sin\alpha+\sqrt{6}\cos\alpha\leq 3.$

2. Dane są punkty $A(2,2) \mathrm{i} B(-1,4)$. Wyznaczyč długośč rzutu prostopadłego

odcinka $\overline{AB}$ na prostą $0$ równaniu $12x+5y=30$. Sporz$\Phi$dzič rysunek.

3. Niech $f(m)$ będzie sumą odwrotności pierwiatków rzeczywistych równania kwadra-

towego $(2^{m}-7)x^{2}-2|2^{m}-4|x+2^{m}=0$, gdzie $m$ jest parametrem rzeczywistym.

Napisač wzór określający $f(m)\mathrm{i}$ narysowač wykres tej funkcji.

4. Dwóch strzelców strzela równocześnie do tego samego celu niezaleznie od siebie.

Pierwszy strzelec trafia za $\mathrm{k}\mathrm{a}\dot{\mathrm{z}}$ dym razem $\mathrm{z}$ prawdopodobieństwem $\displaystyle \frac{2}{3} \mathrm{i}$ oddaje 2

strzały, a drugi trafia $\mathrm{z}$ prawdopodobieństwem $\displaystyle \frac{1}{2} \mathrm{i}$ oddaje 5 strzałów. Ob1iczyč

prawdopodobieństwo, $\dot{\mathrm{z}}\mathrm{e}$ cel zostanie trafiony dokładnie 3 razy.

5. Liczby $a_{1}, a_{2}, a_{n},  n\geq 3$, tworzą ciąg arytmetyczny. Suma wyrazów tego $\mathrm{c}\mathrm{i}_{\Phi \mathrm{g}}\mathrm{u}$

wynosi 28, suma wyrazów $0$ numerach nieparzystych wynosi 16, a $a_{2}\cdot a_{3}=48.$

Wyznaczyč te liczby.

6. $\mathrm{W}$ trójkącie $ABC, \mathrm{w}$ którym $AB=7$ oraz $AC=9$, a kąt przy wierzchołku $A$ jest

dwa razy większy $\mathrm{n}\mathrm{i}\dot{\mathrm{z}}$ kąt przy wierzchołku $B$. Obliczyč stosunek promienia kola

wpisanego do promienia kola opisanego na tym trójk$\Phi$cie. Rozwiązanie zilustrowač

rysunkiem.

7. Zaznaczyč na p{\it l}aszczy $\acute{\mathrm{z}}\mathrm{n}\mathrm{i}\mathrm{e}$ nastepujące zbiory punktów:

$A=\{(x,y):x+y-2\geq|x-2|\},$

$B=\{(x,y):y\leq\sqrt{4x-x^{2}}\}.$

Następnie znalez/č na brzegu zbioru

$P(\displaystyle \frac{5}{2},1)$ jest najmniejsza.

$A\cap B$ punkt $\mathrm{Q}$, którego odleglośč od punktu

8. Przeprowadzič badanie przebiegu i sporządzič wykres funkcji

$f(x)=\displaystyle \frac{1}{2}x^{2}-4+\sqrt{8-x^{2}}.$

6





PRACA KONTROLNA nr 7

kwiecień 2001 $\mathrm{r}$

l. Ile elementów ma zbiór $A$, jeśli liczbajego podzbiorów trójelementowych jest większa

od liczby podzbiorów dwuelementowych $048$ ?

2. $\mathrm{W}$ sześciokqt foremny $0$ boku l wpisano okrąg. $\mathrm{W}$ otrzymany okrqg wpisano sześcio-

$\mathrm{k}_{\Phi^{\mathrm{t}}}$ foremny, $\mathrm{w}$ który znów wpisano okrąg, itd. Obliczyč sumę obwodów wszystkich

otrzymanych okręgów.

3. Dana jest rodzina prostych $0$ równaniach $2x+my-m-2=0,$

prostych tej rodziny są:

a) prostopadłe do prostej $x+4y+2=0,$

b) równoległe do prostej $3x+2y=0,$

c) tworzą $\mathrm{z}$ prostą $x-\sqrt{3}y-1=0$ kąt $\displaystyle \frac{\pi}{3}.$

$m\in R$. Które $\mathrm{z}$

4. Sprawdzič $\mathrm{t}\mathrm{o}\dot{\mathrm{z}}$ samośč: $tg(x-\displaystyle \frac{\pi}{4})-1=\frac{-2}{tgx+1}$. Korzystajqc $\mathrm{z}$ niej sporzadzič wykres

funkcji $f(x)=\displaystyle \frac{1}{tgx+1}\mathrm{w}$ przedziale $[0,\pi].$

5. Dany jest okrąg $K\mathrm{o}$ równaniu $x^{2}+y^{2}-6y=27$. Wyznaczyč równanie krzywej $\Gamma$

bedącej obrazem okręgu $K\mathrm{w}$ powinowactwie prostokątnym $0$ osi Ox $\mathrm{i}$ skali $k=\displaystyle \frac{1}{3}.$

Obliczyč pole figury lezącej ponizej osi odciętych $\mathrm{i}$ ograniczonej łukiem okręgu $\mathrm{K}\mathrm{i}$

$\mathrm{k}\mathrm{r}\mathrm{z}\mathrm{y}\mathrm{w}\Phi^{\Gamma}$. Wykonač rysunek.

6. Wykorzystując nierównośč $2\sqrt{ab}\leq a+b, a, b>0$, wyznaczyč granicę

$\displaystyle \lim_{n\rightarrow\infty}(\frac{\log_{5}16}{\log_{2}3})^{n}$

7. Trylogię skfadającą się $\mathrm{z}$ dwóch powieści dwutomowych oraz jednej jednotomowej

ustawiono przypadkowo na pólce. Jakie jest prawdopodobieństwo tego, $\dot{\mathrm{z}}\mathrm{e}$ tomy a)

obydwu, b) co najmniej jednej, $\mathrm{z}$ dwutomowych powieści znajdujq się obok siebie $\mathrm{i}$

przy tym tom I $\mathrm{z}$ lewej, a tom II $\mathrm{z}$ prawej strony.

8. $\mathrm{W}$ ostrosłupie prawidlowym czworokątnym krawęd $\acute{\mathrm{z}}$ bocznajest nachylona do plasz-

czyzny podstawy pod kqtem $\alpha$, a krawędz$\acute{}$ podstawy ma długośč $a$. Obliczyč pro-

mień kuli stycznej do wszystkich krawędzi tego ostrosfupa. Wykonač odpowiednie

rysunki.

7







KORESPONDENCYJNY KURS Z MATEMATYKI

PRACA KONTROLNA nr l

$\mathrm{p}\mathrm{a}\acute{\mathrm{z}}$dziernik 2$001\mathrm{r}$

l. Dwaj rowerzyści wyruszyli jednocześnie $\mathrm{w}$ drogę, jeden $\mathrm{z}$ A do $\mathrm{B}$, drugi $\mathrm{z}\mathrm{B}$ do $\mathrm{A}$

$\mathrm{i}$ spotkali się po jednej godzinie. Pierwszy $\mathrm{z}$ nich przebywał $\mathrm{w}$ ciągu godziny $03$

km więcej $\mathrm{n}\mathrm{i}\dot{\mathrm{z}}$ drugi $\mathrm{i}$ przyjechal do celu $027$ minut wcześniej $\mathrm{n}\mathrm{i}\dot{\mathrm{z}}$ drugi. Jakie byfy

prędkości obu rowerzystów $\mathrm{i}$ jaka jest odległośč AB?

2. Rozwiązač nierównośč:

$\sqrt{x^{2}-3}>\underline{2}.$

$x$

3. Rysunek przedstawia dach budynku $\mathrm{w}$ rzucie poziomym. $\mathrm{K}\mathrm{a}\dot{\mathrm{z}}$ da $\mathrm{z}$ plaszczyzn nachy-
\begin{center}
\includegraphics[width=48.156mm,height=24.132mm]{./KursMatematyki_PolitechnikaWroclawska_2001_2002_page0_images/image001.eps}
\end{center}
lona jest do płaszczyzny poziomej pod $\mathrm{k}$ tem $30^{0}$ Dłu-

gossc dachu wynosi 18 $\mathrm{m}$, a szerokosc 9 $\mathrm{m}$. Obliczyc po-

le powierzchni dachu oraz cafkowit kubaturę strychu $\mathrm{w}$

tym budynku.

4. Pewna firma przeprowadza co kwartal regulację plac dla swoich pracowników rewa-

$1\mathrm{o}\mathrm{r}\mathrm{y}\mathrm{z}\mathrm{u}\mathrm{j}_{\Phi}\mathrm{c}$ je zgodnie ze wska $\acute{\mathrm{z}}\mathrm{n}\mathrm{i}\mathrm{k}\mathrm{i}\mathrm{e}\mathrm{m}$ inflacji, który jest stafy $\mathrm{i}$ wynosi 1,5\% kwar-

talnie, oraz doliczając stałą kwotę podwyzki 16 $\mathrm{z}\mathrm{l}\mathrm{p}. \mathrm{W}$ styczniu 2001 pan Kowa1ski

otrzymał wynagrodzenie 1600 $\mathrm{z}\mathrm{l}\mathrm{p}$. Jaką pensję otrzyma $\mathrm{w}$ kwietniu 2002? Wyzna-

czyč wzór ogólny na pensję $w_{n}$ pana Kowalskiego $\mathrm{w}\mathrm{n}$-tym kwartale przyjmując, $\dot{\mathrm{z}}\mathrm{e}$

$w_{1}=1600$jest placą $\mathrm{w}$ pierwszym kwartale 2001. Ob1iczyč średniq miesięcznq płacę

pana Kowalskiego $\mathrm{w}$ 2002 roku.

5. Wyznaczyč funkcję odwrotną do $f(x) =x^{3}, x\in R$. Korzystając $\mathrm{z}$ tego wykonač

staranny wykres funkcji $h(x) =$
\begin{center}
\includegraphics[width=36.732mm,height=7.620mm]{./KursMatematyki_PolitechnikaWroclawska_2001_2002_page0_images/image002.eps}
\end{center}
$(|x|-- 1)+1.$

6. Rozwiązač równanie:

-cs  oins 24{\it xx} $=$ 1.

7. Dany jest trójkąt $0$ wierzcholkach $A(-2,1), B(-1,-6), C(2,5)$. Posługując się

rachunkiem wektorowym obliczyč cosinus kąta pomiędzy dwusieczną kata $A\mathrm{i}$ środ-

kową boku $\overline{BC}$. Wykonač rysunek.

8. Przeprowadzič badanie przebiegu $\mathrm{i}$ wykonač wykres funkcji

$f(x)=x+\displaystyle \frac{x}{x-1}+\frac{x}{(x-1)^{2}}+\frac{x}{(x-1)^{3}}+$




PRACA KONTROLNA nr 2

listopad $2001\mathrm{r}$

l. Cena ll paliwa została zmniejszona $0$ 15\%. Po dwóch tygodniach dokonano kolej-

nej zmiany ceny paliwa zwiększając ją $0$ 15\%. $\mathrm{O}$ ile procent końcowa cena paliwa

rózni $\mathrm{s}\mathrm{i}_{9}$ od początkowej?

2. Wyznaczyč $\mathrm{i}$ narysowač zbiór złozony $\mathrm{z}$ punktów $(x,y)$ płaszczyzny spełniających

warunek

$x^{2}+y^{2}=8|x|+6|y|.$

3. Wysokośč ostrosfupa trójkątnego prawidfowego wynosi $h$, a kąt między wysokościa-

mi ścian bocznych jest równy $ 2\alpha$. Obliczyč pole powierzchni bocznej tego ostrosłupa.

Sporządzič odpowiednie rysunki.

4. $\mathrm{Z}$ arkusza blachy $\mathrm{w}$ kształcie równoległoboku $0$ bokach 30 cm $\mathrm{i}60$ cm $\mathrm{i}$ kącie ostrym

$60^{0}$ nalezy odciąč dwa przeciwlegfe trójkqtne naroza $\mathrm{t}\mathrm{a}\mathrm{k}$, aby powstaf romb $0\mathrm{m}\mathrm{o}\dot{\mathrm{z}}$-

liwie największym polu. Określič przez który punkt dfuzszego boku nalez $\mathrm{y}$ prze-

prowadzič cięcie oraz obliczyč kąt ostry otrzymanego rombu zaokrqglajqc wynik do

jednej minuty kątowej.

5. Rozwiązač równanie

$2^{\log_{\sqrt{2}}x}=(\sqrt{2})^{\log_{x}2}$

6. Wyznaczyč dziedzinę i zbiór wartości funkcji

$f(x)=\displaystyle \frac{4}{\sin x+2\cos x+3}.$

7. Znalez$\acute{}$č wszystkie wartości parametru $p$, dla których równanie

$px^{4}-4x^{2}+p+1=0$

ma dwa rózne rozwiązania.

8. Wyznaczyč tangens $\mathrm{k}_{\Phi^{\mathrm{t}\mathrm{a}}}$, pod którym styczna do wykresu funkcji $f(x) = \displaystyle \frac{8}{x^{2}+3} \mathrm{w}$

punkcie $A(3,\displaystyle \frac{2}{3})$ przecina wykres tej funkcji.





PRACA KONTROLNA nr 3

grudzień $2001\mathrm{r}$

l. Dla jakich wartości $\sin x$ liczby $\sin x, \cos x, \sin 2x$ ($\mathrm{w}$ podanym porządku) są ko-

lejnymi wyrazami ciągu geometrycznego. Wyznaczyč czwarte wyrazy tych ciągów.

2. $\mathrm{W}$ pewnych zawodach sportowych startuje 16 druzyn. $\mathrm{W}$ eliminacjach są one losowo

dzielone na 4 grupy po 4 druzyny $\mathrm{k}\mathrm{a}\dot{\mathrm{z}}$ da grupa. Obliczyč prawdopodobieństwo tego,

$\dot{\mathrm{z}}\mathrm{e}$ trzy zwycięskie druzyny $\mathrm{z}$ poprzednich zawodów $\mathrm{z}\mathrm{n}\mathrm{a}\mathrm{j}\mathrm{d}_{\Phi}$ się $\mathrm{k}\mathrm{a}\dot{\mathrm{z}}$ da $\mathrm{w}$ innej grupie.

3. Nie wykonując dzielenia udowodnič, $\dot{\mathrm{z}}\mathrm{e}$ wielomian $(x^{2}+x+1)^{3}-x^{6}-x^{3}-1$ dzieli

się bez reszty przez trójmian $(x+1)^{2}$

4. Wyznaczyč równanie okręgu $0$ promieniu $r$ stycznego do paraboli $y=x^{2}\mathrm{w}$ dwóch

punktach. Dla jakiego $r$ zadanie ma rozwiqzanie? Sporządzič rysunek przyjmujac

$r=3/2.$

5. Stosując zasadę indukcji matematycznej udowodnič prawdziwośč wzoru

$\left(\begin{array}{l}
2\\
2
\end{array}\right) - \left(\begin{array}{l}
3\\
2
\end{array}\right) + \left(\begin{array}{l}
4\\
2
\end{array}\right) - \left(\begin{array}{l}
5\\
2
\end{array}\right) +\ldots+\left(\begin{array}{l}
2n\\
2
\end{array}\right) =n^{2},$

$n\geq 1.$

6. Rozwiązač nierównośč:

$\log_{x}(1-6x^{2})\geq 1.$

7. Środek $S$ okręgu wpisanego $\mathrm{w}$ trapez ABCD jest odlegfy od wierzchofka $B\mathrm{o}SB=$

$\mathrm{d}$, a krótsze ramię $\overline{BC}$ ma dlugośč $BC = \mathrm{c}$. Punkt styczności okręgu $\mathrm{z}$ krótszą

podstawą dzieli ją $\mathrm{w}$ stosunku 1:2. Ob1iczyč po1e tego trapezu. Wykonač rysunek

dla $\mathrm{c}=5\mathrm{i}\mathrm{d}=4.$

8. Wszystkie ściany równoległościanu są rombami $0$ boku $a\mathrm{i}$ kącie ostrym $\beta$. Obliczyč

objętośč tego równoleglościanu. Sporz$\Phi$dzič rysunek. Obliczenia poprzeč stosownym

dowodem.





PRACA KONTROLNA nr 4

styczeń 2002r

l. Obliczyč granicę ciągu 0 wyrazie ogó1nym

{\it an} $=$ -2{\it n}22$++$22{\it n}$+$4 1$++$.. .. . $+.\ +$222{\it n}2{\it n}.

2. Wyznaczyč równanie prostej prostopadfej do danej $2x+3y+3=0\mathrm{i}\mathrm{l}\mathrm{e}\dot{\mathrm{z}}$ qcej $\mathrm{w}$ równej

odleglości od dwóch danych punktów $A(-1,1)\mathrm{i}B(3,3)$. Sporządzič rysunek.

3. Tworząca stozka ma dfugośč $l\mathrm{i}$ widač ją ze środka kuli wpisanej $\mathrm{w}$ ten stozek pod

kątem $\alpha$. Obliczyč objętośč $\mathrm{i}$ kąt rozwarcia stozka. Określič dziedzinę kąta $\alpha.$

4. Bolek kupil jeden długopis $\mathrm{i} k$ zeszytów $\mathrm{i}$ zapłacił $k\mathrm{z}l\mathrm{i}$ 50 gr, a Lolek kupil $k$

dlugopisów $\mathrm{i} 4$ zeszyty $\mathrm{i}$ zapfacif 2, $5k$ zł. Wyznaczyč cenę dfugopisu $\mathrm{i}$ zeszytu $\mathrm{w}$

zalezności od parametru $k$. Znalez/č wszystkie $\mathrm{m}\mathrm{o}\dot{\mathrm{z}}$ liwe wartości tych cen wiedzqc, $\dot{\mathrm{z}}\mathrm{e}$

zeszyt kosztuje nie mniej $\mathrm{n}\mathrm{i}\dot{\mathrm{z}} 50$ gr, długopis jest drozszy od zeszytu, a ceny obydwu

artykułów wyrazają się $\mathrm{w}$ pełnych złotych $\mathrm{i}$ dziesiątkach groszy.

5. Rozwiazač nierównośč:

$\mathrm{t}\mathrm{g}^{3}x\geq\sin 2x.$

6. $\dot{\mathrm{Z}}$ arówki są sprzedawane $\mathrm{w}$ opakowaniach po 6 sztuk. Prawdopodobieństwo, $\dot{\mathrm{z}}\mathrm{e}$ po-

jedyncza $\dot{\mathrm{z}}$ arówka jest sprawna wynosi $\displaystyle \frac{2}{3}$. Jakie jest prawdopodobieństwo tego, $\dot{\mathrm{z}}\mathrm{e}$

$\mathrm{w}$ jednym opakowaniu znajdą się co najmniej 4 sprawne $\dot{\mathrm{z}}$ arówki. $\mathrm{O}$ ile wzrośnie

to prawdopodobieństwo, jeśli jedna, wylosowana $\mathrm{z}$ opakowania $\dot{\mathrm{z}}$ arówka okazafa się

sprawna.

7. Prosta styczna $\mathrm{w}$ punkcie $P$ do okręgu $0$ promieniu 2 $\mathrm{i}$ pólprosta wychodząca ze

środka okręgu mająca $\mathrm{z}$ okręgiem punkt wspólny $S$ przecinają się $\mathrm{w}$ punkcie $A$ pod

kątem $60^{0}$ Znalez/č promień okręgu stycznego do odcinków $AP$, {\it AS} $\mathrm{i}$ łuku $PS.$

Wykonač odpowiedni rysunek.

8. $\mathrm{W}$ ostrosłupie prawidłowym, którego podstawą jest kwadrat, pole $\mathrm{k}\mathrm{a}\dot{\mathrm{z}}$ dej $\mathrm{z}$ pięciu

ścian wynosi l. Ostrosfup ten ścięto pfaszczyzną równolegfq do podstawy $\mathrm{t}\mathrm{a}\mathrm{k}$, aby

uzyskač maksymalny stosunek objętości do pola powierzchni cafkowitej. Obliczyč

pole powierzchni całkowitej otrzymanego ostrosłupa ściętego. Rozwiazanie zilustro-

wač rysunkiem.





PRACA KONTROLNA nr 5

luty $2002\mathrm{r}$

1. $\mathrm{W}$ czworokącie ABCD dane są wktory $AB=\rightarrow(2,-1), BC=\rightarrow(3,3), c^{\rightarrow}D=(-4,1).$

Punkty $K\mathrm{i}M$ są środkami boków $\overline{CD}$ oraz $\overline{AD}$. Posługując się rachunkiem wekto-

rowym obliczyč pole trójkata $KMB$. Wykonač rysunek.

2. Krawędzie oraz przekątna prostopadlościanu $\mathrm{t}\mathrm{w}\mathrm{o}\mathrm{r}\mathrm{z}\Phi$ cztery kolejne wyrazy ciągu

arytmetycznego. Wyznaczyč sumę długości wszystkich krawędzi tego prostopadło-

ścianu, jeśli przekątna ma dfugośč 7 cm.

3. Na pfaszczy $\acute{\mathrm{z}}\mathrm{n}\mathrm{i}\mathrm{e}$ Oxy dane są zbiory:

$A=\{(x,y):y\leq\sqrt{5x-x^{2}}\},B_{s}=\{(x,y):3x+4y=s\}.$

Dla jakich wartości parametru $s$ zbiór $A\cap B_{s}$ nie jest pusty? Sporządzič rysunek.

4. Działka gruntu ma kształt trapezu $0$ bokach 20 $\mathrm{m}, 30\mathrm{m}, 40\mathrm{m}\mathrm{i}60\mathrm{m}$. Właściciel

dziafki twierdzi, $\dot{\mathrm{z}}\mathrm{e}$ polejego dzialki wynosi ponad ll arów. Czy wfaściciel ma rację?

Jeśli tak, to narysowač plan działki $\mathrm{w}$ skali 1:1000 $\mathrm{i}$ podač dokladną wartośčjej pola.

5. Dane jest równanie kwadratowe $\mathrm{z}$ parametrem $m$:

$(m+2)x^{2}+4\sqrt{m}x+(m-3)=0.$

Dla jakiej wartości parametru $m$ kwadrat róznicy pierwiastków rzeczywistych tego

równania jest największy. Podač tę największą wartośč.

6. Stosując zasadę indukcji matematycznej udowodnič, $\dot{\mathrm{z}}\mathrm{e}$ dla $\mathrm{k}\mathrm{a}\dot{\mathrm{z}}$ dego $n \geq 2$ liczba

$2^{2^{n}}-6$ jest podzielna przez 10.

7. Rozwiązač uklad równań

$\left\{\begin{array}{l}
\mathrm{t}\mathrm{g}x+\mathrm{t}\mathrm{g}y=4\\
\cos(x+y)+\cos(x-y)=\frac{1}{2}
\end{array}\right.$

dla $x, y\in[-\pi,\pi].$

8. Równoramienny trójkqt prostokqtny $ABC$ zgięto wzdłuz środkowej $\overline{CD}$ wychodzą-

cej $\mathrm{z}$ wierzchofka kąta prostego $C\mathrm{t}\mathrm{a}\mathrm{k}$, aby obie pofowy tego trójk$\Phi$ta utworzyfy

kąt $60^{0}$ Obliczyč sinusy wszystkich kątów dwuściennych otrzymanego czworościanu

ABCD. Wykonač odpowiednie rysunki $\mathrm{i}$ uzasadnič obliczenia.





PRACA KONTROLNA nr 6

marzec 2002r

l. Wyznaczyč wszystkie wartości parametru rzeczywistego $m$, dla których osią symetrii

wykresu funkcji $p(x)=(m^{2}-2m)x^{2}-(2m-4)x+3$ jest prosta $x=m$. Wykonač

rysunek.

2. $\mathrm{Z}$ kuli $0$ środku $\mathrm{w}$ zerze $\mathrm{i}$ promieniu $R$ wycięto ósmą jej częśč trzema płaszczyznami

ukfadu wspófrzędnych. $\mathrm{W}$ tak $\mathrm{o}\mathrm{t}\mathrm{r}\mathrm{z}\mathrm{y}\mathrm{m}\mathrm{a}\mathrm{n}\Phi$ bryfę wpisano kulę. Obliczyč stosunek

pola powierzchni tej kuli do pola powierzchni bryły.

3. $\mathrm{W}$ trzech pustych urnach $K, \mathrm{L}, \mathrm{M}$ rozmieszczamy losowo 4 rózne ku1e. Ob1iczyč

prawdopodobieństwo tego, $\dot{\mathrm{z}}\mathrm{e}\dot{\mathrm{z}}$ adna $\mathrm{z}$ urn $K\mathrm{i}\mathrm{L}$ nie pozostanie pusta.

4. Dane sa punkty $A(2,6), B(-2,6)\mathrm{i}C(0,0)$, Wyznaczyč równanie linii zawierajacej

wszystkie punkty trójkąta $ABC$, dla których suma kwadratów ich odlegfości od

trzech boków jest stala $\mathrm{i}$ wynosi 9. Sporządzič rysunek.

5. Sporządzič dokładny wykres $\mathrm{i}$ napisač równania asymptot funkcji

$f(x)=\displaystyle \frac{(x+1)^{2}-1}{x|x-1|}$

nie przeprowadzając badania jej przebiegu.

6. Rozwiqzač nierównośč:

$|x|^{2x-1}\displaystyle \leq\frac{1}{x^{2}}.$

7. Styczna do wykresu funkcji $f(x)=\sqrt{3+x}+\sqrt{3-x}\mathrm{w}$ punkcie $A(x_{0},f(x_{0}))$ przecina

oś $\mathrm{x}\mathrm{w}$ punkcie $P$, a oś $\mathrm{y}\mathrm{w}$ punkcie $Q\mathrm{t}\mathrm{a}\mathrm{k}, \dot{\mathrm{z}}\mathrm{e}OP=OQ$. Wyznaczyč $x_{0}.$

8. Trójkat równoboczny $0$ boku $a$ przecięto prostq $l$ na dwie figury, których stosunek

pól jest równy 1:5. Prosta ta przecina bok $\overline{AC}\mathrm{w}$ punkcie $D$ pod kątem $15^{0}$, a bok

$\overline{AB}\mathrm{w}$ punkcie $E$. Wykazač, $\dot{\mathrm{z}}\mathrm{e}AD+AE=a.$





PRACA KONTROLNA nr 7

kwiecień $2002\mathrm{r}$

l. Sześcian $0$ krawędzi dlugości 3 cm ma taką samą objętośčjak dwa sześciany, których

suma dfugości obydwu krawędzi wynosi 4 cm. $\mathrm{O}$ ile $\mathrm{c}\mathrm{m}^{2}$ pole powierzchni $\mathrm{d}\mathrm{u}\dot{\mathrm{z}}$ ego

sześcianu jest mniejsze od sumy pól powierzchni dwóch mniejszych sześcianów.

2. ObliczyČ tangens kąta utworzonego przez przekątne czworokata $0$ wierzchołkach

$\mathrm{A}(1,1), \mathrm{B}(2,0), \mathrm{C}(2,4), \mathrm{D}(0,6)$. Rozwiązanie zilustrowaČ rysunkiem.

3. $\mathrm{W}$ trójkąt prostokątny wpisano okrąg, a $\mathrm{w}$ okrqg ten wpisano podobny trójkąt pro-

stokątny. Wyznaczyč cosinusy kątów ostrych trójk$\Phi$ta, jeśli wiadomo, $\dot{\mathrm{z}}\mathrm{e}$ stosunek

pól obu trójkątów wynosi 9.

4. Wykazač, $\dot{\mathrm{z}}\mathrm{e}$ ciag $a_{n}=\sqrt{n(n+1)}-n$ jest rosnący. Obliczyč jego granice.

5. Rozwiązač nierównośč:

$2\displaystyle \cos^{2}\frac{x}{4}>1.$

6. Rozwiązač równanie

$\displaystyle \log_{2}(1-x)+\log_{4}(x+4)=\log_{4}(x^{3}-x^{2}-3x+5)+\frac{1}{2}$

nie wyznaczając dziedziny $\mathrm{w}$ sposób jawny.

7. $\mathrm{W}$ kulę $0$ promieniu $R$ wpisano stozek $0$ największej objętości. Wyznaczyč promień

podstawy $r\mathrm{i}$ wysokośč $h$ tego stozka. Sporzqdzič rysunek.

8. Znalez/č równania wszystkich prostych, które są styczne jednocześnie do krzywych

$y=-x^{2},y=x^{2}-8x+18.$

Sporządzič rysunek.







KORESPONDENCYJNY KURS Z MATEMATYKI

PRACA KONTROLNA nr l

$\mathrm{p}\mathrm{a}\acute{\mathrm{z}}$dziernik 2$002\mathrm{r}$

l. Narysowač wykres funkcji $y=4+2|x|-x^{2}$ Korzystając $\mathrm{z}$ tego wykresu określič

liczbę rozwi$\Phi$zań równania $4+2|x|-x^{2}=p\mathrm{w}$ zalezności od parametru rzeczywistego

$p.$

2. Pompa napełniająca pusty basen $\mathrm{w}$ pierwszej minucie pracy miała wydajnośč 0,2

$\mathrm{m}^{3}/\mathrm{s}$, a $\mathrm{w}\mathrm{k}\mathrm{a}\dot{\mathrm{z}}$ dej kolejnej minucie jej wydajnośč zwiększano $00,01 \mathrm{m}^{3}/\mathrm{s}$. Pofowa

basenu zostala napełniona po $2n$ minutach, a caly basen po kolejnych $n$ minutach,

gdzie $n$ jest liczbą naturalnq. Wyznaczyč czas napefniania basenu oraz jego pojem-

nośč.

3. Stozek ścięty jest opisany na kuli $0$ promieniu $r=2$ cm. Objętośč kuli stanowi 25\%

objętości stozka. Wyznaczyč średnice podstaw $\mathrm{i}$ dfugośč tworzącej tego stozka.

4. $\mathrm{W}$ trójkącie $ABC$ dane są promień okręgu opisanego $R$, kąt $\angle A=\alpha$ oraz $AB=\displaystyle \frac{8}{5}R.$

Obliczyč pole tego trójkqta.

5. Rozwiązač nierównośč:

$(\sqrt{x})^{\log_{8}x}\geq\sqrt[3]{16x}.$

6. $\mathrm{W}$ czworokącie ABCD odcinki $\overline{AB}\mathrm{i}\overline{BD}$ są prostopadle, $AD = 2AB =a$ oraz

$ AC=\rightarrow \displaystyle \frac{5}{3}AB\rightarrow+\frac{1}{3}AD\rightarrow$. Wyznaczyč cosinus kąta $\angle BCD=\alpha$ oraz obwód czworokąta

ABCD. Sporządzič rysunek.

7. Rozwiązač równanie:

$\displaystyle \frac{1}{\sin x}+\frac{1}{\cos x}=\sqrt{8}.$

8. Wyznaczyč równanie prostej stycznej do wykresu funkcji $y=\displaystyle \frac{1}{x^{2}}\mathrm{w}$ punkcie $P(x_{0},y_{0}),$

$x_{0}>0$, takim, $\dot{\mathrm{z}}\mathrm{e}$ odcinek tej stycznej zawarty $\mathrm{w}$ I čwiartce układu wspólrzędnych

jest najkrótszy. Rozwiązanie zilustrowač stosownym wykresem.

1




PRACA KONTROLNA nr 2

listopad $2002\mathrm{r}$

l. Czy liczby róznych `sfów', jakie $\mathrm{m}\mathrm{o}\dot{\mathrm{z}}$ na utworzyč zmieniając kolejnośč liter $\mathrm{w}$ s{\it l}o-

wach' TANATAN $\mathrm{i}$ AKABARA, są takie same? Uzasadnič odpowied $\acute{\mathrm{z}}$. Przez `sfowo'

rozumiemy tutaj dowolny ciag liter.

2. Reszta $\mathrm{z}$ dzielenia wielomianu $x^{3}+px^{2}-x+q$ przez trójmian $(x+2)^{2}$ wynosi $-x+1.$

Wyznaczyč pierwiastki tego wielomianu.

3. Figura na rysunku ponizej składa się $\mathrm{z}$ łuków $BC, CA$ okręgów $0$ promieniu $a\mathrm{i}$

środkach odpowiednio $\mathrm{w}$ punktach $A, B$, oraz $\mathrm{z}$ odcinka $\overline{AB}0$ dfugości $a$. Obliczyč

promień okręgu stycznego do obu łuków oraz do odcinka $\overline{AB}.$

4. Podstawą pryzmy przedstawionej na rysunku ponizej jest $\mathrm{p}\mathrm{r}\mathrm{o}\mathrm{s}\mathrm{t}\mathrm{o}\mathrm{k}_{\Phi}\mathrm{t}$ ABCD, którego

bok $\overline{AB}$ ma długośč $a$, a bok $\overline{BC}$ długośč $b$, gdzie $a>b$. Wszystkie ściany boczne

pryzmy są nachylone pod kątem $\alpha$ do płaszczyzny podstawy. Obliczyč objetośč tej

pryzmy.

5. Rozwiązač nierównośč

-{\it x}2$<\sqrt{}$5-{\it x}2.

Rozwiązanie zilustrowač wykresami funkcji wystepujqcych po obu stronach nierów-

ności. Zaznaczyč na rysunku otrzymany zbiór rozwiązań.

6. Ciąg $(a_{n})$ jest określony warunkami $\alpha_{1}=4, a_{n+1}=1+2\sqrt{a_{n}}, n\geq 1$. Stosując zasadę

indukcji matematycznej wykazač, $\dot{\mathrm{z}}\mathrm{e}$ ciag $(a_{n})$ jest rosnący oraz dla $n\geq 1$ spefniona

jest nierównośč: $4\leq a_{n}<6.$

7. Na krzywej $0$ równaniu $y=\sqrt{x}$ znalez/č miejsce, którejest połozone najblizej punktu

$P(0,3)$. Sporz$\Phi$dzič rysunek.

8. Wykazač, $\dot{\mathrm{z}}\mathrm{e}$ dla $\mathrm{k}\mathrm{a}\dot{\mathrm{z}}$ dej wartości parametru $\alpha\in R$ równanie kwadratowe

$3x^{2}+4x\sin\alpha-\cos 2\alpha=0$

ma dwa rózne pierwiastki rzeczywiste. Wyznaczyč te wartości parametru $\alpha$, dla

których oba pierwiastki $\mathrm{l}\mathrm{e}\dot{\mathrm{z}}$ ą $\mathrm{w}$ przedziale $(0,1).$

2





PRACA KONTROLNA nr 3

grudzień $2002\mathrm{r}$

l. Suma wyrazów nieskończonego ciqgu geometrycznego zmniejszy się $0$ 25\%, jeśli wy-

kreślimy $\mathrm{z}$ niej skladniki $0$ numerach parzystych niepodzielnych przez 4. Ob1iczyč

sume wszystkich wyrazów tego ciągu wiedząc, $\dot{\mathrm{z}}\mathrm{e}$ jego drugi wyraz wynosi l.

2. $\mathrm{Z}$ kompletu 28 kości do gry $\mathrm{w}$ domino wylosowano dwie kości (bez zwracania).

Obliczyč prawdopodobieństwo tego, $\dot{\mathrm{z}}\mathrm{e}$ kości pasujq do siebie $\mathrm{t}\mathrm{z}\mathrm{n}$. na jednym $\mathrm{z}$ pól

obu kości wystepuje ta sama liczba oczek.

3. Rozwiązač ukfad równań

$\left\{\begin{array}{l}
x\\
5x
\end{array}\right.$

$+2y$

$+my$

3

{\it m}

$\mathrm{w}$ zalezności od parametru rzeczywistego $m$. Wyznaczyč $\mathrm{i}$ narysowač zbiór, jaki

tworzq rozwiązania $(x(m),y(m))$ tego układu, gdy $m$ przebiega zbiór liczb rzeczy-

wistych.

4. $\mathrm{W}$ graniastoslupie prawidłowym sześciokątnym krawędz/ dolnej podstawy $\overline{AB}$ jest

widoczna ze środka górnej podstawy $P$ pod $\mathrm{k}_{\Phi}\mathrm{t}\mathrm{e}\mathrm{m}\alpha$. Wyznaczyč cosinus kąta utwo-

rzonego przez płaszczyznę podstawy $\mathrm{i}$ płaszczyznę zawierającą $\overline{AB}$ oraz przeciwległą

do niej krawęd $\acute{\mathrm{z}}\overline{D'E'}$ górnej podstawy. Obliczenia odpowiednio uzasadnič.

5. Rozwiązač nierównośč

$-1\displaystyle \leq\frac{2^{x+1/2}}{4^{x}-4}\leq 1.$

6. Nie posfugując się tablicami wykazač, $\dot{\mathrm{z}}\mathrm{e}$ tg82030' -tg $7^{0}30'=4+2\sqrt{3}.$

7. Napisač równanie prostej $k$ stycznej do okręgu $x^{2}+y^{2}-2x-2y-3=0\mathrm{w}$ punk-

cie $P(2,3)$. Następnie wyznaczyč równania wszystkich prostych stycznych do tego

okręgu, które tworzą $\mathrm{z}$ prostą $k$ kąt $45^{0}$

8. Dobrač parametry $a>0\mathrm{i}b\in R\mathrm{t}\mathrm{a}\mathrm{k}$, aby funkcja

$f(x)=\displaystyle \{\frac{(a+b}{x^{2}-1}1)+ax-x^{2}$

dla $x\leq a,$

dla $x>a,$

była ciagła i miala pochodnq w punkcie a. Nie przeprowadzając dalszego badania

sporządzič wykres funkcji f(x) oraz stycznej do jej wykresu w punkcie P(a, f(a)).

3





PRACA KONTROLNA nr 4

styczeń $2003\mathrm{r}$

l. Dla jakich wartości parametru rzeczywistego $t$ równanie

$x+3=-(tx+1)^{2}$

ma dokfadnie jedno rozwiązanie.

2. Czworościan foremny $0$ krawędzi $a$ przecięto płaszczyzną równoległą do dwóch prze-

ciwległych krawędzi. Wyrazič pole otrzymanego przekrojujako funkcję długości od-

cinka wyznaczonego przez ten przekrój na jednej $\mathrm{z}$ pozostafych krawędzi. Uzasad-

nič postępowanie. Przedstawič znalezioną funkcję na wykresie $\mathrm{i}$ podačjej największą

wartośč.

3. Zaznaczyč na wykresie zbiór punktów $(x,y)$ pfaszczyzny spelniajqcych warunek

$\log_{xy}|y|\geq 1.$

4. Wyznaczyč równanie linii utworzonej przez wszystkie punkty plaszczyzny, których

odległośč od okręgu $x^{2}+y^{2}=81$ jest $01$ mniejsza $\mathrm{n}\mathrm{i}\dot{\mathrm{z}}$ od punktu $P(8,0)$. Sporządzič

rysunek.

5. Na dziesiątym piętrze pewnego bloku mieszkają Kowalscy $\mathrm{i}$ Nowakowie. Kowalscy

maja dwóch synów $\mathrm{i}$ dwie córki, a Nowakowie jednego syna $\mathrm{i}$ dwie córki. Postanowili

oni wybrač mfodziezowego przedstawiciela swojego piętra. $\mathrm{W}$ tym celu Kowalscy wy-

brali losowo jedno ze swoich dzieci, a Nowakowie jedno ze swoich. Nastepnie spośród

tej dwójki wylosowano jedną osobę. Obliczyč prawdopodobieństwo, $\dot{\mathrm{z}}\mathrm{e}$ przedstawi-

cielem zostal chlopiec.

6. Uzasadnič prawdziwośč nierówności $n+\displaystyle \frac{1}{2}\geq\sqrt{n(n+1)}, n\geq 1$. Korzystając $\mathrm{z}$ niej

oraz $\mathrm{z}$ zasady indukcji matematycznej udowodnič, $\dot{\mathrm{z}}\mathrm{e}$ dla wszystkich $n\geq 1$ jest

$\displaystyle \left(\begin{array}{l}
2n\\
n
\end{array}\right)\geq\frac{4^{n}}{2\sqrt{n}}.$

7. Przeprowadzič badanie przebiegu zmienności funkcji $f(x) = \sqrt{\frac{3x-3}{5-x}}\mathrm{i}$ wykonač jej

wykres.

8. $\mathrm{W}$ trójkacie $ABC$ kąt $A$ ma miarę $\alpha$, kąt $B$ miarę $ 2\alpha$, a $BC=a$. Oznaczmy kolejno

przez $A_{1}$ punkt na boku $\overline{AC}$ taki, $\dot{\mathrm{z}}\mathrm{e}\overline{BA_{1}}$ jest dwusieczną kąta $B$; $B_{1}$ punkt na

boku $\overline{BC}$ taki, $\dot{\mathrm{z}}\mathrm{e}\overline{A_{1}B_{1}}$ jest dwusieczną kąta $A_{1}$, itd. Wyznaczyč długośč łamanej

nieskończonej ABAlBlA2$\ldots.$

4





PRACA KONTROLNA nr 5

luty $2003\mathrm{r}$

l. Jakiej dfugości powinien byč pas napędowy, aby $\mathrm{m}\mathrm{o}\dot{\mathrm{z}}$ na go było $\mathrm{u}\dot{\mathrm{z}}$ yč do pofączenia

dwóch kół $0$ promieniach 20 cm $\mathrm{i}5$ cm, jeśli odlegfośč środków tych kól wynosi 30

cm?

2. Umowa określa wynagrodzenie na kwotę 4000 $\mathrm{z}\mathrm{f}$. Skladka na ubezpieczenie spofeczne

wynosi 18,7\% tej kwoty, a składka na Kasę Chorych 7,75\% kwoty pozostałej po

odliczeniu składki na ubezpieczenie spofeczne. $\mathrm{W}$ celu obliczenia podatku nalezy od

80\% wyjściowej kwoty umowy odjąč skfadkę na ubezpieczenie spoleczne $\mathrm{i}$ wyznaczyč

19\% pozostałej sumy. Podatek jest róznicą tak otrzymanej liczby $\mathrm{i}$ kwoty składki na

Kasę Chorych. Ile wynosi podatek?.

3. Przez punkt $P(1,3)$ poprowadzič prostą $l\mathrm{t}\mathrm{a}\mathrm{k}$, aby odcinek tej prostej zawarty po-

między dwiema danymi prostymi $x-y+3 = 0 \mathrm{i}x+2y-12 = 0$ dzielil się $\mathrm{w}$

punkcie $P$ na polowy. Wyznaczyč równanie ogólne prostej $l\mathrm{i}$ obliczyč pole trójk$\Phi$ta,

jaki prosta $l$ tworzy $\mathrm{z}$ danymi prostymi.

4. Podstawą czworościanu jest trójkąt prostokątny $ABC\mathrm{o}$ kącie ostrym $\alpha \mathrm{i}$ promieniu

okręgu wpisanego $r$. Spodek wysokości opuszczonej $\mathrm{z}$ wierzchołka $D\mathrm{l}\mathrm{e}\dot{\mathrm{z}}\mathrm{y}\mathrm{w}$ punkcie

przecięcia się dwusiecznych trójkąta $ABC$, a ściany boczne wychodzące $\mathrm{z}$ wierzchoł-

ka kąta prostego podstawy tworzą kąt $\beta$. Obliczyč objętośč tego ostrosfupa.

5. Sporzqdzič wykres funkcji

$f(x)=\log_{4}(2|x|-4)^{2}$

Odczytač $\mathrm{z}$ wykresu wszystkie ekstrema lokalne tej funkcji.

6. Rozwiązač równanie $\displaystyle \cos 2x+\frac{\mathrm{t}\mathrm{g}x}{\sqrt{3}+\mathrm{t}\mathrm{g}x}=0.$

7. Dla jakich wartości parametru $a\in R\mathrm{m}\mathrm{o}\dot{\mathrm{z}}$ na określič funkcję $g(x)=f(f(x))$, gdzie

$f(x)=\displaystyle \frac{x^{2}}{ax-1}$. Napisač funkcję $g(x)\mathrm{w}$ jawnej postaci. Wyznaczyč asymptoty funkcji

$g(x)$ dla największej $\mathrm{m}\mathrm{o}\dot{\mathrm{z}}$ liwej cafkowitej wartości parametru $a.$

8. Odcinek $0$ końcach $A(0,3), B(2,y), y \in [0$, 3$]$, obraca się wokół osi Ox. Wyzna-

czyč pole powierzchni bocznej powstałej bryły jako funkcję $y\mathrm{i}$ znalez$\acute{}$č najmniejszq

wartośč tego pola. Sporz$\Phi$dzič rysunek.

5





PRACA KONTROLNA nr 6

marzec $2003\mathrm{r}$

l. Dlajakich wartości parametru rzeczywistego $p$ równanie $\sqrt{x+8p}=\sqrt{x}+2p$ posiada

rozwiązanie?

2. Obrazem okręgu $K$ wjednokładności $0$ środku $S(0,1)\mathrm{i}$ skali $k=-3$ jest okrqg $K_{1}.$

Natomiast obrazem $K_{1} \mathrm{w}$ symetrii względem prostej $0$ równaniu $2x+y+3 = 0$

jest okrąg $0$ tym samym środku co okrąg $K$. Wyznaczyč równanie okręgu $K$, jeśli

wiadomo, $\dot{\mathrm{z}}\mathrm{e}$ okręgi $K\mathrm{i}K_{1}$ są styczne zewnętrznie.

3. $\mathrm{W}$ trapezie równoramiennym dane są promień okręgu opisanego $r$, kąt ostry przy

podstawie $\alpha$ oraz suma długości obu podstaw $d$. Obliczyč dlugośč ramienia tego tra-

pezu. Zbadač warunki rozwiązalności zadania. Wykonač rysunek dla $\alpha=60^{0}, d=$

-25{\it r}.

4. $\mathrm{W}$ ostrosłupie prawidłowym czworokqtnym $\mathrm{k}\mathrm{a}\mathrm{t}$ płaski ściany bocznej przy wierz-

chofku wynosi $ 2\beta$. Przez wierzchofek $A$ podstawy oraz środek przeciwlegfej krawę-

dzi bocznej poprowadzono płaszczyznę równoległą do przekqtnej podstawy wyzna-

czającą przekrój płaski ostrosfupa. Obliczyč objętośč ostroslupa $\mathrm{w}\mathrm{i}\mathrm{e}\mathrm{d}\mathrm{z}\Phi^{\mathrm{C}}, \dot{\mathrm{z}}\mathrm{e}$ pole

przekroju wynosi $S.$

5. Obliczyč granicę

$\displaystyle \lim_{n\rightarrow\infty}\frac{n-\sqrt[3]{n^{3}+n^{\alpha}}}{\sqrt[5]{n^{3}}},$

gdzie $\alpha$ jest najmniejszym dodatnim pierwiastkiem równania 2 $\cos\alpha=-\sqrt{3}.$

6. Rozwiązač nierównośč

$2^{1+2\log_{2}\cos x}-\displaystyle \frac{3}{4}\geq 9^{0.5+\log_{3}\sin x}$

7. Wybrano losowo 4 liczby czterocyfrowe (cyfra tysiecy nie $\mathrm{m}\mathrm{o}\dot{\mathrm{z}}\mathrm{e}$ byč zerem!). Obliczyč

prawdopodobieństwo tego, $\dot{\mathrm{z}}\mathrm{e}$ co najmniej dwie $\mathrm{z}$ tych liczb czytane od przodu lub

od końca będą podzielne przez 4.

8. Zaznaczyč na rysunku zbiór punktów $(x,y)$ płaszczyzny określony warunkami

$|x-3y| <2$ oraz $y^{3}\leq x$. Obliczyč tangens $\mathrm{k}_{\Phi^{\mathrm{t}\mathrm{a}}}$, pod którym przecinają się linie

tworzqce brzeg tego zbioru.

6





PRACA KONTROLNA nr 7

kwiecień 2003r

l. Dwa punkty poruszają się ruchem jednostajnym po okręgu $\mathrm{w}$ tym samym kierunku,

przy czym jeden $\mathrm{z}$ nich wyprzedza drugi co 44 sekund. $\mathrm{J}\mathrm{e}\dot{\mathrm{z}}$ eli zmienič kierunek ruchu

jednego $\mathrm{z}$ tych punktów, to bedą się one spotykač co 8 sekund. Ob1iczyč stosunek

prędkości tych punktów.

2. Dla jakich wartości parametru $p$ nierównośč

$\displaystyle \frac{2px^{2}+2px+1}{x^{2}+x+2-p^{2}}\geq 2$

jest spełniona dla $\mathrm{k}\mathrm{a}\dot{\mathrm{z}}$ dej liczby rzeczywistej $x$?

3. $\mathrm{W}$ równolegloboku dane są $\mathrm{k}\mathrm{a}\mathrm{t}$ ostry $\alpha$, dłuzsza przekątna $d$ oraz róznica boków $r.$

Obliczyč pole równolegloboku.

4. Naczynie $\mathrm{w}$ kształcie półkuli $0$ promieniu $R$ ma trzy nózki $\mathrm{w}$ kształcie kulek $0$

promieniu $r, 4r < R$, przymocowanych do naczynia $\mathrm{w}$ ten sposób, $\dot{\mathrm{z}}\mathrm{e}$ ich środki

tworzą trójkąt równoboczny, a naczynie postawione na płaskiej powierzchni dotyka

ją wjednym punkcie. Obliczyč wzajemnq odleglośč punktów przymocowania kulek.

Wykonač odpowiednie rysunki.

5. Poslugując się rachunkiem rózniczkowym określič liczbę rozwiązań równania

$2x^{3}+1=6|x|-6x^{2}$

6. Nie $\mathrm{s}\mathrm{t}\mathrm{o}\mathrm{s}\mathrm{u}\mathrm{j}_{\Phi}\mathrm{c}$ zasady indukcji matematycznej wykazač, $\dot{\mathrm{z}}\mathrm{e}\mathrm{j}\mathrm{e}\dot{\mathrm{z}}$ eli $n \geq 2$ jest liczbą

naturalną, to $\displaystyle \frac{n^{n}-1}{n-1}$ jest nieparzystą liczbą naturalną.

7. Rozwiązač równanie

$\displaystyle \frac{8}{3}(\sin^{2}x+\sin^{4}x+\ldots)=4-2\cos x+3\cos^{2}x-\frac{9}{2}\cos^{3}x+\ldots$

8. Rozwazmy rodzine prostych normalnych (tj. prostopadfych do stycznych $\mathrm{w}$ punk-

tach styczności) do paraboli $0$ równaniu $2y=x^{2}$ Znalez$\acute{}$č równanie krzywej utwo-

rzonej ze środków odcinków tych normalnych zawartych pomiędzy osią rzędnych $\mathrm{i}$

$\mathrm{w}\mathrm{y}\mathrm{z}\mathrm{n}\mathrm{a}\mathrm{c}\mathrm{z}\mathrm{a}\mathrm{j}_{\Phi}$cymi je punktami paraboli. Sporz$\Phi$dzič rysunek.

7







XXXIII

KORESPONDENCYJNY KURS Z MATEMATYKI

$\mathrm{p}\mathrm{a}\acute{\mathrm{z}}$dziernik 2$003\mathrm{r}.$

PRACA KONTROLNA nr l

l. Podstawą trójk$\Phi$ta równoramiennegojest odcinek $\overline{AB}0$ końcach $A(-1,3), B(1,-1),$

a wierzchołek $C$ tego trójkąta $\mathrm{l}\mathrm{e}\dot{\mathrm{z}}\mathrm{y}$ na prostej $l\mathrm{o}$ równaniu $3x-y-14=0$. Obliczyč

pole trójkąta $ABC.$

2. Pewna liczba sześciocyfrowa zaczyna się ($\mathrm{z}$ lewej strony) cyfrą 3. Jeś1i cyfrę tę

przestawimy $\mathrm{z}$ pierwszej pozycji na ostatnią, to otrzymamy liczbę stanowiacą 25\%

liczby pierwotnej. Znalez/č tę liczbę.

3. $\mathrm{W}$ trapezie opisanym na okregu kąty ostre przy podstawie mają miary $\alpha \mathrm{i}2\alpha, \mathrm{a}$

dlugośč krótszego ramienia wynosi $c$. Obliczyč długośč krótszej podstawy tego

trapezu. Wynik doprowadzič do najprostszej postaci.

4. Rozwiązač nierównośč:

$\displaystyle \frac{1}{x^{2}-x-2}\leq\frac{1}{|x|}.$

5. Zaznaczyč na pfaszczy $\acute{\mathrm{z}}\mathrm{n}\mathrm{i}\mathrm{e}$ zbiór wszystkich punktów $(x,y)$ spelniających nierów-

nośč $\log_{x}(1+(y-1)^{3})\leq 1.$

6. Rozwiązač równanie:

$\sin^{2}3x$ -sin2 $2x=\sin^{2}x.$

7. Wysokośč ostroslupa prawidfowego czworokątnego jest trzy razy dfuzsza od pro-

mienia kuli wpisanej $\mathrm{w}$ ten ostroslup Obliczyč cosinus kata pomiędzy sąsiednimi

ścianami bocznymi tego ostrosłupa.

8. Dany jest nieskończony ciąg geometryczny: $x+1, -x^{2}(x+1), x^{4}(x+1), \ldots$ Wyzna-

czyč najmniejszą $\mathrm{i}$ największą wartośč funkcji $S(x)$ oznaczającej sumę wszystkich

wyrazów tego ciągu.

1




listopad 2003r.

PRACA KONTROLNA nr 2

l. Trójkąt $\mathrm{P}^{\mathrm{r}\mathrm{o}\mathrm{s}\mathrm{t}\mathrm{o}\mathrm{k}}\Phi^{\mathrm{t}\mathrm{n}\mathrm{y}} \mathrm{o}\mathrm{b}\mathrm{r}\mathrm{a}\mathrm{c}\mathrm{a}\mathrm{j}_{\Phi}\mathrm{c}$ się wokół jednej $\mathrm{i}$ drugiej przyprostokątnej daje

bryły $0$ objętościach $V_{1} \mathrm{i} V_{2}$, odpowiednio. Obliczyč objętośč bryły powstałej $\mathrm{z}$

obrotu tego trójkąta wokół dwusiecznej kąta prostego.

2. Czy $\mathrm{m}\mathrm{o}\dot{\mathrm{z}}$ na sumę 42000 z1otych podzie1ič na pewną 1iczbę nagród $\mathrm{t}\mathrm{a}\mathrm{k}$, aby kwoty

tych nagród wyrazaly się $\mathrm{w}$ pelnych setkach złotych, tworzyly ciąg arytmetyczny

oraz najwyzsza nagroda wynosifa 13000 $\mathrm{z}\mathrm{f}$? Jeśli $\mathrm{t}\mathrm{a}\mathrm{k}$, to podač liczbę $\mathrm{i}$ wysokości

tych nagród.

3. Dane sq okregi $0$ równaniach $(x-1)^{2}+(y-1)^{2}=1$ oraz $(x-2)^{2}+(y-1)^{2}=16.$

Wyznaczyč równania wszystkich okręgów stycznych równocześnie do obu danych

okręgów oraz do osi Oy. Sporządzič rysunek.

4. $\mathrm{W}$ równolegloboku kąt ostry miedzy przekqtnymi ma miarę $\beta$, a stosunek dfugości

dfuzszej przekątnej do krótszej przekątnej wynosi $k$. Obliczyč tangens kąta ostrego

tego równoległoboku.

5. Rozwiązač równanie $\sqrt{4x-3}-3=\sqrt{2x-10}.$

6. Dobrač liczby calkowite a,b $\mathrm{t}\mathrm{a}\mathrm{k}$, aby wielomian $6x^{3}-7x^{2}+1$ dzielil się bez reszty

przez trójmian kwadratowy $2x^{2}+ax+b.$

7. Rozwiązač nierównośč $|2^{x}-3|\leq 2^{1-x}$ Rozwiązanie zilustrowač na rysunku wyko-

nując wykresy funkcji występujqcych po obu stronach tej nierówności.

8. Wyznaczyč przedziały monotoniczności funkcji

$f(x)=\displaystyle \sin^{2}x+\frac{\sqrt{3}}{2}x,$

$x\in[-\pi,\pi].$

2





grudzień 2003r.

PRACA KONTROLNA nr 3

l. Obliczyč prawdopodobieństwo tego, $\dot{\mathrm{z}}\mathrm{e}$ gracz losując 7 kart $\mathrm{z}$ talii 24 kart do gry

otrzyma dokładnie cztery karty $\mathrm{w}$ jednym kolorze $\mathrm{w}$ tym asa, króla $\mathrm{i}$ damę.

2. Pewien ostroslup przecięto na trzy części dwiema płaszczyznami równoległymi do

jego podstawy. Pierwsza pfaszczyznajest polozona $\mathrm{w}$ odlegfości $d_{1}=2$ cm, a druga

$\mathrm{w}$ odległości $d_{2}=3$ cm od podstawy. Pola przekrojów ostroslupa tymi plaszczy-

znami równe są odpowiednio $S_{1}=25\mathrm{c}\mathrm{m}^{2}$ oraz $S_{2}=16\mathrm{c}\mathrm{m}^{2}$ Obliczyč objętośč tego

ostroslupa oraz objętośč najmniejszej części.

3. Rozwiązač ukfad równań:

$\left\{\begin{array}{l}
x^{2}+y^{2}=24\\
\frac{2\log x+\log y^{2}}{\log(x+y)}=2
\end{array}\right.$

4. $\mathrm{W}$ trójkącie równoramiennym $ABC$ odległośč środka okręgu wpisanego od wierz-

chofka $C$ wynosi $d$, a podstawę $\overline{AB}$ widač ze środka okręgu wpisanego pod $\mathrm{k}_{\Phi^{\mathrm{t}\mathrm{e}\mathrm{m}}}$

$\alpha$. Obliczyč pole tego trójkąta.

5. Stosując zasadę indukcji matematycznej udowodnič prawdziwośč dla $n\geq 1$ wzoru

$\displaystyle \cos x+\cos 3x+\ldots+\cos(2n-1)x=\frac{\sin 2nx}{2\sin x},\sin x\neq 0.$

6. Wyznaczyč granicę ciągu 0 wyrazie ogólnym

$a_{n}=\displaystyle \frac{\sqrt[6]{4n}}{\sqrt{n}-\sqrt{n+\sqrt[3]{4n^{2}}}},$

$n\geq 1.$

7. Dany jest wierzcholek $A(6,1)$ kwadratu. Wyznaczyč pozostałe wierzchołki tego

kwadratu wiedząc, $\dot{\mathrm{z}}\mathrm{e}$ wierzchofki sąsiadujące $\mathrm{z}A\mathrm{l}\mathrm{e}\mathrm{z}\Phi$jeden na prostej $l:x-2y+1=$

$0$, a jeden na prostej $k:x+3y-4=0$. Sporządzič rysunek.

8. Przeprowadzič badanie $\mathrm{i}$ wykonač wykres funkcji

$f(x)=\displaystyle \frac{x+1}{\sqrt{x}}.$

3





styczeń 2004r.

PRACA KONTROLNA nr 4

l. Statek plynie z Wrocfawia do Szczecina 3 dni, a ze Szczecina do Wrocfawia 5 dni.

Jak długo z Wrocławia do Szczecina płynie woda?

2. Dla jakich wartości rzeczywistych parametru x liczby

$1+\log_{2}3, \log_{x}36,$

$\displaystyle \frac{4}{3}\log_{8}6$

są trzema kolejnymi wyrazami pewnego ciagu geometrycznego.

3. Wanna $0$ pojemności 2001 mająca kszta1t pofowy wa1ca (rozciętego wzdfuz osi) $\mathrm{l}\mathrm{e}\dot{\mathrm{z}}\mathrm{y}$

poziomo na ziemi $\mathrm{i}$ zawiera pewną ilośč wody. Do wanny włozono belkę $\mathrm{w}$ kształcie

walca $0$ średnicy cztery razy mniejszej $\mathrm{n}\mathrm{i}\dot{\mathrm{z}}$ średnica wanny $\mathrm{i}$ długości równej polowie

dlugości wanny. Okazafo się, $\dot{\mathrm{z}}\mathrm{e}$ lustro wody styka się $\mathrm{z}$ belką $\mathrm{z}\mathrm{a}\mathrm{n}\mathrm{u}\mathrm{r}\mathrm{z}\mathrm{o}\mathrm{n}\Phi^{\mathrm{W}}$ wodzie.

Ile wody znajduje się $\mathrm{w}$ wannie? Podač $\mathrm{z}$ dokładnością do 0,11.

4. Wyznaczyč wszystkie wartości parametru $m$, dla których obydwa pierwiastki trój-

mianu kwadratowego $v(x)=x^{2}+mx-m^{2}\mathrm{l}\mathrm{e}\dot{\mathrm{z}}$ ą pomiędzy pierwiastkami trójmianu

$w(x)=x^{2}-(m-1)x-m.$

5. Urna A zawiera trzy kule biafe $\mathrm{i}$ dwie czarne, a urna $\mathrm{B}$ dwie biafe $\mathrm{i}$ trzy czarne.

Wylosowano cztery razy jedną kulę ze zwracaniem $\mathrm{z}$ urny A oraz jedną kulę $\mathrm{z}$ urny

B. Obliczyč prawdopodobieństwo tego, $\dot{\mathrm{z}}\mathrm{e}$ wśród pięciu wylosowanych kul są co

najmniej dwie kule biafe.

6. Rozwiązač równanie:

2 $\sin 2x+2\cos 2x+\mathrm{t}\mathrm{g}x=3.$

7. Danajest funkcja $f(x)=x^{4}-2x^{2}$. Wyznaczyč wszystkie proste styczne do wykresu

tej funkcji zawierające punkt $P(1,-1)$. Określič ile punktów wspólnych $\mathrm{z}$ wykresem

tej funkcji mają wyznaczone styczne. Rozwiązanie zilustrowač rysunkiem.

8. Podstawą ostroslupa ABCS jest trójkąt równoramienny, którego kąt przy wierz-

chołku $C$ ma miarę $\alpha$, a ramię ma długośč $BC=b$. Spodek wysokości ostrosłupa

$\mathrm{l}\mathrm{e}\dot{\mathrm{z}}\mathrm{y}\mathrm{w}$ środku wysokości $\overline{CD}$ podstawy, a kąt pfaski ściany bocznej $ABS$ przy

wierzchofku ma miarę $\alpha$. Obliczyč promień kuli opisanej na tym ostrosfupie oraz

cosinusy katów nachylenia ścian bocznych do podstawy.

4





luty 2004r.

PRACA KONTROLNA nr 5

l. Piąty wyraz rozwinięcia dwumianu $(a+b)^{18}$ jest $0$ 180\% większy od wyrazu trze-

ciego. $\mathrm{O}$ ile procent wyraz ósmy tego rozwinięcia jest mniejszy $\mathrm{b}\text{ą} \mathrm{d}\acute{\mathrm{z}}$ większy od

wyrazu czwartego?

2. Wyznaczyč równanie linii utworzonej przez wszystkie punkty plaszczyzny, dla któ-

rych stosunek kwadratu odległości od prostej $k$ : $x-2y+3 = 0$ do kwadratu

odlegfości od prostej $l:3x+y+2=0$ wynosi 2. Sporządzič rysunek.

3. Obwód trójkąta $ABC$ wynosi 15, a dwusieczna kąta $A$ dzieli bok przeciwlegfy na

odcinki długości 3 oraz 2. Ob1iczyč po1e koła wpisanego $\mathrm{w}$ ten trójkąt.
\begin{center}
\includegraphics[width=182.316mm,height=37.488mm]{./KursMatematyki_PolitechnikaWroclawska_2003_2004_page4_images/image001.eps}
\end{center}
$a_{2}$

$a_{3}$

$a_{4}$

{\it O}

po nieskonczonej famanej jak na rysunku obok,

$p$ ktorej długosci kolejnych odcinkow tworz ci $\mathrm{g}$

cz stka zatrzymała się $\mathrm{w}$ punkcie $P(10,3)$. Jaką

drogę przebyla cz stka?

4. $\mathrm{C}_{\mathrm{Z}\Phi}$stka startuje $\mathrm{z}\mathrm{P}^{\mathrm{o}\mathrm{c}\mathrm{z}}\Phi^{\mathrm{t}\mathrm{k}\mathrm{u}}$ ukfadu wspólrzędnych $\mathrm{i}$ porusza się ze stafą prędkością

$\alpha_{1}$

5. Stosując zasadę indukcji matematycznej udowodnič, $\dot{\mathrm{z}}\mathrm{e}\mathrm{d}\mathrm{l}\mathrm{a}$ wszystkich $n\geq 1$ wie-

lomian $x^{3n+1}+x^{3n-1}+1$ dzieli się $\mathrm{b}\mathrm{e}\mathrm{z}$ reszty przez wielomian $x^{2}+x+1.$

6. Nie przeprowadzajqc badania przebiegu wykonač wykres funkcji

$f(x)=\displaystyle \frac{|x-2|}{x-|x|+2}.$

Podač równania asymptot i ekstrema lokalne tej funkcji.

7. Rozwi$\Phi$zač nierównośč

$|\cos x|^{1+\sqrt{2}\sin x+\sqrt{2}\cos x}\leq 1,$

$x\in[-\pi,\pi].$

8. W stozek wpisano graniastosfup trójkątny prawidłowy 0 wszystkich krawedziach tej

samej dfugości. Przyjakim kącie rozwarcia stozka stosunek objętości graniastosłupa

do objetości stozka jest największy?

5





marzec 2004r.

PRACA KONTROLNA nr 6

1. $\mathrm{W}$ kolo $0$ powierzchni $\displaystyle \frac{5}{4}\pi$ wpisano trójkąt prostokątny $0$ polu l. Obliczyč obwód

tego trójkąta.

2. Sprowadzič do najprostszej postaci wyrazenie

2(sin6 $\alpha+\cos^{6}\alpha$)$-7(\sin^{4}\alpha+\cos^{4}\alpha)+\cos 4\alpha.$

3. Wyznaczyč trójmian kwadratowy, którego wykresem jest parabola styczna do pro-

stej $y=x+2$, przechodząca przez punkt $P(-2,-2)$ oraz symetryczna względem

prostej $x=1$. Sporzqdzič rysunek.

4. $\mathrm{W}$ trapezie ABCD, $\mathrm{w}$ którym $\overline{AB}\Vert\overline{CD}$, dane są $\vec{AC}=(4,7)\rightarrow$ oraz $\vec{BD}=\rightarrow\rightarrow(-6,2).$

Posfugując się rachunkiem wektorowym wyznaczyč wektory AB $\mathrm{i}\vec{CD}$, jeśli $AD\perp BD.$

5. Jaś ma $\mathrm{w}$ portmonetce 3 monetyjednozłotowe, 2 monety dwuzłotowe ijedną pięcio-

złotową. Kupujac zeszyt $\mathrm{w}$ cenie 4 zł wyciaga 1osowo $\mathrm{z}$ portmonetki po jednej mo-

necie tak dlugo, $\mathrm{a}\dot{\mathrm{z}}$ nazbiera się suma wystarczająca do zaplaty za zeszyt. Obliczyč

prawdopodobieństwo, $\dot{\mathrm{z}}\mathrm{e}$ wyciągnie co najmniej trzy monety. Podač odpowiednie

uzasadnienie (nie jest nim $\mathrm{t}\mathrm{z}\mathrm{w}$. drzewko).

6. Narysowač na pfaszczy $\acute{\mathrm{z}}\mathrm{n}\mathrm{i}\mathrm{e}$ zbiór punktów określony następująco

$\mathcal{F}=\{(x,y):\sqrt{4x-x^{2}}\leq y\leq 4-\sqrt{1-2x+x^{2}}\}.$

$\mathrm{W}$ jakiej odleglości od brzegu figury $\mathcal{F}$ znajduje się punkt $P(\displaystyle \frac{3}{2},\frac{5}{2})$ ?

7. Dana jest funkcja $f(x) = \log_{2}(1-x^{2})-\log_{2}(x^{2}-x)$. Nie korzystając $\mathrm{z}$ metod

rachunku rózniczkowego wykazač, $\dot{\mathrm{z}}\mathrm{e}f$ jest rosnąca $\mathrm{w}$ swojej dziedzinie oraz, $\dot{\mathrm{z}}\mathrm{e}$

$g(x)=f(x-\displaystyle \frac{1}{2})$ jest nieparzysta. Wyznaczyč funkcję odwrotną $f^{-1}$, jej dziedzinę

$\mathrm{i}$ zbiór wartości.

8. Pole powierzchni bocznej ostrosłupa prawidfowego czworokątnego wynosi $c^{2}$, a kąt

nachylenia ściany bocznej do podstawy ma miarę $\alpha$. Ostrosłup rozcieto na dwie

części pfaszczyzną przechodzącą przez jeden $\mathrm{z}$ wierzchołków podstawy $\mathrm{i}$ prostopa-

dłą do przeciwległej krawędzi bocznej. Obliczyč objętośč części zawierającej wierz-

cholek ostrosłupa. Kiedy zadanie ma sens?

6





kwiecień 2004r.

PRACA KONTROLNA nr 7

l. Pierwsze dwa wyrazy ciągu geometrycznego są rozwiazaniami równania

$4x^{2}-4px-3p^{2}=0$, gdzie $p$ jest nieznaną $1\mathrm{i}\mathrm{c}\mathrm{z}\mathrm{b}_{\Phi}$. Wyznaczyč ten ciąg, jeśli suma

wszystkich jego wyrazów wynosi 3.

2. Wiedząc, $\dot{\mathrm{z}}\mathrm{e} \cos\varphi = \sqrt{\frac{2}{3}}$ oraz $\varphi \in (\displaystyle \frac{3}{2}\pi,2\pi)$, obliczyč cosinus kąta pomiędzy

prostymi $y=(\displaystyle \sin\frac{\varphi}{2})x, y=(\displaystyle \cos\frac{\varphi}{2})x.$

3. Kostka sześcienna ma krawęd $\acute{\mathrm{z}} 2a$. Aby zmieścič ją $\mathrm{w}$ pojemniku $\mathrm{w}$ kształcie kuli

$0$ średnicy $3a$, ze wszystkich narozy odcięto $\mathrm{w}$ minimalny sposób jednakowe ostro-

slupy prawidfowe trójk$\Phi$tne. Obliczyč dlugośč krawędzi bocznej odciętych czworo-

ścianów?

4. Udowodnič prawdziwośč nierówności

$1+\displaystyle \frac{x}{2}\geq\sqrt{1+x}\geq 1+\frac{x}{2}-\frac{x^{2}}{2}$ dla $x\in[-1,1].$

Zilustrowač $\mathrm{j}_{\Phi}$ na odpowiednim wykresie.

5. Rozwiązač równanie:

--csoins25{\it xx}$=$-sin3{\it x}.

6. Znalez/č równanie okręgu symetrycznego do okręgu $x^{2}-4x+y^{2}+6y=0$ wzglę-

dem stycznej do tego okręgu poprowadzonej $\mathrm{z}$ punktu $P(3,5) \mathrm{i}$ majqcej dodatni

wspófczynnik kierunkowy.

7. $\mathrm{W}$ okrąg $0$ promieniu $r$ wpisano trapez $0$ przekątnej $d\geq r\sqrt{3}\mathrm{i}$ największym ob-

wodzie. Obliczyč pole tego trapezu.

8. Metodą analityczną określič dla jakich wartości parametru $m$ układ równań

$\left\{\begin{array}{l}
mx\\
x
\end{array}\right.$

$-y$

$-2|y|$

$+2=0$

$+2=0$

ma dokladnie jedno rozwiązanie? Wyznaczyč to rozwiązanie w zalezności od m.

Sporządzič rysunek.

7







XXXIV

KORESPONDENCYJNY KURS Z MATEMATYKI

PRACA KONTROLNA nr l

$\mathrm{p}\mathrm{a}\acute{\mathrm{z}}$dziernik 2$004\mathrm{r}.$

l. Staś kupił zeszyty 32-kartkowe po 80 gr za sztukę $\mathrm{i}$ zeszyty 60-kartkowe po 1,20 zł za sztukę

$\mathrm{i}$ zapłacił 13,20 $\mathrm{z}l$. Ile zeszytów 60-kartkowych kupił Staś, jeś1i by1o ich więcej $\mathrm{n}\mathrm{i}\dot{\mathrm{z}}$ zeszytów

32-kartkowych?

2. Rozwiązač nierównośč

--{\it xx}32$+$-{\it xx}$\leq$1.

3. Dana jest parabola $0$ równaniu $y = -x^{2}+2x+3$. Znalez/č równanie paraboli, która jest

symetryczna do danej względem punktu $S(2,1)$, oraz wyznaczyč punkty, $\mathrm{w}$ których przecina

ona osie ukladu wspólrzędnych. Sporządzič rysunek.

4. $\mathrm{W}$ trójkącie prostokątnym równoramiennym $ABC$ dany jest wierzchołek kata prostego $C(1,1),$

a bok $\overline{AB}\mathrm{l}\mathrm{e}\dot{\mathrm{z}}\mathrm{y}$ na prostej $x+5y+7=0$. Wyznaczyč współrzędne wierzcholków A $\mathrm{i}B.$

5. $\mathrm{W}$ ostroslupie prawidlowym sześciokątnym kąty płaskie ścian bocznych przy wierzchołku są

równe $\alpha$. Wyznaczyč cosinus kata między sąsiednimi ścianami bocznymi tego ostroslupa.

6. Dany jest trójkąt równoramienny $0$ kqcie przy podstawie $\alpha \mathrm{i}$ ramieniu $b$. Ramiona tego

trójkąta przecięto prostą odcinając $\mathrm{z}$ niego deltoid. Wyznaczyč katy pozostałego mniejszego

trójkąta oraz jego pole. Kiedy zadanie ma rozwiazanie?

7. Rozwiązač nierównośč

$\sqrt{2^{x-2}+3}\leq 2^{x}-2.$

8. Wyznaczyč dziedzinę oraz narysowač wykres funkcji $s(x)$ danej wzorem

$s(x)=\log_{2}(1-x+x^{2}-x^{3}+\ldots).$

Przy pomocy wykresu określič zbiór wartości tej funkcji.

9. Rozwiązač równanie

tg 3{\it x}$=$ -csoins 42{\it xx}.




PRACA KONTROLNA nr 2

listopad 2004r.

l. Liczby $0$ 45\% mniejsza $\mathrm{i} 0$ 32\% większa od ułamka okresowego 0,(60) sq pierwiastkami

trójmianu kwadratowego $0$ współczynnikach całkowitych względnie pierwszych. Obliczyč

resztę $\mathrm{z}$ dzielenia tego trójmianu przez dwumian $(x-1).$

2. Wykres funkcji $f$ : $[0,5]\rightarrow R$ jest przed-

stawiony na rysunku obok. Narysowac

wykres funkcji $g(x)=f(x)-f(5-x)$

$\mathrm{i}$ zapisač $\mathrm{j}$ wzorem.
\begin{center}
\includegraphics[width=58.524mm,height=35.616mm]{./KursMatematyki_PolitechnikaWroclawska_2004_2005_page1_images/image001.eps}
\end{center}
$y$

2

1

0 1 3 5 $x$

$-1$

3. Obliczyc wartosci $\sin\alpha \mathrm{i}\cos\alpha$, jeśli wiadomo, $\dot{\mathrm{z}}\mathrm{e}$

$\displaystyle \sin\alpha+3\cos\alpha=\frac{1}{\cos\alpha},$

$\displaystyle \alpha\in[0,\pi]\backslash \{\frac{\pi}{2}\}.$

4. Suma 20 pierwszych wyrazów pewnego ciągu arytmetycznego jest równa zeru, a iloczyn dzie-

siqtego $\mathrm{i}$ jedenastego wyrazu wynosi $-1$. Dla jakich liczb naturalnych $n$ suma $n$ pierwszych

wyrazów tego ciągu przekracza 77?

5. Trapez równoramienny jest wpisany $\mathrm{w}$ okrąg $0$ promieniu $R$, a jednq $\mathrm{z}$ jego podstaw jest

średnica tego okręgu. $\mathrm{W}$ trapez ten daje się wpisač okrąg. Wyznaczyč jego promień.

6. Środek kuli opisanej na ostrosłupie prawidłowym trójkatnym $\mathrm{l}\mathrm{e}\dot{\mathrm{z}}\mathrm{y}\mathrm{w}$ odległości $d$ ponad

podstawą ostrosłupa, a kąt nachylenia krawędzi bocznej do podstawy wynosi $\alpha$. Obliczyč

objętośč ostrosłupa.

7. Wyznaczyč wszystkie wartości parametru rzeczywistego $m$, dla których funkcja

$f(x)=\displaystyle \frac{x+1}{x^{2}+mx+4}$

jest dodatnia i rosnąca na odcinku (0,1).

8. Nie korzystając z rachunku rózniczkowego wyznaczyč dziedzinę i zbiór wartości funkcji

$f(x)=\sqrt{\sqrt{2}-\cos x-\sqrt{3}\sin x},$

$x\in[0,\pi].$

9. Rozwiązač uklad równań

$\left\{\begin{array}{l}
|x+1|y\\
x^{2}-4|x|+2y-1
\end{array}\right.$

$=4$

$=0$

Przedstawič ilustrację graficzną obu równań i zaznaczyč na rysunku znalezione rozwiązania.





PRACA KONTROLNA nr 3

grudzień 2004r.

l. W pewnej szkole zapytano uczniów klas maturalnych ile razy w ostatnim miesiqcu ucze-

liczba uczniow stniczyli w imprezie kulturalnej. Wyniki przed-

10 uczniów jest w klasach maturalnych tej szkoly; b)
\begin{center}
\includegraphics[width=78.588mm,height=35.304mm]{./KursMatematyki_PolitechnikaWroclawska_2004_2005_page2_images/image001.eps}
\end{center}
15

5

stawiono na diagramie obok. Obliczyc: a) Ilu

Ile razy średnio w miesi cu uczeń był na imprezie

kulturalnej. Sporz dzic diagram kołowy przedsta-

0 1 2 3 4 5 6 7 wiaj cy procentowo otrzymane wyniki.

2. Turysta zauwazyl, $\dot{\mathrm{z}}\mathrm{e}\mathrm{w}$ pewnym miejscu na odcinku 10 $\mathrm{m}$ potok górski płynie $\mathrm{w}$ korycie

skalnym, które $\mathrm{w}$ przekroju pionowym tworzy trapez $0$ dolnej podstawie 2 $\mathrm{m}\mathrm{i}$ górnej 3 $\mathrm{m}.$

Wysokośč koryta wynosi 50 cm, przy czym woda wypełnia koryto jedynie na głębokośč 10

cm. Turysta ustalil równiez, $\dot{\mathrm{z}}\mathrm{e}$ czas przepływu wody przez koryto wynosi 3 sekundy. I1e

litrów wody przeplywa przez ten potok $\mathrm{w}$ ciągu jednej sekundy?

3. Wykazač, $\dot{\mathrm{z}}\mathrm{e}$ dla dowolnych liczb dodatnich $a, b$ prawdziwa jest nierównośč

$(a+b)^{3}\leq 4(a^{3}+b^{3}).$

Wsk. Podzielič obie strony przez $b^{3}\mathrm{i}$ wprowadzič jednq zmiennq.

4. Boki $\overline{AB}\mathrm{i}\overline{AD}$ równoległoboku $\mathrm{l}\mathrm{e}\dot{\mathrm{z}}$ ą odpowiednio na prostych $3x+4y-7=0\mathrm{i}x-2y+1=0.$

Wyznaczyč współrzędne wierzcholka $C$ tego równolegloboku wiedząc, $\dot{\mathrm{z}}\mathrm{e}$ jego wysokośč do

boku $\overline{AB}$ wynosi 2, a wierzcho1ek $B$ ma współrzędne $(5,-2).$

5. $\mathrm{W}$ trójkącie ostrokątnym $ABC$ dane sq bok $BC=\displaystyle \frac{5}{2}\sqrt{5}$ cm oraz wysokości $BD=\displaystyle \frac{11}{2}$ cm $\mathrm{i}$

$CE=5$ cm. Obliczyč obwód tego trójkąta oraz cosinus kata $\angle BAC.$

6. Spośród dwudziestu najmniejszych, nieparzystych liczb naturalnych wylosowano (bez zwra-

cania) dwie. Obliczyč prawdopodobieństwo, $\dot{\mathrm{z}}\mathrm{e}$ otrzymano: a) dwie liczby pierwsze; b) dwie

liczby względnie pierwsze.

7. Rozwiązač nierównośč $\log_{2} x^{\log_{4}x}\geq\log_{x}16.$

8. Niech $f(m)$ oznacza sumę trzecich potęg pierwiastków rzeczywistych równania kwadratowego

$x^{2}+(m+3)x+m^{2}=0\mathrm{z}$ parametrem $m$. Wyznaczyč wzór funkcji $f(m)$ oraz najmniejszq $\mathrm{i}$

największq wartośč tej funkcji.

9. $\mathrm{W}$ ostrosłupie prawidłowym czworokątnym $\mathrm{k}\mathrm{a}\mathrm{t}$ nachylenia krawędzi bocznej do podstawy

wynosi $\alpha$, a odległosc krawędzi podstawy od przeciwległej sciany

bocznej jest równa $d=3$ cm. Obliczyc wysokosc sciany bocznej.

Czy siatka tego ostrosłupa, jak na rysunku obok, zmiesci $\mathrm{s}\mathrm{i}_{9}$

na arkuszu papieru $\mathrm{w}$ ksztalcie kwadratu $0$ boku 16 cm, jes1i

wiadomo, $\dot{\mathrm{z}}\mathrm{e}\mathrm{t}\mathrm{g}\alpha=2$? Sporz dzic rysunek.
\begin{center}
\includegraphics[width=30.324mm,height=30.384mm]{./KursMatematyki_PolitechnikaWroclawska_2004_2005_page2_images/image002.eps}
\end{center}




PRACA KONTROLNA nr 4

styczeń $2005\mathrm{r}.$

l. Krawędzie oraz przekątna prostopadłościanu tworzq cztery kolejne wyrazy ciągu arytmetycz-

nego, przy czym przekątna ma długośč 7 cm. Jaką najkrótszq drogę musi przebyč mucha,

aby wędrujqc po krawędziach tego prostopadłościanu odwiedzila wszystkie jego wierzchołki.

2. Dany jest wielomian $w(x)=x^{4}-2x^{2}-x+2$. Rozłozyc na czynniki $\mathrm{m}\mathrm{o}\dot{\mathrm{z}}$ liwie najnizszego

stopnia wielomian $p(x)=w(x+1)-w(x).$
\begin{center}
\includegraphics[width=60.504mm,height=64.824mm]{./KursMatematyki_PolitechnikaWroclawska_2004_2005_page3_images/image001.eps}
\end{center}
{\it y}

2

0 2 $x$

3. Na rysunku obok przedstawiono fragment mapy $\mathrm{w}$ ska-

li 1:25000, który zawiera obszar 1asu $L$ ograniczony

czterema drogami. Na mapę jest naniesiona siatka ki-

lometrowa, a dodatkowo umieszczono na niej układ

współrzędnych pokrywaj cy się $\mathrm{z}$ wybranymi liniami

siatki. Zapisac obszar $L\mathrm{w}$ postaci układu nierownosci

liniowych ($\mathrm{w}$ skali mapy). Obliczyc rzeczywiste pole

obszaru $L$ wyrazaj $\mathrm{c}$ go $\mathrm{w}$ hektarach.

4. Na ile sposobów $\mathrm{m}\mathrm{o}\dot{\mathrm{z}}\mathrm{e}$ Krzyś rozdzielič 12 jednakowych cukierków pomiędzy siebie $\mathrm{i}$ trójkę

rodzeństwa, jeśli $\mathrm{k}\mathrm{a}\dot{\mathrm{z}}\mathrm{d}\mathrm{y}$ ma otrzymač co najmniej dwa cukierki.

5. $\mathrm{W}$ stozek wpisano sześcian $0$ krawędzi $a$. Rozwinięcie powierzchni bocznej stozka tworzy

wycinek koła $0$ kącie środkowym 1200. Ob1iczyč tangens kąta podjakim tworzącą tego stozka

widač ze środka sześcianu.

6. $\mathrm{W}$ trójkącie $ABC$ dane $\mathrm{S}\otimes$katy $\alpha \mathrm{i}\beta$ przy podstawie $\overline{AB}$ oraz środkowa $CD=s$ podstawy.

Obliczyč pole tego trójkąta.

7. Rozwiązač równanie $3^{\sin x}+9^{\sin x}+27^{\sin x}+\ldots= \displaystyle \frac{\sqrt{3}+1}{2},$

nieskończonego ciągu geometrycznego.

którego lewa strona jest sumą

8. Stosując zasadę indukcji matematycznej udowodnič nierównośč:

$1-\sqrt{2}+\sqrt{3}-\ldots+\sqrt{2n-1}>\sqrt{\frac{n}{2}},n\geq 1.$

9. Wyznaczyč wszystkie wartości parametru rzeczywistego $p, \mathrm{d}\mathrm{l}\mathrm{a}$ których krzywe $0$ równaniach

$y=\sqrt[3]{x}, y=x^{p}$ przecinajq się $\mathrm{w}$ pewnym punkcie pod kqtem $45^{0}$ Rozwiqzanie zilustrowač

odpowiednim rysunkiem.





PRACA KONTROLNA nr 5

luty $2005\mathrm{r}.$

l. Firma otrzymała zlecenie na wyprodukowanie 80000 sztuk pewnego wyrobu $\mathrm{w}$ terminie 60

$\mathrm{d}\mathrm{n}\mathrm{i}. \mathrm{K}\mathrm{a}\dot{\mathrm{z}}\mathrm{d}\mathrm{y} \mathrm{z} 20$ pracowników firmy $\mathrm{m}\mathrm{o}\dot{\mathrm{z}}\mathrm{e}$ wykonač $\mathrm{w}$ ciągu dnia 50 sztuk tego wyrobu.

Reszta zamówienia $\mathrm{m}\mathrm{o}\dot{\mathrm{z}}\mathrm{e}$ byč zrealizowana przez dotychczasowq załogę, ale za dodatkową

pracę nalezy zapłacič podwójnie. $\mathrm{M}\mathrm{o}\dot{\mathrm{z}}$ na $\mathrm{t}\mathrm{e}\dot{\mathrm{z}}$ zatrudnič pewną liczbę nowych pracowników,

którzy otrzymają 80\% wynagrodzenia sta1ych pracowników. Nowy pracownik $\mathrm{m}\mathrm{o}\dot{\mathrm{z}}\mathrm{e}$ po 4

dniach szkolenia wykonač 26 sztuk wyrobów $\mathrm{w}$ pierwszym dniu $\mathrm{i}$ zwiększač wydajnośč $0$

l sztukę dziennie $\mathrm{a}\dot{\mathrm{z}}$ do osiągnięcia 50 sztuk. I1u nowych pracowników na1ezałoby zatrudnič

wybierajqc drugi wariant $\mathrm{i}$ który wariant jest korzystniejszy dla firmy?

2. Wyznaczyč wszystkie liczby rzeczywiste a $\mathrm{i}b$, których iloczyn oraz róznica kwadratów są

równe ich sumie.

3. Dane są zbiory na p{\it l}aszczy $\acute{\mathrm{z}}\mathrm{n}\mathrm{i}\mathrm{e}A=\{(x,y):(x+y)(y-2x)\leq 0\}$ oraz $B=$

$\{(x,y):y(3-x)\geq x\}$. Zaznaczyč na rysunku zbiór $C=A\cap B$. Podač wszystkie punkty

zbioru $C$, których obie wspólrzędne są liczbami naturalnymi.

4. $\mathrm{W}$ czworokącie wypukłym ABCD przekqtne $\vec{AC}= [7,-1] \mathrm{i} \vec{BD}= [3$, 3$]$ przecinajq się

$\mathrm{w}$ punkcie $O$ odległym $0\sqrt{8}$ od wierzchołków $C\mathrm{i}D$. Wyznaczyč wektory $\overline{AB}\succ \mathrm{i}\overline{B}C\succ$ oraz

narysowač ten czworokąt.

5. Wazon $\mathrm{w}$ ksztalcie graniastosłupa prawidłowego trójkątnego $0$ krawędzi podstawy 4 cm $\mathrm{i}$

wysokości 25 cm napełniono ca1kowicie wodą. Następnie wy1ano częśč wody przechy1ajqc

wazon $\mathrm{w}$ taki sposób, $\dot{\mathrm{z}}\mathrm{e}$ poziom wody na dwóch krawędziach bocznych znajdował się $\mathrm{w}$

odległości 4 cm $\mathrm{i}3$ cm od górnego brzegu wazonu. Jaka wysokośč będzie miał słup wody $\mathrm{w}$

wazonie po ustawieniu go $\mathrm{z}$ powrotem $\mathrm{w}$ pozycji pionowej?

6. Zbadač monotonicznośč ciągu $0$ wyrazie ogólnym

$a_{n}=\displaystyle \frac{2^{n}+2^{n+1}+\ldots+2^{2n+1}}{2+2^{3}+\ldots+2^{2n+1}}.$

7. Sporządzič wykres funkcji $f(x)=\sqrt{5x-x^{2}}-2$ nie przeprowadzając badania jej przebiegu $\mathrm{i}$

podač nazwę otrzymanej krzywej. Na podstawie wykresu określič liczbę rozwiązań równania

$|\sqrt{5x-x^{2}}-2|=p\mathrm{w}$ zalezności od parametru rzeczywistego $p.$

8. Wykazač, $\dot{\mathrm{z}}\mathrm{e}$ równanie kwadratowe $ 3x^{2}+4x\sin\alpha-\cos 2\alpha = 0$ ma dla $\mathrm{k}\mathrm{a}\dot{\mathrm{z}}$ dej wartości

parametru $\alpha$ dwa rózne pierwiastki rzeczywiste. Wyznaczyč wszystkie wartości parametru

$\alpha\in[0,2\pi]$, dla których suma odwrotności pierwiastków tego równania jest nieujemna.

9. Wyznaczyč asymptoty, przedziały monotoniczności oraz ekstrema lokalne funkcji

$f(x)=|x-2|+\displaystyle \frac{5x-4}{2x^{3}}.$





PRACA KONTROLNA nr 6

marzec $2005\mathrm{r}.$

l. Suma cyfr liczby trzycyfrowej wynosi 9. Cyfra setek jest równa 1/81iczby złozonej $\mathrm{z}$ dwu

pozostałych cyfr, a cyfra jednostek jest takze równa 1/81iczby złozonej $\mathrm{z}$ dwu pozostałych

cyfr. Co to za liczba?

2. Obliczyč $\mathrm{t}\mathrm{g}\beta$, gdzie $\beta\in[0,\pi]$, wiedzqc, $\dot{\mathrm{z}}\mathrm{e}\cos\beta=\sin\alpha+\cos\alpha$ oraz $\dot{\mathrm{z}}\mathrm{e}$

tg $\displaystyle \alpha=-\frac{3}{4}, \alpha\in[0,\pi]. \mathrm{W}$ której čwiartce $\mathrm{l}\mathrm{e}\dot{\mathrm{z}}\mathrm{y}$ kąt $\alpha+\beta?$Odpowied $\acute{\mathrm{z}}$ uzasadnič nie wykonujac

obliczeń przyblizonych.

3. Wyznaczyč równania wszystkich parabol przechodzących przez punkt $P(1,\sqrt{3})$, których

wierzchołek $\mathrm{i}$ punkty przecięcia $\mathrm{z}\mathrm{o}\mathrm{s}\mathrm{i}\otimes Ox$ tworzq trójkąt równoboczny $0$ polu $\sqrt{3}$. Sporządzič

rysunek.

4. Rzucamy trzy razy kostką do gry. Jakie jest prawdopodobieństwo, $\dot{\mathrm{z}}\mathrm{e}$ wyniki kolejnych

rzutów utworzą a) ciąg arytmetyczny; b) ciąg rosnący?

5. $\mathrm{Z}$ punktu $P \mathrm{l}\mathrm{e}\dot{\mathrm{z}}$ qcego $\mathrm{w}$ odległości $R$ od powierzchni kuli $0$ promieniu $R$ poprowadzono

trzy pólproste styczne do tej kuli tworzące kąt trójścienny $0$ jednakowych kątach plaskich.

Obliczyč cosinus kata plaskiego tego trójścianu.

6. Okrąg $0$ promieniu $r$ przecina $\mathrm{k}\mathrm{a}\dot{\mathrm{z}}$ de $\mathrm{z}$ ramion kąta ostrego $2\gamma \mathrm{w}$ dwóch punktach $\mathrm{w}$ taki

sposób, $\dot{\mathrm{z}}\mathrm{e}$ wyznaczajq one dwie cięciwy jednakowej długości, a czworokąt utworzony przez

te cztery punkty ma największe pole. Obliczyč odleglośč środka okręgu od wierzcholka kąta?

7. Rozwiązač nierównośč

$\displaystyle \log_{x}\frac{1-2x}{2-x}\geq 1.$

8. Wyznaczyč $\mathrm{i}$ narysowač zbiór wszystkich punktów płaszczyzny, których suma odległości od

osi $Ox\mathrm{i}$ od okręgu $x^{2}+(y-1)^{2}=1$ wynosi 2.

9. Dana jest funkcja $f(x) =\displaystyle \cos 2x+\frac{2}{3}\sin x. |\sin x|$. a) Korzystając $\mathrm{z}$ definicji uzasadnič, $\dot{\mathrm{z}}\mathrm{e}$

$f'(0) = 0$. b) Znalez/č wszystkie punkty $\mathrm{z}$ przedzialu $[-\pi,\pi], \mathrm{w}$ których styczna do wy-

kresu funkcji $f(x)$ jest równoległa do stycznej $\mathrm{w}$ punkcie $x = \displaystyle \frac{\pi}{4}$. Rozwiqzanie zilustrowač

odpowiednim rysunkiem.





PRACA KONTROLNA nr 7

kwiecień $2005\mathrm{r}.$

l. Liczba czteroelementowych podzbiorów zbioru $A$ jest ll razy większa od liczby jego pod-

zbiorów dwuelementowych, a zbiór $B\subset A$ ma tyle samo podzbiorów czteroelementowych co

dwuelementowych. Ile podzbiorów co najwyzej trzyelementowych ma zbiór $A\backslash B$?

2. Reszta $\mathrm{z}$ dzielenia wielomianu $x^{3}+px^{2}-x+q$ przez trójmian $(x+2)^{2}$ wynosi $(-x+1).$

Obliczyč pierwiastki tego wielomianu.

3. Kula $\mathcal{K}$ jest styczna do wszystkich krawędzi czworościanu foremnego $0$ objętości 64 $\mathrm{c}\mathrm{m}^{3}$

Czworościan ten przecięto płaszczyznq równoległą do jednej ze ścian $\mathrm{i}$ styczną do kuli $\mathcal{K}.$

Obliczyč objętośč otrzymanego ostrosłupa ściętego.

4. Znalez$\acute{}$č wszystkie wartości parametru $p$, dla których przedział [1, 2] jest zawarty $\mathrm{w}$ dziedzinie

funkcji

$f(x)=\displaystyle \frac{\sqrt{x^{2}-3px+2p^{2}}}{\sqrt{x+p}}.$

5. Ze zbioru liczb czterocyfrowych wylosowano (ze zwracaniem) 4 liczby. Obliczyč prawdopodo-

bieństwo tego, $\dot{\mathrm{z}}\mathrm{e}$ co najmniej dwie $\mathrm{z}$ wylosowanych liczb czytane od strony lewej do prawej

lub od strony prawej do lewej są podzielne przez 4.

6. Nalezy wykonač stolik $0$ symetrycznym owalnym blacie, jak pokazano na rysunku obok,
\begin{center}
\includegraphics[width=49.128mm,height=29.004mm]{./KursMatematyki_PolitechnikaWroclawska_2004_2005_page6_images/image001.eps}
\end{center}
$0$ długosci l $\mathrm{m}\mathrm{i}$ szerokosci 60 cm. Projektant przyj ł, $\dot{\mathrm{z}}\mathrm{e}$

brzeg blatu będzie się składał $\mathrm{z}$ czterech łuków okręgow,

$\mathrm{k}\mathrm{a}\dot{\mathrm{z}}\mathrm{d}\mathrm{y}0\mathrm{k}$ cie srodkowym $90^{0}$ Jakie powinny byc pro-

mienie tych łuków, aby brzeg blatu $\mathrm{b}\mathrm{y}l$ krzyw gładk?

Podac powierzchnię blatu $\mathrm{z}$ dokladnością do l $\mathrm{c}\mathrm{m}^{2}$

7. Styczna do okręgu $x^{2} + y^{2} 4x 2y 5 = 0 \mathrm{w}$ punkcie $A(-1,2)$, prosta

$3x+4y-10=0$ oraz oś $Ox$ tworzq trójkąt. Obliczyč jego pole $\mathrm{i}$ sporządzič rysunek.

8. Rozwiązač równanie

$\displaystyle \mathrm{c}\mathrm{t}\mathrm{g}^{2}x-\mathrm{c}\mathrm{t}\mathrm{g}^{4}x+\mathrm{c}\mathrm{t}\mathrm{g}^{6}x-\ldots=\frac{1+\cos 3x}{2},$

którego lewa strona jest sumą nieskończonego ciągu geometrycznego.

9. Na walcu obrotowym $0$ wysokości równej średnicy podstawy opisano ostroslup prawidlo-

wy trójkątny $0$ najmniejszej objętości $\mathrm{i}$ taki, $\dot{\mathrm{z}}\mathrm{e}$ jedna $\mathrm{z}$ podstaw walca $\mathrm{l}\mathrm{e}\dot{\mathrm{z}}\mathrm{y}$ na podstawie

ostrosłupa. Obliczyč tangens kąta nachylenia ściany bocznej tego ostroslupa do podstawy.







xxxv

KORESPONDENCYJNY KURS Z MATEMATYKI

PRACA KONTROLNA nr l

$\mathrm{p}\mathrm{a}\acute{\mathrm{z}}$dziernik 2$005\mathrm{r}.$

l. Niech $f(x) = x^{2}+bx+5$. Wyznaczyč wszystkie wartości parametru $b$, dla których:

a) wykres funkcji $f$ jest symetryczny względem prostej $x=2$, b) wierzchofek paraboli

będqcej wykresem funkcji $f\mathrm{l}\mathrm{e}\dot{\mathrm{z}}\mathrm{y}$ na prostej $x+y+1=0$. Sporządzič staranny rysunek.

2. Kilkoro dzieci dostafo torebkę cukierków do równego podziału. Gdyby liczba dzieci byfa

$01$ mniejsza, to $\mathrm{k}\mathrm{a}\dot{\mathrm{z}}$ de $\mathrm{z}$ nich dostafoby $02$ cukierki więcej. Gdyby cukierków byfo dwa

razy więcej, a dzieci $0$ dwoje więcej, to $\mathrm{k}\mathrm{a}\dot{\mathrm{z}}$ de dostałoby $05$ cukierków wiecej. Ile było

dzieci a ile cukierków?

3. Babcia zalozyła swemu rocznemu wnukowi lokatę $\mathrm{w}$ wysokości 1000 $\mathrm{z}l$ oprocentowanq $\mathrm{w}$

wysokości 6\% $\mathrm{w}$ skali roku $\mathrm{z}$ półroczną kapitalizacją odsetek $\mathrm{i}$ postanowiła co 6 miesięcy

wplacač na to konto 100 $\mathrm{z}\mathrm{f}$. Jaka sumę dostanie wnuczek $\mathrm{w}$ dniu swoich osiemnastych

urodzin?

4. Dane są wierzcholki $A(-3,2), C(4,2), D(0,4)$ trapezu równoramiennego ABCD, $\mathrm{w}$ któ-

rym $\overline{AB}||\overline{CD}$. Wyznaczyč współrzędne wierzchołka $B$ oraz pole trapezu. Sporządzič

rysunek.

5. Wyznaczyč stosunek dlugości przekątnych rombu wiedząc, $\dot{\mathrm{z}}\mathrm{e}$ stosunek pola kofa wpisa-

nego $\mathrm{w}$ ten romb do pola rombu wynosi $\displaystyle \frac{\pi}{5}.$

6. Podstawą prostopadłościanu jest prostokąt $0$ dluzszym boku $a$. Przekątna prostopadlo-

ścianu tworzy $\mathrm{z}$ przekątnymi ścian bocznych kąty $\alpha$ oraz $ 2\alpha$. Obliczyč objetośč tego

prostopadfościanu. Dla jakich kątów $\alpha$ zadanie ma rozwiązanie?

7. Dla jakich wartości parametru $p$ funkcja

$f(x)=\displaystyle \frac{x^{3}}{px^{2}+px+1}$

jest określona $\mathrm{i}$ rosnąca na calej prostej rzeczywistej?

8. Rozwiązač równanie

ctg $x=2\sqrt{3}\sin x.$

9. Liczby $a_{1} = (\sqrt{2})^{\log_{\frac{1}{2}}16}$ oraz $a_{2} = 16^{-\log_{\sqrt[3]{2}}\sqrt[4]{2}}$ są odpowiednio pierwszym $\mathrm{i}$ drugim

wyrazem pewnego ciqgu geometrycznego. Rozwiązač nierównośč

$(\sqrt{x})^{\log^{2}x-1}\geq 2S,$

gdzie S oznacza sumę wszystkich wyrazów tego ciągu.




PRACA KONTROLNA nr 2

listopad $2005\mathrm{r}.$

l. Stop zawiera 60\% srebra próby 0,6 $\mathrm{i}$ 30\% srebra próby 0,7 oraz 20 dkg srebra próby 0,8.

a) Ile srebra $\mathrm{i}$ jakiej próby nalez $\mathrm{y}$ dodač, by otrzymač 2,5 kg srebra próby 0,7?

b) Obliczyč próbę stopu, jakim nalezy zastapič połowę danego stopu, by otrzymač stop

$0$ próbie 0,75?

2. Wyznaczyč wszystkie punkty okręgu $0$ środku $(0,0)\mathrm{i}$ promieniu 5, których i1oczyn kwa-

dratów wspólrzędnych jest najmniejszą wspólną wielokrotnościa liczb 12 $\mathrm{i} 14$. Obliczyč

obwód wielokąta, którego wierzchofkami $\mathrm{s}\Phi$ znalezione punkty. Bez $\mathrm{u}\dot{\mathrm{z}}$ ywania kalkulatora

zbadač, czy jest on większy od 30.

3. Dla jakich wartości $a \mathrm{i} b$ wielomian $W(x) = x^{4}-3x^{3}+bx^{2}+ax+b$ jest podzielny

przez trójmian kwadratowy $(x^{2}-1)$ ? Dla znalezionych wartości wspófczynników $a\mathrm{i}b$

rozwiązač nierównośč $W(x)\leq 0.$

4. Wykorzystuj $\Phi^{\mathrm{C}}\mathrm{t}\mathrm{o}\dot{\mathrm{z}}$ samośč trygonometryczną $\displaystyle \sin\alpha+\sin\beta=2\sin\frac{\alpha+\beta}{2}\cos\frac{\alpha-\beta}{2}$ narysowač

staranny wykres funkcji $f(x)=|\sin x+\cos x|$. Korzystając $\mathrm{z}$ tego wykresu, wyznaczyč

najmniejszą $\mathrm{i}$ największa wartośč funkcji $f$ na przedziale $[-\displaystyle \frac{\pi}{2},\pi]$. Wyznaczyč rozwiązania

równania $f(x)=\displaystyle \frac{1}{\sqrt{2}}$ zawarte $\mathrm{w}$ tym przedziale.

5. Pole powierzchni całkowitej stozka jest dwa razy większe od pola powierzchni kuli wpi-

sanej $\mathrm{w}$ ten stozek. Znalez/č cosinus kąta nachylenia $\mathrm{t}\mathrm{w}\mathrm{o}\mathrm{r}\mathrm{z}\Phi^{\mathrm{C}\mathrm{e}\mathrm{j}}$ stozka do podstawy.

6. $\mathrm{W}$ trójkącie równoramiennym suma długości ramienia $\mathrm{i}$ promienia okręgu opisanego

na tym trójkącie równa jest $m$ a wysokośč trójkąta równa jest 2. Wyznaczyč długośč

ramienia jako funkcję parametru $m$ oraz wartośč $m$, dla której kąt przy wierzchofku

trójkąta równy jest $120^{\mathrm{o}}$? Dla jakich wartości $m$ zadanie ma rozwiązanie?

7. Narysowač zbiory $A=\{(x,y):x^{2}+2x+y^{2}\leq 0\}, B= \{(x,y):x^{2}+2y+y^{2}\leq 0\},$

$C=\{(x,y):x\leq 0,y\geq 0,x^{2}+y^{2}\leq 4\}$. Obliczyč pola figur $A\cap B, A\backslash B, C\backslash (A\cup B).$

Podač równania osi symetrii figury $A\cup B.$

8. Rozwiązač nierównośč $\displaystyle \frac{1}{\sqrt{4-x^{2}}}\leq\frac{1}{x-1}.$

9 $\mathrm{W}\mathrm{z}$naczyč równania wszystkich pstych stycznychktóre s$\text{ą} \mathrm{p}\mathrm{o}\mathrm{s}\mathrm{o}$padle d$\mathrm{o}\mathrm{p}$rostej orównaniu {\it x}$+y=0.$Obliczyčdo wykres p$\mathrm{o}1\mathrm{e}\mathrm{r}\text{ó} \mathrm{w}\mathrm{n}\mathrm{o}1\mathrm{e}\mathrm{g}1$obfunkc {\it f}$(x)=\displaystyle \frac{8x}{x^{2}+3,\mathrm{o}\mathrm{k}\mathrm{u}},$

którego wierzchołkami są punkty wspólne tych stycznych $\mathrm{z}$ wykresem funkcji $f(x).$





PRACA KONTROLNA nr 3

grudzień $2005\mathrm{r}$

l. Droge $\mathrm{z}$ miasta $A$ do miasta $B$ rowerzysta pokonuje $\mathrm{w}$ ciqgu 3 godzin. Po długotrwałych

deszczach stan $\displaystyle \frac{3}{5}$ drogi pogorszyf się na tyle, $\dot{\mathrm{z}}\mathrm{e}$ na tym odcinku rowerzysta $\mathrm{m}\mathrm{o}\dot{\mathrm{z}}\mathrm{e}$ jechač

$\mathrm{z}$ prędkością $04\mathrm{k}\mathrm{m}/\mathrm{h}$ mniejszą. By czas podrózy $\mathrm{z}A$ do $B$ nie uległ zmianie, zmuszony

jest na pozostafym odcinku zwiększyč prędkośč $012\mathrm{k}\mathrm{m}/\mathrm{h}$. Jaka jest odległośč $\mathrm{z}A$ do

$B\mathrm{i}\mathrm{z}$ jaką prędkością $\mathrm{j}\mathrm{e}\acute{\mathrm{z}}$dził rowerzysta przed ulewami?

2. Niech $f(x)=|4-|x-2||+1$. Sporządzič staranny wykres funkcji $f\mathrm{i}$ posługując $\mathrm{s}\mathrm{i}\mathrm{e}$ nim:

a) wyznaczyč najmniejszą $\mathrm{i}$ największ$\Phi$ wartośč funkcji $f\mathrm{w}$ przedziale $[0$, 7$]$, b) podač

równanie osi symetrii wykresu funkcji $f$, c) wyznaczyč $a>0\mathrm{t}\mathrm{a}\mathrm{k}$, aby pole figury ogra-

niczonej osiami układu, wykresem funkcji $f$ oraz prostą $x=a$ było równe 32.

3. Promień światfa przechodzi przez punkt $A(1,1)$, odbija się od prostej $0$ równaniu $y=$

$x-2$ (zgodnie $\mathrm{z}$ zasadą mówiącą, $\dot{\mathrm{z}}\mathrm{e}$ kąt padania jest równy kątowi odbicia) $\mathrm{i}$ przechodzi

przez punkt $B(4,6)$. Wyznaczyč wspófrzędne punktu odbicia $P$ oraz równania prostych,

po których biegnie promień przed $\mathrm{i}$ po odbiciu.

4. Na egzaminie uczeń wybiera losowo 4 pytania $\mathrm{z}$ zestawu egzaminacyjnego liczącego 40

pytań. Aby zdač egzamin nalez $\mathrm{y}$ poprawnie odpowiedzieč na co najmniej dwa pytania.

Jakie jest prawdopodobieństwo zdania egzaminu przez ucznia znajqcego odpowiedzi na

40\% pytań $\mathrm{z}$ zestawu egzaminacyjnego?

5. $\mathrm{W}$ ciągu arytmetycznym $(a_{n})$ mamy $a_{1}+a_{3}=3$ oraz $a_{1}a_{4}=1$. Dla jakich $n$ prawdziwa

jest nierównośč $a_{1}+a_{2}+a_{3}+\ldots+a_{n}\leq 93$?

6. Trójk$\Phi$t prostokątny $0$ przyprostokątnych $a, b$ obracamy wokóf środkowej najdluzszego

boku. Obliczyč objętośč otrzymanej bryly.

7. Korzystając $\mathrm{z}$ zasady indukcji matematycznej wykazač, $\dot{\mathrm{z}}\mathrm{e}$ dla $\mathrm{k}\mathrm{a}\dot{\mathrm{z}}$ dej liczby naturalnej

$n$ liczba $7^{n}-(-3)^{n}$ dzieli się przez 10.

8. Dla jakich wartości parametru rzeczywistego $m$ równanie

$2^{2x}-2(m-1)2^{x}+m^{2}-m-2=0$

ma dokładnie jeden pierwiastek rzeczywisty?

9. Wśród graniastosfupów prawidfowych sześciokątnych $0$ danym polu powierzchni cafkowi-

tej $S=27\sqrt{3}\mathrm{d}\mathrm{m}^{2}$ wskazač graniastosłup $0$ największej objętości. Podač objętośč tego

graniastosfupa $\mathrm{z}$ dokfadnością do l $\mathrm{m}1.$





PRACA KONTROLNA nr 4

styczeń 2006r.

l. Rozwiązač uklad równań

$\left\{\begin{array}{l}
x^{2}-y^{2}\\
x^{3}+y^{3}
\end{array}\right.$

$2(x-y)$

$6-(x-y)$

2. Dany jest punkt $P(3,2)$ oraz dwie proste $k\mathrm{i}l\mathrm{o}$ równaniach odpowiednio: $x+y+4=0$

$\mathrm{i}2x-3y-9=0$. Znalez/č taki punkt $Q$ na prostej $l$, aby środek odcinka $\overline{PQ}\mathrm{l}\mathrm{e}\dot{\mathrm{z}}$ af na

prostej $k$. Rozwiqzanie zilustrowač odpowiednim rysunkiem.

3. Dlajakich wartości parametru rzeczywistego $a\neq 0$ pierwiastki wielomianu $w(x)=a^{2}x^{3}-$

$a^{2}x^{2}-(a^{2}+1)x+a^{2}-1$ są trzema pierwszymi wyrazami pewnego ciągu arytmetycznego?

Dla $\mathrm{k}\mathrm{a}\dot{\mathrm{z}}$ dego otrzymanego przypadku obliczyč czwarty wyraz ciągu.

4. Znalez/č liczbę trzycyfrową wiedząc, $\dot{\mathrm{z}}\mathrm{e}$ iloraz $\mathrm{z}$ dzielenia tej liczby przez sumę jej cyfr

jest równy 48, a róznica szukanej 1iczby $\mathrm{i}$ liczby napisanej tymi samymi cyframi, ale $\mathrm{w}$

odwrotnym porządku wynosi 198.

5. $\mathrm{W}$ okrąg wpisano trapez $\mathrm{t}\mathrm{a}\mathrm{k}, \dot{\mathrm{z}}\mathrm{e}$ jedna $\mathrm{z}$ jego podstaw jest średnicą okręgu. Stosunek

długości obwodu trapezu do sumy długości jego podstaw jest równy $\displaystyle \frac{3}{2}$. Obliczyč cosinus

kąta ostrego $\mathrm{w}$ tym trapezie.

6. Na ostrosłupie prawidłowym trójkątnym opisano stozek, a na tym stozku opisano ku-

lę. $K_{\Phi^{\mathrm{t}}}$ przy wierzcholku przekroju osiowego stozka jest równy $\alpha$. Obliczyč stosunek

objętości kuli do objętości ostrosłupa.

7. Rozwiązač nierównośč

$-\infty^{1}1$ -$0_{m}11 m \rightarrow$ -{\it m} l

$1. \perp \mathrm{L}\cdot\Delta \mathrm{v}\mathrm{v}1*^{\Delta}\omega\vee\perp\perp 1\vee\perp\cdot \mathrm{v}\mathrm{v}\perp\perp\cdot \mathrm{o}\mathrm{c}$

$3^{x+\frac{1}{2}}-2^{2x+1}<4^{x}-5\cdot 3^{x-\frac{1}{2}}$

8. Zbadač przebieg zmienności $\mathrm{i}$ sporządzič staranny wykres funkcji $f(x) = \displaystyle \frac{4-x^{2}}{x_{/}^{2}-1}$. Na-

stępnie narysowač wykres funkcji $k=g(m)$, gdzie $k$ jest liczbą pierwiastków rownania

$|$--{\it x}4-2-{\it x}21$|=$ {\it m}.

9. Ze zbioru cyfr $\{0$, 1, 2, 3$\}$ wylosowano dwie $\mathrm{i}$ odrzucono. $\mathrm{Z}$ otrzymanego zbioru wyloso-

wano ze zwracaniem pięč cyfr. Jakie jest prawdopodobieństwo, $\dot{\mathrm{z}}\mathrm{e}$ liczba utworzona $\mathrm{z}$

tych cyfr jest podzielna przez 3?





PRACA KONTROLNA nr 5

luty $2006\mathrm{r}.$

l. Przyprostokqtne trójkąta prostokqtnego mają długości 6 $\mathrm{i}8$ cm. $\mathrm{W}$ trójkąt ten wpisano

kwadrat $\mathrm{t}\mathrm{a}\mathrm{k}, \dot{\mathrm{z}}\mathrm{e}$ dwa jego wierzchofki $ 1\mathrm{e}\mathrm{Z}\otimes$ na przeciwprostokątnej, a dwa pozostafe na

przyprostokątnych. Obliczyč pola figur, na jakie brzeg kwadratu dzieli dany trójkąt.

2. Niech $A$ będzie zbiorem tych punktów $x$ osi liczbowej, których suma odleglości od punk-

tów $-1\mathrm{i}5$jest mniejsza od 12, a $B=\{x\in R:\sqrt{x^{2}-25}-x<1\}$. Znalez/č $\mathrm{i}$ zaznaczyč

na osi liczbowej zbiory $A, B$ oraz $(A\backslash B)\cup(B\backslash A).$

3. Wykazač, $\dot{\mathrm{z}}\mathrm{e}$ liczba $x=\sqrt[3]{2\sqrt{6}+4}-\sqrt[3]{2\sqrt{6}-4}$

Wskazówka: obliczyč $x^{3}$

jest niewymierna.

4. Wyznaczyč zbiór wszystkich wartości parametru $m$, dla których równanie

$\displaystyle \cos x=\frac{3m}{m^{2}-4}$

ma rozwiązanie $\mathrm{w}$ przedziale $[-\displaystyle \frac{\pi}{3},\frac{\pi}{3}]$. Obliczyč ctg $x$ dla cafkowitych $m\mathrm{z}$ tego zbioru.

5. $\mathrm{W}$ ostrosłupie prawidłowym sześciokątnym przekrój $0$ najmniejszym polu płaszczyzną

zawieraj $\Phi^{\mathrm{C}}\Phi$ wysokośč ostrosłupa jest trójkątem równobocznym $0$ boku $2a$. Obliczyč co-

sinus kąta dwuściennego między ścianami bocznymi tego ostroslupa.

6. Dane jest pófkole $0$ średnicy AB $\mathrm{i}$ promieniu długości $|AO| = r$. Na promieniu $AO$

jako na średnicy wewnątrz danego pólkola zakreślono pófokrąg. Na większym pófokręgu

obrano punkt $P \mathrm{i}$ polączono go $\mathrm{z}$ punktami A $\mathrm{i} B$. Odcinek $AP$ przecina mniejszy

pólokrąg $\mathrm{w}$ punkcie $C$. Obliczyč dfugośč odcinka $AP, \mathrm{j}\mathrm{e}\dot{\mathrm{z}}$ eli wiadomo, $\dot{\mathrm{z}}\mathrm{e}|CP|+|PB|=1.$

Przeprowadzič analizę dla jakich wartości $r$ zadanie ma rozwiązanie.

7. Zbadač monotonicznośč ciągu $a_{n} = \displaystyle \frac{n-2}{\sqrt{n^{2}+1}}$. Obliczyč granicę tego ciągu, a następnie

znalez/č wszystkie jego wyrazy odlegle od granicy co najmniej $0\displaystyle \frac{1}{10}.$

8. Wykazač, $\dot{\mathrm{z}}\mathrm{e}$ pole trójkąta ograniczonego styczną do wykresu funkcji $y = \displaystyle \frac{2x-3}{x-2} \mathrm{i}$ jego

asymptotami jest stałe. Sporządzič rysunek.

9. Rozwiązač uklad równań

$\left\{\begin{array}{l}
\log_{(x-y)}[8(x+y)]\\
(x+y)^{\log_{4}(x-y)}
\end{array}\right.$

$-2$

-21







XXXVI

KORESPONDENCYJNY KURS Z MATEMATYKI

PRACA KONTROLNA $\mathrm{n}\mathrm{r}1-$ POZIOM PODSTAWOWY

$\mathrm{p}\mathrm{a}\acute{\mathrm{z}}$dziernik 2$006\mathrm{r}.$

l. Róznica pewnej liczby trzycyfrowej $\mathrm{i}$ liczby otrzymanej za pomocą tych samych cyfr

zapisanych $\mathrm{w}$ odwrotnej kolejności równa jest 495, a suma równa jest 1009. Jaka to

liczba.

2. Obliczyč $p=\displaystyle \frac{64^{\frac{1}{3}}\sqrt{8}+8^{\frac{1}{3}}\sqrt{64}}{\sqrt[3]{64\sqrt{8}}}$. Znalez/č wszystkie liczby naturalne, dla których spełniona

jest nierównośč $x^{3}-2x^{2}-p^{2}x+2p^{2}\leq 0.$

3. Polowę kolekcji letniej sprzedano po zafozonej cenie. Po obnizce ceny $0$ 50\% udalo się

sprzedač pofowę pozostalej części towaru $\mathrm{i}$ dopiero kolejna 50\%-owa obnizka pozwo1ifa

sklepowi pozbyč się produktu.

a) Ile procent zaplanowanego przychodu stanowi uzyskana ze sprzedaz $\mathrm{y}$ kwota?

b) $\mathrm{O}$ ile procent wyjściowa cena towaru powinna byla byč $\mathrm{w}\mathrm{y}\dot{\mathrm{z}}$ sza, by sklep uzyskaf

zaplanowany początkowo przychód? Wyniki podač $\mathrm{z}$ dokładnością do l promila.

4. Dach wiezy kościola ma ksztalt ostrosfupa, którego podstawq jest sześciokąt foremny $0$

boku 2 $\mathrm{m}$ a największy $\mathrm{z}$ przekrojów pfaszczyzną $\mathrm{z}\mathrm{a}\mathrm{w}\mathrm{i}\mathrm{e}\mathrm{r}\mathrm{a}\mathrm{j}_{\Phi}\mathrm{c}\text{ą}$ wysokośč jest trójkątem

równobocznym. Obliczyč kubaturę dachu wiezy kościoła. Ile 2-1itrowych puszek farby

antykorozyjnej trzeba kupič do pomalowania blachy, którą pokryty jest dach, $\mathrm{j}\mathrm{e}\dot{\mathrm{z}}$ eli wia-

domo, $\dot{\mathrm{z}}\mathrm{e} 1$ litr farby wystarcza do pomalowania 6 $\mathrm{m}^{2}$ blachy $\mathrm{i}$ trzeba uwzględnič 8\%

farby na ewentualne straty.

5. Niech

$f(x)=$

dla

dla

$x\leq 1,$

$x>1.$

a) Narysowač wykres funkcji $f\mathrm{i}$ na jego podstawie wyznaczyč zbiór wartości funkcji.

b) Obliczyč $f(\sqrt{3}-1)$ oraz $f(3-\sqrt{3}).$

c) Rozwiqzač nierównośč $2\sqrt{f(x)}\leq 3\mathrm{i}$ zaznaczyč na osi $\mathrm{o}x$ zbiór rozwiązań.

6. Punkt $A=(1,0)$ jest wierzchofkiem rombu $0$ kącie przy tym wierzchołku równym $60^{\mathrm{o}}$

Wyznaczyč wspófrzędne pozostafych wierzchofków rombu wiedząc, $\dot{\mathrm{z}}\mathrm{e}$ dwa $\mathrm{z}$ nich lezą

na prostej $l$ : $2x-y+3=0$. Obliczyč pole rombu. Ile rozwiązań ma to zadanie?




PRACA KONTROLNA $\mathrm{n}\mathrm{r}1-$ POZIOM ROZSZERZONY

l. Rozwiązač nierównośč $\displaystyle \frac{1}{\sqrt{4-x^{2}}}\geq\frac{1}{x-1}\mathrm{i}$ starannie zaznaczyč zbiór rozwiqzań na osi liczbo-

wej.

2. Rozwiązač równanie 2 $\sin 2x+2\sin x-2\cos x=1$. Następnie podač rozwiązania nalezące

do przedziału $[-\pi,\pi].$

3. $\mathrm{Z}$ przystani A wyrusza $\mathrm{z}$ biegiem rzeki statek do przystani $\mathrm{B}$, odlegfej od A $0140$ km. Po

upływie l godziny wyrusza za nim łódz$\acute{}$ motorowa, dopędza statek, po czym wraca do

przystani A $\mathrm{w}$ tym samym momencie, $\mathrm{w}$ którym statek przybija do przystani B. Znalez$\acute{}$č

prędkośč biegu rzeki, $\mathrm{j}\mathrm{e}\dot{\mathrm{z}}$ eli wiadomo, $\dot{\mathrm{z}}\mathrm{e}\mathrm{w}$ stojącej wodzie prędkośč statku wynosi 16

$\mathrm{k}\mathrm{m}/$godz, a prędkośč łodzi 24 $\mathrm{k}\mathrm{m}/$godz.

4. Dane są liczby: $m=\displaystyle \frac{(_{4}^{6})\cdot(_{2}^{8})}{(_{3}^{7})}, n=\displaystyle \frac{(\sqrt{2})^{-4}(\frac{1}{4})^{-\frac{5}{2}}\sqrt[4]{3}}{(\sqrt[4]{16})_{27^{-\frac{1}{4}}}^{3}}.$

a) Sprawdzič, wykonując odpowiednie obliczenia, $\dot{\mathrm{z}}\mathrm{e}m, n$ są liczbami naturalnymi.

b) Wyznaczyč $k\mathrm{t}\mathrm{a}\mathrm{k}$, by liczby $m, k, n$ były odpowiednio: pierwszym, drugim $\mathrm{i}$ trzecim

wyrazem ciągu geometrycznego.

c) Wyznaczyč sumę wszystkich wyrazów nieskończonego ciągu geometrycznego, któ-

rego pierwszymi trzema wyrazami są $m, k, n$. Ile wyrazów tego ciągu nalez $\mathrm{y}$ wziąč,

by ich suma przekroczyła 95\% sumy wszystkich wyrazów?

5. $\mathrm{Z}$ wierzchofka $A$ kwadratu ABCD $0$ boku $a$ poprowadzono dwie proste, które dzielą kąt

przy tym wierzchołku na trzy równe części $\mathrm{i}$ przecinają boki kwadratu $\mathrm{w}$ punktach $K\mathrm{i}$

$L$. Wyznaczyč długości odcinków, najakie te proste dzielą przekątną kwadratu. Znalez$\acute{}$č

promień okręgu wpisanego $\mathrm{w}$ deltoid AKCL.

6. Podstawą pryzmy przedstawionej na rysunku ponizej jest prostokąt ABCD,
\begin{center}
\includegraphics[width=87.432mm,height=33.528mm]{./KursMatematyki_PolitechnikaWroclawska_2006_2007_page1_images/image001.eps}
\end{center}
{\it K} $\mathrm{L}$

C

$\mathrm{b}$

zmy.

A a $\mathrm{B}$

gosč $b$, gdzie $a>b$. Wszystkie ściany boczne

pryzmy $\mathrm{s}$ nachylone pod $\mathrm{k}$ tem $\alpha$ do płasz-

czyzny podstawy. Obliczyc objętosc tej pry-

ktorego bok $AB$ ma dlugośc $a$, a bok $BC$ dfu-





PRACA KONTROLNA $\mathrm{n}\mathrm{r}6-$ POZIOM PODSTAWOWY

marzec 2007r.

l. Boki trójk$\Phi$ta prostokątnego $0$ polu 12 $\mathrm{t}\mathrm{w}\mathrm{o}\mathrm{r}\mathrm{z}\Phi$ ciąg arytmetyczny. Wyznaczyč promień

okręgu wpisanego $\mathrm{w}$ ten trójkąt.

2. Pan Kowalski zaciągnąf 3l grudnia $\mathrm{p}\mathrm{o}\dot{\mathrm{z}}$ yczkę 4000 z1otych oprocentowaną $\mathrm{w}$ wysokości

18\% $\mathrm{w}$ skali roku. Zobowiązaf się splacič ją $\mathrm{w}$ ciągu roku $\mathrm{w}$ trzech równych ratach

płatnych 30 kwietnia, 30 sierpnia $\mathrm{i}30$ grudnia. Oprocentowanie $\mathrm{p}\mathrm{o}\dot{\mathrm{z}}$ yczki liczy się od l

stycznia, a odsetki od kredytu naliczane są $\mathrm{w}$ terminach pfatności rat. Obliczyč wysokośč

tych rat $\mathrm{w}$ zaokrągleniu do pełnych groszy.

3. Narysowač wykres funkcji $f(x)=$

$\mathrm{i}$ na jego podstawie wyznaczyč:

dla

dla

dla

$x<0,$

$x=0,$

$x>0,$

a) zbiór, jaki tworzą wartości funkcji $f(x)$, gdy $x$ przebiega przedzial $(-2,1)$ ;

b) zbiór rozwi$\Phi$zań nierówności $\displaystyle \frac{1}{2}\leq f(x)\leq 2.$

4. Suma wysokości $h$ ostrosłupa prawidłowego czworokątnego $\mathrm{i}$ jego krawędzi bocznej $b$

równa jest 12. D1a jakiej wartości $h$ objętośč tego ostroslupa jest najwieksza? Obliczyč

pole powierzchni cafkowitej ostrosfupa dla tej wartości $h.$

5. Punkty $A(0,4) \mathrm{i}D(3,5)$ są wierzchołkami trapezu równoramiennego ABCD, którego

podstawy $\overline{AB}$ oraz $\overline{CD}$ są prostopadfe do prostej $k\mathrm{o}$ równaniu $x-y-2=0$. Wyznaczyč

wspólrzędne pozostałych wierzchołków wiedząc, $\dot{\mathrm{z}}\mathrm{e}$ wierzchołek $C \mathrm{l}\mathrm{e}\dot{\mathrm{z}}\mathrm{y}$ na prostej $k.$

Znalez$\acute{}$č współrzędne środka oraz promień okręgu opisanego na tym trapezie.

6. Na kole $0$ promieniu $r$ opisano romb. Punkty styczności są wierzcholkami $\mathrm{c}\mathrm{z}\mathrm{w}\mathrm{o}\mathrm{r}\mathrm{o}\mathrm{k}_{\Phi}\mathrm{t}\mathrm{a}$

ABCD. Zakładając, $\dot{\mathrm{z}}\mathrm{e}$ stosunek pola rombu do pola czworokqta równy jest $\displaystyle \frac{8}{3}$, obliczyč

dlugośč boku rombu ijego $\mathrm{p}\mathrm{r}\mathrm{z}\mathrm{e}\mathrm{k}_{\Phi^{\mathrm{t}}}$nych. Obliczyč pole jednego $\mathrm{z}$ obszarów ograniczonych

bokami rombu $\mathrm{i}$ okręgiem.





PRACA KONTROLNA $\mathrm{n}\mathrm{r}6-$ POZIOM ROZSZERZONY

l. Dla jakich wartości parametru $\alpha\in[0,2\pi]$ istnieje dodatnie maksimum funkcji

$ f(x)=(2\cos\alpha-1)x^{2}-2x+\cos\alpha$ ?

2. Granicą ciągu $0$ wyrazie ogólnym $a_{n}=\displaystyle \frac{\sqrt{n^{4}+an^{3}+bn}-n^{2}}{\sqrt{n^{2}+1}}$ jest większy $\mathrm{z}$ pierwiastków

równania $4x^{\log x}+10x^{-\log x}=41$. Wyznaczyč parametry a $\mathrm{i}b.$

3. Wyznaczyč równanie krzywej utworzonej przez punkty, których odlegfośč od osi $0x$ jest

taka sama, jak odległośč od pólokręgu $0$ równaniu $y=\sqrt{2x-x^{2}}$. Sporzqdzič rysunek.

4. $\mathrm{W}$ stozku ściętym $\mathrm{P}^{\mathrm{r}\mathrm{z}\mathrm{e}\mathrm{k}}\Phi^{\mathrm{t}\mathrm{n}\mathrm{e}}$ przekroju osiowego $\mathrm{p}\mathrm{r}\mathrm{z}\mathrm{e}\mathrm{c}\mathrm{i}\mathrm{n}\mathrm{a}\mathrm{j}_{\Phi}$ się pod $\mathrm{k}_{\Phi}\mathrm{t}\mathrm{e}\mathrm{m}$ prostym, $\mathrm{a}$

tworząca $0$ dfugości $l$ nachylona jest do płaszczyzny podstawy dolnej pod kątem $\alpha.$

Obliczyč pole powierzchni bocznej tego stozka ściętego oraz pole powierzchni opisanej

na nim kuli.

5. $\mathrm{W}$ trójkącie $\triangle ABC$ dane są podstawa $|AB|=a$, kąt ostry przy podstawie $\angle CAB=2\alpha$

$\mathrm{i}$ dwusieczna tego kąta $|AD|=d$. Obliczyč pole koła opisanego na tym trójkącie. Podač

warunek istnienia rozwiązania.

6. Zbadač przebieg zmienności funkcji określonej wzorem

$f(x)=\displaystyle \sqrt{x+1}+1+\frac{1}{\sqrt{x+1}}+\ldots,$

gdzie prawa stronajest sumą wyrazów nieskończonego ciągu geometrycznego. Narysowač

jej staranny wykres.





PRACA KONTROLNA $\mathrm{n}\mathrm{r}2-$ POZIOM PODSTAWOWY

listopad $2006\mathrm{r}.$

l. Liczba dwuelementowych podzbiorów zbioru $A$ jest 7 razy większa $\mathrm{n}\mathrm{i}\dot{\mathrm{z}}$ liczba dwuele-

mentowych podzbiorów zbioru $B$. Liczba dwuelementowych podzbiorów zbioru $A$ nie

zawierających ustalonego elementu $a\in A$ jest 5 razy większa $\mathrm{n}\mathrm{i}\dot{\mathrm{z}}$ liczba dwuelemento-

wych podzbiorów zbioru $B$. Ile elementów ma $\mathrm{k}\mathrm{a}\dot{\mathrm{z}}\mathrm{d}\mathrm{y}\mathrm{z}$ tych zbiorów? Ile $\mathrm{k}\mathrm{a}\dot{\mathrm{z}}\mathrm{d}\mathrm{y}\mathrm{z}$ tych

zbiorów ma podzbiorów trzyelementowych?

2. $A\cap B, A\backslash B\mathrm{i}B\backslash $apisa '$\mathrm{w}\mathrm{p}$ostaciNiech {\it A}$=\displaystyle \{x\in 1\mathrm{R}:\frac{1}{x^{2}+23,A\mathrm{z}}\geq\frac{1}{10x,\mathrm{C}}\}$oraz {\it B}$=\mathrm{p}\mathrm{r}!^{x\in 1\mathrm{R}:|x-2|<\frac{7}{2}\}.\mathrm{Z}\mathrm{b}\mathrm{i}\mathrm{o}\mathrm{r}\mathrm{y}A,B,A\cup B}\mathrm{e}\mathrm{d}\mathrm{z}\mathrm{i}\mathrm{a}1\text{ó} \mathrm{w}1$iczbowych izaznaczyč j$\mathrm{e}\mathrm{n}\mathrm{a}\mathrm{o}\mathrm{s}\mathrm{i}$

liczb owej.

3. Stosując wzory skróconego mnozenia sprowadzič do najprostszej postaci wyrazenie

$W=2$ (sin6 $\alpha+\cos^{6}\alpha$)$-(\sin^{4}\alpha+\cos^{4}\alpha).$

Wykorzystując wzór $\cos 2\alpha = \cos^{2}\alpha-\sin^{2}\alpha$

wyrazenie $W$ przyjmuje wartośč $\displaystyle \frac{1}{2}.$

obliczyč, dla jakich wartości kąta $\alpha$

4. Wiadomo, $\dot{\mathrm{z}}\mathrm{e}$ liczby $-1$, 3 są pierwiastkami wielomianu $W(x)=x^{4}-ax^{3}-4x^{2}+bx+3.$

Wyznaczyč $a, b\mathrm{i}$ rozwiązač nierównośč $\sqrt{W(x)}\leq x^{2}-x.$

5. Na kole $0$ promieniu $r$ opisano trapez równoramienny, $\mathrm{w}$ którym stosunek dlugości pod-

staw wynosi 4: 3. Ob1iczyč stosunek po1a kofa do po1a trapezu oraz cosinus kąta ostrego

$\mathrm{w}$ tym trapezie.

6. $\mathrm{W}$ ostroslupie prawidłowym $\mathrm{c}\mathrm{z}\mathrm{w}\mathrm{o}\mathrm{r}\mathrm{o}\mathrm{k}_{\Phi^{\mathrm{t}}}\mathrm{n}\mathrm{y}\mathrm{m}$ wszystkie krawędzie $\mathrm{s}\Phi$ równe $a$. Obliczyč

objętośč tego ostroslupa. Znalez/č cosinus kąta nachylenia ściany bocznej do podstawy

oraz cosinus kata między ścianami bocznymi tego ostrosłupa.





PRACA KONTROLNA $\mathrm{n}\mathrm{r}2-$ POZIOM ROZSZERZONY

l. Trzeci składnik rozwinięcia dwumianu $(\displaystyle \sqrt[3]{x}+\frac{1}{\sqrt{x}})^{n}$ ma współczynnik równy 45. Wyzna-

czyč wszystkie skladniki tego rozwinięcia, $\mathrm{w}$ których $x$ występuje $\mathrm{w}$ potędze $0$ wykfadniku

całkowitym.

2. Niech $A=\{(x,y):y\geq||x-2|-1|\}, B=\{(x,y):y+\sqrt{4x-x^{2}-3}\leq 2\}$. Narysowač

na pfaszczy $\acute{\mathrm{z}}\mathrm{n}\mathrm{i}\mathrm{e}$ zbiór $A\cap B\mathrm{i}$ obliczyč jego pole.

3. Niech $a_{n}=\displaystyle \frac{1+kn}{5+k^{2}n}.$

a) Określič monotonicznośč ciągu $(a_{n})\mathrm{w}$ zalezności od parametru $k.$

b) Niech $S(k)$ oznacza sumę nieskończonego ciągu geometrycznego $0$ pierwszym wyra-

zie $a_{1}=1 \mathrm{i}$ ilorazie $q_{k}=\displaystyle \lim_{n\rightarrow\infty}a_{n}$. Sporządzič wykres funkcji $S(k)\mathrm{i}$ na tej podstawie

wyznaczyč zbiór jej wartości.

4. Dana jest funkcja $f(x)=\cos x$. Wyznaczyč dziedzinę oraz zbiór wartości funkcji

$g(x)=\sqrt{f(\frac{\pi}{2}-x)+\sqrt{3}f(x)-1}.$

5. $\mathrm{C}\mathrm{z}\mathrm{w}\mathrm{o}\mathrm{r}\mathrm{o}\mathrm{k}_{\Phi^{\mathrm{t}}}$ wypukly ABCD, $\mathrm{w}$ którym $AB=1, BC=2, CD=4, DA=3$ jest wpisany

$\mathrm{w}$ okrąg. Obliczyč promień $R$ tego okręgu. Sprawdzič, czy $\mathrm{w}$ czworokąt ten $\mathrm{m}\mathrm{o}\dot{\mathrm{z}}$ na wpisač

$\mathrm{o}\mathrm{k}\mathrm{r}\Phi \mathrm{g}. \mathrm{J}\mathrm{e}\dot{\mathrm{z}}$ eli $\mathrm{t}\mathrm{a}\mathrm{k}$, to obliczyč promień $r$ tego okręgu.

6. Plaszczyzna przechodząca przez jeden $\mathrm{z}$ wierzcholków czworościanu foremnego $\mathrm{i}$ rów-

noległa do jednej $\mathrm{z}$ jego krawędzi dzieli ten czworościan na dwie bryły $0$ takiej samej

objętości. Wyznaczyč pole przekroju oraz cosinus kąta nachylenia tego przekroju do

plaszczyzny podstawy.





PRACA KONTROLNA $\mathrm{n}\mathrm{r}3-$ POZIOM PODSTAWOWY

grudzień $2006\mathrm{r}.$

1. $\mathrm{Z}$ talii 24 kart wy1osowano dwie. Jakie jest prawdopodobieństwo, $\dot{\mathrm{z}}\mathrm{e}$ obie $\mathrm{s}\Phi$ koloru czer-

wonego lub obie są figurami?

2. Panowie X $\mathrm{i}\mathrm{Y}$ zafozyli jednocześnie firmy $\mathrm{i}\mathrm{w}$ pierwszym miesiącu dziafalności $\mathrm{k}\mathrm{a}\dot{\mathrm{z}}$ da

$\mathrm{z}$ nich miała obrot równy 50000 zfotych. Po pięciu miesiącach okazafo się, $\dot{\mathrm{z}}\mathrm{e}$ obrót

firmy pana X rósł $\mathrm{z}$ miesiąca na miesiąc $0$ tę samą kwotę, a obrót firmy pana $\mathrm{Y}$ rósł co

miesiąc $\mathrm{w}$ postępie geometrycznym. Stwierdzili równiez, $\dot{\mathrm{z}}\mathrm{e}\mathrm{w}$ drugim $\mathrm{i}$ trzecim miesiącu

działalności firma pana X miała obrót większy od obrotu firmy pana $\mathrm{Y}\mathrm{o}$ 2000 zł.

a) Jakie były obroty $\mathrm{k}\mathrm{a}\dot{\mathrm{z}}$ dej $\mathrm{z}$ firm $\mathrm{w}$ pieciu początkowych miesiącach?

b) Która $\mathrm{z}$ firm miała większą sumę obrotów $\mathrm{w}$ pierwszych pięciu miesiącach $\mathrm{i}\mathrm{o}$ ile?

c) Po ilu miesiącach obrót jednej $\mathrm{z}$ firm (której?) przekroczy dwukrotnie obrót drugiej

firmy?

3. Tangens kąta ostrego $\alpha$ równy jest $\displaystyle \frac{a}{b}$, gdzie

$\alpha=(\sqrt{2+\sqrt{3}}-\sqrt{2-\sqrt{3}})^{2}b=(\sqrt{\sqrt{2}+1}-\sqrt{\sqrt{2}-1})^{2}$

Wyznaczyč wartości pozostałych funkcji trygonometrycznych tego kata. Wykorzystując

wzór $\sin 2\alpha=2\sin\alpha\cos\alpha$, obliczyč miarę kąta $\alpha.$

4. Narysowač wykres funkcji $f(x)=|2x-4|-\sqrt{x^{2}+4x+4}$. Dlajakiego $m$ pole trójkąta

ograniczonego wykresem funkcji $f$ oraz prostą $y=m$ równe jest 6?

5. Harcerze rozbili 2 namioty, jeden $\mathrm{w}$ odległości 5 $\mathrm{m}$, drugi - 17 $\mathrm{m}$ od prostoliniowego

brzegu rzeki. Odległośč między namiotami równajest 13 $\mathrm{m}. \mathrm{W}$ którym miejscu $\mathrm{n}\mathrm{a}$ samym

brzegu rzeki (licząc od punktu brzegu będqcego rzutem prostopadfym punktu polozenia

pierwszego namiotu) powinni umieścič maszt $\mathrm{z}$ flagą zastępu, by odległośč od masztu do

$\mathrm{k}\mathrm{a}\dot{\mathrm{z}}$ dego $\mathrm{z}$ namiotów byfa taka sama?

6. Wysokośč ostrosłupa trójkątnego prawidłowego wynosi $h$, a kąt między wysokościami

ścian bocznych poprowadzonymi $\mathrm{z}$ wierzchołka ostrosfupa jest równy $ 2\alpha$. Obliczyč pole

powierzchni bocznej $\mathrm{i}$ objętośč tego ostrosfupa.





PRACA KONTROLNA $\mathrm{n}\mathrm{r}3-$ POZIOM ROZSZERZONY

l. Dlajakich wartości rzeczywistego parametru $p$ równanie $(p-2)x^{2}-(p+1)x-p=0$ ma

dwa rózne pierwiastki: a) ujemne? b) będące sinusem $\mathrm{i}$ cosinusem tego samego kąta?

2. Jakie powinny byč wymiary puszki $\mathrm{w}$ kształcie walca $0$ pojemności jednego litra, by jej

pole powierzchni całkowitej bylo najmniejsze?

3. $\mathrm{Z}$ badań statystycznych wynika,$\dot{\mathrm{z}}\mathrm{e}$ 5\% $\mathrm{m}\text{ę}\dot{\mathrm{z}}$ czyzn $\mathrm{i}$ 0,2\% kobiet to daltoniści. Wiadomo,

$\dot{\mathrm{z}}\mathrm{e}$ 55\% mieszkańców Wrocławia stanowia kobiety. Jakie jest prawdopodobieństwo, $\dot{\mathrm{z}}\mathrm{e}$

wśród 31osowo wybranych osób przynajmniej dwie nie odrózniaj$\Phi$ ko1orów?

4. Rozwiązač nierównośč $\displaystyle \log_{x}\frac{2-7x}{2x-7}\geq a$, gdzie $a$ jest granicą ciagu $0$ wyrazach

$a_{n}=\displaystyle \frac{4n(\sqrt{n^{2}+n}-n)}{n+1}.$

5. Pary liczb spefniające uklad równań

$\left\{\begin{array}{l}
-4x^{2}+y^{2}+2y+1=0,\\
-x^{2}+y+4=0
\end{array}\right.$

są wspólrzędnymi wierzchofków czworokata wypukfego ABCD.

a) Wykazač, $\dot{\mathrm{z}}\mathrm{e}$ czworokąt ABCD jest trapezem równoramiennym.

b) Wyznaczyč równanie okręgu opisanego na czworokącie ABCD.

6. Piramida utworzona z pięciu kul, z których cztery maja taki sam promień, jest wpisana

w walec. Przekrój osiowy walca jest kwadratem 0 boku d. Wyznaczyč promienie tych

kul.





PRACA KONTROLNA $\mathrm{n}\mathrm{r}4-$ POZIOM PODSTAWOWY

styczeń $2007\mathrm{r}.$

l. Dwóch robotników $\mathrm{m}\mathrm{o}\dot{\mathrm{z}}\mathrm{e}$ razem wykonač $\mathrm{P}^{\mathrm{e}\mathrm{w}\mathrm{n}}\Phi$ pracę $\mathrm{w}\mathrm{c}\mathrm{i}_{\Phi \mathrm{g}}\mathrm{u}7$ dni pod warunkiem, $\dot{\mathrm{z}}\mathrm{e}$

pierwszy $\mathrm{z}$ nich rozpocznie pracę $0$ póltora dnia wcześniej Gdyby $\mathrm{k}\mathrm{a}\dot{\mathrm{z}}\mathrm{d}\mathrm{y}\mathrm{z}$ nich praco-

waf oddzielnie, to drugi wykonałby calą pracę $03$ dni wcześniej od pierwszego. Ile dni

potrzebuje $\mathrm{k}\mathrm{a}\dot{\mathrm{z}}\mathrm{d}\mathrm{y}\mathrm{z}$ robotników na wykonanie calej pracy?

2. Narysowač na płaszczyz$\acute{}$nie zbiór $\{(x,y):\sqrt{x-1}+x\leq 2,0\leq y^{3}\leq\sqrt{5}-2\}$

jego pole. Wsk. Obliczyč $a=(\displaystyle \frac{\sqrt{5}-1}{2})^{3}$

i obliczyč

3. Obliczyč $a=\mathrm{t}\mathrm{g}\alpha, \mathrm{j}\mathrm{e}\dot{\mathrm{z}}$ eli $\displaystyle \sin\alpha-\cos\alpha=\frac{1}{5}\mathrm{i}\mathrm{k}\mathrm{a}\mathrm{t}\alpha$ spefnia nierównośč $\displaystyle \frac{\pi}{4}<\alpha<\frac{\pi}{2}$. Wyznaczyč

wysokośč trójk$\Phi$ta prostokątnego, $\mathrm{w}$ którym tangens jednego $\mathrm{z}$ k$\Phi$tów ostrych jest równy

$a$ a pole koła opisanego na tym trójkącie wynosi $25\pi.$

4. Kopufa Bazyliki $\acute{\mathrm{S}}\mathrm{w}$. Piotra $\mathrm{w}$ Watykanie ma ksztalt pólsfery $0$ promieniu 28 $\mathrm{m}$. Przed

rozpoczęciem prac renowacyjnych, na centralnie ustawionym rusztowaniu, umocowano

poziomą platformę $\mathrm{w}$ ksztalcie kola. Największa odległośč tej platformy od sklepienia

równa jest 2, 5 $\mathrm{m}$. a najmniejsza 1, 5 $\mathrm{m}$. Jaka jest powierzchnia tej platformy?

5. Trójmian kwadratowy $f(x)=\alpha x^{2}+bx+c$ przyjmuje najmniejszą wartośč równą $-2\mathrm{w}$

punkcie $x=2$ a reszta $\mathrm{z}$ dzielenia tego trójmianu przez dwumian $(x-1)$ równa jest 4.

Wyznaczyč współczynniki $a, b, c$. Narysowač staranny wykres funkcji $g(x) = f(|x|) \mathrm{i}$

wyznaczyč najmniejszq $\mathrm{i}$ najwiekszą wartośč tej funkcji na przedziale [$-1,3].$

6. Pani Zosia odcięfa $\mathrm{z}$ kwadratowego kawafka materiafu $0$ boku l $\mathrm{m}$ wszystkie cztery

narozniki $\mathrm{i}$ otrzymala serwetę $\mathrm{w}$ kształcie ośmiokąta foremnego. Postanowila wykończyč

ją szydelkową koronkq $0$ szerokości 5 cm.

a) Obliczyč dfugośč boku serwety przed $\mathrm{i}$ po jej wykończeniu.

b) Wiedząc, $\dot{\mathrm{z}}\mathrm{e}$ na zrobienie 100 centymetrów kwadratowych koronki potrzebny jest

jeden motek kordonku obliczyč, ile motków musi kupič Pani Zosia, $\mathrm{j}\mathrm{e}\dot{\mathrm{z}}$ eli powinna

uwzględnič 2\% straty materiafu podczas pracy.





PRACA KONTROLNA $\mathrm{n}\mathrm{r}4-$ POZIOM ROZSZERZONY

l. Do zbiornika poprowadzono trzy rury. Pierwsza rura potrzebuje do napełnienia zbiornika

$04$ godziny więcej $\mathrm{n}\mathrm{i}\dot{\mathrm{z}}$ druga, a trzecia napełnia cafy zbiornik $\mathrm{w}$ czasie dwa razy krótszym

$\mathrm{n}\mathrm{i}\dot{\mathrm{z}}$ pierwsza. Wjakim czasie napelnia zbiornik $\mathrm{k}\mathrm{a}\dot{\mathrm{z}}$ da $\mathrm{z}\mathrm{r}\mathrm{u}\mathrm{r}, \mathrm{j}\mathrm{e}\dot{\mathrm{z}}$ eli wiadomo, $\dot{\mathrm{z}}\mathrm{e}$ wszystkie

trzy rury otwarte jednocześnie napefniajq zbiornik $\mathrm{w}$ ciągu 2 godzin $\mathrm{i}40$ minut?

2. Stosując zasadę indukcji matematycznej wykazač prawdziwośč następującego wzoru dla

wszystkich $n\geq 1$

$\displaystyle \frac{1^{2}}{1\cdot 3}+\frac{2^{2}}{3\cdot 5}+\frac{3^{2}}{5\cdot 7}+\ldots+\frac{n^{2}}{(2n-1)(2n+1)}=\frac{n(n+1)}{2(2n+1)}$

3. Nie wykorzystujqc metod rachunku rózniczkowego wyznaczyč przedziały zawarte $\mathrm{w}[0,2\pi],$

na których funkcja

$ f(x)=\cos x+2\cos^{2}x+4\cos^{3}x+8\cos^{4}x+\ldots$

jest rosnąca.

4. Narysowač zbiór $\{(x,y):|x|+|y|\leq 6,|y|\leq 2^{|x|},|y|\geq\log_{2}|x|\}\mathrm{i}$ napisač równaniajego

osi symetrii. Podač odpowiednie uzasadnienie.

5. Pole przekroju ostrosłupa prawidlowego czworokątnego plaszczyznq przechodzącą przez

$\mathrm{P}^{\mathrm{r}\mathrm{z}\mathrm{e}\mathrm{k}}\Phi^{\mathrm{t}\mathrm{n}\text{ą}}$ podstawy $\mathrm{i}$ wierzcholek ostroslupa jest trójk$\Phi$tem równobocznym $0$ polu $S.$

Wyznaczyč stosunek promienia kuli wpisanej $\mathrm{w}$ ten ostrosłup do promienia kuli opisanej

na tym ostroslupie.

6. Punkt $A(1,2)$ jest wierzchołkiem trójkąta równobocznego. Wyznaczyč dwa pozostałe

wierzchołki tego trójkqta wiedząc, $\dot{\mathrm{z}}\mathrm{e}$ jeden $\mathrm{z}$ nich $\mathrm{l}\mathrm{e}\dot{\mathrm{z}}\mathrm{y}$ na prostej $x-y-1=0$, ajeden

$\mathrm{z}$ boków jest równolegly do wektora $\vec{v}= [-1,2]$. Obliczyč pole tego trójkąta. Ile jest

trójkątów spelniających warunki zadania?





PRACA KONTROLNA $\mathrm{n}\mathrm{r}5-$ POZIOM PODSTAWOWY

luty 2007r.

l. Bolek $\mathrm{i}$ Lolek $\mathrm{z}$ okazji swoich 9 $\mathrm{i} 11$ urodzin otrzymali od babci 200 zf do podziafu.

Umówili się, $\dot{\mathrm{z}}\mathrm{e}$ starszy otrzyma większą sumę, ale nie więcej $\mathrm{n}\mathrm{i}\dot{\mathrm{z}}0$ połowę od otrzymanej

przez brata, a ponadto średnia geometryczna obu kwot nie przekroczy iloczynu ich lat

$\dot{\mathrm{z}}$ ycia. Jaką maksymalną $\mathrm{i}$ minimalną kwotę $\mathrm{m}\mathrm{o}\dot{\mathrm{z}}\mathrm{e}$ otrzymač starszy brat.

2. Rozwazmy zbiór wszystkich ciagów binarnych $0$ dlugości 7. Wy1osowano jeden ciąg.

a) Jakie jest prawdopodobieństwo, $\dot{\mathrm{z}}\mathrm{e}$ bedzie zawieraf co najmniej 3 jedynki.

b) Jakie jest prawdopodobieństwo, $\dot{\mathrm{z}}\mathrm{e}\mathrm{w}$ tym ciągu wystqpi seria samych zer lub sa-

mych jedynek $0$ dfugości co najmniej 4.

3. $\mathrm{W}$ trójkącie $ABC$ dane są $\displaystyle \angle CAB=\frac{\pi}{3}$, wysokośč $|CD| =h=5$ oraz $|BD| =d=\sqrt{2}.$

Obliczyč promień okręgu wpisanego $\mathrm{w}$ ten trójkąt.

4. Na jednym rysunku przedstawič staranne wykresy funkcji $f(x) = |\displaystyle \sin(x-\frac{\pi}{9})|$ oraz

$g(x)=-\displaystyle \cos(x+\frac{5\pi}{18})$ na przedziale $I=[-\pi,2\pi].$

a) Odczytač $\mathrm{z}$ wykresu kąt $x_{0}$ taki, $\dot{\mathrm{z}}\mathrm{e}g(x)=\sin(x-x_{0}).$

b) Korzystając $\mathrm{z}$ wykresu oraz punktu a) wyznaczyč wszystkie kąty $x\in I$, dla których

$f(x)=g(x)$ oraz przedziafy, dla których $g(x)>f(x).$

5. Na walcu $0$ wysokości 6 cm $\mathrm{i}$ średnicy podstawy 16 cm opisano stozek $0$ kqcie rozwarcia

$ 2\alpha$ tak, $\dot{\mathrm{z}}\mathrm{e}$ podstawa walca $\mathrm{l}\mathrm{e}\dot{\mathrm{z}}\mathrm{y}$ na podstawie stozka, przy czym $\mathrm{t}\mathrm{g}\alpha= \displaystyle \frac{4}{3}$. Wyznaczyč

minimalne wymiary prostokąta ($\mathrm{z}$ zaokrągleniem $\mathrm{w}$ górę do pelnych cm), $\mathrm{w}$ którym

$\mathrm{m}\mathrm{o}\dot{\mathrm{z}}$ na zmieścič rozciętą powierzchnię boczną stozka $\mathrm{i}$ obliczyč jaki procent pola tego

prostokąta stanowi powierzchnia boczna stozka.

6. Dane są proste $k$ : $2x-3y+6=0$ oraz $l$ : $2x+4y-7=0$. Na prostej $k$ znalez$\acute{}$č punkt,

którego obraz symetryczny względem prostej $l\mathrm{l}\mathrm{e}\dot{\mathrm{z}}\mathrm{y}$ na osi $\mathrm{O}y$. Sporządzič rysunek.





PRACA KONTROLNA $\mathrm{n}\mathrm{r}5-$ POZIOM ROZSZERZONY

l. Stosując zasadę indukcji matematycznej wykazač, $\dot{\mathrm{z}}\mathrm{e}$ liczba $7^{n}-(-3)^{n}$ jest podzielna

przez 10 d1a $\mathrm{k}\mathrm{a}\dot{\mathrm{z}}$ dego naturalnego $n.$

2. Rozwiązač nierównośč 4 logl6 $\cos 2x+2\log_{4}\sin x+\log_{2}\cos x+3<0$ dla $x\displaystyle \in(0,\frac{\pi}{4}).$

3. Róznica ciqgu arytmetycznego $(a_{n})$ jest liczbq mniejszq od l. Wyznaczyč najmniejszą

wartośč wyrazenia $\displaystyle \frac{a_{1}a49}{a_{50}}, \mathrm{w}\mathrm{i}\mathrm{e}\mathrm{d}\mathrm{z}\Phi^{\mathrm{C}}, \dot{\mathrm{z}}\mathrm{e}a_{51}=1.$

4. Cięciwa paraboli $0$ równaniu $y=-a^{2}x^{2}+5ax-4$ jest styczna do krzywej $y=\displaystyle \frac{1}{-x+1}$

$\mathrm{w}$ punkcie $0$ odciętej $x_{o}=2$, który dzieli $\mathrm{t}\mathrm{e}$ cięciwę na połowy. Wyznaczyč parametr $a.$

Podač ilustrację graficzną rozwiązania zadania.

5. Dana jest funkcja $f(x)=\displaystyle \frac{2x^{2}}{(2-x)^{2}}.$

a) Zbadač przebieg zmienności funkcji $f\mathrm{i}$ naszkicowač jej wykres.

b) Sporządzič wykres funkcji $k=g(m)$, gdzie $k$ jest liczbą rozwi$\Phi$zań równania

$\displaystyle \frac{2x^{2}}{(2-|x|)^{2}}=m$

$\mathrm{w}$ zalezności od parametru rzeczywistego $m.$

6. $\mathrm{W}$ kulę $0$ promieniu $R$ wpisano stozek, $\mathrm{w}$ którym tworząca jest równa średnicy pod-

stawy. Obydwie bryły przecieto płaszczyzną równoległą do podstawy stozka. Szerokośč

otrzymanego $\mathrm{w}$ przecięciu pierścienia kofowego zawartego między powierzchnią kulistą

a powierzchnią boczną stozka równa się $m.$

a) Znalez/č odlegfośč pfaszczyzny tnącej od wierzchołka stozka.

b) Przedyskutowač liczbę rozwiązań $\mathrm{w}$ zalezności od $m\mathrm{i}$ podač interpretację geome-

tryczną przypadków szczególnych.







XXXVII

KORESPONDENCYJNY KURS Z MATEMATYKI

PRACA KONTROLNA $\mathrm{n}\mathrm{r}1-$ POZIOM PODSTAWOWY

$\mathrm{p}\mathrm{a}\acute{\mathrm{z}}$dziernik 2$007\mathrm{r}.$

l. Pan Kowalski wpłacił $\mathrm{P}^{\mathrm{e}\mathrm{w}\mathrm{n}}\Phi$ sumę na lokatę oprocentowaną $\mathrm{w}$ wysokości 8\% $\mathrm{w}$ skali

roku, przy czym odsetki naliczane sq kwartalnie. $\mathrm{W}$ ciągu rozwazanego roku inflacja

wyniosła 4\%. Jakie jest rea1ne roczne oprocentowanie 1okaty Pana Kowa1skiego, $\mathrm{t}\mathrm{z}\mathrm{n}. 0$

ile procent więcej warte $\mathrm{s}\Phi \mathrm{p}\mathrm{i}\mathrm{e}\mathrm{n}\mathrm{i}_{\Phi}\mathrm{d}\mathrm{z}\mathrm{e}$, które Pan Kowalski miał na koncie po roku od

tych, które wpłacił? Wynik podač $\mathrm{z}$ dokładnością do setnych części procenta.

2. Liczba $p=\displaystyle \frac{(\sqrt[3]{54}-2)(9\sqrt[3]{4}+6\sqrt[3]{2}+4)-(2-\sqrt{3})^{3}}{\sqrt{3}+(1+\sqrt{3})^{2}}$ jest miejscem zerowym funkcji

$f(x) = ax^{2}+bx+c$. Wyznaczyč wspólczynniki $a, b, c$ oraz drugie miejsce zerowe tej

funkcji $\mathrm{w}\mathrm{i}\mathrm{e}\mathrm{d}\mathrm{z}\Phi^{\mathrm{C}}, \dot{\mathrm{z}}\mathrm{e}$ największ$\Phi$ wartości$\Phi$ funkcji jest 4, a jej wykres jest symetryczny

względem prostej $x=1.$

3. Dwie styczne do okręgu $0$ promieniu 6 przecinają się pod kątem $60^{\mathrm{o}}$. Obliczyč pole obsza-

ru ograniczonego odcinkami tych stycznych $\mathrm{i}$ krótszym $\mathrm{z}$ łuków, najakie $\mathrm{o}\mathrm{k}\mathrm{r}\Phi \mathrm{g}$ podzielony

jest punktami styczności. Wyznaczyč promień okręgu wpisanego $\mathrm{w}$ ten obszar.

4. Niech

$f(x)=$

gdy

gdy

$|x-1|\geq 1,$

$|x-1|<1.$

a) Obliczyč $f(-\displaystyle \frac{2}{3}), f(\displaystyle \frac{1+\sqrt{3}}{2})$ oraz $f(\pi-1).$

b) Narysowač wykres funkcji $f\mathrm{i}$ na jego podstawie podač zbiór wartości funkcji.

c) Rozwi$\Phi$zač nierównośč $f(x)\displaystyle \geq-\frac{1}{2}\mathrm{i}$ zaznaczyč na osi $0x$ zbiór jej rozwi$\Phi$zań.

5. Pole przekroju graniastosłupa prawidlowego $0$ podstawie kwadratowej paszczyz$\Phi$ prze-

$\mathrm{c}\mathrm{h}\mathrm{o}\mathrm{d}\mathrm{z}\Phi^{\mathrm{C}}\Phi$ przez $\mathrm{P}^{\mathrm{r}\mathrm{z}\mathrm{e}\mathrm{k}}\Phi^{\mathrm{t}\mathrm{n}}\Phi$ graniastosłupa $\mathrm{i}$ środki przeciwległych krawędzi bocznych jest

3 razy większe $\mathrm{n}\mathrm{i}\dot{\mathrm{z}}$ pole podstawy. Wyznaczyč tangens kąta nachylenia $\mathrm{P}^{\mathrm{r}\mathrm{z}\mathrm{e}\mathrm{k}}\Phi^{\mathrm{t}\mathrm{n}\mathrm{e}\mathrm{j}}$ grania-

stosłupa do podstawy. Obliczyč pole powierzchni całkowitej tego graniastosłupa $\mathrm{w}\mathrm{i}\mathrm{e}\mathrm{d}\mathrm{z}\Phi^{\mathrm{C}},$

$\dot{\mathrm{z}}\mathrm{e}$ pole rozwazanego przekroju równe jest 10.

6. Jeden $\mathrm{z}$ wierzcholków trójk$\Phi$ta prostokątnego $0$ polu 7, 5 jest punktem przecięcia pro-

stych $k:x-y+3=0$ oraz $l$ : $2x+y=0$. Wyznaczyč pozostałe wierzchołki $\mathrm{w}\mathrm{i}\mathrm{e}\mathrm{d}\mathrm{z}\Phi^{\mathrm{C}},$

$\dot{\mathrm{z}}\mathrm{e}\mathrm{l}\mathrm{e}\mathrm{z}\Phi$ one na prostych $k\mathrm{i}l$, a wierzchołek $\mathrm{k}_{\Phi^{\mathrm{t}\mathrm{a}}}$ prostego jest na prostej $l$. Sporz$\Phi$dzič

staranny rysunek.




PRACA KONTROLNA $\mathrm{n}\mathrm{r} 1 -$ POZIOM ROZSZERZONY

l. Narysowač wykres funkcji $f(x)=$ 

$\mathrm{P}\mathrm{o}\mathrm{s}l\mathrm{u}\mathrm{g}\mathrm{u}\mathrm{j}_{\Phi}\mathrm{c}$ się nim podač

wzór $\mathrm{i}$ narysowač wykres funkcji $g(m)$ określaj$\Phi$cej liczbę rozwi$\Phi$zań równania $f(x)=m,$

gdzie $m$ jest parametrem rzeczywistym.

2. Rozwi$\Phi$zač równanie $\displaystyle \frac{\sin 3x}{\cos x}=$ ctg $x-\mathrm{t}\mathrm{g}x.$

3. Napisač równanie stycznej $k$ do wykresu funkcji $f(x)=x^{2}-4x+3\mathrm{w}$ punkcie $(x_{1},0),$

gdzie $x_{1}$ jest najmniejszym miejscem zerowym tej funkcji. Znalez$\acute{}$č punkt przecięcia tej

stycznej ze $\mathrm{s}\mathrm{t}\mathrm{y}\mathrm{c}\mathrm{z}\mathrm{n}\Phi$ do niej $\mathrm{p}\mathrm{r}\mathrm{o}\mathrm{s}\mathrm{t}\mathrm{o}\mathrm{p}\mathrm{a}\mathrm{d}\text{ł}_{\Phi}$ Sporządzič staranny rysunek.

4. Rozwi$\Phi$zač nierównośč $\log_{2}(x-1)-\log_{\frac{1}{2}}(4-x)-\log_{\sqrt{2}}(x-2)\leq 0.$

5. Rozwi$\Phi$zač nierównośč $\displaystyle \sqrt{x^{2}-1}+1+\frac{1}{\sqrt{x^{2}-1}}+\ldots\geq \displaystyle \frac{9}{2},$

wyrazów nieskończonego $\mathrm{c}\mathrm{i}_{\Phi \mathrm{g}}\mathrm{u}$ geometrycznego.

gdzie lewa strona jest $\mathrm{s}\mathrm{u}\mathrm{m}\Phi$

6. $\mathrm{W}$ stozek wpisano kulę, a następnie $\mathrm{w}$ obszar zawarty między $\mathrm{t}_{\Phi}\mathrm{k}\mathrm{u}1_{\Phi}\mathrm{i}$ wierzchołkiem

stozka wpisano kulę $0$ objętości 8 razy mniejszej. Ob1iczyč stosunek objętości stozka do

objętości kuli na nim opisanej.





PRACA KONTROLNA $\mathrm{n}\mathrm{r}6-$ POZIOM PODSTAWOWY

marzec 2008r.

l. Dwa naczynia zawieraj $\Phi^{\mathrm{W}}$ sumie 401itrów wody. Po prze1aniu pewnej części wody pierw-

szego naczynia do drugiego, $\mathrm{w}$ pierwszym naczyniu zostalo trzy razy mniej wody $\mathrm{n}\mathrm{i}\dot{\mathrm{z}}\mathrm{w}$

drugim. Gdy następnie przelano taką samą częśč wody drugiego naczynia do pierwszego,

okazało się, $\dot{\mathrm{z}}\mathrm{e}\mathrm{w}$ obu naczyniach jest tyle samo płynu. Obliczyč, ile wody było pierwotnie

$\mathrm{w}\mathrm{k}\mathrm{a}\dot{\mathrm{z}}$ dym naczyniu $\mathrm{i}\mathrm{j}\mathrm{a}\mathrm{k}_{\Phi}$ jej częśč przelewano.

2. Obwód trójk$\Phi$ta równoramiennego równy jest 20. Jakie powinny byč jego boki, by obję-

tośč bryły otrzymanej przez obrót tego trójkąta wokóf podstawy była największa?

3. Student opracował 28 spośród 45 przygotowanych na egzamin tematów. Losuje trzy

tematy. $\mathrm{J}\mathrm{e}\dot{\mathrm{z}}$ eli odpowie poprawnie na wszystkie, to dostanie ocenę bardzo $\mathrm{d}\mathrm{o}\mathrm{b}\mathrm{r}\Phi, \mathrm{j}\mathrm{e}\dot{\mathrm{z}}$ eli

na dwa- $\mathrm{d}\mathrm{o}\mathrm{b}\mathrm{r}\Phi$, a $\mathrm{j}\mathrm{e}\dot{\mathrm{z}}$ eli na jedno- $\mathrm{d}\mathrm{o}\mathrm{s}\mathrm{t}\mathrm{a}\mathrm{t}\mathrm{e}\mathrm{c}\mathrm{z}\mathrm{n}\Phi$. Jakie jest prawdopodobieństwo, $\dot{\mathrm{z}}\mathrm{e}$:

a) dostanie przynajmniej db? b) zda egzamin?

4. Narysowač staranny wykres funkcji $f(x)=x^{2}-2|x|-3$, wyznaczyč jej miejsca zerowe $\mathrm{i}$

zbiór wartości. $\mathrm{W}\mathrm{y}\mathrm{k}\mathrm{o}\mathrm{r}\mathrm{z}\mathrm{y}\mathrm{s}\mathrm{t}\mathrm{u}\mathrm{j}_{\Phi}\mathrm{c}$ wykres funkcji $f$:

a) narysowač wykres funkcji $h(x)=x^{2}-2x-2|x-1|-1.$

b) $\mathrm{p}\mathrm{o}\mathrm{s}\text{ł} \mathrm{u}\mathrm{g}\mathrm{u}\mathrm{j}_{\Phi}\mathrm{c}$ się powyzszymi wykresami określič, dla jakich wartości parametru rze-

czywistego $m$ równanie $f(x)=h(x)+m$ ma dokładnie jedno $\mathrm{r}\mathrm{o}\mathrm{z}\mathrm{w}\mathrm{i}_{\Phi}$zanie.

5. Państwo Kowalscy $\mathrm{s}\Phi$ właścicielami działki budowlanej $\mathrm{w}$

kształcie trójk$\Phi$ta prostokątnego $0$ przyprostokątnych dfugości

30 $\mathrm{m}\mathrm{i}40\mathrm{m}$. Postanowili podzielič ją na dwie równej wartości

części zgodnie ze schematem obok. Wyznaczyč długośč odcinka

$\overline{BK}\mathrm{w}\mathrm{i}\mathrm{e}\mathrm{d}\mathrm{z}\Phi^{\mathrm{C}}, \dot{\mathrm{z}}\mathrm{e}$ jeden metr kwadratowy działki $\mathrm{c}\mathrm{z}\mathrm{w}\mathrm{o}\mathrm{r}\mathrm{o}\mathrm{k}_{\Phi}$tnej

jest póltora raza drozszy $\mathrm{n}\mathrm{i}\dot{\mathrm{z}}$ jeden metr kwadratowy dzialki

trójk$\Phi$tnej. Która $\mathrm{z}$ działek ma większy obwód $\mathrm{i}0$ ile? Wynik

podač $\mathrm{z}$ dokładnościq do 10 cm.
\begin{center}
\includegraphics[width=45.312mm,height=37.740mm]{./KursMatematyki_PolitechnikaWroclawska_2007_2008_page10_images/image001.eps}
\end{center}
{\it A}

{\it L}

{\it B  K C}

6. Boki $\overline{AB}, \overline{AC}$ trójk$\Phi$ta zawarte są $\mathrm{w}$ prostych $l$ : $x-y-1=0$ oraz $k$ : $x+2y+2=0.$

Wyznaczyč współrzędne wierzcholków $B, C \mathrm{w}\mathrm{i}\mathrm{e}\mathrm{d}\mathrm{z}\Phi^{\mathrm{C}}, \dot{\mathrm{z}}\mathrm{e}$ punkt $P(1,1)$ jest środkiem

boku $\overline{BC}$. Wyznaczyč współrzędne wierzcholków trójk$\Phi$ta otrzymanego przez odbicie

symetryczne powyzszego trójk$\Phi$ta względem boku $\overline{BC}.$





PRACA KONTROLNA $\mathrm{n}\mathrm{r} 6-$ POZIOM ROZSZERZONY

l. Rozwi$\Phi$zač $\mathrm{i}$ zinterpretowač graficznie układ równań 

1,

1.

2. Niech $f(x)=\log_{2}x, g(x)=x+2, h(x)=|x|.$

a) Narysowač wykresy funkcji $f\mathrm{o}h\mathrm{o}g$ oraz $g0f0h$

b) Rozwi$\Phi$zač nierównośč $(f\mathrm{o}h\mathrm{o}g)(x)<(g\mathrm{o}f\mathrm{o}h)(x).$

3. Rzucamy kolejno trzy razy kostką do gry. Jakie jest prawdopodobieństwo, $\dot{\mathrm{z}}\mathrm{e}\mathrm{w}$ otrzy-

manym $\mathrm{c}\mathrm{i}_{\Phi \mathrm{g}}\mathrm{u}\mathrm{s}\Phi$ przynajmniej dwie,,szóstki'' lub suma oczek przekroczy 14?

4. Dany jest wielomian $W(x) = x^{3}+ax+b$, gdzie $b \neq 0$. Wykazač, $\dot{\mathrm{z}}\mathrm{e} W(x)$ posiada

pierwiastek podwójny wtedy $\mathrm{i}$ tylko wtedy, gdy spełniony jest warunek $4a^{3}+27b^{2}=0.$

Wyrazič pierwiastki za pomocą współczynnika $b.$

5. $\mathrm{W}$ ostrosłupie prawidłowym czworokątnym dany jest kąt $\alpha$ nachylenia ściany bocznej

do podstawy oraz obwód ściany bocznej równy $l$. Obliczyč objętośč tego ostrosłupa.

6. Narysowač staranny wykres funkcji $f(x)=\cos x-\sqrt{3}|\sin x| \mathrm{w}$ przedziale $[0,2\pi] \mathrm{i}$ wy-

znaczyč zbiór jej wartości.

a) $\mathrm{P}\mathrm{o}\mathrm{s}\text{ł} \mathrm{u}\mathrm{g}\mathrm{u}\mathrm{j}_{\Phi}\mathrm{c}$ się wykresem podač liczbę rozwi$\Phi$zań równania $f(x)=m\mathrm{w}$ zalezności

od parametru rzeczywistego $m.$

b) $\mathrm{R}\mathrm{o}\mathrm{z}\mathrm{w}\mathrm{i}_{\Phi}\mathrm{z}\mathrm{u}\mathrm{j}_{\Phi}\mathrm{c}$ odpowiednie równanie $\mathrm{i}$ korzystając $\mathrm{z}$ wykresu podač $\mathrm{r}\mathrm{o}\mathrm{z}\mathrm{w}\mathrm{i}_{\Phi}$zanie nie-

równości $f(x)\leq-\sqrt{2}.$





PRACA KONTROLNA $\mathrm{n}\mathrm{r}2-$ POZIOM PODSTAWOWY

listopad $2007\mathrm{r}.$

l. Trzy liczby dodatnie $\mathrm{t}\mathrm{w}\mathrm{o}\mathrm{r}\mathrm{z}\Phi \mathrm{c}\mathrm{i}_{\Phi \mathrm{g}}$ geometryczny. Suma tych liczb równa jest 26, a suma

ich odwrotności wynosi 0.7(2). Wyznaczyč te 1iczby.

2. Pole powierzchni bocznej ostrosłupa prawidłowego $\mathrm{c}\mathrm{z}\mathrm{w}\mathrm{o}\mathrm{r}\mathrm{o}\mathrm{k}_{\Phi^{\mathrm{t}}}$nego j$\mathrm{e}\mathrm{s}\mathrm{t}2$ razy większe $\mathrm{n}\mathrm{i}\dot{\mathrm{z}}$

pole podstawy. $\mathrm{W}$ trójk$\Phi$t otrzymany $\mathrm{w}$ przekroju ostrosfupa $\mathrm{p}\mathrm{a}$szczyz $\Phi \mathrm{p}\mathrm{r}\mathrm{z}\mathrm{e}\mathrm{c}\mathrm{h}\mathrm{o}\mathrm{d}\mathrm{z}\Phi^{\mathrm{C}}\Phi$

przez jego wysokośč $\mathrm{i} \mathrm{P}^{\mathrm{r}\mathrm{z}\mathrm{e}\mathrm{k}}\Phi^{\mathrm{t}\mathrm{n}\mathrm{q}}$ podstawy wpisano kwadrat, którego jeden bok jest

zawarty $\mathrm{w}\mathrm{P}^{\mathrm{r}\mathrm{z}\mathrm{e}\mathrm{k}}\Phi^{\mathrm{t}\mathrm{n}\mathrm{e}\mathrm{j}}$ podstawy. Obliczyč stosunek pola tego kwadratu do pola podstawy

ostrosłupa. Sporz$\Phi$dzič staranny rysunek.

3. Wykonač działania $\mathrm{i}$ zapisač $\mathrm{w}$ najprostszej postaci wyrazenie

$s(a,b)= (\displaystyle \frac{a^{2}+b^{2}}{a^{2}-b^{2}}-\frac{a^{3}+b^{3}}{a^{3}-b^{3}})$ : $(\displaystyle \frac{a^{2}}{a^{3}-b^{3}}-\frac{a}{a^{2}+ab+b^{2}})$

Wyznaczyč wysokośč trójk$\Phi$ta prostokątnego wpisanego $\mathrm{w}\mathrm{o}\mathrm{k}\mathrm{r}\Phi \mathrm{g}\mathrm{o}$ promieniu 6 opusz-

$\mathrm{c}\mathrm{z}\mathrm{o}\mathrm{n}\Phi \mathrm{z}$ wierzchołka $\mathrm{k}_{\Phi^{\mathrm{t}\mathrm{a}}}$ prostego wiedząc, $\dot{\mathrm{z}}\mathrm{e}$ tangens jednego $\mathrm{z}$ k$\Phi$tów ostrych tego

trójk$\Phi$ta równy jest $s(\sqrt{5}+\sqrt{3},\sqrt{5}-\sqrt{3}).$

4. Wielomian $W(x) =x^{3}-x^{2}+bx+c$ jest podzielny przez $(x+3)$, a reszta $\mathrm{z}$ dzielenia

tego wielomianu przez $(x-3)$ równa jest 6. Wyznaczyč $b\mathrm{i} c$, a następnie rozwi$\Phi$zač

nierównośč $(x+1)W(x-1)-(x+2)W(x-2)\leq 0.$

5. $\mathrm{W}$ ramach przygotowań do EURO 2012 zap1anowano budowe komp1eksu sportowego zło-

$\dot{\mathrm{z}}$ onego $\mathrm{z}$ czterech jednakowych hal sportowych $\mathrm{w}$ kształcie pófkul $0$ środkach $\mathrm{w}$ rogach

kwadratu $0$ boku 100 $\mathrm{m}\mathrm{i}$ piątej hali $\mathrm{w}$ ksztafcie pófkuli stycznej do czterech pozosta-

fych. Jakie powinny byč wymiary tych hal, by koszt ich budowy był najmniejszy, $\mathrm{j}\mathrm{e}\dot{\mathrm{z}}$ eli

wiadomo, $\dot{\mathrm{z}}\mathrm{e}$ jest on proporcjonalny do pola powierzchni dachu hali?

6. $\mathrm{W}$ trójk$\Phi$cie prostokątnym $0$ kącie prostym przy wierzchoku $C$ na przedłuzeniu przeciw-

$\mathrm{p}\mathrm{r}\mathrm{o}\mathrm{s}\mathrm{t}\mathrm{o}\mathrm{k}_{\Phi^{\mathrm{t}}}\mathrm{n}\mathrm{e}\mathrm{j}$ AB odmierzono odcinek $BD\mathrm{t}\mathrm{a}\mathrm{k}, \dot{\mathrm{z}}\mathrm{e}|BD|=|BC|$. Wyznaczyč $|CD|$ oraz

obliczyč pole trójkta $\triangle ACD, \mathrm{j}\mathrm{e}\dot{\mathrm{z}}$ eli $|BC|=5, |AC|=12$. Sporz$\Phi$dzič staranny rysunek.





PRACA KONTROLNA $\mathrm{n}\mathrm{r} 2-$ POZIOM ROZSZERZONY

l. Znalez/č wszystkie wartości parametru rzeczywistego $m$, dla których pierwiastki trójmia-

nu kwadratowego $f(x)=(m-2)x^{2}-(m+1)x-m \mathrm{s}\mathrm{p}\mathrm{e}\mathrm{f}\mathrm{n}\mathrm{i}\mathrm{a}\mathrm{j}_{\Phi}$ nierównośč $|x_{1}|+|x_{2}|\leq 1.$

2. Wyznaczyč dziedzinę funkcji

$f(x)=\displaystyle \frac{\sqrt{2^{4-x^{2}}-4^{x}}}{\log(2-x-x^{2}-\ldots)}.$

3. Grupa l75 robotników miala wykonač pewną pracę $\mathrm{w}$ określonym terminie. Po upływie

30 dni wspólnej pracy przesyłano codziennie po 3 robotników na inne stanowiska, wsku-

tek czego robota została wykonana $\mathrm{z}$ opóz/nieniem 21 $\mathrm{d}\mathrm{n}\mathrm{i}. \mathrm{W}$ ciągu ilu dni miała byč

wykonana praca według planu?

4. Wyznaczyč promień okręgu opisanego na czworokącie ABCD, $\mathrm{w}$ którym $\mathrm{k}_{\Phi^{\mathrm{t}}}$ przy wierz-

cholku $A$ ma miarę $\alpha, \mathrm{k}_{\Phi^{\mathrm{t}\mathrm{y}}}$ przy wierzchołkach $B,  D\mathrm{s}\Phi$ proste oraz $|BC|=a, |AD|=b.$

Sporz$\Phi$dzič staranny rysunek.

5. Narysowač staranny wykres funkcji $f(x)=\displaystyle \frac{\sin 2x-|\sin x|}{\sin x}.$

$\mathrm{W}$ przedziale $[0,\pi]$ wyznaczyč $\mathrm{r}\mathrm{o}\mathrm{z}\mathrm{w}\mathrm{i}_{\Phi}$zania nierówności $f(x)<2(\sqrt{2}-1)\cos^{2}x.$

6. Pole przekroju graniastosłupa prawidlowego $0$ podstawie kwadratowej paszczyz$\Phi$ prze-

$\mathrm{c}\mathrm{h}\mathrm{o}\mathrm{d}\mathrm{z}\Phi^{\mathrm{C}}\Phi$ przez $\mathrm{P}^{\mathrm{r}\mathrm{z}\mathrm{e}\mathrm{k}}\Phi^{\mathrm{t}\mathrm{n}}\Phi$ graniastosłupa $\mathrm{i}$ środek jednej $\mathrm{z}$ krawędzi podstawy jest 3 razy

większe $\mathrm{n}\mathrm{i}\dot{\mathrm{z}}$ pole podstawy. Wyznaczyč tangens $\mathrm{k}_{\Phi^{\mathrm{t}\mathrm{a}}}$ nachylenia $\mathrm{P}^{\mathrm{r}\mathrm{z}\mathrm{e}\mathrm{k}}\Phi^{\mathrm{t}\mathrm{n}\mathrm{e}\mathrm{j}}$ graniastosłu-

pa do podstawy. Obliczyč pole powierzchni całkowitej tego graniastosłupa $\mathrm{w}\mathrm{i}\mathrm{e}\mathrm{d}\mathrm{z}\Phi^{\mathrm{C}}, \dot{\mathrm{z}}\mathrm{e}$

pole rozwazanego przekroju równe jest 15. Sporządzič staranny rysunek.





PRACA KONTROLNA $\mathrm{n}\mathrm{r}3-$ POZIOM PODSTAWOWY

grudzień $2007\mathrm{r}.$

l. Rozwi$\Phi$zač równanie

$\sqrt{3-x}+\sqrt{3x-2}=2.$

2. Sześč kostek sześciennych $0$ objętościach 1, 2, 4, 8, 16 $\mathrm{i}32\mathrm{d}\mathrm{m}^{3}$ ustawiono $\mathrm{w}$ piramidę,

$\mathrm{u}\mathrm{k}l\mathrm{a}\mathrm{d}\mathrm{a}\mathrm{j}_{\Phi}\mathrm{c}\mathrm{j}\mathrm{e}\mathrm{d}\mathrm{n}\Phi$ kostkę na drugiej poczynając od największej. Czy wysokośč piramidy

przekroczy 120 cm? Odpowied $\acute{\mathrm{z}}$ uzasadnič bez prowadzenia obliczeń przyblizonych.

3. ()$\mathrm{P}\mathrm{a}\mathrm{n}\mathrm{W}$ wybrał się na spacer do parku mającego ksztalt $\mathrm{p}\mathrm{r}\mathrm{o}\mathrm{s}\mathrm{t}\mathrm{o}\mathrm{k}_{\Phi^{\mathrm{t}}}\mathrm{a}\mathrm{o}$ wymiarach 400 $\mathrm{m}$

na 300 $\mathrm{m}$, podzielonego alejkami na 12 kwadratów $0$ boku 100 $\mathrm{m}$, jak na rysunku ponizej.

Postanowił przejśč od punktu $A$ do $B, l_{\Phi}$cznie $700\mathrm{m}$, wybierajqc przypadkowo alejkę

na $\mathrm{k}\mathrm{a}\dot{\mathrm{z}}$ dym rozwidleniu. Jakie jest prawdopodobieństwo, $\dot{\mathrm{z}}\mathrm{e}$ Pan $\mathrm{W}$ przejdzie środkow$\Phi$

$\mathrm{a}\mathrm{l}\mathrm{e}\mathrm{j}\mathrm{k}_{\Phi^{\mathrm{O}\mathrm{Z}\mathrm{n}\mathrm{a}\mathrm{c}\mathrm{z}\mathrm{o}\mathrm{n}}\Phi}$ na rysunku $x$?
\begin{center}
\includegraphics[width=36.216mm,height=28.752mm]{./KursMatematyki_PolitechnikaWroclawska_2007_2008_page4_images/image001.eps}
\end{center}
$\sqrt{}^{B}$

4. $\mathrm{P}\mathrm{o}\mathrm{d}\mathrm{s}\mathrm{t}\mathrm{a}\mathrm{w}\Phi$ trójk$\Phi$ta równoramiennego jest odcinek AB $0$ końcach $A(-1,1), B(3,3), \mathrm{a}$

wierzchołek $C\mathrm{l}\mathrm{e}\dot{\mathrm{z}}\mathrm{y}$ na paraboli $0$ równaniu $y^{2}=x+1$. Wyznaczyč współrzędne wierz-

chołka $C$ oraz pole trójk$\Phi$ta $ABC$. Sporz$\Phi$dzič rysunek.

5. Na jednym rysunku sporz$\Phi$dzič dokładne wykresy funkcji $\sin x, \cos x$, tg $x$ oraz ctg $x$

$\mathrm{w}$ przedziale $(0,\displaystyle \frac{\pi}{2}) \mathrm{i}$ zaznaczyč na nich

ctg $(\displaystyle \cos\frac{\pi}{4}), \displaystyle \cos(\sin\frac{\pi}{3}), \displaystyle \sin(\cos\frac{\pi}{3})$, tg $(\displaystyle \sin\frac{\pi}{2})$

Uporz$\Phi$dkowač powyzsze liczby od najmniejszej do największej. Uzasadnič te relacje za

$\mathrm{P}^{\mathrm{o}\mathrm{m}\mathrm{o}\mathrm{c}}\Phi$ odpowiednich nierówności.

6. $\mathrm{W}$ ostrosłupie prawidłowym trójkątnym kąt między ścianami bocznymi ma miarę $\alpha, \mathrm{a}$

odległośč krawędzi podstawy od przeciwległej krawędzi bocznej jest równa $d$. Obliczyč

objętośč ostrosłupa.





PRACA KONTROLNA nr 3 -POZIOM ROZSZERZONY

1. $\mathrm{S}\mathrm{t}\mathrm{o}\mathrm{s}\mathrm{u}\mathrm{j}_{\Phi}\mathrm{c}$ zasadę indukcji matematycznej, udowodnič prawdziwośč wzoru

$\left(\begin{array}{l}
3\\
2
\end{array}\right) + \left(\begin{array}{l}
5\\
2
\end{array}\right) +\ldots+\left(\begin{array}{ll}
2n+ & 1\\
2 & 
\end{array}\right) =\displaystyle \frac{n(n+1)(4n+5)}{6}$

dla $n\geq 1.$

2. Wojtuś wylosował $\mathrm{j}\mathrm{e}\mathrm{d}\mathrm{n}\Phi$ monetę ze skarbonki zawierającej 3 złotówki, 4 dwuzłotówki $\mathrm{i}3$

pięciozlotówki. Następnie, $\mathrm{w}$ zalezności od wyniku pierwszego losowania, wylosował jesz-

cze trzy monety, gdy za pierwszym razem otrzymał złotówkę, dwie monety, gdy pierwsza

była dwuzlotówk$\Phi$ oraz jedną monetę, gdy $\mathrm{w}$ pierwszym losowaniu dostał pięciozłotów-

kę. Obliczyč prawdopodobieństwo, $\dot{\mathrm{z}}\mathrm{e}$, postępuj$\Phi$c $\mathrm{w}$ ten sposób, zgromadził $\text{ł}_{\Phi}$cznie co

najmniej 8 złotych.

3. Jednym $\mathrm{z}$ wierzchołków kwadratujest punkt $A(2,2)$, a środkiemjednego $\mathrm{z}$ przeciwległych

boków jest punkt $M(-\displaystyle \frac{1}{2},-\frac{1}{2})$. Wyznaczyč współrzędne pozostałych wierzchołków oraz

równanie okręgu opisanego na tym kwadracie.

4. Rozwi$\Phi$zač nierównośč

$\displaystyle \frac{1}{\sqrt{3^{x+1}-2}}\geq\frac{1}{4-(\sqrt{3})^{x+2}}.$

5. $\mathrm{W}$ ostrosłup prawidłowy trójkątny wpisano walec, którego podstawa $\mathrm{l}\mathrm{e}\dot{\mathrm{z}}\mathrm{y}$ na podstawie

ostrosłupa. Srednica podstawy walcajest równajego wysokości. Znalez/č tangens $\mathrm{k}_{\Phi^{\mathrm{t}\mathrm{a}}}$ na-

chylenia krawędzi bocznej ostrosłupa do podstawy, dla którego stosunek objętości walca

do objętości ostrosłupa jest największy. Podač ten największy stosunek $\mathrm{w}$ procentach.

6. Długości boków trapezu opisanego na okręgu $0$ promieniu $R\mathrm{t}\mathrm{w}\mathrm{o}\mathrm{r}\mathrm{z}\Phi \mathrm{c}\mathrm{i}_{\Phi \mathrm{g}}$ arytmetyczny,

przy czym najkrótszy bok ma długośč $\displaystyle \frac{3}{4}R$. Obliczyč długości obu podstaw trapezu oraz

cosinus $\mathrm{k}_{\Phi^{\mathrm{t}\mathrm{a}}}$ pomiędzy $\mathrm{P}^{\mathrm{r}\mathrm{z}\mathrm{e}\mathrm{k}}\Phi^{\mathrm{t}\mathrm{n}\mathrm{y}\mathrm{m}\mathrm{i}}$. Sporządzič rysunek $\mathrm{p}\mathrm{r}\mathrm{z}\mathrm{y}\mathrm{j}\mathrm{m}\mathrm{u}\mathrm{j}_{\Phi}\mathrm{c}R=2$ cm.





PRACA KONTROLNA $\mathrm{n}\mathrm{r}4-$ POZIOM PODSTAWOWY

styczeń 2008r.

l. Ramka z drutu 0 długości 1 ma kształt kwadratu zakończonego

trójk$\Phi$tem równoramiennym, jak na rysunku. Bok kwadratu wynosi

a, natomiast ramię trójkąta równe jest b. Wyznaczyč a i b tak, by

pola kwadratu i trójkąta byly jednakowe.

2. Niech

$A=\{(x,y):x\in \mathbb{R},y\in \mathbb{R},y=-x+a,a\in\langle-2,2\rangle\},$

$B=\displaystyle \{(x,y):x\in \mathbb{R},y\in \mathbb{R},y=kx,k\in\langle\frac{1}{2},1\rangle\}.$

$\mathrm{W}\mathrm{P}^{\mathrm{r}\mathrm{o}\mathrm{s}\mathrm{t}\mathrm{o}\mathrm{k}}\Phi^{\mathrm{t}\mathrm{n}\mathrm{y}\mathrm{m}}$ układzie współrzędnych narysowač zbiór $A\cap B\mathrm{i}$ obliczyč jego pole.

Sprawdzič, czy punkt $(\displaystyle \frac{1}{2},\frac{3}{4})$ nalezy do zbioru $A\cap B.$

3. Dany jest stozek ścięty, $\mathrm{w}$ którym pole dolnej podstawy jest 4 razy większe od po1a

górnej. $\mathrm{W}$ stozek wpisano walec $\mathrm{t}\mathrm{a}\mathrm{k}, \dot{\mathrm{z}}\mathrm{e}$ dolna podstawa walca $\mathrm{l}\mathrm{e}\dot{\mathrm{z}}\mathrm{y}$ na dolnej podstawie

stozka, a brzeg górnej podstawy $\mathrm{l}\mathrm{e}\dot{\mathrm{z}}\mathrm{y}$ na jego powierzchni bocznej. $\mathrm{J}\mathrm{a}\mathrm{k}_{\Phi}$ częśč objętości

stozka ściętego stanowi objętośč walca, $\mathrm{j}\mathrm{e}\dot{\mathrm{z}}$ eli wysokośč walca jest 3 razy mniejsza od

wysokości stozka? Odpowied $\acute{\mathrm{z}}$ podač $\mathrm{w}$ procentach $\mathrm{z}$ dokładności$\Phi$ do jednego promila.

Sporz$\Phi$dzič staranny rysunek przekroju osiowego bryly.

4. Rozwi$\Phi$zač nierównośč $f(x)+3x>1$, gdzie $f(x)=\displaystyle \frac{1-3x}{\sqrt{2-\frac{3x+1}{x-2}}}.$

5. Dane $\mathrm{s}\Phi$ dwa $\displaystyle \mathrm{c}\mathrm{i}_{\Phi \mathrm{g}}\mathrm{i}a_{n}=\frac{1}{n}$ oraz $b_{n}=\displaystyle \frac{n-2}{(n+2)(n+4)}$. Zbadač monotonicznośč ciqgu

$c_{n}=(n-1)a_{n+1}+2b_{2n}.$

Czy $\mathrm{c}\mathrm{i}_{\Phi \mathrm{g}}c_{n}$ jest ograniczony? Dla jakich $n$ spefniona jest nierównośč $\displaystyle \frac{3}{4}<c_{n}<1$?

6. Okręgi $0$ promieniach $r\mathrm{i}2r\mathrm{p}\mathrm{r}\mathrm{z}\mathrm{e}\mathrm{c}\mathrm{i}\mathrm{n}\mathrm{a}\mathrm{j}_{\Phi}$ się $\mathrm{w}$ punktach A $\mathrm{i}B, \mathrm{b}\text{ę} \mathrm{d}_{\Phi}$cych wierzchołkami

trójk$\Phi$ta równobocznego $ABC$ wpisanego $\mathrm{w}$ jeden $\mathrm{z}$ okręgów. Obliczyč pole deltoidu

ADBC, którego wierzchołek $D\mathrm{l}\mathrm{e}\dot{\mathrm{z}}\mathrm{y}$ na drugim okręgu oraz wyznaczyč sinus kąta przy

wierzchołku $D.$





PRACA KONTROLNA $\mathrm{n}\mathrm{r} 4-$ POZIOM ROZSZERZONY

l. Dany jest romb ABCD $0$ boku $a\mathrm{i}\mathrm{k}_{\Phi}\mathrm{c}\mathrm{i}\mathrm{e}$ ostrym $\alpha. \mathrm{Z}$ wierzcholka $A\mathrm{k}_{\Phi^{\mathrm{t}\mathrm{a}}}$ ostrego po-

prowadzono dwa jednakowej długości odcinki $0$ końcach zawartych $\mathrm{w}$ bokach $BC\mathrm{i}CD.$

Wyznaczyč długości tych odcinków oraz sinusy kątów, na jaki został podzielony $\mathrm{k}_{\Phi^{\mathrm{t}}}\alpha$

$\mathrm{w}\mathrm{i}\mathrm{e}\mathrm{d}\mathrm{z}\Phi^{\mathrm{C}}, \dot{\mathrm{z}}\mathrm{e}$ pole środkowego deltoidu jest równe połowie pola danego rombu.

2. Napisač równanie stycznej do krzywej $f(x)=\displaystyle \frac{x}{x^{2}-1} \mathrm{w}$ punkcie $x_{0} = 2$. Wykazač, $\dot{\mathrm{z}}\mathrm{e}$

obrazem tej stycznej $\mathrm{w}$ symetrii względem punktu $(0,0)$ jest prosta, która jest styczną

do tej samej krzywej. Wyznaczyč odległośč między tymi stycznymi.

3. Niech

$A=\{(x,y):x\in \mathbb{R},y\in \mathbb{R},|x-1|+x\geq y+|y-2|\},$

$B=\displaystyle \{(x,y):x\in \mathbb{R},y\in \mathbb{R},|x-1|+\frac{1}{4}|y|\leq 1\}.$

Na płaszczy $\acute{\mathrm{z}}\mathrm{n}\mathrm{i}\mathrm{e}OXY$ narysowač zbiory $A\cap B$ oraz $B'\backslash A.$

4. Dane jest równanie

8 $(\sin\alpha+4)x^{2}-8(\sin\alpha+1)x+1=0,$

gdzie $\alpha \in \langle 0,  2\pi\rangle$. Dla jakich wartości $\mathrm{k}_{\Phi^{\mathrm{t}\mathrm{a}}}\alpha$ suma odwrotności pierwiastków tego

równania jest równa co najmniej 8 $(\cos\alpha-(\cos\alpha)^{-1}+1)$ ?

5. Zbadač funkcję $f(m)=\displaystyle \frac{y}{x}$, gdzie para $x\mathrm{i}y$ jest $\mathrm{r}\mathrm{o}\mathrm{z}\mathrm{w}\mathrm{i}_{\Phi}$zaniem układu równań

$\left\{\begin{array}{l}
(m-2)x+(m+2)y=m^{2}-1\\
(m+2)x+(m-2)y=m^{2}+1,
\end{array}\right.$

$\mathrm{z}$ parametrem rzeczywistym $m$. Sporz$\Phi$dzič wykres funkcji $f(m).$

6. $\mathrm{W}$ stozek $0$ promieniu podstawy $r\mathrm{i}\mathrm{t}\mathrm{w}\mathrm{o}\mathrm{r}\mathrm{z}\Phi^{\mathrm{C}\mathrm{e}\mathrm{j}}l$ wpisano ostrosłup prawidłowy trójkątny

$\mathrm{t}\mathrm{a}\mathrm{k}, \dot{\mathrm{z}}\mathrm{e}$ wierzchołek tego ostrosłupa pokrywa się ze środkiem podstawy stozka, a pozo-

stałe wierzchołki $\mathrm{l}\mathrm{e}\mathrm{z}\Phi$ na ścianie bocznej stozka. Jaka jest maksymalna objętośč tego

ostrosłupa? Sporzqdzič staranny rysunek.





PRACA KONTROLNA $\mathrm{n}\mathrm{r}5-$ POZIOM PODSTAWOWY

luty $2008\mathrm{r}.$

l. Ile razy objętośč ostrosłupa trójkątnego prawidfowego opisanego na stozku $0$ objętości $V$

jest większa od objętości ostrosłupa trójkątnego prawidlowego wpisanego $\mathrm{w}$ ten stozek?

2. Rozwi$\Phi$zač nierównośč

$|4x^{2}-4|+2x\geq|1-x|+2.$

3. Kamilek ma 2 latka $\mathrm{i}85$ cm wzrostu. Przez kolejne 31ata będzie rósł średnio 1cm mie-

sięcznie. Potem $\mathrm{w}\mathrm{c}\mathrm{i}_{\Phi \mathrm{g}}\mathrm{u}\mathrm{k}\mathrm{a}\dot{\mathrm{z}}$ dych 10 miesięcy będzie rósł $0$ 10\% wolniej $\mathrm{n}\mathrm{i}\dot{\mathrm{z}}\mathrm{w}$ poprzednim

okresie. Jaki wzrost będzie miał chłopczyk $\mathrm{w}$ dniu swoich 15-tych urodzin? Wynik podač

$\mathrm{z}$ dokładności$\Phi$ do 5 mm.

4. Uzasadnič, wykonuj $\Phi^{\mathrm{C}}$ odpowiednie obliczenia, $\dot{\mathrm{z}}\mathrm{e}\mathrm{z}$ kartki papieru $\mathrm{w}$ kształcie sześciok$\Phi$ta

foremnego $0$ boku $a= 2(1+\sqrt{3}) \mathrm{m}\mathrm{o}\dot{\mathrm{z}}$ na wyci$\Phi$č 19 kółek $0$ promieniu l. Czy istnieje

mniejszy sześciok$\Phi$t foremny, $\mathrm{z}$ którego $\mathrm{m}\mathrm{o}\dot{\mathrm{z}}$ na wyciąč taką $\mathrm{s}\mathrm{a}\mathrm{m}\Phi$ ilośč identycznych kółek?

5. Punkty (l, l) $\mathrm{i} (5,4) \mathrm{s}\Phi$ dwoma wierzchołkami rombu $0$ polu 15. Opisač konstrukcje

wszystkich rombów spełniaj $\Phi^{\mathrm{C}}\mathrm{y}\mathrm{c}\mathrm{h}$ podane warunki. Wyznaczyč wspófrzędne pozostałych

wierzcholków, przy załozeniu, $\dot{\mathrm{z}}\mathrm{e}$ nie wszystkie wierzchołki $\mathrm{l}\mathrm{e}\mathrm{z}\Phi \mathrm{w}$ I čwiartce układu

wspólrzędnych.

6. Wyznaczyč równanie krzywej będ$\Phi$cej zbiorem wszystkich środków cięciw paraboli

$y=(x-1)^{2}+1 \mathrm{p}\mathrm{r}\mathrm{z}\mathrm{e}\mathrm{c}\mathrm{h}\mathrm{o}\mathrm{d}_{\mathrm{Z}\Phi}$cych przez punkt $P(-1,2).$

(Wsk. Zauwazyč, $\dot{\mathrm{z}}\mathrm{e}\mathrm{j}\mathrm{e}\dot{\mathrm{z}}$ eli $x_{1}, x_{2}$ są pierwiastkami trójmianu kwadratowego $y=ax^{2}+bx+c,$

to prawdziwa jest równośč $x_{1}+x_{2}=\displaystyle \frac{-b}{a}.$)





PRACA KONTROLNA $\mathrm{n}\mathrm{r} 5-$ POZIOM ROZSZERZONY

l. Rozwi$\Phi$zač równanie

$\displaystyle \mathrm{t}\mathrm{g}^{2}x+\mathrm{t}\mathrm{g}^{4}x+\cdots=\frac{1}{2},$

$\mathrm{w}$ którym lewa strona jest $\mathrm{s}\mathrm{u}\mathrm{m}\Phi$ wyrazów nieskończonego ciągu geometrycznego.

2. Pani Józefa wpłaciła do banku pewien kapitał $K_{0}$ na okres jednego roku na lokatę opro-

$\mathrm{c}\mathrm{e}\mathrm{n}\mathrm{t}\mathrm{o}\mathrm{w}\mathrm{a}\mathrm{n}\Phi$ {\it P}\% $\mathrm{w}$ skali roku, przy czym kapitalizacja odsetek następuje $N$ razy rocznie.

Uzasadnič indukcyjnie, $\dot{\mathrm{z}}\mathrm{e}$ wzór $K_{n}=K_{0}(1+\displaystyle \frac{P}{100N})^{n}$ okeśla stan konta pani Józefy po

$n$-tym okresie kapitalizacyjnym. Sprawdzič, jaki będzie stan konta pani Józefy po roku

przy załozeniu, $\dot{\mathrm{z}}\mathrm{e}$ wplaci ona 10.000, 00 zł na 6\%, a odsetki kapita1izowane będą co

$\mathrm{m}\mathrm{i}\mathrm{e}\mathrm{s}\mathrm{i}_{\Phi}\mathrm{c}.$

3. Zaznaczyč na płaszczy $\acute{\mathrm{z}}\mathrm{n}\mathrm{i}\mathrm{e}$ zbiór rozwi$\Phi$zań nierówności

$\log_{\frac{1}{2}}(3\log_{x}(2y))\geq 0.$

4. $\mathrm{W}$ koło $0$ promieniu $R$ wpisano trójkqt, którego pole stanowi czwartą częśč pola koła,

ajeden $\mathrm{z}$ k$\Phi$tów ma miarę $\alpha$. Obliczyč iloczyn oraz sumę kwadratów długości boków tego

trójk$\Phi$ta.

5. Wyznaczyč równanie krzywej będ$\Phi$cej zbiorem wszystkich środków okręgów stycznych do

prostej $y=2\mathrm{i}\mathrm{p}\mathrm{r}\mathrm{z}\mathrm{e}\mathrm{c}\mathrm{h}\mathrm{o}\mathrm{d}_{\mathrm{Z}\Phi}$cych przez $\mathrm{P}^{\mathrm{O}\mathrm{C}\mathrm{Z}}\Phi^{\mathrm{t}\mathrm{e}\mathrm{k}}$ układu współrzędnych. Spośród rozwaza-

nych okręgów narysowač wszystkie okręgi styczne do jednej $\mathrm{z}$ osi układu współrzędnych

$\mathrm{i}$ wyznaczyč równanie okręgu przechodzącego przez ich środki.

6. Na dnie naczynia $\mathrm{w}$ kształcie walca umieszczono 6 małych ku1ek $0$ promieniu $R\mathrm{w}$ taki

sposób, $\dot{\mathrm{z}}\mathrm{e}\mathrm{k}\mathrm{a}\dot{\mathrm{z}}$ da $\mathrm{z}$ nich jest styczna do dwu innych kulek $\mathrm{i}$ ściany bocznej naczynia.

Następnie umieszczono $\mathrm{w}$ nim kulę $0$ promieniu $ 2R\mathrm{s}\mathrm{t}\mathrm{y}\mathrm{c}\mathrm{z}\mathrm{n}\Phi$ do $\mathrm{k}\mathrm{a}\dot{\mathrm{z}}$ dej $\mathrm{z}$ małych kulek

oraz górnej podstawy walca. Sprawdzič, ile wody zmieści się $\mathrm{w}$ tak zapełnionym naczyniu.







XXXVIII

KORESPONDENCYJNY KURS

Z MATEMATYKI

$\mathrm{p}\mathrm{a}\acute{\mathrm{z}}$dziernik 2008 $\mathrm{r}.$

PRACA KONTROLNA nr l- POZIOM PODSTAWOWY

l. Ile jest liczb pięciocyfrowych podzielnych przez 9, które $\mathrm{w}$ rozwinieciu dziesiętnym maja:

a) obie cyfry 1, 2 $\mathrm{i}$ tylko $\mathrm{t}\mathrm{e}$? b) obie cyfry 1, 3 $\mathrm{i}$ tylko $\mathrm{t}\mathrm{e}$? c) wszystkie cyfry 1, 2, 3

$\mathrm{i}$ tylko $\mathrm{t}\mathrm{e}$? Odpowiedz/uzasadnič. $\mathrm{W}$ przypadku b) wypisač otrzymane liczby.

2. Uprościč wyrazenie $w(x)=9x^{2}-\sqrt{(-9x^{2})^{2}}+3x-\sqrt{9x^{2}}$, a następnie:

a) obliczyč $w(\displaystyle \frac{\sqrt{2}-1}{\sqrt{2}+1})$ oraz $w(\displaystyle \frac{1}{1-\sqrt{3}})$

nowniku.

i wynik podač bez niewymierności w mia-

b) wyznaczyč liczbę $b\mathrm{t}\mathrm{a}\mathrm{k}$, by pole obszaru ograniczonego osiami układu współrzędnych

$\mathrm{i}$ wykresem funkcji $f(x)=w(x)+b$ byfo równe 3. Sporz$\Phi$dzič wykres funkcji $f(x).$

3. Sprawdzič, $\dot{\mathrm{z}}\mathrm{e}$ liczby: $k=\displaystyle \frac{(\sqrt{2})^{-4}(\frac{1}{4})^{-\frac{5}{2}}\sqrt[4]{3}}{(\sqrt[4]{16})^{3}\cdot 27^{-\frac{1}{4}}}, n=(\sqrt{3}-\sqrt{2})^{2}+(\sqrt{6}+1)^{2}$ są całkowite

$\mathrm{i}$ dodatnie. Wyznaczyč $m\mathrm{t}\mathrm{a}\mathrm{k}$, by liczby $k, m, n$ byfy odpowiednio: pierwszym, drugim

$\mathrm{i}$ trzecim wyrazem rosnącego ciqgu geometrycznego. Ile trzeba wziąč początkowych wy-

razów tego $\mathrm{c}\mathrm{i}_{\Phi \mathrm{g}}\mathrm{u}$, by ich suma przekroczyła 100?

4. Miejscowości $A(1,1) \mathrm{i}B(3,3) \mathrm{c}\mathrm{h}\mathrm{c}\Phi$ wspólnie wybudowač oczyszczalnię ścieków. Zazna-

czyč na płaszczy $\acute{\mathrm{z}}\mathrm{n}\mathrm{i}\mathrm{e}$ zbiór $\mathrm{m}\mathrm{o}\dot{\mathrm{z}}$ liwych punktów umiejscowienia oczyszczalni wiedząc, $\dot{\mathrm{z}}\mathrm{e}$

powinna ona byč jednakowo oddalona od $\mathrm{k}\mathrm{a}\dot{\mathrm{z}}$ dej $\mathrm{z}$ miejscowości $\mathrm{i}$ odlegfośč ta nie $\mathrm{m}\mathrm{o}\dot{\mathrm{z}}\mathrm{e}$

przekraczač 2. Ponadto od1egfośč oczyszcza1ni od prosto1iniowego odcinka rzeki fączącej

punkty $D(-2,-\displaystyle \frac{3}{2}) \mathrm{i}E(4,3)$ nie powinna byč mniejsza $\mathrm{n}\mathrm{i}\dot{\mathrm{z}} 1$. Rozwiązanie zilustrowač

rysunkiem.

5. Jaką bryłę otrzymujemy łqcząc środki ścian sześcianu? Obliczyč stosunek objętości tej

bryfy do objętości wyjściowego sześcianu.

6. Wysokośč opuszczona na ramię trójkąta równoramiennego dzieli jego pole $\mathrm{w}$ stosunku

1 : 3. Wyznaczyč tangens kata przy podstawie oraz stosunek długości promienia okręgu

wpisanego do dfugości promienia okręgu opisanego na tym trójkącie. Sporządzič odpo-

wiednie rysunki.




PRACA KONTROLNA nr l- POZIOM ROZSZERZONY

1. $\mathrm{Z}$ przystani A wyrusza $\mathrm{z}$ biegiem rzeki statek do przystani $\mathrm{B}$, odległej od A $0140$ km.

Po upfywie l godziny wyrusza za nim łódz$\acute{}$ motorowa, dopędza statek $\mathrm{w}$ pofowie drogi,

po czym wraca do przystani A $\mathrm{w}$ tym samym momencie, $\mathrm{w}$ którym statek przybija do

przystani B. Wyznaczyč predkośč statku $\mathrm{i}$ prędkośč lodzi $\mathrm{w}$ wodzie stojacej wiedzac, $\dot{\mathrm{z}}\mathrm{e}$

prędkośč biegu rzeki wynosi 4 $\mathrm{k}\mathrm{m}/$godz.

2. Niech $a(x)=\displaystyle \frac{\sqrt{x-1}+1}{x-2}$. Dla jakich liczb rzeczywistych $x$ zarówno wartośč $a(x)$ jak $\mathrm{i}$

jej odwrotnośč $\mathrm{s}\Phi$ mniejsze $\mathrm{n}\mathrm{i}\dot{\mathrm{z}}2$?

3. Wyznaczyč cosinus kata między ścianami ośmiościanu foremnego. Obliczyč stosunek dlu-

gości promienia kuli wpisanej do dfugości promienia kuli opisanej na tej bryle. Sporządzič

odpowiednie rysunki.

4. Liczby: $a = 4\displaystyle \cos^{2}\frac{\pi}{12}$ -tg $\displaystyle \frac{\pi}{3}, b = \displaystyle \frac{(\sqrt[3]{2})^{54}(\frac{1}{\sqrt{3}})^{-6}-(2\sqrt{2})^{12}(\sqrt[3]{3})^{6}}{2^{3}\cdot(\sqrt[3]{\frac{1}{32}})^{-12}+(4\sqrt{2})^{8}}$ są odpowied-

nio pierwszym $\mathrm{i}$ piątym wyrazem nieskończonego, malejącego ciągu geometrycznego.

Obliczyč wyraz piętnasty oraz sumę wszystkich wyrazów tego ciągu. Ile początkowych

wyrazów tego ciągu nalezy wziqč, by ich suma przekroczyła 85\% sumy wszystkich wy-

razów?

5. $K\mathrm{a}\dot{\mathrm{z}}$ da $\mathrm{z}$ przekqtnych trapezu ma długośč 5, jedna $\mathrm{z}$ podstaw ma długośč 2, a po1e równe

jest 12. Ob1iczyč promień okręgu opisanego na tym trapezie. Sporządzič rysunek.

6. Jednym $\mathrm{z}$ boków trójkąta $ABC$ jest odcinek $AB$, gdzie $A(1,2), B(3,1)$. Wyznaczyč

równanie zbioru wszystkich punktów $C$ takich, $\dot{\mathrm{z}}\mathrm{e}$ kąt $BCA$ ma miarę $45^{\mathrm{o}}$ oraz opisač

konstrukcję wszystkich trójk$\Phi$tów równoramiennych spelniających warunek ten warunek.

Sporządzič rysunek.





XXXVIII

KORESPONDENCYJNY KURS

Z MATEMATYKI

marzec 2009 r.

PRACA KONTROLNA nr 6- POZIOM PODSTAWOWY

l. Obliczyč wartośč wyrazenia

$\displaystyle \frac{b^{2}-1}{a^{3}+b^{3}}$ : $(\displaystyle \frac{a+b}{1+ab-a^{2}-a^{3}b}+\frac{1}{a+b}\frac{ab+1}{a^{2}-1})$

$\mathrm{d}\mathrm{l}\mathrm{a}a=\sqrt{2}+1, b=\sqrt{2}-1.$

2. Pole deltoidu wpisanego $\mathrm{w}$ okrąg $0$ promieniu $r$ równe jest $r^{2} \mathrm{W}$

deltoidu.

Wyznaczyč kąty tego

3. $\mathrm{Z}$ miast A $\mathrm{i}\mathrm{B}$ wyruszyly jednocześnie dwa samochody jadące ze stałymi prędkościami

naprzeciw siebie. Do chwili spotkania pierwszy $\mathrm{z}$ nich przebyl drogę $\mathrm{o}d$ km większą $\mathrm{n}\mathrm{i}\dot{\mathrm{z}}$

drugi. Jadqc dalej $\mathrm{z}$ tymi samymi prędkościami, pierwszy samochód przebyf drogę od $\mathrm{A}$

do $\mathrm{B}\mathrm{w}m$ godzin, drugi zaś $\mathrm{w}n$ godzin. Obliczyč odlegfośč między miastami A $\mathrm{i}$ B.

4. Wyznaczyč wszystkie trójkqty równoramienne $0$ wierzchołkach $A(1,0), B(4,1), \mathrm{w}$ któ-

rych $|AB| = |AC| \mathrm{i}$ środkowa $CD$ boku $AB$ jest zawarta $\mathrm{w}$ prostej $x+y=3$. Znalez/č

wspólrzędne środka cięzkości tego $\mathrm{z}$ trójkątow, który ma najmniejsze pole.

5. Sporządzič staranny wykres funkcji $f$ zadanej wzorem

$f(x)=$

gdy

gdy

$|x-\displaystyle \frac{3}{2}|\leq\frac{3}{2},$

$|x-\displaystyle \frac{3}{2}|>\frac{3}{2}.$

Posfugując się wykresem określič zbiór wartości funkcji $f$. Wyznaczyč najmniejszą $\mathrm{i}$

najwiekszą wartośč funkcji $\mathrm{w}$ przedziale $[2-\sqrt{2},2+\sqrt{2}].$

6. $\mathrm{W}$ stozek wpisano graniastoslup prosty $\mathrm{t}\mathrm{a}\mathrm{k}, \dot{\mathrm{z}}\mathrm{e}$ podstawa dolna graniastosfupa zawiera się

$\mathrm{w}$ podstawie stozka, a wierzchołki górnej podstawy nalezą do powierzchni bocznej stozka.

Podstawą graniastosłupajest trójkąt prostokatny, $\mathrm{w}$ którym stosunek przyprostokqtnych

wynosi 1 : 3, a po1e powierzchni największej ściany bocznej jest 2 razy mniejsze $\mathrm{n}\mathrm{i}\dot{\mathrm{z}}$

pole przekroju osiowego stozka. Obliczyč stosunek objętości graniastosłupa do objętości

stozka.





PRACA KONTROLNA nr 6- POZIOM ROZSZERZONY

l. Sporządzič staranny wykres funkcji

stępowania.

$f(x)= |2\displaystyle \frac{3-|x|}{2}-1|$. Opisač $\mathrm{i}$ uzasadnič sposób po-

2. Rozwiązač nierównośč

$\displaystyle \frac{\sqrt{x^{2}-1}}{x}\leq\frac{\sqrt{6x+36}}{8}.$

3. Punkty $K, L, M$ dzielq odpowiednio boki AB, $BC, CA$ trójkąta $\mathrm{w}$ stosunku 1 : 3 oraz

$\vec{AB} =$ [11, 2], $\vec{AC} =$ [2, 4]. Posługuj $\Phi^{\mathrm{C}}$ się rachunkiem wektorowym, obliczyč cosinus

$\mathrm{k}_{\Phi}\mathrm{t}\mathrm{a}\angle MKL.$

4. Wyznaczyč wszystkie wartości parametru całkowitego $m$, dla których para liczb $(x,y)$

spefniająca ukfad równań

$\left\{\begin{array}{l}
2x+y=4\\
4x+3y=m
\end{array}\right.$

jest rozwiązaniem nierówności $x-\sqrt{8}y\leq 4$ oraz $x\log_{3}2+y\log_{3}5\leq x\log_{3}7.$

5. Podstawą ostrosłupa czworokątnego jest prostokąt $0$ przekątnej długości $d$, a wszyst-

kie krawędzie boczne maja tę samq długośč. Większa ściana boczna jest nachylona do

podstawy pod kątem $\alpha$, a mniejsza pod kątem $\beta$. Obliczyč objętośč $\mathrm{i}$ pole powierzchni

bocznej ostrosłupa.

6. Dany jest ukfad równań

$\left\{\begin{array}{l}
x-3|y+1|=0\\
(x-p)^{2}+y^{2}=5,
\end{array}\right.$

gdzie $p$ jest parametrem rzeczywistym.

a) Rozwiązač algebraicznie powyzszy układ dla $p=2\mathrm{i}$ podač jego interpretację geo-

metryczną. Sporz$\Phi$dzič rysunek.

b) Korzystając $\mathrm{z}$ rysunku $\mathrm{i}$ odpowiednich rozwazań geometrycznych, określič liczbę

rozwiązań danego układu $\mathrm{w}$ zalezności od parametru $p.$





XXXVIII

KORESPONDENCYJNY KURS

Z MATEMATYKI

listopad 2008 r.

PRACA KONTROLNA nr 2- POZIOM PODSTAWOWY

l. Niech $A=\{(x,y):|x|+2y\leq 3\}, B=\{(x,y):|y|\geq x^{2}\}$. Zaznaczyč na p{\it l}aszczy $\acute{\mathrm{z}}\mathrm{n}\mathrm{i}\mathrm{e}$

zbiory $A\cap B, A\backslash B.$

2. Trapez $0$ kątach przy podstawie $30^{\mathrm{o}}$ oraz $45^{\mathrm{o}}$ jest opisany na okręgu $0$ promieniu $R.$

Obliczyč stosunek pola kola do pola trapezu.

3. Dla jakich wartości $\mathrm{k}_{\Phi}\mathrm{t}\mathrm{a}\alpha\in[0,2\pi]$ równanie kwadratowe

$\sin\alpha\cdot x^{2}-2x+2\sin\alpha-1=0$

ma dokfadnie jedno rozwiązanie?

4. Pole powierzchni bocznej ostrosłupa prawidlowego trójkatnego jest 6 razy większe $\mathrm{n}\mathrm{i}\dot{\mathrm{z}}$

pole jego podstawy. Obliczyč cosinus $\mathrm{k}_{\Phi^{\mathrm{t}\mathrm{a}}}$ nachylenia krawędzi bocznej ostrosfupa do

pfaszczyzny podstawy.

5. Iloczyn dwu liczb jest 20 razy wiekszy $\mathrm{n}\mathrm{i}\dot{\mathrm{z}}$ odwrotnośč ich sumy. Suma sześcianów tych

liczb stanowi 325\% i1oczynu tych 1iczb $\mathrm{i}$ ich sumy. Jakie to liczby?

6. Narysowač wykres funkcji

$f(x)=$

gdy

gdy

$|x|\leq 1,$

$|x|>1.$

a) Obliczyč $f(-\displaystyle \frac{1}{1+\sqrt{2}})$ oraz $f(\displaystyle \frac{1+\sqrt{2}}{2})$

mianowniku.

Wynik podač bez niewymierności w

b) Wykorzystując wykres rozwiązač nierównośč

rozwiqzań na osi 0x

$f(x) \geq -\displaystyle \frac{1}{2}$

i zaznaczyč zbiór jej

c) Odczytač z wykresu przedziafy, na których funkcja f jest malejąca.





PRACA KONTROLNA nr 2- POZIOM ROZSZERZONY

l. Zaznaczyč na płaszczy $\acute{\mathrm{z}}\mathrm{n}\mathrm{i}\mathrm{e}$ zbiór $\displaystyle \{(x,y):|x|\leq\frac{3}{2},\log_{\frac{2}{3}}|x|<y<\log_{\frac{3}{2}}|x|\}.$

2. Wykazač, $\dot{\mathrm{z}}\mathrm{e}$ róznica kwadratów dwu dowolnych liczb cafkowitych niepodzielnych przez

3 jest liczbą podzielną przez 3.

3. $\mathrm{W}$ trójkącie równoramiennym $ABC0$ podstawie $AB$ ramię ma długośč $b$, a kąt przy

wierzchofku C- miarę $\gamma. D$ jest takim punktem ramienia $BC, \dot{\mathrm{z}}\mathrm{e}$ odcinek $AD$ dzieli pole

trójkąta na polowę. Wyznaczyč promienie okregów wpisanych $\mathrm{w}$ trójkqty $ABD\mathrm{i}ADC.$

Dla jakiego kąta $\gamma$ promienie te $\mathrm{s}\Phi$ równe?

4. Niech $f(x)=3(x+2)^{4}+x^{2}+4x+p$, gdzie $p$ jest parametrem rzeczywistym.

a) Uzasadnič, $\dot{\mathrm{z}}\mathrm{e}$ wykres funkcji $f(x)$ jest symetryczny względem prostej $x=-2.$

b) Dla jakiego parametru rzeczywistego $p$ najmniejszą wartością funkcji $f(x)$ jest

$y=-2$ ? Odpowied $\acute{\mathrm{z}}$ uzasadnič, nie $\mathrm{s}\mathrm{t}\mathrm{o}\mathrm{s}\mathrm{u}\mathrm{j}_{\Phi}\mathrm{c}$ metod rachunku rózniczkowego.

c) Określič liczbę pierwiastków równania $f(x)=0\mathrm{w}$ zalezności od parametru $p.$

5. Rozwiązač nierównośč $|\sin x-\sqrt{3}\cos x|\geq 1.$

6. Rozwiązač równanie

$1-(\displaystyle \frac{2^{x}}{3^{x}-2^{x}})+(\frac{2^{x}}{3^{x}-2^{x}})^{2}-(\frac{2^{x}}{3^{x}-2^{x}})^{3}+\ldots=\frac{3^{x-2}}{2^{x-1}},$

którego lewa strona jest sumą wyrazów nieskończonego ciągu geometrycznego.





XXXVIII

KORESPONDENCYJNY KURS

Z MATEMATYKI

grudzień 2008 r.

PRACA KONTROLNA nr 3 -POZIOM PODSTAWOWY

l. Boki $a_{n}\mathrm{i}b_{n}$ prostokąta $P_{n}$ są wyrazami ciągów arytmetycznych, $\mathrm{w}$ których $a_{1}=b_{1}=100$

oraz $r_{1}=5\mathrm{i}r_{2}=-5$. Znalez/č wszystkie wartości $n$, dla których pole prostokąta $P_{n}$ jest

mniejsze $0$ co najmniej 40\% od po1a $\mathrm{P}^{\mathrm{r}\mathrm{o}\mathrm{s}\mathrm{t}\mathrm{o}\mathrm{k}}\Phi^{\mathrm{t}\mathrm{a}P_{1}}.$

2. Znalez$\acute{}$č równania dwusiecznych katów zawartych między prostymi $x-7y+6 = 0,$

$x+y-2=0$. Następnie wybrač tę $\mathrm{d}\mathrm{w}\mathrm{u}\mathrm{s}\mathrm{i}\mathrm{e}\mathrm{c}\mathrm{z}\mathrm{n}\Phi$, która tworzy $\mathrm{z}$ osią odciętych mniejszy

$\mathrm{k}_{\Phi^{\mathrm{t}}}$. Sporządzič rysunek.

3. Pudelko zawiera 2l klocków po 7 $\mathrm{w}$ kolorach zółtym, czerwonym $\mathrm{i}$ niebieskim.

Wojtuś ufoz $\mathrm{y}l$ wiez$\cdot$ę $\mathrm{z}8$ przypadkowo wybranych klocków. Jakie jest prawdopodobień-

stwo tego, $\dot{\mathrm{z}}\mathrm{e}\mathrm{w}$ wiezy znalazly się klocki wszystkich trzech kolorów?

4. Nie rozwiązując nierówności wykazač, $\dot{\mathrm{z}}\mathrm{e}$ relacja

$\sqrt{3x-3x^{2}+3}>1+\sqrt[5]{x^{2}+1}$

nie jest spefniona dla $\dot{\mathrm{z}}$ adnej liczby rzeczywistej $x.$

5. $\mathrm{W}$ momencie spostrzezenia samolotu nadlatującego ze stafą prędkością $\mathrm{i}$ na stafej wyso-

kości obserwator widziaf go pod kątem $35^{\mathrm{o}}$ do poziomu. Po jednej minucie kąt ten wzrósl

do $65^{\mathrm{o}}$

a) Po jakim czasie od momentu spostrzezenia samolotu przeleciał on nad głową ob-

serwatora?

b) Przyjmując, $\dot{\mathrm{z}}\mathrm{e}$ samolot leciał $\mathrm{z}$ prędkościq 500 $\mathrm{k}\mathrm{m}/\mathrm{h}$, obliczyč na jakiej wysokości

odbywaf się lot.

Wyniki podač $\mathrm{w}$ zaokrqgleniu do pełnych sekund $\mathrm{i}$ pełnych setek metrów.

6. $\mathrm{W}$ stozek $0$ objętości $V\mathrm{i}$ wysokości $\mathrm{s}\mathrm{t}\mathrm{a}\mathrm{n}\mathrm{o}\mathrm{w}\mathrm{i}_{\Phi}\mathrm{c}\mathrm{e}\mathrm{j}$ 75\% promienia podstawy wpisano walec

$\mathrm{t}\mathrm{a}\mathrm{k}, \dot{\mathrm{z}}\mathrm{e}$ podstawa walca $\mathrm{l}\mathrm{e}\dot{\mathrm{z}}\mathrm{y}$ na podstawie stozka, a wysokośč walca jest równa średnicy

jego podstawy. Obliczyč stosunek pola powierzchni całkowitej walca do pola powierzchni

cafkowitej stozka oraz objętośč kuli opisanej na walcu. Sporządzič odpowiedni rysunek.





PRACA KONTROLNA nr 3 -POZIOM ROZSZERZONY

l. Na diagramie skladającym się $\mathrm{z} 9$ kwadratowych pól $\mathrm{w}$ układzie 3 $\rangle\langle 3$ zaznaczono

$\mathrm{w}$ losowo wybranych polach kófko $\mathrm{i}$ krzyzyk. Jakie jest prawdopodobieństwo tego, $\dot{\mathrm{z}}\mathrm{e}$

oba znaki znalazły się na sąsiednich polach $\mathrm{t}\mathrm{z}\mathrm{n}$. stykających się jednym bokiem.

2. Kąty $\mathrm{c}\mathrm{z}\mathrm{w}\mathrm{o}\mathrm{r}\mathrm{o}\mathrm{k}_{\Phi}\mathrm{t}\mathrm{a}$ wpisanego $\mathrm{w}$ okrąg $0$ promieniu $R\mathrm{t}\mathrm{w}\mathrm{o}\mathrm{r}\mathrm{z}\Phi \mathrm{c}\mathrm{i}_{\Phi \mathrm{g}}$ arytmetyczny, którego

pierwszy wyraz wynosi $\displaystyle \frac{\pi}{4}$. Przekątna czworokąta leząca naprzeciw kąta $\displaystyle \frac{\pi}{4}$ jest prosto-

padla do jednego $\mathrm{z}$ boków. Wyznaczyč kąty, obwód oraz pole tego czworokąta.

3. Trójkąt równoramienny $0$ podstawie $a \mathrm{i}$ kącie przy wierzchołku $36^{\mathrm{o}}$ obraca się wo-

kóf dwusiecznej kąta przy podstawie. Obliczyč objętośč powstafej bryly. Skorzystač

$\mathrm{z}$ twierdzenia $0$ dwusiecznej kąta $\mathrm{w}$ trójkącie. Wynik podač bez $\mathrm{u}\dot{\mathrm{z}}$ ycia funkcji trygono-

metrycznych.

4. Odcinek $0$ końcach $A(1,1)\mathrm{i}B(3,2)$ jest bokiem prostokąta, którego jeden $\mathrm{z}$ wierzchof-

ków $\mathrm{l}\mathrm{e}\dot{\mathrm{z}}\mathrm{y}$ na prostej $l$ : $x-y+1=0$. Znalez/č współrzędne wierzcholków $C\mathrm{i}D$. Obliczyč

cosinus kąta miedzy przekątnymi tego $\mathrm{p}\mathrm{r}\mathrm{o}\mathrm{s}\mathrm{t}\mathrm{o}\mathrm{k}_{\Phi^{\mathrm{t}}}\mathrm{a}$. Sporządzič rysunek.

5. Liczba 2 jest pierwiastkiem podwójnym wielomianu $w(x)=x^{3}+ax^{2}+bx+c$, a funkcja

$f(x) =w(x+1)+p$ jest nieparzysta. Znalez/č ten wielomian $\mathrm{i}$ obliczyč $p$. Na jednym

rysunku sporz$\Phi$dzič wykresy funkcji $f(x)$ oraz $h(x)=|w(x)|.$

6. Wyznaczyč dziedzinę funkcji

$y=\displaystyle \frac{\mathrm{c}\mathrm{t}\mathrm{g}4x}{\cos 2x+\cos 6x}.$





XXXVIII

KORESPONDENCYJNY KURS

Z MATEMATYKI

styczeń 2009 r.

PRACA KONTROLNA nr 4- POZIOM PODSTAWOWY

l. Dane sa funkcje określone wzorami $f(x)=x-3$ oraz $g(x)=4-x, x\in R.$

Rozwiązač nierównośč

$|f(2x-5)+g(x+1)|\displaystyle \leq|f(\frac{x}{2}-1)+g(\frac{x}{2}-4)|-2|g(\frac{x}{2})|.$

2. Wartośč $\mathrm{u}\dot{\mathrm{z}}$ ytkowa pewnego $\mathrm{u}\mathrm{r}\mathrm{z}\Phi^{\mathrm{d}}$Zenia maleje $\mathrm{z}$ roku na rok $\mathrm{w}$ postępie arytmetycz-

nym. $\mathrm{W}$ jakim czasie maszyna będzie całkowicie bezuzyteczna, $\mathrm{j}\mathrm{e}\dot{\mathrm{z}}$ eli po 251atach pracy

jej wartośč byfa trzykrotnie mniejsza, $\mathrm{n}\mathrm{i}\dot{\mathrm{z}}$ jej wartośč po 151atach pracy? Po pewnych

udoskonaleniach wydluzono czas $\mathrm{u}\dot{\mathrm{z}}$ ytkowania takiego urządzenia $0$ pieč lat. $\mathrm{O}$ ile wol-

niej będzie teraz spadač jego wartośč $\mathrm{u}\dot{\mathrm{z}}$ ytkowa rocznie? Wynik podač $\mathrm{w}$ procentach $\mathrm{z}$

dokladności$\Phi$ do jednego miejsca po przecinku.

3. $\mathrm{W}$ okrąg wpisano cztery okręgi $\mathrm{w}$ sposób pokazany na rysunku.

Wyznaczyč stosunek pola rombu, którego wierzchołkami są środki

czterech wpisanych okręgów, do pola kola, $\mathrm{w}$ które wpisano te okręgi.

4. Wyznaczyč wartośč parametru $a$, dla którego funkcja kwadratowa $0$ równaniu

$f(x) = (a-1)x^{2}+(a-2)x+1$ osiqga najmniejszą wartośč równa l. Następnie zna-

lez$\acute{}$č równanie prostej $\mathrm{P}^{\mathrm{r}\mathrm{z}\mathrm{e}\mathrm{c}\mathrm{h}\mathrm{o}\mathrm{d}\mathrm{z}}\Phi^{\mathrm{c}\mathrm{e}\mathrm{j}}$ przez punkt $A(a,2a+1)$ prostopadfej do prostej $0$

równaniu $4y+x+8=0$. Jakie jest wzajemne polozenie otrzymanej prostej $\mathrm{i}$ wykresu

funkcji $f$? Wykonač staranny wykres funkcji $f$ oraz obu prostych.

5. Wyznaczyč dziedzinę funkcji danej wzorem

$f(x)=\displaystyle \frac{x-1}{\sqrt{1-\frac{2x}{x-1}}},$

a następnie rozwiązač równanie $f(x)-f(-x)=2.$

6. $\mathrm{W}$ prawidłowym ostrosłupie trójkątnym ściana boczna ma pole dwa razy większe od

pola podstawy. Promień kuli wpisanej $\mathrm{w}$ ten ostrosfup ma dfugośč $r=1$. Obliczyč sumę

wszystkich wysokości tego ostrosłupa oraz wyznaczyč tangens kąta nachylenia krawędzi

bocznej do pfaszczyzny podstawy.





PRACA KONTROLNA nr 4- POZIOM ROZSZERZONY

l. Janek oszczędza na komputer $\mathrm{i}\mathrm{w}$ tym celu włozyl $4000\mathrm{z}l$ na lokatę roczna. Oprocento-

wanie tej lokaty wynosi 12\% $\mathrm{w}$ skali roku, a odsetki kapitalizowane $\mathrm{s}\varpi$ co $\mathrm{m}\mathrm{i}\mathrm{e}\mathrm{s}\mathrm{i}_{\Phi}\mathrm{c}$. Jaki

dochód przyniesie Jankowi ta lokata? Czy więcej uzyskafby na lokacie 18\%, $\mathrm{w}$ której

odsetki kapitalizowane są co kwartał?

2. Zbadač monotonicznośč $\mathrm{c}\mathrm{i}_{\Phi \mathrm{g}}\mathrm{u}0$ wyrazach $a_{n}=\displaystyle \frac{1}{n+1}+\frac{1}{n+2}+\ldots+\frac{1}{n+n}$. Czy ten

ciag jest ograniczony? Wyznaczyč $a_{1}, a_{2}\mathrm{i}a_{3}.$

3. Udowodnič, stosując zasadę indukcji matematycznej, $\dot{\mathrm{z}}\mathrm{e}$ dla $\mathrm{k}\mathrm{a}\dot{\mathrm{z}}$ dej liczby naturalnej $n$

liczba $8^{n+1}+9^{2n-1}$ jest podzielna przez 73.

4. Obliczyč sumę wszystkich tych pierwiastków równania

$\displaystyle \sin^{2}(x+\frac{\pi}{3})+\cos^{2}(x-\frac{\pi}{3})=\frac{7}{4},$

które nalezą do przedziafu $(-10,10).$

5. $\mathrm{W}$ trójkat równoboczny $ABC$ wpisano trzy kwadraty $\mathrm{w}$ taki sposób, $\dot{\mathrm{z}}\mathrm{e}$ jeden $\mathrm{z}$ boków

$\mathrm{k}\mathrm{a}\dot{\mathrm{z}}$ dego kwadratu zawiera się wjednym $\mathrm{z}$ boków trójkata. Środki tych kwadratów $\mathrm{t}\mathrm{w}\mathrm{o}\mathrm{r}\mathrm{z}\Phi$

trójkąt równoboczny $PQR$. Obliczyč stosunek pola trójkąta $ABC$ do pola trójkąta $PQR.$

6. {\it K}rawęd $\acute{\mathrm{z}}$ kwadratowej podstawy prostopadfościanu ma dlugośč $a$. Prostopadfościan prze-

cięto pfaszczyzną przechodzącą przezjeden $\mathrm{z}$ wierzchołków prostopadfościanu oraz środki

dwóch sąsiednich krawedzi przeciwległej podstawy $\mathrm{t}\mathrm{a}\mathrm{k}, \dot{\mathrm{z}}\mathrm{e}$ otrzymany przekrój jest pię-

ciokątem. Obliczyč obwód oraz pole tego pięciok$\Phi$ta, $\mathrm{j}\mathrm{e}\dot{\mathrm{z}}$ eli pfaszczyzna przekroju jest

nachylona do płaszczyzny podstawy pod kątem $\alpha.$





XXXVIII

KORESPONDENCYJNY KURS

Z MATEMATYKI

luty 2009 r.

PRACA KONTROLNA nr 5- POZIOM PODSTAWOWY

l. Biegacz wyruszył na trase maratonu, pokonując $\mathrm{k}\mathrm{a}\dot{\mathrm{z}}$ de 300 $\mathrm{m} \mathrm{w}$ ciągu l minuty. Po

uplywie 20 minut wyruszy1 za nim rowerzysta $\mathrm{i}$ jadąc ze $\mathrm{s}\mathrm{t}\mathrm{a}l_{\Phi}$ prędkości$\Phi$, dogonif ma-

ratończyka dokładnie 195 $\mathrm{m}$ przed linia mety. Jaka była prędkośč rowerzysty? Po jakim

czasie powinien wyjechač rowerzysta, aby jadąc ze stałq prędkością 30 $\mathrm{k}\mathrm{m}/\mathrm{h}$, przekro-

czyč linię mety równocześnie $\mathrm{z}$ biegaczem? Wynik zaokrąglič $\mathrm{w}$ dóf $\mathrm{z}$ dokfadnością do l

sekundy.

2. Tangens kata ostrego $\alpha$ równy jest $\displaystyle \frac{a}{b}$, gdzie

$a=(\sqrt{2+\sqrt{3}}-\sqrt{2-\sqrt{3}})^{2}b=(\sqrt{\sqrt{2}+1}-\sqrt{\sqrt{2}-1})^{2}$

Wyznaczyč wartości pozostałych funkcji trygonometrycznych tego kąta. Wykorzystując

wzór $\sin 2\alpha=2\sin\alpha\cos\alpha$, obliczyč miarę $\mathrm{k}_{\Phi}\mathrm{t}\mathrm{a}\alpha.$

3. $\mathrm{W}$ walec wpisano trzy wzajemnie styczne kule $\mathrm{w}$ ten sposób, $\dot{\mathrm{z}}\mathrm{e}\mathrm{k}\mathrm{a}\dot{\mathrm{z}}\mathrm{d}\mathrm{a}\mathrm{z}$ nich jest styczna

do ściany bocznej $\mathrm{i}$ obu podstaw walca. Sprawdzič, jaką cześč objętości walca zajmujq

kule. Wynik wyrazony $\mathrm{w}$ procentach podač $\mathrm{z}$ dokfadnością do l promila.

4. Wskazač wszystkie $\mathrm{t}\mathrm{e}$ wyrazy ciągu $(a_{n})$, gdzie

$a_{n}=\displaystyle \frac{\log_{2}^{2}n+\log_{\frac{1}{2}}(n^{3})}{\log_{n}2}-2\log_{4}(\frac{1}{n^{2}}),$

które są równe zero.

5. Dwie klepsydry, mała $\mathrm{i}\mathrm{d}\mathrm{u}\dot{\mathrm{z}}\mathrm{a}$, odmierzają odpowiednio $m\mathrm{i}n, m<n$, pełnych minut. Po

raz pierwszy obrócono je równocześnie $\mathrm{w}$ samo pofudnie. $K\mathrm{a}\dot{\mathrm{z}}\mathrm{d}_{\Phi}\mathrm{z}$ nich obracano, gdy

tylko przesypał się $\mathrm{w}$ niej cały piasek. Czas mierzono do momentu, gdy obie klepsydry

równocześnie przestały działač. Określič, która była wtedy godzina, $\mathrm{j}\mathrm{e}\dot{\mathrm{z}}$ eli wiadomo, $\dot{\mathrm{z}}\mathrm{e}$

mafą obrócono $013$ razy więcej $\mathrm{n}\mathrm{i}\dot{\mathrm{z}}$ duzą, a gdy mafą obracano po raz jedenasty, $\mathrm{d}\mathrm{u}\dot{\mathrm{z}}\mathrm{a}$

wypełniona byla dokładnie $\mathrm{w}$ połowie.

6. $\mathrm{W}$ trójkąt równoboczny $0$ polu $P$ wpisano $\mathrm{o}\mathrm{k}\mathrm{r}\Phi \mathrm{g}$ oraz trzy ma-

fe okręgi - jak na rysunku. Następnie odcięto narozniki trójkąta

wzdłuz łuków małych okregów. Obliczyč pole koła opisanego na

tak powstalej figurze.





PRACA KONTROLNA nr 5- POZIOM ROZSZERZONY

l. Wśród prostokątów $0$ ustalonej dlugości przekątnej $p$ znalez$\acute{}$č ten, którego pole jest

największe. Nie stosowač metod rachunku rózniczkowego.

2. Znalez/č wszystkie liczby rzeczywiste $m\neq 0$, dla których równanie

$\displaystyle \frac{x}{m}+m=\frac{m}{x}+x+1$

ma dwa rózne pierwiastki $x_{1}, x_{2}$ spefniające warunek $|x_{1}-x_{2}|>x_{1}+x_{2}.$

3. Rozwiązač nierównośč

$2^{3x-1}-2^{2x-1}-2^{x+1}+2>0.$

4. Stosując wzór na zamianę podstawy logarytmu uzasadnič, $\dot{\mathrm{z}}\mathrm{e}$ liczba

$S_{n}=\log_{m^{2^{0}}}x+\log_{m^{2^{1}}}x+\log_{m^{2^{2}}}x+\cdots+\log_{m^{2^{n}}}x$, gdzie $x>0$ oraz $m\in \mathbb{N}, m>1,$

jest $\mathrm{s}\mathrm{u}\mathrm{m}\Phi$ częściową pewnego nieskończonego $\mathrm{c}\mathrm{i}_{\Phi \mathrm{g}}\mathrm{u}$ geometrycznego. Obliczyč sumę wszyst-

kich wyrazów tego ciągu $\mathrm{i}$ zbadač, dla jakiego $x$ suma ta wynosi $\displaystyle \frac{1}{2}.$

5. Określič dziedzinę funkcji $f(x)=\log_{x^{2}}$($1-\mathrm{t}\mathrm{g}x$ tg $2x$).

6. $\mathrm{W}$ kulę wpisano 4 identyczne mafe ku1e wzajemnie do siebie styczne. Ob1iczyč, jaką

częśč objętości $\mathrm{d}\mathrm{u}\dot{\mathrm{z}}$ ej kuli wypełniajq małe kule. Wynik wyrazony $\mathrm{w}$ procentach podač

$\mathrm{z}$ dokladności$\Phi$ do l promila.







XXXIX

KORESPONDENCYJNY KURS

Z MATEMATYKI

$\mathrm{p}\mathrm{a}\acute{\mathrm{z}}$dziernik 2009 $\mathrm{r}.$

PRACA KONTROLNA $\mathrm{n}\mathrm{r} 1-$ POZIOM PODSTAWOWY

l. Właściciel hurtowni sprzedał $\displaystyle \frac{1}{3}$ partii bananów po załozonej przez siebie cenie. Okazalo

się, $\dot{\mathrm{z}}\mathrm{e}$ owoce zbyt szybko dojrzewają, więc obnizyf cenę $0$ 30\% $\mathrm{i}$ wówczas sprzedaf 60\%

pozostałej ilości owoców. Resztę bananów udało mu się sprzedač dopiero, gdy ustalił

ich cenę na poziomie $\displaystyle \frac{1}{5}$ ceny początkowej. Ile procent zaplanowanego zysku stanowi

kwota uzyskana ze sprzedaz $\mathrm{y}$? Po ile powinien byf sprzedač pierwszą partię towaru, by

jednokrotna obnizka ich ceny $0$ 25\% pozwoliła na sprzedaz wszystkich owoców $\mathrm{i}$ uzyskanie

zaplanowanego początkowo zysku?

2. Przekątne trapezu $0$ podstawach 3 $\mathrm{i}4$ przecinają się pod kątem prostym. Na $\mathrm{k}\mathrm{a}\dot{\mathrm{z}}$ dym

$\mathrm{z}$ boków trapezu, jako na średnicy, oparto półokrąg. Obliczyč sume pól otrzymanych

czterech pólkoli. Sporządzič rysunek.

3. Uprościč wyrazenie $\displaystyle \frac{1}{\sqrt{a}-\sqrt{b}}(\sqrt[6]{a^{5}}-\frac{b}{\sqrt[6]{\alpha}}) -\displaystyle \frac{a-b}{\sqrt[3]{a^{2}}+\sqrt[6]{\alpha}\sqrt{b}} \mathrm{d}\mathrm{l}\mathrm{a}a, b, \mathrm{d}\mathrm{l}\mathrm{a}$ których ma

ono sens. Następnie obliczyč jego wartośč, przyjmując $a=(4-2\sqrt{3})^{3}\mathrm{i} b=3+2\sqrt{2}.$

4. Podstawą ostrosfupa prawidlowego jest sześciok$\Phi$t foremny $0$ boku $a$. Obliczyč objętośč,

wiedząc, $\dot{\mathrm{z}}\mathrm{e}$ najmniejszy ($\mathrm{w}$ sensie powierzchni) $\mathrm{z}$ przekrojów ostrosłupa płaszczyzną

zawierajqcą wysokośč jest trójkątem równobocznym. Wyznaczyč cosinus kąta między

ścianami bocznymi ostrosfupa. Sporz$\Phi$dzič rysunek.

5. Dana jest funkcja liniowa $f(x)=2x-6.$

a) Dlajakiego $a$ pole trójkąta ograniczonego osiami ukfadu wspófrzędnych $\mathrm{i}$ wykresem

funkcji $h(x)=f(x-a)$ równe jest 4? Sporządzič rysunek.

b) Narysowač zbiór $D=\{(x,y):f(x^{2}+2x)\leq y\leq f(x+2)\}.$

6. Sporządzič wykres funkcji $f(x)=$

dla

dla

$x<0,$

$x\geq 0.$

Posfugując się nim, wyznaczyč przedzialy monotoniczności tej funkcji. Narysowač wy-

kres funkcji $g(m)$ określajacej liczbę rozwiqzań równania $f(x)=|m| \mathrm{w}$ zalezności od

parametru rzeczywistego $m.$




PRACA KONTROLNA nr l- POZIOM ROZSZERZONY

l. Statek wyrusza ($\mathrm{z}$ biegiem rzeki) $\mathrm{z}$ przystani A do odległej $0 140$ km przystani B. Po

uplywie l godziny wyrusza za nim łódz/ motorowa, dopędza statek $\mathrm{w}$ pofowie drogi,

po czym wraca do przystani A $\mathrm{w}$ tym samym momencie, $\mathrm{w}$ którym statek przybija do

przystani B. Wyznaczyč prędkośč statku $\mathrm{i}$ prędkośč lodzi $\mathrm{w}$ wodzie stojącej, wiedzqc, $\dot{\mathrm{z}}\mathrm{e}$

prędkośč nurtu rzeki wynosi 4 $\mathrm{k}\mathrm{m}/$godz.

2. Uprościč wyrazenie (dla $a, b$, dla których ma ono sens)

$(\displaystyle \frac{\sqrt[6]{b}}{\sqrt{b}-\sqrt[6]{a^{3}b^{2}}}-\frac{a}{\sqrt{ab}-a\sqrt[3]{b}})[\frac{1}{\sqrt{a}-\sqrt{b}}(\sqrt[6]{a^{5}}-\frac{b}{\sqrt[6]{a}})-\frac{a-b}{\sqrt[3]{a^{2}}+\sqrt[6]{a}\sqrt{b}}],$

a nastepnie obliczyč jego wartośč dla $a=4\log_{4}81 \mathrm{i} b=(\log_{3}2)^{-1}$

3. Rozwiązač równanie $\sin 2x+\sin x=2+\cos x-2\cos^{2}x.$

4. Rozwiązač nierównośč $\displaystyle \frac{1}{\sqrt{4-x^{2}}}\geq\frac{1}{x-1} \mathrm{i}$ starannie zaznaczyč zbiór rozwi$\Phi$zań na osi

liczb owej.

5. $K\mathrm{a}\dot{\mathrm{z}}\mathrm{d}\mathrm{a}\mathrm{z}$ przekątnych trapezu ma dlugośč 5, jedna $\mathrm{z}$ podstaw ma długośč 2, a po1e równe

jest 12. Ob1iczyč promień okręgu opisanego $\mathrm{n}\mathrm{a}$ tym trapezie. Sporządzič rysunek.

6. $\mathrm{W}$ czworościanie ABCD jedna krawęd $\acute{\mathrm{z}}$ jest $0$ połowę krótsza od pozostałych, które sq

równe. Obliczyč objętośč oraz cosinusy kątów dwuściennych tego czworościanu. Sporzą-

dzič rysunek.





XXXX

KORESPONDENCYJNY KURS

Z MATEMATYKI

marzec 2010 r.

PRACA KONTROLNA nr 6- POZIOM PODSTAWOWY

l. Logarytmy (przy ustalonej podstawie) $\mathrm{z}$ liczb: $a_{1}=\displaystyle \frac{2}{5}x, a_{2}=x-1, a_{3}=x+3$ tworzą ciąg

arytmetyczny. Wyznaczyč $x$. Dla znalezionego $x$ obliczyč sumę początkowych dziesięciu

wyrazów ciągu geometrycznego, którego trzema pierwszymi wyrazami są liczby $a_{1}, a_{2}, a_{3}.$

2. Odcinek $0$ końcach $A(\displaystyle \frac{5}{2},\frac{\sqrt{3}}{2}), B(\displaystyle \frac{5}{2},\frac{3\sqrt{3}}{2})$ jest bokiem wielokąta foremnego wpisanego $\mathrm{w}$

okrąg styczny do osi $Ox$. Wyznaczyc równanie tego okręgu $\mathrm{i}$ wspófrzędne pozostafych

wierzchołków wielokąta. Ile rozwiązań ma to zadanie? Sporządzič rysunek.

3. Dany jest ostroslup prawidlowy trójk$\Phi$tny, $\mathrm{w}$ którym krawęd $\acute{\mathrm{z}}$ bocznajest dwa razy dfuz-

sza $\mathrm{n}\mathrm{i}\dot{\mathrm{z}}$krawed $\acute{\mathrm{z}}$ podstawy. Ostrosłup ten podzielono płaszczyzną przechodzącą przez kra-

$\mathrm{w}\mathrm{e}\mathrm{d}\acute{\mathrm{z}}$ podstawy na dwie bryły $0$ tej samej objętości. Wyznaczyč tangens kąta nachylenia

tej pfaszczyzny do pfaszczyzny podstawy. Sporz$\Phi$dzič rysunek.

4. $\mathrm{O}$ kącie $\alpha$ wiadomo, $\displaystyle \dot{\mathrm{z}}\mathrm{e}\sin\alpha-\cos\alpha=\frac{2}{\sqrt{3}}.$

a) Określič, $\mathrm{w}$ której čwiartce jest kąt $\alpha.$

b) Obliczyč tg $\alpha+$ ctg $\alpha$ oraz $\sin\alpha+\cos\alpha.$

c) Wyznaczyč tg $\alpha.$

5. Dfuzsza przyprostokątna $b$ trójkqta prostokątnego $0$ kącie ostrym $30^{\mathrm{o}}$ jest średnicą pól-

okręgu dzielącego ten trójkqt na dwa obszary. Wyznaczyč stosunek pól tych obszarów

oraz dfugośč promienia okręgu wpisanego $\mathrm{w}$ obszar $\mathrm{z}\mathrm{a}\mathrm{w}\mathrm{i}\mathrm{e}\mathrm{r}\mathrm{a}\mathrm{j}_{\Phi}\mathrm{c}\mathrm{y}$ wierzchofek kąta $60^{\mathrm{o}}$

Sporządzič rysunek.

6. Dwaj turyści wyruszyli jednocześnie: jeden $\mathrm{z}$ punktu $A$ do punktu $B$, drugi-z $B$ do $A.$

$K\mathrm{a}\dot{\mathrm{z}}\mathrm{d}\mathrm{y}\mathrm{z}$ nich szedf ze stafą prędkością $\mathrm{i}$ dotarfszy do mety, natychmiast ruszaf $\mathrm{w}$ drogę

powrotną. Pierwszy raz mineli się $\mathrm{w}$ odległości 12 km od punktu $B$, drugi- po upływie

6 godzin od momentu pierwszego spotkania-w odległości 6 km od punktu $A$. Obliczyč

odległośč punktów $A\mathrm{i}B\mathrm{i}$ prędkości, $\mathrm{z}$ jakimi poruszali się turyści.





PRACA KONTROLNA nr 6- POZIOM ROZSZERZONY

l. Rozwiązač równanie

$\sqrt{x^{2}-3}+\sqrt{5-2x}=4-x.$

2. $\mathrm{Z}$ urny zawierającej 2 ku1e białe, 4 czerwone $\mathrm{i}3$ czarne wylosowanojedną kulę. Następnie

wylosowano jeszcze trzy kule, gdy pierwsza okazała się biala, dwie kule, gdy pierwsza

była czerwona, lub jedną kulę, gdy $\mathrm{w}$ pierwszym losowaniu wypadła czarna. Obliczyč

prawdopodobieństwo, $\dot{\mathrm{z}}\mathrm{e}\mathrm{w}$ urnie nie pozostafa $\dot{\mathrm{z}}$ adna kula biafa.

3. Podstawą graniastosłupa prostego jest trójkąt $0$ bokach $a, b\mathrm{i}$ kącie między nimi $\alpha, \mathrm{a}$

przekątne ścian bocznych, wychodzące $\mathrm{z}$ wierzchołka kąta $\alpha, \mathrm{s}\Phi$ do siebie prostopadfe.

Obliczyč objętośč graniastoslupa.

4. Na jednym rysunku sporzadzič staranne wykresy funkcji

$f(x)=\sqrt{6x-x^{2}}$

oraz

$g(x)=|\displaystyle \frac{3}{2}-f(x+2)|.$

Obliczyč pole figury ograniczonej wykresem funkcji $g(x)\mathrm{i}\mathrm{o}\mathrm{s}\mathrm{i}_{\Phi}Ox.$

5. Podač dziedzinę $\mathrm{i}$ sprawdzič $\mathrm{t}\mathrm{o}\dot{\mathrm{z}}$ samośč

tg2 -$\alpha$2 $=$ -11 $+$-ccooss $\alpha\alpha$.

Cosinus kąta ostrego $\alpha$ wynosi $\displaystyle \frac{1}{8}$. Korzystając $\mathrm{z}$ powyzszej $\mathrm{t}\mathrm{o}\dot{\mathrm{z}}$ samości, obliczyč wartośč

sumy tg $\displaystyle \frac{\alpha}{4}+\mathrm{t}\mathrm{g}\frac{\alpha}{2}+\mathrm{t}\mathrm{g}\frac{3\alpha}{4}+\mathrm{t}\mathrm{g}\alpha$. Wynik podač $\mathrm{w}$ najprostszej postaci.

6. Punkt $C(-2,-1)$ jest wierzchofkiem trójkąta równoramiennego $ABC, \mathrm{w}$ którym $|AC|=$

$|BC|$. Środkowe trójkąta przecinają się $\mathrm{w}$ punkcie $M(1,2)$, a dwusieczne $\mathrm{w}$ punkcie

$S(\displaystyle \frac{1}{2},\frac{3}{2})$. Wyznaczyč wspófrzędne wierzcholków A $\mathrm{i}B.$





XXXIX

KORESPONDENCYJNY KURS

Z MATEMATYKI

listopad 2009 r.

PRACA KONTROLNA $\mathrm{n}\mathrm{r} 2-$ POZIOM PODSTAWOWY

l. Suma $n$ początkowych wyrazów ciagu $(a_{n})$ określona jest wzorem $S_{n} =2n^{2}+5n+c.$

Wyznaczyč stafą $c\mathrm{t}\mathrm{a}\mathrm{k}$, by $(a_{n})\mathrm{b}\mathrm{y}l$ ciągiem arytmetycznym. Obliczyč sumę dwudziestu

jeden pierwszych wyrazów tego ciągu $0$ numerach parzystych.

2. Narysowač zbiory: $A=\{(x,y):(x-1)^{2}\leq y\leq 2-|x-1|\}, B=\{(x,y):|x|+|x-2|\leq 2y\}$

oraz $(A\backslash B)\cup(B\backslash A)$. Ile wynosi pole figury $A\cap B$?

3. Przekrój graniastosłupa prawidlowego czworokątnego płaszczyzną zawierającą przekątną

podstawy ijedną $\mathrm{z}$ krawędzi bocznychjest kwadratem. Obliczyč stosunek pola przekroju

tego graniastosłupa plaszczyzną zawierającą przekątną podstawy dolnej $\mathrm{i}$ przeciwległy

wierzchołek podstawy górnej do pola przekroju płaszczyznq zawierającq przekatną gra-

niastosfupa $\mathrm{i}$ środki przeciwlegfych krawędzi bocznych. Sporz$\Phi$dzič rysunek.

4. Niech $f(x)=$

dla

dla

$x\leq 1,$

$x>1.$

a) Sporządzič wykres funkcji $f\mathrm{i}$ na jego podstawie wyznaczyč zbiór wartości tej funk-

cji.

b) Obliczyč $f(\sqrt{3}-1) \mathrm{i}$ korzystając $\mathrm{z}$ wykresu zaznaczyč na osi $0x$ zbiór rozwiązań

nierówności $f^{2}(x)\leq 4.$

5. Wiadomo, $\dot{\mathrm{z}}\mathrm{e}$ liczby $-1$, 3 są pierwiastkami wielomianu $W(x)=x^{4}-ax^{3}-4x^{2}+bx+3.$

Rozwiązač nierównośč $\sqrt{W(x)}\leq x^{2}-x.$

6. Punkt $A=(1,0)$ jest wierzchofkiem rombu $0$ kącie przy tym wierzcholku równym $60^{\mathrm{o}}$

Wyznaczyč współrzędne pozostałych wierzchołków rombu wiedząc, $\dot{\mathrm{z}}\mathrm{e}$ dwa $\mathrm{z}$ nich lezą

na prostej $l$ : $2x-y+3=0$. Ile rozwiqzań ma to zadanie?





PRACA KONTROLNA nr 2- POZIOM ROZSZERZONY

l. Dane są liczby $m=\displaystyle \frac{\left(\begin{array}{l}
6\\
4
\end{array}\right)\left(\begin{array}{l}
8\\
2
\end{array}\right)}{\left(\begin{array}{l}
7\\
3
\end{array}\right)},$

{\it n}$=$ -($\sqrt{}$($\sqrt{}$24)1-64)(3-41.)2-7-25-$\sqrt{}$4-413.

Wyznaczyč sume wszystkich wyrazów nieskończonego ciqgu geometrycznego, którego

pierwszym wyrazem jest $m$, a piątym $n$. Ile wyrazów tego ciągu nalez $\mathrm{y}$ wziqč, by ich

suma przekroczyla 99\% sumy wszystkich wyrazów?

2. Narysowač zbiory: $A=\{(x,y):x^{2}+2x+y^{2}\leq 3\}, B=\{(x,y):|y|\leq\sqrt{3}x+\sqrt{3}\}$

oraz $(A\backslash B)\cup(B\backslash A)$. Wyznaczyč równanie okręgu wpisanego $\mathrm{w}$ figurę $A\cap B.$

3. Liczby: $a_{1}=\log_{(3-2\sqrt{2})^{2}}(\sqrt{2}-1), \displaystyle \alpha_{2}=\frac{1}{2}\log_{\frac{1}{3}}\frac{\sqrt{3}}{6}, a_{3}=3^{\log_{\sqrt{3}^{\frac{\sqrt{6}}{2}}}}, a_{4}=\log_{(\sqrt{2}-1)}(\sqrt{2}+1),$

$a_{5}=(2^{\sqrt{2}+1})^{\sqrt{2}-1}, a_{6}=\log_{3}2$ są wszystkimi pierwiastkami wielomianu $W(x)$, którego

wyraz wolny jest dodatni.

a) Które $\mathrm{z}$ tych pierwiastków są niewymierne? Odpowiedz/uzasadnič.

b) Wyznaczyč dziedzinę funkcji

nych.

$f(x) = \sqrt{W(x)},$

nie wykonując obliczeń przyblizo-

4. Narysowač wykres funkcji $f$ zadanej wzorem $f(x)=$

Posfugując się wykresem $\mathrm{i}$ odpowiednimi obliczeniami rozwiązač nierównośč

$|f(x)-\displaystyle \frac{1}{2}|<\frac{1}{4}$

5. Na prostej $x+2y=5$ wyznaczyč punkty, $\mathrm{z}$ których okrqg $(x-1)^{2}+(y-1)^{2}=1$ jest

widoczny pod kątem $60^{\mathrm{o}}$. Obliczyč pole obszaru ograniczonego lukiem okręgu $\mathrm{i}$ stycznymi

do niego poprowadzonymi $\mathrm{w}$ znalezionych punktach. Sporządzič rysunek.

6. Na dnie naczynia $\mathrm{w}$ ksztalcie walca umieszczono cztery jednakowe metalowe kulki $0$

$\mathrm{m}\mathrm{o}\dot{\mathrm{z}}$ liwie największej objętości. Następnie do naczynia wrzucono jeszcze $\mathrm{j}\mathrm{e}\mathrm{d}\mathrm{n}\Phi$ kulkę $\mathrm{i}$

okazało się, $\dot{\mathrm{z}}\mathrm{e}$ jest ona styczna do płaskiej pokrywy naczynia. Wyznaczyč promienie

kulek wiedząc, $\dot{\mathrm{z}}\mathrm{e}$ przekrój osiowy walca jest kwadratem $0$ boku $d.$





XXXIX

KORESPONDENCYJNY KURS

Z MATEMATYKI

grudzień 2009 r.

PRACA KONTROLNA nr 3 -POZIOM PODSTAWOWY

l. Sześč kostek sześciennych $0$ objętościach 256, 128, 64, 32, 16 $\mathrm{i}8\mathrm{c}\mathrm{m}^{3}$ ustawiono $\mathrm{w}$ pirami-

dę. Czy $\mathrm{m}\mathrm{o}\dot{\mathrm{z}}$ na tę piramidę umieścič na pófce $0$ wysokości 24 cm? Odpowied $\acute{\mathrm{z}}$ uzasadnič

bez wykonywania obliczeń przyblizonych.

2. Wojtuś postawif przypadkowo cztery pionki na szachownicy $016$ polach. Jakiejest praw-

dopodobieństwo, $\dot{\mathrm{z}}\mathrm{e}$ co najwyzej dwa pionki będą staly $\mathrm{w}$ szeregu (poziomo lub pionowo)?

3. Rozwiązač nierównośč

$|\displaystyle \frac{x^{2}+3x+2}{2x^{2}+7x+6}|\leq 1.$

4. Lamana ABCD jest przedstawiona na rysunku ponizej. Niech E będzie punktem prze-

cięcia się prostych AB iCD. Obliczyč pole trójk$\Phi$ta CBE.

5. Obserwator, stojąc $\mathrm{w}$ pewnej odlegfości, widzi wiezę kościofa pod kątem $60^{\mathrm{o}}$ Po odda-

leniu się $050\mathrm{m}$ kąt widzenia zmniejszył się do $45^{\mathrm{o}}$ Obliczyč cosinus kąta, pod jakim

obserwator będzie widział wiezę kościofa, jeśli oddali się $0$ kolejne 50 $\mathrm{m}.$

6. Wycinek koła ma obwód $2s$, gdzie $s > 0$ jest ustaloną liczbą. Wyrazič pole $P$ tego

wycinka jako funkcję promienia $r$ kofa. Sporządzič wykres funkcji $P=P(r).$





PRACA KONTROLNA nr 3 -POZIOM ROZSZERZONY

l. Sporządzič wykres funkcji $f(m) = \displaystyle \frac{1}{x_{1}}+\frac{1}{x_{2}}$, gdzie $x_{1}, x_{2}$ sa pierwiastkami równania

$x^{2}-2mx+m+2=0$, a $m$ jest parametrem rzeczywistym.

2. Ala ulozyła $\mathrm{z}$ czterech klocków liczbę 2009. Nastepnie spośród tych k1ocków 1osowa-

fa ze zwracaniem cztery razy po jednym klocku. Jakie jest prawdopodobieństwo, $\dot{\mathrm{z}}\mathrm{e}\mathrm{z}$

otrzymanych $\mathrm{w}$ ten sposób cyfr $\mathrm{m}\mathrm{o}\dot{\mathrm{z}}$ na byłoby utworzyč liczbe:

a) podzielną przez 3?

b) podzielną przez 4?

3. Rozwazmy funkcje $ f(x)=4^{x+1}+4^{2x+1}+4^{3x+1}+\ldots$ oraz $g(x)=2^{x}+2^{x-1}+2^{x-2}+\ldots,$

gdzie prawe strony wzorów określających obie funkcje są sumami wyrazów nieskończo-

nych ciągów geometrycznych. Wykazač, $\dot{\mathrm{z}}\mathrm{e}$ funkcja $f(x)$ jest rosnąca. Znalez/č wszystkie

liczby $x$, dla których $f(x)=g(x).$

4. Rozwiązač nierównośč

$\displaystyle \frac{\mathrm{t}\mathrm{g}x+\sin x}{3\mathrm{t}\mathrm{g}x-2\sin x}\geq\cos^{2}\frac{x}{2}.$

5. Okrag styczny do ramion paraboli $y = x^{2}-2x$ jest styczny równocześnie do osi $Ox.$

Znalez/č równania stycznych do okręgu $\mathrm{w}$ punktach jego styczności $\mathrm{z}$ parabolą.

6. $\mathrm{Z}$ odcinków $0$ długościach równych czterem najmniejszym nieparzystym liczbom pierw-

szym zbudowano trapez, którego pole jest liczbą wymierną. Wyznaczyč tangens $\mathrm{k}_{\Phi^{\mathrm{t}\mathrm{a}}}$

między przekątnymi tego trapezu.





XXXIX

KORESPONDENCYJNY KURS

Z MATEMATYKI

styczeń 2010 r.

PRACA KONTROLNA $\mathrm{n}\mathrm{r} 4-$ POZIOM PODSTAWOWY

l. Mamy dwa termosy kawy $\mathrm{z}$ mlekiem. $\mathrm{W}$ pierwszym termosie stosunek objętości mleka

do objętości kawy wynosi 2:3, a $\mathrm{w}$ drugim 3:7. I1e 1itrów p1ynu na1ez $\mathrm{y}$ wziąč $\mathrm{z}\mathrm{k}\mathrm{a}\dot{\mathrm{z}}$ dego

termosu, aby otrzymač 2,41itra kawy $\mathrm{z}$ mlekiem, $\mathrm{w}$ której objętośč kawy bedzie dwa

razy większa $\mathrm{n}\mathrm{i}\dot{\mathrm{z}}$ objętośč mleka?

2. Kwotę l00000 zf wpfacono na lokatę roczną, $\mathrm{w}$ której odsetki doliczane są co kwartaf. Po

roku suma odsetek wyniosła dokładnie 4060,401 $\mathrm{z}l$. Znalez/č oprocentowanie tej lokaty.

Jakie powinno byč oprocentowanie lokaty, aby przy kapitalizacji dokonywanej raz na póf

roku osiągnąč ten sam zysk?

3. Dane są zbiory $A=\{(x,y):x,y,\in \mathbb{R},y^{2}-4x^{2}\geq 0\}\mathrm{i}B=\{(x,y):|x|+|y|\leq 2\}$. Nary-

sowač zbiór $A\cup B$. Znalez$\acute{}$č punkt ze zbioru $A\cup B$ pofozony najblizej puntu $C=(3,2).$

4. Narysowač wykres trójmianu kwadratowego $f(x) = x^{2}+4x-5$ oraz wykres funkcji

$g(x)=4-f(x-2).$

a) Rozwiązač nierównośč $f(x)>g(x).$

b) Znalez/č obraz wykresu funkcji $f(x) \mathrm{w}$ symetrii względem prostej $x=2 \mathrm{i}$ na tej

podstawie podač wzór tej funkcji.

5. $\mathrm{W}$ ostrosfupie prawidłowym trójk$\Phi$tnym $0$ krawędzi podstawy równej $a$ kąt pfaski ściany

bocznej przy wierzchołku jest równy $ 2\alpha$. Obliczyč objętośč tego ostrosłupa oraz sinus

kąta nachylenia ściany bocznej do podstawy.

6. $\mathrm{W}$ trapezie ABCD, $\mathrm{w}$ którym bok $AB$ jest równolegfy do boku $DC$, dane są: $\angle BAD=$

$\displaystyle \frac{\pi}{3}, |AB| =20, |DC| =8$ oraz $|AD| =5$. Obliczyč obwód tego trapezu, $\sin\angle ADB$ oraz

odlegfośč punktu przecięcia się przekątnych tego trapezu od jego podstaw.





PRACA KONTROLNA nr 4- POZIOM ROZSZERZONY

l. Dzieląc wielomian $W(x)$ przez dwumian $x-3$ otrzymujemy resztę równą 2, a dzie1ąc

ten wielomian przez $x-2$ otrzymujemy resztę równą l. Wyznaczyč resztę $\mathrm{z}$ dzielenia

$W(x)$ przez $(x-2)(x-3)$. Znalez/č wielomian trzeciego stopnia spełniający powyzsze

warunki wiedzqc, $\dot{\mathrm{z}}\mathrm{e}x=1$ jest pierwiastkiem tego wielomianu, a suma wyrazu wolnego

$\mathrm{i}$ wspólczynnika przy $x^{3}$ jest równa 0.

2. Znalez/č najmniejszą $\mathrm{i}$ największą wartośč funkcji $f(x)=\displaystyle \sin x-\frac{1}{2}\cos 2x$ na przedziale

$[-\displaystyle \frac{\pi}{2},\frac{\pi}{2}] \mathrm{i}$ rozwiązač nierównośč - $\displaystyle \frac{1}{2}\leq f(x)\leq\frac{1}{4}$. Zadanie rozwiązač bez $\mathrm{u}\dot{\mathrm{z}}$ ywania pojęcia

pochodnej.

3. Rozwiązač nierównośč

$\log_{\frac{1}{\sqrt{2}}}(2^{2x+1}-16^{x})\geq-12x.$

4. $\mathrm{W}$ stozek $\mathrm{o}\mathrm{k}_{\Phi}\mathrm{c}\mathrm{i}\mathrm{e}$ rozwarcia równym $ 2\alpha$ wpisano kulę $0$ promieniu $R$. Wewnątrz stozka

stawiamy na kuli sześcian $0$ maksymalnej objętości $\mathrm{i}$ podstawie równoleglej do podstawy

stozka. Wyznaczyč dlugośč krawędzi tego sześcianu.

5. Stosunek dlugości promienia okręgu wpisanego do dfugości promienia okręgu opisanego

na trójkącie prostokqtnym wynosi $\displaystyle \frac{1}{3+2\sqrt{3}}$. Obliczyč sinusy kątów ostrych tego trójkąta.

6. Ślimak ma do przejścia taśmę $0$ długości 3 metrów zamocowaną $\mathrm{w}$ punkcie startu A. $\mathrm{W}$

ciągu $\mathrm{k}\mathrm{a}\dot{\mathrm{z}}$ dego dnia udaje mu się przejśč l metr, a $\mathrm{k}\mathrm{a}\dot{\mathrm{z}}$ dej nocy gdy śpi, ktoś- ciągnąc

za drugi koniec taśmy- wydłuza $\mathrm{j}\mathrm{a}$ równomiernie $0 1$ metr. Niech $d_{n}$ oznacza długośč

taśmy $\mathrm{w}n$-tym dniu, a $a_{n}$- odległośč ślimaka od punktu A przy końcu $n$-tego dnia.

a) Uzasadnič, $\dot{\mathrm{z}}\mathrm{e}$ ciąg $(a_{n})$ zdefiniowany jest następującym wzorem rekurencyjnym:

$a_{1}=1$ oraz $a_{n+1}=\displaystyle \frac{3+n}{2+n}a_{n}+1$ dla $n\geq 1.$

b) Pokazač, $\displaystyle \dot{\mathrm{z}}\mathrm{e}a_{n}=(n+2)(\frac{1}{3}+\frac{1}{4}+\ldots+\frac{1}{n+2}), n\geq 1.$

c) Czy ślimak dojdzie do końca taśmy? $\mathrm{J}\mathrm{e}\dot{\mathrm{z}}$ eli tak, to $\mathrm{w}$ którym dniu, to znaczy, dla

jakich $n$ prawdziwa jest nierównośč $a_{n}>d_{n}$?





XXXIX

KORESPONDENCYJNY KURS

Z MATEMATYKI

luty 2010 r.

PRACA KONTROLNA $\mathrm{n}\mathrm{r} 5-$ POZIOM PODSTAWOWY

l. Dwie wiewiórki, Kasiai Basia, postanowiły wspólnie zbierač orzechy. $\mathrm{K}\mathrm{a}\dot{\mathrm{z}}$ dego dnia Basia

przynosifa do wspólnej spizarni $04$ orzechy więcej $\mathrm{n}\mathrm{i}\dot{\mathrm{z}}$ Kasia, codziennie tyle samo. Po

30 dniach współpracy wiewiórki pokłóciły się. Basia zostawiła Kasi wszystkie orzechy

$\mathrm{i}$ zafozyła wlasną spizarnię. Od tamtej pory $\mathrm{k}\mathrm{a}\dot{\mathrm{z}}$ da $\mathrm{z}$ wiewiórek przynosi do swojej spizarni

tę samą ilośč orzechów co przedtem, ale Basia codziennie dostaje 6 orzechów od Kasi. Po

50 dniach samodzielnej pracy Kasia ma jeszcze $0100$ orzechów więcej $\mathrm{n}\mathrm{i}\dot{\mathrm{z}}$ Basia. Ustalič,

po ile orzechów zbiera codziennie $\mathrm{k}\mathrm{a}\dot{\mathrm{z}}$ da $\mathrm{z}$ wiewiórek $\mathrm{i}$ oszacowač, po ilu dniach $\mathrm{w}$ spizarni

Basi będzie więcej orzechów $\mathrm{n}\mathrm{i}\dot{\mathrm{z}}\mathrm{u}$ kolezanki.

2. Określič dziedzinę $\mathrm{i}$ zbiór wartości funkcji $f(x)=\sin x\cdot\sin 2x$. (tg $x+$ ctg $x$). Wykonač

staranny wykres funkcji $g(x)=f(x-\displaystyle \frac{\pi}{4})+1\mathrm{i}$ rozwiązač równanie $g(x)=0$. Posfugując

się sporzqdzonym wykresem określič zbiór rozwiązań nierówności $g(x)\geq 0.$

3. Wyznaczyč równania wszystkich prostych, które są styczne jednocześnie do obu okręgów

$(x-1)^{2}+(y-1)^{2}=1$

oraz

$(x-5)^{2}+(y-1)^{2}=1.$

Obliczenia zilustrowač odpowiednim rysunkiem.

4. Rozwiązač nierównośč

$\displaystyle \frac{3\sqrt{4-x}+1}{1-\sqrt{4-x}}>1-2\sqrt{4-x}.$

5. Koszt budowy I kondygnacji biurowca wynosi l0 mln zł., a $\mathrm{k}\mathrm{a}\dot{\mathrm{z}}$ dej kolejnej jest $\mathrm{n}\mathrm{i}\dot{\mathrm{z}}$ szy

$0100\mathrm{t}\mathrm{y}\mathrm{s}. \mathrm{z}\mathrm{f}$. od poprzedniej. Planowany koszt wynajmu powierzchni biurowych $\mathrm{w}$ tym

budynku jest stały do XL kondygnacji $\mathrm{i}$ wynosi 200 $\mathrm{t}\mathrm{y}\mathrm{s}$. zł. za całą kondygnację, $\mathrm{a}$

potem podwaja się co 5 kondygnacji (na ko1ejnych 5 kondygnacjach jest stały). Roczny

koszt wynajmu ostatniej, najbardziej prestizowej $\mathrm{i}$ drozszej od pozostafych kondygnacji

jest równy kosztowi budowy całego XXXVII piętra. Oszacowač, po ilu latach zwróci się

inwestorom koszt budowy tego budynku.

6. $\mathrm{W}$ trapezie równoramiennym kąt przy podstawie ma miarę $\displaystyle \frac{\pi}{3}$, a róznica długości podstaw

wynosi 4. Usta1ič, i1e powinno wynosič po1e tego trapezu, aby $\mathrm{m}\mathrm{o}\dot{\mathrm{z}}$ na było wpisač $\mathrm{w}$ niego

kofo. $\mathrm{W}$ tym przypadku wyznaczyč stosunek pola kofa opisanego na tym trapezie do pola

koła wpisanego.





PRACA KONTROLNA nr 5- POZIOM ROZSZERZONY

l. Znalez$\acute{}$č wszystkie liczby rzeczywiste m, dla których równanie

$\displaystyle \frac{x}{m}+m=\frac{m}{x}+x+1$

ma dwa pierwiastki róznych znaków.

2. Rozwiązač nierównośč

$2^{x^{2}+4}+2^{x^{2}+3}+2^{x^{2}}>5^{x^{2}+1}-25\cdot 2^{x^{2}-2}$

3. Określič dziedzinę $\mathrm{i}$ zbiór wartości funkcji $f(x)=\displaystyle \mathrm{c}\mathrm{t}\mathrm{g}(\pi+x)\mathrm{c}\mathrm{t}\mathrm{g}(x-\frac{\pi}{2})\cos x$. Sporz$\Phi$dzič

staranny wykres funkcji $g(x) =2f(|x-\displaystyle \frac{\pi}{4}|)-1$. Na podstawie wykresu $\mathrm{i}$ niezbędnych

obliczeń rozwiązač nierównośč $g(x)\leq-2$, a zbiór jej rozwiązań zaznaczyč na osi OX.

4. Rozwiązač nierównośč

$\displaystyle \log_{x^{2}}(3x-1)-\log_{x^{2}}(x-1)^{2}+\log_{x^{2}}|x-1|\geq\frac{1}{2}.$

5. $\mathrm{W}$ ostroslupie sześciokątnym prawidfowym kąt dwuścienny utworzony przez pfaszczyzny

przeciwległych ścian bocznych ma miarę $\displaystyle \frac{\pi}{4}$. Wyznaczyč promień $R$ kuli opisanej na tym

ostroslupie jako funkcję dfugości $a$ boku jego podstawy.

6. $\mathrm{W}$ kolo wpisano ośmiokąt foremny, $\mathrm{w}$ ośmiokąt kofo, $\mathrm{w}$ kofo kolejny ośmiokąt foremny

itd. Wysunač hipotezę $0$ wartości pola $n$-tego koła $\mathrm{i}$ uzasadnič $\mathrm{j}\mathrm{a}$ indukcyjnie. Suma pól

nieskończonego $\mathrm{c}\mathrm{i}_{\Phi \mathrm{g}}\mathrm{u}$ kól otrzymanych $\mathrm{w}$ ten sposób jest ośmiokrotnością pola jednego

$\mathrm{z}$ nich. Ustalič którego, nie stosując obliczeń przyblizonych.







XL

KORESPONDENCYJNY KURS

Z MATEMATYKI

wrzesień 2010 r.

PRACA KONTROLNA $\mathrm{n}\mathrm{r} 1-$ POZIOM PODSTAWOWY

l. Ile jest trzycyfrowych liczb naturalnych:

a) podzielnych przez 31ub przez 5?

b) podzielnych przez 31ub przez 6?

c) podzielnych przez 3 $\mathrm{i}$ niepodzielnych przez 5?

2. Renomowany dom mody sprzedaf 40\% kolekcji letniej po zafozonej cenie. Po obnizce

ceny $0$ 50\% udało $\mathrm{s}\mathrm{i}\mathrm{e}$ sprzedač połowę pozostałej części towaru $\mathrm{i}$ dopiero kolejna 50\%-

owa obnizka pozwolifa opróznič magazyny. Ile procent zaplanowanego przychodu stanowi

uzyskana ze sprzedaz $\mathrm{y}$ kwota? $\mathrm{O}$ ile procent wyjściowa cena towaru powinna była byč

$\mathrm{w}\mathrm{y}\dot{\mathrm{z}}$ sza, by sklep uzyskał zaplanowany początkowo przychód?

3. Określič dziedzinę wyrazenia $w(x,y)=\displaystyle \frac{2}{x-y}-\frac{3xy}{x^{3}-y^{3}}-\frac{x-y}{x^{2}+xy+y^{2}}.$

Sprowadzič je do najprostszej postaci $\mathrm{i}$ obliczyč $w(1+\sqrt{2},(1+\sqrt{2})^{-1})$

4. Obliczyč sumę wszystkich liczb pierwszych spelniających nierównośč

$(p-4)x^{2}-4(p-2)x-p\leq 0$, gdzie $p=\displaystyle \frac{64^{\frac{1}{3}}\sqrt{8}+8^{\frac{1}{3}}\sqrt{64}}{\sqrt[3]{64\sqrt{8}}}$

5. Dwa naczynia zawierają $\mathrm{w}$ sumie 401itrów wody. Po prze1aniu pewnej części wody pierw-

szego naczynia do drugiego, $\mathrm{w}$ pierwszym naczyniu zostało trzy razy mniej wody $\mathrm{n}\mathrm{i}\dot{\mathrm{z}}\mathrm{w}$

drugim. Gdy następnie przelano taką samą częśč wody drugiego naczynia do pierwszego,

okazało się, $\dot{\mathrm{z}}\mathrm{e}\mathrm{w}$ obu naczyniach jest tyle samo płynu. Obliczyč, ile wody było pierwotnie

$\mathrm{w}\mathrm{k}\mathrm{a}\dot{\mathrm{z}}$ dym naczyniu $\mathrm{i}\mathrm{j}\mathrm{a}\mathrm{k}_{\Phi}$ jej częśč przelewano.

6. Dwie $\mathrm{g}\mathrm{a}\acute{\mathrm{z}}$dziny, pracując razem, mogą wykonač zamówioną partię pisanek $\mathrm{w}$ ciągu 7

dni pod warunkiem, $\dot{\mathrm{z}}\mathrm{e}$ pierwsza $\mathrm{z}$ nich rozpocznie pracę $0$ póltora dnia wcześniej $\mathrm{n}\mathrm{i}\dot{\mathrm{z}}$

druga. Gdyby $\mathrm{k}\mathrm{a}\dot{\mathrm{z}}\mathrm{d}\mathrm{a}\mathrm{z}$ nich pracowała oddzielnie, to druga wykonalaby cafą pracę $03$

$\mathrm{d}\mathrm{n}\mathrm{i}$ wcześniej od pierwszej. Ile $\mathrm{d}\mathrm{n}\mathrm{i}$ potrzebuje $\mathrm{k}\mathrm{a}\dot{\mathrm{z}}$ da $\mathrm{z}$ kobiet na wykonanie calej pracy?




PRACA KONTROLNA nr l- POZIOM ROZSZERZONY

l. Ile jest liczb pięciocyfrowych podzielnych przez 9, które $\mathrm{w}$ rozwinięciu dziesiętnym mają:

a) obie cyfry 1, 2 $\mathrm{i}$ tylko $\mathrm{t}\mathrm{e}$? b) obie cyfry 1, 3 $\mathrm{i}$ tylko $\mathrm{t}\mathrm{e}$? c) wszystkie cyfry 1, 2, 3

$\mathrm{i}$ tylko $\mathrm{t}\mathrm{e}$? Odpowiedz/uzasadnič. $\mathrm{W}$ przypadku b) wypisač otrzymane liczby.

2. Pan Kowalski zaciqgnąl 3l grudnia $\mathrm{p}\mathrm{o}\dot{\mathrm{z}}$ yczkę 4000 złotych oprocentowaną $\mathrm{w}$ wysoko-

ści 18\% $\mathrm{w}$ skali roku. Zobowiązał się splacič $\mathrm{j}_{\Phi}\mathrm{w}\mathrm{c}\mathrm{i}_{\Phi \mathrm{g}}\mathrm{u}$ roku $\mathrm{w}$ trzech równych ratach

platnych 30 kwietnia, 30 sierpnia $\mathrm{i}30$ grudnia. Oprocentowanie $\mathrm{p}\mathrm{o}\dot{\mathrm{z}}$ yczki liczy się od

l stycznia, a odsetki od kredytu naliczane są $\mathrm{w}$ terminach płatności rat. Obliczyč wyso-

kośč tych rat $\mathrm{w}$ zaokrągleniu do pełnych groszy.

3. Określič dziedzinę wyrazenia

$w(x,y)=\displaystyle \frac{x}{x^{3}+x^{2}y+xy^{2}+y^{3}}+\frac{y}{x^{3}-x^{2}y+xy^{2}-y^{3}}+\frac{1}{x^{2}-y^{2}}-\frac{1}{x^{2}+y^{2}}-\frac{x^{2}+2y^{2}}{x^{4}-y^{4}}.$

Sprowadzič je do najprostszej postaci $\mathrm{i}$ obliczyč $w(\cos 15^{\mathrm{o}},\sin 15^{\mathrm{o}}).$

4. Liczba $p = \displaystyle \frac{(\sqrt[3]{54}-2)(9\sqrt[3]{4}+6\sqrt[3]{2}+4)-(2-\sqrt{3})^{3}}{\sqrt{3}+(1+\sqrt{3})^{2}}$ jest miejscem zerowym funkcji

kwadratowej $f(x)=ax^{2}+bx+c$. Wyznaczyč współczynniki $a, b, c$ oraz drugie miejsce

zerowe tej funkcji wiedząc, $\dot{\mathrm{z}}\mathrm{e}$ największ$\Phi$ wartością funkcji jest 4, a jej wykres jest

symetryczny wzgledem prostej $x=1.$

5. Do zbiornika poprowadzono trzy rury. Pierwsza rura potrzebuje do napefnienia zbiornika

$04$ godziny więcej $\mathrm{n}\mathrm{i}\dot{\mathrm{z}}$ druga, a trzecia napełnia caly zbiornik $\mathrm{w}$ czasie dwa razy krótszym

$\mathrm{n}\mathrm{i}\dot{\mathrm{z}}$ pierwsza. Wjakim czasie napelnia zbiornik $\mathrm{k}\mathrm{a}\dot{\mathrm{z}}$ da $\mathrm{z}\mathrm{r}\mathrm{u}\mathrm{r}, \mathrm{j}\mathrm{e}\dot{\mathrm{z}}$ eli wiadomo, $\dot{\mathrm{z}}\mathrm{e}$ wszystkie

trzy rury otwarte jednocześnie napefniają zbiornik $\mathrm{w}$ ciągu 2 godzin $\mathrm{i}40$ minut?

6. $\mathrm{Z}$ przystani A wyrusza $\mathrm{z}$ biegiem rzeki statek do przystani $\mathrm{B}$, odległej od A $0140$ km. Po

upływie l godziny wyrusza za nim łódz/ motorowa, dopędza statek, po czym wraca do

przystani A $\mathrm{w}$ tym samym momencie, $\mathrm{w}$ którym statek przybija do przystani B. Znalez/č

prędkośč biegu rzeki, $\mathrm{j}\mathrm{e}\dot{\mathrm{z}}$ eli wiadomo, $\dot{\mathrm{z}}\mathrm{e}\mathrm{w}$ stojącej wodzie prędkośč statku wynosi 16

$\mathrm{k}\mathrm{m}/$godz, a prędkośč łodzi 24 $\mathrm{k}\mathrm{m}/$godz.





XL

KORESPONDENCYJNY KURS

Z MATEMATYKI

luty 2011 r.

PRACA KONTROLNA $\mathrm{n}\mathrm{r} 6-$ POZIOM PODSTAWOWY

l. Losujemy liczbę ze zbioru \{1, 2, 3, $\ldots$, 100\}, a następnie liczbę ze zbioru \{2, 3, 4, 5\}. Obli-

czyč prawdopodobieństwo, $\dot{\mathrm{z}}\mathrm{e}$ pierwsza $\mathrm{z}$ wylosowanych liczb jest podzielna przez $\mathrm{d}\mathrm{r}\mathrm{u}\mathrm{g}\Phi.$

2. Liczba 2-elementowych podzbiorów zbioru $A$ jest 7 razy większa $\mathrm{n}\mathrm{i}\dot{\mathrm{z}}$ liczba 2-e1emento-

wych podzbiorów zbioru $B$. Liczba 2-e1ementowych podzbiorów zbioru $A$ nie zawierają-

cych ustalonego elementu $a\in A$ jest 5 razy większa $\mathrm{n}\mathrm{i}\dot{\mathrm{z}}$ liczba 2-e1ementowych podzbiorów

zbioru $B$. Ile elementów ma $\mathrm{k}\mathrm{a}\dot{\mathrm{z}}\mathrm{d}\mathrm{y}\mathrm{z}$ tych zbiorów? Ile $\mathrm{k}\mathrm{a}\dot{\mathrm{z}}\mathrm{d}\mathrm{y}\mathrm{z}$ tych zbiorów ma podzbio-

rów 3-e1ementowych?

3. $\mathrm{W}$ turnieju szachowym $\mathrm{k}\mathrm{a}\dot{\mathrm{z}}\mathrm{d}\mathrm{y}$ uczestnik miał rozegrač $\mathrm{z}$ pozostałymi po jednej partii. Po

rozegraniu trzech partii dwóch szachistów zrezygnowalo $\mathrm{z}$ dalszej $\mathrm{g}\mathrm{r}\mathrm{y}. \mathrm{W}$ sumie rozegra-

no 84 partie. I1u by1o uczestników na początku turnieju, $\mathrm{j}\mathrm{e}\dot{\mathrm{z}}$ eli dwaj zawodnicy, którzy

zrezygnowali, nie grali ze sobq?

4. Suma pierwszego $\mathrm{i}$ trzeciego wyrazu $\mathrm{c}\mathrm{i}_{\Phi \mathrm{g}}\mathrm{u}$ geometrycznego $(a_{n})$ wynosi 20. Znajd $\acute{\mathrm{z}}$ wzór

ogólny ciągu arytmetycznego $(b_{n})$ takiego, $\dot{\mathrm{z}}\mathrm{e}b_{1}=a_{1}, b_{2}=a_{2}, b_{5}=a_{3}.$

5. Rozkład ocen ze sprawdzianu $\mathrm{w}$ klasie IIIa jest opisany tabelką
\begin{center}
\begin{tabular}{l|l|l|l|l|l}
\multicolumn{1}{l|}{ocena}&	\multicolumn{1}{|l|}{$1$}&	\multicolumn{1}{|l|}{ $2$}&	\multicolumn{1}{|l|}{ $3$}&	\multicolumn{1}{|l|}{ $4$}&	\multicolumn{1}{|l}{ $5$}	\\
\hline
\multicolumn{1}{l|}{liczba osób}&	\multicolumn{1}{|l|}{$1$}&	\multicolumn{1}{|l|}{ $2$}&	\multicolumn{1}{|l|}{ $8$}&	\multicolumn{1}{|l|}{ $9$}&	\multicolumn{1}{|l}{ $6$}
\end{tabular}

\end{center}
Jaś otrzymał ocenę 4. Czy wypadł powyzej średniej $\mathrm{w}$ swojej klasie? $\mathrm{W}$ pozostałych kla-

sach średnie punktów wynosily: 3,875 $\mathrm{w}$ IIIb (24 osoby) $\mathrm{i}4,6\mathrm{w}$ IIIc (25 osób). Czy ocena

otrzymana przez Jasia znajduje się powyzej średniej liczonej łącznie wśród wszystkich

uczniów klas trzecich? Ile co najmniej, a ile co najwyzej, osób miało piqtki $\mathrm{w}$ klasie IIIc

(skala ocen to $1,2,\ldots,5$)?

6. Ile liczb czterocyfrowych $0$ wszystkich cyfrach róznych $\mathrm{m}\mathrm{o}\dot{\mathrm{z}}$ na utworzyč $\mathrm{z}$ cyfr 1,2,3,4,5, $\mathrm{a}$

ile $\mathrm{z}$ cyfr 0,1,2,3,4,5,6? $\mathrm{W}$ obu przypadkach obliczyč, ile $\mathrm{m}\mathrm{o}\dot{\mathrm{z}}$ na utworzyč czterocyfrowych

liczb podzielnych przez 5.





PRACA KONTROLNA nr 6- POZIOM ROZSZERZONY

l. Trzeci skladnik rozwinięcia dwumianu $(\displaystyle \sqrt[3]{x}+\frac{1}{\sqrt{x}})^{n}$ ma wspólczynnik równy 45. Wyzna-

czyč wszystkie składniki tego rozwinięcia, $\mathrm{w}$ których $x$ wystepuje $\mathrm{w}$ potędze $0$ wykfadniku

cafkowitym.

2. $\mathrm{W}$ turnieju szachowym rozgrywanym systemem,,kazdy $\mathrm{z}\mathrm{k}\mathrm{a}\dot{\mathrm{z}}$ dym'' dwóch uczestników

nie ukończyfo turnieju, przy czym jeden $\mathrm{z}$ nich rozegra110 partii, a drugi ty1ko jedną. I1u

było zawodników $\mathrm{i}$ czy wspomniani zawodnicy grali ze sobą, $\mathrm{j}\mathrm{e}\dot{\mathrm{z}}$ eli rozegrano 55 partii?

3. $\mathrm{W}$ pudełku jest 400 ku1 $\mathrm{w}$ tym $n$ czerwonych. Wybieramy losowo dwie kule. Prawdopo-

dobieństwo wylosowania dwóch kul czerwonych jest równe $\displaystyle \frac{1}{760}.$

a) Ile kul czerwonych jest $\mathrm{w}$ tym pudełku?

b) Obliczyč prawdopodobieństwo, $\dot{\mathrm{z}}\mathrm{e}\dot{\mathrm{z}}$ adna $\mathrm{z}$ wylosowanych kul nie jest czerwona.

4. Suma wyrazów nieskończonego ciagu geometrycznego zmniejszy $\mathrm{s}\mathrm{i}\mathrm{e}0$ 25\%, $\mathrm{j}\mathrm{e}\dot{\mathrm{z}}$ eli wy-

kreślimy $\mathrm{z}$ niej skfadniki $0$ numerach parzystych niepodzielnych przez 4. Ob1iczyč sumę

wszystkich wyrazów tego ciqgu wiedząc, $\dot{\mathrm{z}}\mathrm{e}$ jego drugi wyraz wynosi l.

5. Stosując zasadę indukcji matematycznej udowodnič prawdziwośč wzoru

$\left(\begin{array}{l}
2\\
2
\end{array}\right) - \left(\begin{array}{l}
3\\
2
\end{array}\right) + \left(\begin{array}{l}
4\\
2
\end{array}\right) - \left(\begin{array}{l}
5\\
2
\end{array}\right) +\ldots+\left(\begin{array}{l}
2n\\
2
\end{array}\right) =n^{2},$

$n\geq 1.$

6. Wśród wszystkich $\mathrm{b}\mathrm{l}\mathrm{i}\acute{\mathrm{z}}$niąt 64\% stanowią bliz$\acute{}$nięta tej samej plci. Prawdopodobieństwo

urodzenia chłopca wynosi 0,51. Ob1iczyč prawdopodobieństwo, $\dot{\mathrm{z}}\mathrm{e}$ drugie $\mathrm{z}\mathrm{b}\mathrm{l}\mathrm{i}\acute{\mathrm{z}}$niąt jest

$\mathrm{d}\mathrm{z}\mathrm{i}\mathrm{e}\mathrm{w}\mathrm{c}\mathrm{z}\mathrm{y}\mathrm{n}\mathrm{k}_{\Phi}$, pod warunkiem, $\dot{\mathrm{z}}\mathrm{e}$:

a) pierwsze jest dziewczynką,

b) pierwsze jest chlopcem.





XL

KORESPONDENCYJNY KURS

Z MATEMATYKI

marzec 2011 r.

PRACA KONTROLNA nr 7- POZIOM PODSTAWOWY

l. Rozwiązač równanie

$1-|x|=\sqrt{1+x} \mathrm{i}$ podač jego ilustrację graficzną.

2. Wyznaczyč wszystkie punkty $x\mathrm{z}$ przedziafu $[0,2\pi]$, dla których spefnionajest nierównośč

$\sin 2x-\mathrm{t}\mathrm{g}x\leq 0$. Podač ilustrację graficzną nierówności.

3. Określič liczbę rozwiązań ukfadu równań

$\left\{\begin{array}{l}
y\\
y
\end{array}\right.$

$|x-2|+1,$

$ax$

$\mathrm{w}$ zalezności od wartości współczynnika kierunkowego prostej $y=ax$. Znalez$\acute{}$č rozwiąza-

nia $\mathrm{w}$ przypadku, gdy jednym $\mathrm{z}$ nich jest para (4, 3). Sporzadzič staranny rysunek.

4. Danajest prosta $l:x+2y-4=0$. Przez punkt (l, l) poprowadzič prostą $k\mathrm{o}$ dodatnim

współczynniku kierunkowym $\mathrm{t}\mathrm{a}\mathrm{k}$, aby pole trójkqta ograniczonego prostymi $l,  k\mathrm{i}\mathrm{o}\mathrm{s}\mathrm{i}\otimes$

$0x$ byfo dwa razy większe $\mathrm{n}\mathrm{i}\dot{\mathrm{z}}$ pole trójk$\Phi$ta ograniczonego tymi prostymi $\mathrm{i}\mathrm{o}\mathrm{s}\mathrm{i}_{\Phi}0y.$

5. Trójkąt równoboczny $ABC0$ boku długości $a$ zgięto wzdluz wysokości $CD$ pod pew-

nym kątem, otrzymujqc $\mathrm{w}$ ten sposób czworościan ABCD. Obliczyč objętośč $\mathrm{i}$ pole

powierzchni calkowitej tego czworościanu wiedząc, $\dot{\mathrm{z}}\mathrm{e}$ tangens kąta nachylenia ściany

$ABC$ do podstawy czworościanu równy jest $\sqrt{6}.$

6. Punkt $(0,2)$ jest środkiem symetrii wykresu funkcji

$\mathrm{i}b$ wiedząc, $\dot{\mathrm{z}}\mathrm{e} f(a)=0.$

$f(x)=x(|x|-2a)+b$. Wyznaczyč $a$





PRACA KONTROLNA nr 7- POZIOM ROZSZERZONY

l. Rozwiązač równanie

$\sqrt{8+2x-x^{2}}=2x-5.$

Zilustrowač je odpowiednim wykresem.

2. Wyznaczyč wszystkie wartości parametru rzeczywistego $p$, dla których rozwiqzania

ukfadu równań

$\left\{\begin{array}{l}
px+2y=p- 2\\
2x+py=p-1
\end{array}\right.$

są zawarte $\mathrm{w}$ kwadracie $K=\{(x,y):|x|+|y|\leq 1\}.$

3. Bok $AB$ trójkąta równoramiennego $ABC\mathrm{l}\mathrm{e}\dot{\mathrm{z}}\mathrm{y}$ na prostej $l$ : $x-3y-4 = 0$. Punkt

$D(4,0)$ jest spodkiem wysokości tego trójkąta, a $S(2,1)$ środkiem boku $AC$. Wyznaczyč

wspólrzędne wierzcholka $B$. Sporządzič rysunek.

4. Podstawą ostrosfupa $0$ wysokości $h$ jest trójk$\Phi$t $\mathrm{P}^{\mathrm{r}\mathrm{o}\mathrm{s}\mathrm{t}\mathrm{o}\mathrm{k}}\Phi^{\mathrm{t}\mathrm{n}\mathrm{y}\mathrm{o}}$ kącie ostrym $\alpha$. Wszystkie

ściany boczne ostrosłupa są nachylone do podstawy pod kątem $\alpha$, a pole powierzchni

całkowitej jest czterokrotnie większe od pola podstawy. Obliczyč objętośč ostroslupa.

Wynik podač $\mathrm{w}$ najprostszej postaci.

5. Rozwiązač nierównośč

sin2 {\it x}$+$ -csoins42 {\it xx}$+$ -csoins64 {\it xx}$+$ -csoins86 {\it xx}$+$... $\geq$ -83,

$\mathrm{w}$ której lewa strona jest sumą nieskończonego ciągu geometrycznego.

6. Jednym $\mathrm{z}$ pierwiastków wielomianu $w(x)=ax^{3}+bx^{2}+cx+d$ jest liczba $-1$. Znalez$\acute{}$č

pozostafe pierwiastki wiedząc, $\dot{\mathrm{z}}\mathrm{e}w(1)=-2\mathrm{i}$ środkiem symetrii wykresu funkcji $w(x)$

jest punkt $S(\displaystyle \frac{1}{4},\frac{5}{2})$. Nie prowadząc dodatkowego badania, sporządzič wykres funkcji

$w(x)$. Dobrač odpowiednio jednostki na osiach ukfadu.





XL

KORESPONDENCYJNY KURS

Z MATEMATYKI

kwiecień 2011 r.

PRACA KONTROLNA $\mathrm{n}\mathrm{r} 8-$ POZIOM PODSTAWOWY

l. Uprościč wyrazenie

$a(x)= (\displaystyle \frac{x+1}{x-2}-\frac{x^{3}+8}{x^{3}-8}\frac{x^{2}+2x+4}{x^{2}-4})$ : $\displaystyle \frac{1}{x-2}$

$\mathrm{i}$ rozwiązač nierównośč $|\alpha(x)|<1.$

2. Trzech robotników ma wykonač $\mathrm{P}^{\mathrm{e}\mathrm{w}\mathrm{n}}\Phi$ pracę. Wiadomo, $\dot{\mathrm{z}}\mathrm{e}$ pierwszy $\mathrm{i}$ drugi robotnik,

pracując razem, wykonaliby calą pracę $\mathrm{w}$ czasie $n\mathrm{d}\mathrm{n}\mathrm{i}$, drugi $\mathrm{i}$ trzeci-w czasie $m\mathrm{d}\mathrm{n}\mathrm{i}, \mathrm{a}$

pierwszy $\mathrm{i}$ trzeci-w czasie $k\mathrm{d}\mathrm{n}\mathrm{i}$. Ile dni potrzebuje $\mathrm{k}\mathrm{a}\dot{\mathrm{z}}\mathrm{d}\mathrm{y}\mathrm{z}$ robotników na samodzielne

wykonanie cafej pracy?

3. Dla jakich $\alpha\in [0,2\pi$) równanie kwadratowe $\cos\alpha\cdot x^{2}-2x+2\cos\alpha-1=0$ ma dwa

rózne pierwiastki?

4. Wierzcholkami czworokąta są punkty, których współrzędne spelniają układ równań

$\left\{\begin{array}{l}
xy+x-y\\
x^{2}-xy+y^{2}
\end{array}\right.$

1,

1.

Obliczyč pole czworokąta oraz wyznaczyč równanie okręgu na nim opisanego.

5. Pole powierzchni bocznej ostrosłupa prawidlowego czworokątnego jest 2 razy większe $\mathrm{n}\mathrm{i}\dot{\mathrm{z}}$

pole podstawy. Wyznaczyč cosinusy kątów dwuściennych przy krawędzi podstawy oraz

krawędzi bocznej. Sporządzič staranny rysunek.

6. Dany jest stozek ściety, $\mathrm{w}$ którym pole dolnej podstawy jest 4 razy wieksze od po1a górnej.

$\mathrm{W}$ stozek wpisano walec $\mathrm{t}\mathrm{a}\mathrm{k}, \dot{\mathrm{z}}\mathrm{e}$ dolna podstawa walca $\mathrm{l}\mathrm{e}\dot{\mathrm{z}}\mathrm{y}$ na dolnej podstawie stozka,

a brzeg górnej podstawy walca $\mathrm{l}\mathrm{e}\dot{\mathrm{z}}\mathrm{y}$ na powierzchni bocznej stozka. Jaką częśč objetości

stozka ściętego stanowi objętośč walca, $\mathrm{j}\mathrm{e}\dot{\mathrm{z}}$ eli wysokośč walca jest 3 razy mniejsza od

wysokości stozka? Odpowied $\acute{\mathrm{z}}$ podač $\mathrm{w}$ procentach $\mathrm{z}$ dokladnością do jednego promila.

Sporządzič staranny rysunek przekroju osiowego bryły.





PRACA KONTROLNA nr 8- POZIOM ROZSZERZONY

l. Rozwiązač nierównośč

2. Rozwiązač ukfad równań

$\displaystyle \frac{1}{x^{2}-2x-3}\geq\frac{1}{|x-2|+3}.$

$\left\{\begin{array}{l}
x^{2}+y^{2}=8,\\
- x1+-y1=1.
\end{array}\right.$

Obliczyč pole wielokąta $0$ wierzchofkach, których wspólrzędne spefniają powyzszy ukfad.

Podač ilustrację graficzną tego układu.

3. Wyznaczyč wszystkie wartości parametru $\alpha\in[-\pi,\pi$), dla których równanie kwadratowe

$(\sin 4\alpha)x^{2}-2(\cos\alpha)x+\sin 2\alpha=0$

ma dwa rózne nieujemne pierwiastki rzeczywiste. Rozwiązania zaznaczyč na kole trygo-

nometrycznym.

4. Udowodnič, $\dot{\mathrm{z}}\mathrm{e}\mathrm{j}\mathrm{e}\dot{\mathrm{z}}$ eli liczby rzeczywiste $a, b, c$ spełniajq warunki $a^{2}+b^{2}= (a+b-c)^{2}$

oraz $b, c\neq 0$, to

--{\it ab}22$++$(({\it ba}--{\it cc}))22$=$--{\it ab}--{\it cc}.

5. Trójkąt równoboczny $ABC0$ boku $a$ wpisano $\mathrm{w}$ okrąg. Na fuku $BC$ wybrano punkt

$D\mathrm{t}\mathrm{a}\mathrm{k}, \dot{\mathrm{z}}\mathrm{e}$ proste AB $\mathrm{i}CD$ przecinają się $\mathrm{w}$ punkcie $E\mathrm{i} |BE| = 2a$. Obliczyč pole $S$

czworokąta ABCD $\mathrm{i}$ wykazač, $\displaystyle \dot{\mathrm{z}}\mathrm{e}S=\frac{1}{4}(|BD|+|CD|)^{2}\sqrt{3}.$

6. Rozwinięcie, powierzchni, bocznej, stozka, ściętego, opisanego na kuli jest przedstawione

na rysunku. Obliczyč objętośč tego
\begin{center}
\includegraphics[width=61.416mm,height=51.456mm]{./KursMatematyki_PolitechnikaWroclawska_2010_2011_page15_images/image001.eps}
\end{center}
$\alpha$

{\it b}

stozka sciętego $\mathrm{i}$ promien kuli opisa-

nej na nim. Podac wynik liczbowy dla

$\displaystyle \alpha=\frac{\pi}{4}, b=4$ cm.





XL

KORESPONDENCYJNY KURS

Z MATEMATYKI

$\mathrm{p}\mathrm{a}\acute{\mathrm{z}}$dziernik 2010 $\mathrm{r}.$

PRACA KONTROLNA $\mathrm{n}\mathrm{r} 2-$ POZIOM PODSTAWOWY

l. Niech $A=\displaystyle \{x\in \mathbb{R}:\frac{1}{x^{2}+23}\geq\frac{1}{10x}\}$ oraz $B=\displaystyle \{x\in \mathbb{R}:|x-2|<\frac{7}{2}\}.$

Zbiory $A, B, A\cup B, A\cap B, A\backslash B\mathrm{i}B\backslash A$ zapisač $\mathrm{w}$ postaci przedzialów liczbowych $\mathrm{i}$

zaznaczyč je na osi liczbowej.

2. Zaznaczyč na pfaszczy $\acute{\mathrm{z}}\mathrm{n}\mathrm{i}\mathrm{e}$ zbiory

$A=\{(x,y):|x|+|y|\leq 2\}$ oraz

$\mathrm{i}$ obliczyč pole zbioru $A\cap B.$

$B=\displaystyle \{(x,y):\frac{1}{|x-1|}\leq\frac{1}{|x+3|},\frac{2}{|y-1|}\geq 1\}$

3. Trójmian kwadratowy $f(x)=ax^{2}+bx+c$ przyjmuje najmniejszą wartośč równą $-1 \mathrm{w}$

punkcie $x=1$ a reszta $\mathrm{z}$ dzielenia tego trójmianu przez dwumian $(x-2)$ równa jest l.

Wyznaczyč wspólczynniki $a, b, c$. Narysowač staranny wykres funkcji $g(x) = f(|x|) \mathrm{i}$

wyznaczyč najmniejszą $\mathrm{i}$ największą wartośč tej funkcji na przedziale [$-1,3].$

4. Tangens kąta ostrego $\alpha$ równy jest $\displaystyle \frac{a}{b}$, gdzie

$\alpha=(\sqrt{2+\sqrt{3}}-\sqrt{2-\sqrt{3}})^{2}b=(\sqrt{\sqrt{2}+1}-\sqrt{\sqrt{2}-1})^{2}$

Wyznaczyč wartości pozostalych funkcji trygonometrycznych tego $\mathrm{k}_{\Phi^{\mathrm{t}\mathrm{a}}}$. Wykorzystując

wzór $\sin 2\alpha=2\sin\alpha\cos\alpha$, obliczyč miarę kąta $\alpha.$

5. Narysowač wykres funkcji $f(x)=\sqrt{4x^{2}-4x+1}-x \mathrm{i}$ rozwiązač nierównośč $f(x)<0.$

$\mathrm{W}$ zalezności od parametru $m$ określič liczbę rozwiązań równania $|f(x)| = m$. Dla

jakiego $a$ pole trójkqta ograniczonego osia $Ox\mathrm{i}$ wykresem funkcji $g(x)=f(x)-a$ równe

jest 6?

6. Niech $f(x)=$

dla

dla

$x\leq 1,$

$x>1.$

a) Narysowač wykres funkcji $f\mathrm{i}$ na jego podstawie wyznaczyč zbiór wartości funkcji.

b) Obliczyč $f(\sqrt{3}-1)$ oraz $f(3-\sqrt{3}).$

c) Rozwiązač nierównośč $2\sqrt{f(x)}\leq 3\mathrm{i}$ zbiór jej rozwiązań zaznaczyč na osi $0x.$





PRACA KONTROLNA nr 2- POZIOM ROZSZERZONY

l. Rozwiązač nierównośč $\displaystyle \frac{1}{\sqrt{5+4x-x^{2}}}\geq\frac{1}{x-2} \mathrm{i}$ zbiór rozwiązań zaznaczyč $\mathrm{n}\mathrm{a}$ prostej.

2. Niech $A=\{(x,y):y\geq||x-2|-1|\}, B=\{(x,y):y+\sqrt{4x-x^{2}-3}\leq 2\}.$

Narysowač $\mathrm{n}\mathrm{a}\mathrm{p}${\it l}aszczy $\acute{\mathrm{z}}\mathrm{n}\mathrm{i}\mathrm{e}$ zbiór $A\cap B\mathrm{i}$ obliczyč jego pole.

3. Dla jakich wartości rzeczywistego parametru $p$ równanie $(p-1)x^{4}+(p-2)x^{2}+p=0$

ma dokladnie $\mathrm{d}\mathrm{w}\mathrm{a}$ rózne pierwiastki?

4. Znalez/č wszystkie wartości parametru rzeczywistego $m, \mathrm{d}\mathrm{l}\mathrm{a}$ których pierwiastki trójmia-

nu kwadratowego $f(x)=(m-2)x^{2}-(m+1)x-m$ spełniają nierównośč $|x_{1}|+|x_{2}|\leq 1.$

5. Narysowač staranny wykres funkcji

$f(x)=\{$

$\sqrt{x^{2}-4x+4}-1$

$-\sqrt{4x-x^{2}-3}$

, gdy

, gdy

$|x-2|\geq 1,$

$|x-2|\leq 1.$

$\mathrm{i}$ rozwiązač nierównośč $|f(x)| > \displaystyle \frac{1}{2}. \mathrm{W}$ zalezności od parametru $m$ określič liczbę roz-

wiązań równania $|f(x)| =m$. Obliczyč pole obszaru ograniczonego wykresem funkcji

$g(x)=|f(x)|\mathrm{i}$ prostą $y=\displaystyle \frac{1}{2}.$

6. Niech

$f(x)=$

gdy

gdy

$|x-1|\geq 1,$

$|x-1|<1.$

a) Obliczyč $f(-\displaystyle \frac{2}{3}), f(\displaystyle \frac{1+\sqrt{3}}{2})$ oraz $f(\pi-1).$

b) Narysowač wykres funkcji $f\mathrm{i}$ na jego podstawie podač zbiór wartości funkcji.

c) Rozwi$\mathfrak{B}$ač nierównośč $f(x)\displaystyle \geq-\frac{1}{2}\mathrm{i}$ zaznaczyč na osi $0x$ zbiór jej rozwi$\Phi$zań.





XL

KORESPONDENCYJNY KURS

Z MATEMATYKI

listopad 2010 r.

PRACA KONTROLNA $\mathrm{n}\mathrm{r} 3-$ POZIOM PODSTAWOWY

1. $\mathrm{W}$ trójkącie prostokątnym $0$ kącie prostym przy wierzchoku $C$ na przedluzeniu przeciw-

prostokqtnej $AB$ odmierzono odcinek $BD\mathrm{t}\mathrm{a}\mathrm{k}, \dot{\mathrm{z}}\mathrm{e}|BD|=|BC|$. Wyznaczyč $|CD|$ oraz

obliczyč pole trójkta $\triangle ACD, \mathrm{j}\mathrm{e}\dot{\mathrm{z}}$ eli $|BC|=5, |AC|=12.$

2. Harcerze rozbili 2 namioty, jeden $\mathrm{w}$ odległości 5 $\mathrm{m}$, drugi - 17 $\mathrm{m}$ od prostoliniowego

brzegu rzeki. Odlegfośč między namiotami równajest 13 $\mathrm{m}. \mathrm{W}$ którym miejscu na samym

brzegu rzeki (liczqc od punktu brzegu będacego rzutem prostopadłym punktu połozenia

pierwszego namiotu) powinni umieścič maszt $\mathrm{z}\mathrm{f}\mathrm{l}\mathrm{a}\mathrm{g}\Phi$ zastępu, by odlegfośč od masztu do

$\mathrm{k}\mathrm{a}\dot{\mathrm{z}}$ dego $\mathrm{z}$ namiotów była taka sama?

3. Na kole $0$ promieniu $r$ opisano trapez równoramienny, $\mathrm{w}$ którym stosunek dlugości pod-

staw wynosi 4: 3. Ob1iczyč stosunek po1a ko1a do po1a trapezu oraz cosinus kąta ostrego

$\mathrm{w}$ tym trapezie.

4. Wielomian $W(x) =x^{3}-x^{2}+bx+c$ jest podzielny przez $(x+3)$, a reszta $\mathrm{z}$ dzielenia

tego wielomianu przez $(x-3)$ równa jest 6. Wyznaczyč $b\mathrm{i} c$, a następnie rozwiązač

nierównośč $(x+1)W(x-1)-(x+2)W(x-2)\leq 0.$

5. Wykonač dziafania $\mathrm{i}$ zapisač $\mathrm{w}$ najprostszej postaci wyrazenie

$s(a,b)= (\displaystyle \frac{a^{2}+b^{2}}{a^{2}-b^{2}}-\frac{a^{3}+b^{3}}{a^{3}-b^{3}})$ : $(\displaystyle \frac{a^{2}}{a^{3}-b^{3}}-\frac{a}{a^{2}+ab+b^{2}})$

Wyznaczyč wysokośč trójkąta prostokątnego wpisanego $\mathrm{w}$ okrąg $0$ promieniu 6 opusz-

czoną $\mathrm{z}$ wierzcholka kąta prostego wiedząc, $\dot{\mathrm{z}}\mathrm{e}$ tangens jednego $\mathrm{z}$ kątów ostrych tego

trójkąta równy jest $s(\sqrt{5}+\sqrt{3},\sqrt{5}-\sqrt{3}).$

6. $\mathrm{W}$ trójkącie $ABC$ dane są $\angle CAB= \displaystyle \frac{\pi}{3}$, wysokośč $|CD| =h=5$ oraz $BD=d=\sqrt{2}.$

Obliczyč odległośč środków okręgów wpisanych $\mathrm{w}$ trójkąty ADC $\mathrm{i}\mathrm{D}\mathrm{B}\mathrm{C}.$





PRACA KONTROLNA nr 3- POZIOM ROZSZERZONY

l. Dany jest wielomian $W(x) = x^{3}+ax+b$, gdzie $b \neq 0$. Wykazač, $\dot{\mathrm{z}}\mathrm{e} W(x)$ posiada

pierwiastek podwójny wtedy $\mathrm{i}$ tylko wtedy, gdy spelniony jest warunek $4a^{3}+27b^{2}=0.$

Wyrazič pierwiastki za pomocą współczynnika $b.$

2. Wyznaczyč promień okręgu opisanego na czworokqcie ABCD, $\mathrm{w}$ którym $\mathrm{k}\mathrm{a}\mathrm{t}$ przy wierz-

chofku $A$ ma miarę $\alpha$, kąty przy wierzchofkach $B, D$ są proste oraz $|BC|=a, |AD|=b.$

Sporządzič staranny rysunek.

3. Narysowač staranny wykres funkcji $f(x)=\displaystyle \frac{\sin 2x-|\sin x|}{\sin x}.$

$\mathrm{W}$ przedziale $[0,\pi]$ wyznaczyč $\mathrm{r}\mathrm{o}\mathrm{z}\mathrm{w}\mathrm{i}_{\Phi}$zania nierówności $f(x)<2(\sqrt{2}-1)\cos^{2}x.$

4. $\mathrm{Z}$ wierzchołka $A$ kwadratu ABCD $0$ boku $a$ poprowadzono dwie proste, które dzielą kąt

przy tym wierzchołku na trzy równe części $\mathrm{i}$ przecinają boki kwadratu $\mathrm{w}$ punktach $K\mathrm{i}$

$L$. Wyznaczyč dfugości odcinków, na jakie te proste dzielą $\mathrm{P}^{\mathrm{r}\mathrm{z}\mathrm{e}\mathrm{k}}\Phi^{\mathrm{t}\mathrm{n}}\Phi$ kwadratu. Znalez/č

promień okręgu wpisanego $\mathrm{w}$ deltoid AKCL.

5. Czworokąt wypukły ABCD, $\mathrm{w}$ którym $AB=1, BC=2, CD=4, DA=3$ jest wpisany

$\mathrm{w}$ okrąg. Obliczyč promień $R$ tego okręgu. Sprawdzič, czy $\mathrm{w}$ czworokąt ten $\mathrm{m}\mathrm{o}\dot{\mathrm{z}}$ na wpisač

okrąg. $\mathrm{J}\mathrm{e}\dot{\mathrm{z}}$ eli $\mathrm{t}\mathrm{a}\mathrm{k}$, to obliczyč promień $r$ tego okręgu.

6. Na boku $BC$ trójkąta równobocznego obrano punkt $D\mathrm{t}\mathrm{a}\mathrm{k}, \dot{\mathrm{z}}\mathrm{e}$ promień okręgu wpisanego

$\mathrm{w}$ trójkąt $ADC$ jest dwa razy mniejszy $\mathrm{n}\mathrm{i}\dot{\mathrm{z}}$ promień okręgu wpisanego $\mathrm{w}$ trójkąt $ABD.$

$\mathrm{W}$ jakim stosunku punkt $D$ dzieli bok $BC$?





XL

KORESPONDENCYJNY KURS

Z MATEMATYKI

grudzień 2010 r.

PRACA KONTROLNA $\mathrm{n}\mathrm{r} 4-$ POZIOM PODSTAWOWY

l. Rozwiązač równanie $\displaystyle \frac{1}{\cos x}+\mathrm{t}\mathrm{g}x-\sin(\frac{\pi}{2}-x)=0$ dla $x\in[-2\pi,2\pi].$

2. Na p{\it l}aszczy $\acute{\mathrm{z}}\mathrm{n}\mathrm{i}\mathrm{e}$ dane są cztery punkty: $A(1,-1), B(5,7), C(4,-4), D(2,4)$. Obliczyč od-

ległośč punktu przecięcia prostych AB $\mathrm{i}CD$ od symetralnej odcinka $BC$. Sporz$\Phi$dzič

rysunek.

3. Rozwiązač uklad równań

$\left\{\begin{array}{l}
y+x^{2}=4\\
4x^{2}-y^{2}+2y=1
\end{array}\right.$

Podač interpretację geometryczną tego ukladu $\mathrm{i}$ wykazač, $\dot{\mathrm{z}}\mathrm{e}$ cztery punkty, które $\mathrm{s}\Phi$

jego rozwiązaniem, wyznaczają na płaszczy $\acute{\mathrm{z}}\mathrm{n}\mathrm{i}\mathrm{e}$ trapez równoramienny. Znalez$\acute{}$č równanie

okręgu opisanego na tym trapezie.

4. $\mathrm{W}$ ostroslupie prawidfowym trójkątnym dlugośč krawędzi podstawy jest równa $a$. Kąt

między krawędzią podstawy, a krawędzią boczną jest równy $\displaystyle \frac{\pi}{4}$. Obliczyč pole przekro-

ju ostrosłupa $\mathrm{p}\mathrm{a}$szczyz $\Phi$ przechodzącą przez krawędz/ podstawy $\mathrm{i}$ środek przeciwleglej

krawędzi bocznej. Sporządzič staranny rysunek.

5. Dane sa dwa okręgi: $K_{1}0$ środku $\mathrm{w}$ punkcie $(0,0)\mathrm{i}$ promieniu 5 $\mathrm{i}K_{2}\mathrm{o}$ równaniu

$x^{2}+6x+y^{2}-12y+5=0$. Obliczyč pole czworokąta wyznaczonego przez środki okręgów

oraz punkty, $\mathrm{w}$ których te okręgi się przecinają. Sporządzič staranny rysunek.

6. Podstawą graniastosłupa jest równolegfobok $0$ bokach dfugości $a\mathrm{i}2a$ oraz kącie ostrym

$\displaystyle \frac{\pi}{3}$. Krótsza przekątna graniastoslupa tworzy $\mathrm{w}$ pfaszczyzną podstawy kąt $\displaystyle \frac{\pi}{6}$. Obliczyč

długośč dłuzszej przekatnej oraz pole powierzchni całkowitej tego graniastoslupa.





PRACA KONTROLNA nr 4- POZIOM ROZSZERZONY

l. Rozwiązač równanie 2 $\sin^{2}x-2\sin x\cos 2x=1.$

2. Dane są dwa wektory $\vec{a}= [2,-3]$ oraz $\vec{b}= [-1,4]$. Pokazač, $\dot{\mathrm{z}}\mathrm{e}$ wektor $\vec{AB}=3\text{{\it ã}}+2\vec{b}$

jest prostopadły do wektora $\vec{BC}=8\text{{\it ã}}+11\vec{b}$. Obliczyč dlugośč środkowej trójkąta $ABC$

rozpiętego na wektorach $\vec{AB}\mathrm{i}\vec{BC}$, poprowadzonej $\mathrm{z}$ wierzchołka $B.$

3. Niech $K$ będzie wierzchofkiem paraboli $f(x)=-\displaystyle \frac{4}{9}x^{2}-\frac{8}{3}x$, a L- wierzcholkiem paraboli

$g(x) = -f(x-7)+7$. Na paraboli $g(x)$ znalez/č taki punkt $N$, aby wektor $\vec{NL}$ był

równolegfy do wektora $\vec{MK}$, gdzie $M=(0,f(0))$. Obliczyč pole czworokąta KMLN.

4. Przekrój sześcianu płaszczyznq jest sześciokątem foremnym. Wyznaczyč kąt nachylenia

tej pfaszczyzny do pfaszczyzny podstawy sześcianu oraz obliczyč pole tego przekroju.

Wykonač odpowiedni rysunek.

5. Dane są dwa okręgi: $K_{1} 0$ środku $\mathrm{w}$ punkcie $P(1,1) \mathrm{i}$ promieniu l oraz $K_{2} 0$ środku

$Q(9,5) \mathrm{i}$ promieniu 3. Zna1ez/č punkt $S$ na odcinku $\overline{PQ}$ oraz dobrač skalę $k\mathrm{t}\mathrm{a}\mathrm{k}$, aby

okrąg $K_{2}$ był obrazem okręgu $K_{1} \mathrm{w}$ jednokładności $0$ środku $S\mathrm{i}$ skali $k$. Wyznaczyč

równania prostych, które są styczne jednocześnie do obu okręgów $\mathrm{i}\mathrm{p}\mathrm{r}\mathrm{z}\mathrm{e}\mathrm{c}\mathrm{h}\mathrm{o}\mathrm{d}\mathrm{z}\Phi$ przez

punkt $S.$

6. $\mathrm{W}$ ostrosłupie prawidłowym czworokatnym pole $\mathrm{k}\mathrm{a}\dot{\mathrm{z}}$ dej $\mathrm{z}$ pięciu ścian jest równe l. Ostro-

slup ten ścięto $\mathrm{w}$ polowie wysokości $\mathrm{p}^{\mathrm{f}\mathrm{n}}$aszczyz $\Phi$ równolegfą do podstawy. Obliczyč ob-

jętośč oraz pole powierzchni całkowitej otrzymanego ostrosłupa ściętego. Wykonač od-

powiedni rysunek.





XL

KORESPONDENCYJNY KURS

Z MATEMATYKI

styczeń 2011 r.

PRACA KONTROLNA $\mathrm{n}\mathrm{r} 5-$ POZIOM PODSTAWOWY

1. $\mathrm{W}$ ciagu arytmetycznym suma poczatkowych dwudziestu jeden wyrazów wynosi $21\sqrt{2},$

a jego $\mathrm{d}\mathrm{z}\mathrm{i}\mathrm{e}\mathrm{s}\mathrm{i}_{\Phi}\mathrm{t}\mathrm{y}$ wyraz równy jest $-2-2\sqrt{2}$. Wyznaczyč najmniejszy dodatni wyraz

tego ciągu.

2. Rozwiazač nierównośč

$-2<\log_{\frac{1}{2}}(5x+2)\leq 2.$

3. Firmy X $\mathrm{i}\mathrm{Y}$ jednocześnie rozpoczęly dzialalnośč. $\mathrm{W}$ pierwszym miesiącu $\mathrm{k}\mathrm{a}\dot{\mathrm{z}}$ da $\mathrm{z}$ nich

miała dochód równy 50000 zf. Po pięciu miesiącach okazało sie, $\dot{\mathrm{z}}\mathrm{e}$ dochód firmy X

rósł $\mathrm{z}$ miesiąca na miesiąc $0$ tę samą kwote, a dochód firmy $\mathrm{Y}$ wzrastał co miesiąc

geometrycznie. $\mathrm{W}$ drugim $\mathrm{i}$ trzecim miesiącu działalnosci firma X miała dochód wiekszy

od dochodu firmy $\mathrm{Y} 0$ 2000 $\mathrm{z}l$. Ustalič, która $\mathrm{z}$ firm miala wiekszą sumę dochodów

$\mathrm{w}$ pierwszych pięciu miesiącach swojej dzialalności.

4. Sporządzič staranny wykres funkcji (za jednostkę przyjąč 2 cm)

$f(x)=(-2x^{2}+3x\displaystyle \frac{|x|}{1-x}$

dla

dla

$|x-1|\geq 1,$

$|x-1|<1.$

Korzystajqc $\mathrm{z}$ niego, określič ilośč rozwiazań równania $f(x) =m \mathrm{w}$ zalezności od rze-

czywistego parametru $m.$

5. Stosując wzór na sinus podwojonego kata oraz wzory redukcyjne, obliczyč wartośč wy-

$\mathrm{r}\mathrm{a}\dot{\mathrm{z}}$ enia

$\displaystyle \cos\frac{\pi}{5}\cdot\cos\frac{2\pi}{5}\cdot\cos\frac{3\pi}{5}\cdot\cos\frac{4\pi}{5}.$

6. Wiedząc, $\dot{\mathrm{z}}\mathrm{e} \displaystyle \sin\frac{\pi}{10} = \displaystyle \frac{1}{4}(\sqrt{5}-1)$, wyznaczyč wszystkie kąty $\alpha \in [0,\pi]$, dla których

spefnione jest równanie

$2^{2+\sin\alpha}=\sqrt{2}\cdot 4^{\cos^{2}\alpha}$





PRACA KONTROLNA nr 5- POZIOM ROZSZERZONY

l. Zaznaczyč na osi liczbowej zbiór rozwiązań nierówności

$\displaystyle \frac{2x-\sqrt{2-x}}{x}\geq x.$

2. Wyznaczyč wszystkie liczby rzeczywiste x, dla których funkcja

$f(x)=2^{x^{2}+2}-2^{x^{2}-1}-2\cdot 7^{x^{2}-1}$

przyjmuje wartości dodatnie.

3. Określič dziedzinę $\mathrm{i}$ sporządzič staranny wykres funkcji $f(x) = 1-\log_{3}(1-x)$. Za

jednostkę przyj$\Phi$č 2 cm. Zna1ez/č obraz tego wykresu $\mathrm{w}$ symetrii osiowej względem prostej

$x=y\mathrm{i}$ podač wzór funkcji, której wykresem jest nowo powstala krzywa.

4. Rozwiązač nierównośč

$\sqrt{\log_{2}(x^{2}-1)}>\log_{2}\sqrt{x^{2}-1}.$

5. Niech $c>0\mathrm{i}c\neq 1$. Znalez/č liczbę naturalną $m$, dla ktorej suma $m$ poczatkowych wyra-

zów ciągu arytmetycznego $a_{n}=\log_{2}(c^{n})$, jest 10100 razy większa od sumy wszystkich

wyrazów ciągu geometrycznego $b_{n}=\log_{2^{3^{n}}}(c).$

6. Korzystajqc ze wzoru

$\sin 5\alpha=5\sin\alpha-20\sin^{3}\alpha+16\sin^{5}\alpha,$

obliczyč wartośč $\displaystyle \sin\frac{\pi}{5}$. Podač wartości wyrazeń $\displaystyle \cos\frac{\pi}{5}, \displaystyle \sin\frac{\pi}{10}$ oraz $\displaystyle \cos\frac{\pi}{10}$. Wyprowa-

dzič wzór na pole dwudziestokąta foremnego wpisanego $\mathrm{w}$ okrąg $0$ promieniu $r.$







XLI

KORESPONDENCYJNY KURS

Z MATEMATYKI

wrzesień 2011 r.

PRACA KONTROLNA $\mathrm{n}\mathrm{r} 1 -$ POZIOM PODSTAWOWY

l. Średni czas przeznaczony na matematykę na dwunastu wydziałach pewnej uczelni wy-

nosi 240 godzin. Utworzono nowy wydziaf $\mathrm{i}$ wówczas średnia liczba godzin matematyki

wzrosła $0$ 5\%. Ile godzin przeznaczono na matematykę na nowym wydziale?

2. Droge $\mathrm{z}$ miasta $A$ do miasta $B$ rowerzysta pokonuje $\mathrm{w}$ czasie 3 godzin. Po dfugotrwa1ych

deszczach stan $\displaystyle \frac{\mathrm{s}}{5}$ drogi pogorszyl się na tyle, $\dot{\mathrm{z}}\mathrm{e}$ na tym odcinku rowerzysta $\mathrm{m}\mathrm{o}\dot{\mathrm{z}}\mathrm{e}$ jechač

$\mathrm{z}$ prędkością $04\mathrm{k}\mathrm{m}/\mathrm{h}$ mniejszą. By czas podrózy $\mathrm{z}A$ do $B$ nie uległ zmianie, zmuszony

jest na pozostafym odcinku zwiększyč prędkośč $012\mathrm{k}\mathrm{m}/\mathrm{h}$. Jaka jest odleglośč $\mathrm{z}A$ do

$B\mathrm{i}\mathrm{z}$ jaką prędkością jez/dził rowerzysta przed ulewami?

3. Trzy klasy pewnego gimnazjum wyjechafy na zieloną szkolę. $K\mathrm{a}\dot{\mathrm{z}}\mathrm{d}\mathrm{y}$ uczeń $\mathrm{z}$ klasy $\mathrm{A}$

wyslaf tę $\mathrm{s}\mathrm{a}\mathrm{m}\Phi$ liczbę SMS-ów. $\mathrm{W}$ klasie $\mathrm{B}$ wysfano taką samą liczbę SMS-ów, ale liczba

uczniów byla $01$ mniejsza, a $\mathrm{k}\mathrm{a}\dot{\mathrm{z}}\mathrm{d}\mathrm{y}\mathrm{z}$ nich wyslał $02$ SMS-y więcej. $\mathrm{Z}$ kolei klasa $\mathrm{C}, \mathrm{w}$

której było $0$ dwóch uczniów więcej $\mathrm{i}\mathrm{k}\mathrm{a}\dot{\mathrm{z}}\mathrm{d}\mathrm{y}$ wysłaf $05$ SMS-ów więcej, wysfała $\mathrm{w}$ sumie

dwa razy więcej wiadomości. Ilu uczniów bylo na zielonej szkole $\mathrm{i}$ ile SMS-ów wyslali?

4. Ile jest czterocyfrowych liczb naturalnych:

a) podzielnych przez 4 $\mathrm{i}$ przez 5?

b) podzielnych przez 41ub przez 5?

c) podzielnych przez 4 $\mathrm{i}$ niepodzielnych przez 5?

5. Umowa określa wynagrodzenie miesięczne pana Kowalskiego na kwotę 4000 $\mathrm{z}\mathrm{f}$. Skfadka

na ubezpieczenie społeczne wynosi 18, 7\% tej kwoty, a składka na ubezpieczenie zdrowot-

ne- 7, 75\% kwoty pozosta1ej po od1iczeniu skfadki na ubezpieczenie społeczne. $\mathrm{W}$ celu

obliczenia podatku nalez $\mathrm{y}$ od 80\% wyjściowej kwoty umowy odjąč składkę na ubezpie-

czenie społeczne $\mathrm{i}$ wyznaczyč 19\% pozostałej sumy. Podatek jest róznicą tak otrzymanej

kwoty $\mathrm{i}$ skfadki na ubezpieczenie zdrowotne. Ile zfotych miesięcznie otrzymuje pan Ko-

walski? Jakie powinno byč jego wynagrodzenie, by co miesiąc dostawal przynajmniej

3000 $\mathrm{z}l$?

6. Uprościč wyrazenie (dla $x, y$, dla których ma ono sens)

-{\it xx}--23{\it y}-21-{\it y}--3221--{\it xx}3231{\it yy}--2331

( -{\it xy})- -32

$\mathrm{i}$ następnie obliczyč jego wartośč dla $x=1+\sqrt{2}, y=7+5\sqrt{2}.$




PRACA KONTROLNA nr l- POZIOM ROZSZERZONY

l. Wiek ojca jest $05$ lat większy $\mathrm{n}\mathrm{i}\dot{\mathrm{z}}$ suma lat trzech jego synów. Za 101at ojciec będzie

2 razy starszy od swego najstarszego syna, za 20 lat będzie 2 razy starszy od swego

średniego syna, a za 301at będzie 2 razy starszy od swego najmłodszego syna. Kiedy

ojciec byf 3 razy starszy od swego najstarszego syna, a kiedy będzie 3 razy starszy od

swego najmfodszego syna?

2. Dwaj rowerzyści wyruszyli jednocześnie $\mathrm{w}$ drogę, jeden $\mathrm{z}$ A do $\mathrm{B}$, drugi $\mathrm{z}\mathrm{B}$ do A $\mathrm{i}$ minęli

się po godzinie. Pierwszy jechał $\mathrm{z}$ prędkości$\Phi 03$ km większ$\Phi \mathrm{n}\mathrm{i}\dot{\mathrm{z}}$ drugi $\mathrm{i}$ przyjechał do

celu $027$ minut wcześniej. Jakie były prędkości obu rowerzystów $\mathrm{i}$ jaka jest odlegfośč od

A do $\mathrm{B}$ ?

3. Pierwszy $\mathrm{i}$ drugi pracownik $\mathrm{w}\mathrm{y}\mathrm{k}\mathrm{o}\mathrm{n}\mathrm{a}\mathrm{j}_{\Phi}$ wspólnie pewną pracę $\mathrm{w}$ czasie $\mathrm{c} \mathrm{d}\mathrm{n}\mathrm{i}$, drugi $\mathrm{i}$

trzeci-w czasie a $\mathrm{d}\mathrm{n}\mathrm{i}$, zaś pierwszy $\mathrm{i}$ trzeci-w czasie $b\mathrm{d}\mathrm{n}\mathrm{i}$? Ile dni potrzebuje $\mathrm{k}\mathrm{a}\dot{\mathrm{z}}\mathrm{d}\mathrm{y}\mathrm{z}$

pracowników na wykonanie tej pracy samodzielnie?

4. Ile jest liczb pięciocyfrowych podzielnych przez 6, które $\mathrm{w}$ zapisie dziesiętnym mają:

a) obie cyfry 1, 2 $\mathrm{i}$ tylko $\mathrm{t}\mathrm{e}$? b) obie cyfry 2, 3 $\mathrm{i}$ tylko $\mathrm{t}\mathrm{e}$? c) wszystkie cyfry 1, 2, 3

$\mathrm{i}$ tylko $\mathrm{t}\mathrm{e}$? Odpowied $\acute{\mathrm{z}}$ uzasadnič.

5. $\mathrm{W}$ hurtowni znajduje się towar, którego a\% sprzedano $\mathrm{z}$ zyskiem p\%, a b\% pozostałej

części sprzedano $\mathrm{z}$ zyskiem q\%. $\mathrm{Z}$ jakim zyskiem nalezy sprzedač resztę towaru, by

cafkowity zysk wyniósl r\%?

6. Uprościč wyrazenie (dla $x, y$, dla których ma ono sens)

( -{\it y} -21 -{\it y} -{\it x}61 -21 {\it y} -31 - -{\it x} -21 {\it y} -21 {\it x-xy} -31)

[ -{\it x} -21 -1 {\it y} -21 ({\it x} -65 - -{\it xy}-61) - -{\it x} -23 {\it x}$+$-{\it xy}-61 {\it y} -21]

$\mathrm{i}$ następnie obliczyč jego wartośč dla $x=5\sqrt{2}-7, y=7+5\sqrt{2}.$





XLI

KORESPONDENCYJNY KURS

Z MATEMATYKI

luty 2012 r.

PRACA KONTROLNA nr 6- POZIOM PODSTAWOWY

l. Obliczyč, ile jest wszystkich liczb czterocyfrowych, których suma cyfr wynosi 20 $\mathrm{i}$ które

$\mathrm{m}\mathrm{a}\mathrm{j}_{\Phi}$ dokfadnie jedno zero wśród swoich cyfr:

a) $\mathrm{j}\mathrm{e}\dot{\mathrm{z}}$ eli wszystkie cyfry muszą byč rózne,

b) $\mathrm{j}\mathrm{e}\dot{\mathrm{z}}$ eli cyfry mogą powtarzač się.

2. Do ponumerowania wszystkich stron grubej ksiązki zecer $\mathrm{z}\mathrm{u}\dot{\mathrm{z}}$ ył 2989 cyfr. I1e stron ma

ta ksiązka?

3. Zbiory $A, B, C$ są skończone, przy czym

$|A|=10,$

$|B|=9,$

$|A\cap B|=3, |A\cap C|=1,$

$|B\cap C|=1$ oraz

$|A\cup B\cup C|=18.$

Wyznaczyč liczbę elementów zbiorów $A\cap B\cap C$ oraz $C.$

4. Na egzamin $\mathrm{z}$ matematyki przygotowano $\mathrm{i}$ ogloszono 45 zadań. Student nauczył się

rozwiązywač tylko $\displaystyle \frac{2}{3}$ spośród nich. Na egzaminie student losuje trzy zadania. Otrzymuje

ocenę bardzo dobrq za poprawne rozwiqzanie trzech zadań, dobrą za rozwiązanie dwóch,

dostateczną za rozwiązanie jednego $\mathrm{i}$ niedostateczną, gdy nie rozwiąze $\dot{\mathrm{z}}$ adnego zadania.

Jakiejest prawdopodobieństwo, $\dot{\mathrm{z}}\mathrm{e}$ uzyska ocenę co najmniej dostateczną, ajakie- bardzo

dobrq?

5. Udowodnič, $\dot{\mathrm{z}}\mathrm{e}$ dla dowolnej liczby naturalnej $n$ liczba

$\displaystyle \frac{1}{25}\cdot 100^{n}+\frac{2}{5}\cdot 10^{n}+1$

jest kwadratem liczby naturalnej $\mathrm{i}$ jest liczbą podzielną przez 9.

6. $\mathrm{W}$ urnie I są dwie kule biafe $\mathrm{i}$ dwie czarne. $\mathrm{W}$ urnie II jest pięč kul bialych $\mathrm{i}$ trzy

czarne. Rzucamy dwiema kostkami do gry. $\mathrm{J}\mathrm{e}\dot{\mathrm{z}}$ eli iloczyn otrzymanych oczek jest liczbq

$\mathrm{n}\mathrm{i}\mathrm{e}\mathrm{p}\mathrm{a}\mathrm{r}\mathrm{z}\mathrm{y}\mathrm{s}\mathrm{t}_{\Phi}$, to losujemy kulę $\mathrm{z}$ urny I, $\mathrm{w}$ przeciwnym przypadku losujemy kulę $\mathrm{z}$ urny II.

a) Obliczyč prawdopodobieństwo wylosowania kuli czarnej?

b) Ile co najmniej razy nalez $\mathrm{y}$ powtórzyč opisane doświadczenie, aby $\mathrm{z}$ prawdopodo-

bieństwem nie mniejszym $\displaystyle \mathrm{n}\mathrm{i}\dot{\mathrm{z}}\frac{5}{7}$, co najmniej raz wyciągnač kulę białą?





PRACA KONTROLNA nr 6- POZIOM ROZSZERZONY

l. Jest pięč biletów po l zloty, trzy bilety po 3 złote $\mathrm{i}$ dwa bilety po 5 złotych. Wybrano

losowo trzy bilety. Obliczyč prawdopodobieństwo, $\dot{\mathrm{z}}\mathrm{e}:\mathrm{a}$) przynajmniej dwa $\mathrm{z}$ tych biletów

mają jednakową wartośč; b) wybrane bilety mają lączną wartośč 7 złotych.

2. Korzystajqc $\mathrm{z}$ zasady indukcji matematycznej udowodnič, $\dot{\mathrm{z}}\mathrm{e}$ nierównośč

-21

-43

$\displaystyle \frac{2n-1}{2n}<\frac{1}{\sqrt{2n+1}}$

jest prawdziwa dla dowolnej liczby naturalnej $n.$

3. Dwie osoby rzucaj $\Phi$ na przemian $\mathrm{m}\mathrm{o}\mathrm{n}\mathrm{e}\mathrm{t}_{\Phi}$. Wygrywa ta osoba, która pierwsza wyrzuci or-

ła. Obliczyč, ile wynoszą prawdopodobieństwa wygranej dla $\mathrm{k}\mathrm{a}\dot{\mathrm{z}}$ dego $\mathrm{z}$ graczy. Następnie

obliczyč prawdopodobieństwa wygranej obu graczy, gdy rozgrywka została zmieniona

$\mathrm{w}$ następujący sposób: pierwszy gracz rzuca jeden raz $\mathrm{m}\mathrm{o}\mathrm{n}\mathrm{e}\mathrm{t}_{\Phi}$, a potem gracze rzucają

monetą po dwa razy (zaczynając od drugiego gracza), $\mathrm{a}\dot{\mathrm{z}}$ do pierwszego wyrzucenia orla.

4. Ze zbioru liczb naturalnych $n$ spefniających warunek $\displaystyle \frac{1}{\log n}+\frac{\mathrm{l}}{1-\log n}>$ llosujemy kolejno

bez zwracania dwie liczby $\mathrm{i}$ tworzymy $\mathrm{z}$ nich liczbę dwucyfrową, $\mathrm{w}$ której cyfrą dziesiątek

jest pierwsza $\mathrm{z}$ wylosowanych liczb. Sprawdzič niezaleznośč zdarzeń: A- utworzona liczba

jest parzysta, B- utworzona liczba jest podzielna przez 3.

5. Obliczyč, ile liczb mniejszych od l00 nie jest podzielnych przez 2, 3, 5 ani przez 7. Udo-

wodnič, $\dot{\mathrm{z}}\mathrm{e}$ wszystkie te liczby oprócz l są pierwsze. Ile jest liczb pierwszych mniejszych

od 100?

6. Dla $\mathrm{k}\mathrm{a}\dot{\mathrm{z}}$ dej druzyny pilkarskiej biorącej udział $\mathrm{w}$ Euro 2012 eksperci wyznaczy1i współ-

czynnik $p$ oznaczaj $\Phi^{\mathrm{c}\mathrm{y}}$ prawdopodobieństwo, $\dot{\mathrm{z}}\mathrm{e}$ Polska pokona tę druzynę. Druzyny po-

dzielono na cztery koszyki. $\mathrm{Z} \mathrm{k}\mathrm{a}\dot{\mathrm{z}}$ dego koszyka do $\mathrm{k}\mathrm{a}\dot{\mathrm{z}}$ dej grupy zostanie wylosowana

jedna druzyna, tak $\dot{\mathrm{z}}\mathrm{e}$ po zakończeniu losowania powstaną cztery grupy po cztery dru-

$\dot{\mathrm{z}}$ yny. Polska znajduje się $\mathrm{w}$ koszyku A. Pozostale koszyki to:

$\mathrm{B}$: Niemcy $(p=0,2)$, Wlochy $(p=0,2)$, Anglia $(p=0,4)$, Rosja $(p=0,5)$ ;

$\mathrm{C}$: Chorwacja $(p=0,6)$, Grecja $(p=0,6)$, Portugalia $(p=0,4)$, Szwecja $(p=0,6)$ ;

$\mathrm{D}$: Dania $(p=0,4)$, Francja $(p=0,4)$, Czechy $(p=0,6)$, Irlandia $(p=0,5).$

a) Jakie jest prawdopodobieństwo, $\dot{\mathrm{z}}\mathrm{e}$ do grupy $\mathrm{z}$ Polskq trafią przynajmniej dwie

druzyny, których $p$ jest większe lub równe 0, 5?

b) Gospodarz Euro 2012, Po1ska, ma prawo do następuj $\Phi^{\mathrm{c}\mathrm{e}\mathrm{j}}$ modyfikacji: $\mathrm{z}$ losowo wy-

branego koszyka zostaną wylosowane do grupy $\mathrm{z}$ nią dwie druzyny, a $\mathrm{z}$ innego losowo

wybranego koszyka nie będzie losowana $\dot{\mathrm{z}}$ adna. Czy Polsce opłaca się skorzystač $\mathrm{z}$

tego prawa, jeśli chce powiększyč prawdopodobieństwo zdarzenia $\mathrm{z}$ punktu a)?





XLI

KORESPONDENCYJNY KURS

Z MATEMATYKI

marzec 2012 r.

PRACA KONTROLNA nr 7- POZIOM PODSTAWOWY

l. Narysowač wykres funkcji $f(x) = |2x-4|-\sqrt{x^{2}+4x+4}$. Określič liczbe rozwiazań

równania $|f(x)| = m \mathrm{w}$ zalezności od parametru $m$. Dla jakiego $m$ pole trójk$\Phi$ta

ograniczonego wykresem funkcji $f$ oraz prostą $y=m$ równe jest 6?

2. Wśród prostokątów 0 ustalonej dfugości przekątnej p wskazač ten, którego pole jest

największe. Nie stosowač metod rachunku rózniczkowego.

3. Wyznaczyč wszystkie liczby rzeczywiste $x$, dla których funkcja $f(x)=x-1-\log_{\frac{1}{3}}(4-$

$3^{x})$ przyjmuje wartości nieujemne.

4. Stosując wzór na cosinus podwojonego kąta, rozwiazač $\mathrm{w}$ przedziale $[0,2\pi]$ nierównośč

$\displaystyle \cos 2x\leq\frac{\cos 2x+\sin x-\cos^{2}x}{1-\sin x}.$

5. Niech $f(x)=$

dla

dla

$x\leq 1,$

$x>1.$

a) Sporządzič wykres funkcji $f\mathrm{i}$ na jego podstawie wyznaczyč zbiór wartości tej funk-

cji.

b) Obliczyč $f(\sqrt{3}-1) \mathrm{i}$ korzystając $\mathrm{z}$ wykresu zaznaczyč na osi $0x$ zbiór rozwiązań

nierówności $f^{2}(x)\leq 4.$

6. $\mathrm{W}$ kulę $0$ promieniu $R$ wpisano stozek $0$ kacie rozwarcia $\displaystyle \frac{\pi}{3}$ oraz walec $0$ tej samej podsta-

wie, co stozek. Obliczyč stosunek pola powierzchni bocznej stozka do pola powierzchni

bocznej walca.





PRACA KONTROLNA nr 7- POZIOM ROZSZERZONY

l. Uzasadnič, $\dot{\mathrm{z}}\mathrm{e}$ punkty przecięcia dwusiecznych kątów wewnętrznych dowolnego równo-

ległoboku są wierzchofkami prostokąta, którego przekątna ma dlugośč równą róznicy

długości sąsiednich boków równoległoboku.

2. Wśród walców wpisanych $\mathrm{w}$ kulę $0$ promieniu $R$ wskazač ten, którego pole powierzchni

bocznej jest największe. Nie stosowač metod rachunku rózniczkowego.

3. Dane są punkty $A(-1,2), B(1,-4)$ oraz $P(2m,4m^{3}-1)$. Wyznaczyč wszystkie wartości

parametru $m$, dla których $\triangle ABP$ jest prostokątny. $\mathrm{R}\mathrm{o}\mathrm{z}\mathrm{w}\mathrm{i}_{\Phi}$zanie zilustrowač starannym

rysunkiem.

4. Rozwiązač układ równań

$\left\{\begin{array}{l}
x^{2}+y^{2}-8=0\\
xy+x-y=0
\end{array}\right.$

$\mathrm{i}$ podač jego interpretację graficzną.

5. $\mathrm{W}$ przedziale $[-\displaystyle \frac{\pi}{2},\frac{3\pi}{2}]$ rozwiązač nierównośč

$1-\displaystyle \mathrm{t}\mathrm{g}x+\mathrm{t}\mathrm{g}^{2}x-\mathrm{t}\mathrm{g}^{3}x+\cdots>\frac{\sqrt{3}}{2}$ ($1-$ ctg $x$),

której lewa strona jest $\mathrm{s}\mathrm{u}\mathrm{m}\Phi$ nieskończonego ciągu geometrycznego.

6. Wyznaczyč wszystkie wartości rzeczywistego parametru $m$, dla których równanie

$(m^{2}-2)4^{x}+2^{x+1}+m=0$

ma dwa rózne rozwiazania.





XLI

KORESPONDENCYJNY KURS

Z MATEMATYKI

kwiecień 2012 r.

PRACA KONTROLNA $\mathrm{n}\mathrm{r} 6-$ POZIOM PODSTAWOWY

l. Wyznaczyč równanie paraboli, której wykres jest symetryczny względem punktu $(-\displaystyle \frac{3}{2},\frac{5}{2})$

do wykresu paraboli $y = (x+2)^{2}$ Wykazač, $\dot{\mathrm{z}}\mathrm{e}$ punkty przecięcia $\mathrm{i}$ wierzchofki obu

parabol są wierzchołkami równoległoboku $\mathrm{i}$ obliczyč jego pole.

2. $\mathrm{W}$ graniastoslup prawidlowy trójkątny $\mathrm{m}\mathrm{o}\dot{\mathrm{z}}$ na wpisač kulę. Wyznaczyč stosunek pola

powierzchni bocznej do sumy pól obu podstaw oraz cosinus kąta nachylenia przekątnej

ściany bocznej do sąsiedniej ściany bocznej.

3. Uzasadnič, $\dot{\mathrm{z}}\mathrm{e}$ dla $\alpha\in\langle 0,  2\pi\rangle$ równanie

$2x^{2}-2(2\cos\alpha-1)x+2\cos^{2}\alpha-5\cos\alpha+2=0$

nie ma pierwiastków tego samego znaku.

4. Punkty przecięcia prostych $x-y=0, x+y-4=0, x-3y=0$ są wierzchołkami trójkąta.

Obliczyč objętośč bryfy powstałej $\mathrm{z}$ obrotu tego trójkąta dookoła najdłuzszego boku.

5. Trzech pracowników ma wykonač pewnq pracę. Aby wykonač tę pracę samodzielnie,

pierwszy $\mathrm{z}$ nich pracowałby $07$ dni dluzej, drugi - $015$ dni dluzej, a trzeci - trzy razy

dłuzej, $\mathrm{n}\mathrm{i}\dot{\mathrm{z}}$ gdyby pracowali razem. $\mathrm{W}$ jakim czasie wykonają tę pracę wspólnie?

6. Wyznaczyč promień kuli stycznej do wszystkich krawędzi czworościanu foremnego $0$

krawędzi $\alpha.$





PRACA KONTROLNA nr 6- POZIOM ROZSZERZONY

l. Rozwiązač nierównośč $\displaystyle \frac{x}{\sqrt{x^{3}-2x+1}}\geq\frac{1}{\sqrt{x+3}}.$

2. Narysowač staranny wykres funkcji

$f(x)=\displaystyle \frac{\sin 2x-|\sin x|}{\sin x}.$

Następnie $\mathrm{w}$ przedziale $[0,\pi]$ wyznaczyč rozwiqzania nierówności

$f(x)<2(\sqrt{2}-1)\cos^{2}x$

3. Rozwiązač nierównośč

$1+\displaystyle \frac{\log_{2}x}{1+\log_{2}x}+(\frac{\log_{2}x}{1+\log_{2}x})^{2}+\cdots\geq 2\log_{2}x,$

której lewa strona jest $\mathrm{s}\mathrm{u}\mathrm{m}\Phi$ nieskończonego szeregu geometrycznego.

4. Objętośč stozka jest 4 razy miejsza $\mathrm{n}\mathrm{i}\dot{\mathrm{z}}$ objętośč opisanej na nim kuli. Wyznaczyč sto-

sunek pola powierzchni całkowitej stozka do pola powierzchni kuli oraz kąt, pod jakim

$\mathrm{t}\mathrm{w}\mathrm{o}\mathrm{r}\mathrm{z}\Phi^{\mathrm{C}\mathrm{a}}$ stozka jest widoczna ze środka kuli.

5. Promień światla przechodzi przez punkt $A(1,1)$, odbija się od prostej $0$ równaniu

$y = x-2$ (zgodnie $\mathrm{z}$ zasadq mówiąca, $\dot{\mathrm{z}}\mathrm{e}$ kąt padania jest równy kątowi odbicia) $\mathrm{i}$

przechodzi przez punkt $B(4,6)$. Wyznaczyč wspófrzędne punktu odbicia $P$ oraz równania

prostych, po których biegnie promień przed $\mathrm{i}$ po odbiciu.

6. Na boku $BC$ trójkąta równobocznego obrano punkt $D\mathrm{t}\mathrm{a}\mathrm{k}, \dot{\mathrm{z}}\mathrm{e}$ promień okręgu wpisanego

$\mathrm{w}$ trójkqt $ADC$ jest dwa razy mniejszy $\mathrm{n}\mathrm{i}\dot{\mathrm{z}}$ promień okręgu wpisanego $\mathrm{w}$ trójkąt $ABD.$

$\mathrm{W}$ jakim stosunku punkt $D$ dzieli bok $BC$?





XLI

KORESPONDENCYJNY KURS

Z MATEMATYKI

$\mathrm{p}\mathrm{a}\acute{\mathrm{z}}$dziernik 2011 $\mathrm{r}.$

PRACA KONTROLNA $\mathrm{n}\mathrm{r} 2-$ POZIOM PODSTAWOWY

l. Niech $A=\displaystyle \{x\in \mathbb{R}:\frac{x}{x^{2}-1}\geq\frac{1}{x}\}$ oraz $B=\{x\in \mathbb{R}:|x+2|<4\}$. Zbiory $A, B, A\cup B,$

$A\cap B, A\backslash B\mathrm{i}B\backslash A$ zapisač $\mathrm{w}$ postaci przedziałów liczbowych $\mathrm{i}$ zaznaczyč je na osi

liczbowej.

2. Zaznaczyč na płaszczy $\acute{\mathrm{z}}\mathrm{n}\mathrm{i}\mathrm{e}$ zbiory $A\cap B, A\backslash B,$

$B=\{(x,y):|y|>x^{2}\}.$

gdzie $A = \{(x,y):|x|+2y\leq 3\},$

3. Suma wysokości $h$ ostrosłupa prawidłowego czworokątnego $\mathrm{i}$ jego krawędzi bocznej $b$

równa jest 12. D1a jakiej wartości $h$ objętośč tego ostroslupa jest największa? Obliczyč

pole powierzchni cafkowitej ostrosfupa dla znalezionej wartości $h.$

4. Wykres trójmianu kwadratowego $f(x)=ax^{2}+bx+c$ jest symetryczny względem prostej

$x=2$, a największ$\Phi$ wartości$\Phi$ tej funkcjijest l. Wyznaczyč wspólczynniki $a, b, c$, wiedząc,

$\dot{\mathrm{z}}\mathrm{e}$ reszta $\mathrm{z}$ dzielenia tego trójmianu przez dwumian $(x+1)$ równa jest $-8$. Narysowač

staranny wykres funkcji $g(x) = f(|x|) \mathrm{i}$ wyznaczyč najmniejszą $\mathrm{i}$ największą wartośč

funkcji $g$ na przedziale [-1, 3].

5. Liczba $p=\displaystyle \frac{(2\sqrt{3}+\sqrt{2})^{3}+(2\sqrt{3}-\sqrt{2})^{3}}{(\sqrt{3}+2)^{2}-(\sqrt{3}-2)^{2}}$ jest kwadratem promienia okręgu opisanego

na trójkqcie prostokqtnym $0$ polu 7,2. Ob1iczyč wysokośč $\mathrm{i}$ tangens mniejszego $\mathrm{z}$ kątów

ostrych tego trójkąta.

6. Narysowač wykres funkcji $f(x)=\sqrt{x^{2}+2x+1}-|2x-4|$. Obliczyč pole obszaru ograni-

czonego wykresem funkcji $f(x)$ oraz wykresem funkcji $g(x)=-f(x)$. Narysowač wykresy

funkcji $f_{1}(x)=|f(x)|$ oraz $f_{2}(x)=f(|x|).$





PRACA KONTROLNA nr 2- POZIOM ROZSZERZONY

l. Dlajakich wartości rzeczywistego parametru $p$ równanie $(p-1)x^{2}-(p+1)x-1=0$ ma

dwa rózne pierwiastki ujemne?

2. Narysowač na płaszczyz/nie zbiór $\{(x,y):\sqrt{x-1}+x\leq 2,0\leq y^{3}\leq\sqrt{5}-2\}$

jego pole. Wsk. Obliczyč $a=(\displaystyle \frac{\sqrt{5}-1}{2})^{3}$

i obliczyč

3. Obliczyč $a=\mathrm{t}\mathrm{g}\alpha, \mathrm{j}\mathrm{e}\dot{\mathrm{z}}$ eli $\displaystyle \sin\alpha-\cos\alpha=\frac{1}{5}\mathrm{i}\mathrm{k}\mathrm{a}\mathrm{t}\alpha$ spełnia nierównośč $\displaystyle \frac{\pi}{4}<\alpha<\frac{\pi}{2}$. Znalez/č

promień kofa wpisanego $\mathrm{w}$ trójkąt $\mathrm{p}\mathrm{r}\mathrm{o}\mathrm{s}\mathrm{t}\mathrm{o}\mathrm{k}_{\Phi^{\mathrm{t}}}\mathrm{n}\mathrm{y}\mathrm{o}$ polu $ 25\pi$, wiedząc, $\dot{\mathrm{z}}\mathrm{e}$ tangens jednego

$\mathrm{z}$ kątów ostrych tego trójkąta jest równy $a.$

4. Narysowač wykres funkcji $f(x) =2|x-1|-\sqrt{x^{2}+2x+1}$. Dla jakiego $m$ pole figury

ograniczonej wykresem funkcji $f$ oraz prostą $y=m$ równe jest 32?

5. Wiadomo, $\dot{\mathrm{z}}\mathrm{e}$ liczby $-1$, 3 sq pierwiastkami wielomianu $W(x)=x^{4}-ax^{3}-4x^{2}+bx+3.$

Wyznaczyč $a, b\mathrm{i}$ rozwiązač nierównośč $\sqrt{W(x)}\leq x^{2}-x.$

6. Narysowač wykres funkcji $f(x)=$

$\mathrm{i}$ na jego podstawie wyznaczyč:

gdy

gdy

$|x-2|\leq 1,$

$|x-2|>1$

a) przedziafy, na których funkcja $f$ jest malejąca,

b) zbiór wartości funkcji $f(x),$

c) zbiór rozwiązań nierówności $|f(x)|\displaystyle \leq\frac{1}{2}.$





XLI

KORESPONDENCYJNY KURS

Z MATEMATYKI

listopad 2011 r.

PRACA KONTROLNA $\mathrm{n}\mathrm{r} 3-$ POZIOM PODSTAWOWY

1. $\mathrm{W}$ trapez równoramienny $0$ obwodzie 20 $\mathrm{i}$ kacie ostrym $\displaystyle \frac{\pi}{6}\mathrm{m}\mathrm{o}\dot{\mathrm{z}}$ na wpisač okrqg. Obliczyč

promień okręgu oraz dfugości boków tego trapezu.

2. Wielomian $W(x)=x^{3}+ax^{2}+bx-64$ ma trzy pierwiastki rzeczywiste, których średnia

arytmetyczna jest równa $\displaystyle \frac{14}{\mathrm{s}}$, a jeden $\mathrm{z}$ pierwiastków jest równy średniej geometrycznej

dwóch pozostafych. Wyznaczyč $a\mathrm{i}b$ oraz pierwiastki tego wielomianu.

3. Na okręgu $0$ promieniu $r$ opisano romb, którego dłuzsza przekątna ma długośč $4r$. Wy-

znaczyč pola wszystkich czterech figur ograniczonych bokami rombu $\mathrm{i}$ odpowiednimi

łukami okręgu.

4. Przez punkt $(-1,1)$ poprowadzono prostq $\mathrm{t}\mathrm{a}\mathrm{k}$, aby środek jej odcinka zawartego między

prostymi $x+2y= 1\mathrm{i}x+2y=3$ nalezaf do prostej $x-y= 1$. Wyznaczyč równanie

symetralnej odcinka.

5. $\mathrm{W}$ okręgu $0$ środku $\mathrm{w}$ punkcie $O\mathrm{i}$ promieniu $r$ poprowadzono dwie wzajemnie prosto-

padłe średnice AB $\mathrm{i}CD$ oraz cięciwe $AE$, która przecina średnicę $CD\mathrm{w}$ punkcie $F.$

Dla jakiego kąta $\angle BAE$, czworokąt OBEF ma dwa razy większe pole od pola trójkąta

$AFO$?

6. Na przeciwprostokątnej $AB$ trójkąta prostokqtnego $ABC$ zbudowano trójkqt równobocz-

ny $ADB$, którego pole jest dwa razy większe od pola trójkąta $ABC$. Wyznaczyč kąty

trójkąta $ABC$ oraz stosunek $|BK|$ : $|KA|$ dfugości odcinków, na jakie punkt styczności

$K$ okregu wpisanego $\mathrm{w}$ trójkąt $ABC$ dzieli przeciwprostokatną.





PRACA KONTROLNA nr 3- POZIOM ROZSZERZONY

l. Napisač równanie okręgu przechodzącego przez punkt (1, 2) stycznego do prostych

$y=-2x\mathrm{i}y=-2x+20.$

2. Na bokach $AC \mathrm{i} BC$ trójkąta $ABC$ zaznaczono odpowiednio punkty $E \mathrm{i} D \mathrm{t}\mathrm{a}\mathrm{k}, \dot{\mathrm{z}}\mathrm{e}$

$\displaystyle \frac{|EC|}{|AE|}=\frac{|DC|}{|BD|}=2$. Wyznaczyč stosunek pola trójkąta $ABC$ do pola trójkąta $ABF$, gdzie

$F$ jest punktem przecięcia odcinków $AD\mathrm{i}$ {\it BE}.

3. Kąt przy wierzchołku $C$ trójkąta $ABC$ jest równy $\displaystyle \frac{\pi}{3}$, a długości boków $AC\mathrm{i}BC$ wyno-

$\mathrm{s}\mathrm{z}\Phi$ odpowiednio 15 cm $\mathrm{i}10$ cm. Na bokach trójk$\Phi$ta zbudowano trójkąty równoboczne

$\mathrm{i}$ otrzymano $\mathrm{w}$ ten sposób wielokąt $0$ dodatkowych wierzcholkach $D, E, F$. Obliczyč

odległośč między wierzchołkami $C\mathrm{i}D, B\mathrm{i}F$ oraz A $\mathrm{i}D$?

4. Wielomian $W(x)=x^{4}-3x^{3}+ax^{2}+bx+c$ ma pierwiastek równy l. Reszta $\mathrm{z}$ dzielenia tego

wielomianu przez $x^{2}-x-2$ równa jest $4x-12$. Wyznaczyč $a, b, c\mathrm{i}$ pozostałe pierwiastki.

Rozwiązač nierównośč $W(x+1)\geq W(x-1).$

5. Dane jest równanie

$(2\sin\alpha-1)x^{2}-2x+\sin\alpha=0,$

$\mathrm{z}$ niewiadomą $x\mathrm{i}$ parametrem $\alpha\in [-\displaystyle \frac{\pi}{2},\frac{\pi}{2}]$. Dlajakich wartości $\alpha$ suma odwrotności pier-

wiastków równania jest większa od 8 $\sin\alpha$, a dla jakich- suma kwadratów odwrotności

pierwiastków jest równa 2 $\sin\alpha$?

6. $\mathrm{W}$ trójkąt równoramienny wpisano okrąg $0$ promieniu $r$. Wyznaczyč pole trójkąta, $\mathrm{j}\mathrm{e}\dot{\mathrm{z}}$ eli

środek okręgu opisanego na tym trójkącie $\mathrm{l}\mathrm{e}\dot{\mathrm{z}}\mathrm{y}$ na okręgu wpisanym $\mathrm{w}$ ten trójkąt.





XLI

KORESPONDENCYJNY KURS

Z MATEMATYKI

grudzień 2011 r.

PRACA KONTROLNA $\mathrm{n}\mathrm{r} 4-$ POZIOM PODSTAWOWY

l. Dane są punkty $A(1,2)$ oraz $B(-1,3)$. Znalez/č współrzędne wierzchołków $C\mathrm{i}D$, jeśli

ABCD jest równoleglobokiem, $\mathrm{w}$ którym $\displaystyle \not\simeq DAB=\frac{\pi}{4}, \displaystyle \mathrm{a}\not\in ADB=\frac{\pi}{2}.$

2. Zaznaczyč na płaszczyz/nie zbiór punktów określony przez uklad nierówności

$\left\{\begin{array}{l}
x^{2}+y^{2}-2|x|>0,\\
|y|\leq 2-x^{2}
\end{array}\right.$

3. $\mathrm{W}$ przedziale $[0,\pi]$ rozwiązač równanie

$\displaystyle \frac{6-12\sin^{2}x}{\mathrm{t}\mathrm{g}^{2}x-1}=8\sin^{4}x-5.$

4. $\mathrm{W}$ sześcian $0$ krawędzi dlugości $a$ wpisano walec, którego przekrój osiowy jest kwadra-

tem, a osią jest przekątna sześcianu. Obliczyč objetośč $V$ walca. Nie wykonując obliczeń

przyblizonych, uzasadnič, $\dot{\mathrm{z}}\mathrm{e}V$ stanowi ponad 25\% objętości sześcianu.

5. Znalez$\acute{}$č równania prostych prostopadłych do prostej $x+2y+4 = 0$ odcinających na

okręgu $(x-2)^{2}+(y-4)^{2} =24$ cięciwy $0$ dfugości 4. Zna1ez$\acute{}$č równanie tej przekątnej

czworokąta wyznaczonego przez otrzymane cięciwy, która tworzy $\mathrm{z}$ osią $Ox$ większy kąt.

6. Wysokośč ostrosfupa prawidłowego sześciokątnego wynosi $H$, a $\mathrm{k}\mathrm{a}\mathrm{t}$ między sqsiednimi

ścianami bocznymi ma miarę $\displaystyle \frac{3}{4}\pi$. Obliczyč objętośč tego ostroslupa oraz tangens $\mathrm{k}_{\Phi^{\mathrm{t}\mathrm{a}}}$

nachylenia ściany bocznej do podstawy.





PRACA KONTROLNA nr 4- POZIOM ROZSZERZONY

l. Znalez$\acute{}$č równania okręgów $0$ promieniu 2 przecinających okrąg $(x+2)^{2}+(y+1)^{2}=25$

$\mathrm{w}$ punkcie $P(1,3)$ pod $\mathrm{k}_{\Phi}\mathrm{t}\mathrm{e}\mathrm{m}$ prostym. Korzystač $\mathrm{z}$ metod rachunku wektorowego.

2. Rozwiązač graficznie układ równań

$\left\{\begin{array}{l}
x^{2}+y^{2}=3+|4x+2|,\\
y^{2}=5-|x|,
\end{array}\right.$

wykonując staranne wykresy krzywych danych powyzszymi równaniami oraz niezbędne

obliczenia.

3. Rozwiązač równanie

$\displaystyle \frac{\cos 6x}{\sin^{4}x-\cos^{4}x}=2\cos 4x+1.$

4. $\mathrm{W}$ trójkącie $ABC$ dany jest wierzchofek $B(-1,3)$. Prosta $y=x+1$ jest symetralnq boku

$BC$, a prosta $9x-3y-2=0$ symetralną boku $AB$. Obliczyč pole trójkąta $ABC$ oraz

tangens $\mathrm{k}_{\Phi}\mathrm{t}\mathrm{a}\alpha$ przy wierzcholku $A$. Uzasadnič, $\displaystyle \dot{\mathrm{z}}\mathrm{e}\frac{5\pi}{12}<\alpha<\frac{\pi}{2}$, nie wykonując obliczeń

przyblizonych.

5. $\mathrm{W}$ walec $0$ promieniu podstawy $R\mathrm{i}$ wysokości $tR$, gdzie $t$ jest parametrem dodatnim,

wpisano mniejszy walec $\mathrm{t}\mathrm{a}\mathrm{k}$, aby byf styczny do powierzchni bocznej $\mathrm{i}$ obu podstaw

większego walca, a jego oś była prostopadla do osi większego walca. Wyrazič stosunek

objętości mniejszego walca do objętości większego jako funkcję parametru $t$. Wyznaczyč

największą wartośč tego stosunku $\mathrm{i}$ odpowiadające mu wymiary obu walców. Podač

warunki rozwiqzalności zadania. Sporządzič odpowiednie rysunki.

6. Promień kuli opisanej na ostroslupie prawidlowym trójkątnym wynosi $R$. Wiadomo, $\dot{\mathrm{z}}\mathrm{e}$

$\mathrm{k}\mathrm{a}\mathrm{t}$ płaski przy wierzcholkujest dwa razy większy $\mathrm{n}\mathrm{i}\dot{\mathrm{z}}\mathrm{k}\mathrm{a}\mathrm{t}$ nachylenia krawędzi bocznej do

podstawy. Obliczyč objętośč ostroslupa $\mathrm{i}$ określič miarę kąta nachylenia ściany bocznej

do podstawy.





XLI

KORESPONDENCYJNY KURS

Z MATEMATYKI

styczeń 2012 r.

PRACA KONTROLNA $\mathrm{n}\mathrm{r} 5-$ POZIOM PODSTAWOWY

l. Wykazač, $\dot{\mathrm{z}}\mathrm{e}$ dla dowolnej liczby naturalnej $n$ liczba

przez 6.

$\displaystyle \frac{1}{4}n^{4}+\frac{1}{2}n^{3}-\frac{1}{4}n^{2}-\frac{1}{2}n$ jest podzielna

2. Niech $a=\log_{\frac{2}{5}}16+\log_{\frac{5}{2}}100$. Rozwiązač nierównośč $\log_{2}(x^{2}+x)+\log_{\frac{1}{2}}a\leq 0.$

3. Rozwiązač równanie $\displaystyle \frac{\sin 4x}{\cos 2x}=-1.$

4. Obliczyč $x, \mathrm{w}\mathrm{i}\mathrm{e}\mathrm{d}\mathrm{z}\Phi^{\mathrm{C}}, \dot{\mathrm{z}}\mathrm{e}\mathrm{t}\mathrm{g}\alpha = 2^{x}, \mathrm{t}\mathrm{g}\beta= 2^{-x}$ oraz $\alpha-\beta= \displaystyle \frac{\pi}{6}$. Wyznaczyč $n\mathrm{t}\mathrm{a}\mathrm{k}$, by

$1+4^{x}+4^{2x}+\cdots+4^{(n-1)x}=121.$

5. Logarytmy $\mathrm{z}$ trzech liczb dodatnich tworzą ciąg arytmetyczny. Suma tych liczb równa

jest 26, a suma ich odwrotności wynosi 0.7(2). Zna1ez$\acute{}$č $\mathrm{t}\mathrm{e}$ liczby.

6. $\mathrm{O}$ kącie $\alpha$ wiadomo, $\displaystyle \dot{\mathrm{z}}\mathrm{e}\sin\alpha+\cos\alpha=\frac{2}{\sqrt{3}}.$

a) Określič, $\mathrm{w}$ której čwiartce jest kąt $\alpha.$

b) Obliczyč $\mathrm{t}\mathrm{g}\alpha+$ ctg $\alpha$ oraz $\sin\alpha-\cos\alpha.$

c) Wyznaczyč $\mathrm{t}\mathrm{g}\alpha.$





PRACA KONTROLNA nr 5- POZIOM ROZSZERZONY

l. Wykorzystując zasade indukcji matematycznej udowodnič, $\dot{\mathrm{z}}\mathrm{e}$ dla $\mathrm{k}\mathrm{a}\dot{\mathrm{z}}$ dej liczby natural-

nej $n$ zachodzi równośč

$\left(\begin{array}{l}
2\\
2
\end{array}\right) + \left(\begin{array}{l}
3\\
2
\end{array}\right) + \left(\begin{array}{l}
4\\
2
\end{array}\right) +\cdots\left(\begin{array}{l}
2n\\
2
\end{array}\right) =\displaystyle \frac{(2n-1)n(2n+1)}{3}.$

2. Dla jakiego parametru $m$ równanie $x^{3}+(m-1)x^{2}-(2m^{2}+m)x+2m^{2}=0$ ma trzy

pierwiastki tworzące ciąg arytmetyczny?

3. Rozwiązač nierównośč $\log(1-2^{x})+x\log 5\leq x(1+\log 2)+\log 6.$

4. Rozwiązač równanie

$\displaystyle \frac{\sin x}{1+\cos x}=2-$ ctg $x.$

Podač interpretację geometryczna, sporządzając wykresy odpowiednich funkcji.

5. Dane są liczby: $m=\displaystyle \frac{\left(\begin{array}{l}
6\\
4
\end{array}\right)\left(\begin{array}{l}
8\\
2
\end{array}\right)}{\left(\begin{array}{l}
7\\
3
\end{array}\right)},$

{\it n}$=$ -($\sqrt{}$($\sqrt{}$24)1-64)(3-41.)2-7-25-$\sqrt{}$4-413.

a) Sprawdzič, wykonując odpowiednie obliczenia, $\dot{\mathrm{z}}\mathrm{e}m, n$ są liczbami naturalnymi.

b) Wyznaczyč $k\mathrm{t}\mathrm{a}\mathrm{k}$, by liczby $m, k, n$ były odpowiednio: pierwszym, drugim $\mathrm{i}$ trzecim

wyrazem $\mathrm{c}\mathrm{i}_{\Phi \mathrm{g}}\mathrm{u}$ geometrycznego.

c) Wyznaczyč sumę wszystkich wyrazów nieskończonego ciągu geometrycznego, któ-

rego pierwszymi trzema wyrazami są $m, k, n$. Ile wyrazów tego ciągu nalez $\mathrm{y}$ wziąč,

by ich suma przekroczyła 95\% sumy wszystkich wyrazów?

6. Rozwiązač równanie

$1-(\displaystyle \frac{2^{x}}{3^{x}-2^{x}})+(\frac{2^{x}}{3^{x}-2^{x}})^{2}-(\frac{2^{x}}{3^{x}-2^{x}})^{3}+\ldots=\frac{3^{x-2}}{2^{x-1}},$

którego lewa strona jest sumą wyrazów nieskończonego ciągu geometrycznego.







XLII

KORESPONDENCYJNY KURS

Z MATEMATYKI

wrzesień 2012 r.

PRACA KONTROLNA $\mathrm{n}\mathrm{r} 1 -$ POZIOM PODSTAWOWY

l. Niech $A=\displaystyle \{x\in \mathbb{R}:\frac{1}{x^{2}+1}\geq\frac{1}{7-x}\}$ oraz $B=\{x\in \mathbb{R}:|x-2|+|x-7|<7\}$. Znalez/č

$\mathrm{i}$ zaznaczyč na osi liczbowej zbiory $A, B$ oraz $(A\backslash B)\cup(B\backslash A).$

2. Liczba $p=\displaystyle \frac{(\sqrt[3]{54}-2)(9\sqrt[3]{4}+6\sqrt[3]{2}+4)-(2-\sqrt{3})^{3}}{\sqrt{3}+(1+\sqrt{3})^{2}}$ jest miejscem zerowym funkcji

$f(x)=ax^{2}+bx+c$. Pole trójkąta, którego wierzcholkami są punkty przecięcia wykresu

$\mathrm{z}$ osiami układu współrzędnych równe jest 20. Wyznaczyč współczynnik $b$ oraz drugie

miejsce zerowe tej funkcji $\mathrm{w}\mathrm{i}\mathrm{e}\mathrm{d}\mathrm{z}\Phi^{\mathrm{C}}, \dot{\mathrm{z}}\mathrm{e}$ wykres funkcji jest symetryczny względem prostej

$x=3.$

3. Trapez $0$ kqtach przy podstawie $30^{\mathrm{o}}$ oraz $45^{\mathrm{o}}$ jest opisany na okręgu $0$ promieniu $R.$

Obliczyč stosunek pola kola do pola trapezu.

4. Niech $f(x)=$

Obliczyč $f(\displaystyle \frac{1+\sqrt{3}}{2})$ oraz $f(\displaystyle \frac{\pi+1}{\pi-2}).$

Narysowač wykres funkcji $f\mathrm{i}$ na jego podstawie podač zbiór wartości funkcji oraz roz-

wiqzač nierównośč $f(x)\displaystyle \geq-\frac{1}{2}.$

5. Tangens kąta ostrego $\alpha$ równy jest $\displaystyle \frac{a}{7b}$, gdzie

$a=(\sqrt{2}+1)^{3}-(\sqrt{2}-1)^{3},b=(\sqrt{\sqrt{2}+1}-\sqrt{\sqrt{2}-1})^{2}$

Wyznaczyč wartości pozostałych funkcji trygonometrycznych tego kąta oraz kąta $2\alpha.$

6. $\mathrm{W}$ trójk$\Phi$t otrzymany $\mathrm{w}$ przekroju ostrosłupa prawidłowego czworokątnego pfaszczyzną

przechodzącą przez wysokośč ostrosłupa $\mathrm{i}$ przekątną jego podstawy wpisano kwadrat,

którego jeden bok jest zawarty $\mathrm{w}$ przekatnej podstawy. Pole kwadratu jest dwa ra-

zy mniejsze $\mathrm{n}\mathrm{i}\dot{\mathrm{z}}$ pole podstawy ostrosfupa. Obliczyč stosunek pola powierzchni bocznej

ostrosłupa do pola jego podstawy oraz cosinus kąta między ścianami bocznymi.




PRACA KONTROLNA nr l- POZIOM ROZSZERZONY

l. Niech $A=\{(x,y):y\geq||x-2|-1|\}, B=\{(x,y):y+\sqrt{4x-x^{2}-3}\leq 2\}$. Narysowač

na pfaszczy $\acute{\mathrm{z}}\mathrm{n}\mathrm{i}\mathrm{e}$ zbiór $A\cap B\mathrm{i}$ obliczyč jego pole.

2. Pole powierzchni bocznej ostrosłupa prawidłowego trójkątnego jest $\mathrm{k}$ razy większe $\mathrm{n}\mathrm{i}\dot{\mathrm{z}}$

pole jego podstawy. Obliczyč cosinus kata nachylenia krawędzi bocznej ostroslupa do

pfaszczyzny podstawy.

3. Dane są liczby: $m = \displaystyle \frac{\left(\begin{array}{l}
6\\
4
\end{array}\right)\left(\begin{array}{l}
8\\
2
\end{array}\right)}{\left(\begin{array}{l}
7\\
3
\end{array}\right)}, n = \displaystyle \frac{(\sqrt{2})^{-4}(\frac{1}{4})^{-\frac{5}{2}}\sqrt[4]{3}}{(\sqrt[4]{16})^{3}\cdot 27^{-\frac{1}{4}}}$. Wyznaczyč $k \mathrm{t}\mathrm{a}\mathrm{k}$, by liczby

$m, k, n$ byly odpowiednio: pierwszym, drugim $\mathrm{i}$ trzecim wyrazem ciqgu geometrycznego,

a nstępnie wyznaczyč sumę wszystkich wyrazów nieskończonego ciągu geometrycznego,

którego pierwszymi trzema wyrazami są $m, k, n$. Ile wyrazów tego ciągu nalezy wziąč,

by ich suma przekroczyła 95\% sumy wszystkich wyrazów?

4. Narysowač wykres funkcji $f(x)=$ 

Poslugujqc się nim podač

wzór $\mathrm{i}$ narysowač wykres funkcji $g(m)$ określającej liczbę rozwiązań równania $f(x)=m,$

gdzie $m$ jest parametrem rzeczywistym.

5. Obliczyč tangens kąta wypukfego $\alpha$ spefniaj $\Phi^{\mathrm{c}\mathrm{e}\mathrm{g}\mathrm{o}}$ warunek $\sin\alpha-\cos\alpha=2\sqrt{6}\sin\alpha\cos\alpha.$

6. $\mathrm{W}$ trójkącie równoramiennym $ABC0$ podstawie $AB$ ramie ma dlugośč $b$, a kąt przy

wierzchofku C- miarę $\gamma. D$ jest takim punktem ramienia $BC, \dot{\mathrm{z}}\mathrm{e}$ odcinek $AD$ dzieli pole

trójkqta na polowę. Wyznaczyč promienie $\rho_{1}, \rho_{2}$ okręgów wpisanych $\mathrm{w}$ trójkąty $ABD\mathrm{i}$

$ADC$. Dla jakiego kąta $\gamma$ promienie te są równe, a dla jakiego $\rho_{1}=2\rho_{2}$?





XLII

KORESPONDENCYJNY KURS

Z MATEMATYKI

luty 2013 r.

PRACA KONTROLNA $\mathrm{n}\mathrm{r} 6-$ POZIOM PODSTAWOWY

l. Rozwiazač równanie

$\sqrt{2^{2x+1}-52^{x}+4}=2^{x+2}-5.$

2. Spośród cyfr liczby 2ll52ll25ll2 wylosowano trzy (bez zwracania). Obliczyč prawdo-

podobieństwo tego, $\dot{\mathrm{z}}\mathrm{e}$ liczba utworzona $\mathrm{z}$ wylosowanych cyfr nie jest podzielna przez

trzy.

3. Wyznaczyč dziedzinę funkcji

$f(x)=\sqrt{-\log_{2}\frac{3x}{x^{2}-4}}.$

4. 20 uczniów posadzono losowo $\mathrm{w}$ sali zawierającej 4 rzędy po 5 krzese1 $\mathrm{w}\mathrm{k}\mathrm{a}\dot{\mathrm{z}}$ dym. Obliczyč

prawdopodobieństwo tego, $\dot{\mathrm{z}}\mathrm{e}$ Bolek będzie siedział przy Lolku, $\mathrm{t}\mathrm{z}\mathrm{n}. \mathrm{z}$ przodu, $\mathrm{z}$ tylu, $\mathrm{z}$

prawej albo $\mathrm{z}$ lewej jego strony.

5. Uzasadnič, $\dot{\mathrm{z}}\mathrm{e}$ dla dowolnego $p$ oraz $x>-1$ prawdziwa jest nierównośč

$p^{2}+(1-p)^{2}x\displaystyle \geq\frac{x}{1+x}.$

Znalez/č $\mathrm{i}$ narysowač na pfaszczy $\acute{\mathrm{z}}\mathrm{n}\mathrm{i}\mathrm{e}$ zbiorów wszystkich par $(p,x)$, dla których $\mathrm{w}$ po-

$\mathrm{w}\mathrm{y}\dot{\mathrm{z}}$ szej nierówności ma miejsce równośč.

6. Trapez równoramienny ABCD $0$ polu $P$, ramieniu $c\mathrm{i}$ kącie ostrym przy podstawie $\alpha$

zgięto wzdluz jego osi symetrii $EF\mathrm{t}\mathrm{a}\mathrm{k}, \dot{\mathrm{z}}\mathrm{e}$ obie pofowy utworzyfy kąt $\alpha$. Obliczyč obję-

tośč powstałego $\mathrm{w}$ ten sposób wielościanu ABCDEF. Obliczyč tangens kąta nachylenia

do podstawy tej ściany bocznej, która nie jest prostopadła do podstawy. Sporz$\Phi$dzič

odpowiednie rysunki. Podač warunki istnienia rozwiązania.





PRACA KONTROLNA nr 6- POZIOM ROZSZERZONY

l. Rozwiązač równanie

$\sqrt{x^{2}-3}+2\sqrt{5-2x}=5-x.$

2. Wybrano losowo trzy krawędzie sześcianu. Obliczyč prawdopodobieństwo tego, $\dot{\mathrm{z}}\mathrm{e}\dot{\mathrm{z}}$ adne

dwie nie mają punktów wspólnych.

3. Gra $\mathrm{w}$ pary. $\mathrm{W}$ skarbonce znajduje się $\mathrm{d}\mathrm{u}\dot{\mathrm{z}}$ a liczba monet $0$ nominalach l $\mathrm{z}l$, 2 zł $\mathrm{i}$

5 $\mathrm{z}l. \mathrm{W}$ pierwszym kroku Jaś losuje trzy monety. Jesli wśród nich są dwie jednakowe,

to wrzuca je do skarbonki. $\mathrm{W}$ kolejnych krokach losuje ze skarbonki $\mathrm{k}\mathrm{a}\dot{\mathrm{z}}$ dorazowo tyle

monet, ile trzyma $\mathrm{w}$ rece, a nastepnie paryjednakowych monet wrzuca do skarbonki. Gra

kończy się, gdy wrzuci do skarbonki wszystkie monety. Obliczyč prawdopodobieństwo

tego, $\dot{\mathrm{z}}\mathrm{e}$ Jaś skończy grę: a) $\mathrm{w}$ drugim kroku; b) $\mathrm{w}$ drugim lub trzecim kroku.

4. Dane są wierzchołki $A(-3,2), C(4,2), D(0,4)$ trapezu równoramiennego ABCD, $\mathrm{w}$ któ-

rym $AB||CD$. Wyznaczyč współrzędne wierzcholka $B$ oraz równanie okręgu opisanego

na trapezie.

5. Udowodnič, $\dot{\mathrm{z}}\mathrm{e}$ dla $x>-1$ prawdziwa jest nierównośč podwójna

$1+\displaystyle \frac{x}{2}-\frac{x^{2}}{2}\leq\sqrt{1+x}\leq 1+\frac{x}{2}.$

Zilustrowač tę nierównośč odpowiednim rysunkiem.

6. $\mathrm{Z}$ dwóch przeciwlegfych wierzchołków prostokąta $0$ polu $P$, bdcego podstawą prosto-

padfościanu $0$ wysokości l, wystawiono po dwie przekątne sąsiednich ścian bocznych.

Wyrazič cosinus kąta pomiędzy plaszczyznami utworzonymi przez te pary przekątnych

jako funkcję sinusa kąta między nimi. Sporządzič rysunki.





XLII

KORESPONDENCYJNY KURS

Z MATEMATYKI

marzec 2013 r.

PRACA KONTROLNA nr 7- POZIOM PODSTAWOWY

l. Wyznaczyč rozwiazanie ogólne równania

$\displaystyle \sin(2x+\frac{\pi}{3})=\cos(x-\frac{\pi}{6}),$

a następnie podač rozwiązania $\mathrm{w}$ przedziale $[-2\pi,2\pi].$

2. Wyrazenie

$(\displaystyle \frac{a-2b}{\sqrt[3]{a^{2}}-\sqrt[3]{4b^{2}}}+\frac{\sqrt[3]{2a^{2}b}+\sqrt[3]{4ab^{2}}}{\sqrt[3]{a^{2}}+\sqrt[3]{4b^{2}}+\sqrt[3]{16ab}})$ : $\displaystyle \frac{a\sqrt[3]{a}+b\sqrt[3]{2b}+b\sqrt[3]{a}+a\sqrt[3]{2b}}{a+b}$

sprowadzič do najprostszej postaci. Przy jakich zalozeniach ma ono sens?

3. Narysowač wykres funkcji $f(x) =2|x|-\sqrt{x^{2}+4x+4}$ oraz wyznaczyč najmniejszą $\mathrm{i}$

największą wartośč funkcji $|f(x)|\mathrm{w}$ przedziale [-1, 2]. D1a jakiego $m$ pole figury ograni-

czonej wykresem funkcji $|f(x)|\mathrm{i}\mathrm{p}\mathrm{r}\mathrm{o}\mathrm{s}\mathrm{t}_{\Phi}y=m$ równe jest 16?

4. Rozwiązač układ równań

$\left\{\begin{array}{l}
x^{2}-4y^{2}+8y=4\\
x^{2}+y^{2}-2y=4
\end{array}\right.$

Podač interpretację geometryczną tego ukladu $\mathrm{i}$ obliczyč pole czworokata, którego wierz-

chofkami są cztery punkty będące jego rozwiązaniem.

5. $\mathrm{W}$ trapezie równoramiennym ABCD, $\mathrm{w}$ którym $BC||AD$ dane są $\vec{AB} = [1,-2]$ oraz

$\vec{AD}=[1$, 1$]$. Obliczyč pole trapezu $\mathrm{i}$ wyznaczyč $\mathrm{k}_{\Phi^{\mathrm{t}}}$ między jego przekątnymi.

6. $\mathrm{W}$ ostrosłupie prawidłowym trójkątnym cosinus kata nachylenia ściany bocznej do pod-

stawy równy jest $\displaystyle \frac{1}{9}$. Obliczyč stosunek pola powierzchni cafkowitej do pola podstawy.

Wykorzystując wzór $\sin 2\alpha=2\sin\alpha\cos\alpha$, wyznaczyč sinus kąta między ścianami bocz-

nymi tego ostrosfupa. Sporządzič rysunki.





PRACA KONTROLNA nr 7- POZIOM ROZSZERZONY

l. Rozwiązač równanie

$\displaystyle \sin x+\cos x=\frac{\cos 2x}{\sin 2x-1}$

2. Wyrazenie

$w(x,y)=\displaystyle \frac{x}{x^{3}+x^{2}y+xy^{2}+y^{3}}+\frac{y}{x^{3}-x^{2}y+xy^{2}-y^{3}}+\frac{1}{x^{2}-y^{2}}-\frac{1}{x^{2}+y^{2}}-\frac{x^{2}+2y^{2}}{x^{4}-y^{4}}$

doprowadzič do najprostszej postaci. Przy jakich zalozeniach ma ono sens? Obliczyč

$w(\cos 15^{\mathrm{o}},\sin 15^{\mathrm{o}}).$

3. Narysowač wykres funkcji

$f(x)=$

dla

dla

$x\leq 1,$

$x>1$

$\mathrm{i}$ posfugujac $\mathrm{s}\mathrm{i}\mathrm{e}$ nim wyznaczyč zbiór wartości funkcji $|f(x)|\mathrm{w}$ przedziale $[-\displaystyle \frac{1}{2},\frac{3}{2}].$

4. Rozwiązač ukfad równań

$\left\{\begin{array}{l}
y+x^{2}=4\\
4x^{2}-y^{2}+2y=1
\end{array}\right.$

Podač interpretację geometryczną tego ukladu $\mathrm{i}$ wykazač, $\dot{\mathrm{z}}\mathrm{e}$ cztery punkty, które są

jego rozwiązaniem, wyznaczaj $\Phi$ na płaszczy $\acute{\mathrm{z}}\mathrm{n}\mathrm{i}\mathrm{e}$ trapez równoramienny. Znalez/č równanie

okręgu opisanego na tym trapezie.

5. Odcinek $0$ końcach $A(0,7)\mathrm{i}B(5,2)$ jest przeciwprostokatna trójkąta prostokqtnego, któ-

rego wierzcholek $C\mathrm{l}\mathrm{e}\dot{\mathrm{z}}\mathrm{y}$ na prostej $x=3$. Posfugując się rachunkiem wektorowym ob-

liczyč cosinus kąta między dwusieczną kąta prostego a wysokością opuszczoną $\mathrm{z}$ wierz-

chofka $C.$

6. Pole powierzchni calkowitej ostrosfupa prawidfowego trójkątnego jest dziesięč razy więk-

sze $\mathrm{n}\mathrm{i}\dot{\mathrm{z}}$ pole jego podstawy. Wyznaczyč cosinus kąta między ścianami bocznymi oraz

stosunek objętości ostroslupa do objętości wpisanej $\mathrm{w}$ niego kuli.





XLII

KORESPONDENCYJNY KURS

Z MATEMATYKI

kwiecień 2013 r.

PRACA KONTROLNA $\mathrm{n}\mathrm{r} 8-$ POZIOM PODSTAWOWY

l. Cztery kolejne współczynniki wielomianu $f(x)=x^{3}+ax^{2}+bx+c$ tworzą $\mathrm{c}\mathrm{i}\otimes \mathrm{g}$ geome-

tryczny. Wiadomo, $\dot{\mathrm{z}}\mathrm{e}-3$ jest pierwiastkiem tego wielomianu. Wyznaczyč wspófczynniki

$a, b, c.$

2. Kolo $x^{2}+y^{2}+4x-2y-1\leq 0$ zostalo przesunięte $0$ wektor $\vec{w}=[3$, 3$]$. Znalez$\acute{}$č równanie

osi symetrii figury, która jest sumą kola $\mathrm{i}$ jego obrazu oraz obliczyč jej pole.

3. Podstawą ostroslupajest trójkąt $0$ bokach $\alpha, b, c$. Wszystkie kąty płaskie przy wierzchołku

ostroslupa są proste. Obliczyč objętośč ostroslupa.

4. Dane są punkty $A(0,2), B(4,4), C(3,6)$. Na prostej przechodzącej przez punkt $C$ rów-

noległej do prostej $AB$ znalez$\acute{}$č punkt $D$, który jest równo odlegly od punktów A $\mathrm{i}B.$

Wykazač, $\dot{\mathrm{z}}\mathrm{e}$ trójkąt $ABD$ jest prostokątny $\mathrm{i}$ napisač równanie okręgu opisanego na nim.

5. Wyznaczyč wartośč parametru $m$, dla którego równanie

$4x^{2}-2x\log_{2}m+1=0$

ma dwa rózne pierwiastki rzeczywiste $x_{1}, x_{2}$ spełniające warunek $x_{1}^{2}+x_{2}^{2}=1.$

6. Dane $\mathrm{s}\Phi$ funkcje $f(x)=4^{x-2}-7\cdot 3^{x-3}, g(x)=3^{3x+2}-5\cdot 4^{3x}$

Rozwiązač nierównośč

$f(x+3)>g(\displaystyle \frac{x}{3})$





PRACA KONTROLNA nr 8- POZIOM ROZSZERZONY

l. Niech $A$ bedzie wierzchołkiem kwadratu, a $M$ środkiem przeciwległego boku. Na prze-

$\mathrm{k}_{\Phi}$tnej kwadratu wychodzącej $\mathrm{z}$ wierzchofka $A$ wybrano punkt $P\mathrm{t}\mathrm{a}\mathrm{k}$, aby $|AP|=|MP|.$

Obliczyč, $\mathrm{w}$ jakim stosunku punkt $P$ dzieli przekatną kwadratu.

2. Stosując zasade indukcji matematycznej udowodnič nierównośč

$\left(\begin{array}{l}
2n\\
n
\end{array}\right) \displaystyle \leq\frac{4^{n}}{\sqrt{2n+2}},$

$n\geq 1.$

3. Wyznaczyč równanie okręgu $0$ środku lezącym na prostej $y-x=0$ oraz stycznego do

prostej $y-3=0\mathrm{i}$ do okręgu $x^{2}+y^{2}-4x+3=0$. Sporządzič rysunek.

4. Liczba $-2$ jest pierwiastkiem dwukrotnym wielomianu $w(x) = \displaystyle \frac{1}{2}x^{3}+ax^{2}+bx+c,$

a punkt $\mathrm{S}(-1,y_{0})$ jest środkiem symetrii wykresu $w(x)$. Wyznaczyč $a, b, c, y_{0}$ oraz trzeci

pierwiastek. Sporzqdzič wykres $w(x)\mathrm{w}$ przedziale $[-3,\displaystyle \frac{3}{2}].$

5. Wycinek kofa $0$ promieniu $3R\mathrm{i}\mathrm{k}_{\Phi}\mathrm{c}\mathrm{i}\mathrm{e}$ środkowym $\alpha$ zwinięto $\mathrm{w}$ powierzchnię boczną

stozka $S_{1}$. Podobnie, wycinek koła $0$ promieniu $R\mathrm{i}$ kqcie środkowym $ 3\alpha$ zwinieto $\mathrm{w}$ po-

wierzchnię boczną stozka $S_{2}$. Następnie obydwa stozki $\mathrm{z}l_{\Phi}$czono podstawami $\mathrm{t}\mathrm{a}\mathrm{k}$, aby

miafy wspólną oś obrotu, a ich wierzchofki byly skierowane $\mathrm{w}$ przeciwnych kierunkach.

Obliczyč promień kuli wpisanej $\mathrm{w}$ otrzymaną brylę. Sporzqdzič rysunek.

6. Podač interpretację geometryczną równania $\sqrt{2x+4}=mx+m+1\mathrm{z}$ parametrem $m.$

Graficznie $\mathrm{i}$ analitycznie określič, dla jakich wartości $m$ równanie ma dwa pierwiastki

$x_{1}=x_{1}(m), x_{2}=x_{2}(m)$. Nie korzystając $\mathrm{z}$ metod rachunku rózniczkowego, wykazač, $\dot{\mathrm{z}}\mathrm{e}$

funkcja $f(m)=x_{1}(m)+x_{2}(m)$ jest malejąca oraz sporządzič jej wykres.





XLII

KORESPONDENCYJNY KURS

Z MATEMATYKI

$\mathrm{p}\mathrm{a}\acute{\mathrm{z}}$dziernik 2012 $\mathrm{r}.$

PRACA KONTROLNA $\mathrm{n}\mathrm{r} 2-$ POZIOM PODSTAWOWY

l. Firma budowlana podpisała umowe na modernizację odcinka autostrady $0$ długości 21

km $\mathrm{w}$ określonym terminie. Ze względu na zblizające się mistrzostwa świata $\mathrm{w}$ rzu-

cie telefonem komórkowym postanowiono zrealizowač zamówienie 10 dni wcześniej, co

oznaczafo koniecznośč zwiększenia średniej normy dziennej $0$ 5\%. $\mathrm{W}$ jakim czasie firma

zamierzafa pierwotnie zrealizowač to zamówienie?

2. Pan Kowalski zaciągnął $\mathrm{w}$ banku kredyt $\mathrm{w}$ wysokości 4000 zł oprocentowany na 16\% $\mathrm{w}$

skali roku. Zgodnie $\mathrm{z}$ umową będzie go spłacaf $\mathrm{w}$ czterech ratach co 3 miesiące, spfacając

za $\mathrm{k}\mathrm{a}\dot{\mathrm{z}}$ dym razem 1000zł oraz 4\% pozosta1ego zadłuzenia. I1e złotych ostatecznie zwróci

bankowi pan Kowalski?

3. Ile jest czterocyfrowych liczb naturalnych:

a) podzielnych przez 2, 31ub przez 5?

b) podzielnych przez dokfadnie dwie spośród powyzszych liczb?

4. Na paraboli $y=x^{2}-6x+11$ znalez/č taki punkt $C, \dot{\mathrm{z}}\mathrm{e}$ pole trójkąta $0$ wierzchołkach

$A=(0,3), B=(4,0), C$ jest najmniejsze.

5. Przy prostoliniowej ulicy (oś Ox) $\mathrm{w}$ punkcie $x=0$ zainstalowano parkomat. $\mathrm{W}$ punkcie

$x=1 \mathrm{m}\mathrm{o}\dot{\mathrm{z}}$ na korzystač $\mathrm{z}$ bankomatu, a $\mathrm{w}$ punkcie $x=-2$ jest wejście do galerii han-

dlowej. $\mathrm{W}$ którym punkcie $x$ ulicy nalezy zaparkowač samochód, aby droga przebyta

od samochodu do parkomatu $\mathrm{i}\mathrm{z}$ powrotem (bilet parkingowy nalez $\mathrm{y}$ pofozyč za szybą

pojazdu), następnie do bankomatu po pieniądze, stąd do galerii $\mathrm{i}$ na końcu $\mathrm{z}$ zakupami

do samochodu, byfa najkrótsza? Jaka będzie odpowied $\acute{\mathrm{z}}$, gdy wejście do galerii będzie $\mathrm{w}$

punkcie $x=2$? $\mathrm{W}$ obu przypadkach podač wzór $\mathrm{i}$ narysowač wykres funkcji określającej

droge przebytq przez klienta domu handlowego $\mathrm{w}$ zalezności od punktu zaparkowania

samochodu.

6. Wykonač działania $\mathrm{i}$ zapisač $\mathrm{w}$ najprostszej postaci wyrazenie

$w(a,b)= (\displaystyle \frac{a}{a^{2}-ab+b^{2}}-\frac{a^{2}}{a^{3}+b^{3}})$ : $(\displaystyle \frac{a^{3}-b^{3}}{\alpha^{3}+b^{3}}-\frac{\alpha^{2}+b^{2}}{a^{2}-b^{2}})$

Wykazač, $\dot{\mathrm{z}}\mathrm{e}$ dla dowolnych $a<0$ zachodzi nierównośč $w(-a,a^{-1})\geq 1$, a dla dowolnych

$a>0$ prawdziwa jest nierównośč $w(-a,a^{-1})\leq 1.$





PRACA KONTROLNA nr 2- POZIOM ROZSZERZONY

l. Rozwiązač nierównośč $\displaystyle \frac{1}{\sqrt{5+4x-x^{2}}}\geq\frac{1}{|x|-2}\mathrm{i}$ zbiór rozwiązań zaznaczyč na osi liczbo-

wej.

2. Dwaj rowerzyści wyjechali jednocześnie naprzeciw siebie $\mathrm{z}$ miast A $\mathrm{i}\mathrm{B}$ odlegfych $030$

kilometrów. Minęli się po godzinie $\mathrm{i}$ nie zatrzymując się podqzyli $\mathrm{z}$ tymi samymi pręd-

kościami $\mathrm{k}\mathrm{a}\dot{\mathrm{z}}\mathrm{d}\mathrm{y}\mathrm{w}$ swoim kierunku. Rowerzysta, który wyjechal $\mathrm{z}$ A dotarf do $\mathrm{B}$ póftorej

godziny wcześniej $\mathrm{n}\mathrm{i}\dot{\mathrm{z}}$ jego kolegajadący $\mathrm{z}\mathrm{B}$ dotarł do A. $\mathrm{Z}$ jakimi prędkościami jechali

rowerzyści?

3. Pan Kowalski zaciągnąf 3l grudnia $\mathrm{p}\mathrm{o}\dot{\mathrm{z}}$ yczkę 4000 z1otych $\mathrm{o}\mathrm{P}^{\mathrm{r}\mathrm{o}\mathrm{c}\mathrm{e}\mathrm{n}\mathrm{t}\mathrm{o}\mathrm{w}\mathrm{a}\mathrm{n}}\Phi^{\mathrm{W}}$ wysokości

16\% $\mathrm{w}$ skali roku. Zobowiqzał się splacič ją $\mathrm{w}$ ciągu roku $\mathrm{w}$ czterech równych ratach

platnych 30. marca, 30. czerwca, 30. września $\mathrm{i}30$. grudnia. Oprocentowanie $\mathrm{p}\mathrm{o}\dot{\mathrm{z}}$ yczki

liczy się od l stycznia, a odsetki od kredytu naliczane są $\mathrm{w}$ terminach pfatności rat.

Obliczyč wysokośč tych rat $\mathrm{w}$ zaokrągleniu do pełnych groszy.

4. Dla jakiego parametru $m$ równanie

$2x^{2}-(2m+1)x+m^{2}-9m+39=0$

ma dwa pierwiastki, $\mathrm{z}$ których jeden jest dwa razy większy $\mathrm{n}\mathrm{i}\dot{\mathrm{z}}$ drugi?

5. Ile jest liczb pięciocyfrowych podzielnych przez 9, które $\mathrm{w}$ rozwinięciu dziesiętnym mają:

a) obie cyfry 1, 2 $\mathrm{i}$ tylko $\mathrm{t}\mathrm{e}$? b) obie cyfry 1, 3 $\mathrm{i}$ tylko $\mathrm{t}\mathrm{e}$? c) wszystkie cyfry 1, 2, 3

$\mathrm{i}$ tylko $\mathrm{t}\mathrm{e}$? Odpowiedz/uzasadnič. $\mathrm{W}$ przypadku b) wypisač otrzymane liczby.

6. $\mathrm{Z}$ przystani A wyrusza $\mathrm{z}$ biegiem rzeki statek do przystani $\mathrm{B}$, odleglej od A $0140$ km. Po

uplywie l godziny wyrusza za nim fódz/ motorowa, dopędza statek, po czym wraca do

przystani A $\mathrm{w}$ tym samym momencie, $\mathrm{w}$ którym statek przybija do przystani B. Predkośč

łodzi $\mathrm{w}$ wodzie stojącej jest póltora raza większa $\mathrm{n}\mathrm{i}\dot{\mathrm{z}}$ prędkośč statku $\mathrm{w}$ wodzie stojącej.

Wyznaczyč te prędkości $\mathrm{w}\mathrm{i}\mathrm{e}\mathrm{d}\mathrm{z}\Phi^{\mathrm{C}}, \dot{\mathrm{z}}\mathrm{e}$ rzeka plynie $\mathrm{z}$ prędkością 4 $\mathrm{k}\mathrm{m}/$godz.





XLII

KORESPONDENCYJNY KURS

Z MATEMATYKI

listopad 2012 r.

PRACA KONTROLNA $\mathrm{n}\mathrm{r} 3-$ POZIOM PODSTAWOWY

1. $\mathrm{Z}$ danych Głównego Urzędu Statystycznego wynika, $\dot{\mathrm{z}}\mathrm{e}$ wzrost Produktu Krajowego Brut-

to (PKB) $\mathrm{w}$ Polsce $\mathrm{w}$ roku 2010 wyniósf 3,7\%, a $\mathrm{w}$ roku 2011- 4,3\%. Jaki powinien byč

wzrost PKB $\mathrm{w}$ roku 2012, by średni roczny wzrost PKB $\mathrm{w}$ tych trzech latach wyniósł

4\%? Podač wynik $\mathrm{z}$ dokładnością do 0, 001\%.

2. Czy liczby $\sqrt{2}$, 2, $2\sqrt{2}$ mogą byč wyrazami (niekoniecznie kolejnymi) ciągu arytme-

tycznego? Odpowiedz/uzasadnič.

3. Wielomian $W(x) = x^{5}+ax^{4}+bx^{3}+4x$ jest podzielny przez $(x^{2}-1)$. Wyznaczyč

wspólczynniki $a, b\mathrm{i}$ rozwiązač nierównośč $W(x-1)\leq W(x)\leq W(x+1).$

4. Niech $f(x)=\sqrt{x}, g(x)=x-2, h(x)=|x|$. Narysowač wykresy funkcji zlozonych: $f0h\circ g,$

$f$ o{\it g}o $h$, {\it go} $f\mathrm{o}h$, {\it goho} $f, h\mathrm{o}f\mathrm{o}g$ {\it oraz hogo} $f.$

5. Przyprostokątną trójkata prostokątnego $ABC$ jest odcinek AB $0$ końcach $A(-2,2) \mathrm{i}$

$B(1,-1)$, a wierzchofek $C$ trójk$\Phi$ta $\mathrm{l}\mathrm{e}\dot{\mathrm{z}}\mathrm{y}$ na prostej $3x-y= 14$. Wyznaczyč równanie

okręgu opisanego na tym trójkącie. Ile rozwiązań ma to zadanie? Sporządzič rysunek.

6. Na prostej $x+2y=5$ wyznaczyč punkty, $\mathrm{z}$ których okrag $(x-1)^{2}+(y-1)^{2}=1$ jest

widoczny pod kątem $60^{\mathrm{o}}$. Obliczyč pole obszaru ograniczonego fukiem okręgu $\mathrm{i}$ stycznymi

do niego poprowadzonymi $\mathrm{w}$ znalezionych punktach. Sporządzič rysunek.





PRACA KONTROLNA nr 3- POZIOM ROZSZERZONY

l. Pan Kowalski umieścił swoje oszczędności na dwu róznych lokatach. Pieniądze, otrzy-

mane jako honorarium za podręcznik, zfozyl na lokacie oprocentowanej $\mathrm{w}$ wysokości 7\%

$\mathrm{w}$ skali roku, a wynagrodzenie za cykl wykładów- na lokacie 9\%. Po roku jego dochód

był $030$ zlotych, a po dwu latach- $070$ złotych $\mathrm{w}\mathrm{y}\dot{\mathrm{z}}$ szy od dochodu, który uzyskałby

skfadając cafą sumę na lokacie 8\%. I1e pieniędzy otrzyma1 pan Kowa1ski za podręcznik,

a ile za wykłady?

2. Czy liczby $\sqrt{2}, \sqrt{3}$, 2 mogą byč wyrazami (niekoniecznie kolejnymi) $\mathrm{c}\mathrm{i}_{\Phi \mathrm{g}}\mathrm{u}$ arytmetycz-

nego? Odpowiedz/uzasadnič.

3. Niech $f(x)=2^{x}, g(x)=2-x, h(x)=|x|$. Narysowač wykresy funkcji złozonych $f\mathrm{o}g\mathrm{o}h$

oraz $g\mathrm{o}f\mathrm{o}h\mathrm{i}$ rozwiązač nierównośč $(f\mathrm{o}g\mathrm{o}h)(x)<6+(g\mathrm{o}f\mathrm{o}h)(x).$

4. Dane są punkty $A(1,2), B(3,1).$

takich, $\dot{\mathrm{z}}\mathrm{e}\mathrm{k}\mathrm{a}\mathrm{t}BCA$ ma miarę $45^{\mathrm{o}}$

Wyznaczyč równanie zbioru wszystkich punktów C

5. Liczby: $a_{1}=\log_{(3-2\sqrt{2})^{2}}(\sqrt{2}-1), a_{2}=\displaystyle \frac{1}{2}\log_{\frac{1}{3}}\frac{\sqrt{3}}{6}, \alpha_{3}=3^{\log_{\sqrt{3}}\frac{\sqrt{6}}{2}}, a_{4}=\log_{(\sqrt{2}-1)}(\sqrt{2}+1),$

$a_{5}=(2^{\sqrt{2}+1})^{\sqrt{2}-1}, a_{6}=\log_{3}2$ są jedynymi pierwiastkami wielomianu $W(x)$, którego

wyraz wolny jest dodatni.

a) Które $\mathrm{z}$ tych pierwiastków są niewymierne? Odpowied $\acute{\mathrm{z}}$ uzasadnič.

b) Wyznaczyč dziedzinę funkcji $f(x)=\sqrt{W(x)}$, nie wykonując obliczeń przyblizonych.

6. Niech $f(x)=3(x+2)^{4}+x^{2}+4x+p$, gdzie $p$ jest parametrem rzeczywistym.

a) Uzasadnič, $\dot{\mathrm{z}}\mathrm{e}$ wykres funkcji $f(x)$ jest symetryczny względem prostej $x=-2.$

b) Dla jakiego parametru $p$ najmniejszą wartością funkcji $f(x)$ jest $y=-2$ ?

Odpowiedz/uzasadnič, nie stosując metod rachunku rózniczkowego.

c) Określič liczbę rozwiqzań równania $f(x)=0\mathrm{w}$ zalezności od parametru $p.$





XLII

KORESPONDENCYJNY KURS

Z MATEMATYKI

grudzień 2012 r.

PRACA KONTROLNA $\mathrm{n}\mathrm{r} 4-$ POZIOM PODSTAWOWY

l. Wyznaczyč wszystkie kąty $\alpha \mathrm{z}$ przedziału $[0,2\pi]$, dla których suma kwadratów pierwiast-

ków rzeczywistych równania $x^{2}+2x\sin\alpha-\cos^{2}\alpha=0$ jest równa co najwyzej 3.

2. Uzasadnič, $\dot{\mathrm{z}}\mathrm{e}$ suma średnic okręgu opisanego na trójkącie prostokqtnym $\mathrm{i}$ okręgu wpisa-

nego $\mathrm{w}$ ten trójkąt jest równa sumie długości przyprostokątnych. Znalez$\acute{}$č dlugości boków

trójkąta, $\mathrm{j}\mathrm{e}\dot{\mathrm{z}}$ eli promienie tych okręgów są równe $R=5\mathrm{i}r=2.$

3. Narysowač wykres funkcji $f(x)=\cos^{2}x+|\sin x|\sin x \mathrm{w}$ przedziale $[-2\pi,2\pi].$

a) Podač zbiór wartości $\mathrm{i}$ miejsca zerowe.

b) Wyznaczyč przedziafy monotoniczności.

c) Rozwiązač nierównośč $|f(x)|\displaystyle \geq\frac{1}{2}.$

4. $\mathrm{W}$ kwadracie $0$ boku dfugości $a$ narysowano cztery pólkola, których średnicami są boki

kwadratu. Pólkola przecinają $\mathrm{s}\mathrm{i}\mathrm{e}$ parami tworząc czterolistną rozetę. Obliczyč pole $\mathrm{i}$

obwód rozety.

5. Dach wiez $\mathrm{y}$ kościola ma kształt ostrosłupa, którego podstawą jest sześciokąt foremny $0$

boku 4 $\mathrm{m}$ a największy $\mathrm{z}$ przekrojów płaszczyzną zawierajqcq wysokośč jest trójkątem

równobocznym. Obliczyč kubaturę dachu wiez $\mathrm{y}$ kościofa. Ile 2-1itrowych puszek farby

antykorozyjnej trzeba kupič do pomalowania blachy, którą pokryty jest dach, $\mathrm{j}\mathrm{e}\dot{\mathrm{z}}$ eli wia-

domo, $\dot{\mathrm{z}}\mathrm{e}$ llitr farby wystarcza do pomalowania 6 $\mathrm{m}^{2}$ blachy $\mathrm{i}$ trzeba uwzględnič 8\%

farby $\mathrm{n}\mathrm{a}$ ewentualne straty.

6. Promień kuli opisanej na ostrosłupie prawidlowym czworokątnym wynosi $R$. Prosto-

padfa wyprowadzona ze środka kuli do ściany bocznej ostroslupa tworzy $\mathrm{z}$ wysokością

ostrosłupa kąt $\alpha$. Wyznaczyč wysokośč ostroslupa.





PRACA KONTROLNA nr 4- POZIOM ROZSZERZONY

l. Dla jakich katów $\alpha \mathrm{z}$ przedziału $[0,\displaystyle \frac{\pi}{2}]$ równanie $x^{2}\sin\alpha+x+\cos\alpha=0$ ma dwa róz-

ne pierwiastki rzeczywiste? Czy iloczyn pierwiastków równania $\mathrm{m}\mathrm{o}\dot{\mathrm{z}}\mathrm{e}$ byč równy $\sqrt{3}$?

Wyznaczyč wszystkie kąty $\alpha$, dla których suma pierwiastków jest większa od $-2.$

2. Przekrój ostroslupa prawidłowego czworokątnego plaszczyzną przechodzącą przez prze-

kątnq podstawy $\mathrm{i}$ wierzcholek ostrosłupajest trójkątem równobocznym. Wyznaczyč sto-

sunek promienia kuli wpisanej $\mathrm{w}$ ostrosfup do promienia kuli opisanej na ostrosfupie.

3. Narysowač wykres funkcji $f(x)=\displaystyle \frac{\sin 2x-|\sin x|}{\sin x}.$

równośč $f(x)<2(\sqrt{2}-1)\cos^{2}x.$

$\mathrm{W}$ przedziale $[0,2\pi]$ rozwiązač nie-

4. $\mathrm{C}\mathrm{z}\mathrm{w}\mathrm{o}\mathrm{r}\mathrm{o}\mathrm{k}_{\Phi^{\mathrm{t}}}$ wypukfy ABCD, $\mathrm{w}$ którym $AB=1, BC=2, CD=4, DA=3$ jest wpisany

$\mathrm{w}$ okrąg. Obliczyč promień $R$ tego okręgu. Sprawdzič, czy $\mathrm{w}$ ten czworokąt $\mathrm{m}\mathrm{o}\dot{\mathrm{z}}$ na wpisač

okrqg. $\mathrm{J}\mathrm{e}\dot{\mathrm{z}}$ eli $\mathrm{t}\mathrm{a}\mathrm{k}$, to obliczyč jego promień.

5. $\mathrm{W}$ kole $K\mathrm{o}$ promieniu 4 cm narysowano 6 kó1 $0$ promieniu 2 cm $\mathrm{P}^{\mathrm{r}\mathrm{z}\mathrm{e}\mathrm{c}\mathrm{h}\mathrm{o}\mathrm{d}\mathrm{z}}\Phi^{\mathrm{c}\mathrm{y}\mathrm{c}\mathrm{h}}$ przez

środek kola $K\mathrm{i}$ stycznych do niego $\mathrm{t}\mathrm{a}\mathrm{k}$, aby środki tych sześciu kół były wierzchołkami

sześciok$\Phi$ta foremnego. Obliczyč pole $\mathrm{i}$ obwód figury, która jest $\mathrm{s}\mathrm{u}\mathrm{m}\Phi$ tych sześciu kóf.

6. Stosunek pola powierzchni bocznej stozka ściętego do pola powierzchni wpisanej $\mathrm{w}$ ten

stozek kuli wyrazič jako funkcję kata nachylenia tworzącej stozka do podstawy.





XLII

KORESPONDENCYJNY KURS

Z MATEMATYKI

styczeń 2013 r.

PRACA KONTROLNA $\mathrm{n}\mathrm{r} 5-$ POZIOM PODSTAWOWY

l. Między $\mathrm{k}\mathrm{a}\dot{\mathrm{z}}$ de dwa kolejne wyrazy pięcioelementowego ciągu arytmetycznego wstawiono

$m$ liczb, otrzymując ciąg arytmetyczny, którego sumajest 13 razy większa $\mathrm{n}\mathrm{i}\dot{\mathrm{z}}$ suma wyj-

ściowego ciągu. Obliczyč $m$. Jaką jednakową ilośč liczb nalez $\mathrm{y}$ wstawič miedzy $\mathrm{k}\mathrm{a}\dot{\mathrm{z}}$ de dwa

kolejne wyrazy $n$ elementowego ciągu arytmetycznego, aby otrzymač ciąg arytmetyczny

$0$ sumie $n$ razy większej $\mathrm{n}\mathrm{i}\dot{\mathrm{z}}$ suma wyjściowego ciągu?

2. Linie kolejowe malują wagony klasy standard na niebiesko, klasy komfort na rózowo,

a klasy biznes na szaro. Na ile sposobów $\mathrm{m}\mathrm{o}\dot{\mathrm{z}}$ na zestawič skfad pięciowagonowy, który

zawiera co najmniej jeden wagon $\mathrm{k}\mathrm{a}\dot{\mathrm{z}}$ dej klasy, a kolejnośč wagonów jest istotna?

3. Niech $n$ będzie liczbą naturalną. $\mathrm{W}$ przedziale $[0,2\pi]$ rozwiqzač równanie

$1+\cos^{2}x+\cos^{4}x+\cdots+\cos^{2n}x=2-\cos^{2n}x.$

4. Zawodnik przebiegf równym tempem pierwsze l0 km biegu maratońskiego $(42\mathrm{k}\mathrm{m})\mathrm{w}$ cza-

sie 45 minut, a $\mathrm{k}\mathrm{a}\dot{\mathrm{z}}\mathrm{d}\mathrm{y}$ kolejny kilometr pokonywał $\mathrm{w}$ czasie $0$ 5\% dluzszym $\mathrm{n}\mathrm{i}\dot{\mathrm{z}}$ poprzedni.

Sprawdzič, czy zawodnik zmieścil się $\mathrm{w}$ sześciogodzinnym limicie czasowym.

5. Rozwiązač nierównośč

$\log_{2}(x+2)-\log_{4}(4-x^{2})\geq 0.$

6. Niech $A=\{(x,y):|x|+2|y|\leq 2\}$. Zbiór $B$ powstaje przez obrót figury $A0$ kąt $\displaystyle \frac{\pi}{2} (\mathrm{w}$

kierunku przeciwnym do ruchu wskazówek zegara) wokół początku układu współrzęd-

nych. Starannie narysowač zbiory $A\cup B$ oraz $A\triangle B=(A\backslash B)\cup(B\backslash A)\mathrm{i}$ obliczyč ich

pola.





PRACA KONTROLNA nr 5- POZIOM ROZSZERZONY

l. Zbadač, dla jakich argumentów funkcja

przyjmuje wartości ujemne.

$g(x)=2^{x^{3}-5} 3^{7x^{2}}\cdot 4^{7x-1}-2^{7x^{2}+1}$

$3^{x^{3}-2} 9^{7x-3}$

2. Rozwiązač nierównośč

$2^{-\sin x}+2^{-2\sin x}+2^{-3\sin x}+\ldots\leq\sqrt{2}+1,$

której lewa strona jest sumą nieskończonego ciągu geometrycznego.

3. Podač dziedzinę i wyznaczyč wszystkie miejsca zerowe funkcji

$f(x)=\displaystyle \log_{x+1}(x-1)-\log_{x+1}(2x-\frac{2}{x})+1.$

4. Dany jest ciąg liczbowy $(a_{n}), \mathrm{w}$ którym $\mathrm{k}\mathrm{a}\dot{\mathrm{z}}\mathrm{d}\mathrm{y}$ wyraz jest sumą podwojonego wyrazu

poprzedniego $\mathrm{i}4$, a jego czwarty wyraz wynosi 36. Podač wzór na n-ty wyraz $\mathrm{c}\mathrm{i}_{\Phi \mathrm{g}}\mathrm{u}\mathrm{i}$

udowodnič go, wykorzystując zasadę indukcji matematycznej.

5. Niech $A=\{(x,y):|x|+2|y|\leq 2\}$. Zbiór $B$ otrzymano przez obrót $A0$ kąt $\displaystyle \frac{\pi}{2}(\mathrm{w}$ kierunku

przeciwnym do ruchu wskazówek zegara) wokóf $\mathrm{P}^{\mathrm{o}\mathrm{c}\mathrm{z}}\Phi^{\mathrm{t}\mathrm{k}\mathrm{u}}$ ukladu wspólrzędnych, a zbiór

C- przez obrót zbioru $A\cup B0$ kąt $\displaystyle \frac{\pi}{4}$ wokól początku układu współrzędnych. Wykonač

staranny rysunek zbioru $A\cup B\cup C$ oraz obliczyč jego pole.

6. Boki $\triangle ABC$ zawarte są $\mathrm{w}$ prostych $y=2x+m, y=mx+1$ oraz $2y=2-x$. Podač wartośč

rzeczywistego parametru $m\displaystyle \in(-\frac{1}{2},2)$, dla której pole rozwazanego trójkąta wynosi $\displaystyle \frac{1}{5}.$

Dla wyznaczonego $m$ wykonač staranny rysunek (przyjąč jednostkę równą 3 cm).







XLIII

KORESPONDENCYJNY KURS

Z MATEMATYKI

wrzesień 2013 r.

PRACA KONTROLNA $\mathrm{n}\mathrm{r} 1 -$ POZIOM PODSTAWOWY

l. Wzrost kursu Euro $\mathrm{w}$ stosunku do złotego spowodował podwyzkę ceny nowego modelu

Volvo $0$ 5\%. Poniewaz nie było popytu na te samochody, więc postanowiono ustalič cenę

promocyjną na poziomie odpowiadającym wzrostowi kursu Euro $0$ 2\%.

a) $\mathrm{O}$ ile procent cena promocyjna byfa $\mathrm{n}\mathrm{i}\dot{\mathrm{z}}$ sza od ceny wynikającej $\mathrm{z}$ faktycznego wzro-

stu kursu Euro $\mathrm{w}$ stosunku do zfotego? Wynik podač $\mathrm{z}$ dokladności$\Phi$ do l promila.

b) Ile pan Kowalski stracif na wzroście kursu Euro, a ile zyskal dzięki cenie promo-

cyjnej, $\mathrm{j}\mathrm{e}\dot{\mathrm{z}}$ eli kupił samochód za 56000? Rachunki prowadzič $\mathrm{z}$ dokładnością do

cafkowitych złotych.

2. $\mathrm{Z}$ obozu A do obozu $\mathrm{B}\mathrm{m}\mathrm{o}\dot{\mathrm{z}}$ na przejśč drogą $\dot{\mathrm{z}}$ wirową lub ściezką przez las, która jest $0$

sześč kilometrów krótsza $\mathrm{n}\mathrm{i}\dot{\mathrm{z}}$ droga $\dot{\mathrm{z}}$ wirowa. Bolek wyszedł $\mathrm{z}$ A $\mathrm{i}$ idąc ściezką $\mathrm{z}$ prędkościq

4 $\mathrm{k}\mathrm{m}/\mathrm{h}$ dotarf do $\mathrm{B} 1$ godzinę wcześniej $\mathrm{n}\mathrm{i}\dot{\mathrm{z}}$ Lolek, który $\mathrm{w}$ tym samym momencie

wyruszył drogą $\dot{\mathrm{z}}$ wirową. Znalez/č dlugośč ściezki, wiedząc, $\dot{\mathrm{z}}\mathrm{e}$ prędkośč, $\mathrm{z}$ jaką porusza

się Lolek wyraza się liczbą całkowitą.

3. Ile jest naturalnych liczb pięciocyfrowych, $\mathrm{w}$ których zapisie dziesiętnym występują do-

kładnie dwa 0 $\mathrm{i}$ dokladnie jedna cyfra l?

4. Niech $A=\displaystyle \{x\in \mathbb{R}:\frac{1}{x^{2}+2}\geq\frac{1}{6-3x}\}$ oraz $B=\{x\in \mathbb{R}:|x-2|+|x+2|<6\}.$

Znalez$\acute{}$č $\mathrm{i}$ zaznaczyč na osi liczbowej zbiory $A, B$ oraz $(A\backslash B)\cup(B\backslash A).$

5. Uprościč wyrazenie

-$\sqrt{}$6{\it a}5-1$\sqrt{}$6{\it a}2{\it b}3($\sqrt{}$6{\it a}5--$\sqrt{}$6{\it ba})--{\it aa}$+-\sqrt{}${\it bab}

dla $a, b$, dla których ma ono sens, a następnie obliczyč jego wartośč, przyjmując

$a=4-2\sqrt{3}\mathrm{i} b=3+2\sqrt{2}.$

6. Grupa l75 robotników firmy pana Kowalskiego miala wykonač pewien odcinek autostra-

dy A4 $\mathrm{w}$ określonym terminie. Po upływie 30 dni wspó1nej pracy okazało się, $\dot{\mathrm{z}}\mathrm{e}$ musi

$\mathrm{m}\mathrm{o}\dot{\mathrm{z}}$ liwie szybko dokonač naprawy oddanego wcześniej odcinka autostrady A2. $\mathrm{W}\mathrm{z}\mathrm{w}\mathrm{i}_{\Phi}\mathrm{z}$-

ku $\mathrm{z}$ tym codziennie odsyłano do tego zadania kolejnych 3 robotników, wskutek czego

prace przy budowie autostrady A4 zakończono $\mathrm{z} 21$-dniowym opóznieniem. $\mathrm{W}$ jakim

czasie planowano pierwotnie wybudowač dany odcinek autostrady A4?




PRACA KONTROLNA nr l- POZIOM ROZSZERZONY

l. Pan Kowalski zaciągną131 grudnia $\mathrm{p}\mathrm{o}\dot{\mathrm{z}}$ yczkę 4000 złotych oprocentowaną $\mathrm{w}$ wysokości

18\% $\mathrm{w}$ skali roku. Zobowiązaf się splacič ją $\mathrm{w}$ ciągu roku $\mathrm{w}$ trzech równych ratach

płatnych 30 kwietnia, 30 sierpnia $\mathrm{i}30$ grudnia. Oprocentowanie $\mathrm{p}\mathrm{o}\dot{\mathrm{z}}$ yczki liczy się od l

stycznia, a odsetki od kredytu naliczane są $\mathrm{w}$ terminach płatności rat. Obliczyč wysokośč

tych rat $\mathrm{w}$ zaokrągleniu do pefnych groszy.

2. $\mathrm{Z}$ dwu stacji wyjezdzają jednocześnie naprzeciw siebie dwa pociągi. Pierwszy jedzie $\mathrm{z}$

prędkości$\Phi$ 15 $\mathrm{k}\mathrm{m}/\mathrm{h}$ większą $\mathrm{n}\mathrm{i}\dot{\mathrm{z}}$ drugi $\mathrm{i}$ spotykają się po 40 minutach. Gdyby drugi

pociąg wyjechaf $09$ minut wcześniej, to, jadąc $\mathrm{z}$ tymi samymi prędkościami, spotkalyby

się $\mathrm{w}$ połowie drogi. Znalez$\acute{}$č odległośč między miejscowościami oraz prędkości $\mathrm{k}\mathrm{a}\dot{\mathrm{z}}$ dego $\mathrm{z}$

pociągów.

3. Ile jest liczb pięciocyfrowych podzielnych przez 9, które $\mathrm{w}$ rozwinięciu dziesiętnym mają:

a) obie cyfry 1, 2 $\mathrm{i}$ tylko $\mathrm{t}\mathrm{e}$? b) obie cyfry 2, 3 $\mathrm{i}$ tylko $\mathrm{t}\mathrm{e}$? c) wszystkie cyfry 1, 2, 3

$\mathrm{i}$ tylko $\mathrm{t}\mathrm{e}$? Odpowiedz/uzasadnič.

4. Narysowač na płaszczyz/nie zbiór $A=\{(x,y):\sqrt{-2x-x^{2}}\leq y\leq\sqrt{3}|x+1|\}$

jego pole.

i obliczyč

5. Uprościč wyrazenie (dla a, b, dla których ma ono sens)

$(\displaystyle \frac{\sqrt[6]{b}}{\sqrt{b}-\sqrt[6]{a^{3}b^{2}}}-\frac{a}{\sqrt{ab}-a\sqrt[3]{b}})[\frac{\sqrt[6]{a}}{\sqrt{b}(\sqrt[6]{a^{5}}-\sqrt[3]{a}\sqrt{b})}(\sqrt[6]{a^{5}}-\frac{b}{\sqrt[6]{a}})-\frac{\sqrt[6]{a}(\alpha-b)}{a\sqrt{b}+b\sqrt{a}}],$

a następnie obliczyč jego wartośč dla $a=6\sqrt{3}-10$

i

$b=10+6\sqrt{3}$

6. Dwaj turyści wyruszyli jednocześnie: jeden $\mathrm{z}$ punktu $A$ do punktu $B$, drugi-z $B$ do $A.$

$K\mathrm{a}\dot{\mathrm{z}}\mathrm{d}\mathrm{y}\mathrm{z}$ nich szedf ze stafą prędkością $\mathrm{i}$ dotarlszy do mety, natychmiast ruszaf $\mathrm{w}$ drogę

powrotną. Pierwszy raz minęli się $\mathrm{w}$ odległości 12 km od punktu $B$, drugi- po uplywie

6 godzin od momentu pierwszego spotkania-w odległości 6 km od punktu $A$. Obliczyč

odlegfośč punktów $A\mathrm{i}B\mathrm{i}$ prędkości, $\mathrm{z}$ jakimi poruszali się turyści.





XLIII

KORESPONDENCYJNY KURS

Z MATEMATYKI

luty 2014 r.

PRACA KONTROLNA $\mathrm{n}\mathrm{r} 6-$ POZIOM PODSTAWOWY

l. Rozwiąz równanie

2 $($log2 $(2-x))^{2}-3\log_{2}(2-x)-2=0.$

2. Rozwiąz nierównośč wykfadniczą

$4^{\frac{1}{2}x^{2}-x}\cdot 3^{x^{2}+7x-2}\leq 9^{x^{2}+2x}\cdot 2^{x-2}$

3. Określ dziedzinę funkcji $f(x)=\displaystyle \frac{-1}{1-\sqrt{5-x^{2}}}-1$. Dlajakich argumentów funkcja przyjmuje

wartości ujemne?

4. $\mathrm{W}$ przedziale $[0,2\pi]$ wyznacz wszystkie liczby spefniające równanie

$\mathrm{t}\mathrm{g}^{2}x=8|\cos x|-1.$

5. Oblicz pole ośmiokąta będącego wspólna cześcią kwadratu $0$ boku dfugości 4 oraz jego

obrazu $\mathrm{w}$ obrocie $0$ kąt $\displaystyle \frac{\pi}{4}$ względem środka kwadratu. Wyznacz promień okręgu opisanego

na tym ośmiokącie $\mathrm{i}\mathrm{s}$porząd $\acute{\mathrm{z}}$ rysunek.

6. Dane $\mathrm{s}\Phi$ punkty $A(0,-2)$ oraz $B(4,0)$. Wyznacz wszystkie punkty $P\mathrm{l}\mathrm{e}\dot{\mathrm{z}}$ ące na paraboli

$y=x^{2}$, dla których $\triangle ABP$ jest prostokątny. Sporząd $\acute{\mathrm{z}}$ rysunek.





PRACA KONTROLNA nr 6- POZIOM ROZSZERZONY

l. Liczby $a_{1}, a_{2}, \ldots, a_{n}$, gdzie $n$ jest pewną liczba parzystą, tworzą ciąg arytmetyczny

$0$ sumie 15. Suma wszystkich wyrazów $0$ numerach parzystych $\mathrm{w}$ tym $\mathrm{c}\mathrm{i}_{\Phi \mathrm{g}}\mathrm{u}$ wynosi 0,

a iloczyn $a_{1}a_{2}=150$. Jakie to liczby?

2. Rozwiąz nierównośč logarytmiczną

$\log_{3}(x^{3}-x^{2}-4x-2)\leq\log_{\sqrt{3}}\sqrt{x+1}.$

3. Rozwiąz nierównośč trygonometryczną

$1-2\displaystyle \sin^{2}2x+4\sin^{4}2x-8\sin^{6}2x+\cdots>\frac{1}{3-2\sin^{2}x},$

której lewa strona jest sumą nieskończonego ciągu geometrycznego. Zaznacz dziedzinę

$\mathrm{i}$ zbiór rozwiązań nierówności na kole trygonometrycznym.

4. Kwadrat $0$ boku długości 4 obrócono $0$ kąt $\displaystyle \frac{\pi}{6}$ względem środka kwadratu, $\mathrm{w}$ kierunku

przeciwnym do ruchu wskazówek zegara. Oblicz pole wspólnej części kwadratu wyjścio-

wego $\mathrm{i}$ jego obrazu $\mathrm{w}$ tym obrocie. Sporząd $\acute{\mathrm{z}}$ rysunek.

5. Wyznacz równania tych stycznych do okręgu $x^{2}+y^{2}=1$, które $\mathrm{w}$ przecięciu $\mathrm{z}$ okregiem

$x^{2}-16x+y^{2}+39=0$ tworzą cięciwy dfugości 8. $\mathrm{s}_{\mathrm{P}^{\mathrm{o}\mathrm{r}\mathrm{z}}\Phi^{\mathrm{d}\acute{\mathrm{z}}\mathrm{r}\mathrm{y}\mathrm{s}\mathrm{u}\mathrm{n}\mathrm{e}\mathrm{k}}}.$

6. Wyznacz $\mathrm{i}$ narysuj funkcję $g(m)$ określającą liczbę rozwiązań równania

$(m-1)\displaystyle \frac{1}{4^{x}}+(m+1)2^{1-x}=2-m$

$\mathrm{w}$ zalezności od rzeczywistego parametru $m.$





XLIII

KORESPONDENCYJNY KURS

Z MATEMATYKI

marzec 2014 r.

PRACA KONTROLNA nr 7- POZIOM PODSTAWOWY

l. Rozwiązač nierównośč $\displaystyle \frac{1}{|x-1|}\leq x+3\mathrm{i}$ podač jej interpretację graficzną.

2. $\mathrm{W}$ przedziale $[0,2\pi]$ rozwiązač nierównośč 2 $\sin^{2}x>1+\cos x$. Zbiór rozwiązań zaznaczyč

na kole trygonometrycznym.

3. Znalez/č równanie okręgu stycznego do obu osi ukfadu wspófrzędnych $\mathrm{i}$ do dodatniej

gałęzi hiperboli $y=\displaystyle \frac{1}{x}$. Sporzqdzič rysunek.

4. Zaznaczyč na płaszczy $\acute{\mathrm{z}}\mathrm{n}\mathrm{i}\mathrm{e}$ zbiory $A = \{(x,y):1-\sqrt{2|x|-x^{2}}\leq|y|\leq 1+\sqrt{2-|x|}\}$

oraz $B=\{(x,y):|x|\leq 1,|y|\leq 1\}\mathrm{i}$ obliczyč pole figury $B\backslash A.$

5. Trapez prostokątny, $\mathrm{w}$ którym stosunek dfugości podstaw wynosi 3 : 2, jest opisany na

okręgu $0$ promieniu $r$. Wyznaczyč stosunek pola koła do pola trapezu oraz cosinus kąta

ostrego $\mathrm{w}$ tym trapezie.

6. Plaszczyzna przechodząca przez krawędz/ podstawy graniastosfupa prawidfowego trój-

kątnego, $\mathrm{w}$ którym wszystkie krawędzie są równe, dzieli ten graniastosłup na dwie bryły

$0$ tej samej objętości. Znalez/č kąt nachylenia plaszczyzny do podstawy. Sporządzič ry-

sunek.





PRACA KONTROLNA nr 7- POZIOM ROZSZERZONY

l. Rozwiązač nierównośč $\displaystyle \frac{3}{x^{2}-2x}\leq\frac{1}{|x|}.$

2. $\mathrm{W}$ przedziale $[0,2\pi]$ rozwiązač nierównośč

zaznaczyč na kole trygonometrycznym.

$\sqrt{\sin^{2}x-\sin x} \geq \cos x$. Zbiór rozwiazań

3. Znalez/č $\mathrm{i}$ zaznaczyč na płaszczy $\acute{\mathrm{z}}\mathrm{n}\mathrm{i}\mathrm{e}$ zbiór punktów $\{(x,y):\log_{x^{2}+y^{2}}(x+2y)\geq 1\}.$

4. Znalez/č równanie okręgu stycznego do osi $Ox$ oraz do obu gałęzi krzywej $0$ równaniu

$y=\displaystyle \frac{1}{x^{2}}$. Sporządzič rysunek. Wskazówka: Skorzystač $\mathrm{z}$ algebraicznego warunku styczności.

5. $\mathrm{W}$ trapezie opisanym na okręgu $0$ promieniu $r$ kąt ostry przy podstawie $1\mathrm{e}\dot{\mathrm{Z}}\mathrm{a}\mathrm{c}\mathrm{y}$ naprzeciw

krótszej przekątnej ma miarę $30^{o}$, a krótsza przekątna tworzy $\mathrm{z}$ podstawą $\mathrm{k}_{\Phi}\mathrm{t}45^{o}$ Obli-

czyč obwód trapezu oraz tangens kąta pomiędzy jego przekątnymi. Sporządzič rysunek.

6. Przez wierzchofek $S$ stozka poprowadzono plaszczyznę przecinajqcąjego podstawę wzdfuz

cięciwy $AB$. Miara kąta $\angle ASB$ jest równa $\alpha$, a miara kąta $\angle AOB$ jest równa $\beta$, gdzie

$O$ jest środkiem podstawy. Obliczyč sinus kata rozwarcia stozka. Podač warunki rozwią-

zalności zadania oraz warunek, aby kąt rozwarcia stozka byf $\mathrm{k}_{\Phi}\mathrm{t}\mathrm{e}\mathrm{m}$ prostym.





KORESPONDENCYJNY KURS

Z MATEMATYKI

$\mathrm{p}\mathrm{a}\acute{\mathrm{z}}$dziernik 2013 $\mathrm{r}.$

PRACA KONTROLNA $\mathrm{n}\mathrm{r} 2-$ POZIOM PODSTAWOWY

l. Rozwiązač nierównośč $x^{3}+nx^{2}-m^{2}x-m^{2}n\leq 0$, gdzie

$m=\displaystyle \frac{64^{\frac{1}{3}}\sqrt{2}+8^{\frac{1}{3}}\sqrt{64}}{\sqrt[3]{64\sqrt{8}}}$

oraz

{\it n}$=$ -($\sqrt{}$($\sqrt{}$24)1-64)(3-41.)2-7-25-$\sqrt{}$-441 3

2. $\dot{\mathrm{D}}$ la jakich wartości $\alpha\in[0,2\pi]$ liczby $\sin\alpha,  6\cos\alpha$, 6 tg $\alpha$ tworzą ciąg geometryczny?

3. Suma pewnej ilości kolejnych liczb naturalnych równa jest 33, a róznica kwadratów

najwiekszej $\mathrm{i}$ najmniejszej wynosi 55. Wyznaczyč te 1iczby.

4. Narysowač wykres funkcji

$f(x)=$

gdy

gdy

$|x-2|\leq 3,$

$|x-2|>3$

$\mathrm{i}$ wyznaczyč zbiór jej wartości. Dla jakich argumentów $x$ wykres funkcji $f(x) \mathrm{l}\mathrm{e}\dot{\mathrm{z}}\mathrm{y}$ pod

prostą $x-2y+10=0$ ? Zilustrowač rozwiązanie graficznie.

5. Dlajakiego parametru $m$ równanie $x^{2}-mx+m^{2}-2m+1=0$ ma dwa rózne pierwiastki

$\mathrm{w}$ przedziale $(0,2)$ ?

6. Wierzchołek $A$ wykresu funkcji $f(x)=ax^{2}+bx+c\mathrm{l}\mathrm{e}\dot{\mathrm{z}}\mathrm{y}$ na prostej $x=3\mathrm{i}$ jest odległy

od początku ukladu współrzędnych $05$. Pole trójkąta, którego wierzchofkami $\mathrm{s}\Phi$ punkty

przecięcia wykresu $\mathrm{z}$ osią $Ox$ oraz punkt $A$ równe jest 8. Podač wzór funkcji, której

wykres jest obrazem paraboli $f(x)\mathrm{w}$ symetrii względem punktu $(1,f(1)).$





PRACA KONTROLNA nr 2- POZIOM ROZSZERZONY

l. Obliczyč $a$ wiedząc, $\dot{\mathrm{z}}\mathrm{e}$ liczba $[\displaystyle \frac{2+9\sqrt{2}}{2\sqrt{2}-2}-\frac{1}{2}(2+\sqrt{2})^{2}]-(\frac{\sqrt[6]{32}}{2\sqrt{2}-2})^{3}$ jest miejscem zero-

wym funkcji $f(x)=2^{x}-a^{3}x.$

2. Dziesiąty wyraz rozwinięcia $(\displaystyle \frac{1}{\sqrt{x}}-\sqrt[3]{x})^{n}$ nie zawiera $x$. Wyznaczyč współczynniki przy

najnizszej $\mathrm{i}$ najwyzszej potędze $x.$

3. Wyznaczyč zbiór wartości funkcji $f(x)=(\displaystyle \log_{2}x)^{3}+\log_{2}\frac{x^{2}}{4}-1$ na przedziale (1, 2).

4. Tangens kąta ostrego $\alpha$ równy jest $\displaystyle \frac{a}{7b}$, gdzie

$a=(\sqrt{2}+1)^{3}-(\sqrt{2}-1)^{3}b=(\sqrt{\sqrt{2}+1}-\sqrt{\sqrt{2}-1})^{2}$

Wyznaczyč wartości pozostalych funkcji trygonometrycznych tego kąta oraz $\mathrm{k}_{\Phi^{\mathrm{t}\mathrm{a}}}2\alpha.$

Jaka jest miara kąta $\alpha$?

5. Trzy liczby $x<y<z$, których suma jest równa 93 tworza ciąg geometryczny. Te same

liczby $\mathrm{m}\mathrm{o}\dot{\mathrm{z}}$ na uwazač za pierwszy, drugi $\mathrm{i}$ siódmy wyraz ciqgu arytmetycznego. Jakie to

liczby?

6. Określič liczbę pierwiastków równania $(2m-3)x^{2}-4m|x|+m-1=0\mathrm{w}$ zalezności od

parametru $m.$





XLIII

KORESPONDENCYJNY KURS

Z MATEMATYKI

listopad 2013 r.

PRACA KONTROLNA $\mathrm{n}\mathrm{r} 3-$ POZIOM PODSTAWOWY

l. Wektory $\vec{AB} = [2$, 2$], \vec{BC} = [-2,3], \vec{CD} = [-2,-4]$ są bokami czworokąta ABCD.

Punkty $K\mathrm{i}M$ są środkami boków $CD$ oraz $AD$. Obliczyč pole trójkąta $KMB$ oraz jego

stosunek do pola całego czworokąta. Sporządzič rysunek.

2. Narysowač wykres funkcji

$f(x)=\displaystyle \frac{1}{\sqrt{1+\mathrm{t}\mathrm{g}^{2}x}}-\frac{1}{2},$

a nastepnie rozwiązač graficznie nierównośč $f(x)<0.$

3. Rozwiązač nierównośč $w(x-2)>w(x-1)$, gdzie

$w(x)=x^{4}-4x^{3}+5x^{2}-2x.$

4. Tangens kąta ostrego $\alpha$ równy jest

$\sqrt{7-4\sqrt{3}}.$

Wyznaczyč wartości pozostałych funkcji trygonometrycznych tego kąta. Wykorzystując

wzór $\sin 2\alpha=2\sin\alpha\cos\alpha$ wyznaczyč miarę kąta $\alpha.$

5. Punkt $B(2,6)$ jest wierzchołkiem trójkąta prostokątnego $0$ polu 25, którego przeciwpro-

stokątna zawarta jest $\mathrm{w}$ prostej $x-2y=0$. Obliczyč wysokośč opuszczoną na przeciw-

$\mathrm{P}^{\mathrm{r}\mathrm{o}\mathrm{s}\mathrm{t}\mathrm{o}\mathrm{k}}\Phi^{\mathrm{t}\mathrm{n}}\Phi^{\mathrm{i}}$ wyznaczyč wspófrzędne pozostafych wierzchofków trójkąta.

6. Dane są punkty $A(-1,-3) \mathrm{i}B(2,-2)$. Na paraboli $y=x^{2}-1$ znalez/č taki punkt $C,$

aby pole trójkąta $ABC$ byfo najmniejsze.





PRACA KONTROLNA nr 3- POZIOM ROZSZERZONY

l. Dla jakich wartości parametru $\alpha\in(0,2\pi)$ funkcja

$ f(x)=\sin\alpha\cdot x^{2}-x+\cos\alpha$

posiada minimum lokalne $\mathrm{i}$ wartośč najmniejsza funkcji jest ujemna?

2. Rozwiązač równanie

$\sqrt{3}+\mathrm{t}\mathrm{g}x=4\sin x.$

3. Wielomian $w(x)=x^{4}+3x^{3}+px^{2}+qx+r$ dzieli się przez $x-2$, a resztą $\mathrm{z}$ jego dzielenia

przez $x^{2}+x-2$ jest $-4x-12$. Wyznaczyč współczynniki $p, q, r\mathrm{i}$ rozwiązač nierównośč

$w(x)\geq 0.$

4. $\mathrm{W}$ czworokącie ABCD dane są $AD=a$ oraz $AB=2a$. Wiadomo, $\dot{\mathrm{z}}\mathrm{e}\vec{AC}=2\vec{AB}+3\vec{AD}$

oraz $\angle BAD=60^{\mathrm{o}}$. Stosując rachunek wektorowy obliczyč cosinus kąta $ABC$ oraz obwód

czworokąta. Rozwiązanie zilustrowač rysunkiem.

5. Punkt $P(-\displaystyle \sqrt{3},\frac{\sqrt{3}}{2})$ jest środkiem boku trójkąta równobocznego. Drugi bok trójkąta $\mathrm{l}\mathrm{e}\dot{\mathrm{z}}\mathrm{y}$

na prostej $y=2x$. Wyznaczyč współrzędne wszystkich wierzchołków trójkąta $\mathrm{i}$ obliczyč

jego pole. Sporzqdzič rysunek.

6. Wyznaczyč zbiór punktów płaszczyzny utworzonych przez środki wszystkich okręgów

stycznych jednocześnie do prostej $y=0$ oraz do okregu $x^{2}+y^{2}-4y+3=0$. Sporzqdzič

rysunek.





XLIII

KORESPONDENCYJNY KURS

Z MATEMATYKI

grudzień 2013 r.

PRACA KONTROLNA $\mathrm{n}\mathrm{r} 4-$ POZIOM PODSTAWOWY

l. Na półkuli $0$ promieniu $r$ opisano stozek $0$ kacie rozwarcia $2\alpha \mathrm{w}$ taki sposób, $\dot{\mathrm{z}}\mathrm{e}$ środek

podstawy stozka znajduje się $\mathrm{w}$ środku pófkuli. Oblicz objętośč $\mathrm{i}$ pole powierzchni stozka.

Jaki jest stosunek objetości stozka do objętości półkuli dla kąta rozwarcia $\pi/3$?

2. Kula jest styczna do wszystkich krawędzi czworościanu foremnego 0 krawędzi a. Oblicz

promień tej kuli.

3. $\mathrm{W}$ kwadrat ABCD wpisano kwadrat EFGH, który zajmuje 3/4 jego powierzchni. $\mathrm{W}$

jakim stosunku wierzchofki kwadratu EFGH dzielą boki kwadratu ABCD?

4. Niech $f(x)=4^{x+4}-7\cdot 3^{x+3}\mathrm{i}g(x)=6\cdot 4^{4x}-3^{4x+2}$

Rozwiąz nierównośč $f(x-3)\displaystyle \leq g(\frac{x}{4}).$

5. Znajd $\acute{\mathrm{z}}$ wymiary trapezu równoramiennego $0$ obwodzie $d\mathrm{i}$ kącie ostrym przy podstawie

$\alpha 0$ największym polu.

6. $\mathrm{W}$ trójkąt równoboczny $0$ boku $a$ wpisujemy trójkąt, którego wierzchołkami są środki

boków naszego trójkąta. Wpisany trójkat kolorujemy na niebiesko. Następnie $\mathrm{w}\mathrm{k}\mathrm{a}\dot{\mathrm{z}}\mathrm{d}\mathrm{y}\mathrm{z}$

niepokolorowanych trójkątów wpisujemy $\mathrm{w}$ ten sam sposób kolejne niebieskie trójk$\Phi$ty,

itd. Znajd $\acute{\mathrm{z}}$ sumę pól niebieskich trójkątów po $n$ krokach. Po ilu krokach niebieskie

trójkąty zajmą co najmniej 50\%, a po i1u- 75\% powierzchni wyjściowego trójkąta?





PRACA KONTROLNA nr 4- POZIOM ROZSZERZONY

1. $\mathrm{W}$ trójkącie prostokqtnym $ABC$ dane sq przyprostokątne $|AC| = 3$ oraz $|CB| = 4.$

Punkt $D$ jest spodkiem wysokości opuszczonej $\mathrm{z}$ wierzcholka kąta prostego, a $E\mathrm{i}$ {\it F}-

punktami przeciecia przeciwprostokątnej $\mathrm{z}$ dwusiecznymi kątów $ACD \mathrm{i} DCB$. Oblicz

długośč odcinka $EF$

2. Sześcian przecinamy pfaszczyzną, która przechodzi przez $\mathrm{P}^{\mathrm{r}\mathrm{z}\mathrm{e}\mathrm{k}}\Phi^{\mathrm{t}\mathrm{n}}\Phi$ jednej ze ścian oraz

środek krawędzi przeciwleglej ściany. Pod jakim kątem przecinają się przekątne otrzy-

manego przekroju?

3. Dane jest równanie kwadratowe $x^{2}+x(1-2^{m})+3(2^{m-2}-4^{m-1})=0$. Dla jakiego pa-

rametru $m$:

a) równanie ma pierwiastki róznych znaków?

b) suma kwadratów pierwiastków równania jest równa co najmniej l?

4. Pole powierzchni bocznej ostrosłupa prawidłowego $0$ podstawie trójkątnej wynosi $\sqrt{39}/4,$

a krawędz/ podstawy ma dlugośč l. Oblicz kąt nachylenia krawędzi bocznej do podstawy.

5. $\mathrm{W}$ trójkącie równoramiennym $ABC\mathrm{o}$ podstawie AB środkowe poprowadzone $\mathrm{z}$ wierz-

cholków $A\mathrm{i}B$ przecinajq się pod kątem prostym. Wyznacz sinus kąta $ACB.$

6. $\mathrm{W}$ trójkąt równoboczny $0$ boku $a$ wpisujemy okrąg. Następnie $\mathrm{w}\mathrm{k}\mathrm{a}\dot{\mathrm{z}}$ dym $\mathrm{z}$ trzech rogów

wpisujemy kolejny okrąg styczny do wpisanego okręgu oraz do dwóch boków trójkqta.

Postepujemy tak nieskończenie wiele razy. Oblicz sumę obwodów wpisanych okręgów.

Jaką powierzchnię trójkąta zajmują wpisane kola?





XLIII

KORESPONDENCYJNY KURS

Z MATEMATYKI

styczeń 2014 r.

PRACA KONTROLNA nr $5-$ POZIOM PODSTAWOWY

l. Na ile sposobów $\mathrm{z}$ grupy 10 ch1opców $\mathrm{i}8$ dziewcząt $\mathrm{m}\mathrm{o}\dot{\mathrm{z}}$ na wybrač dwie sześcioosobowe

druzyny do siatkówki $\mathrm{t}\mathrm{a}\mathrm{k}$, aby $\mathrm{w}\mathrm{k}\mathrm{a}\dot{\mathrm{z}}$ dej druzynie było po trzech chłopców?

2. Rzucamy pięcioma kostkami do gry. Co jest bardziej prawdopodobne: wyrzucenie tej

samej liczby oczek na co najmniej czterech kostkach, czy otrzymaniejednej $\mathrm{z}$ konfiguracji

1, 2, 3, 4, $5\mathrm{l}\mathrm{u}\mathrm{b}2$, 3, 4, 5, 6?

3. Wyznaczyč wszystkie wartości parametru $m$, dla których uklad równań

$\left\{\begin{array}{l}
x^{2}+y^{2}=2\\
4x^{2}-4y+m=0
\end{array}\right.$

ma dokładnie: a) jedno; b) $\mathrm{d}\mathrm{w}\mathrm{a};\mathrm{c}$) trzy rozwiązania. Uzasadnič odpowied $\acute{\mathrm{z}}$. Rozwiązanie

zilustrowač rysunkiem.

4. Obliczyč prawdopodobieństwo, $\dot{\mathrm{z}}\mathrm{e}$ dwie losowo wybrane rózne przekątne ośmiokąta fo-

remnego przecinają się.

5. Dany jest punkt $C(3,3)$. Na prostych $l$ : $x-y+1 = 0$ oraz $k$ : $x+2y-5 = 0,$

przecinających się $\mathrm{w}$ punkcie $M$, znalez/č odpowiednio punkty A $\mathrm{i}B\mathrm{t}\mathrm{a}\mathrm{k}$, aby kąt $\angle ACB$

był prosty, a czworokąt ABCM był trapezem. Sporzadzič rysunek.

6. $\mathrm{W}$ ostrosfupie prawidfowym trójk$\Phi$tnym dane są kąt pfaski $ 2\gamma$ przy wierzchofku oraz

odległośč $d$ krawędzi bocznej od przeciwleglej krawędzi podstawy. Obliczyč objetośč

ostroslupa. Następnie podstawič $2\displaystyle \gamma=\frac{\pi}{6}, d=\sqrt[4]{3} \mathrm{i}$ wynik podač $\mathrm{w}$ najprostszej postaci.





PRACA KONTROLNA nr $5 -$ POZIOM ROZSZERZONY

l. Na ile sposobów $\mathrm{m}\mathrm{o}\dot{\mathrm{z}}$ na ustawič $\mathrm{w}$ rzedzie trzy rózne pary butów $\mathrm{t}\mathrm{a}\mathrm{k}$, aby buty co naj-

mniej jednej pary stafy obok siebie, przy czym but lewy $\mathrm{z}$ lewej strony.

2. Stosując zasadę indukcji matematycznej, udowodnič nierównośč

$1+\displaystyle \sqrt{2}+\sqrt{3}+\ldots+\sqrt{n}\geq\frac{2}{3}n\sqrt{n+1},$

$n\geq 1.$

3. Pan Kowalski wyrusza $\mathrm{z}$ punktu $S$ na spacer po parku, którego plan jest przedstawiony
\begin{center}
\includegraphics[width=34.644mm,height=28.800mm]{./KursMatematyki_PolitechnikaWroclawska_2013_2014_page9_images/image001.eps}
\end{center}
na rysunku. Postanawia isc $\mathrm{k}\mathrm{a}\dot{\mathrm{z}}\mathrm{d}$ alejk co najwyzej jeden $\mathrm{r}\mathrm{a}\mathrm{z}.$

Obliczyc prawdopodobieństwo, $\dot{\mathrm{z}}\mathrm{e}$ przejdzie przez punkt $M,$

$\mathrm{j}\mathrm{e}\dot{\mathrm{z}}$ eli na $\mathrm{k}\mathrm{a}\dot{\mathrm{z}}$ dym skrzyzowaniu alejek wybiera kolejn (jeszcze

nie przebyt) alejk $\mathrm{z}$ tym samym prawdopodobienstwem lub

konczy spacer, gdy nie ma takiej alejki.

4. Uczeń zna odpowiedzi na 20 spośród 30 pytań egzaminacyjnych. Na egzaminie losuje dwa

pytania. $\mathrm{J}\mathrm{e}\dot{\mathrm{z}}$ eli odpowie poprawnie na oba, to egzamin $\mathrm{z}\mathrm{d}\mathrm{a}, \mathrm{j}\mathrm{e}\dot{\mathrm{z}}$ eli na $\dot{\mathrm{z}}$ adne, to nie $\mathrm{z}\mathrm{d}\mathrm{a},$

a $\mathrm{j}\mathrm{e}\dot{\mathrm{z}}$ eli na jedno, to wynik egzaminu rozstrzyga odpowied $\acute{\mathrm{z}}$ na dodatkowe wylosowane

pytanie. Obliczyč prawdopodobieństwo, $\dot{\mathrm{z}}\mathrm{e}$ uczeń zda egzamin.

5. $\mathrm{W}$ trójkąt $0$ wierzchofkach $A(-1,-1), B(3,1), C(1,3)$ wpisano kwadrat $\mathrm{t}\mathrm{a}\mathrm{k}, \dot{\mathrm{z}}\mathrm{e}$ dwa jego

wierzchołki lezą na boku $AB$ trójkąta. Wyznaczyč współrzędne wierzcholków kwadratu

oraz stosunek pola kwadratu do pola trójkąta. Sporz$\Phi$dzič rysunek.

6. Ostrosłup prawidlowy czworokątny ABCDS $0$ krawędzi podstawy $a$ ma pole powierzchni

całkowitej $5\alpha^{2}$ Środkiem krawedzi bocznej $AS$ jest punkt $M$. Obliczyč promień kuli

opisanej na ostroslupie ABCDM oraz cosinus kąta pomiędzy ścianami bocznymi $CDM$

oraz $BCM.$







XLV

KORESPONDENCYJNY KURS

Z MATEMATYKI

$\mathrm{p}\mathrm{a}\acute{\mathrm{z}}$dziernik 2015 $\mathrm{r}.$

PRACA KONTROLNA $\mathrm{n}\mathrm{r} 2-$ POZIOM PODSTAWOWY

l. Czy suma długości przekatnych kwadratów $0$ polach 10 $\mathrm{i} \displaystyle \frac{21}{2}$ jest większa od długości

$\mathrm{P}^{\mathrm{r}\mathrm{z}\mathrm{e}\mathrm{k}}\Phi^{\mathrm{t}\mathrm{n}\mathrm{e}\mathrm{j}}$ kwadratu $0$ polu $\displaystyle \frac{81}{2}$? Odpowied $\acute{\mathrm{z}}$ uzasadnič nie $\mathrm{u}\dot{\mathrm{z}}$ ywając kalkulatora.

2. Grupa sluchaczy wykladu $\mathrm{z}$ algebry liczy 261 osób. Egzamin podstawowy zdała pewna

(dodatnia) ilośč osób. Po egzaminie poprawkowym liczba osób, które zdały, powiekszyla

się $0 5$, 6\%. Ile osób zdafo egzamin podstawowy (wskazówka: pamiętaj, $\dot{\mathrm{z}}\mathrm{e}$ ilośč osób,

które zdały egzamin jest liczbą calkowitą)?

3. Haslo do pewnego systemu komputerowego ma skfadač się $\mathrm{z}$ dokladnie 21iter (do wyboru

$\mathrm{z}26$ małych $\mathrm{i}26\mathrm{d}\mathrm{u}\dot{\mathrm{z}}$ ych liter alfabetu) oraz $\mathrm{z}$ przynajmniej 2 $\mathrm{i}$ co najwyzej 4 cyfr (od 0

do 9). Zarówno 1iteryjak $\mathrm{i}$ liczby mogą się powtarzač. Ilejest róznych hasel spelniajqcych

te warunki?

4. Rozwiązač nierównośč

$x+1\geq\sqrt{5-x}.$

5. Suma 2l pierwszych wyrazów pewnego ciqgu arytmetycznego wynosi zero a iloczyn

dwunastego $\mathrm{i}$ trzynastego wyrazu równy jest 8. D1a jakich 1iczb $n$ suma $n$ pierwszych

wyrazów tego ciagu jest mniejsza od 9?

6. Marcin stoi nad brzegiem morza $\mathrm{i}$ obserwuje $\mathrm{o}\mathrm{d}\mathrm{p}\mathrm{f}\mathrm{y}\mathrm{w}\mathrm{a}\mathrm{j}_{\Phi}\mathrm{c}\mathrm{y}$ statek.

a) Jak daleko będzie statek od (oczu) Marcina $\mathrm{w}$ momencie, $\mathrm{w}$ którym zniknie on za

horyzontem (Marcin przestanie go widzieč)?

b) Najak wysoką wiezę musi on wejśč, $\dot{\mathrm{z}}$ eby jeszcze widzieč statek bedący $\mathrm{w}$ odleglości

10 km od niego?

Przyjąč, $\dot{\mathrm{z}}\mathrm{e}$ Ziemia jest kulą $0$ promieniu 6371 km a oczy Marcina znajdują się na wyso-

kości 170 cm.




PRACA KONTROLNA nr 2- POZ1OM ROZSZERZONY

l. Ulozono dwie wieze $\mathrm{z}$ sześciennych klocków. Pierwszq $\mathrm{z}$ trzech klocków $0$ objętości 72,

8 oraz 3 $cm^{3}$, a drugą $\mathrm{z}$ czterech jednakowych klocków $0$ objętości 8 $cm^{3}$ Która $\mathrm{z}$ nich

jest $\mathrm{w}\mathrm{y}\dot{\mathrm{z}}$ sza? Odpowied $\acute{\mathrm{z}}$ uzasadnič nie $\mathrm{u}\dot{\mathrm{z}}$ ywając kalkulatora.

2. Kod do sejfu $\mathrm{w}$ willi pana Bogackiego jest pięciocyfrowy. Jego córka, korzystając $\mathrm{z}$ chwi-

lowej nieobecności taty, próbuje go otworzyč. Wie jednak tylko, $\dot{\mathrm{z}}\mathrm{e}$ kod ulozony jest

$\mathrm{z}$ dokładnie trzech róznych cyfr $\mathrm{i}$ nie występują $\mathrm{w}$ nim cyfry 1,4 $\mathrm{i}9$. Ile jest róznych

kodów spelniających te warunki?

3. Rozwiązač nierównośč

$x-1>\sqrt{4-\frac{6}{x}}.$

4. Wjednej szklance znajduje się woda, a $\mathrm{w}$ drugiej dokładnie taka sama ilośč wina. $\mathrm{Z}$ pierw-

szej szklanki przelano jedną $\text{ł} \mathrm{y}\dot{\mathrm{z}}$ kę wody do szklanki $\mathrm{z}$ winem $\mathrm{i}$ dokladnie wymieszano.

Następnie przelano jedną $l\mathrm{y}\dot{\mathrm{z}}$ kę powstałej mieszaniny $\mathrm{z}$ powrotem do pierwszej szklanki.

Sprawdzič czy po tych zabiegach jest więcej wody $\mathrm{w}$ winie czy wina $\mathrm{w}$ wodzie.

5. Trzy liczby tworzą $\mathrm{c}\mathrm{i}_{\Phi \mathrm{g}}$ geometryczny. Ich suma równa jest 13, a suma ich odwrotności

wynosi $\displaystyle \frac{13}{9}$. Znalez/č te liczby.

6. Bocian stoi na słupie $0$ wysokości 5 metrów. Magda, której oczy znajdują się na wysokości

160 cm nad $\mathrm{z}\mathrm{i}\mathrm{e}\mathrm{m}\mathrm{i}_{\Phi}$, stoi 10,2 metra od tego s1upai widzi bociana pod kątem 6 stopni. Jak

wysoki jest bocian? Podač wynik $\mathrm{z}$ dokładnością do l cm. $\mathrm{W}$ razie potrzeby odpowiednią

funkcję trygonometryczną kąta $6^{\mathrm{o}}$ przyblizyč za pomocq tablic matematycznych lub

kalkulatora.

Rozwiązania (rękopis) zadań z wybranego poziomu prosimy nadsyłač do

2015r. na adres:

19 $\mathrm{p}\mathrm{a}\acute{\mathrm{z}}$ dziernika

Wydziaf Matematyki

Politechnika Wrocfawska

Wybrzez $\mathrm{e}$ Wyspiańskiego 27

$50-370$ WROCLAW.

Na kopercie prosimy $\underline{\mathrm{k}\mathrm{o}\mathrm{n}\mathrm{i}\mathrm{e}\mathrm{c}\mathrm{z}\mathrm{n}\mathrm{i}\mathrm{e}}$ zaznaczyč wybrany poziom! (np. poziom podsta-

wowy lub rozszerzony). Do rozwiązań nalez $\mathrm{y}$ dolączyč zaadresowaną do siebie koperte

zwrotną $\mathrm{z}$ naklejonym znaczkiem, odpowiednim do wagi listu. Prace niespelniające po-

danych warunków nie będą poprawiane ani odsyłane.

Adres internetowy Kursu: http://www.im.pwr.wroc.pl/kurs







XLVII

KORESPONDENCYJNY KURS

Z MATEMATYKI

$\mathrm{p}\mathrm{a}\acute{\mathrm{z}}$dziernik 2017 $\mathrm{r}.$

PRACA KONTROLNA $\mathrm{n}\mathrm{r} 2-$ POZIOM PODSTAWOWY

l. Rozwiązač nierównośč

$2x-2>\sqrt{7-4x}.$

2. Dla jakich wartości parametru $m$ pierwiastkiem wielomianu

$w(x)=2x^{3}-x^{2}-(m^{2}-2)x+m-1$

jest $x=2$? Dla znalezionych wartości $m$ wyznaczyč pozostafe pierwiastki $w(x).$

3. Narysowač staranny wykres funkcji $f(x)=|\displaystyle \sin x|\cos x-\frac{1}{4}\mathrm{i}$ rozwiązač nierównośč

$f(x)\displaystyle \leq-\frac{1}{2}.$

4. Rozwiązač równanie

$4^{x+\sqrt{x^{2}-2}}-5\cdot 2^{x-1+\sqrt{x^{2}-2}}=6.$

5. $\mathrm{W}$ trójkącie równoramiennym $ABC0$ podstawie $AB$ dane $\mathrm{s}\Phi A(2,-1)$ oraz $B(-1,3).$

Środkowe poprowadzone $\mathrm{z}A\mathrm{i}\mathrm{z}B$ są prostopadłe. $\mathrm{Z}\mathrm{n}\mathrm{a}\mathrm{l}\mathrm{e}\mathrm{z}^{J}\text{č}$ współrzędne punktu $C$ oraz

obliczyč pole $\mathrm{i}$ obwód tego trójkqta.

6. $\mathrm{W}$ okrąg $0$ promieniu $R$ wpisano trzy jednakowe okręgi wzajemnie styczne $\mathrm{w}$ punktach

$A, B, C\mathrm{i}$ styczne do danego okręgu. Obliczyč pole obszaru ograniczonego mniejszymi

fukami AB, $BC\mathrm{i}CA.$




PRACA KONTROLNA nr 2- POZ1OM ROZSZERZONY

l. Rozwiązač nierównośč

$\sqrt{2x^{2}-x}<5-4x.$

2. Rozwiazač układ równań

$\left\{\begin{array}{l}
xy\\
x^{\log y}
\end{array}\right.$

400,

16.

3. Narysowač staranny wykres funkcji $f(x)=|\sin x|-\cos x$, wyznaczyč jej zbior wartości

oraz rozwiązač nierównośč

$\displaystyle \frac{1}{f(x)}\geq 1.$

4. Reszta $\mathrm{z}$ dzielenia wielomianu $w(x)=x^{4}+ax^{3}-bx^{2}+bx$ przez trójmian $x^{2}-9$ wynosi

$-5x+45$. Wyznaczyč wartości parametrów $a\mathrm{i}b$ oraz rozwiązač nierównośč

$w(x-1)\geq w(x+1).$

5. Dany jest punkt $A(2,1)$. Wyznaczyč $\mathrm{i}$ narysowač zbiór tych wszystkich punktów $C$, dla

których czworokąt ABCD jest prostokqtem takim, $\dot{\mathrm{z}}\mathrm{e}$ punkty $B\mathrm{i}D\mathrm{l}\mathrm{e}\dot{\mathrm{z}}$ ą na osiach układu

wspófrzędnych $\mathrm{i}$ nie $\mathrm{n}\mathrm{a}\mathrm{l}\mathrm{e}\mathrm{z}\Phi$ do tego samego boku $\mathrm{p}\mathrm{r}\mathrm{o}\mathrm{s}\mathrm{t}\mathrm{o}\mathrm{k}_{\Phi^{\mathrm{t}}}\mathrm{a}$. Wykonač rysunek.

6. Nad sześcianem $0$ krawędzi $a$ stojącym na pfaszczy $\acute{\mathrm{z}}\mathrm{n}\mathrm{i}\mathrm{e}$ umieszczono punktowe z/ródfo

światła na wysokości $b>a$ (rzut prostopadły punktu, $\mathrm{w}$ którym jest z/ródło światła na

tę pfaszczyznę, zawiera się $\mathrm{w}$ podstawie sześcianu). Obliczyč pole obszaru jaki zajmuje

cień sześcianu lącznie $\mathrm{z}$ jego podstawą na tej płaszczyz/nie.

Rozwiązania (rękopis) zadań z wybranego poziomu prosimy nadsyfač do

2017r. na adres:

18 $\mathrm{p}\mathrm{a}\acute{\mathrm{z}}$dziernika

Wydziaf Matematyki

Politechnika Wrocfawska

Wybrzez $\mathrm{e}$ Wyspiańskiego 27

$50-370$ WROCLAW.

Na kopercie prosimy $\underline{\mathrm{k}\mathrm{o}\mathrm{n}\mathrm{i}\mathrm{e}\mathrm{c}\mathrm{z}\mathrm{n}\mathrm{i}\mathrm{e}}$ zaznaczyč wybrany poziom! (np. poziom podsta-

wowy lub rozszerzony). Do rozwiązań nalez $\mathrm{y}$ dołączyč zaadresowaną do siebie koperte

zwrotną $\mathrm{z}$ naklejonym znaczkiem, odpowiednim do wagi listu. Prace niespełniające po-

danych warunków nie będą poprawiane ani odsyłane.

Adres internetowy Kursu: http://www.im.pwr.wroc.pl/kurs







XLVIII

KORESPONDENCYJNY KURS

Z MATEMATYKI

$\mathrm{p}\mathrm{a}\acute{\mathrm{z}}$dziernik 2018 $\mathrm{r}.$

PRACA KONTROLNA $\mathrm{n}\mathrm{r} 2-$ POZIOM PODSTAWOWY

l. Rozwiązač nierównośč $x-1>\sqrt{x^{2}-3}.$

2. Rozwiązač równanie $\displaystyle \frac{1}{\sin 2x}+\frac{1}{\sin x}=0.$

3. Narysowač zbiór $\{(x,y):-1\leq\log_{\frac{1}{2}}|x|+\log_{2}|y|\leq 1,|x|+|y|\leq 3\}\mathrm{i}$ obliczyč jego pole.

4. Na prostej $l$ : $2x-y-1=0$ wyznaczyč punkty, $\mathrm{z}$ których odcinek $0$ końcach $A(0,1)$ oraz

$B(0,3)$ jest widoczny pod kątem $45^{\mathrm{o}}$

5. $\mathrm{W}$ obszar ograniczony wykresem funkcji kwadratowej $\mathrm{i}$ osią $Ox$ wpisano prostokąt $0$ polu

24, którego jeden $\mathrm{z}$ boków zawarty jest $\mathrm{w}$ osi $Ox$, a dwa wierzchofki lezą na paraboli.

Odległośč między miejscami zerowymi funkcji wynosi 10. Wyznaczyč wzór tej funkcji,

wiedzac, $\dot{\mathrm{z}}\mathrm{e}$ wierzchofek paraboli $\mathrm{l}\mathrm{e}\dot{\mathrm{z}}\mathrm{y}$ na osi $Oy$ ijeden $\mathrm{z}$ boków prostokąta ma dfugośč 6.

Rozwiązanie zilustrowač odpowiednim rysunkiem.

6. $\mathrm{W}$ ostrosłupie, którego podstawą jest romb $0$ boku $\alpha$, jedna $\mathrm{z}$ krawędzi bocznych równiez

ma dfugośč $a\mathrm{i}$ jest prostopadfa do podstawy. Wszystkie pozostałe krawędzie boczne są

równe. Obliczyč objętośč, pole powierzchni całkowitej ostrosłupa oraz sinus kąta nachy-

lenia do podstawy jego pochyłych ścian bocznych.




PRACA KONTROLNA nr $2$ - PozioM R0ZSZERZ0NY

l. Wyznaczyč dziedzinę funkcji $f(x)=\log_{2}(\sqrt{x-1-\sqrt{x^{2}-3x-4}}-1).$

2. Rozwiązač równanie $\sin^{4}x+\cos^{4}x=\sin x\cos x.$

3. Narysowač zbiór $\{(x,y):|x|+|y|\leq 6,|y|\leq 2^{|x|},|y|\geq\log_{2}|x|\}\mathrm{i}$ napisač równaniajego

osi symetrii. Podač odpowiednie uzasadnienie.

4. Niech $f(x) = \displaystyle \frac{2x-1}{x-2}, g(x) = (\sqrt{2})^{\log_{2}(2x-1)^{2}+4\log_{\frac{1}{2}}\sqrt{2-x}}$ Narysowač wykres funkcji

$h(x) = \displaystyle \max\{f(x),g(x)\}$. Czy $\mathrm{m}\mathrm{o}\dot{\mathrm{z}}$ na podač wzór funkcji $h(x)$, wykorzystujac jedynie

$f(x)$ ?

5. Punkt $A(1,1)$ jest wierzchołkiem rombu $0$ polu 10. Przekątna $AC$ rombu jest równo-

legła do wektora $\vec{v}=[1$, 2$]$. Wyznaczyč współrzędne pozostałych wierzchołków rombu,

$\mathrm{w}\mathrm{i}\mathrm{e}\mathrm{d}_{\mathrm{Z}\otimes}\mathrm{c}, \dot{\mathrm{z}}\mathrm{e}$ jeden $\mathrm{z}$ nich $\mathrm{l}\mathrm{e}\dot{\mathrm{z}}\mathrm{y}$ na prostej $y=x-2.$

6. $\mathrm{W}$ ostrosłupie, którego podstawq jest romb $0$ boku $\alpha$, jedna $\mathrm{z}$ krawędzi bocznych równiez

ma dlugośč $a\mathrm{i}$ jest prostopadła do podstawy. Wszystkie pozostafe krawędzie boczne

są równe. Wyznaczyč cosinusy kątów płaskich przy wierzchofku $\mathrm{k}\mathrm{a}\dot{\mathrm{z}}$ dej ściany bocznej

ostrosłupa oraz cosinusy katów między jego ścianami bocznymi

Rozwiązania (rękopis) zadań z wybranego poziomu prosimy nadsyłač do

2018r. na adres:

18 $\mathrm{p}\mathrm{a}\acute{\mathrm{z}}$ dziernika

Wydziaf Matematyki

Politechnika Wrocfawska

Wybrzez $\mathrm{e}$ Wyspiańskiego 27

$50-370$ WROCLAW.

Na kopercie prosimy $\underline{\mathrm{k}\mathrm{o}\mathrm{n}\mathrm{i}\mathrm{e}\mathrm{c}\mathrm{z}\mathrm{n}\mathrm{i}\mathrm{e}}$ zaznaczyč wybrany poziom! (np. poziom podsta-

wowy lub rozszerzony). Do rozwiązań nalez $\mathrm{y}$ dołączyč zaadresowaną do siebie kopertę

zwrotną $\mathrm{z}$ naklejonym znaczkiem, odpowiednim do wagi listu. Prace niespelniające po-

danych warunków nie będą poprawiane ani odsylane.

Uwaga. Wysyłając nam rozwiazania zadań uczestnik Kursu udostępnia Politechnice Wrocławskiej

swoje dane osobowe, które przetwarzamy wyłącznie $\mathrm{w}$ zakresie niezbędnym do jego prowadzenia

(odesłanie zadań, prowadzenie statystyki). Szczególowe informacje $0$ przetwarzaniu przez nas danych

osobowych są dostępne na stronie internetowej Kursu.

Adres internetowy Kursu: http: //www. im. pwr. edu. pl/kurs







XLIX

KORESPONDENCYJNY KURS

Z MATEMATYKI

$\mathrm{p}\mathrm{a}\acute{\mathrm{z}}$dziernik 2019 $\mathrm{r}.$

PRACA KONTROLNA $\mathrm{n}\mathrm{r} 2-$ POZIOM PODSTAWOWY

l. Niech $\alpha$ będzie kątem ostrym takim, $\dot{\mathrm{z}}\mathrm{e}\sin\alpha=\sqrt{15}\cos\alpha$. Wyznaczyč wszystkie wartości

funkcji trygonometrycznych kątów $\alpha$ oraz $2\alpha.$

2. Rozwiązač nierównośč

$x\geq 2+\sqrt{10-3x}.$

3. Wykres trójmianu kwadratowego $f(x)=ax^{2}+bx+c$ jest symetryczny wzgledem prostej

$x=3$, a $\mathrm{r}\mathrm{e}\mathrm{s}\mathrm{z}\mathrm{t}_{\Phi}\mathrm{z}$ jego dzielenia przez wielomian $x-2$ jest -$1$. Wiadomo tez$\cdot, \dot{\mathrm{z}}\mathrm{e}f(0)=3.$

Znalez/č wartości wspólczynników $a, b, c\mathrm{i}$ rozwiązač nierównośč

$\displaystyle \frac{1}{f(x)}\geq\frac{1}{3}.$

4. $\mathrm{W}$ ciqgu arytmetycznym, $\mathrm{w}$ którym trzeci wyraz jest odwrotnością pierwszego, suma

pierwszych ośmiu wyrazów wynosi 25. Ob1iczyč sumę pierwszych 10 wyrazów $0$ numerach

nieparzystych.

5. Pole trapezu równoramiennego, opisanego na okregu $0$ promieniu l, wynosi 5. Ob1iczyč

pole czworokąta, którego wierzchofkami są punkty styczności okręgu $\mathrm{i}$ trapezu.

6. Na szczycie góry, na którą wchodzi Agata po stoku $0$ kacie nachylenia $\beta$, stoi krowa

$0$ wysokości 150 cm. Dziewczynka widzi ją pod kątem $\alpha$, przy czym przyjmujemy tutaj

dla uproszczenia, $\dot{\mathrm{z}}\mathrm{e}$ punkt obserwacji znajduje się na poziomie drogi. Najakiej wysokości

nad poziomem morza stoi Agata, $\mathrm{j}\mathrm{e}\dot{\mathrm{z}}$ eli szczyt jest na wysokości 1520 $\mathrm{m}$ n.p.m.? Podač

wzór $\mathrm{i}$ następnie wykonač obliczenia dla $\beta=43^{\mathrm{o}}, \alpha=2^{\mathrm{o}}$




PRACA KONTROLNA nr 2- POZ1OM ROZSZERZONY

l. W nieskończonym ciągu geometrycznym, którego suma równa jest 4, trzeci wyraz jest

odwrotnością pierwszego. Wyznaczyč pierwszy wyraz i sumę wszystkich wyrazów 0 nu-

merach parzystych.

2. Narysowač wykres funkcji

$f(x)=\displaystyle \frac{\sin x}{\sqrt{1+\mathrm{t}\mathrm{g}^{2}x}}$

$\mathrm{i}$ rozwiązač nierównośč $f(x)\displaystyle \geq\frac{1}{4}.$

3. Rozwiązač nierównośč

$\displaystyle \sqrt{\frac{4x-2}{x+4}}\leq\frac{2}{x-1}-1.$

4. Reszta $\mathrm{z}$ dzielenia wielomianu $w(x)=ax^{5}+bx^{2}+c$ przez $p(x)=x^{3}-x^{2}-2x$ jest wielo-

mian $r(x)=11x^{2}+12x+1$. Wyznaczyč wartości współczynników $a, b, c$ oraz rozwiązač

nierównośč $w(x)\geq(x+1)^{2}$

5. Wyznaczyč pole trójkąta równobocznego, którego wierzcholki lezą na trzech róznych

prostych równolegfych, $\mathrm{z}$ których środkowa jest oddalona od skrajnych $0$ {\it a} $\mathrm{i}b.$

6. $\mathrm{W}$ punktach $A(0,0), B(4,0) \mathrm{i}C(0,4)$ ustawione są trzy znaczniki. Sensory robota po-

zwalają ustalič, $\dot{\mathrm{z}}\mathrm{e}\mathrm{z}$ miejsca, $\mathrm{w}$ którym znajduje się on obecnie odcinek $AB$ widač pod

kątem $\alpha=90^{\mathrm{o}}$, a odcinek $AC$ pod kątem $\beta=60^{\mathrm{o}}$ Ustalič pofozenie robota $\mathrm{w}$ ukfadzie

wspólrzędnych.

Rozwiązania (rękopis) zadań z wybranego poziomu prosimy nadsyłač do

2019r. na adres:

18 $\mathrm{p}\mathrm{a}\acute{\mathrm{z}}$ dziernika

Wydziaf Matematyki

Politechnika Wrocfawska

Wybrzez $\mathrm{e}$ Wyspiańskiego 27

$50-370$ WROCLAW.

Na kopercie prosimy $\underline{\mathrm{k}\mathrm{o}\mathrm{n}\mathrm{i}\mathrm{e}\mathrm{c}\mathrm{z}\mathrm{n}\mathrm{i}\mathrm{e}}$ zaznaczyč wybrany poziom! (np. poziom podsta-

wowy lub rozszerzony). Do rozwiązań nalez $\mathrm{y}$ dołączyč zaadresowana do siebie koperte

zwrotną $\mathrm{z}$ naklejonym znaczkiem, odpowiednim do wagi listu. Prace niespelniające po-

danych warunków nie będą poprawiane ani odsylane.

Uwaga. Wysyłając nam rozwi\S zania zadań uczestnik Kursu udostępnia Politechnice Wrocławskiej

swoje dane osobowe, które przetwarzamy wyłącznie $\mathrm{w}$ zakresie niezbędnym do jego prowadzenia

(odesłanie zadań, prowadzenie statystyki). Szczególowe informacje $0$ przetwarzaniu przez nas danych

osobowych są dostępne na stronie internetowej Kursu.

Adres internetowy Kursu: http: //www. im. pwr. edu. pl/kurs







L

KORESPONDENCYJNY KURS

Z MATEMATYKI

$\mathrm{p}\mathrm{a}\acute{\mathrm{z}}$dziernik 2020 $\mathrm{r}.$

PRACA KONTROLNA $\mathrm{n}\mathrm{r} 2-$ POZIOM PODSTAWOWY

l. Niemieckie przepisy drogowe wymagaja zachowania bezpiecznego odstępu między po-

ruszającymi się $\mathrm{w}$ tym samym kierunku pojazdami. Zalecane jest przy tym zachowanie

zasady,,połowa licznika $\mathrm{j}\mathrm{e}\dot{\mathrm{z}}$ eli dwa pojazdyjadą $\mathrm{z}$ prędkością $x\mathrm{k}\mathrm{m}/\mathrm{h}$, to odstęp między

nimi powinien wynosič przynajmniej $x/2$ metrów. Jaki odstep czasowy powinien zatem

dzielič te dwa pojazdy? Przyjmując, $\dot{\mathrm{z}}\mathrm{e}$ dla samochodujadącego $\mathrm{z}$ prędkości$\Phi v\mathrm{m}/\mathrm{s}$ droga

hamowania wynosi $s_{h}=\displaystyle \frac{v^{2}}{2a}$ metrów (gdzie $a$ jest stałym współczynnikiem hamowania),

sprawd $\acute{\mathrm{z}}$ przy jakiej prędkości $x\mathrm{k}\mathrm{m}/\mathrm{h}$ dojdzie do wypadku, $\mathrm{j}\mathrm{e}\dot{\mathrm{z}}$ eli oba pojazdy jechały

$\mathrm{z}$ minimalnym zalecanym odstępem, pierwszy zatrzymał się nagle (przyjmij $a=10$), $\mathrm{a}$

drugi zaczał hamowač jednq sekundę póz/niej $\mathrm{i}\mathrm{z}$ sila taka, $\dot{\mathrm{z}}\mathrm{e}a=7.$

2. $\displaystyle \frac{\mathrm{J}\mathrm{a}\sqrt{6}}{2}?\mathrm{k}\mathrm{i}\mathrm{m}\mathrm{k}_{\Phi}$tami mogą byč $\alpha \mathrm{i}2\alpha, \mathrm{j}\mathrm{e}\dot{\mathrm{z}}$ eli wiadomo, $\dot{\mathrm{z}}\mathrm{e}\alpha$ jest $\mathrm{k}_{\Phi}\mathrm{t}\mathrm{e}\mathrm{m}$ ostrym oraz $\sin\alpha+\cos\alpha=$

$3$. Rozwazmy funkcję $f(x)=x^{2}-(a+2)x+3(a-1)$. Dla jakich wartości paramertu $a$:

(i) cafy wykres $f(x)\mathrm{l}\mathrm{e}\dot{\mathrm{z}}\mathrm{y}$ ponad prostą $y=-1$?

(ii) oba miejsca zerowe funkcji $f(x)$ sq wieksze od 2?

4. Rozwiąz nierównośč

$x\leq 1+\sqrt{2+x}.$

5. Narysuj starannie zbiór $A\cap B$, gdzie

$A=\{(x,y):2|x|+|y|\leq 2\},$

$B=\{(x,y):y^{2}-y<2\}$

$\mathrm{i}$ oblicz jego pole.

6. Jednym $\mathrm{z}$ wierzchofków kwadratu jest $A(1,-3)$, a jedna $\mathrm{z}$ jego przekątnych zawiera się

$\mathrm{w}$ prostej $y = -2x+2$. Wyznaczyč współrzędne pozostalych wierzchołków kwadratu

$\mathrm{i}$ równanie okręgu wpisanego $\mathrm{w}$ ten kwadrat.




PRACA KONTROLNA nr 2- POZ1OM ROZSZERZONY

l. Wyznacz kąty $\alpha \mathrm{i}2\alpha$ wiedzac, $\mathrm{i}\dot{\mathrm{z}}\alpha$ jest kątem rozwartym takim, $\dot{\mathrm{z}}\mathrm{e}$ tg $\alpha+\mathrm{c}\mathrm{t}\mathrm{g}\alpha=-2\sqrt{2}.$

2. Rozwiąz równanie

$x=\sqrt{5+\sqrt{3+x^{2}}}.$

Nie $\mathrm{u}\dot{\mathrm{z}}$ ywając kalkulatora zbadaj, czy jego rozwiązanie jest liczbą większą $\mathrm{n}\mathrm{i}\dot{\mathrm{z}}3.$

3. Udowodnij, $\dot{\mathrm{z}}\mathrm{e}\mathrm{j}\mathrm{e}\dot{\mathrm{z}}$ eli dwa trójkąty prostokątne mają równe obwody $\mathrm{i}$ dlugości przeciw-

$\mathrm{P}^{\mathrm{r}\mathrm{o}\mathrm{s}\mathrm{t}\mathrm{o}\mathrm{k}}\Phi^{\mathrm{t}\mathrm{n}\mathrm{y}\mathrm{c}\mathrm{h}}$, to $\mathrm{s}\Phi$ przystające.

4. Narysuj starannie zbiór $A\cap B$, gdzie

$A=\{(x,y):x^{2}-8|x|+y^{2}-8|y|+16\geq 0,|x|\leq 4,|y|\leq 4\},$

$B=\{(x,y):x^{2}+y^{2}>16(3-2\sqrt{2})\}$

$\mathrm{i}$ oblicz jego pole.

5. Dla jakich wartości parametrów $p\mathrm{i}q$ do zbioru rozwiązań równania

$x^{3}-3px^{2}+(q+4)x=0,$

nalezą zarówno $p$ jak $\mathrm{i}q$?

6. Napisz równanie prostej $k$ stycznej do okregu $x^{2}-4x+y^{2}+2y=0\mathrm{w}$ punkcie $P(3,1).$

Następnie wyznacz równania wszystkich prostych stycznych do tego okręgu, które tworzą

$\mathrm{z}$ prostą $k$ kąt $45^{\mathrm{o}}$

Rozwiązania (rękopis) zadań z wybranego poziomu prosimy nadsylač do

2020r. na adres:

20 $\mathrm{p}\mathrm{a}\acute{\mathrm{z}}$dziernika

Wydziaf Matematyki

Politechnika Wrocfawska

Wybrzez $\mathrm{e}$ Wyspiańskiego 27

$50-370$ WROCLAW.

Na kopercie prosimy $\underline{\mathrm{k}\mathrm{o}\mathrm{n}\mathrm{i}\mathrm{e}\mathrm{c}\mathrm{z}\mathrm{n}\mathrm{i}\mathrm{e}}$ zaznaczyč wybrany poziom! (np. poziom podsta-

wowy lub rozszerzony). Do rozwiązań nalez $\mathrm{y}$ dołączyč zaadresowaną do siebie kopertę

zwrotną $\mathrm{z}$ naklejonym znaczkiem, odpowiednim do formatu listu. Polecamy stosowanie

kopert formatu C5 $(160\mathrm{x}230\mathrm{m}\mathrm{m})$ ze znaczkiem $0$ wartości 3,30 zł. Na $\mathrm{k}\mathrm{a}\dot{\mathrm{z}}$ dą większą

koperte nalez $\mathrm{y}$ nakleič drozszy znaczek. Prace niespełniające podanych warunków nie

będą poprawiane ani odsyłane.

Uwaga. Wysylając nam rozwi\S zania zadań uczestnik Kursu udostępnia Politechnice Wroclawskiej

swoje dane osobowe, które przetwarzamy wyłącznie $\mathrm{w}$ zakresie niezbędnym do jego prowadzenia

(odesfanie zadań, prowadzenie statystyki). Szczegófowe informacje $0$ przetwarzaniu przez nas danych

osobowych są dostępne na stronie internetowej Kursu.

Adres internetowy Kursu: http://www.im.pwr.edu.pl/kurs







LI KORESPONDENCYJNY KURS

Z MATEMATYKI

$\mathrm{p}\mathrm{a}\acute{\mathrm{z}}$dziernik 2021 $\mathrm{r}.$

PRACA KONTROLNA nr 2- POZIOM PODSTAWOWY

l. Rozwiąz równanie

$\displaystyle \sin 2x=\cos^{4}\frac{x}{2}$ -sin4 $\displaystyle \frac{x}{2}.$

2. Rozwiąz nierównośč

$\sqrt{4-x}\leq x+8.$

3. $\mathrm{W}$ ciągu geometrycznym $(a_{n})\mathrm{z}\mathrm{a}\mathrm{c}\mathrm{h}\mathrm{o}\mathrm{d}\mathrm{z}\Phi$ równości: $a_{4}-a_{2}=18$ oraz $a_{5}-a_{3}=36$. Wyznacz

$a_{3}.$

4. Dla jakich wartości parametru $m$ rozwiązaniem ukladu

$\left\{\begin{array}{l}
2x+3y=4\\
4x+my=2m
\end{array}\right.$

jest para liczb dodatnich?

5. Przekrój poprzeczny dwuspadowego dachu pewnego budynku jest czworokątem ABCD,

$\mathrm{w}$ którym kąt $DAB$ jest kątem prostym, $|AB| =9m$, a obie (nierówne) połacie dachu,

czyli odcinki $BC\mathrm{i}CD$, są nachylone pod kqtem $40^{\mathrm{o}}$ do poziomu (odcinka AB). Oblicz

fączną dfugośč (tzn. $|BC|+|CD|$) obu polaci dachu.

6. Wykaz, $\dot{\mathrm{z}}\mathrm{e}$ miara kąta ostrego $\mathrm{w}$ rombie wynosi $30^{\mathrm{o}}$ wtedy $\mathrm{i}$ tylko wtdy, gdy długośč jego

boku jest równa średniej geometrycznej jego przekątnych.




PRACA KONTROLNA $\mathrm{n}\mathrm{r} 2-$ POZIOM ROZSZERZONY

l. Rozwiąz równanie

tg $x\cdot \mathrm{t}\mathrm{g}(x+1)=1.$

2. Rozwiąz nierównośč

$2-3x>\sqrt{\frac{x+4}{1-x}}.$

3. Huragan znad Oceanu Atlantyckiego zbliza się do wybrzeza Florydy. $\mathrm{J}\mathrm{e}\dot{\mathrm{z}}$ eli jego cen-

trum znajdzie się $\mathrm{w}$ odległości mniejszej $\mathrm{n}\mathrm{i}\dot{\mathrm{z}}60$ km od centrum Miami, to miasto dozna

powaznych zniszczeń. Meteorolog modeluje centrum miastajako ustalony punkt $0$ współ-

rzędnych (240, 200), gdzie $\mathrm{j}\mathrm{e}\mathrm{d}\mathrm{n}\mathrm{o}\mathrm{s}\mathrm{t}\mathrm{k}_{\Phi}$ ukladu wspófrzędnych jest kilometr. Przyjmuje na-

tomiast, $\dot{\mathrm{z}}\mathrm{e}$ centrum huraganu porusza się po prostej $0$ równaniu $y=kx+20$. Dlajakich

wartości parametru $k$ miasto nie dozna powaznych zniszczeń?

4. Zbadaj liczbę rozwiązań równania

-{\it a}2{\it xx}2 $+$-21$\alpha$ - -2 -1{\it ax} $=$ -{\it xa}'

$\mathrm{w}$ zalezności od parametru $a\neq 0.$

5. Pole rombu jest równe $S$, a suma długości jego przekątnych wynosi $m$. Wyznacz długośč

jego boku oraz cosinus kąta ostrego. Jakie warunki $\mathrm{m}\mathrm{u}\mathrm{s}\mathrm{z}\Phi$ spełniač parametry $m\mathrm{i}S\dot{\mathrm{z}}$ eby

zadanie miało rozwiązanie?

6. Dany jest niestały ciag arytmetyczny $(a_{n})$ taki, $\dot{\mathrm{z}}\mathrm{e}$ iloraz

$\displaystyle \frac{\alpha_{1}+a_{2}+\ldots+\alpha_{n}}{a_{n+1}+a_{n+2}+\ldots+a_{2n}}$

jest liczbą stafą $C$. Wyznacz wartośč tej stalej oraz róznicę tego ciqgu, jeśli wiadomo,

$\dot{\mathrm{z}}\mathrm{e}$ jego pierwszym wyrazem jest $\alpha_{1}=p.$

Rozwiązania (rękopis) zadań z wybranego poziomu prosimy nadsyfač do

nika 2021r. na adres:

20 $\mathrm{p}\mathrm{a}\acute{\mathrm{z}}$ dzier-

Wydziaf Matematyki

Politechnika Wrocfawska

Wybrzez $\mathrm{e}$ Wyspiańskiego 27

$50-370$ WROCLAW.

Na kopercie prosimy $\underline{\mathrm{k}\mathrm{o}\mathrm{n}\mathrm{i}\mathrm{e}\mathrm{c}\mathrm{z}\mathrm{n}\mathrm{i}\mathrm{e}}$ zaznaczyč wybrany poziom! (np. poziom podsta-

wowy lub rozszerzony). Do rozwiązań nalez $\mathrm{y}$ dołączyč zaadresowaną do siebie kopertę

zwrotną $\mathrm{z}$ naklejonym znaczkiem, odpowiednim do formatu listu. Polecamy stosowanie

kopert formatu C5 $(160\mathrm{x}230\mathrm{m}\mathrm{m})$ ze znaczkiem $0$ wartości 3,30 zł. Na $\mathrm{k}\mathrm{a}\dot{\mathrm{z}}$ dą większą

kopertę nalez $\mathrm{y}$ nakleič $\mathrm{d}\mathrm{r}\mathrm{o}\dot{\mathrm{z}}$ szy znaczek. Prace niespełniające podanych warunków nie

bedą poprawiane ani odsyłane.

Uwaga. Wysyfajac nam rozwiazania zadań uczestnik Kursu udostępnia Politechnice Wroclawskiej

swoje dane osobowe, które przetwarzamy wyłącznie $\mathrm{w}$ zakresie niezbednym do jego prowadzenia

(odesfanie zadań, prowadzenie statystyki). Szczegófowe informacje $0$ przetwarzaniu przez nas danych

osobowych s\S dostępne na stronie internetowej Kursu.

Adres internetowy Kursu: http: //www. im. pwr. edu. pl/kurs







LII

KORESPONDENCYJNY KURS

Z MATEMATYKI

$\mathrm{p}\mathrm{a}\acute{\mathrm{z}}$dziernik 2022 $\mathrm{r}.$

PRACA KONTROLNA $\mathrm{n}\mathrm{r} 2-$ POZIOM PODSTAWOWY

l. Rozwiąz nierównośč

$1\displaystyle \leq\log_{\frac{1}{3}}\frac{1}{2x-1}<2.$

2. Średnia arytmetyczna czwartego, szóstego $\mathrm{i}$ dziesiątego wyrazu ciągu arytmetycznego

$(a_{n})$, gdzie $ n\geq 1$, wynosi 14, a ciąg $(a_{3},a_{5},a_{11})$ jest geometryczny. Uzasadnij, $\dot{\mathrm{z}}\mathrm{e}$ ciąg

$(a_{4},\alpha_{6},\alpha_{10})$ równiez jest geometryczny.

3. $\mathrm{W}\mathrm{c}\mathrm{i}_{\Phi \mathrm{g}}\mathrm{u}$ arytmetycznym $(a_{n})$, określonym dla $\mathrm{k}\mathrm{a}\dot{\mathrm{z}}$ dej liczby naturalnej $n\geq 1$, mamy

$a_{3}=0$ oraz

$a_{6}=7\sin^{2}\alpha,$

gdzie $\mathrm{t}\mathrm{g}\alpha=3$. Oblicz sumę 50 początkowych wyrazów tego ciągu, których indeksy są

liczbami parzystymi.

4. Bank oferuje kredyt, który nalezy spłacič jednorazowo wraz $\mathrm{z}$ odsetkami po roku. Jaki

jest calkowity koszt tego kredytu, jeśli co miesiąc bank nalicza odsetki $\mathrm{w}$ wysokości

2\% aktualnego zadłuzenia, a dodatkowo $\mathrm{w}$ chwili przyznania kredytu dolicza prowizję

$\mathrm{w}$ wysokości 3\% $\mathrm{p}\mathrm{o}\dot{\mathrm{z}}$ yczanej kwoty? $\mathrm{J}\mathrm{a}\mathrm{k}_{\Phi}$ kwotę trzeba będzie spfacič, jeśli $\mathrm{p}\mathrm{o}\dot{\mathrm{z}}$ yczymy

20000 zł? Prowizja naliczana jest jednorazowo $\mathrm{i}$ powiększa kwotę, którą nalez $\mathrm{y}$ spłacič.

5. Zaznacz na osi liczbowej zbiór wszystkich wartości parametru $t$, dla których funkcja

$f(x)=(\displaystyle \frac{2-t^{2}}{t-3})^{t-x}+1-t$

jest malejąca. Naszkicuj wykres funkcji $f$ dla największej cafkowitej wartości $t\mathrm{z}$ wyzna-

czonego zbioru.

6. Niech $c > 0 \mathrm{i} c \neq 1$. Wyznacz najmniejszą liczbę naturalną $m$, dla której suma $m$

$\mathrm{P}^{\mathrm{O}\mathrm{C}\mathrm{Z}}\varpi$tkowych wyrazów ciągu $(a_{n}), a_{n}=\log_{2}c^{n}$, przekracza liczbę

$\log_{2^{m}}c^{m^{2}}+16\log_{4}c^{2}$




PRACA KONTROLNA $\mathrm{n}\mathrm{r} 2-$ POZIOM ROZSZERZONY

l. Uzasadnij, $\dot{\mathrm{z}}\mathrm{e}\mathrm{c}\mathrm{i}_{\Phi \mathrm{g}}(a_{n})$, którego n-ty wyraz dany jest wzorem

$a_{n}=\displaystyle \frac{1}{2^{1}+3^{1}}+\frac{1}{2^{2}+3^{2}}+\frac{1}{2^{3}+3^{3}}+\cdots+\frac{1}{2^{n}+3^{n}},$

jest ograniczony.

2. Wyznacz dziedzinę $D_{f}$ funkcji

$f(x)=\displaystyle \log_{10+3x-x^{2}}(8-\frac{7}{1-x})$

3. Niech $c>0$. Zbadaj monotonicznośč oraz oblicz sumę wszystkich wyrazów nieskończo-

nego ciągu $(a_{n})$, gdzie

$a_{n}=\log_{3^{3^{n}}}c$ dla $\mathrm{k}\mathrm{a}\dot{\mathrm{z}}$ dego $n\geq 1.$

Ustal, dla jakiej wartości parametru $c$ suma ta jest nie mniejsza od liczby $\log_{9}(c^{2}-2).$

4. Rozwiąz nierównośč

$\displaystyle \sqrt{\log_{\sqrt{x}}(x+2)}>\frac{1}{\log_{\sqrt{x+2}}\sqrt{x}}.$

5. Określ ilośč rozwiązań równania

$|2^{x-1}-1|=m\cdot 2^{x+1}$

$\mathrm{w}$ zalezności od wartości parametru $m.$

6. Opisz metodę konstrukcji $\mathrm{i}$ starannie narysuj wykres funkcji

$f(x)=2+\displaystyle \log_{2}\frac{1}{2-x}.$

Następnie narysuj obraz tej krzywej $\mathrm{w}$ symetrii względem prostej $x =y.$ Wyprowad $\acute{\mathrm{z}}$

wzór funkcji, której wykresem jest powstafa $\mathrm{w}$ ten sposób krzywa.

Rozwiązania (rękopis) zadań $\mathrm{z}$ wybranego poziomu prosimy nadsyfač do $20.10.2022\mathrm{r}$. na

adres:

Wydziaf Matematyki

Politechnika Wrocfawska

Wybrzez $\mathrm{e}$ Wyspiańskiego 27

$50-370$ WROCLAW,

lub elektronicznie, za pośrednictwem portalu talent. $\mathrm{p}\mathrm{w}\mathrm{r}$. edu. pl

Na kopercie prosimy $\underline{\mathrm{k}\mathrm{o}\mathrm{n}\mathrm{i}\mathrm{e}\mathrm{c}\mathrm{z}\mathrm{n}\mathrm{i}\mathrm{e}}$ zaznaczyč wybrany poziom! (np. poziom podsta-

wowy lub rozszerzony). Do rozwiązań nalez $\mathrm{y}$ dołączyč zaadresowaną do siebie koperte

zwrotną $\mathrm{z}$ naklejonym znaczkiem, odpowiednim do formatu listu. Prace niespełniające

podanych warunków nie bedą poprawiane ani odsyłane.

Uwaga. Wysyfajac nam rozwiazania zadań uczestnik Kursu udostępnia Politechnice Wrocfawskiej

swoje dane osobowe, które przetwarzamy wyłącznie $\mathrm{w}$ zakresie niezbędnym do jego prowadzenia

(odeslanie zadań, prowadzenie statystyki). Szczegófowe informacje $0$ przetwarzaniu przez nas danych

osobowych $\mathrm{S}\otimes$ dostępne na stronie internetowej Kursu.

Adres internetowy Kursu: http: //www. im. pwr. edu. pl/kurs







XLIV

KORESPONDENCYJNY KURS

Z MATEMATYKI

listopad 2014 r.

PRACA KONTROLNA nr 3- POZIOM PODSTAWOWY

l. Rozwiąz nierównośč

$x^{5}+x^{4}-8x^{2}+16\geq 8x^{3}-16x.$

2. $\mathrm{W}$ przedziale $[\pi,2\pi]$ rozwiqz równanie

-cs  oins 36{\it xx} $=$ 1.

3. Dane $\mathrm{s}\Phi$ trzy wektory ã$= (1,$ 1$), \vec{b}= (2,-1), \vec{c}= (5,2)$. Dobierz takie liczby $p, q$, aby

$\mathrm{z}$ wektorów $p\vec{a}, q\vec{b}, \vec{c}\mathrm{m}\mathrm{o}\dot{\mathrm{z}}$ na było zbudowač trójkqt.

4. $\mathrm{W}$ przedziale $[0,\pi]$ narysuj wykres funkcji

$f(x)=\displaystyle \frac{1}{|\mathrm{t}\mathrm{g}x+\mathrm{c}\mathrm{t}\mathrm{g}x|}+\sin 2x,$

$\mathrm{i}$ rozwiąz nierównośč $f(x)<\displaystyle \frac{3}{4}.$

5. Na okręgu $x^{2}-2x+y^{2}+4y-4 = 0$ wyznacz punkt, którego odleglośč od prostej

$x-3y+6=0$ jest najmniejsza.

6. $\mathrm{P}\mathrm{r}\mathrm{z}\mathrm{e}\mathrm{k}_{\Phi}\mathrm{t}\mathrm{n}\mathrm{a}$ rombu $0$ polu 9 zawarta jest $\mathrm{w}$ prostej $x-2y+3 = 0$, a jednym $\mathrm{z}$ jego

wierzchofków jest punkt $A(2,-2)$. Wyznacz współrzędne pozostafych wierzchołków tego

rombu.




PRACA KONTROLNA nr 3- POZIOM ROZSZERZONY

l. Resztą $\mathrm{z}$ dzielenia wielomianu $w(x) =x^{4}+px^{3}-3x^{2}+qx-14$ przez $x^{2}-x-2$ jest

$4x-28$. Wyznacz współczynniki $p, q\mathrm{i}$rozwia $\dot{\mathrm{Z}}$ nierównośč $w(x)\geq 0.$

2. Wyznacz $\mathrm{n}\mathrm{a}\mathrm{j}\mathrm{m}\mathrm{n}\mathrm{i}\mathrm{e}\mathrm{j}\mathrm{s}\mathrm{z}\Phi$ wartośč funkcji

$f(x)=($tg $x+$ ctg $x)^{2}$

oraz $\mathrm{r}\mathrm{o}\mathrm{z}\mathrm{w}\mathrm{i}_{\Phi}\dot{\mathrm{z}}$ nierównośč $f(x)\leq f(2x).$

3. Rozwiąz równanie

$\cos x+\cos 2x+2\cos 3x+\cos 4x+\cos 5x=0.$

4. Znajd $\acute{\mathrm{z}}$ kąt między wektorami $\vec{a}\mathrm{i}\vec{b}\mathrm{w}\mathrm{i}\mathrm{e}\mathrm{d}\mathrm{z}\text{ą} \mathrm{c}, \dot{\mathrm{z}}\mathrm{e}$ wektor $5\text{{\it ã}}-4\vec{b}\mathrm{j}\mathrm{e}\mathrm{s}\mathrm{t}$ prostopadły do wektora

$2\vec{\alpha}+4\vec{b}$, a wektor $\text{{\it ã}}-5\vec{b}$ jest prostopadły do wektora $6\vec{\alpha}-2\vec{b}.$

5. $\mathrm{Z}$ wierzchofka $O$ paraboli $y^{2}=2x$ poprowadzono dwie proste wzajemnie prostopadłe $\mathrm{i}$

przecinające parabolę $\mathrm{w}$ punktach $P\mathrm{i}Q$. Wyznacz zbiór punktów pfaszczyzny utworzony

przez środki cięzkości trójkątów $OPQ.$ Sporząd $\acute{\mathrm{z}}$ rysunek.

6. $\mathrm{W}$ trójk$\Phi$cie $0$ wierzchofkach $A(-6,-7), B(8,-9), C(0,10)$ punkt $P$ jest środkiem boku

$BC$, a punkt $S$ jest punktem przecięcia środkowej poprowadzonej $\mathrm{z}$ wierzcholka $A$ oraz

wysokości opuszczonej na bok $AB$. Oblicz pole trójk$\Phi$ta $CSP$ oraz znajd $\acute{\mathrm{z}}$ równanie

okręgu opisanego na nim.

Rozwiązania (rękopis) zadań z wybranego poziomu prosimy nadsyfač do

2014r. na adres:

18 1istopada

Instytut Matematyki $\mathrm{i}$ Informatyki

Politechniki Wrocfawskiej

Wybrzez $\mathrm{e}$ Wyspiańskiego 27

$50-370$ WROCLAW.

Na kopercie prosimy $\underline{\mathrm{k}\mathrm{o}\mathrm{n}\mathrm{i}\mathrm{e}\mathrm{c}\mathrm{z}\mathrm{n}\mathrm{i}\mathrm{e}}$ zaznaczyč wybrany poziom! (np. poziom podsta-

wowy lub rozszerzony). Do rozwiązań nalez $\mathrm{y}$ dołączyč zaadresowaną do siebie kopertę

zwrotną $\mathrm{z}$ naklejonym znaczkiem, odpowiednim do wagi listu. Prace niespełniające po-

danych warunków nie będą poprawiane ani odsyłane.

Adres internetowy Kursu: http://www.im.pwr.wroc.pl/kurs







XLV

KORESPONDENCYJNY KURS

Z MATEMATYKI

listopad 2015 r.

PRACA KONTROLNA nr 3- POZIOM PODSTAWOWY

l. Rozwiązač równanie tg $x-\displaystyle \sin x=\frac{1-\cos x}{2\cos x}.$

2. Narysowač wykres funkcji $f(x)=2\sin x+|\sin x|\mathrm{i}$ rozwiązač nierównośč $|f(x)|\displaystyle \leq\frac{3\sqrt{3}}{2}.$

3. Odcinek $CD$ jest obrazem odcinka $0$ końcach $A(1,1)\mathrm{i}B(2,0)$ wjednokładności $0$ środku

$S(1,-1)\mathrm{i}$ skali $k=-2$. Obliczyč pole czworokąta ABCD. Sporządzič rysunek.

4. Wielomian $W(x)=x^{3}+ax^{2}+bx+c$ jest podzielny przez dwumian $x+1$, ajego wykres

jest symetryczny względem punktu $(0,0)$. Wyznaczyč $a, b, c\mathrm{i}$ rozwiązač nierównośč

$(x-1)W(x+2)-(x-2)W(x+1)\leq 0.$

5. Punkty $\mathrm{A}(1,1), \mathrm{B}(0,3)$ są kolejnymi wierzchofkami rombu ABCD. Wyznaczyč pozostafe

wierzchołki, wiedząc, $\dot{\mathrm{z}}\mathrm{e}$ jeden $\mathrm{z}$ nich $\mathrm{l}\mathrm{e}\dot{\mathrm{z}}\mathrm{y}$ na prostej $x-y-2=0$. Sporzqdzič rysunek.

6. $\mathrm{W}$ trójkąt równoramienny wpisano $\mathrm{o}\mathrm{k}\mathrm{r}\varpi \mathrm{g}\mathrm{o}$ promieniu $r$. Wyznaczyč pole trójk$\Phi$ta, $\mathrm{j}\mathrm{e}\dot{\mathrm{z}}$ eli

środek okręgu opisanego na tym trójkącie $\mathrm{l}\mathrm{e}\dot{\mathrm{z}}\mathrm{y}$ na okręgu wpisanym $\mathrm{w}$ ten trójkąt. Ile

rozwiązań ma to zadanie? Sporządzič rysunek.




PRACA KONTROLNA nr 3- POZIOM ROZSZERZONY

l. Narysowač wykres funkcji $f(x)=\cos 2x-\sin^{2}x \mathrm{i}$ rozwiązač nierównośč $f(x)\displaystyle \geq\frac{1}{4}.$

2. Obliczyč pole trójkąta $ABC 0$ wierzcholkach $A(3,6), B(1,0)$, wiedząc, $\dot{\mathrm{z}}\mathrm{e}$ wysokości

przecinają się $\mathrm{w}$ punkcie (4, 4). Sporządzič rysunek.

3. Dla jakiego kąta ostrego $\alpha$ zachodzi równośč

$\log_{\sin\alpha}(2\cos^{2}\alpha+\sin\alpha\cos\alpha-1)=2$?

4. Dla jakiego parametru $p$ wielomian $W(x) = x^{3}+px^{2}+11x-6$ ma trzy pierwiastki,

$\mathrm{z}$ których jeden jest średnią arytmetyczną pozostafych? Znalez/č wielomian $0$ powyzszej

własności, którego wszystkie pierwiastki są wymierne.

5. Wyznaczyč równania wszystkich prostych stycznych do $\mathrm{k}\mathrm{a}\dot{\mathrm{z}}$ dej $\mathrm{z}$ parabol $y=(x+1)^{2}$

oraz $y=-(x-3)^{2}-2$. Sporz$\Phi$dzič rysunek.

6. $\mathrm{W}$ trójkącie równoramiennym $ABC$ sinus kąta przy wierzcholku $C$ jest równy 3/5. Pod

jakim $\mathrm{k}_{\Phi}\mathrm{t}\mathrm{e}\mathrm{m}$ przecinają się środkowe poprowadzone $\mathrm{z}$ wierzcholków podstawy AB?

Rozwiazania (rękopis) zadań z wybranego poziomu prosimy nadsyłač do

2015r. na adres:

18 1istopada

$\mathrm{W}\mathrm{y}\mathrm{d}\mathrm{z}\mathrm{i}\mathrm{a}\not\subset$ Matematyki

Politechnika Wrocfawska

Wybrzez $\mathrm{e}$ Wyspiańskiego 27

$50-370$ WROCLAW.

Na kopercie prosimy $\underline{\mathrm{k}\mathrm{o}\mathrm{n}\mathrm{i}\mathrm{e}\mathrm{c}\mathrm{z}\mathrm{n}\mathrm{i}\mathrm{e}}$ zaznaczyč wybrany poziom! (np. poziom podsta-

wowy lub rozszerzony). Do rozwiązań nalez $\mathrm{y}$ dołączyč zaadresowaną do siebie koperte

zwrotną $\mathrm{z}$ naklejonym znaczkiem, odpowiednim do wagi listu. Prace niespelniające po-

danych warunków nie będą poprawiane ani odsyłane.

Adres internetowy Kursu: http: //www. im. pwr. edu. pl/kurs







XLVII

KORESPONDENCYJNY KURS

Z MATEMATYKI

listopad 2017 r.

PRACA KONTROLNA $\mathrm{n}\mathrm{r} 3-$ POZIOM PODSTAWOWY

l. Dwaj kolarze jezdzą po torze $\mathrm{w}$ kształcie okręgu ze stałymi prędkościami. $\mathrm{J}\mathrm{e}\dot{\mathrm{z}}$ eli startujq

$\mathrm{z}$ tego samego punktu $\mathrm{i}\mathrm{j}\mathrm{a}\mathrm{d}_{\Phi}\mathrm{w}$ tę $\mathrm{s}\mathrm{a}\mathrm{m}\Phi$ stronę, to szybszy $\mathrm{z}$ nich pierwszy raz ponownie

zrówna się $\mathrm{z}$ wolniejszym, wyprzedzając go ojedno okrązenie, po przejechaniu dokładnie

7 okrązeń. Ilu okrązeń potrzebuje szybszy kolarz $\dot{\mathrm{z}}$ eby dogonič kolegę, $\mathrm{j}\mathrm{e}\dot{\mathrm{z}}$ eli startują $\mathrm{z}$

przeciwlegfych stron toru (tzn. odcinek lączący punkty ich startu jest średnic$\Phi$ kofa)?

2. Liczby $0$ 16\% mniejsza $\mathrm{i}\mathrm{o}$ 43\% większa od ułamka okresowego 0, (75) są pierwiastkami

trójmianu kwadratowego $0$ wspófczynnikach całkowitych względnie pierwszych. Obliczyč

resztę $\mathrm{z}$ dzielenia tego trójmianu przez dwumian $(x-1).$

3. Rozwiązač równanie

$\displaystyle \sin x+\cos x=\frac{1}{\sin x}.$

4. Rozwiązač nierównośč

$\displaystyle \frac{\log_{2}(10-x^{2})}{\log_{2}(4-x)}>2.$

5. Dwa okręgi $0$ promieniach $r\mathrm{i}R$ styczne zewnętrznie $\mathrm{w}$ punkcie $C$, są styczne do prostej

$k\mathrm{w}$ punktach A $\mathrm{i}B$. Wyznaczyč kąt $\angle ACB\mathrm{i}$ promień okręgu opisanego na trójk$\Phi$cie

$ABC.$

6. Dane są punkty $A(2,-2)\mathrm{i}B(8,1)$. Na paraboli $y=x^{2}-x$ znalez/č taki punkt $C, \dot{\mathrm{z}}$ eby

pole trójkąta $ABC$ bylo najmniejsze. Wykonač rysunek.




PRACA KONTROLNA nr 3- POZ1OM ROZSZERZONY

l. Czy wieza zbudowana $\mathrm{z}$ sześciennych klocków $0$ objętościach 1, 3, 9, 27, zmieści $\mathrm{s}\mathrm{i}\mathrm{e}$ na

pófce $0$ wysokości $\displaystyle \frac{15}{2}?$Odpowied $\acute{\mathrm{z}}$ uzasadnič nie $\mathrm{s}\mathrm{t}\mathrm{o}\mathrm{s}\mathrm{u}\mathrm{j}_{\Phi}\mathrm{c}$ obliczeń przyblizonych.

2. Rozwiązač równanie

$\cos 2x=(\sqrt{3}-1)\sin x(\cos x+\sin x).$

3. Sporz$\Phi$dzič staranny wykres funkcji $f(x)=|2^{-|x|+1}-1|-\displaystyle \frac{1}{2}$. Opisač sposób postępowania.

Rozwiązač nierównośč $f(x)>0.$

4. Rozwiązač nierównośč

$\displaystyle \log_{2}x+\log_{2}^{3}x+\log_{2}^{5}x+<\frac{20}{9}.$

5. Podjakim kątem przecinajq się okręgi $0$ równaniach $(x-6)^{2}+y^{2}=9, x^{2}+(y+4)^{2}=25$

(kątem miedzy dwoma okręgami nazywamy kąt między stycznymi $\mathrm{w}$ punkcie przecięcia)?

Znalez/č równanie okręgu, którego środek $\mathrm{l}\mathrm{e}\dot{\mathrm{z}}\mathrm{y}$ na prostej $2x-y=0, \mathrm{i}$ który przecina

$\mathrm{k}\mathrm{a}\dot{\mathrm{z}}\mathrm{d}\mathrm{y}\mathrm{z}$ danych okręgów pod $\mathrm{k}_{\Phi^{\mathrm{t}\mathrm{e}\mathrm{m}}}$ prostym.

6. Boisko do gry $\mathrm{w}$ football amerykański ma kształt prostokąta $0$ dlugości $a\mathrm{i}$ szerokości

$b<a$. Na środku krótszych boków stoją bramki $0$ szerokości $d<b. \mathrm{Z}$ którego miejsca

linii bocznej boiska (czyli dfuzszego boku $\mathrm{P}^{\mathrm{r}\mathrm{o}\mathrm{s}\mathrm{t}\mathrm{o}\mathrm{k}}\Phi^{\mathrm{t}\mathrm{a})}$ widač bramkę pod największym

$\mathrm{m}\mathrm{o}\dot{\mathrm{z}}$ liwym kątem? Wyrazič odpowied $\acute{\mathrm{z}}\mathrm{z}\mathrm{a}$ pomocą wzoru zawierającego symbole $a, b, d,$

a następnie wykonač obliczenia dla wartości $\alpha=110m, b=49m, d=5m.$

Rozwiązania (rękopis) zadań z wybranego poziomu prosimy nadsylač do

2016r. na adres:

18 1istopada

Wydziaf Matematyki

Politechnika Wrocfawska

Wybrzez $\mathrm{e}$ Wyspiańskiego 27

$50-370$ WROCLAW.

Na kopercie prosimy $\underline{\mathrm{k}\mathrm{o}\mathrm{n}\mathrm{i}\mathrm{e}\mathrm{c}\mathrm{z}\mathrm{n}\mathrm{i}\mathrm{e}}$ zaznaczyč wybrany poziom! (np. poziom podsta-

wowy lub rozszerzony). Do rozwiązań nalez $\mathrm{y}$ dolączyč zaadresowaną do siebie koperte

zwrotną $\mathrm{z}$ naklejonym znaczkiem, odpowiednim do wagi listu. Prace niespelniające po-

danych warunków nie będą poprawiane ani odsyłane.

Adres internetowy Kursu: http://www.im.pwr.wroc.pl/kurs







XLVIII

KORESPONDENCYJNY KURS

Z MATEMATYKI

listopad 2018 r.

PRACA KONTROLNA $\mathrm{n}\mathrm{r} 3-$ POZIOM PODSTAWOWY

l. Narysowač wykres funkcji $f(x)=2\cos x-|\cos x|\mathrm{i}$ rozwiazač nierównośč $f(x)<-\displaystyle \frac{3}{2}.$

2. Znalez/č punkt nalezący do paraboli $y^{2}=4x$, którego odleglośč od punktu $A(3,0)$ jest

najmniejsza.

3. Dany jest punkt $A(2,1)$ oraz dwie proste:

$p$: $x+y+2=0, q$: $x-2y-4=0.$

Znalez/č taki punkt $B$ na prostej $q, \dot{\mathrm{z}}\mathrm{e}\mathrm{b}\mathrm{y}$ środek odcinka AB $\mathrm{l}\mathrm{e}\dot{\mathrm{z}}$ af $\mathrm{n}\mathrm{a}$ prostej $p$. Sporządzič

rysunek.

4. Logarytmy liczb l, $3^{x}-2, 3^{x}+4$ tworzą ciąg arytmetyczny ($\mathrm{w}$ podanej kolejności). Ob-

liczyč $x.$

5. Kolejne środki boków czworokąta wypuklego ABCD polączono odcinkami otrzymując

czworokqt EFGH. Jaka figurą jest czworokąt EFGH? Odpowied $\acute{\mathrm{z}}$ uzasadnič. Obliczyč

pole czworokąta ABCD, wiedząc, $\dot{\mathrm{z}}\mathrm{e}$ pole czworokąta EFGH jest równe 5.

6. Rozwiązač nierównośč

$f(x)\displaystyle \leq\frac{4}{f(x)},$

gdzie $f(x)=-\displaystyle \frac{4}{3}x^{2}+2x+\frac{4}{3}.$




PRACA KONTROLNA nr 3- POZ1OM ROZSZERZONY

l. Narysowač wykres funkcji $f(x)=2\displaystyle \cos^{2}x-\sin(2x-\frac{\pi}{2})\mathrm{i}$ rozwiazač nierównośč $|f(x)|<2.$

2. Znalez/č punkt nalezący do paraboli $y^{2}=2x$, którego odlegfośč od prostej $x-2y+6=0$

jest najmniejsza.

3. Wielomian $w(x)=x^{4}+\alpha x^{3}+bx^{2}+cx+d$ jest podzielny przez trójmian $x^{2}-x-2, \mathrm{a}$

jego wykres jest symetryczny względem osi $0y$. Wyznaczyč wartości parametrów $a, b, c, d$

$\mathrm{i}$ rozwiqzač nierównośč $w(x+1)\leq w(x-2).$

4. Rozwiązač nierównośč

$\log x+\log^{3}x+\log^{5}x+\leq 2\sqrt{5}.$

5. Punkt $S$ jest środkiem boku AB $\mathrm{w}$ trójkącie $ABC$. Ponadto $AC\neq BC$ oraz $\angle BAC+$

$\angle SCB=90^{\mathrm{o}}$ Niech $D$ bedzie punktem przeciecia symetralnej AB $\mathrm{z}$ prostą $AC$. Udo-

wodnič, $\dot{\mathrm{z}}\mathrm{e}$ na czworokącie SBDC $\mathrm{m}\mathrm{o}\dot{\mathrm{z}}$ na opisač okrąg. Dlaczego musimy zalozyč, $\dot{\mathrm{z}}\mathrm{e}$

$AC\neq BC$?

6. Wyznaczyč równanie zbioru wszystkich środków tych cięciw paraboli $y = x^{2}$, które

przechodzą przez punkt $A(0,2).$

Rozwiązania (rękopis) zadań z wybranego poziomu prosimy nadsyłač do

2018r. na adres:

18 1istopada

Wydziaf Matematyki

Politechnika Wrocfawska

Wybrzez $\mathrm{e}$ Wyspiańskiego 27

$50-370$ WROCLAW.

Na kopercie prosimy $\underline{\mathrm{k}\mathrm{o}\mathrm{n}\mathrm{i}\mathrm{e}\mathrm{c}\mathrm{z}\mathrm{n}\mathrm{i}\mathrm{e}}$ zaznaczyč wybrany poziom! (np. poziom podsta-

wowy lub rozszerzony). Do rozwiązań nalez $\mathrm{y}$ dołączyč zaadresowaną do siebie kopertę

zwrotną $\mathrm{z}$ naklejonym znaczkiem, odpowiednim do wagi listu. Prace niespełniające po-

danych warunków nie będą poprawiane ani odsyłane.

Uwaga. Wysylając nam rozwiazania zadań uczestnik Kursu udostępnia Politechnice Wroclawskiej

swoje dane osobowe, które przetwarzamy wyłącznie $\mathrm{w}$ zakresie niezbędnym do jego prowadzenia

(odesłanie zadań, prowadzenie statystyki). Szczegółowe informacje $0$ przetwarzaniu przez nas danych

osobowych są dostępne na stronie internetowej Kursu.

Adres internetowy Kursu: http: //www. im. pwr. edu. pl/kurs







XLIX

KORESPONDENCYJNY KURS

Z MATEMATYKI

listopad 2019 r.

PRACA KONTROLNA $\mathrm{n}\mathrm{r} 3-$ POZIOM PODSTAWOWY

l. Znalez$\acute{}$č największą wartośč funkcji

$f(x)=\displaystyle \frac{2}{\sqrt{4x^{2}-12x+11}}$

$\mathrm{i}$ rozwiqzač nierównośč $f(x)\geq 1.$

2. Rozwiązač równanie

$(1+\cos 4x)\sin 2x=\cos^{2}2x.$

3. Rozwiązač równanie

$\log_{\sqrt{5}}(4^{x}-6)-\log_{\sqrt{5}}(2^{x}-2)=2.$

4. Stosunek długości przekątnych rombu jest równy 5:l2. Obliczyč stosunek pola rombu do

do pola koła wpisanego $\mathrm{w}$ ten romb.

5. Dane są punkty $A(1,1)\mathrm{i}B(7,4)$. Na paraboli $y=x^{2}+x+3$ znalez/č taki punkt $C, \dot{\mathrm{z}}$ eby

pole trójkąta $ABC$ było najmniejsze. Wykonač rysunek.

6. Ramiona trójk$\Phi$ta równoramiennego zawarte $\mathrm{s}\Phi^{\mathrm{W}}$ prostych $0$ równaniach $8x-y+17=0$

oraz $4x+7y-59 = 0$, a jego podstawa przechodzi przez punkt $P(0,2)$. Wyznaczyč

równanie prostej zawierajacej podstawę $\mathrm{i}$ obliczyč pole tego trójkqta.




PRACA KONTROLNA nr 3- POZIOM ROZSZERZONY

l. Dla jakich wartości parametru $m$ równanie

$x^{2}-2(m-4)x+m^{2}+5m+6=0$

ma dwa rózne pierwiastki rzeczywiste, których suma odwrotności jest dodatnia?

2. Rozwiązač równanie

$\displaystyle \frac{1}{\sin^{2}2x}+\mathrm{t}\mathrm{g}x-$ ctg $x=2.$

3. Rozwiązač układ równań

$\left\{\begin{array}{l}
- 2\mathrm{l}\mathrm{o}\mathrm{l}\mathrm{o}\mathrm{g}\mathrm{g}2(- x- \mathrm{l}\mathrm{o}y\mathrm{g})(- x+1y)\\
- \mathrm{l}\mathrm{l}\mathrm{o}\mathrm{o}\mathrm{g}\mathrm{g}xy-- \mathrm{l}\mathrm{l}\mathrm{o}\mathrm{o}\mathrm{g}\mathrm{g}73
\end{array}\right.$

$=1$

$=-1.$

4. Dany jest trójkąt $ABC, \mathrm{w}$ którym $\displaystyle \angle ACB=\frac{2\pi}{3}$. Dwusieczna kąta $ACB$ przecina prostą

przechodzqca przez punkt $A\mathrm{i}$ równoległq do boku $BC\mathrm{w}$ punkcie $P$, a prostq przecho-

dzącą przez punkt $B\mathrm{i}$ równolegl$\Phi$ do boku $AC\mathrm{w}$ punkcie $Q$. Udowodnič, $\dot{\mathrm{z}}\mathrm{e}AQ=BP.$

5. Wyznaczyč stosunek promienia okręgu wpisanego $\mathrm{w}$ {\it romb ABCD} $0$ kącie ostrym $\alpha=$

$\angle DAB$ do promienia okregu opisanego na trójkącie $ABD$. Sprawdzič dlajakiego kata $\alpha$

stosunek ten jest najwięszy.

6. Wyznaczyč równanie zbioru wszystkich środków tych cięciw paraboli $y = x^{2}$, które

zaczynają się $\mathrm{w}$ punkcie $A(1,1)$. Rozwiązanie zilustrowač rysunkiem.

Rozwiązania (rękopis) zadań z wybranego poziomu prosimy nadsyfač do

2019r. na adres:

18 1istopada

Wydziaf Matematyki

Politechnika Wrocfawska

Wybrzez $\mathrm{e}$ Wyspiańskiego 27

$50-370$ WROCLAW.

Na kopercie prosimy $\underline{\mathrm{k}\mathrm{o}\mathrm{n}\mathrm{i}\mathrm{e}\mathrm{c}\mathrm{z}\mathrm{n}\mathrm{i}\mathrm{e}}$ zaznaczyč wybrany poziom! (np. poziom podsta-

wowy lub rozszerzony). Do rozwiązań nalez $\mathrm{y}$ dołączyč zaadresowaną do siebie koperte

zwrotną $\mathrm{z}$ naklejonym znaczkiem, odpowiednim do wagi listu. Prace niespelniające po-

danych warunków nie bedą poprawiane ani odsyłane.

Uwaga. Wysylając nam rozwi\S zania zadań uczestnik Kursu udostępnia Politechnice Wroclawskiej

swoje dane osobowe, które przetwarzamy wyłącznie $\mathrm{w}$ zakresie niezbednym do jego prowadzenia

(odesfanie zadań, prowadzenie statystyki). Szczegófowe informacje $0$ przetwarzaniu przez nas danych

osobowych są dostępne na stronie internetowej Kursu.

Adres internetowy Kursu: http: //www. im. pwr. edu. pl/kurs







L

KORESPONDENCYJNY KURS

Z MATEMATYKI

listopad 2020 r.

PRACA KONTROLNA nr 3- POZIOM PODSTAWOWY

l. Punkty $K\mathrm{i}L$ sa środkami boków AB $\mathrm{i}CD$ czworokąta ABCD. Wykaz, $\dot{\mathrm{z}}\mathrm{e}$

$\displaystyle \vec{KL}=\frac{1}{2}(\vec{AD}+\vec{BC}).$

Wykonaj rysunek.

2. $\mathrm{W}$ pewnym ciągu geometrycznym $\mathrm{k}\mathrm{a}\dot{\mathrm{z}}\mathrm{d}\mathrm{y}$ ($\mathrm{z}$ wyjątkiem pierwszego) wyraz jest róznicą

wyrazu następnego $\mathrm{i}$ poprzedniego. Znajd $\acute{\mathrm{z}}$ iloraz tego $\mathrm{c}\mathrm{i}_{\Phi \mathrm{g}}\mathrm{u}.$

3. Rozwiąz nierównośč

$[\log_{0,2}(x-1)]^{2}>4.$

4. Rozwiąz równanie

$\displaystyle \sin^{2}x+\frac{1}{2}\sin 2x=1.$

5. Statek plynie prosto $\mathrm{w}$ kierunku klifu. $K_{\Phi^{\mathrm{t}}}$ elewacji (kąt utworzony przez linię $\mathrm{P}^{\mathrm{o}\mathrm{z}\mathrm{i}\mathrm{o}\mathrm{m}}\Phi$

$\mathrm{i}$ odcinek fączący obserwatora na statku ze szczytem klifu) wynosi początkowo $\alpha$, ale po

przepłynięciu przez statek $d$ metrów wzrasta do $\beta$. Wyznacz wysokośč klifu. Wykonaj

obliczenia dla wartości $\alpha=10^{\mathrm{o}}, \beta=15^{\mathrm{o}}, d=50.$

6. Obliczyč pole cześci wspólnej trzech kól $0$ promieniach $r \mathrm{i}$ środkach $\mathrm{w}$ wierzchołkach

trójk$\Phi$ta równobocznego $0$ boku $r\sqrt{2}.$




PRACA KONTROLNA nr 3- POZIOM ROZSZERZONY

1. Znajd $\acute{\mathrm{z}}$ taki ciąg arytmetyczny, $\mathrm{w}$ którym suma pierwszych $n$ wyrazów równajest $n^{2}$ dla

wszystkich $n\in \mathbb{N}.$

2. $\mathrm{W}$ sześciok$\Phi$cie foremnym ABCDEF punkty $M \mathrm{i} N$ są środkami boków $CD \mathrm{i}$ {\it DE}.

Wyznacz $\mathrm{k}\mathrm{a}\mathrm{t}$ między wektorami $\vec{AM}\mathrm{i}\vec{BN}.$

3. Rozwiąz nierównośč

$\log_{2x}(x^{2}-5x+6)<1.$

4. Rozwiąz równanie

$\displaystyle \cos 2x-3\cos x=4\cos^{2}\frac{x}{2}.$

5. Znajd $\acute{\mathrm{z}}$ najmniejszą wartośč ilorazu pola powierzchni bocznej stozka $\mathrm{i}$ pola powierzchni

kuli wpisanej $\mathrm{w}$ ten stozek oraz kąt rozwarcia stozka realizujący tę wartośč najmniejszq.

6. Na dachu budynku stoi antena, której wysokośč chcemy wyznaczyč nie wchodząc na

górę. Urządzenie pomiarowe ustawione $\mathrm{w}$ pewnej odległości od budynku zmierzyło kąty

między pionem a odcinkiem $l_{\Phi}$czącym punkt pomiaru ze szczytem anteny oraz między

pionem a odcinkiem łączącym punkt pomiaru $\mathrm{z}$ podstawą anteny. Otrzymano kąty $\alpha_{1}$

$\mathrm{i}\beta_{1}$ odpowiednio. Nastepnie przesunięto urządzenie $0d$ metrów $\mathrm{w}$ kierunku budynku

bez zmiany wysokości punktu pomiarowego $\mathrm{i}$ ponowiono pomiary, otrzymując kąty $\alpha_{2}$

$\mathrm{i}\beta_{2}$. Podaj wzór na wysokośč anteny $\mathrm{i}$ wykonaj obliczenia dla kątów $\alpha_{1}=53^{\mathrm{o}}, \beta_{1}=55^{\mathrm{o}},$

$\alpha_{2}=51^{\mathrm{o}}, \beta_{2}=53.04^{\mathrm{o}}$, oraz $d=5m.$

Rozwiązania (rękopis) zadań z wybranego poziomu prosimy nadsyfač do

2020r. na adres:

20 1istopada

Wydziaf Matematyki

Politechnika Wrocfawska

Wybrzez $\mathrm{e}$ Wyspiańskiego 27

$50-370$ WROCLAW.

Na kopercie prosimy $\underline{\mathrm{k}\mathrm{o}\mathrm{n}\mathrm{i}\mathrm{e}\mathrm{c}\mathrm{z}\mathrm{n}\mathrm{i}\mathrm{e}}$ zaznaczyč wybrany poziom! (np. poziom podsta-

wowy lub rozszerzony). Do rozwiązań nalez $\mathrm{y}$ dołączyč zaadresowaną do siebie kopertę

zwrotną $\mathrm{z}$ naklejonym znaczkiem, odpowiednim do formatu listu. Polecamy stosowanie

kopert formatu C5 $(160\mathrm{x}230\mathrm{m}\mathrm{m})$ ze znaczkiem $0$ wartości 3,30 zł. Na $\mathrm{k}\mathrm{a}\dot{\mathrm{z}}$ dą wiekszą

kopertę nalez $\mathrm{y}$ nakleič $\mathrm{d}\mathrm{r}\mathrm{o}\dot{\mathrm{z}}$ szy znaczek. Prace niespełniające podanych warunków nie

będą poprawiane ani odsyłane.

Uwaga. Wysylając nam rozwiazania zadań uczestnik Kursu udostępnia Politechnice Wroclawskiej

swoje dane osobowe, które przetwarzamy wylącznie $\mathrm{w}$ zakresie niezbędnym do jego prowadzenia

(odesfanie zadań, prowadzenie statystyki). Szczegófowe informacje $0$ przetwarzaniu przez nas danych

osobowych są dostępne na stronie internetowej Kursu.

Adres internetowy Kursu: http://www. im.pwr.edu.pl/kurs







LI KORESPONDENCYJNY KURS

Z MATEMATYKI

listopad 2021 r.

PRACA KONTROLNA nr $3$- POZIOM PODSTAWOWY

l. Narysuj staranny wykres funkcji $f(x)=|\sin x|\cos x\mathrm{i}$ rozwiąz nierównośč $|f(x)|\displaystyle \leq\frac{1}{4}.$

2. Wyznacz dziedzinę funkcji

$f(x)=\displaystyle \log_{2}(\frac{3x-5}{x-2}+1)$

$\mathrm{i}$ sprawd $\acute{\mathrm{z}}$ dla jakich argumentów funkcja ta przyjmuje wartości dodatnie.

3. $\mathrm{W}$ trójkącie dane są dlugości dwóch boków a $\mathrm{i}b$. Oblicz długośč trzeciego boku, wiedząc,

$\dot{\mathrm{z}}\mathrm{e}$ suma wysokości poprowadzonych do boków $a\mathrm{i}b$ jest równa trzeciej wysokości.

4. Niech ABCDEF będzie sześciokątem foremnym. Wykaz$\cdot, \dot{\mathrm{z}}\mathrm{e}$

$\vec{AB}+\vec{AC}+\vec{AD}+\vec{AE}+\vec{AF}=3\vec{AD}.$

5. Na krzywej $0$ równaniu $y= \sqrt{2x}$znajd $\acute{\mathrm{z}}$ miejsce, które polozone jest najblizej punktu

$P(3,0).$ Sporząd $\acute{\mathrm{z}}$ rysunek.

6. Dla jakich wartości parametru $m$ pierwiastkiem wielomianu

$w(x)=2x^{3}-7x^{2}-(m^{2}-12)x+m^{2}+m-6$

jest $x=3$? Dla znalezionych wartości $m$ wyznacz pozostałe pierwiastki $w(x).$




PRACA KONTROLNA $\mathrm{n}\mathrm{r} 3-$ POZIOM ROZSZERZONY

l. Dany jest trójkąt $0$ wierzchofkach $A(-1,3), B(-4,-1), \mathrm{i}C(3,0).$ Znajd $\acute{\mathrm{z}}$ kąt pomiędzy

wysokością tego trójkąta poprowadzonq $\mathrm{z}$ wierzcholka $A\mathrm{i}$ bokiem $AC$. Oblicz pole tego

trójkąta.

2. Narysuj wykres funkcji $f(x)=\sin^{2}x-\cos 2x\mathrm{i}\mathrm{r}\mathrm{o}\mathrm{z}\mathrm{w}\mathrm{i}_{\Phi}\dot{\mathrm{z}}$ nierównośč $f(x)\displaystyle \geq-\frac{1}{4}.$

3. Zaznacz na płaszczyz$\acute{}$nie zbiór punktów, których współrzędne spełniajq nierównośč

$\log_{y}(\log_{x}y)>0.$

4. Reszta $\mathrm{z}$ dzielenia wielomianu $w(x)=x^{4}+ax^{3}+(b+2)x^{2}+bx+a-3$ przez trójmian

$x^{2}+2x-8$ wynosi $-5x+40$. Wyznacz wartośč parametrów $a\mathrm{i}b$ oraz rozwiąz nierównośč

$w(x-1)\geq w(x+1).$

5. Dany jest trapez ABCD $0$ podstawach AB $\mathrm{i}CD, \mathrm{w}$ którym $\angle ABC=90^{\mathrm{o}}$ Dwusieczna

kąta BAD przecina odcinek $BC\mathrm{w}$ punkcie $P$. Niech $Q$ będzie rzutem prostopadłym

punktu $P$ na prostą $AD$. Wykaz, $\dot{\mathrm{z}}\mathrm{e}\mathrm{j}\mathrm{e}\dot{\mathrm{z}}$ eli pole czworokqta APCD jest równe polu

trójk$\Phi$ta $ABP$, to $|PC|=|DQ|.$

6. Boisko do gry $\mathrm{w}$ piłkę recznq jest prostokątem $0$ długości $40\mathrm{m}\mathrm{i}$ szerokości $20\mathrm{m}$. Bramki

$\mathrm{m}\mathrm{a}\mathrm{j}_{\Phi}$ szerokośč $3\mathrm{m}\mathrm{i}$ stoją dokfadnie na środku linii bramkowej (krótszego boku pro-

stokąta). $\mathrm{Z}$ jakiego punktu linii bocznej (dłuzszego boku prostokąta) widač bramkę pod

najwiekszym $\mathrm{m}\mathrm{o}\dot{\mathrm{z}}$ liwym kątem?

Rozwiązania (rękopis) zadań z wybranego poziomu prosimy nadsyfač do

2021r. na adres:

201istopada

Wydziaf Matematyki

Politechnika Wrocfawska

Wybrzez $\mathrm{e}$ Wyspiańskiego 27

$50-370$ WROCLAW,

lub elektronicznie, za pośrednictwem portalu talent. $\mathrm{p}\mathrm{w}\mathrm{r}$. edu. pl

Na kopercie prosimy $\underline{\mathrm{k}\mathrm{o}\mathrm{n}\mathrm{i}\mathrm{e}\mathrm{c}\mathrm{z}\mathrm{n}\mathrm{i}\mathrm{e}}$ zaznaczyč wybrany poziom! (np. poziom podsta-

wowy lub rozszerzony). Do rozwiązań nalez $\mathrm{y}$ dołączyč zaadresowaną do siebie koperte

zwrotną $\mathrm{z}$ naklejonym znaczkiem, odpowiednim do formatu listu. Polecamy stosowanie

kopert formatu C5 $(160\mathrm{x}230\mathrm{m}\mathrm{m})$ ze znaczkiem $0$ wartości 3,30 zł. Na $\mathrm{k}\mathrm{a}\dot{\mathrm{z}}$ dą wiekszą

kopertę nalez $\mathrm{y}$ nakleič drozszy znaczek. Prace niespelniające podanych warunków nie

bedą poprawiane ani odsyłane.

Uwaga. Wysyłając nam rozwi\S zania zadań uczestnik Kursu udostępnia Politechnice Wroclawskiej

swoje dane osobowe, które przetwarzamy wyłącznie $\mathrm{w}$ zakresie niezbednym do jego prowadzenia

(odesfanie zadań, prowadzenie statystyki). Szczególowe informacje $0$ przetwarzaniu przez nas danych

osobowych sa dostępne na stronie internetowej Kursu.

Adres internetowy Kursu: http://www.im.pwr.edu.pl/kurs







LII

KORESPONDENCYJNY KURS

Z MATEMATYKI

listopad 2022 r.

PRACA KONTROLNA $\mathrm{n}\mathrm{r} 3-$ POZIOM PODSTAWOWY

1. $\mathrm{W}$ trójk$\Phi$cie $ABC$ wpisanym $\mathrm{w}$ okrąg $0$ środku $S\mathrm{i}$ promieniu $r$ dany jest kąt $\alpha=\angle ABC.$

Oblicz pole trójkqta $ASC.$

2. Rozwiąz równanie

$|\displaystyle \sin x|+|\cos x|=\frac{\sqrt{6}}{2}.$

3. Dana jest funkcja

$f(x)=\displaystyle \cos(2x-\frac{\pi}{6})$

Narysuj starannie wykres funkcji $f(x)$. Rozwiqz nierównośč $(f(x))^{2}\displaystyle \geq\frac{1}{2}.$

4. Niech $\alpha, \beta \mathrm{i}\gamma$ oznaczają kąty pewnego trójkąta. Wykaz, $\dot{\mathrm{z}}\mathrm{e}\mathrm{j}\mathrm{e}\dot{\mathrm{z}}$ eli

-ssiinn $\beta\alpha =$2 cos $\gamma$,

to ten trójkąt jest równoramienny.

5. Na okręgu $0$ promieniu $r$ opisano trapez prostokątny, którego najkrótszy bok jest równy

$\displaystyle \frac{4}{3}r$. Oblicz pole tego trapezu.

6. Pewną górę widač najpierw pod kątem $\alpha$ (jest to kąt między linią poziomą, a odcinkiem

lączącym szczyt $\mathrm{z}$ obserwatorem), a po przyblizeniu się do niej $\mathrm{o}d$ metrów widač $\mathrm{j}\mathrm{a}$ pod

nieco większym kątem $\beta$. Wyznaczyč względną wysokośč tej góry. Wykonač obliczenia

dla wartości $\alpha=41^{\mathrm{o}}, \beta=45^{\mathrm{o}}, d=90\mathrm{m}.$




PRACA KONTROLNA $\mathrm{n}\mathrm{r} 3-$ POZIOM ROZSZERZONY

l. Udowodnij, $\dot{\mathrm{z}}\mathrm{e}$

$\cos 4x=1-8\cos^{2}x+8\cos^{4}x.$

Wykorzystując ten wzór, znajd $\acute{\mathrm{z}}$ wartośč $\displaystyle \cos\frac{\pi}{24}.$

2. Wykaz$\cdot, \dot{\mathrm{z}}\mathrm{e}$ dla $\mathrm{k}\mathrm{a}\dot{\mathrm{z}}$ dego trójkata zachodzi nierównośč

-21  {\it r} $<$ -{\it h}1{\it a} $+$ -{\it h}1{\it b} $<$ -{\it r}1,

gdzie $h_{a}, h_{b}$ sq wysokościami, a $r$ promieniem okregu wpisanego $\mathrm{w}$ ten trójkąt.

3. Dana jest funkcja $f(x) = \sin 4x$ ctg $2x-\displaystyle \frac{1}{2}$. Rozwiąz nierównośč $f(x) \geq$

staranny wykres $f(x).$

l i narysuj

4. Przekątne trapezu dzielą ten trapez na cztery trójkąty. Pola tych dwóch trójkątów, któ-

rych bokami są podstawy trapezu równe są $S_{a}\mathrm{i}S_{b}$. Oblicz pole tego trapezu.

5. Manipulator robota składa się $\mathrm{z}$ dwóch ramion $0$ długościach $l_{1}\mathrm{i}l_{2}$, połączonych prze-

gubem. Pierwsze ramię umieszczono $\mathrm{w}$ poczatku układu wspófrzednych.

Niech $\alpha$ oznacza kąt miedzy pierwszym ramieniem $\mathrm{i}$ osią

$Ox$, a $\beta$ - kąt mi dzy drugim ramieniem $\mathrm{i}$ kierunkiem

pierwszego ramienia (patrz rysunek). Wyznacz wspól-

rzędne końca drugiego ramienia (punktu $P$) $\mathrm{w}$ zalezno-

sci od $\mathrm{k}$ tow $\alpha \mathrm{i}\beta$. Sprawdz, czy punkt $P\mathrm{m}\mathrm{o}\dot{\mathrm{z}}\mathrm{e}$ przesu-

$\mathrm{n} \mathrm{c}$ si do punktow $S(2,1)$ oraz $Q(3,-1)\mathrm{j}\mathrm{e}\dot{\mathrm{z}}$ eli $l_{1} =3,$

$l_{2} = 2$ oraz ruchy manipulatora ograniczone są $\mathrm{t}\mathrm{a}\mathrm{k}, \dot{\mathrm{z}}\mathrm{e}$

$\alpha, \beta\in -\displaystyle \frac{2\pi}{3}, \displaystyle \frac{2\pi}{3} \mathrm{J}\mathrm{e}\dot{\mathrm{z}}$ eli $\mathrm{t}\mathrm{a}\mathrm{k}$, to wskaz konkretne $\mathrm{k}$ ty $\alpha \mathrm{i}$

$\beta$ (podaj przyblizenia, jesli nie $\mathrm{m}\mathrm{o}\dot{\mathrm{z}}$ na okreslic dokladnej

ich wartości), a jeśli $\mathrm{n}\mathrm{i}\mathrm{e}$, to uzasadnij dlaczego.
\begin{center}
\includegraphics[width=60.048mm,height=45.720mm]{./KursMatematyki_PolitechnikaWroclawska_3_2022_page1_images/image001.eps}
\end{center}
$y$

{\it P}

2

$l_{2}$

$\prime\prime\beta_{\rightarrow}$

1

$l_{1}$

$\alpha$

1 $2 3 4 5 x$

$-1$

$-2$

6. Okrąg $0$ promieniu $r$ toczy się wewnętrznie bez poślizgu po okręgu $0$ promieniu $2r$. Ja-

ką linię zakreśla ustalony (dowolnie wybrany) punkt $P$ ruchomego okręgu? Wskazówka:

rozwaz dwa rózne pofozenia mniejszego okręgu $\mathrm{i}\mathrm{s}$prawd $\acute{\mathrm{z}}$ gdzie przesuwa się punkt stycz-

ności, skorzystaj ze związku między dlugością łuku, kqtem środkowym opartym na tym

fuku $\mathrm{i}$ promieniem okręgu.

Rozwiązania (rękopis) zadań $\mathrm{z}$ wybranego poziomu prosimy nadsylač do $20.11.2022\mathrm{r}$. na

adres:

Wydziaf Matematyki

Politechnika Wrocfawska

Wybrzez $\mathrm{e}$ Wyspiańskiego 27

$50-370$ WROCLAW,

lub elektronicznie, za pośrednictwem portalu talent. $\mathrm{p}\mathrm{w}\mathrm{r}$. edu. pl

Na kopercie prosimy $\underline{\mathrm{k}\mathrm{o}\mathrm{n}\mathrm{i}\mathrm{e}\mathrm{c}\mathrm{z}\mathrm{n}\mathrm{i}\mathrm{e}}$ zaznaczyč wybrany poziom! (np. poziom podsta-

wowy $\mathrm{l}\mathrm{u}\mathrm{b}$ rozszerzony). Do rozwiązań nalez $\mathrm{y}$ dolączyč zaadresowaną do siebie kopertę

zwrotną $\mathrm{z}$ naklejonym znaczkiem, odpowiednim do formatu listu. Prace niespełniające

podanych warunków nie będą poprawiane ani odsyłane.

Uwaga. Wysyfaj\S c nam rozwi\S zania zadań uczestnik Kursu udostępnia Politechnice Wrocfawskiej

swoje dane osobowe, które przetwarzamy wyłącznie $\mathrm{w}$ zakresie niezbędnym do jego prowadzenia

(odeslanie zadań, prowadzenie statystyki). Szczególowe informacje $0$ przetwarzaniu przez nas danych

osobowych s\S dostępne na stronie internetowej Kursu.

Adres internetowy Kursu: http: //www. im. pwr. edu. pl/kurs







XLIV

KORESPONDENCYJNY KURS

Z MATEMATYKI

grudzień 2014 r.

PRACA KONTROLNA $\mathrm{n}\mathrm{r} 4-$ POZIOM PODSTAWOWY

l. Dla jakich kątów $\alpha\in\langle 0,  2\pi\rangle$ równanie $2x^{2}-2(2\cos\alpha-1)x+2\cos^{2}\alpha-5\cos\alpha+2=0$

ma dwa rózne pierwiastki rzeczywiste?

2. Dane są punkty $A(-2,0), B(2,4)$ oraz $C(1,5)$. Oblicz pole trapezu ABCD, wiedząc, $\dot{\mathrm{z}}\mathrm{e}$

punkt $D$ jest jednakowo odległy od punktów A $\mathrm{i}B.$

3. $\mathrm{W}$ trójkącie równoramiennym kąt przy podstawie ma miarę $30^{\mathrm{o}}$ Oblicz stosunek dfu-

gości promienia okręgu opisanego na trójkącie do długości promienia okręgu wpisanego

$\mathrm{w}$ trójkąt.

4. Płaszczyzna przechodząca przez środek dolnej podstawy walca jest nachylona do pod-

stawy pod kątem $\alpha \mathrm{i}$ przecina górną podstawe walca wzdłuz cięciwy dlugości $a$. Cięciwa

ta odcina $\mathrm{f}\mathrm{u}\mathrm{k}$, na którym oparty jest $\mathrm{k}_{\Phi^{\mathrm{t}}}$ środkowy $0$ mierze $120^{\mathrm{o}}$ Oblicz objętośč walca.

5. Niech $x_{1}\mathrm{i}x_{2}$ będą pierwiastkami wielomianu $p(x)=x^{2}-x+a$, a $x_{3}\mathrm{i}x_{4}-$ pierwiastkami

wielomianu $q(x)=x^{2}-4x+b$. Dlajakich $a\mathrm{i}b$ liczby $x_{1}, x_{2}, x_{3}, x_{4}$ są kolejnymi wyrazami

ciągu geometrycznego?

6. Na dwóch zewnętrznie stycznych kulach opisano stozek $\mathrm{t}\mathrm{a}\mathrm{k}, \dot{\mathrm{z}}\mathrm{e}$ środki tych kul lezą na

wysokości stozka. Promień mniejszej kulijest równy $r$, a stosunek objętości kul wynosi 8.

Oblicz pole powierzchni bocznej stozka.




PRACA KONTROLNA nr 4- POZ1OM ROZSZERZONY

l. Dane są proste $y = 4x \mathrm{i} y = x-2$ oraz punkt $M = (1,2)$. Wyznacz współrzędne

punktów $A\mathrm{i}B\mathrm{l}\mathrm{e}\dot{\mathrm{z}}$ ących odpowiednio na danych prostych takich, $\dot{\mathrm{z}}\mathrm{e}$ punkty $A, B,  M\mathrm{s}\Phi$

współliniowe oraz $\displaystyle \frac{|AM|}{|BM|}=\frac{2}{3}.$

2. $\mathrm{W}$ równoległoboku $0$ kącie ostrym $60^{\mathrm{o}}$ stosunek kwadratów długości przekątnych wynosi

1:3. Oblicz stosunek dlugości dwóch sąsiednich boków.

3. Niech $a, b, c, d$ będą kolejnymi liczbami naturalnymi. Pokaz, $\dot{\mathrm{z}}\mathrm{e}$ wielomian $w(x)=ax^{3}-$

$bx^{2}-cx+d$ ma trzy pierwiastki rzeczywiste, wśród których co najmniej jeden jest liczbą

cafkowitą. Dla jakich parametrów $a, b, c, d$ suma tych pierwiastków jest największa?

4. Dla jakich kątów $\alpha\in\langle 0,  2\pi\rangle$ spełniona jest nierównośč

$2^{\sin^{2}x}+\sqrt[4]{2}\cdot 2^{\cos^{2}x}\leq\sqrt{2}+\sqrt[4]{8}$?

5. $\mathrm{W}$ ostrosfupie prawidfowym czworokątnym $0$ krawędzi podstawy $a$ stosunek dlugości

krawędzi podstawy do wysokości wynosi 2:3. Ostrosłup przecięto p1aszczyzna przecho-

dzącą przez krawędz/ podstawy $\mathrm{i}$ prostopadlą do przeciwleglej ściany bocznej. Oblicz pole

otrzymanego przekroju.

6. Wierzchołek stozka jest środkiem kuli a brzeg podstawy stozka zawiera $\mathrm{s}\mathrm{i}\mathrm{e}\mathrm{w}$ powierzchni

kuli. Pole powierzchni calkowitej stozka stanowi $\displaystyle \frac{1}{4}$ pola powierzchni kuli. Oblicz stosunek

objętości stozka do objętości kuli.

Rozwiązania (rękopis) zadań z wybranego poziomu prosimy nadsylač do

na adres:

18 grudnia 20l4r.

Instytut Matematyki $\mathrm{i}$ Informatyki

Politechniki Wrocfawskiej

Wybrzez $\mathrm{e}$ Wyspiańskiego 27

$50-370$ WROCLAW.

Na kopercie prosimy $\underline{\mathrm{k}\mathrm{o}\mathrm{n}\mathrm{i}\mathrm{e}\mathrm{c}\mathrm{z}\mathrm{n}\mathrm{i}\mathrm{e}}$ zaznaczyč wybrany poziom! (np. poziom podsta-

wowy lub rozszerzony). Do rozwiązań nalez $\mathrm{y}$ dołączyč zaadresowaną do siebie kopertę

zwrotną $\mathrm{z}$ naklejonym znaczkiem, odpowiednim do wagi listu. Prace niespełniające po-

danych warunków nie będą poprawiane ani odsyłane.

Adres internetowy Kursu: http://www.im.pwr.wroc.pl/kurs







XLV

KORESPONDENCYJNY KURS

Z MATEMATYKI

grudzień 2015 r.

PRACA KONTROLNA $\mathrm{n}\mathrm{r} 4-$ POZIOM PODSTAWOWY

l. Znalez$\acute{}$č miejsca zerowe $\mathrm{i}$ naszkicowač wykres funkcji $f(x)=x^{2}-x-5|x|+5$. Wyznaczyč

najmniejszą $\mathrm{i}$ największ$\Phi$ wartośč tej funkcji na przedziale [-5, 5].

2. Romb $0$ boku $a\mathrm{i}$ kącie ostrym $\alpha$ podzielono na trzy części $0$ równych polach odcinka-

mi majacymi wspólny początek $\mathrm{w}$ wierzchofku kąta ostrego $\mathrm{i}$ końce na bokach rombu.

Obliczyč dlugości tych odcinków. Wykonač odpowiedni rysunek.

3. Odcinek $0$ końcach $A(-1,-1) \mathrm{i} B(3,2)$ jest podstawą trapezu. Druga podstawa jest

trzy razy dluzsza $\mathrm{i}$ ma środek $\mathrm{w}$ punkcie $P(1,5)$. Wyznaczyč wspófrzędne pozostafych

wierzchołków trapezu $\mathrm{i}$ obliczyč jego pole.

4. $\mathrm{W}$ okrqg $0$ promieniu l wpisujemy trójkąt równoboczny $\mathrm{i}$ zakreślamy odcinki kofa, które

$ 1\mathrm{e}\mathrm{Z}\otimes$ na zewnatrz trójkąta. $\mathrm{W}$ otrzymany trójkąt wpisujemy okrąg $\mathrm{i}$ powtarzamy proce-

durę, zaznaczając za $\mathrm{k}\mathrm{a}\dot{\mathrm{z}}$ dym razem odcinki kolejnych kół znajdujące się poza kolejnym

trójk$\Phi$tem. Obliczyč pole zaznaczonego obszaru po sześciu krokach, czyli po narysowaniu

sześciu trójkątów.

5. Sześcian podzielono na dwie bryły plaszczyznq przechodzącą przez krawęd $\acute{\mathrm{z}}$ podsta-

wy. Jedna częśč ma 5, a druga 6 ścian. Po1e powierzchni ca1kowitej bry1y, która ma 5

ścianjest równa połowie pola powierzchni sześcianu. Wyznaczyč tangens kąta nachylenia

płaszczyzny dzielącej sześcian do pfaszczyzny podstawy.

6. Rozwazamy zbiór liczb cafkowitych dodatnich równych co najwyzej l800, które nie dzielą

się ani przez 5 ani przez 6. Ob1iczyč sumę 1iczb $\mathrm{z}$ tego zbioru. Ile $\mathrm{w}$ tym zbiorze jest liczb

parzystych, a ile nieparzystych?




PRACA KONTROLNA nr 4- POZ1OM ROZSZERZONY

l. Punkty $A(2,0)\mathrm{i}B(0,2)$ są wierzchołkami podstawy trójkąta równoramiennego. Znalez/č

wspófrzędne wierzchołka $C, \mathrm{w}\mathrm{i}\mathrm{e}\mathrm{d}\mathrm{z}\Phi^{\mathrm{C}}, \dot{\mathrm{z}}\mathrm{e}$ środkowe $AD\mathrm{i}$ {\it BE} $\mathrm{s}\Phi$ prostopadle.

2. Trzy pierwiastki wielomianu $0$ współczynnikach całkowitych tworzą ciag arytmetyczny.

Suma tych pierwiastków jest równa 21, a i1oczyn 315. Pokazač, $\dot{\mathrm{z}}\mathrm{e}$ wartośč wielomianu

$\mathrm{w}$ dowolnym punkcie, który jest liczbą nieparzystą, jest podzielna przez 48.

3. $\mathrm{W}$ trójkącie równobocznym $0$ boku długości $a$ przeprowadzamy prostą przechodząca

przez środek wysokości $\mathrm{n}\mathrm{a}\mathrm{c}\mathrm{h}\mathrm{y}\mathrm{l}\mathrm{o}\mathrm{n}\Phi$ do niej pod $\mathrm{k}_{\Phi}\mathrm{t}\mathrm{e}\mathrm{m}30^{\mathrm{o}}$ Odcina ona od trójk$\Phi$ta trapez.

Obliczyč pole $\mathrm{i}$ obwód tego trapezu oraz objetośč $\mathrm{i}$ pole powierzchni bryfy powstafej $\mathrm{z}$

jego obrotu dookoła dłuzszej podstawy.

4. $\mathrm{W}$ trójk$\Phi$t równoboczny $0$ boku dlugości l wpisano kwadrat. Następnie $\mathrm{w}$ pozostalą częśč

(nad kwadratem) znów wpisano kwadrat, itd. Jaką dlugośč ma bok kwadratu $\mathrm{w}n$-tym

kroku? Podač wzór ciągu $P_{n}$ określającego sumę pól wpisanych kwadratów po $n$ krokach,

a następnie obliczyč jego granicę.

5. $\mathrm{W}$ okrąg $0$ promieniu $r$ wpisano trapez, którego podstawą jest średnica okręgu. Dla

jakiego kąta przy podstawie pole trapezu jest największe?

6. Znalez/č dziedzinę oraz przedzialy monotoniczności funkcji

$f(x)=1+\displaystyle \frac{2x}{x^{2}-3}+(\frac{2x}{x^{2}-3})^{2}+\ldots.$

Naszkicowač wykres tej funkcji oraz zbadač liczbę rozwi$\Phi$zań równania $f(x) = m \mathrm{w}$

zalezności od parametru $m.$

Rozwiązania (rękopis) zadań z wybranego poziomu prosimy nadsylač do

na adres:

18 grudnia 20l5r.

Wydziaf Matematyki

Politechnika Wrocfawska

Wybrzez $\mathrm{e}$ Wyspiańskiego 27

$50-370$ WROCLAW.

Na kopercie prosimy $\underline{\mathrm{k}\mathrm{o}\mathrm{n}\mathrm{i}\mathrm{e}\mathrm{c}\mathrm{z}\mathrm{n}\mathrm{i}\mathrm{e}}$ zaznaczyč wybrany poziom! (np. poziom podsta-

wowy lub rozszerzony). Do rozwiązań nalez $\mathrm{y}$ dołączyč zaadresowana do siebie kopertę

zwrotną $\mathrm{z}$ naklejonym znaczkiem, odpowiednim do wagi listu. Prace niespełniające po-

danych warunków nie będą poprawiane ani odsyłane.

Adres internetowy Kursu: http://www.im.pwr.wroc.pl/kurs







XLVI

KORESPONDENCYJNY KURS

Z MATEMATYKI

grudzień 2016 r.

PRACA KONTROLNA $\mathrm{n}\mathrm{r} 4-$ POZIOM PODSTAWOWY

l. Dwa samochody wyjechałyjednocześnie zjednego miejsca ijada $\mathrm{w}$ tym samym kierunku.

Pierwszyjedzie $\mathrm{z}$ prędkością 50 $\mathrm{k}\mathrm{m}/\mathrm{h}$, a drugi $\mathrm{z}$ prędkością 40 $\mathrm{k}\mathrm{m}/\mathrm{h}$. Pół godziny póz$\acute{}$niej

$\mathrm{z}$ tego samego miejsca $\mathrm{i}\mathrm{w}$ tym samym kierunku wyruszyl trzeci samochód, który dopędził

pierwszy samochód $0 1$ godzinę $\mathrm{i}30$ minut póz/niej $\mathrm{n}\mathrm{i}\dot{\mathrm{z}}$ drugi. $\mathrm{Z}$ jaka prędkościq jechaf

trzeci samochód?

2. Proste $y = 2, y = 2x+10$ oraz $4x+3y = 0$ wyznaczają trójkąt $ABC$. Otrzymany

trójk$\Phi$t przeksztafcono $\mathrm{u}\dot{\mathrm{z}}$ ywaj $\Phi^{\mathrm{C}}$ najpierw jednokładności $0$ środku $O(0,0)\mathrm{i}$ skali $k=3,$

a następnie symetrii względem osi $OX$. Wyznaczyč współrzędne trójkąta $ABC$ oraz

wspófrzędne obrazów jego wierzcholków. Obliczyč pole trójkqta $ABC\mathrm{i}$ jego obrazu $\mathrm{w}$

tym przeksztalceniu.

3. Rozwazmy zbiór wszystkich prostokątów wpisanych $\mathrm{w}$ kwadrat $0$ boku dlugości $a\mathrm{w}$ taki

sposób, $\dot{\mathrm{z}}\mathrm{e}$ boki tego prostokąta $\mathrm{s}\Phi$ parami równolegfe do przekątnych danego kwadratu.

Obliczyč długości boków tego prostokąta, który ma największe pole.

4. Podstawa trójkqta równobocznego jest średnica koła $0$ promieniu $r$. Obliczyč stosunek

pola powierzchni części trójk$\Phi$ta lezącej na $\mathrm{z}\mathrm{e}\mathrm{w}\mathrm{n}\Phi \mathrm{t}\mathrm{r}\mathrm{z}$ kofa do pola powierzchni części

trójkąta lezącej wewnątrz kola.

5. $\mathrm{W}$ stozku pole podstawy, pole powierzchni kuli wpisanej $\mathrm{w}$ ten stozek $\mathrm{i}$ pole powierzchni

bocznej stozka, tworzą ciag arytmetyczny. Znalez/č cosinus kąta nachylenia tworzqcej

stozka do plaszczyzny jego podstawy.

6. $\mathrm{O}\mathrm{k}\mathrm{r}\Phi \mathrm{g}O_{1}\mathrm{o}$ promieniu l jest styczny do ramion kąta $0$ mierze $\displaystyle \frac{\pi}{3}$. Mniejszy od niego okrąg

$O_{2}$ jest styczny zewnętrznie do niego $\mathrm{i}$ obu ramion tego kąta. Procedurę kontynuujemy.

Znalez$\acute{}$č sumę obwodów pieciu otrzymanych kolejno $\mathrm{w}$ ten sposób okręgów. Dla jakiego

$n$ suma obwodów $\mathrm{c}\mathrm{i}_{\Phi \mathrm{g}}\mathrm{u}$ tych okręgów jest większa od $\displaystyle \frac{299}{100}\pi$?




PRACA KONTROLNA nr 4- POZ1OM ROZSZERZONY

l. Do punktu $A$ po dwóch prostoliniowych drogach jada ze stałymi prędkościami samochód

$\mathrm{i}$ rower. $\mathrm{W}$ chwili początkowej samochód, rower $\mathrm{i}$ punkt $ A\mathrm{t}\mathrm{w}\mathrm{o}\mathrm{r}\mathrm{z}\Phi$ trójk$\Phi$t $\mathrm{p}\mathrm{r}\mathrm{o}\mathrm{s}\mathrm{t}\mathrm{o}\mathrm{k}_{\Phi^{\mathrm{t}}}\mathrm{n}\mathrm{y}.$

Gdy samochód przejechał 25 km trójkąt, którego dwa wierzchołki przesunęły się, stał

się trójkatem równobocznym. Znalez$\acute{}$č odległośč między samochodem a rowerem $\mathrm{w}$ chwili

$\mathrm{P}^{\mathrm{O}\mathrm{C}\mathrm{Z}}\Phi^{\mathrm{t}\mathrm{k}\mathrm{o}\mathrm{w}\mathrm{e}\mathrm{j}}$, jeśli $\mathrm{w}$ momencie dotarcia samochodu do punktu $A$ rower miaf jeszcze do

przejechania 12 km.

2. Na pfaszczy $\acute{\mathrm{z}}\mathrm{n}\mathrm{i}\mathrm{e}$ dane $\mathrm{s}\Phi$ punkty A $\mathrm{i}B$. Udowodnij, $\dot{\mathrm{z}}\mathrm{e}$ złozenie symetrii środkowej wzglę-

dem punktu $A\mathrm{z}$ przesunięciem $0$ wektor $\overline{A}B$ jest symetrią środkową względem środka

odcinka $\overline{AB}.$

3. Wyznaczyč największą wartośč pola $\mathrm{P}^{\mathrm{r}\mathrm{o}\mathrm{s}\mathrm{t}\mathrm{o}\mathrm{k}}\Phi^{\mathrm{t}\mathrm{a}}$, którego dwa wierzchofki $\mathrm{l}\mathrm{e}\dot{\mathrm{z}}$ ą na paraboli

$y=x^{2}-4x+4$, a dwa pozostale na cięciwie paraboli wyznaczonej przez prostą $y=3.$

4. Suma trzech początkowych wyrazów nieskończonego ciągu geometrycznego wynosi 6, $\mathrm{a}$

suma $S$ wszystkich wyrazów tego ciągu równa się $\displaystyle \frac{16}{\mathrm{s}}$. Dlajakich $n$ naturalnych spefniona

jest nierównośč $|S-S_{n}|<\displaystyle \frac{1}{96}$?

5. Dwa jednakowe stozki zfozono podstawami. Obliczyč objętośč powstafej bryfy, jeśli pro-

mień kuli wpisanej $\mathrm{w}$ tę bryłę wynosi $R$, a punkt styczności kuli $\mathrm{i}$ stozka dzieli tworzącą

stozka $\mathrm{w}$ stosunku $m$ do $n$?

6. $\mathrm{W}$ czworościan foremny ABCD $0$ krawędzi dlugości $d$ wpisano kulę. Prowadzimy pfasz-

czyzny równoległe do ścian czworościanu $\mathrm{i}$ styczne do wpisanej kuli odcinając $\mathrm{w}$ ten

sposób cztery $\mathrm{p}\mathrm{r}\mathrm{z}\mathrm{y}\mathrm{s}\mathrm{t}\mathrm{a}\mathrm{j}_{\Phi}\mathrm{c}\mathrm{e}$ czworościany foremne. $\mathrm{W}\mathrm{k}\mathrm{a}\dot{\mathrm{z}}\mathrm{d}\mathrm{y}\mathrm{z}$ nich wpisujemy kulę $\mathrm{i}$ po-

stępujemy analogiczniejak $\mathrm{z}$ kulą wpisaną $\mathrm{w}$ czworościan ABCD. Obliczyč sumę objęto-

ści wszystkich kul wpisanych $\mathrm{w}$ otrzymane czworościany, jeśli proces ten kontynuujemy

nieskończenie wiele razy.

Rozwiązania (rekopis) zadań z wybranego poziomu prosimy nadsyłač do

na adres:

18 grudnia 20l6r.

Wydziaf Matematyki

Politechnika Wrocfawska

Wybrzez $\mathrm{e}$ Wyspiańskiego 27

$50-370$ WROCLAW.

Na kopercie prosimy $\underline{\mathrm{k}\mathrm{o}\mathrm{n}\mathrm{i}\mathrm{e}\mathrm{c}\mathrm{z}\mathrm{n}\mathrm{i}\mathrm{e}}$ zaznaczyč wybrany poziom! (np. poziom podsta-

wowy lub rozszerzony). Do rozwiązań nalez $\mathrm{y}$ dołączyč zaadresowana do siebie kopertę

zwrotną $\mathrm{z}$ naklejonym znaczkiem, odpowiednim do wagi listu. Prace niespelniające po-

danych warunków nie będą poprawiane ani odsyłane.

Adres internetowy Kursu: http://www.wmat.pwr.wroc.pl/kurs







XLVII

KORESPONDENCYJNY KURS

Z MATEMATYKI

grudzień 2017 r.

PRACA KONTROLNA $\mathrm{n}\mathrm{r} 4-$ POZIOM PODSTAWOWY

l. Rodzina składa się $\mathrm{z}$ pięciorga dzieci $\mathrm{i}$ dwojga rodziców. Załózmy, $\dot{\mathrm{z}}\mathrm{e}$ dzieci nie moga

wyjśč na spacer ani nie $\mathrm{m}\mathrm{o}\mathrm{g}\Phi$ zostač $\mathrm{w}$ domu bez opieki któregokolwiek $\mathrm{z}$ rodziców. $\mathrm{W}$ ilu

$\mathrm{m}\mathrm{o}\dot{\mathrm{z}}$ liwych kombinacjach dzieci mogą wyjśč na spacer zakladajqc, $\dot{\mathrm{z}}\mathrm{e}$ przynajmniej jedno

dziecko idzie na spacer?

2. Na bokach prostokąta $0$ stałym obwodzie $4p$ opisano na średnicach pólokręgi $1\mathrm{e}\mathrm{z}\Phi^{\mathrm{C}\mathrm{e}}$

na zewnqtrz prostokąta. Dla jakich wartości boków prostokąta pole figury ograniczonej

$\mathrm{k}\mathrm{r}\mathrm{z}\mathrm{y}\mathrm{w}\Phi \mathrm{z}1_{\mathrm{o}\mathrm{Z}\mathrm{o}\mathrm{n}}\varpi \mathrm{z}$ tych czterech pólokręgów jest najmniejsze? Wykonač staranny rysunek.

3. Punkty $A(1,3), B(5,1), C(4,4)$ są wierzchołkami trójkąta. Obliczyč stosunek pola koła

opisanego na tym trójkqcie do pola kola wpisanego $\mathrm{w}$ ten trójkąt.

4. Liczby $x_{1},  x_{2}\mathrm{s}\Phi$ pierwiastkami równania $x^{2}-3x+A=0$, a liczby $x_{3}, x_{4}$ pierwiastkami

równania $x^{2}-12x+B=0$. Wiadomo, $\dot{\mathrm{z}}\mathrm{e}$ liczby $x_{1}, x_{2}, x_{3}, x_{4}$ tworza ciąg geometryczny.

Znalez/č ten ciąg oraz liczby A $\mathrm{i}B.$

5. Rozwiązač układ równań:

$\{$

a następnie obliczyč pole obszaru, który jest rozwiązaniem ukladu nierówności:

$\left\{\begin{array}{l}
x^{2}+y^{2}-2x-4y+1=0,\\
|x-1|-y=0,\\
x^{2}+y^{2}-2x-4y+1\leq 0,\\
|x-1|-y\leq 0.
\end{array}\right.$

Sporządzič staranny rysunek.

6. $\mathrm{W}$ graniastoslupie prawidfowym czworokqtnym okrąg styczny do dwóch boków podsta-

wy $\mathrm{i}$ przechodzqcy przez jej wierzcholek nielezący na $\dot{\mathrm{z}}$ adnym $\mathrm{z}$ tych boków ma promień

$r=2$. Płaszczyzna $\mathrm{P}^{\mathrm{r}\mathrm{z}\mathrm{e}\mathrm{c}\mathrm{h}\mathrm{o}\mathrm{d}\mathrm{z}}\Phi^{\mathrm{c}\mathrm{a}}$ przez środki krawędzi wychodzących $\mathrm{z}$ jednego wierz-

chołka graniastosłupa tworzy $\mathrm{z}$ płaszczyzną jego podstawy kąt $45^{\mathrm{o}}$ Obliczyč objetośč

graniastosłupa.




PRACA KONTROLNA nr 4- POZ1OM ROZSZERZONY

l. Na ile sposbów $\mathrm{m}\mathrm{o}\dot{\mathrm{z}}$ na umieścič 6 osób $\mathrm{w}$ pokojach dwuosobowych przy zalozeniu, $\dot{\mathrm{z}}\mathrm{e}$

pewne dwie ustalone osoby nie chcą mieszkač razem oraz $\dot{\mathrm{z}}\mathrm{e} \mathrm{a}$) pokoje są jednakowe,

a wiec $\mathrm{w}\mathrm{a}\dot{\mathrm{z}}$ ne jest kto mieszka $\mathrm{z} \mathrm{k}\mathrm{i}\mathrm{m}$, ale niewazne $\mathrm{w}$ którym pokoju; b) pokoje są

istotnie rózne, a więc $\mathrm{w}\mathrm{a}\dot{\mathrm{z}}$ ne jest kto mieszka $\mathrm{w}$ którym pokoju?

2. Rozwiązač $\mathrm{n}\mathrm{a}\mathrm{s}\mathrm{t}\text{ę} \mathrm{p}\mathrm{u}\mathrm{j}_{\Phi^{\mathrm{C}\oplus}}$ nierównośč

$\cos^{2}x+\cos^{3}x+\cos^{4}x+\ldots<\cos x+1$

dla $x\in[0,2\pi].$

3. Pokazač, $\dot{\mathrm{z}}\mathrm{e}$ dla $\mathrm{k}\mathrm{a}\dot{\mathrm{z}}$ dej wartości parametru $m$ wielomian

$w(x)=x^{3}+(2m-1)x^{2}-(3+2m)x+3$

ma pierwiastek całkowity. Dlajakich wartości parametru $m$ pierwiastki tego wielomianu

$\mathrm{t}\mathrm{w}\mathrm{o}\mathrm{r}\mathrm{z}\Phi$ ciąg arytmetyczny?

4. Punkt $A$ nalez $\mathrm{y}$ do obszaru kąta $0$ mierze stopniowej 60. Od1egłości tego punktu od ra-

mion kata są równe a $\mathrm{i}b$. Wyznaczyč odleglośč punktu $A$ od wierzchołka kąta. Następnie

obliczyč tę odlegfośč dla $a=2\mathrm{i}b=\sqrt{3}-1.$

5. $\mathrm{Z}$ punktu $A(1,1)$ wychodzą dwie półproste prostopadłe przecinające oś $OX$ układu

wspólrzędnych. Niech $F$ będzie obszarem $\mathrm{k}_{\Phi^{\mathrm{t}\mathrm{a}}}$ prostego wyznaczonego przez te pól-

proste, $G$ zaś zbiorem wszystkich punktów $0$ obydwóch wspólrzędnych nieujemnych.

Wyznaczyč połozenie półprostych, dla których pole figury $F\cap G$ jest najmniejsze.

6. Znalez/č $\mathrm{n}\mathrm{a}\mathrm{j}\mathrm{m}\mathrm{n}\mathrm{i}\mathrm{e}\mathrm{j}\mathrm{s}\mathrm{z}\Phi \mathrm{m}\mathrm{o}\dot{\mathrm{z}}\mathrm{l}\mathrm{i}\mathrm{w}\Phi$ objętośč stozka opisanego na walcu, którego przekrojem

osiowym jest kwadrat $0$ boku 2.

Rozwiązania (rękopis) zadań z wybranego poziomu prosimy nadsylač do

na adres:

18 grudnia 20l7r.

Wydziaf Matematyki

Politechniki Wrocfawskiej

Wybrzez $\mathrm{e}$ Wyspiańskiego 27

$50-370$ WROCLAW.

Na kopercie prosimy $\underline{\mathrm{k}\mathrm{o}\mathrm{n}\mathrm{i}\mathrm{e}\mathrm{c}\mathrm{z}\mathrm{n}\mathrm{i}\mathrm{e}}$ zaznaczyč wybrany poziom! (np. poziom podsta-

wowy lub rozszerzony). Do rozwiązań nalez $\mathrm{y}$ dołączyč zaadresowaną do siebie kopertę

zwrotną $\mathrm{z}$ naklejonym znaczkiem, odpowiednim do wagi listu. Prace niespelniające po-

danych warunków nie bedą poprawiane ani odsyłane.

Adres internetowy Kursu: http: //www. im. pwr. edu. pl/kurs







XLVIII

KORESPONDENCYJNY KURS

Z MATEMATYKI

grudzień 2018 r.

PRACA KONTROLNA $\mathrm{n}\mathrm{r} 4-$ POZIOM PODSTAWOWY

1. $\mathrm{W}$ zawodach szachowych bierze udział pewna ilośč zawodników, przy czym $\mathrm{k}\mathrm{a}\dot{\mathrm{z}}\mathrm{d}\mathrm{y}$ zawod-

nik gra $\mathrm{z}\mathrm{k}\mathrm{a}\dot{\mathrm{z}}$ dym innym dokladnie $\mathrm{r}\mathrm{a}\mathrm{z}$. Ilu bylo zawodników, jeśli wiadomo, $\dot{\mathrm{z}}\mathrm{e}$ rozegrano

55 partii? Ile rozegranoby partii $\mathrm{w}$ tych zawodach, gdyby jeden $\mathrm{z}$ zawodników zrezygno-

waf $\mathrm{z}$ zawodów rozegrawszy cztery partie?

2. Dane są trzy wektory: ã $= [1,-2], \vec{b}= [-2,-1], \vec{c}= [3$, 4$]$. Dla jakich rzeczywistych

parametrów $t\mathrm{i}s$, {\it wektory Afi}$=${\it tã}, $\overline{B}7=s\vec{b}$ oraz $c\infty=\vec{c}$ tworza trójkąt $ABC$? Zna-

lez/č wspófrzędne środka cięzkości otrzymanego trójk$\Phi$ta, przyjmując $A(0,0)$. Sporządzič

staranny rysunek.

3. Wartośč $\mathrm{u}\dot{\mathrm{z}}$ ytkowa pewnej maszyny maleje $\mathrm{z}$ roku na rok $0$ tę samą wielkośč. Obliczyč

czas, wjakim maszyna straci cafkowitą wartośč $\mathrm{u}\dot{\mathrm{z}}$ ytkową, $\mathrm{j}\mathrm{e}\dot{\mathrm{z}}$ eli wiadomo, $\dot{\mathrm{z}}$ ejej wartośč

po 251atach pracy była trzy razy mniejsza $\mathrm{n}\mathrm{i}\dot{\mathrm{z}}$ jej wartośč po 151atach.

4. Na okręgu $0$ promieniu dfugości $r$ opisano trapez prostokątny, którego najdfuzszy bok

ma długośč $3r$. Obliczyč pole tego trapezu. Sporządzič staranny rysunek.

5. Obliczyč pierwiastek równania

--4{\it x}--6{\it mx}---22{\it xx}$++${\it m}1$=$--26{\it x}-2{\it m-x}--7{\it x}22

wiedząc, $\dot{\mathrm{z}}\mathrm{e}$ jest on $02$ większy od wartości parametru $m.$

6. $\mathrm{Z}$ czworościanu foremnego odcinamy cztery naroz $\mathrm{a}$, których krawędziami bocznymi są

połówki krawędzi czworościanu. Jaki wielościan otrzymujemy? Obliczyč stosunek jego

objętości $\mathrm{i}$ pola powierzchni do objetości $\mathrm{i}$ pola powierzchni czworościanu. Sporządzič

staranny rysunek.




PRACA KONTROLNA nr 4- POZIOM ROZSZERZONY

1. $\mathrm{W}$ zawodach szachowych bierze udział pewna ilośč zawodników, przy czym $\mathrm{k}\mathrm{a}\dot{\mathrm{z}}\mathrm{d}\mathrm{y}$ zawod-

nik gra $\mathrm{z}\mathrm{k}\mathrm{a}\dot{\mathrm{z}}$ dym innym zawodnikiem dokładnie $\mathrm{r}\mathrm{a}\mathrm{z}$. Ilu bylo zawodników tych zawodów,

jeśli rozegrano 84 partie, a dwóch zawodników wycofało się $\mathrm{z}$ zawodów po rozegraniu

przez $\mathrm{k}\mathrm{a}\dot{\mathrm{z}}$ dego trzech partii?

2. Przez środek boku trójkąta równobocznego poprowadzono prostą tworzącą $\mathrm{z}$ tym bokiem

$\mathrm{k}\mathrm{a}\mathrm{t}45^{\mathrm{o}}\mathrm{i}$ dzielącą ten trójkqt na dwie figury. Obliczyč stosunek pól tych figur (większej

do mniejszej). Wynik przedstawič $\mathrm{w}$ najprostszej postaci.

3. Dla jakich wartości parametru $m$, punkty $A(m,-\displaystyle \frac{3}{2}), B(2,0)$ oraz $C(4,-m)$ są wierz-

cholkami trójkąta $ABC$? Zbadač jak zmienia $\mathrm{s}\mathrm{i}\mathrm{e}$ pole tego trójkata $\mathrm{w}$ zalezności od $m.$

Znalez/č, $0$ ile istnieją, najmniejszą $\mathrm{i}$ największ$\Phi$ wartośč tego pola dla $m\in[-2,2].$

4. $\mathrm{Z}$ miast $A\mathrm{i}B$ odległych $0119$ km wyruszają naprzeciw siebie dwaj rowerzyści, przy czym

drugi rowerzysta startuje dwie godziny po wyje $\acute{\mathrm{z}}\mathrm{d}\mathrm{z}\mathrm{i}\mathrm{e}$ pierwszego. Pierwszy rowerzysta,

ruszający $\mathrm{z}$ miasta $A, \mathrm{w}$ ciągu pierwszej godziny przejez $\mathrm{d}\dot{\mathrm{z}}$ a 20 km $\mathrm{i}\mathrm{w}\mathrm{k}\mathrm{a}\dot{\mathrm{z}}$ dej następnej

godzinie przejezdza $02$ km mniej $\mathrm{n}\mathrm{i}\dot{\mathrm{z}}\mathrm{w}$ poprzedniej. Natomiast drugi rowerzysta $\mathrm{w}$ ciągu

pierwszej godziny przejez $\mathrm{d}\dot{\mathrm{z}}$ a 10 km $\mathrm{i}\mathrm{w}\mathrm{k}\mathrm{a}\dot{\mathrm{z}}$ dej następnej godzinie przejezdza $03$ km

więcej $\mathrm{n}\mathrm{i}\dot{\mathrm{z}}\mathrm{w}$ poprzedniej. Po ilu godzinach jazdy się spotkają $\mathrm{i}\mathrm{w}$ jakiej odległości będą

wtedy od obu miast?

5. Wyznaczyč sumę pierwiastków równania

$2^{(m+1)x^{2}-4mx+m+\frac{3}{2}}=\sqrt{2}$

jako funkcję parametru $m$. Wyznaczyč przedziafy, na których funkcja ta jest $\mathrm{r}\mathrm{o}\mathrm{s}\mathrm{n}\Phi^{\mathrm{C}\mathrm{a}}.$

6. $\mathrm{Z}$ sześcianu odcinamy osiem narozy (małych czworościanów), których wierzchołkami

są wierzchołki sześcianu, a bocznymi krawędziami - połówki krawędzi sześcianu. Jaki

wielościan otrzymujemy? Obliczyč stosunekjego objętości $\mathrm{i}$ pola powierzchni do objętości

$\mathrm{i}$ pola powierzchni sześcianu. Znalez/č odległośč między dwoma najbardziej odległymi

wierzchofkami tego wielościanu. Sporządzič staranny rysunek.

Rozwiązania (rękopis) zadań z wybranego poziomu prosimy nadsyłač do

na adres:

18 grudnia 20l8r.

Wydziaf Matematyki

Politechnika Wrocfawska

Wybrzez $\mathrm{e}$ Wyspiańskiego 27

$50-370$ WROCLAW.

Na kopercie prosimy $\underline{\mathrm{k}\mathrm{o}\mathrm{n}\mathrm{i}\mathrm{e}\mathrm{c}\mathrm{z}\mathrm{n}\mathrm{i}\mathrm{e}}$ zaznaczyč wybrany poziom! (np. poziom podsta-

wowy lub rozszerzony). Do rozwiązań nalez $\mathrm{y}$ dołączyč zaadresowaną do siebie koperte

zwrotną $\mathrm{z}$ naklejonym znaczkiem, odpowiednim do wagi listu. Prace niespelniające po-

danych warunków nie będą poprawiane ani odsyłane.

Uwaga. Wysylajac nam rozwiazania zadań uczestnik Kursu udostępnia Politechnice Wroclawskiej

swoje dane osobowe, które przetwarzamy wyłącznie $\mathrm{w}$ zakresie niezbednym do jego prowadzenia

(odesfanie zadań, prowadzenie statystyki). Szczegófowe informacje $0$ przetwarzaniu przez nas danych

osobowych $\mathrm{S}\otimes$ dostępne na stronie internetowej Kursu.

Adres internetowy Kursu: http: //www. im. pwr. edu. pl/kurs







XLIX

KORESPONDENCYJNY KURS

Z MATEMATYKI

grudzień 2019 r.

PRACA KONTROLNA $\mathrm{n}\mathrm{r} 4-$ POZIOM PODSTAWOWY

l. Rozwiązač nierównośč $\sqrt{2^{x}-1}\leq 2^{x}-3.$

2. Trójkąt prostokątny $0$ przyprostokątnych $a, b$ obracamy wokóf $\mathrm{k}\mathrm{a}\dot{\mathrm{z}}$ dej $\mathrm{z}$ przyprostokąt-

nych. Obliczyč stosunek sumy objętości tych stozków do objętości bryly otrzymanej

przez obrót trójkąta wokóf przeciwprostokątnej $\mathrm{i}$ wyrazič go jako funkcję zmiennej $\displaystyle \frac{a}{b}.$

3. Punkty $(-1,1), (0,0), (\sqrt{2},0)$ są trzema kolejnymi wierzcholkami wielokąta foremnego.

Wyznaczyč wspófrzędne pozostalych wierzchofków wielokąta oraz jego pole. Podač rów-

nania okręgów wpisanego $\mathrm{i}$ opisanego na tym wielokącie oraz wyznaczyč stosunek ich

promieni.

4. Niech $f(x)=\{$

$\displaystyle \frac{2-|x|}{|x|-1}$

$\displaystyle \frac{8}{9}x^{2}-1$

gdy

gdy

$|x|>\displaystyle \frac{3}{2}.$

$|x|\displaystyle \leq\frac{3}{2}.$

a) Narysowač wykres funkcji $f\mathrm{i}$ na jego podstawie wyznaczyč zbiór wartości funkcji.

b) Obliczyč $f(\sqrt{2})$ oraz $f(\sqrt{3}).$

c) Rozwiązač nierównośč $f(x)\displaystyle \leq-\frac{1}{2}\mathrm{i}$ zaznaczyč na osi $0x$ zbiór rozwiazań.

5. Punkty $A(0,1), B(4,3) \mathrm{s}\Phi$ dwoma kolejnymi wierzcholkami równolegfoboku ABCD,

a $S(2,3)$ punktem przecięcia przekqtnych. Posługujac się rachunkiem wektorowym, wy-

znaczyč pozostafe wierzchofki równolegfoboku oraz wierzchofki równolegfoboku otrzy-

manego przez obrót ABCD wokól punktu $A090^{\mathrm{o}} \mathrm{w}$ kierunku przeciwnym do ruchu

wskazówek zegara.

6. Ostroslup prawidlowy trójkątny, $\mathrm{w}$ którym bok podstawy $\mathrm{i}$ wysokośč są równe $a$ przecięto

plaszczyzną przechodzącq przez jedną $\mathrm{z}$ krawędzi podstawy na dwie bryły $0$ tej samej

objętości. Wyznaczyč tangens kąta nachylenia tej pfaszczyzny do pfaszczyzny podstawy.

Sporządzič rysunek.




PRACA KONTROLNA nr 4- POZ1OM ROZSZERZONY

l. Punkty $A(0,1), B(4,3)$ są dwoma kolejnymi wierzchołkami równolegloboku ABCD, $\mathrm{a}$

$S(2,3)$ punktem przecięcia przekątnych. Posfugując się rachunkiem wektorowym, wyzna-

czyč pozostałe wierzchołki równoległoboku oraz wierzchołki równoległoboku $A'B'C'D'$

otrzymanego przez obrót ABCD $0$ kąt $90^{\mathrm{o}}$ wokól punktu $(0,0)\mathrm{w}$ kierunku przeciwnym

do ruchu wskazówek zegara. Sprawdzič, $\dot{\mathrm{z}}\mathrm{e}A'B'C'D'$ jest obrazem ABCD $\mathrm{w}$ przeksztal-

ceniu $T_{2}\mathrm{o}O\mathrm{o}T_{1}$, gdzie $T_{1}$ jest przesunięciem $0$ wektor $[0$, 1$]$, O- obrotem $0$ kąt $90^{o}$ wokół

punktu $(0,0)\mathrm{w}$ kierunku przeciwnym do ruchu wskazówek zegara, a $T_{2}$- przesunięciem

$0$ wektor [1, 0].

2. Narysowač wykres funkcji

$f(x)=1-\displaystyle \frac{2^{x}}{3^{x}-2^{x}}+(\frac{2^{x}}{3^{x}-2^{x}})^{2}$

$\mathrm{i}$ uzasadnič, $\dot{\mathrm{z}}\mathrm{e}$ przyjmuje ona $\mathrm{w}\mathrm{y}l_{\Phi}$cznie wartości większe $\displaystyle \mathrm{n}\mathrm{i}\dot{\mathrm{z}}\frac{1}{2}.$

3. Niech $f(x)=$

dla

dla

$x\leq 1,$

$x>1.$

a) Narysowač wykres funkcji $f\mathrm{i}$ na jego podstawie wyznaczyč zbiór wartości funkcji.

b) Obliczyč $f(\displaystyle \log_{\frac{1}{2}}(\sqrt{2}-\frac{1}{2}))$ oraz $f(2^{\sqrt{2}}+\displaystyle \frac{1}{2}).$

c) Rozwiązač nierównośč $f(x)\displaystyle \leq\frac{1}{2}\mathrm{i}$ zaznaczyč na osi $0x$ zbiór rozwiązań.

4. Punkt $C(0,0)$ jest wierzcholkiem trójkąta równoramiennego, $\mathrm{w}$ którym środkowa podsta-

{\it wy AB} $\mathrm{i}$ wysokośč poprowadzona zjednego $\mathrm{z}$ wierzcholków $A, B$ przecinają się $\mathrm{w}$ punkcie

$S(2,1)$. Pole trójkąta $ABS$ jest dwa razy mniejsze $\mathrm{n}\mathrm{i}\dot{\mathrm{z}}$ pole trójkąta $ABC$. Wyznaczyč

wspófrzędne wierzchołków $A, B$ oraz równanie okręgu opisanego na trójkącie $ABC.$

5. $\mathrm{W}$ ośmiościan foremny wpisano dwa sześciany. Wierzchofki pierwszego $\mathrm{z}$ nich lezą na

krawędziach ośmiościanu, a wierzchołki drugiego- na wysokościach ścian bocznych. Ob-

liczyč stosunek objętości tych sześcianów.

6. Prostokąt $0$ bokach $a\mathrm{i}2a$ obraca się wokół przekątnej. Obliczyč pole powierzchni całko-

witej $\mathrm{i}$ objętośč otrzymanej bryły.

Rozwiązania (rękopis) zadań z wybranego poziomu prosimy nadsyłač do

na adres:

18 grudnia 20l9r.

Wydziaf Matematyki

Politechnika Wrocfawska

Wybrzez $\mathrm{e}$ Wyspiańskiego 27

$50-370$ WROCLAW.

Na kopercie prosimy $\underline{\mathrm{k}\mathrm{o}\mathrm{n}\mathrm{i}\mathrm{e}\mathrm{c}\mathrm{z}\mathrm{n}\mathrm{i}\mathrm{e}}$ zaznaczyč wybrany poziom! (np. poziom podsta-

wowy lub rozszerzony). Do rozwiązań nalez $\mathrm{y}$ dołączyč zaadresowaną do siebie kopertę

zwrotną $\mathrm{z}$ naklejonym znaczkiem, odpowiednim do wagi listu. Prace niespelniające po-

danych warunków nie będą poprawiane ani odsylane.

Uwaga. Wysyłając nam rozwiazania zadań uczestnik Kursu udostępnia Politechnice Wrocławskiej

swoje dane osobowe, które przetwarzamy wyłącznie $\mathrm{w}$ zakresie niezbędnym do jego prowadzenia

(odesłanie zadań, prowadzenie statystyki). Szczególowe informacje $0$ przetwarzaniu przez nas danych

osobowych są dostępne na stronie internetowej Kursu.

Adres internetowy Kursu: http: //www. im. pwr. edu. pl/kurs







L KORESPONDENCYJNY KURS

Z MATEMATYKI

grudzień 2020 r.

PRACA KONTROLNA $\mathrm{n}\mathrm{r} 4-$ POZIOM PODSTAWOWY

l. Wykaz$\cdot, \mathrm{z}\mathrm{e}$ dla dowolnej liczby naturalnej $n$ liczba $\displaystyle \vec{3}^{n^{4}}1-\vec{3}^{n^{3}}2-\frac{1}{3}n^{2}+\frac{2}{3}n$ jest podzielna

przez 8.

2. Podaj wzór funkcji kwadratowej, której wykres jest obrazem paraboli $f(x)=-4x(x-1)$

$\mathrm{w}$ symetrii względem punktu $(0,2)$. Uzasadnij poprawnośč znalezionego wzoru $\mathrm{i}\mathrm{s}$porząd $\acute{\mathrm{z}}$

wykresy obu funkcji $\mathrm{w}$ jednym ukladzie wspólrzednych.

3. Wyznacz wielomian $f(x)=x^{3}+ax^{2}+bx+c$ wiedząc, $\dot{\mathrm{z}}\mathrm{e}$ jego pierwiastki są całkowite $\mathrm{i}$

tworzą ciąg geometryczny, a wykres przecina oś $Oy\mathrm{w}$ punkcie $0$ wspólrzędnej $-8.$

4. Narysuj wykres funkcji $f(x)=\displaystyle \frac{|x-1|}{|x|-1}$. Wyznacz zbiór jej wartości $\mathrm{i}$ rozwiąz nierównośč

$|f(x)|\leq 2.$

5. $\mathrm{W}$ zalezności od parametru $a$ określ liczbe rozwiązań układu

Podaj interpretację graficzną dla $a=\sqrt{5}, a=1$ oraz $a=3.$

$\left\{\begin{array}{l}
x^{2}+y^{2}=1\\
|2x-y|=a.
\end{array}\right.$

6. Ostroslup prawidłowy czworokątny, $\mathrm{w}$ którym najmniejszy przekrój płaszczyzną zawie-

rającą wysokośč, prostopadła do płaszczyzny podstawy, jest trójkątem równobocznym,

przecięto płaszczyzną przechodzącą przez jedną $\mathrm{z}$ krawędzi podstawy prostopadlą do

przeciwległej ściany bocznej. Wyznacz stosunek objetości brył, na jakie plaszczyzna ta

$\mathrm{p}\mathrm{o}\mathrm{d}\mathrm{z}\mathrm{i}\mathrm{e}\mathrm{l}\mathrm{i}\ddagger \mathrm{a}$ ostrosłup.




PRACA KONTROLNA nr 4- POZ1OM ROZSZERZONY

l. Trzeci składnik rozwinięcia dwumianu $(\displaystyle \sqrt[3]{x^{2}}+\frac{1}{\sqrt{x}})^{n}$ ma wspófczynnik równy 45. Wy-

znacz wszystkie skladniki tego rozwinięcia, $\mathrm{w}$ których $x$ wystepuje $\mathrm{w}$ potedze $0$ wykład-

niku cafkowitym.

2. Wykres wielomianu $w(x) =x^{3}+ax^{2}+bx+c$ przecina oś $Oy\mathrm{w}$ punkcie $(0,-6) \mathrm{i}$ jest

symetryczny względem punktu $(-1,-2)$. Wyznacz wspófczynniki $a, b, c$ oraz pierwiastki

tego wielomianu. Sporząd $\acute{\mathrm{z}}$ wykres.

3. $\mathrm{W}$ zalezności od parametru $m$ określ liczbę rozwiqzań równania

$4^{x-1}-2^{x+1}\log_{2}m+1=0$

4. Narysuj wykres funkcji

$f(x)=1-\displaystyle \frac{\log_{2}|x-1|}{1-\log_{2}|x-1|}+(\frac{\log_{2}|x-1|}{1-\log_{2}|x-1|})^{2}-(\frac{\log_{2}|x-1|}{1-\log_{2}|x-1|})^{3}+$

gdzie prawa strona jest sumą nieskończonego ciągu geometrycznego.

5. $\mathrm{W}$ zalezności od parametru $a$ określ liczbe rozwiqzań układu

Podaj interpretację graficzną dla $a=0, a=-1$ oraz $a=7.$

$\left\{\begin{array}{l}
xy-y=1\\
x^{2}+y^{2}-2x=a+1.
\end{array}\right.$

6. Dany jest ostroslup prawidłowy trójkątny, $\mathrm{w}$ którym krawęd $\acute{\mathrm{z}}$ boczna jest dwa razy

dłuzsza $\mathrm{n}\mathrm{i}\dot{\mathrm{z}}$ krawędz$\acute{}$ podstawy. Ostrosfup ten podzielono pfaszczyzną przechodzącą przez

krawędz/ podstawy na dwie bryły $0$ tej samej objętości. Wyznacz stosunek objętości kul

wpisanych $\mathrm{w}\mathrm{k}\mathrm{a}\dot{\mathrm{z}}$ dą $\mathrm{z}$ tych brył. Sporzqd $\acute{\mathrm{z}}$ rysunek.

Rozwiązania (rekopis) zadań z wybranego poziomu prosimy nadsyłač do

adres:

31.12.2020r. na

Wydziaf Matematyki

Politechnika Wrocfawska

Wybrzez $\mathrm{e}$ Wyspiańskiego 27

$50-370$ WROCLAW.

Na kopercie prosimy $\underline{\mathrm{k}\mathrm{o}\mathrm{n}\mathrm{i}\mathrm{e}\mathrm{c}\mathrm{z}\mathrm{n}\mathrm{i}\mathrm{e}}$ zaznaczyč wybrany poziom! (np. poziom podsta-

wowy lub rozszerzony). Do rozwiązań nalez $\mathrm{y}$ dołączyč zaadresowaną do siebie kopertę

zwrotną $\mathrm{z}$ naklejonym znaczkiem, odpowiednim do formatu listu. Polecamy stosowanie

kopert formatu C5 $(160\mathrm{x}230\mathrm{m}\mathrm{m})$ ze znaczkiem $0$ wartości 3,30 $\mathrm{z}1$. Na $\mathrm{k}\mathrm{a}\dot{\mathrm{z}}$ dą wiekszą

kopertę nalez $\mathrm{y}$ nakleič drozszy znaczek. Prace niespełniające podanych warunków nie

będą poprawiane ani odsyłane.

Uwaga. Wysyfaj\S c nam rozwiązania zadań uczestnik Kursu udostępnia Politechnice Wrocfawskiej

swoje dane osobowe, które przetwarzamy wylącznie $\mathrm{w}$ zakresie niezbędnym do jego prowadzenia

(odeslanie zadań, prowadzenie statystyki). Szczegółowe informacje $0$ przetwarzaniu przez nas danych

osobowych są dostępne na stronie internetowej Kursu.

Adres internetowy Kursu: http: //www. im. pwr. edu. pl/kurs







LI KORESPONDENCYJNY KURS

Z MATEMATYKI

grudzień 2021 r.

PRACA KONTROLNA nr 4- POZIOM PODSTAWOWY

l. Trzy liczby naturalne $0$ iloczynie 80 tworzą ciąg arytmetyczny. $\mathrm{J}\mathrm{e}\dot{\mathrm{z}}$ eli drugi wyraz tego

ciągu zmniejszymy $0 1$, to liczby te (rozwazane $\mathrm{w}$ tej samej kolejności) utworzą ciąg

geometryczny. Jakie to liczby?

2. Liczby dodatnie $a, b$ spełniają warunek $\alpha^{2}+b^{2}=7ab$. Wykaz, $\dot{\mathrm{z}}\mathrm{e}$

$\log_{3}a+\log_{3}b+2=2\log_{3}(a+b).$

3. Rozwiąz równanie

tg2 {\it x}$=$ -11 $+$-csoins {\it xx}.

4. Narysuj wykres funkcji

$f(x)=\{$

$\displaystyle \frac{2}{3}x^{2}-\frac{8}{3}x+2,$

$|4-2|x-3||,$

gdy

gdy

$|2x-5|\leq 3,$

$|2x-5|>3.$

Na jego podstawie wyznacz: zbiór wartości funkcji $f(x)$ oraz liczbę rozwiqzań równania

$f(x)=m \mathrm{w}$ zalezności od parametru $m.$

5. Punkt $A(0,0)$ jest wierzchołkiem ośmiokąta foremnego wpisanego $\mathrm{w}$ okrąg $x^{2}-2x+y^{2}=0.$

Wyznacz współrzedne pozostafych wierzchołków.

6. Przekrój ostrosfupa prawidfowego $\mathrm{c}\mathrm{z}\mathrm{w}\mathrm{o}\mathrm{r}\mathrm{o}\mathrm{k}_{\Phi^{\mathrm{t}}}$nego plaszczyzną $\mathrm{p}\mathrm{r}\mathrm{z}\mathrm{e}\mathrm{c}\mathrm{h}\mathrm{o}\mathrm{d}\mathrm{z}\text{ą}_{\mathrm{C}\Phi}$ przez wierz-

chołek $\mathrm{i}$ przekątną jego podstawy jest trójkqtem równobocznym. $\mathrm{W}$ ostrosłup wpisano

sześcian, którego dolna podstawa jest zawarta $\mathrm{w}$ podstawie ostrosfupa, a wierzchołki

górnej podstawy sześcianu lezą na krawędziach ostrosłupa. Oblicz stosunek objętości

sześcianu do objetości ostroslupa.




PRACA KONTROLNA $\mathrm{n}\mathrm{r} 4-$ POZIOM ROZSZERZONY

l. Liczby dodatnie $a, b, c$ spełniają warunki: $c>b, a\neq 1, c-b\neq 1, c+b\neq 1$. Wykaz$\cdot, \dot{\mathrm{z}}\mathrm{e}$

równośč

$\displaystyle \log_{c+b}a\cdot\log_{c-b}a=\frac{\log_{c+b}a+\log_{c-b}a}{2}$

zachodzi wtedy $\mathrm{i}$ tylko wtedy, gdy $a^{2}+b^{2}=c^{2}$

2. Rozwiąz nierównośč $\displaystyle \sin^{4}x+\cos^{4}x\leq\frac{3}{4}.$

3. Oblicz sumę wyrazów nieskończonego ciągu geometrycznego, $\mathrm{w}$ którym $\alpha_{1}=1$, a $\mathrm{k}\mathrm{a}\dot{\mathrm{z}}\mathrm{d}\mathrm{y}$

kolejny wyraz jest pofowq róznicy wyrazu następnego $\mathrm{i}$ poprzedniego..

4. Narysuj wykres funkcji $f(x)=\{$

$2^{-x},$

$\log_{2}|x\sqrt{2}|,$

gdy

gdy

$|x+1|\leq 2,$

$|x+1|>2.$

Na podstawie wykresu wyznacz zbiór wartości funkcji $f(x)\mathrm{i}\mathrm{s}$prawd $\acute{\mathrm{z}}$, wjakich punktach

jest ona ciągła. Odpowied $\acute{\mathrm{z}}$ poprzyj odpowiednim rachunkiem.

5. Okręgi $0$ promieniach $r<R$ sq styczne zewnętrznie $\mathrm{w}$ punkcie $M\mathrm{i}$ styczne do prostej $l$

$\mathrm{w}$ punktach A $\mathrm{i}B$. Wyznacz pole trójk$\Phi$ta $ABM\mathrm{w}$ zalezności od $r\mathrm{i}R.$

6. $\mathrm{W}$ ostrosłupie prawidfowym trójkątnym krawędz/ boczna jest nachylona do podstawy

pod kątem $60^{\mathrm{o}}$ Oblicz stosunek objętości kuli wpisanej do objętości kuli opisanej na

ostroslupie.

$\mathrm{R}\mathrm{o}\mathrm{z}\mathrm{w}\mathrm{i}_{\Phi}$zania (rękopis) zadań $\mathrm{z}$ wybranego poziomu prosimy nadsylač do

$2021\mathrm{r}$. na adres:

31 grudnia

Wydziaf Matematyki

Politechnika Wrocfawska

Wybrzez $\mathrm{e}$ Wyspiańskiego 27

$50-370$ WROCLAW,

lub elektronicznie, za pośrednictwem portalu talent. pwr. edu. pl

Na kopercie prosimy $\underline{\mathrm{k}\mathrm{o}\mathrm{n}\mathrm{i}\mathrm{e}\mathrm{c}\mathrm{z}\mathrm{n}\mathrm{i}\mathrm{e}}$ zaznaczyč wybrany poziom! (np. poziom podsta-

wowy lub rozszerzony). Do rozwiązań nalez $\mathrm{y}$ dołączyč zaadresowaną do siebie koperte

zwrotną $\mathrm{z}$ naklejonym znaczkiem, odpowiednim do formatu listu. Polecamy stosowanie

kopert formatu C5 $(160\mathrm{x}230\mathrm{m}\mathrm{m})$ ze znaczkiem $0$ wartości 3,30 zł. Na $\mathrm{k}\mathrm{a}\dot{\mathrm{z}}$ dą wiekszą

kopertę nalez $\mathrm{y}$ nakleič drozszy znaczek. Prace niespełniające podanych warunków nie

będą poprawiane ani odsyłane.

Uwaga. Wysyłając nam rozwiązania zadań uczestnik Kursu udostępnia Politechnice Wrocfawskiej

swoje dane osobowe, które przetwarzamy wyłącznie $\mathrm{w}$ zakresie niezbędnym do jego prowadzenia

(odesłanie zadań, prowadzenie statystyki). Szczególowe informacje $0$ przetwarzaniu przez nas danych

osobowych są dostępne na stronie internetowej Kursu.

Adres internetowy Kursu: http: //www. im. pwr. edu. pl/kurs







LII

KORESPONDENCYJNY KURS

Z MATEMATYKI

grudzień 2022 r.

PRACA KONTROLNA $\mathrm{n}\mathrm{r} 4-$ POZIOM PODSTAWOWY

l. Wyznacz miarę kąta ostrego $\alpha$, wiedząc, $\dot{\mathrm{z}}\mathrm{e} \displaystyle \cos\alpha+\sin\alpha=\frac{1}{\sin\alpha}.$

2. Dane są wierzchofki $A(-1,-2)\mathrm{i}B(6,-1)$ równolegfoboku, którego $\mathrm{P}^{\mathrm{r}\mathrm{z}\mathrm{e}\mathrm{k}}\Phi^{\mathrm{t}\mathrm{n}\mathrm{e}}$ przecinają

się $\mathrm{w}$ punkcie $S(4,0)$. Wyznacz współrzędne pozostałych wierzchołków $\mathrm{i}$ oblicz pole

równolegfoboku.

3. Trójkqt prostokątny $0$ polu 30 jest opisany na okręgu $0$ promieniu 2. Wyznacz dfugości

jego boków.

4. Cięciwy AB $\mathrm{i}CD$ (punkt $C\mathrm{l}\mathrm{e}\dot{\mathrm{z}}\mathrm{y}$ na łuku AB) przecinaj $\Phi$ się pod $\mathrm{k}_{\Phi^{\mathrm{t}}}\mathrm{e}\mathrm{m}$ prostym $\mathrm{w}$ punk-

cie $S$. Pole trójkąta $BSD$ jest równe 4, a po1e trójkąta $ASC$ wynosi 9. Ob1icz po1e

czworokąta ADBC, $\mathrm{j}\mathrm{e}\dot{\mathrm{z}}$ eli suma długości tych cięciw jest równa 15.

5. Dane $\mathrm{s}\Phi$ punkty $A(8,2)\mathrm{i}B(1,6)$. Punkt $C\mathrm{l}\mathrm{e}\dot{\mathrm{z}}\mathrm{y}$ najednej $\mathrm{z}$ osi ukfadu ijest wierzchofkiem

kata prostego $\mathrm{w}$ trójkącie $ABC$. Wyznacz współrzedne punku $C.$

6. $\mathrm{W}$ ostrosłupie prawidlowym trójk$\Phi$tnym zachodzi równośč $\cos\alpha=\sqrt{3}\cos\beta$, gdzie $\alpha$ jest

kątem nachylenia krawędzi bocznej, a $\beta$- kątem nachylenia ściany bocznej do podstawy.

Wykaz, $\dot{\mathrm{z}}\mathrm{e}$ ten ostrosłup jest czworościanem foremnym.




PRACA KONTROLNA $\mathrm{n}\mathrm{r} 4-$ POZIOM ROZSZERZONY

l. Wiedzqc, $\displaystyle \dot{\mathrm{z}}\mathrm{e}\sin 2x=-\frac{3}{4} \mathrm{i}  x\in (\displaystyle \frac{\pi}{2},\pi)$, oblicz wartośč wyrazenia

$\displaystyle \frac{\sin(3x+30^{\mathrm{o}})-\sin(x-30^{\mathrm{o}})}{4\cos^{2}x-2}.$

2. Wektory $\vec{u}, \vec{v}$ mają długośč l $\mathrm{i}$ tworzą kąt $60^{\mathrm{o}}$ Oblicz dlugości przekątnych równoległo-

boku rozpiętego na wektorach $(2\vec{u}-\vec{v})\mathrm{i}(\vec{u}-2\vec{v})$. Wyznacz jego kąt ostry $\mathrm{i}\mathrm{s}$prawd $\acute{\mathrm{z}},$

czy $\mathrm{m}\mathrm{o}\dot{\mathrm{z}}$ na $\mathrm{w}$ ten równoleglobok wpisač okrąg. $\mathrm{J}\mathrm{e}\dot{\mathrm{z}}$ eli $\mathrm{t}\mathrm{a}\mathrm{k}$, to oblicz jego promień.

3. Przekątne trapezu ABCD przecinają się $\mathrm{w}$ takim punkcie $P, \dot{\mathrm{z}}\mathrm{e}$

$|AP|^{2}+|BP|^{2}-|AB|^{2}=\displaystyle \frac{2\sqrt{5}}{3}|AP||BP|.$

$\mathrm{O}$ ile dluzszy jest promień okręgu opisanego na trójkącie $ABP$ od promienia okręgu opi-

sanego na trójkącie $PCD, \mathrm{j}\mathrm{e}\dot{\mathrm{z}}$ eli $|AB|-|CD|=4$?

4. Na okręgu $x^{2}+y^{2}-2x-2y=0$, opisany jest trapez prostokątny ABCD $0$ polu 12.

Wyznacz współrzedne wierzchołków trapezu, wiedzqc, $\dot{\mathrm{z}}\mathrm{e}$ wieksza $\mathrm{z}$ jego podstaw $AB$

jest zawarta jest $\mathrm{w}$ prostej $x+y=0$, a kąt przy wierzchofku $A$ jest prosty.

5. $\mathrm{W}$ trójkącie równoramiennym $ABC$ kąt przy wierzchołku $C$ ma miarę $20^{\mathrm{o}} \mathrm{Z}$ wierzchoł-

ków $A\mathrm{i}B$ poprowadzono półproste pod kqtami $50^{\mathrm{o}}\mathrm{i}60^{\mathrm{o}}$ wzgledem podstawy, przecina-

jące ramiona $AC\mathrm{i}BC\mathrm{w}$ punktach $D\mathrm{i}E$ odpowiednio. Wyznacz miarę $\mathrm{k}_{\Phi}\mathrm{t}\mathrm{a}BDE.$

$\mathrm{W}\mathrm{S}K.$ Poprowad $\acute{\mathrm{z}}$ półprosta $\mathrm{z}$ punktu $A$ przecinająca odcinek $BD\mathrm{w}$ punkcie $G$, a bok

$BC\mathrm{w}$ takim punkcie $F, \dot{\mathrm{z}}\mathrm{e}\angle BAF=60^{\mathrm{o}}\mathrm{i}$ przyjrzyj się czworokątowi DGEF.

6. $\mathrm{W}$ ostrosłupie prawidłowym trójkątnym krawęd $\acute{\mathrm{z}}$ boczna jest dwa razy dluzsza $\mathrm{n}\mathrm{i}\dot{\mathrm{z}}$ kra-

wedz' podstawy. Wyznacz cosinus kata między ścianami bocznymi ostrosłupa oraz sto-

sunek promienia kuli opisanej na ostrosfupie do promienia kuli wpisanej $\mathrm{w}$ ostroslup.

Rozwiązania (rękopis) zadań z wybranego poziomu prosimy nadsyfač do 31.12.2022r.

adres:

na

Wydziaf Matematyki

Politechnika Wrocfawska

Wybrzez $\mathrm{e}$ Wyspiańskiego 27

$50-370$ WROCLAW,

lub elektronicznie, za pośrednictwem portalu talent. $\mathrm{p}\mathrm{w}\mathrm{r}$. edu. pl

Na kopercie prosimy $\underline{\mathrm{k}\mathrm{o}\mathrm{n}\mathrm{i}\mathrm{e}\mathrm{c}\mathrm{z}\mathrm{n}\mathrm{i}\mathrm{e}}$ zaznaczyč wybrany poziom! (np. poziom podsta-

wowy lub rozszerzony). Do rozwiązań nalez $\mathrm{y}$ dołączyč zaadresowana do siebie koperte

zwrotną $\mathrm{z}$ naklejonym znaczkiem, odpowiednim do formatu listu. Prace niespełniające

podanych warunków nie będą poprawiane ani odsyłane.

Uwaga. Wysylajac nam rozwiazania zadań uczestnik Kursu udostępnia Politechnice Wroclawskiej

swoje dane osobowe, które przetwarzamy wyłącznie $\mathrm{w}$ zakresie niezbędnym do jego prowadzenia

(odesfanie zadań, prowadzenie statystyki). Szczegófowe informacje $0$ przetwarzaniu przez nas danych

osobowych $\mathrm{S}\otimes$ dostępne na stronie internetowej Kursu.

Adres internetowy Kursu: http://www.im.pwr.edu.pl/kurs







XLIV

KORESPONDENCYJNY KURS

Z MATEMATYKI

styczeń 2015 r.

PRACA KONTROLNA nr 5- POZIOM PODSTAWOWY

1. $\mathrm{W}$ ciągu arytmetycznym suma wyrazów od drugiego do piqtego wynosi 50 $\mathrm{i}$ jest ona

równa iloczynowi wyrazu czwartego $\mathrm{i}\mathrm{p}\mathrm{i}_{\Phi}$tego. Znajd $\acute{\mathrm{z}}$ pierwszy wyraz $\mathrm{i}$ róznicę $\mathrm{c}\mathrm{i}_{\Phi \mathrm{g}}\mathrm{u}.$

2. Punkt A(l, l) jest wierzchołkiem trójkąta równobocznego wpisanego w okrag 0 środku

w punkcie (2, 0). Wyznacz współrzędne pozostałych wierzchofków trójkąta. Rozwiązanie

zilustruj starannym rysunkiem.

3. $\mathrm{W}$ konkursie matematycznym trzy $\mathrm{P}^{\mathrm{o}\mathrm{c}\mathrm{z}}\Phi^{\mathrm{t}\mathrm{k}\mathrm{o}\mathrm{w}\mathrm{e}}$ miejsca zostafy przyznane Asi, Basi, Kasi,

Kamilowi $\mathrm{i}$ Rafałowi. Ilejest $\mathrm{m}\mathrm{o}\dot{\mathrm{z}}$ liwych rozstrzygnięč konkursu, $\mathrm{j}\mathrm{e}\dot{\mathrm{z}}$ eli wiadomo, $\dot{\mathrm{z}}\mathrm{e}\mathrm{k}\mathrm{a}\dot{\mathrm{z}}$ de

$\mathrm{z}$ miejsc I- III zostało przyznane?

4. Opisz równaniem $\mathrm{i}$ narysuj $\mathrm{w}$ układzie wspólrzędnych zbiór punktów płaszczyzny, któ-

rych odległośč od punktu $A(-2,-1)$ jest dwa razy większa od odleglości od punktu

$B(1,2).$

5. Rozwiqz nierównośč

$5^{1-x^{4}}\cdot 2^{x^{2}(x^{2}-1)}>16^{x^{2}-1}\cdot 5^{5-5x^{2}}$

6. Wyznacz wszystkie liczby $x\mathrm{z}$ przedziału $[0,2\pi]$ spelniajqce równanie

1$+$2 $\displaystyle \sin x+2^{2}\sin^{2}x+\cdots+2^{n-1}\sin^{n-1}x=\frac{1-2^{n}\sin^{n}x}{1-\sqrt{2}\sin 2x}$

dla $\mathrm{k}\mathrm{a}\dot{\mathrm{z}}$ dej liczby naturalnej $n\geq 1.$




PRACA KONTROLNA nr 5- POZIOM ROZSZERZONY

l. Wyznacz wszystkie liczby rzeczywiste x, dla których funkcja

$f(x)=\displaystyle \frac{x^{2}-\sqrt{2-x}}{x-1}-x$

przyjmuje wartości nieujemne.

2. Rozwiąz równanie

$1+3^{-3\sin^{2}x}+3^{-6\sin^{2}x}+3^{-9\sin^{2}x}+\displaystyle \cdots=\frac{3}{3-3^{\sin^{2}x}},$

którego lewa strona jest sumą nieskończonego ciqgu geometrycznego.

3. Danajest liczba $\alpha\in(0,1)\cup(1,\infty)$ oraz ciąg liczbowy $(a_{n})$ taki, $\dot{\mathrm{z}}\mathrm{e}a=2^{a_{1}}$ oraz $a= \sqrt[n]{2^{a_{n}}}$

dla $\mathrm{k}\mathrm{a}\dot{\mathrm{z}}$ dego naturalnego $n$. Wyznacz liczbę naturalną $m$, dla której suma $m\mathrm{P}^{\mathrm{o}\mathrm{c}\mathrm{z}}\Phi^{\mathrm{t}\mathrm{k}\mathrm{o}-}$

wych wyrazów ciągu $(a_{n})$ jest 5050 razy większa od pierwszego wyrazu.

4. Drzewa rosnące przed galerią handlową $\mathrm{z}\mathrm{o}\mathrm{s}\tan\Phi$ przed świętami ozdobione jednobarwny-

mi diodami LED. Na ile sposobów $\mathrm{m}\mathrm{o}\dot{\mathrm{z}}$ na wykonač iluminację świątecznq, jeśli wiadomo,

$\dot{\mathrm{z}}\mathrm{e}$ drzew jest 6, $\mathrm{k}\mathrm{a}\dot{\mathrm{z}}$ de drzewo zostanie podświetlone na jeden $\mathrm{z}3$ kolorów, a $\mathrm{k}\mathrm{a}\dot{\mathrm{z}}\mathrm{d}\mathrm{y}$ kolor

zostanie wykorzystany co najmniej $\mathrm{r}\mathrm{a}\mathrm{z}$?

5. Krzywa $\Gamma$ jest zbiorem punktów pfaszczyzny, których odleglośč od punktu $A(-\displaystyle \frac{2}{3},0)$

jest trzy razy mniejsza od odległości od punktu $B(2,-8)$. Opisz krzywą równaniem

$\mathrm{i}$ zbadaj, dla jakich wartości rzeczywistego parametru $m$ prosta

$mx-y-3m-1=0$

ma dokładnie 2 punkty wspó1ne $\mathrm{z}$ krzywą $\Gamma. \mathrm{R}\mathrm{o}\mathrm{z}\mathrm{w}\mathrm{i}_{\Phi}$zanie zilustruj rysunkiem.

6. Rozwiąz nierównośč

$\sqrt{\frac{1}{2}\log_{2}(x^{4}-2x^{3}+x^{2})}\geq 4\log_{4}\sqrt{x^{2}-x}.$

Rozwiqzania prosimy nadsyłač do dnia

18 stycznia 20l5 na adres:

Katedra Matematyki WPPT

Politechniki Wrocfawskiej

Wybrzez $\mathrm{e}$ Wyspiańskiego 27

$50-370$ Wrocfaw.

Na kopercie prosimy koniecznie zaznaczyč wybrany poziom. Do rozwiązań nalez$\mathrm{y}$ do-

laczyč zaadresowan\S do siebie kopertę zwrotn\S z naklejonym znaczkiem, odpowiednim do wagi listu.

Prace $\mathrm{n}\mathrm{i}\mathrm{e}\mathrm{s}\mathrm{p}\mathrm{e}l\mathrm{n}\mathrm{i}\mathrm{a}\mathrm{j}_{\Phi}\mathrm{c}\mathrm{e}$ podanych warunków nie będą poprawiane ani odsyłane.

Adres internetowy Kursu:

http://www. im.pwr.edu.pl/kurs







XLV

KORESPONDENCYJNY KURS

Z MATEMATYKI

styczeń 2016 r.

PRACA KONTROLNA nr 5- POZIOM PODSTAWOWY

l. Udowodnič, $\dot{\mathrm{z}}\mathrm{e}$ róznica kwadratów dwu dowolnych liczb całkowitych niepodzielnych przez

3 jest podzielna przez 3.

2. Rozwiązač równanie

$\mathrm{w}$ przedziale $[0,2\pi].$

sin2(-$\pi+$2{\it x})-sin(-$\pi$-2{\it x})$+$sin2(-$\pi$-2{\it x})$=$1

3. Dla jakiego parametru $m$ równanie

$(\log_{2}^{2}m-1)\cdot x^{2}+2$ (log2 $m-1$)$\cdot x+2=0$

ma tylko jedno $\mathrm{r}\mathrm{o}\mathrm{z}\mathrm{w}\mathrm{i}_{\Phi}\mathrm{z}\mathrm{a}\mathrm{n}\mathrm{i}\mathrm{e}$?

4. Jedna $\mathrm{z}$ krawędzi bocznych ostrosfupa, którego podstawą jest kwadrat $0$ boku $a$, jest

prostopadła do podstawy. Najdluzsza krawędz/ boczna jest nachylona do podstawy pod

$\mathrm{k}_{\Phi}\mathrm{t}\mathrm{e}\mathrm{m}60^{\mathrm{o}}$. Obliczyč pole powierzchni calkowitej oraz sumę dfugości krawędzi ostrosfupa.

Sporządzič rysunek.

5. $\mathrm{J}\mathrm{a}\mathrm{k}_{\Phi}\mathrm{k}\mathrm{r}\mathrm{z}\mathrm{y}\mathrm{w}\Phi^{\mathrm{t}\mathrm{w}\mathrm{o}\mathrm{r}\mathrm{Z}}\Phi$ punkty plaszczyzny, $\mathrm{z}$ których odcinek $0$ końcach $A(1,0)\mathrm{i}B(0,1)$

jest widoczny pod kątem $30^{\mathrm{o}}$

6. Narysowač wykres funkcji $f(x)=\displaystyle \frac{|x+1|-1}{|x-1|}\mathrm{i}$ na jego podstawie wyznaczyč przedzialy

jej monotoniczności oraz najmniejszą wartośč $\mathrm{w}$ przedziale $[-2,\displaystyle \frac{1}{2}]$




PRACA KONTROLNA nr 4- POZ1OM ROZSZERZONY

l. Udowodnič $\mathrm{t}\mathrm{o}\dot{\mathrm{z}}$ samośč

$x^{3}+y^{3}+z^{3}-3xyz=(x+y+z)(x^{2}+y^{2}+z^{2}-xy-xz-yz)$

$\mathrm{i}$ wykorzystujqc ją, usunąč niewymiernośč $\mathrm{z}$ mianownika ułamka $\displaystyle \frac{1}{1+\sqrt[3]{2}+\sqrt[3]{4}}.$

2. Jaką krzywą tworzą środki wszystkich okręgów stycznych równocześnie do osi $Ox\mathrm{i}$ do

okręgu $x^{2}+(y-1)^{2}=1$ ?

3. Wyznaczyč wszystkie styczne do wykresu funkcji $f(x)=\displaystyle \frac{x-3}{x-2}$ prostopadłe do prostej

$4x+y+1=0$. Sporządzič staranne wykresy wszystkich rozwazanych krzywych.

4. Narysowač wykres funkcji

$ f(x)=1-\displaystyle \frac{2^{x}}{1-2^{x}}+(\frac{2^{x}}{1-2^{x}})^{2}-(\frac{2^{x}}{1-2^{x}})^{3}+(\frac{2^{x}}{1-2^{x}})^{4}-\ldots$

bdcej sumą nieskończonego szeregu geometrycznego. Rozwiązač nierównośč

$f(x)>4^{x}-21\cdot 2^{x-2}+2.$

5. Dla jakiego parametru $m$ równanie

(log4 $m+1$)$\cdot x^{2}+3\log_{4}m\cdot x+2\log_{4}m-1=0$

ma dwa pierwiastki $x_{1}, x_{2}$ spefniające warunki: $x_{1}<x_{2}$, oraz $2(x_{2}^{2}-x_{1}^{2})>x_{2}^{4}-x_{1}^{4}$?

6. $\mathrm{W}$ ostrosłupie prawidfowym trójkątnym tangens kąta nachylenia ściany bocznej do pod-

stawy jest równy $2\sqrt{3}$. Obliczyč stosunek objętości kuli wpisanej do objętości kuli opi-

sanej na ostrosfupie.

Rozwiązania (rękopis) zadań z wybranego poziomu prosimy nadsylač do 18 stycznia 2016 r.

na adres:

Wydziaf Matematyki

Politechnika Wrocfawska

Wybrzez $\mathrm{e}$ Wyspiańskiego 27

$50-370$ WROCLAW.

Na kopercie prosimy $\underline{\mathrm{k}\mathrm{o}\mathrm{n}\mathrm{i}\mathrm{e}\mathrm{c}\mathrm{z}\mathrm{n}\mathrm{i}\mathrm{e}}$ zaznaczyč wybrany poziom! (np. poziom podsta-

wowy lub rozszerzony). Do rozwiązań nalez $\mathrm{y}$ dołączyč zaadresowaną do siebie kopertę

zwrotną $\mathrm{z}$ naklejonym znaczkiem, odpowiednim do wagi listu. Prace niespełniajace po-

danych warunków nie będą poprawiane ani odsyłane.

Adres internetowy Kursu: http://www.im.pwr.wroc.pl/kurs







XLVI

KORESPONDENCYJNY KURS

Z MATEMATYKI

styczeń 2017 r.

PRACA KONTROLNA nr 5- POZIOM PODSTAWOWY

1. $\mathrm{W}$ urnie znajduje się 9 ku1 ponumerowanych od 1 do 9. Losujemy bez zwracania 4

kule $\mathrm{i}$ dodajemy ich numery. Ile jest $\mathrm{m}\mathrm{o}\dot{\mathrm{z}}$ liwych wyników losowania, $\mathrm{w}$ których suma

wylosowanych numerów jest parzysta, a ile wyników losowania prowadzi do uzyskania

liczby nieparzystej?

2. Narysuj na płaszczy $\acute{\mathrm{z}}\mathrm{n}\mathrm{i}\mathrm{e}$ krzywą

$y=|2^{|x-1|}-2|$

i starannie opisz metodę jej konstrukcji.

3. Wyznacz dziedzinę funkcji

$f(x)=\sqrt{\log_{\frac{1}{2}}(2x-1)-2\log_{2}\frac{1}{x-2}}.$

4. Rozwiąz równanie

$(\displaystyle \frac{9}{4})^{x}(\frac{8}{27})^{x-2}\log(27-x)-3\log_{\frac{1}{10}}\frac{1}{\sqrt{27-x}}=0$

5. Narysuj w układzie współrzędnych zbiór

$A=\{(x,y)\in \mathbb{R}^{2}:\sqrt{(x^{2}-y)^{2}}+1<(|x|+1)^{2}\}.$

6. Wśród walców wpisanych w kulę 0 promieniu R wskaz ten 0 największym polu po-

wierzchni bocznej. Podaj jego wymiary oraz stosunek pola jego powierzchni cafkowitej

do pola powierzchni kuli.




PRACA KONTROLNA nr 5- POZIOM ROZSZERZONY

1. $\mathrm{W}$ finale pewnego konkursu bierze udział 10 osób. Prowadzący wybiera 1osowo jedną

$\mathrm{z}$ nich $\mathrm{i}$ zadaje jej pytanie finafowe. Obliczyč prawdopodobieństwo, $\dot{\mathrm{z}}\mathrm{e}$ zapytana osoba

udzieli poprawnej odpowiedzi, jeśli wiadomo, $\dot{\mathrm{z}}\mathrm{e}$ k-ty finalista odpowie poprawnie na

pytanie finałowe $\mathrm{z}$ prawdopodobieństwem $\displaystyle \frac{1}{2^{k}}$, gdzie $k\in\{1, \ldots$, 10$\}.$

2. Rozwiąz równanie

$\mathrm{x}^{\log_{3}x-1}=9.$

3. Zbadaj, dla jakich argumentów x funkcja

$f(x)=(2-x)^{\frac{3x-4}{2-x}}-1$

przyjmuje wartości ujemne.

4. Podaj dziedzinę $\mathrm{i}$ narysuj wykres funkcji

$f(x)=2|\log_{2}\sqrt{|x-1|-1}|.$

Starannie opisz metodę jego konstrukcji. Rozwiąz równanie $f(x)=2.$

5. Narysuj na płaszczyz/nie zbiór

$A=\{(x,y)\in \mathbb{R}^{2}$:

$\log_{|x|}(\log_{y+1}(|x|+1)) \leq 0\}.$

6. Wśród prostopadłościanów wpisanych $\mathrm{w}$ kulę $0$ promieniu $R$, których przekątna tworzy

kąt $\alpha \mathrm{z}$ jednq ze ścian, wskaz ten $0$ największej objętości. Podaj jego wymiary oraz

stosunek jego objętości do objętości kuli. Jaki procent objętości kuli stanowi objętośč

prostopadfościanu dla kąta $\alpha=45^{\mathrm{o}}$? Wynik podač $\mathrm{z}$ dokfadnością do jednego promila.

Rozwiązania prosimy nadsyłač do dnia

18 stycznia 20l7 na adres:

Wydziaf Matematyki

Politechniki Wrocfawskiej

Wybrzez $\mathrm{e}$ Wyspiańskiego 27

$50\rightarrow 370$ Wroclaw.

Na kopercie prosimy koniecznie zaznaczyč wybrany poziom. Do rozwiązań nalez$\mathrm{y}$ do-

f\S czyč zaadresowaną do siebie kopertę zwrotn\S z naklejonym znaczkiem, odpowiednim do wagi listu.

Prace niespelniające podanych warunków nie będą poprawiane ani odsyłane.

Adres internetowy Kursu:

http://www. im.pwr.edu.pl/kur s







XLVII

KORESPONDENCYJNY KURS

Z MATEMATYKI

styczeń 2018 r.

PRACA KONTROLNA $\mathrm{n}\mathrm{r} 5-$ POZIOM PODSTAWOWY

l. Rozwiązač równanie $3^{\log_{\sqrt{3}}(2^{x}-1)}=2^{x+1}+1.$

2. Jaki zbiór tworzą środki wszystkich cięciw $\mathrm{P}^{\mathrm{r}\mathrm{z}\mathrm{e}\mathrm{c}\mathrm{h}\mathrm{o}\mathrm{d}\mathrm{z}}\Phi^{\mathrm{c}\mathrm{y}\mathrm{c}\mathrm{h}}$ przez ustalony punkt zada-

nego okręgu?

3. Narysowač wykres funkcji $f(x) = \displaystyle \frac{|x+2|-1}{x-1}$. Wyznaczyč zbiór jej wartości oraz naj-

mniejszą $\mathrm{i}$ największą wartośč na przedziale $[$-3, $0].$

4. Niech $T$ będzie przeksztalceniem płaszczyzny polegającym na przesunięciu $0$ wektor

[1, 2], a $S-$ symetrią względem prostej $y=x$. Wyznaczyč (analitycznie) obrazy kwadratu

$0$ wierzchofkach $(0,1)$, (1, 1), (1, 2) $\mathrm{i}(0,2)\mathrm{w}$ przeksztafceniach $S0T\mathrm{i}T0S$. Sporz$\Phi$dzič

staranne rysunki.

5. Wspólne styczne do stycznych zewnętrznie okręgów $0$ promieniach $r<R$ przecinają się

pod kątem $ 2\alpha$. Wyznaczyč stosunek pól tych okręgów. Dla jakiego kąta $\alpha \mathrm{d}\mathrm{u}\dot{\mathrm{z}}\mathrm{e}$ kofo ma

9 razy większe pole $\mathrm{n}\mathrm{i}\dot{\mathrm{z}}$ małe?

6. Pole powierzchni cafkowitej ostroslupa prawidlowego trójk$\Phi$tnego jest 4 razy większe od

pola jego podstawy. Obliczyč sinus kąta między ścianami ostroslupa.




PRACA KONTROLNA nr 5- POZIOM ROZSZERZONY

1. $\mathrm{W}$ rozwinięciu $(a+b)^{n} = \displaystyle \sum_{k=0}^{n}\left(\begin{array}{l}
n\\
k
\end{array}\right)a^{n-k}b^{k}$ dla $a = \sqrt{x}, b = \displaystyle \frac{1}{2\sqrt[4]{x}}$ trzy pierwsze

wspólczynniki przy potęgach $x$ tworza ciag arytmetyczny. Znalez$\acute{}$č wszystkie składniki

rozwinięcia, $\mathrm{w}$ którym $x$ występuje $\mathrm{w}$ potędze $0$ wykladniku cafkowitym.

2. Punkty $K, L, M$ dzielą boki AB, $BC, CA$ trójkąta $ABC$ (odpowiednio) $\mathrm{w}$ tym samym

stosunku, $\mathrm{t}\mathrm{z}\mathrm{n}.$

$\displaystyle \frac{|KB|}{|AB|}=\frac{|LC|}{|BC|}=\frac{|MA|}{|CA|}=s$

Wykazač, $\dot{\mathrm{z}}\mathrm{e}$ dla dowolnego punktu $P$ znajdujacego się wewnqtrz trójkąta zachodzi rów-

nośč

$\vec{PK}+\vec{PL}+\vec{PM}=\vec{PA}+\vec{PB}+\vec{PC}.$

3. Narysowač wykres funkcji $f(x)=\displaystyle \frac{(x+1)^{2}-1}{x|x-1|}$. Wyznaczyč styczną do wykresu $\mathrm{w}$ punk-

cie $(-2,f(-2))$ oraz stycznq do niej prostopadła.

4. Końce odcinka AB $0$ długości $l\mathrm{p}\mathrm{o}\mathrm{r}\mathrm{u}\mathrm{s}\mathrm{z}\mathrm{a}\mathrm{j}_{\Phi}$ się po okręgu $0$ promieniu $R (l<2R)$. Na

odcinku obrano punkt $P\mathrm{t}\mathrm{a}\mathrm{k}, \dot{\mathrm{z}}\mathrm{e} \displaystyle \frac{|AP|}{|PB|} = \displaystyle \frac{1}{3}$. Uzasadnič, $\dot{\mathrm{z}}\mathrm{e}$ poruszający $\mathrm{s}\mathrm{i}\mathrm{e}$ punkt $P$

zakreśla okrąg $0$ tym samym środku. Dla jakiego $l$ wycięte $\mathrm{w}$ ten sposób kolo ma pole

dwa razy mniejsze od pola $\mathrm{d}\mathrm{u}\dot{\mathrm{z}}$ ego koła.

5. Rozwazamy zbiór wszystkich trójkątów $0$ polu 10, których jednym $\mathrm{z}$ wierzcholków jest

$A(5,0)$ a pozostale dwa $\mathrm{l}\mathrm{e}\dot{\mathrm{z}}\Phi$ na osi $Oy$. Wyznaczyč zbiór wszystkich punktów pfaszczy-

zny, które są środkami okręgów opisanych na tych trójkątach.

6. $\mathrm{W}$ przeciwlegle narozniki sześcianu $0$ boku l wpisano dwie kule $0$ takich samych pro-

mieniach $\mathrm{t}\mathrm{a}\mathrm{k}, \dot{\mathrm{z}}\mathrm{e}\mathrm{k}\mathrm{a}\dot{\mathrm{z}}$ da $\mathrm{z}$ nich jest styczna do drugiej $\mathrm{i}$ do trzech ścian wychodzących $\mathrm{z}$

odpowiedniego wierzchołka. Jaka jest odleglośč ich środków?

Rozwiązania (rękopis) zadań z wybranego poziomu prosimy nadsylač do

na adres:

18 stycznia 20l8r.

Wydziaf Matematyki

Politechnika Wrocfawska

Wybrzez $\mathrm{e}$ Wyspiańskiego 27

$50-370$ WROCLAW.

Na kopercie prosimy $\underline{\mathrm{k}\mathrm{o}\mathrm{n}\mathrm{i}\mathrm{e}\mathrm{c}\mathrm{z}\mathrm{n}\mathrm{i}\mathrm{e}}$ zaznaczyč wybrany poziom! (np. poziom podsta-

wowy lub rozszerzony). Do rozwiązań nalez $\mathrm{y}$ dołączyč zaadresowaną do siebie koperte

zwrotną $\mathrm{z}$ naklejonym znaczkiem, odpowiednim do wagi listu. Prace niespelniające po-

danych warunków nie będą poprawiane ani odsyłane.

Adres internetowy Kursu: http://www.im.pwr.wroc.pl/kurs







XLVIII

KORESPONDENCYJNY KURS

Z MATEMATYKI

styczeń 2019 r.

PRACA KONTROLNA $\mathrm{n}\mathrm{r} 5-$ POZIOM PODSTAWOWY

l. Znalez$\acute{}$č stuelementowy ciag arytmetyczny, w którym suma wyrazów 0 numerach niepa-

rzystych jest dwa razy większa od sumy wyrazów 0 numerach parzystych io50 mniejsza

od sumy wszystkich wyrazów.

2. Rozwiązač układ równań 

$2^{y-1},$

$\log_{2}(x+2).$

3. Narysowač wykres funkcji $f(x) =x|x|-4|x|+3\mathrm{i}$ określič liczbę rozwiązań równania

$f(x)=m\mathrm{w}$ zalezności od parametru $m.$

4. $\mathrm{W}$ {\it romb ABCD} $0$ kącie ostrym $\alpha$ wpisano czworokąt, którego boki są równoległe do

przekqtnych rombu. Jakie jest $\mathrm{m}\mathrm{o}\dot{\mathrm{z}}$ liwie największe pole takiego czworokąta?

5. Znalez/č równania wspólnych stycznych do wykresów funkcji

$f(x)=-x^{2}+2x\mathrm{i}g(x)=x^{2}+1.$

6. $\mathrm{W}$ stozek $0$ promieniu podstawy $R$ wpisano stozek $0$ osiem razy mniejszej objętości.

Wysokośč malego stozka jest zawarta $\mathrm{w}$ wysokości $\mathrm{d}\mathrm{u}\dot{\mathrm{z}}$ ego stozka, jego wierzchołek jest

$\mathrm{w}$ środku podstawy, a okrąg ograniczający podstawę malego stozka jest zawarty $\mathrm{w}$ po-

wierzchni bocznej $\mathrm{d}\mathrm{u}\dot{\mathrm{z}}$ ego stozka. Obliczyč $\displaystyle \frac{r}{R}$, gdzie $r$ oznacza promień podstawy stozka

wpisanego.




PRACA KONTROLNA nr5 -P0Zi0M R0ZSZERZ0NY

l. Rozwiązač ukfad równań 

16,

16

2. Wyznaczyč równania wszystkich stycznych do wykresu funkcji

są prostopadłe do prostej $x+3y+1=0.$

$f(x) = \displaystyle \frac{2x-1}{x+1},$

które

3. Granicą ciągu $0$ wyrazie ogólnym $a_{n}=n^{2}-\sqrt{n^{4}-an^{2}+bn}$ jest większy $\mathrm{z}$ pierwiastków

równania

$x^{\log_{2}x}-3=4x^{\log_{\frac{1}{2}}x}$

Wyznaczyč parametry a $\mathrm{i}b.$

4. Na boku $BC$ trójkąta równobocznego obrano punkt $D\mathrm{t}\mathrm{a}\mathrm{k}, \dot{\mathrm{z}}\mathrm{e}$ promień okręgu wpisanego

$\mathrm{w}$ trójkąt $ADC$ jest dwa razy mniejszy $\mathrm{n}\mathrm{i}\dot{\mathrm{z}}$ promień okręgu wpisanego $\mathrm{w}$ trójkąt $ABD.$

$\mathrm{W}$ jakim stosunku punkt $D$ dzieli bok $BC$?

5. Rozwiązač nierównośč

$1+\displaystyle \frac{\sin x}{\sqrt{3}+\sin x}+(\frac{\sin x}{\sqrt{3}+\sin x})^{2}+(\frac{\sin x}{\sqrt{3}+\sin x})^{3}+\cdots\leq\cos x,$

której lewa strona jest sumą wszystkich wyrazów nieskończonego ciągu geometrycznego.

6. Jakie wymiary ma walec $\mathrm{o}\mathrm{m}\mathrm{o}\dot{\mathrm{z}}$ liwie największej objętości wpisany $\mathrm{w}$ sześcian $0$ boku $a$

$\mathrm{w}$ taki sposób, $\dot{\mathrm{z}}\mathrm{e}$ jego oś jest zawarta $\mathrm{w}$ przekątnej sześcianu, a $\mathrm{k}\mathrm{a}\dot{\mathrm{z}}$ da $\mathrm{z}$ podstaw jest

styczna do trzech ścian wychodzących $\mathrm{z}$ jednego wierzchofka.

Rozwiązania (rękopis) zadań z wybranego poziomu prosimy nadsyfač do

na adres:

18 stycznia 20l9r.

Wydziaf Matematyki

Politechnika Wrocfawska

Wybrzeže Wyspiańskiego 27

$50-370$ WROCLAW.

Na kopercie prosimy $\underline{\mathrm{k}\mathrm{o}\mathrm{n}\mathrm{i}\mathrm{e}\mathrm{c}\mathrm{z}\mathrm{n}\mathrm{i}\mathrm{e}}$ zaznaczyč wybrany poziom! (np. poziom podsta-

wowy lub rozszerzony). Do rozwiązań nalez $\mathrm{y}$ dołączyč zaadresowaną do siebie koperte

zwrotną $\mathrm{z}$ naklejonym znaczkiem, odpowiednim do wagi listu. Prace niespełniające po-

danych warunków nie będą poprawiane ani odsyłane.

Uwaga. Wysylając nam rozwiązania zadań uczestnik Kursu udostępnia nam swoje dane osobo-

we, które przetwarzamy wyłącznie $\mathrm{w}$ zakresie niezbędnym do jego prowadzenia (odesłanie zadań,

prowadzenie statystyki). Szczególowe informacje $0$ przetwarzaniu przez nas danych osobowych sq

dostępne na stronie internetowej Kursu.

Adres internetowy Kursu: http: //www. im. pwr. edu. pl/kurs







XLIX

KORESPONDENCYJNY KURS

Z MATEMATYKI

styczeń 2020 r.

PRACA KONTROLNA $\mathrm{n}\mathrm{r} 5-$ POZIOM PODSTAWOWY

l. Załózmy, $\dot{\mathrm{z}}\mathrm{e}$ mamy 12 ku1 białych $\mathrm{i}9$ kul czarnych. Na ile sposobów $\mathrm{m}\mathrm{o}\dot{\mathrm{z}}$ na ustawič te

kule $\mathrm{w}$ rzędzie $\mathrm{w}$ taki sposób, aby $\dot{\mathrm{z}}$ adna czarna kula nie sąsiadowala $\mathrm{z}$ czarną? Na ile

róznych sposobów $\mathrm{m}\mathrm{o}\dot{\mathrm{z}}$ na ustawič te kule $\mathrm{w}$ rzędzie $\mathrm{w}$ taki sposób, aby $\dot{\mathrm{z}}$ adna czarna kula

nie sąsiadowafa $\mathrm{z}$ czarną, jeśli kule białe ponumerujemy kolejnymi liczbami parzystymi,

a kule czarne- kolejnymi liczbami nieparzystymi?

2. Ścianki kostki do gry oznaczono liczbami: -$3,$ -$2,$ -$1$, 1, 2, 3. Jakie jest prawdopodobień-

stwo zdarzenia, $\dot{\mathrm{z}}\mathrm{e}$ przy dwóch rzutach tą kostką: a) otrzymana suma liczb wynosi 2; b)

wartośč bezwzględna sumy liczb jest równa co najwyzej 3?

3. Wyznaczyč ciag arytmetyczny $0$ pierwszym wyrazie równym 2, wiedząc, $\dot{\mathrm{z}}\mathrm{e}$ wyrazy:

pierwszy, trzeci $\mathrm{i}$ jedenasty $\mathrm{w}$ podanej kolejności tworzą ciąg geometryczny. Ile pierw-

szych kolejnych wyrazów tego ciqgu nalezy dodač, aby otrzymana suma była większa

$\mathrm{n}\mathrm{i}\dot{\mathrm{z}}$ 1000?

4. $\mathrm{W}$ zbiorze $[0,2\pi]$ rozwiązač nierównośč

$\sin x+\sin 3x\geq\cos x+\cos 3x.$

5. Znalez$\acute{}$č równania okręgów, które są styczne do obu osi układu współrzędnych oraz do

prostej $0$ równaniu $x+y=4$. Wykonač rysunek.

6. Pokazač, $\dot{\mathrm{z}}\mathrm{e}$ stosunek objetości stozka do objętości wpisanej $\mathrm{w}$ ten stozek kuli jest równy

stosunkowi pola powierzchni cafkowitej stozka do pola powierzchni kuli.




PRACA KONTROLNA nr $5$ - PozioM ROZSZERZONY

l. Na ile sposobów $\mathrm{m}\mathrm{o}\dot{\mathrm{z}}$ na wybrač 5 kart $\mathrm{z}$ talii 52 kart $\mathrm{t}\mathrm{a}\mathrm{k}$, aby mieč przynajmniej po

jednej karcie $\mathrm{w}\mathrm{k}\mathrm{a}\dot{\mathrm{z}}$ dym $\mathrm{z}$ czterech kolorów? A jaka jest odpowied $\acute{\mathrm{z}}$, gdy wybieramy 6

kart $\mathrm{z}$ talii?

2. Rozpatrujemy zbiór ciągów $n$-elementowych $0$ wyrazach -$1, 0$ lub l. Obliczyč prawdo-

podobieństwo tego, $\dot{\mathrm{z}}\mathrm{e}$ losowo wybrany ciąg ma co najwyzej jeden wyraz równy 0 $\mathrm{i}$ suma

jego wyrazów równa jest 0.

3. Suma wszystkich wspófczynników wielomianu $W_{n}(x)$ jest równa

$\displaystyle \lim_{n\rightarrow\infty}(1+\frac{1}{2}+\frac{1}{4}+\ldots+\frac{1}{2^{n}}),$

a suma współczynników przy nieparzystych potęgach zmiennej równa jest sumie współ-

czynników przy jej parzystych potęgach. Wyznaczyč resztę $R(x)\mathrm{z}$ dzielenia wielomianu

$W_{n}(x)$ przez dwumian $x^{2}-1.$

4. Rozwiązač nierównośč

$\sin x+\sin 2x+\sin 3x\geq\cos x+\cos 2x+\cos 3x.$

5. Zbadač przebieg zmienności funkcji $f(x) = \displaystyle \frac{4x^{2}-3x-1}{4x^{2}+1} \mathrm{i}$ naszkicowač jej wykres. Na

podstawie sporządzonego wykresu określič liczbę rozwi$\Phi$zań równania $f(x) =m\mathrm{w}$ za-

$\mathrm{l}\mathrm{e}\dot{\mathrm{z}}$ ności od parametru $m.$

6. $\mathrm{W}$ stozku pole podstawy, pole powierzchni kuli wpisanej $\mathrm{w}$ ten stozek $\mathrm{i}$ pole powierzchni

bocznej stozka tworzą ciąg arytmetyczny. Wyznaczyč kąt nachylenia tworzącej stozka

do plaszczyzny jego podstawy. Wykonač rysunek.

Rozwiązania (rękopis) zadań z wybranego poziomu prosimy nadsyfač do

na adres:

18 stycznia 2020r.

Wydziaf Matematyki

Politechnika Wrocfawska

Wybrzez $\mathrm{e}$ Wyspiańskiego 27

$50-370$ WROCLAW.

Na kopercie prosimy $\underline{\mathrm{k}\mathrm{o}\mathrm{n}\mathrm{i}\mathrm{e}\mathrm{c}\mathrm{z}\mathrm{n}\mathrm{i}\mathrm{e}}$ zaznaczyč wybrany poziom! (np. poziom podsta-

wowy lub rozszerzony). Do rozwiązań nalez $\mathrm{y}$ dołączyč zaadresowana do siebie koperte

zwrotną $\mathrm{z}$ naklejonym znaczkiem, odpowiednim do formatu listu. Polecamy stosowanie

kopert formatu C5 $(160\mathrm{x}230\mathrm{m}\mathrm{m})$ ze znaczkiem $0$ wartości 3,30 zł. Na $\mathrm{k}\mathrm{a}\dot{\mathrm{z}}$ dą większą

kopertę nalez $\mathrm{y}$ nakleič $\mathrm{d}\mathrm{r}\mathrm{o}\dot{\mathrm{z}}$ szy znaczek. Prace niespełniające podanych warunków nie

będą poprawiane ani odsyłane.

Uwaga. Wysylaj\S c nam rozwi\S zania zadań uczestnik Kursu udostępnia Politechnice Wrocfawskiej

swoje dane osobowe, które przetwarzamy wyłącznie $\mathrm{w}$ zakresie niezbędnym do jego prowadzenia

(odeslanie zadań, prowadzenie statystyki). Szczególowe informacje $0$ przetwarzaniu przez nas danych

osobowych są dostępne na stronie internetowej Kursu.

Adres internetowy Kursu: http: //www. im. pwr. edu. pl/kurs







L KORESPONDENCYJNY KURS

Z MATEMATYKI

styczeń 2021 r.

PRACA KONTROLNA nr 5- POZIOM PODSTAWOWY

l. Jeden $\mathrm{z}$ wierzchołków trójkąta równobocznego wpisanego $\mathrm{w}$ okrąg $x^{2}+y^{2}=2$ znajduje się

$\mathrm{w}$ punkcie $P(1,1)$. Wyznacz polozenie pozostałych wierzchołków $\mathrm{i}$ sporząd $\acute{\mathrm{z}}$ odpowiedni

rysunek.

2. Zbadaj, dla jakiej wartości parametru $\alpha \in [0,2\pi]$ liczba 0 jest najwiekszą wartościa

funkcji

$f(x)=x^{2}\cos\alpha+x(1+\cos 2\alpha)-1$

$\mathrm{w}$ calej jej dziedzinie.

3. Wyznacz te argumenty funkcji

$g(x)=16\cdot 2^{x^{4}}\cdot 243^{x^{2}}-81\cdot 3^{x^{4}}\cdot 32^{x^{2}}$

dla których funkcja ta przyjmuje wartości nieujemne.

4. Zakładając, $\dot{\mathrm{z}}\mathrm{e}x\in[0,2\pi]$, rozwiąz nierównośč trygonometryczną

16 $\displaystyle \sin^{4}\frac{x}{2}-16\sin^{2}\frac{x}{2}+3\geq 0.$

5. Wyznacz wszystkie punkty wspólne krzywych

$y=\displaystyle \log_{\sqrt{2}}\sqrt{2x-1}+\log_{\frac{1}{2}}\frac{1}{3x+1}$

oraz

$y=1+2\log_{4}(x+1).$

6. Narysuj wykres funkcji

$f(x)=|2-|2-2^{|x|}||$

i precyzyjnie opisz zastosowaną metode jego konstrukcji. Na podstawie rysunku wskaz

lokalne ekstrema funkcji oraz określ jej najmniejszą i największą wartośč w cafej dzie-

dzinie, 0 i1e one istnieją.




PRACA KONTROLNA $\mathrm{n}\mathrm{r} 5-$ POZIOM ROZSZERZONY

l. Jeden $\mathrm{z}$ wierzchołków sześciokąta foremnego wpisanego $\mathrm{w}$ okrąg $x^{2}+y^{2}=2$ znajduje się $\mathrm{w}$

punkcie $P(-1,-1)$. Wyznacz pofozenie pozostalych wierzchołków $\mathrm{i}\mathrm{s}$porząd $\acute{\mathrm{z}}$ odpowiedni

rysunek.

2. Rozwiąz nierównośč

$2^{3x^{3}+x^{2}-3x+1}\cdot 3^{6x^{4}-x^{2}}\geq 3^{x^{3}+6x^{2}-x-1}\cdot 4^{3x^{4}+x^{3}-3x^{2}-x+1}$

3. Określ dziedzinę $\mathrm{i}$ zbadaj, dla jakich argumentów funkcja

$f(x)=\displaystyle \log_{x-1}(x+2)+\log_{x+2}\frac{1}{x-1}$

przyjmuje wartości dodatnie.

4. Rozwiąz nierównośč

$3-3\displaystyle \sin^{2}x+3\sin^{4}x-3\sin^{6}x+\ldots\leq\frac{16\cos^{2}x-16\cos^{4}x}{2-\cos^{2}x},$

której lewa strona jest suma wszystkich wyrazów nieskończonego ciągu geometrycznego.

5. Na stozku $0$ promieniu podstawy $R$ opisano ostrosłup prawidłowy czworokątny, a $\mathrm{w}$ sto-

$\dot{\mathrm{z}}$ ek ten wpisano ostrosłup prawidlowy sześciok$\Phi$tny. Stosunek pól powierzchni bocznych

obu ostrosfupów wynosi $k$. Wyznacz zakres zmienności parametru $k$, a dla $k=\displaystyle \frac{11}{8}$ oblicz

wysokośč stozka $\mathrm{i}$ wykonač staranne rysunki rozwazanych brył.

6. Określ dziedzinę, wyznacz wszystkie asymptoty, przedziały monotoniczności oraz wszyst-

kie lokalne ekstrema funkcji

$f(x)=\displaystyle \frac{x^{3}+x^{2}-x+2}{x^{2}+x-2}.$

$\mathrm{s}_{\mathrm{P}^{\mathrm{o}\mathrm{r}\mathrm{z}}\Phi^{\mathrm{d}\acute{\mathrm{z}}}}$ staranny wykres.

Rozwiązania (rękopis) zadań z wybranego poziomu prosimy nadsyfač do

2021r. na adres:

20 stycznia

Wydziaf Matematyki

Politechnika Wrocfawska

Wybrzez $\mathrm{e}$ Wyspiańskiego 27

$50-370$ WROCLAW.

Na kopercie prosimy $\underline{\mathrm{k}\mathrm{o}\mathrm{n}\mathrm{i}\mathrm{e}\mathrm{c}\mathrm{z}\mathrm{n}\mathrm{i}\mathrm{e}}$ zaznaczyč wybrany poziom! (np. poziom podsta-

wowy lub rozszerzony). Do rozwiązań nalez $\mathrm{y}$ dołączyč zaadresowaną do siebie kopertę

zwrotną $\mathrm{z}$ naklejonym znaczkiem, odpowiednim do formatu listu. Polecamy stosowanie

kopert formatu C5 $(160\mathrm{x}230\mathrm{m}\mathrm{m})$ ze znaczkiem $0$ wartości 3,30 zł. Na $\mathrm{k}\mathrm{a}\dot{\mathrm{z}}$ dą większą

kopertę nalez $\mathrm{y}$ nakleič $\mathrm{d}\mathrm{r}\mathrm{o}\dot{\mathrm{z}}$ szy znaczek. Prace niespełniające podanych warunków nie

bedą poprawiane ani odsyłane.

Uwaga. Wysyfajac nam rozwiazania zadań uczestnik Kursu udostępnia Politechnice Wroclawskiej

swoje dane osobowe, które przetwarzamy wyłącznie $\mathrm{w}$ zakresie niezbednym do jego prowadzenia

(odesfanie zadań, prowadzenie statystyki). Szczegófowe informacje $0$ przetwarzaniu przez nas danych

osobowych s\S dostępne na stronie internetowej Kursu.

Adres internetowy Kursu: http: //www. im. pwr. edu. pl/kurs







LI KORESPONDENCYJNY KURS

Z MATEMATYKI

styczeń 2022 r.

PRACA KONTROLNA nr 5- POZIOM PODSTAWOWY

l. Do sklepu dostarczono ziemniaki $\mathrm{w}$ dwóch gatunkach. II gatunekjest po $a\mathrm{z}1$ za kilogram,

a I gatunek jest $020$ \% drozszy. Lączna wartośč dostarczonych ziemniaków wyniosla $56a$

$\mathrm{z}l. \mathrm{W}$ ciągu dnia sprzedano 1/5 ziemniaków I gatunku $\mathrm{i} 1/4$ ziemniaków II gatunku,

$\mathrm{w}$ sumie za kwotę $12,2\alpha \mathrm{z}l$. Ile kilogramów ziemniaków $\mathrm{k}\mathrm{a}\dot{\mathrm{z}}$ dego gatunku dostarczono do

sklepu?

2. Na loteriijest l00 losów, $\mathrm{z}$ których $5$jest wygrywających. Jakiejest prawdopodobieństwo,

$\dot{\mathrm{z}}\mathrm{e}$ wśród trzech kupionych losów a) dokfadnie jeden wygrywa; b) przynajmniej jeden

wygrywa?

3. Dany jest kwadrat $0$ boku $a$. Do boków tego kwadratu dołączono jednakowe trójkqty

równoramienne $0$ podstawie boku kwadratu. Następnie zfączono wierzchofki trójkątów

$\mathrm{w}$ jeden wierzchołek tworząc ostrosłup $0$ objętości $V$. Wyznacz długośč ramienia dolą-

czonych trójkątów, a następnie wykonaj rachunki, przyjmując $a=3$ cm oraz $V= 18$

$\mathrm{c}\mathrm{m}^{3}$

4. Wysokośč rombu $0$ boku $\alpha$ dzieli jeden $\mathrm{z}$ jego boków na dwie części $\mathrm{w}$ stosunku 1 : 2.

Wyznacz dlugości przekątnych rombu oraz promień okręgu wpisanego $\mathrm{w}$ ten romb.

5. Znajd $\acute{\mathrm{z}}$ współrzędne wierzcholka $C$ trójkąta równoramiennego $ABC0$ podstawie $AB,$

gdzie $A(0,0) \mathrm{i} B(2,0)$, wiedząc, $\dot{\mathrm{z}}\mathrm{e}$ środkowe tego trójkąta $AD \mathrm{i}$ BE są prostopadłe

względem siebie.

6. Prosta $0$ równaniu $x-2y+10 = 0$ przecina parabolę $y = x^{2}-4x+5\mathrm{w}$ punktach

{\it A} $\mathrm{i}B$. Wykaz, $\dot{\mathrm{z}}\mathrm{e}$ trójkąt $ABC$, gdzie $C$ jest wierzchołkiem paraboli, jest prostokątny,

a następnie oblicz pole tego trójkata. Wykonaj staranny rysunek.




PRACA KONTROLNA $\mathrm{n}\mathrm{r} 5-$ POZIOM ROZSZERZONY

1. $K\mathrm{a}\mathrm{t}$ ostry równolegloboku ma miarę $45^{\mathrm{o}}$ Punkt przeciecia przekątnych równoległoboku

jest oddalony od boków $0 1\mathrm{i}\sqrt{2}$. Oblicz pole tego równolegfoboku oraz dlugości jego

przekątnych.

2. Spośród 20 pytań egzaminacyjnych uczeń zna odpowied $\acute{\mathrm{z}}\mathrm{n}\mathrm{a}12$ pytań. Jakie jest prawdo-

podobieństwo, $\dot{\mathrm{z}}\mathrm{e}$ uczeń zda egzamin, jeśli przyjętajest następująca zasada: uczeń losuje

dwa pytania $\mathrm{i}$ jeśli na oba odpowie dobrze, to egzamin jest zdany, a jeśli tylko na jed-

no pytanie odpowie dobrze, to losuje jeszcze jedno pytanie $\mathrm{i}$ musi na nie odpowiedzieč

poprawnie, $\dot{\mathrm{z}}$ eby zdač egzamin?

3. Czworościan rozcięto wzdfuz trzech krawędzi wychodzących $\mathrm{z}$ tego samego wierzchofka

$\mathrm{i}$ po rozprostowaniu otrzymano kwadrat $0$ boku $a$. Oblicz objętośč czworościaniu oraz

wykonaj odpowiedni rysunek.

4. Przez punkt $(-1,2)$ przeprowad $\acute{\mathrm{z}} \mathrm{p}\mathrm{r}\mathrm{o}\mathrm{s}\mathrm{t}_{\Phi}\mathrm{t}\mathrm{a}\mathrm{k}$, aby środek jej odcinka zawartego między

prostymi $x+2y = 3\mathrm{i}x+2y = 5$ nalezał do prostej $x+y = 2$. Wyznacz równanie

symetralnej tego odcinka. Wykonaj staranny rysnuek.

5. Rozwiąz algebraicznie następujący ukfad równań

$\left\{\begin{array}{l}
y=|x^{2}-2x|+1\\
x^{2}+y^{2}+1=2x+2y
\end{array}\right.$

$\mathrm{i}$ podaj jego interpretację graficznq (wykonaj staranny rysunek).

6. Funkcja $f(x) = \displaystyle \frac{x^{2}-4x+4}{2x}$ ma $\mathrm{w}$ punktach $A\mathrm{i}B$ wartości ekstremalne. Znajd $\acute{\mathrm{z}}$ taki

punkt $C$ nalezący do osi odciętych, aby pole trójkąta $ABC$ było równe pierwiastkowi

równania $x^{1-\frac{1}{2}+\frac{1}{4}-\frac{1}{8}} =4$, gdzie $x>0$. Naszkicuj wykres funkcji $f(x)$ wraz $\mathrm{z}$ trójkątem

$ABC.$

Rozwiązania (rękopis) zadań z wybranego poziomu prosimy nadsyłač do

2022r. na adres:

20 stycznia

Wydziaf Matematyki

Politechnika Wrocfawska

Wybrzez $\mathrm{e}$ Wyspiańskiego 27

$50-370$ WROCLAW,

lub elektronicznie, za pośrednictwem portalu talent. $\mathrm{p}\mathrm{w}\mathrm{r}$. edu. pl

Na kopercie prosimy $\underline{\mathrm{k}\mathrm{o}\mathrm{n}\mathrm{i}\mathrm{e}\mathrm{c}\mathrm{z}\mathrm{n}\mathrm{i}\mathrm{e}}$ zaznaczyč wybrany poziom! (np. poziom podsta-

wowy lub rozszerzony). Do rozwiązań nalez $\mathrm{y}$ dołączyč zaadresowaną do siebie kopertę

zwrotną $\mathrm{z}$ naklejonym znaczkiem, odpowiednim do formatu listu. Polecamy stosowanie

kopert formatu C5 $(160\mathrm{x}230\mathrm{m}\mathrm{m})$ ze znaczkiem $0$ wartości 3,30 zł. Na $\mathrm{k}\mathrm{a}\dot{\mathrm{z}}$ dą większą

kopertę nalez $\mathrm{y}$ nakleič $\mathrm{d}\mathrm{r}\mathrm{o}\dot{\mathrm{z}}$ szy znaczek. Prace niespełniające podanych warunków nie

bedą poprawiane ani odsyłane.

Uwaga. Wysyfajac nam rozwiazania zadań uczestnik Kursu udostępnia Politechnice Wroclawskiej

swoje dane osobowe, które przetwarzamy wyłącznie $\mathrm{w}$ zakresie niezbednym do jego prowadzenia

(odesfanie zadań, prowadzenie statystyki). Szczegófowe informacje $0$ przetwarzaniu przez nas danych

osobowych s\S dostępne na stronie internetowej Kursu.

Adres internetowy Kursu: http: //www. im. pwr. edu. pl/kurs







LII

KORESPONDENCYJNY KURS

Z MATEMATYKI

styczeń 2023 r.

PRACA KONTROLNA $\mathrm{n}\mathrm{r} 5-$ POZIOM PODSTAWOWY

l. Rozwiqz nierówność

$\displaystyle \frac{\sqrt{30+x-x^{2}}}{x}<\frac{\sqrt{10}}{5}.$

2. $\mathrm{Z}$ ilu domin składa się komplet klocków do gry $\mathrm{w}$ domino, zawierajqcy pojednym dominie

dla $\mathrm{k}\mathrm{a}\dot{\mathrm{z}}$ dej kombinacji oczek od 0 do 6? A jaka jest odpowiedz' d1a kombinacji oczek od

0 do $n$?

3. $\mathrm{W}$ prostokątnym ukladzie współrzędnych narysuj zbiór $A\cap B, \mathrm{j}\mathrm{e}\dot{\mathrm{z}}$ eli:

$A=\{(x,y):x\in \mathbb{R},y\in \mathbb{R},y=x+b,b\in[-2,2]\},$

$B=\displaystyle \{(x,y):x\in \mathbb{R},y\in \mathbb{R},y=ax,a\in[-3,-\frac{1}{3}]\}.$

Zbadaj, czy punkt $(1,-\displaystyle \frac{1}{2})$ nalezy do zbioru $A\cap B.$

4. Spośród trapezów równoramiennych $0$ danym obwodzie $p\mathrm{i}$ danym kącie $\alpha$ przy podstawie

wyznacz trapez $0$ największym polu.

5. Dane są trzy kolejne wierzcholki prostokąta ABCD: $A(-5,-3), B(-2,0), C(-7,5)$. Na-

pisz równanie okręgu opisanego na tym prostokącie oraz równanie prostej stycznej do

tego okręgu $\mathrm{w}$ punkcie $D.$

6. Kwadrat ABCD jest podstawą prostopadłościanu ABCDEFGH. $\acute{\mathrm{S}}$ rodek $M$ krawędzi

AB łączymy $\mathrm{z}$ wierzchołkiem $G$ otrzymując odcinek dlugości $d$ nachylony do ściany

DCGH pod kątem $\alpha$. Oblicz pole powierzchni bocznej tego prostopadłościanu.




PRACA KONTROLNA $\mathrm{n}\mathrm{r} 5-$ POZIOM ROZSZERZONY

l. Para $(x,y)$ jest rozwiązaniem układu:

$\left\{\begin{array}{l}
x-y=-1-m\\
2x-y=2m.
\end{array}\right.$

Dlajakich wartości $m$ punkt $P(x,y)$ nalezy do wnętrza koła $0$ promieniu długości $r=\sqrt{5}$

$\mathrm{i}$ środku $\mathrm{w}$ początku układu współrzędnych?

2. Na ile sposobów $\mathrm{m}\mathrm{o}\dot{\mathrm{z}}$ na ustawić 5 ksiqzek na trzech półkach, jeś1i $\mathrm{w}\mathrm{a}\dot{\mathrm{z}}$ na jest kolejność

ustawienia ksiązek oraz to, na której pólce stoją?

3. Wyznacz zbiór środków wszystkich cięciw okręgu $0$ równaniu $x^{2}+y^{2}=16$, które prze-

chodzą przez punkt $(0,4)$. Wykonaj staranny rysunek.

4. Wykres funkcji $f(x)=x^{3}-3x^{2}+bx+c$ przechodzi przez punkt $A(2,5)$. Wspólczynnik

kierunkowy stycznej do wykresu funkcju $\mathrm{w}$ punkcie $A$ jest rozwiązaniem równania

$(\displaystyle \frac{4}{9})^{x+1}=(\frac{81}{16})^{x+13}$

Wyznacz najmniejszą $\mathrm{i}$ największą wartość funkcji $\mathrm{w}$ przedziale [-2, 2].

5. Obwód trójkąta równoramiennego jest równy $a$. Przy jakich dlugošciach boków pole

trójkąta jest największe? Podaj największą wartość pola trójkąta dla $a=3+2\sqrt{3}.$

6. $\mathrm{W}$ kole $0$ šrodku $O$ poprowadzono dwie prostopadłe średnice $\overline{AB}\mathrm{i}\overline{CD}$ oraz cięciwę $\overline{AM}$

przecinającq średnicę $\overline{CD}\mathrm{w}$ punkcie $K$. Dlajakiego kąta między šrednicą $\overline{AB}$ a cięciwą

$\overline{AM}\mathrm{w}$ czworokąt OBMK $\mathrm{m}\mathrm{o}\dot{\mathrm{z}}$ na wpisać okrąg?

Rozwiązania (rękopis) zadań z wybranego poziomu prosimy nadsylać do 20.01.2023r.

adres:

na

Wydzial Matematyki

Politechnika $\mathrm{W}\mathrm{r}\mathrm{o}\mathrm{c}\not\subset$awska

Wybrzez $\mathrm{e}$ Wyspiańskiego 27

$50-370$ WROCLAW,

lub elektronicznie, za pośrednictwem portalu talent. $\mathrm{p}\mathrm{w}\mathrm{r}$. edu. pl

Na kopercie prosimy $\underline{\mathrm{k}\mathrm{o}\mathrm{n}\mathrm{i}\mathrm{e}\mathrm{c}\mathrm{z}\mathrm{n}\mathrm{i}\mathrm{e}}$ zaznaczyć wybrany poziom! (np. poziom podsta-

wowy lub rozszerzony). Do rozwiązań nalez $\mathrm{y}$ dołączyć zaadresowaną do siebie kopertę

zwrotną $\mathrm{z}$ naklejonym znaczkiem, odpowiednim do formatu listu. Prace niespełniające

podanych warunków nie będą poprawiane ani odsyłane.

Uwaga. Wysylając nam rozwiązania zadań uczestnik Kursu udostępnia Politechnice Wrocławskiej

swoje dane osobowe, które przetwarzamy wyłącznie $\mathrm{w}$ zakresie niezbędnym do jego prowadzenia

(odesłanie zadań, prowadzenie statystyki). Szczegófowe informacje $0$ przetwarzaniu przez nas danych

osobowych są dostępne na stronie internetowej Kursu.

Adres internetowy Kursu: http: //www. im. pwr. edu. pl/kurs







XLIV

KORESPONDENCYJNY KURS

Z MATEMATYKI

luty 2015 r.

PRACA KONTROLNA nr 6- POZIOM PODSTAWOWY

l. Wyznacz dziedzinę funkcji

$f(x)=\log_{4-x^{2}}(2^{x}+2^{1-x}-3).$

2. $\mathrm{W}$ przedziale $[0,2\pi]$ rozwiąz nierównośč

$\displaystyle \cos^{2}2x+\sin^{2}x\leq\frac{1}{2}.$

3. Obwód trójk$\Phi$ta równoramiennego jest równy 8. Jaka powinna byč dfugośč boków tego

trójkąta, by objętośč bryły powstałej $\mathrm{z}$ jego obrotu dokola podstawy byla największa?

4. Rozwiąz równanie

$\sqrt{1-23^{x}+9^{x}}=3^{2x-1}-7\cdot 3^{x-1}+2.$

5. Punkt $B(1,1)$ jest wierzchołkiem kąta prostego $\mathrm{w}$ trójkącie prostokątnym $0$ polu 2, wpi-

sanym $\mathrm{w}$ okrąg $x^{2}+y^{2}+2x-2y-2=0.$ Znajd $\acute{\mathrm{z}}$ współrzędne pozostałych wierzchołków

tego trójkąta. Rozwiązanie zilustruj starannym rysunkiem.

6. Sporzqd $\acute{\mathrm{z}}$ staranny wykres funkcji

$f(x)=$

dla

dla

$|2x-5|\geq 3,$

$|2x-5|<3,$

$\mathrm{i}$ na jego podstawie wyznacz zbiór wartości tej funkcji. Rozwiąz nierównośč $f^{2}(x) \leq 1$

$\mathrm{i}$ zaznacz zbiór jej rozwiązań na osi $0x.$




PRACA KONTROLNA nr 6- POZ1OM ROZSZERZONY

l. Narysuj staranny wykres funkcji

$f(x)=|2^{|x-1|}-4|-2$

$\mathrm{i}$ opisz dokładnie sposób jego konstrukcji. Korzystając $\mathrm{z}$ rysunku, określ ilośč rozwiązań

równania $f(x)=m\mathrm{w}$ zalezności od parametru $m.$

2. Rozwiąz równanie

2 $\cos 2x+1=\sqrt{2\cos^{2}2x-6\sin^{2}x+5}.$

3. $\mathrm{W}$ trójkącie prostokątnym przeciwprostokątna ma długośč 3. Jakie powinny byč d1ugości

przyprostokątnych, aby objętośč bryły powstafej $\mathrm{z}$ jego obrotu dokołajednej $\mathrm{z}$ nich byla

największa?

4. Rozwiąz nierównośč

$2^{x}(1+\displaystyle \frac{\sqrt{3}}{2})^{\frac{1}{x}}-(2-\sqrt{3})^{-x}\geq 0.$

5. Znajd $\acute{\mathrm{z}}$ równania prostych stycznych do okręgu $x^{2}+y^{2}=25$ przechodzących przez punkt

$S(6,8)$. Wyznacz współrzędne punktów styczności $A, B\mathrm{i}$ oblicz pole obszaru ograniczo-

nego odcinkami AS, $BS$ oraz większym fukiem $AB$. Wykonaj staranny rysunek.

6. Zbadaj przebieg zmienności $\mathrm{i}$ narysuj staranny wykres funkcji

$f(x)=\displaystyle \frac{3x-2}{(x-1)^{2}}.$

Rozwiązania (rękopis) zadań z wybranego poziomu prosimy nadsyfač do

na adres:

181utego 20l5r.

Katedra Matematyki WPPT

Politechniki Wrocfawskiej

Wybrzez $\mathrm{e}$ Wyspiańskiego 27

$50-370$ WROCLAW.

Na kopercie prosimy $\underline{\mathrm{k}\mathrm{o}\mathrm{n}\mathrm{i}\mathrm{e}\mathrm{c}\mathrm{z}\mathrm{n}\mathrm{i}\mathrm{e}}$ zaznaczyč wybrany poziom! (np. poziom podsta-

wowy $\mathrm{l}\mathrm{u}\mathrm{b}$ rozszerzony). Do rozwiązań nalez $\mathrm{y}$ dołączyč zaadresowaną do siebie kopertę

zwrotną $\mathrm{z}$ naklejonym znaczkiem, odpowiednim do wagi listu. Prace niespelniające po-

danych warunków nie będą poprawiane ani odsyłane.

Adres internetowy Kursu: http://www.im.pwr.wroc.pl/kurs







XLV

KORESPONDENCYJNY KURS

Z MATEMATYKI

luty 2016 r.

PRACA KONTROLNA nr 6- POZIOM PODSTAWOWY

l. Andrzej przebiegł maraton, pokonujac drugą połowę trasy 10\% wo1niej od pierwszej.

Bernard, biegnąc początkowo w tempie narzuconym przez Andrzeja, w połowie czasu

biegu zwolnił 010\%. Usta1, który z biegaczy pierwszy przekroczy11inię mety.

2. Niech $p$ będzie liczbą pierwszą, $p\geq 7$. Uzasadnij, $\dot{\mathrm{z}}\mathrm{e}$ liczba $p^{\mathrm{z}}-49$ jest podzielna przez

24.

3. Rozwia $\dot{\mathrm{z}}$ równanie

12 $\cos^{2}3x\cdot\sin^{2}2x+\sin^{2}3x=4\sin^{2}3x\cdot\sin^{2}2x+3\cos^{2}3x.$

4. Wyznacz wszystkie argumenty x, dla których funkcja

$f(x)=\log_{3}(x^{2}-x)$ -log9 $(x^{2}+x-2)$

przyjmuje wartości dodatnie.

5. Przekątna rombu $0$ obwodzie 12 jest zawarta $\mathrm{w}$ prostej $x-2y=0$, a punkt $A(1,3)$ jest

jednym $\mathrm{z}$ jego wierzchofków. Wyznaczyč wspófrzędne pozostalych wierzchofków tego

rombu $\mathrm{i}$ obliczyč jego pole. Wykonač staranny rysunek.

6. Narysuj wykres funkcji

$f(x)=\sin^{2}x+\cos^{2}x+\sin^{4}x+\cos^{4}x+\sin^{6}x+\cos^{6}x.$

Znajd $\acute{\mathrm{z}}$ wszystkie liczby $\mathrm{z}$ przedziafu $[0,2\pi]$ spelniające nierównośč $8f(x)>19$. Zastosuj

wzory $\sin 2\alpha=2\sin\alpha\cdot\cos\alpha$ oraz $\cos 2\alpha=\cos^{2}\alpha-\sin^{2}\alpha.$




PRACA KONTROLNA nr 6- POZIOM ROZSZERZONY

l. Na nowym osiedlu wybudowano sześč budynków. $K\mathrm{a}\dot{\mathrm{z}}\mathrm{d}\mathrm{y}$ zostanie pomalowany na jeden

$\mathrm{z}$ trzech kolorów, a $\mathrm{k}\mathrm{a}\dot{\mathrm{z}}\mathrm{d}\mathrm{y}$ kolor zostanie wykorzystany co najmniej $\mathrm{r}\mathrm{a}\mathrm{z}$. Ustal, na ile

sposobów $\mathrm{m}\mathrm{o}\dot{\mathrm{z}}$ na pomalowač te budynki.

2. Zbadaj, dla jakich argumentów $x$ funkcja

$f(x)=7^{x^{4}}\cdot 49^{x}\cdot 5^{2x^{3}+x^{2}}-5^{x^{4}-2}\cdot 25^{x+1}\cdot 49^{x^{3}+\frac{1}{2}x^{2}}$

przyjmuje wartości dodatnie.

3. Rozwiąz równanie

$\mathrm{t}\mathrm{g}^{2}x=$ ($4\mathrm{t}\mathrm{g}^{2}x+3$ tg $x-1$) ($1$ -tg $ x+\mathrm{t}\mathrm{g}^{2}x-\mathrm{t}\mathrm{g}^{3}x+\ldots$).

4. Wskaz wszystkie wartości $x$, dla których suma nieskończonego ciągu geometrycznego

$ S(x)=2^{-2\sin 3x}+2^{-4\sin 3x}+2^{-6\sin 3x}+\cdots+2^{-2n\sin 3x}+\ldots$

nie przekracza jedności.

5. Rozwiąz nierównośč $\mathrm{l}\mathrm{y}\mathrm{n}$

$\log_{x+1}(x^{3}-x)\geq\log_{x+1}(x+2)+1.$

6. Boki $\triangle ABC$ zawarte są $\mathrm{w}$ prostych $y=4, y= 1-mx$ oraz $y=2(x-m)$. Wyznacz

wszystkie wymierne wartości parametru $m$, dla których pole rozwazanego trójkąta wy-

nosi $|\triangle ABC|=12$. Dla $\mathrm{k}\mathrm{a}\dot{\mathrm{z}}$ dej wyznaczonej wartości $m$ wykonaj odpowiedni rysunek.

Rozwiązania prosimy nadsyłač do dnia

181utego 20l6 na adres:

Wydziaf Matematyki

Politechniki Wrocfawskiej

Wybrzez $\mathrm{e}$ Wyspiańskiego 27

$50\rightarrow 370$ Wroclaw.

Na kopercie prosimy koniecznie zaznaczyč wybrany poziom. Do rozwiązań nalezy do-

l\S czyč zaadresowaną do siebie kopertę zwrotn\S z naklejonym znaczkiem, odpowiednim do wagi listu.

Prace niespelniające podanych warunków nie będą poprawiane ani odsylane.

Adres internetowy Kursu:

http://www. im.pwr.edu.pl/kur s







XLVI

KORESPONDENCYJNY KURS

Z MATEMATYKI

luty 2017 r.

PRACA KONTROLNA nr 6- POZIOM PODSTAWOWY

l. Rozwiązač równanie

$\displaystyle \frac{\sin x}{2\cos^{2}2x-1}=1.$

2. Niech $f(x)=\sqrt{x}$. Podač wzór funkcji:

a) $g(x)$, której wykresjest symetrycznym obrazem wykresu $f(x)$ względem prostej $x=1.$

b) $h(x)$, której wykresjest symetrycznym obrazem wykresu $f(x)$ względem punktu $(0,-1).$

Narysowač wykresy wszystkich funkcji. Uzasadnič, wykonując odpowiednie obliczenia,

$\dot{\mathrm{z}}\mathrm{e}$ znalezione funkcje spełniają podane warunki.

3. Wykazač, $\dot{\mathrm{z}}\mathrm{e}$ dla dowolnego $n\geq 2$ liczba

naturalnej $\mathrm{i}$ jest podzielna przez 81.

$\displaystyle \frac{1}{4} 100^{n}+4\cdot 10^{n}+16$ jest kwadratem liczby

4. Narysowač wykres funkcji

$f(x)=$

, gdy

, gdy

$-1\leq x\leq 1,$

$|x|>1.$

Poslugując się wykresem, podač zbiór wartości funkcji $f$ oraz jej najmniejszą $\mathrm{i}$ największą

wartośč na przedziałach [-1, 2] oraz $[0$, 3$].$

5. Znalez$\acute{}$č równanie stycznej $l$ do paraboli $y=x^{2}$ równolegfej do prostej $y=2x-3.$

Wyznaczyč punkt, $\mathrm{w}$ którym styczna do tej paraboli jest prostopadła do znalezionej

prostej $l$. Sporządzič rysunek.

6. Rozwiązač uklad równań

$\left\{\begin{array}{l}
x^{2}+y^{2}\\
- x1+-y1
\end{array}\right.$

$\mathrm{i}$ podač jego interpretację geometryczną.

8,

1.




PRACA KONTROLNA nr 5- POZIOM ROZSZERZONY

l. Niech $f(x)=\displaystyle \frac{x-1}{x+2}$. Podač $\mathrm{i}$ uzasadnič wzór funkcji, której wykres jest obrazem syme-

trycznym wykresu funkcji $f(x)$ względem prostej $x=2$. Sporządzič wykresy obu funkcji

$\mathrm{w}$ jednym ukladzie współrzędnych.

2. Stosujac zasadę indukcji matematycznej, udowodnič prawdziwośč wzoru

$\left(\begin{array}{l}
2\\
2
\end{array}\right) +\left(\begin{array}{l}
4\\
2
\end{array}\right)+\cdots+\left(\begin{array}{l}
2n\\
2
\end{array}\right) =\displaystyle \frac{n(n+1)(4n-1)}{6}$ dla $n\geq 1.$

3. $\mathrm{W}\mathrm{y}\mathrm{k}\mathrm{o}\mathrm{r}\mathrm{z}\mathrm{y}\mathrm{s}\mathrm{t}\mathrm{u}\mathrm{j}_{\Phi}\mathrm{c}$ metody rachunku rózniczkowego znalez/č zbiór wartości funkcji

$f(x)=x^{3}-3x^{2}-9x+3$

na przedziale [-1, 4]. Wyznaczyč przedzia1y $0$ dlugości l, $\mathrm{w}$ których znajdują się miejsca

zerowe tej funkcji $\mathrm{i}$ sporządzič jej wykres.

4. Znalez/č równanie stycznej $l$ do wykresu funkcji $f(x) = \displaystyle \frac{2}{x}+x^{2}\mathrm{w}$ punkcie przecięcia $\mathrm{z}$

prostą $y=x$. Wyznaczyč wszystkie styczne równolegle do znalezionej prostej $l.$

5. Narysowač wykres funkcji

$f(x)=1+\displaystyle \frac{\sin x}{1+\sin x}+(\frac{\sin x}{1+\sin x})^{2}+(\frac{\sin x}{1+\sin x})^{3}+(\frac{\sin x}{1+\sin x})^{4}+\ldots,$

gdzie prawa stronajest suma wszystkich wyrazów nieskończonego ciągu geometrycznego.

Rozwiazač nierównośč

$f(x)>\sqrt{3}\cos x.$

6. Wyznaczyč liczbę rozwiązań układu równań

$\left\{\begin{array}{l}
x^{2}+y^{2}=2y,\\
y=x^{2}-p.
\end{array}\right.$

$\mathrm{w}$ zalezności od parametru $p$. Podač interpretacje geometryczną układu.

$\mathrm{R}\mathrm{o}\mathrm{z}\mathrm{w}\mathrm{i}_{\Phi}$zania prosimy nadsyfač do dnia

181utego 20l7 na adres:

Wydziaf Matematyki

Politechniki Wrocfawskiej

Wybrzez $\mathrm{e}$ Wyspiańskiego 27

$50-370$ Wrocfaw.

Na kopercie prosimy koniecznie zaznaczyč wybrany poziom. Do rozwiązań nalez$\mathrm{y}$ do-

lączyč zaadresowana do siebie kopertę zwrotnq $\mathrm{z}$ naklejonym znaczkiem, odpowiednim do wagi listu.

Prace niespelniaj\S ce podanych warunków nie będą poprawiane ani odsyfane.

Adres internetowy Kursu: http: //www. im. pwr. edu. pl/kurs







XLVII

KORESPONDENCYJNY KURS

Z MATEMATYKI

luty 2018 r.

PRACA KONTROLNA nr 6- POZIOM PODSTAWOWY

1. Rozwia $\dot{\mathrm{z}}$ nierównośč

$\displaystyle \frac{3x-1}{x}\geq 1+\frac{\sqrt{1-x}}{x}.$

2. $\mathrm{W}$ zagrodzie jest 10 zwierząt, po parze danego gatunku. Ob1icz prawdopodobieństwo,

$\dot{\mathrm{z}}\mathrm{e}\mathrm{w}$ zagrodzie zostanie choč jedno zwierzę $\mathrm{k}\mathrm{a}\dot{\mathrm{z}}$ dego gatunku, jeśli wypuścimy $\mathrm{z}$ niej 4

losowo wybrane zwierzęta.

3. Bez $\mathrm{u}\dot{\mathrm{z}}$ ycia kalkulatora porównaj liczby

$a=\sqrt{11-4\sqrt{7}}$

oraz

$b=\log^{2}2\cdot\log 250+\log^{2}5$. log40.

4. Wyznacz wszystkie argumenty $x$, dla których funkcja

$f(x)=27^{x^{2}}\cdot 4^{x^{2}(x-3)}\cdot 3^{x}-6\cdot 3^{x^{3}+2}\cdot 2^{2x-7}$

przyjmuje wartości dodatnie.

5. Wyznacz skalę podobieństwa trójkąta równobocznego opisanego na okregu do trójkąta

równobocznego wpisanego $\mathrm{w}$ ten okrąg. Jaki jest stosunek pól tych trójk$\Phi$tów, a jaki

stosunek objetości stozka $0$ kącie rozwarcia $60^{\mathrm{o}}$ opisanego na kuli do objętości podobnego

stozka wpisanęgo $\mathrm{w}$ tę kulę?

6. Wśród prostokątów 0 ustalonej długości przekątnej p wskaz ten 0 największym polu.




PRACA KONTROLNA nr 6- POZ1OM ROZSZERZONY

l. Rozwiąz nierównośč

$x+1+\displaystyle \frac{1}{x-1}\geq(1+\frac{1}{x-1})\sqrt{2-x}.$

2. Narysuj wykres funkcji $f(x) = |1+ \displaystyle \log_{2}\frac{1}{|1-|x||}|$, opisz sfownie metodę jego kon-

strukcji oraz zbadaj, dla jakich argumentów spelniona jest nierównośč $f(x)\leq 1.$

3. Rozwiąz równanie logarytmiczne

$\log_{(x+2)^{2}}|x-1|=\log_{|x-1|}\sqrt{x+2}.$

4. Trzech alpinistów atakuje szczyt, wchodząc jednocześnie, niezaleznie od siebie, $\mathrm{z}$ róz-

nych stron góry. Prawdopodobieństwo zdobycia szczytu szlakiem północnym wynosi $\displaystyle \frac{1}{3},$

szlakiem zachodnim- $\displaystyle \frac{1}{2}$, a południowym- $\displaystyle \frac{3}{7}$. Oblicz prawdopodobieństwo, $\dot{\mathrm{z}}\mathrm{e}$ atak się

powiedzie (tzn. przynajmniej jeden $\mathrm{z}$ alpinistów zdobędzie szczyt).

5. Oblicz tangens kąta rozwarcia stozka, dla którego kula wpisana w ten stozek zajmuje

dokfadnie polowę jego objętości.

6. Wyznacz równanie linii będącej zbiorem środków wszystkich okręgów stycznych do pro-

stej $y=0$ ijednocześnie stycznych do okręgu $x^{2}+y^{2}=2$. Wykonaj odpowiedni rysunek.

Rozwiqzania prosimy nadsyłač do dnia 181utego 2018 na adres:

Korespondencyjny Kurs $\mathrm{z}$ Matematyki

POZIOM$\ldots$ (wpisač wfaściwy)

Wydziaf Matematyki

Politechnika Wrocfawska

Wybrzez $\mathrm{e}$ Wyspiańskiego 27

$50-370$ Wrocfaw

Na kopercie prosimy koniecznie zaznaczyč wybrany poziom (podstawowy, rozsze-

rzony lub podstawowy $\mathrm{i}$ rozszerzony). Do rozwiązań nalez$\mathrm{y}$ dofączyč zaadresowaną do siebie

kopertę zwrotn\S z naklejonym znaczkiem, odpowiednim do wagi listu $\mathrm{i}$ rozmiaru koperty. Prace nie-

spełniające podanych warunków nie będa poprawiane ani odsyfane.

Adres internetowy Kursu:

http://www. im.pwr.edu.pl/kur s







XLVIII

KORESPONDENCYJNY KURS

Z MATEMATYKI

luty 2019 r.

PRACA KONTROLNA nr 6- POZIOM PODSTAWOWY

l. Pewnej $\mathrm{m}\mathrm{r}\mathrm{o}\acute{\mathrm{z}}\mathrm{n}\mathrm{e}\mathrm{j}$ zimy trzy przeziębione krasnale kupowały $\mathrm{w}$ aptece lekarstwa. Pierw-

szego męczył straszny ból gardfa. Kupil więc trzy tabletki do ssania, tabletkę na kaszel

$\mathrm{i}$ kropelkę do nosa. Zapłacił za wszyskto 4 grosze. Drugiemu dokuczaf uporczywy kasze1,

za tę samą cenę kupił trzy tabletki na kaszel, tabletkę do ssania $\mathrm{i}$ kropelkę do nosa. Trze-

ci mial straszny katar. Poprosif więc $0$ trzy kropelki do nosa, $0$ tabletkę do ssania oraz

$0$ tabletkę na kaszel. A dowiedziawszy się, $\dot{\mathrm{z}}\mathrm{e}$ ma zaplacič 2 grosze, pomyś1ał chwi1kę,

kichnął $\mathrm{i}$ powiedziaf do aptekarza:,,Pomylił się Pan!'' Uzasadnič, $\dot{\mathrm{z}}\mathrm{e}$ krasnal miał rację.

2. Obliczyč, ile kolejnych dodatnich liczb naturalnych podzielnych przez 3 nalez $\mathrm{y}$ dodač do

siebie, aby otrzymana suma była równa liczbie $115a^{-1}$, gdzie

$a=\displaystyle \frac{1}{3\cdot 5}+\frac{1}{5\cdot 7}+\frac{1}{7\cdot 9}+\ldots+\frac{1}{691\cdot 693}.$

3. Rozwiązač równanie

$\sin^{3}x$ ($1+$ ctg $x$)$+\cos^{3}x(1+\mathrm{t}\mathrm{g}x)=\sin 2x+2\sin^{2}x.$

4. Rzucamy pięč razy jednorodną kostką do gry. Obliczyč prawdopodobieństwo wyrzuce-

nia sumy oczek większej od 20, jeś1i wiadomo, $\dot{\mathrm{z}}\mathrm{e}$ suma oczek wyrzuconych $\mathrm{w}$ trzech

pierwszych rzutach wynosi 10.

5. $\mathrm{W}$ trójkąt równoramienny, którego ramiona sa dwa razy dłuzsze od podstawy, wpisano

$\mathrm{p}\mathrm{r}\mathrm{o}\mathrm{s}\mathrm{t}\mathrm{o}\mathrm{k}_{\Phi}\mathrm{t}\mathrm{w}$ taki sposób, $\dot{\mathrm{z}}$ ejeden $\mathrm{z}$ boków $\mathrm{P}^{\mathrm{r}\mathrm{o}\mathrm{s}\mathrm{t}\mathrm{o}\mathrm{k}}\Phi^{\mathrm{t}\mathrm{a}}$ zawarty jest $\mathrm{w}$ podstawie trójk$\Phi$ta.

Jakie powinny byč wymiary tego prostokąta, aby jego pole było największe? Wyznaczyč

wartośč tego największego pola.

6. Narysowač $\mathrm{w}$ prostokqtnym układzie wspólrzędnych wykresy funkcji

$f(x)=-\displaystyle \frac{2}{x}$

oraz

$g(x)=f(|x|-1)+1.$

Rozwiązač nierównośč $g(x)\geq f(x)\mathrm{i}$ zaznaczyč zbiór jej rozwiązań na osi liczbowej.




PRACA KONTROLNA nr 6- POZIOM ROZSZERZONY

l. Ojciec $\mathrm{i}$ syn obchodzą urodziny tego samego dnia. $\mathrm{W}$ roku 2019 podczas uroczystości

urodzin zapytano jubilatów, ile maja lat. Ojciec odpowiedziaf:,,Jeśli wiek mego syna

przemnozę przez swój wiek za $31\mathrm{l}\mathrm{a}\mathrm{t}$, to otrzymam rok swego urodzenia'' Syn dodaf:

,,Aja otrzymam rok swego urodzenia, jeśli wiek mego ojca sprzed 161at przemnozę przez

swój wiek za 331ata'' $\mathrm{W}$ którym roku urodzif się $\mathrm{k}\mathrm{a}\dot{\mathrm{z}}\mathrm{d}\mathrm{y}\mathrm{z}$ jubilatów?

2. Wyznaczyč dziedzinę naturalną funkcji

$f(x)=\log(3^{3x-1}-3^{2x-1}-3^{x+1}+3).$

3. Rozwiązač równanie

4 $\sin x\cdot\sin 2x\cdot\sin 3x=\sin 4x.$

4. $\mathrm{W}$ dwóch urnach znajdują się kule biafe $\mathrm{i}$ czarne, przy czym $\mathrm{w}$ pierwszej urnie $\mathrm{s}\Phi 4$ kule

białe $\mathrm{i}6$ czarnych, a $\mathrm{w}$ drugiej jest 7 ku1 bia1ych $\mathrm{i}3$ czarne. Rzucamy dwa razy jedno-

rodną $\mathrm{k}\mathrm{o}\mathrm{s}\mathrm{t}\mathrm{k}_{\Phi}$ do gry. Jeśli suma wyrzuconych oczek jest mniejsza $\mathrm{n}\mathrm{i}\dot{\mathrm{z}}6$, losujemy dwie

kule $\mathrm{z}$ pierwszej urny. Jeśli suma wyrzuconych oczek jest większa $\mathrm{n}\mathrm{i}\dot{\mathrm{z}}9$, losujemy dwie

kule $\mathrm{z}$ drugiej urny. $\mathrm{W}$ pozostałych przypadkach losujemy po jednej kuli $\mathrm{z}\mathrm{k}\mathrm{a}\dot{\mathrm{z}}$ dej urny.

Obliczyč prawdopodobieństwo wylosowania dwóch kul bialych.

5. Uzasadnič, $\dot{\mathrm{z}}\mathrm{e}$ dla $\mathrm{k}\mathrm{a}\dot{\mathrm{z}}$ dej liczby naturalnej $n$ liczba $n^{5}-n$ jest podzielna przez 5. Czy

prawdq jest, $\dot{\mathrm{z}}\mathrm{e}$ jest ona $\mathrm{t}\mathrm{e}\dot{\mathrm{z}}$ podzielna przez 30?

6. $\mathrm{W}$ trójkąt równoramienny, którego ramiona są dfugości $a$, a miara kąta zawartego po-

między nimi wynosi $\alpha$, wpisano prostokat $\mathrm{w}$ taki sposób, $\dot{\mathrm{z}}\mathrm{e}$ jeden $\mathrm{z}$ boków prostokąta

zawarty jest $\mathrm{w}$ jednym $\mathrm{z}$ ramion trójkąta. Jakie powinny byč wymiary tego $\mathrm{P}^{\mathrm{r}\mathrm{o}\mathrm{s}\mathrm{t}\mathrm{o}\mathrm{k}}\Phi^{\mathrm{t}\mathrm{a}},$

aby jego pole bylo największe? Wyznaczyč wartośč tego największego pola.

Rozwiązania (rękopis) zadań z wybranego poziomu prosimy nadsyfač do

na adres:

18 stycznia 20l9r.

Wydziaf Matematyki

Politechnika Wrocfawska

Wybrzez $\mathrm{e}$ Wyspiańskiego 27

$50-370$ WROCLAW.

Na kopercie prosimy $\underline{\mathrm{k}\mathrm{o}\mathrm{n}\mathrm{i}\mathrm{e}\mathrm{c}\mathrm{z}\mathrm{n}\mathrm{i}\mathrm{e}}$ zaznaczyč wybrany poziom! (np. poziom podsta-

wowy lub rozszerzony). Do rozwiązań nalez $\mathrm{y}$ dołaczyč zaadresowaną do siebie kopertę

zwrotną $\mathrm{z}$ naklejonym znaczkiem, odpowiednim do wagi listu. Prace niespełniające po-

danych warunków nie bedą poprawiane ani odsyłane.

Uwaga. Wysylaj\S c nam rozwiązania zadań uczestnik Kursu udostępnia nam swoje dane osobo-

we, które przetwarzamy wyłącznie $\mathrm{w}$ zakresie niezbędnym do jego prowadzenia (odeslanie zadań,

prowadzenie statystyki). Szczególowe informacje $0$ przetwarzaniu przez nas danych osobowych sa

dostepne na stronie internetowej Kursu.

Adres internetowy Kursu: http://www.im.pwr.edu.pl/kurs







XLIX

KORESPONDENCYJNY KURS

Z MATEMATYKI

luty 2020 r.

PRACA KONTROLNA $\mathrm{n}\mathrm{r} 6-$ POZIOM PODSTAWOWY

1. $\mathrm{W}$ szufiadzie znajduje się 6 róznych par rękawiczek. Ob1icz prawdopodobieństwo, $\dot{\mathrm{z}}\mathrm{e}$

wśród 51osowo wybranych rękawic jest co najmniej jedna para.

2. Wyznacz dziedzinę $\mathrm{i}$ zbadaj, dla jakich argumentów funkcja

$f(x)=\log_{\sqrt{3}}(x+3)-\log_{3}(9-x^{2})$

przyjmuje wartości ujemne.

3. Wśród prostok$\Phi$tów wpisanych $\mathrm{w}\mathrm{o}\mathrm{k}\mathrm{r}\Phi \mathrm{g}\mathrm{o}$ promieniu $R$ bez $\mathrm{u}\dot{\mathrm{z}}$ ycia metod rachunku róz-

niczkowego wskaz ten, którego pole jest największe.

4. Rozwiąz nierównośč

$4^{x^{3}-x+2}\cdot 5^{2x-3x^{2}}-2^{4-3x^{2}}\cdot 25^{x^{3}}\geq 0.$

5. Powierzchnia boczna stozka po rozcięciu jest wycinkiem kofa $\mathrm{o}\mathrm{k}_{\Phi}\mathrm{c}\mathrm{i}\mathrm{e}216^{\mathrm{o}}$

stawy stozka wynosi $ 6\pi$. Oblicz objętośč kuli wpisanej $\mathrm{w}$ ten stozek.

Obwód pod-

6. Narysuj wykres funkcji

$f(x)=-1+2^{1-|1-|x||}$

i precyzyjnie opisz zastosowaną metodę jego konstrukcji. Na podstawie rysunku wskaz

przedziafy monotoniczności funkcji oraz zbiór jej wartości.




PRACA KONTROLNA $\mathrm{n}\mathrm{r} 6-$ POZIOM ROZSZERZONY

l. Developer chce pomalowač $\mathrm{k}\mathrm{a}\dot{\mathrm{z}}$ de $\mathrm{z} 11$ pięter nowo wybudowanego wiezowca na jeden

$\mathrm{z}3$ kolorów występujących $\mathrm{w}$ jego logo, przy czym $\mathrm{k}\mathrm{a}\dot{\mathrm{z}}\mathrm{d}\mathrm{y}$ kolor ma zostač wykorzystany

co najmniej jeden $\mathrm{r}\mathrm{a}\mathrm{z}$. Obliczyč prawdopodobieństwo, $\dot{\mathrm{z}}\mathrm{e}$ dwaj niezalezni graficy, którym

zlecono zaprojektowanie kolorystyki budynku, przedstawią ten sam projekt. Przyjąč, $\dot{\mathrm{z}}\mathrm{e}$

wybór przez nich $\mathrm{k}\mathrm{a}\dot{\mathrm{z}}$ dego takiego ukladu kolorów jest jednakowo prawdopodobny.

2. Rozwiąz równanie

$8x^{3}=1+6x,$

stosując podstawienie $x=\cos\alpha.$

3. Określ dziedzinę $\mathrm{i}$ zbadaj, dla jakich argumentów funkcja

$f(x)=\displaystyle \log_{x^{2}-1}(x^{2}-2x)-\log_{x^{2}-1}(2-\frac{4}{x})$

przyjmuje wartości nieujemne.

4. Rozwia $\dot{\mathrm{Z}}$ nierównośč

$1+\mathrm{t}\mathrm{g}^{2}2x-\mathrm{t}\mathrm{g}^{4}2x+\mathrm{t}\mathrm{g}^{6}2x-\ldots\leq 3\sin 2x-\sin^{2}2x.$

5. Wśród prostopadfościanów $0$ podstawie kwadratu wpisanych $\mathrm{w}$ kulę $0$ promieniu $R$ wskaz

ten, którego objętośč jest największa.

6. Określ dziedzinę, wyznacz przedzialy monotoniczności oraz wszystkie lokalne ekstrema

funkcji

$f(x)=\displaystyle \frac{(x+1)^{2}}{x(x-2)}.$

$\mathrm{s}_{\mathrm{P}^{\mathrm{o}\mathrm{r}\mathrm{z}}\mathrm{a}}\mathrm{d}\acute{\mathrm{z}}$ jej staranny wykres.

Rozwiqzania (rękopis) zadań z wybranego poziomu prosimy nadsyłač do

na adres:

181utego 2020r.

WydziaX Matematyki

Politechnika Wrocfawska

Wybrzez $\mathrm{e}$ Wyspiańskiego 27

$50-370$ WROCLAW.

Na kopercie prosimy $\underline{\mathrm{k}\mathrm{o}\mathrm{n}\mathrm{i}\mathrm{e}\mathrm{c}z\mathrm{n}\mathrm{i}\mathrm{e}}$ zaznaczyč wybrany poziom! (np. poziom podsta-

wowy lub rozszerzony). Do rozwiązań nalez $\mathrm{y}$ dołączyč zaadresowaną do siebie koperte

zwrotną $\mathrm{z}$ naklejonym znaczkiem, odpowiednim do formatu listu. Polecamy stosowanie

kopert formatu C5 $(160\mathrm{x}230\mathrm{m}\mathrm{m})$ ze znaczkiem $0$ wartości 3,30 zł. Na $\mathrm{k}\mathrm{a}\dot{\mathrm{z}}$ dą większą

koperte nale $\dot{\mathrm{z}}\mathrm{y}$ nakleič drozszy znaczek. Prace niespelniające podanych warunków nie

będą poprawiane ani odsyłane.

Uwaga. Wysylając nam rozwiązania zadań uczestnik Kursu udostępnia Politechnice Wroclawskiej swoje da-

ne osobowe, które przetwarzamy wyłącznie $\mathrm{w}$ zakresie niezbędnym do jego prowadzenia (odeslanie zadań,

prowadzenie statystyki). Szczególowe informacje $0$ przetwarzaniu przez nas danych osobowych są dostepne na

stronie internetowej Kursu.

Adres internetowy Kursu: http://www. im.pwr.edu.pl/kurs







L KORESPONDENCYJNY KURS

Z MATEMATYKI

luty 2021 r.

PRACA KONTROLNA nr 6- POZIOM PODSTAWOWY

l. Suma wszystkich krawędzi prostopadłościanu $0$ podstawie kwadratowej wynosi 16 cm.

Jakie są wymiary tego prostopadfościanu, który ma najwieksze pole powierzchni cafko-

witej?

2. Sporząd $\acute{\mathrm{z}}$ wykres funkcji

$f(x)=|x^{2}-4|-2x$

oraz wyznacz liczbę pierwiastków równania

$f(x)=m$

$\mathrm{w}$ zalezności od parametru $m.$

3. Ze zbioru trzech elementów $\{a,b,c\}$ pobrano ze zwracaniem próbkę $0$ liczności 9 e1e-

mentów. Oblicz prawdopodobieństwo zdarzenia, $\dot{\mathrm{z}}\mathrm{e}\mathrm{w}$ tej próbie $\mathrm{k}\mathrm{a}\dot{\mathrm{z}}\mathrm{d}\mathrm{y}$ element wystąpi

dokładnie trzy razy.

4. Sześciu przyjaciól $A, B, C, D, E, F$ zajmuje sześč kolejnych miejsc $\mathrm{w}$ jednym rzędzie sali

kinowej. Na ile sposobów mogą usiąśč, aby: a) osoby $A, B, C$ siedziałyjedna obok drugiej

($\mathrm{w}$ dowolnej kolejności)? b) $\dot{\mathrm{z}}$ adne dwie $\mathrm{z}$ osób $A, B, C$ nie siedziały obok siebie?

5. Wyznacz wspólrzędne wierzcholków trójk$\Phi$ta $ABC$, którego boki zawieraja się $\mathrm{w}$ pro-

stych: $y=2, 2x-y+10=0, 4x+3y=0$. Następnie wyznacz współrzędne wierzchołków

trójk$\Phi$ta, który jest obrazem trójkąta $ABC$ wjednokfadności $0$ środku $O(0,0)\mathrm{i}$ skali $-2.$

Oblicz pole trójkąta $ABC\mathrm{i}$ jego obrazu $\mathrm{w}$ tym przeksztafceniu.

6. Trójkąt równoboczny $ABC 0$ boku l dzielimy na cztery przystajqce trójkąty, lqczqc

środki jego boków. Usuwamy środkowy trójkąt (krok l). To samo robimy $\mathrm{z} \mathrm{k}\mathrm{a}\dot{\mathrm{z}}$ dym

$\mathrm{z}$ trzech pozostałych trójkątów (krok 2). Proces ten wykonujemy $n$ razy. Jaka jest suma

pól usuniętych trójkątów po trzech krokach? Ile kroków wystarczy wykonač, aby suma

pól usuniętych trójkątów była większa $\mathrm{n}\mathrm{i}\dot{\mathrm{z}}3/4$ pola wyjściowego trójkąta?




PRACA KONTROLNA $\mathrm{n}\mathrm{r} 6-$ POZIOM ROZSZERZONY

l. Ile jest czterocyfrowych kodów PIN, $\mathrm{w}$ których: a) $\dot{\mathrm{z}}$ adna cyfra się nie powtarza? b)

któraś $\mathrm{z}$ cyfr się powtarza? Ile kodów jest więcej: tych, $\mathrm{w}$ których $\dot{\mathrm{z}}$ adna cyfra się nie

powtarza, czy tych, $\mathrm{w}$ których któraś $\mathrm{z}$ cyfr się powtarza?

2. Pięciu wioślarzy $A, B, C, D, E$ plynie lodzią, na której znajduje się pięč poprzecznych

lawek dwuosobowych. Wioslarze $A, B, C$ mogą usiąśč tylko przy prawej burcie, natomiast

wioślarze $D\mathrm{i}$ E- tylko przy lewej. Jakie jest prawdopodobieństwo zdarzenia, $\dot{\mathrm{z}}\mathrm{e}$ miejsca

obok wioślarzy $D\mathrm{i}E$ będą zajęte?

3. Znajd $\acute{\mathrm{z}}$ współrzędne wierzchołka $C$ trójkąta równoramiennego $ABC$, gdzie $A(2,0), B(0,2),$

wiedząc, $\dot{\mathrm{z}}\mathrm{e}$ środkowe $AD\mathrm{i}$ {\it BE} $\mathrm{p}\mathrm{r}\mathrm{z}\mathrm{e}\mathrm{c}\mathrm{i}\mathrm{n}\mathrm{a}\mathrm{j}_{\Phi}$ się pod $\mathrm{k}_{\Phi}\mathrm{t}\mathrm{e}\mathrm{m}$ prostym.

4. $\mathrm{W}$ prostokatnym ukladzie wspólrzędnych dane sq punkty $A(\alpha,0)\mathrm{i}B(b,0)$, gdzie

$0 < a < b.$ Znajd $\acute{\mathrm{z}}$ punkt $C(0,c)$, gdzie $c > 0$, dla którego miara kąta $\angle ACB$ jest

największa.

5. Wyznacz wszystkie styczne do wykresu funkcji $f(x)=\displaystyle \frac{x-1}{x+1}$ równolegle do prostej

$x-2y=0\mathrm{i}$ oblicz pole wielokąta, którego wierzchołkami są punkty przecięcia otrzyma-

nych prostych $\mathrm{z}$ osiami układu. Wykonaj staranny rysunek.

6. Kwadrat ABCD $0$ boku $a$ dzielimy na dziewięč przystających kwadratów, dzieląc $\mathrm{k}\mathrm{a}\dot{\mathrm{z}}\mathrm{d}\mathrm{y}$

$\mathrm{z}$ boków kwadratu na trzy równe części $\mathrm{i}$ usuwamy środkowy kwadrat (krok l). Nastepnie

to samo robimy $\mathrm{w}$ pozostafych ośmiu kwadratach (krok 2). Proces ten powtarzany jest

nieskończenie wiele razy. Jaka jest suma pól kwadratów usuniętych $\mathrm{w}n$ krokach? Ile

kroków wystarczy wykonač, aby suma pól usuniętych kwadratów byfa większa $\mathrm{n}\mathrm{i}\dot{\mathrm{z}}$ połowa

pola wyjściowego kwadratu? Jaka jest suma pól wszystkich usuniętych kwadratów (po

nieskończenie wielu krokach)?

Rozwiązania (rękopis) zadań z wybranego poziomu prosimy nadsyfač do 20.02.2021r. na

adres:

Wydziaf Matematyki

Politechnika Wrocfawska

Wybrzez $\mathrm{e}$ Wyspiańskiego 27

$50-370$ WROCLAW.

Na kopercie prosimy $\underline{\mathrm{k}\mathrm{o}\mathrm{n}\mathrm{i}\mathrm{e}\mathrm{c}\mathrm{z}\mathrm{n}\mathrm{i}\mathrm{e}}$ zaznaczyč wybrany poziom! (np. poziom podsta-

wowy lub rozszerzony). Do rozwiązań nalez $\mathrm{y}$ dołączyč zaadresowaną do siebie kopertę

zwrotną $\mathrm{z}$ naklejonym znaczkiem, odpowiednim do formatu listu. Polecamy stosowanie

kopert formatu C5 $(160\mathrm{x}230\mathrm{m}\mathrm{m})$ ze znaczkiem $0$ wartości 3,30 zł. Na $\mathrm{k}\mathrm{a}\dot{\mathrm{z}}$ dą większą

kopertę nalez $\mathrm{y}$ nakleič $\mathrm{d}\mathrm{r}\mathrm{o}\dot{\mathrm{z}}$ szy znaczek. Prace niespełniające podanych warunków nie

bedą poprawiane ani odsyłane.

Uwaga. Wysyfaj\S c nam rozwi\S zania zadań uczestnik Kursu udostępnia Politechnice Wrocfawskiej

swoje dane osobowe, które przetwarzamy wyłącznie $\mathrm{w}$ zakresie niezbędnym do jego prowadzenia

(odeslanie zadań, prowadzenie statystyki). Szczególowe informacje $0$ przetwarzaniu przez nas danych

osobowych są dostępne na stronie internetowej Kursu.

Adres internetowy Kursu: http: //www. im. pwr. edu. pl/kurs







LI KORESPONDENCYJNY KURS

Z MATEMATYKI

luty 2022 r.

PRACA KONTROLNA nr 6- POZIOM PODSTAWOWY

l. Prawdopodobieństwo, $\dot{\mathrm{z}}\mathrm{e}\mathrm{w}$ dowolnie wybranym przedziale pociągu relacji Warszawa-

Wrocfaw podrózny znajdzie co najmniej jedno wolne miejsce wynosi $\displaystyle \frac{1}{2}$. Podrózny szuka

pierwszego wolnego miejsca, zaglądając do $\mathrm{k}\mathrm{a}\dot{\mathrm{z}}$ dego kolejnego przedziafu. Oblicz praw-

dopodobieństwo zdarzenia, $\dot{\mathrm{z}}\mathrm{e}$ liczba odwiedzonych przez niego przedziałów nie przekro-

czy 4.

2. Rozwiąz nierównośč wykładniczą

$2^{x^{3}}\cdot 9^{2x-1}<3^{x^{3}-2}\cdot 4^{2x}$

3. $\mathrm{W}$ trójkącie równoramiennym $\triangle ABC0$ ramionach $AC\mathrm{i}BC\mathrm{k}_{\Phi}\mathrm{t}$ przy podstawie $AB$ ma

miarę $\alpha$. Na boku $AC$ umieszczono punkt $D\mathrm{w}$ taki sposób, $\dot{\mathrm{z}}\mathrm{e}$ trójkąty $\triangle ABC\mathrm{i}\triangle ABD$

są podobne. Wyznacz skalę podobieństwa tych trójkątów oraz warunki rozwiązalności

zadania. Oblicz stosunek pól tych trójkątów oraz stosunek objętości stozków powstałych

przez obrót tych trójkątów wokół ich osi symetrii.

4. Wyznacz wszystkie $\mathrm{m}\mathrm{o}\dot{\mathrm{z}}$ liwe wartości kąta ostrego $\alpha \mathrm{j}\mathrm{e}\dot{\mathrm{z}}$ eli wiadomo, $\dot{\mathrm{z}}\mathrm{e}$

tg $2\alpha+$ ctg $2\displaystyle \alpha=-\frac{4\sqrt{3}}{3}.$

5. Niech $x\in[0,2\pi]$. Rozwiąz nierównośč

$\sin^{5}x+\cos^{5}x\geq\sin^{4}x\cdot\cos x+\cos^{4}x\cdot\sin x.$

6. Wyznacz wszystkie argumenty $x$, dla których funkcja

$f(x)=\log_{2}(x+2)-2\log_{4}\sqrt{x^{3}+8}$

przyjmuje wartości niedodatnie.




PRACA KONTROLNA $\mathrm{n}\mathrm{r} 6-$ POZIOM ROZSZERZONY

l. Rzucamy cztery razy $\mathrm{j}\mathrm{e}\mathrm{d}\mathrm{d}\mathrm{n}\mathrm{k}\mathrm{o}\mathrm{s}\mathrm{t}\mathrm{k}_{\Phi}$ do gry. Oblicz prawdopodobieństwo, $\dot{\mathrm{z}}\mathrm{e}$ suma

wyrzuconych oczek przekroczy 12, jeś1i wiadomo, $\dot{\mathrm{z}}\mathrm{e}$ suma oczek wyrzuconych $\mathrm{w}$ dwóch

pierwszych rzutach wynosi 8.

2. Rozwiąz równanie trygonometryczne

-csions22{\it xx}.. ssiinn{\it xx} --csoins 22{\it xx}.. ccooss {\it xx} $=$1.

3. Rozwiąz równanie

$5^{\mathrm{t}\mathrm{g}^{2}x-1}+5^{3-\mathrm{t}\mathrm{g}^{2}x}=26.$

4. Rozwiąz nierównośč logarytmiczną

$1+\log_{x-1}x<\log_{x-1}(x+6).$

5. Wyznacz dziedzinę $\mathrm{i}$ miejsca zerowe funkcji

$f(x)=\log_{\sin(-x)}(4\sin x\cdot\cos x-1).$

6. $\mathrm{W}$ trójk$\Phi$cie równoramiennym $\triangle ABC$, którego podstawa $AB$ ma dlugośč 4, miara $\mathrm{k}_{\Phi^{\mathrm{t}\mathrm{a}}}$

pomiędzy ramionami $AC\mathrm{i}BC$ wynosi $30^{\mathrm{o}}$ Oblicz objętośč bryly powstałej przez obrót

tego trójkąta wzgledem jednego $\mathrm{z}$ jego ramion.

Rozwiqzania (rękopis) zadań z wybranego poziomu prosimy nadsylač do

2022r. na adres:

20 1utego

Wydziaf Matematyki

Politechnika Wrocfawska

Wybrzez $\mathrm{e}$ Wyspiańskiego 27

$50-370$ WROCLAW,

lub elektronicznie, za pośrednictwem portalu talent. $\mathrm{p}\mathrm{w}\mathrm{r}$. edu. pl

Na kopercie prosimy $\underline{\mathrm{k}\mathrm{o}\mathrm{n}\mathrm{i}\mathrm{e}\mathrm{c}\mathrm{z}\mathrm{n}\mathrm{i}\mathrm{e}}$ zaznaczyč wybrany poziom! (np. poziom podsta-

wowy lub rozszerzony). Do rozwiązań nalez $\mathrm{y}$ dołączyč zaadresowaną do siebie kopertę

zwrotną $\mathrm{z}$ naklejonym znaczkiem, odpowiednim do formatu listu. Polecamy stosowanie

kopert formatu C5 $(160\mathrm{x}230\mathrm{m}\mathrm{m})$ ze znaczkiem $0$ wartości 3,30 zł. Na $\mathrm{k}\mathrm{a}\dot{\mathrm{z}}$ dą większą

koperte nalez $\mathrm{y}$ nakleić drozszy znaczek. Prace niespelniające podanych warunków nie

będą poprawiane ani odsyłane.

Uwaga. Wysyłając nam rozwi\S zania zadań uczestnik Kursu udostępnia Politechnice Wroclawskiej

swoje dane osobowe, które przetwarzamy wyłącznie $\mathrm{w}$ zakresie niezbednym do jego prowadzenia

(odesfanie zadań, prowadzenie statystyki). Szczegófowe informacje $0$ przetwarzaniu przez nas danych

osobowych sa dostępne na stronie internetowej Kursu.

Adres internetowy Kursu: http: //www. im. pwr. edu. pl/kurs







LII

KORESPONDENCYJNY KURS

Z MATEMATYKI

luty 2023 r.

PRACA KONTROLNA $\mathrm{n}\mathrm{r} 6-$ POZIOM PODSTAWOWY

l. Na ile róznych sposobów $\mathrm{m}\mathrm{o}\dot{\mathrm{z}}\mathrm{e}$ się ustawić do zdjęcia sześcioosobowa rodzina, $\mathrm{j}\mathrm{e}\dot{\mathrm{z}}$ eli wszy-

scy mają stać $\mathrm{w}$ jednym rzędzie, a najmłodsza córka musi stać obok mamy?

2. $\mathrm{J}\mathrm{e}\dot{\mathrm{z}}$ eli $\mathrm{w}$ dwóch rzutach sześcienną kostką do gry gracz otrzyma sumę oczek wynoszącą

przynajmniej 10, to wygrywa 100 $\mathrm{z}l.$, a $\mathrm{j}\mathrm{e}\dot{\mathrm{z}}$ eli otrzyma mniej $\mathrm{n}\mathrm{i}\dot{\mathrm{z}} 10\mathrm{i}$ więcej $\mathrm{n}\mathrm{i}\dot{\mathrm{z}}6$, to

wygrywa 50 $\mathrm{z}l. \mathrm{W}$ pozostalych przypadkach przegrywa $\mathrm{i}$ musi zapfacić $80\mathrm{z}1$. Wyznacz

wartość oczekiwaną wygranej gracza $\mathrm{w}$ tej grze. Jak organizator takiej gry powinien

zmienić oplatę za przegraną $\dot{\mathrm{z}}$ eby mógł liczyć na zarobek po wzięciu $\mathrm{w}$ niej udziału przez

wielu graczy?

3. Uzasadnij, $\dot{\mathrm{z}}\mathrm{e}$ dla $\mathrm{k}\mathrm{a}\dot{\mathrm{z}}$ dego $n$ naturalnego liczba $2n^{3}+3n^{2}+n$ jest podzielna przez 6.

4. Oblicz piąty wyraz ciągu arytmetycznego

$\log_{2}x_{1},\log_{2}x_{2},\log_{2}x_{3},$

wiedząc, $\displaystyle \dot{\mathrm{z}}\mathrm{e}x_{1}+x_{2}+x_{3}=\frac{7}{4}$ oraz $x_{2}=\displaystyle \frac{1}{2}.$

5. Oblicz prawdopodobieństwo, $\dot{\mathrm{z}}\mathrm{e}\mathrm{w} 8$ rzutach monetą pojawi się seria przynajmniej 5

reszek lub 5 orłów pod rząd.

6. Losujemy jedną liczbę spošród liczb l, 2, 2023. Znajd $\acute{\mathrm{z}}$ prawdopodobieństwo, $\dot{\mathrm{z}}\mathrm{e} \mathrm{a}$)

wybrana liczba będzie podzielna przez 5 $\mathrm{i}$ przez ll, b) wybrana liczba będzie podzielna

przez 51ub przez 11.




PRACA KONTROLNA $\mathrm{n}\mathrm{r} 6-$ POZIOM ROZSZERZONY

l. Jakiejest prawdopodobieństwo, $\dot{\mathrm{z}}\mathrm{e}\mathrm{w}$ sześciu rzutach standardową kostką do gry wypadną

wszystkie $\mathrm{m}\mathrm{o}\dot{\mathrm{z}}$ liwe liczby oczek?

2. Dla jakich wartości parametru $p$ równanie

$x^{2}-(2^{p}-1)x-3(4^{p-1}-2^{p-2})=0$

ma dwa pierwiastki rzeczywiste róznych znaków?

3. $\mathrm{Z}$ pierwszej urny zawierajqcej $n$ kul bialych $\mathrm{i}$ cztery czarne losujemy dwie kule $\mathrm{i}$ wrzucamy

je do drugiej urny, początkowo pustej. $\mathrm{Z}$ tej drugiej losujemy wtedy jedną kulę.

a) Dlajakich wartości $n$ prawdopodobieństwo wyciągnięcia bialej kuli $\mathrm{z}$ drugiej urny jest

większe od 3/4?

b) Przyjmując $n=6$ oblicz prawdopodobieństwo, $\dot{\mathrm{z}}\mathrm{e}\mathrm{z}$ pierwszej urny wylosowano dwie

białe kule, ješli wiadomo, $\dot{\mathrm{z}}\mathrm{e}\mathrm{z}$ drugiej urny wylosowano białą kulę.

4. $\mathrm{W}$ urnie jest 15 ku1 ponumerowanych 1iczbami od 1 do 15. Wyciągamy $\mathrm{z}$ niej kolejno pięć

kul bez zwracania. Obliczyć prawdopodobieństwo, $\dot{\mathrm{z}}\mathrm{e}$ numer na drugiej kuli jest liczbą

podzielną przez trzy ijednocześnie numer na piątej kuli jest liczbą podzielną przez pięć.

5. Znajdz' dziedzinę oraz wartości największą $\mathrm{i}$ najmniejszą (ješli istnieją) funkcji

$f(x)=\displaystyle \frac{2-x^{2}}{x^{2}}+(2-x^{2})+(2x^{2}-x^{4})+$

która jest sumą szeregu geometrycznego.

6. $\mathrm{W}$ urniejest 99 ku1 białych ijedna czarna. Agnieszka $\mathrm{i}$ Jacek losują $\mathrm{z}$ tej urny na przemian

po jednej kuli bez zwracania. Wygrywa ten, kto wylosuje czarną kulę. Pierwszą kulę

wyciqga Agnieszka. Jakie jest prawdopodobieństwo, $\dot{\mathrm{z}}\mathrm{e}$ to ona wygra?

Rozwiązania (rękopis) zadań z wybranego poziomu prosimy nadsylać do 20.02.2023r.

adres:

na

Wydzial Matematyki

Politechnika Wroclawska

Wybrzez $\mathrm{e}$ Wyspiańskiego 27

$50-370$ WROCLAW,

$\mathrm{l}\mathrm{u}\mathrm{b}$ elektronicznie, za pośrednictwem portalu talent. $\mathrm{p}\mathrm{w}\mathrm{r}$. edu. pl

Na kopercie prosimy $\underline{\mathrm{k}\mathrm{o}\mathrm{n}\mathrm{i}\mathrm{e}\mathrm{c}\mathrm{z}\mathrm{n}\mathrm{i}\mathrm{e}}$ zaznaczyć wybrany poziom! (rip. poziom podsta-

wowy lub rozszerzony). Do rozwiązań nalez $\mathrm{y}$ dołączyć zaadresowaną do siebie kopertę

zwrotną $\mathrm{z}$ naklejonym znaczkiem, odpowiednim do formatu listu. Prace niespełniające

podanych warunków nie będą poprawiane ani odsyłane.

Uwaga. Wysyłając nam rozwiązania zadań uczestnik Kursu udostępnia Politechnice Wrocławskiej

swoje dane osobowe, które przetwarzamy wyłącznie $\mathrm{w}$ zakresie niezbędnym do jego prowadzenia

(odeslanie zadań, prowadzenie statystyki). Szczególowe informacje $0$ przetwarzaniu przez nas danych

osobowych są dostępne na stronie internetowej Kursu.

Adres internetowy Kursu: http: //www. im. pwr. edu. pl/kurs







XLIV

KORESPONDENCYJNY KURS

Z MATEMATYKI

marzec 2015 r.

PRACA KONTROLNA nr 7 -POZIOM PODSTAWOWY

l. Wspótczynniki $a, b$ trójmianu kwadratowego $x^{2} -2ax+b$ oraz pierwiastki tego

trójmianu, napisane $\mathrm{w}$ odpowiedniej kolejności, sa czterema poczatkowymi wyra-

zami pewnego ciagu arytmetycznego. Dla $a=2$ obliczyč róznice ciagu, wspólczynnik

$b$ oraz pierwiastki trójmianu.

2. Kwadrat $0$ boku $a$ zgieto wzdluz jednej $\mathrm{z}$ przekatnych $\mathrm{t}\mathrm{a}\mathrm{k}$, aby odleglośč pozostalych

wierzcholków byla równa potowie dlugości przekatnej kwadratu. $\mathrm{W}$ tak powstaly

czworościan wpisano dwie identyczne, wzajemnie styczne kule. Obliczyč promień

tych $\mathrm{k}\mathrm{u}\mathrm{l}.$

3. Trzy czerwone, trzy zóite $\mathrm{i}$ jedna zieloną kredke wlozono $\mathrm{w}$ przypadkowy sposób

do pudelka. Obliczyč prawdopodobieństwo tego, $\dot{\mathrm{z}}\mathrm{e}\dot{\mathrm{z}}$ adne dwie kredki tego samego

koloru nie beda $\mathrm{l}\mathrm{e}\dot{\mathrm{z}}$ aly obok siebie.

4. Wyznaczyč dziedzine funkcji $f(x)=\sqrt{\frac{\log_{2}x}{1-\log_{2}x}}$. Uzasadnič, $\dot{\mathrm{z}}\mathrm{e}f(x)$ jest rosnaca.

Korzystajac $\mathrm{z}$ tego faktu, określič zbiór wartości funkcji $f(x).$

5. $\mathrm{W}$ ostrostup prawidIowy czworokatny wpisano prostopadlościan prosty $0$ podstawie

kwadratowej $\mathrm{w}$ ten sposób, $\dot{\mathrm{z}}\mathrm{e}$ wierzcholki jego górnej podstawy $\mathrm{l}\mathrm{e}\dot{\mathrm{z}}$ a $\mathrm{w}$ środkach

$\mathrm{c}\mathrm{i}\mathrm{e}\dot{\mathrm{z}}$ kości ścian bocznych ostroslupa. Pole powierzchni calkowitej prostopadlościanu

stanowi trzecia cześč pola powierzchni calkowitej ostrostupa. Obliczyč tangens kata

nachylenia krawędzi bocznej ostroslupa do podstawy.

6. Rozwiazač uklad równań

$\left\{\begin{array}{l}
x^{2}+y^{2}=2\\
- x1+-y1=2
\end{array}\right.$

Podač interpretacje geometryczna tego ukladu i sporzadzič rysunek.




PRACA KONTROLNA nr 7 -POZIOM ROZSZERZONY

1. Na $\mathrm{k}\mathrm{a}\dot{\mathrm{z}}$ dym $\mathrm{z}$ trzech drutów linii elektrycznej wysokiego napiecia siedzi po pieč

wróbli. $\mathrm{W}$ pewnej chwili odfrunelo przypadkowych sześč wróbli. Obliczyč prawdo-

podobieństwo tego, $\dot{\mathrm{z}}\mathrm{e}$ na co najmniej dwóch drutach pozostaIa taka sama liczba

ptaków.

2. Dolna cześč namiotu ma ksztaIt walca $0$ wysokości $h=2\mathrm{m}$, a górna jest stozkiem $0$

tworzacej $l=\sqrt{15}\mathrm{m}\mathrm{i}$ tym samym promieniu, co cześč dolna. Wyznaczyč pozostale

parametry namiotu $\mathrm{t}\mathrm{a}\mathrm{k}$, aby jego objetośč byla najwieksza. Sporzadzič rysunek.

3. $\mathrm{Z}$ pudeIka zawierającego 10 k1ocków ponumerowanych cyframi od 0 do 9 wy1osowano

dwa klocki $\mathrm{i}$ ustawiono obok siebie $\mathrm{w}$ przypadkowej kolejności, tworzac $\mathrm{w}$ ten sposób

liczbe $k$ (ustawienie 03 rozumiemy jako 1iczbe 3). Nastepnie wy1osowano trzeci

klocek $\mathrm{z}$ pozostaIych $\mathrm{i}$ ustawiono go za tamtymi, gdy suma cyfr liczby $k$ byla mniejsza

$\mathrm{n}\mathrm{i}\dot{\mathrm{z}}10$, lub przed tamtymi, $\mathrm{w}$ przeciwnym wypadku. Obliczyč prawdopodobieństwo

tego, $\dot{\mathrm{z}}\mathrm{e}$ otrzymana liczba jest wieksza od 500.

Wsk. $\mathrm{U}\dot{\mathrm{z}}$ yč wzoru na prawdopodobieństwo catkowite.

4. Stosujac zasade indukcji matematycznej, udowodnič $\mathrm{t}\mathrm{o}\dot{\mathrm{z}}$ samośč

$\sin^{2}\alpha+\sin^{2}3\alpha+ +\displaystyle \sin^{2}(2n-1)\alpha=\frac{n}{2}-\frac{\sin 4n\alpha}{4\sin 2\alpha},$

$n\geq 1,$

gdzie $\displaystyle \alpha\neq k\frac{\pi}{2}, k$ caIkowite.

5. Znalez/č równanie stycznej $l$ do wykresu funkcji $f(x)=\displaystyle \frac{1}{x}+x^{2}\mathrm{w}$ punkcie, $\mathrm{w}$ którym

przecina on oś $Ox$. Wyznaczyč wszystkie styczne, które sa równolegle do prostej

$l$. Znalez/č punkty wspólne tych stycznych $\mathrm{z}$ wykresem funkcji. Rozwiązanie zilu-

strowač odpowiednim rysunkiem.

6. {\it K}rawed $\acute{\mathrm{z}}$ podstawy graniastoslupa trójkatnego prawidIowego ma dIugośč $a$. Oznaczmy

przez $ 2\alpha \mathrm{k}\mathrm{a}\mathrm{t}$ miedzy przekatnymi ścian bocznych wychodzacymi $\mathrm{z}$ jednego wierz-

choIka. Graniastoslup $\mathrm{p}\mathrm{r}\mathrm{z}\mathrm{e}\mathrm{c}\mathrm{i}_{\xi}!\mathrm{t}\mathrm{o}$ na dwie cześci pIaszczyzna przechodzaca przez

krawed $\acute{\mathrm{z}}$ dolnej podstawy $\mathrm{i}$ przeciwlegly wierzchoiek górnej podstawy. Obliczyč tan-

gens kata $\alpha$, dla którego $\mathrm{w}$ wieksza cześč graniastoslupa $\mathrm{m}\mathrm{o}\dot{\mathrm{z}}$ na wpisač kule. Dla

znalezionego kata $\alpha$, obliczyč promień kuli wpisanej $\mathrm{w}$ mniejsza cześč graniastosIupa.







XLV

KORESPONDENCYJNY KURS

Z MATEMATYKI

marzec 2016 r.

PRACA KONTROLNA nr 7 -POZIOM PODSTAWOWY

l. Cztery cyfry 0 $\mathrm{i}$ pięč cyfr l ustawiono $\mathrm{w}$ przypadkowej kolejności. Obliczyč praw-

dopodobieństwo tego, $\dot{\mathrm{z}}\mathrm{e}$ na obu końcach powstafego ciągu znalazfy się jednakowe

cyfry.

2. Drugi wyraz pewnego $\mathrm{c}\mathrm{i}_{\Phi \mathrm{g}}\mathrm{u}$ geometrycznego wynosi 8, a ósmy 2. Ob1iczyč siedemna-

sty wyraz tego ciągu oraz sumę pietnastu wyrazów, poczynając od wyrazu trzeciego.

Wynik zapisač $\mathrm{w}$ najprostszej postaci.

3. Rozwiązač nierównośč

$\sqrt{2^{x-2}-2}\leq 2^{x-1}-5.$

4. Dana jest funkcja $f(x)=\displaystyle \frac{\sqrt{2-x-x^{2}}}{\sqrt{1-x^{2}}}$. Znalez/č wszystkie wartości parametru rze-

czywistego $a$, dla których równanie $f(x)=2^{a}$ posiada rozwiązanie. Sporządzič wy-

kres funkcji $f(x).$

5. Romb $0$ boku $a\mathrm{i}$ kącie ostrym $\alpha$ zgięto wzdluz prostej $l_{\Phi}$czącej środki przeciwlegfych

boków, tak aby obie części rombu byly wzajemnie prostopadle. Obliczyč odległośč

wierzchołków katów ostrych oraz cosinus kąta pomiędzy polowami krótszej przekąt-

nej $\mathrm{w}$ zgiętym rombie.

6. Dlugości boków trapezu opisanego na okręgu są liczbami naturalnymi $\mathrm{i}$ są kolejny-

mi wyrazami ciągu arytmetycznego. Obwód trapezu wynosi 24. Ob1iczyč po1e oraz

dłuzszą przekątna trapezu.




PRACA KONTROLNA nr 7 -POZIOM ROZSZERZONY

l. Spośród 12 pączków, $1\mathrm{e}\mathrm{z}\Phi^{\mathrm{C}}\mathrm{y}\mathrm{c}\mathrm{h}$ na pófmisku, 6 byfo nadziewanych, 61ukrowanych,

a 4 nie miały nadzienia ani nie były 1ukrowane. Franek zjad1 dwa 1osowo wybrane

pączki. Obliczyč prawdopodobieństwo, $\dot{\mathrm{z}}\mathrm{e}$ jadf zarówno pqczka lukrowanego jak $\mathrm{i}$

pączka $\mathrm{z}$ nadzieniem.

2. Na krzywej $0$ równaniu $y = \sqrt{4-x}, x \geq 0$, znalez$\acute{}$č punkt $P$, tak aby odcinek

$\iota_{\Phi^{\mathrm{c}\mathrm{z}\text{ą} \mathrm{c}\mathrm{y}P\mathrm{z}\mathrm{p}^{\mathrm{O}\mathrm{C}\mathrm{Z}}\Phi^{\mathrm{t}\mathrm{k}\mathrm{i}\mathrm{e}\mathrm{m}}}}$ ukfadu wspófrzędnych, przy obrocie wokóf osi $Ox$, zakreślił

powierzchnię $0$ największym polu. Sporządzič rysunek.

3. Wyznaczyč punkty przecięcia się wykresu funkcji $f(x) = \displaystyle \frac{3x-7}{2x-2} \mathrm{z}$ wykresem jej

pochodnej $f'(x). K\mathrm{o}\mathrm{r}\mathrm{z}\mathrm{y}\mathrm{s}\mathrm{t}\mathrm{a}\mathrm{j}_{\Phi}\mathrm{c}$ ze wzoru tg $(\displaystyle \alpha-\beta)=\frac{\mathrm{t}\mathrm{g}\alpha-\mathrm{t}\mathrm{g}\beta}{1+\mathrm{t}\mathrm{g}\alpha \mathrm{t}\mathrm{g}\beta}$, obliczyč tangensy

katów, pod którymi przecinają się te wykresy. Rozwiazanie zilustrowač odpowiednim

rysunkiem.

4. Stosujqc zasadę indukcji matematycznej, udowodnič nierównośč

$2\displaystyle \sqrt{n}-\frac{3}{2}<1+\frac{1}{\sqrt{2}}+\frac{1}{\sqrt{3}}+ +\displaystyle \frac{1}{\sqrt{n}}\leq 2\sqrt{n}-1,$

$n\geq 1.$

Dla jakich $n$ nierównośč ta pozwala na oszacowanie występującej $\mathrm{w}$ niej sumy $\mathrm{z}$

blędem względnym mniejszym $\mathrm{n}\mathrm{i}\dot{\mathrm{z}}$ 1\%.

5. $\mathrm{Z}$ punktu $P$ widač okrąg $0$ środku $O \mathrm{i}$ promieniu $r$ pod kqtem $ 2\alpha$. Prosta $PO$

przecina okrąg $\mathrm{w}$ punktach A $\mathrm{i}C$, a styczne do okręgu, poprowadzone $\mathrm{z}$ punktu $P,$

przechodzą przez punkty $B\mathrm{i}D$ na okręgu. Obliczyč promień okręgu wpisanego $\mathrm{w}$

czworokąt ABCD oraz odległośč środków obu okręgów.

6. Podstawą ostrosfupajest romb $0$ boku 5. Spodek wysokości ostrosfupa $\mathrm{l}\mathrm{e}\dot{\mathrm{z}}\mathrm{y}\mathrm{w}$ środku

podstawy, a krawędzie boczne mają długości 6 $\mathrm{i}7$. Obliczyč objętośč ostroslupa oraz

cosinus kąta nachylenia ściany bocznej do podstawy.







XLVI

KORESPONDENCYJNY KURS

Z MATEMATYKI

marzec 2017 r.

PRACA KONTROLNA nr 7- POZIOM PODSTAWOWY

l. Pierwszym wyrazem ciagu arytmetycznego jest $a_{1}=2017$, a jego róznica jest rozwiąza-

niem równania $\sqrt{2-x}-x = 10$. Obliczyč sume wszystkich dodatnich wyrazów tego

ciagu.

2. Spośród dwucyfrowych liczb nieparzystych mniejszych od 50 wylosowano bez zwracania

dwie. Obliczyč prawdopodobieństwo tego, $\dot{\mathrm{z}}\mathrm{e}$ obie wylosowane liczby sa pierwsze oraz

prawdopodobieństwo tego, $\dot{\mathrm{z}}\mathrm{e}$ iloczyn wylosowanych liczb nie jest podzielny przez 15.

3. Uzasadnič, $\dot{\mathrm{z}}\mathrm{e}$ ciag $0$ wyrazach $a_{n}=\displaystyle \frac{2^{n}+2^{n+1}+..+2^{2n}}{2^{2}+2^{4}+\ldots+2^{2n}}, n\geq 1$, nie jest rosnacy oraz,

$\dot{\mathrm{z}}\mathrm{e}$ jest rosnacy, poczynajac od $n=2.$

4. Znalez/č wszystkie wartości parametru rzeczywistego $m$, dla których proste $0$ równaniach

$x-my+2m=0, 2mx+4y+1=0, mx-y-3m-1=0$ sa parami rózne $\mathrm{i}$ przecinaja

$\mathrm{s}\mathrm{i}\mathrm{e}\mathrm{w}$ jednym punkcie. Sporzadzič odpowiedni rysunek dla najmniejszej ze znalezionych

wartości tego parametru.

5. $\mathrm{W}$ ostrostupie prawidtowym czworokatnym dana jest odleglośč $d$ środka podstawy od

krawedzi bocznej oraz $\mathrm{k}\mathrm{a}\mathrm{t}2\alpha$ miedzy sasiednimi ścianami bocznymi. Obliczyč objetośč

ostroslupa.

6. Podstawa $AB$ trójkata równoramiennego $ABC$ jest krótsza od ramion. Wysokości $AD$

$\mathrm{i}$ CE dziela trojkat na cztery cześci, $\mathrm{z}$ których dwie sa trójkatami prostokatnymi $0$ polach

równych 9 oraz 2. Ob1iczyč po1a pozostatych cz$\xi$!ści oraz obwód trójkata.




XLVI

KORESPONDENCYJNY KURS

Z MATEMATYKI

marzec 2017 r.

PRACA KONTROLNA $\mathrm{n}\mathrm{r} 7$- POZIOM ROZSZERZONY

l. Turysta zabtądzil $\mathrm{w}$ lesie zajmujacym obszar ($\mathrm{w}$ km)

$D=\{(x,y):x^{2}+y^{2}\leq 2y+3,-2y\leq x\leq y\}.$

Wskazač mu najkrótsza droge wyjścia $\mathrm{z}$ lasu, jeśli znaduje $\mathrm{s}\mathrm{i}\mathrm{e}\mathrm{w}$ punkcie $P(-\displaystyle \frac{1}{4},\frac{3}{2})$. Ile

minut bedzie trwala wedrówka, jeśli idzie $\mathrm{z}$ predkościa 4 $\mathrm{k}\mathrm{m}/\mathrm{h}$?

2. Korzystajac $\mathrm{z}$ zasady indukcji matematycznej, udowodnič prawdziwośč nierówności

$1^{5}+2^{5}++n^{5}<\displaystyle \frac{n^{3}(n+1)^{3}}{6},n\geq 1.$

3. Kubuś zaobserwowal, $\dot{\mathrm{z}}\mathrm{e}\mathrm{w}$ pewnej chwili $\mathrm{w}$ trzypietrowej kamienicy po drugiej stronie

ulicy pali $\mathrm{s}\mathrm{i}\mathrm{e}$ światto $\mathrm{w}10$ oknach. Na $\mathrm{k}\mathrm{a}\dot{\mathrm{z}}$ dej kondygnacji kamienicy znajduja $\mathrm{s}\mathrm{i}\mathrm{e}4$ okna.

Zakladamy, $\dot{\mathrm{z}}\mathrm{e}$ okna zapalaja $\mathrm{s}\mathrm{i}\mathrm{e}\mathrm{i}$ gasna losowo. Obliczyč prawdopodobieństwo tego, $\dot{\mathrm{z}}\mathrm{e}$

zarówno na drugim jak $\mathrm{i}$ na trzecim pietrze kamienicy świeca $\mathrm{s}\mathrm{i}\mathrm{e}$ co najmniej dwa okna.

Wsk. Skorzystač ze wzoru $P(A\cup B)=P(A)+P(B)-P(A\cap B).$

4. Podstawa graniastosIupa prostego $0$ wysokości $h=2$jest trójkat, $\mathrm{w}$ którym tangens kata

przy wierzchoIku $A$ wynosi $-\sqrt{2}$. Przekatne $e, f$ sasiednich ścian bocznych, wychodzace

$\mathrm{z}$ wierzcholka $A$, sa do siebie prostopadle, a liczby $h, e, f$ sa kolejnymi wyrazami pewnego

ciągu geometrycznego. Obliczyč objetośč graniastosIupa.

5. Znalez/č dziedzine $\mathrm{i}$ zbiór wartości funkcji

$f(x)=\sqrt{\log_{2}\frac{1}{\cos x+\sqrt{3}\sin x}}.$

6. $\mathrm{K}\mathrm{a}\mathrm{t}$ ptaski przy wierzcholku $D$ ostroslupa prawidlowego trójkatnego $0$ podstawie $ABC$

jest równy $\alpha$. Na krawedzi $BD$ wybrano punkt $E \mathrm{t}\mathrm{a}\mathrm{k}, \dot{\mathrm{z}}\mathrm{e} \triangle ACE$ jest trójkatem

równobocznym. Znalez/č stosunek $k(\alpha)$ objetości ostroslupa ABCE do objetości ostro-

{\it sIupa ACED} $\mathrm{w}$ zalezności od kata $\alpha$. Sporzadzič wykres funkcji $k(\alpha).$

Rozwiazania (rekopis) zadań $\mathrm{z}$ wybranego poziomu prosimy nadsyIač do 18 marca 2017 $\mathrm{r}$. na

adres:

WydziaI Matematyki

Politechniki Wroclawskiej,

$\mathrm{u}1$. Wybrzez $\mathrm{e}$ Wyspiańskiego 27,

50-370 WROCLAW.
\begin{center}
\begin{tabular}{|l|l|l|}
\hline
\multicolumn{1}{|l|}{Na kopercie prosimy $\underline{\mathrm{k}\mathrm{o}\mathrm{n}\mathrm{i}\mathrm{e}\mathrm{c}\mathrm{z}\mathrm{n}\mathrm{i}\mathrm{e}}$ zaznaczyč wybrany poziom!}&	\multicolumn{1}{|l|}{(np.}&	\multicolumn{1}{|l|}{poziom podstawowy lub}	\\
\hline
\end{tabular}

\end{center}
Na kopercie prosimy$\underline{\mathrm{k}\mathrm{o}\mathrm{n}\mathrm{i}\mathrm{e}\mathrm{c}\mathrm{z}\mathrm{n}\mathrm{i}\mathrm{e}}$ zaznaczyč wybrany poziom! (np. poziom podstawowy lub

jonym znaczkiem, odpowiednim do wagi listu (od 1.01.2017nowe ceny znaczków!). Prace nie

spelniajace podanych warunków nie beda poprawiane ani odsylane.

ï{\it r}







XLVI

KORESPONDENCYJNY KURS

Z MATEMATYKI

marzec 2018 r.

PRACA KONTROLNA nr 7- POZIOM PODSTAWOWY

l. Liczba l jest pierwiastkiem wielomianu trzeciego stopnia $w(x)$ oraz wielomianu $w(x+1).$

Środkiem symetrii wykresu $w(x)$ jest punkt $S(0,2)$. Narysowač staranny wykres funkcji

$f(x)= |w(x-1)|. (\acute{\mathrm{S}}$rodkiem symetrii krzywej $0$ równaniu $y(x) =ax^{3}+bx^{2}+cx+d$

jest punkt $S(\displaystyle \frac{-b}{3a},y(\frac{-b}{3a})).$)

2. Sala jest oświetlona 5 $\dot{\mathrm{z}}$ arówkami. Wkrecono losowo $\dot{\mathrm{z}}$ arówki z$\cdot$óIte, czerwone, zielone $\mathrm{i}$

niebieskie. Obliczyč prawdopodobieństwo, $\dot{\mathrm{z}}\mathrm{e}$ wkrecono co najmniej dwie $\dot{\mathrm{z}}$ arówki z$\cdot$ólte

$\mathrm{i}$ co najmniej dwie czerwone.

3. Rozwiazač równanie

$\displaystyle \frac{\cos 5x}{\cos 3x}+1=0.$

4. Wazon $\mathrm{w}$ ksztatcie walca, którego wysokośč jest wieksza od średnicy podstawy, ma

objetośč 1200 $\mathrm{c}\mathrm{m}^{3}$ Napelniony woda wazon przechylono $\mathrm{t}\mathrm{a}\mathrm{k}, \dot{\mathrm{z}}\mathrm{e}$ jego oś symetrii utwo-

rzyla $\mathrm{z}$ pionem $\mathrm{k}\mathrm{a}\mathrm{t}45^{o}$ Wylalo $\mathrm{s}\mathrm{i}\mathrm{e}200\mathrm{c}\mathrm{m}^{3}$ wody. Podač wymiary wazonu (pominač

grubośč ścianek).

5. Podstawa $AB$ trapezu równoramiennego jest średnicą okregu opisanego na nim. Za

pomoca rachunku wektorowego wyznaczyč wspólrzedne wierzchotków $B\mathrm{i}C$, wiedzac,

$\dot{\mathrm{z}}\mathrm{e}|AB|=5, A(1,1), D(3,2)$ oraz $\dot{\mathrm{z}}\mathrm{e}B\mathrm{l}\mathrm{e}\dot{\mathrm{z}}\mathrm{y}\mathrm{w}$ dolnej pólplaszczyz/nie.

6. Krzywa spiralna jest utworzona $\mathrm{z}$ čwiartek okregów, których promienie tworza ciag geo-

{\it O} $p_{0} Oy \mathrm{t}\mathrm{a}\mathrm{k}, \dot{\mathrm{z}}\mathrm{e}$ uki obu okr gów acza si $\mathrm{w}$ punk-

$P_{2} O_{3}$ cie $P_{1}$ (rysunek). Środki kolejnych okregów sa

$P_{3}$ Nast pnie wykonac obliczenia dla $q=\displaystyle \frac{3}{2}.$
\begin{center}
\includegraphics[width=68.940mm,height=69.852mm]{./KursMatematyki_PolitechnikaWroclawska_7_2018_page0_images/image001.eps}
\end{center}
$P_{1}$

$O_{2}$

metryczny $0$ ilorazie $q >$ l. Środek pierw-

szego okr gu znajduje si $\mathrm{w}$ poczatku uk adu

wspo rz dnych, a punkt $P_{0}(2,0)$ jest poczatkiem

krzywej. Srodek $O_{2}$ drugiego okr gu $\mathrm{l}\mathrm{e}\dot{\mathrm{z}}\mathrm{y}$ na osi

tak po $0\dot{\mathrm{z}}$ one, $\dot{\mathrm{z}}\mathrm{e}$ utworzona krzywa jest $\mathrm{g}$ adka $\mathrm{i}$

promien uku mniejszego okregu jest cześcia pro-

mienia uku wiekszego okr gu (rysunek). Znale $\acute{\mathrm{z}}\mathrm{c}$

wspo rz dne srodka $O_{6}$ oraz $\mathrm{d}$ ugośc uku spi-

rali $P_{0}P_{6}$. Wynik podač $\mathrm{w}$ najprostszej postaci.




XLVI

KORESPONDENCYJNY KURS

Z MATEMATYKI

marzec 2018 r.

PRACA KONTROLNA nr 7- POZIOM ROZSZERZONY

l. Stosujac zasade indukcji matematycznej, wykazač, $\dot{\mathrm{z}}\mathrm{e}$ dla wszystkich $n$

$10^{n}+18n-1$ jest podzielna przez 27.

$\geq 1$ liczba

2. Sprawdzič $\mathrm{t}\mathrm{o}\dot{\mathrm{z}}$ samośč

$\displaystyle \frac{\cos^{2}\alpha-\cos^{2}\beta}{\sin^{2}\alpha-\cos^{2}\beta}=\mathrm{t}\mathrm{g}(\alpha+\beta)\mathrm{t}\mathrm{g}(\alpha-\beta)$

$\mathrm{i}$ określič jej dziedzine.

3. Dwóch strzelców oddato $\mathrm{k}\mathrm{a}\dot{\mathrm{z}}\mathrm{d}\mathrm{y}$ po dwa strzaly $\mathrm{i}$ okazato $\mathrm{s}\mathrm{i}\mathrm{e}, \dot{\mathrm{z}}\mathrm{e}$ cel zostal trafiony dokIadnie

dwa razy. Obliczyč prawdopodobieństwo, $\dot{\mathrm{z}}\mathrm{e}$ dwukrotnie trafil pierwszy strzelec, jeśli

za $\mathrm{k}\mathrm{a}\dot{\mathrm{z}}$ dym razem pierwszy trafia $\mathrm{z}$ prawdopodobieństwem $\displaystyle \frac{4}{5}$, a drugi $\mathrm{z}$ prawdopodo-

bieństwem $\displaystyle \frac{3}{5}.$

4. Znalez/č wartośč parametru nieujemnego $p$, dla którego suma kwadratów odwrotności

pierwiastków równania

$x^{2}+(p+1)x-(p+3)=0$

jest najmniejsza.

5. Rozwiazač uktad równań

$\left\{\begin{array}{l}
x^{2}y^{2}=4\\
y^{4}-6y^{2}-x^{2}+9=0
\end{array}\right.$

Podač interpretacje geometryczna tego ukladu $\mathrm{i}$ obliczyč pole wielokata utworzonego

przez jego rozwiazania (interpretowane jako punkty na plaszczy $\acute{\mathrm{z}}\mathrm{n}\mathrm{i}\mathrm{e}$). Sporzadzič rysu-

nek.

6. Podstawa ostrosIupa ABCD jest trójkat równoramienny $0$ kacie przy wierzchoIku $2\alpha.$

Plaszczyzna przechodząca przez wierzcholek $D$ ostroslupa $\mathrm{i}$ wysokośč podstawy jest

pIaszczyzna symetrii ostrosIupa, a przekrój bryIy ta pIaszczyzna jest trójkatem równo-

bocznym $0$ boku $a$. Wykazač, $\dot{\mathrm{z}}\mathrm{e}$ ostrosIup ma jeszcze jedna plaszczyzne symetrii $\mathrm{i}$

obliczyč promień kuli opisanej na nim.

Rozwiązania (rekopis) zadań z wybranego poziomu prosimy nadsylač do 18 marca 2018 r. na

adres:

Wydziaf Matematyki

Politechniki Wrocfawskiej,

ul. Wybrzeze Wyspiańskiego 27,

50-370 WROCLAW.
\begin{center}
\begin{tabular}{|l|l|l|}
\hline
\multicolumn{1}{|l|}{Na kopercie prosimy $\underline{\mathrm{k}\mathrm{o}\mathrm{n}\mathrm{i}\mathrm{e}\mathrm{c}\mathrm{z}\mathrm{n}\mathrm{i}\mathrm{e}}$ zaznaczyč wybrany poziom!}&	\multicolumn{1}{|l|}{(np.}&	\multicolumn{1}{|l|}{poziom podstawowy lub}	\\
\hline
	\\
	\\
\multicolumn{1}{|l|}{beda poprawiane ani odsytane.}	\\
\cline{1-1}
\end{tabular}

\end{center}
Adres Internetowy Kursu: http://www.im.pwr.wroc.pl/kurs







XLVIII

KORESPONDENCYJNY KURS

Z MATEMATYKI

marzec 2019 r.

PRACA KONTROLNA nr 7- POZIOM PODSTAWOWY

1. $\mathrm{W}$ pierwszej godzinie rowerzysta A jedzie $\mathrm{z}$ prędkościq 25 $\mathrm{k}\mathrm{m}/\mathrm{h}$, a $\mathrm{w}\mathrm{k}\mathrm{a}\dot{\mathrm{z}}$ dej kolejnej

godziniejedzie ze stafą prędkości$\Phi$ mniejszą $0$ 20\% $\mathrm{w}$ stosunku do prędkości $\mathrm{w}$ poprzedniej

godzinie. Natomiast rowerzysta $\mathrm{B}$ jedzie $\mathrm{w}$ pierwszej godzinie $\mathrm{z}$ prędkością 8 $\mathrm{k}\mathrm{m}/\mathrm{h},$

a $\mathrm{w}\mathrm{k}\mathrm{a}\dot{\mathrm{z}}$ dej kolejnej godzinie jedzie ze stałq prędkością większq $0$ 20\% $\mathrm{w}$ stosunku do

prędkości $\mathrm{w}$ poprzedniej godzinie. Obaj startują równocześnie $\mathrm{z}$ tego samego punktu.

Który $\mathrm{z}$ nich dotrze prędzej do celu lezącego $\mathrm{w}$ odleglości 100 km od punktu startu?

Po której godzinie jazdy odlegfośč między nimi $\mathrm{w}$ zaokrągleniu do pełnych kilometrów

będzie największa $\mathrm{i}$ ile będzie wynosič? Odpowiedzi uzasadnič bez stosowania obliczeń

przyblizonych.

2. $\mathrm{W}$ skarbonce jest 5 monet 5 zf $\mathrm{i}5$ monet 2 $\mathrm{z}\mathrm{f}$. Kuba wylosowaf ze skarbonki 6 monet.

Obliczyč prawdopodobieństwo tego, $\dot{\mathrm{z}}\mathrm{e}$ wystarczy mu pieniędzy na kupno ksiązki $\mathrm{w}$ cenie

20 $\mathrm{z}1.$

3. Rozwiązač nierównośč

$2\log_{2}(3-\sqrt{2^{x+1}-7})>x.$

4. Dla jakich wartości parametru $m$ liczby $x_{0}, y_{0}$, spełniające uklad równań

$\left\{\begin{array}{l}
x\\
3x
\end{array}\right.$

$+$

$+$

{\it my}

2{\it y}

$=2$

$=m$

są odpowiednio cosinusem $\mathrm{i}$ sinusem tego samego kąta $\alpha \in [0,\pi]$. Podač $x_{0} \mathrm{i} y_{0}$ dla

znalezionych wartości parametru $m.$

5. $\mathrm{W}$ ostrosfupie prawidfowym trójkątnym kąt pomiędzy ścianami bocznymi wynosi $2\alpha.$

Niech $P$ będzie spodkiem wysokości ściany bocznej opuszczonej na krawęd $\acute{\mathrm{z}}$ boczną.

Pfaszczyzna równolegfa do podstawy przechodząca przez $P$ dzieli ostrosfup na dwie czę-

ści, $\mathrm{z}$ których górna ma objetośč $V$. Obliczyč objętośč oraz krawędz/ podstawy ostrosłupa.

Podač dziedzinę kąta $\alpha.$

6. Kąty przy podstawie $AB$ trójkąta sq równe $\alpha$ oraz $2\alpha, \displaystyle \alpha<\frac{\pi}{4}$, a środkowa boku $AB$ ma

dlugośč $d$. Znalez/č dlugości boków trójk$\Phi$ta. Następnie podstawič do wyniku ogólnego

dane $d=\sqrt{11}$ oraz $\displaystyle \sin\alpha=\frac{\sqrt{2}}{4}\mathrm{i}$ wykonač obliczenia.




XLVIII

KORESPONDENCYJNY KURS

Z MATEMATYKI

marzec 2019 r.

PRACA KONTROLNA $\mathrm{n}\mathrm{r} 7$- POZIOM ROZSZERZONY

l. Rozwiqzač nierównośč

$\sqrt{\sin 2x-\cos 2x+1}\leq 2\sin x.$

2. Ze zbioru $\{$1, 2, $3n\}, n\geq 1$, wylosowano bez zwracania dwie liczby. Obliczyč prawdo-

podobieństwo tego, $\dot{\mathrm{z}}\mathrm{e}$ suma otrzymanych liczb jest mniejsza od $4n\mathrm{i}$ co najmniej jedna

$\mathrm{z}$ nich jest większa od $n.$

3. Stosując zasadę indukcji matematycznej, udowodnič prawdziwośč wzoru

$1^{4}+2^{4}++n^{4}+\displaystyle \frac{1^{2}+2^{2}++n^{2}}{5}=\frac{n^{2}(n+1)^{2}(2n+1)}{10},$

$n\geq 1.$

4. Dana jest funkcja $f(x)=\displaystyle \frac{1}{3}x^{3}-\frac{4}{3}x$. Styczna do wykresu tej funkcji $\mathrm{w}$ punkcie $A(1,-1)$

przecina wykres $\mathrm{w}$ punkcie $B(x_{1},f(x_{1}))$, a styczna do jej wykresu $\mathrm{w}$ punkcie $B$ przecina

wykres $\mathrm{w}$ punkcie $C(x_{2},f(x_{2}))$. Znalez/č punkty $B \mathrm{i} C$ oraz obliczyč tangensy katów

trójkąta $\triangle ABC$. Sporządzič rysunek, dobierając odpowiednie skale na obu osiach.

5. $\mathrm{W}$ czworokącie ABCD $0$ bokach $|AB|=a, |AD|=2a$ mamy $\displaystyle \vec{AC}=2\vec{AB}+\frac{1}{2}\vec{AD}$ oraz

$\displaystyle \cos\angle BCD=\frac{1}{4}$. Wykazač, $\dot{\mathrm{z}}\mathrm{e}$ na tym czworokącie $\mathrm{m}\mathrm{o}\dot{\mathrm{z}}$ na opisač okrąg. Obliczyč promień

tego okregu. Sporz$\Phi$dzič rysunek.

6. Podstawą ostroslupa jest trójkąt równoramienny $0$ kącie przy wierzchofku $2\alpha, \alpha<\pi/4,$

$\mathrm{i}$ podstawie $2a$. Dwie ściany boczne są przystajqcymi do siebie trójkątami podobny-

mi, ale nie przystającymi, do podstawy ostroslupa. Znalez/č cosinus kąta pfaskiego przy

wierzchołku trzeciej ściany bocznej oraz objętośč ostroslupa. Narysowač starannie siatkę

tego ostrosłupa dla $\displaystyle \alpha=\frac{\pi}{5}.$

Rozwiązania (rękopis) zadań $\mathrm{z}$ wybranego poziomu prosimy nadsyłaČ do 18 marca 2019 $\mathrm{r}.$

na adres:

Wydziaf Matematyki

Politechniki Wrocfawskiej,

Wybrzez $\mathrm{e}$ Wyspiańskiego 27,

$50-370$ WROCLAW.

Na kopercie prosimy $\underline{\mathrm{k}\mathrm{o}\mathrm{n}\mathrm{i}\mathrm{e}\mathrm{c}\mathrm{z}\mathrm{n}\mathrm{i}\mathrm{e}}$ zaznaczyč wybrany poziom! (np. poziom pod-

stawowy lub rozszerzony). Do rozwiazań nalez $\mathrm{y}$ dolączyč zaadresowaną do siebie

kopertę zwrotną $\mathrm{z}$ naklejonym znaczkiem, odpowiednim do wagi listu. Prace nie

spelniające podanych warunków nie będą poprawiane ani odsyłane.

Uwaga. Wysyłaj\S c nam rozwi\S zania zadań uczestnik Kursu udostępnia nam swoje dane osobowe,

które przetwarzamy wyłącznie $\mathrm{w}$ zakresie niezbednym do jego prowadzenia (odeslanie pracy, prowa-

dzenie statystyki). Szczególowe informacje $0$ przetwarzaniu przez nas danych osobowych są dostępne

na stronie internetowej Kursu.

Adres Internetowy Kursu: http://www.im.pwr.edu.pl/kurs







XLIX

KORESPONDENCYJNY KURS

Z MATEMATYKI

marzec 2020 r.

PRACA KONTROLNA nr 7- POZIOM PODSTAWOWY

l. Wyznaczyč $z$ jako funkcję zmiennej $y$, wiedząc, $\displaystyle \dot{\mathrm{z}}\mathrm{e}x=2\frac{\mathrm{l}}{1-\log_{2}z}$ oraz $y=2\displaystyle \frac{\mathrm{l}}{1-\log_{2}x}$

2. Pokazač, $\dot{\mathrm{z}}\mathrm{e}$ dla $\mathrm{k}\mathrm{a}\dot{\mathrm{z}}$ dej wartości parametru $\alpha \in [0,2\pi]$, dla której istnieje rozwiązanie

równania $x^{2}-2\cos\alpha\cdot x+\sin^{2}\alpha= 0$ suma kwadratów jego pierwiastków jest równa

przynajmniej l.

3. $\mathrm{W}$ zalezności od parametru rzeczywistego $k$ przedyskutowač liczbę rozwiązań ukfadu

równań 

Sporządzič ilustrację graficzną układu dla kilku charakterystycznych $k.$

4. Przekątna $BD$ równoległoboku ABCD jest prostopadla do boku $AD$, a $\mathrm{k}\mathrm{a}\mathrm{t}$ ostry te-

go równolegloboku jest równy kątowi między jego przekątnymi. Wyznaczyč stosunek

dlugości przekqtnych. Sporządzič rysunek.

5. Wyznaczyč zbiór punktów, $\mathrm{z}$ których odcinek $0$ końcach $A(2,0)\mathrm{i}B(1,\sqrt{2})$ jest widoczny

pod kątem $30^{\mathrm{o}}$ Sporządzič rysunek.

6. Podstawą graniastosłupa prostego $0$ wszystkich krawędziach równych $a$, jest romb $0$ kącie

ostrym $\alpha$. Graniastoslup przecięto $\mathrm{p}\mathrm{f}\mathrm{z}\Phi \mathrm{P}^{\mathrm{r}\mathrm{z}\mathrm{e}\mathrm{c}\mathrm{h}\mathrm{o}\mathrm{d}_{\mathrm{Z}}}\text{ą}^{\mathrm{C}}\Phi$ przechodzącą przez dfuzszą

przekqtną $AC$ podstawy dolnej $\mathrm{i}$ przeciwległy wierzchołek podstawy górnej. Wyznaczyč

cosinus $\mathrm{k}_{\Phi^{\mathrm{t}\mathrm{a}}}$ nachylenia tej płaszczyzny do plaszczyzny podstawy $\mathrm{i}$ pole otrzymanego

przekroju. Sporządzič rysunek.




PRACA KONTROLNA nr 3- POZIOM ROZSZERZONY

l. Rozwiązač równanie $(\displaystyle \frac{1}{x})^{2-3\log_{2}x}=\frac{1}{2}x^{1+\log_{x}2}$

2. Dlajakich wartości parametru $m$ równanie $x^{3}+(m-2)x^{2}+(2-m-m^{2})x-(1-m^{2})=0$

ma trzy rózne pierwiastki, których suma kwadratów nie przekracza 5?

3. Czworokąt wypukly ABCD, $\mathrm{w}$ którym $AB=1, BC=2, CD=4, DA=3$ jest wpisany

$\mathrm{w}$ okrąg. Obliczyč promień $R$ tego okręgu. Sprawdzič, czy $\mathrm{w}$ czworokąt ten $\mathrm{m}\mathrm{o}\dot{\mathrm{z}}$ na wpisač

okrąg. $\mathrm{J}\mathrm{e}\dot{\mathrm{z}}$ eli $\mathrm{t}\mathrm{a}\mathrm{k}$, to obliczyč promień $r$ tego okręgu. Sporządzič rysunek.

4. Podstawa graniastosłupa prostego $0$ wszystkich krawedziach równych jest romb $0$ kącie

ostrym $\displaystyle \frac{\pi}{3}$. Graniastosfup ten przecięto dwiema płaszczyznami: plaszczyzną przechodzącą

przez bok $AB$ podstawy dolnej $\mathrm{i}$ wierzchołek $C'$ oraz płaszczyzną przechodzącą przez bok

$AD$ podstawy dolnej $\mathrm{i}$ ten sam wierzchofek $C'$. Wyznaczyč kąt dwuścienny między tymi

plaszczyznami oraz stosunek objętości brył, na jakie zostal podzielony graniastoslup.

Sporzadzič rysunek.

5. $\mathrm{W}$ zalezności od parametru rzeczywistego $p$ przedyskutowač liczbę rozwiązań ukfadu

równań

$\left\{\begin{array}{ll}
x^{4}+y^{4}+2x^{2}y^{2}-4x^{2} & =0,\\
x^{2}+y^{2}-2\sqrt{3}y & =p
\end{array}\right.$

Sporządzič ilustrację graficzną układu dla kilku charakterystycznych $p.$

6. Wykorzystując wzór Newtona $\mathrm{i}\mathrm{o}\mathrm{b}\mathrm{l}\mathrm{i}\mathrm{c}\mathrm{z}\mathrm{a}\mathrm{j}_{\Phi}\mathrm{c}$ pochodną wielomianu $w(x)=(1-x)^{n}$, wy-

kazač, $\dot{\mathrm{z}}\mathrm{e}$ dla dowolnego $n\in 1\mathrm{N}, n\geq 2$ zachodzi równośč

$\left(\begin{array}{l}
n\\
1
\end{array}\right)-2\left(\begin{array}{l}
n\\
2
\end{array}\right)+3\left(\begin{array}{l}
n\\
3
\end{array}\right)-4\left(\begin{array}{l}
n\\
4
\end{array}\right)+\ldots+(-1)^{n-1}n\left(\begin{array}{l}
n\\
n
\end{array}\right)=0$

Wywnioskowač stąd, $\dot{\mathrm{z}}$ ejezeli liczby $a_{1}, a_{2}, \ldots, a_{n}, a_{n+1}$ tworzą ciąg arytmetyczny, to dla

dowolnego $n\in 1\mathrm{N}$ zachodzi równośč

$a_{1}-\left(\begin{array}{l}
n\\
1
\end{array}\right)a_{2}+\left(\begin{array}{l}
n\\
2
\end{array}\right)a_{3}-\left(\begin{array}{l}
n\\
3
\end{array}\right)a_{4}+\ldots+(-1)^{n}\left(\begin{array}{l}
n\\
n
\end{array}\right)a_{n+1}=0$

Rozwiązania (rękopis) zadań z wybranego poziomu prosimy nadsylač do

2019r. na adres:

18 1istopada

Wydziaf Matematyki

Politechnika Wrocfawska

Wybrzez $\mathrm{e}$ Wyspiańskiego 27

$50-370$ WROCLAW.

Na kopercie prosimy $\underline{\mathrm{k}\mathrm{o}\mathrm{n}\mathrm{i}\mathrm{e}\mathrm{c}\mathrm{z}\mathrm{n}\mathrm{i}\mathrm{e}}$ zaznaczyč wybrany poziom! (np. poziom podsta-

wowy lub rozszerzony). Do rozwiązań nalez $\mathrm{y}$ dołączyč zaadresowana do siebie koperte

zwrotną $\mathrm{z}$ naklejonym znaczkiem, odpowiednim do formatu listu. Polecamy stosowanie

kopert formatu C5 $(160\mathrm{x}230\mathrm{m}\mathrm{m})$ ze znaczkiem $0$ wartości 3,30 zł. Na $\mathrm{k}\mathrm{a}\dot{\mathrm{z}}$ dą wiekszą

kopertę nalez $\mathrm{y}$ nakleič drozszy znaczek. Prace niespełniające podanych warunków nie

będą poprawiane ani odsyłane.

Uwaga. Wysyłając nam rozwi\S zania zadań uczestnik Kursu udostępnia Politechnice Wroclawskiej

swoje dane osobowe, które przetwarzamy wyłącznie $\mathrm{w}$ zakresie niezbędnym do jego prowadzenia

(odesfanie zadań, prowadzenie statystyki). Szczególowe informacje $0$ przetwarzaniu przez nas danych

osobowych są dostępne na stronie internetowej Kursu.

Adres internetowy Kursu: http://www.im.pwr.edu.pl/kurs







L KORESPONDENCYJNY KURS

Z MATEMATYKI

marzec 2021 r.

PRACA KONTROLNA nr 7- POZIOM PODSTAWOWY

l. Wykaz$\cdot, \dot{\mathrm{z}}\mathrm{e}$ dla dowolnych liczb $a, b$ róznych od zera, posiadających ten sam znak, praw-

dziwa jest nierównośč

-{\it ab}$+-\alpha${\it b} $> -85^{\cdot}$

2. Wyznacz $\mathrm{t}\mathrm{g}\alpha$, wiedząc, $\dot{\mathrm{z}}\mathrm{e}\alpha$ jest kqtem ostrym spełniającym równanie

$\displaystyle \frac{2\sin\alpha+3\cos\alpha}{\cos\alpha}=2$ ctg $\alpha.$

3. Spośród l0 biafych $\mathrm{i}2$ czarnych kul losujemy bez zwracania $m\mathrm{k}\mathrm{u}\mathrm{l}$. Jaka jest najmniejsza

liczba $m$, dla której prawdopodobieństwo, $\dot{\mathrm{z}}\mathrm{e}$ wśród wylosowanych kul jest przynajmniej

jedna czarna, przekracza $\displaystyle \frac{1}{2}$?

4. Wielomian $W(x)=2x^{3}+px^{2}+qx-2$ ma współczynniki całkowite $\mathrm{i}$ pierwiastek całkowity,

a reszta $\mathrm{z}$ jego dzielenia przez dwumian $x-2$ jest równa 10. D1a jakich $x$ przyjmuje on

wartości dodatnie?

5. Odcinek $0$ końcach $A(1,0) \mathrm{i}B(2,1)$ jest podstawą trójkąta równoramiennego, którego

trzeci wierzchofek $\mathrm{l}\mathrm{e}\dot{\mathrm{z}}\mathrm{y}$ na prostej $y=2x+1$. Podaj równania prostych $\mathrm{z}\mathrm{a}\mathrm{w}\mathrm{i}\mathrm{e}\mathrm{r}\mathrm{a}\mathrm{j}_{\Phi}$cych

ramiona tego trójkąta $\mathrm{i}$ oblicz jego pole.

6. Na bokach $AC\mathrm{i}BC$ trójkqta równoramiennego $ABC$ obrano punkty $M\mathrm{i}N$, których

rzutami prostokątnymi na podstawę AB $\mathrm{s}\Phi$ punkty $S, T$. Wykaz$\cdot, \dot{\mathrm{z}}\mathrm{e}|AB|=2|ST|$ wtedy

$\mathrm{i}$ tylko wtedy, gdy $|AM|=|CN|.$




PRACA KONTROLNA $\mathrm{n}\mathrm{r} 7-$ POZIOM ROZSZERZONY

l. Wykaz, $\dot{\mathrm{z}}\mathrm{e}$ dla dowolnych liczb rzeczywistych $a, b$ równośč $a^{3}-2b^{3}=ab(a+b)$ zachodzi

wtedy $\mathrm{i}$ tylko wtedy, gdy $a=2b.$

2. Rozwiąz równanie $\displaystyle \cos x-\sin x=\frac{\cos 2x}{\sin 2x+1}$

3. Liczba dwuelementowych podzbiorów zbioru $A$ jest 3 razy większa $\mathrm{n}\mathrm{i}\dot{\mathrm{z}}$ liczba dwuele-

mentowych podzbiorów zbioru $B$. Liczba dwuelementowych podzbiorów zbioru $A$ nie

zawierających ustalonego elementu $a\in A$ jest sumą liczby dwuelementowych podzbio-

rów zbioru $B\mathrm{i}$ liczby dwuelementowych podzbiorów zbioru $B$, do którego dodano jeden

element. Ile elementów ma $\mathrm{k}\mathrm{a}\dot{\mathrm{z}}\mathrm{d}\mathrm{y}\mathrm{z}$ tych zbiorów? Ile $\mathrm{k}\mathrm{a}\dot{\mathrm{z}}\mathrm{d}\mathrm{y}\mathrm{z}$ tych zbiorów ma podzbiorów

trzyelementowych?

4. Reszta $\mathrm{z}$ dzielenia wielomianu $W(x)=x^{4}+x^{3}+px^{2}+qx+2$ przez $(x^{2}+1)$ jest równa

$(-2x+6)$. Rozwiąz nierównośč $W(x)>0.$

5. Dwa boki trójkąta zawierają $\mathrm{s}\mathrm{i}\mathrm{e}\mathrm{w}$ prostych $2x-y=0\mathrm{i}x-2y=0$, a proste zawierające

jego wysokości przecinają się $\mathrm{w}$ punkcie $S(5,-2)$. Wyznacz wierzcholki trójk$\Phi$ta $\mathrm{i}$ oblicz

jego pole.

6. Wyznacz równanie krzywej będącej zbiorem środków okręgów, które sq styczne do prostej

$x=2\mathrm{i}$ do okręgu $x^{2}+2x+y^{2}-2y+1=0.$

Rozwiązania (rękopis) zadań z wybranego poziomu prosimy nadsyłač do 20.03.2021r. na

adres:

Wydziaf Matematyki

Politechnika Wrocfawska

Wybrzez $\mathrm{e}$ Wyspiańskiego 27

$50-370$ WROCLAW.

Na kopercie prosimy $\underline{\mathrm{k}\mathrm{o}\mathrm{n}\mathrm{i}\mathrm{e}\mathrm{c}\mathrm{z}\mathrm{n}\mathrm{i}\mathrm{e}}$ zaznaczyč wybrany poziom! (np. poziom podsta-

wowy lub rozszerzony). Do rozwiązań nalez $\mathrm{y}$ dołączyč zaadresowana do siebie koperte

zwrotną $\mathrm{z}$ naklejonym znaczkiem, odpowiednim do formatu listu. Polecamy stosowanie

kopert formatu C5 $(160\mathrm{x}230\mathrm{m}\mathrm{m})$ ze znaczkiem $0$ wartości 3,30 zł. Na $\mathrm{k}\mathrm{a}\dot{\mathrm{z}}$ dą wiekszą

kopertę nalez $\mathrm{y}$ nakleič drozszy znaczek. Prace niespełniające podanych warunków nie

będą poprawiane ani odsyłane.

Uwaga. Wysyłając nam rozwi\S zania zadań uczestnik Kursu udostępnia Politechnice Wroclawskiej

swoje dane osobowe, które przetwarzamy wyłącznie $\mathrm{w}$ zakresie niezbednym do jego prowadzenia

(odesfanie zadań, prowadzenie statystyki). Szczegófowe informacje $0$ przetwarzaniu przez nas danych

osobowych sa dostępne na stronie internetowej Kursu.

Adres internetowy Kursu: http://www.im.pwr.edu.pl/kurs







LI KORESPONDENCYJNY KURS

Z MATEMATYKI

marzec 2022 r.

PRACA KONTROLNA nr 7- POZIOM PODSTAWOWY

l. Grupa przyjaciól postanowiła kupič wspólnie ciekawą grę komputerową za 1920 z1otych.

Gdy zgfosifo sięjeszcze czterech chętnych do korzystania $\mathrm{z}$ tego oprogramowania, okazało

się, $\dot{\mathrm{z}}\mathrm{e}$, przy równym podziale kosztów, $\mathrm{k}\mathrm{a}\dot{\mathrm{z}}\mathrm{d}\mathrm{y}$ będzie mógf zaplacič 80 zfotych mniej. I1e

osób będzie korzystało $\mathrm{z}$ tej gry $\mathrm{i}$ ile $\mathrm{k}\mathrm{a}\dot{\mathrm{z}}\mathrm{d}\mathrm{y}\mathrm{z}$ nich musi za $\mathrm{n}\mathrm{i}\mathrm{a}$ zapłacič?

2. Liczby a, b, c dają przy dzieleniu przez 7 reszty (odpowiednio) - l, 2, 3.

suma kwadratów tych liczb jest podzielna przez 7.

Wykaz, $\dot{\mathrm{z}}\mathrm{e}$

3. Dla jakiego parametru $m$ pierwiastkiem równania

$x^{2}+(2m+1)x+m+4=0$

jest liczba $(-2)$ ? Dla znalezionego $m$ wyznacz drugi pierwiastek tego równania $\mathrm{i}\mathrm{s}$prawd $\acute{\mathrm{z}},$

dlajakich argumentów otrzymana funkcja kwadratowa $f(x)=x^{2}+(2m+1)x+m+4$

spełnia nierównośč

$2f(x)>1+\sqrt{2}.$

4. Oblicz wartośč wyrazeń

$\displaystyle \alpha=\frac{\sin 45^{\mathrm{o}}\cos 15^{\mathrm{o}}-\cos 45^{\mathrm{o}}\sin 15^{\mathrm{o}}}{\sin^{2}20^{\mathrm{o}}+\sin^{2}70^{\mathrm{o}}},$

{\it b}$=$ -ssiinn 7205oo ccooss 7150oo $+$-ccooss 2705oo ssiinn 7105oo.

Wyznacz stosunek promieni okregów wpisanego $\mathrm{i}$ opisanego na trójkącie prostokątnym,

którego przyprostokątne mają dlugości a $\mathrm{i}b.$

5. Punkty $A(1,0), B(5,2), C(3,3) \mathrm{s}\Phi$ trzema kolejnymi wierzchofkami trapezu prostokąt-

nego, $\mathrm{w}$ którym $AB||CD$. Wyznacz współrzędne wierzcholka $D$ oraz równania przekąt-

nych trapezu. $\mathrm{W}$ jakim stosunku $\mathrm{k}\mathrm{a}\dot{\mathrm{z}}$ da $\mathrm{z}$ tych przekątnych dzieli pole trapezu?

6. Krawędz/ boczna ostrosłupa prawidfowego trójkątnego jest dwa razy dłuzsza $\mathrm{n}\mathrm{i}\dot{\mathrm{z}}$ kra-

$\mathrm{w}\mathrm{e}\mathrm{d}\acute{\mathrm{z}}$ podstawy. Oblicz objetośč ostrosłupa $\mathrm{i}$ cosinus kąta nachylenia ściany bocznej do

podstawy, $\mathrm{w}\mathrm{i}\mathrm{e}\mathrm{d}\mathrm{z}\Phi^{\mathrm{C}}, \dot{\mathrm{z}}\mathrm{e}$ suma dlugości wszystkich jego krawędzi jest równa 18.




PRACA KONTROLNA $\mathrm{n}\mathrm{r} 7-$ POZIOM ROZSZERZONY

l. Dlajakich wartości parametru $a$ równanie $4-|x-1|=(a+2)^{2}$ ma dwa rózne rozwiązania?

2. Wykorzystując dwumian Newtona, uzasadnij, $\dot{\mathrm{z}}\mathrm{e}$ liczba $11^{2k}-9^{2k}$ jest podzielna przez

100 dla dowolnej liczby naturalnej $k$ podzielnej przez 5.

3. Wykaz$\cdot, \dot{\mathrm{z}}\mathrm{e}\mathrm{w}$ dowolnym trójk$\Phi$cie prostokątnym wartośč bezwzględna róznicy tangensów

kątów ostrych jest dwa razy większa $\mathrm{n}\mathrm{i}\dot{\mathrm{z}}$ wartośč bezwzględna tangensa kąta, jaki tworzą

wysokośč $\mathrm{i}$ środkowa poprowadzone $\mathrm{z}$ wierzchołka kąta prostego.

4. Dany jest trapez prostokątny $0$ podstawach długości $a\mathrm{i}b$ oraz wysokości $2h$. Wykaz$\cdot,$

$\dot{\mathrm{z}}\mathrm{e}\mathrm{j}\mathrm{e}\dot{\mathrm{z}}$ eli $h^{2}=ab,$ to dłuzsze ramię trapezu jest równe $\alpha+b$, a okrąg, którego jest ono

średnicą, jest styczny do drugiego ramienia.

5. Narysuj wykres funkcji

$ f(x)=1-\displaystyle \frac{x}{x+2}+(\frac{x}{x+2})^{2}-(\frac{x}{x+2})^{3}+\ldots$

która jest sumą nieskończonego szeregu geometrycznego $\mathrm{i}$ wyznacz równanie prostej

stycznej do wykresu prostopadłej do prostej $2x-y=0. \mathrm{S}_{\mathrm{P}^{\mathrm{o}\mathrm{r}\mathrm{Z}\otimes}}\mathrm{d}\acute{\mathrm{z}}$ staranny rysunek.

6. Podstawą ostrosfupa jest trapez $0$ obwodzie 32, którego jedna podstawa jest trzy razy

dluzsza $\mathrm{n}\mathrm{i}\dot{\mathrm{z}}$ druga. Wszystkie krawedzie boczne ostroslupa są nachylone do podstawy

pod $\mathrm{k}_{\Phi}\mathrm{t}\mathrm{e}\mathrm{m}60^{\mathrm{o}}$ Oblicz objętośč ostrosfupa, wiedząc, $\dot{\mathrm{z}}\mathrm{e}\mathrm{w}$ jego podstawę $\mathrm{m}\mathrm{o}\dot{\mathrm{z}}$ na wpisač

okrąg.

Rozwiązania (rękopis) zadań z wybranego poziomu prosimy nadsyłač do

2022r. na adres:

20 marca

Wydziaf Matematyki

Politechnika Wrocfawska

Wybrzez $\mathrm{e}$ Wyspiańskiego 27

$50-370$ WROCLAW,

lub elektronicznie, za pośrednictwem portalu talent. $\mathrm{p}\mathrm{w}\mathrm{r}$. edu. pl

Na kopercie prosimy $\underline{\mathrm{k}\mathrm{o}\mathrm{n}\mathrm{i}\mathrm{e}\mathrm{c}\mathrm{z}\mathrm{n}\mathrm{i}\mathrm{e}}$ zaznaczyč wybrany poziom! (np. poziom podsta-

wowy lub rozszerzony). Do rozwiązań nalez $\mathrm{y}$ dołączyč zaadresowaną do siebie koperte

zwrotną $\mathrm{z}$ naklejonym znaczkiem, odpowiednim do formatu listu. Polecamy stosowanie

kopert formatu C5 $(160\mathrm{x}230\mathrm{m}\mathrm{m})$ ze znaczkiem $0$ wartości 3,30 zł. Na $\mathrm{k}\mathrm{a}\dot{\mathrm{z}}$ dą wiekszą

kopertę nalez $\mathrm{y}$ nakleič $\mathrm{d}\mathrm{r}\mathrm{o}\dot{\mathrm{z}}$ szy znaczek. Prace niespełniające podanych warunków nie

będą poprawiane ani odsyłane.

Uwaga. Wysylajqc nam rozwiazania zadań uczestnik Kursu udostępnia Politechnice Wroclawskiej

swoje dane osobowe, które przetwarzamy wyłącznie $\mathrm{w}$ zakresie niezbędnym do jego prowadzenia

(odesfanie zadań, prowadzenie statystyki). Szczegófowe informacje $0$ przetwarzaniu przez nas danych

osobowych sa dostępne na stronie internetowej Kursu.

Adres internetowy Kursu: http: //www. im. pwr. edu. pl/kurs







LII

KORESPONDENCYJNY KURS

Z MATEMATYKI

marzec 2023 r.

PRACA KONTROLNA $\mathrm{n}\mathrm{r} 7-$ POZIOM PODSTAWOWY

l. Wielomian $W(x) =x^{3}-(k+m)x^{2}-(k-m)x+3$ jest podzielny przez dwumian $(x-1),$

a suma jego współczynników przy parzystych potęgach zmiennej $x$ jest równa sumie

wspófczynników przy nieparzystych potegach zmiennej. Rozwiqz nierównośč

$W(x)\leq x^{2}-1.$

2. Rozwia $\dot{\mathrm{z}}$ algebraicznie układ równań

tację geometryczną.

$\left\{\begin{array}{l}
|y|=2-x^{2},\\
x^{2}+y^{2}=2
\end{array}\right.$

a następnie podaj jego interpre-

3. $\mathrm{W}$ przedziale $[0,2\pi]$ określ liczbę rozwiązań równania

$\cos x$. ctg $x-\sin x=a\cos 2x,$

$\mathrm{w}$ zalezności od parametru $a.$

4. Niech $P(k)$ oznacza pole trójkąta ograniczonego prostą $y=kx\mathrm{i}$ wykresem funkcji

$f(x)=4-2|x|.$

Wyznacz $\mathrm{n}\mathrm{a}\mathrm{j}\mathrm{m}\mathrm{n}\mathrm{i}\mathrm{e}\mathrm{j}\mathrm{s}\mathrm{z}\Phi$ wartośč $P(k).$

5. Punkty $A(0,0)\mathrm{i}B(4,3)$ są wierzchołkami rombu $0$ kącie ostrym $45^{\mathrm{o}}$, który zawarty jest

$\mathrm{w}$ pierwszej čwiartce ukfadu wspólrzędnych. Wyznacz współrzędne jego wierzcholków.

Podaj równanie okręgu wpisanego $\mathrm{w}$ ten romb. Ile jest wszystkich rombów $0$ boku $AB$

$\mathrm{i}$ kącie ostrym $45^{\mathrm{o}}$? Oblicz objętośč bryły otrzymanej przez obrót rombu wokół jego

boku.

6. $\mathrm{W}$ ostroslupie prawidlowym czworokątnym środek podstawy jest odlegly $\mathrm{o}d$ od krawędzi

bocznej a kąt między sąsiednimi ścianami bocznymi ostroslupa jest równy $ 2\alpha$. Oblicz

objętośč ostrosfupa.




PRACA KONTROLNA $\mathrm{n}\mathrm{r} 7-$ POZIOM ROZSZERZONY

l. Dla jakiego parametru $m$ równanie

$mx^{3}-(2m+1)x^{2}+(2-3m)x+3=0$

ma trzy rózne pierwiastki, które są kolejnymi wyrazami $\mathrm{c}\mathrm{i}_{\Phi \mathrm{g}}\mathrm{u}$ arytmetycznego?

2. Rozwiąz równanie

$\displaystyle \frac{1+\mathrm{t}\mathrm{g}x+\mathrm{t}\mathrm{g}^{2}x+\mathrm{t}\mathrm{g}^{3}x+\ldots+\mathrm{t}\mathrm{g}^{n}x+}{1-\mathrm{t}\mathrm{g}x+\mathrm{t}\mathrm{g}^{2}x-\mathrm{t}\mathrm{g}^{3}x+\ldots+(-1)^{n}\mathrm{t}\mathrm{g}^{n}x+}=1+\sin 2x.$

3. Narysuj $\mathrm{w}$ prostokątnym ukfadzie wspófrzędnych zbiór punktów spefniających warunek

$\log_{(x-y)}(x+y)\leq 1.$

4. Podaj równanie prostej $l$ stycznej do wykresu funkcji $f(x)=\displaystyle \frac{3x-2}{(x-1)^{2}} \mathrm{w}$ punkcie jego

przecięcia $\mathrm{z}\mathrm{o}\mathrm{s}\mathrm{i}_{\Phi}Oy\mathrm{i}$ wyznacz równania wszystkich stycznych do wykresu równoleglych

do $l$. Oblicz odleglośč między otrzymanymi prostymi. Sporząd $\acute{\mathrm{z}}$ staranny wykres funkcji

wraz $\mathrm{z}$ otrzymanymi stycznymi.

5. Ostrosłup prawidłowy czworokątny przecięto plaszczyzną przechodzącą przez przekqtną

podstawy $\mathrm{i}$ środek przeciwleglej krawedzi bocznej. Płaszczyzna ta jest nachylona do

plaszczyzny podstawy pod kątem $\alpha$. Wyznacz kąt między ścianami bocznymi.

6. Odcinek $0$ końcach $A(0,0)\mathrm{i}B(8,6)$ jest dłuzszą podstawą trapezu prostokątnego opisa-

nego na okręgu. Wyznacz współrzędne pozostafych wierzcholków trapezu, wiedzqc, $\dot{\mathrm{z}}\mathrm{e}$

bok $CD$ jest dwa razy krótszy od boku $AB$. Podaj równanie okręgu wpisanego $\mathrm{w}$ ten

trapez. Oblicz objętośč bryly otrzymanej przez obrót trapezu wokół ramienia $BC.$

$\mathrm{R}\mathrm{o}\mathrm{z}\mathrm{w}\mathrm{i}_{\Phi}$zania (rękopis) zadań $\mathrm{z}$ wybranego poziomu prosimy nadsyfač do $20.03.2023\mathrm{r}$. na

adres:

Wydziaf Matematyki

Politechnika Wrocfawska

Wybrzez $\mathrm{e}$ Wyspiańskiego 27

$50-370$ WROCLAW,

$\mathrm{l}\mathrm{u}\mathrm{b}$ elektronicznie, za poŚrednictwem portalu talent. $\mathrm{p}\mathrm{w}\mathrm{r}$. edu. pl

Na kopercie prosimy $\underline{\mathrm{k}\mathrm{o}\mathrm{n}\mathrm{i}\mathrm{e}\mathrm{c}\mathrm{z}\mathrm{n}\mathrm{i}\mathrm{e}}$ zaznaczyč wybrany poziom! (np. poziom podsta-

wowy lub rozszerzony). Do rozwiązań nalez $\mathrm{y}$ dołączyč zaadresowana do siebie koperte

zwrotną $\mathrm{z}$ naklejonym znaczkiem, odpowiednim do formatu listu. Prace niespełniające

podanych warunków nie będą poprawiane ani odsyłane.

Uwaga. Wysylając nam rozwiazania zadań uczestnik Kursu udostępnia Politechnice Wroclawskiej

swoje dane osobowe, które przetwarzamy wyłącznie $\mathrm{w}$ zakresie niezbędnym do jego prowadzenia

(odesłanie zadań, prowadzenie statystyki). Szczegółowe informacje $0$ przetwarzaniu przez nas danych

osobowych są dostępne na stronie internetowej Kursu.

Adres internetowy Kursu: http: //www. im. pwr. edu. pl/kurs





\end{document}