\documentclass[a4paper,12pt]{article}
\usepackage{latexsym}
\usepackage{amsmath}
\usepackage{amssymb}
\usepackage{graphicx}
\usepackage{wrapfig}
\pagestyle{plain}
\usepackage{fancybox}
\usepackage{bm}

\begin{document}

Centralna Komisja Egzaminacyjna

Arkusz zawiera informacje prawnie chronione do momentu rozpoczęcia egzaminu.

WPISUJE ZDAJACY

KOD PESEL

{\it Miejsce}

{\it na naklejkę}

{\it z kodem}
\begin{center}
\includegraphics[width=21.432mm,height=9.804mm]{./F1_M_PP_C2012_page0_images/image001.eps}

\includegraphics[width=82.092mm,height=9.804mm]{./F1_M_PP_C2012_page0_images/image002.eps}
\end{center}
\fbox{} dysleksja
\begin{center}
\includegraphics[width=204.060mm,height=216.048mm]{./F1_M_PP_C2012_page0_images/image003.eps}
\end{center}
EGZAMIN MATU LNY

Z MATEMATYKI

CZERWIEC 2012

POZIOM PODSTAWOWY

1. Sprawd $\acute{\mathrm{z}}$, czy arkusz egzaminacyjny zawiera 18 stron

(zadania $1-34$). Ewentualny brak zgłoś przewodniczącemu

zespo nadzorującego egzamin.

2. Rozwiązania zadań i odpowiedzi wpisuj w miejscu na to

przeznaczonym.

3. Odpowiedzi do zadań za niętych (l-24) przenieś

na ka ę odpowiedzi, zaznaczając je w części ka $\mathrm{y}$

przeznaczonej dla zdającego. Zamaluj $\blacksquare$ pola do tego

przeznaczone. Błędne zaznaczenie otocz kółkiem \fcircle$\bullet$

i zaznacz właściwe.

4. Pamiętaj, $\dot{\mathrm{z}}\mathrm{e}$ pominięcie argumentacji lub istotnych

obliczeń w rozwiązaniu zadania otwa ego (25-34) $\mathrm{m}\mathrm{o}\dot{\mathrm{z}}\mathrm{e}$

spowodować, $\dot{\mathrm{z}}\mathrm{e}$ za to rozwiązanie nie będziesz mógł

dostać pełnej liczby punktów.

5. Pisz czytelnie i uzywaj tvlko długopisu lub -Dióra

z czarnym tuszem lub atramentem.

6. Nie uzywaj korektora, a błędne zapisy wyrazínie prze eśl.

7. Pamiętaj, $\dot{\mathrm{z}}\mathrm{e}$ zapisy w brudnopisie nie będą oceniane.

8. $\mathrm{M}\mathrm{o}\dot{\mathrm{z}}$ esz korzystać z zestawu wzorów matematycznych,

cyrkla i linijki oraz kalkulatora.

9. Na tej stronie oraz na karcie odpowiedzi wpisz swój

numer PESEL i przyklej naklejkę z kodem.

10. Nie wpisuj $\dot{\mathrm{z}}$ adnych znaków w części przeznaczonej

dla egzaminatora.

Czas pracy:

170 minut

Liczba punktów

do uzyskania: 50

$\Vert\Vert\Vert\Vert\Vert\Vert\Vert\Vert\Vert\Vert\Vert\Vert\Vert\Vert\Vert\Vert\Vert\Vert\Vert\Vert\Vert\Vert\Vert\Vert|  \mathrm{M}\mathrm{M}\mathrm{A}-\mathrm{P}1_{-}1\mathrm{P}-123$




{\it 2}

{\it Egzamin maturalny z matematyki}

{\it Poziom podstawowy}

ZADANIA ZAMKNIĘTE

{\it Wzadaniach} $\theta d1.$ {\it do 24. wybierz i zaznacz na karcie odpowiedzipoprawnq odpowiedzí}.

Zadanie l. $(1pkt)$

Ułamek $\displaystyle \frac{\sqrt{5}+2}{\sqrt{5}-2}$ jest równy

A. 1 B. $-1$

C. $7+4\sqrt{5}$

D. $9+4\sqrt{5}$

Zadanie 2. $(1pkt)$

Liczbami spełniającymi równanie $|2x+3|=5$ są

A. $1\mathrm{i}-4$

B. l i 2

C. $-1\mathrm{i}4$

D. $-2\mathrm{i}2$

Zadanie 3. $(1pkt)$

Równanie $(x+5)(x-3)(x^{2}+1)=0$ ma

A.

B.

C.

D.

dwa rozwiązania: $x=-5, x=3.$

dwa rozwiązania: $x=-3, x=5.$

cztery rozwiązania: $x=-5, x=-1, x=1, x=3.$

cztery rozwiązania: $x=-3, x=-1, x=1, x=5.$

Zadanie 4. (1pkt)

Marza równa 1,5\% kwoty pozyczonego kapitału była równa 3000 zł.

pozyczono

Wynika stąd, $\dot{\mathrm{z}}\mathrm{e}$

A. 45 zł

B. 2000 zł

C. 200000 zł

D. 450000 zł

Zadanie 5. $(1pkt)$

Najednym z ponizszych rysunków przedstawiono fragment wykresu funkcji $y=x^{2}+2x-3.$

Wskaz ten rysunek.
\begin{center}
\includegraphics[width=4.932mm,height=22.812mm]{./F1_M_PP_C2012_page1_images/image001.eps}

\begin{tabular}{|l|l|}
\hline
\multicolumn{1}{|l|}{ $\begin{array}{l}\mbox{$4$}	\\	\mbox{ $3$}	\\	\mbox{ $2$}	\\	\mbox{ $1$}	\end{array}$}&	\multicolumn{1}{|l|}{ $\mathrm{y}$}	\\
\hline
\multicolumn{1}{|l|}{ $\begin{array}{l}\mbox{ $-4-2-1$}	\\	\mbox{ $-1$}	\\	\mbox{ $-2$}	\\	\mbox{ $-3$}	\\	\mbox{ $-4$}	\end{array}$}&	\multicolumn{1}{|l|}{ $234$}	\\
\hline
\end{tabular}


\begin{tabular}{|l|l|}
\hline
\multicolumn{1}{|l|}{ $\begin{array}{l}\mbox{$4$}	\\	\mbox{ $3$}	\\	\mbox{ $1$}	\end{array}$}&	\multicolumn{1}{|l|}{ $\mathrm{y}$}	\\
\hline
\multicolumn{1}{|l|}{ $-4-3-21^{1}4321$}&	\multicolumn{1}{|l|}{ $124$}	\\
\hline
\end{tabular}


\begin{tabular}{|l|l|}
\hline
\multicolumn{1}{|l|}{ $\begin{array}{l}\mbox{$4$}	\\	\mbox{ $3$}	\\	\mbox{ $2$}	\\	\mbox{ $1$}	\end{array}$}&	\multicolumn{1}{|l|}{ $\mathrm{y}$}	\\
\hline
\multicolumn{1}{|l|}{ $\begin{array}{l}\mbox{ $-43-2-1$}	\\	\mbox{ $-1$}	\\	\mbox{ $-2$}	\\	\mbox{ $-3$}	\\	\mbox{ $-4$}	\end{array}$}&	\multicolumn{1}{|l|}{ $234$}	\\
\hline
\end{tabular}


\includegraphics[width=4.932mm,height=22.812mm]{./F1_M_PP_C2012_page1_images/image002.eps}

\begin{tabular}{|l|l|}
\hline
\multicolumn{1}{|l|}{ $\begin{array}{l}\mbox{$4$}	\\	\mbox{ $3$}	\\	\mbox{ $2$}	\\	\mbox{ $1$}	\end{array}$}&	\multicolumn{1}{|l|}{ $\mathrm{y}$}	\\
\hline
\multicolumn{1}{|l|}{ $\begin{array}{l}\mbox{ $-4-3-2$}	\\	\mbox{ $-1$}	\\	\mbox{ $-3$}	\\	\mbox{ $-4$}	\end{array}$}&	\multicolumn{1}{|l|}{ $124$}	\\
\hline
\end{tabular}


\includegraphics[width=5.232mm,height=22.860mm]{./F1_M_PP_C2012_page1_images/image003.eps}
\end{center}
A.

B.

C.

D.





{\it Egzamin maturalny z matematyki}

{\it Poziom podstawowy}

{\it 11}

Zadanie 27. (2pkt)

Podstawy trapezu prostokątnego mają długości 6 i 10 oraz tangens jego kąta ostrego jest

równy 3. Ob1icz po1e tego trapezu.

Odpowiedzí :

Zadanie 28. $(2pkt)$

Uzasadnij, $\dot{\mathrm{z}}$ ejezeli $\alpha$ jest kątem ostrym, to $\sin^{4}\alpha+\cos^{2}\alpha=\sin^{2}\alpha+\cos^{4}\alpha.$





{\it 12}

{\it Egzamin maturalny z matematyki}

{\it Poziom podstawowy}

Zadanie 29. $(2pkt)$

Uzasadnij, $\dot{\mathrm{z}}\mathrm{e}$ suma kwadratów trzech kolejnych liczb całkowitych przy dzieleniu przez 3 daje

resztę 2.

Zadanie 30. $(2pkt)$

Suma $S_{n}=a_{1}+a_{2}+\ldots+a_{n}$ początkowych $n$ wyrazów pewnego ciągu arytmetycznego $(a_{n})$

jest określona wzorem $S_{n}=n^{2}-2n$ dla $n\geq 1$. Wyznacz wzór na n-ty wyraz tego ciągu.

Odpowied $\acute{\mathrm{z}}$:





{\it Egzamin maturalny z matematyki}

{\it Poziom podstawowy}

{\it 13}

Zadanie 31. $(2pkt)$

Dany jest romb, którego kąt ostry ma miarę $45^{\mathrm{o}}$, a jego pole jest równe $50\sqrt{2}$. Oblicz

wysokość tego rombu.

Odpowied $\acute{\mathrm{z}}$:





{\it 14}

{\it Egzamin maturalny z matematyki}

{\it Poziom podstawowy}

Zadanie 32. $(4pkt)$

Punkty $A=(2,11), B=(8,23), C=(6,14)$ są wierzchołkami trójkąta. Wysokość trójkąta

poprowadzona z wierzchołka $C$ przecina prostą AB w punkcie $D$. Oblicz współrzędne punktu $D.$

Odpowied $\acute{\mathrm{z}}$:





{\it Egzamin maturalny z matematyki}

{\it Poziom podstawowy}

{\it 15}

Zadanie 33. (4pkt)

Oblicz, ile jest liczb naturalnych pięciocyfrowych, w zapisie których nie występuje zero, jest

dokładniejedna cyfra 7 i dokładniejedna cyfra parzysta.

Odpowiedzí :





{\it 16}

{\it Egzamin maturalny z matematyki}

{\it Poziom podstawowy}

Zadanie 34. (4pkt)

Dany jest graniastosłup prawidłowy trójkątny ABCDEF o podstawach ABC i DEF

i krawędziach bocznych AD, BE iCF (zobacz rysunek). Długość krawędzi podstawy AB jest

równa 8, a po1e trójkąta ABFjest równe 52. Ob1icz objętość tego graniastosłupa.





{\it Egzamin maturalny z matematyki}

{\it Poziom podstawowy}

{\it 1}7

Odpowied $\acute{\mathrm{z}}$:





{\it 18}

{\it Egzamin maturalny z matematyki}

{\it Poziom podstawowy}

BRUDNOPIS





{\it Egzamin maturalny z matematyki}

{\it Poziom podstawowy}

{\it 3}

BRUDNOPIS





{\it 4}

{\it Egzamin maturalny z matematyki}

{\it Poziom podstawowy}

Zadanie 6. $(1pkt)$

Wierzchołkiem paraboli będącej wykresem ffinkcji określonej wzorem $f(x)=x^{2}-4x+4$

jest punkt o współrzędnych

A. (0,2)

B. $(0,-2)$

C. $(-2,0)$

D. (2, 0)

Zadanie 7. $(1pkt)$

Jeden kąt trójkąta ma miarę $54^{\mathrm{o}} \mathrm{Z}$ pozostałych dwóch kątów tego trójkątajedenjest 6 razy

większy od drugiego. Miary pozostałych kątów są równe

A. $21^{\mathrm{o}}$ i $105^{\mathrm{o}}$

B. $11^{\mathrm{o}}$ i $66^{\mathrm{o}}$

C. $18^{\mathrm{o}}$ i $108^{\mathrm{o}}$

D. $16^{\mathrm{o}}\mathrm{i}96^{\mathrm{o}}$

Zadanie 8. $(1pkt)$

Krótszy bok prostokąta ma długość 6. Kąt między przekątną prostokąta i dłuzszym bokiem

ma miarę $30^{\mathrm{o}}$. Dłuzszy bok prostokąta ma długość

A. $2\sqrt{3}$

B. $4\sqrt{3}$

C. $6\sqrt{3}$

D. 12

Zadanie 9. (1pkt)

Cięciwa okręgu ma długość 8 cm ijest odda1ona odjego środka o 3 cm. Promień tego okręgu

ma długość

A. 3 cm

B. 4 cm

C. 5 cm

D. 8 cm

Zadanie 10. (1pkt)

Punkt O jest środkiem okręgu. Kąt wpisany BAD ma miarę

A. $150^{\mathrm{o}}$
\begin{center}
\includegraphics[width=44.040mm,height=46.380mm]{./F1_M_PP_C2012_page3_images/image001.eps}
\end{center}
{\it D  C}

$130^{\circ}$

{\it O}

$60^{\circ}$

{\it B}

{\it A}

$115^{\mathrm{o}}$

$120^{\mathrm{o}}$

C.

B.

D. $85^{\mathrm{o}}$

Zadanie ll. (lpkt)

Pięciokąt ABCDE jest foremny. Wskaz trójkąt przystający do trójkąta ECD

A.

$\Delta ABF$

B.

$\Delta CAB$
\begin{center}
\includegraphics[width=55.884mm,height=50.088mm]{./F1_M_PP_C2012_page3_images/image002.eps}
\end{center}
{\it D}

{\it E  I H  C}

{\it J  G}

{\it F}

{\it A B}

$\Delta ABD$

D.

$\Delta IHD$

C.





{\it Egzamin maturalny z matematyki}

{\it Poziom podstawowy}

{\it 5}

BRUDNOPIS





{\it 6}

{\it Egzamin maturalny z matematyki}

{\it Poziom podstawowy}

Zadanie 12. (1pkt)

Punkt O jest środkiem okręgu przedstawionego na rysunku. Równanie tego okręgu ma postać:

A.

B.
\begin{center}
\includegraphics[width=65.436mm,height=64.104mm]{./F1_M_PP_C2012_page5_images/image001.eps}
\end{center}
y

4

2

{\it o}

$-1$  1 2  3 4

x

5

$-2$

D.

C.

Zadanie 13. $(1pkt)$

Wyra $\dot{\mathrm{z}}$ enie $\displaystyle \frac{3x+1}{x-2}-\frac{2x-1}{x+3}$ jest równe

A.

-({\it xx}2-$+$21)5({\it xx} $++$31)

B.

$\displaystyle \frac{x+2}{(x-2)(x+3)}$

$(x-2)^{2}+(y-1)^{2}=9$

$(x-2)^{2}+(y-1)^{2}=3$

$(x+2)^{2}+(y+1)^{2}=9$

$(x+2)^{2}+(y+1)^{2}=3$

C. $\displaystyle \frac{x}{(x-2)(x+3)}$

D.

$\displaystyle \frac{x+2}{-5}$

Zadanie 14. $(1pkt)$

Ciąg $(a_{n})$ jest określony wzorem $a_{n}=\sqrt{2n+4}$ dla $n\geq 1$. Wówczas

A. $a_{8}=2\sqrt{5}$

B. $a_{8}=8$

C. $a_{8}=5\sqrt{2}$

D. $a_{8}=\sqrt{12}$

Zadanie 15. $(1pkt)$

Ciąg $(2\sqrt{2},4,a)$ jest geometryczny. Wówczas

A. $a=8\sqrt{2}$

B. $a=4\sqrt{2}$

C. $a=8-2\sqrt{2}$

D. $a=8+2\sqrt{2}$

Zadanie 16. $(1pkt)$

Kąt $\alpha$ jest ostry i $\mathrm{t}\mathrm{g}\alpha=1$. Wówczas

A. $\alpha<30^{\mathrm{o}}$

B. $\alpha=30^{\mathrm{o}}$

C. $\alpha=45^{\mathrm{o}}$

D. $\alpha>45^{\mathrm{o}}$

Zadanie 17. (1pkt)

Wiadomo, $\dot{\mathrm{z}}\mathrm{e}$ dziedziną funkcji

$(-\infty,2)\cup(2,+\infty)$. Wówczas

$f$ określonej wzorem $f(x)=\displaystyle \frac{x-7}{2x+a}$ jest zbiór

A. $a=2$

B. $a=-2$

C. $a=4$

D. $a=-4$





{\it Egzamin maturalny z matematyki}

{\it Poziom podstawowy}

7

BRUDNOPIS





{\it 8}

{\it Egzamin maturalny z matematyki}

{\it Poziom podstawowy}

Zadanie 18. $(1pkt)$

Jeden z rysunków przedstawia wykres ffinkcji liniowej $f(x)=ax+b$, gdzie $a>0\mathrm{i}b<0$. Wskaz

ten wykres.
\begin{center}
\includegraphics[width=45.108mm,height=45.108mm]{./F1_M_PP_C2012_page7_images/image001.eps}
\end{center}
$\gamma$

{\it x}

0
\begin{center}
\includegraphics[width=45.012mm,height=51.864mm]{./F1_M_PP_C2012_page7_images/image002.eps}
\end{center}
$\gamma$

{\it x}

0

D.

A.
\begin{center}
\includegraphics[width=45.156mm,height=51.864mm]{./F1_M_PP_C2012_page7_images/image003.eps}
\end{center}
$\gamma$

{\it x}

0

B.
\begin{center}
\includegraphics[width=44.904mm,height=51.912mm]{./F1_M_PP_C2012_page7_images/image004.eps}
\end{center}
$\gamma$

{\it x}

0

C.

Zadanie 19. $(1pkt)$

Punkt $S=(2,7)$ jest środkiem odcinka $AB$, w którym $A=(-1,3)$. Punkt $B$ ma współrzędne:

A. $B=(5,11)$ B. $B=(\displaystyle \frac{1}{2},2)$ C. $B=(-\displaystyle \frac{3}{2},-5)$ D. $B=(3,11)$

Zadanie 20. $(1pkt)$

$\mathrm{W}$ kolejnych sześciu rzutach kostką otrzymano następujące wyniki: 6, 3, 1, 2, 5, 5. Mediana

tych wynikówjest równa:

A. 3

B. 3,5

C. 4

D. 5

Zadanie 21. $(1pkt)$

RównoŚć $(a+2\sqrt{2})^{2}=a^{2}+28\sqrt{2}+8$ zachodzi dla

A. $a=14$ B. $a=7\sqrt{2}$ C.

$a=7$

D. $a=2\sqrt{2}$

Zadanie 22. (1pkt)

Trójkąt prostokątny o przyprostokątnych 4 i 6 obracamy wokół dłuzszej przyprostokątnej.

Objętość powstałego stozkajest równa

A. $ 96\pi$

B. $ 48\pi$

C. $ 32\pi$

D. $ 8\pi$

Zadanie 23. $(1pkt)$

$\mathrm{J}\mathrm{e}\dot{\mathrm{z}}$ eli $A \mathrm{i} B$ są zdarzeniami losowymi, $B'$ jest zdarzeniem przeciwnym do $B, P(A)=0,3,$

$P(B')=0,4$ oraz $ A\cap B=\emptyset$, to $P(A\cup B)$ jest równe

A. 0,12

B. 0,18

C. 0,6

D. 0,9

Zadanie 24. $(1pkt)$

Przekrój osiowy walca jest kwadratem o boku $a. \mathrm{J}\mathrm{e}\dot{\mathrm{z}}$ eli $r$ oznacza promień podstawy walca,

$h$ oznacza wysokość walca, to

A. $r+h=a$

B.

$h-r=\displaystyle \frac{a}{2}$

C.

{\it r-h}$=$ -{\it a}2

D. $r^{2}+h^{2}=a^{2}$





{\it Egzamin maturalny z matematyki}

{\it Poziom podstawowy}

{\it 9}

BRUDNOPIS





$ 1\theta$

{\it Egzamin maturalny z matematyki}

{\it Poziom podstawowy}

ZADANIA OTWARTE

{\it Rozwiqzania zadań o numerach od 25. do 34. nalezy zapisać w} $wyznacz\theta nych$ {\it miejscach}

{\it pod treściq zadania}.

Zadanie 25. $(2pkt)$

Rozwiąz nierówność $x^{2}-3x-10<0.$

Odpowiedz:

Zadanie 26. (2pkt)

Średnia wieku w pewnej giupie studentówjest równa 231ata. Średnia wieku tych studentów

i ich opiekunajest równa 241ata. Opiekun ma 391at. Ob1icz, i1u studentówjest w tej giupie.

Odpowiedzí:



\end{document}