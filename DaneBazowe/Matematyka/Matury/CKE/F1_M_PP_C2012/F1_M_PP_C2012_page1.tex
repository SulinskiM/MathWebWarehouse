\documentclass[a4paper,12pt]{article}
\usepackage{latexsym}
\usepackage{amsmath}
\usepackage{amssymb}
\usepackage{graphicx}
\usepackage{wrapfig}
\pagestyle{plain}
\usepackage{fancybox}
\usepackage{bm}

\begin{document}

{\it 2}

{\it Egzamin maturalny z matematyki}

{\it Poziom podstawowy}

ZADANIA ZAMKNIĘTE

{\it Wzadaniach} $\theta d1.$ {\it do 24. wybierz i zaznacz na karcie odpowiedzipoprawnq odpowiedzí}.

Zadanie l. $(1pkt)$

Ułamek $\displaystyle \frac{\sqrt{5}+2}{\sqrt{5}-2}$ jest równy

A. 1 B. $-1$

C. $7+4\sqrt{5}$

D. $9+4\sqrt{5}$

Zadanie 2. $(1pkt)$

Liczbami spełniającymi równanie $|2x+3|=5$ są

A. $1\mathrm{i}-4$

B. l i 2

C. $-1\mathrm{i}4$

D. $-2\mathrm{i}2$

Zadanie 3. $(1pkt)$

Równanie $(x+5)(x-3)(x^{2}+1)=0$ ma

A.

B.

C.

D.

dwa rozwiązania: $x=-5, x=3.$

dwa rozwiązania: $x=-3, x=5.$

cztery rozwiązania: $x=-5, x=-1, x=1, x=3.$

cztery rozwiązania: $x=-3, x=-1, x=1, x=5.$

Zadanie 4. (1pkt)

Marza równa 1,5\% kwoty pozyczonego kapitału była równa 3000 zł.

pozyczono

Wynika stąd, $\dot{\mathrm{z}}\mathrm{e}$

A. 45 zł

B. 2000 zł

C. 200000 zł

D. 450000 zł

Zadanie 5. $(1pkt)$

Najednym z ponizszych rysunków przedstawiono fragment wykresu funkcji $y=x^{2}+2x-3.$

Wskaz ten rysunek.
\begin{center}
\includegraphics[width=4.932mm,height=22.812mm]{./F1_M_PP_C2012_page1_images/image001.eps}

\begin{tabular}{|l|l|}
\hline
\multicolumn{1}{|l|}{ $\begin{array}{l}\mbox{$4$}	\\	\mbox{ $3$}	\\	\mbox{ $2$}	\\	\mbox{ $1$}	\end{array}$}&	\multicolumn{1}{|l|}{ $\mathrm{y}$}	\\
\hline
\multicolumn{1}{|l|}{ $\begin{array}{l}\mbox{ $-4-2-1$}	\\	\mbox{ $-1$}	\\	\mbox{ $-2$}	\\	\mbox{ $-3$}	\\	\mbox{ $-4$}	\end{array}$}&	\multicolumn{1}{|l|}{ $234$}	\\
\hline
\end{tabular}


\begin{tabular}{|l|l|}
\hline
\multicolumn{1}{|l|}{ $\begin{array}{l}\mbox{$4$}	\\	\mbox{ $3$}	\\	\mbox{ $1$}	\end{array}$}&	\multicolumn{1}{|l|}{ $\mathrm{y}$}	\\
\hline
\multicolumn{1}{|l|}{ $-4-3-21^{1}4321$}&	\multicolumn{1}{|l|}{ $124$}	\\
\hline
\end{tabular}


\begin{tabular}{|l|l|}
\hline
\multicolumn{1}{|l|}{ $\begin{array}{l}\mbox{$4$}	\\	\mbox{ $3$}	\\	\mbox{ $2$}	\\	\mbox{ $1$}	\end{array}$}&	\multicolumn{1}{|l|}{ $\mathrm{y}$}	\\
\hline
\multicolumn{1}{|l|}{ $\begin{array}{l}\mbox{ $-43-2-1$}	\\	\mbox{ $-1$}	\\	\mbox{ $-2$}	\\	\mbox{ $-3$}	\\	\mbox{ $-4$}	\end{array}$}&	\multicolumn{1}{|l|}{ $234$}	\\
\hline
\end{tabular}


\includegraphics[width=4.932mm,height=22.812mm]{./F1_M_PP_C2012_page1_images/image002.eps}

\begin{tabular}{|l|l|}
\hline
\multicolumn{1}{|l|}{ $\begin{array}{l}\mbox{$4$}	\\	\mbox{ $3$}	\\	\mbox{ $2$}	\\	\mbox{ $1$}	\end{array}$}&	\multicolumn{1}{|l|}{ $\mathrm{y}$}	\\
\hline
\multicolumn{1}{|l|}{ $\begin{array}{l}\mbox{ $-4-3-2$}	\\	\mbox{ $-1$}	\\	\mbox{ $-3$}	\\	\mbox{ $-4$}	\end{array}$}&	\multicolumn{1}{|l|}{ $124$}	\\
\hline
\end{tabular}


\includegraphics[width=5.232mm,height=22.860mm]{./F1_M_PP_C2012_page1_images/image003.eps}
\end{center}
A.

B.

C.

D.
\end{document}
