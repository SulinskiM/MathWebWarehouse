\documentclass[a4paper,12pt]{article}
\usepackage{latexsym}
\usepackage{amsmath}
\usepackage{amssymb}
\usepackage{graphicx}
\usepackage{wrapfig}
\pagestyle{plain}
\usepackage{fancybox}
\usepackage{bm}

\begin{document}

{\it 12}

{\it Egzamin maturalny z matematyki}

{\it Poziom podstawowy}

Zadanie 29. $(2pkt)$

Uzasadnij, $\dot{\mathrm{z}}\mathrm{e}$ suma kwadratów trzech kolejnych liczb całkowitych przy dzieleniu przez 3 daje

resztę 2.

Zadanie 30. $(2pkt)$

Suma $S_{n}=a_{1}+a_{2}+\ldots+a_{n}$ początkowych $n$ wyrazów pewnego ciągu arytmetycznego $(a_{n})$

jest określona wzorem $S_{n}=n^{2}-2n$ dla $n\geq 1$. Wyznacz wzór na n-ty wyraz tego ciągu.

Odpowied $\acute{\mathrm{z}}$:
\end{document}
