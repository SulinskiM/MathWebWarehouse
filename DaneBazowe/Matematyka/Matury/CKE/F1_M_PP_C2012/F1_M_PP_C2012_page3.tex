\documentclass[a4paper,12pt]{article}
\usepackage{latexsym}
\usepackage{amsmath}
\usepackage{amssymb}
\usepackage{graphicx}
\usepackage{wrapfig}
\pagestyle{plain}
\usepackage{fancybox}
\usepackage{bm}

\begin{document}

{\it 4}

{\it Egzamin maturalny z matematyki}

{\it Poziom podstawowy}

Zadanie 6. $(1pkt)$

Wierzchołkiem paraboli będącej wykresem ffinkcji określonej wzorem $f(x)=x^{2}-4x+4$

jest punkt o współrzędnych

A. (0,2)

B. $(0,-2)$

C. $(-2,0)$

D. (2, 0)

Zadanie 7. $(1pkt)$

Jeden kąt trójkąta ma miarę $54^{\mathrm{o}} \mathrm{Z}$ pozostałych dwóch kątów tego trójkątajedenjest 6 razy

większy od drugiego. Miary pozostałych kątów są równe

A. $21^{\mathrm{o}}$ i $105^{\mathrm{o}}$

B. $11^{\mathrm{o}}$ i $66^{\mathrm{o}}$

C. $18^{\mathrm{o}}$ i $108^{\mathrm{o}}$

D. $16^{\mathrm{o}}\mathrm{i}96^{\mathrm{o}}$

Zadanie 8. $(1pkt)$

Krótszy bok prostokąta ma długość 6. Kąt między przekątną prostokąta i dłuzszym bokiem

ma miarę $30^{\mathrm{o}}$. Dłuzszy bok prostokąta ma długość

A. $2\sqrt{3}$

B. $4\sqrt{3}$

C. $6\sqrt{3}$

D. 12

Zadanie 9. (1pkt)

Cięciwa okręgu ma długość 8 cm ijest odda1ona odjego środka o 3 cm. Promień tego okręgu

ma długość

A. 3 cm

B. 4 cm

C. 5 cm

D. 8 cm

Zadanie 10. (1pkt)

Punkt O jest środkiem okręgu. Kąt wpisany BAD ma miarę

A. $150^{\mathrm{o}}$
\begin{center}
\includegraphics[width=44.040mm,height=46.380mm]{./F1_M_PP_C2012_page3_images/image001.eps}
\end{center}
{\it D  C}

$130^{\circ}$

{\it O}

$60^{\circ}$

{\it B}

{\it A}

$115^{\mathrm{o}}$

$120^{\mathrm{o}}$

C.

B.

D. $85^{\mathrm{o}}$

Zadanie ll. (lpkt)

Pięciokąt ABCDE jest foremny. Wskaz trójkąt przystający do trójkąta ECD

A.

$\Delta ABF$

B.

$\Delta CAB$
\begin{center}
\includegraphics[width=55.884mm,height=50.088mm]{./F1_M_PP_C2012_page3_images/image002.eps}
\end{center}
{\it D}

{\it E  I H  C}

{\it J  G}

{\it F}

{\it A B}

$\Delta ABD$

D.

$\Delta IHD$

C.
\end{document}
