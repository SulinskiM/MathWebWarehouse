\documentclass[a4paper,12pt]{article}
\usepackage{latexsym}
\usepackage{amsmath}
\usepackage{amssymb}
\usepackage{graphicx}
\usepackage{wrapfig}
\pagestyle{plain}
\usepackage{fancybox}
\usepackage{bm}

\begin{document}

{\it 6}

{\it Egzamin maturalny z matematyki}

{\it Poziom podstawowy}

Zadanie 12. (1pkt)

Punkt O jest środkiem okręgu przedstawionego na rysunku. Równanie tego okręgu ma postać:

A.

B.
\begin{center}
\includegraphics[width=65.436mm,height=64.104mm]{./F1_M_PP_C2012_page5_images/image001.eps}
\end{center}
y

4

2

{\it o}

$-1$  1 2  3 4

x

5

$-2$

D.

C.

Zadanie 13. $(1pkt)$

Wyra $\dot{\mathrm{z}}$ enie $\displaystyle \frac{3x+1}{x-2}-\frac{2x-1}{x+3}$ jest równe

A.

-({\it xx}2-$+$21)5({\it xx} $++$31)

B.

$\displaystyle \frac{x+2}{(x-2)(x+3)}$

$(x-2)^{2}+(y-1)^{2}=9$

$(x-2)^{2}+(y-1)^{2}=3$

$(x+2)^{2}+(y+1)^{2}=9$

$(x+2)^{2}+(y+1)^{2}=3$

C. $\displaystyle \frac{x}{(x-2)(x+3)}$

D.

$\displaystyle \frac{x+2}{-5}$

Zadanie 14. $(1pkt)$

Ciąg $(a_{n})$ jest określony wzorem $a_{n}=\sqrt{2n+4}$ dla $n\geq 1$. Wówczas

A. $a_{8}=2\sqrt{5}$

B. $a_{8}=8$

C. $a_{8}=5\sqrt{2}$

D. $a_{8}=\sqrt{12}$

Zadanie 15. $(1pkt)$

Ciąg $(2\sqrt{2},4,a)$ jest geometryczny. Wówczas

A. $a=8\sqrt{2}$

B. $a=4\sqrt{2}$

C. $a=8-2\sqrt{2}$

D. $a=8+2\sqrt{2}$

Zadanie 16. $(1pkt)$

Kąt $\alpha$ jest ostry i $\mathrm{t}\mathrm{g}\alpha=1$. Wówczas

A. $\alpha<30^{\mathrm{o}}$

B. $\alpha=30^{\mathrm{o}}$

C. $\alpha=45^{\mathrm{o}}$

D. $\alpha>45^{\mathrm{o}}$

Zadanie 17. (1pkt)

Wiadomo, $\dot{\mathrm{z}}\mathrm{e}$ dziedziną funkcji

$(-\infty,2)\cup(2,+\infty)$. Wówczas

$f$ określonej wzorem $f(x)=\displaystyle \frac{x-7}{2x+a}$ jest zbiór

A. $a=2$

B. $a=-2$

C. $a=4$

D. $a=-4$
\end{document}
