\documentclass[a4paper,12pt]{article}
\usepackage{latexsym}
\usepackage{amsmath}
\usepackage{amssymb}
\usepackage{graphicx}
\usepackage{wrapfig}
\pagestyle{plain}
\usepackage{fancybox}
\usepackage{bm}

\begin{document}

{\it 8}

{\it Egzamin maturalny z matematyki}

{\it Poziom podstawowy}

Zadanie 18. $(1pkt)$

Jeden z rysunków przedstawia wykres ffinkcji liniowej $f(x)=ax+b$, gdzie $a>0\mathrm{i}b<0$. Wskaz

ten wykres.
\begin{center}
\includegraphics[width=45.108mm,height=45.108mm]{./F1_M_PP_C2012_page7_images/image001.eps}
\end{center}
$\gamma$

{\it x}

0
\begin{center}
\includegraphics[width=45.012mm,height=51.864mm]{./F1_M_PP_C2012_page7_images/image002.eps}
\end{center}
$\gamma$

{\it x}

0

D.

A.
\begin{center}
\includegraphics[width=45.156mm,height=51.864mm]{./F1_M_PP_C2012_page7_images/image003.eps}
\end{center}
$\gamma$

{\it x}

0

B.
\begin{center}
\includegraphics[width=44.904mm,height=51.912mm]{./F1_M_PP_C2012_page7_images/image004.eps}
\end{center}
$\gamma$

{\it x}

0

C.

Zadanie 19. $(1pkt)$

Punkt $S=(2,7)$ jest środkiem odcinka $AB$, w którym $A=(-1,3)$. Punkt $B$ ma współrzędne:

A. $B=(5,11)$ B. $B=(\displaystyle \frac{1}{2},2)$ C. $B=(-\displaystyle \frac{3}{2},-5)$ D. $B=(3,11)$

Zadanie 20. $(1pkt)$

$\mathrm{W}$ kolejnych sześciu rzutach kostką otrzymano następujące wyniki: 6, 3, 1, 2, 5, 5. Mediana

tych wynikówjest równa:

A. 3

B. 3,5

C. 4

D. 5

Zadanie 21. $(1pkt)$

RównoŚć $(a+2\sqrt{2})^{2}=a^{2}+28\sqrt{2}+8$ zachodzi dla

A. $a=14$ B. $a=7\sqrt{2}$ C.

$a=7$

D. $a=2\sqrt{2}$

Zadanie 22. (1pkt)

Trójkąt prostokątny o przyprostokątnych 4 i 6 obracamy wokół dłuzszej przyprostokątnej.

Objętość powstałego stozkajest równa

A. $ 96\pi$

B. $ 48\pi$

C. $ 32\pi$

D. $ 8\pi$

Zadanie 23. $(1pkt)$

$\mathrm{J}\mathrm{e}\dot{\mathrm{z}}$ eli $A \mathrm{i} B$ są zdarzeniami losowymi, $B'$ jest zdarzeniem przeciwnym do $B, P(A)=0,3,$

$P(B')=0,4$ oraz $ A\cap B=\emptyset$, to $P(A\cup B)$ jest równe

A. 0,12

B. 0,18

C. 0,6

D. 0,9

Zadanie 24. $(1pkt)$

Przekrój osiowy walca jest kwadratem o boku $a. \mathrm{J}\mathrm{e}\dot{\mathrm{z}}$ eli $r$ oznacza promień podstawy walca,

$h$ oznacza wysokość walca, to

A. $r+h=a$

B.

$h-r=\displaystyle \frac{a}{2}$

C.

{\it r-h}$=$ -{\it a}2

D. $r^{2}+h^{2}=a^{2}$
\end{document}
