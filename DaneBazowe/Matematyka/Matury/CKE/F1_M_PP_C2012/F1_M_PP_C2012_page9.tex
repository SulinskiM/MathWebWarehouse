\documentclass[a4paper,12pt]{article}
\usepackage{latexsym}
\usepackage{amsmath}
\usepackage{amssymb}
\usepackage{graphicx}
\usepackage{wrapfig}
\pagestyle{plain}
\usepackage{fancybox}
\usepackage{bm}

\begin{document}

$ 1\theta$

{\it Egzamin maturalny z matematyki}

{\it Poziom podstawowy}

ZADANIA OTWARTE

{\it Rozwiqzania zadań o numerach od 25. do 34. nalezy zapisać w} $wyznacz\theta nych$ {\it miejscach}

{\it pod treściq zadania}.

Zadanie 25. $(2pkt)$

Rozwiąz nierówność $x^{2}-3x-10<0.$

Odpowiedz:

Zadanie 26. (2pkt)

Średnia wieku w pewnej giupie studentówjest równa 231ata. Średnia wieku tych studentów

i ich opiekunajest równa 241ata. Opiekun ma 391at. Ob1icz, i1u studentówjest w tej giupie.

Odpowiedzí:
\end{document}
