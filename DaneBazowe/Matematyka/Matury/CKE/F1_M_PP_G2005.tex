\documentclass[a4paper,12pt]{article}
\usepackage{latexsym}
\usepackage{amsmath}
\usepackage{amssymb}
\usepackage{graphicx}
\usepackage{wrapfig}
\pagestyle{plain}
\usepackage{fancybox}
\usepackage{bm}

\begin{document}
\begin{center}
\begin{tabular}{l|l}
\multicolumn{1}{l|}{{\it dysleksja}}&	\multicolumn{1}{|l}{}	\\
\hline
\multicolumn{1}{l|}{ $\begin{array}{l}\mbox{MATERIAL DIAGNOSTYCZNY}	\\	\mbox{Z MATEMATYKI}	\\	\mbox{Arkusz I}	\\	\mbox{POZIOM PODSTAWOWY}	\\	\mbox{Czas pracy 120 minut}	\\	\mbox{Instrukcja dla ucznia}	\\	\mbox{1. $\mathrm{S}\mathrm{p}\mathrm{r}\mathrm{a}\mathrm{w}\mathrm{d}\acute{\mathrm{z}}$, czy arkusz zawiera 12 ponumerowanych stron.}	\\	\mbox{Ewentualny brak zgłoś przewodniczącemu zespo}	\\	\mbox{nadzorującego badanie.}	\\	\mbox{2. Rozwiązania i odpowiedzi zapisz w miejscu na to}	\\	\mbox{przeznaczonym.}	\\	\mbox{3. $\mathrm{W}$ rozwiązaniach zadań przedstaw tok rozumowania}	\\	\mbox{prowadzący do ostatecznego wyniku.}	\\	\mbox{4. Pisz czytelnie. Uzywaj długopisu pióra tylko z czamym}	\\	\mbox{tusze atramentem.}	\\	\mbox{5. Nie uzywaj korektora, a błędne zapisy $\mathrm{w}\mathrm{y}\mathrm{r}\mathrm{a}\acute{\mathrm{z}}\mathrm{n}\mathrm{i}\mathrm{e}$ prze eśl.}	\\	\mbox{6. Pamiętaj, $\dot{\mathrm{z}}\mathrm{e}$ zapisy w brudnopisie nie podlegają ocenie.}	\\	\mbox{7. $\mathrm{M}\mathrm{o}\dot{\mathrm{z}}$ esz korzystać z zestawu wzorów matematycznych, cyrkla}	\\	\mbox{i linijki oraz kalkulatora.}	\\	\mbox{8. Wypełnij tę część ka $\mathrm{y}$ odpowiedzi, którą koduje uczeń. Nie}	\\	\mbox{wpisuj $\dot{\mathrm{z}}$ adnych znaków w części przeznaczonej dla}	\\	\mbox{oceniającego.}	\\	\mbox{9. Na karcie odpowiedzi wpisz swoją datę urodzenia i PESEL.}	\\	\mbox{Zamaluj $\blacksquare$ pola odpowiadające cyfrom numeru PESEL. Błędne}	\\	\mbox{zaznaczenie otocz kółkiem \fcircle i zaznacz właściwe.}	\\	\mbox{{\it Zyczymy} $p\theta wodzenia'$}	\end{array}$}&	\multicolumn{1}{|l}{$\begin{array}{l}\mbox{ARKUSZ I}	\\	\mbox{GRUDZIEN}	\\	\mbox{ROK 2005}	\\	\mbox{Za rozwiązanie}	\\	\mbox{wszystkich zadań}	\\	\mbox{mozna otrzymać}	\\	\mbox{łącznie}	\\	\mbox{50 punktów}	\end{array}$}	\\
\hline
\multicolumn{1}{l|}{$\begin{array}{l}\mbox{W ełnia uczeń rzed roz oczęciem rac}	\\	\mbox{PESEL UCZNIA}	\end{array}$}&	\multicolumn{1}{|l}{$\begin{array}{l}\mbox{Wypełnia uczeń}	\\	\mbox{przed rozpoczęciem}	\\	\mbox{pracy}	\\	\mbox{KOD UCZNIA}	\end{array}$}
\end{tabular}


\includegraphics[width=80.724mm,height=12.756mm]{./F1_M_PP_G2005_page0_images/image001.eps}

\includegraphics[width=23.616mm,height=9.852mm]{./F1_M_PP_G2005_page0_images/image002.eps}
\end{center}



{\it 2}

{\it Materialpomocniczy do doskonalenia nauczycieli w zakresie diagnozowania, oceniania i egzaminowania}

{\it Matematyka}- {\it grudzień 2005 r}.

Zadanie l. $(4pkt)$

Wielomian $P(x)=x^{3}-21x+20$ rozłóz na czynniki liniowe, to znaczy zapisz go w postaci

iloczynu trzech wielomianów stopnia pierwszego.





{\it Materialpomocniczy do doskonalenia nauczycieli w zakresie diagnozowania, oceniania i egzaminowania ll}

{\it Matematyka}- {\it grudzień 2005 r}.

Zadanie 10. $(7pkt)$

Pole powierzchni całkowitej prawidłowego ostrosłupa trójkątnego równa

a polejego powierzchni bocznej $96\sqrt{3}$. Oblicz objętość tego ostrosłupa.

się $144\sqrt{3},$





{\it 12 Materiatpomocniczy do doskonalenia nauczycieli w zakresie diagnozowania, oceniania i egzaminowania}

{\it Matematyka}- {\it grudzień 2005 r}.

BRUDNOPIS





{\it Materialpomocniczy do doskonalenia nauczycieli w zakresie diagnozowania, oceniania i egzaminowania}

{\it Matematyka}- {\it grudzień 2005 r}.

{\it 3}

Zadanie 2. (4pkt)

W roku 2005 na uroczystości urodzin zapytano jubi1ata, i1e ma 1at.

Jubilat odpowiedział:,,Jeśli swój wiek sprzed 101at pomnozę przez swój wiek za 111at,

to otrzymam rok mojego urodzenia'' Ułóz odpowiednie równanie, rozwiąz je i zapisz,

w którym roku urodził się tenjubilat.





{\it 4}

{\it Materialpomocniczy do doskonalenia nauczycieli w zakresie diagnozowania, oceniania i egzaminowania}

{\it Matematyka}- {\it grudzień 2005 r}.

Zadanie 3. $(5pkt)$

Funkcja $f(x)$ jest określona wzorem: $f(x)=$

a) Sprawd $\acute{\mathrm{z}}$, czy liczba $a=(0,25)^{-0,5}$ nalezy do dziedziny funkcji $f(x).$

b) Oblicz $f(2)$ oraz $f(3).$

c) Sporządz$\acute{}$ wykres funkcji $f(x).$

d) Podaj rozwiązanie równania $f(x)=0.$

e) Zapisz zbiór wartości funkcji $f(x).$





{\it Materialpomocniczy do doskonalenia nauczycieli w zakresie diagnozowania, oceniania i egzaminowania}

{\it Matematyka}- {\it grudzień 2005 r}.

{\it 5}

Zadanie 4. $(6pkt)$

$\mathrm{W}$ układzie współrzędnych są dane dwa punkty: $A=(-2,2)\mathrm{i}B=(4,4).$

a) Wyznacz równanie prostej $AB.$

b) Prosta $AB$ oraz prosta o równaniu $9x-6y-26=0$ przecinają się w punkcie

Oblicz współrzędne punktu $C.$

c) Wyznacz równanie symetralnej odcinka $AB.$

{\it C}.





{\it 6}

{\it Materialpomocniczy do doskonalenia nauczycieli w zakresie diagnozowania, oceniania i egzaminowania}

{\it Matematyka}- {\it grudzień 2005 r}.

Zadanie 5. $(5pkt)$

Nieskończony ciąg liczbowy $(a_{n})$ jest określony wzorem $a_{n}=4n-31, n=1,2,3,\ldots.$

Wyrazy $a_{k}, a_{k+1}, a_{k+2}$ danego ciągu $(a_{n})$, wzięte w takim porządku, powiększono: wyraz

$a_{k} 01$, wyraz $a_{k+1} 03$ oraz wyraz $a_{k+2}023. \mathrm{W}$ ten sposób otrzymano trzy pierwsze wyrazy

pewnego ciągu geometrycznego. Wyznacz $k$ oraz czwarty wyraz tego ciągu geometrycznego.





{\it Materialpomocniczy do doskonalenia nauczycieli w zakresie diagnozowania, oceniania i egzaminowania}

{\it Matematyka}- {\it grudzień 2005 r}.

7

Zadanie 6. $(4pkt)$

Do szkolnych zawodów szachowych zgłosiło się 16 uczniów, wśród których było dwóch

faworytów. Organizatorzy zawodów zamierzają losowo podzielić szachistów na dwie

jednakowo liczne grupy eliminacyjne, Niebieską i Zółtą. Oblicz prawdopodobieństwo

zdarzenia polegającego na tym, $\dot{\mathrm{z}}\mathrm{e}$ faworyci tych zawodów nie znajdą się w tej samej grupie

eliminacyjnej. Końcowy wynik obliczeń zapisz w postaci ułamka nieskracalnego.





{\it 8}

{\it Materialpomocniczy do doskonalenia nauczycieli w zakresie diagnozowania, oceniania i egzaminowania}

{\it Matematyka}- {\it grudzień 2005 r}.

Zadanie 7. $(3pkt)$

Aby wyznaczyć wszystkie liczby całkowite $c$, dla których liczba postaci $\displaystyle \frac{c-3}{c-5}$ jest takz $\mathrm{e}$

liczbą całkowitą mozna postąpić w następujący sposób:

a) Wyrazenie w liczniku ułamka zapisujemy w postaci sumy, której jednym

ze składnikówjest wyrazenie z mianownika:

$\displaystyle \frac{c-3}{c-5}=\frac{(c-5)+2}{c-5}$

b) Zapisujemy powyzszy ułamek w postaci sumy liczby l oraz pewnego ułamka:

$\displaystyle \frac{c-5+2}{c-5}=\frac{c-5}{c-5}+\frac{2}{c-5}=1+\frac{2}{c-5}$

c) Zauwazamy, $\dot{\mathrm{z}}\mathrm{e}$ ułamek $\displaystyle \frac{2}{c-5}$ jest liczbą całkowitą wtedy i tylko wtedy, gdy liczba

$(c-5)$ jest całkowitym dzielnikiem liczby 2, czy1i $\dot{\mathrm{z}}\mathrm{e}(c-5)\in\{-1,1,-2,2\}.$

d) Rozwiązujemy kolejno równania $c-5=-1, c-5=1, c-5=-2, c-5=2,$

i otrzymujemy odpowiedzí: liczba postaci $\displaystyle \frac{c-3}{c-5}$ jest całkowita dla:

$c=4,c=6,c=3,c=7.$

Rozumując analogicznie, wyznacz wszystkie liczby całkowite $x$, dla których liczba postaci

$\displaystyle \frac{x}{x-3}$ jest liczbą całkowitą.





{\it Materialpomocniczy do doskonalenia nauczycieli w zakresie diagnozowania, oceniania i egzaminowania}

{\it Matematyka}- {\it grudzień 2005 r}.

{\it 9}

Zadanie 8. $(5pkt)$

$\mathrm{W}$ kwadrat ABCD wpisano kwadrat EFGH, jak pokazano na ponizszym rysunku. Wiedząc,

$\dot{\mathrm{z}}\mathrm{e}|AB|=1$ oraz tangens kąta $AEH$ równa się $\displaystyle \frac{2}{5}$, oblicz pole kwadratu EFGH.

{\it A}
\begin{center}
\includegraphics[width=95.304mm,height=93.168mm]{./F1_M_PP_G2005_page8_images/image001.eps}
\end{center}
{\it D  G  C}

{\it F}

{\it H}

{\it E B}





$ 1\theta$ {\it Materiatpomocniczy do doskonalenia nauczycieli w zakresie diagnozowania, oceniania i egzaminowania}

{\it Matematyka}- {\it grudzień 2005} $r.$

Zadanie 9. $(7pkt)$

Liczbę naturalną $t_{n}$ nazywamy $n$ -tą liczbą trójkątn\% $\mathrm{j}\mathrm{e}\dot{\mathrm{z}}$ eli jest ona sumą $n$

kolejnych,

początkowych liczb naturalnych. Liczbami trójkątnymi są zatem: $t_{1}=1, t_{2}=1+2=3,$

$t_{3}=1+2+3=6, t_{4}=1+2+3+4=10, t_{5}=1+2+3+4+5=15$. Stosując tę definicję:

a) wyznacz liczbę $t_{17}.$

b) ułóz odpowiednie równanie i zbadaj, czy liczba $7626$jest liczbą trójkątną.

c) wyznacz największą czterocyfrową liczbę trójkątną.



\end{document}