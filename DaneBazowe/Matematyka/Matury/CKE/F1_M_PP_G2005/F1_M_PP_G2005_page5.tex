\documentclass[a4paper,12pt]{article}
\usepackage{latexsym}
\usepackage{amsmath}
\usepackage{amssymb}
\usepackage{graphicx}
\usepackage{wrapfig}
\pagestyle{plain}
\usepackage{fancybox}
\usepackage{bm}

\begin{document}

{\it 6}

{\it Materialpomocniczy do doskonalenia nauczycieli w zakresie diagnozowania, oceniania i egzaminowania}

{\it Matematyka}- {\it grudzień 2005 r}.

Zadanie 5. $(5pkt)$

Nieskończony ciąg liczbowy $(a_{n})$ jest określony wzorem $a_{n}=4n-31, n=1,2,3,\ldots.$

Wyrazy $a_{k}, a_{k+1}, a_{k+2}$ danego ciągu $(a_{n})$, wzięte w takim porządku, powiększono: wyraz

$a_{k} 01$, wyraz $a_{k+1} 03$ oraz wyraz $a_{k+2}023. \mathrm{W}$ ten sposób otrzymano trzy pierwsze wyrazy

pewnego ciągu geometrycznego. Wyznacz $k$ oraz czwarty wyraz tego ciągu geometrycznego.
\end{document}
