\documentclass[a4paper,12pt]{article}
\usepackage{latexsym}
\usepackage{amsmath}
\usepackage{amssymb}
\usepackage{graphicx}
\usepackage{wrapfig}
\pagestyle{plain}
\usepackage{fancybox}
\usepackage{bm}

\begin{document}

{\it Materialpomocniczy do doskonalenia nauczycieli w zakresie diagnozowania, oceniania i egzaminowania}

{\it Matematyka}- {\it grudzień 2005 r}.

7

Zadanie 6. $(4pkt)$

Do szkolnych zawodów szachowych zgłosiło się 16 uczniów, wśród których było dwóch

faworytów. Organizatorzy zawodów zamierzają losowo podzielić szachistów na dwie

jednakowo liczne grupy eliminacyjne, Niebieską i Zółtą. Oblicz prawdopodobieństwo

zdarzenia polegającego na tym, $\dot{\mathrm{z}}\mathrm{e}$ faworyci tych zawodów nie znajdą się w tej samej grupie

eliminacyjnej. Końcowy wynik obliczeń zapisz w postaci ułamka nieskracalnego.
\end{document}
