\documentclass[a4paper,12pt]{article}
\usepackage{latexsym}
\usepackage{amsmath}
\usepackage{amssymb}
\usepackage{graphicx}
\usepackage{wrapfig}
\pagestyle{plain}
\usepackage{fancybox}
\usepackage{bm}

\begin{document}

{\it 8}

{\it Materialpomocniczy do doskonalenia nauczycieli w zakresie diagnozowania, oceniania i egzaminowania}

{\it Matematyka}- {\it grudzień 2005 r}.

Zadanie 7. $(3pkt)$

Aby wyznaczyć wszystkie liczby całkowite $c$, dla których liczba postaci $\displaystyle \frac{c-3}{c-5}$ jest takz $\mathrm{e}$

liczbą całkowitą mozna postąpić w następujący sposób:

a) Wyrazenie w liczniku ułamka zapisujemy w postaci sumy, której jednym

ze składnikówjest wyrazenie z mianownika:

$\displaystyle \frac{c-3}{c-5}=\frac{(c-5)+2}{c-5}$

b) Zapisujemy powyzszy ułamek w postaci sumy liczby l oraz pewnego ułamka:

$\displaystyle \frac{c-5+2}{c-5}=\frac{c-5}{c-5}+\frac{2}{c-5}=1+\frac{2}{c-5}$

c) Zauwazamy, $\dot{\mathrm{z}}\mathrm{e}$ ułamek $\displaystyle \frac{2}{c-5}$ jest liczbą całkowitą wtedy i tylko wtedy, gdy liczba

$(c-5)$ jest całkowitym dzielnikiem liczby 2, czy1i $\dot{\mathrm{z}}\mathrm{e}(c-5)\in\{-1,1,-2,2\}.$

d) Rozwiązujemy kolejno równania $c-5=-1, c-5=1, c-5=-2, c-5=2,$

i otrzymujemy odpowiedzí: liczba postaci $\displaystyle \frac{c-3}{c-5}$ jest całkowita dla:

$c=4,c=6,c=3,c=7.$

Rozumując analogicznie, wyznacz wszystkie liczby całkowite $x$, dla których liczba postaci

$\displaystyle \frac{x}{x-3}$ jest liczbą całkowitą.
\end{document}
