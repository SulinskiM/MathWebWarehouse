\documentclass[a4paper,12pt]{article}
\usepackage{latexsym}
\usepackage{amsmath}
\usepackage{amssymb}
\usepackage{graphicx}
\usepackage{wrapfig}
\pagestyle{plain}
\usepackage{fancybox}
\usepackage{bm}

\begin{document}

$ 1\theta$ {\it Materiatpomocniczy do doskonalenia nauczycieli w zakresie diagnozowania, oceniania i egzaminowania}

{\it Matematyka}- {\it grudzień 2005} $r.$

Zadanie 9. $(7pkt)$

Liczbę naturalną $t_{n}$ nazywamy $n$ -tą liczbą trójkątn\% $\mathrm{j}\mathrm{e}\dot{\mathrm{z}}$ eli jest ona sumą $n$

kolejnych,

początkowych liczb naturalnych. Liczbami trójkątnymi są zatem: $t_{1}=1, t_{2}=1+2=3,$

$t_{3}=1+2+3=6, t_{4}=1+2+3+4=10, t_{5}=1+2+3+4+5=15$. Stosując tę definicję:

a) wyznacz liczbę $t_{17}.$

b) ułóz odpowiednie równanie i zbadaj, czy liczba $7626$jest liczbą trójkątną.

c) wyznacz największą czterocyfrową liczbę trójkątną.
\end{document}
