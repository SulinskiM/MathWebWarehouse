\documentclass[a4paper,12pt]{article}
\usepackage{latexsym}
\usepackage{amsmath}
\usepackage{amssymb}
\usepackage{graphicx}
\usepackage{wrapfig}
\pagestyle{plain}
\usepackage{fancybox}
\usepackage{bm}

\begin{document}
\begin{center}
\begin{tabular}{l|l}
\multicolumn{1}{l|}{$\begin{array}{l}\mbox{{\it dysleksja}}	\\	\mbox{Miejsce}	\\	\mbox{na na ejkę}	\\	\mbox{z kodem szkoly}	\end{array}$}&	\multicolumn{1}{|l}{}	\\
\hline
\multicolumn{1}{l|}{ $\begin{array}{l}\mbox{PRÓBNY EGZAMIN}	\\	\mbox{MATURALNY}	\\	\mbox{Z MATEMATYKI}	\\	\mbox{POZIOM PODSTAWOWY}	\\	\mbox{Czas pracy 120 minut}	\\	\mbox{Instrukcja dla zdającego}	\\	\mbox{1. $\mathrm{S}\mathrm{p}\mathrm{r}\mathrm{a}\mathrm{w}\mathrm{d}\acute{\mathrm{z}}$, czy arkusz egzaminacyjny zawiera 15 stron}	\\	\mbox{(zadania $1-11$). Ewentualny brak zgłoś przewodniczącemu}	\\	\mbox{zespo nadzorującego egzamin.}	\\	\mbox{2. Rozwiązania zadań i odpowiedzi zamieść w miejscu na to}	\\	\mbox{przeznaczonym.}	\\	\mbox{3. $\mathrm{W}$ rozwiązaniach zadań przedstaw tok rozumowania}	\\	\mbox{prowadzący do ostatecznego wyniku.}	\\	\mbox{4. Pisz czytelnie. Uzywaj długopisu pióra tylko z czamym}	\\	\mbox{tusze atramentem.}	\\	\mbox{5. Nie uzywaj korektora, a błędne zapisy prze eśl.}	\\	\mbox{6. Pamiętaj, $\dot{\mathrm{z}}\mathrm{e}$ zapisy w brudnopisie nie podlegają ocenie.}	\\	\mbox{7. $\mathrm{M}\mathrm{o}\dot{\mathrm{z}}$ esz korzystać z zestawu wzorów matematycznych, cyrkla}	\\	\mbox{i linijki oraz kalkulatora.}	\\	\mbox{8. Wypełnij tę część ka $\mathrm{y}$ odpowiedzi, którą koduje zdający.}	\\	\mbox{Nie wpisuj $\dot{\mathrm{z}}$ adnych znaków w części przeznaczonej dla}	\\	\mbox{egzaminatora.}	\\	\mbox{9. Na karcie odpowiedzi wpisz swoją datę urodzenia i PESEL.}	\\	\mbox{Zamaluj $\blacksquare$ pola odpowiadające cyfrom numeru PESEL. Błędne}	\\	\mbox{zaznaczenie otocz kółkiem $\mathrm{O}$ i zaznacz właściwe.}	\\	\mbox{{\it Zyczymy} $p\theta wodzenia'$}	\end{array}$}&	\multicolumn{1}{|l}{$\begin{array}{l}\mbox{LISTOPAD}	\\	\mbox{ROK 2006}	\\	\mbox{Za rozwiązanie}	\\	\mbox{wszystkich zadań}	\\	\mbox{mozna otrzymać}	\\	\mbox{łącznie}	\\	\mbox{50 punktów}	\end{array}$}	\\
\hline
\multicolumn{1}{l|}{$\begin{array}{l}\mbox{Wypelnia zdający przed}	\\	\mbox{roz oczęciem racy}	\\	\mbox{PESEL ZDAJACEGO}	\end{array}$}&	\multicolumn{1}{|l}{$\begin{array}{l}\mbox{KOD}	\\	\mbox{ZDAJACEGO}	\end{array}$}
\end{tabular}


\includegraphics[width=21.840mm,height=9.852mm]{./F1_M_PP_L2006_page0_images/image001.eps}

\includegraphics[width=78.792mm,height=13.356mm]{./F1_M_PP_L2006_page0_images/image002.eps}
\end{center}



{\it 2}

{\it Próbny egzamin maturalny z matematyki}

{\it Poziom podstawowy}

Zadanie 1. (3pkt)

Wzrost kursu euro w stosunku do złotego spowodował podwyzkę ceny wycieczki

zagranicznej o 5\%. Poniewaz nowa cena nie była zachęcająca, postanowiono obnizyć ją

0 8\%, ustalając cenę promocyjną równą l449 zł. Oblicz pierwotną cenę wycieczki dla

jednego uczestnika.





{\it Próbny egzamin maturalny z matematyki}

{\it Poziom podstawowy}

{\it 11}

Zadanie 9. $(4pkt)$

Nauczyciele informatyki, chcąc wyłonić reprezentację szkoły na wojewódzki konkurs

informatyczny, przeprowadzili w klasach I A i I $\mathrm{B}$ test z zakresu poznanych wiadomości.

$\mathrm{K}\mathrm{a}\dot{\mathrm{z}}\mathrm{d}\mathrm{y}$ z nich przygotował zestawienie wyników swoich uczniów w innej formie.

Na podstawie analizy przedstawionych ponizej wyników obu klas:

a) oblicz średni wynik z testu $\mathrm{k}\mathrm{a}\dot{\mathrm{z}}$ dej klasy,

b) oblicz, ile procent uczniów klasy I $\mathrm{B}$ uzyskało wynik $\mathrm{w}\mathrm{y}\dot{\mathrm{z}}$ szy $\mathrm{n}\mathrm{i}\dot{\mathrm{z}}$ średni w swojej klasie,

c) podaj medianę wyników uzyskanych w klasie I A.

$\mathrm{w}\mathfrak{n}\mathrm{i}\mathrm{k}\mathrm{l}$ testu $\displaystyle \inf \mathrm{u}-\mathrm{i}\acute{\mathrm{r}}$ kl. lA
\begin{center}
\includegraphics[width=98.244mm,height=74.832mm]{./F1_M_PP_L2006_page10_images/image001.eps}
\end{center}
5

4

1

0

0 1

2 3 4 5 6 7 8

1J
\begin{center}
\begin{tabular}{|l|l|}
\hline
\multicolumn{1}{|l|}{Liczba punktów}&	\multicolumn{1}{|l|}{Liczba uczniów}	\\
\hline
\multicolumn{1}{|l|}{$0$}&	\multicolumn{1}{|l|}{ $1$}	\\
\hline
\multicolumn{1}{|l|}{ $1$}&	\multicolumn{1}{|l|}{ $2$}	\\
\hline
\multicolumn{1}{|l|}{ $2$}&	\multicolumn{1}{|l|}{ $1$}	\\
\hline
\multicolumn{1}{|l|}{ $3$}&	\multicolumn{1}{|l|}{ $2$}	\\
\hline
\multicolumn{1}{|l|}{ $4$}&	\multicolumn{1}{|l|}{ $1$}	\\
\hline
\multicolumn{1}{|l|}{ $5$}&	\multicolumn{1}{|l|}{ $2$}	\\
\hline
\multicolumn{1}{|l|}{ $6$}&	\multicolumn{1}{|l|}{ $4$}	\\
\hline
\multicolumn{1}{|l|}{ $7$}&	\multicolumn{1}{|l|}{ $4$}	\\
\hline
\multicolumn{1}{|l|}{ $8$}&	\multicolumn{1}{|l|}{ $1$}	\\
\hline
\multicolumn{1}{|l|}{ $9$}&	\multicolumn{1}{|l|}{ $2$}	\\
\hline
\multicolumn{1}{|l|}{ $10$}&	\multicolumn{1}{|l|}{ $5$}	\\
\hline
\end{tabular}

\end{center}
Wyniki testu informatycznego

uczniów kl. l B.
\begin{center}
\includegraphics[width=195.168mm,height=145.488mm]{./F1_M_PP_L2006_page10_images/image002.eps}
\end{center}




{\it 12}

{\it Próbny egzamin maturalny z matematyki}

{\it Poziom podstawowy}

Zadanie 10. $(6pkt)$

Dane są zbiory:

$A=\{x\in R:|5-x|\geq 3\}, B=\{x\in R:x^{2}-9\geq 0\} \mathrm{i} C=\displaystyle \{x\in R:\frac{x+1}{x-1}\leq 1\}.$

a) Zaznacz na osi liczbowej zbiory $A, B \mathrm{i}C.$

b) Wyznacz i zapisz za pomocą przedziału liczbowego zbiór $C\backslash (A\cap B).$
\begin{center}
\includegraphics[width=192.072mm,height=72.240mm]{./F1_M_PP_L2006_page11_images/image001.eps}
\end{center}
zbiór A
\begin{center}
\includegraphics[width=192.072mm,height=72.288mm]{./F1_M_PP_L2006_page11_images/image002.eps}
\end{center}
zbiór B
\begin{center}
\includegraphics[width=192.072mm,height=72.240mm]{./F1_M_PP_L2006_page11_images/image003.eps}
\end{center}
zbiór C





{\it Próbny egzamin maturalny z matematyki}

{\it Poziom podstawowy}

{\it 13}
\begin{center}
\includegraphics[width=195.168mm,height=290.784mm]{./F1_M_PP_L2006_page12_images/image001.eps}
\end{center}




{\it 14}

{\it Próbny egzamin maturalny z matematyki}

{\it Poziom podstawowy}

Zadanie ll. $(4pkt)$

Funkcja $f$ przyporządkowuje $\mathrm{k}\mathrm{a}\dot{\mathrm{z}}$ dej liczbie rzeczywistej $x$ z przedziału $\langle-4,-2\rangle$ połowę

kwadratu tej liczby pomniejszoną o 8.

a) Podaj wzór tej funkcji.

b) Wyznacz najmniejszą wartość funkcji $f$ w podanym przedziale.
\begin{center}
\includegraphics[width=195.168mm,height=254.412mm]{./F1_M_PP_L2006_page13_images/image001.eps}
\end{center}




{\it Próbny egzamin maturalny z matematyki}

{\it Poziom podstawowy}

{\it 15}

BRUDNOPIS





{\it Próbny egzamin maturalny z matematyki}

{\it Poziom podstawowy}

{\it 3}

Zadanie 2. $(4pkt)$

Dany jest kwadrat o boku długości $a. \mathrm{W}$ prostokącie ABCD bok $AB$ jest dwa razy dłuzszy $\mathrm{n}\mathrm{i}\dot{\mathrm{z}}$

bok kwadratu, a bok $AD$ jest o 2 cm krótszy od boku kwadratu. Po1e tego prostokąta jest

$012\mathrm{c}\mathrm{m}^{2}$ większe od pola kwadratu. Oblicz długość boku kwadratu.
\begin{center}
\includegraphics[width=195.168mm,height=266.544mm]{./F1_M_PP_L2006_page2_images/image001.eps}
\end{center}




{\it 4}

{\it Próbny egzamin maturalny z matematyki}

{\it Poziom podstawowy}

Zadanie 3. (5pkt)

Z prostokąta o szerokości 60 cm wycina się deta1e w kształcie półko1a o promieniu 60 cm.

Sposób wycinania detali ilustruje ponizszy rysunek.

Oblicz najmniejszą długość prostokąta potrzebnego do wycięcia dwóch takich detali. Wynik

zaokrąglij do pełnego centymetra.
\begin{center}
\includegraphics[width=195.168mm,height=218.136mm]{./F1_M_PP_L2006_page3_images/image001.eps}
\end{center}




{\it Próbny egzamin maturalny z matematyki}

{\it Poziom podstawowy}

{\it 5}

Zadanie 4. $(3pkt)$

Wielomian $W(x)=-2x^{4}+5x^{3}+9x^{2}-15x-9$

Wyznacz pierwiastki tego wielomianu.

jest podzielny przez

dwumian $(2x+1).$
\begin{center}
\includegraphics[width=195.168mm,height=266.544mm]{./F1_M_PP_L2006_page4_images/image001.eps}
\end{center}




{\it 6}

{\it Próbny egzamin maturalny z matematyki}

{\it Poziom podstawowy}

Zadanie 5. $(5pkt)$

Dane sąproste o równaniach $2x-y-3=0\mathrm{i}2x-3y-7=0.$

a) Zaznacz w prostokątnym układzie współrzędnych na płaszczyzínie kąt

układem nierówności 

b) Oblicz odległość punktu przecięcia się tych prostych od punktu $S=(3,-8).$

opisany
\begin{center}
\includegraphics[width=165.204mm,height=151.536mm]{./F1_M_PP_L2006_page5_images/image001.eps}
\end{center}
7 J

5

4

3

2

1

{\it x}

$-7  -5$ -$4  -3$ -$2  -1 0 1$ 2  1 2 3 4 5  7

$-1$

$-2$

$-3$

$-4$

$-5$

$-7$
\begin{center}
\includegraphics[width=195.168mm,height=97.080mm]{./F1_M_PP_L2006_page5_images/image002.eps}
\end{center}




{\it Próbny egzamin maturalny z matematyki}

{\it Poziom podstawowy}

7
\begin{center}
\includegraphics[width=195.168mm,height=290.784mm]{./F1_M_PP_L2006_page6_images/image001.eps}
\end{center}




{\it 8}

{\it Próbny egzamin maturalny z matematyki}

{\it Poziom podstawowy}

Zadanie 6. $(5pkt)$

$\mathrm{W}$ utnie znajdują się kule z kolejnymi liczbami 10, 11, 12, 13, 50, przy czym ku1

z liczbą 10 jest 10, ku1 z 1iczbą 11 jest 11 itd., a ku1 z 1iczbą $50$jest 5$0. \mathrm{Z}$ umy tej losujemy

jedną kulę. Oblicz prawdopodobieństwo, $\dot{\mathrm{z}}\mathrm{e}$ wylosujemy kulę z liczbą parzystą.





{\it Próbny egzamin maturalny z matematyki}

{\it Poziom podstawowy}

{\it 9}

Zadanie 7. $(6pkt)$

$\mathrm{W}$ graniastosłupie prawidłowym czworokątnym przekątna podstawy ma długość 8 cm

i tworzy z przekątną ściany bocznej, z którą ma wspólny wierzchołek kąt, którego cosinus

jest równy $\displaystyle \frac{2}{3}$. Oblicz objętość i pole powierzchni całkowitej tego graniastosłupa.
\begin{center}
\includegraphics[width=195.168mm,height=254.460mm]{./F1_M_PP_L2006_page8_images/image001.eps}
\end{center}




$ 1\theta$

{\it Próbny egzamin maturalny z matematyki}

{\it Poziom podstawowy}

Zadanie 8. $(5pkt)$

Dany jest wykres funkcji $y=f(x)$ określonej dla $x\in\langle-6, 6\rangle.$
\begin{center}
\begin{tabular}{|l|l|}
\hline
\multicolumn{1}{|l|}{ $\begin{array}{l}\mbox{$7$}	\\	\mbox{ $6$}	\\	\mbox{ $5$}	\\	\mbox{ $4$}	\\	\mbox{ $3$}	\\	\mbox{ $2$}	\end{array}$}&	\multicolumn{1}{|l|}{ $\mathrm{y}$}	\\
\hline
\multicolumn{1}{|l|}{ $\begin{array}{l}\mbox{-f $-8 -7 -6 -4 -3 -2$ 1}	\\	\mbox{$-1$}	\\	\mbox{ $-2$}	\\	\mbox{ $-3$}	\\	\mbox{ $-4$}	\\	\mbox{ $-5$}	\\	\mbox{ $-6$}	\\	\mbox{ $-7$}	\end{array}$}&	\multicolumn{1}{|l|}{ $2346789$}	\\
\hline
\end{tabular}


\includegraphics[width=35.508mm,height=42.576mm]{./F1_M_PP_L2006_page9_images/image001.eps}

\includegraphics[width=35.760mm,height=42.576mm]{./F1_M_PP_L2006_page9_images/image002.eps}
\end{center}
Korzystając z wykresu ffinkcji zapisz:

a) maksymalne przedziały, w których funkcjajest rosnąca,

b) zbiór argumentów, dla których ffinkcja przyjmuje wartości dodatnie,

c) największąwartość ffinkcji $f$ w przedziale $\langle-5, 5\rangle,$

d) miejsca zerowe ffinkcji $g(x)=f(x-1),$

e) najmniejszą wartość funkcji $h(x)=f(x)+2.$



\end{document}