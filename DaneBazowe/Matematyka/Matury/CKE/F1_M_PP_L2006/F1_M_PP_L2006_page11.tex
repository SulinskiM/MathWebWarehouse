\documentclass[a4paper,12pt]{article}
\usepackage{latexsym}
\usepackage{amsmath}
\usepackage{amssymb}
\usepackage{graphicx}
\usepackage{wrapfig}
\pagestyle{plain}
\usepackage{fancybox}
\usepackage{bm}

\begin{document}

{\it 12}

{\it Próbny egzamin maturalny z matematyki}

{\it Poziom podstawowy}

Zadanie 10. $(6pkt)$

Dane są zbiory:

$A=\{x\in R:|5-x|\geq 3\}, B=\{x\in R:x^{2}-9\geq 0\} \mathrm{i} C=\displaystyle \{x\in R:\frac{x+1}{x-1}\leq 1\}.$

a) Zaznacz na osi liczbowej zbiory $A, B \mathrm{i}C.$

b) Wyznacz i zapisz za pomocą przedziału liczbowego zbiór $C\backslash (A\cap B).$
\begin{center}
\includegraphics[width=192.072mm,height=72.240mm]{./F1_M_PP_L2006_page11_images/image001.eps}
\end{center}
zbiór A
\begin{center}
\includegraphics[width=192.072mm,height=72.288mm]{./F1_M_PP_L2006_page11_images/image002.eps}
\end{center}
zbiór B
\begin{center}
\includegraphics[width=192.072mm,height=72.240mm]{./F1_M_PP_L2006_page11_images/image003.eps}
\end{center}
zbiór C
\end{document}
