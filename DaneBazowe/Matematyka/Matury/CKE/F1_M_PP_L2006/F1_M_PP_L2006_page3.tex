\documentclass[a4paper,12pt]{article}
\usepackage{latexsym}
\usepackage{amsmath}
\usepackage{amssymb}
\usepackage{graphicx}
\usepackage{wrapfig}
\pagestyle{plain}
\usepackage{fancybox}
\usepackage{bm}

\begin{document}

{\it 4}

{\it Próbny egzamin maturalny z matematyki}

{\it Poziom podstawowy}

Zadanie 3. (5pkt)

Z prostokąta o szerokości 60 cm wycina się deta1e w kształcie półko1a o promieniu 60 cm.

Sposób wycinania detali ilustruje ponizszy rysunek.

Oblicz najmniejszą długość prostokąta potrzebnego do wycięcia dwóch takich detali. Wynik

zaokrąglij do pełnego centymetra.
\begin{center}
\includegraphics[width=195.168mm,height=218.136mm]{./F1_M_PP_L2006_page3_images/image001.eps}
\end{center}\end{document}
