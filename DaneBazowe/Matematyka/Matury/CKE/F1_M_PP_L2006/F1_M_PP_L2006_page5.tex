\documentclass[a4paper,12pt]{article}
\usepackage{latexsym}
\usepackage{amsmath}
\usepackage{amssymb}
\usepackage{graphicx}
\usepackage{wrapfig}
\pagestyle{plain}
\usepackage{fancybox}
\usepackage{bm}

\begin{document}

{\it 6}

{\it Próbny egzamin maturalny z matematyki}

{\it Poziom podstawowy}

Zadanie 5. $(5pkt)$

Dane sąproste o równaniach $2x-y-3=0\mathrm{i}2x-3y-7=0.$

a) Zaznacz w prostokątnym układzie współrzędnych na płaszczyzínie kąt

układem nierówności 

b) Oblicz odległość punktu przecięcia się tych prostych od punktu $S=(3,-8).$

opisany
\begin{center}
\includegraphics[width=165.204mm,height=151.536mm]{./F1_M_PP_L2006_page5_images/image001.eps}
\end{center}
7 J

5

4

3

2

1

{\it x}

$-7  -5$ -$4  -3$ -$2  -1 0 1$ 2  1 2 3 4 5  7

$-1$

$-2$

$-3$

$-4$

$-5$

$-7$
\begin{center}
\includegraphics[width=195.168mm,height=97.080mm]{./F1_M_PP_L2006_page5_images/image002.eps}
\end{center}\end{document}
