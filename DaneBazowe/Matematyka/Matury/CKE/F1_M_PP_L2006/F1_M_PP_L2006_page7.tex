\documentclass[a4paper,12pt]{article}
\usepackage{latexsym}
\usepackage{amsmath}
\usepackage{amssymb}
\usepackage{graphicx}
\usepackage{wrapfig}
\pagestyle{plain}
\usepackage{fancybox}
\usepackage{bm}

\begin{document}

{\it 8}

{\it Próbny egzamin maturalny z matematyki}

{\it Poziom podstawowy}

Zadanie 6. $(5pkt)$

$\mathrm{W}$ utnie znajdują się kule z kolejnymi liczbami 10, 11, 12, 13, 50, przy czym ku1

z liczbą 10 jest 10, ku1 z 1iczbą 11 jest 11 itd., a ku1 z 1iczbą $50$jest 5$0. \mathrm{Z}$ umy tej losujemy

jedną kulę. Oblicz prawdopodobieństwo, $\dot{\mathrm{z}}\mathrm{e}$ wylosujemy kulę z liczbą parzystą.
\end{document}
