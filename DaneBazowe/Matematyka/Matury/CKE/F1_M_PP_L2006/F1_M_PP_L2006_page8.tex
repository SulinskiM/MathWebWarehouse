\documentclass[a4paper,12pt]{article}
\usepackage{latexsym}
\usepackage{amsmath}
\usepackage{amssymb}
\usepackage{graphicx}
\usepackage{wrapfig}
\pagestyle{plain}
\usepackage{fancybox}
\usepackage{bm}

\begin{document}

{\it Próbny egzamin maturalny z matematyki}

{\it Poziom podstawowy}

{\it 9}

Zadanie 7. $(6pkt)$

$\mathrm{W}$ graniastosłupie prawidłowym czworokątnym przekątna podstawy ma długość 8 cm

i tworzy z przekątną ściany bocznej, z którą ma wspólny wierzchołek kąt, którego cosinus

jest równy $\displaystyle \frac{2}{3}$. Oblicz objętość i pole powierzchni całkowitej tego graniastosłupa.
\begin{center}
\includegraphics[width=195.168mm,height=254.460mm]{./F1_M_PP_L2006_page8_images/image001.eps}
\end{center}\end{document}
