\documentclass[a4paper,12pt]{article}
\usepackage{latexsym}
\usepackage{amsmath}
\usepackage{amssymb}
\usepackage{graphicx}
\usepackage{wrapfig}
\pagestyle{plain}
\usepackage{fancybox}
\usepackage{bm}

\begin{document}

$ 1\theta$

{\it Próbny egzamin maturalny z matematyki}

{\it Poziom podstawowy}

Zadanie 8. $(5pkt)$

Dany jest wykres funkcji $y=f(x)$ określonej dla $x\in\langle-6, 6\rangle.$
\begin{center}
\begin{tabular}{|l|l|}
\hline
\multicolumn{1}{|l|}{ $\begin{array}{l}\mbox{$7$}	\\	\mbox{ $6$}	\\	\mbox{ $5$}	\\	\mbox{ $4$}	\\	\mbox{ $3$}	\\	\mbox{ $2$}	\end{array}$}&	\multicolumn{1}{|l|}{ $\mathrm{y}$}	\\
\hline
\multicolumn{1}{|l|}{ $\begin{array}{l}\mbox{-f $-8 -7 -6 -4 -3 -2$ 1}	\\	\mbox{$-1$}	\\	\mbox{ $-2$}	\\	\mbox{ $-3$}	\\	\mbox{ $-4$}	\\	\mbox{ $-5$}	\\	\mbox{ $-6$}	\\	\mbox{ $-7$}	\end{array}$}&	\multicolumn{1}{|l|}{ $2346789$}	\\
\hline
\end{tabular}


\includegraphics[width=35.508mm,height=42.576mm]{./F1_M_PP_L2006_page9_images/image001.eps}

\includegraphics[width=35.760mm,height=42.576mm]{./F1_M_PP_L2006_page9_images/image002.eps}
\end{center}
Korzystając z wykresu ffinkcji zapisz:

a) maksymalne przedziały, w których funkcjajest rosnąca,

b) zbiór argumentów, dla których ffinkcja przyjmuje wartości dodatnie,

c) największąwartość ffinkcji $f$ w przedziale $\langle-5, 5\rangle,$

d) miejsca zerowe ffinkcji $g(x)=f(x-1),$

e) najmniejszą wartość funkcji $h(x)=f(x)+2.$
\end{document}
