\documentclass[a4paper,12pt]{article}
\usepackage{latexsym}
\usepackage{amsmath}
\usepackage{amssymb}
\usepackage{graphicx}
\usepackage{wrapfig}
\pagestyle{plain}
\usepackage{fancybox}
\usepackage{bm}

\begin{document}
\begin{center}
\includegraphics[width=7.212mm,height=14.220mm]{./F1_M_PP_L2009_page0_images/image001.eps}

\includegraphics[width=25.140mm,height=9.900mm]{./F1_M_PP_L2009_page0_images/image002.eps}
\end{center}
Centralna

Komisja

Egzaminacyjna

APiTAtL z l

NA O OWAST A Eclk l

Materiał współfmansowany ze środków Unii Europejskiej

w ramach Europejskiego Funduszu Społecznego

$\displaystyle \mathrm{F}\cup \mathrm{N}\mathrm{O}\cup \mathrm{s}\mathrm{z}\mathrm{o}r\mathrm{s}_{\mathrm{n}}\bigcup_{\mathrm{S}\mathrm{P}}^{\mathrm{N}1\mathrm{A}\mathrm{E}\cup}\mathrm{R}\mathrm{O}_{\varsigma \mathrm{z}\mathrm{N}?}$
\begin{center}
\includegraphics[width=20.772mm,height=13.920mm]{./F1_M_PP_L2009_page0_images/image003.eps}

\begin{tabular}{|l|l|l}
\cline{1-1}
\multicolumn{1}{|l|}{$\begin{array}{l}\mbox{Miejsce}	\\	\mbox{na na ejkę}	\end{array}$}&	\multicolumn{1}{|l|}{$\begin{array}{l}\mbox{{\it ARKUSZ ZA WIE}}	\\	\mbox{{\it INFO ACJE}}	\\	\mbox{{\it P WNIE CHRONIONE}}	\\	\mbox{{\it DO MOMENTU}}	\\	\mbox{{\it ROZPOCZĘCIA}}	\\	\mbox{{\it EGZAMINU}.'}	\end{array}$}&	\multicolumn{1}{|l}{ $\mathrm{M}\mathrm{M}\mathrm{A}-\mathrm{P}1_{-}1\mathrm{P}-095$}	\\
\hline
&	\multicolumn{1}{|l}{$\begin{array}{l}\mbox{LISTOPAD}	\\	\mbox{ROK 2009}	\\	\mbox{Za rozwiązanie}	\\	\mbox{wszystkich zadań}	\\	\mbox{mozna otrzymać}	\\	\mbox{łącznie}	\\	\mbox{50 punktów}	\end{array}$}	\\
\cline{3-3}
&	\multicolumn{1}{|l}{$\begin{array}{l}\mbox{KOD}	\\	\mbox{ZDAJACEGO}	\end{array}$}
\end{tabular}


\includegraphics[width=21.840mm,height=9.852mm]{./F1_M_PP_L2009_page0_images/image004.eps}

\includegraphics[width=78.744mm,height=13.308mm]{./F1_M_PP_L2009_page0_images/image005.eps}
\end{center}



{\it 2}

{\it Próbny egzamin maturalny z matematyki}

{\it Poziom podstawowy}

ZADANIA ZAMKNIĘTE

$W$ {\it zadaniach} $\theta d1. d_{\theta}25$. {\it wybierz i zaznacz na karcie} $\theta dp\theta${\it wiedzijednq}

{\it poprawnq odpowied} $\acute{z}.$

Zadanie l. $(1pkt)$

Wskaz nierówność, która opisuje sumę przedziałów zaznaczonych na osi liczbowej.
\begin{center}
\includegraphics[width=174.852mm,height=13.416mm]{./F1_M_PP_L2009_page1_images/image001.eps}
\end{center}
$-2$  6  {\it x}

A. $|x-2|>4$

B. $|x-2|<4$

C. $|x-4|<2$

D. $|x-4|>2$

Zadanie 2. (1pkt)

Na seans filmowy sprzedano 280 bi1etów, w tym 126 u1gowych. Jaki procent sprzedanych

biletów stanowiły bilety ulgowe?

A. 22\%

B. 33\%

Zadanie 3. (1pkt)

6\% 1iczby x jest równe 9. Wtedy

A. $x=240$

B. $x=150$

Zadanie 4. $(1pkt)$

Iloraz $32^{-3}$ : $(\displaystyle \frac{1}{8})^{4}$ jest równy

A. $2^{-27}$ B. $2^{-3}$

Zadanie 5. $(1pkt)$

$\mathrm{O}$ liczbie $x$ wiadomo, $\dot{\mathrm{z}}\mathrm{e}\log_{3}x=9$. Zatem

A. {\it x}$=$2 B. {\it x}$=- 21$

Zadanie 6. $(1pkt)$

Wyrazenie $27x^{3}+y^{3}$ jest równe iloczynowi

A.

B.

C.

D.

$(3x+y)(9x^{2}-3xy+y^{2})$

$(3x+y)(9x^{2}+3xy+y^{2})$

$(3x-y)(9x^{2}+3xy+y^{2})$

$(3x-y)(9x^{2}-3xy+y^{2})$

C. 45\%

D. 63\%

C. $x=24$

D. $x=15$

C. $2^{3}$

D. $2^{27}$

C. $x=3^{9}$

D. $x=9^{3}$

Zadanie 7. $(1pkt)$

Dane sąwielomiany: $W(x)=x^{3}-3x+1$ oraz $V(x)=2x^{3}$. Wielomian $W(x)\cdot\nabla(x)$ jest równy

A. $2x^{5}-6x^{4}+2x^{3}$

B. $2x^{6}-6x^{4}+2x^{3}$

C. $2x^{5}+3x+1$

D. $2x^{5}+6x^{4}+2x^{3}$





{\it Próbny egzamin maturalny z matematyki}

{\it Poziom podstawowy}

{\it 11}

Zadanie 28. $(2pkt)$

$\mathrm{W}$ układzie współrzędnych na płaszczyzínie punkty $A=(2,5)$ i $\mathrm{C}=(6,7)$ są przeciwległymi

wierzchołkami kwadratu ABCD. Wyznacz równanie prostej $BD.$

Odpowied $\acute{\mathrm{z}}$:

Zadanie 29. $(2pkt)$

Kąt $a$ jest ostry i $\displaystyle \mathrm{t}\mathrm{g}\alpha=\frac{4}{3}$. Oblicz $\sin\alpha+\cos\alpha.$

Odpowiedzí :





{\it 12}

{\it Próbny egzamin maturalny z matematyki}

{\it Poziom podstawowy}

Zadanie 30. $(2pkt)$

Wykaz, $\dot{\mathrm{z}}\mathrm{e}$ dla $\mathrm{k}\mathrm{a}\dot{\mathrm{z}}$ dego $m$ ciąg $(\displaystyle \frac{m+1}{4},\frac{m+3}{6},\frac{m+9}{12})$ jest arytmetyczny.





{\it Próbny egzamin maturalny z matematyki}

{\it Poziom podstawowy}

{\it 13}

Zadanie 31. $(2pkt)$

Trójkąty $ABC\mathrm{i}CDE$ są równoboczne. Punkty $A, C\mathrm{i}E$ lez$\cdot$ą najednej prostej. Punkty $K, L\mathrm{i}M$

są środkami odcinków $AC$, {\it CE} $\mathrm{i} BD$ (zobacz rysunek). Wykaz, $\dot{\mathrm{z}}\mathrm{e}$ punkty $K, L \mathrm{i} M$

są wierzchołkami trójkąta równobocznego.
\begin{center}
\includegraphics[width=116.640mm,height=65.124mm]{./F1_M_PP_L2009_page12_images/image001.eps}
\end{center}
{\it D}

{\it M}

{\it B}

{\it A  E}

{\it K C  L}





{\it 14}

{\it Próbny egzamin maturalny z matematyki}

{\it Poziom podstawowy}

Zadanie 32. $(5pkt)$

Uczeń przeczytał ksiązkę liczącą480 stron, przy czym $\mathrm{k}\mathrm{a}\dot{\mathrm{z}}$ dego dnia czytał jednakową liczbę

stron. Gdyby czytał $\mathrm{k}\mathrm{a}\dot{\mathrm{z}}$ dego dnia o 8 stron więcej, to przeczytałby tę ksiązkę o 3 dni

wcześniej. Oblicz, ile dni uczeń czytał tę ksiązkę.

Odpowiedzí:





{\it Próbny egzamin maturalny z matematyki}

{\it Poziom podstawowy}

{\it 15}

Zadanie 33. $(4pkt)$

Punkty $A=(2,0) \mathrm{i} B=(12,0)$ są wierzchołkami trójkąta prostokątnego $ABC$

o przeciwprostokątnej $AB$. Wierzchołek $C$ lezy na prostej o równaniu $y=x$. Oblicz

współrzędne punktu $C.$

Odpowiedzí :





{\it 16}

{\it Próbny egzamin maturalny z matematyki}

{\it Poziom podstawowy}

Zadanie 34. $(4pkt)$

Pole trójkąta prostokątnego jest równe 60 $\mathrm{c}\mathrm{m}^{2}$ Jedna przyprostokątna jest o 7 cm diuzsza

od drugiej. Oblicz długość przeciwprostokątnej tego trójkąta.

Odpowiedzí:





{\it Próbny egzamin maturalny z matematyki}

{\it Poziom podstawowy}

{\it 1}7

BRUDNOPIS





{\it Próbny egzamin maturalny z matematyki}

{\it Poziom podstawowy}

{\it 3}

BRUDNOPIS





{\it 4}

{\it Próbny egzamin maturalny z matematyki}

{\it Poziom podstawowy}

Zadanie 8. $(1pkt)$

Wierzchołek paraboli o równaniu $y=-3(x+1)^{2}$ ma współrzędne

A. $(-1,0)$ B. $(0,-1)$ C. $($1, $0)$

D. (0,1)

Zadanie 9. $(1pkt)$

Do wykresu funkcji $f(x)=x^{2}+x-2$ nalezy punkt

A. $(-1,-4)$

B. $(-1,1)$

C. $(-1,-1)$

D. $(-1,-2)$

Zadanie 10. $(1pkt)$

Rozwiązaniem równania $\displaystyle \frac{x-5}{x+3}=\frac{2}{3}$ jest liczba

A. 21 B. 7

C.

$\displaystyle \frac{17}{3}$

D. 0

Zadanie ll. $(1pkt)$

Zbiór rozwiązań nierównoŚci $(x+1)(x-3)>0$ przedstawionyjest na rysunku
\begin{center}
\includegraphics[width=170.988mm,height=15.804mm]{./F1_M_PP_L2009_page3_images/image001.eps}
\end{center}
$-1$  3  {\it x}

A.
\begin{center}
\includegraphics[width=171.756mm,height=13.716mm]{./F1_M_PP_L2009_page3_images/image002.eps}
\end{center}
{\it x}

1

$-3$

B.
\begin{center}
\includegraphics[width=171.048mm,height=15.852mm]{./F1_M_PP_L2009_page3_images/image003.eps}
\end{center}
$-1$  3  {\it x}

C.
\begin{center}
\includegraphics[width=171.756mm,height=13.716mm]{./F1_M_PP_L2009_page3_images/image004.eps}
\end{center}
{\it x}

1

$-3$

D.

Zadanie 12. $(1pkt)$

Dla $ n=1,2,3,\ldots$ ciąg $(a_{n})$ jest określony wzorem: $a_{n}=(-1)^{n}\cdot(3-n)$. Wtedy

A. $a_{3}<0$

B. $a_{3}=0$

C. $a_{3}=1$

D. $a_{3}>1$

Zadanie 13. (1pkt)

W ciągu arytmetycznym trzeci wyraz jest równy 14, ajedenasty jest równy 34. Róznica tego

ciągu jest równa

A. 9 B. -25 C. 2 D. -25

Zadanie 14. $(1pkt)$

$\mathrm{W}$ ciągu geometrycznym $(a_{n})$ dane są: $a_{1}=32 \mathrm{i}a_{4}=-4$. Iloraz tego ciągujest równy

A. 12 B. $\displaystyle \frac{1}{2}$ C. $-\displaystyle \frac{1}{2}$ D. $-12$





{\it Próbny egzamin maturalny z matematyki}

{\it Poziom podstawowy}

{\it 5}

BRUDNOPIS





{\it 6}

{\it Próbny egzamin maturalny z matematyki}

{\it Poziom podstawowy}

Zadanie 15. $(1pkt)$

Kąt $\alpha$ jest ostry i $\displaystyle \sin\alpha=\frac{8}{9}$. Wtedy $\cos\alpha$ jest równy

A. -91 B. -98 C. --$\sqrt{}$917

D.

$\displaystyle \frac{\sqrt{65}}{9}$

Zadanie 16. $(1pkt)$

Danyjest trójkąt prostokątny (patrz rysunek). Wtedy tg $\alpha$ jest równy
\begin{center}
\includegraphics[width=57.660mm,height=36.420mm]{./F1_M_PP_L2009_page5_images/image001.eps}
\end{center}
$\sqrt{3}$

1

$\alpha$

A. $\sqrt{2}$

B.

$\sqrt{2}$

$\sqrt{3}$

$\sqrt{2}$

C.

$\sqrt{3}$

$\sqrt{2}$

D.

-$\sqrt{}$12

Zadanie 17. (1pkt)

W trójkącie równoramiennym ABC dane są

opuszczona z wierzchołka C jest równa

$|AC|=|BC|=7$

oraz

$|AB|=12.$

Wysokość

A. $\sqrt{13}$

B. $\sqrt{5}$

C. l

D. 5

Zadanie 18. $(1pkt)$

Oblicz $\mathrm{d}$ gość odcinka $AE$ wiedząc, $\dot{\mathrm{z}}\mathrm{e}AB||CD \mathrm{i} AB=6, AC=4, CD=8.$
\begin{center}
\includegraphics[width=102.108mm,height=50.040mm]{./F1_M_PP_L2009_page5_images/image002.eps}
\end{center}
{\it D}

{\it B}

8

6

{\it E  A}  4  {\it C}

A.

$|AE|=2$

B.

$|AE|=4$

C.

$|AE|=6$

D.

$|AE|=12$

Zadanie 19. $(1pkt)$

Dane sąpunkty $A=(-2,3)$ oraz $B=(4,6)$. Długość odcinka $AB$ jest równa

A. $\sqrt{208}$

B. $\sqrt{52}$

C. $\sqrt{45}$

D. $\sqrt{40}$

Zadanie 20. $(1pkt)$

Promień okręgu o równaniu $(x-1)^{2}+y^{2}=16$ jest równy

A. l

B. 2

C. 3

D. 4





{\it Próbny egzamin maturalny z matematyki}

{\it Poziom podstawowy}

7

BRUDNOPIS





{\it 8}

{\it Próbny egzamin maturalny z matematyki}

{\it Poziom podstawowy}

Zadanie 21. $(1pkt)$

Wykres ffinkcji liniowej określonej wzorem $f(x)=3x+2$ jest prostą prostopadłą do prostej

o równaniu:

A. $y=-\displaystyle \frac{1}{3}x-1$ B. $y=\displaystyle \frac{1}{3}x+1$ C. $y=3x+1$ D. $y=3x-1$

Zadanie 22. $(1pkt)$

Prosta o równaniu $y=-4x+(2m-7)$ przechodzi przez punkt $A=(2,-1)$. Wtedy

A. $m=7$

B.

{\it m}$=$2 -21

C.

{\it m}$=$ - -21

D. $m=-17$

Zadanie 23. $(1pkt)$

Pole powierzchni całkowitej sześcianu jest równe 150 $\mathrm{c}\mathrm{m}^{2}$ Długość krawędzi tego sześcianu

jest równa

A. 3,5 cm

B. 4 cm

C. 4,5 cm

D. 5 cm

Zadanie 24. (1pkt)

Średnia arytmetyczna pięciu liczb: 5, x, 1, 3, 1 jest równa 3. Wtedy

A. $x=2$

B. $x=3$

C. $x=4$

D. $x=5$

Zadanie 25. $(1pkt)$

Wybieramy liczbę $a$ ze zbioru $A=\{2,3,4,5\}$ oraz liczbę $b$ ze zbioru $B=\{1,4\}$. Ilejest takich par

$(a,b), \dot{\mathrm{z}}\mathrm{e}$ iloczyn $a\cdot b$ jest liczbą nieparzystą?

A. 2

B. 3

C. 5

D. 20





{\it Próbny egzamin maturalny z matematyki}

{\it Poziom podstawowy}

{\it 9}

BRUDNOPIS





$ 1\theta$

{\it Próbny egzamin maturalny z matematyki}

{\it Poziom podstawowy}

ZADANIA OTWARTE

{\it Rozwiqzania zadań o numerach od 26. do 34. nalezy zapisać w} $wyznacz\theta nych$ {\it miejscach}

{\it pod treściq zadania}.

Zadanie 26. $(2pkt)$

Rozwiąz nierówność $x^{2}-3x+2\leq 0.$

Odpowiedzí:

Zadanie 27. $(2pkt)$

Rozwiąz równanie $x^{3}-7x^{2}+2x-14=0.$

Odpowied $\acute{\mathrm{z}}$:



\end{document}