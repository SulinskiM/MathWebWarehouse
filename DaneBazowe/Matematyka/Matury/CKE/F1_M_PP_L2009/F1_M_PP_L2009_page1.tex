\documentclass[a4paper,12pt]{article}
\usepackage{latexsym}
\usepackage{amsmath}
\usepackage{amssymb}
\usepackage{graphicx}
\usepackage{wrapfig}
\pagestyle{plain}
\usepackage{fancybox}
\usepackage{bm}

\begin{document}

{\it 2}

{\it Próbny egzamin maturalny z matematyki}

{\it Poziom podstawowy}

ZADANIA ZAMKNIĘTE

$W$ {\it zadaniach} $\theta d1. d_{\theta}25$. {\it wybierz i zaznacz na karcie} $\theta dp\theta${\it wiedzijednq}

{\it poprawnq odpowied} $\acute{z}.$

Zadanie l. $(1pkt)$

Wskaz nierówność, która opisuje sumę przedziałów zaznaczonych na osi liczbowej.
\begin{center}
\includegraphics[width=174.852mm,height=13.416mm]{./F1_M_PP_L2009_page1_images/image001.eps}
\end{center}
$-2$  6  {\it x}

A. $|x-2|>4$

B. $|x-2|<4$

C. $|x-4|<2$

D. $|x-4|>2$

Zadanie 2. (1pkt)

Na seans filmowy sprzedano 280 bi1etów, w tym 126 u1gowych. Jaki procent sprzedanych

biletów stanowiły bilety ulgowe?

A. 22\%

B. 33\%

Zadanie 3. (1pkt)

6\% 1iczby x jest równe 9. Wtedy

A. $x=240$

B. $x=150$

Zadanie 4. $(1pkt)$

Iloraz $32^{-3}$ : $(\displaystyle \frac{1}{8})^{4}$ jest równy

A. $2^{-27}$ B. $2^{-3}$

Zadanie 5. $(1pkt)$

$\mathrm{O}$ liczbie $x$ wiadomo, $\dot{\mathrm{z}}\mathrm{e}\log_{3}x=9$. Zatem

A. {\it x}$=$2 B. {\it x}$=- 21$

Zadanie 6. $(1pkt)$

Wyrazenie $27x^{3}+y^{3}$ jest równe iloczynowi

A.

B.

C.

D.

$(3x+y)(9x^{2}-3xy+y^{2})$

$(3x+y)(9x^{2}+3xy+y^{2})$

$(3x-y)(9x^{2}+3xy+y^{2})$

$(3x-y)(9x^{2}-3xy+y^{2})$

C. 45\%

D. 63\%

C. $x=24$

D. $x=15$

C. $2^{3}$

D. $2^{27}$

C. $x=3^{9}$

D. $x=9^{3}$

Zadanie 7. $(1pkt)$

Dane sąwielomiany: $W(x)=x^{3}-3x+1$ oraz $V(x)=2x^{3}$. Wielomian $W(x)\cdot\nabla(x)$ jest równy

A. $2x^{5}-6x^{4}+2x^{3}$

B. $2x^{6}-6x^{4}+2x^{3}$

C. $2x^{5}+3x+1$

D. $2x^{5}+6x^{4}+2x^{3}$
\end{document}
