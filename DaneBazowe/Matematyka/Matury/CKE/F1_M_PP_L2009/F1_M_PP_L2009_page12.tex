\documentclass[a4paper,12pt]{article}
\usepackage{latexsym}
\usepackage{amsmath}
\usepackage{amssymb}
\usepackage{graphicx}
\usepackage{wrapfig}
\pagestyle{plain}
\usepackage{fancybox}
\usepackage{bm}

\begin{document}

{\it Próbny egzamin maturalny z matematyki}

{\it Poziom podstawowy}

{\it 13}

Zadanie 31. $(2pkt)$

Trójkąty $ABC\mathrm{i}CDE$ są równoboczne. Punkty $A, C\mathrm{i}E$ lez$\cdot$ą najednej prostej. Punkty $K, L\mathrm{i}M$

są środkami odcinków $AC$, {\it CE} $\mathrm{i} BD$ (zobacz rysunek). Wykaz, $\dot{\mathrm{z}}\mathrm{e}$ punkty $K, L \mathrm{i} M$

są wierzchołkami trójkąta równobocznego.
\begin{center}
\includegraphics[width=116.640mm,height=65.124mm]{./F1_M_PP_L2009_page12_images/image001.eps}
\end{center}
{\it D}

{\it M}

{\it B}

{\it A  E}

{\it K C  L}
\end{document}
