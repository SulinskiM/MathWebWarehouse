\documentclass[a4paper,12pt]{article}
\usepackage{latexsym}
\usepackage{amsmath}
\usepackage{amssymb}
\usepackage{graphicx}
\usepackage{wrapfig}
\pagestyle{plain}
\usepackage{fancybox}
\usepackage{bm}

\begin{document}

{\it 14}

{\it Próbny egzamin maturalny z matematyki}

{\it Poziom podstawowy}

Zadanie 32. $(5pkt)$

Uczeń przeczytał ksiązkę liczącą480 stron, przy czym $\mathrm{k}\mathrm{a}\dot{\mathrm{z}}$ dego dnia czytał jednakową liczbę

stron. Gdyby czytał $\mathrm{k}\mathrm{a}\dot{\mathrm{z}}$ dego dnia o 8 stron więcej, to przeczytałby tę ksiązkę o 3 dni

wcześniej. Oblicz, ile dni uczeń czytał tę ksiązkę.

Odpowiedzí:
\end{document}
