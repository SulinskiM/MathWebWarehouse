\documentclass[a4paper,12pt]{article}
\usepackage{latexsym}
\usepackage{amsmath}
\usepackage{amssymb}
\usepackage{graphicx}
\usepackage{wrapfig}
\pagestyle{plain}
\usepackage{fancybox}
\usepackage{bm}

\begin{document}

{\it 4}

{\it Próbny egzamin maturalny z matematyki}

{\it Poziom podstawowy}

Zadanie 8. $(1pkt)$

Wierzchołek paraboli o równaniu $y=-3(x+1)^{2}$ ma współrzędne

A. $(-1,0)$ B. $(0,-1)$ C. $($1, $0)$

D. (0,1)

Zadanie 9. $(1pkt)$

Do wykresu funkcji $f(x)=x^{2}+x-2$ nalezy punkt

A. $(-1,-4)$

B. $(-1,1)$

C. $(-1,-1)$

D. $(-1,-2)$

Zadanie 10. $(1pkt)$

Rozwiązaniem równania $\displaystyle \frac{x-5}{x+3}=\frac{2}{3}$ jest liczba

A. 21 B. 7

C.

$\displaystyle \frac{17}{3}$

D. 0

Zadanie ll. $(1pkt)$

Zbiór rozwiązań nierównoŚci $(x+1)(x-3)>0$ przedstawionyjest na rysunku
\begin{center}
\includegraphics[width=170.988mm,height=15.804mm]{./F1_M_PP_L2009_page3_images/image001.eps}
\end{center}
$-1$  3  {\it x}

A.
\begin{center}
\includegraphics[width=171.756mm,height=13.716mm]{./F1_M_PP_L2009_page3_images/image002.eps}
\end{center}
{\it x}

1

$-3$

B.
\begin{center}
\includegraphics[width=171.048mm,height=15.852mm]{./F1_M_PP_L2009_page3_images/image003.eps}
\end{center}
$-1$  3  {\it x}

C.
\begin{center}
\includegraphics[width=171.756mm,height=13.716mm]{./F1_M_PP_L2009_page3_images/image004.eps}
\end{center}
{\it x}

1

$-3$

D.

Zadanie 12. $(1pkt)$

Dla $ n=1,2,3,\ldots$ ciąg $(a_{n})$ jest określony wzorem: $a_{n}=(-1)^{n}\cdot(3-n)$. Wtedy

A. $a_{3}<0$

B. $a_{3}=0$

C. $a_{3}=1$

D. $a_{3}>1$

Zadanie 13. (1pkt)

W ciągu arytmetycznym trzeci wyraz jest równy 14, ajedenasty jest równy 34. Róznica tego

ciągu jest równa

A. 9 B. -25 C. 2 D. -25

Zadanie 14. $(1pkt)$

$\mathrm{W}$ ciągu geometrycznym $(a_{n})$ dane są: $a_{1}=32 \mathrm{i}a_{4}=-4$. Iloraz tego ciągujest równy

A. 12 B. $\displaystyle \frac{1}{2}$ C. $-\displaystyle \frac{1}{2}$ D. $-12$
\end{document}
