\documentclass[a4paper,12pt]{article}
\usepackage{latexsym}
\usepackage{amsmath}
\usepackage{amssymb}
\usepackage{graphicx}
\usepackage{wrapfig}
\pagestyle{plain}
\usepackage{fancybox}
\usepackage{bm}

\begin{document}

{\it 6}

{\it Próbny egzamin maturalny z matematyki}

{\it Poziom podstawowy}

Zadanie 15. $(1pkt)$

Kąt $\alpha$ jest ostry i $\displaystyle \sin\alpha=\frac{8}{9}$. Wtedy $\cos\alpha$ jest równy

A. -91 B. -98 C. --$\sqrt{}$917

D.

$\displaystyle \frac{\sqrt{65}}{9}$

Zadanie 16. $(1pkt)$

Danyjest trójkąt prostokątny (patrz rysunek). Wtedy tg $\alpha$ jest równy
\begin{center}
\includegraphics[width=57.660mm,height=36.420mm]{./F1_M_PP_L2009_page5_images/image001.eps}
\end{center}
$\sqrt{3}$

1

$\alpha$

A. $\sqrt{2}$

B.

$\sqrt{2}$

$\sqrt{3}$

$\sqrt{2}$

C.

$\sqrt{3}$

$\sqrt{2}$

D.

-$\sqrt{}$12

Zadanie 17. (1pkt)

W trójkącie równoramiennym ABC dane są

opuszczona z wierzchołka C jest równa

$|AC|=|BC|=7$

oraz

$|AB|=12.$

Wysokość

A. $\sqrt{13}$

B. $\sqrt{5}$

C. l

D. 5

Zadanie 18. $(1pkt)$

Oblicz $\mathrm{d}$ gość odcinka $AE$ wiedząc, $\dot{\mathrm{z}}\mathrm{e}AB||CD \mathrm{i} AB=6, AC=4, CD=8.$
\begin{center}
\includegraphics[width=102.108mm,height=50.040mm]{./F1_M_PP_L2009_page5_images/image002.eps}
\end{center}
{\it D}

{\it B}

8

6

{\it E  A}  4  {\it C}

A.

$|AE|=2$

B.

$|AE|=4$

C.

$|AE|=6$

D.

$|AE|=12$

Zadanie 19. $(1pkt)$

Dane sąpunkty $A=(-2,3)$ oraz $B=(4,6)$. Długość odcinka $AB$ jest równa

A. $\sqrt{208}$

B. $\sqrt{52}$

C. $\sqrt{45}$

D. $\sqrt{40}$

Zadanie 20. $(1pkt)$

Promień okręgu o równaniu $(x-1)^{2}+y^{2}=16$ jest równy

A. l

B. 2

C. 3

D. 4
\end{document}
