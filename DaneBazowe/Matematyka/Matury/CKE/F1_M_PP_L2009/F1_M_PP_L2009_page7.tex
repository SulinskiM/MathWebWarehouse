\documentclass[a4paper,12pt]{article}
\usepackage{latexsym}
\usepackage{amsmath}
\usepackage{amssymb}
\usepackage{graphicx}
\usepackage{wrapfig}
\pagestyle{plain}
\usepackage{fancybox}
\usepackage{bm}

\begin{document}

{\it 8}

{\it Próbny egzamin maturalny z matematyki}

{\it Poziom podstawowy}

Zadanie 21. $(1pkt)$

Wykres ffinkcji liniowej określonej wzorem $f(x)=3x+2$ jest prostą prostopadłą do prostej

o równaniu:

A. $y=-\displaystyle \frac{1}{3}x-1$ B. $y=\displaystyle \frac{1}{3}x+1$ C. $y=3x+1$ D. $y=3x-1$

Zadanie 22. $(1pkt)$

Prosta o równaniu $y=-4x+(2m-7)$ przechodzi przez punkt $A=(2,-1)$. Wtedy

A. $m=7$

B.

{\it m}$=$2 -21

C.

{\it m}$=$ - -21

D. $m=-17$

Zadanie 23. $(1pkt)$

Pole powierzchni całkowitej sześcianu jest równe 150 $\mathrm{c}\mathrm{m}^{2}$ Długość krawędzi tego sześcianu

jest równa

A. 3,5 cm

B. 4 cm

C. 4,5 cm

D. 5 cm

Zadanie 24. (1pkt)

Średnia arytmetyczna pięciu liczb: 5, x, 1, 3, 1 jest równa 3. Wtedy

A. $x=2$

B. $x=3$

C. $x=4$

D. $x=5$

Zadanie 25. $(1pkt)$

Wybieramy liczbę $a$ ze zbioru $A=\{2,3,4,5\}$ oraz liczbę $b$ ze zbioru $B=\{1,4\}$. Ilejest takich par

$(a,b), \dot{\mathrm{z}}\mathrm{e}$ iloczyn $a\cdot b$ jest liczbą nieparzystą?

A. 2

B. 3

C. 5

D. 20
\end{document}
