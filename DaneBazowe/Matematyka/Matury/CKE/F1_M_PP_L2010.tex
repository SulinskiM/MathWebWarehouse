\documentclass[a4paper,12pt]{article}
\usepackage{latexsym}
\usepackage{amsmath}
\usepackage{amssymb}
\usepackage{graphicx}
\usepackage{wrapfig}
\pagestyle{plain}
\usepackage{fancybox}
\usepackage{bm}

\begin{document}

$\mathrm{g}$ NARODOWASTRATECIASPóJNOS$\subseteq$lKAPITALL$\cup$DZKl Centralna Komisja Egzaminacyjna $\mathrm{F}\cup \mathrm{N}\mathrm{D}\cup \mathrm{s}\mathrm{z}\mathrm{S}\mathrm{P}\mathrm{O}\mathrm{L}\mathrm{E}\mathrm{C}\mathrm{Z}\mathrm{N}\mathrm{Y}\cup \mathrm{N}\mathrm{l}\mathrm{A}\mathrm{E}\cup \mathrm{R}\mathrm{O}\mathrm{p}\mathrm{E}\mathrm{J}\mathrm{S}\mathrm{K}\mathrm{A}\mathrm{E}\cup \mathrm{R}\mathrm{O}\mathrm{P}\mathrm{E}\rfloor 5\mathrm{K}\mathrm{l}$\fbox{}

Materiał współfinansowany ze środków Unii Europejskiej w ramach Europejskiego Funduszu Społecznego.

Arkusz zawiera informacje prawnie chronione do momentu rozpoczęcia egzaminu.

WPISUJE ZDAJACY

KOD PESEL

{\it Miejsce}

{\it na naklejkę}

{\it z kodem}
\begin{center}
\includegraphics[width=21.432mm,height=9.804mm]{./F1_M_PP_L2010_page0_images/image001.eps}

\includegraphics[width=82.092mm,height=9.804mm]{./F1_M_PP_L2010_page0_images/image002.eps}

\includegraphics[width=204.012mm,height=216.048mm]{./F1_M_PP_L2010_page0_images/image003.eps}
\end{center}
PRÓBNY EGZAMIN MATU

Z MATEMATY

LNY

POZIOM PODSTAWOWY  LISTOPAD 2010

1.

2.

3.

Sprawdzí, czy arkusz egzaminacyjny zawiera 19 stron

(zadania $1-34$). Ewentualny brak zgłoś przewodniczącemu

zespo nadzorującego egzamin.

Rozwiązania zadań i odpowiedzi wpisuj w miejscu na to

przeznaczonym.

Odpowiedzi do zadań zamkniętych (1-25) przenieś

na ka ę odpowiedzi, zaznaczając je w części ka $\mathrm{y}$

przeznaczonej dla zdającego. Zamaluj $\blacksquare$ pola do tego

przeznaczone. Błędne zaznaczenie otocz kółkiem

i zaznacz właściwe.

4. Pamiętaj, $\dot{\mathrm{z}}\mathrm{e}$ pominięcie argumentacji lub istotnych

obliczeń w rozwiązaniu zadania otwa ego (26-34) $\mathrm{m}\mathrm{o}\dot{\mathrm{z}}\mathrm{e}$

spowodować, $\dot{\mathrm{z}}\mathrm{e}$ za to rozwiązanie nie będziesz mógł

dostać pełnej liczby punktów.

5. Pisz cz elnie i $\mathrm{u}\dot{\mathrm{z}}$ aj tvlko długopisu lub -Dióra

z czarnym tuszem lub atramentem.

6. Nie $\mathrm{u}\dot{\mathrm{z}}$ aj korektora, a błędne zapisy wyrazínie prze eśl.

7. Pamiętaj, $\dot{\mathrm{z}}\mathrm{e}$ zapisy w brudnopisie nie będą oceniane.

8. $\mathrm{M}\mathrm{o}\dot{\mathrm{z}}$ esz korzystać z zestawu wzorów matematycznych,

cyrkla i linijki oraz kalkulatora.

9. Na karcie odpowiedzi wpisz i zakoduj swój numer

PESEL.

10. Nie wpisuj $\dot{\mathrm{z}}$ adnych znaków w części przeznaczonej dla

egzaminatora.

Czas pracy:

170 minut

Liczba punktów

do uzyskania: 50

$\Vert\Vert\Vert\Vert\Vert\Vert\Vert\Vert\Vert\Vert\Vert\Vert\Vert\Vert\Vert\Vert\Vert\Vert\Vert\Vert\Vert\Vert\Vert\Vert|  \mathrm{M}\mathrm{M}\mathrm{A}-\mathrm{P}1_{-}1\mathrm{P}-105$




{\it 2}

{\it Próbny egzamin maturalny z matematyki}

{\it Poziom podstawowy}

ZADANIA ZAMKNIĘTE

{\it Wzadaniach od l. do 25. wybierz i zaznacz na karcie odpowiedzijednq}

{\it poprawnq odpowied} $\acute{z}.$

Zadanie l. $(1pkt)$

Liczba $|5-7|-|-3+4|$ jest równa

A. $-3$ B. $-5$

C. l

D. 3

Zadanie 2. $(1pkt)$

Wskaz rysunek, na którym jest przedstawiony zbiór rozwiązań nierówności $|x-2|\geq 3.$
\begin{center}
\includegraphics[width=173.280mm,height=14.532mm]{./F1_M_PP_L2010_page1_images/image001.eps}
\end{center}
$-1$  5  {\it x}

A.
\begin{center}
\includegraphics[width=172.716mm,height=15.648mm]{./F1_M_PP_L2010_page1_images/image002.eps}
\end{center}
$-1$  5  {\it x}

B.
\begin{center}
\includegraphics[width=171.756mm,height=13.104mm]{./F1_M_PP_L2010_page1_images/image003.eps}
\end{center}
3  {\it x}

C.
\begin{center}
\includegraphics[width=172.716mm,height=15.648mm]{./F1_M_PP_L2010_page1_images/image004.eps}
\end{center}
5  {\it x}

D.

Zadanie 3. (1pkt)

Samochód kosztował 30000 zł. Jego cenę obnizono o 10\%, a następnie cenę po tej obnizce

ponownie obnizono o 10\%. Po tych obnizkach samochód kosztował

A. 24400 zł

B. 24700 zł

C. 24000 zł

D. 24300 zł

Zadanie 4. $(1pkt)$

Danajest liczba $x=63^{2}\displaystyle \cdot(\frac{1}{3})^{4}$. Wtedy

A. $x=7^{2}$ B. $x=7^{-2}$

C. $x=3^{8}\cdot 7^{2}$

D. $x=3\cdot 7$

Zadanie 5. $(1pkt)$

Kwadrat liczby $x=5+2\sqrt{3}$ jest równy

A. 37 B. $25+4\sqrt{3}$

C. $37+20\sqrt{3}$

D. 147

Zadanie 6. $(1pkt)$

Liczba $\log_{5}5-\log_{5}125$ jest równa

A. $-2$ B. $-1$

C.

$\displaystyle \frac{1}{25}$

D. 4





{\it Próbny egzamin maturalny z matematyki}

{\it Poziom podstawowy}

{\it 11}

BRUDNOPIS





{\it 12}

{\it Próbny egzamin maturalny z matematyki}

{\it Poziom podstawowy}

ZADANIA OTWARTE

{\it Rozwiqzania zadań o numerach od 26. do 34. nalezy zapisać w wyznaczonych miejscach}

{\it pod treściq zadania}.

Zadanie 26. $(2pkt)$

Rozwiąz nierówność $x^{2}+11x+30\leq 0.$

Odpowiedzí:

Zadanie 27. $(2pkt)$

Rozwiąz równanie $x^{3}+2x^{2}-5x-10=0.$

Odpowiedzí:





{\it Próbny egzamin maturalny z matematyki}

{\it Poziom podstawowy}

{\it 13}

Zadanie 28. (2pkt)

Przeciwprostokątna trójkąta prostokątnego jest dłuzsza od jednej przyprostokątnej o l cm

i od drugiej przyprostokątnej o 32 cm. Ob1icz długości boków tego trójkąta.

Odpowiedzí:





{\it 14}

{\it Próbny egzamin maturalny z matematyki}

{\it Poziom podstawowy}

Zadanie 29. (2pkt)

Dany jest prostokąt ABCD. Okręgi o średnicach AB $\mathrm{i}$ AD przecinają się w punktach $A\mathrm{i}P$

(zobacz rysunek). Wykaz, $\dot{\mathrm{z}}\mathrm{e}$ punkty $B, P\mathrm{i}D$ lez$\cdot$ą najednej prostej.
\begin{center}
\includegraphics[width=81.228mm,height=71.628mm]{./F1_M_PP_L2010_page13_images/image001.eps}
\end{center}
{\it D  C}

{\it P}

{\it A  B}





{\it Próbny egzamin maturalny z matematyki}

{\it Poziom podstawowy}

{\it 15}

Zadanie 30. $(2pkt)$

Uzasadnij, $\dot{\mathrm{z}}$ ejeśli $(a^{2}+b^{2})(c^{2}+d^{2})=(ac+bd)^{2}$, to {\it ad}$=bc.$

Zadanie 31. (2pkt)

Oblicz, ile jest liczb naturalnych czterocyfrowych, w których zapisie pierwsza cyfra jest

parzysta, a pozostałe nieparzyste.

Odpowiedzí:





{\it 16}

{\it Próbny egzamin maturalny z matematyki}

{\it Poziom podstawowy}

Zadanie 32. $(4pkt)$

Ciąg $(1,x,y-1)$ jest arytmetyczny, natomiast

Oblicz $x$ oraz $y$ i podaj ten ciąg geometryczny.

ciąg (x, y, 12)

jest geometryczny.

Odpowiedzí:





{\it Próbny egzamin maturalny z matematyki}

{\it Poziom podstawowy}

{\it 1}7

Zadanie 33. $(4pkt)$

Punkty $A=(1,5), B=(14,31), C=(4,31)$ są wierzchołkami trójkąta. Prosta zawierająca

wysokość tego trójkąta poprowadzona z wierzchołka $C$ przecina prostą AB w punkcie $D.$

Oblicz długość odcinka $BD.$

Odpowiedzí:





{\it 18}

{\it Próbny egzamin maturalny z matematyki}

{\it Poziom podstawowy}

Zadanie 34. $(5pkt)$

Droga z miasta A do miasta $\mathrm{B}$ ma długość 474 km. Samochódjadący z miasta A do miasta $\mathrm{B}$

wyrusza godzinę pózíniej $\mathrm{n}\mathrm{i}\dot{\mathrm{z}}$ samochód z miasta $\mathrm{B}$ do miasta A. Samochody te spotykają się

w odległości 300 km od miasta B. Średnia prędkość samochodu, który wyjechał z miasta $\mathrm{A},$

liczona od chwili wyjazdu z A do momentu spotkania, była o 17 $\mathrm{k}\mathrm{m}/\mathrm{h}$ mniejsza od średniej

prędkości drugiego samochodu liczonej od chwili wyjazdu z $\mathrm{B}$ do chwili spotkania. Oblicz

średniąprędkość $\mathrm{k}\mathrm{a}\dot{\mathrm{z}}$ dego samochodu do chwili spotkania.

Odpowiedzí:





{\it Próbny egzamin maturalny z matematyki}

{\it Poziom podstawowy}

{\it 19}

BRUDNOPIS





{\it Próbny egzamin maturalny z matematyki}

{\it Poziom podstawowy}

{\it 3}

BRUDNOPIS





{\it 4}

{\it Próbny egzamin maturalny z matematyki}

{\it Poziom podstawowy}

{\it W zadaniach 7, 8 i9 wykorzystaj przedstawiony ponizej wykres funkcji f}
\begin{center}
\begin{tabular}{|l|l|l|l|l|l|l|l|l|l|l|l|l|l|l|l|l|l|}
\hline
\multicolumn{1}{|l|}{}&	\multicolumn{1}{|l|}{}&	\multicolumn{1}{|l|}{}&	\multicolumn{1}{|l|}{}&	\multicolumn{1}{|l|}{}&	\multicolumn{1}{|l|}{}&	\multicolumn{1}{|l|}{}&	\multicolumn{1}{|l|}{$y$}&	\multicolumn{1}{|l|}{}&	\multicolumn{1}{|l|}{}&	\multicolumn{1}{|l|}{}&	\multicolumn{1}{|l|}{}&	\multicolumn{1}{|l|}{}&	\multicolumn{1}{|l|}{}&	\multicolumn{1}{|l|}{}&	\multicolumn{1}{|l|}{}&	\multicolumn{1}{|l|}{}&	\multicolumn{1}{|l|}{}	\\
\hline
\multicolumn{1}{|l|}{}&	\multicolumn{1}{|l|}{}&	\multicolumn{1}{|l|}{}&	\multicolumn{1}{|l|}{}&	\multicolumn{1}{|l|}{}&	\multicolumn{1}{|l|}{}&	\multicolumn{1}{|l|}{}&	\multicolumn{1}{|l|}{}&	\multicolumn{1}{|l|}{}&	\multicolumn{1}{|l|}{}&	\multicolumn{1}{|l|}{}&	\multicolumn{1}{|l|}{}&	\multicolumn{1}{|l|}{}&	\multicolumn{1}{|l|}{}&	\multicolumn{1}{|l|}{}&	\multicolumn{1}{|l|}{}&	\multicolumn{1}{|l|}{}&	\multicolumn{1}{|l|}{}	\\
\hline
\multicolumn{1}{|l|}{}&	\multicolumn{1}{|l|}{}&	\multicolumn{1}{|l|}{}&	\multicolumn{1}{|l|}{}&	\multicolumn{1}{|l|}{}&	\multicolumn{1}{|l|}{}&	\multicolumn{1}{|l|}{}&	\multicolumn{1}{|l|}{}&	\multicolumn{1}{|l|}{}&	\multicolumn{1}{|l|}{}&	\multicolumn{1}{|l|}{}&	\multicolumn{1}{|l|}{}&	\multicolumn{1}{|l|}{}&	\multicolumn{1}{|l|}{}&	\multicolumn{1}{|l|}{}&	\multicolumn{1}{|l|}{}&	\multicolumn{1}{|l|}{}&	\multicolumn{1}{|l|}{}	\\
\hline
\multicolumn{1}{|l|}{}&	\multicolumn{1}{|l|}{}&	\multicolumn{1}{|l|}{}&	\multicolumn{1}{|l|}{}&	\multicolumn{1}{|l|}{}&	\multicolumn{1}{|l|}{}&	\multicolumn{1}{|l|}{}&	\multicolumn{1}{|l|}{}&	\multicolumn{1}{|l|}{}&	\multicolumn{1}{|l|}{}&	\multicolumn{1}{|l|}{}&	\multicolumn{1}{|l|}{}&	\multicolumn{1}{|l|}{}&	\multicolumn{1}{|l|}{}&	\multicolumn{1}{|l|}{}&	\multicolumn{1}{|l|}{}&	\multicolumn{1}{|l|}{}&	\multicolumn{1}{|l|}{}	\\
\hline
\multicolumn{1}{|l|}{}&	\multicolumn{1}{|l|}{}&	\multicolumn{1}{|l|}{}&	\multicolumn{1}{|l|}{}&	\multicolumn{1}{|l|}{}&	\multicolumn{1}{|l|}{}&	\multicolumn{1}{|l|}{}&	\multicolumn{1}{|l|}{}&	\multicolumn{1}{|l|}{}&	\multicolumn{1}{|l|}{}&	\multicolumn{1}{|l|}{}&	\multicolumn{1}{|l|}{}&	\multicolumn{1}{|l|}{}&	\multicolumn{1}{|l|}{}&	\multicolumn{1}{|l|}{}&	\multicolumn{1}{|l|}{}&	\multicolumn{1}{|l|}{}&	\multicolumn{1}{|l|}{}	\\
\hline
\multicolumn{1}{|l|}{}&	\multicolumn{1}{|l|}{}&	\multicolumn{1}{|l|}{}&	\multicolumn{1}{|l|}{}&	\multicolumn{1}{|l|}{}&	\multicolumn{1}{|l|}{}&	\multicolumn{1}{|l|}{}&	\multicolumn{1}{|l|}{}&	\multicolumn{1}{|l|}{}&	\multicolumn{1}{|l|}{}&	\multicolumn{1}{|l|}{}&	\multicolumn{1}{|l|}{}&	\multicolumn{1}{|l|}{}&	\multicolumn{1}{|l|}{}&	\multicolumn{1}{|l|}{}&	\multicolumn{1}{|l|}{}&	\multicolumn{1}{|l|}{}&	\multicolumn{1}{|l|}{}	\\
\hline
\multicolumn{1}{|l|}{}&	\multicolumn{1}{|l|}{}&	\multicolumn{1}{|l|}{}&	\multicolumn{1}{|l|}{}&	\multicolumn{1}{|l|}{}&	\multicolumn{1}{|l|}{}&	\multicolumn{1}{|l|}{}&	\multicolumn{1}{|l|}{}&	\multicolumn{1}{|l|}{}&	\multicolumn{1}{|l|}{}&	\multicolumn{1}{|l|}{}&	\multicolumn{1}{|l|}{}&	\multicolumn{1}{|l|}{}&	\multicolumn{1}{|l|}{}&	\multicolumn{1}{|l|}{}&	\multicolumn{1}{|l|}{}&	\multicolumn{1}{|l|}{}&	\multicolumn{1}{|l|}{ $x$}	\\
\hline
\multicolumn{1}{|l|}{}&	\multicolumn{1}{|l|}{}&	\multicolumn{1}{|l|}{}&	\multicolumn{1}{|l|}{}&	\multicolumn{1}{|l|}{}&	\multicolumn{1}{|l|}{}&	\multicolumn{1}{|l|}{}&	\multicolumn{1}{|l|}{}&	\multicolumn{1}{|l|}{}&	\multicolumn{1}{|l|}{}&	\multicolumn{1}{|l|}{}&	\multicolumn{1}{|l|}{}&	\multicolumn{1}{|l|}{}&	\multicolumn{1}{|l|}{}&	\multicolumn{1}{|l|}{}&	\multicolumn{1}{|l|}{}&	\multicolumn{1}{|l|}{ $1$}&	\multicolumn{1}{|l|}{}	\\
\hline
\multicolumn{1}{|l|}{}&	\multicolumn{1}{|l|}{}&	\multicolumn{1}{|l|}{}&	\multicolumn{1}{|l|}{}&	\multicolumn{1}{|l|}{}&	\multicolumn{1}{|l|}{}&	\multicolumn{1}{|l|}{}&	\multicolumn{1}{|l|}{}&	\multicolumn{1}{|l|}{}&	\multicolumn{1}{|l|}{}&	\multicolumn{1}{|l|}{}&	\multicolumn{1}{|l|}{}&	\multicolumn{1}{|l|}{}&	\multicolumn{1}{|l|}{}&	\multicolumn{1}{|l|}{}&	\multicolumn{1}{|l|}{}&	\multicolumn{1}{|l|}{}&	\multicolumn{1}{|l|}{}	\\
\hline
\multicolumn{1}{|l|}{}&	\multicolumn{1}{|l|}{}&	\multicolumn{1}{|l|}{}&	\multicolumn{1}{|l|}{}&	\multicolumn{1}{|l|}{}&	\multicolumn{1}{|l|}{}&	\multicolumn{1}{|l|}{}&	\multicolumn{1}{|l|}{}&	\multicolumn{1}{|l|}{}&	\multicolumn{1}{|l|}{}&	\multicolumn{1}{|l|}{}&	\multicolumn{1}{|l|}{}&	\multicolumn{1}{|l|}{}&	\multicolumn{1}{|l|}{}&	\multicolumn{1}{|l|}{}&	\multicolumn{1}{|l|}{}&	\multicolumn{1}{|l|}{}&	\multicolumn{1}{|l|}{}	\\
\hline
\multicolumn{1}{|l|}{}&	\multicolumn{1}{|l|}{}&	\multicolumn{1}{|l|}{}&	\multicolumn{1}{|l|}{}&	\multicolumn{1}{|l|}{}&	\multicolumn{1}{|l|}{}&	\multicolumn{1}{|l|}{}&	\multicolumn{1}{|l|}{}&	\multicolumn{1}{|l|}{}&	\multicolumn{1}{|l|}{}&	\multicolumn{1}{|l|}{}&	\multicolumn{1}{|l|}{}&	\multicolumn{1}{|l|}{}&	\multicolumn{1}{|l|}{}&	\multicolumn{1}{|l|}{}&	\multicolumn{1}{|l|}{}&	\multicolumn{1}{|l|}{}&	\multicolumn{1}{|l|}{}	\\
\hline
\end{tabular}

\end{center}
Zadanie 7. (1pkt)

Zbiorem wartości ffinkcjifjest

A. $\langle-2,5\rangle$

B. $\langle-4,8\rangle$

C. $\langle-1,4\rangle$

D. $\langle$5, $ 8\rangle$

Zadanie 8. (1pkt)

Korzystając z wykresu ffinkcjif, wskaz nierówność prawdziwą.

A. $f(-1)<f(1)$

B. $f(1)<f(3)$

C. $f(-1)<f(3)$

D. $f(3)<f(0)$

Zadanie 9. $(1pkt)$

Wykres ffinkcji $g$ określonej wzorem $g(x)=f(x)+2$ jest przedstawiony na rysunku

A. B.





{\it Próbny egzamin maturalny z matematyki}

{\it Poziom podstawowy}

{\it 5}

BRUDNOPIS





{\it 6}

{\it Próbny egzamin maturalny z matematyki}

{\it Poziom podstawowy}

Zadanie 10. $(1pkt)$

Liczby $x_{1}$ i $x_{2}$ sąpierwiastkami równania $x^{2}+10x-24=0\mathrm{i}x_{1}<x_{2}$. Oblicz $2x_{1}+x_{2}.$

A. $-22$

B. $-17$

C. 8

D. 13

Zadanie ll. (lpkt)

Liczba 2 jest pierwiastkiem wie1omianu

równy

$W(x)=x^{3}+ax^{2}+6x-4$. Współczynnik $a$ jest

A. 2

B. $-2$

C. 4

D. $-4$

Zadanie 12. $(1pkt)$

Wskaz $m$, dla którego ffinkcja liniowa określona wzorem $f(x)=(m-1)x+3$ jest stała.

A. $m=1$

B. $m=2$

C. $m=3$

D. $m=-1$

Zadanie 13. $(1pkt)$

Zbiorem rozwiązań nierówności $(x-2)(x+3)\geq 0$ jest

A.

B.

C.

D.

$\langle-2,3\rangle$

$\langle-3,2\rangle$

$(-\infty,-3\rangle\cup\langle 2,+\infty)$

$(-\infty,-2\rangle\cup\langle 3,+\infty)$

Zadanie 14. $(1pkt)$

$\mathrm{W}$ ciągu geometrycznym $(a_{n})$ dane są: $a_{1}=2\mathrm{i}a_{2}=12$. Wtedy

A. $a_{4}=26$

B. $a_{4}=432$

C. $a_{4}=32$

D. $a_{4}=2592$

Zadanie 15. $(1pkt)$

$\mathrm{W}$ ciągu arytmetycznym $a_{1}=3$ oraz $a_{20}=7$. Wtedy suma $S_{20}=a_{1}+a_{2}+\ldots+a_{19}+a_{20}$ jest

równa

A. 95

B. 200

C. 230

D. 100

Zadanie 16. $(1pkt)$

Na rysunku zaznaczono długości boków i kąt $\alpha$ trójkąta prostokątnego (zobacz rysunek). Wtedy
\begin{center}
\includegraphics[width=87.984mm,height=32.868mm]{./F1_M_PP_L2010_page5_images/image001.eps}
\end{center}
13

5

12

A.

$\displaystyle \cos\alpha=\frac{5}{13}$

B.

$\displaystyle \mathrm{t}\mathrm{g}\alpha=\frac{13}{12}$

C.

$\displaystyle \cos\alpha=\frac{12}{13}$

D.

$\displaystyle \mathrm{t}\mathrm{g}\alpha=\frac{12}{5}$





{\it Próbny egzamin maturalny z matematyki}

{\it Poziom podstawowy}

7

BRUDNOPIS





{\it 8}

{\it Próbny egzamin maturalny z matematyki}

{\it Poziom podstawowy}

Zadanie 17. (1pkt)

Ogród ma kształt prostokąta o bokach długości 20 m i 40 m. Na dwóch końcach przekątnej

tego prostokąta wbito słupki. Odległość między tymi słupkamijest

A.

B.

C.

D.

równa 40 $\mathrm{m}$

większa $\mathrm{n}\mathrm{i}\dot{\mathrm{z}}50\mathrm{m}$

większa $\mathrm{n}\mathrm{i}\dot{\mathrm{z}}40\mathrm{m}$ i mniejsza $\mathrm{n}\mathrm{i}\dot{\mathrm{z}}45\mathrm{m}$

większa $\mathrm{n}\mathrm{i}\dot{\mathrm{z}}45\mathrm{m}$ i mniejsza $\mathrm{n}\mathrm{i}\dot{\mathrm{z}}50\mathrm{m}$

Zadanie 18. (1pkt)

Pionowy słupek o wysokości 90 cm rzuca cień o długości 60 cm. W tej samej chwi1i stojąca

obok wieza rzuca cień długości 12 m. Jakajest wysokość wiezy?

A. 18 m

B. 8m

C. 9m

D. 16 m

Zadanie 19. $(1pkt)$

Punkty $A, B \mathrm{i} C$ lez$\cdot$ą na okręgu o środku $S$ (zobacz rysunek). Miara zaznaczonego kąta

wpisanego $ACB$ jest równa
\begin{center}
\includegraphics[width=53.796mm,height=52.728mm]{./F1_M_PP_L2010_page7_images/image001.eps}
\end{center}
{\it C}

{\it A  B}

{\it S}

$230^{\mathrm{o}}$

A. $65^{\mathrm{o}}$

B. $100^{\mathrm{o}}$

C. $115^{\mathrm{o}}$

D. $130^{\mathrm{o}}$

Zadanie 20. $(1pkt)$

Dane sąpunkty $S=(2,1), M=(6,4)$. Równanie okręgu o środku $S$ i przechodzącego przez

punkt $M$ ma postać

A.

B.

C.

D.

$(x-2)^{2}+(y-1)^{2}=5$

$(x-2)^{2}+(y-1)^{2}=25$

$(x-6)^{2}+(y-4)^{2}=5$

$(x-6)^{2}+(y-4)^{2}=25$





{\it Próbny egzamin maturalny z matematyki}

{\it Poziom podstawowy}

{\it 9}

BRUDNOPIS





$ 1\theta$

{\it Próbny egzamin maturalny z matematyki}

{\it Poziom podstawowy}

Zadanie 21. $(1pkt)$

Proste o równaniach $y=2x+3$ oraz $y=-\displaystyle \frac{1}{3}x+2$

A. są równoległe i rózne

B. sąprostopadłe

C. przecinają się pod kątem innym $\mathrm{n}\mathrm{i}\dot{\mathrm{z}}$ prosty

D. pokrywają się

Zadanie 22. $(1pkt)$

Wskaz równanie prostej, którajest osią symetrii paraboli o równaniu $y=x^{2}-4x+2010.$

A. $x=4$

B. $x=-4$

C. $x=2$

D. $x=-2$

Zadanie 23. $(1pkt)$

Kąt $\alpha$ jest ostry i $\displaystyle \cos\alpha=\frac{3}{7}$. Wtedy

A.

$\displaystyle \sin\alpha=\frac{2\sqrt{10}}{7}$

B.

$\displaystyle \sin\alpha=\frac{\sqrt{10}}{7}$

C.

$\displaystyle \sin\alpha=\frac{4}{7}$

D.

$\displaystyle \sin\alpha=\frac{3}{4}$

Zadanie 24. (1pkt)

W karcie dań jest 5 zup i 4 drugie dania. Na i1e sposobów mozna zamówić obiad s$\mathbb{H}$adający się

zjednej zupy ijednego drugiego dania?

A. 25

B. 20

C. 16

D. 9

Zadanie 25. (1pkt)

W czterech rzutach sześcienną kostką do gry otrzymano następujące liczby oczek: 6, 3, 1, 4.

Mediana tych danychjest równa

A. 2

B. 2,5

C. 5

D. 3,5



\end{document}