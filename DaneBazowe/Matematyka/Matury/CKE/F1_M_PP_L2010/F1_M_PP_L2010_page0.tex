\documentclass[a4paper,12pt]{article}
\usepackage{latexsym}
\usepackage{amsmath}
\usepackage{amssymb}
\usepackage{graphicx}
\usepackage{wrapfig}
\pagestyle{plain}
\usepackage{fancybox}
\usepackage{bm}

\begin{document}

$\mathrm{g}$ NARODOWASTRATECIASPóJNOS$\subseteq$lKAPITALL$\cup$DZKl Centralna Komisja Egzaminacyjna $\mathrm{F}\cup \mathrm{N}\mathrm{D}\cup \mathrm{s}\mathrm{z}\mathrm{S}\mathrm{P}\mathrm{O}\mathrm{L}\mathrm{E}\mathrm{C}\mathrm{Z}\mathrm{N}\mathrm{Y}\cup \mathrm{N}\mathrm{l}\mathrm{A}\mathrm{E}\cup \mathrm{R}\mathrm{O}\mathrm{p}\mathrm{E}\mathrm{J}\mathrm{S}\mathrm{K}\mathrm{A}\mathrm{E}\cup \mathrm{R}\mathrm{O}\mathrm{P}\mathrm{E}\rfloor 5\mathrm{K}\mathrm{l}$\fbox{}

Materiał współfinansowany ze środków Unii Europejskiej w ramach Europejskiego Funduszu Społecznego.

Arkusz zawiera informacje prawnie chronione do momentu rozpoczęcia egzaminu.

WPISUJE ZDAJACY

KOD PESEL

{\it Miejsce}

{\it na naklejkę}

{\it z kodem}
\begin{center}
\includegraphics[width=21.432mm,height=9.804mm]{./F1_M_PP_L2010_page0_images/image001.eps}

\includegraphics[width=82.092mm,height=9.804mm]{./F1_M_PP_L2010_page0_images/image002.eps}

\includegraphics[width=204.012mm,height=216.048mm]{./F1_M_PP_L2010_page0_images/image003.eps}
\end{center}
PRÓBNY EGZAMIN MATU

Z MATEMATY

LNY

POZIOM PODSTAWOWY  LISTOPAD 2010

1.

2.

3.

Sprawdzí, czy arkusz egzaminacyjny zawiera 19 stron

(zadania $1-34$). Ewentualny brak zgłoś przewodniczącemu

zespo nadzorującego egzamin.

Rozwiązania zadań i odpowiedzi wpisuj w miejscu na to

przeznaczonym.

Odpowiedzi do zadań zamkniętych (1-25) przenieś

na ka ę odpowiedzi, zaznaczając je w części ka $\mathrm{y}$

przeznaczonej dla zdającego. Zamaluj $\blacksquare$ pola do tego

przeznaczone. Błędne zaznaczenie otocz kółkiem

i zaznacz właściwe.

4. Pamiętaj, $\dot{\mathrm{z}}\mathrm{e}$ pominięcie argumentacji lub istotnych

obliczeń w rozwiązaniu zadania otwa ego (26-34) $\mathrm{m}\mathrm{o}\dot{\mathrm{z}}\mathrm{e}$

spowodować, $\dot{\mathrm{z}}\mathrm{e}$ za to rozwiązanie nie będziesz mógł

dostać pełnej liczby punktów.

5. Pisz cz elnie i $\mathrm{u}\dot{\mathrm{z}}$ aj tvlko długopisu lub -Dióra

z czarnym tuszem lub atramentem.

6. Nie $\mathrm{u}\dot{\mathrm{z}}$ aj korektora, a błędne zapisy wyrazínie prze eśl.

7. Pamiętaj, $\dot{\mathrm{z}}\mathrm{e}$ zapisy w brudnopisie nie będą oceniane.

8. $\mathrm{M}\mathrm{o}\dot{\mathrm{z}}$ esz korzystać z zestawu wzorów matematycznych,

cyrkla i linijki oraz kalkulatora.

9. Na karcie odpowiedzi wpisz i zakoduj swój numer

PESEL.

10. Nie wpisuj $\dot{\mathrm{z}}$ adnych znaków w części przeznaczonej dla

egzaminatora.

Czas pracy:

170 minut

Liczba punktów

do uzyskania: 50

$\Vert\Vert\Vert\Vert\Vert\Vert\Vert\Vert\Vert\Vert\Vert\Vert\Vert\Vert\Vert\Vert\Vert\Vert\Vert\Vert\Vert\Vert\Vert\Vert|  \mathrm{M}\mathrm{M}\mathrm{A}-\mathrm{P}1_{-}1\mathrm{P}-105$
\end{document}
