\documentclass[a4paper,12pt]{article}
\usepackage{latexsym}
\usepackage{amsmath}
\usepackage{amssymb}
\usepackage{graphicx}
\usepackage{wrapfig}
\pagestyle{plain}
\usepackage{fancybox}
\usepackage{bm}

\begin{document}

{\it 14}

{\it Próbny egzamin maturalny z matematyki}

{\it Poziom podstawowy}

Zadanie 29. (2pkt)

Dany jest prostokąt ABCD. Okręgi o średnicach AB $\mathrm{i}$ AD przecinają się w punktach $A\mathrm{i}P$

(zobacz rysunek). Wykaz, $\dot{\mathrm{z}}\mathrm{e}$ punkty $B, P\mathrm{i}D$ lez$\cdot$ą najednej prostej.
\begin{center}
\includegraphics[width=81.228mm,height=71.628mm]{./F1_M_PP_L2010_page13_images/image001.eps}
\end{center}
{\it D  C}

{\it P}

{\it A  B}
\end{document}
