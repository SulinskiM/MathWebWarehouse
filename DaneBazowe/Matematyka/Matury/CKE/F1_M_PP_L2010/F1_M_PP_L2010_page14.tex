\documentclass[a4paper,12pt]{article}
\usepackage{latexsym}
\usepackage{amsmath}
\usepackage{amssymb}
\usepackage{graphicx}
\usepackage{wrapfig}
\pagestyle{plain}
\usepackage{fancybox}
\usepackage{bm}

\begin{document}

{\it Próbny egzamin maturalny z matematyki}

{\it Poziom podstawowy}

{\it 15}

Zadanie 30. $(2pkt)$

Uzasadnij, $\dot{\mathrm{z}}$ ejeśli $(a^{2}+b^{2})(c^{2}+d^{2})=(ac+bd)^{2}$, to {\it ad}$=bc.$

Zadanie 31. (2pkt)

Oblicz, ile jest liczb naturalnych czterocyfrowych, w których zapisie pierwsza cyfra jest

parzysta, a pozostałe nieparzyste.

Odpowiedzí:
\end{document}
