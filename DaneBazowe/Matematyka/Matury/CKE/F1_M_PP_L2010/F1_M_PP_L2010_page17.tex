\documentclass[a4paper,12pt]{article}
\usepackage{latexsym}
\usepackage{amsmath}
\usepackage{amssymb}
\usepackage{graphicx}
\usepackage{wrapfig}
\pagestyle{plain}
\usepackage{fancybox}
\usepackage{bm}

\begin{document}

{\it 18}

{\it Próbny egzamin maturalny z matematyki}

{\it Poziom podstawowy}

Zadanie 34. $(5pkt)$

Droga z miasta A do miasta $\mathrm{B}$ ma długość 474 km. Samochódjadący z miasta A do miasta $\mathrm{B}$

wyrusza godzinę pózíniej $\mathrm{n}\mathrm{i}\dot{\mathrm{z}}$ samochód z miasta $\mathrm{B}$ do miasta A. Samochody te spotykają się

w odległości 300 km od miasta B. Średnia prędkość samochodu, który wyjechał z miasta $\mathrm{A},$

liczona od chwili wyjazdu z A do momentu spotkania, była o 17 $\mathrm{k}\mathrm{m}/\mathrm{h}$ mniejsza od średniej

prędkości drugiego samochodu liczonej od chwili wyjazdu z $\mathrm{B}$ do chwili spotkania. Oblicz

średniąprędkość $\mathrm{k}\mathrm{a}\dot{\mathrm{z}}$ dego samochodu do chwili spotkania.

Odpowiedzí:
\end{document}
