\documentclass[a4paper,12pt]{article}
\usepackage{latexsym}
\usepackage{amsmath}
\usepackage{amssymb}
\usepackage{graphicx}
\usepackage{wrapfig}
\pagestyle{plain}
\usepackage{fancybox}
\usepackage{bm}

\begin{document}

{\it 6}

{\it Próbny egzamin maturalny z matematyki}

{\it Poziom podstawowy}

Zadanie 10. $(1pkt)$

Liczby $x_{1}$ i $x_{2}$ sąpierwiastkami równania $x^{2}+10x-24=0\mathrm{i}x_{1}<x_{2}$. Oblicz $2x_{1}+x_{2}.$

A. $-22$

B. $-17$

C. 8

D. 13

Zadanie ll. (lpkt)

Liczba 2 jest pierwiastkiem wie1omianu

równy

$W(x)=x^{3}+ax^{2}+6x-4$. Współczynnik $a$ jest

A. 2

B. $-2$

C. 4

D. $-4$

Zadanie 12. $(1pkt)$

Wskaz $m$, dla którego ffinkcja liniowa określona wzorem $f(x)=(m-1)x+3$ jest stała.

A. $m=1$

B. $m=2$

C. $m=3$

D. $m=-1$

Zadanie 13. $(1pkt)$

Zbiorem rozwiązań nierówności $(x-2)(x+3)\geq 0$ jest

A.

B.

C.

D.

$\langle-2,3\rangle$

$\langle-3,2\rangle$

$(-\infty,-3\rangle\cup\langle 2,+\infty)$

$(-\infty,-2\rangle\cup\langle 3,+\infty)$

Zadanie 14. $(1pkt)$

$\mathrm{W}$ ciągu geometrycznym $(a_{n})$ dane są: $a_{1}=2\mathrm{i}a_{2}=12$. Wtedy

A. $a_{4}=26$

B. $a_{4}=432$

C. $a_{4}=32$

D. $a_{4}=2592$

Zadanie 15. $(1pkt)$

$\mathrm{W}$ ciągu arytmetycznym $a_{1}=3$ oraz $a_{20}=7$. Wtedy suma $S_{20}=a_{1}+a_{2}+\ldots+a_{19}+a_{20}$ jest

równa

A. 95

B. 200

C. 230

D. 100

Zadanie 16. $(1pkt)$

Na rysunku zaznaczono długości boków i kąt $\alpha$ trójkąta prostokątnego (zobacz rysunek). Wtedy
\begin{center}
\includegraphics[width=87.984mm,height=32.868mm]{./F1_M_PP_L2010_page5_images/image001.eps}
\end{center}
13

5

12

A.

$\displaystyle \cos\alpha=\frac{5}{13}$

B.

$\displaystyle \mathrm{t}\mathrm{g}\alpha=\frac{13}{12}$

C.

$\displaystyle \cos\alpha=\frac{12}{13}$

D.

$\displaystyle \mathrm{t}\mathrm{g}\alpha=\frac{12}{5}$
\end{document}
