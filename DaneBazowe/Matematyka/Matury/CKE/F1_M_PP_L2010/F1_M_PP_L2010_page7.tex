\documentclass[a4paper,12pt]{article}
\usepackage{latexsym}
\usepackage{amsmath}
\usepackage{amssymb}
\usepackage{graphicx}
\usepackage{wrapfig}
\pagestyle{plain}
\usepackage{fancybox}
\usepackage{bm}

\begin{document}

{\it 8}

{\it Próbny egzamin maturalny z matematyki}

{\it Poziom podstawowy}

Zadanie 17. (1pkt)

Ogród ma kształt prostokąta o bokach długości 20 m i 40 m. Na dwóch końcach przekątnej

tego prostokąta wbito słupki. Odległość między tymi słupkamijest

A.

B.

C.

D.

równa 40 $\mathrm{m}$

większa $\mathrm{n}\mathrm{i}\dot{\mathrm{z}}50\mathrm{m}$

większa $\mathrm{n}\mathrm{i}\dot{\mathrm{z}}40\mathrm{m}$ i mniejsza $\mathrm{n}\mathrm{i}\dot{\mathrm{z}}45\mathrm{m}$

większa $\mathrm{n}\mathrm{i}\dot{\mathrm{z}}45\mathrm{m}$ i mniejsza $\mathrm{n}\mathrm{i}\dot{\mathrm{z}}50\mathrm{m}$

Zadanie 18. (1pkt)

Pionowy słupek o wysokości 90 cm rzuca cień o długości 60 cm. W tej samej chwi1i stojąca

obok wieza rzuca cień długości 12 m. Jakajest wysokość wiezy?

A. 18 m

B. 8m

C. 9m

D. 16 m

Zadanie 19. $(1pkt)$

Punkty $A, B \mathrm{i} C$ lez$\cdot$ą na okręgu o środku $S$ (zobacz rysunek). Miara zaznaczonego kąta

wpisanego $ACB$ jest równa
\begin{center}
\includegraphics[width=53.796mm,height=52.728mm]{./F1_M_PP_L2010_page7_images/image001.eps}
\end{center}
{\it C}

{\it A  B}

{\it S}

$230^{\mathrm{o}}$

A. $65^{\mathrm{o}}$

B. $100^{\mathrm{o}}$

C. $115^{\mathrm{o}}$

D. $130^{\mathrm{o}}$

Zadanie 20. $(1pkt)$

Dane sąpunkty $S=(2,1), M=(6,4)$. Równanie okręgu o środku $S$ i przechodzącego przez

punkt $M$ ma postać

A.

B.

C.

D.

$(x-2)^{2}+(y-1)^{2}=5$

$(x-2)^{2}+(y-1)^{2}=25$

$(x-6)^{2}+(y-4)^{2}=5$

$(x-6)^{2}+(y-4)^{2}=25$
\end{document}
