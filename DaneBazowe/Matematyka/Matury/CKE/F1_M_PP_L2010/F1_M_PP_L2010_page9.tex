\documentclass[a4paper,12pt]{article}
\usepackage{latexsym}
\usepackage{amsmath}
\usepackage{amssymb}
\usepackage{graphicx}
\usepackage{wrapfig}
\pagestyle{plain}
\usepackage{fancybox}
\usepackage{bm}

\begin{document}

$ 1\theta$

{\it Próbny egzamin maturalny z matematyki}

{\it Poziom podstawowy}

Zadanie 21. $(1pkt)$

Proste o równaniach $y=2x+3$ oraz $y=-\displaystyle \frac{1}{3}x+2$

A. są równoległe i rózne

B. sąprostopadłe

C. przecinają się pod kątem innym $\mathrm{n}\mathrm{i}\dot{\mathrm{z}}$ prosty

D. pokrywają się

Zadanie 22. $(1pkt)$

Wskaz równanie prostej, którajest osią symetrii paraboli o równaniu $y=x^{2}-4x+2010.$

A. $x=4$

B. $x=-4$

C. $x=2$

D. $x=-2$

Zadanie 23. $(1pkt)$

Kąt $\alpha$ jest ostry i $\displaystyle \cos\alpha=\frac{3}{7}$. Wtedy

A.

$\displaystyle \sin\alpha=\frac{2\sqrt{10}}{7}$

B.

$\displaystyle \sin\alpha=\frac{\sqrt{10}}{7}$

C.

$\displaystyle \sin\alpha=\frac{4}{7}$

D.

$\displaystyle \sin\alpha=\frac{3}{4}$

Zadanie 24. (1pkt)

W karcie dań jest 5 zup i 4 drugie dania. Na i1e sposobów mozna zamówić obiad s$\mathbb{H}$adający się

zjednej zupy ijednego drugiego dania?

A. 25

B. 20

C. 16

D. 9

Zadanie 25. (1pkt)

W czterech rzutach sześcienną kostką do gry otrzymano następujące liczby oczek: 6, 3, 1, 4.

Mediana tych danychjest równa

A. 2

B. 2,5

C. 5

D. 3,5
\end{document}
