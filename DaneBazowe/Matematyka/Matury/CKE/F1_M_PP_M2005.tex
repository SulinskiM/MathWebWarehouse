\documentclass[a4paper,12pt]{article}
\usepackage{latexsym}
\usepackage{amsmath}
\usepackage{amssymb}
\usepackage{graphicx}
\usepackage{wrapfig}
\pagestyle{plain}
\usepackage{fancybox}
\usepackage{bm}

\begin{document}
\begin{center}
\begin{tabular}{l|l}
\multicolumn{1}{l|}{$\begin{array}{l}\mbox{{\it dysleksja}}	\\	\mbox{Miejsce}	\\	\mbox{na na ejkę}	\\	\mbox{z kodem szkoly}	\end{array}$}&	\multicolumn{1}{|l}{MMA-PIAIP-052}	\\
\hline
\multicolumn{1}{l|}{$\begin{array}{l}\mbox{EGZAMIN MATURALNY}	\\	\mbox{Z MATEMATYKI}	\\	\mbox{Arkusz I}	\\	\mbox{POZIOM PODSTAWOWY}	\\	\mbox{Czas pracy 120 minut}	\\	\mbox{Instrukcja dla zdającego}	\\	\mbox{1. $\mathrm{S}\mathrm{p}\mathrm{r}\mathrm{a}\mathrm{w}\mathrm{d}\acute{\mathrm{z}}$, czy arkusz egzaminacyjny zawiera 13 stron.}	\\	\mbox{Ewentualny brak zgłoś przewodniczącemu zespo}	\\	\mbox{nadzorującego egzamin.}	\\	\mbox{2. Rozwiązania zadań i odpowiedzi zamieść w miejscu na to}	\\	\mbox{przeznaczonym.}	\\	\mbox{3. $\mathrm{W}$ rozwiązaniach zadań przedstaw tok rozumowania}	\\	\mbox{prowadzący do ostatecznego wyniku.}	\\	\mbox{4. Pisz czytelnie. Uzywaj długopisu pióra tylko z czatnym}	\\	\mbox{tusze atramentem.}	\\	\mbox{5. Nie uzywaj korektora. Błędne zapisy prze eśl.}	\\	\mbox{6. Pamiętaj, $\dot{\mathrm{z}}\mathrm{e}$ zapisy w $\mathrm{b}$ dnopisie nie podlegają ocenie.}	\\	\mbox{7. Obok $\mathrm{k}\mathrm{a}\dot{\mathrm{z}}$ dego zadania podanajest maksymalna liczba punktów,}	\\	\mbox{którą mozesz uzyskać zajego poprawne rozwiązanie.}	\\	\mbox{8. $\mathrm{M}\mathrm{o}\dot{\mathrm{z}}$ esz korzystać z zestawu wzorów matematycznych, cyrkla}	\\	\mbox{i linijki oraz kalkulatora.}	\\	\mbox{9. Wypełnij tę część ka $\mathrm{y}$ odpowiedzi, którą koduje zdający.}	\\	\mbox{Nie wpisuj $\dot{\mathrm{z}}$ adnych znaków w części przeznaczonej}	\\	\mbox{dla egzaminatora.}	\\	\mbox{10. Na karcie odpowiedzi wpisz swoją datę urodzenia i PESEL.}	\\	\mbox{Zamaluj $\blacksquare$ pola odpowiadające cyfrom numeru PESEL. Błędne}	\\	\mbox{zaznaczenie otocz kółkiem i zaznacz właściwe.}	\\	\mbox{{\it Zyczymy powodzenia}.'}	\end{array}$}&	\multicolumn{1}{|l}{$\begin{array}{l}\mbox{ARKUSZ I}	\\	\mbox{MAJ}	\\	\mbox{ROK 2005}	\\	\mbox{Za rozwiązanie}	\\	\mbox{wszystkich zadań}	\\	\mbox{mozna otrzymać}	\\	\mbox{łącznie}	\\	\mbox{50 punktów}	\end{array}$}	\\
\hline
\multicolumn{1}{l|}{$\begin{array}{l}\mbox{Wypelnia zdający przed}	\\	\mbox{roz oczęciem racy}	\\	\mbox{PESEL ZDAJACEGO}	\end{array}$}&	\multicolumn{1}{|l}{$\begin{array}{l}\mbox{tylko}	\\	\mbox{O Kraków,}	\\	\mbox{OKE Wroclaw}	\\	\mbox{KOD}	\\	\mbox{ZDAJACEGO}	\end{array}$}
\end{tabular}


\includegraphics[width=78.792mm,height=13.356mm]{./F1_M_PP_M2005_page0_images/image001.eps}

\includegraphics[width=21.840mm,height=9.804mm]{./F1_M_PP_M2005_page0_images/image002.eps}
\end{center}



{\it 2}

{\it Egzamin maturalny z matematyki}

{\it Arkusz I}

Zadanie 1. (3pkt)

W pudełku są trzy kule białe i pięć kul czarnych. Do pudełka mozna albo dołozyć jedną kulę

białą albo usunąč z niegojedną kulę czarn4 a następnie wy1osować z tego pudełkajedną ku1ę.

W którym z tych przypadków wylosowanie kuli białej jest bardziej prawdopodobne?

Wykonaj odpowiednie obliczenia.
\begin{center}
\includegraphics[width=192.588mm,height=252.684mm]{./F1_M_PP_M2005_page1_images/image001.eps}
\end{center}




{\it Egzamin maturalny z matematyki}

{\it Arkusz I}

{\it 11}

Zadanie 10. $(7pkt)$

$\mathrm{W}$ ostrosłupie czworokątnym prawidłowym wysokości przeciwległych ścian bocznych

poprowadzone z wierzchołka ostrosłupa mają długości $h$ i tworzą kąt o mierze $ 2\alpha$. Oblicz

objętość tego ostrosłupa.
\begin{center}
\includegraphics[width=192.588mm,height=258.720mm]{./F1_M_PP_M2005_page10_images/image001.eps}
\end{center}




{\it 12}

{\it Egzamin maturalny z matematyki}

{\it Arkusz I}

BRUDNOPIS





{\it Egzamin maturalny z matematyki}

{\it Arkusz I}

{\it 13}





{\it Egzamin maturalny z matematyki}

{\it Arkusz I}

{\it 3}
\begin{center}
\includegraphics[width=193.548mm,height=290.220mm]{./F1_M_PP_M2005_page2_images/image001.eps}
\end{center}
Zadanie 2. $(4pkt)$

Dany jest ciąg $(a_{n})$, gdzie $a_{n}=\displaystyle \frac{n+2}{3n+1}$ dla $ n=1,2,3\ldots$ Wyznacz wszystkie wyrazy tego ciągu

większe od $\displaystyle \frac{1}{2}$





{\it 4}

{\it Egzamin maturalny z matematyki}

{\it Arkusz I}

Zadanie 3. (4pkt)

Dany jest wielomian $W(x)=x^{3}+kx^{2}-4.$

a) Wyznacz współczynnik $k$ tego wielomianu wiedząc, $\dot{\mathrm{z}}\mathrm{e}$ wielomian ten jest podzielny

przez dwumian $x+2.$

b) Dla wyznaczonej wartości $k$ rozłóz wielomian na czynniki i podaj wszystkie jego

pierwiastki.
\begin{center}
\includegraphics[width=192.588mm,height=240.696mm]{./F1_M_PP_M2005_page3_images/image001.eps}
\end{center}




{\it Egzamin maturalny z matematyki}

{\it Arkusz I}

{\it 5}

Zadanie 4. $(5pkt)$

Na trzech półkach ustawiono 76 płyt kompaktowych. Okazało się, $\dot{\mathrm{z}}\mathrm{e}$ liczby płyt na półkach

gótnej, środkowej i dolnej tworzą rosnący ciąg geometryczny. Na środkowej półce stoją

24 płyty. Oblicz, ile płyt stoi na półce gótnej, a ile płyt stoi na półce dolnej.
\begin{center}
\includegraphics[width=192.588mm,height=258.720mm]{./F1_M_PP_M2005_page4_images/image001.eps}
\end{center}




{\it 6}

{\it Egzamin maturalny z matematyki}

{\it Arkusz I}

Zadanie 5. $(4pkt)$

Sklep sprowadza z hurtowni kurtki płacąc po 100 zł za sztukę i sprzedaje średnio 40 sztuk

miesięcznie po 160 zł. Zaobserwowano, $\dot{\mathrm{z}}\mathrm{e} \mathrm{k}\mathrm{a}\dot{\mathrm{z}}$ da kolejna obnizka ceny sprzedaz$\mathrm{y}$ kurtki

$01$ zł zwiększa sprzedaz miesięczną o l sztukę. Jaką cenę kurtki powinien ustalić

sprzedawca, abyjego miesięczny zysk był największy?
\begin{center}
\includegraphics[width=192.588mm,height=252.684mm]{./F1_M_PP_M2005_page5_images/image001.eps}
\end{center}




{\it Egzamin maturalny z matematyki}

{\it Arkusz I}

7

Zadanie 6. (6pkt)

Dane są zbiory liczb rzeczywistych:

$A=\{x:|x+2|\langle 3\}$

$B=\{x:(2x-1)^{3}\leq 8x^{3}-13x^{2}+6x+3\}$

Zapisz w postaci przedziałów liczbowych zbiory $A, B, A\cap B$ oraz $B-A.$
\begin{center}
\includegraphics[width=192.588mm,height=240.696mm]{./F1_M_PP_M2005_page6_images/image001.eps}
\end{center}




{\it 8}

{\it Egzamin maturalny z matematyki}

{\it Arkusz I}

Zadanie 7. (5pkt)

W ponizszej tabeli przedstawiono wyniki sondazu przeprowadzonego w grupie uczniów,

dotyczącego czasu przeznaczanego dziennie na przygotowanie zadań domowych.
\begin{center}
\begin{tabular}{|l|l|l|l|l|}
\hline
\multicolumn{1}{|l|}{$\begin{array}{l}\mbox{Czas}	\\	\mbox{(w godzinach)}	\end{array}$}&	\multicolumn{1}{|l|}{ $1$}&	\multicolumn{1}{|l|}{ $2$}&	\multicolumn{1}{|l|}{ $3$}&	\multicolumn{1}{|l|}{ $4$}	\\
\hline
\multicolumn{1}{|l|}{$\begin{array}{l}\mbox{Liczba}	\\	\mbox{uczniów}	\end{array}$}&	\multicolumn{1}{|l|}{ $5$}&	\multicolumn{1}{|l|}{ $10$}&	\multicolumn{1}{|l|}{ $15$}&	\multicolumn{1}{|l|}{ $10$}	\\
\hline
\end{tabular}

\end{center}
a) Naszkicuj diagram s

wyniki tego sondazu.

pkowy ilustrujący

b) Oblicz średnią liczbę godzin, jaką

uczniowie przeznaczają dziennie na

przygotowanie zadań domowych.
\begin{center}
\includegraphics[width=96.972mm,height=96.924mm]{./F1_M_PP_M2005_page7_images/image001.eps}
\end{center}
c)

Oblicz wariancję i odchylenie

standardowe czasu przeznaczonego

dziennie na przygotowanie zadań

domowych. Wynik podaj z dokładnością

do 0,01.
\begin{center}
\includegraphics[width=192.588mm,height=126.492mm]{./F1_M_PP_M2005_page7_images/image002.eps}
\end{center}




{\it Egzamin maturalny z matematyki}

{\it Arkusz I}

{\it 9}

Zadanie 8. (6pkt)

Z kawałka materiału o kształcie i wymiarach

czworokąta ABCD (patrz na rysunek obok)

wycięto okrągłą serwetkę o promieniu 3 dm.

Oblicz, ile procent całego materiału stanowi

jego niewykorzystana część. Wynik podaj

z dokładnością do 0,01 procenta.
\begin{center}
\includegraphics[width=71.376mm,height=82.092mm]{./F1_M_PP_M2005_page8_images/image001.eps}
\end{center}
{\it c}

{\it D}

10

{\it o}

3
\begin{center}
\includegraphics[width=192.588mm,height=204.624mm]{./F1_M_PP_M2005_page8_images/image002.eps}
\end{center}




$ 1\theta$

{\it Egzamin maturalny z matematyki}

{\it Arkusz I}

Zadanie 9. (6pkt)
\begin{center}
\includegraphics[width=193.644mm,height=280.620mm]{./F1_M_PP_M2005_page9_images/image001.eps}
\end{center}
Rodzeństwo w wieku 8 $\mathrm{i} 10$ lat otrzymało razem w spadku 84100 zł. Kwotę tę złozono

w banku, który stosuje kapitalizację roczną przy rocznej stopie procentowej 5\%. $\mathrm{K}\mathrm{a}\dot{\mathrm{z}}$ de

z dzieci otrzyma swoją część spadku z chwilą osiągnięcia wieku 211at. $\dot{\mathrm{Z}}$ yczeniem

spadkodawcy było takie podzielenie kwoty spadku, aby w przyszłości obie wypłacone części

spadku zaokrąglone do l zł były równe. Jak nalez$\mathrm{y}$ podzielić kwotę 84100 zł między

rodzeńs $0$? Za isz wszystkie wykon ane obliczenia.



\end{document}