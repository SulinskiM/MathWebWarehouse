\documentclass[a4paper,12pt]{article}
\usepackage{latexsym}
\usepackage{amsmath}
\usepackage{amssymb}
\usepackage{graphicx}
\usepackage{wrapfig}
\pagestyle{plain}
\usepackage{fancybox}
\usepackage{bm}

\begin{document}

{\it 2}

{\it Egzamin maturalny z matematyki}

{\it Arkusz I}

Zadanie 1. (3pkt)

W pudełku są trzy kule białe i pięć kul czarnych. Do pudełka mozna albo dołozyć jedną kulę

białą albo usunąč z niegojedną kulę czarn4 a następnie wy1osować z tego pudełkajedną ku1ę.

W którym z tych przypadków wylosowanie kuli białej jest bardziej prawdopodobne?

Wykonaj odpowiednie obliczenia.
\begin{center}
\includegraphics[width=192.588mm,height=252.684mm]{./F1_M_PP_M2005_page1_images/image001.eps}
\end{center}\end{document}
