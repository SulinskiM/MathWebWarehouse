\documentclass[a4paper,12pt]{article}
\usepackage{latexsym}
\usepackage{amsmath}
\usepackage{amssymb}
\usepackage{graphicx}
\usepackage{wrapfig}
\pagestyle{plain}
\usepackage{fancybox}
\usepackage{bm}

\begin{document}

{\it Egzamin maturalny z matematyki}

{\it Arkusz I}

{\it 5}

Zadanie 4. $(5pkt)$

Na trzech półkach ustawiono 76 płyt kompaktowych. Okazało się, $\dot{\mathrm{z}}\mathrm{e}$ liczby płyt na półkach

gótnej, środkowej i dolnej tworzą rosnący ciąg geometryczny. Na środkowej półce stoją

24 płyty. Oblicz, ile płyt stoi na półce gótnej, a ile płyt stoi na półce dolnej.
\begin{center}
\includegraphics[width=192.588mm,height=258.720mm]{./F1_M_PP_M2005_page4_images/image001.eps}
\end{center}\end{document}
