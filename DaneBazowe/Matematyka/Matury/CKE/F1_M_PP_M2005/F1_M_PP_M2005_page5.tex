\documentclass[a4paper,12pt]{article}
\usepackage{latexsym}
\usepackage{amsmath}
\usepackage{amssymb}
\usepackage{graphicx}
\usepackage{wrapfig}
\pagestyle{plain}
\usepackage{fancybox}
\usepackage{bm}

\begin{document}

{\it 6}

{\it Egzamin maturalny z matematyki}

{\it Arkusz I}

Zadanie 5. $(4pkt)$

Sklep sprowadza z hurtowni kurtki płacąc po 100 zł za sztukę i sprzedaje średnio 40 sztuk

miesięcznie po 160 zł. Zaobserwowano, $\dot{\mathrm{z}}\mathrm{e} \mathrm{k}\mathrm{a}\dot{\mathrm{z}}$ da kolejna obnizka ceny sprzedaz$\mathrm{y}$ kurtki

$01$ zł zwiększa sprzedaz miesięczną o l sztukę. Jaką cenę kurtki powinien ustalić

sprzedawca, abyjego miesięczny zysk był największy?
\begin{center}
\includegraphics[width=192.588mm,height=252.684mm]{./F1_M_PP_M2005_page5_images/image001.eps}
\end{center}\end{document}
