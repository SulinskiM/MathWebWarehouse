\documentclass[a4paper,12pt]{article}
\usepackage{latexsym}
\usepackage{amsmath}
\usepackage{amssymb}
\usepackage{graphicx}
\usepackage{wrapfig}
\pagestyle{plain}
\usepackage{fancybox}
\usepackage{bm}

\begin{document}

{\it Egzamin maturalny z matematyki}

{\it Arkusz I}

7

Zadanie 6. (6pkt)

Dane są zbiory liczb rzeczywistych:

$A=\{x:|x+2|\langle 3\}$

$B=\{x:(2x-1)^{3}\leq 8x^{3}-13x^{2}+6x+3\}$

Zapisz w postaci przedziałów liczbowych zbiory $A, B, A\cap B$ oraz $B-A.$
\begin{center}
\includegraphics[width=192.588mm,height=240.696mm]{./F1_M_PP_M2005_page6_images/image001.eps}
\end{center}\end{document}
