\documentclass[a4paper,12pt]{article}
\usepackage{latexsym}
\usepackage{amsmath}
\usepackage{amssymb}
\usepackage{graphicx}
\usepackage{wrapfig}
\pagestyle{plain}
\usepackage{fancybox}
\usepackage{bm}

\begin{document}

{\it 8}

{\it Egzamin maturalny z matematyki}

{\it Arkusz I}

Zadanie 7. (5pkt)

W ponizszej tabeli przedstawiono wyniki sondazu przeprowadzonego w grupie uczniów,

dotyczącego czasu przeznaczanego dziennie na przygotowanie zadań domowych.
\begin{center}
\begin{tabular}{|l|l|l|l|l|}
\hline
\multicolumn{1}{|l|}{$\begin{array}{l}\mbox{Czas}	\\	\mbox{(w godzinach)}	\end{array}$}&	\multicolumn{1}{|l|}{ $1$}&	\multicolumn{1}{|l|}{ $2$}&	\multicolumn{1}{|l|}{ $3$}&	\multicolumn{1}{|l|}{ $4$}	\\
\hline
\multicolumn{1}{|l|}{$\begin{array}{l}\mbox{Liczba}	\\	\mbox{uczniów}	\end{array}$}&	\multicolumn{1}{|l|}{ $5$}&	\multicolumn{1}{|l|}{ $10$}&	\multicolumn{1}{|l|}{ $15$}&	\multicolumn{1}{|l|}{ $10$}	\\
\hline
\end{tabular}

\end{center}
a) Naszkicuj diagram s

wyniki tego sondazu.

pkowy ilustrujący

b) Oblicz średnią liczbę godzin, jaką

uczniowie przeznaczają dziennie na

przygotowanie zadań domowych.
\begin{center}
\includegraphics[width=96.972mm,height=96.924mm]{./F1_M_PP_M2005_page7_images/image001.eps}
\end{center}
c)

Oblicz wariancję i odchylenie

standardowe czasu przeznaczonego

dziennie na przygotowanie zadań

domowych. Wynik podaj z dokładnością

do 0,01.
\begin{center}
\includegraphics[width=192.588mm,height=126.492mm]{./F1_M_PP_M2005_page7_images/image002.eps}
\end{center}\end{document}
