\documentclass[a4paper,12pt]{article}
\usepackage{latexsym}
\usepackage{amsmath}
\usepackage{amssymb}
\usepackage{graphicx}
\usepackage{wrapfig}
\pagestyle{plain}
\usepackage{fancybox}
\usepackage{bm}

\begin{document}

{\it Egzamin maturalny z matematyki}

{\it Arkusz I}

{\it 9}

Zadanie 8. (6pkt)

Z kawałka materiału o kształcie i wymiarach

czworokąta ABCD (patrz na rysunek obok)

wycięto okrągłą serwetkę o promieniu 3 dm.

Oblicz, ile procent całego materiału stanowi

jego niewykorzystana część. Wynik podaj

z dokładnością do 0,01 procenta.
\begin{center}
\includegraphics[width=71.376mm,height=82.092mm]{./F1_M_PP_M2005_page8_images/image001.eps}
\end{center}
{\it c}

{\it D}

10

{\it o}

3
\begin{center}
\includegraphics[width=192.588mm,height=204.624mm]{./F1_M_PP_M2005_page8_images/image002.eps}
\end{center}\end{document}
