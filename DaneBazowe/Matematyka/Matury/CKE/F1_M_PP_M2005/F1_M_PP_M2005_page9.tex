\documentclass[a4paper,12pt]{article}
\usepackage{latexsym}
\usepackage{amsmath}
\usepackage{amssymb}
\usepackage{graphicx}
\usepackage{wrapfig}
\pagestyle{plain}
\usepackage{fancybox}
\usepackage{bm}

\begin{document}

$ 1\theta$

{\it Egzamin maturalny z matematyki}

{\it Arkusz I}

Zadanie 9. (6pkt)
\begin{center}
\includegraphics[width=193.644mm,height=280.620mm]{./F1_M_PP_M2005_page9_images/image001.eps}
\end{center}
Rodzeństwo w wieku 8 $\mathrm{i} 10$ lat otrzymało razem w spadku 84100 zł. Kwotę tę złozono

w banku, który stosuje kapitalizację roczną przy rocznej stopie procentowej 5\%. $\mathrm{K}\mathrm{a}\dot{\mathrm{z}}$ de

z dzieci otrzyma swoją część spadku z chwilą osiągnięcia wieku 211at. $\dot{\mathrm{Z}}$ yczeniem

spadkodawcy było takie podzielenie kwoty spadku, aby w przyszłości obie wypłacone części

spadku zaokrąglone do l zł były równe. Jak nalez$\mathrm{y}$ podzielić kwotę 84100 zł między

rodzeńs $0$? Za isz wszystkie wykon ane obliczenia.
\end{document}
