\documentclass[a4paper,12pt]{article}
\usepackage{latexsym}
\usepackage{amsmath}
\usepackage{amssymb}
\usepackage{graphicx}
\usepackage{wrapfig}
\pagestyle{plain}
\usepackage{fancybox}
\usepackage{bm}

\begin{document}
\begin{center}
\begin{tabular}{l|l}
\multicolumn{1}{l|}{$\begin{array}{l}\mbox{{\it dysleksja}}	\\	\mbox{Miejsce}	\\	\mbox{na na ejkę}	\\	\mbox{z kodem szkoly}	\end{array}$}&	\multicolumn{1}{|l}{MMA-PIAIP-062}	\\
\hline
\multicolumn{1}{l|}{ $\begin{array}{l}\mbox{EGZAMIN MATURALNY}	\\	\mbox{Z MATEMATYKI}	\\	\mbox{Arkusz I}	\\	\mbox{POZIOM PODSTAWOWY}	\\	\mbox{Czas pracy 120 minut}	\\	\mbox{Instrukcja dla zdającego}	\\	\mbox{1. Sprawdzí, czy arkusz egzaminacyjny zawiera 14 stron (zadania}	\\	\mbox{$1-11)$. Ewentualny brak zgłoś przewodniczącemu zespo}	\\	\mbox{nadzorującego egzamin.}	\\	\mbox{2. Rozwiązania zadań i odpowiedzi zamieść w miejscu na to}	\\	\mbox{przeznaczonym.}	\\	\mbox{3. $\mathrm{W}$ rozwiązaniach zadań przedstaw tok rozumowania}	\\	\mbox{prowadzący do ostatecznego wyniku.}	\\	\mbox{4. Pisz czytelnie. $\mathrm{U}\dot{\mathrm{z}}$ aj długopisu pióra tylko z czarnym}	\\	\mbox{tusze atramentem.}	\\	\mbox{5. Nie uzywaj korektora, a błędne zapisy prze eśl.}	\\	\mbox{6. Pamiętaj, $\dot{\mathrm{z}}\mathrm{e}$ zapisy w $\mathrm{b}$ dnopisie nie podlegają ocenie.}	\\	\mbox{7. Obok $\mathrm{k}\mathrm{a}\dot{\mathrm{z}}$ dego zadania podanajest maksymalna liczba punktów,}	\\	\mbox{którą mozesz uzyskać zajego poprawne rozwiązanie.}	\\	\mbox{8. $\mathrm{M}\mathrm{o}\dot{\mathrm{z}}$ esz korzystać z zestawu wzorów matematycznych, cyrkla}	\\	\mbox{i linijki oraz kalkulatora.}	\\	\mbox{9. Wypełnij tę część ka $\mathrm{y}$ odpowiedzi, którą koduje zdający.}	\\	\mbox{Nie wpisuj $\dot{\mathrm{z}}$ adnych znaków w części przeznaczonej dla}	\\	\mbox{egzaminatora.}	\\	\mbox{10. Na karcie odpowiedzi wpisz swoją datę urodzenia i PESEL.}	\\	\mbox{Zamaluj $\blacksquare$ pola odpowiadające cyfrom numeru PESEL. Błędne}	\\	\mbox{zaznaczenie otocz kółkiem $\mathrm{O}$ i zaznacz właściwe.}	\\	\mbox{{\it Zyczymy} $p\theta wodzenia'$}	\end{array}$}&	\multicolumn{1}{|l}{$\begin{array}{l}\mbox{ARKUSZ I}	\\	\mbox{MAJ}	\\	\mbox{ROK 2006}	\\	\mbox{Za rozwiązanie}	\\	\mbox{wszystkich zadań}	\\	\mbox{mozna otrzymać}	\\	\mbox{łącznie}	\\	\mbox{50 punktów}	\end{array}$}	\\
\hline
\multicolumn{1}{l|}{$\begin{array}{l}\mbox{Wypelnia zdający przed}	\\	\mbox{roz oczęciem racy}	\\	\mbox{PESEL ZDAJACEGO}	\end{array}$}&	\multicolumn{1}{|l}{$\begin{array}{l}\mbox{KOD}	\\	\mbox{ZDAJACEGO}	\end{array}$}
\end{tabular}


\includegraphics[width=21.840mm,height=9.852mm]{./F1_M_PP_M2006_page0_images/image001.eps}

\includegraphics[width=78.792mm,height=13.356mm]{./F1_M_PP_M2006_page0_images/image002.eps}
\end{center}



{\it 2}

{\it Egzamin maturalny z matematyki}

{\it Arkusz I}

Zadanie l. $(3pkt)$

Dane są zbiory: $A=\{x\in R:|x-4|\geq 7\}, B=\{x\in R$:

a) zbiór $A,$

b) zbiór $B,$

c) zbiór $C=B\backslash A.$

$x^{2}>0$. Zaznacz na osi liczbowej:

a)

b)

c)
\begin{center}
\includegraphics[width=192.072mm,height=72.240mm]{./F1_M_PP_M2006_page1_images/image001.eps}

\includegraphics[width=192.072mm,height=72.240mm]{./F1_M_PP_M2006_page1_images/image002.eps}

\includegraphics[width=192.072mm,height=72.240mm]{./F1_M_PP_M2006_page1_images/image003.eps}

\includegraphics[width=109.932mm,height=17.580mm]{./F1_M_PP_M2006_page1_images/image004.eps}
\end{center}
WypelnÍa

egzaminator!

Nr czynności

Maks. liczba kt

1.1.

1

1.2.

1.3.

1

Uzyskana liczba pkt





{\it Egzamin maturalny z matematyki}

{\it Arkusz I}

{\it 11}

Zadanie 10. $(6pkt)$

Liczby 3 $\mathrm{i}-1$ sąpierwiastkami wielomianu $W(x)=2x^{3}+ax^{2}+bx+30.$

a) Wyznacz wartości współczynników a $\mathrm{i}b.$

b) Oblicz trzeci pierwiastek tego wielomianu.
\begin{center}
\includegraphics[width=192.276mm,height=260.508mm]{./F1_M_PP_M2006_page10_images/image001.eps}

\includegraphics[width=151.788mm,height=17.580mm]{./F1_M_PP_M2006_page10_images/image002.eps}
\end{center}
Wypelnia

egzamÍnator!

Nr czynności

Maks. liczba kt

10.1.

1

10.2.

1

10.3.

1

10.4.

1

10.5.

1

1

Uzyskana liczba pkt





{\it 12}

{\it Egzamin maturalny z matematyki}

{\it Arkusz I}

Zadanie ll. $(3pkt)$

Sumę $S=\displaystyle \frac{3}{1\cdot 4}+\frac{3}{4\cdot 7}+\frac{3}{7\cdot 10}+\ldots+\frac{3}{301\cdot 304}+\frac{3}{304\cdot 307}$ mozna obliczyć w następujący sposób:

a) sumę $S$ zapisujemy w postaci

$S=\displaystyle \frac{4-1}{4\cdot 1}+\frac{7-4}{7\cdot 4}+\frac{10-7}{10\cdot 7}+\ldots+\frac{304-301}{304\cdot 301}+\frac{307-304}{307\cdot 304}$

b) $\mathrm{k}\mathrm{a}\dot{\mathrm{z}}\mathrm{d}\mathrm{y}$ składnik tej sumy przedstawiamy jako róznicę ułamków

$S=(\displaystyle \frac{4}{4\cdot 1}-\frac{1}{4\cdot 1})+(\frac{7}{7\cdot 4}-\frac{4}{7\cdot 4})+(\frac{10}{10\cdot 7}-\frac{7}{10\cdot 7})+\ldots+(\frac{304}{304\cdot 301}-\frac{301}{304\cdot 301})+(\frac{307}{307\cdot 304}-\frac{304}{307\cdot 304})$

stąd $S=(1-\displaystyle \frac{1}{4})+(\frac{1}{4}-\frac{1}{7})+(\frac{1}{7}-\frac{1}{10})+\ldots+(\frac{1}{301}-\frac{1}{304})+(\frac{1}{304}-\frac{1}{307})$

więc $S=1-\displaystyle \frac{1}{4}+\frac{1}{4}-\frac{1}{7}+\frac{1}{7}-\frac{1}{10}+\ldots+\frac{1}{301}-\frac{1}{304}+\frac{1}{304}-\frac{1}{307}$

c) obliczamy sumę, redukując parami wyrazy sąsiednie, poza pierwszym i ostatnim

$S=1-\displaystyle \frac{1}{307}=\frac{306}{307}.$

Postępując w analogiczny sposób, oblicz sumę $S_{1}=\displaystyle \frac{4}{1\cdot 5}+\frac{4}{5\cdot 9}+\frac{4}{9\cdot 13}+\ldots+\frac{4}{281\cdot 285}$
\begin{center}
\includegraphics[width=192.228mm,height=193.956mm]{./F1_M_PP_M2006_page11_images/image001.eps}
\end{center}




{\it Egzamin maturalny z matematyki}

{\it Arkusz I}

{\it 13}
\begin{center}
\includegraphics[width=192.276mm,height=290.784mm]{./F1_M_PP_M2006_page12_images/image001.eps}

\includegraphics[width=109.980mm,height=17.580mm]{./F1_M_PP_M2006_page12_images/image002.eps}
\end{center}
Nr czynno\S ci

WypelnÍa Maks. liczba kt

egzaminator! Uzyskana liczba pkt

11.1.

1

11.2.

11.3.

1





{\it 14}

{\it Egzamin maturalny z matematyki}

{\it Arkusz I}

BRUDNOPIS





{\it Egzamin maturalny z matematyki}

{\it Arkusz I}

{\it 3}
\begin{center}
\includegraphics[width=192.276mm,height=289.200mm]{./F1_M_PP_M2006_page2_images/image001.eps}
\end{center}
Zadanie 2. $(3pkt)$

$\mathrm{W}$ wycieczce szkolnej bierze udział 16 uczniów, wśród których ty1ko czworo zna oko1icę.

Wychowawca chce wybrać w sposób losowy 3 osoby, które mają pójść do sk1epu. Ob1icz

prawdopodobieństwo tego, $\dot{\mathrm{z}}\mathrm{e}$ wśród wybranych trzech osób będą dokładnie dwie znające

okolicę.
\begin{center}
\includegraphics[width=109.980mm,height=17.580mm]{./F1_M_PP_M2006_page2_images/image002.eps}
\end{center}
Nr czynno\S ci

WypelnÍa Maks. liczba $\llcorner\prime \mathrm{t}$

egzaminator! Uzyskana liczba pkt

2.1.

1

2.2.

2.3.

1





{\it 4}

{\it Egzamin maturalny z matematyki}

{\it Arkusz I}

Zadanie 3. (5pkt)

Kostka masła produkowanego przez pewien zakład mleczarski ma nominalną masę

20 dag. W czasie kontroli zakładu zwazono l50 losowo wybranych kostek masła. Wyniki

badań przedstawiono w tabeli.
\begin{center}
\begin{tabular}{|l|l|l|l|l|l|l|}
\hline
\multicolumn{1}{|l|}{Masa kostki masła (w dag)}&	\multicolumn{1}{|l|}{$16$}&	\multicolumn{1}{|l|}{ $18$}&	\multicolumn{1}{|l|}{ $19$}&	\multicolumn{1}{|l|}{ $20$}&	\multicolumn{1}{|l|}{ $21$}&	\multicolumn{1}{|l|}{ $22$}	\\
\hline
\multicolumn{1}{|l|}{Liczba kostek masła}&	\multicolumn{1}{|l|}{$1$}&	\multicolumn{1}{|l|}{ $15$}&	\multicolumn{1}{|l|}{ $24$}&	\multicolumn{1}{|l|}{ $68$}&	\multicolumn{1}{|l|}{ $26$}&	\multicolumn{1}{|l|}{ $16$}	\\
\hline
\end{tabular}

\end{center}
a) Na podstawie danych przedstawionych w tabeli oblicz średnią arytmetyczną oraz

odchylenie standardowe masy kostki masła.

b) Kontrola wypada pozytywnie, jeśli średnia masa kostki masła jest równa masie

nominalnej i odchylenie standardowe nie przekracza l dag. Czy kontrola zakładu

wypadła pozytywnie? Odpowiedzí uzasadnij.
\begin{center}
\includegraphics[width=192.228mm,height=212.088mm]{./F1_M_PP_M2006_page3_images/image001.eps}

\includegraphics[width=109.932mm,height=17.580mm]{./F1_M_PP_M2006_page3_images/image002.eps}
\end{center}
Wypelnia

egzaminator!

Nr czynnoŚci

Maks. liczba kt

3.1.

2

3.2.

2

3.3.

Uzyskana liczba pkt





{\it Egzamin maturalny z matematyki}

{\it Arkusz I}

{\it 5}
\begin{center}
\includegraphics[width=192.276mm,height=286.668mm]{./F1_M_PP_M2006_page4_images/image001.eps}
\end{center}
Zadanie 4. $(4pkt)$

Dany jest rosnący ciąg geometryczny, w którym $a_{1}=12, a_{3}=27.$

a) Wyznacz iloraz tego ciągu.

b) Zapisz wzór, na podstawie którego mozna obliczyć wyraz $a_{n}$, dla $\mathrm{k}\mathrm{a}\dot{\mathrm{z}}$ dej liczby naturalnej

$n\geq 1.$

c) Oblicz wyraz $a_{6}.$
\begin{center}
\includegraphics[width=109.980mm,height=17.580mm]{./F1_M_PP_M2006_page4_images/image002.eps}
\end{center}
Nr czynności

Wypelnia Maks. liczba $\llcorner\prime \mathrm{t}$

egzaminator! Uzyskana liczba pkt

4.1.

2

4.2.

1

4.3.





{\it 6}

{\it Egzamin maturalny z matematyki}

{\it Arkusz I}

Zadanie 5. $(3pkt)$

Wiedząc, $\dot{\mathrm{z}}\mathrm{e}0^{\mathrm{o}}\leq\alpha\leq 360^{\mathrm{o}}, \sin\alpha<0$ oraz 4 tg $\alpha=3\sin^{2}\alpha+3\cos^{2}\alpha$

a) oblicz $\mathrm{t}\mathrm{g}\alpha,$

b) zaznacz w układzie współrzędnych kąt $\alpha$ i podaj współrzędne dowolnego punktu,

róznego od początku układu współrzędnych, który lezy na końcowym ramieniu tego
\begin{center}
\includegraphics[width=84.024mm,height=108.408mm]{./F1_M_PP_M2006_page5_images/image001.eps}
\end{center}
kąta.
\begin{center}
\includegraphics[width=90.732mm,height=145.488mm]{./F1_M_PP_M2006_page5_images/image002.eps}

\includegraphics[width=109.932mm,height=17.628mm]{./F1_M_PP_M2006_page5_images/image003.eps}
\end{center}
Wypelnia

egzaminator!

Nr czynnoŚci

Maks. liczba kl

5.1.

1

5.2.

1

5.3.

Uzyskana liczba pkt





{\it Egzamin maturalny z matematyki}

{\it Arkusz I}

7

Zadanie 6. $(7pkt)$

Państwo Nowakowie przeznaczyli 26000 zł na zakup działki. Do jednej z ofert dołączono

rysunek dwóch przylegających do siebie działek w skali 1:1000. Jeden metr kwadratowy

gruntu w tej ofercie kosztuje 35 zł. Ob1icz, czy przeznaczona przez państwa Nowaków kwota

wystarczy na zakup działki $\mathrm{P}_{2}.$

E
\begin{center}
\includegraphics[width=126.240mm,height=55.020mm]{./F1_M_PP_M2006_page6_images/image001.eps}
\end{center}
D

$\mathrm{P}_{1}$

$\mathrm{P}_{2}$

A  B  C

AE $=5$ cm,

EC $=13$ cm,

BC $=6,5$ cm.
\begin{center}
\includegraphics[width=193.044mm,height=193.908mm]{./F1_M_PP_M2006_page6_images/image002.eps}

\includegraphics[width=165.756mm,height=17.580mm]{./F1_M_PP_M2006_page6_images/image003.eps}
\end{center}
Wypelnia

egzaminator!

Nr czynnoŚci

Maks. liczba kt

1

1

1

1

1

1

Uzyskana liczba pkt





{\it 8}

{\it Egzamin maturalny z matematyki}

{\it Arkusz I}

Zadanie 7. $(5pkt)$

Szkic przedstawia kanał ciepłowniczy, którego przekrój poprzeczny jest prostokątem.

Wewnątrz kanału znajduje się rurociąg składający się z trzech rur, $\mathrm{k}\mathrm{a}\dot{\mathrm{z}}$ da o średnicy

zewnętrznej l $\mathrm{m}$. Oblicz wysokość i szerokość kanału ciepłowniczego. Wysokość zaokrąglij

do 0,01 $\mathrm{m}.$
\begin{center}
\includegraphics[width=192.636mm,height=97.380mm]{./F1_M_PP_M2006_page7_images/image001.eps}

\includegraphics[width=192.228mm,height=157.632mm]{./F1_M_PP_M2006_page7_images/image002.eps}

\includegraphics[width=123.900mm,height=17.580mm]{./F1_M_PP_M2006_page7_images/image003.eps}
\end{center}
Wypelnia

egzamÍnator!

Nr czynności

Maks. liczba kt

7.1.

1

7.2.

1

7.3.

2

7.4.

1

Uzyskana liczba pkt





{\it Egzamin maturalny z matematyki}

{\it Arkusz I}

{\it 9}

Zadanie 8. $(5pkt)$

Dana jest ffinkcja $f(x) =-x^{2} +6x-5.$

a) Naszkicuj wykres funkcji $f$ i podaj jej zbiór wartości.

b) Podaj rozwiązanie nierówności $f(x)\geq 0.$
\begin{center}
\includegraphics[width=176.940mm,height=103.476mm]{./F1_M_PP_M2006_page8_images/image001.eps}

\includegraphics[width=192.276mm,height=151.584mm]{./F1_M_PP_M2006_page8_images/image002.eps}

\includegraphics[width=137.820mm,height=17.580mm]{./F1_M_PP_M2006_page8_images/image003.eps}
\end{center}
WypelnÍa

egzaminator!

Nr czynności

Maks. liczba kt

8.1.

1

8.2.

1

8.3.

1

8.4.

1

8.5.

1

Uzyskana liczba pkt





$ 1\theta$

{\it Egzamin maturalny z matematyki}

{\it Arkusz I}

Zadanie 9. $(6pkt)$

Dach wiez$\mathrm{y}$ ma kształt powierzchni bocznej ostrosłupa prawidłowego czworokątnego,

którego krawędzí podstawy ma długość 4 $\mathrm{m}$. Ściana boczna tego ostrosłupajest nachylona do

płaszczyzny podstawy pod kątem $60^{\mathrm{o}}$

a) Sporządz$\acute{}$ pomocniczy rysunek i zaznacz na nim podane w zadaniu wielkości.

b) Oblicz, ile sztuk dachówek nalez$\mathrm{y}$ kupić, aby pokryć ten dach, wiedząc, $\dot{\mathrm{z}}\mathrm{e}$ do pokrycia

$1\mathrm{m}^{2}$ potrzebne są24 dachówki. Przy zakupie na1ez$\mathrm{y}$ doliczyć 8\% dachówek na zapas.
\begin{center}
\includegraphics[width=192.228mm,height=242.364mm]{./F1_M_PP_M2006_page9_images/image001.eps}

\includegraphics[width=137.868mm,height=17.628mm]{./F1_M_PP_M2006_page9_images/image002.eps}
\end{center}
Nr czynności

Wypelnia Maks. liczba kt

egzaminator! Uzyskana liczba pkt

1

1

1

2

1



\end{document}