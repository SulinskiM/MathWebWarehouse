\documentclass[a4paper,12pt]{article}
\usepackage{latexsym}
\usepackage{amsmath}
\usepackage{amssymb}
\usepackage{graphicx}
\usepackage{wrapfig}
\pagestyle{plain}
\usepackage{fancybox}
\usepackage{bm}

\begin{document}

{\it 12}

{\it Egzamin maturalny z matematyki}

{\it Arkusz I}

Zadanie ll. $(3pkt)$

Sumę $S=\displaystyle \frac{3}{1\cdot 4}+\frac{3}{4\cdot 7}+\frac{3}{7\cdot 10}+\ldots+\frac{3}{301\cdot 304}+\frac{3}{304\cdot 307}$ mozna obliczyć w następujący sposób:

a) sumę $S$ zapisujemy w postaci

$S=\displaystyle \frac{4-1}{4\cdot 1}+\frac{7-4}{7\cdot 4}+\frac{10-7}{10\cdot 7}+\ldots+\frac{304-301}{304\cdot 301}+\frac{307-304}{307\cdot 304}$

b) $\mathrm{k}\mathrm{a}\dot{\mathrm{z}}\mathrm{d}\mathrm{y}$ składnik tej sumy przedstawiamy jako róznicę ułamków

$S=(\displaystyle \frac{4}{4\cdot 1}-\frac{1}{4\cdot 1})+(\frac{7}{7\cdot 4}-\frac{4}{7\cdot 4})+(\frac{10}{10\cdot 7}-\frac{7}{10\cdot 7})+\ldots+(\frac{304}{304\cdot 301}-\frac{301}{304\cdot 301})+(\frac{307}{307\cdot 304}-\frac{304}{307\cdot 304})$

stąd $S=(1-\displaystyle \frac{1}{4})+(\frac{1}{4}-\frac{1}{7})+(\frac{1}{7}-\frac{1}{10})+\ldots+(\frac{1}{301}-\frac{1}{304})+(\frac{1}{304}-\frac{1}{307})$

więc $S=1-\displaystyle \frac{1}{4}+\frac{1}{4}-\frac{1}{7}+\frac{1}{7}-\frac{1}{10}+\ldots+\frac{1}{301}-\frac{1}{304}+\frac{1}{304}-\frac{1}{307}$

c) obliczamy sumę, redukując parami wyrazy sąsiednie, poza pierwszym i ostatnim

$S=1-\displaystyle \frac{1}{307}=\frac{306}{307}.$

Postępując w analogiczny sposób, oblicz sumę $S_{1}=\displaystyle \frac{4}{1\cdot 5}+\frac{4}{5\cdot 9}+\frac{4}{9\cdot 13}+\ldots+\frac{4}{281\cdot 285}$
\begin{center}
\includegraphics[width=192.228mm,height=193.956mm]{./F1_M_PP_M2006_page11_images/image001.eps}
\end{center}\end{document}
