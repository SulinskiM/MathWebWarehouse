\documentclass[a4paper,12pt]{article}
\usepackage{latexsym}
\usepackage{amsmath}
\usepackage{amssymb}
\usepackage{graphicx}
\usepackage{wrapfig}
\pagestyle{plain}
\usepackage{fancybox}
\usepackage{bm}

\begin{document}

{\it Egzamin maturalny z matematyki}

{\it Arkusz I}

{\it 3}
\begin{center}
\includegraphics[width=192.276mm,height=289.200mm]{./F1_M_PP_M2006_page2_images/image001.eps}
\end{center}
Zadanie 2. $(3pkt)$

$\mathrm{W}$ wycieczce szkolnej bierze udział 16 uczniów, wśród których ty1ko czworo zna oko1icę.

Wychowawca chce wybrać w sposób losowy 3 osoby, które mają pójść do sk1epu. Ob1icz

prawdopodobieństwo tego, $\dot{\mathrm{z}}\mathrm{e}$ wśród wybranych trzech osób będą dokładnie dwie znające

okolicę.
\begin{center}
\includegraphics[width=109.980mm,height=17.580mm]{./F1_M_PP_M2006_page2_images/image002.eps}
\end{center}
Nr czynno\S ci

WypelnÍa Maks. liczba $\llcorner\prime \mathrm{t}$

egzaminator! Uzyskana liczba pkt

2.1.

1

2.2.

2.3.

1
\end{document}
