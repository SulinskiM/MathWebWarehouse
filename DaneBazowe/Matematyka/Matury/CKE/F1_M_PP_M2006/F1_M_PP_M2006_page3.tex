\documentclass[a4paper,12pt]{article}
\usepackage{latexsym}
\usepackage{amsmath}
\usepackage{amssymb}
\usepackage{graphicx}
\usepackage{wrapfig}
\pagestyle{plain}
\usepackage{fancybox}
\usepackage{bm}

\begin{document}

{\it 4}

{\it Egzamin maturalny z matematyki}

{\it Arkusz I}

Zadanie 3. (5pkt)

Kostka masła produkowanego przez pewien zakład mleczarski ma nominalną masę

20 dag. W czasie kontroli zakładu zwazono l50 losowo wybranych kostek masła. Wyniki

badań przedstawiono w tabeli.
\begin{center}
\begin{tabular}{|l|l|l|l|l|l|l|}
\hline
\multicolumn{1}{|l|}{Masa kostki masła (w dag)}&	\multicolumn{1}{|l|}{$16$}&	\multicolumn{1}{|l|}{ $18$}&	\multicolumn{1}{|l|}{ $19$}&	\multicolumn{1}{|l|}{ $20$}&	\multicolumn{1}{|l|}{ $21$}&	\multicolumn{1}{|l|}{ $22$}	\\
\hline
\multicolumn{1}{|l|}{Liczba kostek masła}&	\multicolumn{1}{|l|}{$1$}&	\multicolumn{1}{|l|}{ $15$}&	\multicolumn{1}{|l|}{ $24$}&	\multicolumn{1}{|l|}{ $68$}&	\multicolumn{1}{|l|}{ $26$}&	\multicolumn{1}{|l|}{ $16$}	\\
\hline
\end{tabular}

\end{center}
a) Na podstawie danych przedstawionych w tabeli oblicz średnią arytmetyczną oraz

odchylenie standardowe masy kostki masła.

b) Kontrola wypada pozytywnie, jeśli średnia masa kostki masła jest równa masie

nominalnej i odchylenie standardowe nie przekracza l dag. Czy kontrola zakładu

wypadła pozytywnie? Odpowiedzí uzasadnij.
\begin{center}
\includegraphics[width=192.228mm,height=212.088mm]{./F1_M_PP_M2006_page3_images/image001.eps}

\includegraphics[width=109.932mm,height=17.580mm]{./F1_M_PP_M2006_page3_images/image002.eps}
\end{center}
Wypelnia

egzaminator!

Nr czynnoŚci

Maks. liczba kt

3.1.

2

3.2.

2

3.3.

Uzyskana liczba pkt
\end{document}
