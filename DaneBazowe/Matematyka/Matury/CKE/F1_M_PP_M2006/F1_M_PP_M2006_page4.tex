\documentclass[a4paper,12pt]{article}
\usepackage{latexsym}
\usepackage{amsmath}
\usepackage{amssymb}
\usepackage{graphicx}
\usepackage{wrapfig}
\pagestyle{plain}
\usepackage{fancybox}
\usepackage{bm}

\begin{document}

{\it Egzamin maturalny z matematyki}

{\it Arkusz I}

{\it 5}
\begin{center}
\includegraphics[width=192.276mm,height=286.668mm]{./F1_M_PP_M2006_page4_images/image001.eps}
\end{center}
Zadanie 4. $(4pkt)$

Dany jest rosnący ciąg geometryczny, w którym $a_{1}=12, a_{3}=27.$

a) Wyznacz iloraz tego ciągu.

b) Zapisz wzór, na podstawie którego mozna obliczyć wyraz $a_{n}$, dla $\mathrm{k}\mathrm{a}\dot{\mathrm{z}}$ dej liczby naturalnej

$n\geq 1.$

c) Oblicz wyraz $a_{6}.$
\begin{center}
\includegraphics[width=109.980mm,height=17.580mm]{./F1_M_PP_M2006_page4_images/image002.eps}
\end{center}
Nr czynności

Wypelnia Maks. liczba $\llcorner\prime \mathrm{t}$

egzaminator! Uzyskana liczba pkt

4.1.

2

4.2.

1

4.3.
\end{document}
