\documentclass[a4paper,12pt]{article}
\usepackage{latexsym}
\usepackage{amsmath}
\usepackage{amssymb}
\usepackage{graphicx}
\usepackage{wrapfig}
\pagestyle{plain}
\usepackage{fancybox}
\usepackage{bm}

\begin{document}

{\it 6}

{\it Egzamin maturalny z matematyki}

{\it Arkusz I}

Zadanie 5. $(3pkt)$

Wiedząc, $\dot{\mathrm{z}}\mathrm{e}0^{\mathrm{o}}\leq\alpha\leq 360^{\mathrm{o}}, \sin\alpha<0$ oraz 4 tg $\alpha=3\sin^{2}\alpha+3\cos^{2}\alpha$

a) oblicz $\mathrm{t}\mathrm{g}\alpha,$

b) zaznacz w układzie współrzędnych kąt $\alpha$ i podaj współrzędne dowolnego punktu,

róznego od początku układu współrzędnych, który lezy na końcowym ramieniu tego
\begin{center}
\includegraphics[width=84.024mm,height=108.408mm]{./F1_M_PP_M2006_page5_images/image001.eps}
\end{center}
kąta.
\begin{center}
\includegraphics[width=90.732mm,height=145.488mm]{./F1_M_PP_M2006_page5_images/image002.eps}

\includegraphics[width=109.932mm,height=17.628mm]{./F1_M_PP_M2006_page5_images/image003.eps}
\end{center}
Wypelnia

egzaminator!

Nr czynnoŚci

Maks. liczba kl

5.1.

1

5.2.

1

5.3.

Uzyskana liczba pkt
\end{document}
