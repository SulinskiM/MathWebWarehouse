\documentclass[a4paper,12pt]{article}
\usepackage{latexsym}
\usepackage{amsmath}
\usepackage{amssymb}
\usepackage{graphicx}
\usepackage{wrapfig}
\pagestyle{plain}
\usepackage{fancybox}
\usepackage{bm}

\begin{document}

{\it 8}

{\it Egzamin maturalny z matematyki}

{\it Arkusz I}

Zadanie 7. $(5pkt)$

Szkic przedstawia kanał ciepłowniczy, którego przekrój poprzeczny jest prostokątem.

Wewnątrz kanału znajduje się rurociąg składający się z trzech rur, $\mathrm{k}\mathrm{a}\dot{\mathrm{z}}$ da o średnicy

zewnętrznej l $\mathrm{m}$. Oblicz wysokość i szerokość kanału ciepłowniczego. Wysokość zaokrąglij

do 0,01 $\mathrm{m}.$
\begin{center}
\includegraphics[width=192.636mm,height=97.380mm]{./F1_M_PP_M2006_page7_images/image001.eps}

\includegraphics[width=192.228mm,height=157.632mm]{./F1_M_PP_M2006_page7_images/image002.eps}

\includegraphics[width=123.900mm,height=17.580mm]{./F1_M_PP_M2006_page7_images/image003.eps}
\end{center}
Wypelnia

egzamÍnator!

Nr czynności

Maks. liczba kt

7.1.

1

7.2.

1

7.3.

2

7.4.

1

Uzyskana liczba pkt
\end{document}
