\documentclass[a4paper,12pt]{article}
\usepackage{latexsym}
\usepackage{amsmath}
\usepackage{amssymb}
\usepackage{graphicx}
\usepackage{wrapfig}
\pagestyle{plain}
\usepackage{fancybox}
\usepackage{bm}

\begin{document}

$ 1\theta$

{\it Egzamin maturalny z matematyki}

{\it Arkusz I}

Zadanie 9. $(6pkt)$

Dach wiez$\mathrm{y}$ ma kształt powierzchni bocznej ostrosłupa prawidłowego czworokątnego,

którego krawędzí podstawy ma długość 4 $\mathrm{m}$. Ściana boczna tego ostrosłupajest nachylona do

płaszczyzny podstawy pod kątem $60^{\mathrm{o}}$

a) Sporządz$\acute{}$ pomocniczy rysunek i zaznacz na nim podane w zadaniu wielkości.

b) Oblicz, ile sztuk dachówek nalez$\mathrm{y}$ kupić, aby pokryć ten dach, wiedząc, $\dot{\mathrm{z}}\mathrm{e}$ do pokrycia

$1\mathrm{m}^{2}$ potrzebne są24 dachówki. Przy zakupie na1ez$\mathrm{y}$ doliczyć 8\% dachówek na zapas.
\begin{center}
\includegraphics[width=192.228mm,height=242.364mm]{./F1_M_PP_M2006_page9_images/image001.eps}

\includegraphics[width=137.868mm,height=17.628mm]{./F1_M_PP_M2006_page9_images/image002.eps}
\end{center}
Nr czynności

Wypelnia Maks. liczba kt

egzaminator! Uzyskana liczba pkt

1

1

1

2

1
\end{document}
