\documentclass[a4paper,12pt]{article}
\usepackage{latexsym}
\usepackage{amsmath}
\usepackage{amssymb}
\usepackage{graphicx}
\usepackage{wrapfig}
\pagestyle{plain}
\usepackage{fancybox}
\usepackage{bm}

\begin{document}
\begin{center}
\begin{tabular}{l|l}
\multicolumn{1}{l|}{$\begin{array}{l}\mbox{{\it dysleksja}}	\\	\mbox{Miejsce}	\\	\mbox{na naklejkę}	\\	\mbox{z kodem szkoly}	\end{array}$}&	\multicolumn{1}{|l}{ $\mathrm{M}\mathrm{M}\mathrm{A}-\mathrm{P}1_{-}1\mathrm{P}-072$}	\\
\hline
\multicolumn{1}{l|}{$\begin{array}{l}\mbox{EGZAMIN MATURALNY}	\\	\mbox{Z MATEMATYKI}	\\	\mbox{POZIOM PODSTAWOWY}	\\	\mbox{Czas pracy 120 minut}	\\	\mbox{Instrukcja dla zdającego}	\\	\mbox{1. Sprawdzí, czy arkusz egzaminacyjny zawiera 15 stron (zadania}	\\	\mbox{$1-11)$. Ewentualny brak zgłoś przewodniczącemu zespołu}	\\	\mbox{nadzorującego egzamin.}	\\	\mbox{2. Rozwiązania zadań i odpowiedzi zamieść w miejscu na to}	\\	\mbox{przeznaczonym.}	\\	\mbox{3. $\mathrm{W}$ rozwiązaniach zadań przedstaw tok rozumowania}	\\	\mbox{prowadzący do ostatecznego wyniku.}	\\	\mbox{4. Pisz czytelnie. Uzywaj $\mathrm{d}$ gopisu pióra tylko z czatnym}	\\	\mbox{tusze atramentem.}	\\	\mbox{5. Nie uzywaj korektora, a błędne zapisy prze eśl.}	\\	\mbox{6. Pamiętaj, $\dot{\mathrm{z}}\mathrm{e}$ zapisy w brudnopisie nie podlegają ocenie.}	\\	\mbox{7. Obok $\mathrm{k}\mathrm{a}\dot{\mathrm{z}}$ dego zadania podanajest maksymalna liczba punktów,}	\\	\mbox{którą mozesz uzyskać zajego poprawne rozwiązanie.}	\\	\mbox{8. $\mathrm{M}\mathrm{o}\dot{\mathrm{z}}$ esz korzystać z zestawu wzorów matematycznych, cyrkla}	\\	\mbox{i linijki oraz kalkulatora.}	\\	\mbox{9. Wypełnij tę część ka $\mathrm{y}$ odpowiedzi, którą koduje zdający.}	\\	\mbox{Nie wpisuj $\dot{\mathrm{z}}$ adnych znaków w części przeznaczonej dla}	\\	\mbox{egzaminatora.}	\\	\mbox{10. Na karcie odpowiedzi wpisz swoją datę urodzenia i PESEL.}	\\	\mbox{Zamaluj $\blacksquare$ pola odpowiadające cyfrom numeru PESEL. Błędne}	\\	\mbox{zaznaczenie otocz kółkiem $\mathrm{O}$ i zaznacz właściwe.}	\\	\mbox{{\it Zyczymy powodzenia}.'}	\end{array}$}&	\multicolumn{1}{|l}{$\begin{array}{l}\mbox{MAJ}	\\	\mbox{ROK 2007}	\\	\mbox{Za rozwiązanie}	\\	\mbox{wszystkich zadań}	\\	\mbox{mozna otrzymać}	\\	\mbox{łącznie}	\\	\mbox{50 punktów}	\end{array}$}	\\
\hline
\multicolumn{1}{l|}{$\begin{array}{l}\mbox{Wypelnia zdający}	\\	\mbox{rzed roz oczęciem racy}	\\	\mbox{PESEL ZDAJACEGO}	\end{array}$}&	\multicolumn{1}{|l}{$\begin{array}{l}\mbox{KOD}	\\	\mbox{ZDAJACEGO}	\end{array}$}
\end{tabular}


\includegraphics[width=21.840mm,height=9.852mm]{./F1_M_PP_M2007_page0_images/image001.eps}

\includegraphics[width=78.792mm,height=13.356mm]{./F1_M_PP_M2007_page0_images/image002.eps}
\end{center}



{\it 2}

{\it Egzamin maturalny z matematyki}

{\it Poziom podstawowy}

Zadanie 1. (5pkt)

Znajdzí wzór funkcji kwadratowej $y=f(x)$, której wykresem jest parabola o wierzchołku

$(1,-9)$ przechodząca przez punkt o współrzędnych $(2,-8)$. Otrzymaną funkcję przedstaw

w postaci kanonicznej. Obliczjej miejsca zerowe i naszkicuj wykres.
\begin{center}
\includegraphics[width=137.868mm,height=17.628mm]{./F1_M_PP_M2007_page1_images/image001.eps}
\end{center}
Nr czynnoŚci

Wypelnia Maks. liczba kt

egzaminator! Uzyskana liczba pkt

1.1.

1

1.2.

1

1.3.

1

1.4.

1

1.5.

1





{\it Egzamin maturalny z matematyki}

{\it Poziom podstawowy}

{\it 11}
\begin{center}
\includegraphics[width=151.788mm,height=17.580mm]{./F1_M_PP_M2007_page10_images/image001.eps}
\end{center}
WypelnÍa

egzaminator!

Nr czynności

Maks. lÍczba kt

1

1

1

1

1

Uzyskana liczba pkt





{\it 12}

{\it Egzamin maturalny z matematyki}

{\it Poziom podstawowy}

Zadanie 10. (5pkt)

Dany jest graniastosłup czworokątny prosty ABCDEFGH o podstawach ABCD $\mathrm{i}$ {\it EFGH oraz}

krawędziach bocznych $AE, BF, CG, DH$. Podstawa ABCD graniastosłupajest rombem o boku

długości 8 cm i kątach ostrych $A \mathrm{i} C$ o mierze $60^{\circ}$ Przekątna graniastosłupa $CE$ jest

nachylona do płaszczyzny podstawy pod kątem $60^{\circ}$ Sporządz$\acute{}$ rysunek pomocniczy i zaznacz

na nim wymienione w zadaniu kąty. Oblicz objętość tego graniastosłupa.





{\it Egzamin maturalny z matematyki}

{\it Poziom podstawowy}

{\it 13}
\begin{center}
\includegraphics[width=137.820mm,height=17.580mm]{./F1_M_PP_M2007_page12_images/image001.eps}
\end{center}
Wypelnia

egzaminator!

Nr czynno\S ci

Maks. liczba kt

10.1.

1

10.2.

1

10.3.

1

10.4.

1

10.5.

1

Uzyskana liczba pkt





{\it 14}

{\it Egzamin maturalny z matematyki}

{\it Poziom podstawowy}

Zadanie 11. (4pkt)

Dany jest rosnący ciąg geometryczny $(a_{n})$ dla

Oblicz $x$ oraz $y, \mathrm{j}\mathrm{e}\dot{\mathrm{z}}$ eli wiadomo, $\dot{\mathrm{z}}\mathrm{e}x+y=35$

$n\geq 1$, w którym $a_{1}=x, a_{2}=14, a_{3}=y.$
\begin{center}
\includegraphics[width=123.900mm,height=17.628mm]{./F1_M_PP_M2007_page13_images/image001.eps}
\end{center}
Wypelnia

egzaminator!

Nr czynności

Maks. liczba kt

11.1.

1

1

11.4.

1

Uzyskana liczba pkt





{\it Egzamin maturalny z matematyki}

{\it Poziom podstawowy}

{\it 15}

BRUDNOPIS





{\it Egzamin maturalny z matematyki}

{\it Poziom podstawowy}

{\it 3}

Zadanie 2. (3pkt)

Wysokość prowizji, którą klient płaci w pewnym biurze maklerskim przy $\mathrm{k}\mathrm{a}\dot{\mathrm{z}}$ dej zawieranej

transakcji kupna lub sprzedaz$\mathrm{y}$ akcji jest uzalezniona od wartości transakcji. Zalezność ta

została przedstawiona w tabeli:
\begin{center}
\begin{tabular}{|l|l|}
\hline
\multicolumn{1}{|l|}{Wartość transakcji}&	\multicolumn{1}{|l|}{Wysokość rowizji}	\\
\hline
\multicolumn{1}{|l|}{do 500 zł}&	\multicolumn{1}{|l|}{15 zł}	\\
\hline
\multicolumn{1}{|l|}{od 500,01 zł do 3000 zł}&	\multicolumn{1}{|l|}{2\% wartości $\mathrm{t}\mathrm{r}\mathrm{a}\mathrm{n}\mathrm{s}\mathrm{a}\mathrm{k}\mathrm{c}\mathrm{j}\mathrm{i}+5$ zł}	\\
\hline
\multicolumn{1}{|l|}{od 3000,01 zł do 8000 zł}&	\multicolumn{1}{|l|}{1,5\% wartości $\mathrm{t}\mathrm{r}\mathrm{a}\mathrm{n}\mathrm{s}\mathrm{a}\mathrm{k}\mathrm{c}\mathrm{j}\mathrm{i}+20$ zł}	\\
\hline
\multicolumn{1}{|l|}{od 8000,01 zł do 15000 zł}&	\multicolumn{1}{|l|}{1\% wartości $\mathrm{t}\mathrm{r}\mathrm{a}\mathrm{n}\mathrm{s}\mathrm{a}\mathrm{k}\mathrm{c}\mathrm{j}\mathrm{i}+60$ zł}	\\
\hline
\multicolumn{1}{|l|}{powyzej 15000 zł}&	\multicolumn{1}{|l|}{0,7\% wartości $\mathrm{t}\mathrm{r}\mathrm{a}\mathrm{n}\mathrm{s}\mathrm{a}\mathrm{k}\mathrm{c}\mathrm{j}\mathrm{i}+105$ zł}	\\
\hline
\end{tabular}

\end{center}
Klient zakupił za pośrednictwem tego biura maklerskiego 530 akcji w cenie 25 zł za jedną

akcję. Po roku sprzedał wszystkie kupione akcje po 45 zł zajedną sztukę. Ob1icz, i1e zarobił

na tych transakcjach po uwzględnieniu prowizji, które zapłacił.
\begin{center}
\includegraphics[width=109.980mm,height=17.580mm]{./F1_M_PP_M2007_page2_images/image001.eps}
\end{center}
Nr czynności

Wypelnia Maks. liczba kt

egzaminator! Uzyskana liczba pkt

2.1.

1

2.2.

1

2.3.

1





{\it 4}

{\it Egzamin maturalny z matematyki}

{\it Poziom podstawowy}

Zadanie 3. (4pkt)

Korzystając z danych przedstawionych na rysunku, oblicz wartość wyrazenia:

$\mathrm{t}\mathrm{g}^{2}\beta-5\sin\beta$. ctg $\alpha+\sqrt{1-\cos^{2}\alpha}.$

{\it C}
\begin{center}
\includegraphics[width=81.684mm,height=35.256mm]{./F1_M_PP_M2007_page3_images/image001.eps}
\end{center}
8  6

$\beta$

{\it A}  $\alpha$  {\it B}
\begin{center}
\includegraphics[width=123.900mm,height=17.628mm]{./F1_M_PP_M2007_page3_images/image002.eps}
\end{center}
Wypelnia

egzaminator!

Nr czynności

Maks. liczba kt

3.1.

1

3.2.

1

3.3.

1

3.4.

1

Uzyskana liczba pkt





{\it Egzamin maturalny z matematyki}

{\it Poziom podstawowy}

{\it 5}

Zadanie 4. (5pkt)

Samochód przebył w pewnym czasie 210 km. Gdybyjechał ze średnią prędkością o 10 km/h

większib to czas przejazdu skróciłby się o pół godziny. Oblicz, z jaką średnią prędkością

jechał ten samochód.
\begin{center}
\includegraphics[width=137.820mm,height=17.580mm]{./F1_M_PP_M2007_page4_images/image001.eps}
\end{center}
Wypelnia

egzaminator!

Nr czynno\S ci

Maks. liczba kt

4.1.

1

4.2.

1

4.3.

1

4.4.

1

4.5.

1

Uzyskana liczba pkt





{\it 6}

{\it Egzamin maturalny z matematyki}

{\it Poziom podstawowy}

Zadanie 5. (5pkt)

Dany jest ciąg arytmetyczny $(a_{n})$, gdzie $n\geq 1$. Wiadomo, $\dot{\mathrm{z}}\mathrm{e}$ dla $\mathrm{k}\mathrm{a}\dot{\mathrm{z}}$ dego $n\geq 1$

$n$ początkowych wyrazów $S_{n}=a_{1}+a_{2}+\ldots+a_{n}$ wyraza się wzorem: $S_{n}=-n^{2}+13n.$

a) Wyznacz wzór na $n-\mathrm{t}\mathrm{y}$ wyraz ciągu $(a_{n}).$

b) Oblicz a200$7^{\cdot}$

c) Wyznacz liczbę $n$, dla której $a_{n}=0.$

suma
\begin{center}
\includegraphics[width=137.868mm,height=17.580mm]{./F1_M_PP_M2007_page5_images/image001.eps}
\end{center}
Nr czynności

Wypelnia Maks. liczba kt

egzaminator! Uzyskana liczba pkt

5.1.

5.2.

1

5.3.

1

5.4.

1

5.5.

1





{\it Egzamin maturalny z matematyki}

{\it Poziom podstawowy}

7

Zadanie 6. (4pkt)

Dany jest wielomian $W(x)=2x^{3}+ax^{2}-14x+b.$

a) Dla $a=0 \mathrm{i} b=0$ otrzymamy wielomian $W(x)=2x^{3}-14x$. Rozwiąz równanie

$2x^{3}-14x=0.$

b) Dobierz wartości $a\mathrm{i}b$ tak, aby wielomian $W(x)$ był podzielny jednocześnie przez $x-2$

oraz przez $x+3.$
\begin{center}
\includegraphics[width=123.900mm,height=17.580mm]{./F1_M_PP_M2007_page6_images/image001.eps}
\end{center}
Nr czynności

Wypelnia Maks. liczba kt

egzamÍnator! Uzyskana liczba pkt

1

1

1





{\it 8}

{\it Egzamin maturalny z matematyki}

{\it Poziom podstawowy}

Zadanie 7. (5pkt)

Dany jest punkt $C=(2,3)$ i prosta o równaniu $y=2x-8$ będąca symetralną odcinka $BC.$

Wyznacz współrzędne punktu $B$. Wykonaj obliczenia uzasadniające odpowiedz.
\begin{center}
\includegraphics[width=137.868mm,height=17.628mm]{./F1_M_PP_M2007_page7_images/image001.eps}
\end{center}
Nr czynnoŚci

Wypelnia Maks. liczba kt

egzaminator! Uzyskana liczba pkt

7.1.

7.2.

1

7.3.

1

7.4.

1

7.5.

1





{\it Egzamin maturalny z matematyki}

{\it Poziom podstawowy}

{\it 9}

Zadanie 8. (4pkt)

Na stole $\mathrm{l}\mathrm{e}\dot{\mathrm{z}}$ ało 14 banknotów: 2 banknoty o nomina1e 100 zł, 2 banknoty o nomina1e 50 zł

$\mathrm{i} 10$ banknotów o nominale 20 zł. Wiatr zdmuchnął na podłogę 5 banknotów. Ob1icz

prawdopodobieństwo tego, $\dot{\mathrm{z}}\mathrm{e}$ na podłodze lezy dokładnie 130 zł. Odpowied $\acute{\mathrm{z}}$ podaj w postaci

ułamka nieskracalnego.
\begin{center}
\includegraphics[width=123.900mm,height=17.628mm]{./F1_M_PP_M2007_page8_images/image001.eps}
\end{center}
Nr czynnoŚci

Wypelnia Maks. liczba kt

egzaminator! Uzyskana liczba pkt

8.1.

1

8.2.

8.3.

1

8.4.

1





$ 1\theta$

{\it Egzamin maturalny z matematyki}

{\it Poziom podstawowy}

Zadanie 9. (6pkt)

Oblicz pole czworokąta wypukłego ABCD, w którym kąty wewnętrzne mają odpowiednio

miary: $4A=90^{\circ}, \triangleleft B=75^{\circ}, \triangleleft C=60^{\circ}, \triangleleft D=135^{\circ}$, a boki AB $\mathrm{i}$ AD mają długość 3 cm.

Sporządzí rysunek pomocniczy.



\end{document}