\documentclass[a4paper,12pt]{article}
\usepackage{latexsym}
\usepackage{amsmath}
\usepackage{amssymb}
\usepackage{graphicx}
\usepackage{wrapfig}
\pagestyle{plain}
\usepackage{fancybox}
\usepackage{bm}

\begin{document}

{\it 2}

{\it Egzamin maturalny z matematyki}

{\it Poziom podstawowy}

Zadanie 1. (5pkt)

Znajdzí wzór funkcji kwadratowej $y=f(x)$, której wykresem jest parabola o wierzchołku

$(1,-9)$ przechodząca przez punkt o współrzędnych $(2,-8)$. Otrzymaną funkcję przedstaw

w postaci kanonicznej. Obliczjej miejsca zerowe i naszkicuj wykres.
\begin{center}
\includegraphics[width=137.868mm,height=17.628mm]{./F1_M_PP_M2007_page1_images/image001.eps}
\end{center}
Nr czynnoŚci

Wypelnia Maks. liczba kt

egzaminator! Uzyskana liczba pkt

1.1.

1

1.2.

1

1.3.

1

1.4.

1

1.5.

1
\end{document}
