\documentclass[a4paper,12pt]{article}
\usepackage{latexsym}
\usepackage{amsmath}
\usepackage{amssymb}
\usepackage{graphicx}
\usepackage{wrapfig}
\pagestyle{plain}
\usepackage{fancybox}
\usepackage{bm}

\begin{document}

{\it 12}

{\it Egzamin maturalny z matematyki}

{\it Poziom podstawowy}

Zadanie 10. (5pkt)

Dany jest graniastosłup czworokątny prosty ABCDEFGH o podstawach ABCD $\mathrm{i}$ {\it EFGH oraz}

krawędziach bocznych $AE, BF, CG, DH$. Podstawa ABCD graniastosłupajest rombem o boku

długości 8 cm i kątach ostrych $A \mathrm{i} C$ o mierze $60^{\circ}$ Przekątna graniastosłupa $CE$ jest

nachylona do płaszczyzny podstawy pod kątem $60^{\circ}$ Sporządz$\acute{}$ rysunek pomocniczy i zaznacz

na nim wymienione w zadaniu kąty. Oblicz objętość tego graniastosłupa.
\end{document}
