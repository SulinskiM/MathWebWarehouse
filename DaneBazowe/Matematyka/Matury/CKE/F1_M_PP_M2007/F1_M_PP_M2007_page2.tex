\documentclass[a4paper,12pt]{article}
\usepackage{latexsym}
\usepackage{amsmath}
\usepackage{amssymb}
\usepackage{graphicx}
\usepackage{wrapfig}
\pagestyle{plain}
\usepackage{fancybox}
\usepackage{bm}

\begin{document}

{\it Egzamin maturalny z matematyki}

{\it Poziom podstawowy}

{\it 3}

Zadanie 2. (3pkt)

Wysokość prowizji, którą klient płaci w pewnym biurze maklerskim przy $\mathrm{k}\mathrm{a}\dot{\mathrm{z}}$ dej zawieranej

transakcji kupna lub sprzedaz$\mathrm{y}$ akcji jest uzalezniona od wartości transakcji. Zalezność ta

została przedstawiona w tabeli:
\begin{center}
\begin{tabular}{|l|l|}
\hline
\multicolumn{1}{|l|}{Wartość transakcji}&	\multicolumn{1}{|l|}{Wysokość rowizji}	\\
\hline
\multicolumn{1}{|l|}{do 500 zł}&	\multicolumn{1}{|l|}{15 zł}	\\
\hline
\multicolumn{1}{|l|}{od 500,01 zł do 3000 zł}&	\multicolumn{1}{|l|}{2\% wartości $\mathrm{t}\mathrm{r}\mathrm{a}\mathrm{n}\mathrm{s}\mathrm{a}\mathrm{k}\mathrm{c}\mathrm{j}\mathrm{i}+5$ zł}	\\
\hline
\multicolumn{1}{|l|}{od 3000,01 zł do 8000 zł}&	\multicolumn{1}{|l|}{1,5\% wartości $\mathrm{t}\mathrm{r}\mathrm{a}\mathrm{n}\mathrm{s}\mathrm{a}\mathrm{k}\mathrm{c}\mathrm{j}\mathrm{i}+20$ zł}	\\
\hline
\multicolumn{1}{|l|}{od 8000,01 zł do 15000 zł}&	\multicolumn{1}{|l|}{1\% wartości $\mathrm{t}\mathrm{r}\mathrm{a}\mathrm{n}\mathrm{s}\mathrm{a}\mathrm{k}\mathrm{c}\mathrm{j}\mathrm{i}+60$ zł}	\\
\hline
\multicolumn{1}{|l|}{powyzej 15000 zł}&	\multicolumn{1}{|l|}{0,7\% wartości $\mathrm{t}\mathrm{r}\mathrm{a}\mathrm{n}\mathrm{s}\mathrm{a}\mathrm{k}\mathrm{c}\mathrm{j}\mathrm{i}+105$ zł}	\\
\hline
\end{tabular}

\end{center}
Klient zakupił za pośrednictwem tego biura maklerskiego 530 akcji w cenie 25 zł za jedną

akcję. Po roku sprzedał wszystkie kupione akcje po 45 zł zajedną sztukę. Ob1icz, i1e zarobił

na tych transakcjach po uwzględnieniu prowizji, które zapłacił.
\begin{center}
\includegraphics[width=109.980mm,height=17.580mm]{./F1_M_PP_M2007_page2_images/image001.eps}
\end{center}
Nr czynności

Wypelnia Maks. liczba kt

egzaminator! Uzyskana liczba pkt

2.1.

1

2.2.

1

2.3.

1
\end{document}
