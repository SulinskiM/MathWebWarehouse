\documentclass[a4paper,12pt]{article}
\usepackage{latexsym}
\usepackage{amsmath}
\usepackage{amssymb}
\usepackage{graphicx}
\usepackage{wrapfig}
\pagestyle{plain}
\usepackage{fancybox}
\usepackage{bm}

\begin{document}

{\it Egzamin maturalny z matematyki}

{\it Poziom podstawowy}

{\it 5}

Zadanie 4. (5pkt)

Samochód przebył w pewnym czasie 210 km. Gdybyjechał ze średnią prędkością o 10 km/h

większib to czas przejazdu skróciłby się o pół godziny. Oblicz, z jaką średnią prędkością

jechał ten samochód.
\begin{center}
\includegraphics[width=137.820mm,height=17.580mm]{./F1_M_PP_M2007_page4_images/image001.eps}
\end{center}
Wypelnia

egzaminator!

Nr czynno\S ci

Maks. liczba kt

4.1.

1

4.2.

1

4.3.

1

4.4.

1

4.5.

1

Uzyskana liczba pkt
\end{document}
