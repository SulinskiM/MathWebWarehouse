\documentclass[a4paper,12pt]{article}
\usepackage{latexsym}
\usepackage{amsmath}
\usepackage{amssymb}
\usepackage{graphicx}
\usepackage{wrapfig}
\pagestyle{plain}
\usepackage{fancybox}
\usepackage{bm}

\begin{document}

{\it 6}

{\it Egzamin maturalny z matematyki}

{\it Poziom podstawowy}

Zadanie 5. (5pkt)

Dany jest ciąg arytmetyczny $(a_{n})$, gdzie $n\geq 1$. Wiadomo, $\dot{\mathrm{z}}\mathrm{e}$ dla $\mathrm{k}\mathrm{a}\dot{\mathrm{z}}$ dego $n\geq 1$

$n$ początkowych wyrazów $S_{n}=a_{1}+a_{2}+\ldots+a_{n}$ wyraza się wzorem: $S_{n}=-n^{2}+13n.$

a) Wyznacz wzór na $n-\mathrm{t}\mathrm{y}$ wyraz ciągu $(a_{n}).$

b) Oblicz a200$7^{\cdot}$

c) Wyznacz liczbę $n$, dla której $a_{n}=0.$

suma
\begin{center}
\includegraphics[width=137.868mm,height=17.580mm]{./F1_M_PP_M2007_page5_images/image001.eps}
\end{center}
Nr czynności

Wypelnia Maks. liczba kt

egzaminator! Uzyskana liczba pkt

5.1.

5.2.

1

5.3.

1

5.4.

1

5.5.

1
\end{document}
