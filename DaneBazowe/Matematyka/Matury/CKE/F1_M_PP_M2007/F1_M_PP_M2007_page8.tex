\documentclass[a4paper,12pt]{article}
\usepackage{latexsym}
\usepackage{amsmath}
\usepackage{amssymb}
\usepackage{graphicx}
\usepackage{wrapfig}
\pagestyle{plain}
\usepackage{fancybox}
\usepackage{bm}

\begin{document}

{\it Egzamin maturalny z matematyki}

{\it Poziom podstawowy}

{\it 9}

Zadanie 8. (4pkt)

Na stole $\mathrm{l}\mathrm{e}\dot{\mathrm{z}}$ ało 14 banknotów: 2 banknoty o nomina1e 100 zł, 2 banknoty o nomina1e 50 zł

$\mathrm{i} 10$ banknotów o nominale 20 zł. Wiatr zdmuchnął na podłogę 5 banknotów. Ob1icz

prawdopodobieństwo tego, $\dot{\mathrm{z}}\mathrm{e}$ na podłodze lezy dokładnie 130 zł. Odpowied $\acute{\mathrm{z}}$ podaj w postaci

ułamka nieskracalnego.
\begin{center}
\includegraphics[width=123.900mm,height=17.628mm]{./F1_M_PP_M2007_page8_images/image001.eps}
\end{center}
Nr czynnoŚci

Wypelnia Maks. liczba kt

egzaminator! Uzyskana liczba pkt

8.1.

1

8.2.

8.3.

1

8.4.

1
\end{document}
