\documentclass[a4paper,12pt]{article}
\usepackage{latexsym}
\usepackage{amsmath}
\usepackage{amssymb}
\usepackage{graphicx}
\usepackage{wrapfig}
\pagestyle{plain}
\usepackage{fancybox}
\usepackage{bm}

\begin{document}

{\it ARKUSZ ZA WIERA INFORMACJE} $PRA$ {\it WNIE CHRONIONE}

{\it DO MOMENTU ROZPOCZĘCIA EGZAMINU}.$\displaystyle \int$
\begin{center}
\begin{tabular}{|l|l|l}
\cline{1-1}
\multicolumn{1}{|l|}{$\begin{array}{l}\mbox{Miejsce}	\\	\mbox{na na ejkę}	\end{array}$}&	\multicolumn{1}{|l|}{}&	\multicolumn{1}{|l}{ $\mathrm{M}\mathrm{M}\mathrm{A}-\mathrm{P}1_{-}1\mathrm{P}-082$}	\\
\hline
&	\multicolumn{1}{|l}{$\begin{array}{l}\mbox{MAJ}	\\	\mbox{ROK 2008}	\\	\mbox{Za rozwiązanie}	\\	\mbox{wszystkich zadań}	\\	\mbox{mozna otrzymać}	\\	\mbox{łącznie}	\\	\mbox{50 punktów}	\end{array}$}	\\
\cline{3-3}
&	\multicolumn{1}{|l}{$\begin{array}{l}\mbox{KOD}	\\	\mbox{ZDAJACEGO}	\end{array}$}
\end{tabular}


\includegraphics[width=21.840mm,height=9.852mm]{./F1_M_PP_M2008_page0_images/image001.eps}

\includegraphics[width=78.792mm,height=13.356mm]{./F1_M_PP_M2008_page0_images/image002.eps}
\end{center}



{\it 2 Egzamin maturalny z matematyki}

{\it Poziom podstawowy}

Zadanie l. $(4pkt)$

Na ponizszym rysunku przedstawiono łamaną ABCD, którajest wykresem ffinkcji $y=f(x).$
\begin{center}
\includegraphics[width=108.108mm,height=107.952mm]{./F1_M_PP_M2008_page1_images/image001.eps}
\end{center}
{\it y}

{\it C  D}

3

1

$-3 -2  -1  0 1_{1} 2_{1}$ 3  $1_{1} 2_{1}$  4  {\it x}

1

$-2$

{\it A  B}  $-4$

Korzystając z tego wykresu:

a) zapisz w postaci przedziału zbiór wartości funkcji $f,$

b) podaj wartość funkcji $f$ dla argumentu $x=1-\sqrt{10},$

c) wyznacz równanie prostej $BC,$

d) oblicz długość odcinka $BC.$





{\it Egzamin maturalny z matematyki ll}

{\it Poziom podstawowy}
\begin{center}
\includegraphics[width=123.900mm,height=17.832mm]{./F1_M_PP_M2008_page10_images/image001.eps}
\end{center}
Nr zadania

Wypelnia Maks. liczba kt

egzaminator! Uzyskana liczba pkt

7.1

1

7.2

1

7.3

1

7.4

1





{\it 12 Egzamin maturalny z matematyki}

{\it Poziom podstawowy}

Zadanie 8. $(4pkt)$

Dany jest wielomian $W(x)=x^{3}-5x^{2}-9x+45.$

a) Sprawdzí, czy punkt $A=(1$, 30$)$ nalezy do wykresu tego wielomianu.

b) Zapisz wielomian $W$ w postaci iloczynu trzech wielomianów stopnia pierwszego.
\begin{center}
\includegraphics[width=123.948mm,height=17.784mm]{./F1_M_PP_M2008_page11_images/image001.eps}
\end{center}
Nr zadania

Wypelnia Maks. liczba kt

egzaminator! Uzyskana lÍczba pkt

8.1

1

8.2

1

8.3

1

8.4

1





{\it Egzamin maturalny z matematyki 13}

{\it Poziom podstawowy}

Zadanie 9. (5pkt)

Oblicz najmniejszą i

w przedziale $\langle-2, 2\rangle.$

największą wartość

ffinkcji kwadratowej

$f(x)=(2x+1)(x-2)$
\begin{center}
\includegraphics[width=137.928mm,height=17.832mm]{./F1_M_PP_M2008_page12_images/image001.eps}
\end{center}
Wypelnia

egzaminator!

Nr zadania

Maks. liczba kt

1

1

Uzyskana liczba pkt





{\it 14 Egzamin maturalny z matematyki}

{\it Poziom podstawowy}

Zadanie 10. $(3pkt)$

Rysunek przedstawia fragment wykresu funkcji $h$, określonej wzorem $h(x)=\displaystyle \frac{a}{x}$ dla $x\neq 0.$

Wiadomo, $\dot{\mathrm{z}}\mathrm{e}$ do wykresu ffinkcji $h$ nalezy punkt $P=(2,5).$

a) Oblicz wartość współczynnika $a.$

b) Ustal, czy liczba $h(\pi)-h(-\pi)$ jest dodatnia czy ujemna.

c) Rozwiąz nierówność $h(x)>5.$
\begin{center}
\includegraphics[width=140.664mm,height=112.824mm]{./F1_M_PP_M2008_page13_images/image001.eps}
\end{center}
{\it y}

1

1  {\it x}





{\it Egzamin maturalny z matematyki 15}

{\it Poziom podstawowy}
\begin{center}
\includegraphics[width=109.980mm,height=17.832mm]{./F1_M_PP_M2008_page14_images/image001.eps}
\end{center}
Nr zadania

Wypelnia Maks. liczba kt

egzaminator! Uzyskana liczba pkt

10.1

10.2

10.3

1





{\it 16 Egzamin maturalny z matematyki}

{\it Poziom podstawowy}

Zadanie ll. $(5pkt)$

Pole powierzchni bocznej ostrosłupa prawidłowego trójkątnego równa się $\displaystyle \frac{a^{2}\sqrt{15}}{4}$, gdzie

$a$ oznacza długość krawędzi podstawy tego ostrosłupa. Zaznacz na ponizszym rysunku kąt

nachylenia ściany bocznej ostrosłupa do płaszczyzny jego podstawy. Miarę tego kąta oznacz

symbolem $\beta$. Oblicz $\cos\beta$ i korzystając z tablic funkcji trygonometrycznych odczytaj

przyblizoną wartość $\beta$ z dokładnością do $1^{\mathrm{o}}$





{\it Egzamin maturalny z matematyki 17}

{\it Poziom podstawowy}
\begin{center}
\includegraphics[width=137.928mm,height=17.832mm]{./F1_M_PP_M2008_page16_images/image001.eps}
\end{center}
Wypelnia

egzaminator!

Nr zadania

Maks. liczba kt

1

11.2

1

11.3

1

11.4

1

11.5

Uzyskana liczba pkt





{\it 18 Egzamin maturalny z matematyki}

{\it Poziom podstawowy}

Zadanie 12. $(4pkt)$

Rzucamy dwa razy symetryczną sześcienną kostką do gry. Oblicz prawdopodobieństwo

$\mathrm{k}\mathrm{a}\dot{\mathrm{z}}$ dego z następujących zdarzeń:

a) $A-\mathrm{w}\mathrm{k}\mathrm{a}\dot{\mathrm{z}}$ dym rzucie wypadnie nieparzysta liczba oczek.

b) $B-$ suma oczek otrzymanych w obu rzutachjest liczbą większą od 9.

c) $C-$ suma oczek otrzymanych w obu rzutachjest liczbą nieparzystą i większą od 9.
\begin{center}
\includegraphics[width=123.948mm,height=17.832mm]{./F1_M_PP_M2008_page17_images/image001.eps}
\end{center}
Wypelnia

egzaminator!

Nr zadania

Maks. liczba kt

12.1

1

12.2

1

12.3

1

12.4

1

Uzyskana liczba pkt





{\it Egzamin maturalny z matematyki 19}

{\it Poziom podstawowy}

BRUDNOPIS





{\it Egzamin maturalny z matematyki 3}

{\it Poziom podstawowy}
\begin{center}
\includegraphics[width=123.900mm,height=17.832mm]{./F1_M_PP_M2008_page2_images/image001.eps}
\end{center}
Nr zadania

Wypelnia Maks. liczba kt

egzaminator! Uzyskana liczba pkt

1.1

1

1.2

1

1.3

1

1.4

1





{\it 4 Egzamin maturalny z matematyki}

{\it Poziom podstawowy}

Zadanie 2. (4pkt)

Liczba przekątnych wielokąta wypukłego, w którymjest $n$ boków i $n\geq 3$ wyraza się wzorem

$P(n)=\displaystyle \frac{n(n-3)}{2}.$

Wykorzystując ten wzór:

a) oblicz liczbę przekątnych w dwudziestokącie wypukłym.

b) oblicz, ile boków ma wielokąt wypukły, w którym liczba przekątnych jest pięć razy

większa od liczby boków.

c) sprawd $\acute{\mathrm{z}}$, czy jest prawdziwe następujące stwierdzenie:

{\it Kazdy wielokqt wypukly o parzystej liczbie boków ma parzystq liczbę przekqtnych}.

Odpowied $\acute{\mathrm{z}}$ uzasadnij.
\begin{center}
\includegraphics[width=123.948mm,height=17.784mm]{./F1_M_PP_M2008_page3_images/image001.eps}
\end{center}
Nr zadania

Wypelnia Maks. liczba kt

egzamÍnator! Uzyskana lÍczba pkt

2.1

1

2.2

1

2.3

1

2.4

1





{\it Egzamin maturalny z matematyki 5}

{\it Poziom podstawowy}

Zadanie 3. $(4pkt)$

Rozwiąz. równanie $4^{23}x-32^{9}x=16^{4}\cdot(4^{4})^{4}$

Zapisz rozwiązanie tego równania w postaci $2^{k}$, gdzie kjest liczbą całkowitą.
\begin{center}
\includegraphics[width=123.900mm,height=17.784mm]{./F1_M_PP_M2008_page4_images/image001.eps}
\end{center}
Nr zadania

Wypelnia Maks. liczba kt

egzaminator! Uzyskana liczba pkt

3.1

1

3.2

1

3.3

1

3.4

1





{\it 6 Egzamin maturalny z matematyki}

{\it Poziom podstawowy}

Zadanie 4. (3pkt)

Koncetn paliwowy podnosił dwukrotnie w jednym tygodniu cenę benzyny, pierwszy raz

010\%, a drugi raz o 5\%. Po obu tych podwyzkachjeden litr benzyny, wyprodukowanej przez

ten koncern, kosztuje 4,62 zł. Ob1icz cenę jednego 1itra benzyny przed omawianymi

podwyzkami.
\begin{center}
\includegraphics[width=109.932mm,height=17.832mm]{./F1_M_PP_M2008_page5_images/image001.eps}
\end{center}
Wypelnia

egzaminator!

Nr zadania

Maks. liczba kt

4.2

4.3

1

Uzyskana liczba pkt





{\it Egzamin maturalny z matematyki 7}

{\it Poziom podstawowy}

Zadanie 5. $(5pkt)$

Nieskończony ciąg liczbowy $(a_{n})$ jest określony wzorem $a_{n}=2-\displaystyle \frac{1}{n}, n=1$, 2, 3,$\ldots.$

a) Oblicz, ile wyrazów ciągu $(a_{n})$ jest mniejszych od 1,975.

b) Dla pewnej liczby $x$ trzywyrazowy ciąg $(a_{2},a_{7},x)$ jest arytmetyczny. Oblicz $x.$
\begin{center}
\includegraphics[width=137.928mm,height=17.784mm]{./F1_M_PP_M2008_page6_images/image001.eps}
\end{center}
Nr zadania

Wypelnia Maks. liczba $\mathrm{k}\iota$

egzaminator! Uzyskana lÍczba pkt

5.1

1

5.2

1

5.3

1

5.4

5.5

1





{\it 8 Egzamin maturalny z matematyki}

{\it Poziom podstawowy}

Zadanie 6. $(5pkt)$

Prosta o równaniu $5x+4y-10=0$ przecina oś $Ox$ układu współrzędnych w punkcie $A$ oraz

oś $Oy$ w punkcie $B$. Oblicz współrzędne wszystkich punktów $C$ lez$\cdot$ących na osi $Ox$ i takich,

$\dot{\mathrm{z}}\mathrm{e}$ trójkąt $ABC$ ma pole równe 35.





{\it Egzamin maturalny z matematyki 9}

{\it Poziom podstawowy}
\begin{center}
\includegraphics[width=137.928mm,height=17.832mm]{./F1_M_PP_M2008_page8_images/image001.eps}
\end{center}
Wypelnia

egzaminator!

Nr zadania

Maks. liczba kt

1

1

1

1

Uzyskana liczba pkt





$ 1\theta$ {\it Egzamin maturalny z matematyki}

{\it Poziom podstawowy}

Zadanie 7. $(4pkt)$

Dany jest trapez, w którym podstawy mają długość 4 cm i 10 cm oraz ramiona tworzą

z dłuzsząpodstawą kąty o miarach $30^{\mathrm{o}}$ i $45^{\mathrm{o}}$. Oblicz wysokość tego trapezu.



\end{document}