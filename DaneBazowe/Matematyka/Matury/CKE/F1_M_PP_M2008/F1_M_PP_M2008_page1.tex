\documentclass[a4paper,12pt]{article}
\usepackage{latexsym}
\usepackage{amsmath}
\usepackage{amssymb}
\usepackage{graphicx}
\usepackage{wrapfig}
\pagestyle{plain}
\usepackage{fancybox}
\usepackage{bm}

\begin{document}

{\it 2 Egzamin maturalny z matematyki}

{\it Poziom podstawowy}

Zadanie l. $(4pkt)$

Na ponizszym rysunku przedstawiono łamaną ABCD, którajest wykresem ffinkcji $y=f(x).$
\begin{center}
\includegraphics[width=108.108mm,height=107.952mm]{./F1_M_PP_M2008_page1_images/image001.eps}
\end{center}
{\it y}

{\it C  D}

3

1

$-3 -2  -1  0 1_{1} 2_{1}$ 3  $1_{1} 2_{1}$  4  {\it x}

1

$-2$

{\it A  B}  $-4$

Korzystając z tego wykresu:

a) zapisz w postaci przedziału zbiór wartości funkcji $f,$

b) podaj wartość funkcji $f$ dla argumentu $x=1-\sqrt{10},$

c) wyznacz równanie prostej $BC,$

d) oblicz długość odcinka $BC.$
\end{document}
