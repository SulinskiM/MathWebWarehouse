\documentclass[a4paper,12pt]{article}
\usepackage{latexsym}
\usepackage{amsmath}
\usepackage{amssymb}
\usepackage{graphicx}
\usepackage{wrapfig}
\pagestyle{plain}
\usepackage{fancybox}
\usepackage{bm}

\begin{document}

{\it Egzamin maturalny z matematyki 13}

{\it Poziom podstawowy}

Zadanie 9. (5pkt)

Oblicz najmniejszą i

w przedziale $\langle-2, 2\rangle.$

największą wartość

ffinkcji kwadratowej

$f(x)=(2x+1)(x-2)$
\begin{center}
\includegraphics[width=137.928mm,height=17.832mm]{./F1_M_PP_M2008_page12_images/image001.eps}
\end{center}
Wypelnia

egzaminator!

Nr zadania

Maks. liczba kt

1

1

Uzyskana liczba pkt
\end{document}
