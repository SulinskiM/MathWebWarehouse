\documentclass[a4paper,12pt]{article}
\usepackage{latexsym}
\usepackage{amsmath}
\usepackage{amssymb}
\usepackage{graphicx}
\usepackage{wrapfig}
\pagestyle{plain}
\usepackage{fancybox}
\usepackage{bm}

\begin{document}

{\it 14 Egzamin maturalny z matematyki}

{\it Poziom podstawowy}

Zadanie 10. $(3pkt)$

Rysunek przedstawia fragment wykresu funkcji $h$, określonej wzorem $h(x)=\displaystyle \frac{a}{x}$ dla $x\neq 0.$

Wiadomo, $\dot{\mathrm{z}}\mathrm{e}$ do wykresu ffinkcji $h$ nalezy punkt $P=(2,5).$

a) Oblicz wartość współczynnika $a.$

b) Ustal, czy liczba $h(\pi)-h(-\pi)$ jest dodatnia czy ujemna.

c) Rozwiąz nierówność $h(x)>5.$
\begin{center}
\includegraphics[width=140.664mm,height=112.824mm]{./F1_M_PP_M2008_page13_images/image001.eps}
\end{center}
{\it y}

1

1  {\it x}
\end{document}
