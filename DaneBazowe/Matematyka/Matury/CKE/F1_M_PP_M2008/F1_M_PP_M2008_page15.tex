\documentclass[a4paper,12pt]{article}
\usepackage{latexsym}
\usepackage{amsmath}
\usepackage{amssymb}
\usepackage{graphicx}
\usepackage{wrapfig}
\pagestyle{plain}
\usepackage{fancybox}
\usepackage{bm}

\begin{document}

{\it 16 Egzamin maturalny z matematyki}

{\it Poziom podstawowy}

Zadanie ll. $(5pkt)$

Pole powierzchni bocznej ostrosłupa prawidłowego trójkątnego równa się $\displaystyle \frac{a^{2}\sqrt{15}}{4}$, gdzie

$a$ oznacza długość krawędzi podstawy tego ostrosłupa. Zaznacz na ponizszym rysunku kąt

nachylenia ściany bocznej ostrosłupa do płaszczyzny jego podstawy. Miarę tego kąta oznacz

symbolem $\beta$. Oblicz $\cos\beta$ i korzystając z tablic funkcji trygonometrycznych odczytaj

przyblizoną wartość $\beta$ z dokładnością do $1^{\mathrm{o}}$
\end{document}
