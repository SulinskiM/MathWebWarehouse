\documentclass[a4paper,12pt]{article}
\usepackage{latexsym}
\usepackage{amsmath}
\usepackage{amssymb}
\usepackage{graphicx}
\usepackage{wrapfig}
\pagestyle{plain}
\usepackage{fancybox}
\usepackage{bm}

\begin{document}

{\it 18 Egzamin maturalny z matematyki}

{\it Poziom podstawowy}

Zadanie 12. $(4pkt)$

Rzucamy dwa razy symetryczną sześcienną kostką do gry. Oblicz prawdopodobieństwo

$\mathrm{k}\mathrm{a}\dot{\mathrm{z}}$ dego z następujących zdarzeń:

a) $A-\mathrm{w}\mathrm{k}\mathrm{a}\dot{\mathrm{z}}$ dym rzucie wypadnie nieparzysta liczba oczek.

b) $B-$ suma oczek otrzymanych w obu rzutachjest liczbą większą od 9.

c) $C-$ suma oczek otrzymanych w obu rzutachjest liczbą nieparzystą i większą od 9.
\begin{center}
\includegraphics[width=123.948mm,height=17.832mm]{./F1_M_PP_M2008_page17_images/image001.eps}
\end{center}
Wypelnia

egzaminator!

Nr zadania

Maks. liczba kt

12.1

1

12.2

1

12.3

1

12.4

1

Uzyskana liczba pkt
\end{document}
