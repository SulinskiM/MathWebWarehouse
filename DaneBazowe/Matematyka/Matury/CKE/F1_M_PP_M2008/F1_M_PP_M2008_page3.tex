\documentclass[a4paper,12pt]{article}
\usepackage{latexsym}
\usepackage{amsmath}
\usepackage{amssymb}
\usepackage{graphicx}
\usepackage{wrapfig}
\pagestyle{plain}
\usepackage{fancybox}
\usepackage{bm}

\begin{document}

{\it 4 Egzamin maturalny z matematyki}

{\it Poziom podstawowy}

Zadanie 2. (4pkt)

Liczba przekątnych wielokąta wypukłego, w którymjest $n$ boków i $n\geq 3$ wyraza się wzorem

$P(n)=\displaystyle \frac{n(n-3)}{2}.$

Wykorzystując ten wzór:

a) oblicz liczbę przekątnych w dwudziestokącie wypukłym.

b) oblicz, ile boków ma wielokąt wypukły, w którym liczba przekątnych jest pięć razy

większa od liczby boków.

c) sprawd $\acute{\mathrm{z}}$, czy jest prawdziwe następujące stwierdzenie:

{\it Kazdy wielokqt wypukly o parzystej liczbie boków ma parzystq liczbę przekqtnych}.

Odpowied $\acute{\mathrm{z}}$ uzasadnij.
\begin{center}
\includegraphics[width=123.948mm,height=17.784mm]{./F1_M_PP_M2008_page3_images/image001.eps}
\end{center}
Nr zadania

Wypelnia Maks. liczba kt

egzamÍnator! Uzyskana lÍczba pkt

2.1

1

2.2

1

2.3

1

2.4

1
\end{document}
