\documentclass[a4paper,12pt]{article}
\usepackage{latexsym}
\usepackage{amsmath}
\usepackage{amssymb}
\usepackage{graphicx}
\usepackage{wrapfig}
\pagestyle{plain}
\usepackage{fancybox}
\usepackage{bm}

\begin{document}

{\it 6 Egzamin maturalny z matematyki}

{\it Poziom podstawowy}

Zadanie 4. (3pkt)

Koncetn paliwowy podnosił dwukrotnie w jednym tygodniu cenę benzyny, pierwszy raz

010\%, a drugi raz o 5\%. Po obu tych podwyzkachjeden litr benzyny, wyprodukowanej przez

ten koncern, kosztuje 4,62 zł. Ob1icz cenę jednego 1itra benzyny przed omawianymi

podwyzkami.
\begin{center}
\includegraphics[width=109.932mm,height=17.832mm]{./F1_M_PP_M2008_page5_images/image001.eps}
\end{center}
Wypelnia

egzaminator!

Nr zadania

Maks. liczba kt

4.2

4.3

1

Uzyskana liczba pkt
\end{document}
