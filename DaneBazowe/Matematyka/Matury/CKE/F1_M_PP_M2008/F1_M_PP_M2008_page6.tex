\documentclass[a4paper,12pt]{article}
\usepackage{latexsym}
\usepackage{amsmath}
\usepackage{amssymb}
\usepackage{graphicx}
\usepackage{wrapfig}
\pagestyle{plain}
\usepackage{fancybox}
\usepackage{bm}

\begin{document}

{\it Egzamin maturalny z matematyki 7}

{\it Poziom podstawowy}

Zadanie 5. $(5pkt)$

Nieskończony ciąg liczbowy $(a_{n})$ jest określony wzorem $a_{n}=2-\displaystyle \frac{1}{n}, n=1$, 2, 3,$\ldots.$

a) Oblicz, ile wyrazów ciągu $(a_{n})$ jest mniejszych od 1,975.

b) Dla pewnej liczby $x$ trzywyrazowy ciąg $(a_{2},a_{7},x)$ jest arytmetyczny. Oblicz $x.$
\begin{center}
\includegraphics[width=137.928mm,height=17.784mm]{./F1_M_PP_M2008_page6_images/image001.eps}
\end{center}
Nr zadania

Wypelnia Maks. liczba $\mathrm{k}\iota$

egzaminator! Uzyskana lÍczba pkt

5.1

1

5.2

1

5.3

1

5.4

5.5

1
\end{document}
