\documentclass[a4paper,12pt]{article}
\usepackage{latexsym}
\usepackage{amsmath}
\usepackage{amssymb}
\usepackage{graphicx}
\usepackage{wrapfig}
\pagestyle{plain}
\usepackage{fancybox}
\usepackage{bm}

\begin{document}

{\it ARKUSZ ZA WIERA INFORMACJE} $PRA$ {\it WNIE CHRONIONE}

{\it DO MOMENTU ROZPOCZĘCIA EGZAMINU}.$\displaystyle \int$
\begin{center}
\includegraphics[width=192.024mm,height=288.084mm]{./F1_M_PP_M2009_page0_images/image001.eps}
\end{center}
Miejsce

na na ejkę

MMA-PI IP-092

EGZAMIN MATURALNY

MAJ

Z MATEMATYKI

POZIOM PODSTAWOWY

Czas pracy 120 minut

Instrukcja dla zdającego

1.

2.

3.

4.

5.

6.

7.

8.

9.

Sprawd $\acute{\mathrm{z}}$, czy arkusz egzaminacyjny zawiera 16 stron (zadania

$1-11)$. Ewentualny brak zgłoś przewodniczącemu zespołu

nadzorującego egzamin.

Rozwiązania zadań i odpowiedzi zamieść w miejscu na to

przeznaczonym.

W rozwiązaniach zadań przedstaw tok rozumowania

prowadzący do ostatecznego wyniku.

Pisz czytelnie. Uzywaj $\mathrm{d}$ gopisu pióra tylko z czatnym

tusze atramentem.

Nie uzywaj korektora, a błędne zapisy prze eśl.

Pamiętaj, $\dot{\mathrm{z}}\mathrm{e}$ zapisy w brudnopisie nie podlegają ocenie.

Obok $\mathrm{k}\mathrm{a}\dot{\mathrm{z}}$ dego zadania podanajest maksymalna liczba punktów,

którą $\mathrm{m}\mathrm{o}\dot{\mathrm{z}}$ esz uzyskać zajego poprawne rozwiązanie.

$\mathrm{M}\mathrm{o}\dot{\mathrm{z}}$ esz korzystać z zesta wzorów matematycznych, cyrkla

i linijki oraz kalkulatora.

Na karcie odpowiedzi wpisz swoją datę urodzenia i PESEL.

Nie wpisuj $\dot{\mathrm{z}}$ adnych znaków w części przeznaczonej

dla egzaminatora.

Za rozwiązanie

wszystkich zadań

mozna otrzymać

łącznie

50 punktów

{\it Zyczymy} $pow\theta dzenia'$

Wypelnia zdający

rzed roz oczęciem racy

PESEL ZDAJACEGO

KOD

ZDAJACEGO




{\it 2}

{\it Egzamin maturalny z matematyki}

{\it Poziom podstawowy}

Zadanie l. $(5pkt)$

Funkcja $f$ określona jest wzorem $f(x)=$

a) Uzupełnij tabelę:

dla $x<2$

dla $2\leq x\leq 4$
\begin{center}
\begin{tabular}{|l|l|l|l|}
\hline
\multicolumn{1}{|l|}{$x$}&	\multicolumn{1}{|l|}{ $-3$}&	\multicolumn{1}{|l|}{ $3$}&	\multicolumn{1}{|l|}{}	\\
\hline
\multicolumn{1}{|l|}{ $f(x)$}&	\multicolumn{1}{|l|}{}&	\multicolumn{1}{|l|}{}&	\multicolumn{1}{|l|}{ $0$}	\\
\hline
\end{tabular}

\end{center}
b) Narysuj wykres funkcji $f.$

c) Podaj wszystkie liczby całkowite $x$, spełniające nierówność $f(x)\geq-6.$
\begin{center}
\includegraphics[width=137.868mm,height=17.832mm]{./F1_M_PP_M2009_page1_images/image001.eps}
\end{center}
Nr zadania

Wypelnia Maks. liczba kt

egzaminator! Uzyskana liczba pkt

1.1

1

1

1.3

1

1.4

1

1.5





{\it Egzamin maturalny z matematyki}

{\it Poziom podstawowy}

{\it 11}
\begin{center}
\includegraphics[width=192.276mm,height=290.784mm]{./F1_M_PP_M2009_page10_images/image001.eps}

\includegraphics[width=123.900mm,height=17.784mm]{./F1_M_PP_M2009_page10_images/image002.eps}
\end{center}
Nr zadania

Wypelnia Maks. liczba kt

egzamÍnator! Uzyskana liczba pkt

8.1

1

8.2

1

8.3

1

8.4

1





{\it 12}

{\it Egzamin maturalny z matematyki}

{\it Poziom podstawowy}

Zadanie 9. $(4pkt)$

Punkty $B=(0,10) \mathrm{i} O=(0,0)$ są wierzchołkami trójkąta prostokątnego $OAB$, w którym

$|\neq OAB|=90^{\mathrm{o}}$ Przyprostokątna $OA$ zawiera się w prostej o równaniu

współrzędne punktu $A$ i długość przyprostokątnej $OA.$

$y=\displaystyle \frac{1}{2}x$. Oblicz
\begin{center}
\includegraphics[width=192.228mm,height=254.460mm]{./F1_M_PP_M2009_page11_images/image001.eps}

\includegraphics[width=123.948mm,height=17.784mm]{./F1_M_PP_M2009_page11_images/image002.eps}
\end{center}
Nr zadania

Wypelnia Maks. liczba kt

egzamÍnator! Uzyskana lÍczba pkt

1

1

1

1





{\it Egzamin maturalny z matematyki}

{\it Poziom podstawowy}

{\it 13}

Zadanie 10. $(5pkt)$

Tabela przedstawia wyniki części teoretycznej egzaminu na prawo jazdy. Zdający uzyskał

wynik pozytywny, $\mathrm{j}\mathrm{e}\dot{\mathrm{z}}$ eli popełnił co najwyzej dwa błędy.
\begin{center}
\begin{tabular}{|l|l|l|l|l|l|l|l|l|l|}
\hline
\multicolumn{1}{|l|}{liczba błędów}&	\multicolumn{1}{|l|}{$0$}&	\multicolumn{1}{|l|}{ $1$}&	\multicolumn{1}{|l|}{ $2$}&	\multicolumn{1}{|l|}{ $3$}&	\multicolumn{1}{|l|}{ $4$}&	\multicolumn{1}{|l|}{ $5$}&	\multicolumn{1}{|l|}{ $6$}&	\multicolumn{1}{|l|}{ $7$}&	\multicolumn{1}{|l|}{ $8$}	\\
\hline
\multicolumn{1}{|l|}{liczba zdających}&	\multicolumn{1}{|l|}{$8$}&	\multicolumn{1}{|l|}{ $5$}&	\multicolumn{1}{|l|}{ $8$}&	\multicolumn{1}{|l|}{ $5$}&	\multicolumn{1}{|l|}{ $2$}&	\multicolumn{1}{|l|}{ $1$}&	\multicolumn{1}{|l|}{ $0$}&	\multicolumn{1}{|l|}{ $0$}&	\multicolumn{1}{|l|}{ $1$}	\\
\hline
\end{tabular}

\end{center}
a) Oblicz średnią arytmetyczną liczby błędów popełnionych przez zdających ten egzamin.

Wynik podaj w zaokrągleniu do całości.

b) Oblicz prawdopodobieństwo, $\dot{\mathrm{z}}\mathrm{e}$ wśród dwóch losowo wybranych zdających tylko jeden

uzyskał wynik pozytywny. Wynik zapisz w postaci ułamka zwykłego nieskracalnego.
\begin{center}
\includegraphics[width=192.276mm,height=218.136mm]{./F1_M_PP_M2009_page12_images/image001.eps}

\includegraphics[width=137.928mm,height=17.832mm]{./F1_M_PP_M2009_page12_images/image002.eps}
\end{center}
Nr zadania

Wypelnia Maks. liczba kt

egzaminator! Uzyskana liczba pkt

10.1

1

10.2

1

10.3

1

10.4

1

10.5





{\it 14}

{\it Egzamin maturalny z matematyki}

{\it Poziom podstawowy}

Zadanie ll. $(5pkt)$

Powierzchnia boczna walca po rozwinięciu na płaszczyznę jest prostokątem. Przekątna tego

prostokąta ma długość 12 i tworzy z bokiem, którego długość jest równa wysokości wa1ca,

kąt o mierze $30^{\circ}$

a) Oblicz pole powierzchni bocznej tego walca.

b) Sprawdzí, czy objętość tego walcajest większa od $18\sqrt{3}$. Odpowiedzí uzasadnij.
\begin{center}
\includegraphics[width=192.228mm,height=272.640mm]{./F1_M_PP_M2009_page13_images/image001.eps}
\end{center}




{\it Egzamin maturalny z matematyki}

{\it Poziom podstawowy}

{\it 15}
\begin{center}
\includegraphics[width=192.276mm,height=290.784mm]{./F1_M_PP_M2009_page14_images/image001.eps}

\includegraphics[width=137.928mm,height=17.784mm]{./F1_M_PP_M2009_page14_images/image002.eps}
\end{center}
Nr zadania

Wypelnia Maks. liczba kt

egzaminator! Uzyskana lÍczba pkt

11.1

1

11.2

1

11.3

11.4

11.5

1





{\it 16}

{\it Egzamin maturalny z matematyki}

{\it Poziom podstawowy}

BRUDNOPIS





{\it Egzamin maturalny z matematyki}

{\it Poziom podstawowy}

{\it 3}

Zadanie 2. $(3pkt)$

Dwaj rzemieślnicy przyjęli zlecenie wykonania wspólnie 980 deta1i. Zap1anowa1i, $\dot{\mathrm{z}}\mathrm{e}$

$\mathrm{k}\mathrm{a}\dot{\mathrm{z}}$ dego dnia pierwszy z nich wykona $m$, a drugi $n$ detali. Obliczyli, $\dot{\mathrm{z}}\mathrm{e}$ razem wykonają

zlecenie w ciągu 7 dni. Po pierwszym dniu pracy pierwszy z rzemieś1ników rozchorował się

i wtedy drugi, aby wykonać całe zlecenie, musiał pracować o 8 dni dłuzej $\mathrm{n}\mathrm{i}\dot{\mathrm{z}}$ planował, (nie

zmieniając liczby wykonywanych codziennie detali). Oblicz $m \mathrm{i} n.$
\begin{center}
\includegraphics[width=192.276mm,height=248.364mm]{./F1_M_PP_M2009_page2_images/image001.eps}

\includegraphics[width=109.980mm,height=17.784mm]{./F1_M_PP_M2009_page2_images/image002.eps}
\end{center}
Nr zadania

Wypelnia Maks. liczba kt

egzaminator! Uzyskana lÍczba pkt

2.1

2.2

1

2.3

1





{\it 4}

{\it Egzamin maturalny z matematyki}

{\it Poziom podstawowy}

Zadanie 3. $(5pkt)$

Wykres funkcji $f$ danej wzorem $f(x)=-2x^{2}$ przesunięto wzdłuz osi $Ox 0 3$ jednostki

w prawo oraz wzdłuz osi $Oy\mathrm{o}$ 8jednostek w górę, otrzymując wykres funkcji $g.$

a) Rozwiąz nierówność $f(x)+5<3x.$

b) Podaj zbiór wartości funkcji $g.$

c) Funkcja $g$ określonajest wzorem $g(x)=-2x^{2}+bx+c$. Oblicz $b\mathrm{i}c.$
\begin{center}
\includegraphics[width=192.228mm,height=260.508mm]{./F1_M_PP_M2009_page3_images/image001.eps}
\end{center}




{\it Egzamin maturalny z matematyki}

{\it Poziom podstawowy}

{\it 5}
\begin{center}
\includegraphics[width=192.276mm,height=290.784mm]{./F1_M_PP_M2009_page4_images/image001.eps}

\includegraphics[width=137.928mm,height=17.784mm]{./F1_M_PP_M2009_page4_images/image002.eps}
\end{center}
Nr zadanÍa

Wypelnia Maks. liczba kt

egzaminator! Uzyskana lÍczba pkt

1

3.2

1

3.3

3.4

3.5

1





{\it 6}

{\it Egzamin maturalny z matematyki}

{\it Poziom podstawowy}

Zadanie 4. $(3pkt)$

Wykaz, $\dot{\mathrm{z}}\mathrm{e}$ liczba $3^{54}$jest rozwiązaniem równania $243^{11}-81^{14}+7x=9^{27}$
\begin{center}
\includegraphics[width=109.932mm,height=17.832mm]{./F1_M_PP_M2009_page5_images/image001.eps}
\end{center}
Wypelnia

egzaminator!

Nr zadania

Maks. liczba kt

4.2

4.3

1

Uzyskana liczba pkt





{\it Egzamin maturalny z matematyki}

{\it Poziom podstawowy}

7

Zadanie 5. $(5pkt)$

Wielomian $W$ dany jest wzorem $W(x)=x^{3}+ax^{2}-4x+b.$

a) Wyznacz $a, b$ oraz $c$ tak, aby wielomian $W$ był równy wielomianowi $P$, gdy

$P(x)=x^{3}+(2a+3)x^{2}+(a+b+c)x-1.$

b) Dla $a=3 \mathrm{i} b=0$ zapisz wielomian $W$ w postaci iloczynu trzech wielomianów stopnia

pierwszego.
\begin{center}
\includegraphics[width=137.928mm,height=17.832mm]{./F1_M_PP_M2009_page6_images/image001.eps}
\end{center}
Nr zadania

Wypelnia Maks. liczba kt

egzaminator! Uzyskana liczba pkt

5.1

1

5.2

1

5.3

1

5.4

1

5.5





{\it 8}

{\it Egzamin maturalny z matematyki}

{\it Poziom podstawowy}

Zadanie 6. $(5pkt)$

Miarajednego z kątów ostrych w trójkącie prostokątnymjest równa $\alpha.$

a) Uzasadnij, ze spełnionajest nierówność $\sin\alpha-\mathrm{t}\mathrm{g}\alpha<0.$

b) Dla $\displaystyle \sin\alpha=\frac{\mathrm{z}\sqrt{2}}{3}$ oblicz wartość wyrazenia $\cos^{3}\alpha+\cos\alpha\cdot\sin^{2}\alpha.$
\begin{center}
\includegraphics[width=192.228mm,height=254.460mm]{./F1_M_PP_M2009_page7_images/image001.eps}

\includegraphics[width=137.868mm,height=17.832mm]{./F1_M_PP_M2009_page7_images/image002.eps}
\end{center}
Nr zadania

Wypelnia Maks. liczba kt

egzaminator! Uzyskana liczba pkt

1

1





{\it Egzamin maturalny z matematyki}

{\it Poziom podstawowy}

{\it 9}

Zadanie 7. $(6pkt)$

Dany jest ciąg arytmetyczny $(a_{n})$ dla $n\geq 1$, w którym $a_{7}=1, a_{11}=9.$

a) Oblicz pierwszy wyraz $a_{1}$ i róznicę $r$ ciągu $(a_{n}).$

b) Sprawdzí, czy ciąg $(a_{7},a_{8},a_{11})$ jest geometryczny.

c) Wyznacz takie $n$, aby suma $n$ początkowych wyrazów ciągu

najmniejszą.

$(a_{n})$

miała wartość
\begin{center}
\includegraphics[width=192.276mm,height=242.316mm]{./F1_M_PP_M2009_page8_images/image001.eps}

\includegraphics[width=151.836mm,height=17.784mm]{./F1_M_PP_M2009_page8_images/image002.eps}
\end{center}
Wypelnia

egzaminator!

Nr zadania

Maks. liczba kt

7.1

1

7.2

1

7.3

1

7.4

1

7.5

1

1

Uzyskana liczba pkt





$ 1\theta$

{\it Egzamin maturalny z matematyki}

{\it Poziom podstawowy}

Zadanie 8. (4pkt)

W trapezie ABCD długość podstawy CD jest równa 18, a długości ramion trapezu AD iBC

są odpowiednio równe 25 i 15. Kąty ADBi DCB, zaznaczone na rysunku, mają równe miary.

Oblicz obwód tego trapezu.
\begin{center}
\includegraphics[width=136.704mm,height=55.680mm]{./F1_M_PP_M2009_page9_images/image001.eps}
\end{center}
{\it D  C}

{\it A  B}



\end{document}