\documentclass[a4paper,12pt]{article}
\usepackage{latexsym}
\usepackage{amsmath}
\usepackage{amssymb}
\usepackage{graphicx}
\usepackage{wrapfig}
\pagestyle{plain}
\usepackage{fancybox}
\usepackage{bm}

\begin{document}

{\it 2}

{\it Egzamin maturalny z matematyki}

{\it Poziom podstawowy}

Zadanie l. $(5pkt)$

Funkcja $f$ określona jest wzorem $f(x)=$

a) Uzupełnij tabelę:

dla $x<2$

dla $2\leq x\leq 4$
\begin{center}
\begin{tabular}{|l|l|l|l|}
\hline
\multicolumn{1}{|l|}{$x$}&	\multicolumn{1}{|l|}{ $-3$}&	\multicolumn{1}{|l|}{ $3$}&	\multicolumn{1}{|l|}{}	\\
\hline
\multicolumn{1}{|l|}{ $f(x)$}&	\multicolumn{1}{|l|}{}&	\multicolumn{1}{|l|}{}&	\multicolumn{1}{|l|}{ $0$}	\\
\hline
\end{tabular}

\end{center}
b) Narysuj wykres funkcji $f.$

c) Podaj wszystkie liczby całkowite $x$, spełniające nierówność $f(x)\geq-6.$
\begin{center}
\includegraphics[width=137.868mm,height=17.832mm]{./F1_M_PP_M2009_page1_images/image001.eps}
\end{center}
Nr zadania

Wypelnia Maks. liczba kt

egzaminator! Uzyskana liczba pkt

1.1

1

1

1.3

1

1.4

1

1.5
\end{document}
