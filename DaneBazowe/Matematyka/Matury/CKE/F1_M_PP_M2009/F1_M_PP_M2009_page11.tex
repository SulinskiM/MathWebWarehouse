\documentclass[a4paper,12pt]{article}
\usepackage{latexsym}
\usepackage{amsmath}
\usepackage{amssymb}
\usepackage{graphicx}
\usepackage{wrapfig}
\pagestyle{plain}
\usepackage{fancybox}
\usepackage{bm}

\begin{document}

{\it 12}

{\it Egzamin maturalny z matematyki}

{\it Poziom podstawowy}

Zadanie 9. $(4pkt)$

Punkty $B=(0,10) \mathrm{i} O=(0,0)$ są wierzchołkami trójkąta prostokątnego $OAB$, w którym

$|\neq OAB|=90^{\mathrm{o}}$ Przyprostokątna $OA$ zawiera się w prostej o równaniu

współrzędne punktu $A$ i długość przyprostokątnej $OA.$

$y=\displaystyle \frac{1}{2}x$. Oblicz
\begin{center}
\includegraphics[width=192.228mm,height=254.460mm]{./F1_M_PP_M2009_page11_images/image001.eps}

\includegraphics[width=123.948mm,height=17.784mm]{./F1_M_PP_M2009_page11_images/image002.eps}
\end{center}
Nr zadania

Wypelnia Maks. liczba kt

egzamÍnator! Uzyskana lÍczba pkt

1

1

1

1
\end{document}
