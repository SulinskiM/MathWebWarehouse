\documentclass[a4paper,12pt]{article}
\usepackage{latexsym}
\usepackage{amsmath}
\usepackage{amssymb}
\usepackage{graphicx}
\usepackage{wrapfig}
\pagestyle{plain}
\usepackage{fancybox}
\usepackage{bm}

\begin{document}

{\it Egzamin maturalny z matematyki}

{\it Poziom podstawowy}

{\it 13}

Zadanie 10. $(5pkt)$

Tabela przedstawia wyniki części teoretycznej egzaminu na prawo jazdy. Zdający uzyskał

wynik pozytywny, $\mathrm{j}\mathrm{e}\dot{\mathrm{z}}$ eli popełnił co najwyzej dwa błędy.
\begin{center}
\begin{tabular}{|l|l|l|l|l|l|l|l|l|l|}
\hline
\multicolumn{1}{|l|}{liczba błędów}&	\multicolumn{1}{|l|}{$0$}&	\multicolumn{1}{|l|}{ $1$}&	\multicolumn{1}{|l|}{ $2$}&	\multicolumn{1}{|l|}{ $3$}&	\multicolumn{1}{|l|}{ $4$}&	\multicolumn{1}{|l|}{ $5$}&	\multicolumn{1}{|l|}{ $6$}&	\multicolumn{1}{|l|}{ $7$}&	\multicolumn{1}{|l|}{ $8$}	\\
\hline
\multicolumn{1}{|l|}{liczba zdających}&	\multicolumn{1}{|l|}{$8$}&	\multicolumn{1}{|l|}{ $5$}&	\multicolumn{1}{|l|}{ $8$}&	\multicolumn{1}{|l|}{ $5$}&	\multicolumn{1}{|l|}{ $2$}&	\multicolumn{1}{|l|}{ $1$}&	\multicolumn{1}{|l|}{ $0$}&	\multicolumn{1}{|l|}{ $0$}&	\multicolumn{1}{|l|}{ $1$}	\\
\hline
\end{tabular}

\end{center}
a) Oblicz średnią arytmetyczną liczby błędów popełnionych przez zdających ten egzamin.

Wynik podaj w zaokrągleniu do całości.

b) Oblicz prawdopodobieństwo, $\dot{\mathrm{z}}\mathrm{e}$ wśród dwóch losowo wybranych zdających tylko jeden

uzyskał wynik pozytywny. Wynik zapisz w postaci ułamka zwykłego nieskracalnego.
\begin{center}
\includegraphics[width=192.276mm,height=218.136mm]{./F1_M_PP_M2009_page12_images/image001.eps}

\includegraphics[width=137.928mm,height=17.832mm]{./F1_M_PP_M2009_page12_images/image002.eps}
\end{center}
Nr zadania

Wypelnia Maks. liczba kt

egzaminator! Uzyskana liczba pkt

10.1

1

10.2

1

10.3

1

10.4

1

10.5
\end{document}
