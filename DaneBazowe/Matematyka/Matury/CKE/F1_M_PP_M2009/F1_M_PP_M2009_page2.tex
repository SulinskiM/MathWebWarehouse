\documentclass[a4paper,12pt]{article}
\usepackage{latexsym}
\usepackage{amsmath}
\usepackage{amssymb}
\usepackage{graphicx}
\usepackage{wrapfig}
\pagestyle{plain}
\usepackage{fancybox}
\usepackage{bm}

\begin{document}

{\it Egzamin maturalny z matematyki}

{\it Poziom podstawowy}

{\it 3}

Zadanie 2. $(3pkt)$

Dwaj rzemieślnicy przyjęli zlecenie wykonania wspólnie 980 deta1i. Zap1anowa1i, $\dot{\mathrm{z}}\mathrm{e}$

$\mathrm{k}\mathrm{a}\dot{\mathrm{z}}$ dego dnia pierwszy z nich wykona $m$, a drugi $n$ detali. Obliczyli, $\dot{\mathrm{z}}\mathrm{e}$ razem wykonają

zlecenie w ciągu 7 dni. Po pierwszym dniu pracy pierwszy z rzemieś1ników rozchorował się

i wtedy drugi, aby wykonać całe zlecenie, musiał pracować o 8 dni dłuzej $\mathrm{n}\mathrm{i}\dot{\mathrm{z}}$ planował, (nie

zmieniając liczby wykonywanych codziennie detali). Oblicz $m \mathrm{i} n.$
\begin{center}
\includegraphics[width=192.276mm,height=248.364mm]{./F1_M_PP_M2009_page2_images/image001.eps}

\includegraphics[width=109.980mm,height=17.784mm]{./F1_M_PP_M2009_page2_images/image002.eps}
\end{center}
Nr zadania

Wypelnia Maks. liczba kt

egzaminator! Uzyskana lÍczba pkt

2.1

2.2

1

2.3

1
\end{document}
