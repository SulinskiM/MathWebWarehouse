\documentclass[a4paper,12pt]{article}
\usepackage{latexsym}
\usepackage{amsmath}
\usepackage{amssymb}
\usepackage{graphicx}
\usepackage{wrapfig}
\pagestyle{plain}
\usepackage{fancybox}
\usepackage{bm}

\begin{document}

{\it 4}

{\it Egzamin maturalny z matematyki}

{\it Poziom podstawowy}

Zadanie 3. $(5pkt)$

Wykres funkcji $f$ danej wzorem $f(x)=-2x^{2}$ przesunięto wzdłuz osi $Ox 0 3$ jednostki

w prawo oraz wzdłuz osi $Oy\mathrm{o}$ 8jednostek w górę, otrzymując wykres funkcji $g.$

a) Rozwiąz nierówność $f(x)+5<3x.$

b) Podaj zbiór wartości funkcji $g.$

c) Funkcja $g$ określonajest wzorem $g(x)=-2x^{2}+bx+c$. Oblicz $b\mathrm{i}c.$
\begin{center}
\includegraphics[width=192.228mm,height=260.508mm]{./F1_M_PP_M2009_page3_images/image001.eps}
\end{center}\end{document}
