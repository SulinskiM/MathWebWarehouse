\documentclass[a4paper,12pt]{article}
\usepackage{latexsym}
\usepackage{amsmath}
\usepackage{amssymb}
\usepackage{graphicx}
\usepackage{wrapfig}
\pagestyle{plain}
\usepackage{fancybox}
\usepackage{bm}

\begin{document}

{\it Egzamin maturalny z matematyki}

{\it Poziom podstawowy}

7

Zadanie 5. $(5pkt)$

Wielomian $W$ dany jest wzorem $W(x)=x^{3}+ax^{2}-4x+b.$

a) Wyznacz $a, b$ oraz $c$ tak, aby wielomian $W$ był równy wielomianowi $P$, gdy

$P(x)=x^{3}+(2a+3)x^{2}+(a+b+c)x-1.$

b) Dla $a=3 \mathrm{i} b=0$ zapisz wielomian $W$ w postaci iloczynu trzech wielomianów stopnia

pierwszego.
\begin{center}
\includegraphics[width=137.928mm,height=17.832mm]{./F1_M_PP_M2009_page6_images/image001.eps}
\end{center}
Nr zadania

Wypelnia Maks. liczba kt

egzaminator! Uzyskana liczba pkt

5.1

1

5.2

1

5.3

1

5.4

1

5.5
\end{document}
