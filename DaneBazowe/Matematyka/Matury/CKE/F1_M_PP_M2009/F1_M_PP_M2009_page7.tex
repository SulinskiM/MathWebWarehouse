\documentclass[a4paper,12pt]{article}
\usepackage{latexsym}
\usepackage{amsmath}
\usepackage{amssymb}
\usepackage{graphicx}
\usepackage{wrapfig}
\pagestyle{plain}
\usepackage{fancybox}
\usepackage{bm}

\begin{document}

{\it 8}

{\it Egzamin maturalny z matematyki}

{\it Poziom podstawowy}

Zadanie 6. $(5pkt)$

Miarajednego z kątów ostrych w trójkącie prostokątnymjest równa $\alpha.$

a) Uzasadnij, ze spełnionajest nierówność $\sin\alpha-\mathrm{t}\mathrm{g}\alpha<0.$

b) Dla $\displaystyle \sin\alpha=\frac{\mathrm{z}\sqrt{2}}{3}$ oblicz wartość wyrazenia $\cos^{3}\alpha+\cos\alpha\cdot\sin^{2}\alpha.$
\begin{center}
\includegraphics[width=192.228mm,height=254.460mm]{./F1_M_PP_M2009_page7_images/image001.eps}

\includegraphics[width=137.868mm,height=17.832mm]{./F1_M_PP_M2009_page7_images/image002.eps}
\end{center}
Nr zadania

Wypelnia Maks. liczba kt

egzaminator! Uzyskana liczba pkt

1

1
\end{document}
