\documentclass[a4paper,12pt]{article}
\usepackage{latexsym}
\usepackage{amsmath}
\usepackage{amssymb}
\usepackage{graphicx}
\usepackage{wrapfig}
\pagestyle{plain}
\usepackage{fancybox}
\usepackage{bm}

\begin{document}

$ 1\theta$

{\it Egzamin maturalny z matematyki}

{\it Poziom podstawowy}

Zadanie 8. (4pkt)

W trapezie ABCD długość podstawy CD jest równa 18, a długości ramion trapezu AD iBC

są odpowiednio równe 25 i 15. Kąty ADBi DCB, zaznaczone na rysunku, mają równe miary.

Oblicz obwód tego trapezu.
\begin{center}
\includegraphics[width=136.704mm,height=55.680mm]{./F1_M_PP_M2009_page9_images/image001.eps}
\end{center}
{\it D  C}

{\it A  B}
\end{document}
