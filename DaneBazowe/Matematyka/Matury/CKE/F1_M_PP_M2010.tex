\documentclass[a4paper,12pt]{article}
\usepackage{latexsym}
\usepackage{amsmath}
\usepackage{amssymb}
\usepackage{graphicx}
\usepackage{wrapfig}
\pagestyle{plain}
\usepackage{fancybox}
\usepackage{bm}

\begin{document}

Centralna Komisja Egzaminacyjna

Arkusz zawiera informacje prawnie chronione do momentu rozpoczęcia egzaminu.

WPISUJE ZDAJACY

KOD PESEL

{\it Miejsce}

{\it na naklejkę}

{\it z kodem}
\begin{center}
\includegraphics[width=21.432mm,height=9.852mm]{./F1_M_PP_M2010_page0_images/image001.eps}

\includegraphics[width=82.092mm,height=9.852mm]{./F1_M_PP_M2010_page0_images/image002.eps}

\includegraphics[width=204.060mm,height=216.048mm]{./F1_M_PP_M2010_page0_images/image003.eps}
\end{center}
EGZAMIN MATU

Z MATEMATY

LNY

POZIOM PODSTAWOWY  MAJ 2010

l. Sprawdzí, czy arkusz egzaminacyjny zawiera 20 stron

(zadania $1-34$). Ewentualny brak zgłoś przewodniczącemu

zespo nadzo jącego egzamin.

2. Rozwiązania zadań i odpowiedzi wpisuj w miejscu na to

przeznaczonym.

3. Odpowiedzi do zadań za iętych (l-25) przenieś

na ka ę odpowiedzi, zaznaczając je w części ka $\mathrm{y}$

przeznaczonej dla zdającego. Zamaluj $\blacksquare$ pola do tego

przeznaczone. Błędne zaznaczenie otocz kółkiem \fcircle$\bullet$

i zaznacz właściwe.

4. Pamiętaj, $\dot{\mathrm{z}}\mathrm{e}$ pominięcie argumentacji lub istotnych

obliczeń w rozwiązaniu zadania otwa ego (26-34) $\mathrm{m}\mathrm{o}\dot{\mathrm{z}}\mathrm{e}$

spowodować, $\dot{\mathrm{z}}\mathrm{e}$ za to rozwiązanie nie będziesz mógł

dostać pełnej liczby punktów.

5. Pisz czytelnie i $\mathrm{u}\dot{\mathrm{z}}$ aj tvlko $\mathrm{d}$ gopisu lub -Dióra

z czamym tuszem lub atramentem.

6. Nie uzywaj korektora, a błędne zapisy wyra $\acute{\mathrm{z}}\mathrm{n}\mathrm{i}\mathrm{e}$ prze eśl.

7. Pamiętaj, $\dot{\mathrm{z}}\mathrm{e}$ zapisy w brudnopisie nie będą oceniane.

8. $\mathrm{M}\mathrm{o}\dot{\mathrm{z}}$ esz korzystać z zesta wzorów matematycznych,

cyrkla i linijki oraz kalkulatora.

9. Na karcie odpowiedzi wpisz swój numer PESEL i przyklej

naklejkę z kodem.

10. Nie wpisuj $\dot{\mathrm{z}}$ adnych znaków w części przeznaczonej dla

egzaminatora.

Czas pracy:

170 minut

Liczba punktów

do uzyskania: 50

$\Vert\Vert\Vert\Vert\Vert\Vert\Vert\Vert\Vert\Vert\Vert\Vert\Vert\Vert\Vert\Vert\Vert\Vert\Vert\Vert\Vert\Vert\Vert\Vert|  \mathrm{M}\mathrm{M}\mathrm{A}-\mathrm{P}1_{-}1\mathrm{P}-102$




{\it 2}

{\it Egzamin maturalny z matematyki}

{\it Poziom podstawowy}

ZADANIA ZAMKNIĘTE

{\it Wzadaniach} $\theta d1.$ {\it do 25. wybierz i zaznacz na karcie odpowiedzipoprawnq odpowied} $\acute{z}.$

Zadanie l. $(1pkt)$

Wskaz rysunek, na którymjest przedstawiony zbiór rozwiązań nierówności $|x+7|>5.$
\begin{center}
\includegraphics[width=173.280mm,height=13.212mm]{./F1_M_PP_M2010_page1_images/image001.eps}
\end{center}
$-12$  2  {\it x}

A.
\begin{center}
\includegraphics[width=175.008mm,height=13.812mm]{./F1_M_PP_M2010_page1_images/image002.eps}
\end{center}
2  12  {\it x}

B.
\begin{center}
\includegraphics[width=173.280mm,height=13.260mm]{./F1_M_PP_M2010_page1_images/image003.eps}
\end{center}
$-12  -2$  {\it x}

C.
\begin{center}
\includegraphics[width=171.756mm,height=13.104mm]{./F1_M_PP_M2010_page1_images/image004.eps}
\end{center}
$-2$  12  {\it x}

D.

Zadanie 2. (1pkt)

Spodnie po obnizce ceny o 30\% kosztują 126 zł. I1e kosztowały spodnie przed obnizką?

A. 163,80 zł

B. 180 zł

C. 294 zł

D. 420 zł

Zadanie 3. $(1pkt)$

Liczba $(\displaystyle \frac{2^{-2}\cdot 3^{-1}}{2^{-1}3^{-2}})^{0}$ jest równa

A. I B. 4

C. 9

D. 36

Zadanie 4. (1pkt)

Liczba $\log_{4}8+\log_{4}2$ jest równa

A. l

B. 2

C. $\log_{4}6$

D. log410

Zadanie 5. $(1pkt)$

Dane są wielomiany $W(x)=-2x^{3}+5x^{2}-3$ oraz $P(x)=2x^{3}+12x$. Wielomian $W(x)+P(x)$

jest równy

A. $5x^{2}+12x-3$

B. $4x^{3}+5x^{2}+12x-3$

C. $4x^{6}+5x^{2}+12x-3$

D. $4x^{3}+12x^{2}-3$





{\it Egzamin maturalny z matematyki}

{\it Poziom podstawowy}

{\it 11}

Zadanie 28. (2pkt)

Trójkąty prostokątne równoramienne $ABC\mathrm{i}CDE$ są połozone tak, jak na ponizszym rysunku

(w obu trójkątach kąt przy wierzchołku $C$ jest prosty). Wykaz, $\dot{\mathrm{z}}\mathrm{e}|AD|=|BE|.$

{\it C}
\begin{center}
\includegraphics[width=76.404mm,height=36.072mm]{./F1_M_PP_M2010_page10_images/image001.eps}
\end{center}
{\it E}

{\it D}

{\it A  B}
\begin{center}
\includegraphics[width=109.980mm,height=17.832mm]{./F1_M_PP_M2010_page10_images/image002.eps}
\end{center}
Nr zadani,`

Wypelnia Maks. liczba kt

egzaminator

Uzyskana liczba pkt

2

27.

2

28.

2





{\it 12}

{\it Egzamin maturalny z matematyki}

{\it Poziom podstawowy}

Zadanie 29. $(2pkt)$

Kąt $\alpha$ jest ostry i $\displaystyle \mathrm{t}\mathrm{g}\alpha=\frac{5}{12}$. Oblicz $\cos\alpha.$

Odpowied $\acute{\mathrm{z}}$:

Zadanie 30. $(2pkt)$

Wyka $\dot{\mathrm{z}}$, ze jeśli $a>0$, to $\displaystyle \frac{a^{2}+1}{a+1}\geq\frac{a+1}{2}.$





{\it Egzamin maturalny z matematyki}

{\it Poziom podstawowy}

{\it 13}

Zadanie 31. (2pkt)

W trapezie prostokątnym krótsza przekątna dzieli go na trójkąt prostokątny

równoboczny. Dłuzsza podstawa trapezujest równa 6. Ob1icz obwód tego trapezu.

i trójkąt

Odpowiedzí :
\begin{center}
\includegraphics[width=109.980mm,height=17.784mm]{./F1_M_PP_M2010_page12_images/image001.eps}
\end{center}
Nr zadania

Wypelnia Maks. liczba kt

egzaminator

Uzyskana lÍczba pkt

2

30.

2

31.

2





{\it 14}

{\it Egzamin maturalny z matematyki}

{\it Poziom podstawowy}

Zadanie 32. $(4pkt)$

Podstawą ostrosłupa ABCD jest trójkąt $ABC$. Krawędzí AD jest wysokością ostrosłupa (zobacz

rysunek). Oblicz objętość ostrostupa ABCD, jeśli wiadomo, $\dot{\mathrm{z}}\mathrm{e} |AD|=12, |BC|=6,$

$|BD|=|CD|=13.$

{\it D}





{\it Egzamin maturalny z matematyki}

{\it Poziom podstawowy}

{\it 15}

Odpowiedzí :
\begin{center}
\includegraphics[width=82.044mm,height=17.832mm]{./F1_M_PP_M2010_page14_images/image001.eps}
\end{center}
Wypelnia

egzaminator

Nr zadania

Maks. liczba kt

32.

4

Uzyskana liczba pkt





{\it 16}

{\it Egzamin maturalny z matematyki}

{\it Poziom podstawowy}

Zadanie 33. $(4pkt)$

Doświadczenie losowe polega na dwukrotnym rzucie symetryczną sześcienną kostką do gry.

Oblicz prawdopodobieństwo zdarzenia $A$ polegającego na tym, $\dot{\mathrm{z}}\mathrm{e}$ w pierwszym rzucie

otrzymamy parzystą liczbę oczek i iloczyn liczb oczek w obu rzutach będzie podzielny przez 12.

Wynik przedstaw w postaci ułamka zwykłego nieskracalnego.





{\it Egzamin maturalny z matematyki}

{\it Poziom podstawowy}

{\it 1}7

Odpowiedzí :
\begin{center}
\includegraphics[width=82.044mm,height=17.784mm]{./F1_M_PP_M2010_page16_images/image001.eps}
\end{center}
Wypelnia

egzaminator

Nr zadania

Maks. lÍczba kt

33.

4

Uzyskana lÍczba pkt





{\it 18}

{\it Egzamin maturalny z matematyki}

{\it Poziom podstawowy}

Zadanie 34. $(5pkt)$

$\mathrm{W}$ dwóch hotelach wybudowano prostokątne baseny. Basen w pierwszym hotelu

ma powierzchnię 240 $\mathrm{m}^{2}$ Basen w drugim hotelu ma powierzchnię 350 $\mathrm{m}^{2}$ oraz jest o 5 $\mathrm{m}$

dłuzszy i 2 $\mathrm{m}$ szerszy $\mathrm{n}\mathrm{i}\dot{\mathrm{z}}$ w pierwszym hotelu. Oblicz, jakie wymiary mogą mieć baseny

w obu hotelach. Podaj wszystkie $\mathrm{m}\mathrm{o}\dot{\mathrm{z}}$ liwe odpowiedzi.





{\it Egzamin maturalny z matematyki}

{\it Poziom podstawowy}

{\it 19}

Odpowiedzí :
\begin{center}
\includegraphics[width=82.044mm,height=17.832mm]{./F1_M_PP_M2010_page18_images/image001.eps}
\end{center}
Wypelnia

egzaminator

Nr zadania

Maks. liczba kt

34.

5

Uzyskana liczba pkt





$ 2\theta$

{\it Egzamin maturalny z matematyki}

{\it Poziom podstawowy}

BRUDNOPIS





{\it Egzamin maturalny z matematyki}

{\it Poziom podstawowy}

{\it 3}

BRUDNOPIS





{\it 4}

{\it Egzamin maturalny z matematyki}

{\it Poziom podstawowy}

Zadanie 6. $(1pkt)$

Rozwiązaniem równania $\displaystyle \frac{3x-1}{7x+1}=\frac{2}{5}$ jest

A. 1 B. -73

C.

-47

D. 7

Zadanie 7. $(1pkt)$

Do zbioru rozwiązań nierównoŚci $(x-2)(x+3)<0$ nalezy liczba

A. 9 B. 7 C. 4

D. l

Zadanie 8. $(1pkt)$

Wykresem funkcji kwadratowej $f(x)=-3x^{2}+3$ jest parabola o wierzchołku w punkcie

A. $($3, $0)$ B. $(0,3)$ C. $(-3,0)$ D. $(0,-3)$

Zadanie 9. $(1pkt)$

Prosta o równaniu $y=-2x+(3m+3)$ przecina w układzie współrzędnych oś $Oy$ w punkcie

(0,2). Wtedy

A. {\it m}$=$--23

B.

{\it m}$=$- -31

C.

{\it m}$=$ -31

D.

{\it m}$=$ -35

Zadanie 10. $(1pkt)$

Na rysunku jest przedstawiony wykres funkcji $y=f(x).$
\begin{center}
\begin{tabular}{|l|l|l|l|l|l|l|l|l|l|l|l|l|l|l|}
\hline
\multicolumn{1}{|l|}{}&	\multicolumn{1}{|l|}{}&	\multicolumn{1}{|l|}{}&	\multicolumn{1}{|l|}{$y$}&	\multicolumn{1}{|l|}{}&	\multicolumn{1}{|l|}{}&	\multicolumn{1}{|l|}{}&	\multicolumn{1}{|l|}{}&	\multicolumn{1}{|l|}{}&	\multicolumn{1}{|l|}{}&	\multicolumn{1}{|l|}{}&	\multicolumn{1}{|l|}{}&	\multicolumn{1}{|l|}{}&	\multicolumn{1}{|l|}{}&	\multicolumn{1}{|l|}{}	\\
\hline
\multicolumn{1}{|l|}{}&	\multicolumn{1}{|l|}{}&	\multicolumn{1}{|l|}{}&	\multicolumn{1}{|l|}{}&	\multicolumn{1}{|l|}{}&	\multicolumn{1}{|l|}{}&	\multicolumn{1}{|l|}{}&	\multicolumn{1}{|l|}{}&	\multicolumn{1}{|l|}{}&	\multicolumn{1}{|l|}{}&	\multicolumn{1}{|l|}{}&	\multicolumn{1}{|l|}{}&	\multicolumn{1}{|l|}{}&	\multicolumn{1}{|l|}{}&	\multicolumn{1}{|l|}{}	\\
\hline
\multicolumn{1}{|l|}{}&	\multicolumn{1}{|l|}{}&	\multicolumn{1}{|l|}{}&	\multicolumn{1}{|l|}{}&	\multicolumn{1}{|l|}{}&	\multicolumn{1}{|l|}{}&	\multicolumn{1}{|l|}{}&	\multicolumn{1}{|l|}{}&	\multicolumn{1}{|l|}{}&	\multicolumn{1}{|l|}{}&	\multicolumn{1}{|l|}{}&	\multicolumn{1}{|l|}{}&	\multicolumn{1}{|l|}{}&	\multicolumn{1}{|l|}{}&	\multicolumn{1}{|l|}{}	\\
\hline
\multicolumn{1}{|l|}{}&	\multicolumn{1}{|l|}{}&	\multicolumn{1}{|l|}{}&	\multicolumn{1}{|l|}{}&	\multicolumn{1}{|l|}{}&	\multicolumn{1}{|l|}{}&	\multicolumn{1}{|l|}{}&	\multicolumn{1}{|l|}{}&	\multicolumn{1}{|l|}{}&	\multicolumn{1}{|l|}{}&	\multicolumn{1}{|l|}{}&	\multicolumn{1}{|l|}{}&	\multicolumn{1}{|l|}{}&	\multicolumn{1}{|l|}{}&	\multicolumn{1}{|l|}{}	\\
\hline
\multicolumn{1}{|l|}{}&	\multicolumn{1}{|l|}{}&	\multicolumn{1}{|l|}{}&	\multicolumn{1}{|l|}{}&	\multicolumn{1}{|l|}{}&	\multicolumn{1}{|l|}{}&	\multicolumn{1}{|l|}{}&	\multicolumn{1}{|l|}{}&	\multicolumn{1}{|l|}{}&	\multicolumn{1}{|l|}{}&	\multicolumn{1}{|l|}{}&	\multicolumn{1}{|l|}{}&	\multicolumn{1}{|l|}{}&	\multicolumn{1}{|l|}{}&	\multicolumn{1}{|l|}{}	\\
\hline
\multicolumn{1}{|l|}{}&	\multicolumn{1}{|l|}{}&	\multicolumn{1}{|l|}{}&	\multicolumn{1}{|l|}{}&	\multicolumn{1}{|l|}{}&	\multicolumn{1}{|l|}{}&	\multicolumn{1}{|l|}{}&	\multicolumn{1}{|l|}{}&	\multicolumn{1}{|l|}{}&	\multicolumn{1}{|l|}{}&	\multicolumn{1}{|l|}{}&	\multicolumn{1}{|l|}{}&	\multicolumn{1}{|l|}{}&	\multicolumn{1}{|l|}{}&	\multicolumn{1}{|l|}{}	\\
\hline
\multicolumn{1}{|l|}{}&	\multicolumn{1}{|l|}{}&	\multicolumn{1}{|l|}{}&	\multicolumn{1}{|l|}{}&	\multicolumn{1}{|l|}{}&	\multicolumn{1}{|l|}{}&	\multicolumn{1}{|l|}{}&	\multicolumn{1}{|l|}{}&	\multicolumn{1}{|l|}{}&	\multicolumn{1}{|l|}{}&	\multicolumn{1}{|l|}{}&	\multicolumn{1}{|l|}{}&	\multicolumn{1}{|l|}{}&	\multicolumn{1}{|l|}{}&	\multicolumn{1}{|l|}{}	\\
\hline
\multicolumn{1}{|l|}{}&	\multicolumn{1}{|l|}{}&	\multicolumn{1}{|l|}{}&	\multicolumn{1}{|l|}{}&	\multicolumn{1}{|l|}{}&	\multicolumn{1}{|l|}{}&	\multicolumn{1}{|l|}{}&	\multicolumn{1}{|l|}{}&	\multicolumn{1}{|l|}{}&	\multicolumn{1}{|l|}{}&	\multicolumn{1}{|l|}{}&	\multicolumn{1}{|l|}{}&	\multicolumn{1}{|l|}{}&	\multicolumn{1}{|l|}{}&	\multicolumn{1}{|l|}{}	\\
\hline
\multicolumn{1}{|l|}{}&	\multicolumn{1}{|l|}{}&	\multicolumn{1}{|l|}{}&	\multicolumn{1}{|l|}{}&	\multicolumn{1}{|l|}{}&	\multicolumn{1}{|l|}{}&	\multicolumn{1}{|l|}{}&	\multicolumn{1}{|l|}{}&	\multicolumn{1}{|l|}{}&	\multicolumn{1}{|l|}{}&	\multicolumn{1}{|l|}{}&	\multicolumn{1}{|l|}{}&	\multicolumn{1}{|l|}{}&	\multicolumn{1}{|l|}{}&	\multicolumn{1}{|l|}{ $x$}	\\
\hline
\multicolumn{1}{|l|}{}&	\multicolumn{1}{|l|}{}&	\multicolumn{1}{|l|}{ $0$}&	\multicolumn{1}{|l|}{}&	\multicolumn{1}{|l|}{}&	\multicolumn{1}{|l|}{}&	\multicolumn{1}{|l|}{}&	\multicolumn{1}{|l|}{}&	\multicolumn{1}{|l|}{}&	\multicolumn{1}{|l|}{}&	\multicolumn{1}{|l|}{}&	\multicolumn{1}{|l|}{}&	\multicolumn{1}{|l|}{ $1$}&	\multicolumn{1}{|l|}{O 1}&	\multicolumn{1}{|l|}{$1$}	\\
\hline
\multicolumn{1}{|l|}{}&	\multicolumn{1}{|l|}{}&	\multicolumn{1}{|l|}{}&	\multicolumn{1}{|l|}{}&	\multicolumn{1}{|l|}{}&	\multicolumn{1}{|l|}{}&	\multicolumn{1}{|l|}{}&	\multicolumn{1}{|l|}{}&	\multicolumn{1}{|l|}{}&	\multicolumn{1}{|l|}{}&	\multicolumn{1}{|l|}{}&	\multicolumn{1}{|l|}{}&	\multicolumn{1}{|l|}{}&	\multicolumn{1}{|l|}{}&	\multicolumn{1}{|l|}{}	\\
\hline
\end{tabular}

\end{center}
Które równanie ma dokładnie trzy rozwiązania?

A. $f(x)=0$

B. $f(x)=1$

C. $f(x)=2$

D. $f(x)=3$

Zadanie ll. $(1pkt)$

$\mathrm{W}$ ciągu arytmetycznym $(a_{n})$ dane są: $a_{3}=13\mathrm{i}a_{5}=39$. Wtedy wyraz $a_{1}$ jest równy

A. 13

B. 0

C. $-13$

D. $-26$

Zadanie 12. $(1pkt)$

$\mathrm{W}$ ciągu geometrycznym $(a_{n})$ dane są: $a_{1}=3\mathrm{i}a_{4}=24$. Iloraz tego ciągujest równy

A. 8 B. 2 C. -81 D. --21





{\it Egzamin maturalny z matematyki}

{\it Poziom podstawowy}

{\it 5}

BRUDNOPIS





{\it 6}

{\it Egzamin maturalny z matematyki}

{\it Poziom podstawowy}

Zadanie 13. (1pkt)

Liczba przekątnych siedmiokąta foremnegojest równa

A. 7

B. 14

C. 21

D. 28

Zadanie 14. $(1pkt)$

Kąt $\alpha$ jest ostry i $\displaystyle \sin\alpha=\frac{3}{4}$. Wartość wyrazenia $ 2-\cos^{2}\alpha$ jest równa

A. --2165 B. -23 C. --1176 D.

$\displaystyle \frac{31}{16}$

Zadanie 15. (1pkt)

Okrąg opisany na kwadracie ma promień 4. Długość boku tego kwadratujest równa

A. $4\sqrt{2}$

B. $2\sqrt{2}$

C. 8

D. 4

Zadanie 16. (1pkt)

Podstawa trójkąta równoramiennego ma długość 6, a ramię ma długość 5.

opuszczona na podstawę ma długość

Wysokość

A. 3

B. 4

C. $\sqrt{34}$

D. $\sqrt{61}$

Zadanie 17. (1pkt)

Odcinki AB i DE są równoległe. Długości odcinków CD, DE i AB są odpowiednio równe

1, 3 i 9. Długość odcinka AD jest równa
\begin{center}
\includegraphics[width=90.624mm,height=47.652mm]{./F1_M_PP_M2010_page5_images/image001.eps}
\end{center}
{\it C}

1

{\it D E}

3

{\it A}  9  {\it B}

A. 2

B. 3

C. 5

D. 6

Zadanie 18. $(1pkt)$

Punkty $A, B, C$ lez$\cdot$ące na okręgu o środku $S$ są wierzchołkami trójkąta równobocznego. Miara

zaznaczonego na rysunku kąta środkowego $ASB$ jest równa
\begin{center}
\includegraphics[width=65.436mm,height=70.968mm]{./F1_M_PP_M2010_page5_images/image002.eps}
\end{center}
{\it C}

{\it S}

{\it A  B}

B. $90^{\mathrm{o}}$  C. $60^{\mathrm{o}}$

A. $120^{\mathrm{o}}$

D. $30^{\mathrm{o}}$





{\it Egzamin maturalny z matematyki}

{\it Poziom podstawowy}

7

BRUDNOPIS





{\it 8}

{\it Egzamin maturalny z matematyki}

{\it Poziom podstawowy}

Zadanie 19. (1pkt)

Latawiec ma wymiary podane na

zacieniowanego trójkątajest równa

rysunku. Powierzchnia

A. 3200 $\mathrm{c}\mathrm{m}^{2}$

B. 6400 $\mathrm{c}\mathrm{m}^{2}$
\begin{center}
\includegraphics[width=33.480mm,height=80.676mm]{./F1_M_PP_M2010_page7_images/image001.eps}
\end{center}
30

1600 $\mathrm{c}\mathrm{m}^{2}$

800 $\mathrm{c}\mathrm{m}^{2}$

C.

D.

Zadanie 20. $(1pkt)$

Współczynnik kierunkowy prostej równoległej do prostej o równaniu $y=-3x+5$ jest równy:

A.

- -31

B. $-3$

C.

-31

D. 3

Zadanie 21. (1pkt)

Wskaz równanie okręgu o promieniu 6.

A. $x^{2}+y^{2}=3$

B. $x^{2}+y^{2}=6$

C. $x^{2}+y^{2}=12$

D. $x^{2}+y^{2}=36$

Zadanie 22. $(1pkt)$

Punkty $A=(-5,2) \mathrm{i} B=(3,-2)$ są wierzchołkami trójkąta równobocznego $ABC$. Obwód

tego trójkątajest równy

A. 30

B. $4\sqrt{5}$

C. $12\sqrt{5}$

D. 36

Zadanie 23. $(1pkt)$

Pole powierzchni całkowitej prostopadłoŚcianu o wymiarach $5\times 3\times 4$ jest równe

A. 94

B. 60

C. 47

D. 20

Zadanie 24. (1pkt)

Ostrosłup ma 18 wierzchołków. Liczba wszystkich krawędzi tego ostrosłupajest równa

A. ll

B. 18

C. 27

D. 34

Zadanie 25. (1pkt)

Średnia arytmetyczna dziesięciu liczb x, 3, 1, 4, 1, 5, 1, 4, 1, 5jest równa 3. Wtedy

A. $x=2$

B. $x=3$

C. $x=4$

D. $x=5$





{\it Egzamin maturalny z matematyki}

{\it Poziom podstawowy}

{\it 9}

BRUDNOPIS





$ 1\theta$

{\it Egzamin maturalny z matematyki}

{\it Poziom podstawowy}

ZADANIA OTWARTE

{\it Rozwiqzania zadań o numerach od 26. do 34. nalezy zapisać w} $wyznacz\theta nych$ {\it miejscach}

{\it pod treściq zadania}.

Zadanie 26. $(2pkt)$

Rozwiąz nierówność $x^{2}-x-2\leq 0.$

Odpowied $\acute{\mathrm{z}}$:

Zadanie 27. $(2pkt)$

Rozwiąz równanie $x^{3}-7x^{2}-4x+28=0.$

Odpowied $\acute{\mathrm{z}}$:



\end{document}