\documentclass[a4paper,12pt]{article}
\usepackage{latexsym}
\usepackage{amsmath}
\usepackage{amssymb}
\usepackage{graphicx}
\usepackage{wrapfig}
\pagestyle{plain}
\usepackage{fancybox}
\usepackage{bm}

\begin{document}

{\it 2}

{\it Egzamin maturalny z matematyki}

{\it Poziom podstawowy}

ZADANIA ZAMKNIĘTE

{\it Wzadaniach} $\theta d1.$ {\it do 25. wybierz i zaznacz na karcie odpowiedzipoprawnq odpowied} $\acute{z}.$

Zadanie l. $(1pkt)$

Wskaz rysunek, na którymjest przedstawiony zbiór rozwiązań nierówności $|x+7|>5.$
\begin{center}
\includegraphics[width=173.280mm,height=13.212mm]{./F1_M_PP_M2010_page1_images/image001.eps}
\end{center}
$-12$  2  {\it x}

A.
\begin{center}
\includegraphics[width=175.008mm,height=13.812mm]{./F1_M_PP_M2010_page1_images/image002.eps}
\end{center}
2  12  {\it x}

B.
\begin{center}
\includegraphics[width=173.280mm,height=13.260mm]{./F1_M_PP_M2010_page1_images/image003.eps}
\end{center}
$-12  -2$  {\it x}

C.
\begin{center}
\includegraphics[width=171.756mm,height=13.104mm]{./F1_M_PP_M2010_page1_images/image004.eps}
\end{center}
$-2$  12  {\it x}

D.

Zadanie 2. (1pkt)

Spodnie po obnizce ceny o 30\% kosztują 126 zł. I1e kosztowały spodnie przed obnizką?

A. 163,80 zł

B. 180 zł

C. 294 zł

D. 420 zł

Zadanie 3. $(1pkt)$

Liczba $(\displaystyle \frac{2^{-2}\cdot 3^{-1}}{2^{-1}3^{-2}})^{0}$ jest równa

A. I B. 4

C. 9

D. 36

Zadanie 4. (1pkt)

Liczba $\log_{4}8+\log_{4}2$ jest równa

A. l

B. 2

C. $\log_{4}6$

D. log410

Zadanie 5. $(1pkt)$

Dane są wielomiany $W(x)=-2x^{3}+5x^{2}-3$ oraz $P(x)=2x^{3}+12x$. Wielomian $W(x)+P(x)$

jest równy

A. $5x^{2}+12x-3$

B. $4x^{3}+5x^{2}+12x-3$

C. $4x^{6}+5x^{2}+12x-3$

D. $4x^{3}+12x^{2}-3$
\end{document}
