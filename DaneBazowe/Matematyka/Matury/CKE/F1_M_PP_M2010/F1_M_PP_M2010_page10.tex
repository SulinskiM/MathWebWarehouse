\documentclass[a4paper,12pt]{article}
\usepackage{latexsym}
\usepackage{amsmath}
\usepackage{amssymb}
\usepackage{graphicx}
\usepackage{wrapfig}
\pagestyle{plain}
\usepackage{fancybox}
\usepackage{bm}

\begin{document}

{\it Egzamin maturalny z matematyki}

{\it Poziom podstawowy}

{\it 11}

Zadanie 28. (2pkt)

Trójkąty prostokątne równoramienne $ABC\mathrm{i}CDE$ są połozone tak, jak na ponizszym rysunku

(w obu trójkątach kąt przy wierzchołku $C$ jest prosty). Wykaz, $\dot{\mathrm{z}}\mathrm{e}|AD|=|BE|.$

{\it C}
\begin{center}
\includegraphics[width=76.404mm,height=36.072mm]{./F1_M_PP_M2010_page10_images/image001.eps}
\end{center}
{\it E}

{\it D}

{\it A  B}
\begin{center}
\includegraphics[width=109.980mm,height=17.832mm]{./F1_M_PP_M2010_page10_images/image002.eps}
\end{center}
Nr zadani,`

Wypelnia Maks. liczba kt

egzaminator

Uzyskana liczba pkt

2

27.

2

28.

2
\end{document}
