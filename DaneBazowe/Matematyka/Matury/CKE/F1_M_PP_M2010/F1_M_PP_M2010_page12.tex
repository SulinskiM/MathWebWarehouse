\documentclass[a4paper,12pt]{article}
\usepackage{latexsym}
\usepackage{amsmath}
\usepackage{amssymb}
\usepackage{graphicx}
\usepackage{wrapfig}
\pagestyle{plain}
\usepackage{fancybox}
\usepackage{bm}

\begin{document}

{\it Egzamin maturalny z matematyki}

{\it Poziom podstawowy}

{\it 13}

Zadanie 31. (2pkt)

W trapezie prostokątnym krótsza przekątna dzieli go na trójkąt prostokątny

równoboczny. Dłuzsza podstawa trapezujest równa 6. Ob1icz obwód tego trapezu.

i trójkąt

Odpowiedzí :
\begin{center}
\includegraphics[width=109.980mm,height=17.784mm]{./F1_M_PP_M2010_page12_images/image001.eps}
\end{center}
Nr zadania

Wypelnia Maks. liczba kt

egzaminator

Uzyskana lÍczba pkt

2

30.

2

31.

2
\end{document}
