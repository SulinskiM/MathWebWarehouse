\documentclass[a4paper,12pt]{article}
\usepackage{latexsym}
\usepackage{amsmath}
\usepackage{amssymb}
\usepackage{graphicx}
\usepackage{wrapfig}
\pagestyle{plain}
\usepackage{fancybox}
\usepackage{bm}

\begin{document}

{\it 16}

{\it Egzamin maturalny z matematyki}

{\it Poziom podstawowy}

Zadanie 33. $(4pkt)$

Doświadczenie losowe polega na dwukrotnym rzucie symetryczną sześcienną kostką do gry.

Oblicz prawdopodobieństwo zdarzenia $A$ polegającego na tym, $\dot{\mathrm{z}}\mathrm{e}$ w pierwszym rzucie

otrzymamy parzystą liczbę oczek i iloczyn liczb oczek w obu rzutach będzie podzielny przez 12.

Wynik przedstaw w postaci ułamka zwykłego nieskracalnego.
\end{document}
