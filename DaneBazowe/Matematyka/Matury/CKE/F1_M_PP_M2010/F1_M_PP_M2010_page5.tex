\documentclass[a4paper,12pt]{article}
\usepackage{latexsym}
\usepackage{amsmath}
\usepackage{amssymb}
\usepackage{graphicx}
\usepackage{wrapfig}
\pagestyle{plain}
\usepackage{fancybox}
\usepackage{bm}

\begin{document}

{\it 6}

{\it Egzamin maturalny z matematyki}

{\it Poziom podstawowy}

Zadanie 13. (1pkt)

Liczba przekątnych siedmiokąta foremnegojest równa

A. 7

B. 14

C. 21

D. 28

Zadanie 14. $(1pkt)$

Kąt $\alpha$ jest ostry i $\displaystyle \sin\alpha=\frac{3}{4}$. Wartość wyrazenia $ 2-\cos^{2}\alpha$ jest równa

A. --2165 B. -23 C. --1176 D.

$\displaystyle \frac{31}{16}$

Zadanie 15. (1pkt)

Okrąg opisany na kwadracie ma promień 4. Długość boku tego kwadratujest równa

A. $4\sqrt{2}$

B. $2\sqrt{2}$

C. 8

D. 4

Zadanie 16. (1pkt)

Podstawa trójkąta równoramiennego ma długość 6, a ramię ma długość 5.

opuszczona na podstawę ma długość

Wysokość

A. 3

B. 4

C. $\sqrt{34}$

D. $\sqrt{61}$

Zadanie 17. (1pkt)

Odcinki AB i DE są równoległe. Długości odcinków CD, DE i AB są odpowiednio równe

1, 3 i 9. Długość odcinka AD jest równa
\begin{center}
\includegraphics[width=90.624mm,height=47.652mm]{./F1_M_PP_M2010_page5_images/image001.eps}
\end{center}
{\it C}

1

{\it D E}

3

{\it A}  9  {\it B}

A. 2

B. 3

C. 5

D. 6

Zadanie 18. $(1pkt)$

Punkty $A, B, C$ lez$\cdot$ące na okręgu o środku $S$ są wierzchołkami trójkąta równobocznego. Miara

zaznaczonego na rysunku kąta środkowego $ASB$ jest równa
\begin{center}
\includegraphics[width=65.436mm,height=70.968mm]{./F1_M_PP_M2010_page5_images/image002.eps}
\end{center}
{\it C}

{\it S}

{\it A  B}

B. $90^{\mathrm{o}}$  C. $60^{\mathrm{o}}$

A. $120^{\mathrm{o}}$

D. $30^{\mathrm{o}}$
\end{document}
