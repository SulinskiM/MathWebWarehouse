\documentclass[a4paper,12pt]{article}
\usepackage{latexsym}
\usepackage{amsmath}
\usepackage{amssymb}
\usepackage{graphicx}
\usepackage{wrapfig}
\pagestyle{plain}
\usepackage{fancybox}
\usepackage{bm}

\begin{document}

{\it 8}

{\it Egzamin maturalny z matematyki}

{\it Poziom podstawowy}

Zadanie 19. (1pkt)

Latawiec ma wymiary podane na

zacieniowanego trójkątajest równa

rysunku. Powierzchnia

A. 3200 $\mathrm{c}\mathrm{m}^{2}$

B. 6400 $\mathrm{c}\mathrm{m}^{2}$
\begin{center}
\includegraphics[width=33.480mm,height=80.676mm]{./F1_M_PP_M2010_page7_images/image001.eps}
\end{center}
30

1600 $\mathrm{c}\mathrm{m}^{2}$

800 $\mathrm{c}\mathrm{m}^{2}$

C.

D.

Zadanie 20. $(1pkt)$

Współczynnik kierunkowy prostej równoległej do prostej o równaniu $y=-3x+5$ jest równy:

A.

- -31

B. $-3$

C.

-31

D. 3

Zadanie 21. (1pkt)

Wskaz równanie okręgu o promieniu 6.

A. $x^{2}+y^{2}=3$

B. $x^{2}+y^{2}=6$

C. $x^{2}+y^{2}=12$

D. $x^{2}+y^{2}=36$

Zadanie 22. $(1pkt)$

Punkty $A=(-5,2) \mathrm{i} B=(3,-2)$ są wierzchołkami trójkąta równobocznego $ABC$. Obwód

tego trójkątajest równy

A. 30

B. $4\sqrt{5}$

C. $12\sqrt{5}$

D. 36

Zadanie 23. $(1pkt)$

Pole powierzchni całkowitej prostopadłoŚcianu o wymiarach $5\times 3\times 4$ jest równe

A. 94

B. 60

C. 47

D. 20

Zadanie 24. (1pkt)

Ostrosłup ma 18 wierzchołków. Liczba wszystkich krawędzi tego ostrosłupajest równa

A. ll

B. 18

C. 27

D. 34

Zadanie 25. (1pkt)

Średnia arytmetyczna dziesięciu liczb x, 3, 1, 4, 1, 5, 1, 4, 1, 5jest równa 3. Wtedy

A. $x=2$

B. $x=3$

C. $x=4$

D. $x=5$
\end{document}
