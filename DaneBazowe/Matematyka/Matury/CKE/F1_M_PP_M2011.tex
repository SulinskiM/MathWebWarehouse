\documentclass[a4paper,12pt]{article}
\usepackage{latexsym}
\usepackage{amsmath}
\usepackage{amssymb}
\usepackage{graphicx}
\usepackage{wrapfig}
\pagestyle{plain}
\usepackage{fancybox}
\usepackage{bm}

\begin{document}

Centralna Komisja Egzaminacyjna

Arkusz zawiera informacje prawnie chronione do momentu rozpoczęcia egzaminu.

WPISUJE ZDAJACY

KOD PESEL

{\it Miejsce}

{\it na naklejkę}

{\it z kodem}
\begin{center}
\includegraphics[width=21.432mm,height=9.804mm]{./F1_M_PP_M2011_page0_images/image001.eps}

\includegraphics[width=82.092mm,height=9.804mm]{./F1_M_PP_M2011_page0_images/image002.eps}

\includegraphics[width=204.060mm,height=216.048mm]{./F1_M_PP_M2011_page0_images/image003.eps}
\end{center}
EGZAMIN MATU

Z MATEMATY

LNY

POZIOM PODSTAWOWY  MAJ 2011

1. Sprawd $\acute{\mathrm{z}}$, czy arkusz egzaminacyjny zawiera 20 stron

(zadania $1-33$). Ewentualny brak zgłoś przewodniczącemu

zespo nadzorującego egzamin.

2. Rozwiązania zadań i odpowiedzi wpisuj w miejscu na to

przeznaczonym.

3. Odpowiedzi do zadań za iętych (l-23) przenieś

na ka ę odpowiedzi, zaznaczając je w części ka $\mathrm{y}$

przeznaczonej dla zdającego. Zamaluj $\blacksquare$ pola do tego

przeznaczone. Błędne zaznaczenie otocz kółkiem \fcircle$\bullet$

i zaznacz właściwe.

4. Pamiętaj, $\dot{\mathrm{z}}\mathrm{e}$ pominięcie argumentacji lub istotnych

obliczeń w rozwiązaniu zadania otwa ego (24-33) $\mathrm{m}\mathrm{o}\dot{\mathrm{z}}\mathrm{e}$

spowodować, $\dot{\mathrm{z}}\mathrm{e}$ za to rozwiązanie nie będziesz mógł

dostać pełnej liczby punktów.

5. Pisz czytelnie i $\mathrm{u}\dot{\mathrm{z}}$ aj tvlko $\mathrm{d}$ gopisu lub -Dióra

z czatnym tuszem lub atramentem.

6. Nie uzywaj korektora, a błędne zapisy wyrazínie prze eśl.

7. Pamiętaj, $\dot{\mathrm{z}}\mathrm{e}$ zapisy w brudnopisie nie będą oceniane.

8. $\mathrm{M}\mathrm{o}\dot{\mathrm{z}}$ esz korzystać z zestawu wzorów matematycznych,

cyrkla i linijki oraz kalkulatora.

9. Na karcie odpowiedzi wpisz swój numer PESEL i przyklej

naklejkę z kodem.

10. Nie wpisuj $\dot{\mathrm{z}}$ adnych znaków w części przeznaczonej dla

egzaminatora.

Czas pracy:

170 minut

Liczba punktów

do uzyskania: 50

$\Vert\Vert\Vert\Vert\Vert\Vert\Vert\Vert\Vert\Vert\Vert\Vert\Vert\Vert\Vert\Vert\Vert\Vert\Vert\Vert\Vert\Vert\Vert\Vert|  \mathrm{M}\mathrm{M}\mathrm{A}-\mathrm{P}1_{-}1\mathrm{P}-112$




{\it 2}

{\it Egzamin maturalny z matematyki}

{\it Poziom podstawowy}

ZADANIA ZAMKNIĘTE

{\it Wzadaniach} $\theta d1.$ {\it do 23. wybierz i zaznacz na karcie odpowiedzipoprawnq odpowied} $\acute{z}.$

Zadanie l. $(1pkt)$

Wska $\dot{\mathrm{z}}$ nierówność, którą spełnia liczba $\pi.$

A. $|x+1|>5$ B. $|x-1|<2$

C.

$|x+\displaystyle \frac{2}{3}|\leq 4$

D.

$|x-\displaystyle \frac{1}{3}|\geq 3$

Zadanie 2. (1pkt)

Pierwsza rata, która stanowi 9\% ceny roweru, jest równa 189zł. Rower kosztuje

A. 1701 zł.

B. 2100 zł.

C. 1890 zł.

D. 2091 zł.

Zadanie 3. $(1pkt)$

Wyrazenie $5a^{2}-10ab+15a$ jest równe iloczynowi

A. $5a^{2}(1-10b+3)$

B. $5a(a-2b+3)$

C. $5a(a-10b+15)$

D. $5(a-2b+3)$

Zadanie 4. (1pkt)

Układ równań 

A. $a=-1$

B. $a=0$

C. $a=2$

D. $a=3$

Zadanie 5. $(1pkt)$

Rozwiązanie równania $x(x+3)-49=x(x-4)$ nalezy do przedziału

A.

$(-\infty,3)$

B. $(10,+\infty)$

C. $(-5,-1)$

D. $(2,+\infty)$

Zadanie 6. $(1pkt)$

Najmniejszą liczbą całkowitą nalezącą do zbioru rozwiązań nierówności $\displaystyle \frac{3}{8}+\frac{x}{6}<\frac{5x}{12}$ jest

A. l

B. 2

C. $-1$

D. $-2$

Zadanie 7. $(1pkt)$

Wskaz, który zbiór przedstawiony na osi liczbowej jest zbiorem liczb spełniających

jednocześnie następujące nierówności: 3 $(x-1)(x-5)\leq 0 \mathrm{i} x>1.$
\begin{center}
\includegraphics[width=45.264mm,height=7.320mm]{./F1_M_PP_M2011_page1_images/image001.eps}
\end{center}
A.

B.

1

$\check{}$6

$\underline{x}$
\begin{center}
\includegraphics[width=36.168mm,height=7.320mm]{./F1_M_PP_M2011_page1_images/image002.eps}

\includegraphics[width=84.228mm,height=14.172mm]{./F1_M_PP_M2011_page1_images/image003.eps}
\end{center}
{\it x}

1 5

D.

$\underline{x}$

-$\check{}$5

C.

1





{\it Egzamin maturalny z matematyki}

{\it Poziom podstawowy}

{\it 11}

Zadanie 26. (2pkt)

Na rysunku przedstawiono wykres funkcjif.
\begin{center}
\includegraphics[width=154.836mm,height=87.372mm]{./F1_M_PP_M2011_page10_images/image001.eps}
\end{center}
Odczytaj z wykresu i zapisz:

a) zbiór wartości funkcjif,

b) przedział maksymalnej długości, w którym funkcja f jest malejąca.

Odpowied $\acute{\mathrm{z}}$:
\begin{center}
\includegraphics[width=109.980mm,height=17.784mm]{./F1_M_PP_M2011_page10_images/image002.eps}
\end{center}
Nr zadania

Wypelnia Maks. liczba kt

egzaminator

Uzyskana lÍczba pkt

24.

2

25.

2

2





{\it 12}

{\it Egzamin maturalny z matematyki}

{\it Poziom podstawowy}

Zadanie 27. $(2pkt)$

Liczby $x, y$, 19 w podanej kolejności tworzą ciąg arytmetyczny, przy czym $x+y=8.$

Oblicz $x\mathrm{i}y.$

Odpowied $\acute{\mathrm{z}}$:

Zadanie 28. $(2pkt)$

Kąt $\alpha$ jest ostry i $\displaystyle \frac{\sin\alpha}{\cos\alpha}+\frac{\cos\alpha}{\sin\alpha}=2$. Oblicz wartość wyrazenia $\sin\alpha\cdot\cos\alpha.$

Odpowiedzí:





{\it Egzamin maturalny z matematyki}

{\it Poziom podstawowy}

{\it 13}

Zadanie 29. $(2pkt)$

Dany jest czworokąt ABCD, w którym AB $\Vert$ CD. Na boku $BC$ wybrano taki punkt $E,$

$\dot{\mathrm{z}}\mathrm{e}|EC|=|CD|\mathrm{i}|EB|=|BA|$. Wykaz$\cdot, \dot{\mathrm{z}}\mathrm{e}$ kąt $AED$ jest prosty.

Odpowiedzí :
\begin{center}
\includegraphics[width=109.980mm,height=17.832mm]{./F1_M_PP_M2011_page12_images/image001.eps}
\end{center}
Nr zadania

Wypelnia Maks. liczba kt

egzaminator

Uzyskana liczba pkt

27.

2

28.

2

2





{\it 14}

{\it Egzamin maturalny z matematyki}

{\it Poziom podstawowy}

Zadanie 30. (2pkt)

Ze zbioru liczb \{1, 2, 3 7\} 1osujemy ko1ejno dwa razy po jednej 1iczbie ze zwracaniem.

Oblicz prawdopodobieństwo wylosowania liczb, których sumajest podzielna przez 3.

Odpowied $\acute{\mathrm{z}}$:





{\it Egzamin maturalny z matematyki}

{\it Poziom podstawowy}

{\it 15}

Zadanie 31. $(4pkt)$

Okrąg o środku w punkcie $S=(3,7)$ jest styczny do prostej o równaniu $y=2x-3$. Oblicz

współrzędne punktu styczności.

Odpowied $\acute{\mathrm{z}}$:
\begin{center}
\includegraphics[width=96.012mm,height=17.784mm]{./F1_M_PP_M2011_page14_images/image001.eps}
\end{center}
WypelnÍa

egzaminator

Nr zadania

Maks. liczba kt

30.

2

31.

4

Uzyskana liczba pkt





{\it 16}

{\it Egzamin maturalny z matematyki}

{\it Poziom podstawowy}

Zadanie 32. $(5pkt)$

Pewien turysta pokonał trasę 112 km, przechodząc $\mathrm{k}\mathrm{a}\dot{\mathrm{z}}$ dego dnia tę samą liczbę kilometrów.

Gdyby mógł przeznaczyć na tę wędrówkę o 3 dni więcej, to w ciągu $\mathrm{k}\mathrm{a}\dot{\mathrm{z}}$ dego dnia mógłby

przechodzić o 12 km mniej. Ob1icz, i1e ki1ometrów dziennie przechodził ten turysta.





{\it Egzamin maturalny z matematyki}

{\it Poziom podstawowy}

17

Odpowiedzí :
\begin{center}
\includegraphics[width=82.044mm,height=17.832mm]{./F1_M_PP_M2011_page16_images/image001.eps}
\end{center}
Wypelnia

egzaminator

Nr zadania

Maks. liczba kt

32.

5

Uzyskana liczba pkt





{\it 18}

{\it Egzamin maturalny z matematyki}

{\it Poziom podstawowy}

Zadanie 33. (4pkt)

Punkty K, L iM są środkami krawędzi BC, GHi AE szeŚcianu ABCDEFGH o krawędzi

długości l (zobacz rysunek). Oblicz pole trójkąta KLM.





{\it Egzamin maturalny z matematyki}

{\it Poziom podstawowy}

{\it 19}

Odpowiedzí :
\begin{center}
\includegraphics[width=82.044mm,height=17.832mm]{./F1_M_PP_M2011_page18_images/image001.eps}
\end{center}
Wypelnia

egzaminator

Nr zadania

Maks. liczba kt

33.

4

Uzyskana liczba pkt





$ 2\theta$

{\it Egzamin maturalny z matematyki}

{\it Poziom podstawowy}

BRUDNOPIS





{\it Egzamin maturalny z matematyki}

{\it Poziom podstawowy}

{\it 3}

BRUDNOPIS





$\blacksquare$

$\blacksquare$

$\Vert\Vert\Vert\Vert\Vert\Vert\Vert\Vert\Vert\Vert\Vert\Vert\Vert\Vert\Vert\Vert\Vert\Vert\Vert\Vert\Vert\Vert\Vert\Vert|$
\begin{center}
\includegraphics[width=79.452mm,height=15.804mm]{./F1_M_PP_M2011_page20_images/image001.eps}
\end{center}
PESEL

$\mathrm{M}\mathrm{M}\mathrm{A}-\mathrm{P}1_{-}1$ P-112

WYPELNIA ZDAJACY
\begin{center}
\begin{tabular}{|l|l|l|l|l|}
\cline{1-1}
\multicolumn{1}{|l|}{$\begin{array}{l}\mbox{Nr}	\\	\mbox{zad.}	\end{array}$}	\\
\cline{1-1}
\multicolumn{1}{|l|}{ $1$}&	\multicolumn{1}{|l|}{ $\fbox{$\mathrm{A}$}$}&	\multicolumn{1}{|l|}{ $\fbox{$\mathrm{B}$}$}&	\multicolumn{1}{|l|}{ $\underline{\mathrm{H}\mathrm{c}}-$}&	\multicolumn{1}{|l|}{ $\Gamma \mathrm{D}\lrcorner$}	\\
\hline
\multicolumn{1}{|l|}{ $2$}&	\multicolumn{1}{|l|}{ $\fbox{$\mathrm{A}$}$}&	\multicolumn{1}{|l|}{[‡L]}&	\multicolumn{1}{|l|}{$\fbox{$\mathrm{c}$}$}&	\multicolumn{1}{|l|}{ $\fbox{$\mathrm{D}$}$}	\\
\hline
\multicolumn{1}{|l|}{ $3$}&	\multicolumn{1}{|l|}{ $\fbox{$\mathrm{A}$}$}&	\multicolumn{1}{|l|}{ $\fbox{$\mathrm{B}$}$}&	\multicolumn{1}{|l|}{ $\underline{\mathrm{H}\mathrm{c}}-$}&	\multicolumn{1}{|l|}{ $\Gamma \mathrm{D}\lrcorner$}	\\
\hline
\multicolumn{1}{|l|}{ $4$}&	\multicolumn{1}{|l|}{ $\fbox{$\mathrm{A}$}$}&	\multicolumn{1}{|l|}{ $\fbox{$\mathrm{B}$}$}&	\multicolumn{1}{|l|}{ $\fbox{$\mathrm{c}$}$}&	\multicolumn{1}{|l|}{ $\fbox{$\mathrm{D}$}$}	\\
\hline
\multicolumn{1}{|l|}{ $5$}&	\multicolumn{1}{|l|}{ $\displaystyle \prod$}&	\multicolumn{1}{|l|}{ $\fbox{$\mathrm{B}$}$}&	\multicolumn{1}{|l|}{ $\fbox{$\mathrm{c}$}$}&	\multicolumn{1}{|l|}{ $\mathrm{g}$}	\\
\hline
\multicolumn{1}{|l|}{ $6$}&	\multicolumn{1}{|l|}{ $\fbox{$\mathrm{A}$}$}&	\multicolumn{1}{|l|}{ $\fbox{$\mathrm{B}$}$}&	\multicolumn{1}{|l|}{ $\fbox{$\mathrm{c}$}$}&	\multicolumn{1}{|l|}{ $\fbox{$\mathrm{D}$}$}	\\
\hline
\multicolumn{1}{|l|}{ $7$}&	\multicolumn{1}{|l|}{ $\fbox{$\mathrm{A}$}$}&	\multicolumn{1}{|l|}{ $\fbox{$\mathrm{B}$}$}&	\multicolumn{1}{|l|}{ $\fbox{$\mathrm{c}$}$}&	\multicolumn{1}{|l|}{ $\fbox{$\mathrm{D}$}$}	\\
\hline
\multicolumn{1}{|l|}{ $8$}&	\multicolumn{1}{|l|}{ $\cap$}&	\multicolumn{1}{|l|}{ $\fbox{$\mathrm{B}$}$}&	\multicolumn{1}{|l|}{ $\fbox{$\mathrm{c}$}$}&	\multicolumn{1}{|l|}{ $\fbox{$\mathrm{D}$}$}	\\
\hline
\multicolumn{1}{|l|}{ $9$}&	\multicolumn{1}{|l|}{ $\fbox{$\mathrm{A}$}$}&	\multicolumn{1}{|l|}{ $\fbox{$\mathrm{B}$}$}&	\multicolumn{1}{|l|}{ $\fbox{$\mathrm{c}$}$}&	\multicolumn{1}{|l|}{ $\fbox{$\mathrm{D}$}$}	\\
\hline
\multicolumn{1}{|l|}{ $10$}&	\multicolumn{1}{|l|}{ $\fbox{$\mathrm{A}$}$}&	\multicolumn{1}{|l|}{ $\fbox{$\mathrm{B}$}$}&	\multicolumn{1}{|l|}{ $\fbox{$\mathrm{c}$}$}&	\multicolumn{1}{|l|}{ $\fbox{$\mathrm{D}$}$}	\\
\hline
\multicolumn{1}{|l|}{ $11$}&	\multicolumn{1}{|l|}{ $\fbox{$\mathrm{A}$}$}&	\multicolumn{1}{|l|}{ $\fbox{$\mathrm{B}$}$}&	\multicolumn{1}{|l|}{ $\underline{\mathrm{H}\mathrm{c}}-$}&	\multicolumn{1}{|l|}{ $\fbox{$\mathrm{D}$}$}	\\
\hline
\multicolumn{1}{|l|}{ $12$}&	\multicolumn{1}{|l|}{ $\fbox{$\mathrm{A}$}$}&	\multicolumn{1}{|l|}{ $\fbox{$\mathrm{B}$}$}&	\multicolumn{1}{|l|}{ $\fbox{$\mathrm{c},$}$}&	\multicolumn{1}{|l|}{ $\fbox{$\mathrm{D}$}$}	\\
\hline
\multicolumn{1}{|l|}{ $13$}&	\multicolumn{1}{|l|}{ $\displaystyle \prod$}&	\multicolumn{1}{|l|}{ $\fbox{$\mathrm{B}$}$}&	\multicolumn{1}{|l|}{ $\fbox{$\mathrm{c}$}$}&	\multicolumn{1}{|l|}{ $\fbox{$\mathrm{D}$}$}	\\
\hline
\multicolumn{1}{|l|}{ $14$}&	\multicolumn{1}{|l|}{ $\fbox{$\mathrm{A}$}$}&	\multicolumn{1}{|l|}{[‡N]}&	\multicolumn{1}{|l|}{$\fbox{$\zeta \mathrm{i},$}$}&	\multicolumn{1}{|l|}{ $\fbox{$\mathrm{D}$}$}	\\
\hline
\multicolumn{1}{|l|}{ $15$}&	\multicolumn{1}{|l|}{ $\fbox{$\mathrm{A}$}$}&	\multicolumn{1}{|l|}{ $\fbox{$\mathrm{B}$}$}&	\multicolumn{1}{|l|}{ $\mathrm{H}$-c-}&	\multicolumn{1}{|l|}{$\fbox{$\mathrm{D}$}$}	\\
\hline
\multicolumn{1}{|l|}{ $16$}&	\multicolumn{1}{|l|}{ $\fbox{$\mathrm{A}$}$}&	\multicolumn{1}{|l|}{[‡N]}&	\multicolumn{1}{|l|}{$\fbox{$\mathrm{c}$}$}&	\multicolumn{1}{|l|}{ $\fbox{$\mathrm{D}$}$}	\\
\hline
\multicolumn{1}{|l|}{ $17$}&	\multicolumn{1}{|l|}{ $\fbox{$\mathrm{A}$}$}&	\multicolumn{1}{|l|}{ $\fbox{$\mathrm{B}$}$}&	\multicolumn{1}{|l|}{ $\underline{\mathrm{H}\mathrm{c}}-$}&	\multicolumn{1}{|l|}{ $\fbox{$\ulcorner)$}$}	\\
\hline
\multicolumn{1}{|l|}{ $18$}&	\multicolumn{1}{|l|}{ $\fbox{$\mathrm{A}$}$}&	\multicolumn{1}{|l|}{[‡N]}&	\multicolumn{1}{|l|}{$\fbox{$\mathrm{c}$}$}&	\multicolumn{1}{|l|}{ $\fbox{$\mathrm{D}$}$}	\\
\hline
\multicolumn{1}{|l|}{ $19$}&	\multicolumn{1}{|l|}{ $\fbox{$\mathrm{A}$}$}&	\multicolumn{1}{|l|}{ $\fbox{$\mathrm{B}$}$}&	\multicolumn{1}{|l|}{ $\fbox{$\mathrm{c}$}$}&	\multicolumn{1}{|l|}{ $\fbox{$\ulcorner)$}$}	\\
\hline
\multicolumn{1}{|l|}{ $20$}&	\multicolumn{1}{|l|}{ $\fbox{$\mathrm{A}$}$}&	\multicolumn{1}{|l|}{ $\fbox{$\mathrm{B}$}$}&	\multicolumn{1}{|l|}{ $\fbox{$\mathrm{c}$}$}&	\multicolumn{1}{|l|}{ $\fbox{$\mathrm{D}$}$}	\\
\hline
\multicolumn{1}{|l|}{ $21$}&	\multicolumn{1}{|l|}{ $\displaystyle \prod$}&	\multicolumn{1}{|l|}{ $\fbox{$\mathrm{B}$}$}&	\multicolumn{1}{|l|}{ $\fbox{$\mathrm{c}$}$}&	\multicolumn{1}{|l|}{ $\ulcorner \mathrm{D}\rfloor$}	\\
\hline
\multicolumn{1}{|l|}{ $22$}&	\multicolumn{1}{|l|}{ $\fbox{$\mathrm{A}$}$}&	\multicolumn{1}{|l|}{ $\fbox{$\mathrm{B}$}$}&	\multicolumn{1}{|l|}{ $\fbox{$\mathrm{c}$}$}&	\multicolumn{1}{|l|}{ $\fbox{$\mathrm{D}$}$}	\\
\hline
\multicolumn{1}{|l|}{ $23$}&	\multicolumn{1}{|l|}{ $\fbox{$\mathrm{A}$}$}&	\multicolumn{1}{|l|}{ $\fbox{$\mathrm{B}$}$}&	\multicolumn{1}{|l|}{ $\fbox{$\mathrm{c}$}$}&	\multicolumn{1}{|l|}{ $\fbox{$ 1\supset$}$}	\\
\hline
\end{tabular}

\end{center}
Miejsce na naKlej$\kappa$e

z rr PESE-

WYPELNIA EGZAMINATOR
\begin{center}
\begin{tabular}{|l|l|l|l|l|l|l|}
	\\
&	\multicolumn{1}{|l|}{$0$}&	\multicolumn{1}{|l|}{ $1$}&	\multicolumn{1}{|l|}{ $2$}&	\multicolumn{1}{|l|}{ $3$}&	\multicolumn{1}{|l|}{ $4$}&	\multicolumn{1}{|l|}{ $5$}	\\
\cline{2-7}
\multicolumn{1}{|l|}{ $24$}&	\multicolumn{1}{|l|}{ $\square $}&	\multicolumn{1}{|l|}{ $\square $}&	\multicolumn{1}{|l|}{ $\square $}&	\multicolumn{1}{|l|}{}&	\multicolumn{1}{|l|}{}&	\multicolumn{1}{|l|}{}	\\
\hline
\multicolumn{1}{|l|}{ $25$}&	\multicolumn{1}{|l|}{ $\square $}&	\multicolumn{1}{|l|}{ $\square $}&	\multicolumn{1}{|l|}{ $\square $}&	\multicolumn{1}{|l|}{}&	\multicolumn{1}{|l|}{}&	\multicolumn{1}{|l|}{}	\\
\hline
\multicolumn{1}{|l|}{ $26$}&	\multicolumn{1}{|l|}{ $\square $}&	\multicolumn{1}{|l|}{ $\square $}&	\multicolumn{1}{|l|}{ $\square $}&	\multicolumn{1}{|l|}{}&	\multicolumn{1}{|l|}{}&	\multicolumn{1}{|l|}{}	\\
\hline
\multicolumn{1}{|l|}{ $27$}&	\multicolumn{1}{|l|}{ $\square $}&	\multicolumn{1}{|l|}{ $\square $}&	\multicolumn{1}{|l|}{ $\square $}&	\multicolumn{1}{|l|}{}&	\multicolumn{1}{|l|}{}&	\multicolumn{1}{|l|}{}	\\
\hline
\multicolumn{1}{|l|}{ $28$}&	\multicolumn{1}{|l|}{ $\square $}&	\multicolumn{1}{|l|}{ $\square $}&	\multicolumn{1}{|l|}{ $\square $}&	\multicolumn{1}{|l|}{}&	\multicolumn{1}{|l|}{}&	\multicolumn{1}{|l|}{}	\\
\hline
\multicolumn{1}{|l|}{ $29$}&	\multicolumn{1}{|l|}{ $\square $}&	\multicolumn{1}{|l|}{ $\square $}&	\multicolumn{1}{|l|}{ $\square $}&	\multicolumn{1}{|l|}{}&	\multicolumn{1}{|l|}{}&	\multicolumn{1}{|l|}{}	\\
\hline
\multicolumn{1}{|l|}{ $30$}&	\multicolumn{1}{|l|}{ $\square $}&	\multicolumn{1}{|l|}{ $\square $}&	\multicolumn{1}{|l|}{ $\square $}&	\multicolumn{1}{|l|}{}&	\multicolumn{1}{|l|}{}&	\multicolumn{1}{|l|}{}	\\
\hline
\multicolumn{1}{|l|}{ $31$}&	\multicolumn{1}{|l|}{ $\square $}&	\multicolumn{1}{|l|}{ $\square $}&	\multicolumn{1}{|l|}{ $\square $}&	\multicolumn{1}{|l|}{ $\square $}&	\multicolumn{1}{|l|}{ $\square $}&	\multicolumn{1}{|l|}{}	\\
\hline
\multicolumn{1}{|l|}{ $32$}&	\multicolumn{1}{|l|}{ $\square $}&	\multicolumn{1}{|l|}{ $\square $}&	\multicolumn{1}{|l|}{ $\square $}&	\multicolumn{1}{|l|}{ $\square $}&	\multicolumn{1}{|l|}{ $\square $}&	\multicolumn{1}{|l|}{ $\square $}	\\
\hline
\multicolumn{1}{|l|}{ $33$}&	\multicolumn{1}{|l|}{ $\square $}&	\multicolumn{1}{|l|}{ $\square $}&	\multicolumn{1}{|l|}{ $\square $}&	\multicolumn{1}{|l|}{ $\square $}&	\multicolumn{1}{|l|}{ $\square $}&	\multicolumn{1}{|l|}{}	\\
\hline
\end{tabular}


\includegraphics[width=14.580mm,height=9.852mm]{./F1_M_PP_M2011_page20_images/image002.eps}
\end{center}
$\blacksquare$

$\blacksquare$

SUMA

PUNKTÓW

D \square  \square  \square  \square  \square  \square  \square  \square  \square  \square 

J

0 1 2 3 4 5 6 7 8 9

0 1 2 3 4 5 6 7 8 9

$\blacksquare$




\begin{center}
\includegraphics[width=73.152mm,height=11.028mm]{./F1_M_PP_M2011_page21_images/image001.eps}
\end{center}
KOD EGZAMINATORA

Czytelny podpis egzaminatora
\begin{center}
\includegraphics[width=21.840mm,height=9.852mm]{./F1_M_PP_M2011_page21_images/image002.eps}
\end{center}
KOD ZDAJACEGO





{\it 4}

{\it Egzamin maturalny z matematyki}

{\it Poziom podstawowy}

Zadanie 8. $(1pkt)$

Wyrazenie $\log_{4}(2x-1)$ jest określone dla wszystkich liczb $x$ spełniających warunek

A.

$x\displaystyle \leq\frac{1}{2}$

B.

$x>\displaystyle \frac{1}{2}$

C. $x\leq 0$

D. $x>0$

Zadanie 9. $(1pkt)$

Dane są funkcje liniowe $f(x)=x-2$ oraz $g(x)=x+4$ określone dla wszystkich liczb

rzeczywistych $x$. Wskaz, który z ponizszych wykresów jest wykresem funkcji

$h(x)=f(x)\cdot g(x).$
\begin{center}
\includegraphics[width=29.868mm,height=49.380mm]{./F1_M_PP_M2011_page3_images/image001.eps}
\end{center}
{\it y}

{\it x}

$-4$  2
\begin{center}
\includegraphics[width=30.024mm,height=49.380mm]{./F1_M_PP_M2011_page3_images/image002.eps}
\end{center}
{\it y}

$-2$

{\it x}

4
\begin{center}
\includegraphics[width=29.868mm,height=49.380mm]{./F1_M_PP_M2011_page3_images/image003.eps}
\end{center}
{\it y}

{\it x}

$-4$  2
\begin{center}
\includegraphics[width=29.868mm,height=49.380mm]{./F1_M_PP_M2011_page3_images/image004.eps}
\end{center}
{\it y}

$-2$

{\it X}

4

A.

B.

C.

D.

Zadanie 10 $(1pkt)$

Funkcja liniowa określona jest wzorem $f(x)=-\sqrt{2}x+4$. Miejscem zerowym tej funkcjijest

liczba

A. $-2\sqrt{2}$

B.

-$\sqrt{}$22

C.

- -$\sqrt{}$22

D. $2\sqrt{2}$

Zadanie ll. $(1pkt)$

Danyjest nieskończony ciąg geometryczny $(a_{n})$, w którym $a_{3}=1 \displaystyle \mathrm{i}a_{4}=\frac{2}{3}$. Wtedy

A. {\it a}1$=- 23$ B. {\it a}1$=- 49$ C. {\it a}1$=$-23 D. {\it a}1$=$-49

Zadanie 12. $(1pkt)$

Danyjest nieskończony rosnący ciąg arytmetyczny $(a_{n})$ o wyrazach dodatnich. Wtedy

A. $a_{4}+a_{7}=a_{10}$

B. $a_{4}+a_{6}=a_{3}+a_{8}$

C. $a_{2}+a_{9}=a_{3}+a_{8}$

D. $a_{5}+a_{7}=2a_{8}$

Zadanie 13. $(1pkt)$

Kąt $\alpha$ jest ostry i $\displaystyle \cos\alpha=\frac{5}{13}$. Wtedy

A. $\displaystyle \sin\alpha=\frac{12}{13}$ oraz $\displaystyle \mathrm{t}\mathrm{g}\alpha=\frac{12}{5}$

C. $\displaystyle \sin\alpha=\frac{12}{5}$ oraz $\displaystyle \mathrm{t}\mathrm{g}\alpha=\frac{12}{13}$

B. $\displaystyle \sin\alpha=\frac{12}{13}$ oraz $\displaystyle \mathrm{t}\mathrm{g}\alpha=\frac{5}{12}$

D. $\displaystyle \sin\alpha=\frac{5}{12}$ oraz $\displaystyle \mathrm{t}\mathrm{g}\alpha=\frac{12}{13}$





{\it Egzamin maturalny z matematyki}

{\it Poziom podstawowy}

{\it 5}

BRUDNOPIS





{\it 6}

{\it Egzamin maturalny z matematyki}

{\it Poziom podstawowy}

Zadanie 14. $(1pkt)$

Wartość wyrazenia $\displaystyle \frac{\sin^{2}38^{\mathrm{o}}+\cos^{2}38^{\mathrm{o}}-1}{\sin^{2}52^{\mathrm{o}}+\cos^{2}52^{\mathrm{o}}+1}$ jest równa

A.

-21

B. 0

C.

- -21

D. l

Zadanie 15. $(1pkt)$

$\mathrm{W}$ prostopadłoŚcianie ABCDEFGH mamy: $|AB|=5, |AD|=4, |AE|=3$. Który z odcinków

{\it AB}, $BG, GE, EB$ jest najdłuzszy?

A.

{\it AB}

B.

{\it BG}

C.

{\it GE}

{\it D. EB}

Zadanie 16. $(1pkt)$

Punkt $O$ jest środkiem okręgu. Kąt wpisany $\alpha$ ma miarę
\begin{center}
\includegraphics[width=66.348mm,height=60.912mm]{./F1_M_PP_M2011_page5_images/image001.eps}
\end{center}
{\it B}

$\alpha$

{\it A}

$160^{\mathrm{o}}$  {\it C}

{\it O}

A. $80^{\mathrm{o}}$

B. $100^{\mathrm{o}}$

C. $110^{\mathrm{o}}$

D. $120^{\mathrm{o}}$

Zadanie 17. $(1pkt)$

Wysokość rombu o boku długości 6 i kącie ostrym $60^{\mathrm{o}}$ jest równa

A. $3\sqrt{3}$

B. 3

C. $6\sqrt{3}$

D. 6

Zadanie 18. $(1pkt)$

Prosta $k$ ma równanie $y=2x-3$. Wskaz równanie prostej $l$ równoległej do prostej $k$

i przechodzącej przez punkt $D$ o współrzędnych $(-2,1).$

A. $y=-2x+3$

B. $y=2x+1$

C. $y=2x+5$

D. $y=-x+1$





{\it Egzamin maturalny z matematyki}

{\it Poziom podstawowy}

7

BRUDNOPIS





{\it 8}

{\it Egzamin maturalny z matematyki}

{\it Poziom podstawowy}

Zadanie 19. $(1pkt)$

Styczną do okręgu $(x-1)^{2}+y^{2}-4=0$ jest prosta o równaniu

A. $x=1$

B. $x=3$

C. $y=0$

D. $y=4$

Zadanie 20. (1pkt)

Pole powierzchni całkowitej sześcianu jest równe 54. Długość przekątnej tego sześcianu jest

równa

A. $\sqrt{6}$

B. 3

C. 9

D. $3\sqrt{3}$

Zadanie 21. (1pkt)

Objętość stozka o wysokości 8 i średnicy podstawy 12jest równa

A. $ 124\pi$

B. $ 96\pi$

C. $ 64\pi$

D. $ 32\pi$

Zadanie 22. (1pkt)

Rzucamy dwa razy symetryczną sześcienną kostką do gry. Prawdopodobieństwo otrzymania

sumy oczek równej trzy wynosi

A.

-61

B.

-91

C.

$\displaystyle \frac{1}{12}$

D.

$\displaystyle \frac{1}{18}$

Zadanie 23. (1pkt)

Uczniowie pewnej klasy zostali poproszeni o odpowiedzí na pytanie:,,Ile osób liczy twoja

rodzina?'' Wyniki przedstawiono w tabeli:
\begin{center}
\begin{tabular}{|l|l|}
\hline
\multicolumn{1}{|l|}{$\begin{array}{l}\mbox{Liczba osób}	\\	\mbox{w rodzinie}	\end{array}$}&	\multicolumn{1}{|l|}{$\begin{array}{l}\mbox{liczba}	\\	\mbox{uczniów}	\end{array}$}	\\
\hline
\multicolumn{1}{|l|}{ $3$}&	\multicolumn{1}{|l|}{ $6$}	\\
\hline
\multicolumn{1}{|l|}{ $4$}&	\multicolumn{1}{|l|}{ $12$}	\\
\hline
\multicolumn{1}{|l|}{ $x$}&	\multicolumn{1}{|l|}{ $2$}	\\
\hline
\end{tabular}

\end{center}
Średnia liczba osób w rodzinie dla uczniów tej klasyjest równa 4. Wtedy 1iczba x jest równa

A. 3

B. 4

C. 5

D. 7





{\it Egzamin maturalny z matematyki}

{\it Poziom podstawowy}

{\it 9}

BRUDNOPIS





$ 1\theta$

{\it Egzamin maturalny z matematyki}

{\it Poziom podstawowy}

ZADANIA OTWARTE

{\it Rozwiqzania zadań o numerach od 24. do 33. nalezy zapisać w} $wyznacz\theta nych$ {\it miejscach}

{\it pod treściq zadania}.

Zadanie 24. $(2pkt)$

Rozwiąz nierówność $3x^{2}-10x+3\leq 0.$

Odpowiedzí:

Zadanie 25. $(2pkt)$

Uzasadnij, $\dot{\mathrm{z}}\mathrm{e}\mathrm{j}\mathrm{e}\dot{\mathrm{z}}$ eli $a+b=1$

$\mathrm{i} a^{2}+b^{2}=7$, to $a^{4}+b^{4}=31.$



\end{document}