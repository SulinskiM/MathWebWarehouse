\documentclass[a4paper,12pt]{article}
\usepackage{latexsym}
\usepackage{amsmath}
\usepackage{amssymb}
\usepackage{graphicx}
\usepackage{wrapfig}
\pagestyle{plain}
\usepackage{fancybox}
\usepackage{bm}

\begin{document}

{\it 2}

{\it Egzamin maturalny z matematyki}

{\it Poziom podstawowy}

ZADANIA ZAMKNIĘTE

{\it Wzadaniach} $\theta d1.$ {\it do 23. wybierz i zaznacz na karcie odpowiedzipoprawnq odpowied} $\acute{z}.$

Zadanie l. $(1pkt)$

Wska $\dot{\mathrm{z}}$ nierówność, którą spełnia liczba $\pi.$

A. $|x+1|>5$ B. $|x-1|<2$

C.

$|x+\displaystyle \frac{2}{3}|\leq 4$

D.

$|x-\displaystyle \frac{1}{3}|\geq 3$

Zadanie 2. (1pkt)

Pierwsza rata, która stanowi 9\% ceny roweru, jest równa 189zł. Rower kosztuje

A. 1701 zł.

B. 2100 zł.

C. 1890 zł.

D. 2091 zł.

Zadanie 3. $(1pkt)$

Wyrazenie $5a^{2}-10ab+15a$ jest równe iloczynowi

A. $5a^{2}(1-10b+3)$

B. $5a(a-2b+3)$

C. $5a(a-10b+15)$

D. $5(a-2b+3)$

Zadanie 4. (1pkt)

Układ równań 

A. $a=-1$

B. $a=0$

C. $a=2$

D. $a=3$

Zadanie 5. $(1pkt)$

Rozwiązanie równania $x(x+3)-49=x(x-4)$ nalezy do przedziału

A.

$(-\infty,3)$

B. $(10,+\infty)$

C. $(-5,-1)$

D. $(2,+\infty)$

Zadanie 6. $(1pkt)$

Najmniejszą liczbą całkowitą nalezącą do zbioru rozwiązań nierówności $\displaystyle \frac{3}{8}+\frac{x}{6}<\frac{5x}{12}$ jest

A. l

B. 2

C. $-1$

D. $-2$

Zadanie 7. $(1pkt)$

Wskaz, który zbiór przedstawiony na osi liczbowej jest zbiorem liczb spełniających

jednocześnie następujące nierówności: 3 $(x-1)(x-5)\leq 0 \mathrm{i} x>1.$
\begin{center}
\includegraphics[width=45.264mm,height=7.320mm]{./F1_M_PP_M2011_page1_images/image001.eps}
\end{center}
A.

B.

1

$\check{}$6

$\underline{x}$
\begin{center}
\includegraphics[width=36.168mm,height=7.320mm]{./F1_M_PP_M2011_page1_images/image002.eps}

\includegraphics[width=84.228mm,height=14.172mm]{./F1_M_PP_M2011_page1_images/image003.eps}
\end{center}
{\it x}

1 5

D.

$\underline{x}$

-$\check{}$5

C.

1
\end{document}
