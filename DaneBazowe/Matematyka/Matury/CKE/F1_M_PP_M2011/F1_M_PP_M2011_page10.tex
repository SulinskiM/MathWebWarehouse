\documentclass[a4paper,12pt]{article}
\usepackage{latexsym}
\usepackage{amsmath}
\usepackage{amssymb}
\usepackage{graphicx}
\usepackage{wrapfig}
\pagestyle{plain}
\usepackage{fancybox}
\usepackage{bm}

\begin{document}

{\it Egzamin maturalny z matematyki}

{\it Poziom podstawowy}

{\it 11}

Zadanie 26. (2pkt)

Na rysunku przedstawiono wykres funkcjif.
\begin{center}
\includegraphics[width=154.836mm,height=87.372mm]{./F1_M_PP_M2011_page10_images/image001.eps}
\end{center}
Odczytaj z wykresu i zapisz:

a) zbiór wartości funkcjif,

b) przedział maksymalnej długości, w którym funkcja f jest malejąca.

Odpowied $\acute{\mathrm{z}}$:
\begin{center}
\includegraphics[width=109.980mm,height=17.784mm]{./F1_M_PP_M2011_page10_images/image002.eps}
\end{center}
Nr zadania

Wypelnia Maks. liczba kt

egzaminator

Uzyskana lÍczba pkt

24.

2

25.

2

2
\end{document}
