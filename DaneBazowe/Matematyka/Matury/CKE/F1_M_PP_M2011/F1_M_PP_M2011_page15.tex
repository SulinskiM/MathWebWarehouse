\documentclass[a4paper,12pt]{article}
\usepackage{latexsym}
\usepackage{amsmath}
\usepackage{amssymb}
\usepackage{graphicx}
\usepackage{wrapfig}
\pagestyle{plain}
\usepackage{fancybox}
\usepackage{bm}

\begin{document}

{\it 16}

{\it Egzamin maturalny z matematyki}

{\it Poziom podstawowy}

Zadanie 32. $(5pkt)$

Pewien turysta pokonał trasę 112 km, przechodząc $\mathrm{k}\mathrm{a}\dot{\mathrm{z}}$ dego dnia tę samą liczbę kilometrów.

Gdyby mógł przeznaczyć na tę wędrówkę o 3 dni więcej, to w ciągu $\mathrm{k}\mathrm{a}\dot{\mathrm{z}}$ dego dnia mógłby

przechodzić o 12 km mniej. Ob1icz, i1e ki1ometrów dziennie przechodził ten turysta.
\end{document}
