\documentclass[a4paper,12pt]{article}
\usepackage{latexsym}
\usepackage{amsmath}
\usepackage{amssymb}
\usepackage{graphicx}
\usepackage{wrapfig}
\pagestyle{plain}
\usepackage{fancybox}
\usepackage{bm}

\begin{document}

{\it 4}

{\it Egzamin maturalny z matematyki}

{\it Poziom podstawowy}

Zadanie 8. $(1pkt)$

Wyrazenie $\log_{4}(2x-1)$ jest określone dla wszystkich liczb $x$ spełniających warunek

A.

$x\displaystyle \leq\frac{1}{2}$

B.

$x>\displaystyle \frac{1}{2}$

C. $x\leq 0$

D. $x>0$

Zadanie 9. $(1pkt)$

Dane są funkcje liniowe $f(x)=x-2$ oraz $g(x)=x+4$ określone dla wszystkich liczb

rzeczywistych $x$. Wskaz, który z ponizszych wykresów jest wykresem funkcji

$h(x)=f(x)\cdot g(x).$
\begin{center}
\includegraphics[width=29.868mm,height=49.380mm]{./F1_M_PP_M2011_page3_images/image001.eps}
\end{center}
{\it y}

{\it x}

$-4$  2
\begin{center}
\includegraphics[width=30.024mm,height=49.380mm]{./F1_M_PP_M2011_page3_images/image002.eps}
\end{center}
{\it y}

$-2$

{\it x}

4
\begin{center}
\includegraphics[width=29.868mm,height=49.380mm]{./F1_M_PP_M2011_page3_images/image003.eps}
\end{center}
{\it y}

{\it x}

$-4$  2
\begin{center}
\includegraphics[width=29.868mm,height=49.380mm]{./F1_M_PP_M2011_page3_images/image004.eps}
\end{center}
{\it y}

$-2$

{\it X}

4

A.

B.

C.

D.

Zadanie 10 $(1pkt)$

Funkcja liniowa określona jest wzorem $f(x)=-\sqrt{2}x+4$. Miejscem zerowym tej funkcjijest

liczba

A. $-2\sqrt{2}$

B.

-$\sqrt{}$22

C.

- -$\sqrt{}$22

D. $2\sqrt{2}$

Zadanie ll. $(1pkt)$

Danyjest nieskończony ciąg geometryczny $(a_{n})$, w którym $a_{3}=1 \displaystyle \mathrm{i}a_{4}=\frac{2}{3}$. Wtedy

A. {\it a}1$=- 23$ B. {\it a}1$=- 49$ C. {\it a}1$=$-23 D. {\it a}1$=$-49

Zadanie 12. $(1pkt)$

Danyjest nieskończony rosnący ciąg arytmetyczny $(a_{n})$ o wyrazach dodatnich. Wtedy

A. $a_{4}+a_{7}=a_{10}$

B. $a_{4}+a_{6}=a_{3}+a_{8}$

C. $a_{2}+a_{9}=a_{3}+a_{8}$

D. $a_{5}+a_{7}=2a_{8}$

Zadanie 13. $(1pkt)$

Kąt $\alpha$ jest ostry i $\displaystyle \cos\alpha=\frac{5}{13}$. Wtedy

A. $\displaystyle \sin\alpha=\frac{12}{13}$ oraz $\displaystyle \mathrm{t}\mathrm{g}\alpha=\frac{12}{5}$

C. $\displaystyle \sin\alpha=\frac{12}{5}$ oraz $\displaystyle \mathrm{t}\mathrm{g}\alpha=\frac{12}{13}$

B. $\displaystyle \sin\alpha=\frac{12}{13}$ oraz $\displaystyle \mathrm{t}\mathrm{g}\alpha=\frac{5}{12}$

D. $\displaystyle \sin\alpha=\frac{5}{12}$ oraz $\displaystyle \mathrm{t}\mathrm{g}\alpha=\frac{12}{13}$
\end{document}
