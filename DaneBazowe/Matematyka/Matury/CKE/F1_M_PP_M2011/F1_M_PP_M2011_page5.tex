\documentclass[a4paper,12pt]{article}
\usepackage{latexsym}
\usepackage{amsmath}
\usepackage{amssymb}
\usepackage{graphicx}
\usepackage{wrapfig}
\pagestyle{plain}
\usepackage{fancybox}
\usepackage{bm}

\begin{document}

{\it 6}

{\it Egzamin maturalny z matematyki}

{\it Poziom podstawowy}

Zadanie 14. $(1pkt)$

Wartość wyrazenia $\displaystyle \frac{\sin^{2}38^{\mathrm{o}}+\cos^{2}38^{\mathrm{o}}-1}{\sin^{2}52^{\mathrm{o}}+\cos^{2}52^{\mathrm{o}}+1}$ jest równa

A.

-21

B. 0

C.

- -21

D. l

Zadanie 15. $(1pkt)$

$\mathrm{W}$ prostopadłoŚcianie ABCDEFGH mamy: $|AB|=5, |AD|=4, |AE|=3$. Który z odcinków

{\it AB}, $BG, GE, EB$ jest najdłuzszy?

A.

{\it AB}

B.

{\it BG}

C.

{\it GE}

{\it D. EB}

Zadanie 16. $(1pkt)$

Punkt $O$ jest środkiem okręgu. Kąt wpisany $\alpha$ ma miarę
\begin{center}
\includegraphics[width=66.348mm,height=60.912mm]{./F1_M_PP_M2011_page5_images/image001.eps}
\end{center}
{\it B}

$\alpha$

{\it A}

$160^{\mathrm{o}}$  {\it C}

{\it O}

A. $80^{\mathrm{o}}$

B. $100^{\mathrm{o}}$

C. $110^{\mathrm{o}}$

D. $120^{\mathrm{o}}$

Zadanie 17. $(1pkt)$

Wysokość rombu o boku długości 6 i kącie ostrym $60^{\mathrm{o}}$ jest równa

A. $3\sqrt{3}$

B. 3

C. $6\sqrt{3}$

D. 6

Zadanie 18. $(1pkt)$

Prosta $k$ ma równanie $y=2x-3$. Wskaz równanie prostej $l$ równoległej do prostej $k$

i przechodzącej przez punkt $D$ o współrzędnych $(-2,1).$

A. $y=-2x+3$

B. $y=2x+1$

C. $y=2x+5$

D. $y=-x+1$
\end{document}
