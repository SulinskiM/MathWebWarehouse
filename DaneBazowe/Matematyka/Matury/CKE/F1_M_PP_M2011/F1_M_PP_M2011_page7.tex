\documentclass[a4paper,12pt]{article}
\usepackage{latexsym}
\usepackage{amsmath}
\usepackage{amssymb}
\usepackage{graphicx}
\usepackage{wrapfig}
\pagestyle{plain}
\usepackage{fancybox}
\usepackage{bm}

\begin{document}

{\it 8}

{\it Egzamin maturalny z matematyki}

{\it Poziom podstawowy}

Zadanie 19. $(1pkt)$

Styczną do okręgu $(x-1)^{2}+y^{2}-4=0$ jest prosta o równaniu

A. $x=1$

B. $x=3$

C. $y=0$

D. $y=4$

Zadanie 20. (1pkt)

Pole powierzchni całkowitej sześcianu jest równe 54. Długość przekątnej tego sześcianu jest

równa

A. $\sqrt{6}$

B. 3

C. 9

D. $3\sqrt{3}$

Zadanie 21. (1pkt)

Objętość stozka o wysokości 8 i średnicy podstawy 12jest równa

A. $ 124\pi$

B. $ 96\pi$

C. $ 64\pi$

D. $ 32\pi$

Zadanie 22. (1pkt)

Rzucamy dwa razy symetryczną sześcienną kostką do gry. Prawdopodobieństwo otrzymania

sumy oczek równej trzy wynosi

A.

-61

B.

-91

C.

$\displaystyle \frac{1}{12}$

D.

$\displaystyle \frac{1}{18}$

Zadanie 23. (1pkt)

Uczniowie pewnej klasy zostali poproszeni o odpowiedzí na pytanie:,,Ile osób liczy twoja

rodzina?'' Wyniki przedstawiono w tabeli:
\begin{center}
\begin{tabular}{|l|l|}
\hline
\multicolumn{1}{|l|}{$\begin{array}{l}\mbox{Liczba osób}	\\	\mbox{w rodzinie}	\end{array}$}&	\multicolumn{1}{|l|}{$\begin{array}{l}\mbox{liczba}	\\	\mbox{uczniów}	\end{array}$}	\\
\hline
\multicolumn{1}{|l|}{ $3$}&	\multicolumn{1}{|l|}{ $6$}	\\
\hline
\multicolumn{1}{|l|}{ $4$}&	\multicolumn{1}{|l|}{ $12$}	\\
\hline
\multicolumn{1}{|l|}{ $x$}&	\multicolumn{1}{|l|}{ $2$}	\\
\hline
\end{tabular}

\end{center}
Średnia liczba osób w rodzinie dla uczniów tej klasyjest równa 4. Wtedy 1iczba x jest równa

A. 3

B. 4

C. 5

D. 7
\end{document}
