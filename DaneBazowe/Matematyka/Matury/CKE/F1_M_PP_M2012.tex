\documentclass[a4paper,12pt]{article}
\usepackage{latexsym}
\usepackage{amsmath}
\usepackage{amssymb}
\usepackage{graphicx}
\usepackage{wrapfig}
\pagestyle{plain}
\usepackage{fancybox}
\usepackage{bm}

\begin{document}

$1-$

$-1\cup 1$

$-\mapsto 1$

$\mathrm{r}--$

Centralna Komisja Egzaminacyjna

Arkusz zawiera informacje prawnie chronione do momentu rozpoczęcia egzaminu.

WPISUJE ZDAJACY

KOD PESEL

{\it Miejsce}

{\it na naklejkę}

{\it z kodem}
\begin{center}
\includegraphics[width=21.432mm,height=9.804mm]{./F1_M_PP_M2012_page0_images/image001.eps}

\includegraphics[width=82.092mm,height=9.804mm]{./F1_M_PP_M2012_page0_images/image002.eps}
\end{center}
\fbox{} dysleksja
\begin{center}
\includegraphics[width=204.060mm,height=216.048mm]{./F1_M_PP_M2012_page0_images/image003.eps}
\end{center}
EGZAMIN MATU LNY

Z MATEMATYKI

MAJ 2012

POZIOM PODSTAWOWY

1. Sprawd $\acute{\mathrm{z}}$, czy arkusz egzaminacyjny zawiera 18 stron

(zadania $1-34$). Ewentualny brak zgłoś przewodniczącemu

zespo nadzorującego egzamin.

2. Rozwiązania zadań i odpowiedzi wpisuj w miejscu na to

przeznaczonym.

3. Odpowiedzi do zadań za niętych (l-25) przenieś

na ka ę odpowiedzi, zaznaczając je w części ka $\mathrm{y}$

przeznaczonej dla zdającego. Zamaluj $\blacksquare$ pola do tego

przeznaczone. Błędne zaznaczenie otocz kółkiem \fcircle$\bullet$

i zaznacz właściwe.

4. Pamiętaj, $\dot{\mathrm{z}}\mathrm{e}$ pominięcie argumentacji lub istotnych

obliczeń w rozwiązaniu zadania otwa ego (26-34) $\mathrm{m}\mathrm{o}\dot{\mathrm{z}}\mathrm{e}$

spowodować, $\dot{\mathrm{z}}\mathrm{e}$ za to rozwiązanie nie będziesz mógł

dostać pełnej liczby punktów.

5. Pisz czytelnie i uzywaj tvlko długopisu lub -Dióra

z czarnym tuszem lub atramentem.

6. Nie uzywaj korektora, a błędne zapisy wyrazínie prze eśl.

7. Pamiętaj, $\dot{\mathrm{z}}\mathrm{e}$ zapisy w brudnopisie nie będą oceniane.

8. $\mathrm{M}\mathrm{o}\dot{\mathrm{z}}$ esz korzystać z zestawu wzorów matematycznych,

cyrkla i linijki oraz kalkulatora.

9. Na tej stronie oraz na karcie odpowiedzi wpisz swój

numer PESEL i przyklej naklejkę z kodem.

10. Nie wpisuj $\dot{\mathrm{z}}$ adnych znaków w części przeznaczonej

dla egzaminatora.

Czas pracy:

170 minut

Liczba punktów

do uzyskania: 50

$\Vert\Vert\Vert\Vert\Vert\Vert\Vert\Vert\Vert\Vert\Vert\Vert\Vert\Vert\Vert\Vert\Vert\Vert\Vert\Vert\Vert\Vert\Vert\Vert|  \mathrm{M}\mathrm{M}\mathrm{A}-\mathrm{P}1_{-}1\mathrm{P}-122$




{\it 2}

{\it Egzamin maturalny z matematyki}

{\it Poziom podstawowy}

ZADANIA ZAMKNIĘTE

{\it Wzadaniach} $\theta d1.$ {\it do 25. wybierz i zaznacz na karcie odpowiedzipoprawnq odpowied} $\acute{z}.$

Zadanie l. (lpkt)

Cenę nart obnizono o 20\%, a po miesiącu nową cenę obnizono o da1sze 30\%. W wyniku obu

obnizek cena nart zmniejszyła się o

A. 44\%

B. 50\%

C. 56\%

D. 60\%

Zadanie 2. $(1pkt)$

3

Liczba $\sqrt[3]{(-8)^{-1}}\cdot 16^{\overline{4}}$ jest równa

A. $-8$

B. $-4$

C. 2

D. 4

Zadanie 3. $(1pkt)$

Liczba $(3-\sqrt{2})^{2}+4(2-\sqrt{2})$ jest równa

A. $19-10\sqrt{2}$

B. $17-4\sqrt{2}$

C. $15+14\sqrt{2}$

D. $19+6\sqrt{2}$

Zadanie 4. $(1pkt)$

Iloczyn 2$\cdot\log_{1}9$ jest równy

-3

A. $-6$ B. $-4$

C. $-1$

D. l

Zadanie 5. $(1pkt)$

Wska $\dot{\mathrm{z}}$ liczbę, która spełnia równanie $|3x+1|=4x.$

A. $x=-1$

B. $x=1$

C. $x=2$

D. $x=-2$

Zadanie 6. $(1pkt)$

Liczby $x_{1}, x_{2}$ sąróz$\cdot$nymi rozwiązaniami równania $2x^{2}+3x-7=0$. Suma $x_{1}+x_{2}$ jest równa

A.

- -27

B.

- -47

C.

- -23

D.

- -43

Zadanie 7. $(1pkt)$

Miejscami zerowymi ffinkcji kwadratowej $y=-3(x-7\mathrm{X}x+2)$ są

A. $x=7, x=-2$

B. $x=-7, x=-2$

C. $x=7, x=2$

D. $x=-7, x=2$

Zadanie 8. $(1pkt)$

Funkcja liniowafjest określona wzorem $f(x)=ax+6$, gdzie $a>0$. Wówczas spełniony jest

warunek

A. $f(1)>1$

B. $f(2)=2$

C. $f(3)<3$

D. $f(4)=4$





{\it Egzamin maturalny z matematyki}

{\it Poziom podstawowy}

{\it 11}

Zadanie 28. $(2pkt)$

Liczby $x_{1}=-4 \mathrm{i} x_{2}=3$ są pierwiastkami

trzeci pierwiastek tego wielomianu.

Odpowiedzí :

wielomianu $W(x)=x^{3}+4x^{2}-9x-36$. Oblicz

Zadanie 29. $(2pkt)$

Wyznacz równanie symetralnej odcinka o końcach $A=(-2,2)\mathrm{i}B=(2,10).$

Odpowiedzí :
\begin{center}
\includegraphics[width=123.900mm,height=17.832mm]{./F1_M_PP_M2012_page10_images/image001.eps}
\end{center}
Nr zadania

Wypelnia Maks. liczba kt

egzaminator

Uzyskana liczba pkt

2

27.

2

28.

2

2





{\it 12}

{\it Egzamin maturalny z matematyki}

{\it Poziom podstawowy}

Zadanie 30. $(2pkt)$

$\mathrm{W}$ trójkącie $ABC$ poprowadzono dwusieczne kątów A $\mathrm{i}B$. Dwusieczne te przecinają się

w punkcie $P$. Uzasadnij, $\dot{\mathrm{z}}\mathrm{e}$ kąt $APB$ jest rozwarty.





{\it Egzamin maturalny z matematyki}

{\it Poziom podstawowy}

{\it 13}

Zadanie 31. (2pkt)

Ze zbioru liczb \{1,2,3,4,5,6,7\} 1osujemy dwa razy po jednej 1iczbie ze zwracaniem. Ob1icz

prawdopodobieństwo zdarzenia A, polegającego na wylosowaniu liczb, których iloczyn jest

podzielny przez 6.

Odpowied $\acute{\mathrm{z}}$:
\begin{center}
\includegraphics[width=95.964mm,height=17.784mm]{./F1_M_PP_M2012_page12_images/image001.eps}
\end{center}
Wypelnia

egzaminator

Nr zadania

Maks. liczba kt

30.

2

31.

2

Uzyskana liczba pkt





{\it 14}

{\it Egzamin maturalny z matematyki}

{\it Poziom podstawowy}

Zadanie 32. (4pkt)

Ciąg (9, x,19) jest arytmetyczny, a ciąg (x,42,y,z) jest geometryczny. Ob1icz x, y oraz z.

Odpowiedzí:





{\it Egzamin maturalny z matematyki}

{\it Poziom podstawowy}

{\it 15}

Zadanie 33. $(4pkt)$

$\mathrm{W}$ graniastosłupie prawidłowym czworokątnym ABCDEFGH przekątna $AC$ podstawy

ma długość 4. Kąt ACE jest równy $60^{\mathrm{o}}$. Oblicz objętość ostrosłupa ABCDE przedstawionego

na ponizszym rysunku.

Odpowiedzí :
\begin{center}
\includegraphics[width=95.964mm,height=17.784mm]{./F1_M_PP_M2012_page14_images/image001.eps}
\end{center}
Wypelnia

egzaminator

Nr zadania

Maks. liczba kt

32.

4

33.

4

Uzyskana liczba pkt





{\it 16}

{\it Egzamin maturalny z matematyki}

{\it Poziom podstawowy}

Zadanie 34. $(5pkt)$

Miasto $A$ i miasto $B$ łączy linia kolejowa długości 210 km. Średnia prędkość pociągu

pospiesznego na tej trasie jest o 24 $\mathrm{k}\mathrm{m}/\mathrm{h}$ większa od średniej prędkości pociągu osobowego.

Pociąg pospieszny pokonuje tę trasę o l godzinę krócej $\mathrm{n}\mathrm{i}\dot{\mathrm{z}}$ pociąg osobowy. Oblicz czas

pokonania tej drogi przez pociąg pospieszny.





{\it Egzamin maturalny z matematyki}

{\it Poziom podstawowy}

{\it 1}7

Odpowied $\acute{\mathrm{z}}$:
\begin{center}
\includegraphics[width=82.044mm,height=17.832mm]{./F1_M_PP_M2012_page16_images/image001.eps}
\end{center}
Wypelnia

egzaminator

Nr zadania

Maks. liczba kt

34.

5

Uzyskana liczba pkt





{\it 18}

{\it Egzamin maturalny z matematyki}

{\it Poziom podstawowy}

BRUDNOPIS





{\it Egzamin maturalny z matematyki}

{\it Poziom podstawowy}

{\it 3}

BRUDNOPIS





{\it 4}

{\it Egzamin maturalny z matematyki}

{\it Poziom podstawowy}

Zadanie 9. $(1pkt)$

Wskaz wykres funkcji, która w przedziale $\langle-4,4\rangle$ ma dokładniejedno miejsce zerowe.

A.
\begin{center}
\includegraphics[width=57.408mm,height=58.368mm]{./F1_M_PP_M2012_page3_images/image001.eps}
\end{center}
4

3

y

2

1

$-4$ -$3  -2$

$-1$

$-1$

1 2

$\mathrm{x}$

$3\backslash ^{4}$

$-2$

$-3$

$-4$

C.
\begin{center}
\includegraphics[width=58.668mm,height=58.980mm]{./F1_M_PP_M2012_page3_images/image002.eps}
\end{center}
y

3

2

1

x

$-4$ -$3  -2  -1$  1 2 3 4

$-2$

$-3$

Zadanie 10. $(1pkt)$

Liczba tg $30^{\mathrm{o}}-\sin 30^{\mathrm{o}}$ jest równa

A. $\sqrt{3}-1$

B.

- -$\sqrt{}$63

B.
\begin{center}
\includegraphics[width=57.144mm,height=58.164mm]{./F1_M_PP_M2012_page3_images/image003.eps}
\end{center}
4  y

2

1

$-4$ -$3  -2$

$-1$

$-1$

1 2  3 4

$-2$

$-3$

$-4$

D.
\begin{center}
\includegraphics[width=56.088mm,height=57.300mm]{./F1_M_PP_M2012_page3_images/image004.eps}
\end{center}
4  y

2

1

$-4  -2$

$-1$

$-1$

1 3  4

$-2$

$-3$

$-4$

C.

$\displaystyle \frac{\sqrt{3}-1}{6}$

D.

$\displaystyle \frac{2\sqrt{3}-3}{6}$

Zadanie ll. $(1pkt)$

$\mathrm{W}$ trójkącie prostokątnym $ABC$ odcinek $AB$ jest przeciwprostokątną

$|BC|=12$. Wówczas sinus kąta ABCjest równy

i

$|AB|=13$

oraz

A.

-1123

B.

$\displaystyle \frac{5}{13}$

C.

$\displaystyle \frac{5}{12}$

D.

$\displaystyle \frac{13}{12}$

Zadanie 12. (1pkt)

W trójkącie równoramiennym ABC dane

Podstawa AB tego trójkąta ma długość

są

$|AC|=|BC|=5$

oraz wysokość

$|CD|=2.$

A. 6

B. $\mathrm{z}\sqrt{21}$

C. $\mathrm{z}\sqrt{29}$

D. 14





{\it Egzamin maturalny z matematyki}

{\it Poziom podstawowy}

{\it 5}

BRUDNOPIS





{\it 6}

{\it Egzamin maturalny z matematyki}

{\it Poziom podstawowy}

Zadanie 13. $(1pkt)$

$\mathrm{W}$ trójkącie prostokątnym dwa dłuzsze boki mają długości 5 $\mathrm{i}7$. Obwód tego trójkąta jest

równy

A. $16\sqrt{6}$ B. $14\sqrt{6}$ C. $12+4\sqrt{6}$ D. $12+2\sqrt{6}$

Zadanie 14. $(1pkt)$

Odcinki AB $\mathrm{i}$ CD są równoległe i $|AB|=5, |AC|=2, |CD|=7$ (zobacz rysunek). Długość

odcinka $AE$ jest równa

A.

$\displaystyle \frac{10}{7}$

B.

$\displaystyle \frac{14}{5}$
\begin{center}
\includegraphics[width=68.628mm,height=61.116mm]{./F1_M_PP_M2012_page5_images/image001.eps}
\end{center}
{\it D}

{\it B}

7

5

{\it E  A} 2  {\it C}

5

C. 3

D. 5

Zadanie 15. (1pkt)

Pole kwadratu wpisanego w okrąg o promieniu 5jest równe

A. 25

B. 50

C. 75

D. 100

Zadanie 16. $(1pkt)$

Punkty $A, B, C, D$ dzielą okrąg na 4 równe łuki. Miara zaznaczonego na rysunku kąta

wpisanego $ACD$ jest równa

A. $90^{\mathrm{o}}$

B. $60^{\mathrm{o}}$
\begin{center}
\includegraphics[width=50.388mm,height=50.388mm]{./F1_M_PP_M2012_page5_images/image002.eps}
\end{center}
{\it C}

$D$

{\it B}

{\it A}

D. $30^{\mathrm{o}}$

C. $45^{\mathrm{o}}$

Zadanie 17. (1pkt)

Miary kątów czworokąta tworzą ciąg arytmetyczny o róznicy

czworokąta ma miarę

$20^{\mathrm{o}}$ Najmniejszy kąt tego

A. $40^{\mathrm{o}}$

B. $50^{\mathrm{o}}$

C. $60^{\mathrm{o}}$

D. $70^{\mathrm{o}}$

Zadanie 18. $(1pkt)$

Dany jest ciąg $(a_{n})$ określony wzorem $a_{n}=(-1)^{n}\displaystyle \cdot\frac{2-n}{n^{2}}$ dla $n\geq 1$. Wówczas wyraz $a_{5}$ tego

ciągujest równy

A. - $\displaystyle \frac{3}{25}$ B. $\displaystyle \frac{3}{25}$ C. - $\displaystyle \frac{7}{25}$ D. $\displaystyle \frac{7}{25}$





{\it Egzamin maturalny z matematyki}

{\it Poziom podstawowy}

7

BRUDNOPIS





{\it 8}

{\it Egzamin maturalny z matematyki}

{\it Poziom podstawowy}

Zadanie 19. (1pkt)

Pole powierzchni jednej ściany sześcianujest równe 4. Objętość tego sześcianujest równa

A. 6

B. 8

C. 24

D. 64

Zadanie 20. $(1pkt)$

Tworząca stozka ma długość 4 i jest nachy1ona do płaszczyzny podstawy pod kątem $45^{\mathrm{o}}$

Wysokość tego stozkajest równa

A. $2\sqrt{2}$

B. $ 16\pi$

C. $4\sqrt{2}$

D. $ 8\pi$

Zadanie 21. $(1pkt)$

Wskaz równanie prostej równoległej do prostej o równaniu $3x-6y+7=0.$

A. {\it y}$=$-21{\it x} B. {\it y}$=$--21{\it x} C. {\it y}$=$2{\it x} D. {\it y}$=- 2x$

Zadanie 22. (1pkt)

Punkt A ma współrzędne (5,2012). Punkt B jest symetryczny do punktu A wzg1ędem osi Ox,

a punkt Cjest symetryczny do punktu B względem osi Oy. Punkt C ma współrzędne

A. $(-5,-2012)$

B. $(-2012,-5)$

C. $(-5$, 2012$)$

D. $(-2012,5)$

Zadanie 23. $(1pkt)$

Na okręgu o równaniu $(x-2)^{2}+(y+7)^{2}=4\mathrm{l}\mathrm{e}\dot{\mathrm{z}}\mathrm{y}$ punkt

A. $A=(-2,5)$

B. $B=(2,-5)$

C. $C=(2,-7)$

D. $D=(7,-2)$

Zadanie 24. (1pkt)

Flagę, takąjak pokazano na rysunku, nalezy zszyć

z trzech jednakowej szerokości pasów kolorowej

tkaniny. Oba pasy zewnętrzne mają być tego

samego koloru, a pas znajdujący się między nimi

ma być innego koloru.

Liczba róznych takich flag, które mozna uszyć,

mając do dyspozycji tkaniny w 10 ko1orach, jest

równa

A. 100

B. 99

C. 90

D. 19

Zadanie 25. (1pkt)

Średnia arytmetyczna cen sześciu akcji na giełdzie jest równa 500 zł. Za pięć z tych akcji

zapłacono 2300 zł. Cena szóstej akcjijest równa

A. 400 zł

B. 500 zł

C. 600 zł

D. 700 zł





{\it Egzamin maturalny z matematyki}

{\it Poziom podstawowy}

{\it 9}

BRUDNOPIS





$ 1\theta$

{\it Egzamin maturalny z matematyki}

{\it Poziom podstawowy}

ZADANIA OTWARTE

{\it Rozwiqzania zadań o numerach od 26. do 34. nalezy zapisać w} $wyznacz\theta nych$ {\it miejscach}

{\it pod treściq zadania}.

Zadanie 26. $(2pkt)$

Rozwiąz nierówność $x^{2}+8x+15>0.$

Odpowiedzí:

Zadanie 27. $(2pkt)$

Uzasadnij, $\dot{\mathrm{z}}\mathrm{e}$ jeśli liczby rzeczywiste $a,$

$\displaystyle \frac{a+b+c}{3}>\frac{a+b}{2}.$

$b, c$ spełniają nierówności $0<a<b<c$, to



\end{document}