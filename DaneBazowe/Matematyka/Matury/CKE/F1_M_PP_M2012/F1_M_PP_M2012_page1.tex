\documentclass[a4paper,12pt]{article}
\usepackage{latexsym}
\usepackage{amsmath}
\usepackage{amssymb}
\usepackage{graphicx}
\usepackage{wrapfig}
\pagestyle{plain}
\usepackage{fancybox}
\usepackage{bm}

\begin{document}

{\it 2}

{\it Egzamin maturalny z matematyki}

{\it Poziom podstawowy}

ZADANIA ZAMKNIĘTE

{\it Wzadaniach} $\theta d1.$ {\it do 25. wybierz i zaznacz na karcie odpowiedzipoprawnq odpowied} $\acute{z}.$

Zadanie l. (lpkt)

Cenę nart obnizono o 20\%, a po miesiącu nową cenę obnizono o da1sze 30\%. W wyniku obu

obnizek cena nart zmniejszyła się o

A. 44\%

B. 50\%

C. 56\%

D. 60\%

Zadanie 2. $(1pkt)$

3

Liczba $\sqrt[3]{(-8)^{-1}}\cdot 16^{\overline{4}}$ jest równa

A. $-8$

B. $-4$

C. 2

D. 4

Zadanie 3. $(1pkt)$

Liczba $(3-\sqrt{2})^{2}+4(2-\sqrt{2})$ jest równa

A. $19-10\sqrt{2}$

B. $17-4\sqrt{2}$

C. $15+14\sqrt{2}$

D. $19+6\sqrt{2}$

Zadanie 4. $(1pkt)$

Iloczyn 2$\cdot\log_{1}9$ jest równy

-3

A. $-6$ B. $-4$

C. $-1$

D. l

Zadanie 5. $(1pkt)$

Wska $\dot{\mathrm{z}}$ liczbę, która spełnia równanie $|3x+1|=4x.$

A. $x=-1$

B. $x=1$

C. $x=2$

D. $x=-2$

Zadanie 6. $(1pkt)$

Liczby $x_{1}, x_{2}$ sąróz$\cdot$nymi rozwiązaniami równania $2x^{2}+3x-7=0$. Suma $x_{1}+x_{2}$ jest równa

A.

- -27

B.

- -47

C.

- -23

D.

- -43

Zadanie 7. $(1pkt)$

Miejscami zerowymi ffinkcji kwadratowej $y=-3(x-7\mathrm{X}x+2)$ są

A. $x=7, x=-2$

B. $x=-7, x=-2$

C. $x=7, x=2$

D. $x=-7, x=2$

Zadanie 8. $(1pkt)$

Funkcja liniowafjest określona wzorem $f(x)=ax+6$, gdzie $a>0$. Wówczas spełniony jest

warunek

A. $f(1)>1$

B. $f(2)=2$

C. $f(3)<3$

D. $f(4)=4$
\end{document}
