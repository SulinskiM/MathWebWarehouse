\documentclass[a4paper,12pt]{article}
\usepackage{latexsym}
\usepackage{amsmath}
\usepackage{amssymb}
\usepackage{graphicx}
\usepackage{wrapfig}
\pagestyle{plain}
\usepackage{fancybox}
\usepackage{bm}

\begin{document}

{\it Egzamin maturalny z matematyki}

{\it Poziom podstawowy}

{\it 11}

Zadanie 28. $(2pkt)$

Liczby $x_{1}=-4 \mathrm{i} x_{2}=3$ są pierwiastkami

trzeci pierwiastek tego wielomianu.

Odpowiedzí :

wielomianu $W(x)=x^{3}+4x^{2}-9x-36$. Oblicz

Zadanie 29. $(2pkt)$

Wyznacz równanie symetralnej odcinka o końcach $A=(-2,2)\mathrm{i}B=(2,10).$

Odpowiedzí :
\begin{center}
\includegraphics[width=123.900mm,height=17.832mm]{./F1_M_PP_M2012_page10_images/image001.eps}
\end{center}
Nr zadania

Wypelnia Maks. liczba kt

egzaminator

Uzyskana liczba pkt

2

27.

2

28.

2

2
\end{document}
