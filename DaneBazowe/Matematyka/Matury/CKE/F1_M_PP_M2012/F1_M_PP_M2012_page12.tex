\documentclass[a4paper,12pt]{article}
\usepackage{latexsym}
\usepackage{amsmath}
\usepackage{amssymb}
\usepackage{graphicx}
\usepackage{wrapfig}
\pagestyle{plain}
\usepackage{fancybox}
\usepackage{bm}

\begin{document}

{\it Egzamin maturalny z matematyki}

{\it Poziom podstawowy}

{\it 13}

Zadanie 31. (2pkt)

Ze zbioru liczb \{1,2,3,4,5,6,7\} 1osujemy dwa razy po jednej 1iczbie ze zwracaniem. Ob1icz

prawdopodobieństwo zdarzenia A, polegającego na wylosowaniu liczb, których iloczyn jest

podzielny przez 6.

Odpowied $\acute{\mathrm{z}}$:
\begin{center}
\includegraphics[width=95.964mm,height=17.784mm]{./F1_M_PP_M2012_page12_images/image001.eps}
\end{center}
Wypelnia

egzaminator

Nr zadania

Maks. liczba kt

30.

2

31.

2

Uzyskana liczba pkt
\end{document}
