\documentclass[a4paper,12pt]{article}
\usepackage{latexsym}
\usepackage{amsmath}
\usepackage{amssymb}
\usepackage{graphicx}
\usepackage{wrapfig}
\pagestyle{plain}
\usepackage{fancybox}
\usepackage{bm}

\begin{document}

{\it 16}

{\it Egzamin maturalny z matematyki}

{\it Poziom podstawowy}

Zadanie 34. $(5pkt)$

Miasto $A$ i miasto $B$ łączy linia kolejowa długości 210 km. Średnia prędkość pociągu

pospiesznego na tej trasie jest o 24 $\mathrm{k}\mathrm{m}/\mathrm{h}$ większa od średniej prędkości pociągu osobowego.

Pociąg pospieszny pokonuje tę trasę o l godzinę krócej $\mathrm{n}\mathrm{i}\dot{\mathrm{z}}$ pociąg osobowy. Oblicz czas

pokonania tej drogi przez pociąg pospieszny.
\end{document}
