\documentclass[a4paper,12pt]{article}
\usepackage{latexsym}
\usepackage{amsmath}
\usepackage{amssymb}
\usepackage{graphicx}
\usepackage{wrapfig}
\pagestyle{plain}
\usepackage{fancybox}
\usepackage{bm}

\begin{document}

{\it 4}

{\it Egzamin maturalny z matematyki}

{\it Poziom podstawowy}

Zadanie 9. $(1pkt)$

Wskaz wykres funkcji, która w przedziale $\langle-4,4\rangle$ ma dokładniejedno miejsce zerowe.

A.
\begin{center}
\includegraphics[width=57.408mm,height=58.368mm]{./F1_M_PP_M2012_page3_images/image001.eps}
\end{center}
4

3

y

2

1

$-4$ -$3  -2$

$-1$

$-1$

1 2

$\mathrm{x}$

$3\backslash ^{4}$

$-2$

$-3$

$-4$

C.
\begin{center}
\includegraphics[width=58.668mm,height=58.980mm]{./F1_M_PP_M2012_page3_images/image002.eps}
\end{center}
y

3

2

1

x

$-4$ -$3  -2  -1$  1 2 3 4

$-2$

$-3$

Zadanie 10. $(1pkt)$

Liczba tg $30^{\mathrm{o}}-\sin 30^{\mathrm{o}}$ jest równa

A. $\sqrt{3}-1$

B.

- -$\sqrt{}$63

B.
\begin{center}
\includegraphics[width=57.144mm,height=58.164mm]{./F1_M_PP_M2012_page3_images/image003.eps}
\end{center}
4  y

2

1

$-4$ -$3  -2$

$-1$

$-1$

1 2  3 4

$-2$

$-3$

$-4$

D.
\begin{center}
\includegraphics[width=56.088mm,height=57.300mm]{./F1_M_PP_M2012_page3_images/image004.eps}
\end{center}
4  y

2

1

$-4  -2$

$-1$

$-1$

1 3  4

$-2$

$-3$

$-4$

C.

$\displaystyle \frac{\sqrt{3}-1}{6}$

D.

$\displaystyle \frac{2\sqrt{3}-3}{6}$

Zadanie ll. $(1pkt)$

$\mathrm{W}$ trójkącie prostokątnym $ABC$ odcinek $AB$ jest przeciwprostokątną

$|BC|=12$. Wówczas sinus kąta ABCjest równy

i

$|AB|=13$

oraz

A.

-1123

B.

$\displaystyle \frac{5}{13}$

C.

$\displaystyle \frac{5}{12}$

D.

$\displaystyle \frac{13}{12}$

Zadanie 12. (1pkt)

W trójkącie równoramiennym ABC dane

Podstawa AB tego trójkąta ma długość

są

$|AC|=|BC|=5$

oraz wysokość

$|CD|=2.$

A. 6

B. $\mathrm{z}\sqrt{21}$

C. $\mathrm{z}\sqrt{29}$

D. 14
\end{document}
