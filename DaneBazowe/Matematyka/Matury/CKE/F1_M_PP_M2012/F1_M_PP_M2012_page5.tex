\documentclass[a4paper,12pt]{article}
\usepackage{latexsym}
\usepackage{amsmath}
\usepackage{amssymb}
\usepackage{graphicx}
\usepackage{wrapfig}
\pagestyle{plain}
\usepackage{fancybox}
\usepackage{bm}

\begin{document}

{\it 6}

{\it Egzamin maturalny z matematyki}

{\it Poziom podstawowy}

Zadanie 13. $(1pkt)$

$\mathrm{W}$ trójkącie prostokątnym dwa dłuzsze boki mają długości 5 $\mathrm{i}7$. Obwód tego trójkąta jest

równy

A. $16\sqrt{6}$ B. $14\sqrt{6}$ C. $12+4\sqrt{6}$ D. $12+2\sqrt{6}$

Zadanie 14. $(1pkt)$

Odcinki AB $\mathrm{i}$ CD są równoległe i $|AB|=5, |AC|=2, |CD|=7$ (zobacz rysunek). Długość

odcinka $AE$ jest równa

A.

$\displaystyle \frac{10}{7}$

B.

$\displaystyle \frac{14}{5}$
\begin{center}
\includegraphics[width=68.628mm,height=61.116mm]{./F1_M_PP_M2012_page5_images/image001.eps}
\end{center}
{\it D}

{\it B}

7

5

{\it E  A} 2  {\it C}

5

C. 3

D. 5

Zadanie 15. (1pkt)

Pole kwadratu wpisanego w okrąg o promieniu 5jest równe

A. 25

B. 50

C. 75

D. 100

Zadanie 16. $(1pkt)$

Punkty $A, B, C, D$ dzielą okrąg na 4 równe łuki. Miara zaznaczonego na rysunku kąta

wpisanego $ACD$ jest równa

A. $90^{\mathrm{o}}$

B. $60^{\mathrm{o}}$
\begin{center}
\includegraphics[width=50.388mm,height=50.388mm]{./F1_M_PP_M2012_page5_images/image002.eps}
\end{center}
{\it C}

$D$

{\it B}

{\it A}

D. $30^{\mathrm{o}}$

C. $45^{\mathrm{o}}$

Zadanie 17. (1pkt)

Miary kątów czworokąta tworzą ciąg arytmetyczny o róznicy

czworokąta ma miarę

$20^{\mathrm{o}}$ Najmniejszy kąt tego

A. $40^{\mathrm{o}}$

B. $50^{\mathrm{o}}$

C. $60^{\mathrm{o}}$

D. $70^{\mathrm{o}}$

Zadanie 18. $(1pkt)$

Dany jest ciąg $(a_{n})$ określony wzorem $a_{n}=(-1)^{n}\displaystyle \cdot\frac{2-n}{n^{2}}$ dla $n\geq 1$. Wówczas wyraz $a_{5}$ tego

ciągujest równy

A. - $\displaystyle \frac{3}{25}$ B. $\displaystyle \frac{3}{25}$ C. - $\displaystyle \frac{7}{25}$ D. $\displaystyle \frac{7}{25}$
\end{document}
