\documentclass[a4paper,12pt]{article}
\usepackage{latexsym}
\usepackage{amsmath}
\usepackage{amssymb}
\usepackage{graphicx}
\usepackage{wrapfig}
\pagestyle{plain}
\usepackage{fancybox}
\usepackage{bm}

\begin{document}

{\it 8}

{\it Egzamin maturalny z matematyki}

{\it Poziom podstawowy}

Zadanie 19. (1pkt)

Pole powierzchni jednej ściany sześcianujest równe 4. Objętość tego sześcianujest równa

A. 6

B. 8

C. 24

D. 64

Zadanie 20. $(1pkt)$

Tworząca stozka ma długość 4 i jest nachy1ona do płaszczyzny podstawy pod kątem $45^{\mathrm{o}}$

Wysokość tego stozkajest równa

A. $2\sqrt{2}$

B. $ 16\pi$

C. $4\sqrt{2}$

D. $ 8\pi$

Zadanie 21. $(1pkt)$

Wskaz równanie prostej równoległej do prostej o równaniu $3x-6y+7=0.$

A. {\it y}$=$-21{\it x} B. {\it y}$=$--21{\it x} C. {\it y}$=$2{\it x} D. {\it y}$=- 2x$

Zadanie 22. (1pkt)

Punkt A ma współrzędne (5,2012). Punkt B jest symetryczny do punktu A wzg1ędem osi Ox,

a punkt Cjest symetryczny do punktu B względem osi Oy. Punkt C ma współrzędne

A. $(-5,-2012)$

B. $(-2012,-5)$

C. $(-5$, 2012$)$

D. $(-2012,5)$

Zadanie 23. $(1pkt)$

Na okręgu o równaniu $(x-2)^{2}+(y+7)^{2}=4\mathrm{l}\mathrm{e}\dot{\mathrm{z}}\mathrm{y}$ punkt

A. $A=(-2,5)$

B. $B=(2,-5)$

C. $C=(2,-7)$

D. $D=(7,-2)$

Zadanie 24. (1pkt)

Flagę, takąjak pokazano na rysunku, nalezy zszyć

z trzech jednakowej szerokości pasów kolorowej

tkaniny. Oba pasy zewnętrzne mają być tego

samego koloru, a pas znajdujący się między nimi

ma być innego koloru.

Liczba róznych takich flag, które mozna uszyć,

mając do dyspozycji tkaniny w 10 ko1orach, jest

równa

A. 100

B. 99

C. 90

D. 19

Zadanie 25. (1pkt)

Średnia arytmetyczna cen sześciu akcji na giełdzie jest równa 500 zł. Za pięć z tych akcji

zapłacono 2300 zł. Cena szóstej akcjijest równa

A. 400 zł

B. 500 zł

C. 600 zł

D. 700 zł
\end{document}
