\documentclass[a4paper,12pt]{article}
\usepackage{latexsym}
\usepackage{amsmath}
\usepackage{amssymb}
\usepackage{graphicx}
\usepackage{wrapfig}
\pagestyle{plain}
\usepackage{fancybox}
\usepackage{bm}

\begin{document}

$ 1\theta$

{\it Egzamin maturalny z matematyki}

{\it Poziom podstawowy}

ZADANIA OTWARTE

{\it Rozwiqzania zadań o numerach od 26. do 34. nalezy zapisać w} $wyznacz\theta nych$ {\it miejscach}

{\it pod treściq zadania}.

Zadanie 26. $(2pkt)$

Rozwiąz nierówność $x^{2}+8x+15>0.$

Odpowiedzí:

Zadanie 27. $(2pkt)$

Uzasadnij, $\dot{\mathrm{z}}\mathrm{e}$ jeśli liczby rzeczywiste $a,$

$\displaystyle \frac{a+b+c}{3}>\frac{a+b}{2}.$

$b, c$ spełniają nierówności $0<a<b<c$, to
\end{document}
