\documentclass[a4paper,12pt]{article}
\usepackage{latexsym}
\usepackage{amsmath}
\usepackage{amssymb}
\usepackage{graphicx}
\usepackage{wrapfig}
\pagestyle{plain}
\usepackage{fancybox}
\usepackage{bm}

\begin{document}

Centralna Komisja Egzaminacyjna

Arkusz zawiera informacje prawnie chronione do momentu rozpoczęcia egzaminu.

WPISUJE ZDAJACY

KOD PESEL

{\it Miejsce}

{\it na naklejkę}

{\it z kodem}
\begin{center}
\includegraphics[width=21.432mm,height=9.804mm]{./F1_M_PP_M2013_page0_images/image001.eps}

\includegraphics[width=82.092mm,height=9.804mm]{./F1_M_PP_M2013_page0_images/image002.eps}
\end{center}
\fbox{} dysleksja
\begin{center}
\includegraphics[width=204.060mm,height=216.048mm]{./F1_M_PP_M2013_page0_images/image003.eps}
\end{center}
EGZAMIN MATU LNY

Z MATEMATYKI

MAJ 2013

POZIOM PODSTAWOWY

1.

2.

3.

Sprawd $\acute{\mathrm{z}}$, czy ar sz egzaminacyjny zawiera 22 strony

(zadania $1-34$). Ewentualny brak zgłoś przewodniczącemu

zespo nadzorującego egzamin.

Rozwiązania zadań i odpowiedzi wpisuj w miejscu na to

przeznaczonym.

Odpowiedzi do zadań za iętych (1-25) przenieś

na ka ę odpowiedzi, zaznaczając je w części ka $\mathrm{y}$

przeznaczonej dla zdającego. Zamaluj $\blacksquare$ pola do tego

przeznaczone. Błędne zaznaczenie otocz kółkiem

i zaznacz właściwe.

4. Pamiętaj, $\dot{\mathrm{z}}\mathrm{e}$ pominięcie argumentacji lub istotnych

obliczeń w rozwiązaniu zadania otwa ego (26-34) $\mathrm{m}\mathrm{o}\dot{\mathrm{z}}\mathrm{e}$

spowodować, $\dot{\mathrm{z}}\mathrm{e}$ za to rozwiązanie nie będziesz mógł

dostać pełnej liczby punktów.

5. Pisz czytelnie i uzywaj tvlko długopisu lub -Dióra

z czamym tuszem lub atramentem.

6. Nie uzywaj korektora, a błędne zapisy wyra $\acute{\mathrm{z}}\mathrm{n}\mathrm{i}\mathrm{e}$ przekreśl.

7. Pamiętaj, $\dot{\mathrm{z}}\mathrm{e}$ zapisy w brudnopisie nie będą oceniane.

8. $\mathrm{M}\mathrm{o}\dot{\mathrm{z}}$ esz korzystać z zestawu wzorów matematycznych,

cyrkla i linijki oraz kalkulatora.

9. Na tej stronie oraz na karcie odpowiedzi wpisz swój

numer PESEL i przyklej naklejkę z kodem.

10. Nie wpisuj $\dot{\mathrm{z}}$ adnych znaków w części przeznaczonej

dla egzaminatora.

Czas pracy:

170 minut

Liczba punktów

do uzyskania: 50

$\Vert\Vert\Vert\Vert\Vert\Vert\Vert\Vert\Vert\Vert\Vert\Vert\Vert\Vert\Vert\Vert\Vert\Vert\Vert\Vert\Vert\Vert\Vert\Vert|  \mathrm{M}\mathrm{M}\mathrm{A}-\mathrm{P}1_{-}1\mathrm{P}-132$




{\it 2}

{\it Egzamin maturalny z matematyki}

{\it Poziom podstawowy}

ZADANIA ZAMKNIĘTE

{\it Wzadaniach l-25 wybierz i zaznacz na karcie odpowiedzipoprawnq odpowiedzí}.

Zadanie l. $(1pkt)$

Wskaz rysunek, na którym zaznaczony

spełniających nierówność $|x+4|<5.$

jest zbiór

wszystkich liczb rzeczywistych
\begin{center}
\includegraphics[width=165.552mm,height=12.240mm]{./F1_M_PP_M2013_page1_images/image001.eps}
\end{center}
A.
\begin{center}
\includegraphics[width=165.612mm,height=17.832mm]{./F1_M_PP_M2013_page1_images/image002.eps}
\end{center}
$-9  -4$  1  {\it X}

B.
\begin{center}
\includegraphics[width=165.552mm,height=18.036mm]{./F1_M_PP_M2013_page1_images/image003.eps}
\end{center}
$-1$  4 9  {\it X}

C.
\begin{center}
\includegraphics[width=165.552mm,height=17.784mm]{./F1_M_PP_M2013_page1_images/image004.eps}
\end{center}
$-9  -5  -1$  {\it X}

1 5  9  {\it X}

D.

Zadanie 2. $(1pkt)$

Liczby $a\mathrm{i}b$ są dodatnie oraz 12\% 1iczby $a$ jest równe 15\% 1iczby $b$. Stąd wynika, $\dot{\mathrm{z}}\mathrm{e}a$ jest

równe

A. 103\% 1iczby $b$ B. 125\% 1iczby $b$ C. 150\% 1iczby $b$ D. 153\% 1iczby $b$

Zadanie 3. $(1pkt)$

Liczba $\log 100-\log_{2}8$ jest równa

A. $-2$

B. $-1$

C. 0

D. l

Zadanie 4. $(1pkt)$

Rozwiązaniem układu równań 

A. $x=-3 \mathrm{i}y=4$

B. $x=-3 \mathrm{i}y=6$

C. $x=3 \mathrm{i}y=-4$

D. $x=9 \mathrm{i}y=4$

Zadanie 5. $(1pkt)$

Punkt $A=(0,1)$ lezy na wykresie ffinkcji liniowej $f(x)=(m-2)x+m-3$. Stąd wynika, $\dot{\mathrm{z}}\mathrm{e}$

A. $m=1$

B. $m=2$

C. $m=3$

D. $m=4$

Zadanie 6. $(1pkt)$

Wierzchołkiem paraboli o równaniu $y=-3(x-2)^{2}+4$ jest punkt o współrzędnych

A. $(-2,-4)$

B. $(-2,4)$

C. $(2,-4)$

D. (2, 4)

Zadanie 7. $(1pkt)$

Dla $\mathrm{k}\mathrm{a}\dot{\mathrm{z}}$ dej liczby rzeczywistej $x$, wyrazenie $4x^{2}-12x+9$ jest równe

A. $(4x+3)(x+3)$

B. $(2x-3)(2x+3)$

C. $(2x-3)(2x-3)$

D. $(x-3)(4x-3)$





{\it Egzamin maturalny z matematyki}

{\it Poziom podstawowy}

{\it 11}

Zadanie 27. $(2pkt)$

Kąt $\alpha$ jest ostry i $\displaystyle \sin\alpha=\frac{\sqrt{3}}{2}$. Oblicz wartość wyrazenia $\sin^{2}\alpha-3\cos^{2}\alpha.$

Odpowied $\acute{\mathrm{z}}$:
\begin{center}
\includegraphics[width=95.964mm,height=17.832mm]{./F1_M_PP_M2013_page10_images/image001.eps}
\end{center}
Wypelnia

egzaminator

2

27.

2

Uzyskana liczba pkt





{\it 12}

{\it Egzamin maturalny z matematyki}

{\it Poziom podstawowy}

Zadanie 28. $(2pkt)$

Udowodnij, $\dot{\mathrm{z}}\mathrm{e}$ dla dowolnych liczb rzeczywistych $x, y, z$ takich, $\dot{\mathrm{z}}\mathrm{e}x+y+z=0$, prawdziwa

jest nierówność $xy+yz+zx\leq 0.$

$\mathrm{M}\mathrm{o}\dot{\mathrm{z}}$ esz skorzystać z $\mathrm{t}\mathrm{o}\dot{\mathrm{z}}$ samości $(x+y+z)^{2}=x^{2}+y^{2}+z^{2}+2xy+2xz+2yz.$





{\it Egzamin maturalny z matematyki}

{\it Poziom podstawowy}

{\it 13}

Zadanie 29. $(2pkt)$

Na rysunku przedstawiony jest wykres funkcji $f(x)$ określonej dla $x\in\langle-7,8\rangle.$
\begin{center}
\includegraphics[width=162.564mm,height=98.292mm]{./F1_M_PP_M2013_page12_images/image001.eps}
\end{center}
Odczytaj z wykresu i zapisz:

a) największą wartość funkcji f,

b) zbiór rozwiązań nierówności $f(x)<0.$
\begin{center}
\includegraphics[width=96.012mm,height=17.832mm]{./F1_M_PP_M2013_page12_images/image002.eps}
\end{center}
Wypelnia

egzaminator

Nr zadania

Maks. liczba kt

28.

2

2

Uzyskana liczba pkt





{\it 14}

{\it Egzamin maturalny z matematyki}

{\it Poziom podstawowy}

Zadanie 30. $(2pkt)$

Rozwiąz nierówność $2x^{2}-7x+5\geq 0.$

Odpowiedzí:





{\it Egzamin maturalny z matematyki}

{\it Poziom podstawowy}

{\it 15}

Zadanie 31. $(2pkt)$

Wykaz, $\dot{\mathrm{z}}\mathrm{e}$ liczba $6^{100}-2\cdot 6^{99}+10\cdot 6^{98}$ jest podzielna przez 17.
\begin{center}
\includegraphics[width=95.964mm,height=17.784mm]{./F1_M_PP_M2013_page14_images/image001.eps}
\end{center}
Wypelnia

egzaminator

Nr zadania

Maks. liczba kt

30.

2

31.

2

Uzyskana liczba pkt





{\it 16}

{\it Egzamin maturalny z matematyki}

{\it Poziom podstawowy}

Zadanie 32. (4pkt)

Punkt S jest środkiem okręgu opisanego na trójkącie ostrokątnym ABC. Kąt ACS jest trzy razy

większy od kąta BAS, a kąt CBSjest dwa razy większy od kąta BAS. Oblicz kąty trójkąta ABC.
\begin{center}
\includegraphics[width=93.828mm,height=90.168mm]{./F1_M_PP_M2013_page15_images/image001.eps}
\end{center}
{\it C}

{\it S}

{\it A  B}





{\it Egzamin maturalny z matematyki}

{\it Poziom podstawowy}

17

Odpowied $\acute{\mathrm{z}}$:
\begin{center}
\includegraphics[width=82.044mm,height=17.832mm]{./F1_M_PP_M2013_page16_images/image001.eps}
\end{center}
Wypelnia

egzaminator

Nr zadania

Maks. liczba kt

32.

4

Uzyskana liczba pkt





{\it 18}

{\it Egzamin maturalny z matematyki}

{\it Poziom podstawowy}

Zadanie 33. $(4pkt)$

Pole podstawy ostrosłupa prawidłowego czworokątnego jest równe 100

pole powierzchni bocznej jest równe 260 $\mathrm{c}\mathrm{m}^{2}$. Oblicz objętość tego ostrosłupa.

$\mathrm{c}\mathrm{m}^{2}$, a jego





{\it Egzamin maturalny z matematyki}

{\it Poziom podstawowy}

{\it 19}

Odpowiedzí :
\begin{center}
\includegraphics[width=82.044mm,height=17.832mm]{./F1_M_PP_M2013_page18_images/image001.eps}
\end{center}
Wypelnia

egzaminator

Nr zadania

Maks. liczba kt

33.

4

Uzyskana liczba pkt





$ 2\theta$

{\it Egzamin maturalny z matematyki}

{\it Poziom podstawowy}

Zadanie 34. $(5pkt)$

Dwa miasta łączy linia kolejowa o długości 336 ki1ometrów. Pierwszy pociąg przebył tę trasę

w czasie o 40 minut krótszym $\mathrm{n}\mathrm{i}\dot{\mathrm{z}}$ drugi pociąg. Średnia prędkość pierwszego pociągu na tej

trasie była o 9 $\mathrm{k}\mathrm{n}\vee \mathrm{h}$ większa od średniej prędkości drugiego pociągu. Oblicz średnią

prędkość $\mathrm{k}\mathrm{a}\dot{\mathrm{z}}$ dego z tych pociągów na tej trasie.





{\it Egzamin maturalny z matematyki}

{\it Poziom podstawowy}

{\it 3}

BRUDNOPIS





{\it Egzamin maturalny z matematyki}

{\it Poziom podstawowy}

{\it 21}

Odpowiedzí :
\begin{center}
\includegraphics[width=82.044mm,height=17.832mm]{./F1_M_PP_M2013_page20_images/image001.eps}
\end{center}
Wypelnia

egzaminator

Nr zadania

Maks. liczba kt

34.

5

Uzyskana liczba pkt





{\it 22}

{\it Egzamin maturalny z matematyki}

{\it Poziom podstawowy}

BRUDNOPIS





{\it 4}

{\it Egzamin maturalny z matematyki}

{\it Poziom podstawowy}

Zadanie 8. $(1pkt)$

Prosta o równaniu $y=\displaystyle \frac{2}{m}x+1$ jest prostopadła do prostej o równaniu $y=-\displaystyle \frac{3}{2}x-1$. Stąd

wynika, $\dot{\mathrm{z}}\mathrm{e}$

A. $m=-3$

B.

{\it m}$=$ -23

C.

{\it m}$=$ -23

D. $m=3$

Zadanie 9. $(1pkt)$

Na rysunku przedstawiony jest fragment wykresu pewnej funkcji liniowej $y=ax+b.$
\begin{center}
\includegraphics[width=66.036mm,height=50.748mm]{./F1_M_PP_M2013_page3_images/image001.eps}
\end{center}
$y$

0  {\it x}

Jakie znaki mają współczynniki a ib?

A. $a<0 \mathrm{i}b<0$

B. $a<0 \mathrm{i}b>0$

C. $a>0 \mathrm{i}b<0$

D. $a>0\mathrm{i}b>0$

Zadanie 10. (1pkt)

Najmniejszą liczbą całkowitą spełniającą nierówność $\displaystyle \frac{x}{2}\leq\frac{2x}{3}+\frac{1}{4}$ jest

A. $-2$

B. $-1$

C. 0

D. l

Zadanie ll. $(1pkt)$

Na rysunku l przedstawiony jest wykres funkcji $y=f(x)$ określonej dla $x\in\langle-7,4\rangle.$
\begin{center}
\includegraphics[width=184.500mm,height=59.280mm]{./F1_M_PP_M2013_page3_images/image002.eps}
\end{center}
Rysunek 2 przedstawia wykres ffinkcji

A. $y=f(x+2)$ B. $y=f(x)-2$

C. $y=f(x-2)$

D. $y=f(x)+2$

Zadanie 12. $(1pkt)$

Ciąg $($27, 18, $x+5)$ jest geometryczny. Wtedy

A. $x=4$

B. $x=5$

C. $x=7$

D. $x=9$





{\it Egzamin maturalny z matematyki}

{\it Poziom podstawowy}

{\it 5}

BRUDNOPIS





{\it 6}

{\it Egzamin maturalny z matematyki}

{\it Poziom podstawowy}

Zadanie 13. $(1pkt)$

Ciąg $(a_{n})$ określony dla $n\geq 1$ jest arytmetyczny oraz $a_{3}=10 \mathrm{i}a_{4}=14$. Pierwszy wyraz tego

ciągu jest równy

A. $a_{1}=-2$ B. $a_{1}=2$ C. $a_{1}=6$ D. $a_{1}=12$

Zadanie 14. $(1pkt)$

Kąt $\alpha$ jest ostry i $\displaystyle \sin\alpha=\frac{\sqrt{3}}{2}$. Wartość wyrazenia $\cos^{2}\alpha-2$ jest równa

A.

- -47

B.

- -41

C.

-21

D.

-$\sqrt{}$23

Zadanie 15. $(1pkt)$

Średnice AB $\mathrm{i}$ CD okręgu o środku $S$ przecinają się pod kątem $50^{\mathrm{o}}$ (takjak na rysunku).
\begin{center}
\includegraphics[width=65.124mm,height=65.628mm]{./F1_M_PP_M2013_page5_images/image001.eps}
\end{center}
{\it B}

{\it D}

$\alpha$

{\it S  M}

$50^{\mathrm{o}}$

{\it C}

{\it A}

Miara kąta $\alpha$ jest równa

A. $25^{\mathrm{o}}$

B. $30^{\mathrm{o}}$

C. $40^{\mathrm{o}}$

D. $50^{\mathrm{o}}$

Zadanie 16. $(1pkt)$

Liczba rzeczywistych rozwiązań równania $(x+1)(x+2)(x^{2}+3)=0$ jest równa

A. 0

B. l

C. 2

D. 4

Zadanie 17. $(1pkt)$

Punkty $A=(-1,2) \mathrm{i}B=(5,-2)$ są dwoma sąsiednimi wierzchołkami rombu ABCD. Obwód

tego rombujest równy

A. $\sqrt{13}$

B. 13

C. 676

D. $8\sqrt{13}$

Zadanie 18. $(1pkt)$

Punkt $S=(-4,7)$ jest środkiem odcinka

współrzędne

$PQ$, gdzie $Q=(17,12)$. Zatem punkt $P$ ma

A. $P=(2,-25)$

B. $P=(38,17)$

C. $P=(-25,2)$

D. $P=(-12,4)$





{\it Egzamin maturalny z matematyki}

{\it Poziom podstawowy}

7

BRUDNOPIS





{\it 8}

{\it Egzamin maturalny z matematyki}

{\it Poziom podstawowy}

Zadanie 19. $(1pkt)$

Odległość między środkami okręgów o równaniach $(x+1)^{2}+(y-2)^{2}=9$ oraz $x^{2}+y^{2}=10$

jest równa

A. $\sqrt{5}$

B. $\sqrt{10}-3$

C. 3

D. 5

Zadanie 20. $(1pkt)$

Liczba wszystkich krawędzi graniastosłupajest o 10 większa od 1iczby wszystkichjego ścian

bocznych. Stąd wynika, $\dot{\mathrm{z}}\mathrm{e}$ podstawą tego graniastosłupajest

A. czworokąt

B. pięciokąt

C. sześciokąt

D. dziesięciokąt

Zadanie 21. (1pkt)

Pole powierzchni bocznej stozka o wysokości 4 i promieniu podstawy 3 jest równe

A. $ 9\pi$

B. $ 12\pi$

C. $ 15\pi$

D. $ 16\pi$

Zadanie 22. $(1pkt)$

Rzucamy dwa razy symetryczną sześcienną kostką do gry. Niech $p$ oznacza

prawdopodobieństwo zdarzenia, $\dot{\mathrm{z}}\mathrm{e}$ iloczyn liczb wyrzuconych oczekjest równy 5. Wtedy

A.

$p=\displaystyle \frac{1}{36}$

B.

$p=\displaystyle \frac{1}{18}$

C.

$p=\displaystyle \frac{1}{12}$

D.

{\it p}$=$ -91

Zadanie 23. $(1pkt)$

Liczba $\displaystyle \frac{\sqrt{50}-\sqrt{18}}{\sqrt{2}}$ jest równa

A. $2\sqrt{2}$ B. 2

C. 4

D. $\sqrt{10}-\sqrt{6}$

Zadanie 24. (1pkt)

Mediana uporządkowanego niemalejąco zestawu sześciu liczb:

Wtedy

1, 2, 3, x, 5, 8 jest równa 4.

A. $x=2$

B. $x=3$

C. $x=4$

D. $x=5$

Zadanie 25. $(1pkt)$

Objętość graniastosłupa prawidłowego trójkątnego o wysokości $7$jest równa $28\sqrt{3}$. Długość

krawędzi podstawy tego graniastosłupajest równa

A. 2

B. 4

C. 8

D. 16





{\it Egzamin maturalny z matematyki}

{\it Poziom podstawowy}

{\it 9}

BRUDNOPIS





$ 1\theta$

{\it Egzamin maturalny z matematyki}

{\it Poziom podstawowy}

ZADANIA OTWARTE

{\it Rozwiqzania zadań} $26-34$ {\it nalezy zapisać w wyznaczonych miejscach} $p\theta d$ {\it treściq zadania}.

Zadanie 26. $(2pkt)$

Rozwiąz równanie $x^{3}+2x^{2}-8x-16=0$

Odpowiedzí:



\end{document}