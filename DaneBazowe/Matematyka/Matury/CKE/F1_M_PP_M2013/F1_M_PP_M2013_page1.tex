\documentclass[a4paper,12pt]{article}
\usepackage{latexsym}
\usepackage{amsmath}
\usepackage{amssymb}
\usepackage{graphicx}
\usepackage{wrapfig}
\pagestyle{plain}
\usepackage{fancybox}
\usepackage{bm}

\begin{document}

{\it 2}

{\it Egzamin maturalny z matematyki}

{\it Poziom podstawowy}

ZADANIA ZAMKNIĘTE

{\it Wzadaniach l-25 wybierz i zaznacz na karcie odpowiedzipoprawnq odpowiedzí}.

Zadanie l. $(1pkt)$

Wskaz rysunek, na którym zaznaczony

spełniających nierówność $|x+4|<5.$

jest zbiór

wszystkich liczb rzeczywistych
\begin{center}
\includegraphics[width=165.552mm,height=12.240mm]{./F1_M_PP_M2013_page1_images/image001.eps}
\end{center}
A.
\begin{center}
\includegraphics[width=165.612mm,height=17.832mm]{./F1_M_PP_M2013_page1_images/image002.eps}
\end{center}
$-9  -4$  1  {\it X}

B.
\begin{center}
\includegraphics[width=165.552mm,height=18.036mm]{./F1_M_PP_M2013_page1_images/image003.eps}
\end{center}
$-1$  4 9  {\it X}

C.
\begin{center}
\includegraphics[width=165.552mm,height=17.784mm]{./F1_M_PP_M2013_page1_images/image004.eps}
\end{center}
$-9  -5  -1$  {\it X}

1 5  9  {\it X}

D.

Zadanie 2. $(1pkt)$

Liczby $a\mathrm{i}b$ są dodatnie oraz 12\% 1iczby $a$ jest równe 15\% 1iczby $b$. Stąd wynika, $\dot{\mathrm{z}}\mathrm{e}a$ jest

równe

A. 103\% 1iczby $b$ B. 125\% 1iczby $b$ C. 150\% 1iczby $b$ D. 153\% 1iczby $b$

Zadanie 3. $(1pkt)$

Liczba $\log 100-\log_{2}8$ jest równa

A. $-2$

B. $-1$

C. 0

D. l

Zadanie 4. $(1pkt)$

Rozwiązaniem układu równań 

A. $x=-3 \mathrm{i}y=4$

B. $x=-3 \mathrm{i}y=6$

C. $x=3 \mathrm{i}y=-4$

D. $x=9 \mathrm{i}y=4$

Zadanie 5. $(1pkt)$

Punkt $A=(0,1)$ lezy na wykresie ffinkcji liniowej $f(x)=(m-2)x+m-3$. Stąd wynika, $\dot{\mathrm{z}}\mathrm{e}$

A. $m=1$

B. $m=2$

C. $m=3$

D. $m=4$

Zadanie 6. $(1pkt)$

Wierzchołkiem paraboli o równaniu $y=-3(x-2)^{2}+4$ jest punkt o współrzędnych

A. $(-2,-4)$

B. $(-2,4)$

C. $(2,-4)$

D. (2, 4)

Zadanie 7. $(1pkt)$

Dla $\mathrm{k}\mathrm{a}\dot{\mathrm{z}}$ dej liczby rzeczywistej $x$, wyrazenie $4x^{2}-12x+9$ jest równe

A. $(4x+3)(x+3)$

B. $(2x-3)(2x+3)$

C. $(2x-3)(2x-3)$

D. $(x-3)(4x-3)$
\end{document}
