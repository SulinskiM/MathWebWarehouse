\documentclass[a4paper,12pt]{article}
\usepackage{latexsym}
\usepackage{amsmath}
\usepackage{amssymb}
\usepackage{graphicx}
\usepackage{wrapfig}
\pagestyle{plain}
\usepackage{fancybox}
\usepackage{bm}

\begin{document}

{\it Egzamin maturalny z matematyki}

{\it Poziom podstawowy}

{\it 13}

Zadanie 29. $(2pkt)$

Na rysunku przedstawiony jest wykres funkcji $f(x)$ określonej dla $x\in\langle-7,8\rangle.$
\begin{center}
\includegraphics[width=162.564mm,height=98.292mm]{./F1_M_PP_M2013_page12_images/image001.eps}
\end{center}
Odczytaj z wykresu i zapisz:

a) największą wartość funkcji f,

b) zbiór rozwiązań nierówności $f(x)<0.$
\begin{center}
\includegraphics[width=96.012mm,height=17.832mm]{./F1_M_PP_M2013_page12_images/image002.eps}
\end{center}
Wypelnia

egzaminator

Nr zadania

Maks. liczba kt

28.

2

2

Uzyskana liczba pkt
\end{document}
