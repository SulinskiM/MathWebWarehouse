\documentclass[a4paper,12pt]{article}
\usepackage{latexsym}
\usepackage{amsmath}
\usepackage{amssymb}
\usepackage{graphicx}
\usepackage{wrapfig}
\pagestyle{plain}
\usepackage{fancybox}
\usepackage{bm}

\begin{document}

{\it 4}

{\it Egzamin maturalny z matematyki}

{\it Poziom podstawowy}

Zadanie 8. $(1pkt)$

Prosta o równaniu $y=\displaystyle \frac{2}{m}x+1$ jest prostopadła do prostej o równaniu $y=-\displaystyle \frac{3}{2}x-1$. Stąd

wynika, $\dot{\mathrm{z}}\mathrm{e}$

A. $m=-3$

B.

{\it m}$=$ -23

C.

{\it m}$=$ -23

D. $m=3$

Zadanie 9. $(1pkt)$

Na rysunku przedstawiony jest fragment wykresu pewnej funkcji liniowej $y=ax+b.$
\begin{center}
\includegraphics[width=66.036mm,height=50.748mm]{./F1_M_PP_M2013_page3_images/image001.eps}
\end{center}
$y$

0  {\it x}

Jakie znaki mają współczynniki a ib?

A. $a<0 \mathrm{i}b<0$

B. $a<0 \mathrm{i}b>0$

C. $a>0 \mathrm{i}b<0$

D. $a>0\mathrm{i}b>0$

Zadanie 10. (1pkt)

Najmniejszą liczbą całkowitą spełniającą nierówność $\displaystyle \frac{x}{2}\leq\frac{2x}{3}+\frac{1}{4}$ jest

A. $-2$

B. $-1$

C. 0

D. l

Zadanie ll. $(1pkt)$

Na rysunku l przedstawiony jest wykres funkcji $y=f(x)$ określonej dla $x\in\langle-7,4\rangle.$
\begin{center}
\includegraphics[width=184.500mm,height=59.280mm]{./F1_M_PP_M2013_page3_images/image002.eps}
\end{center}
Rysunek 2 przedstawia wykres ffinkcji

A. $y=f(x+2)$ B. $y=f(x)-2$

C. $y=f(x-2)$

D. $y=f(x)+2$

Zadanie 12. $(1pkt)$

Ciąg $($27, 18, $x+5)$ jest geometryczny. Wtedy

A. $x=4$

B. $x=5$

C. $x=7$

D. $x=9$
\end{document}
