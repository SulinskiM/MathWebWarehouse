\documentclass[a4paper,12pt]{article}
\usepackage{latexsym}
\usepackage{amsmath}
\usepackage{amssymb}
\usepackage{graphicx}
\usepackage{wrapfig}
\pagestyle{plain}
\usepackage{fancybox}
\usepackage{bm}

\begin{document}

{\it 6}

{\it Egzamin maturalny z matematyki}

{\it Poziom podstawowy}

Zadanie 13. $(1pkt)$

Ciąg $(a_{n})$ określony dla $n\geq 1$ jest arytmetyczny oraz $a_{3}=10 \mathrm{i}a_{4}=14$. Pierwszy wyraz tego

ciągu jest równy

A. $a_{1}=-2$ B. $a_{1}=2$ C. $a_{1}=6$ D. $a_{1}=12$

Zadanie 14. $(1pkt)$

Kąt $\alpha$ jest ostry i $\displaystyle \sin\alpha=\frac{\sqrt{3}}{2}$. Wartość wyrazenia $\cos^{2}\alpha-2$ jest równa

A.

- -47

B.

- -41

C.

-21

D.

-$\sqrt{}$23

Zadanie 15. $(1pkt)$

Średnice AB $\mathrm{i}$ CD okręgu o środku $S$ przecinają się pod kątem $50^{\mathrm{o}}$ (takjak na rysunku).
\begin{center}
\includegraphics[width=65.124mm,height=65.628mm]{./F1_M_PP_M2013_page5_images/image001.eps}
\end{center}
{\it B}

{\it D}

$\alpha$

{\it S  M}

$50^{\mathrm{o}}$

{\it C}

{\it A}

Miara kąta $\alpha$ jest równa

A. $25^{\mathrm{o}}$

B. $30^{\mathrm{o}}$

C. $40^{\mathrm{o}}$

D. $50^{\mathrm{o}}$

Zadanie 16. $(1pkt)$

Liczba rzeczywistych rozwiązań równania $(x+1)(x+2)(x^{2}+3)=0$ jest równa

A. 0

B. l

C. 2

D. 4

Zadanie 17. $(1pkt)$

Punkty $A=(-1,2) \mathrm{i}B=(5,-2)$ są dwoma sąsiednimi wierzchołkami rombu ABCD. Obwód

tego rombujest równy

A. $\sqrt{13}$

B. 13

C. 676

D. $8\sqrt{13}$

Zadanie 18. $(1pkt)$

Punkt $S=(-4,7)$ jest środkiem odcinka

współrzędne

$PQ$, gdzie $Q=(17,12)$. Zatem punkt $P$ ma

A. $P=(2,-25)$

B. $P=(38,17)$

C. $P=(-25,2)$

D. $P=(-12,4)$
\end{document}
