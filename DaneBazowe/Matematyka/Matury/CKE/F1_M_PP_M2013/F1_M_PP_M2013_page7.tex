\documentclass[a4paper,12pt]{article}
\usepackage{latexsym}
\usepackage{amsmath}
\usepackage{amssymb}
\usepackage{graphicx}
\usepackage{wrapfig}
\pagestyle{plain}
\usepackage{fancybox}
\usepackage{bm}

\begin{document}

{\it 8}

{\it Egzamin maturalny z matematyki}

{\it Poziom podstawowy}

Zadanie 19. $(1pkt)$

Odległość między środkami okręgów o równaniach $(x+1)^{2}+(y-2)^{2}=9$ oraz $x^{2}+y^{2}=10$

jest równa

A. $\sqrt{5}$

B. $\sqrt{10}-3$

C. 3

D. 5

Zadanie 20. $(1pkt)$

Liczba wszystkich krawędzi graniastosłupajest o 10 większa od 1iczby wszystkichjego ścian

bocznych. Stąd wynika, $\dot{\mathrm{z}}\mathrm{e}$ podstawą tego graniastosłupajest

A. czworokąt

B. pięciokąt

C. sześciokąt

D. dziesięciokąt

Zadanie 21. (1pkt)

Pole powierzchni bocznej stozka o wysokości 4 i promieniu podstawy 3 jest równe

A. $ 9\pi$

B. $ 12\pi$

C. $ 15\pi$

D. $ 16\pi$

Zadanie 22. $(1pkt)$

Rzucamy dwa razy symetryczną sześcienną kostką do gry. Niech $p$ oznacza

prawdopodobieństwo zdarzenia, $\dot{\mathrm{z}}\mathrm{e}$ iloczyn liczb wyrzuconych oczekjest równy 5. Wtedy

A.

$p=\displaystyle \frac{1}{36}$

B.

$p=\displaystyle \frac{1}{18}$

C.

$p=\displaystyle \frac{1}{12}$

D.

{\it p}$=$ -91

Zadanie 23. $(1pkt)$

Liczba $\displaystyle \frac{\sqrt{50}-\sqrt{18}}{\sqrt{2}}$ jest równa

A. $2\sqrt{2}$ B. 2

C. 4

D. $\sqrt{10}-\sqrt{6}$

Zadanie 24. (1pkt)

Mediana uporządkowanego niemalejąco zestawu sześciu liczb:

Wtedy

1, 2, 3, x, 5, 8 jest równa 4.

A. $x=2$

B. $x=3$

C. $x=4$

D. $x=5$

Zadanie 25. $(1pkt)$

Objętość graniastosłupa prawidłowego trójkątnego o wysokości $7$jest równa $28\sqrt{3}$. Długość

krawędzi podstawy tego graniastosłupajest równa

A. 2

B. 4

C. 8

D. 16
\end{document}
