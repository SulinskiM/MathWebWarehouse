\documentclass[a4paper,12pt]{article}
\usepackage{latexsym}
\usepackage{amsmath}
\usepackage{amssymb}
\usepackage{graphicx}
\usepackage{wrapfig}
\pagestyle{plain}
\usepackage{fancybox}
\usepackage{bm}

\begin{document}

$--\cdot\backslash \tau^{\mathrm{l}}\cdots-\cdot i1\dot{\text{‡C}}$

$:_{}^{\prime:=_{1\text{‡@}}^{1}}\overline{\iota}_{:\dot{!^{\mathrm{f}}:}!_{\vee}}^{-1}..$

-r$\equiv$:$\grave{}=$-J-$=$.-$\acute{}$z--,-.-[‡@]n-w$\Omega$-.:-.R-.J--n-$\llcorner\ulcorner-\simeq-\breve{}$.---$\lrcorner$.--[‡@]R-[‡@]-.--{\it l}-$\iota$-

Arkusz zawiera informacje prawnie chronione do momentu rozpoczęcia egzaminu.

WPISUJE ZDAJACY

KOD PESEL

{\it Miejsce}

{\it na naklejkę}

{\it z kodem}
\begin{center}
\includegraphics[width=21.432mm,height=9.852mm]{./F1_M_PP_M2014_page0_images/image001.eps}

\includegraphics[width=82.044mm,height=9.852mm]{./F1_M_PP_M2014_page0_images/image002.eps}
\end{center}
\fbox{} dysleksja
\begin{center}
\includegraphics[width=204.060mm,height=219.000mm]{./F1_M_PP_M2014_page0_images/image003.eps}
\end{center}
EGZAMIN MATU LNY

Z MATEMATYKI

MAJ 2014

POZIOM PODSTAWOWY

1. Sprawd $\acute{\mathrm{z}}$, czy arkusz egzaminacyjny zawiera 19 stron

(zadania $1-34$). Ewentualny brak zgłoś przewodniczącemu

zespo nadzo jącego egzamin.

2. Rozwiązania zadań i odpowiedzi wpisuj w miejscu na to

przeznaczonym.

3. Odpowiedzi do zadań za niętych (l-25) przenieś

na ka ę odpowiedzi, zaznaczając je w części ka $\mathrm{y}$

przeznaczonej dla zdającego. Zamaluj $\blacksquare$ pola do tego

przeznaczone. Błędne zaznaczenie otocz kółkiem \fcircle$\bullet$

i zaznacz właściwe.

4. Pamiętaj, $\dot{\mathrm{z}}\mathrm{e}$ pominięcie argumentacji lub istotnych

obliczeń w rozwiązaniu zadania otwa ego (26-34) $\mathrm{m}\mathrm{o}\dot{\mathrm{z}}\mathrm{e}$

spowodować, $\dot{\mathrm{z}}\mathrm{e}$ za to rozwiązanie nie otrzymasz pełnej

liczby punktów.

5. Pisz czytelnie i uzywaj tvlko długopisu lub -Dióra

z czatnym tuszem lub atramentem.

6. Nie uzywaj korektora, a błędne zapisy wyrazínie prze eśl.

7. Pamiętaj, $\dot{\mathrm{z}}\mathrm{e}$ zapisy w brudnopisie nie będą oceniane.

8. $\mathrm{M}\mathrm{o}\dot{\mathrm{z}}$ esz korzystać z zestawu wzorów matematycznych,

cyrkla i linijki oraz kalkulatora.

9. Na tej stronie oraz na karcie odpowiedzi wpisz swój

numer PESEL i przyklej naklejkę z kodem.

10. Nie wpisuj $\dot{\mathrm{z}}$ adnych znaków w części przeznaczonej

dla egzaminatora.

Czas pracy:

170 minut

Liczba punktów

do uzyskania: 50

$\Vert\Vert\Vert\Vert\Vert\Vert\Vert\Vert\Vert\Vert\Vert\Vert\Vert\Vert\Vert\Vert\Vert\Vert\Vert\Vert\Vert\Vert\Vert\Vert|  \mathrm{M}\mathrm{M}\mathrm{A}-\mathrm{P}1_{-}1\mathrm{P}-142$




{\it 2}

{\it Egzamin maturalny z matematyki}

{\it Poziom podstawowy}

ZADANIA ZAMKNIĘTE

{\it Wzadaniach od l. do 25. wybierz i zaznacz na karcie odpowiedzipoprawnq} $odp\theta wied\acute{z}.$

Zadanie l. $(1pkt)$

Na rysunku przedstawiono geometryczną interpretację jednego z $\mathrm{n}\mathrm{i}\dot{\mathrm{z}}$ ej zapisanych układów

równań.
\begin{center}
\includegraphics[width=62.844mm,height=50.340mm]{./F1_M_PP_M2014_page1_images/image001.eps}
\end{center}
4  {\it y}

$-3$

2

$-2$ -$1$

0

$-1$

1 2 3  {\it x}

Wskaz ten układ.

A.

$\left\{\begin{array}{l}
y=x+1\\
y=-2x+4
\end{array}\right.$

B.

$\left\{\begin{array}{l}
y=x-1\\
y=2x+4
\end{array}\right.$

C.

$\left\{\begin{array}{l}
y=x-1\\
y=-2x+4
\end{array}\right.$

D.

$\left\{\begin{array}{l}
y=x+1\\
y=2x+4
\end{array}\right.$

Zadanie 2. $(1pkt)$

$\mathrm{J}\mathrm{e}\dot{\mathrm{z}}$ eli liczba $78$jest o 50\% większa od 1iczby $c$, to

A. $c=60$

B. $c=52$

C. $c=48$

D. $c=39$

Zadanie 3. $(1pkt)$

Wartość wyrazenia $\displaystyle \frac{2}{\sqrt{3}-1}-\frac{2}{\sqrt{3}+1}$ jest równa

A. $-2$ B. $-2\sqrt{3}$

C. 2

D. $2\sqrt{3}$

Zadanie $4.(1pkt)$

Suma $\log_{8}16+1$ jest równa

A. 3

B.

-23

C. log817

D.

-73

Zadanie 5. $(1pkt)$

Wspólnym pierwiastkiem równań $(x^{2}-1)(x-10)(x-5)=0$ oraz $\displaystyle \frac{2x-10}{x-1}=0$ jest liczba

A. $-1$

B. l

C. 5

D. 10





{\it Egzamin maturalny z matematyki}

{\it Poziom podstawowy}

{\it 11}

Zadanie 27. $(2pkt)$

Rozwiąz równanie $9x^{3}+18x^{2}-4x-8=0.$

Odpowiedzí :
\begin{center}
\includegraphics[width=90.372mm,height=17.580mm]{./F1_M_PP_M2014_page10_images/image001.eps}
\end{center}
Wypelnia

egzamÍnator

Nr zadania

Maks. liczba kt

2

27.

2

Uzyskana liczba pkt





{\it 12}

{\it Egzamin maturalny z matematyki}

{\it Poziom podstawowy}

Zadanie 28. $(2pkt)$

Udowodnij, $\dot{\mathrm{z}}\mathrm{e}\mathrm{k}\mathrm{a}\dot{\mathrm{z}}$ da liczba całkowita $k$, która przy dzieleniu przez 7 daje resztę 2, ma tę

własność, $\dot{\mathrm{z}}\mathrm{e}$ reszta z dzielenia liczby $3k^{2}$ przez $7$jest równa 5.





{\it Egzamin maturalny z matematyki}

{\it Poziom podstawowy}

{\it 13}

Zadanie 29. $(2pkt)$

Na rysunku przedstawiono fragment wykresu ffinkcji $f$, który powstał w wyniku przesunięcia

wykresu funkcji określonej wzorem $y=\displaystyle \frac{1}{x}$ dla $\mathrm{k}\mathrm{a}\dot{\mathrm{z}}$ dej liczby rzeczywistej $x\neq 0.$
\begin{center}
\includegraphics[width=98.760mm,height=87.372mm]{./F1_M_PP_M2014_page12_images/image001.eps}
\end{center}
a) Odczytaj z wykresu i zapisz zbiór tych wszystkich argumentów, dla których wartości

funkcji $f$ są większe od 0.

b) Podaj miejsce zerowe funkcji $g$ określonej wzorem $g(x)=f(x-3).$

Odpowied $\acute{\mathrm{z}}:\mathrm{a})$

b)
\begin{center}
\includegraphics[width=90.372mm,height=17.580mm]{./F1_M_PP_M2014_page12_images/image002.eps}
\end{center}
Wypelnia

egzamÍnator

Nr zadania

Maks. liczba kt

28.

2

2

Uzyskana liczba pkt





{\it 14}

{\it Egzamin maturalny z matematyki}

{\it Poziom podstawowy}

Zadanie 30. (2pkt)

Ze zbioiu liczb \{1, 2, 3, 4, 5, 6, 7, 8\} 1osujemy dwa razy po jednej 1iczbie ze zwracaniem.

Oblicz prawdopodobieństwo zdarzenia A, polegającego na wylosowaniu liczb, z których

pierwszajest większa od drugiej o 41ub 6.

Odpowied $\acute{\mathrm{z}}$:





{\it Egzamin maturalny z matematyki}

{\it Poziom podstawowy}

{\it 15}

Zadanie 31. (2pkt)

Środek S okręgu opisanego na trójkącie równoramiennym ABC, o ramionach ACiBC, lezy

wewnątrz tego trójkąta (zobacz rysunek).
\begin{center}
\includegraphics[width=60.708mm,height=65.076mm]{./F1_M_PP_M2014_page14_images/image001.eps}
\end{center}
{\it C}

{\it S}

{\it A  B}

{\it ASB}

kąta wypukłego

Wykaz, $\dot{\mathrm{z}}\mathrm{e}$ miara

wypukłego $SBC.$

est cztery

razy większa

od miary kąta
\begin{center}
\includegraphics[width=90.372mm,height=17.580mm]{./F1_M_PP_M2014_page14_images/image002.eps}
\end{center}
Wypelnia

egzamÍnator

Nr zadania

Maks. liczba kt

30.

2

31.

2

Uzyskana liczba pkt





{\it 16}

{\it Egzamin maturalny z matematyki}

{\it Poziom podstawowy}

Zadanie 32. (4pkt)

Pole powierzchni całkowitej prostopadłościanu jest równe 198. Stosunki długości krawędzi

prostopadłościanu wychodzących z tego samego wierzchołka prostopadłościanu to 1: 2: 3.

Oblicz długość przekątnej tego prostopadłościanu.

Odpowied $\acute{\mathrm{z}}$:





{\it Egzamin maturalny z matematyki}

{\it Poziom podstawowy}

{\it 1}7

Zadanie 33. $(5pkt)$

Turysta zwiedzał zamek stojący na wzgórzu. Droga łącząca parking z zamkiem ma długość

2,1 km. Lączny czas wędrówki turysty z parkingu do zamku i z powrotem, nie licząc czasu

poświęconego na zwiedzanie, był równy l godzinę i 4 minuty. Ob1icz, z jaką średnią

prędkością turysta wchodził na wzgórze, $\mathrm{j}\mathrm{e}\dot{\mathrm{z}}$ eli prędkość ta była o $1 \displaystyle \frac{\mathrm{k}\mathrm{m}}{\mathrm{h}}$ mniejsza od średniej

prędkości, zjaką schodził ze wzgórza.

Odpowied $\acute{\mathrm{z}}$:
\begin{center}
\includegraphics[width=90.372mm,height=17.628mm]{./F1_M_PP_M2014_page16_images/image001.eps}
\end{center}
Wypelnia

egzaminator

Nr zadania

Maks. liczba kt

32.

4

33.

5

Uzyskana liczba pkt





{\it 18}

{\it Egzamin maturalny z matematyki}

{\it Poziom podstawowy}

Zadanie 34. $(4pkt)$

Kąt CAB trójkąta prostokątnego $ACB$ ma miarę $30^{\mathrm{o}}$. Pole kwadratu DEFG, wpisanego w ten

trójkąt (zobacz rysunek), jest równe 4. Ob1icz po1e trójkąta $ACB.$
\begin{center}
\includegraphics[width=68.880mm,height=43.692mm]{./F1_M_PP_M2014_page17_images/image001.eps}
\end{center}
{\it B}

{\it F}

{\it E}

{\it G}

$30^{\mathrm{o}}$

{\it C D  A}

Odpowiedzí :
\begin{center}
\includegraphics[width=78.840mm,height=17.580mm]{./F1_M_PP_M2014_page17_images/image002.eps}
\end{center}
Wypelnia

egzaminator

Nr zadania

Maks. liczba kt

34.

4

Uzyskana liczba pkt





{\it Egzamin maturalny z matematyki}

{\it Poziom podstawowy}

{\it 19}

BRUDNOPIS





{\it Egzamin maturalny z matematyki}

{\it Poziom podstawowy}

{\it 3}

BRUDNOPIS





{\it 4}

{\it Egzamin maturalny z matematyki}

{\it Poziom podstawowy}

Zadanie 6. $(1pkt)$

Funkcja liniowa $f(x)=(m^{2}-4)x+2$ jest malejąca, gdy

A. $m\in\{-2,2\}$

B. $m\in(-2,2)$

C.

$m\in(-\infty,-2)$

D. $m\in(2,+\infty)$

Zadanie 7. (1pkt)

Na rysunku przedstawiono fragment wykresu funkcji kwadratowej f
\begin{center}
\includegraphics[width=56.436mm,height=49.788mm]{./F1_M_PP_M2014_page3_images/image001.eps}
\end{center}
{\it y}

{\it x}

0

Funkcjafjest określona wzorem

A.

C.

$f(x)=\displaystyle \frac{1}{2}(x+3)(x-1)$

$f(x)=-\displaystyle \frac{1}{2}(x+3)(x-1)$

B.

D.

$f(x)=\displaystyle \frac{1}{2}(x-3)(x+1)$

$f(x)=-\displaystyle \frac{1}{2}(x-3)(x+1)$

Zadanie 8. $(1pkt)$

Punkt $C=(0,2)$ jest wierzchołkiem trapezu ABCD, którego podstawa $AB$ jest zawarta

w prostej o równaniu $y=2x-4$. Wskaz równanie prostej zawierającej podstawę CD.

A. $y=\displaystyle \frac{1}{2}x+2$ B. $y=-2x+2$ C. $y=-\displaystyle \frac{1}{2}x+2$ D. $y=2x+2$

Zadanie 9. $(1pkt)$

Dla $\mathrm{k}\mathrm{a}\dot{\mathrm{z}}$ dej liczby $x$, spełniającej warunek-3$<x<0$, wyrazenie $\displaystyle \frac{|x+3|-x+3}{x}$ jest równe

A. 2 B. 3 C. --{\it x}6 D. -{\it x}6

Zadanie 10. $(1pkt)$

Pierwiastki $x_{1}, x_{2}$ równania $2(x+2)(x-2)=0$ spełniają warunek

A.

$\underline{1}\underline{1}+=-1$

$x_{1} x_{2}$

B.

$\underline{1}+\underline{1}=0$

$x_{1} x_{2}$

C.

-{\it x}1  1 $+$ -{\it x}12 $=$ -41

D.

-{\it x}1  1 $+$ -{\it x}12 $=$ -21

Zadanie ll. $(1pkt)$

Liczby $2, -1, -4$ są trzema początkowymi wyrazami ciągu arytmetycznego

określonego dla liczb naturalnych $n\geq 1$. Wzór ogólny tego ciągu ma postać

A. $a_{n}=-3n+5$ B. $a_{n}=n-3$ C. $a_{n}=-n+3$ D. $a_{n}=3n-5$

$(a_{n}),$





{\it Egzamin maturalny z matematyki}

{\it Poziom podstawowy}

{\it 5}

BRUDNOPIS





{\it 6}

{\it Egzamin maturalny z matematyki}

{\it Poziom podstawowy}

Zadanie 12. $(1pkt)$

$\mathrm{J}\mathrm{e}\dot{\mathrm{z}}$ eli trójkąty $ABC \mathrm{i} A'B'C'$ są podobne, a ich pola $\mathrm{S}i\mathrm{L}$ odpowiednio, równe 25 $\mathrm{c}\mathrm{m}^{2}$

$\mathrm{i}50\mathrm{c}\mathrm{m}^{2}$, to skala podobieństwa $\displaystyle \frac{A'B'}{AB}$ jest równa

A. 2 B. -21 C. $\sqrt{}$2 D. --$\sqrt{}$22

Zadanie 13. $(1pkt)$

Liczby: $x-2$, 6, 12, w podanej kolejności,

geometrycznego. Liczba $x$ jest równa

są trzema

kolejnymi wyrazami

ciągu

A. 0

B. 2

C. 3

D. 5

Zadanie 14. $(1pkt)$

$\mathrm{J}\mathrm{e}\dot{\mathrm{z}}$ eli $\alpha$ jest kątem ostrym oraz $\displaystyle \mathrm{t}\mathrm{g}\alpha=\frac{2}{5}$, to wartość wyrazenia $\displaystyle \frac{3\cos\alpha-2\sin\alpha}{\sin\alpha-5\cos\alpha}$ jest równa

A.

$-\displaystyle \frac{11}{23}$

B.

$\displaystyle \frac{24}{5}$

C.

- -2131

D.

$\displaystyle \frac{5}{24}$

Zadanie 15. (1pkt)

Liczba punktów wspólnych okręgu

współrzędnychjest równa

A. 0 B. 1

o równaniu $(x+2)^{2}+(y-3)^{2}=4$

C. 2 D.

z osiami układu

4

Zadanie 16. $(1pkt)$

Wysokość trapezu równoramiennego o kącie ostrym $60^{\mathrm{o}}$ i ramieniu długości $2\sqrt{3}$ jest równa

A. $\sqrt{3}$ B. 3 C. $2\sqrt{3}$ D. 2

Zadanie 17. $(1pkt)$

Kąt środkowy oparty na iuku, którego diugośćjest równa $\displaystyle \frac{4}{9}$ diugości okręgu, ma miarę

A. $160^{\mathrm{o}}$

B. $80^{\mathrm{o}}$

C. $40^{\mathrm{o}}$

D. $20^{\mathrm{o}}$

Zadanie 18. $(1pkt)$

$\mathrm{O}$ funkcji liniowej $f$ wiadomo, $\dot{\mathrm{z}}\mathrm{e}f(1)=2$. Do wykresu tej funkcji nalez$\mathrm{y}$ punkt $P=(-2,3).$

Wzór funkcji $f$ to

A. $f(x)=-\displaystyle \frac{1}{3}x+\frac{7}{3}$ B. $f(x)=-\displaystyle \frac{1}{2}x+2$ C. $f(x)=-3x+7$ D. $f(x)=-2x+4$

Zadanie 19. $(1pkt)$

$\mathrm{J}\mathrm{e}\dot{\mathrm{z}}$ eli ostrosłup ma 10 krawędzi, to 1iczba ścian bocznychjest równa

A. 5

B. 7

C. 8

D. 10





{\it Egzamin maturalny z matematyki}

{\it Poziom podstawowy}

7

BRUDNOPIS





{\it 8}

{\it Egzamin maturalny z matematyki}

{\it Poziom podstawowy}

Zadanie 20. (1pkt)

Stozek i walec mają takie same podstawy i równe pola powierzchni bocznych. Wtedy

tworząca stozka jest

A. sześć razy dłuzsza od wysokości walca.

B. trzy razy dłuzsza od wysokości walca.

C. dwa razy dłuzsza od wysokości walca.

D. równa wysokości walca.

Zadanie 21. $(1pkt)$

Liczba $(\displaystyle \frac{1}{(\sqrt[3]{729}+\sqrt[4]{256}+2)^{0}})^{-2}$ jest równa

A. $\displaystyle \frac{1}{225}$ B. $\displaystyle \frac{1}{15}$

C. l

D. 15

Zadanie 22. $(1pkt)$

Do wykresu ffinkcji, określonej dla wszystkich liczb rzeczywistych wzorem $y=-2^{x-2}$, nalez$\mathrm{y}$

punkt

A. $A=(1,-2)$ B. $B=(2,-1)$ C. $C=(1,\displaystyle \frac{1}{2})$ D. $D=(4,4)$

Zadanie 23. $(1pkt)$

$\mathrm{J}\mathrm{e}\dot{\mathrm{z}}$ eli $A$ jest zdarzeniem losowym, $\mathrm{a}$

zachodzi równość $P(A)=2\cdot P(A^{\uparrow})$, to

A. $P(A)=\displaystyle \frac{2}{3}$ B. $P(A)=\displaystyle \frac{1}{2}$

A ` -zdarzeniem przeciwnym do zdarzenia A oraz

C. $P(A)=\displaystyle \frac{1}{3}$ D. $P(A)=\displaystyle \frac{1}{6}$

Zadanie 24. (1pkt)

Na ile sposobów mozna wybrać dwóch graczy spośród 10 zawodników?

A. 100 B. 90 C. 45 D.

20

Zadanie 25. $(1pkt)$

Mediana zestawu danych 2, 12, $a$, 10, 5, 3 jest równa 7. Wówczas

A. $a=4$ B. $a=6$ C. $a=7$

D. $a=9$





{\it Egzamin maturalny z matematyki}

{\it Poziom podstawowy}

{\it 9}

BRUDNOPIS





$ 1\theta$

{\it Egzamin maturalny z matematyki}

{\it Poziom podstawowy}

ZADANIA OTWARTE

{\it Rozwiqzania zadań o numerach od 26. do 34. nalezy zapisać}

{\it w wyznaczonych miejscach} $p\theta d$ {\it treściq zadania}.

Zadanie 26. $(2pkt)$

Wykresem funkcji kwadratowej $f(x)=2x^{2}+bx+c$ jest parabola, której wierzchołkiemjest

punkt $W=(4,0)$. Oblicz wartości współczynników $b\mathrm{i}c.$

Odpowied $\acute{\mathrm{z}}$:



\end{document}