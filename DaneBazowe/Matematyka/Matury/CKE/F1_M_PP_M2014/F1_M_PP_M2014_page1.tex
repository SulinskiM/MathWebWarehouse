\documentclass[a4paper,12pt]{article}
\usepackage{latexsym}
\usepackage{amsmath}
\usepackage{amssymb}
\usepackage{graphicx}
\usepackage{wrapfig}
\pagestyle{plain}
\usepackage{fancybox}
\usepackage{bm}

\begin{document}

{\it 2}

{\it Egzamin maturalny z matematyki}

{\it Poziom podstawowy}

ZADANIA ZAMKNIĘTE

{\it Wzadaniach od l. do 25. wybierz i zaznacz na karcie odpowiedzipoprawnq} $odp\theta wied\acute{z}.$

Zadanie l. $(1pkt)$

Na rysunku przedstawiono geometryczną interpretację jednego z $\mathrm{n}\mathrm{i}\dot{\mathrm{z}}$ ej zapisanych układów

równań.
\begin{center}
\includegraphics[width=62.844mm,height=50.340mm]{./F1_M_PP_M2014_page1_images/image001.eps}
\end{center}
4  {\it y}

$-3$

2

$-2$ -$1$

0

$-1$

1 2 3  {\it x}

Wskaz ten układ.

A.

$\left\{\begin{array}{l}
y=x+1\\
y=-2x+4
\end{array}\right.$

B.

$\left\{\begin{array}{l}
y=x-1\\
y=2x+4
\end{array}\right.$

C.

$\left\{\begin{array}{l}
y=x-1\\
y=-2x+4
\end{array}\right.$

D.

$\left\{\begin{array}{l}
y=x+1\\
y=2x+4
\end{array}\right.$

Zadanie 2. $(1pkt)$

$\mathrm{J}\mathrm{e}\dot{\mathrm{z}}$ eli liczba $78$jest o 50\% większa od 1iczby $c$, to

A. $c=60$

B. $c=52$

C. $c=48$

D. $c=39$

Zadanie 3. $(1pkt)$

Wartość wyrazenia $\displaystyle \frac{2}{\sqrt{3}-1}-\frac{2}{\sqrt{3}+1}$ jest równa

A. $-2$ B. $-2\sqrt{3}$

C. 2

D. $2\sqrt{3}$

Zadanie $4.(1pkt)$

Suma $\log_{8}16+1$ jest równa

A. 3

B.

-23

C. log817

D.

-73

Zadanie 5. $(1pkt)$

Wspólnym pierwiastkiem równań $(x^{2}-1)(x-10)(x-5)=0$ oraz $\displaystyle \frac{2x-10}{x-1}=0$ jest liczba

A. $-1$

B. l

C. 5

D. 10
\end{document}
