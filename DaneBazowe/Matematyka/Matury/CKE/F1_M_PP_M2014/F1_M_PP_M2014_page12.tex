\documentclass[a4paper,12pt]{article}
\usepackage{latexsym}
\usepackage{amsmath}
\usepackage{amssymb}
\usepackage{graphicx}
\usepackage{wrapfig}
\pagestyle{plain}
\usepackage{fancybox}
\usepackage{bm}

\begin{document}

{\it Egzamin maturalny z matematyki}

{\it Poziom podstawowy}

{\it 13}

Zadanie 29. $(2pkt)$

Na rysunku przedstawiono fragment wykresu ffinkcji $f$, który powstał w wyniku przesunięcia

wykresu funkcji określonej wzorem $y=\displaystyle \frac{1}{x}$ dla $\mathrm{k}\mathrm{a}\dot{\mathrm{z}}$ dej liczby rzeczywistej $x\neq 0.$
\begin{center}
\includegraphics[width=98.760mm,height=87.372mm]{./F1_M_PP_M2014_page12_images/image001.eps}
\end{center}
a) Odczytaj z wykresu i zapisz zbiór tych wszystkich argumentów, dla których wartości

funkcji $f$ są większe od 0.

b) Podaj miejsce zerowe funkcji $g$ określonej wzorem $g(x)=f(x-3).$

Odpowied $\acute{\mathrm{z}}:\mathrm{a})$

b)
\begin{center}
\includegraphics[width=90.372mm,height=17.580mm]{./F1_M_PP_M2014_page12_images/image002.eps}
\end{center}
Wypelnia

egzamÍnator

Nr zadania

Maks. liczba kt

28.

2

2

Uzyskana liczba pkt
\end{document}
