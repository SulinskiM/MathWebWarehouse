\documentclass[a4paper,12pt]{article}
\usepackage{latexsym}
\usepackage{amsmath}
\usepackage{amssymb}
\usepackage{graphicx}
\usepackage{wrapfig}
\pagestyle{plain}
\usepackage{fancybox}
\usepackage{bm}

\begin{document}

{\it Egzamin maturalny z matematyki}

{\it Poziom podstawowy}

{\it 15}

Zadanie 31. (2pkt)

Środek S okręgu opisanego na trójkącie równoramiennym ABC, o ramionach ACiBC, lezy

wewnątrz tego trójkąta (zobacz rysunek).
\begin{center}
\includegraphics[width=60.708mm,height=65.076mm]{./F1_M_PP_M2014_page14_images/image001.eps}
\end{center}
{\it C}

{\it S}

{\it A  B}

{\it ASB}

kąta wypukłego

Wykaz, $\dot{\mathrm{z}}\mathrm{e}$ miara

wypukłego $SBC.$

est cztery

razy większa

od miary kąta
\begin{center}
\includegraphics[width=90.372mm,height=17.580mm]{./F1_M_PP_M2014_page14_images/image002.eps}
\end{center}
Wypelnia

egzamÍnator

Nr zadania

Maks. liczba kt

30.

2

31.

2

Uzyskana liczba pkt
\end{document}
