\documentclass[a4paper,12pt]{article}
\usepackage{latexsym}
\usepackage{amsmath}
\usepackage{amssymb}
\usepackage{graphicx}
\usepackage{wrapfig}
\pagestyle{plain}
\usepackage{fancybox}
\usepackage{bm}

\begin{document}

{\it Egzamin maturalny z matematyki}

{\it Poziom podstawowy}

{\it 1}7

Zadanie 33. $(5pkt)$

Turysta zwiedzał zamek stojący na wzgórzu. Droga łącząca parking z zamkiem ma długość

2,1 km. Lączny czas wędrówki turysty z parkingu do zamku i z powrotem, nie licząc czasu

poświęconego na zwiedzanie, był równy l godzinę i 4 minuty. Ob1icz, z jaką średnią

prędkością turysta wchodził na wzgórze, $\mathrm{j}\mathrm{e}\dot{\mathrm{z}}$ eli prędkość ta była o $1 \displaystyle \frac{\mathrm{k}\mathrm{m}}{\mathrm{h}}$ mniejsza od średniej

prędkości, zjaką schodził ze wzgórza.

Odpowied $\acute{\mathrm{z}}$:
\begin{center}
\includegraphics[width=90.372mm,height=17.628mm]{./F1_M_PP_M2014_page16_images/image001.eps}
\end{center}
Wypelnia

egzaminator

Nr zadania

Maks. liczba kt

32.

4

33.

5

Uzyskana liczba pkt
\end{document}
