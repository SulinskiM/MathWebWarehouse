\documentclass[a4paper,12pt]{article}
\usepackage{latexsym}
\usepackage{amsmath}
\usepackage{amssymb}
\usepackage{graphicx}
\usepackage{wrapfig}
\pagestyle{plain}
\usepackage{fancybox}
\usepackage{bm}

\begin{document}

{\it 4}

{\it Egzamin maturalny z matematyki}

{\it Poziom podstawowy}

Zadanie 6. $(1pkt)$

Funkcja liniowa $f(x)=(m^{2}-4)x+2$ jest malejąca, gdy

A. $m\in\{-2,2\}$

B. $m\in(-2,2)$

C.

$m\in(-\infty,-2)$

D. $m\in(2,+\infty)$

Zadanie 7. (1pkt)

Na rysunku przedstawiono fragment wykresu funkcji kwadratowej f
\begin{center}
\includegraphics[width=56.436mm,height=49.788mm]{./F1_M_PP_M2014_page3_images/image001.eps}
\end{center}
{\it y}

{\it x}

0

Funkcjafjest określona wzorem

A.

C.

$f(x)=\displaystyle \frac{1}{2}(x+3)(x-1)$

$f(x)=-\displaystyle \frac{1}{2}(x+3)(x-1)$

B.

D.

$f(x)=\displaystyle \frac{1}{2}(x-3)(x+1)$

$f(x)=-\displaystyle \frac{1}{2}(x-3)(x+1)$

Zadanie 8. $(1pkt)$

Punkt $C=(0,2)$ jest wierzchołkiem trapezu ABCD, którego podstawa $AB$ jest zawarta

w prostej o równaniu $y=2x-4$. Wskaz równanie prostej zawierającej podstawę CD.

A. $y=\displaystyle \frac{1}{2}x+2$ B. $y=-2x+2$ C. $y=-\displaystyle \frac{1}{2}x+2$ D. $y=2x+2$

Zadanie 9. $(1pkt)$

Dla $\mathrm{k}\mathrm{a}\dot{\mathrm{z}}$ dej liczby $x$, spełniającej warunek-3$<x<0$, wyrazenie $\displaystyle \frac{|x+3|-x+3}{x}$ jest równe

A. 2 B. 3 C. --{\it x}6 D. -{\it x}6

Zadanie 10. $(1pkt)$

Pierwiastki $x_{1}, x_{2}$ równania $2(x+2)(x-2)=0$ spełniają warunek

A.

$\underline{1}\underline{1}+=-1$

$x_{1} x_{2}$

B.

$\underline{1}+\underline{1}=0$

$x_{1} x_{2}$

C.

-{\it x}1  1 $+$ -{\it x}12 $=$ -41

D.

-{\it x}1  1 $+$ -{\it x}12 $=$ -21

Zadanie ll. $(1pkt)$

Liczby $2, -1, -4$ są trzema początkowymi wyrazami ciągu arytmetycznego

określonego dla liczb naturalnych $n\geq 1$. Wzór ogólny tego ciągu ma postać

A. $a_{n}=-3n+5$ B. $a_{n}=n-3$ C. $a_{n}=-n+3$ D. $a_{n}=3n-5$

$(a_{n}),$
\end{document}
