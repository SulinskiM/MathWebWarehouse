\documentclass[a4paper,12pt]{article}
\usepackage{latexsym}
\usepackage{amsmath}
\usepackage{amssymb}
\usepackage{graphicx}
\usepackage{wrapfig}
\pagestyle{plain}
\usepackage{fancybox}
\usepackage{bm}

\begin{document}

{\it 6}

{\it Egzamin maturalny z matematyki}

{\it Poziom podstawowy}

Zadanie 12. $(1pkt)$

$\mathrm{J}\mathrm{e}\dot{\mathrm{z}}$ eli trójkąty $ABC \mathrm{i} A'B'C'$ są podobne, a ich pola $\mathrm{S}i\mathrm{L}$ odpowiednio, równe 25 $\mathrm{c}\mathrm{m}^{2}$

$\mathrm{i}50\mathrm{c}\mathrm{m}^{2}$, to skala podobieństwa $\displaystyle \frac{A'B'}{AB}$ jest równa

A. 2 B. -21 C. $\sqrt{}$2 D. --$\sqrt{}$22

Zadanie 13. $(1pkt)$

Liczby: $x-2$, 6, 12, w podanej kolejności,

geometrycznego. Liczba $x$ jest równa

są trzema

kolejnymi wyrazami

ciągu

A. 0

B. 2

C. 3

D. 5

Zadanie 14. $(1pkt)$

$\mathrm{J}\mathrm{e}\dot{\mathrm{z}}$ eli $\alpha$ jest kątem ostrym oraz $\displaystyle \mathrm{t}\mathrm{g}\alpha=\frac{2}{5}$, to wartość wyrazenia $\displaystyle \frac{3\cos\alpha-2\sin\alpha}{\sin\alpha-5\cos\alpha}$ jest równa

A.

$-\displaystyle \frac{11}{23}$

B.

$\displaystyle \frac{24}{5}$

C.

- -2131

D.

$\displaystyle \frac{5}{24}$

Zadanie 15. (1pkt)

Liczba punktów wspólnych okręgu

współrzędnychjest równa

A. 0 B. 1

o równaniu $(x+2)^{2}+(y-3)^{2}=4$

C. 2 D.

z osiami układu

4

Zadanie 16. $(1pkt)$

Wysokość trapezu równoramiennego o kącie ostrym $60^{\mathrm{o}}$ i ramieniu długości $2\sqrt{3}$ jest równa

A. $\sqrt{3}$ B. 3 C. $2\sqrt{3}$ D. 2

Zadanie 17. $(1pkt)$

Kąt środkowy oparty na iuku, którego diugośćjest równa $\displaystyle \frac{4}{9}$ diugości okręgu, ma miarę

A. $160^{\mathrm{o}}$

B. $80^{\mathrm{o}}$

C. $40^{\mathrm{o}}$

D. $20^{\mathrm{o}}$

Zadanie 18. $(1pkt)$

$\mathrm{O}$ funkcji liniowej $f$ wiadomo, $\dot{\mathrm{z}}\mathrm{e}f(1)=2$. Do wykresu tej funkcji nalez$\mathrm{y}$ punkt $P=(-2,3).$

Wzór funkcji $f$ to

A. $f(x)=-\displaystyle \frac{1}{3}x+\frac{7}{3}$ B. $f(x)=-\displaystyle \frac{1}{2}x+2$ C. $f(x)=-3x+7$ D. $f(x)=-2x+4$

Zadanie 19. $(1pkt)$

$\mathrm{J}\mathrm{e}\dot{\mathrm{z}}$ eli ostrosłup ma 10 krawędzi, to 1iczba ścian bocznychjest równa

A. 5

B. 7

C. 8

D. 10
\end{document}
