\documentclass[a4paper,12pt]{article}
\usepackage{latexsym}
\usepackage{amsmath}
\usepackage{amssymb}
\usepackage{graphicx}
\usepackage{wrapfig}
\pagestyle{plain}
\usepackage{fancybox}
\usepackage{bm}

\begin{document}

{\it 8}

{\it Egzamin maturalny z matematyki}

{\it Poziom podstawowy}

Zadanie 20. (1pkt)

Stozek i walec mają takie same podstawy i równe pola powierzchni bocznych. Wtedy

tworząca stozka jest

A. sześć razy dłuzsza od wysokości walca.

B. trzy razy dłuzsza od wysokości walca.

C. dwa razy dłuzsza od wysokości walca.

D. równa wysokości walca.

Zadanie 21. $(1pkt)$

Liczba $(\displaystyle \frac{1}{(\sqrt[3]{729}+\sqrt[4]{256}+2)^{0}})^{-2}$ jest równa

A. $\displaystyle \frac{1}{225}$ B. $\displaystyle \frac{1}{15}$

C. l

D. 15

Zadanie 22. $(1pkt)$

Do wykresu ffinkcji, określonej dla wszystkich liczb rzeczywistych wzorem $y=-2^{x-2}$, nalez$\mathrm{y}$

punkt

A. $A=(1,-2)$ B. $B=(2,-1)$ C. $C=(1,\displaystyle \frac{1}{2})$ D. $D=(4,4)$

Zadanie 23. $(1pkt)$

$\mathrm{J}\mathrm{e}\dot{\mathrm{z}}$ eli $A$ jest zdarzeniem losowym, $\mathrm{a}$

zachodzi równość $P(A)=2\cdot P(A^{\uparrow})$, to

A. $P(A)=\displaystyle \frac{2}{3}$ B. $P(A)=\displaystyle \frac{1}{2}$

A ` -zdarzeniem przeciwnym do zdarzenia A oraz

C. $P(A)=\displaystyle \frac{1}{3}$ D. $P(A)=\displaystyle \frac{1}{6}$

Zadanie 24. (1pkt)

Na ile sposobów mozna wybrać dwóch graczy spośród 10 zawodników?

A. 100 B. 90 C. 45 D.

20

Zadanie 25. $(1pkt)$

Mediana zestawu danych 2, 12, $a$, 10, 5, 3 jest równa 7. Wówczas

A. $a=4$ B. $a=6$ C. $a=7$

D. $a=9$
\end{document}
