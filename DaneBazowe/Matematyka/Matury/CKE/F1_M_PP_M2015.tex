\documentclass[a4paper,12pt]{article}
\usepackage{latexsym}
\usepackage{amsmath}
\usepackage{amssymb}
\usepackage{graphicx}
\usepackage{wrapfig}
\pagestyle{plain}
\usepackage{fancybox}
\usepackage{bm}

\begin{document}

Arkusz zawiera informacje prawnie chronione do momentu rozpoczęcia egzaminu.

UZUPELNIA ZDAJACY

KOD PESEL

{\it Miejsce}

{\it na naklejkę}

{\it z kodem}
\begin{center}
\includegraphics[width=21.432mm,height=9.852mm]{./F1_M_PP_M2015_page0_images/image001.eps}

\includegraphics[width=82.092mm,height=9.852mm]{./F1_M_PP_M2015_page0_images/image002.eps}
\end{center}
\fbox{} dysleksja
\begin{center}
\includegraphics[width=204.060mm,height=216.048mm]{./F1_M_PP_M2015_page0_images/image003.eps}
\end{center}
EGZAMIN MATU LNY

Z MATEMATYKI

POZIOM PODSTAWOWY  5 MAJA 20I5

Instrukcja dla zdającego

l. Sprawdzí, czy arkusz egzaminacyjny zawiera 24 strony

(zadania $1-34$). Ewentualny brak zgłoś przewodniczącemu

zespo nadzorującego egzamin.

2. Rozwiązania zadań i odpowiedzi wpisuj w miejscu na to

przeznaczonym.

3. Odpowiedzi do zadań za niętych (l-25) przenieś

na ka ę odpowiedzi, zaznaczając je w części ka $\mathrm{y}$

przeznaczonej dla zdającego. Zamaluj $\blacksquare$ pola do tego

przeznaczone. Błędne zaznaczenie otocz kółkiem \fcircle$\bullet$

i zaznacz właściwe.

4. Pamiętaj, $\dot{\mathrm{z}}\mathrm{e}$ pominięcie argumentacji lub istotnych

obliczeń w rozwiązaniu zadania otwartego (26-34) $\mathrm{m}\mathrm{o}\dot{\mathrm{z}}\mathrm{e}$

spowodować, $\dot{\mathrm{z}}\mathrm{e}$ za to rozwiązanie nie będziesz mógł

dostać pełnej liczby punktów.

5. Pisz czytelnie i $\mathrm{u}\dot{\mathrm{z}}$ aj tvlko długopisu lub -Dióra

z czamym tuszem lub atramentem.

6. Nie uzywaj korektora, a błędne zapisy wyra $\acute{\mathrm{z}}\mathrm{n}\mathrm{i}\mathrm{e}$ prze eśl.

7. Pamiętaj, $\dot{\mathrm{z}}\mathrm{e}$ zapisy w brudnopisie nie będą oceniane.

8. $\mathrm{M}\mathrm{o}\dot{\mathrm{z}}$ esz korzystać z zestawu wzorów matematycznych,

cyrkla i linijki oraz kalkulatora prostego.

9. Na tej stronie oraz na karcie odpowiedzi wpisz swój

numer PESEL i przyklej naklejkę z kodem.

10. Nie wpisuj $\dot{\mathrm{z}}$ adnych znaków w części przeznaczonej dla

egzaminatora.

Godzina rozpoczęcia:

Czas pracy:

170 minut

Liczba punktów

do uzyskania: 50

$\Vert\Vert\Vert\Vert\Vert\Vert\Vert\Vert\Vert\Vert\Vert\Vert\Vert\Vert\Vert\Vert\Vert\Vert\Vert\Vert\Vert\Vert\Vert\Vert|  \mathrm{M}\mathrm{M}\mathrm{A}-\mathrm{P}1_{-}1\mathrm{P}-152$




{\it Egzamin maturalny z matematyki}

{\it Poziom podstawowy}

{\it Wzadaniach od l. do 25. wybierz i zaznacz na karcie odpowiedzi poprawnq odpowiedzí}.

Zadanie l. (lpkt)

Cena pewnego towaru wraz z 7-procentowym podatkiem VAT jest równa 34347 zł. Cena

tego samego towaru wraz z 23-procentowym podatkiem VAT będzie równa

A. 37236 zł

B. 39842, 52 zł

C. 39483 zł

D. 42246, 81 zł

Zadanie 2. $(1pkt)$

Najmniejszą liczbą całkowitą dodatnią spełniającą nierówność $|x+4,5|\geq 6$ jest

A. $x=1$

B. $x=2$

C. $x=3$

D. $x=6$

Zadanie 3. $(1pkt)$

Liczba $2^{\frac{4}{3}}. \sqrt[3]{2^{5}}$ jest równa

A.

$2^{\frac{20}{3}}$

B. 2

C.

2-45

D. $2^{3}$

Zadanie 4. $(1pkt)$

Liczba 2 $\log_{5}10-\log_{5}4$ jest równa

A. 2 B. 1og596

C. $2\log_{5}6$

D. 5

$\mathrm{Z}\mathrm{a}\mathrm{d}\mathrm{a}\mathrm{n}\mathrm{i}\varepsilon 5. (1pkt)$

Zbiór wszystkich liczb rzeczywistych spełniających nierówność $\displaystyle \frac{3}{5}-\frac{2x}{3}\geq\frac{x}{6}$ jest przedziałem

A.

$\displaystyle \langle\frac{9}{15},+\infty)$

B.

$(-\displaystyle \infty,\frac{18}{25}\}$

C.

$\displaystyle \{\frac{1}{30},+\infty)$

D.

(-$\infty$ , -95$\rangle$

Zadanie 6. $(1pkt)$

Dziedziną funkcji $f$ określonej wzorem $f(x)=\displaystyle \frac{x+4}{x^{2}-4x}\mathrm{m}\mathrm{o}\dot{\mathrm{z}}\mathrm{e}$ być zbiór

A. wszystkich liczb rzeczywistych róznych od 0 i od 4.

B. wszystkich liczb rzeczywistych róznych od $-4$ i od 4.

C. wszystkich liczb rzeczywistych róznych od -A i od 0.

D. wszystkich liczb rzeczywistych.

$\mathrm{Z}\mathrm{a}\mathrm{d}\mathrm{a}\mathrm{n}\mathrm{i}\varepsilon 7. (1pkt)$

Rozwiązaniem równania $\displaystyle \frac{2x-4}{3-x}=\frac{4}{3}$ jest liczba

A. $x=0$

B.

$x=\displaystyle \frac{12}{5}$

C. $x=2$

Strona 2 z24

D.

{\it x}$=$ -2151

MMA-IP





{\it Egzamin maturalny z matematyki}

{\it Poziom podstawowy}

{\it BRUDNOPIS} ({\it nie podlega ocenie})

MMA-IP

Strona ll z24





{\it Egzamin maturalny z matematyki}

{\it Poziom podstawowy}

Zadanie $2\not\in. (2pki)$

Wykaz, $\dot{\mathrm{z}}\mathrm{e}$ dla $\mathrm{k}\mathrm{a}\dot{\mathrm{z}}$ dej liczby rzeczywistej $x$ i dla $\mathrm{k}\mathrm{a}\dot{\mathrm{z}}$ dej liczby rzeczywistej $y$ prawdziwajest

nierówność $4x^{2}-8xy+5y^{2}\geq 0.$

Strona 12 z24

MMA-IP





{\it Egzamin maturalny z matematyki}

{\it Poziom podstawowy}

Zadanie 27. $(2pkt)$

Rozwiąz nierówność $2x^{2}-4x\geq x-2.$

Odpowied $\acute{\mathrm{z}}$:
\begin{center}
\includegraphics[width=96.012mm,height=17.784mm]{./F1_M_PP_M2015_page12_images/image001.eps}
\end{center}
Wypelnia

egzaminator

Nr zadania

Maks. liczba kt

2

27.

2

Uzyskana liczba pkt

MMA-IP

Strona 13 z24





{\it Egzamin maturalny z matematyki}

{\it Poziom podstawowy}

Zadanie $2\mathrm{S}. (2pkt)$

Rozwiąz równanie $4x^{3}+4x^{2}-x-1=0.$

Odpowiedzí:

Strona 14 z24

MMA-IP





{\it Egzamin maturalny z matematyki}

{\it Poziom podstawowy}

Zadanie $29_{n}(2pkt)$

Na rysunku przedstawiono wykres funkcji $f.$
\begin{center}
\includegraphics[width=143.052mm,height=110.136mm]{./F1_M_PP_M2015_page14_images/image001.eps}
\end{center}
$y$

5

4

3

2

1

{\it x}

$-4$ -$3  -2$ -$1 0$  1 2  3 4  5 6

$-1$

$-2$

Funkcja $h$ określona jest dla $x\in\langle-3,  5\rangle$ wzorem $h(x)=f(x)+q$, gdzie $q$ jest pewną liczbą

rzeczywistą. Wiemy, $\dot{\mathrm{z}}$ ejednym z miejsc zerowych funkcji $h$ jest liczba $x_{0}=-1.$

a) Wyznacz q.

b) Podaj wszystkie pozostałe miejsca zerowe funkcji h.

Odpowiedzí :
\begin{center}
\includegraphics[width=96.012mm,height=23.676mm]{./F1_M_PP_M2015_page14_images/image002.eps}
\end{center}
Wypelnia

egzaminator

Nr zadania

Maks. liczba kt

28.

2

2

Uzyskana liczba pkt

Strona 15 z24

MMA-IP





{\it Egzamin maturalny z matematyki}

{\it Poziom podstawowy}

Zadanie 30. $(2pki)$

Dany jest skończony ciąg, w którym pierwszy wyraz jest równy 444, a ostatni jest

równy 653. $\mathrm{K}\mathrm{a}\dot{\mathrm{z}}\mathrm{d}\mathrm{y}$ wyraz tego ciągu, począwszy od drugiego, jest o ll większy od wyrazu

bezpośrednio go poprzedzającego. Oblicz sumę wszystkich wyrazów tego ciągu.

Odpowiedzí:

Strona 16 z24

MMA-IP





{\it Egzamin maturalny z matematyki}

{\it Poziom podstawowy}

Zadanie 31. $(2pkt)$

Dany jest okrąg o środku w punkcie $O$. Prosta $KL$ jest styczna do tego okręgu w punkcie $L,$

a środek $O$ tego okręgu lezy na odcinku KM (zob. rysunek). Udowodnij, $\dot{\mathrm{z}}\mathrm{e}$ kąt $KML$ ma

miarę $31^{\mathrm{o}}$
\begin{center}
\includegraphics[width=112.320mm,height=75.180mm]{./F1_M_PP_M2015_page16_images/image001.eps}
\end{center}
{\it L}

{\it M} ?

{\it O}

$28^{\mathrm{o}}$

{\it K}
\begin{center}
\includegraphics[width=96.012mm,height=17.784mm]{./F1_M_PP_M2015_page16_images/image002.eps}
\end{center}
Wypelnia

egzaminator

Nr zadania

Maks. liczba kt

30.

2

31.

2

Uzyskana liczba pkt

MMA-IP

Strona 17 z24





{\it Egzamin maturalny z matematyki}

{\it Poziom podstawowy}

Zadanie 32. $(4pki)$

Wysokość graniastosłupa prawidłowego czworokątnego jest równa 16. Przekątna graniastosłupa

jest nachylona do płaszczyzny jego podstawy pod kątem, którego cosinus jest równy $\displaystyle \frac{3}{5}$. Oblicz

pole powierzchni całkowitej tego graniastosłupa.

Strona 18 z24

MMA-IP





{\it Egzamin maturalny z matematyki}

{\it Poziom podstawowy}

Odpowied $\acute{\mathrm{z}}$:
\begin{center}
\includegraphics[width=82.044mm,height=17.832mm]{./F1_M_PP_M2015_page18_images/image001.eps}
\end{center}
Nr zadania

Wypelnia Maks. liczba kt

egzaminator

Uzyskana liczba pkt

32.

4

MMA-IP

Strona 19 z24





{\it Egzamin maturalny z matematyki}

{\it Poziom podstawowy}

Zadanie 33. (4pkt)

Wśród 115 osób przeprowadzono badania ankietowe, związane z zakupami w pewnym

kiosku. W ponizszej tabeli przedstawiono informacje o tym, ile osób kupiło bilety

tramwajowe ulgowe oraz ile osób kupiło bilety tramwajowe normalne.
\begin{center}
\begin{tabular}{|l|l|}
\hline
\multicolumn{1}{|l|}{$\begin{array}{l}\mbox{Rodzaj kupionych}	\\	\mbox{biletów}	\end{array}$}&	\multicolumn{1}{|l|}{Liczba osób}	\\
\hline
\multicolumn{1}{|l|}{ulgowe}&	\multicolumn{1}{|l|}{$76$}	\\
\hline
\multicolumn{1}{|l|}{normalne}&	\multicolumn{1}{|l|}{$41$}	\\
\hline
\end{tabular}

\end{center}
Uwaga! 27 osób spośród ankietowanych kupiło oba rodzaje bi1etów.

Oblicz prawdopodobieństwo zdarzenia polegającego na tym, $\dot{\mathrm{z}}\mathrm{e}$ osoba losowo wybrana

spośród ankietowanych nie kupiła $\dot{\mathrm{z}}$ adnego biletu. Wynik przedstaw w formie nieskracalnego

ułamka.

Strona 20 z24

MMA-IP





{\it Egzamin maturalny z matematyki}

{\it Poziom podstawowy}

{\it BRUDNOPIS} ({\it nie podlega ocenie})

MMA-IP

Strona 3 z24





{\it Egzamin maturalny z matematyki}

{\it Poziom podstawowy}

Odpowied $\acute{\mathrm{z}}$:
\begin{center}
\includegraphics[width=82.044mm,height=17.832mm]{./F1_M_PP_M2015_page20_images/image001.eps}
\end{center}
Wypelnia

egzaminator

Nr zadania

Maks. liczba kt

33.

4

Uzyskana liczba pkt

MMA-IP

Strona 21 z24





{\it Egzamin maturalny z matematyki}

{\it Poziom podstawowy}

Zadanie 34. $\beta 5pkt$)

Biegacz narciarski Borys wyruszył na trasę biegu o 10 minut pózíniej $\mathrm{n}\mathrm{i}\dot{\mathrm{z}}$ inny zawodnik,

Adam. Metę zawodów, po przebyciu 15-ki1ometrowej trasy biegu, obaj zawodnicy pokona1i

równocześnie. Okazało się, $\dot{\mathrm{z}}\mathrm{e}$ wartość średniej prędkości na całej trasie w przypadku Borysa

była o $4,5 \displaystyle \frac{\mathrm{k}\mathrm{m}}{\mathrm{h}}$ większa $\mathrm{n}\mathrm{i}\dot{\mathrm{z}}$ w przypadku Adama. Oblicz, wjakim czasie Adam pokonał całą

trasę biegu.

Strona 22 z24

MMA-IP





{\it Egzamin maturalny z matematyki}

{\it Poziom podstawowy}

Odpowied $\acute{\mathrm{z}}$:
\begin{center}
\includegraphics[width=82.044mm,height=17.832mm]{./F1_M_PP_M2015_page22_images/image001.eps}
\end{center}
Nr zadania

Wypelnia Maks. liczba kt

egzaminator

Uzyskana liczba pkt

34.

5

MMA-IP

Strona 23 z24





{\it Egzamin maturalny z matematyki}

{\it Poziom podstawowy}

{\it BRUDNOPIS} ({\it nie podlega ocenie})

Strona 24 z24

MMA-IP





{\it Egzamin maturalny z matematyki}

{\it Poziom podstawowy}

Zadanie 8. $(1pkt)$

Miejscem zerowym funkcji liniowej określonej wzorem $f(x)=-\displaystyle \frac{2}{3}x+4$ jest

A. 0

B. 6

C. 4

D. $-6$

ZadanÍe 9. $(1pkt)$

Punkt $M=(\displaystyle \frac{1}{2},3)$

nalezy do

wykresu funkcji

liniowej określonej

wzorem

$f(x)=(3-2a)x+2$. Wtedy

A.

{\it a}$=$- -21

B. $a=2$

C.

{\it a}$=$ -21

D. $a=-2$

Zadanie $l0. (1pkt)$

Na rysunku przedstawiono fragment prostej o równaniu $y=ax+b.$
\begin{center}
\includegraphics[width=125.328mm,height=84.840mm]{./F1_M_PP_M2015_page3_images/image001.eps}
\end{center}
{\it y}

7

6

5

$P=(2,5)$

4  $Q=(5,3)$

3

2

1

{\it x}

$-1$

0

$-1$

1 2 3 4  5 6 7 8  9 1

Współczynnik kierunkowy tej prostej jest równy

A.

{\it a}$=$- -23

B.

{\it a}$=$- -23

C.

{\it a}$=$- -25

D.

{\it a}$=$- -53

Zadanie ll. (lpkt)

$\mathrm{W}$ ciągu arytmetycznym $(a_{n})$ określonym dla

wyrazem tego ciągujest liczba 156?

$n\geq 1$ dane są $a_{1}=-4$

i

$r=2$. Którym

A. 81.

B. 80.

C. 76.

D. 77.

Zadanie 12. (1pkt)

W rosnącym ciągu geometrycznym

$(a_{n})$, określonym dla $n\geq 1$, spełniony jest warunek

$a_{4}=3a_{1}$. Iloraz $q$ tego ciągu jest równy

A.

{\it q}$=$ -31

B.

{\it q}$=$ -$\sqrt{}$313

C. $q=\sqrt[3]{3}$

Strona 4 $\mathrm{z}24$

D. $q=3$

MMA-IP





{\it Egzamin maturalny z matematyki}

{\it Poziom podstawowy}

{\it BRUDNOPIS} ({\it nie podlega ocenie})

MMA-IP

Strona 5 z24





{\it Egzamin maturalny z matematyki}

{\it Poziom podstawowy}

Zadanie 13. (1pkt)

Drabinę o długości 4 metrów oparto o pionowy mur,

w odległości 1,30 m od tego muru (zobacz rysunek).

a jej podstawę umieszczono
\begin{center}
\includegraphics[width=27.072mm,height=53.388mm]{./F1_M_PP_M2015_page5_images/image001.eps}
\end{center}
4m

$\alpha$

1,30 $\mathrm{m}$

Kąt $\alpha$, podjakim ustawiono drabinę, spełnia warunek

A. $0^{\mathrm{o}}<\alpha<30^{\mathrm{o}}$

B. $30^{\mathrm{o}}<\alpha<45^{\mathrm{o}}$

C. $45^{\mathrm{o}}<\alpha<60^{\mathrm{o}}$

D. $60^{\mathrm{o}}<\alpha<90^{\mathrm{o}}$

Zadanie 14. $(1pkt)$

Kąt $\alpha$jest ostry i $\displaystyle \sin\alpha=\frac{2}{5}$. Wówczas $\cos\alpha$ jest równy

A. -25 B. --$\sqrt{}$421 C. -53

D.

$\displaystyle \frac{\sqrt{21}}{5}$

Zadanie $15_{\mathfrak{v}}(1pkt)$

$\mathrm{W}$ trójkącie równoramiennym $ABC$ spełnione są warunki: $|AC|=|BC|, |\neq CAB|=50^{\mathrm{o}}$

Odcinek $BD$ jest dwusieczną kąta $ABC$, a odcinek $BE$ jest wysokoŚcią opuszczoną

z wierzchołka $B$ na bok $AC$. Miara kąta $EBD$ jest równa
\begin{center}
\includegraphics[width=136.200mm,height=91.392mm]{./F1_M_PP_M2015_page5_images/image002.eps}
\end{center}
{\it C}

{\it E}

{\it D}

?

$50^{\mathrm{o}}$

{\it A  B}

B. 12, $5^{\mathrm{o}}$

A. $10^{\mathrm{o}}$

C. 13, $5^{\mathrm{o}}$

D. $15^{\mathrm{o}}$

Strona 6 z24

MMA-IP





{\it Egzamin maturalny z matematyki}

{\it Poziom podstawowy}

{\it BRUDNOPIS} ({\it nie podlega ocenie})

MMA-IP

Strona 7 z24





{\it Egzamin maturalny z matematyki}

{\it Poziom podstawowy}

Zadanie $l\not\in. (1pki)$

Przedstawione na rysunku trójkąty są podobne.
\begin{center}
\includegraphics[width=60.912mm,height=31.092mm]{./F1_M_PP_M2015_page7_images/image001.eps}
\end{center}
{\it a}

4

$\alpha  \beta$
\begin{center}
\includegraphics[width=121.416mm,height=61.980mm]{./F1_M_PP_M2015_page7_images/image002.eps}
\end{center}
{\it b}

$\alpha  \beta$

6

15

12

Wówczas

A. $a=13, b=17$

B. $a=10, b=18$

C. $a=9, b=19$

D. $a=11, b=13$

Zadauie 17. $(1pkt)$

Proste o równaniach: $y=2mx-m^{2}-1$ oraz $y=4m^{2}x+m^{2}+1$ są prostopadłe dla

A. {\it m}$=$--21 B. {\it m}$=$-21 C. {\it m}$=$1 D. {\it m}$=$2

Zadanie 18. (1pkt)

Dane są punkty $M=(3,-5)$ oraz $N=(-1,7)$. Prosta przechodząca przez te punkty ma

równanie

A. $y=-3x+4$

B. $y=3x-4$

C.

$y=-\displaystyle \frac{1}{3}x+4$

D. $y=3x+4$

ZadaBie 19. $(1pkt)$

Dane są punkty: $P=(-2,-2), Q=(3$, 3$)$. Odległość punktu $P$ od punktu $Q$ jest równa

A. I B. 5 C. $5\sqrt{2}$ D. $2\sqrt{5}$

Zadanie 20. $(1pkt)$

Punkt $K=(-4,4)$ jest końcem odcinka $KL$, punkt $L$ lezy na osi $Ox$, a środek $S$ tego odcinka

lezy na osi $Oy$. Wynika stąd, $\dot{\mathrm{z}}\mathrm{e}$

A. $S=(0,2)$

B. $S=(-2,0)$

C. $S=(4,0)$

D. $S=(0,4)$

Strona 8 z24

MMA-IP





{\it Egzamin maturalny z matematyki}

{\it Poziom podstawowy}

{\it BRUDNOPIS} ({\it nie podlega ocenie})

MMA-IP

Strona 9 z24





{\it Egzamin maturalny z matematyki}

{\it Poziom podstawowy}

Zadanie 21. $(1pki)$

Okrąg przedstawiony na rysunku ma środek w punkcie $O=(3,1)$ i przechodzi przez punkty

$S=(0,4)\mathrm{i}T=(0,-2)$. Okrąg tenjest opisany przez równanie
\begin{center}
\includegraphics[width=99.420mm,height=89.460mm]{./F1_M_PP_M2015_page9_images/image001.eps}
\end{center}
{\it y}

6

5

4 {\it S}

3

2

1

{\it O}

{\it x}

0

1

1 2  3 4 5 6  8

$-2$  {\it T}

A. $(x+3)^{2}+(y+1)^{2}=18$

B. $(x-3)^{2}+(y+1)^{2}=18$

C. $(x-3)^{2}+(y-1)^{2}=18$

D. $(x+3)^{2}+(y-1)^{2}=18$

Zadanie 22. (1pkt)

Przekątna ściany sześcianu ma długość 2. Po1e powierzchni całkowitej tego sześcianu jest

równe

A. 24

B. $12\sqrt{2}$

C. 12

D. $16\sqrt{2}$

Zadanie 23. $(1pkt)$

Kula o promieniu 5 cm i stozek o promieniu podstawy

Wysokość stozkajest równa

A. $\displaystyle \frac{25}{\pi}$ cm B. $10\mathrm{c}\mathrm{m}$ C. $\displaystyle \frac{10}{\pi}$ cm

10 cm mają równe objętości.

D. 5 cm

Zadanie 24. (1pki)

Średnia arytmetyczna zestawu danych:

2, 4, 7, 8, 9

jest taka sama jak średnia arytmetyczna zestawu danych:

2, 4, 7, 8, 9, $x.$

Wynika stąd, $\dot{\mathrm{z}}\mathrm{e}$

A. $x=0$

B. $x=3$

C. $x=5$

D. $x=6$

Zadanie $25_{\mathfrak{v}}(1pkt)$

$\mathrm{W}$ pewnej klasie stosunek liczby dziewcząt do liczby chłopców jest równy 4: 5. Losujemy

jedną osobę z tej klasy. Prawdopodobieństwo tego, $\dot{\mathrm{z}}\mathrm{e}$ będzie to dziewczyna, jest równe

A. -45 B. -49 C. -41 D. -91 MMA-1P

Strona 10 $\mathrm{z}24$



\end{document}