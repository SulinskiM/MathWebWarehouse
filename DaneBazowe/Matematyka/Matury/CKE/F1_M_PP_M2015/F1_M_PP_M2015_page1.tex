\documentclass[a4paper,12pt]{article}
\usepackage{latexsym}
\usepackage{amsmath}
\usepackage{amssymb}
\usepackage{graphicx}
\usepackage{wrapfig}
\pagestyle{plain}
\usepackage{fancybox}
\usepackage{bm}

\begin{document}

{\it Egzamin maturalny z matematyki}

{\it Poziom podstawowy}

{\it Wzadaniach od l. do 25. wybierz i zaznacz na karcie odpowiedzi poprawnq odpowiedzí}.

Zadanie l. (lpkt)

Cena pewnego towaru wraz z 7-procentowym podatkiem VAT jest równa 34347 zł. Cena

tego samego towaru wraz z 23-procentowym podatkiem VAT będzie równa

A. 37236 zł

B. 39842, 52 zł

C. 39483 zł

D. 42246, 81 zł

Zadanie 2. $(1pkt)$

Najmniejszą liczbą całkowitą dodatnią spełniającą nierówność $|x+4,5|\geq 6$ jest

A. $x=1$

B. $x=2$

C. $x=3$

D. $x=6$

Zadanie 3. $(1pkt)$

Liczba $2^{\frac{4}{3}}. \sqrt[3]{2^{5}}$ jest równa

A.

$2^{\frac{20}{3}}$

B. 2

C.

2-45

D. $2^{3}$

Zadanie 4. $(1pkt)$

Liczba 2 $\log_{5}10-\log_{5}4$ jest równa

A. 2 B. 1og596

C. $2\log_{5}6$

D. 5

$\mathrm{Z}\mathrm{a}\mathrm{d}\mathrm{a}\mathrm{n}\mathrm{i}\varepsilon 5. (1pkt)$

Zbiór wszystkich liczb rzeczywistych spełniających nierówność $\displaystyle \frac{3}{5}-\frac{2x}{3}\geq\frac{x}{6}$ jest przedziałem

A.

$\displaystyle \langle\frac{9}{15},+\infty)$

B.

$(-\displaystyle \infty,\frac{18}{25}\}$

C.

$\displaystyle \{\frac{1}{30},+\infty)$

D.

(-$\infty$ , -95$\rangle$

Zadanie 6. $(1pkt)$

Dziedziną funkcji $f$ określonej wzorem $f(x)=\displaystyle \frac{x+4}{x^{2}-4x}\mathrm{m}\mathrm{o}\dot{\mathrm{z}}\mathrm{e}$ być zbiór

A. wszystkich liczb rzeczywistych róznych od 0 i od 4.

B. wszystkich liczb rzeczywistych róznych od $-4$ i od 4.

C. wszystkich liczb rzeczywistych róznych od -A i od 0.

D. wszystkich liczb rzeczywistych.

$\mathrm{Z}\mathrm{a}\mathrm{d}\mathrm{a}\mathrm{n}\mathrm{i}\varepsilon 7. (1pkt)$

Rozwiązaniem równania $\displaystyle \frac{2x-4}{3-x}=\frac{4}{3}$ jest liczba

A. $x=0$

B.

$x=\displaystyle \frac{12}{5}$

C. $x=2$

Strona 2 z24

D.

{\it x}$=$ -2151

MMA-IP
\end{document}
