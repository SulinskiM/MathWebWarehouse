\documentclass[a4paper,12pt]{article}
\usepackage{latexsym}
\usepackage{amsmath}
\usepackage{amssymb}
\usepackage{graphicx}
\usepackage{wrapfig}
\pagestyle{plain}
\usepackage{fancybox}
\usepackage{bm}

\begin{document}

{\it Egzamin maturalny z matematyki}

{\it Poziom podstawowy}

Zadanie $29_{n}(2pkt)$

Na rysunku przedstawiono wykres funkcji $f.$
\begin{center}
\includegraphics[width=143.052mm,height=110.136mm]{./F1_M_PP_M2015_page14_images/image001.eps}
\end{center}
$y$

5

4

3

2

1

{\it x}

$-4$ -$3  -2$ -$1 0$  1 2  3 4  5 6

$-1$

$-2$

Funkcja $h$ określona jest dla $x\in\langle-3,  5\rangle$ wzorem $h(x)=f(x)+q$, gdzie $q$ jest pewną liczbą

rzeczywistą. Wiemy, $\dot{\mathrm{z}}$ ejednym z miejsc zerowych funkcji $h$ jest liczba $x_{0}=-1.$

a) Wyznacz q.

b) Podaj wszystkie pozostałe miejsca zerowe funkcji h.

Odpowiedzí :
\begin{center}
\includegraphics[width=96.012mm,height=23.676mm]{./F1_M_PP_M2015_page14_images/image002.eps}
\end{center}
Wypelnia

egzaminator

Nr zadania

Maks. liczba kt

28.

2

2

Uzyskana liczba pkt

Strona 15 z24

MMA-IP
\end{document}
