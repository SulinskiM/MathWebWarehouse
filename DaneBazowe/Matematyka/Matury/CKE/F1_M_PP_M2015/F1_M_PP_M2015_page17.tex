\documentclass[a4paper,12pt]{article}
\usepackage{latexsym}
\usepackage{amsmath}
\usepackage{amssymb}
\usepackage{graphicx}
\usepackage{wrapfig}
\pagestyle{plain}
\usepackage{fancybox}
\usepackage{bm}

\begin{document}

{\it Egzamin maturalny z matematyki}

{\it Poziom podstawowy}

Zadanie 32. $(4pki)$

Wysokość graniastosłupa prawidłowego czworokątnego jest równa 16. Przekątna graniastosłupa

jest nachylona do płaszczyzny jego podstawy pod kątem, którego cosinus jest równy $\displaystyle \frac{3}{5}$. Oblicz

pole powierzchni całkowitej tego graniastosłupa.

Strona 18 z24

MMA-IP
\end{document}
