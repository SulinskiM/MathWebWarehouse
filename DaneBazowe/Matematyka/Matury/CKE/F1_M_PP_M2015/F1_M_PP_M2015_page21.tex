\documentclass[a4paper,12pt]{article}
\usepackage{latexsym}
\usepackage{amsmath}
\usepackage{amssymb}
\usepackage{graphicx}
\usepackage{wrapfig}
\pagestyle{plain}
\usepackage{fancybox}
\usepackage{bm}

\begin{document}

{\it Egzamin maturalny z matematyki}

{\it Poziom podstawowy}

Zadanie 34. $\beta 5pkt$)

Biegacz narciarski Borys wyruszył na trasę biegu o 10 minut pózíniej $\mathrm{n}\mathrm{i}\dot{\mathrm{z}}$ inny zawodnik,

Adam. Metę zawodów, po przebyciu 15-ki1ometrowej trasy biegu, obaj zawodnicy pokona1i

równocześnie. Okazało się, $\dot{\mathrm{z}}\mathrm{e}$ wartość średniej prędkości na całej trasie w przypadku Borysa

była o $4,5 \displaystyle \frac{\mathrm{k}\mathrm{m}}{\mathrm{h}}$ większa $\mathrm{n}\mathrm{i}\dot{\mathrm{z}}$ w przypadku Adama. Oblicz, wjakim czasie Adam pokonał całą

trasę biegu.

Strona 22 z24

MMA-IP
\end{document}
