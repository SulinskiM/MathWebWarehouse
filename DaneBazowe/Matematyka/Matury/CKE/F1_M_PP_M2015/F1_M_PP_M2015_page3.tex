\documentclass[a4paper,12pt]{article}
\usepackage{latexsym}
\usepackage{amsmath}
\usepackage{amssymb}
\usepackage{graphicx}
\usepackage{wrapfig}
\pagestyle{plain}
\usepackage{fancybox}
\usepackage{bm}

\begin{document}

{\it Egzamin maturalny z matematyki}

{\it Poziom podstawowy}

Zadanie 8. $(1pkt)$

Miejscem zerowym funkcji liniowej określonej wzorem $f(x)=-\displaystyle \frac{2}{3}x+4$ jest

A. 0

B. 6

C. 4

D. $-6$

ZadanÍe 9. $(1pkt)$

Punkt $M=(\displaystyle \frac{1}{2},3)$

nalezy do

wykresu funkcji

liniowej określonej

wzorem

$f(x)=(3-2a)x+2$. Wtedy

A.

{\it a}$=$- -21

B. $a=2$

C.

{\it a}$=$ -21

D. $a=-2$

Zadanie $l0. (1pkt)$

Na rysunku przedstawiono fragment prostej o równaniu $y=ax+b.$
\begin{center}
\includegraphics[width=125.328mm,height=84.840mm]{./F1_M_PP_M2015_page3_images/image001.eps}
\end{center}
{\it y}

7

6

5

$P=(2,5)$

4  $Q=(5,3)$

3

2

1

{\it x}

$-1$

0

$-1$

1 2 3 4  5 6 7 8  9 1

Współczynnik kierunkowy tej prostej jest równy

A.

{\it a}$=$- -23

B.

{\it a}$=$- -23

C.

{\it a}$=$- -25

D.

{\it a}$=$- -53

Zadanie ll. (lpkt)

$\mathrm{W}$ ciągu arytmetycznym $(a_{n})$ określonym dla

wyrazem tego ciągujest liczba 156?

$n\geq 1$ dane są $a_{1}=-4$

i

$r=2$. Którym

A. 81.

B. 80.

C. 76.

D. 77.

Zadanie 12. (1pkt)

W rosnącym ciągu geometrycznym

$(a_{n})$, określonym dla $n\geq 1$, spełniony jest warunek

$a_{4}=3a_{1}$. Iloraz $q$ tego ciągu jest równy

A.

{\it q}$=$ -31

B.

{\it q}$=$ -$\sqrt{}$313

C. $q=\sqrt[3]{3}$

Strona 4 $\mathrm{z}24$

D. $q=3$

MMA-IP
\end{document}
