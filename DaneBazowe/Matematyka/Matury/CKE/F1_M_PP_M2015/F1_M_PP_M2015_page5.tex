\documentclass[a4paper,12pt]{article}
\usepackage{latexsym}
\usepackage{amsmath}
\usepackage{amssymb}
\usepackage{graphicx}
\usepackage{wrapfig}
\pagestyle{plain}
\usepackage{fancybox}
\usepackage{bm}

\begin{document}

{\it Egzamin maturalny z matematyki}

{\it Poziom podstawowy}

Zadanie 13. (1pkt)

Drabinę o długości 4 metrów oparto o pionowy mur,

w odległości 1,30 m od tego muru (zobacz rysunek).

a jej podstawę umieszczono
\begin{center}
\includegraphics[width=27.072mm,height=53.388mm]{./F1_M_PP_M2015_page5_images/image001.eps}
\end{center}
4m

$\alpha$

1,30 $\mathrm{m}$

Kąt $\alpha$, podjakim ustawiono drabinę, spełnia warunek

A. $0^{\mathrm{o}}<\alpha<30^{\mathrm{o}}$

B. $30^{\mathrm{o}}<\alpha<45^{\mathrm{o}}$

C. $45^{\mathrm{o}}<\alpha<60^{\mathrm{o}}$

D. $60^{\mathrm{o}}<\alpha<90^{\mathrm{o}}$

Zadanie 14. $(1pkt)$

Kąt $\alpha$jest ostry i $\displaystyle \sin\alpha=\frac{2}{5}$. Wówczas $\cos\alpha$ jest równy

A. -25 B. --$\sqrt{}$421 C. -53

D.

$\displaystyle \frac{\sqrt{21}}{5}$

Zadanie $15_{\mathfrak{v}}(1pkt)$

$\mathrm{W}$ trójkącie równoramiennym $ABC$ spełnione są warunki: $|AC|=|BC|, |\neq CAB|=50^{\mathrm{o}}$

Odcinek $BD$ jest dwusieczną kąta $ABC$, a odcinek $BE$ jest wysokoŚcią opuszczoną

z wierzchołka $B$ na bok $AC$. Miara kąta $EBD$ jest równa
\begin{center}
\includegraphics[width=136.200mm,height=91.392mm]{./F1_M_PP_M2015_page5_images/image002.eps}
\end{center}
{\it C}

{\it E}

{\it D}

?

$50^{\mathrm{o}}$

{\it A  B}

B. 12, $5^{\mathrm{o}}$

A. $10^{\mathrm{o}}$

C. 13, $5^{\mathrm{o}}$

D. $15^{\mathrm{o}}$

Strona 6 z24

MMA-IP
\end{document}
