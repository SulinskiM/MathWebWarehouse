\documentclass[a4paper,12pt]{article}
\usepackage{latexsym}
\usepackage{amsmath}
\usepackage{amssymb}
\usepackage{graphicx}
\usepackage{wrapfig}
\pagestyle{plain}
\usepackage{fancybox}
\usepackage{bm}

\begin{document}

{\it Egzamin maturalny z matematyki}

{\it Poziom podstawowy}

Zadanie $l\not\in. (1pki)$

Przedstawione na rysunku trójkąty są podobne.
\begin{center}
\includegraphics[width=60.912mm,height=31.092mm]{./F1_M_PP_M2015_page7_images/image001.eps}
\end{center}
{\it a}

4

$\alpha  \beta$
\begin{center}
\includegraphics[width=121.416mm,height=61.980mm]{./F1_M_PP_M2015_page7_images/image002.eps}
\end{center}
{\it b}

$\alpha  \beta$

6

15

12

Wówczas

A. $a=13, b=17$

B. $a=10, b=18$

C. $a=9, b=19$

D. $a=11, b=13$

Zadauie 17. $(1pkt)$

Proste o równaniach: $y=2mx-m^{2}-1$ oraz $y=4m^{2}x+m^{2}+1$ są prostopadłe dla

A. {\it m}$=$--21 B. {\it m}$=$-21 C. {\it m}$=$1 D. {\it m}$=$2

Zadanie 18. (1pkt)

Dane są punkty $M=(3,-5)$ oraz $N=(-1,7)$. Prosta przechodząca przez te punkty ma

równanie

A. $y=-3x+4$

B. $y=3x-4$

C.

$y=-\displaystyle \frac{1}{3}x+4$

D. $y=3x+4$

ZadaBie 19. $(1pkt)$

Dane są punkty: $P=(-2,-2), Q=(3$, 3$)$. Odległość punktu $P$ od punktu $Q$ jest równa

A. I B. 5 C. $5\sqrt{2}$ D. $2\sqrt{5}$

Zadanie 20. $(1pkt)$

Punkt $K=(-4,4)$ jest końcem odcinka $KL$, punkt $L$ lezy na osi $Ox$, a środek $S$ tego odcinka

lezy na osi $Oy$. Wynika stąd, $\dot{\mathrm{z}}\mathrm{e}$

A. $S=(0,2)$

B. $S=(-2,0)$

C. $S=(4,0)$

D. $S=(0,4)$

Strona 8 z24

MMA-IP
\end{document}
