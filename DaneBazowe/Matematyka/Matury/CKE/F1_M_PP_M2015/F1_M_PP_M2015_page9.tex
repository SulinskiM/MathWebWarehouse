\documentclass[a4paper,12pt]{article}
\usepackage{latexsym}
\usepackage{amsmath}
\usepackage{amssymb}
\usepackage{graphicx}
\usepackage{wrapfig}
\pagestyle{plain}
\usepackage{fancybox}
\usepackage{bm}

\begin{document}

{\it Egzamin maturalny z matematyki}

{\it Poziom podstawowy}

Zadanie 21. $(1pki)$

Okrąg przedstawiony na rysunku ma środek w punkcie $O=(3,1)$ i przechodzi przez punkty

$S=(0,4)\mathrm{i}T=(0,-2)$. Okrąg tenjest opisany przez równanie
\begin{center}
\includegraphics[width=99.420mm,height=89.460mm]{./F1_M_PP_M2015_page9_images/image001.eps}
\end{center}
{\it y}

6

5

4 {\it S}

3

2

1

{\it O}

{\it x}

0

1

1 2  3 4 5 6  8

$-2$  {\it T}

A. $(x+3)^{2}+(y+1)^{2}=18$

B. $(x-3)^{2}+(y+1)^{2}=18$

C. $(x-3)^{2}+(y-1)^{2}=18$

D. $(x+3)^{2}+(y-1)^{2}=18$

Zadanie 22. (1pkt)

Przekątna ściany sześcianu ma długość 2. Po1e powierzchni całkowitej tego sześcianu jest

równe

A. 24

B. $12\sqrt{2}$

C. 12

D. $16\sqrt{2}$

Zadanie 23. $(1pkt)$

Kula o promieniu 5 cm i stozek o promieniu podstawy

Wysokość stozkajest równa

A. $\displaystyle \frac{25}{\pi}$ cm B. $10\mathrm{c}\mathrm{m}$ C. $\displaystyle \frac{10}{\pi}$ cm

10 cm mają równe objętości.

D. 5 cm

Zadanie 24. (1pki)

Średnia arytmetyczna zestawu danych:

2, 4, 7, 8, 9

jest taka sama jak średnia arytmetyczna zestawu danych:

2, 4, 7, 8, 9, $x.$

Wynika stąd, $\dot{\mathrm{z}}\mathrm{e}$

A. $x=0$

B. $x=3$

C. $x=5$

D. $x=6$

Zadanie $25_{\mathfrak{v}}(1pkt)$

$\mathrm{W}$ pewnej klasie stosunek liczby dziewcząt do liczby chłopców jest równy 4: 5. Losujemy

jedną osobę z tej klasy. Prawdopodobieństwo tego, $\dot{\mathrm{z}}\mathrm{e}$ będzie to dziewczyna, jest równe

A. -45 B. -49 C. -41 D. -91 MMA-1P

Strona 10 $\mathrm{z}24$
\end{document}
