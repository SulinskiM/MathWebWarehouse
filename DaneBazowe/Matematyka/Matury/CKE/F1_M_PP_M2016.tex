\documentclass[a4paper,12pt]{article}
\usepackage{latexsym}
\usepackage{amsmath}
\usepackage{amssymb}
\usepackage{graphicx}
\usepackage{wrapfig}
\pagestyle{plain}
\usepackage{fancybox}
\usepackage{bm}

\begin{document}

Arkusz zawiera informacje prawnie chronione do momentu rozpoczęcia egzaminu.

UZUPELNIA ZDAJACY

KOD PESEL

{\it miejsce}

{\it na naklejkę}
\begin{center}
\includegraphics[width=21.432mm,height=9.852mm]{./F1_M_PP_M2016_page0_images/image001.eps}

\includegraphics[width=82.092mm,height=9.852mm]{./F1_M_PP_M2016_page0_images/image002.eps}
\end{center}
\square  dyskalkulia

\fbox{} dysleksja
\begin{center}
\includegraphics[width=204.060mm,height=216.048mm]{./F1_M_PP_M2016_page0_images/image003.eps}
\end{center}
EGZAMIN MATU

Z MATEMATY

LNY

POZIOM PODSTAWOWY  5 MAJA 20I

Instrukcja dla zdającego

l. Sprawdzí, czy arkusz egzaminacyjny zawiera 24 strony

(zadania $1-34$). Ewentualny brak zgłoś przewodniczącemu

zespo nadzorującego egzamin.

2. Rozwiązania zadań i odpowiedzi wpisuj w miejscu na to

przeznaczonym.

3. Odpowiedzi do zadań zamkniętych $(1-25)$ zaznacz

na karcie odpowiedzi, w części ka $\mathrm{y}$ przeznaczonej dla

zdającego. Zamaluj $\blacksquare$ pola do tego przeznaczone. Błędne

zaznaczenie otocz kółkiem $\mathrm{O}$ i zaznacz właściwe.

4. Pamiętaj, $\dot{\mathrm{z}}\mathrm{e}$ pominięcie argumentacji lub istotnych

obliczeń w rozwiązaniu zadania otwa ego (26-34) $\mathrm{m}\mathrm{o}\dot{\mathrm{z}}\mathrm{e}$

spowodować, $\dot{\mathrm{z}}\mathrm{e}$ za to rozwiązanie nie będziesz mógł

dostać pełnej liczby punktów.

5. Pisz czytelnie i $\mathrm{u}\dot{\mathrm{z}}$ aj tylko $\mathrm{d}$ gopisu lub pióra

z czarnym tuszem lub atramentem.

6. Nie $\mathrm{u}\dot{\mathrm{z}}$ aj korektora, a błędne zapisy wyra $\acute{\mathrm{z}}\mathrm{n}\mathrm{i}\mathrm{e}$ prze eśl.

7. Pamiętaj, $\dot{\mathrm{z}}\mathrm{e}$ zapisy w brudnopisie nie będą oceniane.

8. $\mathrm{M}\mathrm{o}\dot{\mathrm{z}}$ esz korzystać z zestawu wzorów matematycznych,

cyrkla i linijki oraz kalkulatora prostego.

9. Na tej stronie oraz na karcie odpowiedzi wpisz swój

numer PESEL i przyklej naklejkę z kodem.

10. Nie wpisuj $\dot{\mathrm{z}}$ adnych znaków w części przeznaczonej dla

egzaminatora.

Godzina rozpoczęcia:

Czas pracy:

170 minut

Liczba punktów

do uzyskania: 50

$\Vert\Vert\Vert\Vert\Vert\Vert\Vert\Vert\Vert\Vert\Vert\Vert\Vert\Vert\Vert\Vert\Vert\Vert\Vert\Vert\Vert\Vert\Vert\Vert|  \mathrm{M}\mathrm{M}\mathrm{A}-\mathrm{P}1_{-}1\mathrm{P}-162$




{\it Egzamin maturalny z matematyki}

{\it Poziom podstawowy}

ZADANIA ZAMKNIĘTE

{\it Wzadaniach od l. do 25. wybierz i zaznacz na karcie odpowiedzipoprawnq} $odp\theta wied\acute{z}.$

Zadanie l, (l pkţ)

Dla $\mathrm{k}\mathrm{a}\dot{\mathrm{z}}$ dej dodatniej liczby $a$ iloraz $\displaystyle \frac{a^{-2,6}}{a^{1,3}}$ jest równy

A.

$a^{-3,9}$

B.

$a^{-2}$

C.

$a^{-1,3}$

D.

$a^{1,3}$

Zadanie 2. $(1pkt)$

Liczba $\log_{\sqrt{2}}(2\sqrt{2})$ jest równa

A.

-23

B. 2

C.

-25

D. 3

Zadanie 3. $(1pkt)$

Liczby $a\mathrm{i}c$ są dodatnie. Liczba $b$ stanowi 48\% 1iczby $a$ oraz 32\% 1iczby $c$. Wynika stąd, $\dot{\mathrm{z}}\mathrm{e}$

A. $c=1,5a$

B. $c=1,6a$

C. $c=0,8a$

D. $c=0,16a$

ZadanÍe 4. $(1pkt)$

RównoŚć $(2\sqrt{2}-a)^{2}=17-12\sqrt{2}$ jest prawdziwa dla

A. $a=3$

B. $a=1$

C. $a=-2$

D. $a=-3$

Zadanie 5. $(1pktJ$

Jedną z liczb, które spełniają nierówność $-x^{5}+x^{3}-x<-2$, jest

A. l

B. $-1$

C. 2

D. $-2$

Zadanie $\epsilon. (1pkt)$

Proste o równaniach $2x-3y=4\mathrm{i}5x-6y=7$ przecinają się w punkcie $P$. Stąd wynika, $\dot{\mathrm{z}}\mathrm{e}$

A. $P=(1,2)$

B. $P=(-1,2)$

C. $P=(-1,-2)$

D. $P=(1,-2)$

ZadanÍe 7. (1pkt)

Punkty ABCD $\mathrm{l}\mathrm{e}\dot{\mathrm{z}}$ ą na o ęgu o środku $S$ (zobacz

Miara kąta $BDC$ jest równa

A. $91^{\mathrm{o}}$

sunek).

B. $72,5^{\mathrm{o}}$
\begin{center}
\includegraphics[width=78.132mm,height=79.452mm]{./F1_M_PP_M2016_page1_images/image001.eps}
\end{center}
{\it D}

{\it C}

$27^{\mathrm{o}}$

{\it S}

18

{\it B}

{\it A}

Strona 2 z24

D. $32^{\mathrm{o}}$

C. $18^{\mathrm{o}}$

MMA-IP





{\it Egzamin maturalny z matematyki}

{\it Poziom podstawowy}

{\it BRUDNOPIS} ({\it nie podlega ocenie})

MMA-IP

Strona ll z24





{\it Egzamin maturalny z matematyki}

{\it Poziom podstawowy}

ZADANIA OTWARTE

{\it Rozwiqzania zadań o numerach od 26. do 34. nalezy zapisać w wyznaczonych miejscach pod treściq}

{\it zadania}.

Zadanie 26. (2pkt)

Rozwiąz nierównoŚć $2x^{2}+5x-3>0.$

Odpowied $\acute{\mathrm{z}}$:

Strona 12 $\mathrm{z}24$

MMA-IP





{\it Egzamin maturalny z matematyki}

{\it Poziom podstawowy}

Zadanie 27. (2pkt)

Rozwiąz równanie $x^{3}+3x^{2}+2x+6=0.$

Odpowied $\acute{\mathrm{z}}$:
\begin{center}
\includegraphics[width=96.012mm,height=17.832mm]{./F1_M_PP_M2016_page12_images/image001.eps}
\end{center}
Wypelnia

egzaminator

Nr zadania

Maks. liczba kt

2

27.

2

Uzyskana liczba pkt

MMA-IP

Strona 13 z24





{\it Egzamin maturalny z matematyki}

{\it Poziom podstawowy}

Zadanie 28. (2pktJ

Kąt $\alpha$ jest ostry $\displaystyle \mathrm{i}(\sin\alpha+\cos\alpha)^{2}=\frac{3}{2}$. Oblicz wartość wyrazenia $\sin\alpha\cdot\cos\alpha.$

Odpowiedzí :

Strona 14 z24

MMA-IP





{\it Egzamin maturalny z matematyki}

{\it Poziom podstawowy}

Zadanie 29. (2pkt)

Dany jest trójkąt prostokątny $ABC$. Na przyprostokątnych $AC\mathrm{i}$ AB tego trójkąta obrano

odpowiednio punkty $D\mathrm{i}G$. Na przeciwprostokątnej $BC$ wyznaczono punkty $E\mathrm{i}F$ takie, $\dot{\mathrm{z}}\mathrm{e}$

$|\wedge DEC|=|\triangleleft BGF|=90^{\mathrm{o}}$ (zobacz rysunek). Wykaz, $\dot{\mathrm{z}}\mathrm{e}$ trójkąt $CDE$ jest podobny do

trójkąta $FBG.$
\begin{center}
\includegraphics[width=87.528mm,height=55.476mm]{./F1_M_PP_M2016_page14_images/image001.eps}
\end{center}
{\it C}

{\it E}

{\it F}

{\it D}

{\it A  G B}
\begin{center}
\includegraphics[width=96.012mm,height=17.784mm]{./F1_M_PP_M2016_page14_images/image002.eps}
\end{center}
Wypelnia

egzaminator

Nr zadania

Maks. liczba kt

28.

2

2

Uzyskana liczba pkt

MMA-IP

Strona 15 z24





{\it Egzamin maturalny z matematyki}

{\it Poziom podstawowy}

Zadanie 30. (2pktJ

Ciąg $(a_{n})$ jest określony wzorem $a_{n}=2n^{2}+2n$ dla $n\geq 1$. Wykaz, $\dot{\mathrm{z}}\mathrm{e}$ suma $\mathrm{k}\mathrm{a}\dot{\mathrm{z}}$ dych dwóch

kolejnych wyrazów tego ciągu jest kwadratem liczby naturalnej.

Strona 16 z24

MMA-IP





{\it Egzamin maturalny z matematyki}

{\it Poziom podstawowy}

{\it Zadanie 3l}. ({\it 2pktJ}

$\mathrm{W}$ skończonym ciągu arytmetycznym $(a_{n})$ pierwszy wyraz $a_{1}$ jest równy 7 oraz ostatni

wyraz $a_{n}$ jest równy 89. Suma wszystkich wyrazów tego ciągujest równa 2016.

Oblicz, ile wyrazów ma ten ciąg.

Odpowied $\acute{\mathrm{z}}$:
\begin{center}
\includegraphics[width=96.012mm,height=17.832mm]{./F1_M_PP_M2016_page16_images/image001.eps}
\end{center}
Wypelnia

egzaminator

Nr zadania

Maks. liczba kt

30.

2

31.

2

Uzyskana liczba pkt

MMA-IP

Strona 17 z24





{\it Egzamin maturalny z matematyki}

{\it Poziom podstawowy}

Zadanie 32. (4pktJ

Jeden z kątów trójkąta jest trzy razy większy od mniejszego z dwóch pozostałych kątów,

które róznią się o $50^{\mathrm{o}}$. Oblicz kąty tego trójkąta.

Strona 18 z24

MMA-IP





{\it Egzamin maturalny z matematyki}

{\it Poziom podstawowy}

Odpowiedzí :
\begin{center}
\includegraphics[width=82.044mm,height=17.784mm]{./F1_M_PP_M2016_page18_images/image001.eps}
\end{center}
Nr zadanÍa

WypelnÍa Maks. liczba kt

egzaminator

Uzyskana liczba pkt

32.

4

MMA-IP

Strona 19 z24





{\it Egzamin maturalny z matematyki}

{\it Poziom podstawowy}

Zadanie 33. (5pktJ

Grupa znajomych wyjez $\mathrm{d}\dot{\mathrm{z}}$ ających na biwak wynajęła bus. Koszt wynajęcia busa jest równy

960 złotych i tę kwotę rozłozono po równo pomiędzy uczestników wyjazdu. Do grupy

wyjez $\mathrm{d}\dot{\mathrm{z}}$ ających dołączyło w ostatniej chwili dwóch znajomych. Wtedy koszt wyjazdu

przypadający na jednego uczestnika zmniejszył się o 16 złotych. Ob1icz, i1e osób wyjechało

na biwak.

Strona 20 z24

MMA-IP





{\it Egzamin maturalny z matematyki}

{\it Poziom podstawowy}

{\it BRUDNOPIS} ({\it nie podlega ocenie})

MMA-IP





{\it Egzamin maturalny z matematyki}

{\it Poziom podstawowy}
\begin{center}
\includegraphics[width=82.044mm,height=17.832mm]{./F1_M_PP_M2016_page20_images/image001.eps}
\end{center}
WypelnÍa

egzaminator

Nr zadanÍa

Maks. liczba kt

33.

5

Uzyskana liczba pkt

MMA-IP

Strona 21 z24





{\it Egzamin maturalny z matematyki}

{\it Poziom podstawowy}

Zadanie 34. (4pktJ

Ze zbioru wszystkich liczb naturalnych dwucyfrowych losujemy kolejno dwa razy po jednej

liczbie bez zwracania. Oblicz prawdopodobieństwo zdarzenia polegającego na tym, $\dot{\mathrm{z}}\mathrm{e}$ suma

wylosowanych liczb będzie równa 30. Wynik zapisz w postaci ułamka zwykłego

nieskracalnego.

Strona 22 z24

MMA-IP





{\it Egzamin maturalny z matematyki}

{\it Poziom podstawowy}

Odpowiedzí :
\begin{center}
\includegraphics[width=82.044mm,height=17.832mm]{./F1_M_PP_M2016_page22_images/image001.eps}
\end{center}
Wypelnia

egzaminator

Nr zadania

Maks. liczba kt

34.

4

Uzyskana liczba pkt

MMA-IP

Strona 23 z24





{\it Egzamin maturalny z matematyki}

{\it Poziom podstawowy}

{\it BRUDNOPIS} ({\it nie podlega ocenie})

Strona 24 z24

MMA-IP





{\it Egzamin maturalny z matematyki}

{\it Poziom podstawowy}

Zadam$\mathrm{e}8. (1pkt)$

Danajest ffinkcja liniowa $f(x)=\displaystyle \frac{3}{4}x+6$. Miejscem zerowym tej funkcjijest liczba

A. 8

B. 6

C. $-6$

D. $-8$

Zadanie $g. (1pktJ$

Równanie wymierne $\displaystyle \frac{3x-1}{x+5}=3$, gdzie $x\neq-5,$

A.

B.

C.

D.

nie ma rozwiązań rzeczywistych.

ma dokładniejedno rozwiązanie rzeczywiste.

ma dokładnie dwa rozwiązania rzeczywiste.

ma dokładnie trzy rozwiązania rzeczywiste.

Informacja do zadań 10. $\mathrm{i}l1.$

Na rysunku przedstawiony jest fragment paraboli będącej wykresem funkcji kwadratowej $f.$

Wierzchołkiem tej parabolijest punkt $W=(1,9)$. Liczby $-2\mathrm{i}4$ to miejsca zerowe funkcji $f.$
\begin{center}
\includegraphics[width=192.228mm,height=118.104mm]{./F1_M_PP_M2016_page3_images/image001.eps}
\end{center}
Zadanie 10. (1pkt)

Zbiorem wartości funkcji f jest przedział

A.

$(-\infty'-2\rangle$

B. $\langle-2,  4\rangle$

C.

$\langle 4,+\infty)$

D. $(-\infty$' $ 9\rangle$

Zadanie $ll. (1pkt)$

Najmniejsza wartość funkcji $f$ w przedziale $\langle-1,2\rangle$ jest równa

A. 2

B. 5

C. 8

D. 9

Strona 4 z24

MMA-IP





{\it Egzamin maturalny z matematyki}

{\it Poziom podstawowy}

{\it BRUDNOPIS} ({\it nie podlega ocenie})

MMA-IP

Strona 5 z24





{\it Egzamin maturalny z matematyki}

{\it Poziom podstawowy}

Zadanie $l2. (1pkt)$

Funkcja $f$ określona jest wzorem $f(x)=\displaystyle \frac{2x^{3}}{x^{6}+1}$ dla kazdej liczby rzeczywistej $x$. Wtedy

$f(-\sqrt[3]{3})$ jest równa

A.

$-\displaystyle \frac{\sqrt[3]{9}}{2}$

B.

- -53

C.

-53

D.

$\displaystyle \frac{\sqrt[3]{3}}{2}$

Zadanie 13. $(1pktJ$

$\mathrm{W}$ okręgu o środku w punkcie $S$ poprowadzono cięciwę AB, która utworzyła z promieniem

$AS$ kąt o mierze $31^{\mathrm{o}}$ (zobacz rysunek). Promień tego okręgu ma długość 10. Od1egłość punktu

$S$ od cięciwy $AB$ jest liczbą z przedziału

A. $\displaystyle \{\frac{9}{2},\frac{11}{2}\}$

B. $\displaystyle \frac{11}{2}, \displaystyle \frac{13}{2}$

C. $\displaystyle \frac{13}{2}, \displaystyle \frac{19}{2}$
\begin{center}
\includegraphics[width=72.588mm,height=76.200mm]{./F1_M_PP_M2016_page5_images/image001.eps}
\end{center}
$B$

{\it K}

{\it S}

31

{\it A}

$\displaystyle \frac{19}{2}, \displaystyle \frac{37}{2}$

D.

Zadanie 14. $(1pkt)$

Czternasty wyraz ciągu arytmetycznegojest równy 8, a róznica tego ciągujest równa $(-\displaystyle \frac{3}{2}).$

Siódmy wyraz tego ciągu jest równy

A.

$\displaystyle \frac{37}{2}$

B.

$-\displaystyle \frac{37}{2}$

C.

- -25

D.

-25

Zadanie 15. $(1pki)$

Ciąg $(x,2x+3,4x+3)$ jest geometryczny. Pierwszy wyraz tego ciągujest równy

A. $-4$

B. l

C. 0

D. $-1$

Zadanie 16. (1pkt)

Przedstawione na rysunku trójkąty ABCi PQR są podobne. Bok AB trójkąta ABC ma długość

A. 8

B. 8,5

C. 9,5
\begin{center}
\includegraphics[width=105.660mm,height=60.504mm]{./F1_M_PP_M2016_page5_images/image002.eps}
\end{center}
18

{\it Q}  $62^{\mathrm{o}}$  {\it R}

{\it C}

17

9

$70^{\mathrm{o}}$

$70^{\mathrm{o}}  48^{\mathrm{o}}$

{\it A B}

{\it x  P}

D. 10

Strona 6 z24

MMA-IP





{\it Egzamin maturalny z matematyki}

{\it Poziom podstawowy}

{\it BRUDNOPIS} ({\it nie podlega ocenie})

MMA-IP

Strona 7 z24





{\it Egzamin maturalny z matematyki}

{\it Poziom podstawowy}

Zadanie 17. $(1pkt)$

Kąt $\alpha$ jest ostry i $\displaystyle \mathrm{t}\mathrm{g}\alpha=\frac{2}{3}$. Wtedy

A.

$\displaystyle \sin\alpha=\frac{3\sqrt{13}}{26}$

B.

$\displaystyle \sin\alpha=\frac{\sqrt{13}}{13}$

C.

$\mathrm{s}$i$\displaystyle \mathrm{n}\alpha=\frac{2\sqrt{13}}{13}$

D.

$\displaystyle \sin\alpha=\frac{3\sqrt{13}}{13}$

Zadanie $l\mathrm{S}. (1pkt)$

$\mathrm{Z}$ odcinków o długościach: 5, $2a+1, a-1$ mozna zbudować trójkąt równoramienny. Wynika

stąd, $\dot{\mathrm{z}}\mathrm{e}$

A. $a=6$

B. $a=4$

C. $a=3$

D. $a=2$

Zadanie 19. (1pRt)

Okręgi o promieniach 3 i 4 są styczne zewnętrznie. Prosta styczna do okręgu

o promieniu 4 w punkcie P przechodzi przez środek okręgu o promieniu 3 (zobacz rysunek).
\begin{center}
\includegraphics[width=171.504mm,height=116.184mm]{./F1_M_PP_M2016_page7_images/image001.eps}
\end{center}
{\it P}

$O_{1}$  3 4  $O_{2}$

Pole trójkąta, którego wierzchołkami są środki okręgów i punkt styczności P, jest równe

A. 14

B. $2\sqrt{33}$

C. $4\sqrt{33}$

D. 12

Zadanie 20. $(1pkt)$

Proste opisane równaniami $y=\displaystyle \frac{2}{m-1}x+m-2$ oraz $y=mx+\displaystyle \frac{1}{m+1}$ są prostopadłe, gdy

A. $m=2$

B.

{\it m}$=$ -21

C.

{\it m}$=$ -31

D. $m=-2$

Strona 8 z24

MMA-IP





{\it Egzamin maturalny z matematyki}

{\it Poziom podstawowy}

{\it BRUDNOPIS} ({\it nie podlega ocenie})

MMA-IP

Strona 9 z24





{\it Egzamin maturalny z matematyki}

{\it Poziom podstawowy}

Zadanie $2l. (1pkt)$

$\mathrm{W}$ układzie współrzędnych dane są punkty $A=(a,6)$ oraz $B=(7,b)$. Środkiem odcinka $AB$

jest punkt $M=(3,4)$. Wynika stąd, $\dot{\mathrm{z}}\mathrm{e}$

A. $a=5 \mathrm{i}b=5$

B. $a=-1 \mathrm{i}b=2$

C. $a=4\mathrm{i}b=10$

D. $a=-4 \mathrm{i}b=-2$

Zadanie 22. (Ipkt)

Rzucamy trzy razy symetryczną monetą. Niech p oznacza prawdopodobieństwo otrzymania

dokładnie dwóch orłów w tych trzech rzutach. Wtedy

A. $0\leq p<0,2$

B. $0,2\leq p\leq 0,35$

C. $0,35<p\leq 0,5$

D. $0,5<p\leq 1$

Zadanie 23. $(1pki)$

Kąt rozwarcia stozka ma miarę $120^{\mathrm{o}}$, a tworząca tego stozka ma długość 4. Objętość tego

stozkajest równa

A. $ 36\pi$

B. $ 18\pi$

C. $ 24\pi$

D. $ 8\pi$

Zadanie 24. (1pki)

Przekątna podstawy graniastosłupa prawidłowego czworokątnego jest dwa razy dłuzsza od

wysokości graniastosłupa. Graniastosłup przecięto płaszczyzną przechodzącą przez przekątną

podstawy ijeden wierzchołek drugiej podstawy (patrz rysunek).

Płaszczyzna przekroju tworzy z podstawą graniastosłupa kąt $\alpha$ o mierze

A. $30^{\mathrm{o}}$

B. $45^{\mathrm{o}}$

C. $60^{\mathrm{o}}$

D. $75^{\mathrm{o}}$

Zadanie 25. $(1pki)$

Średnia arytmetyczna sześciu liczb naturalnych: 31, 16, 25, 29, 27, $x$, jest równa $\displaystyle \frac{x}{2}$. Mediana

tych liczb jest równa

A. 26

B. 27

C. 28

D. 29

Strona 10 z24

MMA-IP



\end{document}