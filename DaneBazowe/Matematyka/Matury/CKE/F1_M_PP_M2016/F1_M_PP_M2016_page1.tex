\documentclass[a4paper,12pt]{article}
\usepackage{latexsym}
\usepackage{amsmath}
\usepackage{amssymb}
\usepackage{graphicx}
\usepackage{wrapfig}
\pagestyle{plain}
\usepackage{fancybox}
\usepackage{bm}

\begin{document}

{\it Egzamin maturalny z matematyki}

{\it Poziom podstawowy}

ZADANIA ZAMKNIĘTE

{\it Wzadaniach od l. do 25. wybierz i zaznacz na karcie odpowiedzipoprawnq} $odp\theta wied\acute{z}.$

Zadanie l, (l pkţ)

Dla $\mathrm{k}\mathrm{a}\dot{\mathrm{z}}$ dej dodatniej liczby $a$ iloraz $\displaystyle \frac{a^{-2,6}}{a^{1,3}}$ jest równy

A.

$a^{-3,9}$

B.

$a^{-2}$

C.

$a^{-1,3}$

D.

$a^{1,3}$

Zadanie 2. $(1pkt)$

Liczba $\log_{\sqrt{2}}(2\sqrt{2})$ jest równa

A.

-23

B. 2

C.

-25

D. 3

Zadanie 3. $(1pkt)$

Liczby $a\mathrm{i}c$ są dodatnie. Liczba $b$ stanowi 48\% 1iczby $a$ oraz 32\% 1iczby $c$. Wynika stąd, $\dot{\mathrm{z}}\mathrm{e}$

A. $c=1,5a$

B. $c=1,6a$

C. $c=0,8a$

D. $c=0,16a$

ZadanÍe 4. $(1pkt)$

RównoŚć $(2\sqrt{2}-a)^{2}=17-12\sqrt{2}$ jest prawdziwa dla

A. $a=3$

B. $a=1$

C. $a=-2$

D. $a=-3$

Zadanie 5. $(1pktJ$

Jedną z liczb, które spełniają nierówność $-x^{5}+x^{3}-x<-2$, jest

A. l

B. $-1$

C. 2

D. $-2$

Zadanie $\epsilon. (1pkt)$

Proste o równaniach $2x-3y=4\mathrm{i}5x-6y=7$ przecinają się w punkcie $P$. Stąd wynika, $\dot{\mathrm{z}}\mathrm{e}$

A. $P=(1,2)$

B. $P=(-1,2)$

C. $P=(-1,-2)$

D. $P=(1,-2)$

ZadanÍe 7. (1pkt)

Punkty ABCD $\mathrm{l}\mathrm{e}\dot{\mathrm{z}}$ ą na o ęgu o środku $S$ (zobacz

Miara kąta $BDC$ jest równa

A. $91^{\mathrm{o}}$

sunek).

B. $72,5^{\mathrm{o}}$
\begin{center}
\includegraphics[width=78.132mm,height=79.452mm]{./F1_M_PP_M2016_page1_images/image001.eps}
\end{center}
{\it D}

{\it C}

$27^{\mathrm{o}}$

{\it S}

18

{\it B}

{\it A}

Strona 2 z24

D. $32^{\mathrm{o}}$

C. $18^{\mathrm{o}}$

MMA-IP
\end{document}
