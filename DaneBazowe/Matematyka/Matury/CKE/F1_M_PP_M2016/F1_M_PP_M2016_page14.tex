\documentclass[a4paper,12pt]{article}
\usepackage{latexsym}
\usepackage{amsmath}
\usepackage{amssymb}
\usepackage{graphicx}
\usepackage{wrapfig}
\pagestyle{plain}
\usepackage{fancybox}
\usepackage{bm}

\begin{document}

{\it Egzamin maturalny z matematyki}

{\it Poziom podstawowy}

Zadanie 29. (2pkt)

Dany jest trójkąt prostokątny $ABC$. Na przyprostokątnych $AC\mathrm{i}$ AB tego trójkąta obrano

odpowiednio punkty $D\mathrm{i}G$. Na przeciwprostokątnej $BC$ wyznaczono punkty $E\mathrm{i}F$ takie, $\dot{\mathrm{z}}\mathrm{e}$

$|\wedge DEC|=|\triangleleft BGF|=90^{\mathrm{o}}$ (zobacz rysunek). Wykaz, $\dot{\mathrm{z}}\mathrm{e}$ trójkąt $CDE$ jest podobny do

trójkąta $FBG.$
\begin{center}
\includegraphics[width=87.528mm,height=55.476mm]{./F1_M_PP_M2016_page14_images/image001.eps}
\end{center}
{\it C}

{\it E}

{\it F}

{\it D}

{\it A  G B}
\begin{center}
\includegraphics[width=96.012mm,height=17.784mm]{./F1_M_PP_M2016_page14_images/image002.eps}
\end{center}
Wypelnia

egzaminator

Nr zadania

Maks. liczba kt

28.

2

2

Uzyskana liczba pkt

MMA-IP

Strona 15 z24
\end{document}
