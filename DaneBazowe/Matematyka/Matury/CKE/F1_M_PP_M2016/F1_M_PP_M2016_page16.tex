\documentclass[a4paper,12pt]{article}
\usepackage{latexsym}
\usepackage{amsmath}
\usepackage{amssymb}
\usepackage{graphicx}
\usepackage{wrapfig}
\pagestyle{plain}
\usepackage{fancybox}
\usepackage{bm}

\begin{document}

{\it Egzamin maturalny z matematyki}

{\it Poziom podstawowy}

{\it Zadanie 3l}. ({\it 2pktJ}

$\mathrm{W}$ skończonym ciągu arytmetycznym $(a_{n})$ pierwszy wyraz $a_{1}$ jest równy 7 oraz ostatni

wyraz $a_{n}$ jest równy 89. Suma wszystkich wyrazów tego ciągujest równa 2016.

Oblicz, ile wyrazów ma ten ciąg.

Odpowied $\acute{\mathrm{z}}$:
\begin{center}
\includegraphics[width=96.012mm,height=17.832mm]{./F1_M_PP_M2016_page16_images/image001.eps}
\end{center}
Wypelnia

egzaminator

Nr zadania

Maks. liczba kt

30.

2

31.

2

Uzyskana liczba pkt

MMA-IP

Strona 17 z24
\end{document}
