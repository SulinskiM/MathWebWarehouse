\documentclass[a4paper,12pt]{article}
\usepackage{latexsym}
\usepackage{amsmath}
\usepackage{amssymb}
\usepackage{graphicx}
\usepackage{wrapfig}
\pagestyle{plain}
\usepackage{fancybox}
\usepackage{bm}

\begin{document}

{\it Egzamin maturalny z matematyki}

{\it Poziom podstawowy}

Zadanie $2l. (1pkt)$

$\mathrm{W}$ układzie współrzędnych dane są punkty $A=(a,6)$ oraz $B=(7,b)$. Środkiem odcinka $AB$

jest punkt $M=(3,4)$. Wynika stąd, $\dot{\mathrm{z}}\mathrm{e}$

A. $a=5 \mathrm{i}b=5$

B. $a=-1 \mathrm{i}b=2$

C. $a=4\mathrm{i}b=10$

D. $a=-4 \mathrm{i}b=-2$

Zadanie 22. (Ipkt)

Rzucamy trzy razy symetryczną monetą. Niech p oznacza prawdopodobieństwo otrzymania

dokładnie dwóch orłów w tych trzech rzutach. Wtedy

A. $0\leq p<0,2$

B. $0,2\leq p\leq 0,35$

C. $0,35<p\leq 0,5$

D. $0,5<p\leq 1$

Zadanie 23. $(1pki)$

Kąt rozwarcia stozka ma miarę $120^{\mathrm{o}}$, a tworząca tego stozka ma długość 4. Objętość tego

stozkajest równa

A. $ 36\pi$

B. $ 18\pi$

C. $ 24\pi$

D. $ 8\pi$

Zadanie 24. (1pki)

Przekątna podstawy graniastosłupa prawidłowego czworokątnego jest dwa razy dłuzsza od

wysokości graniastosłupa. Graniastosłup przecięto płaszczyzną przechodzącą przez przekątną

podstawy ijeden wierzchołek drugiej podstawy (patrz rysunek).

Płaszczyzna przekroju tworzy z podstawą graniastosłupa kąt $\alpha$ o mierze

A. $30^{\mathrm{o}}$

B. $45^{\mathrm{o}}$

C. $60^{\mathrm{o}}$

D. $75^{\mathrm{o}}$

Zadanie 25. $(1pki)$

Średnia arytmetyczna sześciu liczb naturalnych: 31, 16, 25, 29, 27, $x$, jest równa $\displaystyle \frac{x}{2}$. Mediana

tych liczb jest równa

A. 26

B. 27

C. 28

D. 29

Strona 10 z24

MMA-IP
\end{document}
