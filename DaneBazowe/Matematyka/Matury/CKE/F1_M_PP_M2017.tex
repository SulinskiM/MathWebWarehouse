\documentclass[a4paper,12pt]{article}
\usepackage{latexsym}
\usepackage{amsmath}
\usepackage{amssymb}
\usepackage{graphicx}
\usepackage{wrapfig}
\pagestyle{plain}
\usepackage{fancybox}
\usepackage{bm}

\begin{document}

Arkusz zawiera informacje prawnie chronione do momentu rozpoczęcia egzaminu.

UZUPELNIA ZDAJACY

KOD PESEL

{\it miejsce}

{\it na naklejkę}
\begin{center}
\includegraphics[width=21.432mm,height=9.852mm]{./F1_M_PP_M2017_page0_images/image001.eps}

\includegraphics[width=82.092mm,height=9.852mm]{./F1_M_PP_M2017_page0_images/image002.eps}

\includegraphics[width=204.060mm,height=216.048mm]{./F1_M_PP_M2017_page0_images/image003.eps}
\end{center}
EGZAMIN MATU

Z MATEMATY

LNY

POZIOM PODSTAWOWY

Instrukcja dla zdającego

l. Sprawdzí, czy arkusz egzaminacyjny zawiera 26 stron

(zadania $1-34$). Ewentualny brak zgłoś przewodniczącemu

zespo nadzorującego egzamin.

2. Rozwiązania zadań i odpowiedzi wpisuj w miejscu na to

przeznaczonym.

3. Odpowiedzi do zadań zamkniętych $(1-25)$ zaznacz

na karcie odpowiedzi, w części ka $\mathrm{y}$ przeznaczonej dla

zdającego. Zamaluj $\blacksquare$ pola do tego przeznaczone. Błędne

zaznaczenie otocz kółkiem $\mathrm{O}$ i zaznacz właściwe.

4. Pamiętaj, $\dot{\mathrm{z}}\mathrm{e}$ pominięcie argumentacji lub istotnych

obliczeń w rozwiązaniu zadania otwa ego (26-34) $\mathrm{m}\mathrm{o}\dot{\mathrm{z}}\mathrm{e}$

spowodować, $\dot{\mathrm{z}}\mathrm{e}$ za to rozwiązanie nie otrzymasz pełnej

liczby punktów.

5. Pisz czytelnie i $\mathrm{u}\dot{\mathrm{z}}$ aj tylko $\mathrm{d}$ gopisu lub pióra

z czarnym tuszem lub atramentem.

6. Nie $\mathrm{u}\dot{\mathrm{z}}$ aj korektora, a błędne zapisy wyra $\acute{\mathrm{z}}\mathrm{n}\mathrm{i}\mathrm{e}$ prze eśl.

7. Pamiętaj, $\dot{\mathrm{z}}\mathrm{e}$ zapisy w brudnopisie nie będą oceniane.

8. $\mathrm{M}\mathrm{o}\dot{\mathrm{z}}$ esz korzystać z zestawu wzorów matematycznych,

cyrkla i linijki oraz kalkulatora prostego.

9. Na tej stronie oraz na karcie odpowiedzi wpisz swój

numer PESEL i przyklej naklejkę z kodem.

10. Nie wpisuj $\dot{\mathrm{z}}$ adnych znaków w części przeznaczonej dla

egzaminatora.

5 MAJA 20I7

Godzina rozpoczęcia:

9:00

Czas pracy:

170 minut

Liczba punktów

do uzyskania: 50

$\Vert\Vert\Vert\Vert\Vert\Vert\Vert\Vert\Vert\Vert\Vert\Vert\Vert\Vert\Vert\Vert\Vert\Vert\Vert\Vert\Vert\Vert\Vert\Vert|  \mathrm{M}\mathrm{M}\mathrm{A}-\mathrm{P}1_{-}1\mathrm{P}-172$




{\it Egzamin maturalny z matematyki}

{\it Poziom podstawowy}

ZADANIA ZAMKNIĘTE

{\it Wzadaniach od l. do 25. wybierz i zaznacz na karcie odpowiedzipoprawnq} $odp\theta wied\acute{z}.$

Zadanie l. $(1pktJ$

Liczba $5^{8}.16^{-2}$ jest równa

A. $(\displaystyle \frac{5}{2})^{8}$ B.

-25

Zadanie 2. $(1pktJ$

Liczba $\sqrt[3]{54}-\sqrt[3]{2}$ jest równa

A. $\sqrt[3]{52}$

B. 3

Zadanie 3. $(1pktJ$

Liczba 2 $\log_{2}3-2\log_{2}5$ jest równa

A.

$\displaystyle \log_{2}\frac{9}{25}$

B.

$\log_{2} \displaystyle \frac{3}{5}$

C. $10^{8}$

D. 10

C. $\mathrm{z}\sqrt[3]{2}$

D. 2

C.

$\log_{2} \displaystyle \frac{9}{5}$

D.

$\displaystyle \log_{2}\frac{6}{25}$

Zadanie 4. (1pktJ

Liczba osobników pewnego zagrozonego wyginięciem gatunku zwierząt wzrosła w stosunku

do liczby tych zwierząt z 31 grudnia 2011 r. 0120\% i obecnie jest równa 8910. I1e zwierząt

liczyła populacja tego gatunku w ostatnim dniu 2011 roku?

A. 4050

B. 1782

C. 7425

D. 7128

Zadame 5. $(1pkt)$

Równość $(x\sqrt{2}-2)^{2}=(2+\sqrt{2})^{2}$ jest

A. prawdziwa dla $x=-\sqrt{2}.$

B. prawdziwa dla $x=\sqrt{2}.$

C. prawdziwa dla $x=-1.$

D. fałszywa dla $\mathrm{k}\mathrm{a}\dot{\mathrm{z}}$ dej liczby $x.$

Strona 2 z 26

MMA-IP





{\it Egzamin maturalny z matematyki}

{\it Poziom podstawowy}

{\it BRUDNOPIS} ({\it nie podlega ocenie})

MMA-IP

Strona ll z 26





{\it Egzamin maturalny z matematyki}

{\it Poziom podstawowy}

Zadanie 18. $(1pkt)$

Na rysunku przedstawiona jest prosta $k$ o równaniu $y=ax$, przechodząca przez punkt

$A=(2,-3)$ i przez początek układu współrzędnych, oraz zaznaczony jest kąt $\alpha$ nachylenia

tej prostej do osi $Ox.$
\begin{center}
\includegraphics[width=70.716mm,height=67.512mm]{./F1_M_PP_M2017_page11_images/image001.eps}
\end{center}
{\it k}

{\it y}

5

4

3

2

1

$\alpha$

{\it x}

$-5$ -$4  -3$ -$2$

$-1 0$ 1

$-1$

2 3  4 5

$-2$

$-3  -A$

$-4$

Zatem

A.

{\it a}$=$- -23

B.

{\it a}$=$- -23

C.

{\it a}$=$ -23

D.

{\it a}$=$ -23

Zadanie $l9*(1pkt)$

Na płaszczyz$\acute{}$nie z układem współrzędnych proste $k\mathrm{i} l$ przecinają się pod kątem prostym

w punkcie $A=(-2,4)$. Prosta $k$ jest określona równaniem $y=-\displaystyle \frac{1}{4}x+\frac{7}{2}$ Zatem prostą $l$

opisuje równanie

A.

{\it y}$=$ -41 {\it x}$+$ -27

B.

{\it y}$=$- -41 {\it x}- -27

C. $y=4x-12$

D. $y=4x+12$

Zadanie 20. $(1pkt)$

Dany jest okrąg o środku $S=(2,3)$ i promieniu $r=5$. Który z podanych punktów lezy na

tym okręgu?

A. $A=(-1,7)$

B. $B=(2,-3)$

C. $C=(3,2)$

D. $D=(5,3)$

Zadanie 21. (1pkt)

Pole powierzchni całkowitej graniastosiupa prawidłowego czworokątnego, w którym

wysokość jest 3 razy dłuzsza od krawędzi podstawy, jest równe 140. Zatem krawędz$\acute{}$

podstawy tego graniastosłupajest równa

A. $\sqrt{10}$

B.

$3\sqrt{10}$

C. $\sqrt{42}$

D. $3\sqrt{42}$

Strona 12 z 26

MMA-IP





{\it Egzamin maturalny z matematyki}

{\it Poziom podstawowy}

{\it BRUDNOPIS} ({\it nie podlega ocenie})

MMA-IP

Strona 13 z 26





{\it Egzamin maturalny z matematyki}

{\it Poziom podstawowy}

Zadanie 22. (1pkt)

Promień AS podstawy walca jest równy wysokości OS tego walca. Sinus kąta OAS (zobacz

rysunek) jest równy
\begin{center}
\includegraphics[width=49.272mm,height=39.984mm]{./F1_M_PP_M2017_page13_images/image001.eps}
\end{center}
{\it O}

{\it S}

{\it A}

A.

-21

B.

-$\sqrt{}$22

C.

-$\sqrt{}$23

D. l

Zadanie 23. (1pkt)

Dany jest stozek o wysokości 4 i średnicy podstawy 12. Objętość tego stozkajest równa

A. $ 576\pi$

B. $ 192\pi$

C. $ 144\pi$

D. $ 48\pi$

Zadanie 24. (1pkt)

Średnia arytmetyczna ośmiu liczb: 3, 5, 7, 9, x, 15, 17, 19jest równa 11. Wtedy

A. $x=1$

B. $x=2$

C. $x=11$

D. $x=13$

Zadanie 25. $(1pkt)$

Ze zbioru dwudziesm czterech kolejnych liczb naturalnych od l do 241osujemy jedną 1iczbę.

Niech $A$ oznacza zdarzenie, $\dot{\mathrm{z}}\mathrm{e}$ wylosowana liczba będzie dzielnikiem liczby 24. Wtedy

prawdopodobieństwo zdarzenia $A$ jest równe

A.

-41

B.

-31

C.

-81

D.

-61

Strona 14 z26

MMA-IP





{\it Egzamin maturalny z matematyki}

{\it Poziom podstawowy}

{\it BRUDNOPIS} ({\it nie podlega ocenie})

MMA-IP

Strona 15 z 26





{\it Egzamin maturalny z matematyki}

{\it Poziom podstawowy}

Zadanie 26. $(2pktJ$

Rozwiąz nierówność $8x^{2}-72x\leq 0.$

Odpowied $\acute{\mathrm{z}}$:

Strona 16 $\mathrm{z}26$

MMA-IP





{\it Egzamin maturalny z matematyki}

{\it Poziom podstawowy}

Zadanie 27, $(2pktJ$

Wykaz, $\dot{\mathrm{z}}\mathrm{e}$ liczba $4^{2017}+4^{2018}+4^{2019}+4^{2020}$ jest podzielna przez 17.
\begin{center}
\includegraphics[width=96.012mm,height=17.784mm]{./F1_M_PP_M2017_page16_images/image001.eps}
\end{center}
Wypelnia

egzamÍnator

Nr zadania

Maks. liczba kt

2

27.

2

Uzyskana liczba pkt

MMA-IP

Strona 17 z26





{\it Egzamin maturalny z matematyki}

{\it Poziom podstawowy}

Zadanie 2{\$}. $(2pktJ$

Dane są dwa okręgi o środkach w punktach $P \mathrm{i} R$, styczne zewnętrznie w punkcie $C.$

Prosta $AB$ jest styczna do obu okręgów odpowiednio w punktach $A \mathrm{i}B$ oraz $|\triangleleft APC|=\alpha$

$\mathrm{i}|<ABC|=\beta$ (zobacz rysunek). Wykaz, $\dot{\mathrm{z}}\mathrm{e}\alpha=180^{\mathrm{o}}-2\beta.$
\begin{center}
\includegraphics[width=190.908mm,height=44.904mm]{./F1_M_PP_M2017_page17_images/image001.eps}
\end{center}
{\it P}

$\alpha$  {\it C  R}

$(\beta$

{\it A}  -{\it B}

Strona 18 z26

MMA-IP





{\it Egzamin maturalny z matematyki}

{\it Poziom podstawowy}

Zadanie 29. $(4pkt)$

Funkcja kwadratowa $f$ jest określona dla wszystkich liczb rzeczywistych $x$ wzorem

$f(x)=ax^{2}+bx+c$. Największa wartość funkcji $f$ jest równa 6 oraz $f(-6)=f(0)=\displaystyle \frac{3}{2}.$

Oblicz wartość współczynnika $a.$

Odpowiedzí :
\begin{center}
\includegraphics[width=96.012mm,height=17.784mm]{./F1_M_PP_M2017_page18_images/image001.eps}
\end{center}
Wypelnia

egzamÍnator

Nr zadani,

Maks. liczba kt

28.

2

4

Uzyskana liczba pkt

MMA-IP

Strona 19 z26





{\it Egzamin maturalny z matematyki}

{\it Poziom podstawowy}

Zadanie 30. (2pkt)

Przeciwprostokątna trójkąta prostokątnego ma długość 26 cm, a jedna z przyprostokątnych

jest o 14 cm dłuzsza od drugiej. Ob1icz obwód tego trójkąta.

Odpowied $\acute{\mathrm{z}}$:

Strona 20 $\mathrm{z}26$

MMA-IP





{\it Egzamin maturalny z matematyki}

{\it Poziom podstawowy}

{\it BRUDNOPIS} ({\it nie podlega ocenie})

MMA-IP

Strona 3 z 26





{\it Egzamin maturalny z matematyki}

{\it Poziom podstawowy}

Zadanie 31. (2pkt)

$\mathrm{W}$ ciągu arytmetycznym $(a_{n})$, określonym dla $n\geq 1$, dane są: wyraz $a_{1}=8$ i suma trzech

początkowych wyrazów tego ciągu $S_{3}=33$. Oblicz róznicę $a_{16}-a_{13}.$

Odpowiedzí :
\begin{center}
\includegraphics[width=96.012mm,height=17.784mm]{./F1_M_PP_M2017_page20_images/image001.eps}
\end{center}
Wypelnia

egzamÍnator

Nr zadani,

Maks. liczba kt

30.

2

31.

2

Uzyskana liczba pkt

MMA-IP

Strona 21 z26





{\it Egzamin maturalny z matematyki}

{\it Poziom podstawowy}

Zadanie 32. $(SpktJ$

Dane są punkty $A=(-4,0) \mathrm{i}M=(2,9)$ oraz prosta $k$ o równaniu $y=-2x+10$. Wierzchołek

$B$ trójkąta $ABC$ to punkt przecięcia prostej $k$ z osią $Ox$ układu współrzędnych, a wierzchołek

$C$ jest punktem przecięcia prostej $k$ z prostą AM. Oblicz pole trójkąta $ABC.$

Odpowied $\acute{\mathrm{z}}$:

Strona 22 $\mathrm{z}26$

MMA-IP





{\it Egzamin maturalny z matematyki}

{\it Poziom podstawowy}

Zadanie 33. $(2pkt)$

Ze zbioru wszystkich liczb naturalnych dwucyfrowych losujemy jedną liczbę. Oblicz

prawdopodobieństwo zdarzenia, $\dot{\mathrm{z}}\mathrm{e}$ wylosujemy liczbę, która jest równocześnie mniejsza od

40 i podzielna przez 3. Wynik zapisz w postaci ułamka zwykłego nieskracalnego.

Odpowiedzí:
\begin{center}
\includegraphics[width=96.012mm,height=17.784mm]{./F1_M_PP_M2017_page22_images/image001.eps}
\end{center}
Wypelnia

egzamÍnator

Nr zadania

Maks. liczba kt

32.

5

33.

2

Uzyskana liczba pkt

MMA-IP

Strona 23 z 26





{\it Egzamin maturalny z matematyki}

{\it Poziom podstawowy}

Zadanie 34. $(4pktJ$

$\mathrm{W}$ ostrosłupie prawidłowym trójkątnym wysokość ściany bocznej prostopadła do krawędzi

podstawy ostrosłupa jest równa $\displaystyle \frac{5\sqrt{3}}{4}$, a pole powierzchni bocznej tego ostrosłupa jest

równe $\displaystyle \frac{15\sqrt{3}}{4}$. Oblicz objętość tego ostrosłupa.

Strona 24 z26

MMA-IP





{\it Egzamin maturalny z matematyki}

{\it Poziom podstawowy}

Odpowied $\acute{\mathrm{z}}$:
\begin{center}
\includegraphics[width=82.044mm,height=17.832mm]{./F1_M_PP_M2017_page24_images/image001.eps}
\end{center}
Wypelnia

egzaminator

Nr zadanÍa

Maks. lÍczba kt

34.

4

Uzyskana liczba pkt

MMA-IP

Strona 25 z26





{\it Egzamin maturalny z matematyki}

{\it Poziom podstawowy}

{\it BRUDNOPIS} ({\it nie podlega ocenie})

Strona 26 z26

MMA-IP





{\it Egzamin maturalny z matematyki}

{\it Poziom podstawowy}

Zadanie 6. $(1pkt)$

Do zbioru rozwiązań nierówności $(x^{4}+1)(2-x)>0$ nie nalez$\mathrm{v}$ liczba

A. $-3$

B. $-1$

C. l

D. 3

Zadam$\mathrm{e}7. (1pkt)$

Wskaz rysunek, na którym jest przedstawiony zbiór wszystkich rozwiązań nierówności

$2-3x\geq 4.$

A.
\begin{center}
\includegraphics[width=167.940mm,height=17.676mm]{./F1_M_PP_M2017_page3_images/image001.eps}
\end{center}
-23  {\it x}

B.
\begin{center}
\includegraphics[width=167.940mm,height=17.784mm]{./F1_M_PP_M2017_page3_images/image002.eps}
\end{center}
-23  {\it x}

C.
\begin{center}
\includegraphics[width=168.000mm,height=17.832mm]{./F1_M_PP_M2017_page3_images/image003.eps}
\end{center}
- -23  {\it x}

D.
\begin{center}
\includegraphics[width=168.048mm,height=17.832mm]{./F1_M_PP_M2017_page3_images/image004.eps}
\end{center}
- -23  {\it x}

Zadanie $S, (1pktJ$

Równanie $x(x^{2}-4)(x^{2}+4)=0$ z niewiadomą $x$

A. nie ma rozwiązań w zbiorze liczb rzeczywistych.

B. ma dokładnie dwa rozwiązania w zbiorze liczb rzeczywistych.

C. ma dokładnie trzy rozwiązania w zbiorze liczb rzeczywistych.

D. ma dokładnie pięć rozwiązań w zbiorze liczb rzeczywistych.

{\it Zadanie g}. ({\it lpkt})

Miejscem zerowym funkcji liniowej

$f(x)=\sqrt{3}(x+1)-12$ jest liczba

A. $\sqrt{3}-4$

B. $-2\sqrt{3}+1$

C. $4\sqrt{3}-1$

D. $-\sqrt{3}+12$

Strona 4 z 26

MMA-IP





{\it Egzamin maturalny z matematyki}

{\it Poziom podstawowy}

{\it BRUDNOPIS} ({\it nie podlega ocenie})

MMA-IP

Strona 5 z 26





{\it Egzamin maturalny z matematyki}

{\it Poziom podstawowy}

Zadanie 10. $(1pktJ$

Na rysunku przedstawiono fragment wykresu funkcji

o miejscach zerowych: $-3 \mathrm{i}1.$

kwadratowej $f(x)=ax^{2}+bx+c,$
\begin{center}
\includegraphics[width=86.004mm,height=100.380mm]{./F1_M_PP_M2017_page5_images/image001.eps}
\end{center}
{\it 5y}

)4

3

2

1

{\it X}

$\rightarrow 2$

$-4$

$-5$

Współczynnik c we wzorze funkcji f jest równy

A. l

B. 2

C. 3

D. 4

Zadanie ll. $(Ipkt)$

Na rysunku przedstawiono fragment wykresu funkcji wykładniczej $f$ określonej wzorem

$f(x)=a^{x}$. Punkt $A=(1,2)$ nalezy do tego wykresu funkcji.
\begin{center}
\includegraphics[width=143.052mm,height=75.588mm]{./F1_M_PP_M2017_page5_images/image002.eps}
\end{center}
Podstawa a potęgijest równa

A.

- -21

B.

-21

C. $-2$

D. 2

Strona 6 z 26

MMA-IP





{\it Egzamin maturalny z matematyki}

{\it Poziom podstawowy}

{\it BRUDNOPIS} ({\it nie podlega ocenie})

MMA-IP

Strona 7 z 26





{\it Egzamin maturalny z matematyki}

{\it Poziom podstawowy}

Zadanie 12. $(1pkt)$

$\mathrm{W}$ ciągu arytmetycznym $(a_{n})$, określonym dla $n\geq 1$, dane są: $a_{1}=5, a_{2}=11$. Wtedy

A. $a_{14}=71$

B. $a_{12}=71$

C. $a_{11}=71$

D. $a_{10}=71$

Zadanie 13. $(1pkt)$

Dany jest trzywyrazowy ciąg geometryczny $($24, 6, $a-1)$. Stąd wynika, $\dot{\mathrm{z}}\mathrm{e}$

A.

{\it a}$=$ -25

B.

{\it a}$=$ -25

C.

{\it a}$=$ -23

D.

{\it a}$=$ -23

Zadanie 14. $(1pkt)$

Jeśli $m=\sin 50^{\mathrm{o}}$, to

A.

$m=\sin 40^{\mathrm{o}}$

B. $m=\cos 40^{\mathrm{o}}$

C. $m=\cos 50^{\mathrm{o}}$

D. $m=\mathrm{t}\mathrm{g}50^{\mathrm{o}}$

Zadanie 15. (I pkt)

Na okręgu o środku w punkcie O lezy punkt C (zobacz rysunek). Odcinek AB jest średnicą

tego okręgu. Zaznaczony na rysunku kąt środkowy a ma miarę
\begin{center}
\includegraphics[width=70.260mm,height=66.552mm]{./F1_M_PP_M2017_page7_images/image001.eps}
\end{center}
{\it C}

$56^{\mathrm{o}}$

{\it A}

$\alpha$

{\it O}

{\it B}

A. $116^{\mathrm{o}}$

B. $114^{\mathrm{o}}$

C. $112^{\mathrm{o}}$

D. $110^{\mathrm{o}}$

Strona 8 z 26

MMA-IP





{\it Egzamin maturalny z matematyki}

{\it Poziom podstawowy}

{\it BRUDNOPIS} ({\it nie podlega ocenie})

MMA-IP

Strona 9 z 26





{\it Egzamin maturalny z matematyki}

{\it Poziom podstawowy}

Zadanie 16. $(1pktJ$

$\mathrm{W}$ trójkącie $ABC$ punkt $D$ lezy na boku $BC$, a punkt $E$ lezy na boku $AB$. Odcinek $DE$ jest

równoległy do boku $AC$, a ponadto $|BD|=10, |BC|=12 \mathrm{i}|AC|=24$ (zobacz rysunek).
\begin{center}
\includegraphics[width=117.756mm,height=49.020mm]{./F1_M_PP_M2017_page9_images/image001.eps}
\end{center}
{\it B}

10

{\it D}

2

{\it C}

{\it E}

{\it A}

24

Długość odcinka DE jest równa

A. 22 B. 20

C. 12

D. ll

{\it Zadanie l7}. ({\it lpktJ}

Obwód trójkąta ABC, przedstawionego na rysunku, jest równy

A. $(3+\displaystyle \frac{\sqrt{3}}{2})a$
\begin{center}
\includegraphics[width=78.132mm,height=48.816mm]{./F1_M_PP_M2017_page9_images/image002.eps}
\end{center}
{\it C}

{\it a}

$30^{\mathrm{o}}$

{\it A  B}

C. $(3+\sqrt{3})a$

B. $(2+\displaystyle \frac{\sqrt{2}}{2})a$

D. $(2+\sqrt{2})a$

Strona 10 z 26

MMA-IP



\end{document}