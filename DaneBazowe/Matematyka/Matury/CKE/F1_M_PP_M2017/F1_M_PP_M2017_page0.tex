\documentclass[a4paper,12pt]{article}
\usepackage{latexsym}
\usepackage{amsmath}
\usepackage{amssymb}
\usepackage{graphicx}
\usepackage{wrapfig}
\pagestyle{plain}
\usepackage{fancybox}
\usepackage{bm}

\begin{document}

Arkusz zawiera informacje prawnie chronione do momentu rozpoczęcia egzaminu.

UZUPELNIA ZDAJACY

KOD PESEL

{\it miejsce}

{\it na naklejkę}
\begin{center}
\includegraphics[width=21.432mm,height=9.852mm]{./F1_M_PP_M2017_page0_images/image001.eps}

\includegraphics[width=82.092mm,height=9.852mm]{./F1_M_PP_M2017_page0_images/image002.eps}

\includegraphics[width=204.060mm,height=216.048mm]{./F1_M_PP_M2017_page0_images/image003.eps}
\end{center}
EGZAMIN MATU

Z MATEMATY

LNY

POZIOM PODSTAWOWY

Instrukcja dla zdającego

l. Sprawdzí, czy arkusz egzaminacyjny zawiera 26 stron

(zadania $1-34$). Ewentualny brak zgłoś przewodniczącemu

zespo nadzorującego egzamin.

2. Rozwiązania zadań i odpowiedzi wpisuj w miejscu na to

przeznaczonym.

3. Odpowiedzi do zadań zamkniętych $(1-25)$ zaznacz

na karcie odpowiedzi, w części ka $\mathrm{y}$ przeznaczonej dla

zdającego. Zamaluj $\blacksquare$ pola do tego przeznaczone. Błędne

zaznaczenie otocz kółkiem $\mathrm{O}$ i zaznacz właściwe.

4. Pamiętaj, $\dot{\mathrm{z}}\mathrm{e}$ pominięcie argumentacji lub istotnych

obliczeń w rozwiązaniu zadania otwa ego (26-34) $\mathrm{m}\mathrm{o}\dot{\mathrm{z}}\mathrm{e}$

spowodować, $\dot{\mathrm{z}}\mathrm{e}$ za to rozwiązanie nie otrzymasz pełnej

liczby punktów.

5. Pisz czytelnie i $\mathrm{u}\dot{\mathrm{z}}$ aj tylko $\mathrm{d}$ gopisu lub pióra

z czarnym tuszem lub atramentem.

6. Nie $\mathrm{u}\dot{\mathrm{z}}$ aj korektora, a błędne zapisy wyra $\acute{\mathrm{z}}\mathrm{n}\mathrm{i}\mathrm{e}$ prze eśl.

7. Pamiętaj, $\dot{\mathrm{z}}\mathrm{e}$ zapisy w brudnopisie nie będą oceniane.

8. $\mathrm{M}\mathrm{o}\dot{\mathrm{z}}$ esz korzystać z zestawu wzorów matematycznych,

cyrkla i linijki oraz kalkulatora prostego.

9. Na tej stronie oraz na karcie odpowiedzi wpisz swój

numer PESEL i przyklej naklejkę z kodem.

10. Nie wpisuj $\dot{\mathrm{z}}$ adnych znaków w części przeznaczonej dla

egzaminatora.

5 MAJA 20I7

Godzina rozpoczęcia:

9:00

Czas pracy:

170 minut

Liczba punktów

do uzyskania: 50

$\Vert\Vert\Vert\Vert\Vert\Vert\Vert\Vert\Vert\Vert\Vert\Vert\Vert\Vert\Vert\Vert\Vert\Vert\Vert\Vert\Vert\Vert\Vert\Vert|  \mathrm{M}\mathrm{M}\mathrm{A}-\mathrm{P}1_{-}1\mathrm{P}-172$
\end{document}
