\documentclass[a4paper,12pt]{article}
\usepackage{latexsym}
\usepackage{amsmath}
\usepackage{amssymb}
\usepackage{graphicx}
\usepackage{wrapfig}
\pagestyle{plain}
\usepackage{fancybox}
\usepackage{bm}

\begin{document}

{\it Egzamin maturalny z matematyki}

{\it Poziom podstawowy}

Zadanie 18. $(1pkt)$

Na rysunku przedstawiona jest prosta $k$ o równaniu $y=ax$, przechodząca przez punkt

$A=(2,-3)$ i przez początek układu współrzędnych, oraz zaznaczony jest kąt $\alpha$ nachylenia

tej prostej do osi $Ox.$
\begin{center}
\includegraphics[width=70.716mm,height=67.512mm]{./F1_M_PP_M2017_page11_images/image001.eps}
\end{center}
{\it k}

{\it y}

5

4

3

2

1

$\alpha$

{\it x}

$-5$ -$4  -3$ -$2$

$-1 0$ 1

$-1$

2 3  4 5

$-2$

$-3  -A$

$-4$

Zatem

A.

{\it a}$=$- -23

B.

{\it a}$=$- -23

C.

{\it a}$=$ -23

D.

{\it a}$=$ -23

Zadanie $l9*(1pkt)$

Na płaszczyz$\acute{}$nie z układem współrzędnych proste $k\mathrm{i} l$ przecinają się pod kątem prostym

w punkcie $A=(-2,4)$. Prosta $k$ jest określona równaniem $y=-\displaystyle \frac{1}{4}x+\frac{7}{2}$ Zatem prostą $l$

opisuje równanie

A.

{\it y}$=$ -41 {\it x}$+$ -27

B.

{\it y}$=$- -41 {\it x}- -27

C. $y=4x-12$

D. $y=4x+12$

Zadanie 20. $(1pkt)$

Dany jest okrąg o środku $S=(2,3)$ i promieniu $r=5$. Który z podanych punktów lezy na

tym okręgu?

A. $A=(-1,7)$

B. $B=(2,-3)$

C. $C=(3,2)$

D. $D=(5,3)$

Zadanie 21. (1pkt)

Pole powierzchni całkowitej graniastosiupa prawidłowego czworokątnego, w którym

wysokość jest 3 razy dłuzsza od krawędzi podstawy, jest równe 140. Zatem krawędz$\acute{}$

podstawy tego graniastosłupajest równa

A. $\sqrt{10}$

B.

$3\sqrt{10}$

C. $\sqrt{42}$

D. $3\sqrt{42}$

Strona 12 z 26

MMA-IP
\end{document}
