\documentclass[a4paper,12pt]{article}
\usepackage{latexsym}
\usepackage{amsmath}
\usepackage{amssymb}
\usepackage{graphicx}
\usepackage{wrapfig}
\pagestyle{plain}
\usepackage{fancybox}
\usepackage{bm}

\begin{document}

{\it Egzamin maturalny z matematyki}

{\it Poziom podstawowy}

Zadanie 2{\$}. $(2pktJ$

Dane są dwa okręgi o środkach w punktach $P \mathrm{i} R$, styczne zewnętrznie w punkcie $C.$

Prosta $AB$ jest styczna do obu okręgów odpowiednio w punktach $A \mathrm{i}B$ oraz $|\triangleleft APC|=\alpha$

$\mathrm{i}|<ABC|=\beta$ (zobacz rysunek). Wykaz, $\dot{\mathrm{z}}\mathrm{e}\alpha=180^{\mathrm{o}}-2\beta.$
\begin{center}
\includegraphics[width=190.908mm,height=44.904mm]{./F1_M_PP_M2017_page17_images/image001.eps}
\end{center}
{\it P}

$\alpha$  {\it C  R}

$(\beta$

{\it A}  -{\it B}

Strona 18 z26

MMA-IP
\end{document}
