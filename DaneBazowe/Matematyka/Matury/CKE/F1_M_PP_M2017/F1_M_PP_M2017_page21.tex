\documentclass[a4paper,12pt]{article}
\usepackage{latexsym}
\usepackage{amsmath}
\usepackage{amssymb}
\usepackage{graphicx}
\usepackage{wrapfig}
\pagestyle{plain}
\usepackage{fancybox}
\usepackage{bm}

\begin{document}

{\it Egzamin maturalny z matematyki}

{\it Poziom podstawowy}

Zadanie 32. $(SpktJ$

Dane są punkty $A=(-4,0) \mathrm{i}M=(2,9)$ oraz prosta $k$ o równaniu $y=-2x+10$. Wierzchołek

$B$ trójkąta $ABC$ to punkt przecięcia prostej $k$ z osią $Ox$ układu współrzędnych, a wierzchołek

$C$ jest punktem przecięcia prostej $k$ z prostą AM. Oblicz pole trójkąta $ABC.$

Odpowied $\acute{\mathrm{z}}$:

Strona 22 $\mathrm{z}26$

MMA-IP
\end{document}
