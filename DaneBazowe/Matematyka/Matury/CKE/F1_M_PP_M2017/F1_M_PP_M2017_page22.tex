\documentclass[a4paper,12pt]{article}
\usepackage{latexsym}
\usepackage{amsmath}
\usepackage{amssymb}
\usepackage{graphicx}
\usepackage{wrapfig}
\pagestyle{plain}
\usepackage{fancybox}
\usepackage{bm}

\begin{document}

{\it Egzamin maturalny z matematyki}

{\it Poziom podstawowy}

Zadanie 33. $(2pkt)$

Ze zbioru wszystkich liczb naturalnych dwucyfrowych losujemy jedną liczbę. Oblicz

prawdopodobieństwo zdarzenia, $\dot{\mathrm{z}}\mathrm{e}$ wylosujemy liczbę, która jest równocześnie mniejsza od

40 i podzielna przez 3. Wynik zapisz w postaci ułamka zwykłego nieskracalnego.

Odpowiedzí:
\begin{center}
\includegraphics[width=96.012mm,height=17.784mm]{./F1_M_PP_M2017_page22_images/image001.eps}
\end{center}
Wypelnia

egzamÍnator

Nr zadania

Maks. liczba kt

32.

5

33.

2

Uzyskana liczba pkt

MMA-IP

Strona 23 z 26
\end{document}
