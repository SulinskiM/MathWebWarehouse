\documentclass[a4paper,12pt]{article}
\usepackage{latexsym}
\usepackage{amsmath}
\usepackage{amssymb}
\usepackage{graphicx}
\usepackage{wrapfig}
\pagestyle{plain}
\usepackage{fancybox}
\usepackage{bm}

\begin{document}

{\it Egzamin maturalny z matematyki}

{\it Poziom podstawowy}

Zadanie 6. $(1pkt)$

Do zbioru rozwiązań nierówności $(x^{4}+1)(2-x)>0$ nie nalez$\mathrm{v}$ liczba

A. $-3$

B. $-1$

C. l

D. 3

Zadam$\mathrm{e}7. (1pkt)$

Wskaz rysunek, na którym jest przedstawiony zbiór wszystkich rozwiązań nierówności

$2-3x\geq 4.$

A.
\begin{center}
\includegraphics[width=167.940mm,height=17.676mm]{./F1_M_PP_M2017_page3_images/image001.eps}
\end{center}
-23  {\it x}

B.
\begin{center}
\includegraphics[width=167.940mm,height=17.784mm]{./F1_M_PP_M2017_page3_images/image002.eps}
\end{center}
-23  {\it x}

C.
\begin{center}
\includegraphics[width=168.000mm,height=17.832mm]{./F1_M_PP_M2017_page3_images/image003.eps}
\end{center}
- -23  {\it x}

D.
\begin{center}
\includegraphics[width=168.048mm,height=17.832mm]{./F1_M_PP_M2017_page3_images/image004.eps}
\end{center}
- -23  {\it x}

Zadanie $S, (1pktJ$

Równanie $x(x^{2}-4)(x^{2}+4)=0$ z niewiadomą $x$

A. nie ma rozwiązań w zbiorze liczb rzeczywistych.

B. ma dokładnie dwa rozwiązania w zbiorze liczb rzeczywistych.

C. ma dokładnie trzy rozwiązania w zbiorze liczb rzeczywistych.

D. ma dokładnie pięć rozwiązań w zbiorze liczb rzeczywistych.

{\it Zadanie g}. ({\it lpkt})

Miejscem zerowym funkcji liniowej

$f(x)=\sqrt{3}(x+1)-12$ jest liczba

A. $\sqrt{3}-4$

B. $-2\sqrt{3}+1$

C. $4\sqrt{3}-1$

D. $-\sqrt{3}+12$

Strona 4 z 26

MMA-IP
\end{document}
