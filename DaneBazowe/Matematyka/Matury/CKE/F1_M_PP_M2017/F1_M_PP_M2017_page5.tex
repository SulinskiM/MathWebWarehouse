\documentclass[a4paper,12pt]{article}
\usepackage{latexsym}
\usepackage{amsmath}
\usepackage{amssymb}
\usepackage{graphicx}
\usepackage{wrapfig}
\pagestyle{plain}
\usepackage{fancybox}
\usepackage{bm}

\begin{document}

{\it Egzamin maturalny z matematyki}

{\it Poziom podstawowy}

Zadanie 10. $(1pktJ$

Na rysunku przedstawiono fragment wykresu funkcji

o miejscach zerowych: $-3 \mathrm{i}1.$

kwadratowej $f(x)=ax^{2}+bx+c,$
\begin{center}
\includegraphics[width=86.004mm,height=100.380mm]{./F1_M_PP_M2017_page5_images/image001.eps}
\end{center}
{\it 5y}

)4

3

2

1

{\it X}

$\rightarrow 2$

$-4$

$-5$

Współczynnik c we wzorze funkcji f jest równy

A. l

B. 2

C. 3

D. 4

Zadanie ll. $(Ipkt)$

Na rysunku przedstawiono fragment wykresu funkcji wykładniczej $f$ określonej wzorem

$f(x)=a^{x}$. Punkt $A=(1,2)$ nalezy do tego wykresu funkcji.
\begin{center}
\includegraphics[width=143.052mm,height=75.588mm]{./F1_M_PP_M2017_page5_images/image002.eps}
\end{center}
Podstawa a potęgijest równa

A.

- -21

B.

-21

C. $-2$

D. 2

Strona 6 z 26

MMA-IP
\end{document}
