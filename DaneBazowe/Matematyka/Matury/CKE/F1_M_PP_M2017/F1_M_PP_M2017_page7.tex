\documentclass[a4paper,12pt]{article}
\usepackage{latexsym}
\usepackage{amsmath}
\usepackage{amssymb}
\usepackage{graphicx}
\usepackage{wrapfig}
\pagestyle{plain}
\usepackage{fancybox}
\usepackage{bm}

\begin{document}

{\it Egzamin maturalny z matematyki}

{\it Poziom podstawowy}

Zadanie 12. $(1pkt)$

$\mathrm{W}$ ciągu arytmetycznym $(a_{n})$, określonym dla $n\geq 1$, dane są: $a_{1}=5, a_{2}=11$. Wtedy

A. $a_{14}=71$

B. $a_{12}=71$

C. $a_{11}=71$

D. $a_{10}=71$

Zadanie 13. $(1pkt)$

Dany jest trzywyrazowy ciąg geometryczny $($24, 6, $a-1)$. Stąd wynika, $\dot{\mathrm{z}}\mathrm{e}$

A.

{\it a}$=$ -25

B.

{\it a}$=$ -25

C.

{\it a}$=$ -23

D.

{\it a}$=$ -23

Zadanie 14. $(1pkt)$

Jeśli $m=\sin 50^{\mathrm{o}}$, to

A.

$m=\sin 40^{\mathrm{o}}$

B. $m=\cos 40^{\mathrm{o}}$

C. $m=\cos 50^{\mathrm{o}}$

D. $m=\mathrm{t}\mathrm{g}50^{\mathrm{o}}$

Zadanie 15. (I pkt)

Na okręgu o środku w punkcie O lezy punkt C (zobacz rysunek). Odcinek AB jest średnicą

tego okręgu. Zaznaczony na rysunku kąt środkowy a ma miarę
\begin{center}
\includegraphics[width=70.260mm,height=66.552mm]{./F1_M_PP_M2017_page7_images/image001.eps}
\end{center}
{\it C}

$56^{\mathrm{o}}$

{\it A}

$\alpha$

{\it O}

{\it B}

A. $116^{\mathrm{o}}$

B. $114^{\mathrm{o}}$

C. $112^{\mathrm{o}}$

D. $110^{\mathrm{o}}$

Strona 8 z 26

MMA-IP
\end{document}
