\documentclass[a4paper,12pt]{article}
\usepackage{latexsym}
\usepackage{amsmath}
\usepackage{amssymb}
\usepackage{graphicx}
\usepackage{wrapfig}
\pagestyle{plain}
\usepackage{fancybox}
\usepackage{bm}

\begin{document}

{\it Egzamin maturalny z matematyki}

{\it Poziom podstawowy}

Zadanie 16. $(1pktJ$

$\mathrm{W}$ trójkącie $ABC$ punkt $D$ lezy na boku $BC$, a punkt $E$ lezy na boku $AB$. Odcinek $DE$ jest

równoległy do boku $AC$, a ponadto $|BD|=10, |BC|=12 \mathrm{i}|AC|=24$ (zobacz rysunek).
\begin{center}
\includegraphics[width=117.756mm,height=49.020mm]{./F1_M_PP_M2017_page9_images/image001.eps}
\end{center}
{\it B}

10

{\it D}

2

{\it C}

{\it E}

{\it A}

24

Długość odcinka DE jest równa

A. 22 B. 20

C. 12

D. ll

{\it Zadanie l7}. ({\it lpktJ}

Obwód trójkąta ABC, przedstawionego na rysunku, jest równy

A. $(3+\displaystyle \frac{\sqrt{3}}{2})a$
\begin{center}
\includegraphics[width=78.132mm,height=48.816mm]{./F1_M_PP_M2017_page9_images/image002.eps}
\end{center}
{\it C}

{\it a}

$30^{\mathrm{o}}$

{\it A  B}

C. $(3+\sqrt{3})a$

B. $(2+\displaystyle \frac{\sqrt{2}}{2})a$

D. $(2+\sqrt{2})a$

Strona 10 z 26

MMA-IP
\end{document}
