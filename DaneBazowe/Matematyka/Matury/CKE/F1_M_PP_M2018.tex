\documentclass[a4paper,12pt]{article}
\usepackage{latexsym}
\usepackage{amsmath}
\usepackage{amssymb}
\usepackage{graphicx}
\usepackage{wrapfig}
\pagestyle{plain}
\usepackage{fancybox}
\usepackage{bm}

\begin{document}

CENTRALNA

KOMISJA

EGZAMINACYJNA

Arkusz zawiera informacje prawnie chronione do momentu rozpoczęcia egzaminu.

UZUPELNIA ZDAJACY

KOD PESEL

{\it miejsce}

{\it na naklejkę}
\begin{center}
\includegraphics[width=21.432mm,height=9.852mm]{./F1_M_PP_M2018_page0_images/image001.eps}

\includegraphics[width=82.140mm,height=9.852mm]{./F1_M_PP_M2018_page0_images/image002.eps}

\includegraphics[width=204.060mm,height=216.048mm]{./F1_M_PP_M2018_page0_images/image003.eps}
\end{center}
EGZAMIN MATU LNY

Z MATEMATYKI

POZIOM PODSTAWOWY

Instrukcja dla zdającego

1. Sprawd $\acute{\mathrm{z}}$, czy arkusz egzaminacyjny zawiera 26 stron

(zadania $1-34$). Ewentualny brak zgłoś przewodniczącemu

zespo nadzorującego egzamin.

2. Rozwiązania zadań i odpowiedzi wpisuj w miejscu na to

przeznaczonym.

3. Odpowiedzi do zadań zam iętych $(1-25)$ zaznacz

na karcie odpowiedzi, w części ka $\mathrm{y}$ przeznaczonej dla

zdającego. Zamaluj $\blacksquare$ pola do tego przeznaczone. Błędne

zaznaczenie otocz kółkiem $\mathrm{O}$ i zaznacz właściwe.

4. Pamiętaj, $\dot{\mathrm{z}}\mathrm{e}$ pominięcie argumentacji lub istotnych

obliczeń w rozwiązaniu zadania otwa ego (26-34) $\mathrm{m}\mathrm{o}\dot{\mathrm{z}}\mathrm{e}$

spowodować, $\dot{\mathrm{z}}\mathrm{e}$ za to rozwiązanie nie otrzymasz pełnej

liczby punktów.

5. Pisz czytelnie i uzywaj tylko długopisu lub pióra

z czarnym tuszem lub atramentem.

6. Nie uzywaj korektora, a błędne zapisy wyrazínie prze eśl.

7. Pamiętaj, $\dot{\mathrm{z}}\mathrm{e}$ zapisy w brudnopisie nie będą oceniane.

8. $\mathrm{M}\mathrm{o}\dot{\mathrm{z}}$ esz korzystać z zestawu wzorów matematycznych,

cyrkla i linijki oraz kalkulatora prostego.

9. Na tej stronie oraz na karcie odpowiedzi wpisz swój

numer PESEL i przyklej naklejkę z kodem.

10. Nie wpisuj $\dot{\mathrm{z}}$ adnych znaków w części przeznaczonej dla

egzaminatora.

7 MAJA 20I8

Godzina rozpoczęcia:

Czas pracy:

170 minut

Liczba punktów

do uzyskania: 50

$\Vert\Vert\Vert\Vert\Vert\Vert\Vert\Vert\Vert\Vert\Vert\Vert\Vert\Vert\Vert\Vert\Vert\Vert\Vert\Vert\Vert\Vert\Vert\Vert|  \mathrm{M}\mathrm{M}\mathrm{A}-\mathrm{P}1_{-}1\mathrm{P}-182$




{\it Egzamin maturalny z matematyki}

{\it Poziom podstawowy}

ZADANIA ZAMKNIĘTE

$W$ {\it kazdym z zadań od l. do 25. wybierz i zaznacz na karcie odpowiedzipoprawnq odpowied} $\acute{z}.$

Zadanie l. $(1pktJ$

Liczba 2 $\log_{3}6-\log_{3}4$ jest równa

A. 4

B. 2

Zadanie 2. $(1pkt)$

Liczba $\sqrt[3]{\frac{7}{3}}\cdot\sqrt[3]{\frac{81}{56}}$ jest równa

A.

-$\sqrt{}$23

B.

$\displaystyle \frac{3}{2\sqrt[3]{21}}$

C. $2\log_{3}2$

D. $\log_{3}8$

C.

-23

D.

-49

Zadanie 3. $(1pkt)$

Dane są liczby $a=3,6\cdot 10^{-12}$ oraz $b=2,4\cdot 10^{-20}$. Wtedy iloraz $\displaystyle \frac{a}{b}$ jest równy

A. $8,64\cdot 10^{-32}$

B. $1,5\cdot 10^{-8}$

C. $1,5\cdot 10^{8}$

D. $8,64\cdot 10^{32}$

Zadame4. (1pkt)

Cena roweru po obnizce o 15\% była równa 850 zł. Przed tą obnizką rower ten kosztował

A. 865,00 zł

B. 850,15 zł

C. 1000,00 zł

D. 977,50 zł

Zadanie 5. $(1pkt)$

Zbiorem wszystkich rozwiązań nierówności $\displaystyle \frac{1-2x}{2}>\frac{1}{3}$ jest przedział

A.

(-$\infty$' -61)

B.

(-$\infty$' -23)

C.

$(\displaystyle \frac{1}{6},+\infty)$

D.

$(\displaystyle \frac{2}{3},+\infty)$

{\it Zadanie 6}. ({\it lpkt})

Funkcja kwadratowa określona jest wzorem

róznymi miejscami zerowymi ffinkcjif. Zatem

$f(x)=-2(x+3)(x-5)$. Liczby

$x_{1}, x_{2}$

są

A. $x_{1}+x_{2}=-8$

B. $x_{1}+x_{2}=-2$

C. $x_{1}+x_{2}=2$

D. $x_{1}+x_{2}=8$

Strona 2 z 26

MMA-IP





{\it Egzamin maturalny z matematyki}

{\it Poziom podstawowy}

{\it BRUDNOPIS} ({\it nie podlega ocenie})

MMA-IP

Strona ll z 26





{\it Egzamin maturalny z matematyki}

{\it Poziom podstawowy}

Zadanie 23. $(1pktJ$

$\mathrm{W}$ zestawie $\displaystyle \frac{2,2,2,\ldots,2}{m1\mathrm{i}\mathrm{c}\mathrm{z}\mathrm{b}}\frac{4,4,4,\ldots,4}{m1\mathrm{i}\mathrm{c}\mathrm{z}\mathrm{b}}$ jest $2m$ liczb $(m\geq 1)$ ` w tym $m$ liczb 2 $\mathrm{i} m$ liczb 4.

Odchylenie standardowe tego zestawu liczb jest równe

A. 2

B. l

C.

-$\sqrt{}$12

D. $\sqrt{2}$

Zadanie 24. $(1pktJ$

Ile jest wszystkich liczb naturalnych czterocyfrowych mniejszych $\mathrm{n}\mathrm{i}\dot{\mathrm{z}}$ 2018 i podzielnych

przez 5?

A. 402

B. 403

C. 203

D. 204

Zadanie 25, $(1pktJ$

$\mathrm{W}$ pudełku jest 50 kuponów, wśród których jest 15 kuponów przegrywających, a pozostałe

kupony są wygrywające. $\mathrm{Z}$ tego pudełka w sposób losowy wyciągamy jeden kupon.

Prawdopodobieństwo zdarzenia polegającego na tym, $\dot{\mathrm{z}}\mathrm{e}$ wyciągniemy kupon wygrywający, jest

równe

A.

$\displaystyle \frac{15}{35}$

B.

$\displaystyle \frac{1}{50}$

C.

$\displaystyle \frac{15}{50}$

D.

$\displaystyle \frac{35}{50}$

Strona 12 z 26

MMA-IP





{\it Egzamin maturalny z matematyki}

{\it Poziom podstawowy}

{\it BRUDNOPIS} ({\it nie podlega ocenie})

MMA-IP

Strona 13 z 26





{\it Egzamin maturalny z matematyki}

{\it Poziom podstawowy}

Zadanie 26. $(2pktJ$

Rozwiąz nierówność $2x^{2}-3x>5.$

Odpowiedzí :

Strona 14 z 26

MMA-IP





{\it Egzamin maturalny z matematyki}

{\it Poziom podstawowy}

Zadanie 27, $(2pktJ$

Rozwiąz równanie $x^{3}-7x^{2}-4x+28=0.$

Odpowiedzí :
\begin{center}
\includegraphics[width=96.012mm,height=17.832mm]{./F1_M_PP_M2018_page14_images/image001.eps}
\end{center}
Wypelnia

egzaminator

Nr zadania

Maks. liczba kt

2

27.

2

Uzyskana liczba pkt

MMA-IP

Strona 15 z 26





{\it Egzamin maturalny z matematyki}

{\it Poziom podstawowy}

Zadanie 2{\$}. $(2pktJ$

Udowodnij, $\dot{\mathrm{z}}\mathrm{e}$ dla dowolnych liczb dodatnich $a, b$ prawdziwajest nierówność

$\displaystyle \frac{1}{2a}+\frac{1}{2f_{i}}\geq\frac{2}{a+b}.$

Strona 16 z 26

MMA-IP





{\it Egzamin maturalny z matematyki}

{\it Poziom podstawowy}

Zadanie 29. $(2pktJ$

Okręgi o środkach odpowiednio $A\mathrm{i}B$ są styczne zewnętrznie i $\mathrm{k}\mathrm{a}\dot{\mathrm{z}}\mathrm{d}\mathrm{y}$ z nichjest styczny do obu

ramion danego kąta prostego (zobacz rysunek). Promień okręgu o środku $A$ jest równy 2.

{\it A}.

{\it B}.

Uzasadnij, $\dot{\mathrm{z}}\mathrm{e}$ promień okręgu o środku $B$ jest mniejszy od $\sqrt{2}-1.$
\begin{center}
\includegraphics[width=96.012mm,height=17.784mm]{./F1_M_PP_M2018_page16_images/image001.eps}
\end{center}
Wypelnia

egzaminator

Nr zadania

Maks. liczba kt

28.

2

2

Uzyskana liczba pkt

MMA-IP

Strona 17 z 26





{\it Egzamin maturalny z matematyki}

{\it Poziom podstawowy}

Zadanie 30. $(2pkt)$

Do wykresu funkcji wykładniczej, określonej

$f(x)=a^{x}$ (gdzie $a>0 \mathrm{i} a\neq 1$), nalezy punkt

ffinkcji $g$, określonej wzorem $g(x)=f(x)-2$

dla $\mathrm{k}\mathrm{a}\dot{\mathrm{z}}$ dej liczby rzeczywistej $x$ wzorem

$P=(2,9)$. Oblicz $a$ i zapisz zbiór wartości

Odpowiedzí :

Strona 18 z 26

MMA-IP





{\it Egzamin maturalny z matematyki}

{\it Poziom podstawowy}

Zadanie 31. $(2pktJ$

Dwunasty wyraz ciągu arytmetycznego $(a_{n})$, określonego dla $n\geq 1$, jest równy 30, a sumajego

dwunastu początkowych wyrazówjest równa 162. Ob1icz pierwszy wyraz tego ciągu.

Odpowiedzí :
\begin{center}
\includegraphics[width=96.012mm,height=17.832mm]{./F1_M_PP_M2018_page18_images/image001.eps}
\end{center}
Wypelnia

egzaminator

Nr zadania

Maks. liczba kt

30.

2

31.

2

Uzyskana liczba pkt

MMA-IP

Strona 19 z 26





{\it Egzamin maturalny z matematyki}

{\it Poziom podstawowy}

Zadanie 32. $(SpktJ$

$\mathrm{W}$ układzie współrzędnych punkty $A=(4,3) \mathrm{i} B=(10,5)$ są wierzchołkami trójkąta $ABC.$

Wierzchołek $C$ lezy na prostej o równaniu $y=2x+3$. Oblicz współrzędne punktu $C$, dla którego

kąt $ABC$ jest prosty.

Strona 20 z 26

MMA-IP





{\it Egzamin maturalny z matematyki}

{\it Poziom podstawowy}

{\it BRUDNOPIS} ({\it nie podlega ocenie})

MMA-IP

Strona 3 z 26





{\it Egzamin maturalny z matematyki}

{\it Poziom podstawowy}

Odpowiedzí :
\begin{center}
\includegraphics[width=82.044mm,height=17.832mm]{./F1_M_PP_M2018_page20_images/image001.eps}
\end{center}
Wypelnia

egzaminator

Nr zadania

Maks. liczba kt

32.

5

Uzyskana liczba pkt

MMA-IP

Strona 21 z 26





{\it Egzamin maturalny z matematyki}

{\it Poziom podstawowy}

Zadanie 33. $(4pktJ$

Dane są dwa zbiory: $A=\{100$, 200, 300, 400, 500, 600, 700$\} \mathrm{i} B=\{10$, 11, 12, 13, 14, 15, 16$\}.$

$\mathrm{Z}\mathrm{k}\mathrm{a}\dot{\mathrm{z}}$ dego z nich losujemyjedną liczbę. Oblicz prawdopodobieństwo zdarzenia polegającego na

tym, $\dot{\mathrm{z}}\mathrm{e}$ suma wylosowanych liczb będzie podzielna przez 3. Ob1iczone prawdopodobieństwo

zapisz w postaci nieskracalnego ułamka zwykłego.

Strona 22 z 26

MMA-IP





{\it Egzamin maturalny z matematyki}

{\it Poziom podstawowy}

Odpowiedzí :
\begin{center}
\includegraphics[width=82.044mm,height=17.832mm]{./F1_M_PP_M2018_page22_images/image001.eps}
\end{center}
Wypelnia

egzaminator

Nr zadania

Maks. liczba kt

33.

4

Uzyskana liczba pkt

MMA-IP

Strona 23 z 26





{\it Egzamin maturalny z matematyki}

{\it Poziom podstawowy}

Zadanie 34. $(4pktJ$

Dany jest graniastosłup prawidłowy trójkątny (zobacz rysunek). Pole powierzchni całkowitej

tego graniastosłupa jest równe $45\sqrt{3}$. Pole podstawy graniastosłupa jest równe polu jednej

ściany bocznej. Oblicz objętość tego graniastosłupa.
\begin{center}
\includegraphics[width=61.980mm,height=42.828mm]{./F1_M_PP_M2018_page23_images/image001.eps}
\end{center}
{\it F}

{\it E}

{\it C  D}

{\it B}

{\it A}

Strona 24 z 26

MMA-IP





{\it Egzamin maturalny z matematyki}

{\it Poziom podstawowy}

Odpowiedzí :
\begin{center}
\includegraphics[width=82.044mm,height=17.832mm]{./F1_M_PP_M2018_page24_images/image001.eps}
\end{center}
Wypelnia

egzaminator

Nr zadania

Maks. liczba kt

34.

4

Uzyskana liczba pkt

MMA-IP

Strona 25 z 26





{\it Egzamin maturalny z matematyki}

{\it Poziom podstawowy}

{\it BRUDNOPIS} ({\it nie podlega ocenie})

Strona 26 z 26

MMA-IP





{\it Egzamin maturalny z matematyki}

{\it Poziom podstawowy}

Zadanie 7. $(1pkt)$

Równanie $\displaystyle \frac{x^{2}+2x}{x^{2}-4}=0$

A. ma trzy rozwiązania: $x=-2, x=0, x=2$

B. ma dwa rozwiązania: $x=0, x=2$

C. ma dwa rozwiązania: $x=-2, x=2$

D. majedno rozwiązanie: $x=0$

Zadanie S, (lpkt)

Funkcja liniowa f określona jest wzorem

rzeczywistych x. Wskaz zdanie prawdziwe.

$f(x)=\displaystyle \frac{1}{3}x-1,$

dla wszystkich

liczb

A. Funkcja $f$ jest malejącaijej wykres przecina oś $oy$ w punkcie $P=(0,\displaystyle \frac{1}{3}).$

B. Funkcja $f$ jest malejącaijej wykres przecina oś $Oy$ w punkcie $P=(0,-1).$

C. Funkcja $f$ jest rosnąca ijej wykres przecina oś $oy$ w punkcie $P=(0,\displaystyle \frac{1}{3}).$

D. Funkcja $f$ jest rosnącaijej wykres przecina oś $Oy$ w punkcie $P=(0,-1).$

Zadam$\mathrm{e}9. (1pkt)$

Wykresem funkcji kwadratowej $f(x)=x^{2}-6x-3$ jest parabola, której wierzchołkiem jest

punkt o współrzędnych

A. $(-6,-3)$

B. $(-6,69)$

C. $(3,-12)$

D. $(6,-3)$

Zadanie $l0. (1pkt)$

Liczba l jest miejscem zerowym funkcji liniowej $f(x)=ax+b$, a punkt $M=(3,-2)$ nalezy

do wykresu tej funkcji. Współczynnik $a$ we wzorze tej funkcjijest równy

A. l

B.

-23

C.

- -23

D. $-1$

Zadanie ll. $(1pktJ$

Dany jest ciąg $(a_{n})$ jest określony wzorem $a_{n}=\displaystyle \frac{5-2n}{6}$ dla $n\geq 1$. Ciąg tenjest

A.

B.

C.

D.

arytmetyczny ijego róznicajest równa $r=-\displaystyle \frac{1}{3}$

arytmetyczny ijego róznicajest równa $r=-2.$

geometryczny ijego iloraz jest równy $q=-\displaystyle \frac{1}{3}.$

geometryczny ijego iloraz jest równy $q=\displaystyle \frac{5}{6}$

Strona 4 $\mathrm{z}26$

MMA-IP





{\it Egzamin maturalny z matematyki}

{\it Poziom podstawowy}

{\it BRUDNOPIS} ({\it nie podlega ocenie})

MMA-IP

Strona 5 z 26





{\it Egzamin maturalny z matematyki}

{\it Poziom podstawowy}

Zadanie 12. $(1pktJ$

Dla ciągu arytmetycznego $(a_{n})$, określonego dla $n\geq 1$, jest spetniony wamnek $a_{4}+a_{5}+a_{6}=12.$

Wtedy

A. $a_{5}=4$

B. $a_{5}=3$

C. $a_{5}=6$

D. $a_{5}=5$

Zadanie $l3. (1pktJ$

Dany jest ciąg geometryczny $(a_{n})$, określony dla $n\geq 1$, w którym $a_{1}=\sqrt{2},$

$a_{3}=4\sqrt{2}$. Wzór na n-ty wyraz tego ciągu ma postać

$a_{2}=2\sqrt{2},$

A. $a_{n}=(\sqrt{2})^{n}$

B.

{\it an}$=$ -$\sqrt{}$22{\it n}

C.

{\it an}$=$(-$\sqrt{}$22){\it n}

D.

$a_{n}=\displaystyle \frac{(\sqrt{2})}{2}n$

Zadanie 14. (1pkt)

Przyprostokątna LM trójkąta prostokątnego KLM ma długość 3, a przeciwprostokątna KL ma

długość 8 (zobacz rysunek).

3
\begin{center}
\includegraphics[width=82.656mm,height=35.808mm]{./F1_M_PP_M2018_page5_images/image001.eps}
\end{center}
{\it L}

8

$\alpha$

{\it M  K}

Wówczas miara $\alpha$ kąta ostrego $LMK$ tego trójkąta spełnia waiunek

A. $27^{\mathrm{o}}<\alpha\leq 30^{\mathrm{o}}$

B. $24^{\mathrm{o}}<\alpha\leq 27^{\mathrm{o}}$

C. $21^{\mathrm{o}}<\alpha\leq 24^{\mathrm{o}}$

D. $18^{\mathrm{o}}<\alpha\leq 21^{\mathrm{o}}$

Zadanie 15. $(1pkt)$

Dany jest trójkąt o bokach długości: $2\sqrt{5}, 3\sqrt{5}, 4\sqrt{5}$. Trójkątem podobnym do tego trójkąta

jest trójkąt, którego boki mają długości

A. 10, 15, 20

B. 20, 45, 80

C. $\sqrt{2}, \sqrt{3}, \sqrt{4}$

D. $\sqrt{5}, 2\sqrt{5}, 3\sqrt{5}$

Strona 6 z 26

MMA-IP





{\it Egzamin maturalny z matematyki}

{\it Poziom podstawowy}

{\it BRUDNOPIS} ({\it nie podlega ocenie})

MMA-IP

Strona 7 z 26





{\it Egzamin maturalny z matematyki}

{\it Poziom podstawowy}

Zadanie 16. $(1pkt)$

Dany jest okrąg o środku $S$. Punkty $K, L\mathrm{i}M$ lez$\cdot$ą na tym okręgu. Na łuku $KL$ tego okręgu są

oparte kąty $KSL \mathrm{i} KML$ (zobacz rysunek), których miary a $\mathrm{i} \beta$, spełniają warunek

$\alpha+\beta=111^{\mathrm{o}}$. Wynika stąd, $\dot{\mathrm{z}}\mathrm{e}$
\begin{center}
\includegraphics[width=64.368mm,height=61.620mm]{./F1_M_PP_M2018_page7_images/image001.eps}
\end{center}
{\it M}

{\it K  L}

A. $\alpha=74^{\mathrm{o}}$

B. $\alpha=76^{\mathrm{o}}$

C. $\alpha=70^{\mathrm{o}}$

D. $\alpha=72^{\mathrm{o}}$

Zadanie 17. $(1pkt)$

Dany jest trapez prostokątny KLMN, którego podstawy mają długoŚci $|KL|=a, |MN|=b,$

$a>b$. Kąt $KLM$ ma miarę $60^{\mathrm{o}}$. DługoŚć ramienia $LM$ tego trapezujest równa

{\it N b}

{\it M}
\begin{center}
\includegraphics[width=89.712mm,height=41.556mm]{./F1_M_PP_M2018_page7_images/image002.eps}
\end{center}
{\it K  L}

{\it a}

A. $a-b$

B. $2(a-b)$

C.

$a+\displaystyle \frac{1}{2}b$

D.

-{\it a} $+$2 {\it b}

Zadanie $l8.(1pkt)$

Średnicą okręgu jest odcinek $Kl$, gdzie $K=(6,8), L=(-6,-8)$. Równanie tego okręgu ma

postać

A. $x^{2}+y^{2}=200$

B. $x^{2}+y^{2}=100$

C. $x^{2}+y^{2}=400$

D. $x^{2}+y^{2}=300$

Zadanie $1\vartheta. (1pkt)$

Proste o równaniach $y=(m+2)x+3$ oraz $y=(2m-1)x-3$ są równoległe, gdy

A. $m=2$

B. $m=3$

C. $m=0$

D. $m=1$

Strona 8 z 26

MMA-IP





{\it Egzamin maturalny z matematyki}

{\it Poziom podstawowy}

{\it BRUDNOPIS} ({\it nie podlega ocenie})

MMA-IP

Strona 9 z 26





{\it Egzamin maturalny z matematyki}

{\it Poziom podstawowy}

Zadanie 20. $(1pktJ$

Podstawą ostrosłupa jest kwadrat KLMN o boku długości 4. Wysokością tego ostrosłupajest

krawędzí $NS$, ajej długość $\mathrm{t}\mathrm{e}\dot{\mathrm{z}}$ jest równa 4 (zobacz rysunek).

Kąt $\alpha$, jaki tworzą krawędzie $KS\mathrm{i}MS$, spełnia warunek

A. $\alpha=45^{\mathrm{o}}$

B. $45^{\mathrm{o}}<\alpha<60^{\mathrm{o}}$

C. $a>60^{\mathrm{o}}$

D. $\alpha=60^{\mathrm{o}}$

Zadanie 21. $(1pkt)$

Podstawą graniastosłupa prostegojest prostokąt o bokach długości 3 $\mathrm{i}4$. Kąt $\alpha$, jaki przekątna

tego graniastosłupa tworzy zjego podstawą, jest równy $45^{\mathrm{o}}$ (zobacz rysunek).

Wysokość graniastosłupa jest równa

A. 5

B. $3\sqrt{2}$

C. $5\sqrt{2}$

D.

$\displaystyle \frac{5\sqrt{3}}{3}$

Zadanie 22. (1pkt)

Na rysunku przedstawiono bryłę zbudowaną z walca i półkuli. Wysokość walcajest równa r

ijest taka samajak promień półkuli oraz taka sama jak promień podstawy walca.

Objętość tej bryłyjest równa

A.

$\displaystyle \frac{5}{3}\pi r^{3}$

B.

$\displaystyle \frac{4}{3}\pi r^{3}$

C.

$\displaystyle \frac{2}{3}\pi r^{3}$

D.

$\displaystyle \frac{1}{3}\pi r^{3}$

Strona 10 z 26

MMA-IP



\end{document}