\documentclass[a4paper,12pt]{article}
\usepackage{latexsym}
\usepackage{amsmath}
\usepackage{amssymb}
\usepackage{graphicx}
\usepackage{wrapfig}
\pagestyle{plain}
\usepackage{fancybox}
\usepackage{bm}

\begin{document}

{\it Egzamin maturalny z matematyki}

{\it Poziom podstawowy}

ZADANIA ZAMKNIĘTE

$W$ {\it kazdym z zadań od l. do 25. wybierz i zaznacz na karcie odpowiedzipoprawnq odpowied} $\acute{z}.$

Zadanie l. $(1pktJ$

Liczba 2 $\log_{3}6-\log_{3}4$ jest równa

A. 4

B. 2

Zadanie 2. $(1pkt)$

Liczba $\sqrt[3]{\frac{7}{3}}\cdot\sqrt[3]{\frac{81}{56}}$ jest równa

A.

-$\sqrt{}$23

B.

$\displaystyle \frac{3}{2\sqrt[3]{21}}$

C. $2\log_{3}2$

D. $\log_{3}8$

C.

-23

D.

-49

Zadanie 3. $(1pkt)$

Dane są liczby $a=3,6\cdot 10^{-12}$ oraz $b=2,4\cdot 10^{-20}$. Wtedy iloraz $\displaystyle \frac{a}{b}$ jest równy

A. $8,64\cdot 10^{-32}$

B. $1,5\cdot 10^{-8}$

C. $1,5\cdot 10^{8}$

D. $8,64\cdot 10^{32}$

Zadame4. (1pkt)

Cena roweru po obnizce o 15\% była równa 850 zł. Przed tą obnizką rower ten kosztował

A. 865,00 zł

B. 850,15 zł

C. 1000,00 zł

D. 977,50 zł

Zadanie 5. $(1pkt)$

Zbiorem wszystkich rozwiązań nierówności $\displaystyle \frac{1-2x}{2}>\frac{1}{3}$ jest przedział

A.

(-$\infty$' -61)

B.

(-$\infty$' -23)

C.

$(\displaystyle \frac{1}{6},+\infty)$

D.

$(\displaystyle \frac{2}{3},+\infty)$

{\it Zadanie 6}. ({\it lpkt})

Funkcja kwadratowa określona jest wzorem

róznymi miejscami zerowymi ffinkcjif. Zatem

$f(x)=-2(x+3)(x-5)$. Liczby

$x_{1}, x_{2}$

są

A. $x_{1}+x_{2}=-8$

B. $x_{1}+x_{2}=-2$

C. $x_{1}+x_{2}=2$

D. $x_{1}+x_{2}=8$

Strona 2 z 26

MMA-IP
\end{document}
