\documentclass[a4paper,12pt]{article}
\usepackage{latexsym}
\usepackage{amsmath}
\usepackage{amssymb}
\usepackage{graphicx}
\usepackage{wrapfig}
\pagestyle{plain}
\usepackage{fancybox}
\usepackage{bm}

\begin{document}

{\it Egzamin maturalny z matematyki}

{\it Poziom podstawowy}

Zadanie 23. $(1pktJ$

$\mathrm{W}$ zestawie $\displaystyle \frac{2,2,2,\ldots,2}{m1\mathrm{i}\mathrm{c}\mathrm{z}\mathrm{b}}\frac{4,4,4,\ldots,4}{m1\mathrm{i}\mathrm{c}\mathrm{z}\mathrm{b}}$ jest $2m$ liczb $(m\geq 1)$ ` w tym $m$ liczb 2 $\mathrm{i} m$ liczb 4.

Odchylenie standardowe tego zestawu liczb jest równe

A. 2

B. l

C.

-$\sqrt{}$12

D. $\sqrt{2}$

Zadanie 24. $(1pktJ$

Ile jest wszystkich liczb naturalnych czterocyfrowych mniejszych $\mathrm{n}\mathrm{i}\dot{\mathrm{z}}$ 2018 i podzielnych

przez 5?

A. 402

B. 403

C. 203

D. 204

Zadanie 25, $(1pktJ$

$\mathrm{W}$ pudełku jest 50 kuponów, wśród których jest 15 kuponów przegrywających, a pozostałe

kupony są wygrywające. $\mathrm{Z}$ tego pudełka w sposób losowy wyciągamy jeden kupon.

Prawdopodobieństwo zdarzenia polegającego na tym, $\dot{\mathrm{z}}\mathrm{e}$ wyciągniemy kupon wygrywający, jest

równe

A.

$\displaystyle \frac{15}{35}$

B.

$\displaystyle \frac{1}{50}$

C.

$\displaystyle \frac{15}{50}$

D.

$\displaystyle \frac{35}{50}$

Strona 12 z 26

MMA-IP
\end{document}
