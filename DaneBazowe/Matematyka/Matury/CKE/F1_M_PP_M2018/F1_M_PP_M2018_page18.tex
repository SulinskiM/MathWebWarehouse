\documentclass[a4paper,12pt]{article}
\usepackage{latexsym}
\usepackage{amsmath}
\usepackage{amssymb}
\usepackage{graphicx}
\usepackage{wrapfig}
\pagestyle{plain}
\usepackage{fancybox}
\usepackage{bm}

\begin{document}

{\it Egzamin maturalny z matematyki}

{\it Poziom podstawowy}

Zadanie 31. $(2pktJ$

Dwunasty wyraz ciągu arytmetycznego $(a_{n})$, określonego dla $n\geq 1$, jest równy 30, a sumajego

dwunastu początkowych wyrazówjest równa 162. Ob1icz pierwszy wyraz tego ciągu.

Odpowiedzí :
\begin{center}
\includegraphics[width=96.012mm,height=17.832mm]{./F1_M_PP_M2018_page18_images/image001.eps}
\end{center}
Wypelnia

egzaminator

Nr zadania

Maks. liczba kt

30.

2

31.

2

Uzyskana liczba pkt

MMA-IP

Strona 19 z 26
\end{document}
