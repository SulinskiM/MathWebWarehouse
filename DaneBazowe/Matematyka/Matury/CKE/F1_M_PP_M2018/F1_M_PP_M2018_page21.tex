\documentclass[a4paper,12pt]{article}
\usepackage{latexsym}
\usepackage{amsmath}
\usepackage{amssymb}
\usepackage{graphicx}
\usepackage{wrapfig}
\pagestyle{plain}
\usepackage{fancybox}
\usepackage{bm}

\begin{document}

{\it Egzamin maturalny z matematyki}

{\it Poziom podstawowy}

Zadanie 33. $(4pktJ$

Dane są dwa zbiory: $A=\{100$, 200, 300, 400, 500, 600, 700$\} \mathrm{i} B=\{10$, 11, 12, 13, 14, 15, 16$\}.$

$\mathrm{Z}\mathrm{k}\mathrm{a}\dot{\mathrm{z}}$ dego z nich losujemyjedną liczbę. Oblicz prawdopodobieństwo zdarzenia polegającego na

tym, $\dot{\mathrm{z}}\mathrm{e}$ suma wylosowanych liczb będzie podzielna przez 3. Ob1iczone prawdopodobieństwo

zapisz w postaci nieskracalnego ułamka zwykłego.

Strona 22 z 26

MMA-IP
\end{document}
