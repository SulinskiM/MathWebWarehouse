\documentclass[a4paper,12pt]{article}
\usepackage{latexsym}
\usepackage{amsmath}
\usepackage{amssymb}
\usepackage{graphicx}
\usepackage{wrapfig}
\pagestyle{plain}
\usepackage{fancybox}
\usepackage{bm}

\begin{document}

{\it Egzamin maturalny z matematyki}

{\it Poziom podstawowy}

Zadanie 7. $(1pkt)$

Równanie $\displaystyle \frac{x^{2}+2x}{x^{2}-4}=0$

A. ma trzy rozwiązania: $x=-2, x=0, x=2$

B. ma dwa rozwiązania: $x=0, x=2$

C. ma dwa rozwiązania: $x=-2, x=2$

D. majedno rozwiązanie: $x=0$

Zadanie S, (lpkt)

Funkcja liniowa f określona jest wzorem

rzeczywistych x. Wskaz zdanie prawdziwe.

$f(x)=\displaystyle \frac{1}{3}x-1,$

dla wszystkich

liczb

A. Funkcja $f$ jest malejącaijej wykres przecina oś $oy$ w punkcie $P=(0,\displaystyle \frac{1}{3}).$

B. Funkcja $f$ jest malejącaijej wykres przecina oś $Oy$ w punkcie $P=(0,-1).$

C. Funkcja $f$ jest rosnąca ijej wykres przecina oś $oy$ w punkcie $P=(0,\displaystyle \frac{1}{3}).$

D. Funkcja $f$ jest rosnącaijej wykres przecina oś $Oy$ w punkcie $P=(0,-1).$

Zadam$\mathrm{e}9. (1pkt)$

Wykresem funkcji kwadratowej $f(x)=x^{2}-6x-3$ jest parabola, której wierzchołkiem jest

punkt o współrzędnych

A. $(-6,-3)$

B. $(-6,69)$

C. $(3,-12)$

D. $(6,-3)$

Zadanie $l0. (1pkt)$

Liczba l jest miejscem zerowym funkcji liniowej $f(x)=ax+b$, a punkt $M=(3,-2)$ nalezy

do wykresu tej funkcji. Współczynnik $a$ we wzorze tej funkcjijest równy

A. l

B.

-23

C.

- -23

D. $-1$

Zadanie ll. $(1pktJ$

Dany jest ciąg $(a_{n})$ jest określony wzorem $a_{n}=\displaystyle \frac{5-2n}{6}$ dla $n\geq 1$. Ciąg tenjest

A.

B.

C.

D.

arytmetyczny ijego róznicajest równa $r=-\displaystyle \frac{1}{3}$

arytmetyczny ijego róznicajest równa $r=-2.$

geometryczny ijego iloraz jest równy $q=-\displaystyle \frac{1}{3}.$

geometryczny ijego iloraz jest równy $q=\displaystyle \frac{5}{6}$

Strona 4 $\mathrm{z}26$

MMA-IP
\end{document}
