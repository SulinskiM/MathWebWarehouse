\documentclass[a4paper,12pt]{article}
\usepackage{latexsym}
\usepackage{amsmath}
\usepackage{amssymb}
\usepackage{graphicx}
\usepackage{wrapfig}
\pagestyle{plain}
\usepackage{fancybox}
\usepackage{bm}

\begin{document}

{\it Egzamin maturalny z matematyki}

{\it Poziom podstawowy}

Zadanie 12. $(1pktJ$

Dla ciągu arytmetycznego $(a_{n})$, określonego dla $n\geq 1$, jest spetniony wamnek $a_{4}+a_{5}+a_{6}=12.$

Wtedy

A. $a_{5}=4$

B. $a_{5}=3$

C. $a_{5}=6$

D. $a_{5}=5$

Zadanie $l3. (1pktJ$

Dany jest ciąg geometryczny $(a_{n})$, określony dla $n\geq 1$, w którym $a_{1}=\sqrt{2},$

$a_{3}=4\sqrt{2}$. Wzór na n-ty wyraz tego ciągu ma postać

$a_{2}=2\sqrt{2},$

A. $a_{n}=(\sqrt{2})^{n}$

B.

{\it an}$=$ -$\sqrt{}$22{\it n}

C.

{\it an}$=$(-$\sqrt{}$22){\it n}

D.

$a_{n}=\displaystyle \frac{(\sqrt{2})}{2}n$

Zadanie 14. (1pkt)

Przyprostokątna LM trójkąta prostokątnego KLM ma długość 3, a przeciwprostokątna KL ma

długość 8 (zobacz rysunek).

3
\begin{center}
\includegraphics[width=82.656mm,height=35.808mm]{./F1_M_PP_M2018_page5_images/image001.eps}
\end{center}
{\it L}

8

$\alpha$

{\it M  K}

Wówczas miara $\alpha$ kąta ostrego $LMK$ tego trójkąta spełnia waiunek

A. $27^{\mathrm{o}}<\alpha\leq 30^{\mathrm{o}}$

B. $24^{\mathrm{o}}<\alpha\leq 27^{\mathrm{o}}$

C. $21^{\mathrm{o}}<\alpha\leq 24^{\mathrm{o}}$

D. $18^{\mathrm{o}}<\alpha\leq 21^{\mathrm{o}}$

Zadanie 15. $(1pkt)$

Dany jest trójkąt o bokach długości: $2\sqrt{5}, 3\sqrt{5}, 4\sqrt{5}$. Trójkątem podobnym do tego trójkąta

jest trójkąt, którego boki mają długości

A. 10, 15, 20

B. 20, 45, 80

C. $\sqrt{2}, \sqrt{3}, \sqrt{4}$

D. $\sqrt{5}, 2\sqrt{5}, 3\sqrt{5}$

Strona 6 z 26

MMA-IP
\end{document}
