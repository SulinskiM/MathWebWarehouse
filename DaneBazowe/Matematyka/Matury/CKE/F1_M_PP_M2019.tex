\documentclass[a4paper,12pt]{article}
\usepackage{latexsym}
\usepackage{amsmath}
\usepackage{amssymb}
\usepackage{graphicx}
\usepackage{wrapfig}
\pagestyle{plain}
\usepackage{fancybox}
\usepackage{bm}

\begin{document}

CENTRALNA

KOMISJA

EGZAMiNACYJNA

Arkusz zawiera informacje prawnie chronione do momentu rozpoczęcia egzaminu.

UZUPELNIA ZDAJACY

KOD PESEL

{\it miejsce}

{\it na naklejkę}
\begin{center}
\includegraphics[width=21.432mm,height=9.852mm]{./F1_M_PP_M2019_page0_images/image001.eps}

\includegraphics[width=82.140mm,height=9.852mm]{./F1_M_PP_M2019_page0_images/image002.eps}

\includegraphics[width=204.060mm,height=216.108mm]{./F1_M_PP_M2019_page0_images/image003.eps}
\end{center}
EGZAMIN MATU LNY

Z MATEMATYKI

POZIOM PODSTAWOWY

Instrukcja dla zdającego

1. Sprawd $\acute{\mathrm{z}}$, czy arkusz egzaminacyjny zawiera 26 stron

(zadania $1-34$). Ewentualny brak zgłoś przewodniczącemu

zespo nadzorującego egzamin.

2. Rozwiązania zadań i odpowiedzi wpisuj w miejscu na to

przeznaczonym.

3. Odpowiedzi do zadań zam iętych $(1-25)$ zaznacz

na karcie odpowiedzi, w części ka $\mathrm{y}$ przeznaczonej dla

zdającego. Zamaluj $\blacksquare$ pola do tego przeznaczone. Błędne

zaznaczenie otocz kółkiem \copyright i zaznacz właściwe.

4. Pamiętaj, $\dot{\mathrm{z}}\mathrm{e}$ pominięcie argumentacji lub istotnych

obliczeń w rozwiązaniu zadania otwa ego (26-34) $\mathrm{m}\mathrm{o}\dot{\mathrm{z}}\mathrm{e}$

spowodować, $\dot{\mathrm{z}}\mathrm{e}$ za to rozwiązanie nie otrzymasz pełnej

liczby punktów.

5. Pisz czytelnie i uzywaj tylko długopisu lub pióra

z czamym tuszem lub atramentem.

6. Nie uzywaj korektora, a błędne zapisy wyra $\acute{\mathrm{z}}\mathrm{n}\mathrm{i}\mathrm{e}$ prze eśl.

7. Pamiętaj, $\dot{\mathrm{z}}\mathrm{e}$ zapisy w brudnopisie nie będą oceniane.

8. $\mathrm{M}\mathrm{o}\dot{\mathrm{z}}$ esz korzystać z zestawu wzorów matematycznych,

cyrkla i linijki oraz kalkulatora prostego.

9. Na tej stronie oraz na karcie odpowiedzi wpisz swój

numer PESEL i przyklej naklejkę z kodem.

10. Nie wpisuj $\dot{\mathrm{z}}$ adnych znaków w części przeznaczonej dla

egzaminatora.

Godzina rozpoczęcia:

9:00

Czas pracy:

170 minut

Liczba punktów

do uzyskania: 50

$\Vert\Vert\Vert\Vert\Vert\Vert\Vert\Vert\Vert\Vert\Vert\Vert\Vert\Vert\Vert\Vert\Vert\Vert\Vert\Vert\Vert\Vert\Vert\Vert|  \mathrm{M}\mathrm{M}\mathrm{A}-\mathrm{P}1_{-}1\mathrm{P}-192$




{\it Egzamin maturalny z matematyki}

{\it Poziom podstawowy}

ZADANIA ZAMKNIĘTE

$W$ {\it kazdym z zadań} $\theta d1.$ {\it do 25. wybierz i zaznacz na karcie} $\theta owiedipprawnq$ {\it odpowiedz}$\acute{}$.{\it í}

Zadanie l. $(1pkt)$

Liczba $\log_{\sqrt{2}}2$ jest równa

A. 2

B. 4

C.

$\sqrt{2}$

D.

-21

Zadanie 2. $(1pkt)$

Liczba naturalna $n=2^{14}\cdot 5^{15}$ w zapisie dziesiętnym ma

A. 14 cyfr

B. 15 cyfr

C. 16 cyfr

D. 30 cyfr

$\mathrm{Z}_{\vartheta}\mathrm{d}\mathrm{a}\mathrm{n}\mathrm{i}\S 3. (1pkt)$

$\mathrm{W}$ pewnym banku prowizja od udzielanych kredytów hipotecznych przez cały styczeń była

równa 4\%. Na początku 1utego ten bank obnizył wysokość prowizji od wszystkich kredytów

$0 1$ punkt procentowy. Oznacza to, $\dot{\mathrm{z}}\mathrm{e}$ prowizja od kredytów hipotecznych w tym banku

zmniejszyła się o

A. l\%

B. 25\%

C. 33\%

D. 75\%

Zadanie 4, $(1pkt)$

Równość $\displaystyle \frac{1}{4}+\frac{1}{5}+\frac{1}{a}=1$ jest prawdziwa dla

A.

$a=\displaystyle \frac{11}{20}$

B.

{\it a}$=$ -98

C.

{\it a}$=$ -98

D.

{\it a}$=$ -2101

Zadanie 5. $(1pkt)$

Para liczb $x=2 \mathrm{i}y=2$ jest rozwiązaniem układu równań 

A. $a=-1$

B. $a=1$

C. $a=-2$

D. $a=2$

Zadanie $\epsilon. (1pkt)$

Równanie $\displaystyle \frac{(x-1)(x+2)}{x-3}=0$

A. ma trzy rózne rozwiązania: $x=1, x=3, x=-2.$

B. ma trzy rózne rozwiązania: $x=-1, x=-3, x=2.$

C. ma dwa rózne rozwiązania: $x=1, x=-2.$

D. ma dwa rózne rozwiązania: $x=-1, x=2.$

Strona 2 z26

MMA-IP





{\it Egzamin maturalny z matematyki}

{\it Poziom podstawowy}

{\it BRUDNOPIS} ({\it nie podlega ocenie})

Strona ll z 26





{\it Egzamin maturalny z matematyki}

{\it Poziom podstawowy}

Zadanie 22. (1pktJ

Podstawą ostrosłupa prawidłowego czworokątnego ABCDS jest kwadrat ABCD. Wszystkie

ściany boczne tego ostrosłupa są trójkątami równobocznymi.

Miara kąta SAC jest równa

A. $90^{\mathrm{o}}$

B. $75^{\mathrm{o}}$

C. $60^{\mathrm{o}}$

D. $45^{\mathrm{o}}$

Zadanie 23. (1pkt)

Mediana zestawu sześciu danych liczb: 4, 8, 21, a, 16, 25, jest równa 14. Zatem

A. $a=7$

B. $a=12$

C. $a=14$

D. $a=20$

Zadanie 24. (1pkt)

Wszystkich liczb pięciocyfrowych, w których występują wyłącznie cyfry 0, 2, 5, jest

A.

12

B. 36

C. 162

D. 243

Zadanie 25. $(1pkt)$

$\mathrm{W}$ pudełku jest 40 ku1. Wśród nich jest 35 ku1 białych, a pozostałe to ku1e czerwone.

Prawdopodobieństwo wylosowania $\mathrm{k}\mathrm{a}\dot{\mathrm{z}}$ dej kulijest takie samo. $\mathrm{Z}$ pudełka losujemyjedną kulę.

Prawdopodobieństwo zdarzenia polegającego na tym, $\dot{\mathrm{z}}\mathrm{e}$ otrzymamy kulę czerwoną, jest równe

A.

-81

B.

-51

C.

$\displaystyle \frac{1}{40}$

D.

$\displaystyle \frac{1}{35}$

Strona 12 z 26

MMA-IP





{\it Egzamin maturalny z matematyki}

{\it Poziom podstawowy}

{\it BRUDNOPIS} ({\it nie podlega ocenie})

Strona 13 z 26





{\it Egzamin maturalny z matematyki}

{\it Poziom podstawowy}

Zadanie 26. $(2pktJ$

Rozwiąz równanie $x^{3}-5x^{2}-9x+45=0.$

Odpowied $\acute{\mathrm{z}}$:

Strona 14 $\mathrm{z}26$





{\it Egzamin maturalny z matematyki}

{\it Poziom podstawowy}

Zadanie 27, $(2pktJ$

Rozwiąz nierównoŚć $3x^{2}-16x+16>0.$

Odpowiedzí :
\begin{center}
\includegraphics[width=96.012mm,height=17.784mm]{./F1_M_PP_M2019_page14_images/image001.eps}
\end{center}
Wypelnia

egzaminator

Nr zadania

Maks. liczba kt

2

27.

2

Uzyskana liczba pkt

MMA-IP

Strona 15 z 26





{\it Egzamin maturalny z matematyki}

{\it Poziom podstawowy}

Zadanie 2{\$}. $(2pktJ$

Wykaz, $\dot{\mathrm{z}}\mathrm{e}$ dla dowolnych liczb rzeczywistych $a\mathrm{i}b$ prawdziwajest nierówność

$3a^{2}-2ab+3b^{2}\geq 0.$

Strona 16 z 26





{\it Egzamin maturalny z matematyki}

{\it Poziom podstawowy}

Zadanie 29. $(2pktJ$

Danyjest okrąg o środku w punkcie $S$ i promieniu $r$. Na przedłuzeniu cięciwy $AB$ poza punkt $B$

odłozono odcinek $BC$ równy promieniowi danego okręgu. Przez punkty $C\mathrm{i}S$ poprowadzono

prostą. Prosta $CS$ przecina dany okrąg w punktach $D\mathrm{i}E$ (zobacz rysunek). Wykaz, $\dot{\mathrm{z}}$ ejezeli

miara kąta ACSjest równa $\alpha$, to miara kąta $ASD$ jest równa $3\alpha.$
\begin{center}
\includegraphics[width=96.924mm,height=62.580mm]{./F1_M_PP_M2019_page16_images/image001.eps}
\end{center}
{\it D  r}

{\it S}

{\it r}

{\it E}

{\it r  r}

{\it C}

{\it A  B}
\begin{center}
\includegraphics[width=96.012mm,height=17.832mm]{./F1_M_PP_M2019_page16_images/image002.eps}
\end{center}
Wypelnia

egzaminator

Nr zadania

Maks. liczba kt

28.

2

2

Uzyskana liczba pkt

MMA-IP

Strona 17 z 26





{\it Egzamin maturalny z matematyki}

{\it Poziom podstawowy}

Zadanie 30. (2pktJ

Ze zbioru liczb \{1, 2, 3, 4, 5\} 1osujemy dwa razy po jednej 1iczbie ze zwracaniem. Ob1icz

prawdopodobieństwo zdarzenia A polegającego na wylosowaniu liczb, których iloczyn jest

liczbą nieparzystą.

Odpowied $\acute{\mathrm{z}}$:

Strona 18 $\mathrm{z}26$

MMA-IP





{\it Egzamin maturalny z matematyki}

{\it Poziom podstawowy}

Zadanie 31. $(2pktJ$

$\mathrm{W}$ trapezie prostokątnym ABCD dłuzsza podstawa $AB$ ma długość 8. Przekątna $AC$ tego trapezu

ma długość 4 i tworzy z krótszą podstawą trapezu kąt o mierze $30^{\mathrm{o}}$ (zobacz rysunek). Oblicz

długość przekątnej $BD$ tego trapezu.
\begin{center}
\includegraphics[width=106.728mm,height=35.352mm]{./F1_M_PP_M2019_page18_images/image001.eps}
\end{center}
{\it D  C}

4

{\it A}  8  {\it B}

Odpowied $\acute{\mathrm{z}}$:
\begin{center}
\includegraphics[width=96.012mm,height=17.832mm]{./F1_M_PP_M2019_page18_images/image002.eps}
\end{center}
Wypelnia

egzaminator

Nr zadania

Maks. liczba kt

30.

2

31.

2

Uzyskana liczba pkt

MMA-IP

Strona 19 z 26





{\it Egzamin maturalny z matematyki}

{\it Poziom podstawowy}

Zadanie 32. $(4pktJ$

Ciąg arytmetyczny $(a_{n})$ jest określony dla $\mathrm{k}\mathrm{a}\dot{\mathrm{z}}$ dej liczby naturalnej $n\geq 1$. Róznicą tego

ciągujest liczba $r=-4$, a średnia arytmetyczna początkowych sześciu wyrazów tego ciągu:

$a_{1}, a_{2}, a_{3}, a_{4}, a_{5}, a_{6}$, jest równa 16.

a) Oblicz pierwszy wyraz tego ciągu.

b) Oblicz liczbę $k$, dla której $a_{k}=-78.$

Strona 20 z 26

MMA-IP





{\it Egzamin maturalny z matematyki}

{\it Poziom podstawowy}

{\it BRUDNOPIS} ({\it nie podlega ocenie})

Strona 3 z26





Odpowiedzí :

{\it Egzamin maturalny z matematyki}

{\it Poziom podstawowy}
\begin{center}
\includegraphics[width=82.044mm,height=17.832mm]{./F1_M_PP_M2019_page20_images/image001.eps}
\end{center}
Wypelnia

egzaminator

Nr zadania

Maks. liczba kt

32.

4

Uzyskana liczba pkt

MMA-IP

Strona 21 z 26





{\it Egzamin maturalny z matematyki}

{\it Poziom podstawowy}

Zadanie 33. $(4pktJ$

Danyjest punkt $A=(-18,10)$. Prosta o równaniu $y=3x$ jest symetralną odcinka $AB$. Wyznacz

współrzędne punktu $B.$

Strona 22 z 26

MMA-IP





{\it Egzamin maturalny z matematyki}

{\it Poziom podstawowy}

Odpowied $\acute{\mathrm{z}}$:
\begin{center}
\includegraphics[width=82.044mm,height=17.832mm]{./F1_M_PP_M2019_page22_images/image001.eps}
\end{center}
Nr zadania

Wypelnia Maks. liczba kt

egzaminator

Uzyskana liczba pkt

33.

4

MMA-IP

Strona 23 z 26





{\it Egzamin maturalny z matematyki}

{\it Poziom podstawowy}

Zadanie 34. $(SpktJ$

Długość krawędzi podstawy ostrosłupa prawidłowego czworokątnego jest równa 6. Po1e

powierzchni całkowitej tego ostrosłupajest cztery razy większe od polajego podstawy. Kąt $\alpha$

jest kątem nachylenia krawędzi bocznej tego ostrosłupa do płaszczyzny podstawy (zobacz

rysunek). Oblicz cosinus kąta $\alpha.$

Strona 24 z 26

MMA-IP





{\it Egzamin maturalny z matematyki}

{\it Poziom podstawowy}

Odpowied $\acute{\mathrm{z}}$:
\begin{center}
\includegraphics[width=82.044mm,height=17.832mm]{./F1_M_PP_M2019_page24_images/image001.eps}
\end{center}
Nr zadania

Wypelnia Maks. liczba kt

egzaminator

Uzyskana liczba pkt

34.

5

MMA-IP

Strona 25 z 26





{\it Egzamin maturalny z matematyki}

{\it Poziom podstawowy}

{\it BRUDNOPIS} ({\it nie podlega ocenie})

Strona 26 z 26





{\it Egzamin maturalny z matematyki}

{\it Poziom podstawowy}

Zadanie 7. $(1pkt)$

Miejscem zerowym funkcji liniowej $f$ określonej wzorem $f(x)=3(x+1)-6\sqrt{3}$ jest liczba

A. $3-6\sqrt{3}$

B.

$1-6\sqrt{3}$

C. $2\sqrt{3}-1$

D.

$2\displaystyle \sqrt{3}-\frac{1}{3}$

Informacja do zadań S.-10.

Na rysunku przedstawiony jest fragment paraboli będącej wykresem funkcji kwadratowej $f.$

Wierzchołkiem tej parabolijest punkt $W=(2,-4)$. Liczby 0 $\mathrm{i}4$ to miejsca zerowe funkcji $f.$
\begin{center}
\includegraphics[width=127.248mm,height=105.060mm]{./F1_M_PP_M2019_page3_images/image001.eps}
\end{center}
{\it y}

4

3

1

{\it x}

$-3 -2$

$-1 0$

$-1$

1 2 3 4  5 6

$-2$

$-3$

{\it W}

$\langle 0,  4\rangle$

B.

$(-\infty,  0\rangle$

A.

Zadanie 8. (1pkt)

Zbiorem wartości funkcji f jest przedział

C.

$\langle-4, +\infty)$

D. $\langle 4, +\infty)$

Zadam$\mathrm{e}9\cdot(1pkt)$

Największa wartość funkcji $f$ w przedziale $\langle$1, $ 4\rangle$ jest równa

A. $-3$

B. $-4$

C. 4

D. 0

Zadanie 10. (1pkt)

Osią symetrii wykresu funkcji f jest prosta o równaniu

A. $y=-4$

B. $x=-4$

C. $y=2$

D. $x=2$

Strona 4 z26

MMA-IP





{\it Egzamin maturalny z matematyki}

{\it Poziom podstawowy}

{\it BRUDNOPIS} ({\it nie podlega ocenie})

Strona 5 z 26





{\it Egzamin maturalny z matematyki}

{\it Poziom podstawowy}

Zadanie ll. $(1pktJ$

$\mathrm{W}$ ciągu arytmetycznym $(a_{n})$, określonym dla $n\geq 1$, dane są dwa wyrazy: $a_{1}=7\mathrm{i}a_{8}=-49.$

Suma ośmiu początkowych wyrazów tego ciągujest równa

A. $-168$

B. $-189$

C. $-21$

D. $-42$

Zadanie $l2. (1pkt)$

Dany jest ciąg geometryczny $(a_{n})$, określony dla $n\geq 1$. Wszystkie wyrazy tego ciągu są

dodatnie i spełnionyjest watunek $\displaystyle \frac{a_{5}}{a_{3}}=\frac{1}{9}$. Iloraz tego ciągujest równy

A.

-31

B.

-$\sqrt{}$13

C. 3

D. $\sqrt{3}$

Zadanie 13. $(1pktJ$

Sinus kąta ostrego $\alpha$ jest równy $\displaystyle \frac{4}{5}$. Wtedy

A.

$\displaystyle \cos\alpha=\frac{5}{4}$

B.

$\displaystyle \cos\alpha=\frac{1}{5}$

C.

$\displaystyle \cos\alpha=\frac{9}{25}$

D.

$\displaystyle \cos\alpha=\frac{3}{5}$

Zadanie 14. $(1pktJ$

Punkty $D\mathrm{i}E$ lez$\cdot$ą na okręgu opisanym na trójkącie równobocznym $ABC$ (zobacz rysunek).

Odcinek $CD$ jest średnicą tego okręgu. Kąt wpisany $DEB$ ma miarę $\alpha.$
\begin{center}
\includegraphics[width=46.788mm,height=52.680mm]{./F1_M_PP_M2019_page5_images/image001.eps}
\end{center}
{\it C}

{\it E}

$\alpha$

{\it A  B}

{\it D}

Zatem

A. $\alpha=30^{\mathrm{o}}$

B. $\alpha<30^{\mathrm{o}}$

C. $\alpha>45^{\mathrm{o}}$

D. $\alpha=45^{\mathrm{o}}$

Strona 6 z26

MMA-IP





{\it Egzamin maturalny z matematyki}

{\it Poziom podstawowy}

{\it BRUDNOPIS} ({\it nie podlega ocenie})

Strona 7 z 26





{\it Egzamin maturalny z matematyki}

{\it Poziom podstawowy}

Zadanie 15. (1pktJ

Dane są dwa okręgi: okrąg o środku w punkcie O i promieniu 5 oraz okrąg o środku

w punkcie P i promieniu 3. Odcinek OP ma długość 16. Prosta AB jest styczna do tych okręgów

w punktach A iB. Ponadto prosta AB przecina odcinek OP w punkcie K(zobacz rysunek).
\begin{center}
\includegraphics[width=155.292mm,height=65.436mm]{./F1_M_PP_M2019_page7_images/image001.eps}
\end{center}
{\it B}

{\it O  K}

{\it P}

{\it A}

Wtedy

A.

$|OK|=6$

B.

$|OK|=8$

C.

$|OK|=10$

D.

$|OK|=12$

Zadanie $1\mathrm{f}\cdot(1pkt)$

Dany jest romb o boku długości 4 i kącie rozwartym $150^{\mathrm{o}}$. Pole tego rombujest równe

A. 8

B. 12

C. $8\sqrt{3}$

D. 16

Zadanie $l7. (1pktJ$

Proste o równaniach $y=(2m+2)x-2019$ oraz $y=(3m-3)x+2019$ są równoległe, gdy

A. $m=-1$

B. $m=0$

C. $m=1$

D. $m=5$

Zadanie 18. (1pktJ

Prosta o równaniu $y=ax+b$ jest prostopadła do prostej o równaniu $y=-4x+1$ i przechodzi

przez punkt $P=(\displaystyle \frac{1}{2},0)$, gdy

A. $a=-4\mathrm{i}b=-2$

B. {\it a}$=$-41i{\it b}$=$--81

C. $a=-4\mathrm{i}b=2$

D. {\it a}$=$-41i{\it b}$=$-21

Strona 8 z 26

MMA-IP





{\it Egzamin maturalny z matematyki}

{\it Poziom podstawowy}

{\it BRUDNOPIS} ({\it nie podlega ocenie})

Strona 9 z 26





{\it Egzamin maturalny z matematyki}

{\it Poziom podstawowy}

Zadanie 19. $(1pktJ$

Na rysunku przedstawiony jest fragment wykresu funkcji liniowej $f$ Na wykresie tej ffinkcji

$\mathrm{l}\mathrm{e}\dot{\mathrm{z}}$ ą punkty $A=(0,4)\mathrm{i}B=(2,2).$
\begin{center}
\includegraphics[width=65.376mm,height=67.920mm]{./F1_M_PP_M2019_page9_images/image001.eps}
\end{center}
$y$

$5$

-$4^{A}$

3

2

$B1$

1

{\it x}

$-4  -3$ -$2$ -$1$ -$10$  1 2 3 4  $-5$

$-2$

$-3$

$-4$

Obrazem prostej AB w symetrii względem początku układu współrzędnych jest wykres

funkcji g określonej wzorem

A. $g(x)=x+4$

B. $g(x)=x-4$

C. $g(x)=-x-4$

D. $g(x)=-x+4$

Zadanie 20. $(1pktJ$

Dane są punkty o współrzędnych $A=(-2,5)$ oraz $B=(4,-1)$. Średnica okręgu wpisanego

w kwadrat o boku $AB$ jest równa

A. 12

B. 6

C.

$6\sqrt{2}$

D. $2\sqrt{6}$

Zadanie 21. (1pkt)

Promień AS podstawy walca jest równy połowie wysokości OS tego walca. Sinus kąta OAS

(zobacz rysunek) jest równy
\begin{center}
\includegraphics[width=47.904mm,height=76.404mm]{./F1_M_PP_M2019_page9_images/image002.eps}
\end{center}
{\it O}

{\it S}

{\it A}

A.

-$\sqrt{}$25

B.

$\displaystyle \frac{2\sqrt{5}}{5}$

C.

-21

D. l

Strona 10 z 26

MMA-IP



\end{document}