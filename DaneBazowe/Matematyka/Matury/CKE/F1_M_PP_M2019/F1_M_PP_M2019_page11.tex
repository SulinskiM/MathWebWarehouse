\documentclass[a4paper,12pt]{article}
\usepackage{latexsym}
\usepackage{amsmath}
\usepackage{amssymb}
\usepackage{graphicx}
\usepackage{wrapfig}
\pagestyle{plain}
\usepackage{fancybox}
\usepackage{bm}

\begin{document}

{\it Egzamin maturalny z matematyki}

{\it Poziom podstawowy}

Zadanie 22. (1pktJ

Podstawą ostrosłupa prawidłowego czworokątnego ABCDS jest kwadrat ABCD. Wszystkie

ściany boczne tego ostrosłupa są trójkątami równobocznymi.

Miara kąta SAC jest równa

A. $90^{\mathrm{o}}$

B. $75^{\mathrm{o}}$

C. $60^{\mathrm{o}}$

D. $45^{\mathrm{o}}$

Zadanie 23. (1pkt)

Mediana zestawu sześciu danych liczb: 4, 8, 21, a, 16, 25, jest równa 14. Zatem

A. $a=7$

B. $a=12$

C. $a=14$

D. $a=20$

Zadanie 24. (1pkt)

Wszystkich liczb pięciocyfrowych, w których występują wyłącznie cyfry 0, 2, 5, jest

A.

12

B. 36

C. 162

D. 243

Zadanie 25. $(1pkt)$

$\mathrm{W}$ pudełku jest 40 ku1. Wśród nich jest 35 ku1 białych, a pozostałe to ku1e czerwone.

Prawdopodobieństwo wylosowania $\mathrm{k}\mathrm{a}\dot{\mathrm{z}}$ dej kulijest takie samo. $\mathrm{Z}$ pudełka losujemyjedną kulę.

Prawdopodobieństwo zdarzenia polegającego na tym, $\dot{\mathrm{z}}\mathrm{e}$ otrzymamy kulę czerwoną, jest równe

A.

-81

B.

-51

C.

$\displaystyle \frac{1}{40}$

D.

$\displaystyle \frac{1}{35}$

Strona 12 z 26

MMA-IP
\end{document}
