\documentclass[a4paper,12pt]{article}
\usepackage{latexsym}
\usepackage{amsmath}
\usepackage{amssymb}
\usepackage{graphicx}
\usepackage{wrapfig}
\pagestyle{plain}
\usepackage{fancybox}
\usepackage{bm}

\begin{document}

{\it Egzamin maturalny z matematyki}

{\it Poziom podstawowy}

Zadanie 27, $(2pktJ$

Rozwiąz nierównoŚć $3x^{2}-16x+16>0.$

Odpowiedzí :
\begin{center}
\includegraphics[width=96.012mm,height=17.784mm]{./F1_M_PP_M2019_page14_images/image001.eps}
\end{center}
Wypelnia

egzaminator

Nr zadania

Maks. liczba kt

2

27.

2

Uzyskana liczba pkt

MMA-IP

Strona 15 z 26
\end{document}
