\documentclass[a4paper,12pt]{article}
\usepackage{latexsym}
\usepackage{amsmath}
\usepackage{amssymb}
\usepackage{graphicx}
\usepackage{wrapfig}
\pagestyle{plain}
\usepackage{fancybox}
\usepackage{bm}

\begin{document}

{\it Egzamin maturalny z matematyki}

{\it Poziom podstawowy}

Zadanie 31. $(2pktJ$

$\mathrm{W}$ trapezie prostokątnym ABCD dłuzsza podstawa $AB$ ma długość 8. Przekątna $AC$ tego trapezu

ma długość 4 i tworzy z krótszą podstawą trapezu kąt o mierze $30^{\mathrm{o}}$ (zobacz rysunek). Oblicz

długość przekątnej $BD$ tego trapezu.
\begin{center}
\includegraphics[width=106.728mm,height=35.352mm]{./F1_M_PP_M2019_page18_images/image001.eps}
\end{center}
{\it D  C}

4

{\it A}  8  {\it B}

Odpowied $\acute{\mathrm{z}}$:
\begin{center}
\includegraphics[width=96.012mm,height=17.832mm]{./F1_M_PP_M2019_page18_images/image002.eps}
\end{center}
Wypelnia

egzaminator

Nr zadania

Maks. liczba kt

30.

2

31.

2

Uzyskana liczba pkt

MMA-IP

Strona 19 z 26
\end{document}
