\documentclass[a4paper,12pt]{article}
\usepackage{latexsym}
\usepackage{amsmath}
\usepackage{amssymb}
\usepackage{graphicx}
\usepackage{wrapfig}
\pagestyle{plain}
\usepackage{fancybox}
\usepackage{bm}

\begin{document}

{\it Egzamin maturalny z matematyki}

{\it Poziom podstawowy}

Zadanie 32. $(4pktJ$

Ciąg arytmetyczny $(a_{n})$ jest określony dla $\mathrm{k}\mathrm{a}\dot{\mathrm{z}}$ dej liczby naturalnej $n\geq 1$. Róznicą tego

ciągujest liczba $r=-4$, a średnia arytmetyczna początkowych sześciu wyrazów tego ciągu:

$a_{1}, a_{2}, a_{3}, a_{4}, a_{5}, a_{6}$, jest równa 16.

a) Oblicz pierwszy wyraz tego ciągu.

b) Oblicz liczbę $k$, dla której $a_{k}=-78.$

Strona 20 z 26

MMA-IP
\end{document}
