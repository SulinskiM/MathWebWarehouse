\documentclass[a4paper,12pt]{article}
\usepackage{latexsym}
\usepackage{amsmath}
\usepackage{amssymb}
\usepackage{graphicx}
\usepackage{wrapfig}
\pagestyle{plain}
\usepackage{fancybox}
\usepackage{bm}

\begin{document}

{\it Egzamin maturalny z matematyki}

{\it Poziom podstawowy}

Zadanie 7. $(1pkt)$

Miejscem zerowym funkcji liniowej $f$ określonej wzorem $f(x)=3(x+1)-6\sqrt{3}$ jest liczba

A. $3-6\sqrt{3}$

B.

$1-6\sqrt{3}$

C. $2\sqrt{3}-1$

D.

$2\displaystyle \sqrt{3}-\frac{1}{3}$

Informacja do zadań S.-10.

Na rysunku przedstawiony jest fragment paraboli będącej wykresem funkcji kwadratowej $f.$

Wierzchołkiem tej parabolijest punkt $W=(2,-4)$. Liczby 0 $\mathrm{i}4$ to miejsca zerowe funkcji $f.$
\begin{center}
\includegraphics[width=127.248mm,height=105.060mm]{./F1_M_PP_M2019_page3_images/image001.eps}
\end{center}
{\it y}

4

3

1

{\it x}

$-3 -2$

$-1 0$

$-1$

1 2 3 4  5 6

$-2$

$-3$

{\it W}

$\langle 0,  4\rangle$

B.

$(-\infty,  0\rangle$

A.

Zadanie 8. (1pkt)

Zbiorem wartości funkcji f jest przedział

C.

$\langle-4, +\infty)$

D. $\langle 4, +\infty)$

Zadam$\mathrm{e}9\cdot(1pkt)$

Największa wartość funkcji $f$ w przedziale $\langle$1, $ 4\rangle$ jest równa

A. $-3$

B. $-4$

C. 4

D. 0

Zadanie 10. (1pkt)

Osią symetrii wykresu funkcji f jest prosta o równaniu

A. $y=-4$

B. $x=-4$

C. $y=2$

D. $x=2$

Strona 4 z26

MMA-IP
\end{document}
