\documentclass[a4paper,12pt]{article}
\usepackage{latexsym}
\usepackage{amsmath}
\usepackage{amssymb}
\usepackage{graphicx}
\usepackage{wrapfig}
\pagestyle{plain}
\usepackage{fancybox}
\usepackage{bm}

\begin{document}

{\it Egzamin maturalny z matematyki}

{\it Poziom podstawowy}

Zadanie 15. (1pktJ

Dane są dwa okręgi: okrąg o środku w punkcie O i promieniu 5 oraz okrąg o środku

w punkcie P i promieniu 3. Odcinek OP ma długość 16. Prosta AB jest styczna do tych okręgów

w punktach A iB. Ponadto prosta AB przecina odcinek OP w punkcie K(zobacz rysunek).
\begin{center}
\includegraphics[width=155.292mm,height=65.436mm]{./F1_M_PP_M2019_page7_images/image001.eps}
\end{center}
{\it B}

{\it O  K}

{\it P}

{\it A}

Wtedy

A.

$|OK|=6$

B.

$|OK|=8$

C.

$|OK|=10$

D.

$|OK|=12$

Zadanie $1\mathrm{f}\cdot(1pkt)$

Dany jest romb o boku długości 4 i kącie rozwartym $150^{\mathrm{o}}$. Pole tego rombujest równe

A. 8

B. 12

C. $8\sqrt{3}$

D. 16

Zadanie $l7. (1pktJ$

Proste o równaniach $y=(2m+2)x-2019$ oraz $y=(3m-3)x+2019$ są równoległe, gdy

A. $m=-1$

B. $m=0$

C. $m=1$

D. $m=5$

Zadanie 18. (1pktJ

Prosta o równaniu $y=ax+b$ jest prostopadła do prostej o równaniu $y=-4x+1$ i przechodzi

przez punkt $P=(\displaystyle \frac{1}{2},0)$, gdy

A. $a=-4\mathrm{i}b=-2$

B. {\it a}$=$-41i{\it b}$=$--81

C. $a=-4\mathrm{i}b=2$

D. {\it a}$=$-41i{\it b}$=$-21

Strona 8 z 26

MMA-IP
\end{document}
