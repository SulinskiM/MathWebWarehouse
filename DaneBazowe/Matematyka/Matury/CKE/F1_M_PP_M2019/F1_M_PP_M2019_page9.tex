\documentclass[a4paper,12pt]{article}
\usepackage{latexsym}
\usepackage{amsmath}
\usepackage{amssymb}
\usepackage{graphicx}
\usepackage{wrapfig}
\pagestyle{plain}
\usepackage{fancybox}
\usepackage{bm}

\begin{document}

{\it Egzamin maturalny z matematyki}

{\it Poziom podstawowy}

Zadanie 19. $(1pktJ$

Na rysunku przedstawiony jest fragment wykresu funkcji liniowej $f$ Na wykresie tej ffinkcji

$\mathrm{l}\mathrm{e}\dot{\mathrm{z}}$ ą punkty $A=(0,4)\mathrm{i}B=(2,2).$
\begin{center}
\includegraphics[width=65.376mm,height=67.920mm]{./F1_M_PP_M2019_page9_images/image001.eps}
\end{center}
$y$

$5$

-$4^{A}$

3

2

$B1$

1

{\it x}

$-4  -3$ -$2$ -$1$ -$10$  1 2 3 4  $-5$

$-2$

$-3$

$-4$

Obrazem prostej AB w symetrii względem początku układu współrzędnych jest wykres

funkcji g określonej wzorem

A. $g(x)=x+4$

B. $g(x)=x-4$

C. $g(x)=-x-4$

D. $g(x)=-x+4$

Zadanie 20. $(1pktJ$

Dane są punkty o współrzędnych $A=(-2,5)$ oraz $B=(4,-1)$. Średnica okręgu wpisanego

w kwadrat o boku $AB$ jest równa

A. 12

B. 6

C.

$6\sqrt{2}$

D. $2\sqrt{6}$

Zadanie 21. (1pkt)

Promień AS podstawy walca jest równy połowie wysokości OS tego walca. Sinus kąta OAS

(zobacz rysunek) jest równy
\begin{center}
\includegraphics[width=47.904mm,height=76.404mm]{./F1_M_PP_M2019_page9_images/image002.eps}
\end{center}
{\it O}

{\it S}

{\it A}

A.

-$\sqrt{}$25

B.

$\displaystyle \frac{2\sqrt{5}}{5}$

C.

-21

D. l

Strona 10 z 26

MMA-IP
\end{document}
