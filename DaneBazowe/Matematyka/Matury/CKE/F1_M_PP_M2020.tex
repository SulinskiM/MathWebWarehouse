\documentclass[a4paper,12pt]{article}
\usepackage{latexsym}
\usepackage{amsmath}
\usepackage{amssymb}
\usepackage{graphicx}
\usepackage{wrapfig}
\pagestyle{plain}
\usepackage{fancybox}
\usepackage{bm}

\begin{document}

CENTRALNA

KOMISJA

EGZAMINACYJNA

Arkusz zawiera informacje prawnie chronione do momentu rozpoczęcia egzaminu.

UZUPELNIA ZDAJACY

KOD PESEL

{\it miejsce}

{\it na naklejkę}
\begin{center}
\includegraphics[width=21.432mm,height=9.852mm]{./F1_M_PP_M2020_page0_images/image001.eps}

\includegraphics[width=82.140mm,height=9.852mm]{./F1_M_PP_M2020_page0_images/image002.eps}

\includegraphics[width=204.060mm,height=216.048mm]{./F1_M_PP_M2020_page0_images/image003.eps}
\end{center}
EGZAMIN MATU LNY

Z MATEMATYKI

POZIOM PODSTAWOWY

Instrukcja dla zdającego

1. Sprawd $\acute{\mathrm{z}}$, czy arkusz egzaminacyjny zawiera 26 stron

(zadania $1-34$). Ewentualny brak zgłoś przewodniczącemu

zespo nadzo jącego egzamin.

2. Rozwiązania zadań i odpowiedzi wpisuj w miejscu na to

przeznaczonym.

3. Odpowiedzi do zadań zamkniętych $(1-25)$ zaznacz

na karcie odpowiedzi, w części ka $\mathrm{y}$ przeznaczonej dla

zdającego. Zamaluj $\blacksquare$ pola do tego przeznaczone. Błędne

zaznaczenie otocz kółkiem $\mathrm{O}$ i zaznacz właściwe.

4. Pamiętaj, $\dot{\mathrm{z}}\mathrm{e}$ pominięcie argumentacji lub istotnych

obliczeń w rozwiązaniu zadania otwartego (26-34) $\mathrm{m}\mathrm{o}\dot{\mathrm{z}}\mathrm{e}$

spowodować, $\dot{\mathrm{z}}\mathrm{e}$ za to rozwiązanie nie otrzymasz pełnej

liczby punktów.

5. Pisz czytelnie i uzywaj tylko długopisu lub pióra

z czatnym tuszem lub atramentem.

6. Nie uzywaj korektora, a błędne zapisy wyra $\acute{\mathrm{z}}\mathrm{n}\mathrm{i}\mathrm{e}$ prze eśl.

7. Pamiętaj, $\dot{\mathrm{z}}\mathrm{e}$ zapisy w brudnopisie nie będą oceniane.

8. $\mathrm{M}\mathrm{o}\dot{\mathrm{z}}$ esz korzystać z zestawu wzorów matematycznych,

cyrkla i linijki oraz kalkulatora prostego.

9. Na tej stronie oraz na karcie odpowiedzi wpisz swój

numer PESEL i przyklej naklejkę z kodem.

10. Nie wpisuj $\dot{\mathrm{z}}$ adnych znaków w części przeznaczonej dla

egzaminatora.

5 MAJA 2020

Godzina rozpoczęcia:

Czas pracy:

170 minut

Liczba punktów

do uzyskania: 50

$\Vert\Vert\Vert\Vert\Vert\Vert\Vert\Vert\Vert\Vert\Vert\Vert\Vert\Vert\Vert\Vert\Vert\Vert\Vert\Vert\Vert\Vert\Vert\Vert|  \mathrm{M}\mathrm{M}\mathrm{A}-\mathrm{P}1_{-}1\mathrm{P}-202$




{\it Egzamin maturalny z matematyki}

{\it Poziom podstawowy}

ZADANIA ZAMKNIETE

$W$ {\it kazdym z zadań od l. do 25. wybierz i zaznacz na karcie odpowiedzipoprawnq odpowied} $\acute{z}.$

Zadanie l. $(1pkt)$

Wartość wyrazenia $x^{2}-6x+9$ dla $x=\sqrt{3}+3$ jest równa

A. l

B. 3

Zadanie 2. $(1pkt)$

Liczba $\displaystyle \frac{2^{50}\cdot 3^{40}}{36^{10}}$ jest równa

A. $6^{70}$

B. $6^{45}$

Zadanie 3. $(1pkt)$

Liczba $\log_{5}\sqrt{125}$ jest równa

A.

-23

B. 2

C. $1+2\sqrt{3}$

D. $1-2\sqrt{3}$

C. $2^{30}\cdot 3^{20}$

D. $2^{10}\cdot 3^{20}$

C. 3

D.

-23

Zadanie 4. (1pkt)

Cenę x pewnego towaru obnizono o 20\% i otrzymano cenę y. Aby przywrócić cenę x, nową

cenę y nalezy podnieść o

A. 25\%

B. 20\%

C. 15\%

D. 12\%

Zadanie 5. $(1pkt)$

Zbiorem wszystkich rozwiązań nierówności 3 $(1-x)>2(3x-1)-12x$ jest przedział

A.

$(-\displaystyle \frac{5}{3},+\infty)$

B.

(-$\infty$, -35)

C.

$(\displaystyle \frac{5}{3},+\infty)$

D.

(-$\infty$'- -35)

Zadanie $\epsilon. (1pkt)$

Suma wszystkich rozwiązań równania $x(x-3)(x+2)=0$ jest równa

A. 0

B. l

C. 2

D. 3

Strona 2 z26

MMA-IP





{\it Egzamin maturalny z matematyki}

{\it Poziom podstawowy}

{\it BRUDNOPIS} ({\it nie podlega ocenie})

MMA-IP

Strona ll z26





{\it Egzamin maturalny z matematyki}

{\it Poziom podstawowy}

Zadanie 23. (1pktJ

Cztery liczby: 2, 3, a, 8, tworzące zestaw danych, są uporządkowane rosnąco. Mediana tego

zestawu czterech danychjest równa medianie zestawu pięciu danych: 5, 3, 6, 8, 2. Zatem

A. $a=7$

B. $a=6$

C. $a=5$

D. $a=4$

Zadanie 24. $(1pkt)$

Dany jest sześcian ABCDEFGH. Sinus kąta $\alpha$ nachylenia przekątnej $HB$ tego sześcianu do

płaszczyzny podstawy ABCD (zobacz rysunek) jest równy

A.

$\displaystyle \frac{\sqrt{3}}{3}$

B.

$\displaystyle \frac{\sqrt{6}}{3}$

C.

-$\sqrt{}$22

D.

-$\sqrt{}$26
\begin{center}
\includegraphics[width=61.572mm,height=58.116mm]{./F1_M_PP_M2020_page11_images/image001.eps}
\end{center}
{\it H  G}

{\it E  F}

{\it D  C}

$\alpha$

{\it A  B}

Zadanie 25. $(1pkt)$

Danyjest stozek o objętości $ 18\pi$, którego przekrojem osiowymjest trójkąt ABC(zobacz rysunek).

Kąt $CBA$ jest kątem nachylenia tworzącej $l$ tego stozka do płaszczyzny jego podstawy.

Tangens kąta $CBA$ jest równy 2.
\begin{center}
\includegraphics[width=64.464mm,height=48.408mm]{./F1_M_PP_M2020_page11_images/image002.eps}
\end{center}
{\it C}

{\it l}

{\it h}

{\it A  B}

Wynika stąd, $\dot{\mathrm{z}}\mathrm{e}$ wysokość $h$ tego stozkajest równa

A. 12

B. 6

C. 4

D. 2

Strona 12 z26

MMA-IP





{\it Egzamin maturalny z matematyki}

{\it Poziom podstawowy}

{\it BRUDNOPIS} ({\it nie podlega ocenie})

MMA-IP

Strona 13 z26





{\it Egzamin maturalny z matematyki}

{\it Poziom podstawowy}

Zadanie 26. $(2pktJ$

Rozwiąz nierówność 2 $(x-1)(x+3)>x-1.$

Odpowied $\acute{\mathrm{z}}$:

Strona 14 $\mathrm{z}26$

MMA-IP





{\it Egzamin maturalny z matematyki}

{\it Poziom podstawowy}

Zadanie 27. $(2pktJ$

Rozwiąz równanie $x^{3}-9x^{2}-4x+36=0.$

Odpowiedzí:
\begin{center}
\includegraphics[width=96.012mm,height=17.784mm]{./F1_M_PP_M2020_page14_images/image001.eps}
\end{center}
WypelnÍa

egzaminator

Nr zadania

Maks. lÍczba kt

2

27.

2

Uzyskana liczba pkt

MMA-IP

Strona 15 z26





{\it Egzamin maturalny z matematyki}

{\it Poziom podstawowy}

Zadanie 28. $(2pktJ$

Wykaz, ze dla kazdych dwóch róznych liczb rzeczywistych $a\mathrm{i}b$ prawdziwajest nierówność

$a(a-2b)+2b^{2}>0.$

Strona 16 z26

MMA-IP





{\it Egzamin maturalny z matematyki}

{\it Poziom podstawowy}

Zadanie 29. $(2pktJ$

Trójkąt ABCjest równoboczny. Punkt $E$ lezy na wysokości $CD$ tego trójkąta oraz $|CE|=\displaystyle \frac{3}{4}|CD|.$

Punkt $F$ lezy na boku $BC$ i odcinek $EF$ jest prostopadły do $BC$ (zobacz rysunek).
\begin{center}
\includegraphics[width=82.140mm,height=70.812mm]{./F1_M_PP_M2020_page16_images/image001.eps}
\end{center}
{\it C}

{\it F}

{\it E}

{\it A  D  B}

Wykaz, $\displaystyle \dot{\mathrm{z}}\mathrm{e}|CF|=\frac{9}{16}|CB|.$
\begin{center}
\includegraphics[width=96.012mm,height=17.784mm]{./F1_M_PP_M2020_page16_images/image002.eps}
\end{center}
Wypelnia

egzaminator

Nr zadania

Maks. liczba kt

28.

2

2

Uzyskana liczba pkt

MMA-IP

Strona 17 z26





{\it Egzamin maturalny z matematyki}

{\it Poziom podstawowy}

Zadanie 30. $(2pktJ$

Rzucamy dwa razy symetryczną sześcienną kostką do gry, która na $\mathrm{k}\mathrm{a}\dot{\mathrm{z}}$ dej ściance ma inną

liczbę oczek-odjednego oczka do sześciu oczek. Oblicz prawdopodobieństwo zdarzenia $A$

polegającego na tym, ze co najmniej jeden raz wypadnie ścianka z pięcioma oczkami.

Odpowiedzí:

Strona 18 z26

MMA-IP





{\it Egzamin maturalny z matematyki}

{\it Poziom podstawowy}

Zadanie 31. $(2pktJ$

Kąt $\alpha$ jest ostry i spełnia warunek $\displaystyle \frac{2\sin\alpha+3\cos\alpha}{\cos\alpha}=4$. Oblicz tangens kąta $\alpha.$

Odpowied $\acute{\mathrm{z}}$:
\begin{center}
\includegraphics[width=96.012mm,height=17.784mm]{./F1_M_PP_M2020_page18_images/image001.eps}
\end{center}
WypelnÍa

egzaminator

Nr zadanÍa

Maks. lÍczba kt

30.

2

31.

2

Uzyskana liczba pkt

MMA-IP

Strona 19 z26





{\it Egzamin maturalny z matematyki}

{\it Poziom podstawowy}

Zadanie 32. $(4pktJ$

Dany jest kwadrat ABCD, w którym $A=(5,-\displaystyle \frac{5}{3})$. Przekątna $BD$ tego kwadratu jest zawarta

w prostej o równaniu $y=\displaystyle \frac{4}{3}x$. Oblicz współrzędne punktu przecięcia przekątnych $AC\mathrm{i}BD$ oraz

pole kwadratu ABCD.

Strona 20 z26

MMA-IP





{\it Egzamin maturalny z matematyki}

{\it Poziom podstawowy}

{\it BRUDNOPIS} ({\it nie podlega ocenie})

MMA-IP

Strona 3 z26





{\it Egzamin maturalny z matematyki}

{\it Poziom podstawowy}

Odpowiedzí:
\begin{center}
\includegraphics[width=82.044mm,height=17.832mm]{./F1_M_PP_M2020_page20_images/image001.eps}
\end{center}
Wypelnia

egzaminator

Nr zadania

Maks. liczba kt

32.

4

Uzyskana liczba pkt

MMA-IP

Strona 21 z 26





{\it Egzamin maturalny z matematyki}

{\it Poziom podstawowy}

Zadanie 33. $(4pktJ$

Wszystkie wyrazy ciągu geometrycznego $(a_{n})$, określonego dla $n\geq 1$, są dodatnie. Wyrazy tego

ciągu spełniają warunek $6a_{1}-5a_{2}+a_{3}=0$. Oblicz iloraz

$\langle 2\sqrt{2}, 3\sqrt{2}\rangle.$

q tego ciągu nalezący do przedziału

Strona 22 z 26

MMA-IP





{\it Egzamin maturalny z matematyki}

{\it Poziom podstawowy}

Odpowiedzí:
\begin{center}
\includegraphics[width=82.044mm,height=17.832mm]{./F1_M_PP_M2020_page22_images/image001.eps}
\end{center}
Wypelnia

egzaminator

Nr zadania

Maks. liczba kt

33.

4

Uzyskana liczba pkt

MMA-IP

Strona 23 z 26





{\it Egzamin maturalny z matematyki}

{\it Poziom podstawowy}

Zadanie 34. $(SpktJ$

Dany jest ostrosłup prawidłowy czworokątny ABCDS, którego krawędzí boczna ma długość 6

(zobacz rysunek). Ściana boczna tego ostrosłupajest nachylona do płaszczyzny podstawy pod

kątem, którego tangensjest równy $\sqrt{7}$. Oblicz objętość tego ostrosłupa.

Strona 24 z 26

MMA-IP





{\it Egzamin maturalny z matematyki}

{\it Poziom podstawowy}

Odpowiedzí:
\begin{center}
\includegraphics[width=82.044mm,height=17.832mm]{./F1_M_PP_M2020_page24_images/image001.eps}
\end{center}
Wypelnia

egzaminator

Nr zadania

Maks. liczba kt

34.

5

Uzyskana liczba pkt

MMA-IP

Strona 25 z 26





{\it Egzamin maturalny z matematyki}

{\it Poziom podstawowy}

{\it BRUDNOPIS} ({\it nie podlega ocenie})

Strona 26 z 26

MMA-IP





{\it Egzamin maturalny z matematyki}

{\it Poziom podstawowy}

Informacja do zadań 7.$-9.$

Funkcja kwadratowa

f jest określona

wzorem

$f(x)=a(x-1)(x-3)$. Na rysunku

przedstawiono fragment paraboli będącej wykresem tej ffinkcji. Wierzchołkiem tej parabolijest

punkt $W=(2,1).$
\begin{center}
\includegraphics[width=117.960mm,height=97.128mm]{./F1_M_PP_M2020_page3_images/image001.eps}
\end{center}
4  {\it y}

3

2

{\it W}

1

$-3  -2 -1 0$  2 3  4 5 x

$-1$

$-2$

$-3$

Zadanie 7. (1pkt)

Współczynnik a we wzorze funkcji f jest równy

A. l

B. 2

C. $-2$

D. $-1$

Zadanie 8. $(1pkt)$

Największa wartość funkcji $f$ w przedziale $\langle$1, $ 4\rangle$ jest równa

A. $-3$

B. 0

C. l

D. 2

Zadanie 9. (1pkt)

Osią symetrii paraboli będącej wykresem funkcji f jest prosta o równaniu

A. $x=1$

B. $x=2$

C.

$y=1$

D. $y=2$

Strona 4 z26

MMA-IP





{\it Egzamin maturalny z matematyki}

{\it Poziom podstawowy}

{\it BRUDNOPIS} ({\it nie podlega ocenie})

MMA-IP

Strona 5 z26





{\it Egzamin maturalny z matematyki}

{\it Poziom podstawowy}

Zadanie 10. $(1pktJ$

Równanie $x(x-2)=(x-2)^{2}$ w zbiorze liczb rzeczywistych

A. nie ma rozwiązań.

B. ma dokładniejedno rozwiązanie: $x=2.$

C. ma dokładniejedno rozwiązanie: $x=0.$

D. ma dwa rózne rozwiązania: $x=1 \mathrm{i}x=2.$

Zadanie $l1. (1pktJ$

Na iysunku przedstawiono fiiagment wykresu funkcji liniowej $f$ określonej wzorem $f(x)=ax+b.$
\begin{center}
\includegraphics[width=118.056mm,height=97.584mm]{./F1_M_PP_M2020_page5_images/image001.eps}
\end{center}
4  {\it y}

3

1

$-3 -2$

$-1 0$

$-1$

1 2 3 4  5  {\it x}

$-2$

$-3$

Współczynniki a oraz b we wzorze funkcji f spełniają zalezność

A. $a+b>0$

B. $a+b=0$

C. $a\cdot b>0$

D. $a\cdot b<0$

Zadanie 12. $(1pktJ$

Funkcja $f$ jest określona wzorem $f(x)=4^{-x}+1$ dla $\mathrm{k}\mathrm{a}\dot{\mathrm{z}}$ dej liczby rzeczywistej $x$. Liczba $f(\displaystyle \frac{1}{2})$

jest równa

A.

-21

B.

-23

C. 3

D. 17

Zadanie 13. $(1pktJ$

Proste o równaniach $y=(m-2)x$ oraz $y=\displaystyle \frac{3}{4}x+7$ są równoległe. Wtedy

A.

{\it m}$=$- -45

B.

{\it m}$=$ -23

C.

$m=\displaystyle \frac{11}{4}$

D.

$m=\displaystyle \frac{10}{3}$

Strona 6 z26

MMA-IP





{\it Egzamin maturalny z matematyki}

{\it Poziom podstawowy}

{\it BRUDNOPIS} ({\it nie podlega ocenie})

MMA-IP

Strona 7 z 26





{\it Egzamin maturalny z matematyki}

{\it Poziom podstawowy}

Zadanie $1_{[}4. (1pktJ$

Ciąg $(a_{n})$ jest określony wzorem $a_{n}=2n^{2}$ dla $n\geq 1$. Róz$\cdot$nica $a_{5}-a_{4}$ jest równa

A. 4

B. 20

C. 36

D. 18

Zadanie 15. $(1pkt)$

$\mathrm{W}$ ciągu arytmetycznym $(a_{n})$, określonym dla $n\geq 1$, czwarty wyraz jest równy 3, a róznica

tego ciągujest równa 5. Suma $a_{1}+a_{2}+a_{3}+a_{4}$ jest równa

A. $-42$

B. $-36$

C. $-18$

D. 6

Zadanie $l6. (1pkt)$

Punkt $A=(\displaystyle \frac{1}{3},-1)$ nalezy do wykresu ffinkcji liniowej $f$ określonej wzorem $f(x)=3x+b.$

Wynika stąd, $\dot{\mathrm{z}}\mathrm{e}$

A. $b=2$

B. $b=1$

C. $b=-1$

D. $b=-2$

Zadanie $17_{c}(1pkt)$

Punkty $A, B, C, D$ lez$\cdot$ą na okręgu o środku w punkcie $O$. Kąt środkowy DOC ma miarę $118^{\mathrm{o}}$

(zobacz sunek).
\begin{center}
\includegraphics[width=57.612mm,height=61.572mm]{./F1_M_PP_M2020_page7_images/image001.eps}
\end{center}
{\it B} $D$

{\it O}  $118^{\mathrm{o}}$

{\it A C}

Miara kąta ABC jest równa

A. $59^{\mathrm{o}}$

B. $48^{\mathrm{o}}$

C. $62^{\mathrm{o}}$

D. $31^{\mathrm{o}}$

Zadanie 18. $(1pkt)$

Prosta przechodząca przez punkty $A=(3,-2)\mathrm{i}B=(-1,6)$ jest określona równaniem

A.

$y=-2x+4$

B. $y=-2x-8$

C.

$y=2x+8$

D. $y=2x-4$

Strona 8 z 26

MMA-IP





{\it Egzamin maturalny z matematyki}

{\it Poziom podstawowy}

{\it BRUDNOPIS} ({\it nie podlega ocenie})

MMA-IP

Strona 9 z 26





{\it Egzamin maturalny z matematyki}

{\it Poziom podstawowy}

Zadanie 19. $(1pktJ$

Danyjest trójkąt prostokątny o kątach ostrych $\alpha \mathrm{i}\beta$ (zobacz rysunek).

Wyrazenie $ 2\cos\alpha-\sin\beta$ jest równe

A. $ 2\sin\beta$

B.

$\cos\alpha$

C. 0

D. 2

Zadanie 20. $(1pktJ$

Punkt $B$ jest obrazem punktu $A=(-3,5) \mathrm{w}$

współrzędnych. Długość odcinka $AB$ jest równa

symetrii względem

początku układu

A. $2\sqrt{34}$

B. 8

C. $\sqrt{34}$

D. 12

Zadanie 21. (1pktJ

Ilejest wszystkich dwucyfrowych liczb naturalnych utworzonych z cyfr: 1, 3, 5, 7, 9, w których

cyfry się nie powtarzają?

A. 10

B. 15

C. 20

D. 25

Zadanie 22. $(1pkt)$

Pole prostokąta ABCDjest równe 90. Na bokach AB $\mathrm{i}$ CD wybrano- odpowiednio- punkty $P\mathrm{i}R,$

takie, $\displaystyle \dot{\mathrm{z}}\mathrm{e}\frac{|AP|}{|PB|}=\frac{|CR|}{|RD|}=\frac{3}{2}$ (zobacz rysunek).
\begin{center}
\includegraphics[width=78.180mm,height=48.672mm]{./F1_M_PP_M2020_page9_images/image001.eps}
\end{center}
{\it D R  C}

{\it A  P B}

Pole czworokąta APCR jest równe

A. 36

B. 40

C. 54

D. 60

Strona 10 z 26

MMA-IP



\end{document}