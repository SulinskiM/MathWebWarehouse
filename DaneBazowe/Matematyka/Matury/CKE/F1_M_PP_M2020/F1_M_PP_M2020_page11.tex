\documentclass[a4paper,12pt]{article}
\usepackage{latexsym}
\usepackage{amsmath}
\usepackage{amssymb}
\usepackage{graphicx}
\usepackage{wrapfig}
\pagestyle{plain}
\usepackage{fancybox}
\usepackage{bm}

\begin{document}

{\it Egzamin maturalny z matematyki}

{\it Poziom podstawowy}

Zadanie 23. (1pktJ

Cztery liczby: 2, 3, a, 8, tworzące zestaw danych, są uporządkowane rosnąco. Mediana tego

zestawu czterech danychjest równa medianie zestawu pięciu danych: 5, 3, 6, 8, 2. Zatem

A. $a=7$

B. $a=6$

C. $a=5$

D. $a=4$

Zadanie 24. $(1pkt)$

Dany jest sześcian ABCDEFGH. Sinus kąta $\alpha$ nachylenia przekątnej $HB$ tego sześcianu do

płaszczyzny podstawy ABCD (zobacz rysunek) jest równy

A.

$\displaystyle \frac{\sqrt{3}}{3}$

B.

$\displaystyle \frac{\sqrt{6}}{3}$

C.

-$\sqrt{}$22

D.

-$\sqrt{}$26
\begin{center}
\includegraphics[width=61.572mm,height=58.116mm]{./F1_M_PP_M2020_page11_images/image001.eps}
\end{center}
{\it H  G}

{\it E  F}

{\it D  C}

$\alpha$

{\it A  B}

Zadanie 25. $(1pkt)$

Danyjest stozek o objętości $ 18\pi$, którego przekrojem osiowymjest trójkąt ABC(zobacz rysunek).

Kąt $CBA$ jest kątem nachylenia tworzącej $l$ tego stozka do płaszczyzny jego podstawy.

Tangens kąta $CBA$ jest równy 2.
\begin{center}
\includegraphics[width=64.464mm,height=48.408mm]{./F1_M_PP_M2020_page11_images/image002.eps}
\end{center}
{\it C}

{\it l}

{\it h}

{\it A  B}

Wynika stąd, $\dot{\mathrm{z}}\mathrm{e}$ wysokość $h$ tego stozkajest równa

A. 12

B. 6

C. 4

D. 2

Strona 12 z26

MMA-IP
\end{document}
