\documentclass[a4paper,12pt]{article}
\usepackage{latexsym}
\usepackage{amsmath}
\usepackage{amssymb}
\usepackage{graphicx}
\usepackage{wrapfig}
\pagestyle{plain}
\usepackage{fancybox}
\usepackage{bm}

\begin{document}

{\it Egzamin maturalny z matematyki}

{\it Poziom podstawowy}

Zadanie 29. $(2pktJ$

Trójkąt ABCjest równoboczny. Punkt $E$ lezy na wysokości $CD$ tego trójkąta oraz $|CE|=\displaystyle \frac{3}{4}|CD|.$

Punkt $F$ lezy na boku $BC$ i odcinek $EF$ jest prostopadły do $BC$ (zobacz rysunek).
\begin{center}
\includegraphics[width=82.140mm,height=70.812mm]{./F1_M_PP_M2020_page16_images/image001.eps}
\end{center}
{\it C}

{\it F}

{\it E}

{\it A  D  B}

Wykaz, $\displaystyle \dot{\mathrm{z}}\mathrm{e}|CF|=\frac{9}{16}|CB|.$
\begin{center}
\includegraphics[width=96.012mm,height=17.784mm]{./F1_M_PP_M2020_page16_images/image002.eps}
\end{center}
Wypelnia

egzaminator

Nr zadania

Maks. liczba kt

28.

2

2

Uzyskana liczba pkt

MMA-IP

Strona 17 z26
\end{document}
