\documentclass[a4paper,12pt]{article}
\usepackage{latexsym}
\usepackage{amsmath}
\usepackage{amssymb}
\usepackage{graphicx}
\usepackage{wrapfig}
\pagestyle{plain}
\usepackage{fancybox}
\usepackage{bm}

\begin{document}

{\it Egzamin maturalny z matematyki}

{\it Poziom podstawowy}

Zadanie 30. $(2pktJ$

Rzucamy dwa razy symetryczną sześcienną kostką do gry, która na $\mathrm{k}\mathrm{a}\dot{\mathrm{z}}$ dej ściance ma inną

liczbę oczek-odjednego oczka do sześciu oczek. Oblicz prawdopodobieństwo zdarzenia $A$

polegającego na tym, ze co najmniej jeden raz wypadnie ścianka z pięcioma oczkami.

Odpowiedzí:

Strona 18 z26

MMA-IP
\end{document}
