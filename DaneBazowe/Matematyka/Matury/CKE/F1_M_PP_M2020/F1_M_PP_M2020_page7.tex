\documentclass[a4paper,12pt]{article}
\usepackage{latexsym}
\usepackage{amsmath}
\usepackage{amssymb}
\usepackage{graphicx}
\usepackage{wrapfig}
\pagestyle{plain}
\usepackage{fancybox}
\usepackage{bm}

\begin{document}

{\it Egzamin maturalny z matematyki}

{\it Poziom podstawowy}

Zadanie $1_{[}4. (1pktJ$

Ciąg $(a_{n})$ jest określony wzorem $a_{n}=2n^{2}$ dla $n\geq 1$. Róz$\cdot$nica $a_{5}-a_{4}$ jest równa

A. 4

B. 20

C. 36

D. 18

Zadanie 15. $(1pkt)$

$\mathrm{W}$ ciągu arytmetycznym $(a_{n})$, określonym dla $n\geq 1$, czwarty wyraz jest równy 3, a róznica

tego ciągujest równa 5. Suma $a_{1}+a_{2}+a_{3}+a_{4}$ jest równa

A. $-42$

B. $-36$

C. $-18$

D. 6

Zadanie $l6. (1pkt)$

Punkt $A=(\displaystyle \frac{1}{3},-1)$ nalezy do wykresu ffinkcji liniowej $f$ określonej wzorem $f(x)=3x+b.$

Wynika stąd, $\dot{\mathrm{z}}\mathrm{e}$

A. $b=2$

B. $b=1$

C. $b=-1$

D. $b=-2$

Zadanie $17_{c}(1pkt)$

Punkty $A, B, C, D$ lez$\cdot$ą na okręgu o środku w punkcie $O$. Kąt środkowy DOC ma miarę $118^{\mathrm{o}}$

(zobacz sunek).
\begin{center}
\includegraphics[width=57.612mm,height=61.572mm]{./F1_M_PP_M2020_page7_images/image001.eps}
\end{center}
{\it B} $D$

{\it O}  $118^{\mathrm{o}}$

{\it A C}

Miara kąta ABC jest równa

A. $59^{\mathrm{o}}$

B. $48^{\mathrm{o}}$

C. $62^{\mathrm{o}}$

D. $31^{\mathrm{o}}$

Zadanie 18. $(1pkt)$

Prosta przechodząca przez punkty $A=(3,-2)\mathrm{i}B=(-1,6)$ jest określona równaniem

A.

$y=-2x+4$

B. $y=-2x-8$

C.

$y=2x+8$

D. $y=2x-4$

Strona 8 z 26

MMA-IP
\end{document}
