\documentclass[a4paper,12pt]{article}
\usepackage{latexsym}
\usepackage{amsmath}
\usepackage{amssymb}
\usepackage{graphicx}
\usepackage{wrapfig}
\pagestyle{plain}
\usepackage{fancybox}
\usepackage{bm}

\begin{document}

$1-$

$-1\cup 1$

$-\mapsto 1$

$\mathrm{r}--$

Centralna Komisja Egzaminacyjna

Arkusz zawiera informacje prawnie chronione do momentu rozpoczęcia egzaminu.

WPISUJE ZDAJACY

KOD PESEL

{\it Miejsce}

{\it na naklejkę}

{\it z kodem}
\begin{center}
\includegraphics[width=21.432mm,height=9.804mm]{./F1_M_PP_S2012_page0_images/image001.eps}

\includegraphics[width=82.092mm,height=9.804mm]{./F1_M_PP_S2012_page0_images/image002.eps}
\end{center}
\fbox{} dysleksja
\begin{center}
\includegraphics[width=204.060mm,height=216.048mm]{./F1_M_PP_S2012_page0_images/image003.eps}
\end{center}
EGZAMIN MATU LNY

Z MATEMATYKI

SIERPIE $\acute{\mathrm{N}}$ 2012

POZIOM PODSTAWOWY

1. Sprawd $\acute{\mathrm{z}}$, czy arkusz egzaminacyjny zawiera 20 stron

(zadania $1-34$). Ewentualny brak zgłoś przewodniczącemu

zespo nadzorującego egzamin.

2. Rozwiązania zadań i odpowiedzi wpisuj w miejscu na to

przeznaczonym.

3. Odpowiedzi do zadań za niętych (l-25) przenieś

na ka ę odpowiedzi, zaznaczając je w części ka $\mathrm{y}$

przeznaczonej dla zdającego. Zamaluj $\blacksquare$ pola do tego

przeznaczone. Błędne zaznaczenie otocz kółkiem \fcircle$\bullet$

i zaznacz właściwe.

4. Pamiętaj, $\dot{\mathrm{z}}\mathrm{e}$ pominięcie argumentacji lub istotnych

obliczeń w rozwiązaniu zadania otwa ego (26-34) $\mathrm{m}\mathrm{o}\dot{\mathrm{z}}\mathrm{e}$

spowodować, $\dot{\mathrm{z}}\mathrm{e}$ za to rozwiązanie nie będziesz mógł

dostać pełnej liczby punktów.

5. Pisz czytelnie i uzywaj tvlko długopisu lub -Dióra

z czarnym tuszem lub atramentem.

6. Nie uzywaj korektora, a błędne zapisy wyrazínie prze eśl.

7. Pamiętaj, $\dot{\mathrm{z}}\mathrm{e}$ zapisy w brudnopisie nie będą oceniane.

8. $\mathrm{M}\mathrm{o}\dot{\mathrm{z}}$ esz korzystać z zestawu wzorów matematycznych,

cyrkla i linijki oraz kalkulatora.

9. Na tej stronie oraz na karcie odpowiedzi wpisz swój

numer PESEL i przyklej naklejkę z kodem.

10. Nie wpisuj $\dot{\mathrm{z}}$ adnych znaków w części przeznaczonej

dla egzaminatora.

Czas pracy:

170 minut

Liczba punktów

do uzyskania: 50

$\Vert\Vert\Vert\Vert\Vert\Vert\Vert\Vert\Vert\Vert\Vert\Vert\Vert\Vert\Vert\Vert\Vert\Vert\Vert\Vert\Vert\Vert\Vert\Vert|  \mathrm{M}\mathrm{M}\mathrm{A}-\mathrm{P}1_{-}1\mathrm{P}-124$




{\it 2}

{\it Egzamin maturalny z matematyki}

{\it Poziom podstawowy}

ZADANIA ZAMKNIĘTE

{\it Wzadaniach} $\theta d1.$ {\it do 25. wybierz i zaznacz na karcie odpowiedzipoprawnq odpowied} $\acute{z}.$

Zadanie l. $(1pkt)$

Długość boku kwadratu $k_{2}$ jest o 10\% większa od długości boku kwadratu $k_{1}$. Wówczas pole

kwadratu $k_{2}$ jest większe od pola kwadratu $k_{1}$

A. 010\%

B. 0110\%

C. 021\%

D. 0121\%

Zadanie 2. $(1pkt)$

Iloczyn $9^{-5}\cdot 3^{8}$ jest równy

A. $3^{-4}$

B. $3^{-9}$

C. $9^{-1}$

D. $9^{-9}$

Zadanie 3. $(1pkt)$

Liczba $\log_{3}27-\log_{3}1$ jest równa

A. 0

B. l

C. 2

D. 3

Zadanie 4. $(1pkt)$

Liczba $(2-3\sqrt{2})^{2}$ jest równa

A. $-14$ B. 22

$\mathrm{C}.\ -14-12\sqrt{2}$

D. $22-12\sqrt{2}$

Zadanie 5. $(1pkt)$

Liczba $(-2)$ jest miejscem zerowym ffinkcji liniowej $f(x)=mx+2$. Wtedy

A. $m=3$

B. $m=1$

C. $m=-2$

D. $m=-4$

Zadanie 6. $(1pkt)$

Wskaz rysunek, na którym jest przedstawiony zbiór rozwiązań nierówności $|x+4|\leq 7.$
\begin{center}
\includegraphics[width=172.464mm,height=13.260mm]{./F1_M_PP_S2012_page1_images/image001.eps}
\end{center}
$-11$  3  {\it x}

A.
\begin{center}
\includegraphics[width=174.804mm,height=13.416mm]{./F1_M_PP_S2012_page1_images/image002.eps}
\end{center}
$-3$  11  {\it x}

B.
\begin{center}
\includegraphics[width=175.560mm,height=13.212mm]{./F1_M_PP_S2012_page1_images/image003.eps}
\end{center}
$-11$  3  {\it x}

C.
\begin{center}
\includegraphics[width=174.804mm,height=13.356mm]{./F1_M_PP_S2012_page1_images/image004.eps}
\end{center}
$-3$  11  {\it x}

D.





{\it Egzamin maturalny z matematyki}

{\it Poziom podstawowy}

{\it 11}

Zadanie 28. (2pkt)

Pierwszy wyraz ciągu arytmetycznegojest równy 3, czwarty wyraz tego ciągu jest równy 15.

Oblicz sumę szeŚciu początkowych wyrazów tego ciągu.

Odpowiedzí :

Zadanie 29. $(2pkt)$

$\mathrm{W}$ trójkącie równoramiennym $ABC$ dane są $|AC|=|BC|=6 \mathrm{i}|\wedge ACB|=30^{\mathrm{o}}$ (zobacz rysunek).

Oblicz wysokoŚć AD trójkąta opuszczoną z wierzchołka $A$ na bok $BC.$
\begin{center}
\includegraphics[width=39.168mm,height=62.736mm]{./F1_M_PP_S2012_page10_images/image001.eps}
\end{center}
{\it C}

$30^{\mathrm{o}}$

{\it D}

{\it A B}

Odpowiedzí :





{\it 12}

{\it Egzamin maturalny z matematyki}

{\it Poziom podstawowy}

Zadanie 30. (2pkt)

Dany jest równoległobok ABCD. Na przedłuzeniu przekątnej $AC$ wybrano punkt $E$ tak, $\dot{\mathrm{z}}\mathrm{e}$

$|CE|=\displaystyle \frac{1}{2}|AC|$ (zobacz rysunek). Uzasadnij, $\dot{\mathrm{z}}\mathrm{e}$ pole równoległoboku ABCD jest cztery razy

większe od pola trójkąta $DCE.$
\begin{center}
\includegraphics[width=123.948mm,height=46.332mm]{./F1_M_PP_S2012_page11_images/image001.eps}
\end{center}
{\it E}

{\it D}

{\it C}

{\it A  B}





{\it Egzamin maturalny z matematyki}

{\it Poziom podstawowy}

{\it 13}

Zadanie 31. $(2pkt)$

Wykaz, $\dot{\mathrm{z}}\mathrm{e}\mathrm{j}\mathrm{e}\dot{\mathrm{z}}$ eli $c<0$, to trójmian kwadratowy

$y=x^{2}+bx+c$ ma dwa rózne miejsca

zerowe.





{\it 14}

{\it Egzamin maturalny z matematyki}

{\it Poziom podstawowy}

Zadanie 32. (4pkt)

Dany jest trójkąt równoramienny $ABC$, w którym $|AC|=|BC|$ oraz $A=(2,1) \mathrm{i} C=(1,9).$

Podstawa $AB$ tego trójkątajest zawarta w prostej $y=\displaystyle \frac{1}{2}x$. Oblicz współrzędne wierzchołka $B.$





{\it Egzamin maturalny z matematyki}

{\it Poziom podstawowy}

{\it 15}

Odpowied $\acute{\mathrm{z}}$:





{\it 16}

{\it Egzamin maturalny z matematyki}

{\it Poziom podstawowy}

Zadanie 33. (4pkt)

W ostrosłupie prawidłowym czworokątnym ABCDS o podstawie ABCD i wierzchołku S

trójkąt ACS jest równoboczny i ma bok długości 8. Ob1icz sinus kąta nachy1enia ściany

bocznej do płaszczyzny podstawy tego ostrosłupa (zobacz rysunek).





{\it Egzamin maturalny z matematyki}

{\it Poziom podstawowy}

{\it 1}7

Odpowied $\acute{\mathrm{z}}$:





{\it 18}

{\it Egzamin maturalny z matematyki}

{\it Poziom podstawowy}

Zadanie 34. (5pkt)

Kolarz pokonał trasę 114 km. Gdyby jechał ze średnią prędkością mniejszą o 9,5 km/h,

to pokonałby tę trasę w czasie o 2 godziny dłuzszym. Ob1icz, zjaką średnią prędkościąjechał

ten kolarz.





{\it Egzamin maturalny z matematyki}

{\it Poziom podstawowy}

{\it 19}

Odpowied $\acute{\mathrm{z}}$:





$ 2\theta$

{\it Egzamin maturalny z matematyki}

{\it Poziom podstawowy}

BRUDNOPIS





{\it Egzamin maturalny z matematyki}

{\it Poziom podstawowy}

{\it 3}

BRUDNOPIS





{\it 4}

{\it Egzamin maturalny z matematyki}

{\it Poziom podstawowy}

Zadanie 7. (1pkt)

Dana jest parabola o równaniu

paraboli jest równa

$y=x^{2}+8x-14$. Pierwsza współrzędna wierzchołka tej

A. $x=-8$

B. $x=-4$

C. $x=4$

D. $x=8$

Zadanie 8. $(1pkt)$

Wskaz fragment wykresu funkcji kwadratowej, której zbiorem wartościjest $\langle-2,+\infty$).
\begin{center}
\includegraphics[width=142.236mm,height=52.524mm]{./F1_M_PP_S2012_page3_images/image001.eps}
\end{center}
A.  B.  C.

3
\begin{center}
\includegraphics[width=43.788mm,height=52.476mm]{./F1_M_PP_S2012_page3_images/image002.eps}
\end{center}
D.

Zadanie 9. $(1pkt)$

Zbiorem rozwiązań nierówności $x(x+6)<0$ jest

A.

B.

C.

D.

$(-6,0)$

$(0,6)$

$(-\infty,-6)\cup(0,+\infty)$

$(-\infty,0)\cup(6,+\infty)$

Zadanie 10. (1pkt)

Wielomian $W(x)=x^{6}+x^{3}-2$ jest równy iloczynowi

A. $(x^{3}+1)(x^{2}-2)$

B. $(x^{3}-1)(x^{3}+2)$

C. $(x^{2}+2)(x^{4}-1)$

D. $(x^{4}-2)(x+1)$

Zadanie ll. (lpkt)

Równanie $\displaystyle \frac{(x+3)(x-2)}{(x-3)(x+2)}=0$ ma

A.

B.

C.

D.

dokładnie jedno rozwiązanie

dokładnie dwa rozwiązania

dokładnie trzy rozwiązania

dokładnie cztery rozwiązania

Zadanie 12. $(1pkt)$

Danyjest ciąg $(a_{n})$ określony wzorem $a_{n}=\displaystyle \frac{n}{(-2)^{n}}$ dla $n\geq 1$. Wówczas

A.

{\it a}3$=$ -21

B.

{\it a}3$=$ - -21

C.

{\it a}3$=$ -83

D.

{\it a}3$=$- -83





{\it Egzamin maturalny z matematyki}

{\it Poziom podstawowy}

{\it 5}

BRUDNOPIS





{\it 6}

{\it Egzamin maturalny z matematyki}

{\it Poziom podstawowy}

Zadanie 13. $(1pkt)$

$\mathrm{W}$ ciągu geometrycznym $(a_{n})$ dane są: $a_{1}=36, a_{2}=18$. Wtedy

A. $a_{4}=-18$

B. $a_{4}=0$

C. $a_{4}=4,5$

D. $a_{4}=144$

Zadanie 14. $(1pkt)$

Kąt $\alpha$ jest ostry i $\displaystyle \sin\alpha=\frac{7}{13}$. Wtedy $\mathrm{t}\mathrm{g}\alpha$ jest równy

A.

-76

B.

$\displaystyle \frac{7\cdot 13}{120}$

C.

$\displaystyle \frac{7}{\sqrt{120}}$

D.

$\displaystyle \frac{7}{13\sqrt{120}}$

Zadanie 15. (1pkt)

W trójkącie prostokątnym dane są długości boków (zobacz rysunek). Wtedy
\begin{center}
\includegraphics[width=37.344mm,height=72.384mm]{./F1_M_PP_S2012_page5_images/image001.eps}
\end{center}
$\alpha$

11

9

$2\sqrt{10}$

C.

$\displaystyle \sin\alpha=\frac{9}{11}$

B.

$\displaystyle \cos\alpha=\frac{9}{11}$

A.

$\displaystyle \sin\alpha=\frac{11}{2\sqrt{10}}$

D.

$\displaystyle \cos\alpha=\frac{2\sqrt{10}}{11}$

Zadanie 16. (1pkt)

Przekątna AC prostokąta ABCD ma długość

Diugość boku BC jest równa

14. Bok AB tego prostokąta ma długość 6.

A. 8

B. $4\sqrt{10}$

C. $\mathrm{z}\sqrt{58}$

D. 10

Zadanie 17. $(1pkt)$

Punkty $A, B \mathrm{i} C$ lez$\cdot$ą na okręgu o środku $S$ (zobacz rysunek). Miara zaznaczonego kąta

wpisanego $ACB$ jest równa
\begin{center}
\includegraphics[width=54.864mm,height=53.952mm]{./F1_M_PP_S2012_page5_images/image002.eps}
\end{center}
{\it C}

{\it A  B}

{\it S}

$230^{\mathrm{o}}$

A. $65^{\mathrm{o}}$

B. $100^{\mathrm{o}}$

C. $115^{\mathrm{o}}$

D. $130^{\mathrm{o}}$





{\it Egzamin maturalny z matematyki}

{\it Poziom podstawowy}

7

BRUDNOPIS





{\it 8}

{\it Egzamin maturalny z matematyki}

{\it Poziom podstawowy}

Zadanie 18. $(1pkt)$

Długość boku trójkąta równobocznego jest równa $24\sqrt{3}$. Promień okręgu wpisanego w ten

trójkątjest równy

A. 36

B. 18

C. 12

D. 6

Zadanie 19. $(1pkt)$

Wskaz równanie prostej przechodzącej przez początek układu współrzędnych i prostopadłej

do prostej o równaniu $y=-\displaystyle \frac{1}{3}x+2.$

A. $y=3x$ B. $y=-3x$ C. $y=3x+2$ D. $y=\displaystyle \frac{1}{3}x+2$

Zadanie 20. $(1pkt)$

Punkty $B=(-2,4) \mathrm{i} C=(5,1)$ są dwoma sąsiednimi wierzchołkami kwadratu ABCD. Pole

tego kwadratu jest równe

A. 74

B. 58

C. 40

D. 29

Zadanie 21. $(1pkt)$

Danyjest okrąg o równaniu $(x+4)^{2}+(y-6)^{2}=100$. Środek tego okręgu ma współrzędne

A. $(-4,-6)$

B. (4, 6)

C. $(4,-6)$

D. $(-4,6)$

Zadanie 22. (1pkt)

Objętość sześcianujest równa 64. Po1e powierzchni całkowitej tego sześcianu jest równe

A. 512

B. 384

C. 96

D. 16

Zadanie 23. (1pkt)

Przekrój osiowy stozka jest trójkątem równobocznym o boku $a$. Objętość tego stozka wyraz $\mathrm{a}$

się wzorem

A. $\displaystyle \frac{\sqrt{3}}{6}\pi a^{3}$ B. $\displaystyle \frac{\sqrt{3}}{8}\pi a^{3}$ C. $\displaystyle \frac{\sqrt{3}}{12}\pi a^{3}$ D. $\displaystyle \frac{\sqrt{3}}{24}\pi a^{3}$

Zadanie 24. $(1pkt)$

Pewna firma zatrudnia 6 osób. Dyrektor zarabia 8000 zł, a pensje pozostałych pracowników

są równe: 2000 zł, 2800 zł, 3400 zł, 3600 zł, 4200 zł. Mediana zarobków tych 6 osób jest

równa

A. 3400 zł

B. 3500 zł

C. 6000 zł

D. 7000 zł

Zadanie 25. (1pkt)

Ze zbioru \{1, 2, 3, 4, 5, 6, 7, 8, 9, 10, 11, 12, 13, 14, 15\} wybieramy 1osowo jedną 1iczbę. Niech

p oznacza prawdopodobieństwo otrzymania liczby podzielnej przez 4. Wówczas

A.

{\it p}$<$ -51

B.

{\it p}$=$ -51

C.

{\it p}$=$ -41

D.

{\it p}$>$ -41





{\it Egzamin maturalny z matematyki}

{\it Poziom podstawowy}

{\it 9}

BRUDNOPIS





$ 1\theta$

{\it Egzamin maturalny z matematyki}

{\it Poziom podstawowy}

ZADANIA OTWARTE

{\it Rozwiqzania zadań o numerach od 26. do 34. nalezy zapisać w wyznaczonych miejscach}

{\it pod treściq zadania}.

Zadanie 26. $(2pkt)$

Rozwiąz nierówność $x^{2}-8x+7\geq 0.$

Odpowiedzí:

Zadanie 27. $(2pkt)$

Rozwiąz równanie $x^{3}-6x^{2}-9x+54=0.$

Odpowiedzí:



\end{document}