\documentclass[a4paper,12pt]{article}
\usepackage{latexsym}
\usepackage{amsmath}
\usepackage{amssymb}
\usepackage{graphicx}
\usepackage{wrapfig}
\pagestyle{plain}
\usepackage{fancybox}
\usepackage{bm}

\begin{document}

{\it 2}

{\it Egzamin maturalny z matematyki}

{\it Poziom podstawowy}

ZADANIA ZAMKNIĘTE

{\it Wzadaniach} $\theta d1.$ {\it do 25. wybierz i zaznacz na karcie odpowiedzipoprawnq odpowied} $\acute{z}.$

Zadanie l. $(1pkt)$

Długość boku kwadratu $k_{2}$ jest o 10\% większa od długości boku kwadratu $k_{1}$. Wówczas pole

kwadratu $k_{2}$ jest większe od pola kwadratu $k_{1}$

A. 010\%

B. 0110\%

C. 021\%

D. 0121\%

Zadanie 2. $(1pkt)$

Iloczyn $9^{-5}\cdot 3^{8}$ jest równy

A. $3^{-4}$

B. $3^{-9}$

C. $9^{-1}$

D. $9^{-9}$

Zadanie 3. $(1pkt)$

Liczba $\log_{3}27-\log_{3}1$ jest równa

A. 0

B. l

C. 2

D. 3

Zadanie 4. $(1pkt)$

Liczba $(2-3\sqrt{2})^{2}$ jest równa

A. $-14$ B. 22

$\mathrm{C}.\ -14-12\sqrt{2}$

D. $22-12\sqrt{2}$

Zadanie 5. $(1pkt)$

Liczba $(-2)$ jest miejscem zerowym ffinkcji liniowej $f(x)=mx+2$. Wtedy

A. $m=3$

B. $m=1$

C. $m=-2$

D. $m=-4$

Zadanie 6. $(1pkt)$

Wskaz rysunek, na którym jest przedstawiony zbiór rozwiązań nierówności $|x+4|\leq 7.$
\begin{center}
\includegraphics[width=172.464mm,height=13.260mm]{./F1_M_PP_S2012_page1_images/image001.eps}
\end{center}
$-11$  3  {\it x}

A.
\begin{center}
\includegraphics[width=174.804mm,height=13.416mm]{./F1_M_PP_S2012_page1_images/image002.eps}
\end{center}
$-3$  11  {\it x}

B.
\begin{center}
\includegraphics[width=175.560mm,height=13.212mm]{./F1_M_PP_S2012_page1_images/image003.eps}
\end{center}
$-11$  3  {\it x}

C.
\begin{center}
\includegraphics[width=174.804mm,height=13.356mm]{./F1_M_PP_S2012_page1_images/image004.eps}
\end{center}
$-3$  11  {\it x}

D.
\end{document}
