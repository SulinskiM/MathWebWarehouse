\documentclass[a4paper,12pt]{article}
\usepackage{latexsym}
\usepackage{amsmath}
\usepackage{amssymb}
\usepackage{graphicx}
\usepackage{wrapfig}
\pagestyle{plain}
\usepackage{fancybox}
\usepackage{bm}

\begin{document}

{\it Egzamin maturalny z matematyki}

{\it Poziom podstawowy}

{\it 11}

Zadanie 28. (2pkt)

Pierwszy wyraz ciągu arytmetycznegojest równy 3, czwarty wyraz tego ciągu jest równy 15.

Oblicz sumę szeŚciu początkowych wyrazów tego ciągu.

Odpowiedzí :

Zadanie 29. $(2pkt)$

$\mathrm{W}$ trójkącie równoramiennym $ABC$ dane są $|AC|=|BC|=6 \mathrm{i}|\wedge ACB|=30^{\mathrm{o}}$ (zobacz rysunek).

Oblicz wysokoŚć AD trójkąta opuszczoną z wierzchołka $A$ na bok $BC.$
\begin{center}
\includegraphics[width=39.168mm,height=62.736mm]{./F1_M_PP_S2012_page10_images/image001.eps}
\end{center}
{\it C}

$30^{\mathrm{o}}$

{\it D}

{\it A B}

Odpowiedzí :
\end{document}
