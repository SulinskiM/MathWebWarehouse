\documentclass[a4paper,12pt]{article}
\usepackage{latexsym}
\usepackage{amsmath}
\usepackage{amssymb}
\usepackage{graphicx}
\usepackage{wrapfig}
\pagestyle{plain}
\usepackage{fancybox}
\usepackage{bm}

\begin{document}

{\it 12}

{\it Egzamin maturalny z matematyki}

{\it Poziom podstawowy}

Zadanie 30. (2pkt)

Dany jest równoległobok ABCD. Na przedłuzeniu przekątnej $AC$ wybrano punkt $E$ tak, $\dot{\mathrm{z}}\mathrm{e}$

$|CE|=\displaystyle \frac{1}{2}|AC|$ (zobacz rysunek). Uzasadnij, $\dot{\mathrm{z}}\mathrm{e}$ pole równoległoboku ABCD jest cztery razy

większe od pola trójkąta $DCE.$
\begin{center}
\includegraphics[width=123.948mm,height=46.332mm]{./F1_M_PP_S2012_page11_images/image001.eps}
\end{center}
{\it E}

{\it D}

{\it C}

{\it A  B}
\end{document}
