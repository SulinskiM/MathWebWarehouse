\documentclass[a4paper,12pt]{article}
\usepackage{latexsym}
\usepackage{amsmath}
\usepackage{amssymb}
\usepackage{graphicx}
\usepackage{wrapfig}
\pagestyle{plain}
\usepackage{fancybox}
\usepackage{bm}

\begin{document}

{\it 4}

{\it Egzamin maturalny z matematyki}

{\it Poziom podstawowy}

Zadanie 7. (1pkt)

Dana jest parabola o równaniu

paraboli jest równa

$y=x^{2}+8x-14$. Pierwsza współrzędna wierzchołka tej

A. $x=-8$

B. $x=-4$

C. $x=4$

D. $x=8$

Zadanie 8. $(1pkt)$

Wskaz fragment wykresu funkcji kwadratowej, której zbiorem wartościjest $\langle-2,+\infty$).
\begin{center}
\includegraphics[width=142.236mm,height=52.524mm]{./F1_M_PP_S2012_page3_images/image001.eps}
\end{center}
A.  B.  C.

3
\begin{center}
\includegraphics[width=43.788mm,height=52.476mm]{./F1_M_PP_S2012_page3_images/image002.eps}
\end{center}
D.

Zadanie 9. $(1pkt)$

Zbiorem rozwiązań nierówności $x(x+6)<0$ jest

A.

B.

C.

D.

$(-6,0)$

$(0,6)$

$(-\infty,-6)\cup(0,+\infty)$

$(-\infty,0)\cup(6,+\infty)$

Zadanie 10. (1pkt)

Wielomian $W(x)=x^{6}+x^{3}-2$ jest równy iloczynowi

A. $(x^{3}+1)(x^{2}-2)$

B. $(x^{3}-1)(x^{3}+2)$

C. $(x^{2}+2)(x^{4}-1)$

D. $(x^{4}-2)(x+1)$

Zadanie ll. (lpkt)

Równanie $\displaystyle \frac{(x+3)(x-2)}{(x-3)(x+2)}=0$ ma

A.

B.

C.

D.

dokładnie jedno rozwiązanie

dokładnie dwa rozwiązania

dokładnie trzy rozwiązania

dokładnie cztery rozwiązania

Zadanie 12. $(1pkt)$

Danyjest ciąg $(a_{n})$ określony wzorem $a_{n}=\displaystyle \frac{n}{(-2)^{n}}$ dla $n\geq 1$. Wówczas

A.

{\it a}3$=$ -21

B.

{\it a}3$=$ - -21

C.

{\it a}3$=$ -83

D.

{\it a}3$=$- -83
\end{document}
