\documentclass[a4paper,12pt]{article}
\usepackage{latexsym}
\usepackage{amsmath}
\usepackage{amssymb}
\usepackage{graphicx}
\usepackage{wrapfig}
\pagestyle{plain}
\usepackage{fancybox}
\usepackage{bm}

\begin{document}

{\it 6}

{\it Egzamin maturalny z matematyki}

{\it Poziom podstawowy}

Zadanie 13. $(1pkt)$

$\mathrm{W}$ ciągu geometrycznym $(a_{n})$ dane są: $a_{1}=36, a_{2}=18$. Wtedy

A. $a_{4}=-18$

B. $a_{4}=0$

C. $a_{4}=4,5$

D. $a_{4}=144$

Zadanie 14. $(1pkt)$

Kąt $\alpha$ jest ostry i $\displaystyle \sin\alpha=\frac{7}{13}$. Wtedy $\mathrm{t}\mathrm{g}\alpha$ jest równy

A.

-76

B.

$\displaystyle \frac{7\cdot 13}{120}$

C.

$\displaystyle \frac{7}{\sqrt{120}}$

D.

$\displaystyle \frac{7}{13\sqrt{120}}$

Zadanie 15. (1pkt)

W trójkącie prostokątnym dane są długości boków (zobacz rysunek). Wtedy
\begin{center}
\includegraphics[width=37.344mm,height=72.384mm]{./F1_M_PP_S2012_page5_images/image001.eps}
\end{center}
$\alpha$

11

9

$2\sqrt{10}$

C.

$\displaystyle \sin\alpha=\frac{9}{11}$

B.

$\displaystyle \cos\alpha=\frac{9}{11}$

A.

$\displaystyle \sin\alpha=\frac{11}{2\sqrt{10}}$

D.

$\displaystyle \cos\alpha=\frac{2\sqrt{10}}{11}$

Zadanie 16. (1pkt)

Przekątna AC prostokąta ABCD ma długość

Diugość boku BC jest równa

14. Bok AB tego prostokąta ma długość 6.

A. 8

B. $4\sqrt{10}$

C. $\mathrm{z}\sqrt{58}$

D. 10

Zadanie 17. $(1pkt)$

Punkty $A, B \mathrm{i} C$ lez$\cdot$ą na okręgu o środku $S$ (zobacz rysunek). Miara zaznaczonego kąta

wpisanego $ACB$ jest równa
\begin{center}
\includegraphics[width=54.864mm,height=53.952mm]{./F1_M_PP_S2012_page5_images/image002.eps}
\end{center}
{\it C}

{\it A  B}

{\it S}

$230^{\mathrm{o}}$

A. $65^{\mathrm{o}}$

B. $100^{\mathrm{o}}$

C. $115^{\mathrm{o}}$

D. $130^{\mathrm{o}}$
\end{document}
