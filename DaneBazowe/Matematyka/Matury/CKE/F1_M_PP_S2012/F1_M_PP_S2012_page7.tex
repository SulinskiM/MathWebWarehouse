\documentclass[a4paper,12pt]{article}
\usepackage{latexsym}
\usepackage{amsmath}
\usepackage{amssymb}
\usepackage{graphicx}
\usepackage{wrapfig}
\pagestyle{plain}
\usepackage{fancybox}
\usepackage{bm}

\begin{document}

{\it 8}

{\it Egzamin maturalny z matematyki}

{\it Poziom podstawowy}

Zadanie 18. $(1pkt)$

Długość boku trójkąta równobocznego jest równa $24\sqrt{3}$. Promień okręgu wpisanego w ten

trójkątjest równy

A. 36

B. 18

C. 12

D. 6

Zadanie 19. $(1pkt)$

Wskaz równanie prostej przechodzącej przez początek układu współrzędnych i prostopadłej

do prostej o równaniu $y=-\displaystyle \frac{1}{3}x+2.$

A. $y=3x$ B. $y=-3x$ C. $y=3x+2$ D. $y=\displaystyle \frac{1}{3}x+2$

Zadanie 20. $(1pkt)$

Punkty $B=(-2,4) \mathrm{i} C=(5,1)$ są dwoma sąsiednimi wierzchołkami kwadratu ABCD. Pole

tego kwadratu jest równe

A. 74

B. 58

C. 40

D. 29

Zadanie 21. $(1pkt)$

Danyjest okrąg o równaniu $(x+4)^{2}+(y-6)^{2}=100$. Środek tego okręgu ma współrzędne

A. $(-4,-6)$

B. (4, 6)

C. $(4,-6)$

D. $(-4,6)$

Zadanie 22. (1pkt)

Objętość sześcianujest równa 64. Po1e powierzchni całkowitej tego sześcianu jest równe

A. 512

B. 384

C. 96

D. 16

Zadanie 23. (1pkt)

Przekrój osiowy stozka jest trójkątem równobocznym o boku $a$. Objętość tego stozka wyraz $\mathrm{a}$

się wzorem

A. $\displaystyle \frac{\sqrt{3}}{6}\pi a^{3}$ B. $\displaystyle \frac{\sqrt{3}}{8}\pi a^{3}$ C. $\displaystyle \frac{\sqrt{3}}{12}\pi a^{3}$ D. $\displaystyle \frac{\sqrt{3}}{24}\pi a^{3}$

Zadanie 24. $(1pkt)$

Pewna firma zatrudnia 6 osób. Dyrektor zarabia 8000 zł, a pensje pozostałych pracowników

są równe: 2000 zł, 2800 zł, 3400 zł, 3600 zł, 4200 zł. Mediana zarobków tych 6 osób jest

równa

A. 3400 zł

B. 3500 zł

C. 6000 zł

D. 7000 zł

Zadanie 25. (1pkt)

Ze zbioru \{1, 2, 3, 4, 5, 6, 7, 8, 9, 10, 11, 12, 13, 14, 15\} wybieramy 1osowo jedną 1iczbę. Niech

p oznacza prawdopodobieństwo otrzymania liczby podzielnej przez 4. Wówczas

A.

{\it p}$<$ -51

B.

{\it p}$=$ -51

C.

{\it p}$=$ -41

D.

{\it p}$>$ -41
\end{document}
