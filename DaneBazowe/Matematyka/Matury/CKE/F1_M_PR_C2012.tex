\documentclass[a4paper,12pt]{article}
\usepackage{latexsym}
\usepackage{amsmath}
\usepackage{amssymb}
\usepackage{graphicx}
\usepackage{wrapfig}
\pagestyle{plain}
\usepackage{fancybox}
\usepackage{bm}

\begin{document}

Centralna Komisja Egzaminacyjna

Arkusz zawiera informacje prawnie chronione do momentu rozpoczęcia egzaminu.

WPISUJE ZDAJACY

KOD PESEL

{\it Miejsce}

{\it na naklejkę}

{\it z kodem}
\begin{center}
\includegraphics[width=21.432mm,height=9.804mm]{./F1_M_PR_C2012_page0_images/image001.eps}

\includegraphics[width=82.092mm,height=9.804mm]{./F1_M_PR_C2012_page0_images/image002.eps}
\end{center}
\fbox{} dysleksja
\begin{center}
\includegraphics[width=204.060mm,height=216.048mm]{./F1_M_PR_C2012_page0_images/image003.eps}
\end{center}
EGZAMIN MATU LNY

Z MATEMATYKI

CZERWIEC 2012

POZIOM ROZSZERZONY

1.

3.

Sprawd $\acute{\mathrm{z}}$, czy arkusz egzaminacyjny zawiera 20 stron

(zadania $1-12$). Ewentualny brak zgłoś

przewodniczącemu zespołu nadzorującego egzamin.

Rozwiązania zadań i odpowiedzi wpisuj w miejscu na to

przeznaczonym.

Pamiętaj, $\dot{\mathrm{z}}\mathrm{e}$ pominięcie argumentacji lub istotnych

obliczeń w rozwiązaniu zadania otwa ego $\mathrm{m}\mathrm{o}\dot{\mathrm{z}}\mathrm{e}$

spowodować, $\dot{\mathrm{z}}\mathrm{e}$ za to rozwiązanie nie będziesz mógł

dostać pełnej liczby punktów.

Pisz czytelnie i uzywaj tvlko długopisu lub -Dióra

z czarnym tuszem lub atramentem.

Nie uzywaj korektora, a błędne zapisy wyra $\acute{\mathrm{z}}\mathrm{n}\mathrm{i}\mathrm{e}$ prze eśl.

Pamiętaj, $\dot{\mathrm{z}}\mathrm{e}$ zapisy w brudnopisie nie będą oceniane.

$\mathrm{M}\mathrm{o}\dot{\mathrm{z}}$ esz korzystać z zestawu wzorów matematycznych,

cyrkla i linijki oraz kalkulatora.

Na tej stronie oraz na karcie odpowiedzi wpisz swój

numer PESEL i przyklej naklejkę z kodem.

Nie wpisuj $\dot{\mathrm{z}}$ adnych znaków w części przeznaczonej

dla egzaminatora.

Czas pracy:

180 minut

2.

4.

5.

6.

7.

8.

9.

Liczba punktów

do uzyskania: 50

$\Vert\Vert\Vert\Vert\Vert\Vert\Vert\Vert\Vert\Vert\Vert\Vert\Vert\Vert\Vert\Vert\Vert\Vert\Vert\Vert\Vert\Vert\Vert\Vert|  \mathrm{M}\mathrm{M}\mathrm{A}-\mathrm{R}1_{-}1\mathrm{P}-123$




{\it 2}

{\it Egzamin maturalny z matematyki}

{\it Poziom rozszerzony}

Zadanie l. $(4pkt)$

Rozwiąz nierówność $|x-2|+|x+1|\geq 3x-3.$





{\it Egzamin maturalny z matematyki}

{\it Poziom rozszerzony}

{\it 11}

Odpowiedzí :





{\it 12}

{\it Egzamin maturalny z matematyki}

{\it Poziom rozszerzony}

Zadanie 8. $(5pkt)$

$\mathrm{W}$ czworokącie ABCD dane są długości boków: $|AB|=24, |CD|=15, |AD|=7$. Ponadto kąty

$DAB$ oraz $BCD$ sąproste. Oblicz pole tego czworokąta oraz długościjego przekątnych.





{\it Egzamin maturalny z matematyki}

{\it Poziom rozszerzony}

{\it 13}

Odpowiedzí :





{\it 14}

{\it Egzamin maturalny z matematyki}

{\it Poziom rozszerzony}

Zadanie 9. (3pkt)

Oblicz, ile jest liczb naturalnych trzycyfrowych podzielnych przez 6 1ub

przez 15.

podzielnych

Odpowiedzí:





{\it Egzamin maturalny z matematyki}

{\it Poziom rozszerzony}

{\it 15}

Zadanie 10. $(4pkt)$

Na płaszczyzínie dane są punkty $A=(3,-2) \mathrm{i}B=(11,4)$. Na prostej o równaniu $y=8x+10$

znajdz$\ovalbox{\tt\small REJECT}$ punkt $P$, dla którego suma $|AP|^{2}+|BP|^{2}$ jest najmniejsza.

Odpowiedzí :





{\it 16}

{\it Egzamin maturalny z matematyki}

{\it Poziom rozszerzony}

Zadanie ll. $(5pkt)$

Podstawą ostrosłupa ABCS jest trójkąt równoramienny $ABC$, w którym $|AB|=30,$

$|BC|=|AC|=39$ i spodek wysokości ostrosłupa nalez$\mathrm{y}$ do jego podstawy. $\mathrm{K}\mathrm{a}\dot{\mathrm{z}}$ da wysokość

ściany bocznej poprowadzona z wierzchołka $S$ ma długość 26. Ob1icz objętość tego

ostrosłupa.





{\it Egzamin maturalny z matematyki}

{\it Poziom rozszerzony}

17

Odpowied $\acute{\mathrm{z}}$:





{\it 18}

{\it Egzamin maturalny z matematyki}

{\it Poziom rozszerzony}

Zadanie 12. $(3pkt)$

Zdarzenia losowe $A, B$ są zawarte w $\Omega$ oraz $P(A\cap B')=0,1 \mathrm{i}P(A'\cap B)=0,2$. Wykaz, $\dot{\mathrm{z}}\mathrm{e}$

$P(A\cap B)\leq 0,7$ (A'oznacza zdarzenie przeciwne do zdarzenia

przeciwne do zdarzenia $B$).

A, B'oznacza zdarzenie





{\it Egzamin maturalny z matematyki}

{\it Poziom rozszerzony}

{\it 19}

Odpowied $\acute{\mathrm{z}}$:





$ 2\theta$

{\it Egzamin maturalny z matematyki}

{\it Poziom rozszerzony}

BRUDNOPIS





{\it Egzamin maturalny z matematyki}

{\it Poziom rozszerzony}

{\it 3}

Odpowied $\acute{\mathrm{z}}$:





{\it 4}

{\it Egzamin maturalny z matematyki}

{\it Poziom rozszerzony}

Zadanie 2. $(4pkt)$

Wielomian $W(x)=x^{4}+ax^{3}+bx^{2}-24x+9$ jest kwadratem wielomianu $P(x)=x^{2}+cx+d.$

Oblicz $a$ oraz $b.$

Odpowiedzí:





{\it Egzamin maturalny z matematyki}

{\it Poziom rozszerzony}

{\it 5}

Zadanie 3. $(5pkt)$

Kąt $\alpha$ jest taki, $\displaystyle \dot{\mathrm{z}}\mathrm{e}\cos\alpha+\sin\alpha=\frac{4}{3}$. Oblicz wartość wyrazenia $|\cos\alpha-\sin\alpha|.$

Odpowied $\acute{\mathrm{z}}$:





{\it 6}

{\it Egzamin maturalny z matematyki}

{\it Poziom rozszerzony}

Zadanie 4. $(5pkt)$

Wyznacz wszystkie wartości parametru $m$, dla których równanie $2x^{2}+(3-2m)x-m+1=0$

ma dwa rózne pierwiastki $x_{1}, x_{2}$ takie, $\dot{\mathrm{z}}\mathrm{e}|x_{1}-x_{2}|=3.$





{\it Egzamin maturalny z matematyki}

{\it Poziom rozszerzony}

7

Odpowied $\acute{\mathrm{z}}$:





{\it 8}

{\it Egzamin maturalny z matematyki}

{\it Poziom rozszerzony}

Zadanie 5. (5pkt)

W ciągu arytmetycznym

$(a_{n})$, dla $n\geq 1$, dane są

$a_{1}=-2$ oraz róz$\mathrm{m}\mathrm{c}\mathrm{a} r=3$. Oblicz

największe $n$ takie, $\dot{\mathrm{z}}\mathrm{e}a_{1}+a_{2}+\ldots+a_{n}<2012.$

Odpowiedzí:





{\it Egzamin maturalny z matematyki}

{\it Poziom rozszerzony}

{\it 9}

Zadanie 6. $(3pkt)$

Udowodnij, $\dot{\mathrm{z}}\mathrm{e}$ dla dowolnych liczb dodatnich $a, b, c \mathrm{i} d$ prawdziwa jest nierówność

$ac+bd\leq\sqrt{a^{2}+b^{2}}\cdot\sqrt{c^{2}+d^{2}}.$





$ 1\theta$

{\it Egzamin maturalny z matematyki}

{\it Poziom rozszerzony}

Zadanie 7. $(4pkt)$

Okrąg jest styczny do osi układu współrzędnych w punktach $A=(0,2) \mathrm{i}B=(2,0)$ oraz jest

styczny do prostej $l$ w punkcie $C=(1,a)$, gdzie $a>1$. Wyznacz równanie prostej $l.$



\end{document}