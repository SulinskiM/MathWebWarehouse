\documentclass[a4paper,12pt]{article}
\usepackage{latexsym}
\usepackage{amsmath}
\usepackage{amssymb}
\usepackage{graphicx}
\usepackage{wrapfig}
\pagestyle{plain}
\usepackage{fancybox}
\usepackage{bm}

\begin{document}

{\it 16}

{\it Egzamin maturalny z matematyki}

{\it Poziom rozszerzony}

Zadanie ll. $(5pkt)$

Podstawą ostrosłupa ABCS jest trójkąt równoramienny $ABC$, w którym $|AB|=30,$

$|BC|=|AC|=39$ i spodek wysokości ostrosłupa nalez$\mathrm{y}$ do jego podstawy. $\mathrm{K}\mathrm{a}\dot{\mathrm{z}}$ da wysokość

ściany bocznej poprowadzona z wierzchołka $S$ ma długość 26. Ob1icz objętość tego

ostrosłupa.
\end{document}
