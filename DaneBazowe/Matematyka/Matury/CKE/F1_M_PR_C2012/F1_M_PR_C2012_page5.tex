\documentclass[a4paper,12pt]{article}
\usepackage{latexsym}
\usepackage{amsmath}
\usepackage{amssymb}
\usepackage{graphicx}
\usepackage{wrapfig}
\pagestyle{plain}
\usepackage{fancybox}
\usepackage{bm}

\begin{document}

{\it 6}

{\it Egzamin maturalny z matematyki}

{\it Poziom rozszerzony}

Zadanie 4. $(5pkt)$

Wyznacz wszystkie wartości parametru $m$, dla których równanie $2x^{2}+(3-2m)x-m+1=0$

ma dwa rózne pierwiastki $x_{1}, x_{2}$ takie, $\dot{\mathrm{z}}\mathrm{e}|x_{1}-x_{2}|=3.$
\end{document}
