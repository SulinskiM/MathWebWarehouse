\documentclass[a4paper,12pt]{article}
\usepackage{latexsym}
\usepackage{amsmath}
\usepackage{amssymb}
\usepackage{graphicx}
\usepackage{wrapfig}
\pagestyle{plain}
\usepackage{fancybox}
\usepackage{bm}

\begin{document}
\begin{center}
\begin{tabular}{l|l}
\multicolumn{1}{l|}{{\it dysleksja}}&	\multicolumn{1}{|l}{}	\\
\hline
\multicolumn{1}{l|}{ $\begin{array}{l}\mbox{MATERIAL DIAGNOSTYCZNY}	\\	\mbox{Z MATEMATYKI}	\\	\mbox{Arkusz II}	\\	\mbox{POZIOM ROZSZERZONY}	\\	\mbox{Czas pracy 150 minut}	\\	\mbox{Instrukcja dla ucznia}	\\	\mbox{1. $\mathrm{S}\mathrm{p}\mathrm{r}\mathrm{a}\mathrm{w}\mathrm{d}\acute{\mathrm{z}}$, czy arkusz zawiera 12 ponumerowanych stron.}	\\	\mbox{Ewentualny brak zgłoś przewodniczącemu zespo}	\\	\mbox{nadzorującego badanie.}	\\	\mbox{2. Rozwiązania i odpowiedzi zapisz w miejscu na to}	\\	\mbox{przeznaczonym.}	\\	\mbox{3. $\mathrm{W}$ rozwiązaniach zadań przedstaw tok rozumowania}	\\	\mbox{prowadzący do ostatecznego wyniku.}	\\	\mbox{4. Pisz czytelnie. Uzywaj długopisu pióra tylko z czamym}	\\	\mbox{tusze atramentem.}	\\	\mbox{5. Nie uzywaj korektora, a błędne zapisy $\mathrm{w}\mathrm{y}\mathrm{r}\mathrm{a}\acute{\mathrm{z}}\mathrm{n}\mathrm{i}\mathrm{e}$ prze eśl.}	\\	\mbox{6. Pamiętaj, $\dot{\mathrm{z}}\mathrm{e}$ zapisy w brudnopisie nie podlegają ocenie.}	\\	\mbox{7. $\mathrm{M}\mathrm{o}\dot{\mathrm{z}}$ esz korzystać z zestawu wzorów matematycznych, cyrkla}	\\	\mbox{i linijki oraz kalkulatora.}	\\	\mbox{8. Wypełnij tę część ka $\mathrm{y}$ odpowiedzi, którą koduje uczeń. Nie}	\\	\mbox{wpisuj $\dot{\mathrm{z}}$ adnych znaków w części przeznaczonej dla}	\\	\mbox{oceniającego.}	\\	\mbox{9. Na karcie odpowiedzi wpisz swoją datę urodzenia i PESEL.}	\\	\mbox{Zamaluj $\blacksquare$ pola odpowiadające cyfrom numeru PESEL. Błędne}	\\	\mbox{zaznaczenie otocz kółkiem \fcircle i zaznacz właściwe.}	\\	\mbox{{\it Zyczymy} $p\theta wodzenia'$}	\end{array}$}&	\multicolumn{1}{|l}{$\begin{array}{l}\mbox{ARKUSZ II}	\\	\mbox{GRUDZIEN}	\\	\mbox{ROK 2005}	\\	\mbox{Za rozwiązanie}	\\	\mbox{wszystkich zadań}	\\	\mbox{mozna otrzymać}	\\	\mbox{łącznie}	\\	\mbox{50 punktów}	\end{array}$}	\\
\hline
\multicolumn{1}{l|}{$\begin{array}{l}\mbox{W ełnia uczeń rzed roz oczęciem rac}	\\	\mbox{PESEL UCZNIA}	\end{array}$}&	\multicolumn{1}{|l}{$\begin{array}{l}\mbox{Wypełnia uczeń}	\\	\mbox{przed rozpoczęciem}	\\	\mbox{pracy}	\\	\mbox{KOD UCZNIA}	\end{array}$}
\end{tabular}


\includegraphics[width=80.724mm,height=12.756mm]{./F1_M_PR_G2005_page0_images/image001.eps}

\includegraphics[width=23.616mm,height=9.852mm]{./F1_M_PR_G2005_page0_images/image002.eps}
\end{center}



{\it 2}

{\it Materialpomocniczy do doskonalenia nauczycieli w zakresie diagnozowania, oceniania i egzaminowania}

{\it Matematyka}- {\it grudzień 2005 r}.

Zadanie 11. (6pkt)

Wyznacz wszystkie liczby całkowite $k$, dla których funkcja

przyjmuje wartości dodatnie dla $\mathrm{k}\mathrm{a}\dot{\mathrm{z}}$ dego $x\in R.$

$f(x)=x^{2}-2^{k}\displaystyle \cdot x+2^{k}+\frac{5}{4}$





{\it Materialpomocniczy do doskonalenia nauczycieli w zakresie diagnozowania, oceniania i egzaminowania ll}

{\it Matematyka}- {\it grudzień 2005 r}.

Zadanie 19. (6pkt)

Korzystając z zasady indukcji matematycznej, udowodnij, $\dot{\mathrm{z}}\mathrm{e}\mathrm{k}\mathrm{a}\dot{\mathrm{z}}$ da liczba naturalna $n\geq 5$

spełnia nierówność $2^{n}>n^{2}+n-1.$





{\it 12 Materiatpomocniczy do doskonalenia nauczycieli w zakresie diagnozowania, oceniania i egzaminowania}

{\it Matematyka}- {\it grudzień 2005 r}.

BRUDNOPIS





{\it Materialpomocniczy do doskonalenia nauczycieli w zakresie diagnozowania, oceniania i egzaminowania}

{\it Matematyka}- {\it grudzień 2005 r}.

{\it 3}

Zadanie 12. (5pkt)
\begin{center}
\includegraphics[width=128.976mm,height=120.444mm]{./F1_M_PR_G2005_page2_images/image001.eps}
\end{center}
y

$-2$  1  x

Powyzszy rysunek przedstawia fragment wykresu pewnej funkcji wielomianowej $W(x)$

stopnia trzeciego. Jedynymi miejscami zerowymi tego wielomianu są liczby $(-2)$ oraz l,

a pochodna $W'(-2)=18.$

a) Wyznacz wzór wielomianu $W(x).$

b) Wyznacz równanie prostej stycznej do wykresu tego wielomianu w punkcie o odciętej

$x=3.$





{\it 4}

{\it Materialpomocniczy do doskonalenia nauczycieli w zakresie diagnozowania, oceniania i egzaminowania}

{\it Matematyka}- {\it grudzień 2005 r}.





{\it Materialpomocniczy do doskonalenia nauczycieli w zakresie diagnozowania, oceniania i egzaminowania}

{\it Matematyka}- {\it grudzień 2005 r}.

{\it 5}

Zadanie 13. (5pkt)

Sporządzí wykres funkcji $f(x)=|\displaystyle \frac{x-4}{x-2}|$, a następnie korzystając z tego wykresu, wyznacz

wszystkie wartości parametru $k$, dla których równanie $|\displaystyle \frac{x-4}{x-2}|=k$, ma dwa rozwiązania,

których iloczyn jest liczbą ujemną.





{\it 6}

{\it Materialpomocniczy do doskonalenia nauczycieli w zakresie diagnozowania, oceniania i egzaminowania}

{\it Matematyka}- {\it grudzień 2005 r}.

Zadanie 14. (4pkt)

Niech $A, B \subset \Omega$ będą zdarzeniami losowymi, takimi $\displaystyle \dot{\mathrm{z}}\mathrm{e}P(A)=\frac{5}{12}$ oraz $P(B)=\displaystyle \frac{7}{11}$

Zbadaj, czy zdarzenia $A\mathrm{i}B$ są rozłączne.





{\it Materialpomocniczy do doskonalenia nauczycieli w zakresie diagnozowania, oceniania i egzaminowania}

{\it Matematyka}- {\it grudzień 2005 r}.

7

Zadanie 15. $(5pkt)$

Dany jest nieskończony ciąg geometryczny postaci: 2, $\displaystyle \frac{2}{p-1}, \displaystyle \frac{2}{(p-1)^{2}}, \displaystyle \frac{2}{(p-1)^{3}}$, .

Wyznacz wszystkie wartości $p$, dla których granicą tego ciągu jest liczba:

a) 0.

b) 2.





{\it 8}

{\it Materialpomocniczy do doskonalenia nauczycieli w zakresie diagnozowania, oceniania i egzaminowania}

{\it Matematyka}- {\it grudzień 2005 r}.

Zadanie 16. (7pkt)

Danejest równanie postaci $(\cos x-1)\cdot(\cos x+p+1)=0$, gdzie $p\in R$ jest parametrem.

a) Dla $p=-1$ wypisz wszystkie rozwiązania tego równania nalezące do przedziału $\langle 0;5\rangle.$

b) Wyznacz wszystkie wartości parametru $p$, dla których dane równanie

ma w przedziale $\langle-\pi;\pi\rangle$ trzy rózne rozwiązania.





{\it Materialpomocniczy do doskonalenia nauczycieli w zakresie diagnozowania, oceniania i egzaminowania}

{\it Matematyka}- {\it grudzień 2005 r}.

{\it 9}

Zadanie 17. $(4pkt)$

$\mathrm{W}$ trójkącie prostokątnym $ABC(\triangleleft BCA=90^{\circ})$ dane są długości przyprostokątnych: $|BC|=a$

$\mathrm{i} |CA|=b$. Dwusieczna kąta prostego tego trójkąta przecina przeciwprostokątną

$AB$ w punkcie $D$. Wykaz, $\dot{\mathrm{z}}\mathrm{e}$ długość odcinka $CD$ jest równa $\displaystyle \frac{a\cdot b}{a+b}.\sqrt{2}$. Sporządzí

pomocniczy rysunek uwzględniając podane oznaczenia.





$ 1\theta$ {\it Materiatpomocniczy do doskonalenia nauczycieli w zakresie diagnozowania, oceniania i egzaminowania}

{\it Matematyka}- {\it grudzień 2005} $r.$

Zadanie 18. (8pkt)

Oblicz miary kątów dowolnego czworokąta wpisanego w okrąg o promieniu $R=5\sqrt{2},$

wiedząc ponadto, $\dot{\mathrm{z}}$ ejedna z przekątnych tego czworokąta ma długość 10, zaś i1oczyn sinusów

wszystkichjego kątów wewnętrznych równa się $\displaystyle \frac{3}{8}$



\end{document}