\documentclass[a4paper,12pt]{article}
\usepackage{latexsym}
\usepackage{amsmath}
\usepackage{amssymb}
\usepackage{graphicx}
\usepackage{wrapfig}
\pagestyle{plain}
\usepackage{fancybox}
\usepackage{bm}

\begin{document}

{\it Materialpomocniczy do doskonalenia nauczycieli w zakresie diagnozowania, oceniania i egzaminowania}

{\it Matematyka}- {\it grudzień 2005 r}.

{\it 3}

Zadanie 12. (5pkt)
\begin{center}
\includegraphics[width=128.976mm,height=120.444mm]{./F1_M_PR_G2005_page2_images/image001.eps}
\end{center}
y

$-2$  1  x

Powyzszy rysunek przedstawia fragment wykresu pewnej funkcji wielomianowej $W(x)$

stopnia trzeciego. Jedynymi miejscami zerowymi tego wielomianu są liczby $(-2)$ oraz l,

a pochodna $W'(-2)=18.$

a) Wyznacz wzór wielomianu $W(x).$

b) Wyznacz równanie prostej stycznej do wykresu tego wielomianu w punkcie o odciętej

$x=3.$
\end{document}
