\documentclass[a4paper,12pt]{article}
\usepackage{latexsym}
\usepackage{amsmath}
\usepackage{amssymb}
\usepackage{graphicx}
\usepackage{wrapfig}
\pagestyle{plain}
\usepackage{fancybox}
\usepackage{bm}

\begin{document}

{\it Materialpomocniczy do doskonalenia nauczycieli w zakresie diagnozowania, oceniania i egzaminowania}

{\it Matematyka}- {\it grudzień 2005 r}.

7

Zadanie 15. $(5pkt)$

Dany jest nieskończony ciąg geometryczny postaci: 2, $\displaystyle \frac{2}{p-1}, \displaystyle \frac{2}{(p-1)^{2}}, \displaystyle \frac{2}{(p-1)^{3}}$, .

Wyznacz wszystkie wartości $p$, dla których granicą tego ciągu jest liczba:

a) 0.

b) 2.
\end{document}
