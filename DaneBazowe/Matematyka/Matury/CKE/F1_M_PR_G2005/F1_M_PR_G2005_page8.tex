\documentclass[a4paper,12pt]{article}
\usepackage{latexsym}
\usepackage{amsmath}
\usepackage{amssymb}
\usepackage{graphicx}
\usepackage{wrapfig}
\pagestyle{plain}
\usepackage{fancybox}
\usepackage{bm}

\begin{document}

{\it Materialpomocniczy do doskonalenia nauczycieli w zakresie diagnozowania, oceniania i egzaminowania}

{\it Matematyka}- {\it grudzień 2005 r}.

{\it 9}

Zadanie 17. $(4pkt)$

$\mathrm{W}$ trójkącie prostokątnym $ABC(\triangleleft BCA=90^{\circ})$ dane są długości przyprostokątnych: $|BC|=a$

$\mathrm{i} |CA|=b$. Dwusieczna kąta prostego tego trójkąta przecina przeciwprostokątną

$AB$ w punkcie $D$. Wykaz, $\dot{\mathrm{z}}\mathrm{e}$ długość odcinka $CD$ jest równa $\displaystyle \frac{a\cdot b}{a+b}.\sqrt{2}$. Sporządzí

pomocniczy rysunek uwzględniając podane oznaczenia.
\end{document}
