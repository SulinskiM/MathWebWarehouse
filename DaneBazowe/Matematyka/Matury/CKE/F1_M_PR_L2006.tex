\documentclass[a4paper,12pt]{article}
\usepackage{latexsym}
\usepackage{amsmath}
\usepackage{amssymb}
\usepackage{graphicx}
\usepackage{wrapfig}
\pagestyle{plain}
\usepackage{fancybox}
\usepackage{bm}

\begin{document}
\begin{center}
\begin{tabular}{l|l}
\multicolumn{1}{l|}{$\begin{array}{l}\mbox{{\it dysleksja}}	\\	\mbox{Miejsce}	\\	\mbox{na na ejkę}	\\	\mbox{z kodem szkoly}	\end{array}$}&	\multicolumn{1}{|l}{}	\\
\hline
\multicolumn{1}{l|}{ $\begin{array}{l}\mbox{PRÓBNY EGZAMIN}	\\	\mbox{MATURALNY}	\\	\mbox{Z MATEMATYKI}	\\	\mbox{POZIOM ROZSZERZONY}	\\	\mbox{Czas pracy 180 minut}	\\	\mbox{Instrukcja dla zdającego}	\\	\mbox{1. $\mathrm{S}\mathrm{p}\mathrm{r}\mathrm{a}\mathrm{w}\mathrm{d}\acute{\mathrm{z}}$, czy arkusz egzaminacyjny zawiera 16 stron}	\\	\mbox{(zadania $1-12$). Ewentualny brak zgłoś przewodniczącemu}	\\	\mbox{zespo nadzorującego egzamin.}	\\	\mbox{2. Rozwiązania zadań i odpowiedzi zamieść w miejscu na to}	\\	\mbox{przeznaczonym.}	\\	\mbox{3. $\mathrm{W}$ rozwiązaniach zadań przedstaw tok rozumowania}	\\	\mbox{prowadzący do ostatecznego wyniku.}	\\	\mbox{4. Pisz czytelnie. Uzywaj długopisu pióra tylko z czamym}	\\	\mbox{tusze atramentem.}	\\	\mbox{5. Nie uzywaj korektora, a błędne zapisy prze eśl.}	\\	\mbox{6. Pamiętaj, $\dot{\mathrm{z}}\mathrm{e}$ zapisy w brudnopisie nie podlegają ocenie.}	\\	\mbox{7. $\mathrm{M}\mathrm{o}\dot{\mathrm{z}}$ esz korzystać z zestawu wzorów matematycznych, cyrkla}	\\	\mbox{i linijki oraz kalkulatora.}	\\	\mbox{8. Wypełnij tę część ka $\mathrm{y}$ odpowiedzi, którą koduje zdający.}	\\	\mbox{Nie wpisuj $\dot{\mathrm{z}}$ adnych znaków w części przeznaczonej dla}	\\	\mbox{egzaminatora.}	\\	\mbox{9. Na karcie odpowiedzi wpisz swoją datę urodzenia i PESEL.}	\\	\mbox{Zamaluj $\blacksquare$ pola odpowiadające cyfrom numeru PESEL. Błędne}	\\	\mbox{zaznaczenie otocz kółkiem $\mathrm{O}$ i zaznacz właściwe.}	\\	\mbox{{\it Zyczymy} $p\theta wodzenia'$}	\end{array}$}&	\multicolumn{1}{|l}{$\begin{array}{l}\mbox{LISTOPAD}	\\	\mbox{ROK 2006}	\\	\mbox{Za rozwiązanie}	\\	\mbox{wszystkich zadań}	\\	\mbox{mozna otrzymać}	\\	\mbox{łącznie}	\\	\mbox{50 punktów}	\end{array}$}	\\
\hline
\multicolumn{1}{l|}{$\begin{array}{l}\mbox{Wypelnia zdający przed}	\\	\mbox{roz oczęciem racy}	\\	\mbox{PESEL ZDAJACEGO}	\end{array}$}&	\multicolumn{1}{|l}{$\begin{array}{l}\mbox{KOD}	\\	\mbox{ZDAJACEGO}	\end{array}$}
\end{tabular}


\includegraphics[width=21.840mm,height=9.852mm]{./F1_M_PR_L2006_page0_images/image001.eps}

\includegraphics[width=78.792mm,height=13.356mm]{./F1_M_PR_L2006_page0_images/image002.eps}
\end{center}



{\it 2 Próbny egzamin maturalny z matematyki}

{\it Poziom rozszerzony}

Zadanie l. $(5pkt)$

Funkcja homograficzna

parametrem i $|p|\neq\sqrt{3}.$

f jest

określona

wzorem

$f(x)=\underline{px-3},$

$x-p$

gdzie

$p\in R$

jest

a) Dla $p=1$ zapisz wzór ffinkcji w postaci $f(x)=k+\displaystyle \frac{m}{x-1}$, gdzie $k$ oraz $m$

są liczbami rzeczywistymi.

b) Wyznacz wszystkie wartości parametru $p$, dla których w przedziale $(p,+\infty)$ funkcja $f$

jest malejąca.





{\it Próbny egzamin maturalny z matematyki ll}

{\it Poziom rozszerzony}

Zadanie 9. (3pkt)

Niech $ A\subset\Omega \mathrm{i}  B\subset\Omega$ będą zdarzeniami losowymi. Mając dane prawdopodobieństwa

zdarzeń: $P(A)=0,5, P(B)=0,4 \mathrm{i} P(A\backslash B)=0,3$, zbadaj, czy $A \mathrm{i} B$ są zdarzeniami

niezaleznymi.
\begin{center}
\includegraphics[width=195.168mm,height=254.460mm]{./F1_M_PR_L2006_page10_images/image001.eps}
\end{center}




{\it 12 Próbny egzamin maturalny z matematyki}

{\it Poziom rozszerzony}

Zadanie 10. (5pkt)

Ciąg liczbowy

$(a_{n})$

jest określony

dla $\mathrm{k}\mathrm{a}\dot{\mathrm{z}}$ dej

liczby naturalnej

$n\geq 1$ wzorem

$a_{n}=(n-3)(2-p^{2})$, gdzie $p\in R.$

a) Wykaz, $\dot{\mathrm{z}}\mathrm{e}$ dla $\mathrm{k}\mathrm{a}\dot{\mathrm{z}}$ dej wartości $p$ ciąg $(a_{n})$ jest arytmetyczny.

b) Dla $p=2$ oblicz sumę $a_{20}+a_{21}+a_{22}\cdots+a_{40}.$

c) Wyznacz wszystkie wartości $p$, dla których ciąg $(b_{n})$ określony wzorem $b_{n}=a_{n}-pn$

jest stały.
\begin{center}
\includegraphics[width=195.168mm,height=224.184mm]{./F1_M_PR_L2006_page11_images/image001.eps}
\end{center}




{\it Próbny egzamin maturalny z matematyki 13}

{\it Poziom rozszerzony}

Zadanie 11. (3pkt)

Funkcja $f$ przyporządkowuje $\mathrm{k}\mathrm{a}\dot{\mathrm{z}}$ dej liczbie naturalnej $n>1$ największą liczbę całkowitą

spełniającą nierówność $x^{2}-3nx+2n^{2}<0$ o niewiadomej $x$. Wyznacz wzór funkcji $f$





{\it 14 Próbny egzamin maturalny z matematyki}

{\it Poziom rozszerzony}

Zadanie 12. (4pkt)

Dwa okręgi, $\mathrm{k}\mathrm{a}\dot{\mathrm{z}}\mathrm{d}\mathrm{y}$ o promieniu 8, są styczne zewnętrznie. Ze środka jednego z nich

poprowadzono styczne do drugiego okręgu. Oblicz pole zacieniowanej figury (patrz rysunek).
\begin{center}
\includegraphics[width=96.516mm,height=84.432mm]{./F1_M_PR_L2006_page13_images/image001.eps}
\end{center}
{\it A  B}
\begin{center}
\includegraphics[width=195.168mm,height=163.632mm]{./F1_M_PR_L2006_page13_images/image002.eps}
\end{center}




{\it Próbny egzamin maturalny z matematyki 15}

{\it Poziom rozszerzony}
\begin{center}
\includegraphics[width=195.168mm,height=290.724mm]{./F1_M_PR_L2006_page14_images/image001.eps}
\end{center}




{\it 16 Próbny egzamin maturalny z matematyki}

{\it Poziom rozszerzony}

BRUDNOPIS





{\it Próbny egzamin maturalny z matematyki 3}

{\it Poziom rozszerzony}

Zadanie 2. (5pkt)

Wyznacz wszystkie wartości

$W(x)=(x^{2}-8x+12)\cdot(x-k)$ są

geometrycznego.

$k\in R,$

trzema

dla których pierwiastki wielomianu

kolejnymi wyrazami rosnącego ciągu
\begin{center}
\includegraphics[width=195.168mm,height=260.508mm]{./F1_M_PR_L2006_page2_images/image001.eps}
\end{center}




{\it 4 Próbny egzamin maturalny z matematyki}

{\it Poziom rozszerzony}

Zadanie 3. (4pkt)

Na rysunku ponizej przedstawiono wykres funkcji logarytmicznej f.

Rozwiąz równanie $(f(x))^{2}-16=0.$
\begin{center}
\includegraphics[width=195.168mm,height=157.584mm]{./F1_M_PR_L2006_page3_images/image001.eps}
\end{center}




{\it Próbny egzamin maturalny z matematyki 5}

{\it Poziom rozszerzony}
\begin{center}
\includegraphics[width=195.168mm,height=290.724mm]{./F1_M_PR_L2006_page4_images/image001.eps}
\end{center}




{\it 6 Próbny egzamin maturalny z matematyki}

{\it Poziom rozszerzony}

Zadanie 4. $(7pkt)$

Trójkąt prostokątny $ABC$, w którym $|\infty BCA|=90^{\circ} \mathrm{i} |\infty CAB|=30^{\circ}$, jest opisany na okręgu

o promieniu $\sqrt{3}$. Oblicz odległość wierzchołka $C$ trójkąta od punktu styczności tego okręgu

z przeciwprostokątną. Wykonaj odpowiedni rysunek.





{\it Próbny egzamin maturalny z matematyki 7}

{\it Poziom rozszerzony}

Zadanie 5. $(3pkt)$

Sporządzí wykres funkcji $f$ danej wzorem $f(x)=2|x|-x^{2}$, a następnie, korzystając z niego,

podaj wszystkie wartości $x$, dla których funkcja $f$ przyjmuje maksima lokalne i wszystkie

wartości $x$, dla których przyjmuje minima lokalne.
\begin{center}
\includegraphics[width=195.168mm,height=260.508mm]{./F1_M_PR_L2006_page6_images/image001.eps}
\end{center}




{\it 8 Próbny egzamin maturalny z matematyki}

{\it Poziom rozszerzony}

Zadanie 6. $(4pkt)$

Podstawa $AB$ trapezu ABCD jest zawarta w osi $Ox$, wierzchołek $D$ jest punktem przecięcia

paraboli o równaniu $y=-\displaystyle \frac{1}{3}x^{2}+x+6$ z osią $oy$. Pozostałe wierzchołki trapezu równiez $\mathrm{l}\mathrm{e}\dot{\mathrm{z}}$ ą

na tej paraboli (patrz rysunek). Oblicz pole tego trapezu.
\begin{center}
\includegraphics[width=83.724mm,height=69.444mm]{./F1_M_PR_L2006_page7_images/image001.eps}
\end{center}
{\it y}

{\it D C}

{\it A  B}

0  {\it x}
\begin{center}
\includegraphics[width=195.168mm,height=169.728mm]{./F1_M_PR_L2006_page7_images/image002.eps}
\end{center}




{\it Próbny egzamin maturalny z matematyki 9}

{\it Poziom rozszerzony}

Zadanie 7. (3pkt)

Wyznacz wszystkie rozwiązania równania $2\cos^{2}x=\cos x$ nalezące do przedziału $\langle 0,2\pi\rangle.$
\begin{center}
\includegraphics[width=195.168mm,height=266.544mm]{./F1_M_PR_L2006_page8_images/image001.eps}
\end{center}




$ 1\theta$ {\it Próbny egzamin maturalny z matematyki}

{\it Poziom rozszerzony}

Zadanie 8. (4pkt)

Uczeń analizował własności funkcji $f$, której dziedziną jest zbiór wszystkich liczb

rzeczywistych i która ma pochodną $f'(x)$ dla $\mathrm{k}\mathrm{a}\dot{\mathrm{z}}$ dego $x\in R$. Wyniki tej analizy zapisał

w tabeli.
\begin{center}
\begin{tabular}{|l|l|l|l|l|l|l|l|}
\hline
\multicolumn{1}{|l|}{$x$}&	\multicolumn{1}{|l|}{ $(-\infty,-1)$}&	\multicolumn{1}{|l|}{ $-1$}&	\multicolumn{1}{|l|}{ $(-1,2)$}&	\multicolumn{1}{|l|}{ $2$}&	\multicolumn{1}{|l|}{ $(2,3)$}&	\multicolumn{1}{|l|}{ $3$}&	\multicolumn{1}{|l|}{ $(3,+\infty)$}	\\
\hline
\multicolumn{1}{|l|}{ $f'(x)$}&	\multicolumn{1}{|l|}{ $(+)$}&	\multicolumn{1}{|l|}{ $0$}&	\multicolumn{1}{|l|}{ $(-)$}&	\multicolumn{1}{|l|}{ $0$}&	\multicolumn{1}{|l|}{ $(-)$}&	\multicolumn{1}{|l|}{ $0$}&	\multicolumn{1}{|l|}{ $(-)$}	\\
\hline
\multicolumn{1}{|l|}{ $f(x)$}&	\multicolumn{1}{|l|}{}&	\multicolumn{1}{|l|}{ $2$}&	\multicolumn{1}{|l|}{}&	\multicolumn{1}{|l|}{ $-1$}&	\multicolumn{1}{|l|}{}&	\multicolumn{1}{|l|}{ $1$}&	\multicolumn{1}{|l|}{}	\\
\hline
\end{tabular}

\end{center}
Niestety, wpisując znaki pochodnej, popełniłjeden błąd.

a) Przekreśl błędnie wpisany znak pochodnej i wstaw obok prawidłowy.

b) Napisz, czy po poprawieniu błędu w tabeli, zawarte w niej dane pozwolą określić

dokładną liczbę miejsc zerowych ffinkcji $f$. Uzasadniając swoją odpowiedzí $\mathrm{m}\mathrm{o}\dot{\mathrm{z}}$ esz

naszkicować przykładowe wykresy funkcji.



\end{document}