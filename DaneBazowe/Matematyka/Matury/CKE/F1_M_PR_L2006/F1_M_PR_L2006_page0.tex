\documentclass[a4paper,12pt]{article}
\usepackage{latexsym}
\usepackage{amsmath}
\usepackage{amssymb}
\usepackage{graphicx}
\usepackage{wrapfig}
\pagestyle{plain}
\usepackage{fancybox}
\usepackage{bm}

\begin{document}
\begin{center}
\begin{tabular}{l|l}
\multicolumn{1}{l|}{$\begin{array}{l}\mbox{{\it dysleksja}}	\\	\mbox{Miejsce}	\\	\mbox{na na ejkę}	\\	\mbox{z kodem szkoly}	\end{array}$}&	\multicolumn{1}{|l}{}	\\
\hline
\multicolumn{1}{l|}{ $\begin{array}{l}\mbox{PRÓBNY EGZAMIN}	\\	\mbox{MATURALNY}	\\	\mbox{Z MATEMATYKI}	\\	\mbox{POZIOM ROZSZERZONY}	\\	\mbox{Czas pracy 180 minut}	\\	\mbox{Instrukcja dla zdającego}	\\	\mbox{1. $\mathrm{S}\mathrm{p}\mathrm{r}\mathrm{a}\mathrm{w}\mathrm{d}\acute{\mathrm{z}}$, czy arkusz egzaminacyjny zawiera 16 stron}	\\	\mbox{(zadania $1-12$). Ewentualny brak zgłoś przewodniczącemu}	\\	\mbox{zespo nadzorującego egzamin.}	\\	\mbox{2. Rozwiązania zadań i odpowiedzi zamieść w miejscu na to}	\\	\mbox{przeznaczonym.}	\\	\mbox{3. $\mathrm{W}$ rozwiązaniach zadań przedstaw tok rozumowania}	\\	\mbox{prowadzący do ostatecznego wyniku.}	\\	\mbox{4. Pisz czytelnie. Uzywaj długopisu pióra tylko z czamym}	\\	\mbox{tusze atramentem.}	\\	\mbox{5. Nie uzywaj korektora, a błędne zapisy prze eśl.}	\\	\mbox{6. Pamiętaj, $\dot{\mathrm{z}}\mathrm{e}$ zapisy w brudnopisie nie podlegają ocenie.}	\\	\mbox{7. $\mathrm{M}\mathrm{o}\dot{\mathrm{z}}$ esz korzystać z zestawu wzorów matematycznych, cyrkla}	\\	\mbox{i linijki oraz kalkulatora.}	\\	\mbox{8. Wypełnij tę część ka $\mathrm{y}$ odpowiedzi, którą koduje zdający.}	\\	\mbox{Nie wpisuj $\dot{\mathrm{z}}$ adnych znaków w części przeznaczonej dla}	\\	\mbox{egzaminatora.}	\\	\mbox{9. Na karcie odpowiedzi wpisz swoją datę urodzenia i PESEL.}	\\	\mbox{Zamaluj $\blacksquare$ pola odpowiadające cyfrom numeru PESEL. Błędne}	\\	\mbox{zaznaczenie otocz kółkiem $\mathrm{O}$ i zaznacz właściwe.}	\\	\mbox{{\it Zyczymy} $p\theta wodzenia'$}	\end{array}$}&	\multicolumn{1}{|l}{$\begin{array}{l}\mbox{LISTOPAD}	\\	\mbox{ROK 2006}	\\	\mbox{Za rozwiązanie}	\\	\mbox{wszystkich zadań}	\\	\mbox{mozna otrzymać}	\\	\mbox{łącznie}	\\	\mbox{50 punktów}	\end{array}$}	\\
\hline
\multicolumn{1}{l|}{$\begin{array}{l}\mbox{Wypelnia zdający przed}	\\	\mbox{roz oczęciem racy}	\\	\mbox{PESEL ZDAJACEGO}	\end{array}$}&	\multicolumn{1}{|l}{$\begin{array}{l}\mbox{KOD}	\\	\mbox{ZDAJACEGO}	\end{array}$}
\end{tabular}


\includegraphics[width=21.840mm,height=9.852mm]{./F1_M_PR_L2006_page0_images/image001.eps}

\includegraphics[width=78.792mm,height=13.356mm]{./F1_M_PR_L2006_page0_images/image002.eps}
\end{center}\end{document}
