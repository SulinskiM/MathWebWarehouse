\documentclass[a4paper,12pt]{article}
\usepackage{latexsym}
\usepackage{amsmath}
\usepackage{amssymb}
\usepackage{graphicx}
\usepackage{wrapfig}
\pagestyle{plain}
\usepackage{fancybox}
\usepackage{bm}

\begin{document}

{\it 2 Próbny egzamin maturalny z matematyki}

{\it Poziom rozszerzony}

Zadanie l. $(5pkt)$

Funkcja homograficzna

parametrem i $|p|\neq\sqrt{3}.$

f jest

określona

wzorem

$f(x)=\underline{px-3},$

$x-p$

gdzie

$p\in R$

jest

a) Dla $p=1$ zapisz wzór ffinkcji w postaci $f(x)=k+\displaystyle \frac{m}{x-1}$, gdzie $k$ oraz $m$

są liczbami rzeczywistymi.

b) Wyznacz wszystkie wartości parametru $p$, dla których w przedziale $(p,+\infty)$ funkcja $f$

jest malejąca.
\end{document}
