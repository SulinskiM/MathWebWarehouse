\documentclass[a4paper,12pt]{article}
\usepackage{latexsym}
\usepackage{amsmath}
\usepackage{amssymb}
\usepackage{graphicx}
\usepackage{wrapfig}
\pagestyle{plain}
\usepackage{fancybox}
\usepackage{bm}

\begin{document}

{\it Próbny egzamin maturalny z matematyki ll}

{\it Poziom rozszerzony}

Zadanie 9. (3pkt)

Niech $ A\subset\Omega \mathrm{i}  B\subset\Omega$ będą zdarzeniami losowymi. Mając dane prawdopodobieństwa

zdarzeń: $P(A)=0,5, P(B)=0,4 \mathrm{i} P(A\backslash B)=0,3$, zbadaj, czy $A \mathrm{i} B$ są zdarzeniami

niezaleznymi.
\begin{center}
\includegraphics[width=195.168mm,height=254.460mm]{./F1_M_PR_L2006_page10_images/image001.eps}
\end{center}\end{document}
