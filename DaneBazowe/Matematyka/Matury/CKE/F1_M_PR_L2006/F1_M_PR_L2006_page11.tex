\documentclass[a4paper,12pt]{article}
\usepackage{latexsym}
\usepackage{amsmath}
\usepackage{amssymb}
\usepackage{graphicx}
\usepackage{wrapfig}
\pagestyle{plain}
\usepackage{fancybox}
\usepackage{bm}

\begin{document}

{\it 12 Próbny egzamin maturalny z matematyki}

{\it Poziom rozszerzony}

Zadanie 10. (5pkt)

Ciąg liczbowy

$(a_{n})$

jest określony

dla $\mathrm{k}\mathrm{a}\dot{\mathrm{z}}$ dej

liczby naturalnej

$n\geq 1$ wzorem

$a_{n}=(n-3)(2-p^{2})$, gdzie $p\in R.$

a) Wykaz, $\dot{\mathrm{z}}\mathrm{e}$ dla $\mathrm{k}\mathrm{a}\dot{\mathrm{z}}$ dej wartości $p$ ciąg $(a_{n})$ jest arytmetyczny.

b) Dla $p=2$ oblicz sumę $a_{20}+a_{21}+a_{22}\cdots+a_{40}.$

c) Wyznacz wszystkie wartości $p$, dla których ciąg $(b_{n})$ określony wzorem $b_{n}=a_{n}-pn$

jest stały.
\begin{center}
\includegraphics[width=195.168mm,height=224.184mm]{./F1_M_PR_L2006_page11_images/image001.eps}
\end{center}\end{document}
