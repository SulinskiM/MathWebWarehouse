\documentclass[a4paper,12pt]{article}
\usepackage{latexsym}
\usepackage{amsmath}
\usepackage{amssymb}
\usepackage{graphicx}
\usepackage{wrapfig}
\pagestyle{plain}
\usepackage{fancybox}
\usepackage{bm}

\begin{document}

{\it 8 Próbny egzamin maturalny z matematyki}

{\it Poziom rozszerzony}

Zadanie 6. $(4pkt)$

Podstawa $AB$ trapezu ABCD jest zawarta w osi $Ox$, wierzchołek $D$ jest punktem przecięcia

paraboli o równaniu $y=-\displaystyle \frac{1}{3}x^{2}+x+6$ z osią $oy$. Pozostałe wierzchołki trapezu równiez $\mathrm{l}\mathrm{e}\dot{\mathrm{z}}$ ą

na tej paraboli (patrz rysunek). Oblicz pole tego trapezu.
\begin{center}
\includegraphics[width=83.724mm,height=69.444mm]{./F1_M_PR_L2006_page7_images/image001.eps}
\end{center}
{\it y}

{\it D C}

{\it A  B}

0  {\it x}
\begin{center}
\includegraphics[width=195.168mm,height=169.728mm]{./F1_M_PR_L2006_page7_images/image002.eps}
\end{center}\end{document}
