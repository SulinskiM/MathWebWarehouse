\documentclass[a4paper,12pt]{article}
\usepackage{latexsym}
\usepackage{amsmath}
\usepackage{amssymb}
\usepackage{graphicx}
\usepackage{wrapfig}
\pagestyle{plain}
\usepackage{fancybox}
\usepackage{bm}

\begin{document}
\begin{center}
\begin{tabular}{l|l}
\multicolumn{1}{l|}{$\begin{array}{l}\mbox{{\it dysleksja}}	\\	\mbox{Miejsce}	\\	\mbox{na na ejkę}	\\	\mbox{z kodem szkoly}	\end{array}$}&	\multicolumn{1}{|l}{MMA-RIAIP-052}	\\
\hline
\multicolumn{1}{l|}{$\begin{array}{l}\mbox{EGZAMIN MATURALNY}	\\	\mbox{Z MATEMATYKI}	\\	\mbox{Arkusz II}	\\	\mbox{POZIOM ROZSZERZONY}	\\	\mbox{Czas pracy 150 minut}	\\	\mbox{Instrukcja dla zdającego}	\\	\mbox{1. $\mathrm{S}\mathrm{p}\mathrm{r}\mathrm{a}\mathrm{w}\mathrm{d}\acute{\mathrm{z}}$, czy arkusz egzaminacyjny zawiera 15 stron.}	\\	\mbox{Ewentualny brak zgłoś przewodniczącemu zespo}	\\	\mbox{nadzorującego egzamin.}	\\	\mbox{2. Rozwiązania zadań i odpowiedzi zamieść w miejscu na to}	\\	\mbox{przeznaczonym.}	\\	\mbox{3. $\mathrm{W}$ rozwiązaniach zadań przedstaw tok rozumowania}	\\	\mbox{prowadzący do ostatecznego wyniku.}	\\	\mbox{4. Pisz czytelnie. Uzywaj długopisu pióra tylko z czatnym}	\\	\mbox{tusze atramentem.}	\\	\mbox{5. Nie uzywaj korektora. Błędne zapisy prze eśl.}	\\	\mbox{6. Pamiętaj, $\dot{\mathrm{z}}\mathrm{e}$ zapisy w $\mathrm{b}$ dnopisie nie podlegają ocenie.}	\\	\mbox{7. Obok $\mathrm{k}\mathrm{a}\dot{\mathrm{z}}$ dego zadania podanajest maksymalna liczba punktów,}	\\	\mbox{którą mozesz uzyskać zajego poprawne rozwiązanie.}	\\	\mbox{8. $\mathrm{M}\mathrm{o}\dot{\mathrm{z}}$ esz korzystać z zestawu wzorów matematycznych, cyrkla}	\\	\mbox{i linijki oraz kalkulatora.}	\\	\mbox{9. Wypełnij tę część ka $\mathrm{y}$ odpowiedzi, którą koduje zdający.}	\\	\mbox{Nie wpisuj $\dot{\mathrm{z}}$ adnych znaków w części przeznaczonej}	\\	\mbox{dla egzaminatora.}	\\	\mbox{10. Na karcie odpowiedzi wpisz swoją datę urodzenia i PESEL.}	\\	\mbox{Zamaluj $\blacksquare$ pola odpowiadające cyfrom numeru PESEL. Błędne}	\\	\mbox{zaznaczenie otocz kółkiem i zaznacz właściwe.}	\\	\mbox{{\it Zyczymy powodzenia}.'}	\end{array}$}&	\multicolumn{1}{|l}{$\begin{array}{l}\mbox{ARKUSZ II}	\\	\mbox{MAJ}	\\	\mbox{ROK 2005}	\\	\mbox{Za rozwiązanie}	\\	\mbox{wszystkich zadań}	\\	\mbox{mozna otrzymać}	\\	\mbox{łącznie}	\\	\mbox{50 punktów}	\end{array}$}	\\
\hline
\multicolumn{1}{l|}{$\begin{array}{l}\mbox{Wypelnia zdający przed}	\\	\mbox{roz oczęciem racy}	\\	\mbox{PESEL ZDAJACEGO}	\end{array}$}&	\multicolumn{1}{|l}{$\begin{array}{l}\mbox{tylko}	\\	\mbox{O Kraków,}	\\	\mbox{OKE Wroclaw}	\\	\mbox{KOD}	\\	\mbox{ZDAJACEGO}	\end{array}$}
\end{tabular}


\includegraphics[width=78.792mm,height=13.356mm]{./F1_M_PR_M2005_page0_images/image001.eps}

\includegraphics[width=21.840mm,height=9.804mm]{./F1_M_PR_M2005_page0_images/image002.eps}
\end{center}



{\it 2}

{\it Egzamin maturalny z matematyki}

{\it Arkusz II}

Zadanie 11. (3pkt)

Wyznacz dziedzinę funkcji

przedziałów liczbowych.

$f(x)=\log_{x^{2}-3}(x^{3}+4x^{2}-x-4)$ i zapisz ją w postaci sumy
\begin{center}
\includegraphics[width=192.588mm,height=258.720mm]{./F1_M_PR_M2005_page1_images/image001.eps}
\end{center}




{\it Egzamin maturalny z matematyki}

{\it Arkusz II}

{\it 11}
\begin{center}
\includegraphics[width=192.588mm,height=294.792mm]{./F1_M_PR_M2005_page10_images/image001.eps}
\end{center}




{\it 12}

{\it Egzamin maturalny z matematyki}

{\it Arkusz II}

Zadanie 19. $(1\theta pkt)$

Dane jest równanie: $x^{2}+(m-5)x+m^{2}+m+\displaystyle \frac{1}{4}=0.$

Zbadaj, dla jakich wartości parametru $m$ stosunek sumy pierwiastków rzeczywistych

równania do ich iloczynu przyjmuje wartość najmniejszą. Wyznacz tę wartość.
\begin{center}
\includegraphics[width=192.588mm,height=258.720mm]{./F1_M_PR_M2005_page11_images/image001.eps}
\end{center}




{\it Egzamin maturalny z matematyki}

{\it Arkusz II}

{\it 13}
\begin{center}
\includegraphics[width=192.588mm,height=294.792mm]{./F1_M_PR_M2005_page12_images/image001.eps}
\end{center}




{\it 14}

{\it Egzamin maturalny z matematyki}

{\it Arkusz II}

BRUDNOPIS





{\it Egzamin maturalny z matematyki}

{\it Arkusz II}

{\it 15}





{\it Egzamin maturalny z matematyki}

{\it Arkusz II}

{\it 3}

Zadanie 12. (4pkt)

Dana jest funkcja: $f(x)=\cos x-\sqrt{3}\sin x,$

a) Naszkicuj wykres funkcji $f.$

b) Rozwiąz równanie: $f(x)=1.$

$x\in R.$
\begin{center}
\includegraphics[width=202.740mm,height=101.652mm]{./F1_M_PR_M2005_page2_images/image001.eps}
\end{center}
i

2

-2

$\rightarrow 2$
\begin{center}
\includegraphics[width=192.588mm,height=156.516mm]{./F1_M_PR_M2005_page2_images/image002.eps}
\end{center}




{\it 4}

{\it Egzamin maturalny z matematyki}

{\it Arkusz II}

Zadanie 13. (4pkt)

Rzucamy $n$ razy dwiema symetrycznymi sześciennymi kostkami do gry. Oblicz, dlajakich $n$

prawdopodobieństwo otrzymania co najmniej raz tej samej liczby oczek na obu kostkachjest

mniejsze od $\displaystyle \frac{671}{1296}.$
\begin{center}
\includegraphics[width=192.588mm,height=252.732mm]{./F1_M_PR_M2005_page3_images/image001.eps}
\end{center}




{\it Egzamin maturalny z matematyki}

{\it Arkusz II}

{\it 5}

Zadanie 14. (5pkt)

Ob1icz: {\it n}1i$\rightarrow$m$\infty$ -51 $++$47 $++$79 $++$...... $++$((32{\it nn} -$+$23)).
\begin{center}
\includegraphics[width=192.588mm,height=258.720mm]{./F1_M_PR_M2005_page4_images/image001.eps}
\end{center}




{\it 6}

{\it Egzamin maturalny z matematyki}

{\it Arkusz II}

Zadanie 15. (4pkt)

W dowolnym trójkącie ABC punkty MiN są odpowiednio środkami boków ACiBC (Rys. l).

{\it c}
\begin{center}
\includegraphics[width=86.712mm,height=43.992mm]{./F1_M_PR_M2005_page5_images/image001.eps}
\end{center}
Rys. l

{\it A  B}

Zapoznaj się uwaznie z następującym rozumowaniem:

Korzystając z własności wektorów i działań na wektorach, zapisujemy równoŚci:

oraz

$\vec{MN}=\vec{MA}+\vec{AB}+\vec{BN}$ (1)

$\vec{MN}=\vec{MC}+\vec{CN}$ (2)

Po dodaniu równości (l) $\mathrm{i}$ (2) stronami otrzymujemy:

2. $\vec{MN}=\vec{MA}+\vec{MC}+\vec{AB}+\vec{BN}+\vec{CN}$

Poniewaz $\vec{MC}=-\vec{MA}$ oraz $\vec{CN}=-\vec{BN}$, więc:

2. $\vec{MN}=\vec{MA}-\vec{MA}+\vec{AB}+\vec{BN}-\vec{BN}$

2. $\vec{MN}=\vec{\text{Õ}}+\vec{AB}+\vec{0}$

$\displaystyle \vec{MN}=\frac{1}{2}\cdot\vec{AB}.$

Wykorzystując własności iloczynu wektora przez liczbę, ostatnią równość

zinterpretować następująco:

mozna

odcinek lączący środki dwóch boków dowolnego trójkąta jest równolegly do trzeciego

boku tego trójkąta, zaś jego dlugośćjest równa polowie dlugości tego boku.

Przeprowadzając analogiczne rozumowanie, ustal związek pomiędzy wektorem $\vec{MN}$ oraz

wektorami $\vec{AB} \mathrm{i} \vec{DC}$, wiedząc, $\dot{\mathrm{z}}\mathrm{e}$ czworokąt ABCD jest dowolnym trapezem, zaś punkty

$M\mathrm{i}N$ są odpowiednio środkami ramion AD $\mathrm{i}BC$ tego trapezu (Rys. 2).

Rys. 2
\begin{center}
\includegraphics[width=91.536mm,height=46.020mm]{./F1_M_PR_M2005_page5_images/image002.eps}
\end{center}
{\it A}

Podaj interpretację otrzymanego wyniku.





{\it Egzamin maturalny z matematyki}

{\it Arkusz II}

7
\begin{center}
\includegraphics[width=192.588mm,height=294.792mm]{./F1_M_PR_M2005_page6_images/image001.eps}
\end{center}




{\it 8}

{\it Egzamin maturalny z matematyki}

{\it Arkusz II}

Zadanie 16. (5pkt)

Sześcian o krawędzi długości $a$ przecięto płaszczyzną przechodzącą przez przekątną

podstawy i nachyloną do płaszczyzny podstawy pod kątem $\displaystyle \frac{\pi}{3}$. Sporządz$\acute{}$ odpowiedni rysunek.

Oblicz pole otrzymanego przekroju.
\begin{center}
\includegraphics[width=192.588mm,height=252.732mm]{./F1_M_PR_M2005_page7_images/image001.eps}
\end{center}




{\it Egzamin maturalny z matematyki}

{\it Arkusz II}

{\it 9}

Zadanie 17. (7pkt)

Wykaz, bez uzycia kalkulatora i tablic, $\dot{\mathrm{z}}\mathrm{e}\sqrt[3]{5\sqrt{2}+7}-\sqrt[3]{5\sqrt{2}-7}$jest liczbą całkowitą.
\begin{center}
\includegraphics[width=192.588mm,height=264.720mm]{./F1_M_PR_M2005_page8_images/image001.eps}
\end{center}




$ 1\theta$

{\it Egzamin maturalny z matematyki}

{\it Arkusz II}

Zadanie 18. (8pkt)

Pary liczb $(x,y)$ spełniające układ równań:

$\left\{\begin{array}{l}
-4x^{2}+y^{2}+2y+1=0\\
-x^{2}+y+4=0
\end{array}\right.$

są współrzędnymi wierzchołków czworokąta wypukłego ABCD.

a) Wyznacz współrzędne punktów: $A, B, C, D.$

b) Wykaz, $\dot{\mathrm{z}}\mathrm{e}$ czworokąt ABCD jest trapezem równoramiennym.

c) Wyznacz równanie okręgu opisanego na czworokącie ABCD.
\begin{center}
\includegraphics[width=192.588mm,height=228.648mm]{./F1_M_PR_M2005_page9_images/image001.eps}
\end{center}


\end{document}