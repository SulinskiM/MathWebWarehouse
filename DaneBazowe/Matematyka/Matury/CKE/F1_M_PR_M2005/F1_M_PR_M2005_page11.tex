\documentclass[a4paper,12pt]{article}
\usepackage{latexsym}
\usepackage{amsmath}
\usepackage{amssymb}
\usepackage{graphicx}
\usepackage{wrapfig}
\pagestyle{plain}
\usepackage{fancybox}
\usepackage{bm}

\begin{document}

{\it 12}

{\it Egzamin maturalny z matematyki}

{\it Arkusz II}

Zadanie 19. $(1\theta pkt)$

Dane jest równanie: $x^{2}+(m-5)x+m^{2}+m+\displaystyle \frac{1}{4}=0.$

Zbadaj, dla jakich wartości parametru $m$ stosunek sumy pierwiastków rzeczywistych

równania do ich iloczynu przyjmuje wartość najmniejszą. Wyznacz tę wartość.
\begin{center}
\includegraphics[width=192.588mm,height=258.720mm]{./F1_M_PR_M2005_page11_images/image001.eps}
\end{center}\end{document}
