\documentclass[a4paper,12pt]{article}
\usepackage{latexsym}
\usepackage{amsmath}
\usepackage{amssymb}
\usepackage{graphicx}
\usepackage{wrapfig}
\pagestyle{plain}
\usepackage{fancybox}
\usepackage{bm}

\begin{document}

{\it 6}

{\it Egzamin maturalny z matematyki}

{\it Arkusz II}

Zadanie 15. (4pkt)

W dowolnym trójkącie ABC punkty MiN są odpowiednio środkami boków ACiBC (Rys. l).

{\it c}
\begin{center}
\includegraphics[width=86.712mm,height=43.992mm]{./F1_M_PR_M2005_page5_images/image001.eps}
\end{center}
Rys. l

{\it A  B}

Zapoznaj się uwaznie z następującym rozumowaniem:

Korzystając z własności wektorów i działań na wektorach, zapisujemy równoŚci:

oraz

$\vec{MN}=\vec{MA}+\vec{AB}+\vec{BN}$ (1)

$\vec{MN}=\vec{MC}+\vec{CN}$ (2)

Po dodaniu równości (l) $\mathrm{i}$ (2) stronami otrzymujemy:

2. $\vec{MN}=\vec{MA}+\vec{MC}+\vec{AB}+\vec{BN}+\vec{CN}$

Poniewaz $\vec{MC}=-\vec{MA}$ oraz $\vec{CN}=-\vec{BN}$, więc:

2. $\vec{MN}=\vec{MA}-\vec{MA}+\vec{AB}+\vec{BN}-\vec{BN}$

2. $\vec{MN}=\vec{\text{Õ}}+\vec{AB}+\vec{0}$

$\displaystyle \vec{MN}=\frac{1}{2}\cdot\vec{AB}.$

Wykorzystując własności iloczynu wektora przez liczbę, ostatnią równość

zinterpretować następująco:

mozna

odcinek lączący środki dwóch boków dowolnego trójkąta jest równolegly do trzeciego

boku tego trójkąta, zaś jego dlugośćjest równa polowie dlugości tego boku.

Przeprowadzając analogiczne rozumowanie, ustal związek pomiędzy wektorem $\vec{MN}$ oraz

wektorami $\vec{AB} \mathrm{i} \vec{DC}$, wiedząc, $\dot{\mathrm{z}}\mathrm{e}$ czworokąt ABCD jest dowolnym trapezem, zaś punkty

$M\mathrm{i}N$ są odpowiednio środkami ramion AD $\mathrm{i}BC$ tego trapezu (Rys. 2).

Rys. 2
\begin{center}
\includegraphics[width=91.536mm,height=46.020mm]{./F1_M_PR_M2005_page5_images/image002.eps}
\end{center}
{\it A}

Podaj interpretację otrzymanego wyniku.
\end{document}
