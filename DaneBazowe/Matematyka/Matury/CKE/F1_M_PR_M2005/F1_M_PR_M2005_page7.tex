\documentclass[a4paper,12pt]{article}
\usepackage{latexsym}
\usepackage{amsmath}
\usepackage{amssymb}
\usepackage{graphicx}
\usepackage{wrapfig}
\pagestyle{plain}
\usepackage{fancybox}
\usepackage{bm}

\begin{document}

{\it 8}

{\it Egzamin maturalny z matematyki}

{\it Arkusz II}

Zadanie 16. (5pkt)

Sześcian o krawędzi długości $a$ przecięto płaszczyzną przechodzącą przez przekątną

podstawy i nachyloną do płaszczyzny podstawy pod kątem $\displaystyle \frac{\pi}{3}$. Sporządz$\acute{}$ odpowiedni rysunek.

Oblicz pole otrzymanego przekroju.
\begin{center}
\includegraphics[width=192.588mm,height=252.732mm]{./F1_M_PR_M2005_page7_images/image001.eps}
\end{center}\end{document}
