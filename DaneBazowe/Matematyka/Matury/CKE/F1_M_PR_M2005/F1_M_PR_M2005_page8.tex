\documentclass[a4paper,12pt]{article}
\usepackage{latexsym}
\usepackage{amsmath}
\usepackage{amssymb}
\usepackage{graphicx}
\usepackage{wrapfig}
\pagestyle{plain}
\usepackage{fancybox}
\usepackage{bm}

\begin{document}

{\it Egzamin maturalny z matematyki}

{\it Arkusz II}

{\it 9}

Zadanie 17. (7pkt)

Wykaz, bez uzycia kalkulatora i tablic, $\dot{\mathrm{z}}\mathrm{e}\sqrt[3]{5\sqrt{2}+7}-\sqrt[3]{5\sqrt{2}-7}$jest liczbą całkowitą.
\begin{center}
\includegraphics[width=192.588mm,height=264.720mm]{./F1_M_PR_M2005_page8_images/image001.eps}
\end{center}\end{document}
