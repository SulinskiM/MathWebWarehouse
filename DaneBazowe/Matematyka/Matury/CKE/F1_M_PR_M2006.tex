\documentclass[a4paper,12pt]{article}
\usepackage{latexsym}
\usepackage{amsmath}
\usepackage{amssymb}
\usepackage{graphicx}
\usepackage{wrapfig}
\pagestyle{plain}
\usepackage{fancybox}
\usepackage{bm}

\begin{document}
\begin{center}
\begin{tabular}{l|l}
\multicolumn{1}{l|}{$\begin{array}{l}\mbox{{\it dysleksja}}	\\	\mbox{Miejsce}	\\	\mbox{na na ejkę}	\\	\mbox{z kodem szkoly}	\end{array}$}&	\multicolumn{1}{|l}{MMA-RIAIP-02}	\\
\hline
\multicolumn{1}{l|}{ $\begin{array}{l}\mbox{EGZAMIN MATURALNY}	\\	\mbox{Z MATEMATYKI}	\\	\mbox{Arkusz II}	\\	\mbox{POZIOM ROZSZERZONY}	\\	\mbox{Czas pracy 150 minut}	\\	\mbox{Instrukcja dla zdającego}	\\	\mbox{1. Sprawdzí, czy arkusz egzaminacyjny zawiera 14 stron}	\\	\mbox{(zadania $12-21$). Ewentualny brak zgłoś przewodniczącemu}	\\	\mbox{zespo nadzorującego egzamin.}	\\	\mbox{2. Rozwiązania zadań i odpowiedzi zamieść w miejscu na to}	\\	\mbox{przeznaczonym.}	\\	\mbox{3. $\mathrm{W}$ rozwiązaniach zadań przedstaw tok rozumowania}	\\	\mbox{prowadzący do ostatecznego wyniku.}	\\	\mbox{4. Pisz czytelnie. $\mathrm{U}\dot{\mathrm{z}}$ aj długopisu pióra tylko z czarnym}	\\	\mbox{tusze atramentem.}	\\	\mbox{5. Nie uzywaj korektora, a błędne zapisy prze eśl.}	\\	\mbox{6. Pamiętaj, $\dot{\mathrm{z}}\mathrm{e}$ zapisy w $\mathrm{b}$ dnopisie nie podlegają ocenie.}	\\	\mbox{7. Obok $\mathrm{k}\mathrm{a}\dot{\mathrm{z}}$ dego zadania podanajest maksymalna liczba punktów,}	\\	\mbox{którą mozesz uzyskać zajego poprawne rozwiązanie.}	\\	\mbox{8. $\mathrm{M}\mathrm{o}\dot{\mathrm{z}}$ esz korzystać z zestawu wzorów matematycznych, cyrkla}	\\	\mbox{i linijki oraz kalkulatora.}	\\	\mbox{9. Wypełnij tę część ka $\mathrm{y}$ odpowiedzi, którą koduje zdający.}	\\	\mbox{Nie wpisuj $\dot{\mathrm{z}}$ adnych znaków w części przeznaczonej dla}	\\	\mbox{egzaminatora.}	\\	\mbox{10. Na karcie odpowiedzi wpisz swoją datę urodzenia i PESEL.}	\\	\mbox{Zamaluj $\blacksquare$ pola odpowiadające cyfrom numeru PESEL. Błędne}	\\	\mbox{zaznaczenie otocz kółkiem $\mathrm{O}$ i zaznacz właściwe.}	\\	\mbox{{\it Zyczymy} $p\theta wodzenia'$}	\end{array}$}&	\multicolumn{1}{|l}{$\begin{array}{l}\mbox{ARKUSZ II}	\\	\mbox{MAJ}	\\	\mbox{ROK 2006}	\\	\mbox{Za rozwiązanie}	\\	\mbox{wszystkich zadań}	\\	\mbox{mozna otrzymać}	\\	\mbox{łącznie}	\\	\mbox{50 punktów}	\end{array}$}	\\
\hline
\multicolumn{1}{l|}{$\begin{array}{l}\mbox{Wypelnia zdający przed}	\\	\mbox{roz oczęciem racy}	\\	\mbox{PESEL ZDAJACEGO}	\end{array}$}&	\multicolumn{1}{|l}{$\begin{array}{l}\mbox{KOD}	\\	\mbox{ZDAJACEGO}	\end{array}$}
\end{tabular}


\includegraphics[width=21.840mm,height=9.852mm]{./F1_M_PR_M2006_page0_images/image001.eps}

\includegraphics[width=78.792mm,height=13.356mm]{./F1_M_PR_M2006_page0_images/image002.eps}
\end{center}



{\it 2}

{\it Egzamin maturalny z matematyki}

{\it Arkusz II}
\begin{center}
\includegraphics[width=192.228mm,height=288.036mm]{./F1_M_PR_M2006_page1_images/image001.eps}
\end{center}
Zadanie 12. $(5pkt)$

Korzystając z zasady indukcji matematycznej wykaz, $\dot{\mathrm{z}}\mathrm{e}$ dla $\mathrm{k}\mathrm{a}\dot{\mathrm{z}}$ dej liczby naturalnej $n\geq 1$

prawdziwy jest wzór: l$\cdot$ 3$\cdot(1!)^{2}+2\cdot 4\cdot(2!)^{2}+\cdots+n(n+2)(n!)^{2}=[(n+1)!]^{2}-1.$
\begin{center}
\includegraphics[width=137.868mm,height=17.580mm]{./F1_M_PR_M2006_page1_images/image002.eps}
\end{center}
Nr czynnoścÍ

WypelnÍa Maks. liczba kt

egzaminator! Uzyskana liczba pkt

12.1.

1

12.2.

1

12.3.

12.4.

1

12.5.

1





{\it Egzamin maturalny z matematyki}

{\it Arkusz II}

{\it 11}

Zadanie 20. $(4pkt)$

Dane są funkcje $f(x)=3^{x^{2}-5x} \mathrm{i} g(x)=(\displaystyle \frac{1}{9})^{-2x^{2}-3x+2}$

Oblicz, dla których argumentów $x$ wartości funkcji $f$ sąwiększe od wartości funkcji $g.$
\begin{center}
\includegraphics[width=192.276mm,height=260.508mm]{./F1_M_PR_M2006_page10_images/image001.eps}

\includegraphics[width=123.900mm,height=17.628mm]{./F1_M_PR_M2006_page10_images/image002.eps}
\end{center}
Nr czynnoŚci

Wypelnia Maks. liczba kt

egzaminator! Uzyskana liczba pkt

20.1.

1

20.2.

1

20.3.

1

20.4.

1





{\it 12}

{\it Egzamin maturalny z matematyki}

{\it Arkusz II}

Zadanie 21. $(5pkt)$

$\mathrm{W}$ trakcie badania przebiegu zmienności funkcji ustalono, $\dot{\mathrm{z}}\mathrm{e}$ ffinkcja

własności:

- jej dziedzinąjest zbiór wszystkich liczb rzeczywistych,

- $f$ jest funkcją nieparzyst\%

- $f$ jest funkcją ciągłą

oraz:

$f'(x)<0$ dla $x\in(-8,-3),$

$f'(x)>0$ dla $x\in(-3,-1),$

f ma następujące

$f'(x)<0$ dla $x\in(-1,0),$

$f'(-3)=f'(-1)=0,$

$f(-8)=0,$

$f(-3)=-2,$

$f(-2)=0,$

$f(-1)=1.$

$\mathrm{W}$ prostokątnym układzie współrzędnych na płaszczyz$\acute{}$nie naszkicuj wykres funkcji $f$

w przedziale $\langle-8,8\rangle$, wykorzystując podane powyzej informacje ojej własnościach.
\begin{center}
\includegraphics[width=192.228mm,height=48.672mm]{./F1_M_PR_M2006_page11_images/image001.eps}
\end{center}




{\it Egzamin maturalny z matematyki}

{\it Arkusz II}

{\it 13}
\begin{center}
\includegraphics[width=192.276mm,height=290.724mm]{./F1_M_PR_M2006_page12_images/image001.eps}

\includegraphics[width=109.980mm,height=17.580mm]{./F1_M_PR_M2006_page12_images/image002.eps}
\end{center}
Nr czynno\S ci

WypelnÍa Maks. liczba kt

egzaminator! Uzyskana liczba pkt

21.1.

1

21.2.

2

21.3.

2





{\it 14}

{\it Egzamin maturalny z matematyki}

{\it Arkusz II}

BRUDNOPIS





{\it Egzamin maturalny z matematyki}

{\it Arkusz II}

{\it 3}

Zadanie 13. $(5pkt)$

Dany jest ciąg $(a_{n})$, gdzie $a_{n}=\displaystyle \frac{5n+6}{10(n+1)}$ dla $\mathrm{k}\mathrm{a}\dot{\mathrm{z}}$ dej liczby naturalnej $n\geq 1.$

a) Zbadaj monotoniczność ciągu $(a_{n}).$

b) Oblicz $\displaystyle \lim_{n\rightarrow\infty}a_{n}.$

c) Podaj największą liczbę $a$ i najnmiejszą liczbę $b$ takie, $\dot{\mathrm{z}}\mathrm{e}$ dla $\mathrm{k}\mathrm{a}\dot{\mathrm{z}}$ dego $n$ spełniony jest

warunek $a\leq a_{n}\leq b.$
\begin{center}
\includegraphics[width=192.276mm,height=236.268mm]{./F1_M_PR_M2006_page2_images/image001.eps}

\includegraphics[width=137.928mm,height=17.580mm]{./F1_M_PR_M2006_page2_images/image002.eps}
\end{center}
Nr czynności

Wypelnia Maks. liczba kt

egzaminator! Uzyskana liczba pkt

13.1.

1

13.2.

13.3.

13.4.

1

13.5.

1





{\it 4}

{\it Egzamin maturalny z matematyki}

{\it Arkusz II}

Zadanie 14. $(4pkt)$

a) Naszkicuj wykres funkcji $y=\sin 2x$ w przedziale $<-2\pi,2\pi>.$
\begin{center}
\includegraphics[width=192.228mm,height=121.308mm]{./F1_M_PR_M2006_page3_images/image001.eps}
\end{center}
b)

Naszkicuj wykres funkcji $y=\displaystyle \frac{|\sin 2x|}{\sin 2x}$ w przedziale $<-2\pi,2\pi>$

i zapisz, dla których liczb z tego przedziału spełnionajest nierówność $\displaystyle \frac{|\sin 2x|}{\sin 2x}<0.$
\begin{center}
\includegraphics[width=192.228mm,height=121.308mm]{./F1_M_PR_M2006_page3_images/image002.eps}
\end{center}




{\it Egzamin maturalny z matematyki}

{\it Arkusz II}

{\it 5}
\begin{center}
\includegraphics[width=192.276mm,height=290.724mm]{./F1_M_PR_M2006_page4_images/image001.eps}

\includegraphics[width=123.900mm,height=17.580mm]{./F1_M_PR_M2006_page4_images/image002.eps}
\end{center}
Nr czynnoścÍ

Wypelnia Maks. liczba kt

egzaminator! Uzyskana liczba pkt

14.1.

1

14.2.

1

14.3.

1

14.4.

1





{\it 6}

{\it Egzamin maturalny z matematyki}

{\it Arkusz II}

Zadanie 15. $(4pkt)$

Uczniowie dojez $\mathrm{d}\dot{\mathrm{z}}$ ający do szkoły zaobserwowali, $\dot{\mathrm{z}}\mathrm{e}$ spózínienie autobusu zalez$\mathrm{y}$ od tego,

który z trzech kierowców prowadzi autobus. Przeprowadzili badania statystyczne i obliczyli,

$\dot{\mathrm{z}}\mathrm{e}$ w przypadku, gdy autobus prowadzi kierowca $\mathrm{A}$, spózínienie zdarza się w 5\% jego kursów,

gdy prowadzi kierowca $\mathrm{B}$ w 20\% jego kursów, a gdy prowadzi kierowca $\mathrm{C}$ w 50\% jego

kursów. $\mathrm{W}$ ciągu 5-dniowego tygodnia nauki dwa razy prowadzi autobus kierowca $\mathrm{A}$, dwa

razy kierowca $\mathrm{B}$ i jeden raz kierowca C. Oblicz prawdopodobieństwo spózínienia się

szkolnego autobusu w losowo wybrany dzień nauki.
\begin{center}
\includegraphics[width=192.228mm,height=242.364mm]{./F1_M_PR_M2006_page5_images/image001.eps}

\includegraphics[width=123.948mm,height=17.580mm]{./F1_M_PR_M2006_page5_images/image002.eps}
\end{center}
Nr czynno\S ci

Wypelnia Maks. liczba kt

egzamÍnator! Uzyskana liczba pkt

15.1.

1

15.2.

1

15.3.

1

15.4.

1





{\it Egzamin maturalny z matematyki}

{\it Arkusz II}

7

Zadanie 16. $(3pkt)$

Obiekty $A\mathrm{i}B$ lez$\cdot$ą po dwóch stronach jeziora. $\mathrm{W}$ terenie dokonano pomiarów odpowiednich

kątów i ich wyniki przedstawiono na rysunku. Odległość między obiektami $B\mathrm{i}C$ jest równa

400 $\mathrm{m}$. Oblicz odległość w linii prostej między obiektami $A\mathrm{i}B$ i podaj wynik, zaokrąglając

go do jednego metra.
\begin{center}
\includegraphics[width=67.968mm,height=35.808mm]{./F1_M_PR_M2006_page6_images/image001.eps}
\end{center}
{\it A}
\begin{center}
\includegraphics[width=192.276mm,height=212.088mm]{./F1_M_PR_M2006_page6_images/image002.eps}

\includegraphics[width=109.980mm,height=17.628mm]{./F1_M_PR_M2006_page6_images/image003.eps}
\end{center}
Nr czynności

egzaminator! Uzyskana liczba pkt

1

1

16.3.





{\it 8}

{\it Egzamin maturalny z matematyki}

{\it Arkusz II}

Zadanie 17. $(6pkt)$

Na okręgu o promieniu $r$ opisano trapez równoramienny ABCD o dłuzszej podstawie $AB$

i krótszej $CD$. Punkt styczności $S$ dzieli ramię $BC$ tak, $\displaystyle \dot{\mathrm{z}}\mathrm{e}\frac{|CS|}{|SB|}=\frac{2}{5}.$

a) Wyznacz długość ramienia tego trapezu.

b) Oblicz cosinus $|\wedge CBD|.$
\begin{center}
\includegraphics[width=192.228mm,height=248.412mm]{./F1_M_PR_M2006_page7_images/image001.eps}

\includegraphics[width=151.896mm,height=17.580mm]{./F1_M_PR_M2006_page7_images/image002.eps}
\end{center}
Nr czynno\S ci

Wypelnia Maks. liczba kt

egzamÍnator! Uzyskana liczba pkt

17.1.

1

17.2.

1

17.3.

1

17.4.

1

17.5.

1

1





{\it Egzamin maturalny z matematyki}

{\it Arkusz II}

{\it 9}

Zadanie 18. $(7pkt)$

Wśród wszystkich graniastosłupów prawidłowych trójkątnych o objętości równej 2 $\mathrm{m}^{3}$

istnieje taki, którego pole powierzchni całkowitej jest najmniejsze. Wyznacz długości

krawędzi tego graniastosłupa.
\begin{center}
\includegraphics[width=192.276mm,height=260.508mm]{./F1_M_PR_M2006_page8_images/image001.eps}

\includegraphics[width=165.864mm,height=17.628mm]{./F1_M_PR_M2006_page8_images/image002.eps}
\end{center}
Wypelnia

egzaminator!

Nr czynności

Maks. liczba kt

18.1.

1

18.2.

1

18.3.

18.4.

18.5.

18.6.

18.7.

1

Uzyskana liczba pkt





$ 1\theta$

{\it Egzamin maturalny z matematyki}

{\it Arkusz II}

Zadanie 19. (7pkt)

Nieskończony ciąg

geometryczny

$(a_{n})$

jest

zdefiniowany

wzorem

rekurencyjnym: $a_{1}=2, a_{n+1}=a_{n}\cdot\log_{2}(k-2)$, dla $\mathrm{k}\mathrm{a}\dot{\mathrm{z}}$ dej liczby naturalnej $n\geq 1$. Wszystkie

wyrazy tego ciągu są rózne od zera. Wyznacz wszystkie wartości parametru $k$, dla których

istnieje suma wszystkich wyrazów nieskończonego ciągu $(a_{n}).$
\begin{center}
\includegraphics[width=192.228mm,height=254.460mm]{./F1_M_PR_M2006_page9_images/image001.eps}

\includegraphics[width=151.896mm,height=17.580mm]{./F1_M_PR_M2006_page9_images/image002.eps}
\end{center}
Nr czynności

Wypelnia Maks. liczba kt

egzamÍnator! Uzyskana liczba pkt

1

1

1

1

2

1



\end{document}