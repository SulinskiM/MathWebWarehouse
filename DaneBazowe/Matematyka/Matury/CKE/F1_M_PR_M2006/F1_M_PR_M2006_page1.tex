\documentclass[a4paper,12pt]{article}
\usepackage{latexsym}
\usepackage{amsmath}
\usepackage{amssymb}
\usepackage{graphicx}
\usepackage{wrapfig}
\pagestyle{plain}
\usepackage{fancybox}
\usepackage{bm}

\begin{document}

{\it 2}

{\it Egzamin maturalny z matematyki}

{\it Arkusz II}
\begin{center}
\includegraphics[width=192.228mm,height=288.036mm]{./F1_M_PR_M2006_page1_images/image001.eps}
\end{center}
Zadanie 12. $(5pkt)$

Korzystając z zasady indukcji matematycznej wykaz, $\dot{\mathrm{z}}\mathrm{e}$ dla $\mathrm{k}\mathrm{a}\dot{\mathrm{z}}$ dej liczby naturalnej $n\geq 1$

prawdziwy jest wzór: l$\cdot$ 3$\cdot(1!)^{2}+2\cdot 4\cdot(2!)^{2}+\cdots+n(n+2)(n!)^{2}=[(n+1)!]^{2}-1.$
\begin{center}
\includegraphics[width=137.868mm,height=17.580mm]{./F1_M_PR_M2006_page1_images/image002.eps}
\end{center}
Nr czynnoścÍ

WypelnÍa Maks. liczba kt

egzaminator! Uzyskana liczba pkt

12.1.

1

12.2.

1

12.3.

12.4.

1

12.5.

1
\end{document}
