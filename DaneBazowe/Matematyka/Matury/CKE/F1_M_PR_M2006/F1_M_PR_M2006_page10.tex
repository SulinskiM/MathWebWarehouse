\documentclass[a4paper,12pt]{article}
\usepackage{latexsym}
\usepackage{amsmath}
\usepackage{amssymb}
\usepackage{graphicx}
\usepackage{wrapfig}
\pagestyle{plain}
\usepackage{fancybox}
\usepackage{bm}

\begin{document}

{\it Egzamin maturalny z matematyki}

{\it Arkusz II}

{\it 11}

Zadanie 20. $(4pkt)$

Dane są funkcje $f(x)=3^{x^{2}-5x} \mathrm{i} g(x)=(\displaystyle \frac{1}{9})^{-2x^{2}-3x+2}$

Oblicz, dla których argumentów $x$ wartości funkcji $f$ sąwiększe od wartości funkcji $g.$
\begin{center}
\includegraphics[width=192.276mm,height=260.508mm]{./F1_M_PR_M2006_page10_images/image001.eps}

\includegraphics[width=123.900mm,height=17.628mm]{./F1_M_PR_M2006_page10_images/image002.eps}
\end{center}
Nr czynnoŚci

Wypelnia Maks. liczba kt

egzaminator! Uzyskana liczba pkt

20.1.

1

20.2.

1

20.3.

1

20.4.

1
\end{document}
