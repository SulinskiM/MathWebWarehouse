\documentclass[a4paper,12pt]{article}
\usepackage{latexsym}
\usepackage{amsmath}
\usepackage{amssymb}
\usepackage{graphicx}
\usepackage{wrapfig}
\pagestyle{plain}
\usepackage{fancybox}
\usepackage{bm}

\begin{document}

{\it 12}

{\it Egzamin maturalny z matematyki}

{\it Arkusz II}

Zadanie 21. $(5pkt)$

$\mathrm{W}$ trakcie badania przebiegu zmienności funkcji ustalono, $\dot{\mathrm{z}}\mathrm{e}$ ffinkcja

własności:

- jej dziedzinąjest zbiór wszystkich liczb rzeczywistych,

- $f$ jest funkcją nieparzyst\%

- $f$ jest funkcją ciągłą

oraz:

$f'(x)<0$ dla $x\in(-8,-3),$

$f'(x)>0$ dla $x\in(-3,-1),$

f ma następujące

$f'(x)<0$ dla $x\in(-1,0),$

$f'(-3)=f'(-1)=0,$

$f(-8)=0,$

$f(-3)=-2,$

$f(-2)=0,$

$f(-1)=1.$

$\mathrm{W}$ prostokątnym układzie współrzędnych na płaszczyz$\acute{}$nie naszkicuj wykres funkcji $f$

w przedziale $\langle-8,8\rangle$, wykorzystując podane powyzej informacje ojej własnościach.
\begin{center}
\includegraphics[width=192.228mm,height=48.672mm]{./F1_M_PR_M2006_page11_images/image001.eps}
\end{center}\end{document}
