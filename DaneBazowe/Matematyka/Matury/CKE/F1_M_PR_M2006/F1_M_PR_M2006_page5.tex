\documentclass[a4paper,12pt]{article}
\usepackage{latexsym}
\usepackage{amsmath}
\usepackage{amssymb}
\usepackage{graphicx}
\usepackage{wrapfig}
\pagestyle{plain}
\usepackage{fancybox}
\usepackage{bm}

\begin{document}

{\it 6}

{\it Egzamin maturalny z matematyki}

{\it Arkusz II}

Zadanie 15. $(4pkt)$

Uczniowie dojez $\mathrm{d}\dot{\mathrm{z}}$ ający do szkoły zaobserwowali, $\dot{\mathrm{z}}\mathrm{e}$ spózínienie autobusu zalez$\mathrm{y}$ od tego,

który z trzech kierowców prowadzi autobus. Przeprowadzili badania statystyczne i obliczyli,

$\dot{\mathrm{z}}\mathrm{e}$ w przypadku, gdy autobus prowadzi kierowca $\mathrm{A}$, spózínienie zdarza się w 5\% jego kursów,

gdy prowadzi kierowca $\mathrm{B}$ w 20\% jego kursów, a gdy prowadzi kierowca $\mathrm{C}$ w 50\% jego

kursów. $\mathrm{W}$ ciągu 5-dniowego tygodnia nauki dwa razy prowadzi autobus kierowca $\mathrm{A}$, dwa

razy kierowca $\mathrm{B}$ i jeden raz kierowca C. Oblicz prawdopodobieństwo spózínienia się

szkolnego autobusu w losowo wybrany dzień nauki.
\begin{center}
\includegraphics[width=192.228mm,height=242.364mm]{./F1_M_PR_M2006_page5_images/image001.eps}

\includegraphics[width=123.948mm,height=17.580mm]{./F1_M_PR_M2006_page5_images/image002.eps}
\end{center}
Nr czynno\S ci

Wypelnia Maks. liczba kt

egzamÍnator! Uzyskana liczba pkt

15.1.

1

15.2.

1

15.3.

1

15.4.

1
\end{document}
