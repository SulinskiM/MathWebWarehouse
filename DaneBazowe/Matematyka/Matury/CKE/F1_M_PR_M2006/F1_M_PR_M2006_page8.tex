\documentclass[a4paper,12pt]{article}
\usepackage{latexsym}
\usepackage{amsmath}
\usepackage{amssymb}
\usepackage{graphicx}
\usepackage{wrapfig}
\pagestyle{plain}
\usepackage{fancybox}
\usepackage{bm}

\begin{document}

{\it Egzamin maturalny z matematyki}

{\it Arkusz II}

{\it 9}

Zadanie 18. $(7pkt)$

Wśród wszystkich graniastosłupów prawidłowych trójkątnych o objętości równej 2 $\mathrm{m}^{3}$

istnieje taki, którego pole powierzchni całkowitej jest najmniejsze. Wyznacz długości

krawędzi tego graniastosłupa.
\begin{center}
\includegraphics[width=192.276mm,height=260.508mm]{./F1_M_PR_M2006_page8_images/image001.eps}

\includegraphics[width=165.864mm,height=17.628mm]{./F1_M_PR_M2006_page8_images/image002.eps}
\end{center}
Wypelnia

egzaminator!

Nr czynności

Maks. liczba kt

18.1.

1

18.2.

1

18.3.

18.4.

18.5.

18.6.

18.7.

1

Uzyskana liczba pkt
\end{document}
