\documentclass[a4paper,12pt]{article}
\usepackage{latexsym}
\usepackage{amsmath}
\usepackage{amssymb}
\usepackage{graphicx}
\usepackage{wrapfig}
\pagestyle{plain}
\usepackage{fancybox}
\usepackage{bm}

\begin{document}

$ 1\theta$

{\it Egzamin maturalny z matematyki}

{\it Arkusz II}

Zadanie 19. (7pkt)

Nieskończony ciąg

geometryczny

$(a_{n})$

jest

zdefiniowany

wzorem

rekurencyjnym: $a_{1}=2, a_{n+1}=a_{n}\cdot\log_{2}(k-2)$, dla $\mathrm{k}\mathrm{a}\dot{\mathrm{z}}$ dej liczby naturalnej $n\geq 1$. Wszystkie

wyrazy tego ciągu są rózne od zera. Wyznacz wszystkie wartości parametru $k$, dla których

istnieje suma wszystkich wyrazów nieskończonego ciągu $(a_{n}).$
\begin{center}
\includegraphics[width=192.228mm,height=254.460mm]{./F1_M_PR_M2006_page9_images/image001.eps}

\includegraphics[width=151.896mm,height=17.580mm]{./F1_M_PR_M2006_page9_images/image002.eps}
\end{center}
Nr czynności

Wypelnia Maks. liczba kt

egzamÍnator! Uzyskana liczba pkt

1

1

1

1

2

1
\end{document}
