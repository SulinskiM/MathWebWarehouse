\documentclass[a4paper,12pt]{article}
\usepackage{latexsym}
\usepackage{amsmath}
\usepackage{amssymb}
\usepackage{graphicx}
\usepackage{wrapfig}
\pagestyle{plain}
\usepackage{fancybox}
\usepackage{bm}

\begin{document}
\begin{center}
\begin{tabular}{l|l}
\multicolumn{1}{l|}{$\begin{array}{l}\mbox{{\it dysleksja}}	\\	\mbox{Miejsce}	\\	\mbox{na naklejkę}	\\	\mbox{z kodem szkoly}	\end{array}$}&	\multicolumn{1}{|l}{ $\mathrm{M}\mathrm{M}\mathrm{A}-\mathrm{R}1_{-}1\mathrm{P}-072$}	\\
\hline
\multicolumn{1}{l|}{$\begin{array}{l}\mbox{EGZAMIN MATURALNY}	\\	\mbox{Z MATEMATYKI}	\\	\mbox{POZIOM ROZSZERZONY}	\\	\mbox{Czas pracy 180 minut}	\\	\mbox{Instrukcja dla zdającego}	\\	\mbox{1. Sprawdzí, czy arkusz egzaminacyjny zawiera 15 stron}	\\	\mbox{(zadania $1-11$). Ewentualny brak zgłoś przewodniczącemu}	\\	\mbox{zespo nadzo jącego egzamin.}	\\	\mbox{2. Rozwiązania zadań i odpowiedzi zamieść w miejscu na to}	\\	\mbox{przeznaczonym.}	\\	\mbox{3. $\mathrm{W}$ rozwiązaniach zadań przedstaw tok rozumowania}	\\	\mbox{prowadzący do ostatecznego wyniku.}	\\	\mbox{4. Pisz czytelnie. Uzywaj $\mathrm{d}$ gopisu pióra tylko z czatnym}	\\	\mbox{tusze atramentem.}	\\	\mbox{5. Nie uzywaj korektora, a błędne zapisy prze eśl.}	\\	\mbox{6. Pamiętaj, $\dot{\mathrm{z}}\mathrm{e}$ zapisy w brudnopisie nie podlegają ocenie.}	\\	\mbox{7. Obok $\mathrm{k}\mathrm{a}\dot{\mathrm{z}}$ dego zadania podanajest maksymalna liczba punktów,}	\\	\mbox{którą mozesz uzyskać zajego poprawne rozwiązanie.}	\\	\mbox{8. $\mathrm{M}\mathrm{o}\dot{\mathrm{z}}$ esz korzystać z zestawu wzorów matematycznych, cyrkla}	\\	\mbox{i linijki oraz kalkulatora.}	\\	\mbox{9. Wypełnij tę część ka $\mathrm{y}$ odpowiedzi, którą koduje zdający.}	\\	\mbox{Nie wpisuj $\dot{\mathrm{z}}$ adnych znaków w części przeznaczonej dla}	\\	\mbox{egzaminatora.}	\\	\mbox{10. Na karcie odpowiedzi wpisz swoją datę urodzenia i PESEL.}	\\	\mbox{Zamaluj $\blacksquare$ pola odpowiadające cyfrom numeru PESEL. Błędne}	\\	\mbox{zaznaczenie otocz kółkiem $\mathrm{O}$ i zaznacz właściwe.}	\\	\mbox{{\it Zyczymy powodzenia}.'}	\end{array}$}&	\multicolumn{1}{|l}{$\begin{array}{l}\mbox{MAJ}	\\	\mbox{ROK 2007}	\\	\mbox{Za rozwiązanie}	\\	\mbox{wszystkich zadań}	\\	\mbox{mozna otrzymać}	\\	\mbox{łącznie}	\\	\mbox{50 punktów}	\end{array}$}	\\
\hline
\multicolumn{1}{l|}{$\begin{array}{l}\mbox{Wypelnia zdający}	\\	\mbox{rzed roz oczęciem racy}	\\	\mbox{PESEL ZDAJACEGO}	\end{array}$}&	\multicolumn{1}{|l}{$\begin{array}{l}\mbox{KOD}	\\	\mbox{ZDAJACEGO}	\end{array}$}
\end{tabular}


\includegraphics[width=21.840mm,height=9.852mm]{./F1_M_PR_M2007_page0_images/image001.eps}

\includegraphics[width=78.792mm,height=13.356mm]{./F1_M_PR_M2007_page0_images/image002.eps}
\end{center}



{\it 2}

{\it Egzamin maturalny z matematyki}

{\it Poziom rozszerzony}

Zadanie 1. (5pkt)

Danajest funkcja $f(x)=|x-1|-|x+2|$ dla $x\in R.$

a) Wyznacz zbiór wartości funkcji $f$ dla $x\in(-\infty,-2).$

b) Naszkicuj wykres tej funkcji.

c) Podaj jej miejsca zerowe.

d) Wyznacz wszystkie wartości parametru $m$, dla których równanie $f(x)=m$ nie ma

rozwiązania.
\begin{center}
\includegraphics[width=137.868mm,height=17.580mm]{./F1_M_PR_M2007_page1_images/image001.eps}
\end{center}
Nr czynno\S ci

Wypelnia Maks. liczba kt

egzaminator! Uzyskana liczba pkt

1.1.

1

1.2.

1.3.

1

1.4.

1

1.5.

1





{\it Egzamin maturalny z matematyki}

{\it Poziom rozszerzony}

{\it 11}
\begin{center}
\includegraphics[width=109.980mm,height=17.580mm]{./F1_M_PR_M2007_page10_images/image001.eps}
\end{center}
Nr czynności

Wypelnia Maks. liczba kt

egzaminator! Uzyskana liczba pkt

8.1.

1

8.2.

1

8.3.

1





{\it 12}

{\it Egzamin maturalny z matematyki}

{\it Poziom rozszerzony}

Zadanie 9. (3pkt)

Przedstaw wielomian $W(x)=x^{4}-2x^{3}-3x^{2}+4x-1$ w postaci iloczynu dwóch wielomianów

stopnia drugiego o współczynnikach całkowitych i takich, $\dot{\mathrm{z}}\mathrm{e}$ współczynniki przy drugich

potęgach są równe jeden.
\begin{center}
\includegraphics[width=109.932mm,height=17.628mm]{./F1_M_PR_M2007_page11_images/image001.eps}
\end{center}
Wypelnia

egzaminator!

Nr czynności

Maks. liczba kt

1

1

1

Uzyskana liczba pkt





{\it Egzamin maturalny z matematyki}

{\it Poziom rozszerzony}

{\it 13}

Zadanie 10. $(4pkt)$

Na kole opisany jest romb. Stosunek pola koła do pola rombu wynosi $\displaystyle \frac{\pi\sqrt{3}}{8}$. Wyznacz miarę

kąta ostrego rombu.
\begin{center}
\includegraphics[width=123.900mm,height=17.628mm]{./F1_M_PR_M2007_page12_images/image001.eps}
\end{center}
Nr czynności

Wypelnia Maks. liczba kt

egzaminator! Uzyskana liczba pkt

10.2.

1

10.3.

1

10.4.

1





{\it 14}

{\it Egzamin maturalny z matematyki}

{\it Poziom rozszerzony}

Zadanie ll. $(4pkt)$

Suma $n$ początkowych wyrazów ciągu arytmetycznego $(a_{n})$

$S_{n}=2n^{2}+n$ dla $n\geq 1.$

a) Oblicz sumę 50 początkowych wyrazów tego ciągu o

$a_{2}+a_{4}+a_{6}+\ldots+a_{100}.$

b) Oblicz $\displaystyle \lim_{n\rightarrow\infty}\frac{S_{n}}{3n^{2}-2}.$

wyraza się wzorem

numerach parzystych:
\begin{center}
\includegraphics[width=123.900mm,height=17.628mm]{./F1_M_PR_M2007_page13_images/image001.eps}
\end{center}
Wypelnia

egzaminator!

Nr czynności

Maks. liczba kt

11.1.

1

11.2.

1

1

11.4.

1

Uzyskana liczba pkt





{\it Egzamin maturalny z matematyki}

{\it Poziom rozszerzony}

{\it 15}

BRUDNOPIS





{\it Egzamin maturalny z matematyki}

{\it Poziom rozszerzony}

{\it 3}

Zadanie 2. $(5pkt)$

Rozwiąz nierówność: $\log_{\frac{1}{3}}(x^{2}-1)+\log_{\frac{1}{3}}(5-x)>\log_{\frac{1}{3}}(3(x+1)).$
\begin{center}
\includegraphics[width=137.928mm,height=17.580mm]{./F1_M_PR_M2007_page2_images/image001.eps}
\end{center}
Wypelnia

egzaminator!

Nr czynności

Maks. liczba kt

2.1.

1

2.2.

2.3.

1

2.4.

1

2.5.

1

Uzyskana liczba pkt





{\it 4}

{\it Egzamin maturalny z matematyki}

{\it Poziom rozszerzony}

Zadanie 3. $(5pkt)$

Kapsuła lądownika ma kształt stozka zakończonego w podstawie półkulą o tym samym

promieniu co promień podstawy stozka. Wysokość stozka jest o l $\mathrm{m}$ większa $\mathrm{n}\mathrm{i}\dot{\mathrm{z}}$ promień

półkuli. Objętość stozka stanowi $\displaystyle \frac{2}{3}$ objętości całej kapsuły. Oblicz objętość kapsuły

lądownika.
\begin{center}
\includegraphics[width=137.868mm,height=17.580mm]{./F1_M_PR_M2007_page3_images/image001.eps}
\end{center}
Nr czynno\S ci

Wypelnia Maks. liczba kt

egzaminator! Uzyskana liczba pkt

3.1.

1

3.2.

3.3.

1

3.4.

1

3.5.

1





{\it Egzamin maturalny z matematyki}

{\it Poziom rozszerzony}

{\it 5}

Zadanie 4. $(3pkt)$

Dany jest trójkąt o bokach długości l, $\displaystyle \frac{3}{2}$, 2. Oblicz cosinus i sinus kąta lez$\cdot$ącego naprzeciw

najkrótszego boku tego trójkąta.
\begin{center}
\includegraphics[width=109.980mm,height=17.580mm]{./F1_M_PR_M2007_page4_images/image001.eps}
\end{center}
Nr czynności

Wypelnia Maks. liczba kt

egzaminator! Uzyskana liczba pkt

4.1.

1

4.2.

1

4.3.

1





{\it 6}

{\it Egzamin maturalny z matematyki}

{\it Poziom rozszerzony}

Zadanie 5. $(7pkt)$

Wierzchołki trójkąta równobocznego $ABC$ są punktami paraboli $y=-x^{2}+6x$. Punkt $C$ jest

jej wierzchołkiem, a bok $AB$ jest równoległy do osi $\mathrm{O}x$. Sporządzí rysunek w układzie

współrzędnych i wyznacz współrzędne wierzchołków tego trójkąta.
\begin{center}
\includegraphics[width=165.816mm,height=17.628mm]{./F1_M_PR_M2007_page5_images/image001.eps}
\end{center}
Nr czynności

Wypelnia Maks. liczba kt

egzaminator! Uzyskana liczba pkt

5.1.

1

5.2.

1

5.3.

1

5.4.

1

5.5.

1

5.7.

1





{\it Egzamin maturalny z matematyki}

{\it Poziom rozszerzony}

7

Zadanie 6. (4pkt)

Niech $A, B$ będą zdarzeniami o prawdopodobieństwach $P(A) \mathrm{i} P(B)$. Wykaz, $\dot{\mathrm{z}}\mathrm{e}\mathrm{j}\mathrm{e}\dot{\mathrm{z}}$ eli

$P(A)=0,85 \mathrm{i} P(B)=0,75$, to prawdopodobieństwo warunkowe spełnia nierówność

$P(A|B)\geq 0,8.$
\begin{center}
\includegraphics[width=123.900mm,height=17.580mm]{./F1_M_PR_M2007_page6_images/image001.eps}
\end{center}
Nr czynności

Wypelnia Maks. liczba kt

egzamÍnator! Uzyskana liczba pkt

1

1

1





{\it 8}

{\it Egzamin maturalny z matematyki}

{\it Poziom rozszerzony}

Zadanie 7. $(7pkt)$

Dany jest układ równań: 

Dla $\mathrm{k}\mathrm{a}\dot{\mathrm{z}}$ dej wartości parametru $m$ wyznacz parę liczb $(x,y)$, która jest rozwiązaniem tego

układu równań. Wyznacz najmniejszą wartość sumy $x+y$ dla $m\in\langle 2,4\rangle.$





{\it Egzamin maturalny z matematyki}

{\it Poziom rozszerzony}

{\it 9}
\begin{center}
\includegraphics[width=165.864mm,height=17.580mm]{./F1_M_PR_M2007_page8_images/image001.eps}
\end{center}
Nr czynności

Wypelnia Maks. lÍczba kt

egzaminator! Uzyskana liczba pkt

7.1.

1

7.2.

1

7.3.

1

7.4.

7.5.

1

7.7.

1





$ 1\theta$

{\it Egzamin maturalny z matematyki}

{\it Poziom rozszerzony}

Zadanie 8. $(3pkt)$

Danajest funkcja $f$ określona wzorem $f(x)=\displaystyle \frac{\sin^{2}x-|\sin x|}{\sin x}$ dla $x\in(0,\pi)\cup(\pi,2\pi).$

a) Naszkicuj wykres funkcji $f.$

b) Wyznacz miejsca zerowe funkcji $f$



\end{document}