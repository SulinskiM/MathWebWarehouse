\documentclass[a4paper,12pt]{article}
\usepackage{latexsym}
\usepackage{amsmath}
\usepackage{amssymb}
\usepackage{graphicx}
\usepackage{wrapfig}
\pagestyle{plain}
\usepackage{fancybox}
\usepackage{bm}

\begin{document}

{\it 12}

{\it Egzamin maturalny z matematyki}

{\it Poziom rozszerzony}

Zadanie 9. (3pkt)

Przedstaw wielomian $W(x)=x^{4}-2x^{3}-3x^{2}+4x-1$ w postaci iloczynu dwóch wielomianów

stopnia drugiego o współczynnikach całkowitych i takich, $\dot{\mathrm{z}}\mathrm{e}$ współczynniki przy drugich

potęgach są równe jeden.
\begin{center}
\includegraphics[width=109.932mm,height=17.628mm]{./F1_M_PR_M2007_page11_images/image001.eps}
\end{center}
Wypelnia

egzaminator!

Nr czynności

Maks. liczba kt

1

1

1

Uzyskana liczba pkt
\end{document}
