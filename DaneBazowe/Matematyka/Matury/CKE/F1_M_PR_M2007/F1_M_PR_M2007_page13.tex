\documentclass[a4paper,12pt]{article}
\usepackage{latexsym}
\usepackage{amsmath}
\usepackage{amssymb}
\usepackage{graphicx}
\usepackage{wrapfig}
\pagestyle{plain}
\usepackage{fancybox}
\usepackage{bm}

\begin{document}

{\it 14}

{\it Egzamin maturalny z matematyki}

{\it Poziom rozszerzony}

Zadanie ll. $(4pkt)$

Suma $n$ początkowych wyrazów ciągu arytmetycznego $(a_{n})$

$S_{n}=2n^{2}+n$ dla $n\geq 1.$

a) Oblicz sumę 50 początkowych wyrazów tego ciągu o

$a_{2}+a_{4}+a_{6}+\ldots+a_{100}.$

b) Oblicz $\displaystyle \lim_{n\rightarrow\infty}\frac{S_{n}}{3n^{2}-2}.$

wyraza się wzorem

numerach parzystych:
\begin{center}
\includegraphics[width=123.900mm,height=17.628mm]{./F1_M_PR_M2007_page13_images/image001.eps}
\end{center}
Wypelnia

egzaminator!

Nr czynności

Maks. liczba kt

11.1.

1

11.2.

1

1

11.4.

1

Uzyskana liczba pkt
\end{document}
