\documentclass[a4paper,12pt]{article}
\usepackage{latexsym}
\usepackage{amsmath}
\usepackage{amssymb}
\usepackage{graphicx}
\usepackage{wrapfig}
\pagestyle{plain}
\usepackage{fancybox}
\usepackage{bm}

\begin{document}

{\it 4}

{\it Egzamin maturalny z matematyki}

{\it Poziom rozszerzony}

Zadanie 3. $(5pkt)$

Kapsuła lądownika ma kształt stozka zakończonego w podstawie półkulą o tym samym

promieniu co promień podstawy stozka. Wysokość stozka jest o l $\mathrm{m}$ większa $\mathrm{n}\mathrm{i}\dot{\mathrm{z}}$ promień

półkuli. Objętość stozka stanowi $\displaystyle \frac{2}{3}$ objętości całej kapsuły. Oblicz objętość kapsuły

lądownika.
\begin{center}
\includegraphics[width=137.868mm,height=17.580mm]{./F1_M_PR_M2007_page3_images/image001.eps}
\end{center}
Nr czynno\S ci

Wypelnia Maks. liczba kt

egzaminator! Uzyskana liczba pkt

3.1.

1

3.2.

3.3.

1

3.4.

1

3.5.

1
\end{document}
