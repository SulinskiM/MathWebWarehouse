\documentclass[a4paper,12pt]{article}
\usepackage{latexsym}
\usepackage{amsmath}
\usepackage{amssymb}
\usepackage{graphicx}
\usepackage{wrapfig}
\pagestyle{plain}
\usepackage{fancybox}
\usepackage{bm}

\begin{document}

{\it 6}

{\it Egzamin maturalny z matematyki}

{\it Poziom rozszerzony}

Zadanie 5. $(7pkt)$

Wierzchołki trójkąta równobocznego $ABC$ są punktami paraboli $y=-x^{2}+6x$. Punkt $C$ jest

jej wierzchołkiem, a bok $AB$ jest równoległy do osi $\mathrm{O}x$. Sporządzí rysunek w układzie

współrzędnych i wyznacz współrzędne wierzchołków tego trójkąta.
\begin{center}
\includegraphics[width=165.816mm,height=17.628mm]{./F1_M_PR_M2007_page5_images/image001.eps}
\end{center}
Nr czynności

Wypelnia Maks. liczba kt

egzaminator! Uzyskana liczba pkt

5.1.

1

5.2.

1

5.3.

1

5.4.

1

5.5.

1

5.7.

1
\end{document}
