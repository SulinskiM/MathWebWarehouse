\documentclass[a4paper,12pt]{article}
\usepackage{latexsym}
\usepackage{amsmath}
\usepackage{amssymb}
\usepackage{graphicx}
\usepackage{wrapfig}
\pagestyle{plain}
\usepackage{fancybox}
\usepackage{bm}

\begin{document}

{\it Egzamin maturalny z matematyki}

{\it Poziom rozszerzony}

7

Zadanie 6. (4pkt)

Niech $A, B$ będą zdarzeniami o prawdopodobieństwach $P(A) \mathrm{i} P(B)$. Wykaz, $\dot{\mathrm{z}}\mathrm{e}\mathrm{j}\mathrm{e}\dot{\mathrm{z}}$ eli

$P(A)=0,85 \mathrm{i} P(B)=0,75$, to prawdopodobieństwo warunkowe spełnia nierówność

$P(A|B)\geq 0,8.$
\begin{center}
\includegraphics[width=123.900mm,height=17.580mm]{./F1_M_PR_M2007_page6_images/image001.eps}
\end{center}
Nr czynności

Wypelnia Maks. liczba kt

egzamÍnator! Uzyskana liczba pkt

1

1

1
\end{document}
