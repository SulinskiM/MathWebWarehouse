\documentclass[a4paper,12pt]{article}
\usepackage{latexsym}
\usepackage{amsmath}
\usepackage{amssymb}
\usepackage{graphicx}
\usepackage{wrapfig}
\pagestyle{plain}
\usepackage{fancybox}
\usepackage{bm}

\begin{document}

{\it ARKUSZ ZA WIERA INFORMACJE PRA WNIE CHRONIONE}

{\it DO MOMENTU ROZPOCZĘCIA EGZAMINU}.'
\begin{center}
\begin{tabular}{|l|l|l}
\cline{1-1}
\multicolumn{1}{|l|}{$\begin{array}{l}\mbox{Miejsce}	\\	\mbox{na na ejkę}	\end{array}$}&	\multicolumn{1}{|l|}{}&	\multicolumn{1}{|l}{ $\mathrm{M}\mathrm{M}\mathrm{A}-\mathrm{R}1_{-}1\mathrm{P}-082$}	\\
\hline
&	\multicolumn{1}{|l}{$\begin{array}{l}\mbox{MAJ}	\\	\mbox{ROK 2008}	\\	\mbox{Za rozwiązanie}	\\	\mbox{wszystkich zadań}	\\	\mbox{mozna otrzymać}	\\	\mbox{łącznie}	\\	\mbox{50 punktów}	\end{array}$}	\\
\cline{3-3}
&	\multicolumn{1}{|l}{$\begin{array}{l}\mbox{KOD}	\\	\mbox{ZDAJACEGO}	\end{array}$}
\end{tabular}


\includegraphics[width=21.840mm,height=9.852mm]{./F1_M_PR_M2008_page0_images/image001.eps}

\includegraphics[width=78.792mm,height=13.356mm]{./F1_M_PR_M2008_page0_images/image002.eps}
\end{center}



{\it 2 Egzamin maturalny z matematyki}

{\it Poziom rozszerzony}

Zadanie l. $(4pkt)$

Wielomian $f$, którego fragment wykresu przedstawiono na ponizszym rysunku spełnia

warunek $f(0)=90$. Wielomian $g$ dany jest wzorem $g(x)=x^{3}-14x^{2}+63x-90$. Wykaz$\cdot,$

$\dot{\mathrm{z}}\mathrm{e}g(x)=-f(-x)$ dla $x\in R.$
\begin{center}
\includegraphics[width=160.428mm,height=128.724mm]{./F1_M_PR_M2008_page1_images/image001.eps}
\end{center}
{\it y}

{\it f}

1

$-5  -3$  0 1  {\it x}





{\it Egzamin maturalny z matematyki ll}

{\it Poziom rozszerzony}
\begin{center}
\includegraphics[width=195.168mm,height=284.688mm]{./F1_M_PR_M2008_page10_images/image001.eps}

\includegraphics[width=123.900mm,height=17.832mm]{./F1_M_PR_M2008_page10_images/image002.eps}
\end{center}
Nr zadania

Wypelnia Maks. liczba kt

egzaminator! Uzyskana liczba pkt

8.1

1

8.2

1

8.3

1

8.4

1





{\it 12 Egzamin maturalny z matematyki}

{\it Poziom rozszerzony}

Zadanie 9. $(4pkt)$

Wyznacz dziedzinę i najmniejszą wartość funkcji $f(x)=\log_{\frac{\sqrt{2}}{2}}(8x-x^{2}).$
\begin{center}
\includegraphics[width=195.228mm,height=260.508mm]{./F1_M_PR_M2008_page11_images/image001.eps}

\includegraphics[width=123.948mm,height=17.832mm]{./F1_M_PR_M2008_page11_images/image002.eps}
\end{center}
Wypelnia

egzamÍnator!

Nr zadania

Maks. liczba kt

1

1

1

1

Uzyskana liczba pkt





{\it Egzamin maturalny z matematyki 13}

{\it Poziom rozszerzony}

Zadanie 10. $(4pkt)$

$\mathrm{Z}$ pewnej grupy osób, w której jest dwa razy więcej męzczyzn $\mathrm{n}\mathrm{i}\dot{\mathrm{z}}$ kobiet, wybrano losowo

dwuosobową delegację. Prawdopodobieństwo tego, $\dot{\mathrm{z}}\mathrm{e}$ w delegacji znajdą się tylko kobiety

jest równe 0,1. Ob1icz, i1e kobiet i i1u męzczyzn jest w tej grupie.
\begin{center}
\includegraphics[width=195.168mm,height=254.412mm]{./F1_M_PR_M2008_page12_images/image001.eps}

\includegraphics[width=123.900mm,height=17.784mm]{./F1_M_PR_M2008_page12_images/image002.eps}
\end{center}
Nr zadania

Wypelnia Maks. liczba kt

egzaminator! Uzyskana liczba pkt

10.1

1

10.2

1

10.3

1

10.4

1





{\it 14 Egzamin maturalny z matematyki}

{\it Poziom rozszerzony}

Zadanie ll. $(5pkt)$

$\mathrm{W}$ ostrosłupie prawidłowym czworokątnym dane są: $H$ -wysokość ostrosłupa oraz

$\alpha-$ miara kąta utworzonego przez krawędz$\acute{}$ boczną i krawędz$\acute{}$ podstawy $(45^{\circ}<\alpha<90^{\circ}).$

a) Wykaz, $\dot{\mathrm{z}}\mathrm{e}$ objętość $V$ tego ostrosłupajest równa $\displaystyle \frac{4}{3}.\frac{H^{3}}{\mathrm{t}\mathrm{g}^{2}\alpha-1}.$

b) Oblicz miarę kąta $\alpha$, dla której objętość $V$ danego ostrosłupajest równa $\displaystyle \frac{2}{9}H^{3}$ Wynik

podaj w zaokrągleniu do całkowitej liczby stopni.
\begin{center}
\includegraphics[width=195.228mm,height=103.128mm]{./F1_M_PR_M2008_page13_images/image001.eps}
\end{center}




{\it Egzamin maturalny z matematyki 15}

{\it Poziom rozszerzony}
\begin{center}
\includegraphics[width=195.168mm,height=284.688mm]{./F1_M_PR_M2008_page14_images/image001.eps}

\includegraphics[width=137.928mm,height=17.832mm]{./F1_M_PR_M2008_page14_images/image002.eps}
\end{center}
Wypelnia

egzaminator!

Nr zadania

Maks. liczba kt

1

11.2

1

11.3

1

11.4

1

11.5

Uzyskana liczba pkt





{\it 16 Egzamin maturalny z matematyki}

{\it Poziom rozszerzony}

Zadanie 12. $(4pkt)$

$\mathrm{W}$ trójkącie prostokątnym $ABC$ przyprostokątne mają długości: $|BC|=9, |CA|=12$. Na boku

$AB$ wybrano punkt $D$ tak, $\dot{\mathrm{z}}\mathrm{e}$ odcinki $BC \mathrm{i}$ CD mają równe długości. Oblicz długość

odcinka $AD.$
\begin{center}
\includegraphics[width=195.228mm,height=266.544mm]{./F1_M_PR_M2008_page15_images/image001.eps}
\end{center}




{\it Egzamin maturalny z matematyki 17}

{\it Poziom rozszerzony}
\begin{center}
\includegraphics[width=195.168mm,height=284.688mm]{./F1_M_PR_M2008_page16_images/image001.eps}

\includegraphics[width=123.900mm,height=17.832mm]{./F1_M_PR_M2008_page16_images/image002.eps}
\end{center}
Nr zadania

Wypelnia Maks. liczba kt

egzaminator! Uzyskana liczba pkt

12.1

1

12.2

1

12.3

1

1





{\it 18 Egzamin maturalny z matematyki}

{\it Poziom rozszerzony}

BRUDNOPIS





{\it Egzamin maturalny z matematyki 3}

{\it Poziom rozszerzony}
\begin{center}
\includegraphics[width=123.900mm,height=17.832mm]{./F1_M_PR_M2008_page2_images/image001.eps}
\end{center}
Nr zadania

Wypelnia Maks. liczba kt

egzaminator! Uzyskana liczba pkt

1.1

1

1.2

1

1.3

1

1.4

1





{\it 4 Egzamin maturalny z matematyki}

{\it Poziom rozszerzony}

Zadanie 2. $(4pkt)$

Rozwiąz nierówność $|x-2|+|3x-6|<|x|.$
\begin{center}
\includegraphics[width=195.228mm,height=266.544mm]{./F1_M_PR_M2008_page3_images/image001.eps}

\includegraphics[width=123.948mm,height=17.784mm]{./F1_M_PR_M2008_page3_images/image002.eps}
\end{center}
Nr zadania

Wypelnia Maks. liczba kt

egzaminator! Uzyskana lÍczba pkt

2.1

1

2.2

1

2.3

1

2.4

1





{\it Egzamin maturalny z matematyki 5}

{\it Poziom rozszerzony}

Zadanie 3. $(5pkt)$

Liczby $x_{1}=5+\sqrt{23}\mathrm{i}x_{2}=5-\sqrt{23}$ sąrozwiązaniami równania $x^{2}-(p^{2}+q^{2})x+(p+q)=0$

z niewiadomą $x$. Oblicz wartości $p \mathrm{i}q.$
\begin{center}
\includegraphics[width=195.168mm,height=260.508mm]{./F1_M_PR_M2008_page4_images/image001.eps}

\includegraphics[width=137.928mm,height=17.832mm]{./F1_M_PR_M2008_page4_images/image002.eps}
\end{center}
Wypelnia

egzaminator!

Nr zadania

Maks. liczba kt

3.1

1

3.2

1

3.3

1

3.4

1

3.5

Uzyskana liczba pkt





{\it 6 Egzamin maturalny z matematyki}

{\it Poziom rozszerzony}

Zadanie 4. $(4pkt)$

Rozwiąz równanie 4 $\cos^{2}x=4\sin x+1$ w przedziale $\langle 0,2\pi\rangle.$
\begin{center}
\includegraphics[width=195.228mm,height=266.544mm]{./F1_M_PR_M2008_page5_images/image001.eps}

\includegraphics[width=123.948mm,height=17.784mm]{./F1_M_PR_M2008_page5_images/image002.eps}
\end{center}
Nr zadania

Wypelnia Maks. liczba kt

egzaminator! Uzyskana lÍczba pkt

4.1

1

4.2

1

4.3

1

4.4

1





{\it Egzamin maturalny z matematyki 7}

{\it Poziom rozszerzony}

Zadanie 5. $(5pkt)$

Dane jest równanie $|\displaystyle \frac{2}{x}+3|=p$ z niewiadomą $x.$

w zalezności od parametru $p.$

Wyznacz liczbę rozwiązań tego równania
\begin{center}
\includegraphics[width=195.168mm,height=254.460mm]{./F1_M_PR_M2008_page6_images/image001.eps}

\includegraphics[width=137.928mm,height=17.784mm]{./F1_M_PR_M2008_page6_images/image002.eps}
\end{center}
Wypelnia

egzaminator!

Nr zadanÍa

Maks. liczba kt

1

5.2

1

5.3

5.4

5.5

1

Uzyskana lÍczba pkt





{\it 8 Egzamin maturalny z matematyki}

{\it Poziom rozszerzony}

Zadanie 6. $(3pkt)$

Udowodnij, $\dot{\mathrm{z}}\mathrm{e} \mathrm{j}\mathrm{e}\dot{\mathrm{z}}$ eli

to $a=b=c.$

Ciąg

(a, b, c) jest jednocześnie

arytmetyczny

i

geometryczny,
\begin{center}
\includegraphics[width=195.228mm,height=260.508mm]{./F1_M_PR_M2008_page7_images/image001.eps}

\includegraphics[width=109.932mm,height=17.832mm]{./F1_M_PR_M2008_page7_images/image002.eps}
\end{center}
Wypelnia

egzaminator!

Nr zadania

Maks. liczba kt

1

1

1

Uzyskana liczba pkt





{\it Egzamin maturalny z matematyki 9}

{\it Poziom rozszerzony}

Zadanie 7. $(4pkt)$

Uzasadnij, $\dot{\mathrm{z}}\mathrm{e}\mathrm{k}\mathrm{a}\dot{\mathrm{z}}\mathrm{d}\mathrm{y}$ punkt paraboli o równaniu $y=\displaystyle \frac{1}{4}x^{2}+1$ jest równoodległy od osi

punktu $F=(0,2).$

Ox i od
\begin{center}
\includegraphics[width=195.168mm,height=254.460mm]{./F1_M_PR_M2008_page8_images/image001.eps}

\includegraphics[width=123.900mm,height=17.784mm]{./F1_M_PR_M2008_page8_images/image002.eps}
\end{center}
Nr zadania

Wypelnia Maks. liczba kt

egzamÍnator! Uzyskana liczba pkt

7.1

1

7.2

1

7.3

1

7.4

1





$ 1\theta$ {\it Egzamin maturalny z matematyki}

{\it Poziom rozszerzony}

Zadanie 8. $(4pkt)$

Wyznacz współrzędne środka jednokładności, w której obrazem okręgu o równaniu

$(x-16)^{2}+y^{2}=4$ jest okrąg o równaniu $(x-6)^{2}+(y-4)^{2}=16$, a skala tej jednokładności

jest liczbą ujemną.
\begin{center}
\includegraphics[width=195.228mm,height=260.556mm]{./F1_M_PR_M2008_page9_images/image001.eps}
\end{center}


\end{document}