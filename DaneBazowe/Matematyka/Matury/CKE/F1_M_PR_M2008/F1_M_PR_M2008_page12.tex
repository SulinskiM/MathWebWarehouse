\documentclass[a4paper,12pt]{article}
\usepackage{latexsym}
\usepackage{amsmath}
\usepackage{amssymb}
\usepackage{graphicx}
\usepackage{wrapfig}
\pagestyle{plain}
\usepackage{fancybox}
\usepackage{bm}

\begin{document}

{\it Egzamin maturalny z matematyki 13}

{\it Poziom rozszerzony}

Zadanie 10. $(4pkt)$

$\mathrm{Z}$ pewnej grupy osób, w której jest dwa razy więcej męzczyzn $\mathrm{n}\mathrm{i}\dot{\mathrm{z}}$ kobiet, wybrano losowo

dwuosobową delegację. Prawdopodobieństwo tego, $\dot{\mathrm{z}}\mathrm{e}$ w delegacji znajdą się tylko kobiety

jest równe 0,1. Ob1icz, i1e kobiet i i1u męzczyzn jest w tej grupie.
\begin{center}
\includegraphics[width=195.168mm,height=254.412mm]{./F1_M_PR_M2008_page12_images/image001.eps}

\includegraphics[width=123.900mm,height=17.784mm]{./F1_M_PR_M2008_page12_images/image002.eps}
\end{center}
Nr zadania

Wypelnia Maks. liczba kt

egzaminator! Uzyskana liczba pkt

10.1

1

10.2

1

10.3

1

10.4

1
\end{document}
