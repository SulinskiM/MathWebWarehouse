\documentclass[a4paper,12pt]{article}
\usepackage{latexsym}
\usepackage{amsmath}
\usepackage{amssymb}
\usepackage{graphicx}
\usepackage{wrapfig}
\pagestyle{plain}
\usepackage{fancybox}
\usepackage{bm}

\begin{document}

{\it 14 Egzamin maturalny z matematyki}

{\it Poziom rozszerzony}

Zadanie ll. $(5pkt)$

$\mathrm{W}$ ostrosłupie prawidłowym czworokątnym dane są: $H$ -wysokość ostrosłupa oraz

$\alpha-$ miara kąta utworzonego przez krawędz$\acute{}$ boczną i krawędz$\acute{}$ podstawy $(45^{\circ}<\alpha<90^{\circ}).$

a) Wykaz, $\dot{\mathrm{z}}\mathrm{e}$ objętość $V$ tego ostrosłupajest równa $\displaystyle \frac{4}{3}.\frac{H^{3}}{\mathrm{t}\mathrm{g}^{2}\alpha-1}.$

b) Oblicz miarę kąta $\alpha$, dla której objętość $V$ danego ostrosłupajest równa $\displaystyle \frac{2}{9}H^{3}$ Wynik

podaj w zaokrągleniu do całkowitej liczby stopni.
\begin{center}
\includegraphics[width=195.228mm,height=103.128mm]{./F1_M_PR_M2008_page13_images/image001.eps}
\end{center}\end{document}
