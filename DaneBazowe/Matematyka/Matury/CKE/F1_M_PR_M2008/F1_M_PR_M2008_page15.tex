\documentclass[a4paper,12pt]{article}
\usepackage{latexsym}
\usepackage{amsmath}
\usepackage{amssymb}
\usepackage{graphicx}
\usepackage{wrapfig}
\pagestyle{plain}
\usepackage{fancybox}
\usepackage{bm}

\begin{document}

{\it 16 Egzamin maturalny z matematyki}

{\it Poziom rozszerzony}

Zadanie 12. $(4pkt)$

$\mathrm{W}$ trójkącie prostokątnym $ABC$ przyprostokątne mają długości: $|BC|=9, |CA|=12$. Na boku

$AB$ wybrano punkt $D$ tak, $\dot{\mathrm{z}}\mathrm{e}$ odcinki $BC \mathrm{i}$ CD mają równe długości. Oblicz długość

odcinka $AD.$
\begin{center}
\includegraphics[width=195.228mm,height=266.544mm]{./F1_M_PR_M2008_page15_images/image001.eps}
\end{center}\end{document}
