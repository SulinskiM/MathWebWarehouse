\documentclass[a4paper,12pt]{article}
\usepackage{latexsym}
\usepackage{amsmath}
\usepackage{amssymb}
\usepackage{graphicx}
\usepackage{wrapfig}
\pagestyle{plain}
\usepackage{fancybox}
\usepackage{bm}

\begin{document}

{\it Egzamin maturalny z matematyki 5}

{\it Poziom rozszerzony}

Zadanie 3. $(5pkt)$

Liczby $x_{1}=5+\sqrt{23}\mathrm{i}x_{2}=5-\sqrt{23}$ sąrozwiązaniami równania $x^{2}-(p^{2}+q^{2})x+(p+q)=0$

z niewiadomą $x$. Oblicz wartości $p \mathrm{i}q.$
\begin{center}
\includegraphics[width=195.168mm,height=260.508mm]{./F1_M_PR_M2008_page4_images/image001.eps}

\includegraphics[width=137.928mm,height=17.832mm]{./F1_M_PR_M2008_page4_images/image002.eps}
\end{center}
Wypelnia

egzaminator!

Nr zadania

Maks. liczba kt

3.1

1

3.2

1

3.3

1

3.4

1

3.5

Uzyskana liczba pkt
\end{document}
