\documentclass[a4paper,12pt]{article}
\usepackage{latexsym}
\usepackage{amsmath}
\usepackage{amssymb}
\usepackage{graphicx}
\usepackage{wrapfig}
\pagestyle{plain}
\usepackage{fancybox}
\usepackage{bm}

\begin{document}

{\it 8 Egzamin maturalny z matematyki}

{\it Poziom rozszerzony}

Zadanie 6. $(3pkt)$

Udowodnij, $\dot{\mathrm{z}}\mathrm{e} \mathrm{j}\mathrm{e}\dot{\mathrm{z}}$ eli

to $a=b=c.$

Ciąg

(a, b, c) jest jednocześnie

arytmetyczny

i

geometryczny,
\begin{center}
\includegraphics[width=195.228mm,height=260.508mm]{./F1_M_PR_M2008_page7_images/image001.eps}

\includegraphics[width=109.932mm,height=17.832mm]{./F1_M_PR_M2008_page7_images/image002.eps}
\end{center}
Wypelnia

egzaminator!

Nr zadania

Maks. liczba kt

1

1

1

Uzyskana liczba pkt
\end{document}
