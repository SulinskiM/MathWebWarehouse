\documentclass[a4paper,12pt]{article}
\usepackage{latexsym}
\usepackage{amsmath}
\usepackage{amssymb}
\usepackage{graphicx}
\usepackage{wrapfig}
\pagestyle{plain}
\usepackage{fancybox}
\usepackage{bm}

\begin{document}

{\it Egzamin maturalny z matematyki 9}

{\it Poziom rozszerzony}

Zadanie 7. $(4pkt)$

Uzasadnij, $\dot{\mathrm{z}}\mathrm{e}\mathrm{k}\mathrm{a}\dot{\mathrm{z}}\mathrm{d}\mathrm{y}$ punkt paraboli o równaniu $y=\displaystyle \frac{1}{4}x^{2}+1$ jest równoodległy od osi

punktu $F=(0,2).$

Ox i od
\begin{center}
\includegraphics[width=195.168mm,height=254.460mm]{./F1_M_PR_M2008_page8_images/image001.eps}

\includegraphics[width=123.900mm,height=17.784mm]{./F1_M_PR_M2008_page8_images/image002.eps}
\end{center}
Nr zadania

Wypelnia Maks. liczba kt

egzamÍnator! Uzyskana liczba pkt

7.1

1

7.2

1

7.3

1

7.4

1
\end{document}
