\documentclass[a4paper,12pt]{article}
\usepackage{latexsym}
\usepackage{amsmath}
\usepackage{amssymb}
\usepackage{graphicx}
\usepackage{wrapfig}
\pagestyle{plain}
\usepackage{fancybox}
\usepackage{bm}

\begin{document}

{\it ARKUSZ ZA WIERA INFORMACJE} $PRA$ {\it WNIE CHRONIONE}

{\it DO MOMENTU ROZPOCZĘCIA EGZAMINU}.$\displaystyle \int$
\begin{center}
\includegraphics[width=192.024mm,height=288.084mm]{./F1_M_PR_M2009_page0_images/image001.eps}
\end{center}
Miejsce

na na ejkę

EGZAMIN MATURALNY

Z MATEMATYKI

MAJ

POZIOM ROZSZERZONY

Czas pracy 180 minut

Instrukcja dla zdającego

1.

2.

3.

4.

5.

6.

7.

8.

9.

Sprawd $\acute{\mathrm{z}}$, czy arkusz egzaminacyjny zawiera 16 stron

(zadania $1-11$). Ewentualny brak zgłoś przewodniczącemu

zespo nadzorującego egzamin.

Rozwiązania zadań i odpowiedzi zamieść w miejscu na to

przeznaczonym.

W rozwiązaniach zadań przedstaw tok rozumowania

prowadzący do ostatecznego wyniku.

Pisz czytelnie. Uzywaj $\mathrm{d}$ gopisu pióra tylko z czatnym

tusze atramentem.

Nie uzywaj korektora, a błędne zapisy prze eśl.

Pamiętaj, $\dot{\mathrm{z}}\mathrm{e}$ zapisy w brudnopisie nie podlegają ocenie.

Obok $\mathrm{k}\mathrm{a}\dot{\mathrm{z}}$ dego zadania podanajest maksymalna liczba punktów,

którą $\mathrm{m}\mathrm{o}\dot{\mathrm{z}}$ esz uzyskać zajego poprawne rozwiązanie.

$\mathrm{M}\mathrm{o}\dot{\mathrm{z}}$ esz korzystać z zestawu wzorów matematycznych, cyrkla

i linijki oraz kalkulatora.

Na karcie odpowiedzi wpisz swoją datę urodzenia i PESEL.

Nie wpisuj $\dot{\mathrm{z}}$ adnych znaków w części przeznaczonej dla

egzaminatora.

Za rozwiązanie

wszystkich zadań

mozna otrzymać

łącznie

50 punktów

{\it Zyczymy powodzenia}.'

Wypelnia zdający

rzed roz oczęciem racy

PESEL ZDAJACEGO

KOD

ZDAJACEGO




{\it 2}

{\it Egzamin maturalny z matematyki}

{\it Poziom rozszerzony}

Zadanie l. $(4pkt)$

Funkcja liniowa $f$ określona jest wzorem $f(x)=ax+b$ dla $x\in R.$

a) Dla $a=2008\mathrm{i}b=2009$ zbadaj, czy do wykresu tej ffinkcji nalezypunkt $P=(2009,2009^{2}).$

b) Narysuj w układzie współrzędnych zbiór

$A=\displaystyle \{(x,y):x\in\langle-1,3\rangle\mathrm{i}y=-\frac{1}{2}x+b\mathrm{i}b\in\langle-2,1\rangle\}.$
\begin{center}
\includegraphics[width=123.900mm,height=17.628mm]{./F1_M_PR_M2009_page1_images/image001.eps}
\end{center}
Wypelnia

egzaminator!

Nr czynności

Maks. liczba kt

1

1.2.

1.3.

1.4.

1

Uzyskana liczba pkt





{\it Egzamin maturalny z matematyki}

{\it Poziom rozszerzony}

{\it 11}
\begin{center}
\includegraphics[width=123.900mm,height=17.628mm]{./F1_M_PR_M2009_page10_images/image001.eps}
\end{center}
Nr czynnoŚci

Wypelnia Maks. liczba kt

egzaminator! Uzyskana lÍczba pkt

8.1.

1

8.2.

1

8.3.

1

8.4.

1





{\it 12}

{\it Egzamin maturalny z matematyki}

{\it Poziom rozszerzony}

Zadanie 9. $(5pkt)$

$\mathrm{W}$ układzie współrzędnych narysuj okrąg o równaniu $(x+2)^{2}+(y-3)^{2}=4$ oraz zaznacz

punkt $A=(0,-1)$. Prosta o równaniu $x=0$ jest jedną ze stycznych do tego okręgu

przechodzących przez punkt $A$. Wyznacz równanie drugiej stycznej do tego okręgu,

przechodzącej przez punkt $A.$
\begin{center}
\includegraphics[width=137.868mm,height=17.580mm]{./F1_M_PR_M2009_page11_images/image001.eps}
\end{center}
Nr czynnoŚci

Wypelnia Maks. liczba kt

egzaminator! Uzyskana lÍczba pkt

1

1

1





{\it Egzamin maturalny z matematyki}

{\it Poziom rozszerzony}

{\it 13}

Zadanie 10. $(4pkt)$

$\mathrm{W}$ urnie znajdują się jedynie kule białe i czarne. Kul białych jest trzy razy więcej

$\mathrm{n}\mathrm{i}\dot{\mathrm{z}}$ czarnych. Oblicz, ile jest kul w umie, jeśli przy jednoczesnym losowaniu dwóch kul

prawdopodobieństwo otrzymania kul o róznych kolorachjest większe od $\displaystyle \frac{9}{22}$
\begin{center}
\includegraphics[width=123.900mm,height=17.628mm]{./F1_M_PR_M2009_page12_images/image001.eps}
\end{center}
Nr czynnoŚci

Wypelnia Maks. liczba kt

egzaminator! Uzyskana lÍczba pkt

10.1.

1

10.2.

10.3.

10.4.

1





{\it 14}

{\it Egzamin maturalny z matematyki}

{\it Poziom rozszerzony}

Zadanie 11. (6pkt)

Dany jest ostrosłup prawidłowy trójkątny, w którym krawędzí podstawy ma długość a

i krawędzí bocznajest od niej dwa razy dłuzsza. Oblicz cosinus kąta między krawędzią boczną

i krawędzią podstawy ostrosłupa. Narysuj przekrój ostrosłupa płaszczyzną przechodzącą

przez krawędzí podstawy i środek przeciwległej krawędzi bocznej i oblicz pole tego przekroju.





{\it Egzamin maturalny z matematyki}

{\it Poziom rozszerzony}

{\it 15}
\begin{center}
\includegraphics[width=151.788mm,height=17.580mm]{./F1_M_PR_M2009_page14_images/image001.eps}
\end{center}
Wypelnia

egzamÍnator!

Nr czynnoścÍ

Maks. liczba kt

11.1.

1

11.2.

1

11.3.

1

11.4.

11.5.

1

1

Uzyskana liczba pkt





{\it 16}

{\it Egzamin maturalny z matematyki}

{\it Poziom rozszerzony}

BRUDNOPIS





{\it Egzamin maturalny z matematyki}

{\it Poziom rozszerzony}

{\it 3}

Zadanie 2. $(4pkt)$

Przy dzieleniu wielomianu $W(x)$ przez dwumian $(x-1)$ otrzymujemy

$Q(x)=8x^{2}+4x-14$ oraz resztę $R(x)=-5$. Oblicz pierwiastki wielomianu $W(x).$

iloraz
\begin{center}
\includegraphics[width=123.900mm,height=17.580mm]{./F1_M_PR_M2009_page2_images/image001.eps}
\end{center}
Nr czynnoŚci

Wypelnia Maks. liczba kt

egzamÍnator! Uzyskana lÍczba pkt

2.1.

1

2.2.

2.3.

2.4.

1





{\it 4}

{\it Egzamin maturalny z matematyki}

{\it Poziom rozszerzony}

Zadanie 3. $(4pkt)$

Na rysunku przedstawiony jest wykres funkcji wykładniczej $f(x)=a^{x}$ dla $x\in R.$
\begin{center}
\includegraphics[width=112.272mm,height=113.388mm]{./F1_M_PR_M2009_page3_images/image001.eps}
\end{center}
$\gamma$

$5$

4

3

2

$-4 -3  -2 -1$  0 1

$111$

$1$

$1$

$1$

$1$

$1$

$1$

$1$

$\rangle$

$1$

$1$

$111$

2

3 4 x

$-1$

$-2$

$-3$

a) Oblicz $a.$

b) Narysuj wykres funkcji $g(x)=|f(x)-2|$ i podaj wszystkie wartości parametru $m\in R,$

dla których równanie $g(x)=m$ ma dokładniejedno rozwiązanie.





{\it Egzamin maturalny z matematyki}

{\it Poziom rozszerzony}

{\it 5}
\begin{center}
\includegraphics[width=123.900mm,height=17.580mm]{./F1_M_PR_M2009_page4_images/image001.eps}
\end{center}
Nr czynności

Wypelnia Maks. liczba kt

egzamÍnator! Uzyskana liczba pkt

3.1.

1

3.2.

3.3.

3.4.

1





{\it 6}

{\it Egzamin maturalny z matematyki}

{\it Poziom rozszerzony}

Zadanie 4. $(5pkt)$

$\mathrm{W}$ skarbcu królewskim było $k$ monet. Pierwszego dnia rano skarbnik dorzucił 25 monet,

a $\mathrm{k}\mathrm{a}\dot{\mathrm{z}}$ dego następnego ranka dorzucał o 2 monety więcej $\mathrm{n}\mathrm{i}\dot{\mathrm{z}}$ dnia poprzedniego. Jednocześnie

ze skarbca król zabierał w południe $\mathrm{k}\mathrm{a}\dot{\mathrm{z}}$ dego dnia 50 monet. Ob1icz najmniejszą 1iczbę $k$, dla

której w kazdym dniu w skarbcu była co najmniej jedna moneta, a następnie dla tej wartości $k$

oblicz, w którym dniu w skarbcu była najmniejsza liczba monet.
\begin{center}
\includegraphics[width=137.868mm,height=17.628mm]{./F1_M_PR_M2009_page5_images/image001.eps}
\end{center}
Nr czynnoŚci

Wypelnia Maks. liczba kt

egzaminator! Uzyskana lÍczba pkt

4.1.

1

4.2.

1

4.3.

1

4.4.

1

4.5.





{\it Egzamin maturalny z matematyki}

{\it Poziom rozszerzony}

7

Zadanie 5. $(3pkt)$

Wykaz, $\dot{\mathrm{z}}\mathrm{e}\mathrm{j}\mathrm{e}\dot{\mathrm{z}}$ eli $A=3^{4\sqrt{2}+2}$

$\mathrm{i} B=3^{2\sqrt{2}+3}$, to $B=9\sqrt{A}.$
\begin{center}
\includegraphics[width=109.980mm,height=17.580mm]{./F1_M_PR_M2009_page6_images/image001.eps}
\end{center}
Nr czynności

Wypelnia Maks. liczba kt

egzaminator! Uzyskana liczba pkt

5.1.

1

5.2.

1

5.3.

1





{\it 8}

{\it Egzamin maturalny z matematyki}

{\it Poziom rozszerzony}

Zadanie 6. $(5pkt)$

Wyznacz dziedzinę funkcji $f(x)=\log_{2\cos x}(9-x^{2})$ i zapisz ją w postaci sumy przedziałów

liczbowych.
\begin{center}
\includegraphics[width=137.868mm,height=17.580mm]{./F1_M_PR_M2009_page7_images/image001.eps}
\end{center}
Nr czynności

Wypelnia Maks. liczba kt

egzaminator! Uzyskana liczba pkt

1

1

1

1





{\it Egzamin maturalny z matematyki}

{\it Poziom rozszerzony}

{\it 9}

Zadanie 7. $(6pkt)$

Ciąg $(x-3,x+3,6x+2,\ldots)$

jest nieskończonym ciągiem geometrycznym o wyrazach

dodatnich. Oblicz iloraz tego ciągu i uzasadnij,

n początkowych wyrazów tego ciągu.

$\dot{\mathrm{z}}\mathrm{e} \displaystyle \frac{S_{19}}{S_{20}}<\frac{1}{4}$, gdzie $S_{n}$ oznacza sumę
\begin{center}
\includegraphics[width=151.788mm,height=17.628mm]{./F1_M_PR_M2009_page8_images/image001.eps}
\end{center}
Wypelnia

egzaminator!

Nr czynnoŚci

Maks. liczba kt

7.1.

1

7.2.

1

7.3.

1

7.4.

7.5.

1

1

Uzyskana lÍczba pkt





$ 1\theta$

{\it Egzamin maturalny z matematyki}

{\it Poziom rozszerzony}

Zadanie 8. $(4pkt)$

Dwa okręgi o środkach $A\mathrm{i}B$ są styczne zewnętrznie i $\mathrm{k}\mathrm{a}\dot{\mathrm{z}}\mathrm{d}\mathrm{y}$ z nichjestjednocześnie styczny

do ramion tego samego kąta prostego (patrz rysunek). Udowodnij, $\dot{\mathrm{z}}\mathrm{e}$ stosunek promienia

większego z tych okręgów do promienia mniejszegojest równy $3+2\sqrt{2}.$

{\it B}.

{\it A}.



\end{document}