\documentclass[a4paper,12pt]{article}
\usepackage{latexsym}
\usepackage{amsmath}
\usepackage{amssymb}
\usepackage{graphicx}
\usepackage{wrapfig}
\pagestyle{plain}
\usepackage{fancybox}
\usepackage{bm}

\begin{document}

{\it ARKUSZ ZA WIERA INFORMACJE} $PRA$ {\it WNIE CHRONIONE}

{\it DO MOMENTU ROZPOCZĘCIA EGZAMINU}.$\displaystyle \int$
\begin{center}
\includegraphics[width=192.024mm,height=288.084mm]{./F1_M_PR_M2009_page0_images/image001.eps}
\end{center}
Miejsce

na na ejkę

EGZAMIN MATURALNY

Z MATEMATYKI

MAJ

POZIOM ROZSZERZONY

Czas pracy 180 minut

Instrukcja dla zdającego

1.

2.

3.

4.

5.

6.

7.

8.

9.

Sprawd $\acute{\mathrm{z}}$, czy arkusz egzaminacyjny zawiera 16 stron

(zadania $1-11$). Ewentualny brak zgłoś przewodniczącemu

zespo nadzorującego egzamin.

Rozwiązania zadań i odpowiedzi zamieść w miejscu na to

przeznaczonym.

W rozwiązaniach zadań przedstaw tok rozumowania

prowadzący do ostatecznego wyniku.

Pisz czytelnie. Uzywaj $\mathrm{d}$ gopisu pióra tylko z czatnym

tusze atramentem.

Nie uzywaj korektora, a błędne zapisy prze eśl.

Pamiętaj, $\dot{\mathrm{z}}\mathrm{e}$ zapisy w brudnopisie nie podlegają ocenie.

Obok $\mathrm{k}\mathrm{a}\dot{\mathrm{z}}$ dego zadania podanajest maksymalna liczba punktów,

którą $\mathrm{m}\mathrm{o}\dot{\mathrm{z}}$ esz uzyskać zajego poprawne rozwiązanie.

$\mathrm{M}\mathrm{o}\dot{\mathrm{z}}$ esz korzystać z zestawu wzorów matematycznych, cyrkla

i linijki oraz kalkulatora.

Na karcie odpowiedzi wpisz swoją datę urodzenia i PESEL.

Nie wpisuj $\dot{\mathrm{z}}$ adnych znaków w części przeznaczonej dla

egzaminatora.

Za rozwiązanie

wszystkich zadań

mozna otrzymać

łącznie

50 punktów

{\it Zyczymy powodzenia}.'

Wypelnia zdający

rzed roz oczęciem racy

PESEL ZDAJACEGO

KOD

ZDAJACEGO
\end{document}
