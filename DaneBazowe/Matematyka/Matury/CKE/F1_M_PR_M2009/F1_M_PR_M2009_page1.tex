\documentclass[a4paper,12pt]{article}
\usepackage{latexsym}
\usepackage{amsmath}
\usepackage{amssymb}
\usepackage{graphicx}
\usepackage{wrapfig}
\pagestyle{plain}
\usepackage{fancybox}
\usepackage{bm}

\begin{document}

{\it 2}

{\it Egzamin maturalny z matematyki}

{\it Poziom rozszerzony}

Zadanie l. $(4pkt)$

Funkcja liniowa $f$ określona jest wzorem $f(x)=ax+b$ dla $x\in R.$

a) Dla $a=2008\mathrm{i}b=2009$ zbadaj, czy do wykresu tej ffinkcji nalezypunkt $P=(2009,2009^{2}).$

b) Narysuj w układzie współrzędnych zbiór

$A=\displaystyle \{(x,y):x\in\langle-1,3\rangle\mathrm{i}y=-\frac{1}{2}x+b\mathrm{i}b\in\langle-2,1\rangle\}.$
\begin{center}
\includegraphics[width=123.900mm,height=17.628mm]{./F1_M_PR_M2009_page1_images/image001.eps}
\end{center}
Wypelnia

egzaminator!

Nr czynności

Maks. liczba kt

1

1.2.

1.3.

1.4.

1

Uzyskana liczba pkt
\end{document}
