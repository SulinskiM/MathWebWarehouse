\documentclass[a4paper,12pt]{article}
\usepackage{latexsym}
\usepackage{amsmath}
\usepackage{amssymb}
\usepackage{graphicx}
\usepackage{wrapfig}
\pagestyle{plain}
\usepackage{fancybox}
\usepackage{bm}

\begin{document}

{\it 12}

{\it Egzamin maturalny z matematyki}

{\it Poziom rozszerzony}

Zadanie 9. $(5pkt)$

$\mathrm{W}$ układzie współrzędnych narysuj okrąg o równaniu $(x+2)^{2}+(y-3)^{2}=4$ oraz zaznacz

punkt $A=(0,-1)$. Prosta o równaniu $x=0$ jest jedną ze stycznych do tego okręgu

przechodzących przez punkt $A$. Wyznacz równanie drugiej stycznej do tego okręgu,

przechodzącej przez punkt $A.$
\begin{center}
\includegraphics[width=137.868mm,height=17.580mm]{./F1_M_PR_M2009_page11_images/image001.eps}
\end{center}
Nr czynnoŚci

Wypelnia Maks. liczba kt

egzaminator! Uzyskana lÍczba pkt

1

1

1
\end{document}
