\documentclass[a4paper,12pt]{article}
\usepackage{latexsym}
\usepackage{amsmath}
\usepackage{amssymb}
\usepackage{graphicx}
\usepackage{wrapfig}
\pagestyle{plain}
\usepackage{fancybox}
\usepackage{bm}

\begin{document}

{\it Egzamin maturalny z matematyki}

{\it Poziom rozszerzony}

{\it 13}

Zadanie 10. $(4pkt)$

$\mathrm{W}$ urnie znajdują się jedynie kule białe i czarne. Kul białych jest trzy razy więcej

$\mathrm{n}\mathrm{i}\dot{\mathrm{z}}$ czarnych. Oblicz, ile jest kul w umie, jeśli przy jednoczesnym losowaniu dwóch kul

prawdopodobieństwo otrzymania kul o róznych kolorachjest większe od $\displaystyle \frac{9}{22}$
\begin{center}
\includegraphics[width=123.900mm,height=17.628mm]{./F1_M_PR_M2009_page12_images/image001.eps}
\end{center}
Nr czynnoŚci

Wypelnia Maks. liczba kt

egzaminator! Uzyskana lÍczba pkt

10.1.

1

10.2.

10.3.

10.4.

1
\end{document}
