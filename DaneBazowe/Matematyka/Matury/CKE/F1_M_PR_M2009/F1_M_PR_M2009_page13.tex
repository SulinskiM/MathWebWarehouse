\documentclass[a4paper,12pt]{article}
\usepackage{latexsym}
\usepackage{amsmath}
\usepackage{amssymb}
\usepackage{graphicx}
\usepackage{wrapfig}
\pagestyle{plain}
\usepackage{fancybox}
\usepackage{bm}

\begin{document}

{\it 14}

{\it Egzamin maturalny z matematyki}

{\it Poziom rozszerzony}

Zadanie 11. (6pkt)

Dany jest ostrosłup prawidłowy trójkątny, w którym krawędzí podstawy ma długość a

i krawędzí bocznajest od niej dwa razy dłuzsza. Oblicz cosinus kąta między krawędzią boczną

i krawędzią podstawy ostrosłupa. Narysuj przekrój ostrosłupa płaszczyzną przechodzącą

przez krawędzí podstawy i środek przeciwległej krawędzi bocznej i oblicz pole tego przekroju.
\end{document}
