\documentclass[a4paper,12pt]{article}
\usepackage{latexsym}
\usepackage{amsmath}
\usepackage{amssymb}
\usepackage{graphicx}
\usepackage{wrapfig}
\pagestyle{plain}
\usepackage{fancybox}
\usepackage{bm}

\begin{document}

{\it 4}

{\it Egzamin maturalny z matematyki}

{\it Poziom rozszerzony}

Zadanie 3. $(4pkt)$

Na rysunku przedstawiony jest wykres funkcji wykładniczej $f(x)=a^{x}$ dla $x\in R.$
\begin{center}
\includegraphics[width=112.272mm,height=113.388mm]{./F1_M_PR_M2009_page3_images/image001.eps}
\end{center}
$\gamma$

$5$

4

3

2

$-4 -3  -2 -1$  0 1

$111$

$1$

$1$

$1$

$1$

$1$

$1$

$1$

$\rangle$

$1$

$1$

$111$

2

3 4 x

$-1$

$-2$

$-3$

a) Oblicz $a.$

b) Narysuj wykres funkcji $g(x)=|f(x)-2|$ i podaj wszystkie wartości parametru $m\in R,$

dla których równanie $g(x)=m$ ma dokładniejedno rozwiązanie.
\end{document}
