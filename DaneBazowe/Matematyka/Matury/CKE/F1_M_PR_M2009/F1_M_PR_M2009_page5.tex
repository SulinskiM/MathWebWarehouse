\documentclass[a4paper,12pt]{article}
\usepackage{latexsym}
\usepackage{amsmath}
\usepackage{amssymb}
\usepackage{graphicx}
\usepackage{wrapfig}
\pagestyle{plain}
\usepackage{fancybox}
\usepackage{bm}

\begin{document}

{\it 6}

{\it Egzamin maturalny z matematyki}

{\it Poziom rozszerzony}

Zadanie 4. $(5pkt)$

$\mathrm{W}$ skarbcu królewskim było $k$ monet. Pierwszego dnia rano skarbnik dorzucił 25 monet,

a $\mathrm{k}\mathrm{a}\dot{\mathrm{z}}$ dego następnego ranka dorzucał o 2 monety więcej $\mathrm{n}\mathrm{i}\dot{\mathrm{z}}$ dnia poprzedniego. Jednocześnie

ze skarbca król zabierał w południe $\mathrm{k}\mathrm{a}\dot{\mathrm{z}}$ dego dnia 50 monet. Ob1icz najmniejszą 1iczbę $k$, dla

której w kazdym dniu w skarbcu była co najmniej jedna moneta, a następnie dla tej wartości $k$

oblicz, w którym dniu w skarbcu była najmniejsza liczba monet.
\begin{center}
\includegraphics[width=137.868mm,height=17.628mm]{./F1_M_PR_M2009_page5_images/image001.eps}
\end{center}
Nr czynnoŚci

Wypelnia Maks. liczba kt

egzaminator! Uzyskana lÍczba pkt

4.1.

1

4.2.

1

4.3.

1

4.4.

1

4.5.
\end{document}
