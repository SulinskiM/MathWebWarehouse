\documentclass[a4paper,12pt]{article}
\usepackage{latexsym}
\usepackage{amsmath}
\usepackage{amssymb}
\usepackage{graphicx}
\usepackage{wrapfig}
\pagestyle{plain}
\usepackage{fancybox}
\usepackage{bm}

\begin{document}

{\it Egzamin maturalny z matematyki}

{\it Poziom rozszerzony}

{\it 9}

Zadanie 7. $(6pkt)$

Ciąg $(x-3,x+3,6x+2,\ldots)$

jest nieskończonym ciągiem geometrycznym o wyrazach

dodatnich. Oblicz iloraz tego ciągu i uzasadnij,

n początkowych wyrazów tego ciągu.

$\dot{\mathrm{z}}\mathrm{e} \displaystyle \frac{S_{19}}{S_{20}}<\frac{1}{4}$, gdzie $S_{n}$ oznacza sumę
\begin{center}
\includegraphics[width=151.788mm,height=17.628mm]{./F1_M_PR_M2009_page8_images/image001.eps}
\end{center}
Wypelnia

egzaminator!

Nr czynnoŚci

Maks. liczba kt

7.1.

1

7.2.

1

7.3.

1

7.4.

7.5.

1

1

Uzyskana lÍczba pkt
\end{document}
