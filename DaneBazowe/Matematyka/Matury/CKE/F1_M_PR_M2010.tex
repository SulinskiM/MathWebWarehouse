\documentclass[a4paper,12pt]{article}
\usepackage{latexsym}
\usepackage{amsmath}
\usepackage{amssymb}
\usepackage{graphicx}
\usepackage{wrapfig}
\pagestyle{plain}
\usepackage{fancybox}
\usepackage{bm}

\begin{document}

Centralna Komisja Egzaminacyjna

Arkusz zawiera informacje prawnie chronione do momentu rozpoczęcia egzaminu.

WPISUJE ZDAJACY

KOD PESEL

{\it Miejsce}

{\it na naklejkę}

{\it z kodem}
\begin{center}
\includegraphics[width=21.432mm,height=9.852mm]{./F1_M_PR_M2010_page0_images/image001.eps}

\includegraphics[width=82.092mm,height=9.852mm]{./F1_M_PR_M2010_page0_images/image002.eps}

\includegraphics[width=204.060mm,height=216.048mm]{./F1_M_PR_M2010_page0_images/image003.eps}
\end{center}
EGZAMIN MATU  LNY

Z MATEMATY  MAJ 2010

POZIOM ROZSZERZONY

1.

Czas pracy:

180 minut

3.

Sprawdzí, czy arkusz egzaminacyjny zawiera 24 strony

(zadania $1-11$). Ewentualny brak zgłoś

przewodniczącemu zespo nadzo jącego egzamin.

Rozwiązania zadań i odpowiedzi wpisuj w miejscu na to

przeznaczonym.

Pamiętaj, $\dot{\mathrm{z}}\mathrm{e}$ pominięcie argumentacji lub istotnych

obliczeń w rozwiązaniu zadania otwa ego $\mathrm{m}\mathrm{o}\dot{\mathrm{z}}\mathrm{e}$

spowodować, $\dot{\mathrm{z}}\mathrm{e}$ za to rozwiązanie nie będziesz mógł

dostać pełnej liczby punktów.

Pisz czytelnie i $\mathrm{u}\dot{\mathrm{z}}$ aj tvlko $\mathrm{d}$ gopisu lub -Dióra

z czatnym tuszem lub atramentem.

Nie uzywaj korektora, a błędne zapisy wyrazínie prze eśl.

Pamiętaj, $\dot{\mathrm{z}}\mathrm{e}$ zapisy w brudnopisie nie będą oceniane.

$\mathrm{M}\mathrm{o}\dot{\mathrm{z}}$ esz korzystać z zestawu wzorów matematycznych,

cyrkla i linijki oraz kalkulatora.

Na karcie odpowiedzi wpisz swój numer PESEL i przyklej

naklejkę z kodem.

Nie wpisuj $\dot{\mathrm{z}}$ adnych znaków w części przeznaczonej dla

egzaminatora.

2.

4.

5.

6.

7.

8.

9.

Liczba punktów

do uzyskania: 50

$\Vert\Vert\Vert\Vert\Vert\Vert\Vert\Vert\Vert\Vert\Vert\Vert\Vert\Vert\Vert\Vert\Vert\Vert\Vert\Vert\Vert\Vert\Vert\Vert|  \mathrm{M}\mathrm{M}\mathrm{A}-\mathrm{R}1_{-}1\mathrm{P}-102$




{\it 2}

{\it Egzamin maturalny z matematyki}

{\it Poziom rozszerzony}

Zadanie l. $(4pkt)$

Rozwiąz nierówność $|2x+4|+|x-1|\leq 6.$





{\it Egzamin maturalny z matematyki}

{\it Poziom rozszerzony}

{\it 11}
\begin{center}
\includegraphics[width=82.044mm,height=17.832mm]{./F1_M_PR_M2010_page10_images/image001.eps}
\end{center}
Wypelnia

egzaminator

Nr zadania

Maks. liczba kt

5.

5

Uzyskana liczba pkt





{\it 12}

{\it Egzamin maturalny z matematyki}

{\it Poziom rozszerzony}

Zadanie 6. $(5pkt)$

Wyznacz wszystkie wartości parametru $m$, dla których równanie $x^{2}+mx+2=0$ ma dwa

rózne pierwiastki rzeczywiste takie, $\dot{\mathrm{z}}\mathrm{e}$ suma ich kwadratówjest większa od $2m^{2}-13.$





{\it Egzamin maturalny z matematyki}

{\it Poziom rozszerzony}

{\it 13}
\begin{center}
\includegraphics[width=82.044mm,height=17.832mm]{./F1_M_PR_M2010_page12_images/image001.eps}
\end{center}
Wypelnia

egzaminator

Nr zadania

Maks. liczba kt

5

Uzyskana liczba pkt





{\it 14}

{\it Egzamin maturalny z matematyki}

{\it Poziom rozszerzony}

Zadanie 7. $(6pkt)$

Punkt $A=(-2,5)$ jest jednym z wierzchołków trójkąta równoramiennego $ABC$, w którym

$|AC|=|BC|$. Pole tego trójkąta jest równe 15. Bok $BC$ jest zawarty w prostej o równaniu

$y=x+1$. Oblicz współrzędne wierzchołka $C.$





{\it Egzamin maturalny z matematyki}

{\it Poziom rozszerzony}

{\it 15}
\begin{center}
\includegraphics[width=82.044mm,height=17.832mm]{./F1_M_PR_M2010_page14_images/image001.eps}
\end{center}
Wypelnia

egzaminator

Nr zadania

Maks. liczba kt

7.

Uzyskana liczba pkt





{\it 16}

{\it Egzamin maturalny z matematyki}

{\it Poziom rozszerzony}

Zadanie 8. (5pkt)

Rysunek przedstawia fragment

wykresu funkcji

$f(x)=\displaystyle \frac{1}{x^{2}}.$

Przeprowadzono prostą

równoległą do osi $Ox$, która przecięła wykres tej funkcji w punktach $A$

$C=(3,-1)$. Wykaz, $\dot{\mathrm{z}}\mathrm{e}$ pole trójkąta ABCjest większe lub równe 2.

i B. Niech
\begin{center}
\includegraphics[width=161.748mm,height=105.252mm]{./F1_M_PR_M2010_page15_images/image001.eps}
\end{center}




{\it Egzamin maturalny z matematyki}

{\it Poziom rozszerzony}

17
\begin{center}
\includegraphics[width=82.044mm,height=17.832mm]{./F1_M_PR_M2010_page16_images/image001.eps}
\end{center}
Wypelnia

egzamÍnator

Nr zadania

Maks. liczba kt

8.

5

Uzyskana liczba pkt





{\it 18}

{\it Egzamin maturalny z matematyki}

{\it Poziom rozszerzony}

Zadanie 9. $(4pkt)$

Na bokach $BC\mathrm{i}$ CD równoległoboku ABCD zbudowano kwadraty CDEF $\mathrm{i}$ {\it BCGH} ({\it zobacz}

rysunek). Udowodnij, $\dot{\mathrm{z}}\mathrm{e}|AC|=|FG|.$
\begin{center}
\includegraphics[width=96.216mm,height=93.324mm]{./F1_M_PR_M2010_page17_images/image001.eps}
\end{center}
{\it E} $F$

{\it D  C}

{\it G}

{\it A  B}

{\it H}





{\it Egzamin maturalny z matematyki}

{\it Poziom rozszerzony}

{\it 19}
\begin{center}
\includegraphics[width=82.044mm,height=17.832mm]{./F1_M_PR_M2010_page18_images/image001.eps}
\end{center}
Wypelnia

egzaminator

Nr zadania

Maks. liczba kt

4

Uzyskana liczba pkt





$ 2\theta$

{\it Egzamin maturalny z matematyki}

{\it Poziom rozszerzony}

Zadanie 10. (4pkt)

Oblicz prawdopodobieństwo tego, ze w trzech rzutach symetryczną sześcienną kostką do gry suma

kwadratów liczb uzyskanych oczek bę\& ie podzielna przez 3.





{\it Egzamin maturalny z matematyki}

{\it Poziom rozszerzony}

{\it 3}
\begin{center}
\includegraphics[width=82.044mm,height=17.832mm]{./F1_M_PR_M2010_page2_images/image001.eps}
\end{center}
Wypelnia

egzamÍnator

Nr zadania

Maks. liczba kt

1.

4

Uzyskana liczba pkt





{\it Egzamin maturalny z matematyki}

{\it Poziom rozszerzony}

{\it 21}
\begin{center}
\includegraphics[width=82.044mm,height=17.832mm]{./F1_M_PR_M2010_page20_images/image001.eps}
\end{center}
Wypelnia

egzaminator

Nr zadania

Maks. liczba kt

10.

4

Uzyskana liczba pkt





{\it 22}

{\it Egzamin maturalny z matematyki}

{\it Poziom rozszerzony}

Zadanie ll. $(5pkt)$

$\mathrm{W}$ ostrosłupie prawidłowym trójkątnym krawędzí podstawy ma długość $a$. Ściany boczne są

trójkątami ostrokątnymi. Miara kąta między sąsiednimi ścianami bocznymi jest równa $2\alpha.$

Wyznacz objętość tego ostrosłupa.





{\it Egzamin maturalny z matematyki}

{\it Poziom rozszerzony}

{\it 23}
\begin{center}
\includegraphics[width=82.044mm,height=17.832mm]{./F1_M_PR_M2010_page22_images/image001.eps}
\end{center}
Wypelnia

egzaminator

Nr zadania

Maks. liczba kt

11.

5

Uzyskana liczba pkt





{\it 24}

{\it Egzamin maturalny z matematyki}

{\it Poziom rozszerzony}

BRUDNOPIS





{\it 4}

{\it Egzamin maturalny z matematyki}

{\it Poziom rozszerzony}

Zadanie 2. $(4pkt)$

Wyznacz wszystkie rozwiązania równania 2 $\cos^{2}x-5\sin x-4=0$

$\langle 0, 2\pi\rangle.$

nalezące do przedziału





{\it Egzamin maturalny z matematyki}

{\it Poziom rozszerzony}

{\it 5}
\begin{center}
\includegraphics[width=82.044mm,height=17.832mm]{./F1_M_PR_M2010_page4_images/image001.eps}
\end{center}
Wypelnia

egzaminator

Nr zadania

Maks. liczba kt

2.

4

Uzyskana liczba pkt





{\it 6}

{\it Egzamin maturalny z matematyki}

{\it Poziom rozszerzony}

Zadanie 3. $(4pkt)$

Bok kwadratu ABCD ma długość l. Na bokach $BC\mathrm{i}$ CD wybrano odpowiednio punkty $E\mathrm{i}F$

umieszczone tak, by $|CE|=2|DF|$. Oblicz wartość $x=|DF|$, dla której pole trójkąta $AEF$

jest najmniejsze.





{\it Egzamin maturalny z matematyki}

{\it Poziom rozszerzony}

7
\begin{center}
\includegraphics[width=82.044mm,height=17.832mm]{./F1_M_PR_M2010_page6_images/image001.eps}
\end{center}
Wypelnia

egzaminator

Nr zadania

Maks. liczba kt

3.

4

Uzyskana liczba pkt





{\it 8}

{\it Egzamin maturalny z matematyki}

{\it Poziom rozszerzony}

Zadanie 4. $(4pkt)$

Wyznacz wartości $a\mathrm{i}b$ współczynników wielomianu $W(x)=x^{3}+ax^{2}+bx+1$

wiedząc, $\dot{\mathrm{z}}\mathrm{e}$

$W(2)=7$ oraz, $\dot{\mathrm{z}}\mathrm{e}$ reszta z dzielenia $W(x)$ przez $(x-3)$ jest równa 10.





{\it Egzamin maturalny z matematyki}

{\it Poziom rozszerzony}

{\it 9}
\begin{center}
\includegraphics[width=82.044mm,height=17.832mm]{./F1_M_PR_M2010_page8_images/image001.eps}
\end{center}
Wypelnia

egzaminator

Nr zadania

Maks. liczba kt

4.

4

Uzyskana liczba pkt





$ 1\theta$

{\it Egzamin maturalny z matematyki}

{\it Poziom rozszerzony}

Zadanie 5. $(5pkt)$

$\mathrm{O}$ liczbach $a, b, c$ wiemy, $\dot{\mathrm{z}}\mathrm{e}$ ciąg $(a,b,c)$ jest arytmetyczny

$(a+1,b+4,c+19)$ jest geometryczny. Wyznacz te liczby.

i

$a+c=10$, zaś ciąg



\end{document}