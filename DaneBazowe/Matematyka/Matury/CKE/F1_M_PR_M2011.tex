\documentclass[a4paper,12pt]{article}
\usepackage{latexsym}
\usepackage{amsmath}
\usepackage{amssymb}
\usepackage{graphicx}
\usepackage{wrapfig}
\pagestyle{plain}
\usepackage{fancybox}
\usepackage{bm}

\begin{document}

Centralna Komisja Egzaminacyjna

Arkusz zawiera informacje prawnie chronione do momentu rozpoczęcia egzaminu.

WPISUJE ZDAJACY

KOD PESEL

{\it Miejsce}

{\it na naklejkę}

{\it z kodem}
\begin{center}
\includegraphics[width=21.432mm,height=9.804mm]{./F1_M_PR_M2011_page0_images/image001.eps}

\includegraphics[width=82.092mm,height=9.804mm]{./F1_M_PR_M2011_page0_images/image002.eps}

\includegraphics[width=204.060mm,height=216.048mm]{./F1_M_PR_M2011_page0_images/image003.eps}
\end{center}
EGZAMIN MATU

Z MATEMATY

LNY

MAJ 2011

POZIOM ROZSZE ONY

1.

Czas pracy:

180 minut

3.

Sprawd $\acute{\mathrm{z}}$, czy arkusz egzaminacyjny zawiera 19 stron

(zadania $1-12$). Ewentualny brak zgłoś

przewodniczącemu zespo nadzorującego egzamin.

Rozwiązania zadań i odpowiedzi wpisuj w miejscu na to

przeznaczonym.

Pamiętaj, $\dot{\mathrm{z}}\mathrm{e}$ pominięcie argumentacji lub istotnych

obliczeń w rozwiązaniu zadania otwa ego $\mathrm{m}\mathrm{o}\dot{\mathrm{z}}\mathrm{e}$

spowodować, $\dot{\mathrm{z}}\mathrm{e}$ za to rozwiązanie nie będziesz mógł

dostać pełnej liczby punktów.

Pisz czytelnie i $\mathrm{u}\dot{\mathrm{z}}$ aj tvlko $\mathrm{d}$ gopisu lub -Dióra

z czamym tuszem lub atramentem.

Nie $\mathrm{u}\dot{\mathrm{z}}$ aj korektora, a błędne zapisy wyra $\acute{\mathrm{z}}\mathrm{n}\mathrm{i}\mathrm{e}$ prze eśl.

Pamiętaj, $\dot{\mathrm{z}}\mathrm{e}$ zapisy w brudnopisie nie będą oceniane.

$\mathrm{M}\mathrm{o}\dot{\mathrm{z}}$ esz korzystać z zestawu wzorów matematycznych,

cyrkla i linijki oraz kalkulatora.

Na karcie odpowiedzi wpisz swój numer PESEL i przyklej

naklejkę z kodem.

Nie wpisuj $\dot{\mathrm{z}}$ adnych znaków w części przeznaczonej dla

egzaminatora.

2.

4.

5.

6.

7.

8.

9.

Liczba punktów

do uzyskania: 50

$\Vert\Vert\Vert\Vert\Vert\Vert\Vert\Vert\Vert\Vert\Vert\Vert\Vert\Vert\Vert\Vert\Vert\Vert\Vert\Vert\Vert\Vert\Vert\Vert|  \mathrm{M}\mathrm{M}\mathrm{A}-\mathrm{R}1_{-}1\mathrm{P}-112$




{\it 2}

{\it Egzamin maturalny z matematyki}

{\it Poziom rozszerzony}

Zadanie l. $(4pkt)$

Uzasadnij, $\dot{\mathrm{z}}\mathrm{e}$ dla $\mathrm{k}\mathrm{a}\dot{\mathrm{z}}$ dej liczby całkowitej $k$ liczba $k^{6}-2k^{4}+k^{2}$ jest podzielna przez 36.





{\it Egzamin maturalny z matematyki}

{\it Poziom rozszerzony}

{\it 11}

Odpowiedzí :
\begin{center}
\includegraphics[width=82.044mm,height=17.784mm]{./F1_M_PR_M2011_page10_images/image001.eps}
\end{center}
Wypelnia

egzaminator

Nr zadania

Maks. lÍczba kt

7.

4

Uzyskana lÍczba pkt





{\it 12}

{\it Egzamin maturalny z matematyki}

{\it Poziom rozszerzony}

Zadanie 8. (4pkt)

Wśród wszystkich graniastosłupów prawidłowych sześciokątnych, w których suma długości

wszystkich krawędzi jest równa 24, jest taki, który ma największe po1e powierzchni bocznej.

Oblicz długość krawędzi podstawy tego graniastosłupa.





{\it Egzamin maturalny z matematyki}

{\it Poziom rozszerzony}

{\it 13}

Odpowiedzí :
\begin{center}
\includegraphics[width=82.044mm,height=17.784mm]{./F1_M_PR_M2011_page12_images/image001.eps}
\end{center}
Wypelnia

egzaminator

Nr zadania

Maks. lÍczba kt

8.

4

Uzyskana lÍczba pkt





{\it 14}

{\it Egzamin maturalny z matematyki}

{\it Poziom rozszerzony}

Zadanie 9. (4pkt)

Oblicz, ile jest liczb ośmiocyfrowych, w zapisie których nie występuje zero, natomiast

występują dwie dwójki i występują trzy trójki.

Odpowiedzí:





{\it Egzamin maturalny z matematyki}

{\it Poziom rozszerzony}

{\it 15}

Zadanie 10. $(3pkt)$

Dany jest czworokąt wypukły ABCD niebędący równoległobokiem. Punkty $M, N$ są

odpowiednio środkami boków AB $\mathrm{i}$ CD. Punkty $P, Q$ są odpowiednio środkami przekątnych

$AC\mathrm{i}BD$. Uzasadnij, $\dot{\mathrm{z}}\mathrm{e}MQ||PN.$
\begin{center}
\includegraphics[width=95.964mm,height=17.832mm]{./F1_M_PR_M2011_page14_images/image001.eps}
\end{center}
Wypelnia

egzaminator

Nr zadania

Maks. liczba kt

4

10.

3

Uzyskana liczba pkt





{\it 16}

{\it Egzamin maturalny z matematyki}

{\it Poziom rozszerzony}

Zadanie ll. $(6pkt)$

Dany jest ostrosłup prawidłowy czworokątny ABCDS o podstawie ABCD. $\mathrm{W}$ trójkącie

równoramiennym $ASC$ stosunek długości podstawy do długości ramienia jest równy

$|AC|:|AS|=6:5$. Oblicz sinus kąta nachylenia ściany bocznej do płaszczyzny podstawy.





{\it Egzamin maturalny z matematyki}

{\it Poziom rozszerzony}

{\it 1}7

Odpowiedzí :
\begin{center}
\includegraphics[width=82.044mm,height=17.784mm]{./F1_M_PR_M2011_page16_images/image001.eps}
\end{center}
Wypelnia

egzaminator

Nr zadania

Maks. lÍczba kt

11.

Uzyskana lÍczba pkt





{\it 18}

{\it Egzamin maturalny z matematyki}

{\it Poziom rozszerzony}

Zadanie 12. $(3pkt)$

$A, B$ są zdarzeniami losowymi zawartymi w $\Omega$. Wykaz, $\dot{\mathrm{z}}$ ejezeli $P(A)=0,9 \mathrm{i}P(B)=0,7,$

to $P(A\cap B')\leq 0,3$ ($B'$ oznacza zdarzenie przeciwne do zdarzenia $B$).

Odpowiedzí:
\begin{center}
\includegraphics[width=81.996mm,height=17.784mm]{./F1_M_PR_M2011_page17_images/image001.eps}
\end{center}
Nr zadania

Wypelnia Maks. liczba kt

egzaminator

Uzyskana liczba pkt

12.

3





{\it Egzamin maturalny z matematyki}

{\it Poziom rozszerzony}

{\it 19}

BRUDNOPIS










{\it Egzamin maturalny z matematyki}

{\it Poziom rozszerzony}

{\it 3}

Zadanie 2. $(4pkt)$

Uzasadnij, $\dot{\mathrm{z}}$ ejezeli $a\neq b, a\neq c, b\neq c\mathrm{i}a+b=2c$, to $\displaystyle \frac{a}{a-c}+\frac{b}{b-c}=2.$
\begin{center}
\includegraphics[width=95.964mm,height=17.832mm]{./F1_M_PR_M2011_page2_images/image001.eps}
\end{center}
Wypelnia

egzaminator

VIaks. liczba kt

1.

4

2.

4

Uzyskana liczba pkt





$\blacksquare$

$\blacksquare$

$\blacksquare$

$\Vert\Vert\Vert\Vert\Vert\Vert\Vert\Vert\Vert\Vert\Vert\Vert\Vert\Vert\Vert\Vert\Vert\Vert\Vert\Vert\Vert\Vert\Vert\Vert|$
\begin{center}
\includegraphics[width=79.452mm,height=15.804mm]{./F1_M_PR_M2011_page20_images/image001.eps}
\end{center}
PESEL

$\mathrm{M}\mathrm{M}\mathrm{A}-\mathrm{R}1_{-}1$ P-112

WYPELNIA ZDAJACY

Miejsce na naKlej$\kappa$e

z rr PESE-

WYPELNIA EGZAMINATOR
\begin{center}
\begin{tabular}{|l|l|l|l|l|l|l|l|}
	\\
&	\multicolumn{1}{|l|}{$0$}&	\multicolumn{1}{|l|}{ $1$}&	\multicolumn{1}{|l|}{ $2$}&	\multicolumn{1}{|l|}{ $3$}&	\multicolumn{1}{|l|}{ $4$}&	\multicolumn{1}{|l|}{ $5$}&	\multicolumn{1}{|l|}{ $6$}	\\
\cline{2-8}
\multicolumn{1}{|l|}{ $1$}&	\multicolumn{1}{|l|}{ $\square $}&	\multicolumn{1}{|l|}{ $\square $}&	\multicolumn{1}{|l|}{ $\square $}&	\multicolumn{1}{|l|}{ $\square $}&	\multicolumn{1}{|l|}{ $\square $}&	\multicolumn{1}{|l|}{}&	\multicolumn{1}{|l|}{}	\\
\hline
\multicolumn{1}{|l|}{ $2$}&	\multicolumn{1}{|l|}{ $\square $}&	\multicolumn{1}{|l|}{ $\square $}&	\multicolumn{1}{|l|}{ $\square $}&	\multicolumn{1}{|l|}{ $\square $}&	\multicolumn{1}{|l|}{ $\square $}&	\multicolumn{1}{|l|}{}&	\multicolumn{1}{|l|}{}	\\
\hline
\multicolumn{1}{|l|}{ $3$}&	\multicolumn{1}{|l|}{ $\square $}&	\multicolumn{1}{|l|}{ $\square $}&	\multicolumn{1}{|l|}{ $\square $}&	\multicolumn{1}{|l|}{ $\square $}&	\multicolumn{1}{|l|}{ $\square $}&	\multicolumn{1}{|l|}{ $\square $}&	\multicolumn{1}{|l|}{ $\square $}	\\
\hline
\multicolumn{1}{|l|}{ $4$}&	\multicolumn{1}{|l|}{ $\square $}&	\multicolumn{1}{|l|}{ $\square $}&	\multicolumn{1}{|l|}{ $\square $}&	\multicolumn{1}{|l|}{ $\square $}&	\multicolumn{1}{|l|}{ $\square $}&	\multicolumn{1}{|l|}{}&	\multicolumn{1}{|l|}{}	\\
\hline
\multicolumn{1}{|l|}{ $5$}&	\multicolumn{1}{|l|}{ $\square $}&	\multicolumn{1}{|l|}{ $\square $}&	\multicolumn{1}{|l|}{ $\square $}&	\multicolumn{1}{|l|}{ $\square $}&	\multicolumn{1}{|l|}{ $\square $}&	\multicolumn{1}{|l|}{}&	\multicolumn{1}{|l|}{}	\\
\hline
\multicolumn{1}{|l|}{ $6$}&	\multicolumn{1}{|l|}{ $\square $}&	\multicolumn{1}{|l|}{ $\square $}&	\multicolumn{1}{|l|}{ $\square $}&	\multicolumn{1}{|l|}{ $\square $}&	\multicolumn{1}{|l|}{ $\square $}&	\multicolumn{1}{|l|}{}&	\multicolumn{1}{|l|}{}	\\
\hline
\multicolumn{1}{|l|}{ $7$}&	\multicolumn{1}{|l|}{ $\square $}&	\multicolumn{1}{|l|}{ $\square $}&	\multicolumn{1}{|l|}{ $\square $}&	\multicolumn{1}{|l|}{ $\square $}&	\multicolumn{1}{|l|}{ $\square $}&	\multicolumn{1}{|l|}{}&	\multicolumn{1}{|l|}{}	\\
\hline
\multicolumn{1}{|l|}{ $8$}&	\multicolumn{1}{|l|}{ $\square $}&	\multicolumn{1}{|l|}{ $\square $}&	\multicolumn{1}{|l|}{ $\square $}&	\multicolumn{1}{|l|}{ $\square $}&	\multicolumn{1}{|l|}{ $\square $}&	\multicolumn{1}{|l|}{}&	\multicolumn{1}{|l|}{}	\\
\hline
\multicolumn{1}{|l|}{ $9$}&	\multicolumn{1}{|l|}{ $\square $}&	\multicolumn{1}{|l|}{ $\square $}&	\multicolumn{1}{|l|}{ $\square $}&	\multicolumn{1}{|l|}{ $\square $}&	\multicolumn{1}{|l|}{ $\square $}&	\multicolumn{1}{|l|}{}&	\multicolumn{1}{|l|}{}	\\
\hline
\multicolumn{1}{|l|}{ $10$}&	\multicolumn{1}{|l|}{ $\square $}&	\multicolumn{1}{|l|}{ $\square $}&	\multicolumn{1}{|l|}{ $\square $}&	\multicolumn{1}{|l|}{ $\square $}&	\multicolumn{1}{|l|}{}&	\multicolumn{1}{|l|}{}&	\multicolumn{1}{|l|}{}	\\
\hline
\multicolumn{1}{|l|}{ $11$}&	\multicolumn{1}{|l|}{ $\square $}&	\multicolumn{1}{|l|}{ $\square $}&	\multicolumn{1}{|l|}{ $\square $}&	\multicolumn{1}{|l|}{ $\square $}&	\multicolumn{1}{|l|}{ $\square $}&	\multicolumn{1}{|l|}{ $\square $}&	\multicolumn{1}{|l|}{ $\square $}	\\
\hline
\multicolumn{1}{|l|}{ $12$}&	\multicolumn{1}{|l|}{ $\square $}&	\multicolumn{1}{|l|}{ $\square $}&	\multicolumn{1}{|l|}{ $\square $}&	\multicolumn{1}{|l|}{ $\square $}&	\multicolumn{1}{|l|}{}&	\multicolumn{1}{|l|}{}&	\multicolumn{1}{|l|}{}	\\
\hline
\end{tabular}

\end{center}
SUMA

PUNKTÓW
\begin{center}
\includegraphics[width=14.532mm,height=9.756mm]{./F1_M_PR_M2011_page20_images/image002.eps}
\end{center}
D $\square  \square  \square  \square  \square  \square  \square  \square  \square  \square $

0 1 2 3 4 5 6 7 8 9

J $\square  \square  \square  \square  \square  \square  \square  \square  \square  \square $

0 1 2 3 4 5 6 7 8 9

$\blacksquare$

$\blacksquare$




\begin{center}
\includegraphics[width=73.152mm,height=11.028mm]{./F1_M_PR_M2011_page21_images/image001.eps}
\end{center}
KOD EGZAMINATORA

Czytelny podpis egzaminatora
\begin{center}
\includegraphics[width=21.840mm,height=9.852mm]{./F1_M_PR_M2011_page21_images/image002.eps}
\end{center}
KOD ZDAJACEGO





{\it 4}

{\it Egzamin maturalny z matematyki}

{\it Poziom rozszerzony}

Zadanie 3. $(6pkt)$

Wyznacz wszystkie wartości parametru $m$, dla których

$x^{2}-4mx-m^{3}+6m^{2}+m-2=0$ ma dwa rózne pierwiastki rzeczywiste $x_{1}, x_{2}$

$(x_{1}-x_{2})^{2}<8(m+1).$

równanie

takie, $\dot{\mathrm{z}}\mathrm{e}$





{\it Egzamin maturalny z matematyki}

{\it Poziom rozszerzony}

{\it 5}

Odpowiedzí :
\begin{center}
\includegraphics[width=82.044mm,height=17.784mm]{./F1_M_PR_M2011_page4_images/image001.eps}
\end{center}
Wypelnia

egzaminator

Nr zadania

Maks. lÍczba kt

3.

Uzyskana lÍczba pkt





{\it 6}

{\it Egzamin maturalny z matematyki}

{\it Poziom rozszerzony}

Zadanie 4. $(4pkt)$

Rozwiąz równanie 2 $\sin^{2}x-2\sin^{2}x\cos x=1-\cos x$ w przedziale $\langle 0,2\pi\rangle.$

Odpowiedzí:





{\it Egzamin maturalny z matematyki}

{\it Poziom rozszerzony}

7

Zadanie 5. $(4pkt)$

$\mathrm{O}$ ciągu $(x_{n})$ dla $n\geq 1$ wiadomo, $\dot{\mathrm{z}}\mathrm{e}$:

a) ciąg $(a_{n})$ określony wzorem $a_{n}=3^{x_{n}}$ dla $n\geq 1$ jest geometryczny o ilorazie $q=27.$

b) $x_{1}+x_{2}+\ldots+x_{10}=145.$

Oblicz $x_{1}.$
\begin{center}
\includegraphics[width=95.964mm,height=17.832mm]{./F1_M_PR_M2011_page6_images/image001.eps}
\end{center}
Wypelnia

egzaminator

Nr zadania

Maks. liczba kt

4.

4

5.

4

Uzyskana liczba pkt





{\it 8}

{\it Egzamin maturalny z matematyki}

{\it Poziom rozszerzony}

Zadanie 6. $(4pkt)$

Podstawa $AB$ trójkąta równoramiennego $ABC$ ma długość 8 oraz $|\neq BAC|=30^{\mathrm{o}}$

długość środkowej $AD$ tego trójkąta.

Oblicz





{\it Egzamin maturalny z matematyki}

{\it Poziom rozszerzony}

{\it 9}

Odpowiedzí :
\begin{center}
\includegraphics[width=82.044mm,height=17.784mm]{./F1_M_PR_M2011_page8_images/image001.eps}
\end{center}
Wypelnia

egzaminator

Nr zadania

Maks. lÍczba kt

4

Uzyskana lÍczba pkt





$ 1\theta$

{\it Egzamin maturalny z matematyki}

{\it Poziom rozszerzony}

Zadanie 7. $(4pkt)$

Oblicz miarę kąta między stycznymi do okręgu $x^{2}+y^{2}+2x-2y-3=0$ poprowadzonymi

przez punkt $A=(2,$ 0$).$



\end{document}