\documentclass[a4paper,12pt]{article}
\usepackage{latexsym}
\usepackage{amsmath}
\usepackage{amssymb}
\usepackage{graphicx}
\usepackage{wrapfig}
\pagestyle{plain}
\usepackage{fancybox}
\usepackage{bm}

\begin{document}

{\it Egzamin maturalny z matematyki}

{\it Poziom rozszerzony}

{\it 15}

Zadanie 10. $(3pkt)$

Dany jest czworokąt wypukły ABCD niebędący równoległobokiem. Punkty $M, N$ są

odpowiednio środkami boków AB $\mathrm{i}$ CD. Punkty $P, Q$ są odpowiednio środkami przekątnych

$AC\mathrm{i}BD$. Uzasadnij, $\dot{\mathrm{z}}\mathrm{e}MQ||PN.$
\begin{center}
\includegraphics[width=95.964mm,height=17.832mm]{./F1_M_PR_M2011_page14_images/image001.eps}
\end{center}
Wypelnia

egzaminator

Nr zadania

Maks. liczba kt

4

10.

3

Uzyskana liczba pkt
\end{document}
