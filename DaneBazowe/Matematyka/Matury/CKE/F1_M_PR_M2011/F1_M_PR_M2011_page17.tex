\documentclass[a4paper,12pt]{article}
\usepackage{latexsym}
\usepackage{amsmath}
\usepackage{amssymb}
\usepackage{graphicx}
\usepackage{wrapfig}
\pagestyle{plain}
\usepackage{fancybox}
\usepackage{bm}

\begin{document}

{\it 18}

{\it Egzamin maturalny z matematyki}

{\it Poziom rozszerzony}

Zadanie 12. $(3pkt)$

$A, B$ są zdarzeniami losowymi zawartymi w $\Omega$. Wykaz, $\dot{\mathrm{z}}$ ejezeli $P(A)=0,9 \mathrm{i}P(B)=0,7,$

to $P(A\cap B')\leq 0,3$ ($B'$ oznacza zdarzenie przeciwne do zdarzenia $B$).

Odpowiedzí:
\begin{center}
\includegraphics[width=81.996mm,height=17.784mm]{./F1_M_PR_M2011_page17_images/image001.eps}
\end{center}
Nr zadania

Wypelnia Maks. liczba kt

egzaminator

Uzyskana liczba pkt

12.

3
\end{document}
