\documentclass[a4paper,12pt]{article}
\usepackage{latexsym}
\usepackage{amsmath}
\usepackage{amssymb}
\usepackage{graphicx}
\usepackage{wrapfig}
\pagestyle{plain}
\usepackage{fancybox}
\usepackage{bm}

\begin{document}

{\it Egzamin maturalny z matematyki}

{\it Poziom rozszerzony}

7

Zadanie 5. $(4pkt)$

$\mathrm{O}$ ciągu $(x_{n})$ dla $n\geq 1$ wiadomo, $\dot{\mathrm{z}}\mathrm{e}$:

a) ciąg $(a_{n})$ określony wzorem $a_{n}=3^{x_{n}}$ dla $n\geq 1$ jest geometryczny o ilorazie $q=27.$

b) $x_{1}+x_{2}+\ldots+x_{10}=145.$

Oblicz $x_{1}.$
\begin{center}
\includegraphics[width=95.964mm,height=17.832mm]{./F1_M_PR_M2011_page6_images/image001.eps}
\end{center}
Wypelnia

egzaminator

Nr zadania

Maks. liczba kt

4.

4

5.

4

Uzyskana liczba pkt
\end{document}
