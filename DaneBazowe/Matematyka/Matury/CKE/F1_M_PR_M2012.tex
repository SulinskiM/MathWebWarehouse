\documentclass[a4paper,12pt]{article}
\usepackage{latexsym}
\usepackage{amsmath}
\usepackage{amssymb}
\usepackage{graphicx}
\usepackage{wrapfig}
\pagestyle{plain}
\usepackage{fancybox}
\usepackage{bm}

\begin{document}

$1-$

$-1\cup 1$

$-\mapsto 1$

$\mathrm{r}--$

Centralna Komisja Egzaminacyjna

Arkusz zawiera informacje prawnie chronione do momentu rozpoczęcia egzaminu.

WPISUJE ZDAJACY

KOD PESEL

{\it Miejsce}

{\it na naklejkę}

{\it z kodem}
\begin{center}
\includegraphics[width=21.432mm,height=9.804mm]{./F1_M_PR_M2012_page0_images/image001.eps}

\includegraphics[width=82.092mm,height=9.804mm]{./F1_M_PR_M2012_page0_images/image002.eps}
\end{center}
\fbox{} dysleksja
\begin{center}
\includegraphics[width=204.060mm,height=216.048mm]{./F1_M_PR_M2012_page0_images/image003.eps}
\end{center}
EGZAMIN MATU LNY

Z MATEMATYKI

MAJ 2012

POZIOM ROZSZERZONY

1.

3.

Sprawd $\acute{\mathrm{z}}$, czy arkusz egzaminacyjny zawiera 19 stron

(zadania $1-11$). Ewentualny brak zgłoś

przewodniczącemu zespołu nadzorującego egzamin.

Rozwiązania zadań i odpowiedzi wpisuj w miejscu na to

przeznaczonym.

Pamiętaj, $\dot{\mathrm{z}}\mathrm{e}$ pominięcie argumentacji lub istotnych

obliczeń w rozwiązaniu zadania otwa ego $\mathrm{m}\mathrm{o}\dot{\mathrm{z}}\mathrm{e}$

spowodować, $\dot{\mathrm{z}}\mathrm{e}$ za to rozwiązanie nie będziesz mógł

dostać pełnej liczby punktów.

Pisz czytelnie i uzywaj tvlko długopisu lub -Dióra

z czarnym tuszem lub atramentem.

Nie uzywaj korektora, a błędne zapisy wyra $\acute{\mathrm{z}}\mathrm{n}\mathrm{i}\mathrm{e}$ prze eśl.

Pamiętaj, $\dot{\mathrm{z}}\mathrm{e}$ zapisy w brudnopisie nie będą oceniane.

$\mathrm{M}\mathrm{o}\dot{\mathrm{z}}$ esz korzystać z zestawu wzorów matematycznych,

cyrkla i linijki oraz kalkulatora.

Na tej stronie oraz na karcie odpowiedzi wpisz swój

numer PESEL i przyklej naklejkę z kodem.

Nie wpisuj $\dot{\mathrm{z}}$ adnych znaków w części przeznaczonej

dla egzaminatora.

Czas pracy:

180 minut

2.

4.

5.

6.

7.

8.

9.

Liczba punktów

do uzyskania: 50

$\Vert\Vert\Vert\Vert\Vert\Vert\Vert\Vert\Vert\Vert\Vert\Vert\Vert\Vert\Vert\Vert\Vert\Vert\Vert\Vert\Vert\Vert\Vert\Vert|  \mathrm{M}\mathrm{M}\mathrm{A}-\mathrm{R}1_{-}1\mathrm{P}-122$




{\it 2}

{\it Egzamin maturalny z matematyki}

{\it Poziom rozszerzony}

Zadanie l. $(4pkt)$

Wyznacz cztery kolejne liczby całkowite takie, $\dot{\mathrm{z}}\mathrm{e}$ największa z nich jest równa sumie

kwadratów trzech pozostałych liczb.





{\it Egzamin maturalny z matematyki}

{\it Poziom rozszerzony}

{\it 11}

Odpowied $\acute{\mathrm{z}}$:
\begin{center}
\includegraphics[width=82.044mm,height=17.832mm]{./F1_M_PR_M2012_page10_images/image001.eps}
\end{center}
Wypelnia

egzaminator

Nr zadania

Maks. liczba kt

Uzyskana liczba pkt





{\it 12}

{\it Egzamin maturalny z matematyki}

{\it Poziom rozszerzony}

Zadanie 7. $(3pkt)$

Udowodnij, $\dot{\mathrm{z}}$ ejezeli $a+b\geq 0$, to prawdziwajest nierówność $a^{3}+b^{3}\geq a^{2}b+ab^{2}$





{\it Egzamin maturalny z matematyki}

{\it Poziom rozszerzony}

{\it 13}

Zadanie 8. $(4pkt)$

Oblicz, ile jest liczb naturalnych ośmiocyfrowych takich, $\dot{\mathrm{z}}\mathrm{e}$ iloczyn cyfr w ich zapisie

dziesiętnymjest równy 12.

Odpowied $\acute{\mathrm{z}}$:
\begin{center}
\includegraphics[width=95.964mm,height=17.832mm]{./F1_M_PR_M2012_page12_images/image001.eps}
\end{center}
Wypelnia

egzaminator

Nr zadania

Maks. liczba kt

7.

3

8.

4

Uzyskana liczba pkt





{\it 14}

{\it Egzamin maturalny z matematyki}

{\it Poziom rozszerzony}

Zadanie 9. $(5pkt)$

Dany jest prostokąt ABCD, w którym $|AB|=a, |BC|=b \mathrm{i}a>b$. Odcinek $AE$ jest wysokością

trójkąta $DAB$ opuszczoną najego bok $BD.$ Wyra $\acute{\mathrm{z}}$ pole trójkąta $AED$ za pomocą a $\mathrm{i}b.$





{\it Egzamin maturalny z matematyki}

{\it Poziom rozszerzony}

{\it 15}

Odpowiedzí :
\begin{center}
\includegraphics[width=82.044mm,height=17.784mm]{./F1_M_PR_M2012_page14_images/image001.eps}
\end{center}
Wypelnia

egzamÍnator

Nr zadania

Maks. liczba kt

5

Uzyskana liczba pkt





{\it 16}

{\it Egzamin maturalny z matematyki}

{\it Poziom rozszerzony}

Zadanie 10. $(5pkt)$

Podstawą ostrosłupa ABCS jest trójkąt równoramienny $ABC$. Krawędzí AS jest wysokością

ostrosiupa oraz $|AS|=8\sqrt{210}, |BS|=118, |CS|=131$. Oblicz objętość tego ostrosłupa.





{\it Egzamin maturalny z matematyki}

{\it Poziom rozszerzony}

17

Odpowied $\acute{\mathrm{z}}$:
\begin{center}
\includegraphics[width=82.044mm,height=17.832mm]{./F1_M_PR_M2012_page16_images/image001.eps}
\end{center}
Wypelnia

egzaminator

Nr zadania

Maks. liczba kt

10.

5

Uzyskana liczba pkt





{\it 18}

{\it Egzamin maturalny z matematyki}

{\it Poziom rozszerzony}

Zadanie ll. $(3pkt)$

Zdarzenia losowe $A, B$ są zawarte w $\Omega$ oraz $P(A\cap B')=0,7 (A'$ oznacza zdarzenie

przeciwne do zdarzenia $A, B'$ oznacza zdarzenie przeciwne do zdarzenia $B$).

Wykaz, $\dot{\mathrm{z}}\mathrm{e}P(A'\cap B)\leq 0,3.$
\begin{center}
\includegraphics[width=81.996mm,height=17.784mm]{./F1_M_PR_M2012_page17_images/image001.eps}
\end{center}
Nr zadania

Wypelnia Maks. liczba kt

egzaminator

Uzyskana liczba pkt

11.

3





{\it Egzamin maturalny z matematyki}

{\it Poziom rozszerzony}

{\it 19}

BRUDNOPIS





{\it Egzamin maturalny z matematyki}

{\it Poziom rozszerzony}

{\it 3}

Odpowiedzí:
\begin{center}
\includegraphics[width=82.044mm,height=17.832mm]{./F1_M_PR_M2012_page2_images/image001.eps}
\end{center}
Wypelnia

egzaminator

Nr zadania

Maks. liczba kt

1.

4

Uzyskana liczba pkt





{\it 4}

{\it Egzamin maturalny z matematyki}

{\it Poziom rozszerzony}

Zadanie 2. $(4pkt)$

Rozwiąz nierówność $x^{4}+x^{2}\geq 2x.$

Odpowiedzí:





{\it Egzamin maturalny z matematyki}

{\it Poziom rozszerzony}

{\it 5}

Zadanie 3. $(4pkt)$

Rozwiąz równanie $\cos 2x+2=3\cos x.$

Odpowiedzí :
\begin{center}
\includegraphics[width=95.964mm,height=17.784mm]{./F1_M_PR_M2012_page4_images/image001.eps}
\end{center}
Wypelnia

egzamÍnator

Nr zadania

Maks. liczba kt

2.

4

3.

4

Uzyskana liczba pkt





{\it 6}

{\it Egzamin maturalny z matematyki}

{\it Poziom rozszerzony}

Zadanie 4. $(6pkt)$

Oblicz wszystkie wartości parametru $m$, dla których równanie $x^{2}-(m+2)x+m+4=0$

ma dwa rózne pierwiastki rzeczywiste $x_{1}, x_{2}$ takie, $\dot{\mathrm{z}}\mathrm{e}x_{1}^{4}+x_{2}^{4}=4m^{3}+6m^{2}-32m+12.$





{\it Egzamin maturalny z matematyki}

{\it Poziom rozszerzony}

7

Odpowied $\acute{\mathrm{z}}$:
\begin{center}
\includegraphics[width=82.044mm,height=17.832mm]{./F1_M_PR_M2012_page6_images/image001.eps}
\end{center}
Wypelnia

egzaminator

Nr zadania

Maks. liczba kt

4.

Uzyskana liczba pkt





{\it 8}

{\it Egzamin maturalny z matematyki}

{\it Poziom rozszerzony}

Zadanie 5. $(6pkt)$

Trzy liczby tworzą ciąg geometryczny. $\mathrm{J}\mathrm{e}\dot{\mathrm{z}}$ eli do drugiej liczby dodamy 8, to ciąg ten zmieni

się w arytmetyczny. $\mathrm{J}\mathrm{e}\dot{\mathrm{z}}$ eli zaś do ostatniej liczby nowego ciągu arytmetycznego dodamy 64,

to tak otrzymany ciąg będzie znów geometryczny. Znajdzí te liczby. Uwzględnij wszystkie

$\mathrm{m}\mathrm{o}\dot{\mathrm{z}}$ liwości.





{\it Egzamin maturalny z matematyki}

{\it Poziom rozszerzony}

{\it 9}

Odpowied $\acute{\mathrm{z}}$:
\begin{center}
\includegraphics[width=82.044mm,height=17.832mm]{./F1_M_PR_M2012_page8_images/image001.eps}
\end{center}
Wypelnia

egzaminator

Nr zadania

Maks. liczba kt

5.

Uzyskana liczba pkt





$ 1\theta$

{\it Egzamin maturalny z matematyki}

{\it Poziom rozszerzony}

Zadanie 6. $(6pkt)$

$\mathrm{W}$ układzie współrzędnych rozwazmy wszystkie punkty $P$ postaci: $P=(\displaystyle \frac{1}{2}m+\frac{5}{2},m),$

gdzie $ m\in\langle-1,7\rangle$. Oblicz najmniejszą i największąwartość $|PQ|^{2}$, gdzie $Q=(\displaystyle \frac{55}{2},0).$



\end{document}