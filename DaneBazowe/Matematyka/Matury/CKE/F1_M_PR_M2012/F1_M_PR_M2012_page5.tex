\documentclass[a4paper,12pt]{article}
\usepackage{latexsym}
\usepackage{amsmath}
\usepackage{amssymb}
\usepackage{graphicx}
\usepackage{wrapfig}
\pagestyle{plain}
\usepackage{fancybox}
\usepackage{bm}

\begin{document}

{\it 6}

{\it Egzamin maturalny z matematyki}

{\it Poziom rozszerzony}

Zadanie 4. $(6pkt)$

Oblicz wszystkie wartości parametru $m$, dla których równanie $x^{2}-(m+2)x+m+4=0$

ma dwa rózne pierwiastki rzeczywiste $x_{1}, x_{2}$ takie, $\dot{\mathrm{z}}\mathrm{e}x_{1}^{4}+x_{2}^{4}=4m^{3}+6m^{2}-32m+12.$
\end{document}
