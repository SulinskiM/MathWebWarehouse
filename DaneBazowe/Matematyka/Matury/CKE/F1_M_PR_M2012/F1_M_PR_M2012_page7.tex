\documentclass[a4paper,12pt]{article}
\usepackage{latexsym}
\usepackage{amsmath}
\usepackage{amssymb}
\usepackage{graphicx}
\usepackage{wrapfig}
\pagestyle{plain}
\usepackage{fancybox}
\usepackage{bm}

\begin{document}

{\it 8}

{\it Egzamin maturalny z matematyki}

{\it Poziom rozszerzony}

Zadanie 5. $(6pkt)$

Trzy liczby tworzą ciąg geometryczny. $\mathrm{J}\mathrm{e}\dot{\mathrm{z}}$ eli do drugiej liczby dodamy 8, to ciąg ten zmieni

się w arytmetyczny. $\mathrm{J}\mathrm{e}\dot{\mathrm{z}}$ eli zaś do ostatniej liczby nowego ciągu arytmetycznego dodamy 64,

to tak otrzymany ciąg będzie znów geometryczny. Znajdzí te liczby. Uwzględnij wszystkie

$\mathrm{m}\mathrm{o}\dot{\mathrm{z}}$ liwości.
\end{document}
