\documentclass[a4paper,12pt]{article}
\usepackage{latexsym}
\usepackage{amsmath}
\usepackage{amssymb}
\usepackage{graphicx}
\usepackage{wrapfig}
\pagestyle{plain}
\usepackage{fancybox}
\usepackage{bm}

\begin{document}

Centralna Komisja Egzaminacyjna

Arkusz zawiera informacje prawnie chronione do momentu rozpoczęcia egzaminu.

WPISUJE ZDAJACY

KOD PESEL

{\it Miejsce}

{\it na naklejkę}

{\it z kodem}
\begin{center}
\includegraphics[width=21.432mm,height=9.804mm]{./F1_M_PR_M2013_page0_images/image001.eps}

\includegraphics[width=82.092mm,height=9.804mm]{./F1_M_PR_M2013_page0_images/image002.eps}
\end{center}
\fbox{} dysleksja
\begin{center}
\includegraphics[width=204.060mm,height=216.048mm]{./F1_M_PR_M2013_page0_images/image003.eps}
\end{center}
EGZAMIN MATU LNY

Z MATEMATYKI

MAJ 2013

POZIOM ROZSZERZONY

1.

3.

Sprawdzí, czy arkusz egzaminacyjny zawiera 20 stron

(zadania $1-12$). Ewentualny brak zgłoś

przewodniczącemu zespołu nadzorującego egzamin.

Rozwiązania zadań i odpowiedzi wpisuj w miejscu na to

przeznaczonym.

Pamiętaj, $\dot{\mathrm{z}}\mathrm{e}$ pominięcie argumentacji lub istotnych

obliczeń w rozwiązaniu zadania otwa ego $\mathrm{m}\mathrm{o}\dot{\mathrm{z}}\mathrm{e}$

spowodować, $\dot{\mathrm{z}}\mathrm{e}$ za to rozwiązanie nie będziesz mógł

dostać pełnej liczby punktów.

Pisz czytelnie i uzywaj tvlko długopisu lub -Dióra

z czatnym tuszem lub atramentem.

Nie $\mathrm{u}\dot{\mathrm{z}}$ aj korektora, a błędne zapisy wyrazínie przekreśl.

Pamiętaj, $\dot{\mathrm{z}}\mathrm{e}$ zapisy w brudnopisie nie będą oceniane.

$\mathrm{M}\mathrm{o}\dot{\mathrm{z}}$ esz korzystać z zestawu wzorów matematycznych,

cyrkla i linijki oraz kalkulatora.

Na tej stronie oraz na karcie odpowiedzi wpisz swój

numer PESEL i przyklej naklejkę z kodem.

Nie wpisuj $\dot{\mathrm{z}}$ adnych znaków w części przeznaczonej

dla egzaminatora.

Czas pracy:

180 minut

2.

4.

5.

6.

7.

8.

9.

Liczba punktów

do uzyskania: 50

$\Vert\Vert\Vert\Vert\Vert\Vert\Vert\Vert\Vert\Vert\Vert\Vert\Vert\Vert\Vert\Vert\Vert\Vert\Vert\Vert\Vert\Vert\Vert\Vert|  \mathrm{M}\mathrm{M}\mathrm{A}-\mathrm{R}1_{-}1\mathrm{P}-132$




{\it 2}

{\it Egzamin maturalny z matematyki}

{\it Poziom rozszerzony}

Zadanie l. $(4pkt)$

Rozwiąz nierównoŚć $|2x-5|-|x+4|\leq 2-2x.$

Odpowiedzí:





{\it Egzamin maturalny z matematyki}

{\it Poziom rozszerzony}

{\it 11}

Odpowied $\acute{\mathrm{z}}$:
\begin{center}
\includegraphics[width=82.044mm,height=17.784mm]{./F1_M_PR_M2013_page10_images/image001.eps}
\end{center}
WypelnÍa

egzaminator

Nr zadania

Maks. liczba kt

7.

4

Uzyskana liczba pkt





{\it 12}

{\it Egzamin maturalny z matematyki}

{\it Poziom rozszerzony}

Zadanie 8. $(4pkt)$

Reszta z dzielenia wielomianu $W(x)=4x^{3}-5x^{2}-23x+m$ przez dwumian $x+1$ jest równa 20.

Oblicz wartość współczynnika $m$ oraz pierwiastki tego wielomianu.

Odpowiedzí:





{\it Egzamin maturalny z matematyki}

{\it Poziom rozszerzony}

{\it 13}

Zadanie 9. $(5pkt)$

Dany jest trójkąt $ABC$, w którym $|AC|=17 \mathrm{i} |BC|=10$. Na boku AB lezy punkt $D$ taki, $\dot{\mathrm{z}}\mathrm{e}$

$|AD|:|DB|=3:4$ oraz $|DC|=10$. Oblicz pole trójkąta $ABC.$

Odpowied $\acute{\mathrm{z}}$:
\begin{center}
\includegraphics[width=95.964mm,height=17.832mm]{./F1_M_PR_M2013_page12_images/image001.eps}
\end{center}
Wypelnia

egzaminator

Nr zadania

Maks. liczba kt

8.

4

5

Uzyskana liczba pkt





{\it 14}

{\it Egzamin maturalny z matematyki}

{\it Poziom rozszerzony}

Zadanie 10. (4pkt)

W ostrosłupie ABCS podstawa ABC jest trójkątem równobocznym o boku długości a.

Krawędzí AS jest prostopadła do płaszczyzny podstawy. Odległość wierzchołka A od ściany

BCSjest równa d. Wyznacz objętość tego ostrosłupa.





{\it Egzamin maturalny z matematyki}

{\it Poziom rozszerzony}

{\it 15}

Odpowied $\acute{\mathrm{z}}$:
\begin{center}
\includegraphics[width=82.044mm,height=17.784mm]{./F1_M_PR_M2013_page14_images/image001.eps}
\end{center}
WypelnÍa

egzaminator

Nr zadania

Maks. liczba kt

10.

4

Uzyskana liczba pkt





{\it 16}

{\it Egzamin maturalny z matematyki}

{\it Poziom rozszerzony}

Zadanie ll. $(4pkt)$

Rzucamy cztery razy symetryczną sześcienną kostką do gry. Oblicz prawdopodobieństwo

zdarzenia polegającego na tym, $\dot{\mathrm{z}}\mathrm{e}$ iloczyn liczb oczek otrzymanych we wszystkich czterech

rzutach będzie równy 60.





{\it Egzamin maturalny z matematyki}

{\it Poziom rozszerzony}

17

Odpowied $\acute{\mathrm{z}}$:
\begin{center}
\includegraphics[width=82.044mm,height=17.784mm]{./F1_M_PR_M2013_page16_images/image001.eps}
\end{center}
WypelnÍa

egzaminator

Nr zadania

Maks. liczba kt

11.

4

Uzyskana liczba pkt





{\it 18}

{\it Egzamin maturalny z matematyki}

{\it Poziom rozszerzony}

Zadanie 12. $(3pkt)$

Na rysunku przedstawiony jest fragment wykresu funkcji logarytmicznej $f$ określonej wzorem

$f(x)=\log_{2}(x-p).$
\begin{center}
\includegraphics[width=117.852mm,height=97.536mm]{./F1_M_PR_M2013_page17_images/image001.eps}
\end{center}
a) Podaj wartoŚć p.

b) Narysuj wykres funkcji określonej wzorem $y=|f(x)|.$

c) Podaj wszystkie wartości parametru $m$, dla których równanie

rozwiązania o przeciwnych znakach.

$|f(x)|=m$ ma dwa





{\it Egzamin maturalny z matematyki}

{\it Poziom rozszerzony}

{\it 19}

Odpowied $\acute{\mathrm{z}}$:
\begin{center}
\includegraphics[width=82.044mm,height=17.784mm]{./F1_M_PR_M2013_page18_images/image001.eps}
\end{center}
WypelnÍa

egzaminator

Nr zadania

Maks. liczba kt

12.

3

Uzyskana liczba pkt





$ 2\theta$

{\it Egzamin maturalny z matematyki}

{\it Poziom rozszerzony}

BRUDNOPIS





{\it Egzamin maturalny z matematyki}

{\it Poziom rozszerzony}

{\it 3}

Zadanie 2. $(4pkt)$

Trapez równoramienny ABCD o podstawach AB $\mathrm{i}$ CD jest opisany na okręgu o promieniu $r.$

Wykaz, $\dot{\mathrm{z}}\mathrm{e}4r^{2}=|AB|\cdot|CD|.$
\begin{center}
\includegraphics[width=95.964mm,height=17.784mm]{./F1_M_PR_M2013_page2_images/image001.eps}
\end{center}
Wypelnia

egzaminator

Nr zadania

Maks. liczba kt

1.

4

2.

4

Uzyskana liczba pkt





{\it 4}

{\it Egzamin maturalny z matematyki}

{\it Poziom rozszerzony}

Zadanie 3. (3pkt)

Oblicz, ile jest liczb naturalnych sześciocyfrowych, w zapisie których występuje dokładnie

trzy razy cyfra 0 i dokładnie raz występuje cyfra 5.





{\it Egzamin maturalny z matematyki}

{\it Poziom rozszerzony}

{\it 5}

Odpowiedzí:
\begin{center}
\includegraphics[width=82.044mm,height=17.832mm]{./F1_M_PR_M2013_page4_images/image001.eps}
\end{center}
Wypelnia

egzaminator

Nr zadania

Maks. liczba kt

3.

3

Uzyskana liczba pkt





{\it 6}

{\it Egzamin maturalny z matematyki}

{\it Poziom rozszerzony}

Zadanie 4. $(4pkt)$

Rozwiąz równanie $\cos 2x+\cos x+1=0$ dla $x\in\langle 0,2\pi\rangle.$

Odpowied $\acute{\mathrm{z}}$:





{\it Egzamin maturalny z matematyki}

{\it Poziom rozszerzony}

7

Zadanie 5. $(5pkt)$

Ciąg liczbowy $(a,b,c)$ jest arytmetyczny i $a+b+c=33$, natomiast ciąg $(a-1,b+5,c+19)$

jest geometryczny. Oblicz $a, b, c.$

Odpowiedzí:
\begin{center}
\includegraphics[width=95.964mm,height=17.832mm]{./F1_M_PR_M2013_page6_images/image001.eps}
\end{center}
Wypelnia

egzaminator

Nr zadania

Maks. liczba kt

4.

4

5.

5

Uzyskana liczba pkt





{\it 8}

{\it Egzamin maturalny z matematyki}

{\it Poziom rozszerzony}

Zadanie 6. $(6pkt)$

Wyznacz wszystkie wartości parametru $m$, dla których równanie $x^{2}+2(1-m)x+m^{2}-m=0$

ma dwa rózne rozwiązania rzeczywiste $x_{1}, x_{2}$ spełniające warunek $x_{1}\cdot x_{2}\leq 6m\leq x_{1}^{2}+x_{2}^{2}.$





{\it Egzamin maturalny z matematyki}

{\it Poziom rozszerzony}

{\it 9}

Odpowied $\acute{\mathrm{z}}$:
\begin{center}
\includegraphics[width=82.044mm,height=17.784mm]{./F1_M_PR_M2013_page8_images/image001.eps}
\end{center}
WypelnÍa

egzaminator

Nr zadania

Maks. liczba kt

Uzyskana liczba pkt





$ 1\theta$

{\it Egzamin maturalny z matematyki}

{\it Poziom rozszerzony}

Zadanie 7. $(4pkt)$

Prosta o równaniu $3x-4y-36=0$ przecina okrąg o Środku $S=(3,12)$ w punktach A $\mathrm{i}B.$

Długość odcinka $AB$ jest równa 40. Wyznacz równanie tego okręgu.



\end{document}