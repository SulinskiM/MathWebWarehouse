\documentclass[a4paper,12pt]{article}
\usepackage{latexsym}
\usepackage{amsmath}
\usepackage{amssymb}
\usepackage{graphicx}
\usepackage{wrapfig}
\pagestyle{plain}
\usepackage{fancybox}
\usepackage{bm}

\begin{document}

{\it 18}

{\it Egzamin maturalny z matematyki}

{\it Poziom rozszerzony}

Zadanie 12. $(3pkt)$

Na rysunku przedstawiony jest fragment wykresu funkcji logarytmicznej $f$ określonej wzorem

$f(x)=\log_{2}(x-p).$
\begin{center}
\includegraphics[width=117.852mm,height=97.536mm]{./F1_M_PR_M2013_page17_images/image001.eps}
\end{center}
a) Podaj wartoŚć p.

b) Narysuj wykres funkcji określonej wzorem $y=|f(x)|.$

c) Podaj wszystkie wartości parametru $m$, dla których równanie

rozwiązania o przeciwnych znakach.

$|f(x)|=m$ ma dwa
\end{document}
