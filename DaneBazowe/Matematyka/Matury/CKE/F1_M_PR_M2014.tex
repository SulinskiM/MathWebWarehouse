\documentclass[a4paper,12pt]{article}
\usepackage{latexsym}
\usepackage{amsmath}
\usepackage{amssymb}
\usepackage{graphicx}
\usepackage{wrapfig}
\pagestyle{plain}
\usepackage{fancybox}
\usepackage{bm}

\begin{document}

Arkusz zawiera informacje prawnie chronione do momentu rozpoczęcia egzaminu.

WPISUJE ZDAJACY

KOD PESEL

{\it Miejsce}

{\it na naklejkę}

{\it z kodem}
\begin{center}
\includegraphics[width=21.432mm,height=9.852mm]{./F1_M_PR_M2014_page0_images/image001.eps}

\includegraphics[width=82.092mm,height=9.852mm]{./F1_M_PR_M2014_page0_images/image002.eps}
\end{center}
\fbox{} dysleksja
\begin{center}
\includegraphics[width=204.060mm,height=217.272mm]{./F1_M_PR_M2014_page0_images/image003.eps}
\end{center}
EGZAMIN MATU

Z MATEMATY

LNY  MAJ 2014

POZIOM ROZSZERZONY

1.

2.

3.

4.

Sprawdzí, czy arkusz egzaminacyjny zawiera 19 stron

(zadania $1-11$). Ewentualny brak zgłoś

przewodniczącemu zespo nadzo jącego egzamin.

Rozwiązania zadań i odpowiedzi wpisuj w miejscu na to

przeznaczonym.

Pamiętaj, $\dot{\mathrm{z}}\mathrm{e}$ pominięcie argumentacji lub istotnych

obliczeń w rozwiązaniu zadania otwa ego $\mathrm{m}\mathrm{o}\dot{\mathrm{z}}\mathrm{e}$

spowodować, $\dot{\mathrm{z}}\mathrm{e}$ za to rozwiązanie nie otrzymasz pełnej

liczby punktów.

Pisz czytelnie i $\mathrm{u}\dot{\mathrm{z}}$ aj tvlko $\mathrm{d}$ gopisu lub -Dióra

z czatnym tuszem lub atramentem.

Nie uzywaj korektora, a błędne zapisy wyrazínie prze eśl.

Pamiętaj, $\dot{\mathrm{z}}\mathrm{e}$ zapisy w brudnopisie nie będą oceniane.

$\mathrm{M}\mathrm{o}\dot{\mathrm{z}}$ esz korzystać z zestawu wzorów matematycznych,

cyrkla i linijki oraz kalkulatora.

Na karcie odpowiedzi wpisz swój numer PESEL i przyklej

naklejkę z kodem.

Nie wpisuj $\dot{\mathrm{z}}$ adnych znaków w części przeznaczonej dla

egzaminatora.

Czas pracy:

180 minut

5.

6.

7.

8.

9.

Liczba punktów

do uzyskania: 50

$\Vert\Vert\Vert\Vert\Vert\Vert\Vert\Vert\Vert\Vert\Vert\Vert\Vert\Vert\Vert\Vert\Vert\Vert\Vert\Vert\Vert\Vert\Vert\Vert|  \mathrm{M}\mathrm{M}\mathrm{A}-\mathrm{R}1_{-}1\mathrm{P}-142$




{\it 2}

{\it Egzamin maturalny z matematyki}

{\it Poziom rozszerzony}

Zadanie l. $(4pkt)$

Dana jest ffinkcja $f$ określona wzorem $f(x)=\displaystyle \frac{|x+3|+|x-3|}{x}$ dla $\mathrm{k}\mathrm{a}\dot{\mathrm{z}}$ dej liczby rzeczywistej

$x\neq 0$. Wyznacz zbiór wartości tej funkcji.





{\it Egzamin maturalny z matematyki}

{\it Poziom rozszerzony}

{\it 11}

Odpowied $\acute{\mathrm{z}}$:
\begin{center}
\includegraphics[width=78.840mm,height=17.628mm]{./F1_M_PR_M2014_page10_images/image001.eps}
\end{center}
Wypelnia

egzaminator

Nr zadania

Maks. liczba kt

3

Uzyskana liczba pkt





{\it 12}

{\it Egzamin maturalny z matematyki}

{\it Poziom rozszerzony}

Zadanie 7. $(6pkt)$

Ciąg geometryczny $(a_{n})$ ma 100 wyrazów i są one 1iczbami dodatnimi. Suma wszystkich

wyrazów o numerach nieparzystych jest sto razy większa od sumy wszystkich wyrazów

o numerach parzystych oraz $\log a_{1}+\log a_{2}+\log a_{3}+\ldots+\log a_{100}=100$. Oblicz $a_{1}.$

Odpowiedzí :





{\it Egzamin maturalny z matematyki}

{\it Poziom rozszerzony}

{\it 13}

Zadanie 8. $(4pkt)$

Punkty $A, B, C, D, E, F$ są kolejnymi wierzchołkami sześciokąta foremnego, przy czym

$A=(0,2\sqrt{3}), B=(2,0)$, a $C$ lezy na osi $ox$. Wyznacz równanie stycznej do okręgu

opisanego na tym sześciokącie przechodzącej przez wierzchołek $E.$

Odpowiedzí :
\begin{center}
\includegraphics[width=90.276mm,height=17.580mm]{./F1_M_PR_M2014_page12_images/image001.eps}
\end{center}
Wypelnia

egzaminator

Nr zadania

Maks. liczba kt

7.

8.

4

Uzyskana liczba pkt





{\it 14}

{\it Egzamin maturalny z matematyki}

{\it Poziom rozszerzony}

Zadanie 9. (6pkt)

Oblicz objętość ostrosłupa trójkątnego ABCS, którego siatkę przedstawiono na rysunku.





{\it Egzamin maturalny z matematyki}

{\it Poziom rozszerzony}

{\it 15}

Odpowied $\acute{\mathrm{z}}$:
\begin{center}
\includegraphics[width=78.840mm,height=17.628mm]{./F1_M_PR_M2014_page14_images/image001.eps}
\end{center}
Wypelnia

egzaminator

Nr zadania

Maks. liczba kt

Uzyskana liczba pkt





{\it 16}

{\it Egzamin maturalny z matematyki}

{\it Poziom rozszerzony}

Zadanie 10. $(5pkt)$

Wyznacz wszystkie całkowite wartości parametru $m$, dla których równanie

$(x^{3}+2x^{2}+2x+1)[x^{2}-(2m+1)x+m^{2}+m]=0$ ma trzy, parami rózne, pierwiastki

rzeczywiste, takie $\dot{\mathrm{z}}$ ejeden z nichjest średnią arytmetyczną dwóch pozostałych.





{\it Egzamin maturalny z matematyki}

{\it Poziom rozszerzony}

17

Odpowied $\acute{\mathrm{z}}$:
\begin{center}
\includegraphics[width=78.840mm,height=17.628mm]{./F1_M_PR_M2014_page16_images/image001.eps}
\end{center}
Wypelnia

egzaminator

Nr zadania

Maks. liczba kt

10.

5

Uzyskana liczba pkt





{\it 18}

{\it Egzamin maturalny z matematyki}

{\it Poziom rozszerzony}

Zadanie ll. $(4pkt)$

$\mathrm{Z}$ urny zawierającej 10 ku1 ponumerowanych ko1ejnymi 1iczbami od 1 do 101osujemy

jednocześnie trzy kule. Oblicz prawdopodobieństwo zdarzenia $A$ polegającego na tym, $\dot{\mathrm{z}}\mathrm{e}$

numerjednej z wylosowanych kuljest równy sumie numerów dwóch pozostałych kul.

Odpowiedzí :
\begin{center}
\includegraphics[width=78.840mm,height=17.580mm]{./F1_M_PR_M2014_page17_images/image001.eps}
\end{center}
Wypelnia

egzaminator

Nr zadania

Maks. liczba kt

11.

4

Uzyskana liczba pkt





{\it Egzamin maturalny z matematyki}

{\it Poziom rozszerzony}

{\it 19}

BRUDNOPIS





{\it Egzamin maturalny z matematyki}

{\it Poziom rozszerzony}

{\it 3}

Odpowied $\acute{\mathrm{z}}$:
\begin{center}
\includegraphics[width=78.840mm,height=17.628mm]{./F1_M_PR_M2014_page2_images/image001.eps}
\end{center}
Wypelnia

egzaminator

Nr zadania

Maks. liczba kt

1.

4

Uzyskana liczba pkt





{\it 4}

{\it Egzamin maturalny z matematyki}

{\it Poziom rozszerzony}

Zadanie 2. $(6pkt)$

Wyznacz wszystkie wartości parametru $m$, dla których funkcja kwadratowa

$f(x)=x^{2}-(2m+2)x+2m+5$ ma dwa rózne pierwiastki $x_{1}, x_{2}$ takie, $\dot{\mathrm{z}}\mathrm{e}$ suma kwadratów

odległości punktów $A=(x_{1}$, 0$) \mathrm{i}B=(x_{2}$, 0$)$ od prostej o równaniu $x+y+1=0$ jest równa 6.





{\it Egzamin maturalny z matematyki}

{\it Poziom rozszerzony}

{\it 5}

Odpowied $\acute{\mathrm{z}}$:
\begin{center}
\includegraphics[width=78.840mm,height=17.628mm]{./F1_M_PR_M2014_page4_images/image001.eps}
\end{center}
Wypelnia

egzaminator

Nr zadania

Maks. liczba kt

2.

Uzyskana liczba pkt





{\it 6}

{\it Egzamin maturalny z matematyki}

{\it Poziom rozszerzony}

Zadanie 3. $(4pkt)$

Rozwiąz równanie $\sqrt{3}\cdot\cos x=1+\sin x$ w przedziale $\langle 0, 2\pi\rangle.$

Odpowied $\acute{\mathrm{z}}$:





{\it Egzamin maturalny z matematyki}

{\it Poziom rozszerzony}

7

Zadanie 4. $(3pkt)$

Udowodnij, $\dot{\mathrm{z}}\mathrm{e}$ dla $\mathrm{k}\mathrm{a}\dot{\mathrm{z}}$ dych dwóch liczb rzeczywistych dodatnich $x, y$ prawdziwa jest

nierówność $(x+1)\displaystyle \frac{x}{y}+(y+1)\frac{y}{x}>2.$
\begin{center}
\includegraphics[width=90.276mm,height=17.580mm]{./F1_M_PR_M2014_page6_images/image001.eps}
\end{center}
Wypelnia

egzaminator

Nr zadania

Maks. liczba kt

3.

4

4.

3

Uzyskana liczba pkt





{\it 8}

{\it Egzamin maturalny z matematyki}

{\it Poziom rozszerzony}

Zadanie 5. $(5pkt)$

Dane są trzy okręgi o środkach $A, B, C$ i promieniach równych odpowiednio $r, 2r, 3r. \mathrm{K}\mathrm{a}\dot{\mathrm{z}}$ de

dwa z tych okręgów są zewnętrznie styczne: pierwszy z drugim w punkcie $K$, drugi z trzecim

w punkcie $L$ i trzeci z pierwszym w punkcie $M$. Oblicz stosunek pola trójkąta $KLM$ do pola

trójkąta $ABC.$





{\it Egzamin maturalny z matematyki}

{\it Poziom rozszerzony}

{\it 9}

Odpowied $\acute{\mathrm{z}}$:
\begin{center}
\includegraphics[width=78.840mm,height=17.628mm]{./F1_M_PR_M2014_page8_images/image001.eps}
\end{center}
Wypelnia

egzaminator

Nr zadania

Maks. liczba kt

5.

5

Uzyskana liczba pkt





$ 1\theta$

{\it Egzamin maturalny z matematyki}

{\it Poziom rozszerzony}

Zadanie 6. $(3pkt)$

Trójkąt $ABC$ jest wpisany w okrąg o środku $S$. Kąty wewnętrzne CAB, $ABC\mathrm{i}BCA$ tego

trójkąta są równe, odpowiednio, $\alpha,  2\alpha \mathrm{i}  4\alpha$. Wykaz, $\dot{\mathrm{z}}\mathrm{e}$ trójkąt $ABC$ jest rozwartokątny,

i udowodnij, $\dot{\mathrm{z}}\mathrm{e}$ miary wypukłych kątów środkowych $ASB, ASC\mathrm{i}BSC$ tworzą w podanej

kolejności ciąg arytmetyczny.



\end{document}