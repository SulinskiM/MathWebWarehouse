\documentclass[a4paper,12pt]{article}
\usepackage{latexsym}
\usepackage{amsmath}
\usepackage{amssymb}
\usepackage{graphicx}
\usepackage{wrapfig}
\pagestyle{plain}
\usepackage{fancybox}
\usepackage{bm}

\begin{document}

Arkusz zawiera informacje prawnie chronione do momentu rozpoczęcia egzaminu.

WPISUJE ZDAJACY

KOD PESEL

{\it Miejsce}

{\it na naklejkę}

{\it z kodem}
\begin{center}
\includegraphics[width=21.432mm,height=9.852mm]{./F1_M_PR_M2014_page0_images/image001.eps}

\includegraphics[width=82.092mm,height=9.852mm]{./F1_M_PR_M2014_page0_images/image002.eps}
\end{center}
\fbox{} dysleksja
\begin{center}
\includegraphics[width=204.060mm,height=217.272mm]{./F1_M_PR_M2014_page0_images/image003.eps}
\end{center}
EGZAMIN MATU

Z MATEMATY

LNY  MAJ 2014

POZIOM ROZSZERZONY

1.

2.

3.

4.

Sprawdzí, czy arkusz egzaminacyjny zawiera 19 stron

(zadania $1-11$). Ewentualny brak zgłoś

przewodniczącemu zespo nadzo jącego egzamin.

Rozwiązania zadań i odpowiedzi wpisuj w miejscu na to

przeznaczonym.

Pamiętaj, $\dot{\mathrm{z}}\mathrm{e}$ pominięcie argumentacji lub istotnych

obliczeń w rozwiązaniu zadania otwa ego $\mathrm{m}\mathrm{o}\dot{\mathrm{z}}\mathrm{e}$

spowodować, $\dot{\mathrm{z}}\mathrm{e}$ za to rozwiązanie nie otrzymasz pełnej

liczby punktów.

Pisz czytelnie i $\mathrm{u}\dot{\mathrm{z}}$ aj tvlko $\mathrm{d}$ gopisu lub -Dióra

z czatnym tuszem lub atramentem.

Nie uzywaj korektora, a błędne zapisy wyrazínie prze eśl.

Pamiętaj, $\dot{\mathrm{z}}\mathrm{e}$ zapisy w brudnopisie nie będą oceniane.

$\mathrm{M}\mathrm{o}\dot{\mathrm{z}}$ esz korzystać z zestawu wzorów matematycznych,

cyrkla i linijki oraz kalkulatora.

Na karcie odpowiedzi wpisz swój numer PESEL i przyklej

naklejkę z kodem.

Nie wpisuj $\dot{\mathrm{z}}$ adnych znaków w części przeznaczonej dla

egzaminatora.

Czas pracy:

180 minut

5.

6.

7.

8.

9.

Liczba punktów

do uzyskania: 50

$\Vert\Vert\Vert\Vert\Vert\Vert\Vert\Vert\Vert\Vert\Vert\Vert\Vert\Vert\Vert\Vert\Vert\Vert\Vert\Vert\Vert\Vert\Vert\Vert|  \mathrm{M}\mathrm{M}\mathrm{A}-\mathrm{R}1_{-}1\mathrm{P}-142$
\end{document}
