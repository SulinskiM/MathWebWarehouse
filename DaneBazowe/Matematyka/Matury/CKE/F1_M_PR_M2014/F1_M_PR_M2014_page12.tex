\documentclass[a4paper,12pt]{article}
\usepackage{latexsym}
\usepackage{amsmath}
\usepackage{amssymb}
\usepackage{graphicx}
\usepackage{wrapfig}
\pagestyle{plain}
\usepackage{fancybox}
\usepackage{bm}

\begin{document}

{\it Egzamin maturalny z matematyki}

{\it Poziom rozszerzony}

{\it 13}

Zadanie 8. $(4pkt)$

Punkty $A, B, C, D, E, F$ są kolejnymi wierzchołkami sześciokąta foremnego, przy czym

$A=(0,2\sqrt{3}), B=(2,0)$, a $C$ lezy na osi $ox$. Wyznacz równanie stycznej do okręgu

opisanego na tym sześciokącie przechodzącej przez wierzchołek $E.$

Odpowiedzí :
\begin{center}
\includegraphics[width=90.276mm,height=17.580mm]{./F1_M_PR_M2014_page12_images/image001.eps}
\end{center}
Wypelnia

egzaminator

Nr zadania

Maks. liczba kt

7.

8.

4

Uzyskana liczba pkt
\end{document}
