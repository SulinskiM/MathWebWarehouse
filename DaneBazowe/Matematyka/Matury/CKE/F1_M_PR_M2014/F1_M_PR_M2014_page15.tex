\documentclass[a4paper,12pt]{article}
\usepackage{latexsym}
\usepackage{amsmath}
\usepackage{amssymb}
\usepackage{graphicx}
\usepackage{wrapfig}
\pagestyle{plain}
\usepackage{fancybox}
\usepackage{bm}

\begin{document}

{\it 16}

{\it Egzamin maturalny z matematyki}

{\it Poziom rozszerzony}

Zadanie 10. $(5pkt)$

Wyznacz wszystkie całkowite wartości parametru $m$, dla których równanie

$(x^{3}+2x^{2}+2x+1)[x^{2}-(2m+1)x+m^{2}+m]=0$ ma trzy, parami rózne, pierwiastki

rzeczywiste, takie $\dot{\mathrm{z}}$ ejeden z nichjest średnią arytmetyczną dwóch pozostałych.
\end{document}
