\documentclass[a4paper,12pt]{article}
\usepackage{latexsym}
\usepackage{amsmath}
\usepackage{amssymb}
\usepackage{graphicx}
\usepackage{wrapfig}
\pagestyle{plain}
\usepackage{fancybox}
\usepackage{bm}

\begin{document}

{\it 18}

{\it Egzamin maturalny z matematyki}

{\it Poziom rozszerzony}

Zadanie ll. $(4pkt)$

$\mathrm{Z}$ urny zawierającej 10 ku1 ponumerowanych ko1ejnymi 1iczbami od 1 do 101osujemy

jednocześnie trzy kule. Oblicz prawdopodobieństwo zdarzenia $A$ polegającego na tym, $\dot{\mathrm{z}}\mathrm{e}$

numerjednej z wylosowanych kuljest równy sumie numerów dwóch pozostałych kul.

Odpowiedzí :
\begin{center}
\includegraphics[width=78.840mm,height=17.580mm]{./F1_M_PR_M2014_page17_images/image001.eps}
\end{center}
Wypelnia

egzaminator

Nr zadania

Maks. liczba kt

11.

4

Uzyskana liczba pkt
\end{document}
