\documentclass[a4paper,12pt]{article}
\usepackage{latexsym}
\usepackage{amsmath}
\usepackage{amssymb}
\usepackage{graphicx}
\usepackage{wrapfig}
\pagestyle{plain}
\usepackage{fancybox}
\usepackage{bm}

\begin{document}

{\it 4}

{\it Egzamin maturalny z matematyki}

{\it Poziom rozszerzony}

Zadanie 2. $(6pkt)$

Wyznacz wszystkie wartości parametru $m$, dla których funkcja kwadratowa

$f(x)=x^{2}-(2m+2)x+2m+5$ ma dwa rózne pierwiastki $x_{1}, x_{2}$ takie, $\dot{\mathrm{z}}\mathrm{e}$ suma kwadratów

odległości punktów $A=(x_{1}$, 0$) \mathrm{i}B=(x_{2}$, 0$)$ od prostej o równaniu $x+y+1=0$ jest równa 6.
\end{document}
