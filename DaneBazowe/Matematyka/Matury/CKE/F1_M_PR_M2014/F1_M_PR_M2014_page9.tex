\documentclass[a4paper,12pt]{article}
\usepackage{latexsym}
\usepackage{amsmath}
\usepackage{amssymb}
\usepackage{graphicx}
\usepackage{wrapfig}
\pagestyle{plain}
\usepackage{fancybox}
\usepackage{bm}

\begin{document}

$ 1\theta$

{\it Egzamin maturalny z matematyki}

{\it Poziom rozszerzony}

Zadanie 6. $(3pkt)$

Trójkąt $ABC$ jest wpisany w okrąg o środku $S$. Kąty wewnętrzne CAB, $ABC\mathrm{i}BCA$ tego

trójkąta są równe, odpowiednio, $\alpha,  2\alpha \mathrm{i}  4\alpha$. Wykaz, $\dot{\mathrm{z}}\mathrm{e}$ trójkąt $ABC$ jest rozwartokątny,

i udowodnij, $\dot{\mathrm{z}}\mathrm{e}$ miary wypukłych kątów środkowych $ASB, ASC\mathrm{i}BSC$ tworzą w podanej

kolejności ciąg arytmetyczny.
\end{document}
