\documentclass[a4paper,12pt]{article}
\usepackage{latexsym}
\usepackage{amsmath}
\usepackage{amssymb}
\usepackage{graphicx}
\usepackage{wrapfig}
\pagestyle{plain}
\usepackage{fancybox}
\usepackage{bm}

\begin{document}

Arkusz zawiera informacje prawnie chronione do momentu rozpoczęcia egzaminu.

UZUPELNIA ZDAJACY

KOD PESEL

{\it Miejsce}

{\it na naklejkę}

{\it z kodem}
\begin{center}
\includegraphics[width=21.432mm,height=9.852mm]{./F1_M_PR_M2015_page0_images/image001.eps}

\includegraphics[width=82.092mm,height=9.852mm]{./F1_M_PR_M2015_page0_images/image002.eps}
\end{center}
\fbox{} dysleksja
\begin{center}
\includegraphics[width=204.060mm,height=197.820mm]{./F1_M_PR_M2015_page0_images/image003.eps}
\end{center}
EGZAMIN MATU LNY

Z MATEMATYKI

POZIOM ROZSZERZONY  8 MAJA 20I5

Instrukcja dla zdającego

l. Sprawdzí, czy arkusz egzaminacyjny zawiera 17 stron

(zadania $1-11$). Ewentualny brak zgłoś przewodniczącemu

zespo nadzorującego egzamin.

2. Rozwiązania zadań i odpowiedzi wpisuj w miejscu na to

przeznaczonym.

3. Pamiętaj, $\dot{\mathrm{z}}\mathrm{e}$ pominięcie argumentacji lub istotnych

obliczeń w rozwiązaniu zadania otwa ego $\mathrm{m}\mathrm{o}\dot{\mathrm{z}}\mathrm{e}$

spowodować, $\dot{\mathrm{z}}\mathrm{e}$ za to rozwiązanie nie będziesz mógł

dostać pełnej liczby punktów.

4. Pisz czytelnie i uzywaj tvlko długopisu lub -Dióra

z czarnym tuszem lub atramentem.

5. Nie uzywaj korektora, a błędne zapisy wyrazínie prze eśl.

6. Pamiętaj, $\dot{\mathrm{z}}\mathrm{e}$ zapisy w brudnopisie nie będą oceniane.

7. $\mathrm{M}\mathrm{o}\dot{\mathrm{z}}$ esz korzystać z zestawu wzorów matematycznych,

cyrkla i linijki oraz kalkulatora prostego.

8. Na tej stronie oraz na karcie odpowiedzi wpisz swój

numer PESEL i przyklej naklejkę z kodem.

9. Nie wpisuj $\dot{\mathrm{z}}$ adnych znaków w części przeznaczonej dla

egzaminatora.

Godzina rozpoczęcia:

Czas pracy:

180 minut

Liczba punktów

do uzyskania: 50

$\Vert\Vert\Vert\Vert\Vert\Vert\Vert\Vert\Vert\Vert\Vert\Vert\Vert\Vert\Vert\Vert\Vert\Vert\Vert\Vert\Vert\Vert\Vert\Vert|  \mathrm{M}\mathrm{M}\mathrm{A}-\mathrm{R}1_{-}1\mathrm{P}-152$




{\it Egzamin maturalny z matematyki}

{\it Poziom rozszerzony}

Zadanie l.$(3pkt)$

Wykaz, $\dot{\mathrm{z}}\mathrm{e}$ dla $\mathrm{k}\mathrm{a}\dot{\mathrm{z}}$ dej dodatniej liczby rzeczywistej $x$ róz$\cdot$nej od l oraz dla $\mathrm{k}\mathrm{a}\dot{\mathrm{z}}$ dej dodatniej

liczby rzeczywistej $y$ róz$\cdot$nej od l prawdziwajest równość

$\displaystyle \log_{x}(xy)\cdot\log_{y}(\frac{y}{x})=\log_{y}(xy)\cdot\log_{x}(\frac{y}{x}).$

Strona 2 z 17

MMA-IR





Odpowied $\acute{\mathrm{z}}$:

{\it Egzamin maturalny z matematyki}

{\it Poziom rozszerzony}
\begin{center}
\includegraphics[width=82.044mm,height=17.832mm]{./F1_M_PR_M2015_page10_images/image001.eps}
\end{center}
Wypelnia

egzaminator

Nr zadania

Maks. liczba kt

8.

4

Uzyskana liczba pkt

MMA-IR

Strona ll z 17





{\it Egzamin maturalny z matematyki}

{\it Poziom rozszerzony}

Zadanie 9, $(5pkt)$

Wyznacz równania prostych stycznych do okręgu o równaniu

i zarazem prostopadłych do prostej $x+2y-6=0.$

$x^{2}+y^{2}+4x-6y-3=0$

Strona 12 z 17

MMA-IR





Odpowied $\acute{\mathrm{z}}$:

{\it Egzamin maturalny z matematyki}

{\it Poziom rozszerzony}
\begin{center}
\includegraphics[width=82.044mm,height=17.784mm]{./F1_M_PR_M2015_page12_images/image001.eps}
\end{center}
Wypelnia

egzamÍnator

Nr zadania

Maks. liczba kt

5

Uzyskana liczba pkt

MMA-IR

Strona 13 z 17





{\it Egzamin maturalny z matematyki}

{\it Poziom rozszerzony}

Zadanie $l0. (6pki)$

Krawędzí podstawy ostrosłupa prawidłowego czworokątnego ABCDS ma długość $a$. Ściana

boczna jest nachylona do płaszczyzny podstawy ostrosłupa pod kątem $ 2\alpha$. Ostrosłup ten

przecięto płaszczyzną, która przechodzi przez krawędzí podstawy i dzieli na połowy kąt

pomiędzy ścianą boczną i podstawą. Oblicz pole powstałego przekroju tego ostrosłupa.

Strona 14 z 17

MMA-IR





Odpowied $\acute{\mathrm{z}}$:

{\it Egzamin maturalny z matematyki}

{\it Poziom rozszerzony}
\begin{center}
\includegraphics[width=82.044mm,height=17.784mm]{./F1_M_PR_M2015_page14_images/image001.eps}
\end{center}
Wypelnia

egzamÍnator

Nr zadania

Maks. liczba kt

10.

Uzyskana liczba pkt

MMA-IR

Strona 15 z 17





{\it Egzamin maturalny z matematyki}

{\it Poziom rozszerzony}

Zadanie $l1_{1}. (3pkt)$

Rozwazmy rzut sześcioma kostkami do gry, z których $\mathrm{k}\mathrm{a}\dot{\mathrm{z}}$ da ma inny kolor. Oblicz

prawdopodobieństwo zdarzenia polegającego na tym, $\dot{\mathrm{z}}\mathrm{e}$ uzyskany wynik rzutu spełnia

równoczeŚnie trzy warunki:

dokładnie na dwóch kostkach otrzymano pojednym oczku;

dokładnie na trzech kostkach otrzymano po sześć oczek;

suma wszystkich otrzymanych liczb oczekjest parzysta.

Odpowiedzí:
\begin{center}
\includegraphics[width=82.044mm,height=17.784mm]{./F1_M_PR_M2015_page15_images/image001.eps}
\end{center}
Nr zadania

Wypelnia Maks. liczba kt

egzaminator

Uzyskana liczba pkt

11.

3

Strona 16 z 17

MMA-IR





{\it Egzamin maturalny z matematyki}

{\it Poziom rozszerzony}

{\it BRUDNOPIS} ({\it nie podlega ocenie})

MMA-IR

Strona 17 z 17





{\it Egzamin maturalny z matematyki}

{\it Poziom rozszerzony}

Zadanie 2. $(5pkt)$

Dany jest wielomian $W(x)=x^{3}-3mx^{2}+(3m^{2}-1)x-9m^{2}+20m+4$. Wykres tego

wielomianu, po przesunięciu o wektor $u=[-3,0]$, przechodzi przez początek układu

współrzędnych. Wyznacz wszystkie pierwiastki wielomianu $W.$

Odpowied $\acute{\mathrm{z}}$:
\begin{center}
\includegraphics[width=96.012mm,height=17.784mm]{./F1_M_PR_M2015_page2_images/image001.eps}
\end{center}
Wypelnia

egzaminator

Nr zadania

Maks. liczba kt

1.

3

2.

5

Uzyskana liczba pkt

MMA-IR

Strona 3 z 17





{\it Egzamin maturalny z matematyki}

{\it Poziom rozszerzony}

Zadanie 3. $(6pkt)$

Wyznacz wszystkie wartości parametru $m$, dla których równanie $(m^{2}-m)x^{2}-x+1=0$ ma

dwa rózne rozwiązania rzeczywiste $x_{1}, x_{2}$ takie, $\displaystyle \dot{\mathrm{z}}\mathrm{e}\frac{1}{x_{1}+x_{2}}\leq\frac{m}{3}\leq\frac{1}{x_{1}}+\frac{1}{x_{2}}$

Strona 4 z17

MMA-IR





Odpowied $\acute{\mathrm{z}}$:

{\it Egzamin maturalny z matematyki}

{\it Poziom rozszerzony}
\begin{center}
\includegraphics[width=82.044mm,height=17.784mm]{./F1_M_PR_M2015_page4_images/image001.eps}
\end{center}
Wypelnia

egzamÍnator

Nr zadania

Maks. liczba kt

3.

Uzyskana liczba pkt

MMA-IR

Strona 5 z 17





{\it Egzamin maturalny z matematyki}

{\it Poziom rozszerzony}

Zadanie 4. (6pkt)

Trzy liczby tworzą ciąg arytmetyczny. Jeśli do pierwszej z nich dodamy 5, do diugiej 3, a do

trzeciej 4, to otrzymamy rosnący ciąg geometryczny, w którym trzeci wyraz jest cztery razy

większy od pierwszego. Znajdzí te liczby.

Odpowiedzí:

Strona 6 z17

MMA-IR





{\it Egzamin maturalny z matematyki}

{\it Poziom rozszerzony}

Zadanie 5. $(4pkt)$

Rozwiąz równanie $\sin^{2}2x-4\sin^{2}x+1=0$ w przedziale $\langle 0,2\pi\rangle.$

Odpowied $\acute{\mathrm{z}}$:
\begin{center}
\includegraphics[width=96.012mm,height=17.784mm]{./F1_M_PR_M2015_page6_images/image001.eps}
\end{center}
Wypelnia

egzaminator

Nr zadania

Maks. liczba kt

4.

5.

4

Uzyskana liczba pkt

MMA-IR

Strona 7 z 17





{\it Egzamin maturalny z matematyki}

{\it Poziom rozszerzony}

Zadanie 6. $(4pkt)$

Rozwiąz nierówność $|2x-6|+|x+7|\geq 17.$

Odpowiedzí:

Strona 8 z17

MMA-IR





{\it Egzamin maturalny z matematyki}

{\it Poziom rozszerzony}

Zadanie 7. $(4pkt)$

$\mathrm{O}$ trapezie ABCD wiadomo, $\dot{\mathrm{z}}\mathrm{e}$ mozna w niego wpisać okrąg, a ponadto długościjego boków

{\it AB}, $BC$, {\it CD}, $AD-\mathrm{w}$ podanej kolejności- tworzą ciąg geometryczny. Uzasadnij, $\dot{\mathrm{z}}\mathrm{e}$ trapez

ABCD jest rombem.
\begin{center}
\includegraphics[width=96.012mm,height=17.832mm]{./F1_M_PR_M2015_page8_images/image001.eps}
\end{center}
Wypelnia

egzaminator

Nr zadania

Maks. liczba kt

4

7.

4

Uzyskana liczba pkt

MMA-IR

Strona 9 z 17





{\it Egzamin maturalny z matematyki}

{\it Poziom rozszerzony}

$\mathrm{Z}\mathrm{a}\mathrm{d}\mathrm{a}\mathrm{n}\mathrm{i}\varepsilon 8. (4pkt)$

Na boku $AB$ trójkąta równobocznego $ABC$ wybrano punkt $D$ taki, $\dot{\mathrm{z}}\mathrm{e}|AD|$ : $|DB|=2:3.$

Oblicz tangens kąta $ACD.$

Strona 10 z 17

MMA-IR



\end{document}