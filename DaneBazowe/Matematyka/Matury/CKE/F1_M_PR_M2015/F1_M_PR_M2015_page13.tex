\documentclass[a4paper,12pt]{article}
\usepackage{latexsym}
\usepackage{amsmath}
\usepackage{amssymb}
\usepackage{graphicx}
\usepackage{wrapfig}
\pagestyle{plain}
\usepackage{fancybox}
\usepackage{bm}

\begin{document}

{\it Egzamin maturalny z matematyki}

{\it Poziom rozszerzony}

Zadanie $l0. (6pki)$

Krawędzí podstawy ostrosłupa prawidłowego czworokątnego ABCDS ma długość $a$. Ściana

boczna jest nachylona do płaszczyzny podstawy ostrosłupa pod kątem $ 2\alpha$. Ostrosłup ten

przecięto płaszczyzną, która przechodzi przez krawędzí podstawy i dzieli na połowy kąt

pomiędzy ścianą boczną i podstawą. Oblicz pole powstałego przekroju tego ostrosłupa.

Strona 14 z 17

MMA-IR
\end{document}
