\documentclass[a4paper,12pt]{article}
\usepackage{latexsym}
\usepackage{amsmath}
\usepackage{amssymb}
\usepackage{graphicx}
\usepackage{wrapfig}
\pagestyle{plain}
\usepackage{fancybox}
\usepackage{bm}

\begin{document}

{\it Egzamin maturalny z matematyki}

{\it Poziom rozszerzony}

Zadanie $l1_{1}. (3pkt)$

Rozwazmy rzut sześcioma kostkami do gry, z których $\mathrm{k}\mathrm{a}\dot{\mathrm{z}}$ da ma inny kolor. Oblicz

prawdopodobieństwo zdarzenia polegającego na tym, $\dot{\mathrm{z}}\mathrm{e}$ uzyskany wynik rzutu spełnia

równoczeŚnie trzy warunki:

dokładnie na dwóch kostkach otrzymano pojednym oczku;

dokładnie na trzech kostkach otrzymano po sześć oczek;

suma wszystkich otrzymanych liczb oczekjest parzysta.

Odpowiedzí:
\begin{center}
\includegraphics[width=82.044mm,height=17.784mm]{./F1_M_PR_M2015_page15_images/image001.eps}
\end{center}
Nr zadania

Wypelnia Maks. liczba kt

egzaminator

Uzyskana liczba pkt

11.

3

Strona 16 z 17

MMA-IR
\end{document}
