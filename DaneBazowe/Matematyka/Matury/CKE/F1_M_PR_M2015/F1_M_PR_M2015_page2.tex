\documentclass[a4paper,12pt]{article}
\usepackage{latexsym}
\usepackage{amsmath}
\usepackage{amssymb}
\usepackage{graphicx}
\usepackage{wrapfig}
\pagestyle{plain}
\usepackage{fancybox}
\usepackage{bm}

\begin{document}

{\it Egzamin maturalny z matematyki}

{\it Poziom rozszerzony}

Zadanie 2. $(5pkt)$

Dany jest wielomian $W(x)=x^{3}-3mx^{2}+(3m^{2}-1)x-9m^{2}+20m+4$. Wykres tego

wielomianu, po przesunięciu o wektor $u=[-3,0]$, przechodzi przez początek układu

współrzędnych. Wyznacz wszystkie pierwiastki wielomianu $W.$

Odpowied $\acute{\mathrm{z}}$:
\begin{center}
\includegraphics[width=96.012mm,height=17.784mm]{./F1_M_PR_M2015_page2_images/image001.eps}
\end{center}
Wypelnia

egzaminator

Nr zadania

Maks. liczba kt

1.

3

2.

5

Uzyskana liczba pkt

MMA-IR

Strona 3 z 17
\end{document}
