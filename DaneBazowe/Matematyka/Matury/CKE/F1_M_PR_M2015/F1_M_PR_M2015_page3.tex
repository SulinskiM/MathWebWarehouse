\documentclass[a4paper,12pt]{article}
\usepackage{latexsym}
\usepackage{amsmath}
\usepackage{amssymb}
\usepackage{graphicx}
\usepackage{wrapfig}
\pagestyle{plain}
\usepackage{fancybox}
\usepackage{bm}

\begin{document}

{\it Egzamin maturalny z matematyki}

{\it Poziom rozszerzony}

Zadanie 3. $(6pkt)$

Wyznacz wszystkie wartości parametru $m$, dla których równanie $(m^{2}-m)x^{2}-x+1=0$ ma

dwa rózne rozwiązania rzeczywiste $x_{1}, x_{2}$ takie, $\displaystyle \dot{\mathrm{z}}\mathrm{e}\frac{1}{x_{1}+x_{2}}\leq\frac{m}{3}\leq\frac{1}{x_{1}}+\frac{1}{x_{2}}$

Strona 4 z17

MMA-IR
\end{document}
