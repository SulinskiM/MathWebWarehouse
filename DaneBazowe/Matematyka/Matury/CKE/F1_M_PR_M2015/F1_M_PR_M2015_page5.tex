\documentclass[a4paper,12pt]{article}
\usepackage{latexsym}
\usepackage{amsmath}
\usepackage{amssymb}
\usepackage{graphicx}
\usepackage{wrapfig}
\pagestyle{plain}
\usepackage{fancybox}
\usepackage{bm}

\begin{document}

{\it Egzamin maturalny z matematyki}

{\it Poziom rozszerzony}

Zadanie 4. (6pkt)

Trzy liczby tworzą ciąg arytmetyczny. Jeśli do pierwszej z nich dodamy 5, do diugiej 3, a do

trzeciej 4, to otrzymamy rosnący ciąg geometryczny, w którym trzeci wyraz jest cztery razy

większy od pierwszego. Znajdzí te liczby.

Odpowiedzí:

Strona 6 z17

MMA-IR
\end{document}
