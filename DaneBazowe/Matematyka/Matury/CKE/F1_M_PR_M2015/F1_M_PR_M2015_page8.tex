\documentclass[a4paper,12pt]{article}
\usepackage{latexsym}
\usepackage{amsmath}
\usepackage{amssymb}
\usepackage{graphicx}
\usepackage{wrapfig}
\pagestyle{plain}
\usepackage{fancybox}
\usepackage{bm}

\begin{document}

{\it Egzamin maturalny z matematyki}

{\it Poziom rozszerzony}

Zadanie 7. $(4pkt)$

$\mathrm{O}$ trapezie ABCD wiadomo, $\dot{\mathrm{z}}\mathrm{e}$ mozna w niego wpisać okrąg, a ponadto długościjego boków

{\it AB}, $BC$, {\it CD}, $AD-\mathrm{w}$ podanej kolejności- tworzą ciąg geometryczny. Uzasadnij, $\dot{\mathrm{z}}\mathrm{e}$ trapez

ABCD jest rombem.
\begin{center}
\includegraphics[width=96.012mm,height=17.832mm]{./F1_M_PR_M2015_page8_images/image001.eps}
\end{center}
Wypelnia

egzaminator

Nr zadania

Maks. liczba kt

4

7.

4

Uzyskana liczba pkt

MMA-IR

Strona 9 z 17
\end{document}
