\documentclass[a4paper,12pt]{article}
\usepackage{latexsym}
\usepackage{amsmath}
\usepackage{amssymb}
\usepackage{graphicx}
\usepackage{wrapfig}
\pagestyle{plain}
\usepackage{fancybox}
\usepackage{bm}

\begin{document}

Arkusz zawiera informacje prawnie chronione do momentu rozpoczęcia egzaminu.

UZUPELNIA ZDAJACY

KOD PESEL

{\it miejsce}

{\it na naklejkę}
\begin{center}
\includegraphics[width=21.432mm,height=9.852mm]{./F1_M_PR_M2016_page0_images/image001.eps}

\includegraphics[width=82.092mm,height=9.852mm]{./F1_M_PR_M2016_page0_images/image002.eps}
\end{center}
\fbox{} dysleksja
\begin{center}
\includegraphics[width=204.060mm,height=197.868mm]{./F1_M_PR_M2016_page0_images/image003.eps}
\end{center}
EGZAMIN MATU

Z MATEMATY

LNY

Instrukc.ia dla zda.iacego

1. Sprawd $\acute{\mathrm{z}}$, czy arkusz egzaminacyjny zawiera 24 strony

(zadania $1-11$). Ewentualny brak zgłoś przewodniczącemu

zespo nadzorującego egzamin.

2. Rozwiązania zadań i odpowiedzi wpisuj w miejscu na to

przeznaczonym.

3. Pamiętaj, $\dot{\mathrm{z}}\mathrm{e}$ pominięcie argumentacji lub istotnych

obliczeń w rozwiązaniu zadania otwartego $\mathrm{m}\mathrm{o}\dot{\mathrm{z}}\mathrm{e}$

spowodować, $\dot{\mathrm{z}}\mathrm{e}$ za to rozwiązanie nie będziesz mógł

dostać pełnej liczby punktów.

4. Pisz czytelnie i $\mathrm{u}\dot{\mathrm{z}}$ aj tylko $\mathrm{d}$ gopisu lub pióra

z czarnym tuszem lub atramentem.

5. Nie $\mathrm{u}\dot{\mathrm{z}}$ aj korektora, a błędne zapisy wyra $\acute{\mathrm{z}}\mathrm{n}\mathrm{i}\mathrm{e}$ prze eśl.

6. Pamiętaj, $\dot{\mathrm{z}}\mathrm{e}$ zapisy w brudnopisie nie będą oceniane.

7. $\mathrm{M}\mathrm{o}\dot{\mathrm{z}}$ esz korzystać z zestawu wzorów matematycznych,

cyrkla i linijki oraz kalkulatora prostego.

8. Na tej stronie oraz na karcie odpowiedzi wpisz swój

numer PESEL i przyklej naklejkę z kodem.

9. Nie wpisuj $\dot{\mathrm{z}}$ adnych znaków w części przeznaczonej dla

egzaminatora.

Godzina rozpoczęcia:

9:00

Czas pracy:

180 minut

Liczba punktów

do uzyskania: 50

$\Vert\Vert\Vert\Vert\Vert\Vert\Vert\Vert\Vert\Vert\Vert\Vert\Vert\Vert\Vert\Vert\Vert\Vert\Vert\Vert\Vert\Vert\Vert\Vert|  \mathrm{M}\mathrm{M}\mathrm{A}-\mathrm{R}1_{-}1\mathrm{P}-162$




{\it Egzamin maturalny z matematyki}

{\it Poziom rozszerzony}

Zadanie 1. (3pkt)

Niech $\log_{7}4=a$. Wyznacz $\log_{\sqrt{2}}49$ w zalezności od $a.$

Strona 2 z24

MMA-IR





Odpowiedzí:

{\it Egzamin maturalny z matematyki}

{\it Poziom rozszerzony}
\begin{center}
\includegraphics[width=82.044mm,height=17.784mm]{./F1_M_PR_M2016_page10_images/image001.eps}
\end{center}
Wypelnia

egzaminator

Nr zadania

Maks. liczba kt

5.

Uzyskana liczba pkt

MMA-IR

Strona ll z24





{\it Egzamin maturalny z matematyki}

{\it Poziom rozszerzony}

Zadaníe 6. (6pkt)

Punkty $A=(1,1) \mathrm{i} B=(6,2)$ są wierzchołkami trójkąta $ABC$. Wysokości trójkąta $ABC$

przecinają się w punkcie $M=(3,3)$. Oblicz pole tego trójkąta.

Strona 12 z24

MMA-IR





Odpowiedzí:

{\it Egzamin maturalny z matematyki}

{\it Poziom rozszerzony}
\begin{center}
\includegraphics[width=82.044mm,height=17.784mm]{./F1_M_PR_M2016_page12_images/image001.eps}
\end{center}
Wypelnia

egzaminator

Nr zadania

Maks. liczba kt

Uzyskana liczba pkt

MMA-IR

Strona 13 z24





{\it Egzamin maturalny z matematyki}

{\it Poziom rozszerzony}

Zadaníe 7. (3pkt)

Reszta z dzielenia liczby naturalnej $a$ przez 6 jest równa 1. Reszta z dzie1enia 1iczby

naturalnej $b$ przez $6$jest równa 5. Uzasadnij, $\dot{\mathrm{z}}\mathrm{e}$ liczba $a^{2}-b^{2}$ jest podzielna przez 24.

Strona 14 z24

MMA-IR





{\it Egzamin maturalny z matematyki}

{\it Poziom rozszerzony}
\begin{center}
\includegraphics[width=82.044mm,height=17.784mm]{./F1_M_PR_M2016_page14_images/image001.eps}
\end{center}
Wypelnia

egzaminator

Nr zadania

Maks. liczba kt

7.

3

Uzyskana liczba pkt

MMA-IR

Strona 15 z24





{\it Egzamin maturalny z matematyki}

{\it Poziom rozszerzony}

Zadaníe 8. (6pkt)

$\mathrm{W}$ ostrosłupie prawidłowym czworokątnym ABCDS o podstawie ABCD wysokość jest równa 5,

a kąt między sąsiednimi ścianami bocznymi ostrosłupa ma miarę $120^{\mathrm{o}}$ Oblicz objętość tego

ostrosłupa.

Strona 16 z24

MMA-IR





Odpowiedzí:

{\it Egzamin maturalny z matematyki}

{\it Poziom rozszerzony}
\begin{center}
\includegraphics[width=82.044mm,height=17.784mm]{./F1_M_PR_M2016_page16_images/image001.eps}
\end{center}
Wypelnia

egzaminator

Nr zadania

Maks. liczba kt

8.

Uzyskana liczba pkt

MMA-IR

Strona 17 z24





{\it Egzamin maturalny z matematyki}

{\it Poziom rozszerzony}

Zadaníe 9. (3pkt)

Dany jest okrąg o średnicy $AB$ i środku $S$ oraz dwa okręgi o średnicach AS $\mathrm{i}BS$. Okrąg

o środku $M$ i promieniu $r$ ma z $\mathrm{k}\mathrm{a}\dot{\mathrm{z}}$ dym z danych okręgów dokładnie jeden punkt wspólny

(zobacz rysunek). Wykaz, $\displaystyle \dot{\mathrm{z}}\mathrm{e}r=\frac{1}{6}|AB|.$
\begin{center}
\includegraphics[width=91.392mm,height=82.752mm]{./F1_M_PR_M2016_page17_images/image001.eps}
\end{center}
{\it M}

{\it A  K S  L  B}

Strona 18 z24

MMA-IR





{\it Egzamin maturalny z matematyki}

{\it Poziom rozszerzony}
\begin{center}
\includegraphics[width=82.044mm,height=17.784mm]{./F1_M_PR_M2016_page18_images/image001.eps}
\end{center}
Wypelnia

egzaminator

Nr zadania

Maks. liczba kt

3

Uzyskana liczba pkt

MMA-IR

Strona 19 z24





{\it Egzamin maturalny z matematyki}

{\it Poziom rozszerzony}

Zadanie 10. (5pkt)

$\mathrm{W}$ urnie znajduje się 20 ku1: 9 białych, 9 czerwonych i 2 zie1one. $\mathrm{Z}$ tej umy losujemy bez

zwracania 3 ku1e. Ob1icz prawdopodobieństwo zdarzenia po1egającego na tym, $\dot{\mathrm{z}}\mathrm{e}$ co najmniej

dwie z wylosowanych kul są tego samego koloru.

Strona 20 z24

MMA-IR





Odpowiedzí:

{\it Egzamin maturalny z matematyki}

{\it Poziom rozszerzony}
\begin{center}
\includegraphics[width=82.044mm,height=17.784mm]{./F1_M_PR_M2016_page2_images/image001.eps}
\end{center}
Wypelnia

egzaminator

Nr zadania

Maks. liczba kt

1.

3

Uzyskana liczba pkt

MMA-IR

Strona 3 z24





Odpowiedzí:

{\it Egzamin maturalny z matematyki}

{\it Poziom rozszerzony}
\begin{center}
\includegraphics[width=82.044mm,height=17.832mm]{./F1_M_PR_M2016_page20_images/image001.eps}
\end{center}
Wypelnia

egzaminator

Nr zadania

Maks. liczba kt

5

Uzyskana liczba pkt

MMA-IR

Strona 21 z24





{\it Egzamin maturalny z matematyki}

{\it Poziom rozszerzony}

Zadanie 11. (3pkt)

Rozpatrujemy wszystkie liczby naturalne dziesięciocyfrowe, w zapisie których mogą

występować wyłącznie cyfry 1, 2, 3, przy czym cyfra 1 występuje dokładnie trzy razy.

Uzasadnij, $\dot{\mathrm{z}}\mathrm{e}$ takich liczb jest 15360.

Strona 22 z24

MMA-IR





Odpowiedzí:

{\it Egzamin maturalny z matematyki}

{\it Poziom rozszerzony}
\begin{center}
\includegraphics[width=82.044mm,height=17.832mm]{./F1_M_PR_M2016_page22_images/image001.eps}
\end{center}
Wypelnia

egzaminator

Nr zadania

Maks. liczba kt

3

Uzyskana liczba pkt

MMA-IR

Strona 23 z24





{\it Egzamin maturalny z matematyki}

{\it Poziom rozszerzony}

{\it BRUDNOPIS} ({\it nie podlega ocenie})

Strona 24 z24

MMA-IR





{\it Egzamin maturalny z matematyki}

{\it Poziom rozszerzony}

Zadanie 2. (5pkt)

Wielomian $W(x)=2x^{3}+mx^{2}-22x+n$ jest podzielny przez $\mathrm{k}\mathrm{a}\dot{\mathrm{z}}\mathrm{d}\mathrm{y}$ z dwumianów $x+3$

$\mathrm{i}x-4$. Oblicz wartości współczynników $n\mathrm{i}m$ oraz rozwiąz nierówność $W(x)\geq 0.$

Strona 4 z24

MMA-IR





Odpowiedzí:

{\it Egzamin maturalny z matematyki}

{\it Poziom rozszerzony}
\begin{center}
\includegraphics[width=82.044mm,height=17.784mm]{./F1_M_PR_M2016_page4_images/image001.eps}
\end{center}
Wypelnia

egzaminator

Nr zadania

Maks. liczba kt

2.

5

Uzyskana liczba pkt

MMA-IR

Strona 5 z24





{\it Egzamin maturalny z matematyki}

{\it Poziom rozszerzony}

Zadaníe 3. (4pkt)

Rozwiąz równanie $-2\cos^{2}x+3\sin x+3=0$ w przedziale$<0,2\pi>.$

Strona 6 z24

MMA-IR





Odpowiedzí:

{\it Egzamin maturalny z matematyki}

{\it Poziom rozszerzony}
\begin{center}
\includegraphics[width=82.044mm,height=17.784mm]{./F1_M_PR_M2016_page6_images/image001.eps}
\end{center}
Wypelnia

egzaminator

Nr zadania

Maks. liczba kt

3.

4

Uzyskana liczba pkt

MMA-IR

Strona 7 z24





{\it Egzamin maturalny z matematyki}

{\it Poziom rozszerzony}

Zadaníe 4. (6pkt)

Ciąg $(a,4,b,c)$ jest arytmetyczny, a ciąg $(a,a+b,4c)$jest geometryczny. Oblicz $a, b\mathrm{i}c.$

Strona 8 z24

MMA-IR





Odpowiedzí:

{\it Egzamin maturalny z matematyki}

{\it Poziom rozszerzony}
\begin{center}
\includegraphics[width=82.044mm,height=17.784mm]{./F1_M_PR_M2016_page8_images/image001.eps}
\end{center}
Wypelnia

egzaminator

Nr zadania

Maks. liczba kt

4.

Uzyskana liczba pkt

MMA-IR

Strona 9 z24





{\it Egzamin maturalny z matematyki}

{\it Poziom rozszerzony}

Zadaníe 5. (6pkt)

$\mathrm{W}$ trapezie równoramiennym ABCD, w którym AB $\Vert$ CD, dane są

$|BC|=|AD|=40$. Oblicz promień okręgu wpisanego w trójkąt $ABP,$

przecięcia przekątnych tego trapezu.

$|AB|=84, |CD|=36,$

gdzie P jest punktem

Strona 10 z24

MMA-IR



\end{document}