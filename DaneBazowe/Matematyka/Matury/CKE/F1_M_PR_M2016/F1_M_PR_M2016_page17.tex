\documentclass[a4paper,12pt]{article}
\usepackage{latexsym}
\usepackage{amsmath}
\usepackage{amssymb}
\usepackage{graphicx}
\usepackage{wrapfig}
\pagestyle{plain}
\usepackage{fancybox}
\usepackage{bm}

\begin{document}

{\it Egzamin maturalny z matematyki}

{\it Poziom rozszerzony}

Zadaníe 9. (3pkt)

Dany jest okrąg o średnicy $AB$ i środku $S$ oraz dwa okręgi o średnicach AS $\mathrm{i}BS$. Okrąg

o środku $M$ i promieniu $r$ ma z $\mathrm{k}\mathrm{a}\dot{\mathrm{z}}$ dym z danych okręgów dokładnie jeden punkt wspólny

(zobacz rysunek). Wykaz, $\displaystyle \dot{\mathrm{z}}\mathrm{e}r=\frac{1}{6}|AB|.$
\begin{center}
\includegraphics[width=91.392mm,height=82.752mm]{./F1_M_PR_M2016_page17_images/image001.eps}
\end{center}
{\it M}

{\it A  K S  L  B}

Strona 18 z24

MMA-IR
\end{document}
