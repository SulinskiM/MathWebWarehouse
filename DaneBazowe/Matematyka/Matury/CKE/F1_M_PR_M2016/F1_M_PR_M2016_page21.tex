\documentclass[a4paper,12pt]{article}
\usepackage{latexsym}
\usepackage{amsmath}
\usepackage{amssymb}
\usepackage{graphicx}
\usepackage{wrapfig}
\pagestyle{plain}
\usepackage{fancybox}
\usepackage{bm}

\begin{document}

{\it Egzamin maturalny z matematyki}

{\it Poziom rozszerzony}

Zadanie 11. (3pkt)

Rozpatrujemy wszystkie liczby naturalne dziesięciocyfrowe, w zapisie których mogą

występować wyłącznie cyfry 1, 2, 3, przy czym cyfra 1 występuje dokładnie trzy razy.

Uzasadnij, $\dot{\mathrm{z}}\mathrm{e}$ takich liczb jest 15360.

Strona 22 z24

MMA-IR
\end{document}
