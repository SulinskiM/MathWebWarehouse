\documentclass[a4paper,12pt]{article}
\usepackage{latexsym}
\usepackage{amsmath}
\usepackage{amssymb}
\usepackage{graphicx}
\usepackage{wrapfig}
\pagestyle{plain}
\usepackage{fancybox}
\usepackage{bm}

\begin{document}

Arkusz zawiera informacje prawnie chronione do momentu rozpoczęcia egzaminu.

UZUPELNIA ZDAJACY

KOD PESEL

{\it miejsce}

{\it na naklejkę}
\begin{center}
\includegraphics[width=21.432mm,height=9.852mm]{./F1_M_PR_M2017_page0_images/image001.eps}

\includegraphics[width=82.092mm,height=9.852mm]{./F1_M_PR_M2017_page0_images/image002.eps}

\includegraphics[width=204.060mm,height=197.820mm]{./F1_M_PR_M2017_page0_images/image003.eps}
\end{center}
EGZAMIN MATU LNY

Z MATEMATYKI

POZIOM ROZSZERZONY

1.

2.

Sprawdzí, czy arkusz egzaminacyjny zawiera 20 stron

(zadania $1-11$). Ewentualny brak zgłoś przewodniczącemu

zespo nadzorującego egzamin.

Rozwiązania zadań i odpowiedzi wpisuj w miejscu na to

przeznaczonym.

Pamiętaj, $\dot{\mathrm{z}}\mathrm{e}$ pominięcie argumentacji lub istotnych

obliczeń w rozwiązaniu zadania otwa ego $\mathrm{m}\mathrm{o}\dot{\mathrm{z}}\mathrm{e}$

spowodować, $\dot{\mathrm{z}}\mathrm{e}$ za to rozwiązanie nie otrzymasz pełnej

liczby punktów.

Pisz czytelnie i uzywaj tvlko długopisu lub -Dióra

z czarnym tuszem lub atramentem.

Nie uzywaj korektora, a błędne zapisy wyrazínie prze eśl.

Pamiętaj, $\dot{\mathrm{z}}\mathrm{e}$ zapisy w brudnopisie nie będą oceniane.

$\mathrm{M}\mathrm{o}\dot{\mathrm{z}}$ esz korzystać z zestawu wzorów matematycznych,

cyrkla i linijki oraz kalkulatora prostego.

Na tej stronie oraz na karcie odpowiedzi wpisz swój

numer PESEL i przyklej naklejkę z kodem.

Nie wpisuj $\dot{\mathrm{z}}$ adnych znaków w części przeznaczonej dla

egzaminatora.

9 MAJA 20I7

3.

Godzina rozpoczęcia:

4.

5.

6.

7.

8.

9.

Czas pracy:

180 minut

Liczba punktów

do uzyskania: 50

$\Vert\Vert\Vert\Vert\Vert\Vert\Vert\Vert\Vert\Vert\Vert\Vert\Vert\Vert\Vert\Vert\Vert\Vert\Vert\Vert\Vert\Vert\Vert\Vert|  \mathrm{M}\mathrm{M}\mathrm{A}-\mathrm{R}1_{-}1\mathrm{P}-172$




{\it Egzamin maturalny z matematyki}

{\it Poziom rozszerzony}

Zadanie l. $(4pkt)$

Rozwiąz nierówność $|x-1|+|x-5|\leq 10-2x.$

Strona 2 z20

MMA-IR





{\it Egzamin maturalny z matematyki}

{\it Poziom rozszerzony}

Zadanie 6. $(3pkt)$

$\mathrm{W}$ trójkącie ostrokątnym $ABC$ bok $AB$ ma długość $c$, długość boku $BC$ jest równa $a$ oraz

$|\neq ABC|=\beta$. Dwusieczna kąta $ABC$ przecina bok $AC$ trójkąta w punkcie $E.$

Wykaz, $\dot{\mathrm{z}}\mathrm{e}$ długość odcinka $BE$ jest równa $\displaystyle \frac{2ac\cdot\cos\frac{\beta}{2}}{a+c}$
\begin{center}
\includegraphics[width=96.012mm,height=17.832mm]{./F1_M_PR_M2017_page10_images/image001.eps}
\end{center}
Wypelnia

egzaminator

5.

3

3

Jzyskana liczba pkt

MMA-IR

Strona ll z20





{\it Egzamin maturalny z matematyki}

{\it Poziom rozszerzony}

Zadanie 7. (4pkt)

Oblicz, ile jest liczb sześciocyfrowych, w których zapisie nie występuje zero, natomiast

występują dwie dziewiątki, jedna szóstka i suma wszystkich cyfrjest równa 30.

Odpowiedzí:

$ 0\neg$trona 1$2\mathrm{z}20$

MMA-IR





{\it Egzamin maturalny z matematyki}

{\it Poziom rozszerzony}

Zadanie 8. (3pkt)

W dwóch pudełkach umieszczono po pięć kul, przy czym w pierwszym pudełku: 2 ku1e białe

i3 ku1e czerwone, a w drugim pudełku: 1 ku1ę białą i 4 ku1e czerwone. Z pierwszego pudełka

losujemy jedną kulę i bez oglądania wkładamy ją do drugiego pudełka. Następnie

losujemyjedną kulę z drugiego pudełka. Oblicz prawdopodobieństwo wylosowania kuli

białej z drugiego pudełka.

Odpowiedzí:
\begin{center}
\includegraphics[width=96.012mm,height=17.784mm]{./F1_M_PR_M2017_page12_images/image001.eps}
\end{center}
Wypelnia

egzaminator

Nr zadania

Maks. liczba kt

7.

4

8.

3

Uzyskana liczba pkt

MMA-IR

Strona 13 z20





{\it Egzamin maturalny z matematyki}

{\it Poziom rozszerzony}

ZadanÍe 9. (6pkt)

W trójkącie równoramiennym wysokość opuszczona na podstawę jest równa 36, a promień

okręgu wpisanego w ten trójkąt jest równy 10. Ob1icz długości boków tego trójkąta i promień

okręgu opisanego na tym trójkącie.

Strona 14 z20

MMA-IR





Odpowied $\acute{\mathrm{z}}$:

{\it Egzamin maturalny z matematyki}

{\it Poziom rozszerzony}
\begin{center}
\includegraphics[width=82.044mm,height=17.784mm]{./F1_M_PR_M2017_page14_images/image001.eps}
\end{center}
Wypelnia

egzamÍnator

Nr zadania

Maks. liczba kt

Uzyskana liczba pkt

MMA-IR

Strona 15 z20





{\it Egzamin maturalny z matematyki}

{\it Poziom rozszerzony}

Zadanie $l0. (6pki)$

Przekątne sąsiednich ścian bocznych prostopadłościanu wychodzące z jednego wierzchołka

tworzą zjego podstawą kąty o miarach $\displaystyle \frac{\pi}{3} \mathrm{i} \alpha$. Cosinus kąta między tymi przekątnymi jest

równy $\displaystyle \frac{\sqrt{6}}{4}$. Wyznacz miarę kąta $a.$

Strona 16 z20

MMA-IR





Odpowied $\acute{\mathrm{z}}$:

{\it Egzamin maturalny z matematyki}

{\it Poziom rozszerzony}
\begin{center}
\includegraphics[width=82.044mm,height=17.784mm]{./F1_M_PR_M2017_page16_images/image001.eps}
\end{center}
Wypelnia

egzamÍnator

Nr zadania

Maks. liczba kt

10.

Uzyskana liczba pkt

MMA-IR

Strona 17 z20





{\it Egzamin maturalny z matematyki}

{\it Poziom rozszerzony}

Zadanie $l1_{1}. (5pkt)$

Wyznacz równanie okręgu przechodzącego przez punkty $A=(-5,3) \mathrm{i} B=(0,6)$, którego

środek lezy na prostej o równaniu $x-3y+1=0.$

Strona 18 z20

MMA-IR





{\it Egzamin maturalny z matematyki}

{\it Poziom rozszerzony}

Odpowied $\acute{\mathrm{z}}$:
\begin{center}
\includegraphics[width=82.044mm,height=17.784mm]{./F1_M_PR_M2017_page18_images/image001.eps}
\end{center}
Wypelnia

egzamÍnator

Nr zadania

Maks. liczba kt

11.

5

Uzyskana liczba pkt

MMA-IR

Strona 19 z20





{\it Egzamin maturalny z matematyki}

{\it Poziom rozszerzony}

{\it BRUDNOPIS} ({\it nie podlega ocenie})

Strona 20 z20

MMA-IR





Odpowied $\acute{\mathrm{z}}$:

{\it Egzamin maturalny z matematyki}

{\it Poziom rozszerzony}
\begin{center}
\includegraphics[width=82.044mm,height=17.784mm]{./F1_M_PR_M2017_page2_images/image001.eps}
\end{center}
Wypelnia

egzamÍnator

Nr zadania

Maks. liczba kt

1.

4

Uzyskana liczba pkt

MMA-IR

Strona 3 z20





{\it Egzamin maturalny z matematyki}

{\it Poziom rozszerzony}

Zadanie 2. $(Spkt)$

Dany jest wielomian $W(x)=2x^{3}+ax^{2}-13x+b$. Liczba 3 jest jednym z pierwiastków tego

wielomianu. Reszta z dzielenia wielomianu $W(x)$ przez $(x+2)$ jest równa 20. Ob1icz

współczynniki $a\mathrm{i}b$ oraz pozostałe pierwiastki wielomianu $W(x).$

Strona 4 z20

MMA-IR





Odpowied $\acute{\mathrm{z}}$:

{\it Egzamin maturalny z matematyki}

{\it Poziom rozszerzony}
\begin{center}
\includegraphics[width=82.044mm,height=17.784mm]{./F1_M_PR_M2017_page4_images/image001.eps}
\end{center}
Wypelnia

egzamÍnator

Nr zadania

Maks. liczba kt

2.

5

Uzyskana liczba pkt

MMA-IR

Strona 5 z20





{\it Egzamin maturalny z matematyki}

{\it Poziom rozszerzony}

Zadanie 3. $(Spkt)$

Wyznacz wszystkie wartości parametru $m$, dla których równanie

$4x^{2}-6mx+(2m+3)(m-3)=0$

ma dwa rózne rozwiązania rzeczywiste $x_{1}$ i $x_{2}$, przy czym $x_{1}<x_{2}$, spełniające warunek

$(4x_{1}-4x_{2}-1)(4x_{1}-4x_{2}+1)<0.$

Strona 6 z20

MMA-IR





Odpowied $\acute{\mathrm{z}}$:

{\it Egzamin maturalny z matematyki}

{\it Poziom rozszerzony}
\begin{center}
\includegraphics[width=82.044mm,height=17.784mm]{./F1_M_PR_M2017_page6_images/image001.eps}
\end{center}
Wypelnia

egzamÍnator

Nr zadania

Maks. liczba kt

3.

5

Uzyskana liczba pkt

MMA-IR

Strona 7 z20





{\it Egzamin maturalny z matematyki}

{\it Poziom rozszerzony}

Zadanie 4. $(6pkt)$

Liczby $a, b, c$ są - odpowiednio - pierwszym, drugim i trzecim wyrazem ciągu

arytmetycznego. Suma tych liczb jest równa 27. Ciąg $(a-2,b,2c+1)$ jest geometryczny.

Wyznacz liczby $a, b, c.$

Strona 8 z20

MMA-IR





Odpowied $\acute{\mathrm{z}}$:

{\it Egzamin maturalny z matematyki}

{\it Poziom rozszerzony}
\begin{center}
\includegraphics[width=82.044mm,height=17.784mm]{./F1_M_PR_M2017_page8_images/image001.eps}
\end{center}
Wypelnia

egzamÍnator

Nr zadania

Maks. liczba kt

4.

Uzyskana liczba pkt

MMA-IR

Strona 9 z20





{\it Egzamin maturalny z matematyki}

{\it Poziom rozszerzony}

Zadanie 5. $(3pkt)$

Udowodnij, $\dot{\mathrm{z}}\mathrm{e}$ dla dowolnych róznych liczb rzeczywistych $x, y$ prawdziwajest nierówność

$x^{2}y^{2}+2x^{2}+2y^{2}-8xy+4>0.$

Strona 10 z20

MMA-IR



\end{document}