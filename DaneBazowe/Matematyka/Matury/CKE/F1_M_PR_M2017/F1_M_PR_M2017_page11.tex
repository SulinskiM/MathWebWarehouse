\documentclass[a4paper,12pt]{article}
\usepackage{latexsym}
\usepackage{amsmath}
\usepackage{amssymb}
\usepackage{graphicx}
\usepackage{wrapfig}
\pagestyle{plain}
\usepackage{fancybox}
\usepackage{bm}

\begin{document}

{\it Egzamin maturalny z matematyki}

{\it Poziom rozszerzony}

Zadanie 7. (4pkt)

Oblicz, ile jest liczb sześciocyfrowych, w których zapisie nie występuje zero, natomiast

występują dwie dziewiątki, jedna szóstka i suma wszystkich cyfrjest równa 30.

Odpowiedzí:

$ 0\neg$trona 1$2\mathrm{z}20$

MMA-IR
\end{document}
