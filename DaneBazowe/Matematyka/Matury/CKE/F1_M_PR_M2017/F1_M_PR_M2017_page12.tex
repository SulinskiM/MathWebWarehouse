\documentclass[a4paper,12pt]{article}
\usepackage{latexsym}
\usepackage{amsmath}
\usepackage{amssymb}
\usepackage{graphicx}
\usepackage{wrapfig}
\pagestyle{plain}
\usepackage{fancybox}
\usepackage{bm}

\begin{document}

{\it Egzamin maturalny z matematyki}

{\it Poziom rozszerzony}

Zadanie 8. (3pkt)

W dwóch pudełkach umieszczono po pięć kul, przy czym w pierwszym pudełku: 2 ku1e białe

i3 ku1e czerwone, a w drugim pudełku: 1 ku1ę białą i 4 ku1e czerwone. Z pierwszego pudełka

losujemy jedną kulę i bez oglądania wkładamy ją do drugiego pudełka. Następnie

losujemyjedną kulę z drugiego pudełka. Oblicz prawdopodobieństwo wylosowania kuli

białej z drugiego pudełka.

Odpowiedzí:
\begin{center}
\includegraphics[width=96.012mm,height=17.784mm]{./F1_M_PR_M2017_page12_images/image001.eps}
\end{center}
Wypelnia

egzaminator

Nr zadania

Maks. liczba kt

7.

4

8.

3

Uzyskana liczba pkt

MMA-IR

Strona 13 z20
\end{document}
