\documentclass[a4paper,12pt]{article}
\usepackage{latexsym}
\usepackage{amsmath}
\usepackage{amssymb}
\usepackage{graphicx}
\usepackage{wrapfig}
\pagestyle{plain}
\usepackage{fancybox}
\usepackage{bm}

\begin{document}

CENTRALNA

KOMISJA

EGZAMINACYJNA

Arkusz zawiera informacje prawnie chronione do momentu rozpoczęcia egzaminu.

UZUPELNIA ZDAJACY

KOD PESEL

{\it miejsce}

{\it na naklejkę}
\begin{center}
\includegraphics[width=21.432mm,height=9.852mm]{./F1_M_PR_M2018_page0_images/image001.eps}

\includegraphics[width=82.140mm,height=9.852mm]{./F1_M_PR_M2018_page0_images/image002.eps}

\includegraphics[width=204.060mm,height=197.820mm]{./F1_M_PR_M2018_page0_images/image003.eps}
\end{center}
EGZAMIN MATU LNY

Z MATEMATYKI

POZIOM ROZSZERZONY

1.

2.

Sprawdzí, czy arkusz egzaminacyjny zawiera 20 stron

(zadania $1-11$). Ewenmalny brak zgłoś przewodniczącemu

zespo nadzorującego egzamin.

Rozwiązania zadań i odpowiedzi wpisuj w miejscu na to

przeznaczonym.

Pamiętaj, $\dot{\mathrm{z}}\mathrm{e}$ pominięcie argumentacji lub istotnych

obliczeń w rozwiązaniu zadania otwa ego $\mathrm{m}\mathrm{o}\dot{\mathrm{z}}\mathrm{e}$

spowodować, $\dot{\mathrm{z}}\mathrm{e}$ za to rozwiązanie nie otrzymasz pełnej

liczby punktów.

Pisz czytelnie i uzywaj tvlko długopisu lub -Dióra

z czarnym tuszem lub atramentem.

Nie uzywaj korektora, a błędne zapisy wyrazínie prze eśl.

Pamiętaj, $\dot{\mathrm{z}}\mathrm{e}$ zapisy w brudnopisie nie będą oceniane.

$\mathrm{M}\mathrm{o}\dot{\mathrm{z}}$ esz korzystać z zestawu wzorów matematycznych,

cyrkla i linijki oraz kalkulatora prostego.

Na tej stronie oraz na karcie odpowiedzi wpisz swój

numer PESEL i przyklej naklejkę z kodem.

Nie wpisuj $\dot{\mathrm{z}}$ adnych znaków w części przeznaczonej dla

egzaminatora.

9 MAJA 20I8

3.

Godzina rozpoczęcia:

4.

5.

6.

7.

8.

9.

Czas pracy:

180 minut

Liczba punktów

do uzyskania: 50

$\Vert\Vert\Vert\Vert\Vert\Vert\Vert\Vert\Vert\Vert\Vert\Vert\Vert\Vert\Vert\Vert\Vert\Vert\Vert\Vert\Vert\Vert\Vert\Vert|  \mathrm{M}\mathrm{M}\mathrm{A}-\mathrm{R}1_{-}1\mathrm{P}-182$




{\it Egzamin maturalny z matematyki}

{\it Poziom rozszerzony}

Zadanie l.$(4pkt)$

Rozwiąz równanie $3|x+2|=|x-3|+11.$

Strona 2 z20

$\mathrm{M}\mathrm{M}_{p}$





{\it Egzamin maturalny z matematyki}

{\it Poziom rozszerzony}

Zadanie 6. $(3pkt)$

Udowodnij, $\dot{\mathrm{z}}\mathrm{e}$ dla $\mathrm{k}\mathrm{a}\dot{\mathrm{z}}$ dej liczby całkowitej $k$ i dla $\mathrm{k}\mathrm{a}\dot{\mathrm{z}}$ dej liczby całkowitej $m$ liczba $k^{3}m-km^{3}$

jest podzielna przez 6.
\begin{center}
\includegraphics[width=96.012mm,height=17.784mm]{./F1_M_PR_M2018_page10_images/image001.eps}
\end{center}
Wypelnia

egzaminator

Nr zadania

Maks. liczba kt

5.

3

3

Uzyskana liczba pkt

MMA-IR

Strona ll z20





{\it Egzamin maturalny z matematyki}

{\it Poziom rozszerzony}

Zadanie 7. $(4pkt)$

Rozwiąz równanie $2\cos^{2}x+3\sin x=0$ w przedziale $\displaystyle \{-\frac{\pi}{2},\frac{3\pi}{2}\}.$

Odpowied $\acute{\mathrm{z}}$:

Strona 12 z20

MMA-IR





{\it Egzamin maturalny z matematyki}

{\it Poziom rozszerzony}

Zadanie 8. $(5pkt)$

Liczba $\displaystyle \frac{2}{5}$ jest pierwiastkiem wielomianu $W(x)=5x^{3}-7x^{2}-3x+p$. Wyznacz pozostałe

pierwiastki tego wielomianu i rozwiąz nierówność $W(x)>0.$

Odpowied $\acute{\mathrm{z}}$:
\begin{center}
\includegraphics[width=96.012mm,height=17.784mm]{./F1_M_PR_M2018_page12_images/image001.eps}
\end{center}
Wypelnia

egzaminator

Nr zadania

Maks. liczba kt

7.

4

8.

5

Uzyskana liczba pkt

MMA-IR

Strona 13 z20





{\it Egzamin maturalny z matematyki}

{\it Poziom rozszerzony}

ZadanÍe 9. $(6pkt)$

Wyznacz wszystkie wartości parametru $m$, dla których równanie $x^{2}+(m+1)x-m^{2}+1=0$ ma

dwa rozwiązania rzeczywiste $x_{1} \mathrm{i}x_{2}(x_{1}\neq x_{2})$, spełniające waiunek $x_{1}^{3}+x_{2}^{3}>-7x_{1}x_{2}.$

Strona 14 z20

MMA-IR





{\it Egzamin maturalny z matematyki}

{\it Poziom rozszerzony}

Odpowiedzí :
\begin{center}
\includegraphics[width=82.044mm,height=17.832mm]{./F1_M_PR_M2018_page14_images/image001.eps}
\end{center}
Wypelnia

egzaminator

Nr zadania

Maks. liczba kt

Uzyskana liczba pkt

MMA-IR

Strona 15 z20





{\it Egzamin maturalny z matematyki}

{\it Poziom rozszerzony}

Zadanie $l0. (6pki)$

Punkt $A=(7,-1)$ jest wierzchołkiem trójkąta równoramiennego $ABC$, w którym $|AC|=|BC|.$

Obie współrzędne wierzchołka $C$ są liczbami ujemnymi. Okrąg wpisany w trójkąt $ABC$ ma

równanie $x^{2}+y^{2}=10$. Oblicz współrzędne wierzchołków $B\mathrm{i}C$ tego trójkąta.

Strona 16 z20

MMA-IR





{\it Egzamin maturalny z matematyki}

{\it Poziom rozszerzony}

Odpowiedzí :
\begin{center}
\includegraphics[width=82.044mm,height=17.832mm]{./F1_M_PR_M2018_page16_images/image001.eps}
\end{center}
Wypelnia

egzaminator

Nr zadania

Maks. liczba kt

10.

Uzyskana liczba pkt

MMA-IR

Strona 17 z20





{\it Egzamin maturalny z matematyki}

{\it Poziom rozszerzony}

Zadanie $l1. (5pktJ$

Przekrój ostrosłupa prawidłowego trójkątnego ABCS płaszczyzną przechodzącą przez

wierzchołek $S$ i wysokości dwóch ścian bocznych jest trójkątem równobocznym. Krawędzí

boczna tego ostrosłupa ma długość $\displaystyle \frac{4\sqrt{3}}{3}$. Oblicz objętość tego ostrosłupa.

Strona 18 z20

MMA-IR





{\it Egzamin maturalny z matematyki}

{\it Poziom rozszerzony}

Odpowiedzí :
\begin{center}
\includegraphics[width=82.044mm,height=17.832mm]{./F1_M_PR_M2018_page18_images/image001.eps}
\end{center}
Wypelnia

egzaminator

Nr zadania

Maks. liczba kt

11.

5

Uzyskana liczba pkt

MMA-IR

Strona 19 z20





{\it Egzamin maturalny z matematyki}

{\it Poziom rozszerzony}

{\it BRUDNOPIS} ({\it nie podlega ocenie})

Strona 20 z20

MM





{\it Egzamin maturalny z matematyki}

{\it Poziom rozszerzony}

Odpowiedzí:
\begin{center}
\includegraphics[width=82.044mm,height=17.832mm]{./F1_M_PR_M2018_page2_images/image001.eps}
\end{center}
Wypelnia

egzaminator

Nr zadania

Maks. liczba kt

1.

4

Uzyskana liczba pkt

MMA-IR

$\urcorner$trona 3$\mathrm{z}20$





{\it Egzamin maturalny z matematyki}

{\it Poziom rozszerzony}

Zadanie 2. $(Spkt)$

Liczby $a, b, c$, spełniające warunek $3a+b+3c=77$, są odpowiednio pierwszym, drugim

i trzecim wyrazem ciągu arytmetycznego. Ciąg $(a,b+1,2c)$ jest geometryczny. Wyznacz

liczby $a, b, c$ oraz podaj wyrazy ciągu geometrycznego.

Strona 4 z20

MMA-IR





{\it Egzamin maturalny z matematyki}

{\it Poziom rozszerzony}

Odpowiedzí:
\begin{center}
\includegraphics[width=82.044mm,height=17.832mm]{./F1_M_PR_M2018_page4_images/image001.eps}
\end{center}
Wypelnia

egzaminator

Nr zadania

Maks. liczba kt

2.

5

Uzyskana liczba pkt

MMA-IR

$\urcorner$trona 5$\mathrm{z}20$





{\it Egzamin maturalny z matematyki}

{\it Poziom rozszerzony}

Zadanie 3. (Spkt)

Dany jest czworokąt wypukły ABCD, w którym

$|AD|=|AB|=|BC|=a, |\triangleleft BAD|=60^{\mathrm{o}}$

$\mathrm{i}|4ADC|=135^{\mathrm{o}}$. Oblicz pole czworokąta ABCD.

Strona 6 z20

MMA-IR





{\it Egzamin maturalny z matematyki}

{\it Poziom rozszerzony}

Odpowiedzí:
\begin{center}
\includegraphics[width=82.044mm,height=17.832mm]{./F1_M_PR_M2018_page6_images/image001.eps}
\end{center}
Wypelnia

egzaminator

Nr zadania

Maks. liczba kt

3.

5

Uzyskana liczba pkt

MMA-IR

$\urcorner$trona 7$\mathrm{z}20$





{\it Egzamin maturalny z matematyki}

{\it Poziom rozszerzony}

Zadanie 4. $(4pkt)$

$\mathrm{Z}$ liczb ośmioelementowego zbioru $Z=\{1$, 2, 3, 4, 5, 6, 7, 9$\}$ tworzymy ośmiowyrazowy ciąg,

którego wyrazy nie powtarzają się. Oblicz prawdopodobieństwo zdarzenia polegającego na

tym, $\dot{\mathrm{z}}\mathrm{e}\dot{\mathrm{z}}$ adne dwie liczby parzyste nie są sąsiednimi wyrazami utworzonego ciągu. Wynik

przedstaw w postaci ułamka zwykłego nieskracalnego.

Strona 8 z20

MMA-IR





{\it Egzamin maturalny z matematyki}

{\it Poziom rozszerzony}

Odpowiedzí:
\begin{center}
\includegraphics[width=82.044mm,height=17.832mm]{./F1_M_PR_M2018_page8_images/image001.eps}
\end{center}
Wypelnia

egzaminator

Nr zadania

Maks. liczba kt

4.

4

Uzyskana liczba pkt

MMA-IR

$\urcorner$trona 9$\mathrm{z}20$





{\it Egzamin maturalny z matematyki}

{\it Poziom rozszerzony}

Zadanie 5. $(3pkt)$

Trójkąt $ABC$ jest ostrokątny oraz $|AC|>|BC|$. Dwusieczna $d_{c}$ kąta $ACB$ przecina bok $AB$

w punkcie $K$. Punkt $L$ jest obrazem punktu $K$ w symetrii osiowej względem dwusiecznej $d_{A}$

kąta $BAC$, punkt Mjest obrazem punktu $L$ w symetrii osiowej względem dwusiecznej $d_{c}$ kąta

$ACB$, a punkt $N$ jest obrazem punktu $M$ w symetrii osiowej względem dwusiecznej $d_{B}$ kąta

$ABC$ (zobacz rysunek).
\begin{center}
\includegraphics[width=88.344mm,height=83.976mm]{./F1_M_PR_M2018_page9_images/image001.eps}
\end{center}
{\it C}

{\it L}

{\it M}

{\it A  K N  B}

Udowodnij, $\dot{\mathrm{z}}\mathrm{e}$ na czworokącie KNML mozna opisać okrąg.

Strona 10 z20

MMA-IR



\end{document}