\documentclass[a4paper,12pt]{article}
\usepackage{latexsym}
\usepackage{amsmath}
\usepackage{amssymb}
\usepackage{graphicx}
\usepackage{wrapfig}
\pagestyle{plain}
\usepackage{fancybox}
\usepackage{bm}

\begin{document}

{\it Egzamin maturalny z matematyki}

{\it Poziom rozszerzony}

ZadanÍe 9. $(6pkt)$

Wyznacz wszystkie wartości parametru $m$, dla których równanie $x^{2}+(m+1)x-m^{2}+1=0$ ma

dwa rozwiązania rzeczywiste $x_{1} \mathrm{i}x_{2}(x_{1}\neq x_{2})$, spełniające waiunek $x_{1}^{3}+x_{2}^{3}>-7x_{1}x_{2}.$

Strona 14 z20

MMA-IR
\end{document}
