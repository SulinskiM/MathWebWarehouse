\documentclass[a4paper,12pt]{article}
\usepackage{latexsym}
\usepackage{amsmath}
\usepackage{amssymb}
\usepackage{graphicx}
\usepackage{wrapfig}
\pagestyle{plain}
\usepackage{fancybox}
\usepackage{bm}

\begin{document}

{\it Egzamin maturalny z matematyki}

{\it Poziom rozszerzony}

Zadanie 4. $(4pkt)$

$\mathrm{Z}$ liczb ośmioelementowego zbioru $Z=\{1$, 2, 3, 4, 5, 6, 7, 9$\}$ tworzymy ośmiowyrazowy ciąg,

którego wyrazy nie powtarzają się. Oblicz prawdopodobieństwo zdarzenia polegającego na

tym, $\dot{\mathrm{z}}\mathrm{e}\dot{\mathrm{z}}$ adne dwie liczby parzyste nie są sąsiednimi wyrazami utworzonego ciągu. Wynik

przedstaw w postaci ułamka zwykłego nieskracalnego.

Strona 8 z20

MMA-IR
\end{document}
