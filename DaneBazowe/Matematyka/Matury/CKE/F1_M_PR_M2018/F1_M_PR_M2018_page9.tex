\documentclass[a4paper,12pt]{article}
\usepackage{latexsym}
\usepackage{amsmath}
\usepackage{amssymb}
\usepackage{graphicx}
\usepackage{wrapfig}
\pagestyle{plain}
\usepackage{fancybox}
\usepackage{bm}

\begin{document}

{\it Egzamin maturalny z matematyki}

{\it Poziom rozszerzony}

Zadanie 5. $(3pkt)$

Trójkąt $ABC$ jest ostrokątny oraz $|AC|>|BC|$. Dwusieczna $d_{c}$ kąta $ACB$ przecina bok $AB$

w punkcie $K$. Punkt $L$ jest obrazem punktu $K$ w symetrii osiowej względem dwusiecznej $d_{A}$

kąta $BAC$, punkt Mjest obrazem punktu $L$ w symetrii osiowej względem dwusiecznej $d_{c}$ kąta

$ACB$, a punkt $N$ jest obrazem punktu $M$ w symetrii osiowej względem dwusiecznej $d_{B}$ kąta

$ABC$ (zobacz rysunek).
\begin{center}
\includegraphics[width=88.344mm,height=83.976mm]{./F1_M_PR_M2018_page9_images/image001.eps}
\end{center}
{\it C}

{\it L}

{\it M}

{\it A  K N  B}

Udowodnij, $\dot{\mathrm{z}}\mathrm{e}$ na czworokącie KNML mozna opisać okrąg.

Strona 10 z20

MMA-IR
\end{document}
