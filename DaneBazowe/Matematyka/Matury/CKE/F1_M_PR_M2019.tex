\documentclass[a4paper,12pt]{article}
\usepackage{latexsym}
\usepackage{amsmath}
\usepackage{amssymb}
\usepackage{graphicx}
\usepackage{wrapfig}
\pagestyle{plain}
\usepackage{fancybox}
\usepackage{bm}

\begin{document}

CENTRALNA

KOMfSJA

EGZAMiNACYJNA

Arkusz zawiera informacje prawnie chronione do momentu rozpoczęcia egzaminu.

UZUPELNIA ZDAJACY

KOD PESEL

{\it miejsce}

{\it na naklejkę}
\begin{center}
\includegraphics[width=21.432mm,height=9.852mm]{./F1_M_PR_M2019_page0_images/image001.eps}

\includegraphics[width=82.140mm,height=9.852mm]{./F1_M_PR_M2019_page0_images/image002.eps}

\includegraphics[width=204.060mm,height=197.868mm]{./F1_M_PR_M2019_page0_images/image003.eps}
\end{center}
EGZAMIN MATU LNY

Z MATEMATYKI

POZIOM ROZSZERZONY

Instrukcja dla zdającego

1.

3.

Sprawd $\acute{\mathrm{z}}$, czy arkusz egzaminacyjny zawiera 24 strony

(zadania $1-11$). Ewentualny brak zgłoś przewodniczącemu

zespo nadzorującego egzamin.

Rozwiązania zadań i odpowiedzi wpisuj w miejscu na to

przeznaczonym.

Pamiętaj, $\dot{\mathrm{z}}\mathrm{e}$ pominięcie argumentacji lub istotnych

obliczeń w rozwiązaniu zadania otwa ego $\mathrm{m}\mathrm{o}\dot{\mathrm{z}}\mathrm{e}$

spowodować, $\dot{\mathrm{z}}\mathrm{e}$ za to rozwiązanie nie otrzymasz pełnej

liczby punktów.

Pisz czytelnie i uzywaj tvlko długopisu lub -Dióra

z czarnym tuszem lub atramentem.

Nie uzywaj korektora, a błędne zapisy wyra $\acute{\mathrm{z}}\mathrm{n}\mathrm{i}\mathrm{e}$ prze eśl.

Pamiętaj, $\dot{\mathrm{z}}\mathrm{e}$ zapisy w brudnopisie nie będą oceniane.

$\mathrm{M}\mathrm{o}\dot{\mathrm{z}}$ esz korzystać z zestawu wzorów matematycznych,

cyrkla i linijki oraz kalkulatora prostego.

Na tej stronie oraz na karcie odpowiedzi wpisz swój

numer PESEL i przyklej naklejkę z kodem.

Nie wpisuj $\dot{\mathrm{z}}$ adnych znaków w części przeznaczonej dla

egzaminatora.

Godzina rozpoczęcia:

9:00

4.

5.

6.

7.

8.

9.

Czas pracy:

180 minut

Liczba punktów

do uzyskania: 50

$\Vert\Vert\Vert\Vert\Vert\Vert\Vert\Vert\Vert\Vert\Vert\Vert\Vert\Vert\Vert\Vert\Vert\Vert\Vert\Vert\Vert\Vert\Vert\Vert|  \mathrm{M}\mathrm{M}\mathrm{A}-\mathrm{R}1_{-}1\mathrm{P}-192$




{\it Egzamin maturalny z matematyki}

{\it Poziom rozszerzony}

Zadanie l. $(5pkt)$

Funkcja $f$ jest określona wzorem $f(x)=\displaystyle \frac{|x+2|}{x+2}-x+3|x-1|$, dla $\mathrm{k}\mathrm{a}\dot{\mathrm{z}}$ dej liczby rzeczywistej

$x\neq-2$. Wyznacz zbiór wartości tej ffinkcji.

Strona 2 z24

MMA-IR





Odpowied $\acute{\mathrm{z}}$:

{\it Egzamin maturalny z matematyki}

{\it Poziom rozszerzony}
\begin{center}
\includegraphics[width=82.044mm,height=17.784mm]{./F1_M_PR_M2019_page10_images/image001.eps}
\end{center}
Wypelnia

egzamÍnator

Nr zadania

Maks. liczba kt

5.

Uzyskana liczba pkt

MMA-IR

Strona ll z24





{\it Egzamin maturalny z matematyki}

{\it Poziom rozszerzony}

Zadanie 6. $(5pkt)$

Wielomian określony wzorem $W(x)=2x^{3}+(m^{3}+2)x^{2}-11x-2(2m+1)$ jest podzielny przez

dwumian $(x-2)$ oraz przy dzieleniu przez dwumian $(x+1)$ daje resztę 6. Ob1icz $m$ oraz

pierwiastki wielomianu $W$ dla wyznaczonej wartości $m.$

Strona 12 z24

MMA-IR





Odpowied $\acute{\mathrm{z}}$:

{\it Egzamin maturalny z matematyki}

{\it Poziom rozszerzony}
\begin{center}
\includegraphics[width=82.044mm,height=17.784mm]{./F1_M_PR_M2019_page12_images/image001.eps}
\end{center}
Wypelnia

egzamÍnator

Nr zadania

Maks. liczba kt

5

Uzyskana liczba pkt

MMA-IR

Strona 13 z24





{\it Egzamin maturalny z matematyki}

{\it Poziom rozszerzony}

Zadanie 7. $(4pkt)$

Rozwiąz równanie $\cos 2x=\sin x+1$ w przedziale $\langle 0,2\pi\rangle.$

Strona 14 z24





Odpowiedzí:

{\it Egzamin maturalny z matematyki}

{\it Poziom rozszerzony}
\begin{center}
\includegraphics[width=82.044mm,height=17.832mm]{./F1_M_PR_M2019_page14_images/image001.eps}
\end{center}
Wypelnia

egzaminator

Nr zadania

Maks. liczba kt

7.

4

Uzyskana liczba pkt

MMA-IR

Strona 15 z24





{\it Egzamin maturalny z matematyki}

{\it Poziom rozszerzony}

Zadanie 8. $(4pkt)$

Punkt $D$ lezy na boku $AB$ trójkąta $ABC$ oraz $|AC|=16, |AD|=6, |CD|=14 \mathrm{i} |BC|=|BD|.$

Oblicz obwód trójkąta $ABC.$

Strona 16 z24

MMA-IR





Odpowied $\acute{\mathrm{z}}$:

{\it Egzamin maturalny z matematyki}

{\it Poziom rozszerzony}
\begin{center}
\includegraphics[width=82.044mm,height=17.784mm]{./F1_M_PR_M2019_page16_images/image001.eps}
\end{center}
Wypelnia

egzamÍnator

Nr zadania

Maks. liczba kt

8.

4

Uzyskana liczba pkt

MMA-IR

Strona 17 z24





{\it Egzamin maturalny z matematyki}

{\it Poziom rozszerzony}

Zadanie 9. $(6pkt)$

Wyznacz wszystkie wartości parametru $m$, dla których funkcja kwadratowa

wzorem

$f(x)=(2m+1)x^{2}+(m+2)x+m-3$

f określona

ma dwa rózne pierwiastki rzeczywiste $x_{1}, x_{2}$ spełniające warunek $(x_{1}-x_{2})^{2}+5x_{1}x_{2}\geq 1.$

Strona 18 z24

MMA-IR





Odpowiedzí:

{\it Egzamin maturalny z matematyki}

{\it Poziom rozszerzony}
\begin{center}
\includegraphics[width=82.044mm,height=17.832mm]{./F1_M_PR_M2019_page18_images/image001.eps}
\end{center}
Wypelnia

egzaminator

Nr zadania

Maks. liczba kt

Uzyskana liczba pkt

MMA-IR

Strona 19 z24





{\it Egzamin maturalny z matematyki}

{\it Poziom rozszerzony}

Zadanie 10. $(3pkt)$

Ze zbioru \{1, 2, 3, 4, 5, 6, 7, 8, 9\} 1osujemy ko1ejno ze zwracaniem trzy 1iczby. Ob1icz

prawdopodobieństwo zdarzenia polegającego na tym, $\dot{\mathrm{z}}\mathrm{e}$ dokładnie dwie spośród trzech

wylosowanych liczb będą równe. Wynik zapisz w postaci ułamka nieskracalnego.

Strona 20 z24

MMA-IR





Odpowiedzí:

{\it Egzamin maturalny z matematyki}

{\it Poziom rozszerzony}
\begin{center}
\includegraphics[width=82.044mm,height=17.784mm]{./F1_M_PR_M2019_page2_images/image001.eps}
\end{center}
Wypelnia

egzamÍnator

Nr zadania

Maks. liczba kt

1.

5

Uzyskana liczba pkt

MMA-IR

Strona 3 z24





Odpowiedzí:

{\it Egzamin maturalny z matematyki}

{\it Poziom rozszerzony}
\begin{center}
\includegraphics[width=82.044mm,height=17.832mm]{./F1_M_PR_M2019_page20_images/image001.eps}
\end{center}
Wypelnia

egzaminator

Nr zadania

Maks. liczba kt

10.

3

Uzyskana liczba pkt

MMA-IR

Strona 21 z24





{\it Egzamin maturalny z matematyki}

{\it Poziom rozszerzony}

Zadanie ll. $(6pkt)$

Podstawą ostrosłupa ABCDS jest prostokąt ABCD, którego boki mają długości $|AB|=32$

$\mathrm{i}|BC|=18$. Ściany boczne $ABS\mathrm{i}CDS$ są trójkątami przystającymi i $\mathrm{k}\mathrm{a}\dot{\mathrm{z}}$ da z nichjest nachylona

do płaszczyzny podstawy ostrosłupa pod kątem $\alpha$. Ściany boczne $BCS\mathrm{i}ADS$ są trójkątami

przystającymi i $\mathrm{k}\mathrm{a}\dot{\mathrm{z}}$ da z nich jest nachylona do płaszczyzny podstawy pod kątem $\beta$. Miary

kątów $\alpha \mathrm{i} \beta$ spełniają warunek: $\alpha+\beta=90^{\mathrm{o}}$ Oblicz pole powierzchni całkowitej tego

ostrosłupa.

Strona 22 z24

MMA-IR





Odpowiedzí:

{\it Egzamin maturalny z matematyki}

{\it Poziom rozszerzony}
\begin{center}
\includegraphics[width=82.044mm,height=17.832mm]{./F1_M_PR_M2019_page22_images/image001.eps}
\end{center}
Wypelnia

egzaminator

Nr zadania

Maks. liczba kt

11.

Uzyskana liczba pkt

MMA-IR

Strona 23 z24





{\it Egzamin maturalny z matematyki}

{\it Poziom rozszerzony}

{\it BRUDNOPIS} ({\it nie podlega ocenie})

Strona 24 z24





{\it Egzamin maturalny z matematyki}

{\it Poziom rozszerzony}

Zadanie 2. $(3pkt)$

Udowodnij, $\dot{\mathrm{z}}\mathrm{e}$ dla dowolnych dodatnich liczb rzeczywistych $x\mathrm{i}y$, takich $\dot{\mathrm{z}}\mathrm{e}x<y$, i dowolnej

dodatniej liczby rzeczywistej $a$ prawdziwajest nierówność $\displaystyle \frac{x+a}{y+a}+\frac{y}{x}>2.$

Strona 4 z24

MMA-IR





{\it Egzamin maturalny z matematyki}

{\it Poziom rozszerzony}
\begin{center}
\includegraphics[width=82.044mm,height=17.784mm]{./F1_M_PR_M2019_page4_images/image001.eps}
\end{center}
Wypelnia

egzamÍnator

Nr zadania

Maks. liczba kt

2.

3

Uzyskana liczba pkt

Strona 5 z24





{\it Egzamin maturalny z matematyki}

{\it Poziom rozszerzony}

Zadanie 3. $(3pkt)$

Dany jest trójkąt równoramienny $ABC$, w którym $|AC|=|BC|$. Na ramieniu $AC$ tego trójkąta

wybrano punkt $M(M\neq A\mathrm{i}M\neq C)$, a na ramieniu $BC$ wybrano punkt $N$, w taki sposób, $\dot{\mathrm{z}}\mathrm{e}$

$|AM|=|CN|$. Przez punkty $M\mathrm{i}N$ poprowadzono proste prostopadłe do podstawy $AB$ tego

trójkąta, które wyznaczają na niej punkty $S\mathrm{i}T$. Udowodnij, $\displaystyle \dot{\mathrm{z}}\mathrm{e}|ST|=\frac{1}{2}|AB|.$

Strona 6 z24

MMA-IR





{\it Egzamin maturalny z matematyki}

{\it Poziom rozszerzony}
\begin{center}
\includegraphics[width=82.044mm,height=17.784mm]{./F1_M_PR_M2019_page6_images/image001.eps}
\end{center}
Nr zadania

Wypelnia Maks. liczba kt

egzaminator

Uzyskana liczba pkt

3.

3

Strona 7 z24





{\it Egzamin maturalny z matematyki}

{\it Poziom rozszerzony}

Zadanie 4. $(5pkt)$

Ciąg $(a,b,c)$ jest geometryczny, ciąg $(a+1,b+5,c)$ jest malejącym ciągiem arytmetycznym

oraz $a+b+c=39$. Oblicz $a, b, c.$

Strona 8 z24

MMA-IR





Odpowied $\acute{\mathrm{z}}$:

{\it Egzamin maturalny z matematyki}

{\it Poziom rozszerzony}
\begin{center}
\includegraphics[width=82.044mm,height=17.784mm]{./F1_M_PR_M2019_page8_images/image001.eps}
\end{center}
Nr zadania

Wypelnia Maks. liczba kt

egzamÍnator

Uzyskana liczba pkt

4.

5

MMA-IR

Strona 9 z24





{\it Egzamin maturalny z matematyki}

{\it Poziom rozszerzony}

Zadanie 5. $(6pkt)$

Dane są okręgi o równaniach $x^{2}+y^{2}-12x-8y+43=0 \mathrm{i} x^{2}+y^{2}-2ax+4y+a^{2}-77=0.$

Wyznacz wszystkie wartości parametru $a$, dla których te okręgi mają dokładnie jeden punkt

wspólny. Rozwaz wszystkie przypadki.

Strona 10 z24

MMA-IR



\end{document}