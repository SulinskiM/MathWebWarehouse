\documentclass[a4paper,12pt]{article}
\usepackage{latexsym}
\usepackage{amsmath}
\usepackage{amssymb}
\usepackage{graphicx}
\usepackage{wrapfig}
\pagestyle{plain}
\usepackage{fancybox}
\usepackage{bm}

\begin{document}

{\it Egzamin maturalny z matematyki}

{\it Poziom rozszerzony}

Zadanie 9. $(6pkt)$

Wyznacz wszystkie wartości parametru $m$, dla których funkcja kwadratowa

wzorem

$f(x)=(2m+1)x^{2}+(m+2)x+m-3$

f określona

ma dwa rózne pierwiastki rzeczywiste $x_{1}, x_{2}$ spełniające warunek $(x_{1}-x_{2})^{2}+5x_{1}x_{2}\geq 1.$

Strona 18 z24

MMA-IR
\end{document}
