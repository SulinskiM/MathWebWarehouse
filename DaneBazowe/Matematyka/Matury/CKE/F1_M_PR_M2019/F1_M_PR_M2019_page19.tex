\documentclass[a4paper,12pt]{article}
\usepackage{latexsym}
\usepackage{amsmath}
\usepackage{amssymb}
\usepackage{graphicx}
\usepackage{wrapfig}
\pagestyle{plain}
\usepackage{fancybox}
\usepackage{bm}

\begin{document}

{\it Egzamin maturalny z matematyki}

{\it Poziom rozszerzony}

Zadanie 10. $(3pkt)$

Ze zbioru \{1, 2, 3, 4, 5, 6, 7, 8, 9\} 1osujemy ko1ejno ze zwracaniem trzy 1iczby. Ob1icz

prawdopodobieństwo zdarzenia polegającego na tym, $\dot{\mathrm{z}}\mathrm{e}$ dokładnie dwie spośród trzech

wylosowanych liczb będą równe. Wynik zapisz w postaci ułamka nieskracalnego.

Strona 20 z24

MMA-IR
\end{document}
