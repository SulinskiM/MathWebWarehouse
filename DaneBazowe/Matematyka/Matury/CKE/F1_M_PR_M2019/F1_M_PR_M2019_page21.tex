\documentclass[a4paper,12pt]{article}
\usepackage{latexsym}
\usepackage{amsmath}
\usepackage{amssymb}
\usepackage{graphicx}
\usepackage{wrapfig}
\pagestyle{plain}
\usepackage{fancybox}
\usepackage{bm}

\begin{document}

{\it Egzamin maturalny z matematyki}

{\it Poziom rozszerzony}

Zadanie ll. $(6pkt)$

Podstawą ostrosłupa ABCDS jest prostokąt ABCD, którego boki mają długości $|AB|=32$

$\mathrm{i}|BC|=18$. Ściany boczne $ABS\mathrm{i}CDS$ są trójkątami przystającymi i $\mathrm{k}\mathrm{a}\dot{\mathrm{z}}$ da z nichjest nachylona

do płaszczyzny podstawy ostrosłupa pod kątem $\alpha$. Ściany boczne $BCS\mathrm{i}ADS$ są trójkątami

przystającymi i $\mathrm{k}\mathrm{a}\dot{\mathrm{z}}$ da z nich jest nachylona do płaszczyzny podstawy pod kątem $\beta$. Miary

kątów $\alpha \mathrm{i} \beta$ spełniają warunek: $\alpha+\beta=90^{\mathrm{o}}$ Oblicz pole powierzchni całkowitej tego

ostrosłupa.

Strona 22 z24

MMA-IR
\end{document}
