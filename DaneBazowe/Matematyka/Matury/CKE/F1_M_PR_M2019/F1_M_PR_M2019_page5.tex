\documentclass[a4paper,12pt]{article}
\usepackage{latexsym}
\usepackage{amsmath}
\usepackage{amssymb}
\usepackage{graphicx}
\usepackage{wrapfig}
\pagestyle{plain}
\usepackage{fancybox}
\usepackage{bm}

\begin{document}

{\it Egzamin maturalny z matematyki}

{\it Poziom rozszerzony}

Zadanie 3. $(3pkt)$

Dany jest trójkąt równoramienny $ABC$, w którym $|AC|=|BC|$. Na ramieniu $AC$ tego trójkąta

wybrano punkt $M(M\neq A\mathrm{i}M\neq C)$, a na ramieniu $BC$ wybrano punkt $N$, w taki sposób, $\dot{\mathrm{z}}\mathrm{e}$

$|AM|=|CN|$. Przez punkty $M\mathrm{i}N$ poprowadzono proste prostopadłe do podstawy $AB$ tego

trójkąta, które wyznaczają na niej punkty $S\mathrm{i}T$. Udowodnij, $\displaystyle \dot{\mathrm{z}}\mathrm{e}|ST|=\frac{1}{2}|AB|.$

Strona 6 z24

MMA-IR
\end{document}
