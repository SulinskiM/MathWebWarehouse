\documentclass[a4paper,12pt]{article}
\usepackage{latexsym}
\usepackage{amsmath}
\usepackage{amssymb}
\usepackage{graphicx}
\usepackage{wrapfig}
\pagestyle{plain}
\usepackage{fancybox}
\usepackage{bm}

\begin{document}

{\it Egzamin maturalny z matematyki}

{\it Poziom rozszerzony}

Zadanie 5. $(6pkt)$

Dane są okręgi o równaniach $x^{2}+y^{2}-12x-8y+43=0 \mathrm{i} x^{2}+y^{2}-2ax+4y+a^{2}-77=0.$

Wyznacz wszystkie wartości parametru $a$, dla których te okręgi mają dokładnie jeden punkt

wspólny. Rozwaz wszystkie przypadki.

Strona 10 z24

MMA-IR
\end{document}
