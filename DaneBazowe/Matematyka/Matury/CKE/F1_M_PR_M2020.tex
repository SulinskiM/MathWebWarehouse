\documentclass[a4paper,12pt]{article}
\usepackage{latexsym}
\usepackage{amsmath}
\usepackage{amssymb}
\usepackage{graphicx}
\usepackage{wrapfig}
\pagestyle{plain}
\usepackage{fancybox}
\usepackage{bm}

\begin{document}

{\it f}

CENTRALNA

f KOMfSJA

EGZAMINACYJNA

Arkusz zawiera informacje prawnie chronione do momentu rozpoczęcia egzaminu.

UZUPELNIA ZDAJACY

KOD PESEL

{\it miejsce}

{\it na naklejkę}
\begin{center}
\includegraphics[width=21.432mm,height=9.852mm]{./F1_M_PR_M2020_page0_images/image001.eps}

\includegraphics[width=82.140mm,height=9.852mm]{./F1_M_PR_M2020_page0_images/image002.eps}

\includegraphics[width=204.060mm,height=197.868mm]{./F1_M_PR_M2020_page0_images/image003.eps}
\end{center}
EGZAMIN MATU LNY

Z MATEMATYKI

POZIOM ROZSZERZONY

Instrukcja dla zdającego

1.

2.  7 MAJA 2020

3.

Sprawd $\acute{\mathrm{z}}$, czy arkusz egzaminacyjny zawiera 22 strony

(zadania $1-12$). Ewentualny brak zgłoś przewodniczącemu

zespo nadzo jącego egzamin.

Rozwiązania zadań i odpowiedzi wpisuj w miejscu na to

przeznaczonym.

Pamiętaj, $\dot{\mathrm{z}}\mathrm{e}$ pominięcie argumentacji lub istotnych

obliczeń w rozwiązaniu zadania otwartego $\mathrm{m}\mathrm{o}\dot{\mathrm{z}}\mathrm{e}$

spowodować, $\dot{\mathrm{z}}\mathrm{e}$ za to rozwiązanie nie otrzymasz pełnej

liczby punktów.

Pisz czytelnie i uzywaj tvlko długopisu lub -Dióra

z czamym tuszem lub atramentem.

Nie uzywaj korektora, a błędne zapisy wyra $\acute{\mathrm{z}}\mathrm{n}\mathrm{i}\mathrm{e}$ prze eśl.

Pamiętaj, $\dot{\mathrm{z}}\mathrm{e}$ zapisy w brudnopisie nie będą oceniane.

$\mathrm{M}\mathrm{o}\dot{\mathrm{z}}$ esz korzystać z zestawu wzorów matematycznych,

cyrkla i linijki oraz kalkulatora prostego.

Na tej stronie oraz na karcie odpowiedzi wpisz swój

numer PESEL i przyklej naklejkę z kodem.

Nie wpisuj $\dot{\mathrm{z}}$ adnych znaków w części przeznaczonej dla

egzaminatora.

Godzina rozpoczęcia:

9:00

4.

5.

6.

7.

8.

9.

Czas pracy:

180 minut

Liczba punktów

do uzyskania: 50

$\Vert\Vert\Vert\Vert\Vert\Vert\Vert\Vert\Vert\Vert\Vert\Vert\Vert\Vert\Vert\Vert\Vert\Vert\Vert\Vert\Vert\Vert\Vert\Vert|  \mathrm{M}\mathrm{M}\mathrm{A}-\mathrm{R}1_{-}1\mathrm{P}-202$




{\it Egzamin maturalny z matematyki}

{\it Poziom rozszerzony}

Zadanie l. $(4pkt)$

Rozwiąz nierówność $(\displaystyle \frac{1}{x}-1)^{-1}\leq 1.$

Odpowied $\acute{\mathrm{z}}$:

Strona 2 z22





Odpowiedzí:

{\it Egzamin maturalny z matematyki}

{\it Poziom rozszerzony}
\begin{center}
\includegraphics[width=82.044mm,height=17.832mm]{./F1_M_PR_M2020_page10_images/image001.eps}
\end{center}
Wypelnia

egzaminator

Nr zadania

Maks. liczba kt

7.

4

Uzyskana liczba pkt

MMA-IR

Strona ll z22





{\it Egzamin maturalny z matematyki}

{\it Poziom rozszerzony}

Zadanie 8. $(4pkt)$

$\mathrm{W}$ trójkącie równoramiennym $ABC$: $|AC|=|BC|=10$, a miara kąta $ABC$ jest równa $30^{\mathrm{o}}$

Na boku $BC$ wybrano punkt $P$, taki, $\dot{\mathrm{z}}\mathrm{e} \displaystyle \frac{|BP|}{|PC|}=\frac{2}{3}$. Oblicz sinus kąta $\alpha$ (zobacz rysunek).
\begin{center}
\includegraphics[width=141.120mm,height=46.020mm]{./F1_M_PR_M2020_page11_images/image001.eps}
\end{center}
{\it C}

{\it P}

{\it A}  $\alpha$  {\it B}

Strona 12 z22

MMA-IR





Odpowiedzí:

{\it Egzamin maturalny z matematyki}

{\it Poziom rozszerzony}
\begin{center}
\includegraphics[width=82.044mm,height=17.832mm]{./F1_M_PR_M2020_page12_images/image001.eps}
\end{center}
Wypelnia

egzaminator

Nr zadania

Maks. liczba kt

8.

4

Uzyskana liczba pkt

MMA-IR

Strona 13 z22





{\it Egzamin maturalny z matematyki}

{\it Poziom rozszerzony}

Zadanie 9. $(5pkt)$

Prosta o równaniu $x+y-10=0$

przecina okrąg

o równaniu $x^{2}+y^{2}-8x-6y+8=0$

wpunktach $K\mathrm{i}L$. Punkt $S$ jest środkiem cięciwy $KL$. Wyznacz równanie obrazu tego okręgu

wjednokładności o środku $S$ i skali $k=-3.$

Strona 14 z22

MMA-IR





Odpowiedzí:

{\it Egzamin maturalny z matematyki}

{\it Poziom rozszerzony}
\begin{center}
\includegraphics[width=82.044mm,height=17.832mm]{./F1_M_PR_M2020_page14_images/image001.eps}
\end{center}
Wypelnia

egzaminator

Nr zadania

Maks. liczba kt

5

Uzyskana liczba pkt

MMA-IR

Strona 15 z22





{\it Egzamin maturalny z matematyki}

{\it Poziom rozszerzony}

Zadanie 10. $(5pkt)$

Dany jest kwadrat ABCD o boku długości 2. Na bokach $BC\mathrm{i}$ CD tego kwadratu wybrano

- odpowiednio- punkty $P\mathrm{i}Q$, takie, $\dot{\mathrm{z}}\mathrm{e}$ długość odcinka $|PC|=|QD|=x$ (zobacz rysunek).

Wyznacz tę wartość $x$, dla której pole trójkąta $APQ$ osiąga wartość najmniejszą. Oblicz to

najmniejsze pole.

{\it Q}

2
\begin{center}
\includegraphics[width=69.396mm,height=62.784mm]{./F1_M_PR_M2020_page15_images/image001.eps}
\end{center}
{\it D  x}  $2-x$  {\it C}

{\it x}

{\it P}

{\it A} 2  {\it B}

Strona 16 z22

MMA-IR





Odpowiedzí:

{\it Egzamin maturalny z matematyki}

{\it Poziom rozszerzony}
\begin{center}
\includegraphics[width=82.044mm,height=17.832mm]{./F1_M_PR_M2020_page16_images/image001.eps}
\end{center}
Wypelnia

egzaminator

Nr zadania

Maks. liczba kt

10.

5

Uzyskana liczba pkt

MMA-IR

Strona 17 z22





{\it Egzamin maturalny z matematyki}

{\it Poziom rozszerzony}

Zadanie 11. (4pkt)

Oblicz, ilejest wszystkich siedmiocyfrowych liczb naturalnych, w których zapisie dziesiętnym

występują dokładnie trzy cyfry l i dokładnie dwie cyfry 2.

Strona 18 z22

MMA-IR





Odpowiedzí:

{\it Egzamin maturalny z matematyki}

{\it Poziom rozszerzony}
\begin{center}
\includegraphics[width=82.044mm,height=17.832mm]{./F1_M_PR_M2020_page18_images/image001.eps}
\end{center}
Wypelnia

egzaminator

Nr zadania

Maks. liczba kt

11.

4

Uzyskana liczba pkt

MMA-IR

Strona 19 z22





{\it Egzamin maturalny z matematyki}

{\it Poziom rozszerzony}

Zadanie 12. $(6pkt)$

Podstawą ostrosłupa czworokątnego ABCDS jest trapez ABCD (AB $||$ CD). Ramiona tego

trapezu mają długości $|AD|=10 \mathrm{i}|BC|=16$, a miara kąta $ABC$ jest równa $30^{\mathrm{o}}. \mathrm{K}\mathrm{a}\dot{\mathrm{z}}$ da ściana

boczna tego ostrosłupa tworzy z płaszczyzną podstawy kąt $\alpha$, taki, ze $\displaystyle \mathrm{t}\mathrm{g}\alpha=\frac{9}{2}$. Oblicz objętość

tego ostrosłupa.

Strona 20 z22

MMA-IR





{\it Egzamin maturalny z matematyki}

{\it Poziom rozszerzony}

Zadanie 2. $(3pkt)$

Wyznacz wszystkie wartości parametru $a$, dla których równanie $|x-5|=(a-1)^{2}-4$ ma dwa

rózne rozwiązania dodatnie.

Odpowiedzí:
\begin{center}
\includegraphics[width=96.012mm,height=17.784mm]{./F1_M_PR_M2020_page2_images/image001.eps}
\end{center}
Wypelnia

egzaminator

Nr zadania

Maks. lÍczba kt

1.

4

2.

3

Uzyskana liczba pkt

MMA-IR

Strona 3 z22





Odpowiedzí:

{\it Egzamin maturalny z matematyki}

{\it Poziom rozszerzony}
\begin{center}
\includegraphics[width=82.044mm,height=17.832mm]{./F1_M_PR_M2020_page20_images/image001.eps}
\end{center}
Wypelnia

egzaminator

Nr zadania

Maks. liczba kt

12.

Uzyskana liczba pkt

MMA-IR

Strona 21 z22





{\it Egzamin maturalny z matematyki}

{\it Poziom rozszerzony}

{\it BRUDNOPIS} ({\it nie podlega ocenie})

Strona 22 z22















{\it Egzamin maturalny z matematyki}

{\it Poziom rozszerzony}

Zadanie 3. $(3pkt)$

Liczby dodatnie $a\mathrm{i}b$ spełniają równość $a^{2}+2a=4b^{2}+4b$. Wykaz, $\dot{\mathrm{z}}\mathrm{e}a=2b.$

Strona 4 z22





{\it Egzamin maturalny z matematyki}

{\it Poziom rozszerzony}

Zadanie 4. $(3pkt)$

Dany jest trójkąt równoramienny $ABC$, w którym $|AC|=|BC|=6$, a punkt $D$ jest środkiem

podstawy $AB$. Okrąg o środku $D$ jest styczny do prostej $AC$ w punkcie $M$. Punkt $K$ lezy na boku

$AC$, punkt $L$ lezy na boku $BC$, odcinek $KL$ jest styczny do rozwazanego okręgu oraz $|KC|=|LC|=2$

(zobacz rysunek).
\begin{center}
\includegraphics[width=101.496mm,height=64.260mm]{./F1_M_PR_M2020_page4_images/image001.eps}
\end{center}
{\it C}

{\it K  L}

{\it M}

{\it A  D  B}

Wykaz, $\displaystyle \dot{\mathrm{z}}\mathrm{e}\frac{|AM|}{|MC|}=\frac{4}{5}$
\begin{center}
\includegraphics[width=96.012mm,height=17.832mm]{./F1_M_PR_M2020_page4_images/image002.eps}
\end{center}
Wypelnia

egzaminator

Nr zadania

Maks. lÍczba kt

3.

3

4.

3

Uzyskana liczba pkt

MMA-IR

Strona 5 z22





{\it Egzamin maturalny z matematyki}

{\it Poziom rozszerzony}

Zadanie 5. $(5pkt)$

$\mathrm{W}$ trzywyrazowym ciągu geometrycznym $(a_{1},a_{2},a_{3})$ spełnionajest równość $a_{1}+a_{2}+a_{3}=\displaystyle \frac{21}{4}$

Wyrazy $a_{1}, a_{2}, a_{3}$ są- odpowiednio-czwartym, drugim i pierwszym wyrazem rosnącego

ciągu arytmetycznego. Oblicz $a_{1}.$

Strona 6 z22

MMA-IR





Odpowied $\acute{\mathrm{z}}$:

{\it Egzamin maturalny z matematyki}

{\it Poziom rozszerzony}
\begin{center}
\includegraphics[width=82.044mm,height=17.784mm]{./F1_M_PR_M2020_page6_images/image001.eps}
\end{center}
Nr zadania

Wypelnia Maks. liczba kt

egzamÍnator

Uzyskana liczba pkt

5.

5

MMA-IR

Strona 7 z22





{\it Egzamin maturalny z matematyki}

{\it Poziom rozszerzony}

Zadanie 6. $(4pkt)$

Rozwiąz równanie $3\cos 2x+10\cos^{2}x=24\sin x-3$ dla $x\in\langle 0, 2\pi\rangle.$

Strona 8 z22





Odpowied $\acute{\mathrm{z}}$:

{\it Egzamin maturalny z matematyki}

{\it Poziom rozszerzony}
\begin{center}
\includegraphics[width=82.044mm,height=17.784mm]{./F1_M_PR_M2020_page8_images/image001.eps}
\end{center}
Wypelnia

egzamÍnator

Nr zadania

Maks. liczba kt

4

Uzyskana liczba pkt

MMA-IR

Strona 9 z22





{\it Egzamin maturalny z matematyki}

{\it Poziom rozszerzony}

Zadanie 7. $(4pkt)$

Dane jest równanie kwadratowe $x^{2}-(3m+2)x+2m^{2}+7m-15=0$ z niewiadomą $x$. Wyznacz

wszystkie wartoŚci parametru $m$, dla których rózne rozwiązania $x_{1}$ i $x_{2}$ tego równania

i spełniają warunek

$2x_{1}^{2}+5x_{1}x_{2}+2x_{2}^{2}=2.$

istnieją

Strona 10 z22

MMA-IR



\end{document}