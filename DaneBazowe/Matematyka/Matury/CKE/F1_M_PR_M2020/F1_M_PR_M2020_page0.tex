\documentclass[a4paper,12pt]{article}
\usepackage{latexsym}
\usepackage{amsmath}
\usepackage{amssymb}
\usepackage{graphicx}
\usepackage{wrapfig}
\pagestyle{plain}
\usepackage{fancybox}
\usepackage{bm}

\begin{document}

{\it f}

CENTRALNA

f KOMfSJA

EGZAMINACYJNA

Arkusz zawiera informacje prawnie chronione do momentu rozpoczęcia egzaminu.

UZUPELNIA ZDAJACY

KOD PESEL

{\it miejsce}

{\it na naklejkę}
\begin{center}
\includegraphics[width=21.432mm,height=9.852mm]{./F1_M_PR_M2020_page0_images/image001.eps}

\includegraphics[width=82.140mm,height=9.852mm]{./F1_M_PR_M2020_page0_images/image002.eps}

\includegraphics[width=204.060mm,height=197.868mm]{./F1_M_PR_M2020_page0_images/image003.eps}
\end{center}
EGZAMIN MATU LNY

Z MATEMATYKI

POZIOM ROZSZERZONY

Instrukcja dla zdającego

1.

2.  7 MAJA 2020

3.

Sprawd $\acute{\mathrm{z}}$, czy arkusz egzaminacyjny zawiera 22 strony

(zadania $1-12$). Ewentualny brak zgłoś przewodniczącemu

zespo nadzo jącego egzamin.

Rozwiązania zadań i odpowiedzi wpisuj w miejscu na to

przeznaczonym.

Pamiętaj, $\dot{\mathrm{z}}\mathrm{e}$ pominięcie argumentacji lub istotnych

obliczeń w rozwiązaniu zadania otwartego $\mathrm{m}\mathrm{o}\dot{\mathrm{z}}\mathrm{e}$

spowodować, $\dot{\mathrm{z}}\mathrm{e}$ za to rozwiązanie nie otrzymasz pełnej

liczby punktów.

Pisz czytelnie i uzywaj tvlko długopisu lub -Dióra

z czamym tuszem lub atramentem.

Nie uzywaj korektora, a błędne zapisy wyra $\acute{\mathrm{z}}\mathrm{n}\mathrm{i}\mathrm{e}$ prze eśl.

Pamiętaj, $\dot{\mathrm{z}}\mathrm{e}$ zapisy w brudnopisie nie będą oceniane.

$\mathrm{M}\mathrm{o}\dot{\mathrm{z}}$ esz korzystać z zestawu wzorów matematycznych,

cyrkla i linijki oraz kalkulatora prostego.

Na tej stronie oraz na karcie odpowiedzi wpisz swój

numer PESEL i przyklej naklejkę z kodem.

Nie wpisuj $\dot{\mathrm{z}}$ adnych znaków w części przeznaczonej dla

egzaminatora.

Godzina rozpoczęcia:

9:00

4.

5.

6.

7.

8.

9.

Czas pracy:

180 minut

Liczba punktów

do uzyskania: 50

$\Vert\Vert\Vert\Vert\Vert\Vert\Vert\Vert\Vert\Vert\Vert\Vert\Vert\Vert\Vert\Vert\Vert\Vert\Vert\Vert\Vert\Vert\Vert\Vert|  \mathrm{M}\mathrm{M}\mathrm{A}-\mathrm{R}1_{-}1\mathrm{P}-202$
\end{document}
