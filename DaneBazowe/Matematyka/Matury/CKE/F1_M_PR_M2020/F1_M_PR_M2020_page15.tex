\documentclass[a4paper,12pt]{article}
\usepackage{latexsym}
\usepackage{amsmath}
\usepackage{amssymb}
\usepackage{graphicx}
\usepackage{wrapfig}
\pagestyle{plain}
\usepackage{fancybox}
\usepackage{bm}

\begin{document}

{\it Egzamin maturalny z matematyki}

{\it Poziom rozszerzony}

Zadanie 10. $(5pkt)$

Dany jest kwadrat ABCD o boku długości 2. Na bokach $BC\mathrm{i}$ CD tego kwadratu wybrano

- odpowiednio- punkty $P\mathrm{i}Q$, takie, $\dot{\mathrm{z}}\mathrm{e}$ długość odcinka $|PC|=|QD|=x$ (zobacz rysunek).

Wyznacz tę wartość $x$, dla której pole trójkąta $APQ$ osiąga wartość najmniejszą. Oblicz to

najmniejsze pole.

{\it Q}

2
\begin{center}
\includegraphics[width=69.396mm,height=62.784mm]{./F1_M_PR_M2020_page15_images/image001.eps}
\end{center}
{\it D  x}  $2-x$  {\it C}

{\it x}

{\it P}

{\it A} 2  {\it B}

Strona 16 z22

MMA-IR
\end{document}
