\documentclass[a4paper,12pt]{article}
\usepackage{latexsym}
\usepackage{amsmath}
\usepackage{amssymb}
\usepackage{graphicx}
\usepackage{wrapfig}
\pagestyle{plain}
\usepackage{fancybox}
\usepackage{bm}

\begin{document}

{\it Egzamin maturalny z matematyki}

{\it Poziom rozszerzony}

Zadanie 12. $(6pkt)$

Podstawą ostrosłupa czworokątnego ABCDS jest trapez ABCD (AB $||$ CD). Ramiona tego

trapezu mają długości $|AD|=10 \mathrm{i}|BC|=16$, a miara kąta $ABC$ jest równa $30^{\mathrm{o}}. \mathrm{K}\mathrm{a}\dot{\mathrm{z}}$ da ściana

boczna tego ostrosłupa tworzy z płaszczyzną podstawy kąt $\alpha$, taki, ze $\displaystyle \mathrm{t}\mathrm{g}\alpha=\frac{9}{2}$. Oblicz objętość

tego ostrosłupa.

Strona 20 z22

MMA-IR
\end{document}
