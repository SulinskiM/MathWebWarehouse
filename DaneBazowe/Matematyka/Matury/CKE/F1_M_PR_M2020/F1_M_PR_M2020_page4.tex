\documentclass[a4paper,12pt]{article}
\usepackage{latexsym}
\usepackage{amsmath}
\usepackage{amssymb}
\usepackage{graphicx}
\usepackage{wrapfig}
\pagestyle{plain}
\usepackage{fancybox}
\usepackage{bm}

\begin{document}

{\it Egzamin maturalny z matematyki}

{\it Poziom rozszerzony}

Zadanie 4. $(3pkt)$

Dany jest trójkąt równoramienny $ABC$, w którym $|AC|=|BC|=6$, a punkt $D$ jest środkiem

podstawy $AB$. Okrąg o środku $D$ jest styczny do prostej $AC$ w punkcie $M$. Punkt $K$ lezy na boku

$AC$, punkt $L$ lezy na boku $BC$, odcinek $KL$ jest styczny do rozwazanego okręgu oraz $|KC|=|LC|=2$

(zobacz rysunek).
\begin{center}
\includegraphics[width=101.496mm,height=64.260mm]{./F1_M_PR_M2020_page4_images/image001.eps}
\end{center}
{\it C}

{\it K  L}

{\it M}

{\it A  D  B}

Wykaz, $\displaystyle \dot{\mathrm{z}}\mathrm{e}\frac{|AM|}{|MC|}=\frac{4}{5}$
\begin{center}
\includegraphics[width=96.012mm,height=17.832mm]{./F1_M_PR_M2020_page4_images/image002.eps}
\end{center}
Wypelnia

egzaminator

Nr zadania

Maks. lÍczba kt

3.

3

4.

3

Uzyskana liczba pkt

MMA-IR

Strona 5 z22
\end{document}
