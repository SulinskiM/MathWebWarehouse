% This LaTeX document needs to be compiled with XeLaTeX.
\documentclass[10pt]{article}
\usepackage[utf8]{inputenc}
\usepackage{graphicx}
\usepackage[export]{adjustbox}
\graphicspath{ {./images/} }
\usepackage{amsmath}
\usepackage{amsfonts}
\usepackage{amssymb}
\usepackage[version=4]{mhchem}
\usepackage{stmaryrd}
\usepackage{multirow}
\usepackage[fallback]{xeCJK}
\usepackage{polyglossia}
\usepackage{fontspec}
\IfFontExistsTF{Noto Serif CJK JP}
{\setCJKmainfont{Noto Serif CJK JP}}
{\IfFontExistsTF{STSong}
  {\setCJKmainfont{STSong}}
  {\IfFontExistsTF{Droid Sans Fallback}
    {\setCJKmainfont{Droid Sans Fallback}}
    {\setCJKmainfont{SimSun}}
}}

\setmainlanguage{polish}
\IfFontExistsTF{CMU Serif}
{\setmainfont{CMU Serif}}
{\IfFontExistsTF{DejaVu Sans}
  {\setmainfont{DejaVu Sans}}
  {\setmainfont{Georgia}}
}

\author{TERMIN: marzec 2021 r.}
\date{}


\begin{document}
\maketitle
\section*{WYPEŁNIA ZDAJĄCY}
\section*{KOD}
PESEL\\
\includegraphics[max width=\textwidth, center]{2025_02_09_52b35926cd28a35c3960g-01}

\section*{Miejsce na naklejkę.}
Sprawdź, czy kod na naklejce to E-100.

Jeżeli tak - przyklej naklejkę. Jeżeli nie - zgłoś to nauczycielowi.

\section*{EGZAMIN MATURALNY MATEMATYKA - POZIOM PODSTAWOWY}
\section*{TEST DIAGNOSTYCZNY}
CzAS PRACY: 170 minut\\
LICZBA PUNKTÓW DO UZYSKANIA:\\
45

\section*{WYPEとNIA ZESPÓt NADZORUJĄCY}
Uprawnienia zdającego do:

\(\square \quad\)\begin{tabular}{l}
nieprzenoszenia \\
zaznaczeń na kartę \\
\end{tabular} dostosowania zasad oceniania dostosowania w zw. z dyskalkulią.

\begin{center}
\includegraphics[max width=\textwidth]{2025_02_09_52b35926cd28a35c3960g-01(1)}
\end{center}

\section*{Instrukcja dla zdającego}
\begin{enumerate}
  \item Sprawdź, czy arkusz egzaminacyjny zawiera 23 strony (zadania 1-35). Ewentualny brak zgłoś przewodniczącemu zespołu nadzorującego egzamin.
  \item Rozwiązania zadań i odpowiedzi wpisuj w miejscu na to przeznaczonym.
  \item Odpowiedzi do zadań zamkniętych (1-28) zaznacz na karcie odpowiedzi w części karty przeznaczonej dla zdajacego. Zamaluj \(\square\) pola do tego przeznaczone. Błędne zaznaczenie otocz kółkiem i zaznacz właściwe.
  \item Pamiętaj, że pominięcie argumentacji lub istotnych obliczeń w rozwiązaniu zadania otwartego (29-35) może spowodować, że za to rozwiązanie nie otrzymasz pełnej liczby punktów.
  \item Pisz czytelnie i używaj tylko długopisu lub pióra z czarnym tuszem lub atramentem.
  \item Nie używaj korektora, a błędne zapisy wyraźnie przekreśl.
  \item Pamiętaj, że zapisy w brudnopisie nie będą oceniane.
  \item Możesz korzystać z zestawu wzorów matematycznych, cyrkla i linijki oraz kalkulatora prostego.
  \item Na tej stronie oraz na karcie odpowiedzi wpisz swój numer PESEL i przyklej naklejkę z kodem.
  \item Nie wpisuj żadnych znaków w części przeznaczonej dla egzaminatora.
\end{enumerate}

W każdym z zadań od 1. do 28. wybierz i zaznacz na karcie odpowiedzi poprawną odpowiedź.

\section*{Zadanie 1. (0-1)}
Liczba \((\sqrt{6}-\sqrt{2})^{2}-2 \sqrt{3}\) jest równa\\
A. \(8-6 \sqrt{3}\)\\
B. \(8-2 \sqrt{3}\)\\
C. \(4-2 \sqrt{3}\)\\
D. \(8-4 \sqrt{3}\)

\section*{Zadanie 2. (0-1)}
Liczba \(2 \log _{5} 4-3 \log _{5} \frac{1}{2}\) jest równa\\
A. \(-\log _{5} \frac{7}{2}\)\\
B. \(7 \log _{5} 2\)\\
C. \(-\log _{5} 2\)\\
D. \(\log _{5} 2\)

\section*{Zadanie 3. (0-1)}
Medyczna maseczka ochronna wielokrotnego użytku z wymiennymi filtrami wskutek podwyżki zdrożała o \(40 \%\) i kosztuje obecnie 106,40 zł. Cena maseczki przed podwyżką była równa\\
A. \(63,84 \mathrm{zł}\)\\
B. \(65,40 \mathrm{zł}\)\\
C. \(76,00 \mathrm{Z} \nmid\)\\
D. \(66,40 \mathrm{zł}\)

\section*{Zadanie 4. (0-1)}
Dla każdej dodatniej liczby \(b\) wyrażenie \((\sqrt[2]{b} \cdot \sqrt[4]{b})^{\frac{1}{3}}\) jest równe\\
A. \(b^{2}\)\\
B. \(b^{0,25}\)\\
C. \(b^{\frac{8}{3}}\)\\
D. \(b^{\frac{4}{3}}\)

\section*{Zadanie 5. (0-1)}
Para liczb \(x=1, y=-3\) spełnia układ równań \(\left\{\begin{array}{l}x-y=a^{2} \\ (1+a) x-3 y=-4 a\end{array}\right.\)\\
Wtedy a jest równe\\
A. 2\\
B. -2\\
C. \(\sqrt{2}\)\\
D. \(-\sqrt{2}\)

\section*{Zadanie 6. (0-1)}
lloczyn wszystkich rozwiązań równania \(2(x-4)\left(x^{2}-1\right)=0\) jest równy\\
A. -8\\
B. -4\\
C. 4\\
D. 8

\section*{Zadanie 7. (0-1)}
Zbiorem rozwiązań nierówności \(\frac{12-5 x}{2}<3\left(1-\frac{1}{2} x\right)+7 x\) jest\\
A. \(\left(-\infty, \frac{2}{7}\right)\)\\
B. \(\left(\frac{2}{7},+\infty\right)\)\\
C. \(\left(-\infty, \frac{3}{8}\right)\)\\
D. \(\left(\frac{3}{8},+\infty\right)\)

BRUDNOPIS (nie podlega ocenie)\\
\includegraphics[max width=\textwidth, center]{2025_02_09_52b35926cd28a35c3960g-03}

\section*{Zadanie 8. (0-1)}
Funkcja liniowa \(f(x)=(a-1) x+3\) osiąga wartość najmniejszą równą 3 . Wtedy\\
A. \(a=-1\)\\
B. \(a=0\)\\
C. \(a=1\)\\
D. \(a=3\)

\section*{Zadanie 9. (0-1)}
Na wykresie przedstawiono wykres funkcji \(f\).\\
\includegraphics[max width=\textwidth, center]{2025_02_09_52b35926cd28a35c3960g-04}

Wskaż zdanie prawdziwe.\\
A. Dziedziną funkcji \(f\) jest przedział \((-4,5)\).\\
B. Funkcja \(f\) ma dwa miejsca zerowe.\\
C. Funkcja \(f\) dla argumentu 1 przyjmuje wartość ( -1 ).\\
D. Zbiorem wartości funkcji \(f\) jest przedział ( \(-4,5\rangle\).

\section*{Zadanie 10. (0-1)}
Funkcja \(f\) jest określona wzorem \(f(x)=\frac{8 x-7}{2 x^{2}+1}\) dla każdej liczby rzeczywistej \(x\). Wartość funkcji \(f\) dla argumentu 1 jest równa\\
A. \(\frac{1}{5}\)\\
B. \(\frac{1}{3}\)\\
C. 1\\
D. 2

BRUDNOPIS (nie podlega ocenie)\\
\includegraphics[max width=\textwidth, center]{2025_02_09_52b35926cd28a35c3960g-05}

\section*{Zadanie 11. (0-1)}
Ciąg \((x, y, z)\) jest geometryczny. lloczyn wszystkich wyrazów tego ciągu jest równy 64 . Stąd wynika, że \(y\) jest równe\\
A. \(3 \cdot 64\)\\
B. \(\frac{64}{3}\)\\
C. 4\\
D. 3

\section*{Zadanie 12. (0-1)}
Ciąg \(\left(a_{n}\right)\), określony dla każdej liczby naturalnej \(n \geq 1\), jest arytmetyczny. Różnica tego ciągu jest równa 5 , a pierwszy wyraz tego ciągu jest równy ( -3 ). Wtedy iloraz \(\frac{a_{4}}{a_{2}}\) jest równy\\
A. \(\frac{5}{3}\)\\
B. 2\\
C. 6\\
D. 25

\section*{Zadanie 13. (0-1)}
Trójkąt \(A B C\) jest wpisany w okrąg o środku \(O\). Miara kąta \(C A O\) jest równa \(70^{\circ}\) (zobacz rysunek). Wtedy miara kąta \(A B C\) jest równa\\
A. \(20^{\circ}\)\\
B. \(25^{\circ}\)\\
C. \(30^{\circ}\)\\
D. \(35^{\circ}\)\\
\includegraphics[max width=\textwidth, center]{2025_02_09_52b35926cd28a35c3960g-06}

\section*{Zadanie 14. (0-1)}
Ciągi \(\left(a_{n}\right),\left(b_{n}\right)\) oraz \(\left(c_{n}\right)\) są określone dla każdej liczby naturalnej \(n \geq 1\) następująco:

\begin{itemize}
  \item \(a_{n}=6 n^{2}-n^{3}\)
  \item \(b_{n}=2 n+13\)
  \item \(c_{n}=2^{n}\)
\end{itemize}

Wskaż zdanie prawdziwe.\\
A. Ciąg ( \(a_{n}\) ) jest arytmetyczny.\\
B. Ciąg ( \(b_{n}\) ) jest arytmetyczny.\\
C. Ciąg ( \(c_{n}\) ) jest arytmetyczny.\\
D. Wśród ciągów \(\left(a_{n}\right),\left(b_{n}\right),\left(c_{n}\right)\) nie ma ciągu arytmetycznego.

BRUDNOPIS (nie podlega ocenie)\\
\includegraphics[max width=\textwidth, center]{2025_02_09_52b35926cd28a35c3960g-07}

\section*{Zadanie 15. (0-1)}
Ciąg \(\left(a_{n}\right)\) jest określony wzorem \(a_{n}=(-2)^{n} \cdot n+1\) dla każdej liczby naturalnej \(n \geq 1\). Wtedy trzeci wyraz tego ciągu jest równy\\
A. -24\\
B. -17\\
C. -32\\
D. -23

\section*{Zadanie 16. (0-1)}
W romb o boku \(2 \sqrt{3}\) i kącie \(60^{\circ}\) wpisano okrąg. Promień tego okręgu jest równy\\
A. 3\\
B. \(\frac{1}{2}\)\\
C. \(\frac{3}{4}\)\\
D. \(\frac{3}{2}\)

\section*{Zadanie 17. (0-1)}
Przez punkt przecięcia wysokości trójkąta równobocznego \(A B C\) poprowadzono prostą \(D E\) równoległą do podstawy \(A B\) (zobacz rysunek).\\
\includegraphics[max width=\textwidth, center]{2025_02_09_52b35926cd28a35c3960g-08}

Stosunek pola trójkąta \(A B C\) do pola trójkąta \(C D E\) jest równy\\
A. \(9: 4\)\\
B. \(4: 1\)\\
C. \(4: 9\)\\
D. \(3: 2\)

\section*{Zadanie 18. (0-1)}
Końcami odcinka \(P R\) są punkty \(P=(4,7)\) i \(R=(-2,-3)\). Odległość punktu \(T=(3,-1)\) od środka odcinka \(P R\) jest równa\\
A. \(\sqrt{3}\)\\
B. \(\sqrt{13}\)\\
C. \(\sqrt{17}\)\\
D. \(6 \sqrt{2}\)

BRUDNOPIS (nie podlega ocenie)\\
\includegraphics[max width=\textwidth, center]{2025_02_09_52b35926cd28a35c3960g-09}

\section*{Zadanie 19. (0-1)}
Kąt \(\alpha\) jest ostry oraz \(\sin \alpha=\frac{4}{5}\). Wtedy\\
A. \(\cos \alpha=\frac{1}{5}\)\\
B. \(\cos \alpha=-\frac{1}{5}\)\\
C. \(\cos \alpha=-\frac{3}{5}\)\\
D. \(\cos \alpha=\frac{3}{5}\)

\section*{Zadanie 20. (0-1)}
Dane są punkty \(M=(6,0), N=(6,8)\) oraz \(O=(0,0)\). Tangens kąta ostrego MON jest równy\\
A. \(\frac{4}{3}\)\\
B. \(\frac{6}{10}\)\\
C. \(\frac{3}{4}\)\\
D. \(\frac{8}{10}\)

\section*{Zadanie 21. (0-1)}
Proste o równaniach \(y=3 a x-2\) i \(y=2 x+3 a\) są prostopadłe. Wtedy \(a\) jest równe\\
A. \(\frac{2}{3}\)\\
B. \(-\frac{1}{6}\)\\
C. \(\frac{3}{2}\)\\
D. -5

\section*{Zadanie 22. (0-1)}
Dany jest trapez \(A B C D\), w którym boki \(A B\) i \(C D\) są równoległe oraz \(C=(3,5)\). Wierzchołki \(A\) i \(B\) tego trapezu leżą na prostej o równaniu \(y=5 x+3\). Wtedy bok \(C D\) tego trapezu zawiera się w prostej o równaniu\\
A. \(y=3 x+5\)\\
B. \(y=-\frac{1}{5} x+3\)\\
C. \(y=5 x-10\)\\
D. \(y=-\frac{1}{5} x+\frac{28}{5}\)

\section*{Zadanie 23. (0-1)}
W trapezie równoramiennym \(A B C D\) podstawy \(A B\) i \(C D\) mają długości równe odpowiednio \(a\) i \(b\) (przy czym \(a>b\) ). Miara kąta ostrego trapezu jest równa \(30^{\circ}\). Wtedy wysokość tego trapezu jest równa\\
A. \(\frac{a-b}{2} \cdot \sqrt{3}\)\\
B. \(\frac{a-b}{6} \cdot \sqrt{3}\)\\
C. \(\frac{a+b}{2}\)\\
D. \(\frac{a+b}{4}\)

\section*{Zadanie 24. (0-1)}
Przekątna sześcianu ma długość \(5 \sqrt{3}\). Wtedy objętość tego sześcianu jest równa\\
A. 125\\
B. 75\\
C. \(375 \sqrt{3}\)\\
D. \(125 \sqrt{3}\)

BRUDNOPIS (nie podlega ocenie)\\
\includegraphics[max width=\textwidth, center]{2025_02_09_52b35926cd28a35c3960g-11}

\section*{Zadanie 25. (0-1)}
Ostrosłupy prawidłowe trójkątne \(O_{1}\) i \(O_{2}\) mają takie same wysokości. Długość krawędzi podstawy ostrosłupa \(O_{1}\) jest trzy razy dłuższa od długości krawędzi podstawy ostrosłupa \(O_{2}\). Stosunek objętości ostrosłupa \(O_{1}\) do objętości ostrosłupa \(O_{2}\) jest równy\\
A. \(3: 1\)\\
B. \(1: 3\)\\
C. \(9: 1\)\\
D. 1:9

\section*{Zadanie 26. (0-1)}
Wszystkich liczb naturalnych trzycyfrowych parzystych, w których cyfra 7 występuje dokładnie jeden raz, jest\\
A. 85\\
B. 90\\
C. 100\\
D. 150

\section*{Zadanie 27. (0-1)}
Ze zbioru liczb naturalnych dwucyfrowych losujemy jedną liczbę. Prawdopodobieństwo zdarzenia polegającego na tym, że wylosowana liczba jest podzielna przez 5, jest równe\\
A. \(\frac{2}{5}\)\\
B. \(\frac{5}{100}\)\\
C. \(\frac{5}{90}\)\\
D. \(\frac{18}{90}\)

\section*{Zadanie 28. (0-1)}
Liczba \(x\) jest dodatnia. Mediana zestawu czterech liczb: \(1+x, 1+2 x, 4+3 x, 1\), jest równa 10. Wtedy\\
A. \(x=6\)\\
B. \(x=5,5\)\\
C. \(x=2,5\)\\
D. \(x=1\)

BRUDNOPIS (nie podlega ocenie)\\
\includegraphics[max width=\textwidth, center]{2025_02_09_52b35926cd28a35c3960g-13}

Zadanie 29. (0-2)\\
Rozwiąż nierówność:

\[
3 x(x+1)>x^{2}+x+24
\]

\begin{center}
\begin{tabular}{|c|c|c|c|c|c|c|c|c|c|c|c|c|c|c|c|c|c|c|c|c|c|}
\hline
 &  &  &  &  &  &  &  &  &  &  &  &  &  &  &  &  &  &  &  &  &  \\
\hline
 &  &  &  &  &  &  &  &  &  &  &  &  &  &  &  &  &  &  &  &  &  \\
\hline
 &  &  &  &  &  &  &  &  &  &  &  &  &  &  &  &  &  &  &  &  &  \\
\hline
 &  &  &  &  &  &  &  &  &  &  &  &  &  &  &  &  &  &  &  &  &  \\
\hline
 &  &  &  &  &  &  &  &  &  &  &  &  &  &  &  &  &  &  &  &  &  \\
\hline
 &  &  &  &  &  &  &  &  &  &  &  &  &  &  &  &  &  &  &  &  &  \\
\hline
 &  &  &  &  &  &  &  &  &  &  &  &  &  &  &  &  &  &  &  &  &  \\
\hline
 &  &  &  &  &  &  &  &  &  &  &  &  &  &  &  &  &  &  &  &  &  \\
\hline
 &  &  &  &  &  &  &  &  &  &  &  &  &  &  &  &  &  &  &  &  &  \\
\hline
 &  &  &  &  &  &  &  &  &  &  &  &  &  &  &  &  &  &  &  &  &  \\
\hline
 &  &  &  &  &  &  &  &  &  &  &  &  &  &  &  &  &  &  &  &  &  \\
\hline
 &  &  &  &  &  &  &  &  &  &  &  &  &  &  &  &  &  &  &  &  &  \\
\hline
 &  &  &  &  &  &  &  &  &  &  &  &  &  &  &  &  &  &  &  &  &  \\
\hline
 &  &  &  &  &  &  &  &  &  &  &  &  &  &  &  &  &  &  &  &  &  \\
\hline
 &  &  &  &  &  &  &  &  &  &  &  &  &  &  &  &  &  &  &  &  &  \\
\hline
 &  &  &  &  &  &  &  &  &  &  &  &  &  &  &  &  &  &  &  &  &  \\
\hline
 &  &  &  &  &  &  &  &  &  &  &  &  &  &  &  &  &  &  &  &  &  \\
\hline
 &  &  &  &  &  &  &  &  &  &  &  &  &  &  &  &  &  &  &  &  &  \\
\hline
 &  &  &  &  &  &  &  &  &  &  &  &  &  &  &  &  &  &  &  &  &  \\
\hline
 &  &  &  &  &  &  &  &  &  &  &  &  &  &  &  &  &  &  &  &  &  \\
\hline
 &  &  &  &  &  &  &  &  &  &  &  &  &  &  &  &  &  &  &  &  &  \\
\hline
 &  &  &  &  &  &  &  &  &  &  &  &  &  &  &  &  &  &  &  &  &  \\
\hline
 &  &  &  &  &  &  &  &  &  &  &  &  &  &  &  &  &  &  &  &  &  \\
\hline
 &  &  &  &  &  &  &  &  &  &  &  &  &  &  &  &  &  &  &  &  &  \\
\hline
 &  &  &  &  &  &  &  &  &  &  &  &  &  &  &  &  &  &  &  &  &  \\
\hline
 &  &  &  &  &  &  &  &  &  &  &  &  &  &  &  &  &  &  &  &  &  \\
\hline
 &  &  &  &  &  &  &  &  &  &  &  &  &  &  &  &  &  &  &  &  &  \\
\hline
 &  &  &  &  &  &  &  &  &  &  &  &  &  &  &  &  &  &  &  &  &  \\
\hline
 &  &  &  &  &  &  &  &  &  &  &  &  &  &  &  &  &  &  &  &  &  \\
\hline
 &  &  &  &  &  &  &  &  &  &  &  &  &  &  &  &  &  &  &  &  &  \\
\hline
 &  &  &  &  &  &  &  &  &  &  &  &  &  &  &  &  &  &  &  &  &  \\
\hline
 &  &  &  &  &  &  &  &  &  &  &  &  &  &  &  &  &  &  &  &  &  \\
\hline
 &  &  &  &  &  &  &  &  &  &  &  &  &  &  &  &  &  &  &  &  &  \\
\hline
 &  &  &  &  &  &  &  &  &  &  &  &  &  &  &  &  &  &  &  &  &  \\
\hline
 &  &  &  &  &  &  &  &  &  &  &  &  &  &  &  &  &  &  &  &  &  \\
\hline
 &  &  &  &  &  &  &  &  &  &  &  &  &  &  &  &  &  &  &  &  &  \\
\hline
 &  &  &  &  &  &  &  &  &  &  &  &  &  &  &  &  &  &  &  &  &  \\
\hline
 &  &  &  &  &  &  &  &  &  &  &  &  &  &  &  &  &  &  &  &  &  \\
\hline
 &  &  &  &  &  &  &  &  &  &  &  &  &  &  &  &  &  &  &  &  &  \\
\hline
 &  &  &  &  &  &  &  &  &  &  &  &  &  &  &  &  &  &  &  &  &  \\
\hline
 &  &  &  &  &  &  &  &  &  &  &  &  &  &  &  &  &  &  &  &  &  \\
\hline
\end{tabular}
\end{center}

Odpowiedź:

Zadanie 30. (0-2)\\
Rozwiąż równanie:

\[
\frac{6 x-1}{3 x-2}=3 x+2
\]

\begin{center}
\begin{tabular}{|c|c|c|c|c|c|c|c|c|c|c|c|c|c|c|c|c|c|c|c|c|c|c|}
\hline
 &  &  &  &  &  &  &  &  &  &  &  &  &  &  &  &  &  &  &  &  &  &  \\
\hline
 &  &  &  &  &  &  &  &  &  &  &  &  &  &  &  &  &  &  &  &  &  &  \\
\hline
 &  &  &  &  &  &  &  &  &  &  &  &  &  &  &  &  &  &  &  &  &  &  \\
\hline
 &  &  &  &  &  &  &  &  &  &  &  &  &  &  &  &  &  &  &  &  &  &  \\
\hline
 &  &  &  &  &  &  &  &  &  &  &  &  &  &  &  &  &  &  &  &  &  &  \\
\hline
 &  &  &  &  &  &  &  &  &  &  &  &  &  &  &  &  &  &  &  &  &  &  \\
\hline
 &  &  &  &  &  &  &  &  &  &  &  &  &  &  &  &  &  &  &  &  &  &  \\
\hline
 &  &  &  &  &  &  &  &  &  &  &  &  &  &  &  &  &  &  &  &  &  &  \\
\hline
 &  &  &  &  &  &  &  &  &  &  &  &  &  &  &  &  &  &  &  &  &  &  \\
\hline
 &  &  &  &  &  &  &  &  &  &  &  &  &  &  &  &  &  &  &  &  &  &  \\
\hline
 &  &  &  &  &  &  &  &  &  &  &  &  &  &  &  &  &  &  &  &  &  &  \\
\hline
 &  &  &  &  &  &  &  &  &  &  &  &  &  &  &  &  &  &  &  &  &  &  \\
\hline
 &  &  &  &  &  &  &  &  &  &  &  &  &  &  &  &  &  &  &  &  &  &  \\
\hline
 &  &  &  &  &  &  &  &  &  &  &  &  &  &  &  &  &  &  &  &  &  &  \\
\hline
 &  &  &  &  &  &  &  &  &  &  &  &  &  &  &  &  &  &  &  &  &  &  \\
\hline
 &  &  &  &  &  &  &  &  &  &  &  &  &  &  &  &  &  &  &  &  &  &  \\
\hline
 &  &  &  &  &  &  &  &  &  &  &  &  &  &  &  &  &  &  &  &  &  &  \\
\hline
 &  &  &  &  &  &  &  &  &  &  &  &  &  &  &  &  &  &  &  &  &  &  \\
\hline
 &  &  &  &  &  &  &  &  &  &  &  &  &  &  &  &  &  &  &  &  &  &  \\
\hline
 &  &  &  &  &  &  &  &  &  &  &  &  &  &  &  &  &  &  &  &  &  &  \\
\hline
 &  &  &  &  &  &  &  &  &  &  &  &  &  &  &  &  &  &  &  &  &  &  \\
\hline
 &  &  &  &  &  &  &  &  &  &  &  &  &  &  &  &  &  &  &  &  &  &  \\
\hline
 &  &  &  &  &  &  &  &  &  &  &  &  &  &  &  &  &  &  &  &  &  &  \\
\hline
 &  &  &  &  &  &  &  &  &  &  &  &  &  &  &  &  &  &  &  &  &  &  \\
\hline
 &  &  &  &  &  &  &  &  &  &  &  &  &  &  &  &  &  &  &  &  &  &  \\
\hline
 &  &  &  &  &  &  &  &  &  &  &  &  &  &  &  &  &  &  &  &  &  &  \\
\hline
 &  &  &  &  &  &  &  &  &  &  &  &  &  &  &  &  &  &  &  &  &  &  \\
\hline
 &  &  &  &  &  &  &  &  &  &  &  &  &  &  &  &  &  &  &  &  &  &  \\
\hline
 &  &  &  &  &  &  &  &  &  &  &  &  &  &  &  &  &  &  &  &  &  &  \\
\hline
 &  &  &  &  &  &  &  &  &  &  &  &  &  &  &  &  &  &  &  &  &  &  \\
\hline
 &  &  &  &  &  &  &  &  &  &  &  &  &  &  &  &  &  &  &  &  &  &  \\
\hline
 &  &  &  &  &  &  &  &  &  &  &  &  &  &  &  &  &  &  &  &  &  &  \\
\hline
 &  &  &  &  &  &  &  &  &  &  &  &  &  &  &  &  &  &  &  &  &  &  \\
\hline
 &  &  &  &  &  &  &  &  &  &  &  &  &  &  &  &  &  &  &  &  &  &  \\
\hline
 &  &  &  &  &  &  &  &  &  &  &  &  &  &  &  &  &  &  &  &  &  &  \\
\hline
 &  &  &  &  &  &  &  &  &  &  &  &  &  &  &  &  &  &  &  &  &  &  \\
\hline
\end{tabular}
\end{center}

Odpowiedź:

\section*{Wypełnia egzaminator}
\begin{center}
\begin{tabular}{|l|c|c|}
\hline
Nr zadania & 29. & 30. \\
\hline
Maks. liczba pkt & 2 & 2 \\
\hline
Uzyskana liczba pkt &  &  \\
\hline
\end{tabular}
\end{center}

\section*{Zadanie 31. (0-2)}
Dany jest trójkąt prostokątny, którego przyprostokątne mają długości \(a\) i \(b\). Punkt \(O\) leży na przeciwprostokątnej tego trójkąta i jest środkiem okręgu stycznego do przyprostokątnych tego trójkąta (zobacz rysunek).

Wykaż, że promień \(\quad r\) tego okręgu jest równy \(\frac{a b}{a+b}\).\\
\includegraphics[max width=\textwidth, center]{2025_02_09_52b35926cd28a35c3960g-16}\\
b\\
\includegraphics[max width=\textwidth, center]{2025_02_09_52b35926cd28a35c3960g-16(1)}

Zadanie 32. (0-2)\\
Kąt \(\alpha\) jest ostry i \(\sin \alpha+\cos \alpha=\frac{7}{5}\). Oblicz wartość wyrażenia \(2 \sin \alpha \cos \alpha\).

\begin{center}
\begin{tabular}{|c|c|c|c|c|c|c|c|c|c|c|c|c|c|c|c|c|c|c|c|c|c|c|}
\hline
 &  &  &  &  &  &  &  &  &  &  &  &  &  &  &  &  &  &  &  &  &  &  \\
\hline
 &  &  &  &  &  &  &  &  &  &  &  &  &  &  &  &  &  &  &  &  &  &  \\
\hline
 &  &  &  &  &  &  &  &  &  &  &  &  &  &  &  &  &  &  &  &  &  &  \\
\hline
 &  &  &  &  &  &  &  &  &  &  &  &  &  &  &  &  &  &  &  &  &  &  \\
\hline
 &  &  &  &  &  &  &  &  &  &  &  &  &  &  &  &  &  &  &  &  &  &  \\
\hline
 &  &  &  &  &  &  &  &  &  &  &  &  &  &  &  &  &  &  &  &  &  &  \\
\hline
 &  &  &  &  &  &  &  &  &  &  &  &  &  &  &  &  &  &  &  &  &  &  \\
\hline
 &  &  &  &  &  &  &  &  &  &  &  &  &  &  &  &  &  &  &  &  &  &  \\
\hline
 &  &  &  &  &  &  &  &  &  &  &  &  &  &  &  &  &  &  &  &  &  &  \\
\hline
 &  &  &  &  &  &  &  &  &  &  &  &  &  &  &  &  &  &  &  &  &  &  \\
\hline
 &  &  &  &  &  &  &  &  &  &  &  &  &  &  &  &  &  &  &  &  &  &  \\
\hline
 &  &  &  &  &  &  &  &  &  &  &  &  &  &  &  &  &  &  &  &  &  &  \\
\hline
 &  &  &  &  &  &  &  &  &  &  &  &  &  &  &  &  &  &  &  &  &  &  \\
\hline
 &  &  &  &  &  &  &  &  &  &  &  &  &  &  &  &  &  &  &  &  &  &  \\
\hline
 &  &  &  &  &  &  &  &  &  &  &  &  &  &  &  &  &  &  &  &  &  &  \\
\hline
 &  &  &  &  &  &  &  &  &  &  &  &  &  &  &  &  &  &  &  &  &  &  \\
\hline
 &  &  &  &  &  &  &  &  &  &  &  &  &  &  &  &  &  &  &  &  &  &  \\
\hline
 &  &  &  &  &  &  &  &  &  &  &  &  &  &  &  &  &  &  &  &  &  &  \\
\hline
 &  &  &  &  &  &  &  &  &  &  &  &  &  &  &  &  &  &  &  &  &  &  \\
\hline
 &  &  &  &  &  &  &  &  &  &  &  &  &  &  &  &  &  &  &  &  &  &  \\
\hline
 &  &  &  &  &  &  &  &  &  &  &  &  &  &  &  &  &  &  &  &  &  &  \\
\hline
 &  &  &  &  &  &  &  &  &  &  &  &  &  &  &  &  &  &  &  &  &  &  \\
\hline
 &  &  &  &  &  &  &  &  &  &  &  &  &  &  &  &  &  &  &  &  &  &  \\
\hline
 &  &  &  &  &  &  &  &  &  &  &  &  &  &  &  &  &  &  &  &  &  &  \\
\hline
 &  &  &  &  &  &  &  &  &  &  &  &  &  &  &  &  &  &  &  &  &  &  \\
\hline
 &  &  &  &  &  &  &  &  &  &  &  &  &  &  &  &  &  &  &  &  &  &  \\
\hline
 &  &  &  &  &  &  &  &  &  &  &  &  &  &  &  &  &  &  &  &  &  &  \\
\hline
 &  &  &  &  &  &  &  &  &  &  &  &  &  &  &  &  &  &  &  &  &  &  \\
\hline
 &  &  &  &  &  &  &  &  &  &  &  &  &  &  &  &  &  &  &  &  &  &  \\
\hline
 &  &  &  &  &  &  &  &  &  &  &  &  &  &  &  &  &  &  &  &  &  &  \\
\hline
 &  &  &  &  &  &  &  &  &  &  &  &  &  &  &  &  &  &  &  &  &  &  \\
\hline
 &  &  &  &  &  &  &  &  &  &  &  &  &  &  &  &  &  &  &  &  &  &  \\
\hline
 &  &  &  &  &  &  &  &  &  &  &  &  &  &  &  &  &  &  &  &  &  &  \\
\hline
 &  &  &  &  &  &  &  &  &  &  &  &  &  &  &  &  &  &  &  &  &  &  \\
\hline
 &  &  &  &  &  &  &  &  &  &  &  &  &  &  &  &  &  &  &  &  &  &  \\
\hline
 &  &  &  &  &  &  &  &  &  &  &  &  &  &  &  &  &  &  &  &  &  &  \\
\hline
 &  &  &  &  &  &  &  &  &  &  &  &  &  &  &  &  &  &  &  &  &  &  \\
\hline
 &  &  &  &  &  &  &  &  &  &  &  &  &  &  &  &  &  &  &  &  &  &  \\
\hline
\end{tabular}
\end{center}

Odpowiedź:

\begin{center}
\begin{tabular}{|l|l|c|c|}
\hline
\multirow{3}{*}{\begin{tabular}{l}
Wypełnia \\
egzaminator \\
\end{tabular}} & Nr zadania & 31. & 32. \\
\cline { 2 - 4 }
 & Maks. liczba pkt & 2 & 2 \\
\cline { 2 - 4 }
 & Uzyskana liczba pkt &  &  \\
\hline
\end{tabular}
\end{center}

Zadanie 33. (0-2)\\
Dany jest czworokąt \(A B C D\), w którym \(|B C|=|C D|=|A D|=13\) (zobacz rysunek). Przekątna \(B D\) tego czworokąta ma długość 10 i jest prostopadła do boku \(A D\). Oblicz pole czworokąta \(A B C D\).\\
\includegraphics[max width=\textwidth, center]{2025_02_09_52b35926cd28a35c3960g-18}

Odpowiedź: \(\qquad\) .

Zadanie 34. (0-2)\\
Funkcja kwadratowa \(f(x)=x^{2}+b x+c\) nie ma miejsc zerowych. Wykaż, że \(1+c>b\).\\
\(\qquad\)

\begin{center}
\begin{tabular}{|l|l|c|c|}
\hline
\multirow{3}{*}{\begin{tabular}{l}
Wypełnia \\
egzaminator \\
\end{tabular}} & Nr zadania & 33. & 34 \\
\cline { 2 - 4 }
 & Maks. liczba pkt & 2 & 2 \\
\cline { 2 - 4 }
 & Uzyskana liczba pkt &  &  \\
\hline
\end{tabular}
\end{center}

Zadanie 35. (0-5)\\
Rosnący ciąg arytmetyczny \(\left(a_{n}\right)\) jest określony dla każdej liczby naturalnej \(n \geq 1\). Suma pierwszych pięciu wyrazów tego ciągu jest równa 10 . Wyrazy \(a_{3}, a_{5}, a_{13}\) tworzą - w podanej kolejności - ciąg geometryczny. Wyznacz wzór na \(n\)-ty wyraz ciągu arytmetycznego \(\left(a_{n}\right)\).

\begin{center}
\begin{tabular}{|c|c|c|c|c|c|c|c|c|c|c|c|c|c|c|c|c|c|c|c|c|c|c|c|c|c|c|c|}
\hline
 &  &  &  &  &  &  &  &  &  &  &  &  &  &  &  &  &  &  &  &  &  &  &  &  &  &  &  \\
\hline
 &  &  &  &  &  &  &  &  &  &  &  &  &  &  &  &  &  &  &  &  &  &  &  &  &  &  &  \\
\hline
 &  &  &  &  &  &  &  &  &  &  &  &  &  &  &  &  &  &  &  &  &  &  &  &  &  &  &  \\
\hline
 &  &  &  &  &  &  &  &  &  &  &  &  &  &  &  &  &  &  &  &  &  &  &  &  &  &  &  \\
\hline
 &  &  &  &  &  &  &  &  &  &  &  &  &  &  &  &  &  &  &  &  &  &  &  &  &  &  &  \\
\hline
 &  &  &  &  &  &  &  &  &  &  &  &  &  &  &  &  &  &  &  &  &  &  &  &  &  &  &  \\
\hline
 &  &  &  &  &  &  &  &  &  &  &  &  &  &  &  &  &  &  &  &  &  &  &  & \includegraphics[max width=\textwidth]{2025_02_09_52b35926cd28a35c3960g-20}
 &  &  &  \\
\hline
 &  &  &  &  &  &  &  &  &  &  &  &  &  &  &  &  &  &  &  &  &  &  &  &  &  &  &  \\
\hline
 &  &  &  &  &  &  &  &  &  &  &  &  &  &  &  &  &  &  &  &  &  &  &  &  &  &  &  \\
\hline
 &  &  &  &  &  &  &  &  &  &  &  &  &  &  &  &  &  &  &  &  &  &  &  &  &  &  &  \\
\hline
 &  &  &  &  &  &  &  &  &  &  &  &  &  &  &  &  &  &  &  &  &  &  &  &  &  &  &  \\
\hline
 &  &  &  &  &  &  &  &  &  &  &  &  &  &  &  &  &  &  &  &  &  &  &  &  &  &  &  \\
\hline
 &  &  &  &  &  &  &  &  &  &  &  &  &  &  &  &  &  &  &  &  &  &  &  &  &  &  &  \\
\hline
 &  &  &  &  &  &  &  &  &  &  &  &  &  &  &  &  &  &  &  &  &  &  &  &  &  &  &  \\
\hline
 &  &  &  &  &  &  &  &  &  &  &  &  &  &  &  &  &  &  &  &  &  &  &  &  &  &  &  \\
\hline
 &  &  &  &  &  &  &  &  &  &  &  &  &  &  &  &  &  &  &  &  &  &  &  &  &  &  &  \\
\hline
 &  &  &  &  &  &  &  &  &  &  &  &  &  &  &  &  &  &  &  &  &  &  &  &  &  &  &  \\
\hline
 &  &  &  &  &  &  &  &  &  &  &  &  &  &  &  &  &  &  &  &  &  &  &  &  &  &  &  \\
\hline
 &  &  &  &  &  &  &  &  &  &  &  &  &  &  &  &  &  &  &  &  &  &  &  &  &  &  &  \\
\hline
 &  &  &  &  &  &  &  &  &  &  &  &  &  &  &  &  &  &  &  &  &  &  &  &  &  &  &  \\
\hline
 &  &  &  &  &  &  &  &  &  &  &  &  &  &  &  &  &  &  &  &  &  &  &  &  &  &  &  \\
\hline
 &  &  &  &  &  &  &  &  &  &  &  &  &  &  &  &  &  &  &  &  &  &  &  &  &  &  &  \\
\hline
 &  &  &  &  &  &  &  &  &  &  &  &  &  &  &  &  &  &  &  &  &  &  &  &  &  &  &  \\
\hline
 &  &  &  &  &  &  &  &  &  &  &  &  &  &  &  &  &  &  &  &  &  &  &  &  &  &  &  \\
\hline
 &  &  &  &  &  &  &  &  &  &  &  &  &  &  &  &  &  &  &  &  &  &  &  &  &  &  &  \\
\hline
 &  &  &  &  &  &  &  &  &  &  &  &  &  &  &  &  &  &  &  &  &  &  &  &  &  &  &  \\
\hline
 &  &  &  &  &  &  &  &  &  &  &  &  &  &  &  &  &  &  &  &  &  &  &  &  &  &  &  \\
\hline
 &  &  &  &  &  &  &  &  &  &  &  &  &  &  &  &  &  &  &  &  &  &  &  &  &  &  &  \\
\hline
 &  &  &  &  &  &  &  &  &  &  &  &  &  &  &  &  &  &  &  &  &  &  &  &  &  &  &  \\
\hline
 &  &  &  &  &  &  &  &  &  &  &  &  &  &  &  &  &  &  &  &  &  &  &  &  &  &  &  \\
\hline
 &  &  &  &  &  &  &  &  &  &  &  &  &  &  &  &  &  &  &  &  &  &  &  &  &  &  &  \\
\hline
 &  &  &  &  &  &  &  &  &  &  &  &  &  &  &  &  &  &  &  &  &  &  &  &  &  &  &  \\
\hline
 &  &  &  &  &  &  &  &  &  &  &  &  &  &  &  &  &  &  &  &  &  &  &  &  &  &  &  \\
\hline
 &  &  &  &  &  &  &  &  &  &  &  &  &  &  &  &  &  &  &  &  &  &  &  &  &  &  &  \\
\hline
 &  &  &  &  &  &  &  &  &  &  &  &  &  &  &  &  &  &  &  &  &  &  &  &  &  &  &  \\
\hline
 &  &  &  &  &  &  &  &  &  &  &  &  &  &  &  &  &  &  &  &  &  &  &  &  &  &  &  \\
\hline
 &  &  &  &  &  &  &  &  &  &  &  &  &  &  &  &  &  &  &  &  &  &  &  &  &  &  &  \\
\hline
 &  &  &  &  &  &  &  &  &  &  &  &  &  &  &  &  &  &  &  &  &  &  &  &  &  &  &  \\
\hline
 &  &  &  &  &  &  &  &  &  &  &  &  &  &  &  &  &  &  &  &  &  &  &  &  &  &  &  \\
\hline
 &  &  &  &  &  &  &  &  &  &  &  &  &  &  &  &  &  &  &  &  &  &  &  &  &  &  &  \\
\hline
 &  &  &  &  &  &  &  &  &  &  &  &  &  &  &  &  &  &  &  &  &  &  &  &  &  &  &  \\
\hline
 &  &  &  &  &  &  &  &  &  &  &  &  &  &  &  &  &  &  &  &  &  &  &  &  &  &  &  \\
\hline
 &  &  &  &  &  &  &  &  &  &  &  &  &  &  &  &  &  &  &  &  &  &  &  &  &  &  &  \\
\hline
\end{tabular}
\end{center}

\begin{center}
\includegraphics[max width=\textwidth]{2025_02_09_52b35926cd28a35c3960g-21}
\end{center}

Odpowiedź: \(\qquad\)

\begin{center}
\begin{tabular}{|l|l|c|}
\hline
\multirow{3}{*}{\begin{tabular}{l}
Wypełnia \\
egzaminator \\
\end{tabular}} & Nr zadania & 35. \\
\cline { 2 - 3 }
 & Maks. liczba pkt & 5 \\
\cline { 2 - 3 }
 & Uzyskana liczba pkt &  \\
\hline
\end{tabular}
\end{center}

Strona 21 z 23

BRUDNOPIS (nie podlega ocenie)\\
\includegraphics[max width=\textwidth, center]{2025_02_09_52b35926cd28a35c3960g-22}\\
\includegraphics[max width=\textwidth, center]{2025_02_09_52b35926cd28a35c3960g-23}


\end{document}