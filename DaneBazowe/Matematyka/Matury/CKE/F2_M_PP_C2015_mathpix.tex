\documentclass[10pt]{article}
\usepackage[polish]{babel}
\usepackage[utf8]{inputenc}
\usepackage[T1]{fontenc}
\usepackage{graphicx}
\usepackage[export]{adjustbox}
\graphicspath{ {./images/} }
\usepackage{amsmath}
\usepackage{amsfonts}
\usepackage{amssymb}
\usepackage[version=4]{mhchem}
\usepackage{stmaryrd}

\author{Data: 2 czerwca 2015 r.\\
Godzina rozpoczęcia: 9:00\\
CZAS PRACY: \(\mathbf{1 7 0}\) minut\\
LicZba punktów do uzyskania: 50}
\date{}


\newcommand\Varangle{\mathop{{<\!\!\!\!\!\text{\small)}}\:}\nolimits}

\begin{document}
\maketitle
\begin{center}
\includegraphics[max width=\textwidth]{2025_02_10_691d6ddac4a5a8149712g-01(1)}
\end{center}

\section*{EGZAMIN MATURALNY Z MATEMATYKI POZIOM PODSTAWOWY}


\section*{Instrukcja dla zdającego}
\begin{enumerate}
  \item Sprawdź, czy arkusz egzaminacyjny zawiera 22 strony (zadania 1-34). Ewentualny brak zgłoś przewodniczącemu zespołu nadzorującego egzamin.
  \item Rozwiązania zadań i odpowiedzi wpisuj w miejscu na to przeznaczonym.
  \item Odpowiedzi do zadań zamkniętych (1-25) przenieś na kartę odpowiedzi, zaznaczając je w części karty przeznaczonej dla zdającego. Zamaluj pola do tego przeznaczone. Błędne zaznaczenie otocz kółkiem © i zaznacz właściwe.
  \item Pamiętaj, że pominięcie argumentacji lub istotnych obliczeń w rozwiązaniu zadania otwartego (26-34) może spowodować, że za to rozwiązanie nie otrzymasz pełnej liczby punktów.
  \item Pisz czytelnie i używaj tylko długopisu lub pióra z czarnym tuszem lub atramentem.
  \item Nie używaj korektora, a błędne zapisy wyraźnie przekreśl.
  \item Pamiętaj, że zapisy w brudnopisie nie będą oceniane.
  \item Możesz korzystać z zestawu wzorów matematycznych, cyrkla i linijki oraz kalkulatora prostego.
  \item Na tej stronie oraz na karcie odpowiedzi wpisz swój numer PESEL i przyklej naklejkę z kodem.
  \item Nie wpisuj żadnych znaków w części przeznaczonej dla egzaminatora.\\
\includegraphics[max width=\textwidth, center]{2025_02_10_691d6ddac4a5a8149712g-01}
\end{enumerate}

\section*{ZADANIA ZAMKNIETE}
W zadaniach od 1. do 25. wybierz poprawna odpowiedź i zaznacz ja na karcie odpowiedzi.

\section*{Zadanie 1. (0-1)}
Liczba \(2 \sqrt{18}-\sqrt{32}\) jest równa\\
A. \(2^{-\frac{3}{2}}\)\\
B. \(2^{-\frac{1}{2}}\)\\
C. \(2^{\frac{1}{2}}\)\\
D. \(2^{\frac{3}{2}}\)

\section*{Zadanie 2. (0-1)}
Wartość wyrażenia \(\frac{\sqrt[5]{-32} \cdot 2^{-1}}{4} \cdot 2^{2}\) jest równa\\
A. \(-\frac{1}{2}\)\\
B. \(\frac{1}{2}\)\\
C. 1\\
D. -1

\section*{Zadanie 3. (0-1)}
Przy 23-procentowej stawce podatku VAT cena brutto samochodu jest równa 45018 zł. Jaka jest cena netto tego samochodu?\\
A. \(34663,86 \mathrm{zf}\)\\
B. 36600 zt\\
C. \(44995 \mathrm{zł}\)\\
D. \(55372,14 \mathrm{zl}\)

\section*{Zadanie 4. (0-1)}
Wyrażenie \(3 a^{2}-12 a b+12 b^{2}\) może być przekształcone do postaci\\
A. \(3\left(a^{2}-b^{2}\right)^{2}\)\\
B. \(3\left(a-2 b^{2}\right)^{2}\)\\
C. \(3(a-2 b)^{2}\)\\
D. \(3(a+2 b)^{2}\)

\section*{Zadanie 5. (0-1)}
Para liczb \(x=2\) i \(y=1\) jest rozwiązaniem układu równań \(\left\{\begin{array}{l}x+a y=5 \\ 2 x-y=3\end{array}\right.\), gdy\\
A. \(a=-3\)\\
B. \(a=-2\)\\
C. \(a=2\)\\
D. \(a=3\)

\section*{Zadanie 6. (0-1)}
Równanie \(2 x^{2}+11 x+3=0\)\\
A. nie ma rozwiązań rzeczywistych.\\
B. ma dokładnie jedno rozwiązanie rzeczywiste.\\
C. ma dwa dodatnie rozwiązania rzeczywiste.\\
D. ma dwa ujemne rozwiązania rzeczywiste.

BRUDNOPIS (nie podlega ocenie)\\
\includegraphics[max width=\textwidth, center]{2025_02_10_691d6ddac4a5a8149712g-03}

\section*{Zadanie 7. (0-1)}
Wartość wyrażenia \(\sin 120^{\circ}-\cos 30^{\circ}\) jest równa\\
A. \(\sin 90^{\circ}\)\\
B. \(\sin 150^{\circ}\)\\
C. \(\sin 0^{\circ}\)\\
D. \(\sin 60^{\circ}\)

\section*{Zadanie 8. (0-1)}
Wyrażenie \(3 \sin ^{3} \alpha \cos \alpha+3 \sin \alpha \cos ^{3} \alpha\) może być przekształcone do postaci\\
A. 3\\
B. \(3 \sin \alpha \cos \alpha\)\\
C. \(3 \sin ^{3} \alpha \cos ^{3} \alpha\)\\
D. \(6 \sin ^{4} \alpha \cos ^{4} \alpha\)

\section*{Zadanie 9. (0-1)}
Na rysunku przedstawiony jest fragment prostej o równaniu \(y=a x+b\) przechodzącej przez punkty \((0,-2)\) i \((6,2)\).\\
\includegraphics[max width=\textwidth, center]{2025_02_10_691d6ddac4a5a8149712g-04}

Wtedy\\
A. \(a=\frac{2}{3}, b=-2\)\\
B. \(a=3, b=-2\)\\
C. \(a=\frac{3}{2}, b=2\)\\
D. \(a=-3, b=2\)

\section*{Zadanie 10. (0-1)}
Prosta \(k\) przecina oś \(O y\) układu współrzędnych w punkcie \((0,6)\) i jest równoległa do prostej o równaniu \(y=-3 x\). Wówczas prosta \(k\) przecina oś \(O x\) układu współrzędnych w punkcie\\
A. \((-12,0)\)\\
B. \((-2,0)\)\\
C. \((2,0)\)\\
D. \((6,0)\)

\section*{Zadanie 11. (0-1)}
Liczba niewymiernych rozwiązań równania \(x^{2}(x+5)(2 x-3)\left(x^{2}-7\right)=0\) jest równa\\
A. 0\\
B. 1\\
C. 5\\
D. 2

BRUDNOPIS (nie podlega ocenie)\\
\includegraphics[max width=\textwidth, center]{2025_02_10_691d6ddac4a5a8149712g-05}

\section*{Zadanie 12. (0-1)}
Na rysunku przedstawiono wykres funkcji \(f\).\\
\includegraphics[max width=\textwidth, center]{2025_02_10_691d6ddac4a5a8149712g-06}

Funkcja \(f\) jest rosnąca w przedziale\\
A. \(\langle-1,1\rangle\)\\
B. \(\langle 1,5\rangle\)\\
C. \(\langle 5,6\rangle\)\\
D. \(\langle 6,8\rangle\)

\section*{Zadanie 13. (0-1)}
Ciąg geometryczny \(\left(a_{n}\right)\) jest określony wzorem \(a_{n}=2^{n}\) dla \(n \geq 1\). Suma dziesięciu początkowych kolejnych wyrazów tego ciągu jest równa\\
A. \(2\left(1-2^{10}\right)\)\\
B. \(-2\left(1-2^{10}\right)\)\\
C. \(2\left(1+2^{10}\right)\)\\
D. \(-2\left(1+2^{10}\right)\)

\section*{Zadanie 14. (0-1)}
Suma pierwszego i szóstego wyrazu pewnego ciągu arytmetycznego jest równa 13. Wynika stąd, że suma trzeciego i czwartego wyrazu tego ciągu jest równa\\
A. 13\\
B. 12\\
C. 7\\
D. 6

\section*{Zadanie 15. (0-1)}
Miary kątów wewnętrznych pewnego trójkąta pozostają w stosunku 3:4:5. Najmniejszy kąt wewnętrzny tego trójkąta ma miarę\\
A. \(45^{\circ}\)\\
B. \(90^{\circ}\)\\
C. \(75^{\circ}\)\\
D. \(60^{\circ}\)

BRUDNOPIS (nie podlega ocenie)\\
\includegraphics[max width=\textwidth, center]{2025_02_10_691d6ddac4a5a8149712g-07}

\section*{Zadanie 16. (0-1)}
W trójkącie \(A B C\), w którym \(|A C|=|B C|\), na boku \(A B\) wybrano punkt \(D\) taki, że \(|B D|=|C D|\) oraz \(|\Varangle A C D|=21^{\circ}\) (zobacz rysunek).\\
\includegraphics[max width=\textwidth, center]{2025_02_10_691d6ddac4a5a8149712g-08(1)}

Wynika stąd, że kąt \(B C D\) ma miare\\
A. \(57^{\circ}\)\\
B. \(53^{\circ}\)\\
C. \(51^{\circ}\)\\
D. \(55^{\circ}\)

\section*{Zadanie 17. (0-1)}
Długości boków trójkąta są liczbami całkowitymi. Jeden bok ma 7 cm , a drugi ma 2 cm . Trzeci bok tego trójkąta może mieć długość\\
A. 12 cm\\
B. 9 cm\\
C. 6 cm\\
D. 3 cm

\section*{Zadanie 18. (0-1)}
Boki trójkąta mają długości 20 i 12, a kąt między tymi bokami ma miarę \(120^{\circ}\). Pole tego trójkąta jest równe\\
A. 60\\
B. 120\\
C. \(60 \sqrt{3}\)\\
D. \(120 \sqrt{3}\)

\section*{Zadanie 19. (0-1)}
Tworząca stożka o promieniu podstawy 3 ma długość 6 (zobacz rysunek).\\
\includegraphics[max width=\textwidth, center]{2025_02_10_691d6ddac4a5a8149712g-08}

Kąt \(\alpha\) rozwarcia tego stożka jest równy\\
A. \(30^{\circ}\)\\
B. \(45^{\circ}\)\\
C. \(60^{\circ}\)\\
D. \(90^{\circ}\)

BRUDNOPIS (nie podlega ocenie)\\
\includegraphics[max width=\textwidth, center]{2025_02_10_691d6ddac4a5a8149712g-09}

Zadanie 20. (0-1)\\
Graniastosłup o podstawie ośmiokąta ma dokładnie\\
A. 16 wierzchołków.\\
B. 9 wierzchołków.\\
C. 16 krawędzi.\\
D. 8 krawędzi.

\section*{Zadanie 21. (0-1)}
W ostrosłupie czworokątnym, w którym wszystkie krawędzie mają tę samą długość, kąt nachylenia krawędzi bocznej do płaszczyzny podstawy ma miarę\\
A. \(30^{\circ}\)\\
B. \(45^{\circ}\)\\
C. \(60^{\circ}\)\\
D. \(75^{\circ}\)

\section*{Zadanie 22. (0-1)}
Liczba 0,3 jest jednym z przybliżén liczby \(\frac{5}{16}\). Błąd względny tego przybliżenia, wyrażony w procentach, jest równy\\
A. \(4 \%\)\\
B. \(0,04 \%\)\\
C. \(2,5 \%\)\\
D. \(0,025 \%\)

\section*{Zadanie 23. (0-1)}
Średnia arytmetyczna zestawu danych: 2, 4, 7, 8, x jest równa \(n\), natomiast średnia arytmetyczna zestawu danych: \(2,4,7,8, x, 2 x\) jest równa \(2 n\). Wynika stąd, że\\
A. \(x=49\)\\
B. \(x=21\)\\
C. \(x=14\)\\
D. \(x=7\)

\section*{Zadanie 24. (0-1)}
Ile jest wszystkich liczb naturalnych dwucyfrowych podzielnych przez 6 i niepodzielnych przez 9 ?\\
A. 6\\
B. 10\\
C. 12\\
D. 15

\section*{Zadanie 25. (0-1)}
Na loterię przygotowano pulę 100 losów, w tym 4 wygrywające. Po wylosowaniu pewnej liczby losów, wśród których był dokładnie jeden wygrywający, szansa na wygraną była taka sama jak przed rozpoczęciem loterii. Stąd wynika, że wylosowano\\
A. 4 losy.\\
B. 20 losów.\\
C. 50 losów.\\
D. 25 losów.

BRUDNOPIS (nie podlega ocenie)\\
\includegraphics[max width=\textwidth, center]{2025_02_10_691d6ddac4a5a8149712g-11}

Zadanie 26. (0-2)\\
Rozwiąż nierówność \(3 x^{2}-9 x \leq x-3\).

\begin{center}
\begin{tabular}{|c|c|c|c|c|c|c|c|c|c|c|c|c|c|c|c|c|c|c|c|c|c|}
\hline
 &  &  &  &  &  &  &  &  &  &  &  &  &  &  &  &  &  &  &  &  &  \\
\hline
 &  &  &  &  &  &  &  &  &  &  &  &  &  &  &  &  &  &  &  &  &  \\
\hline
 &  &  &  &  &  &  &  &  &  &  &  &  &  &  &  &  &  &  &  &  &  \\
\hline
 &  &  &  &  &  &  &  &  &  &  &  &  &  &  &  &  &  &  &  &  &  \\
\hline
 &  &  &  &  &  &  &  &  &  &  &  &  &  &  &  &  &  &  &  &  &  \\
\hline
 &  &  &  &  &  &  &  &  &  &  &  &  &  &  &  &  &  &  &  &  &  \\
\hline
 &  &  &  &  &  &  &  &  &  &  &  &  &  &  &  &  &  &  &  &  &  \\
\hline
 &  &  &  &  &  &  &  &  &  &  &  &  &  &  &  &  &  &  &  &  &  \\
\hline
 &  &  &  &  &  &  &  &  &  &  &  &  &  &  &  &  &  &  &  &  &  \\
\hline
 &  &  &  &  &  &  &  &  &  &  &  &  &  &  &  &  &  &  &  &  &  \\
\hline
 &  &  &  &  &  &  &  &  &  &  &  &  &  &  &  &  &  &  &  &  &  \\
\hline
 &  &  &  &  &  &  &  &  &  &  &  &  &  &  &  &  &  &  &  &  &  \\
\hline
 &  &  &  &  &  &  &  &  &  &  &  &  &  &  &  &  &  &  &  &  &  \\
\hline
 &  &  &  &  &  &  &  &  &  &  &  &  &  &  &  &  &  &  &  &  &  \\
\hline
 &  &  &  &  &  &  &  &  &  &  &  &  &  &  &  &  &  &  &  &  &  \\
\hline
 &  &  &  &  &  &  &  &  &  &  &  &  &  &  &  &  &  &  &  &  &  \\
\hline
 &  &  &  &  &  &  &  &  &  &  &  &  &  &  &  &  &  &  &  &  &  \\
\hline
 &  &  &  &  &  &  &  &  &  &  &  &  &  &  &  &  &  &  &  &  &  \\
\hline
 &  &  &  &  &  &  &  &  &  &  &  &  &  &  &  &  &  &  &  &  &  \\
\hline
 &  &  &  &  &  &  &  &  &  &  &  &  &  &  &  &  &  &  &  &  &  \\
\hline
 &  &  &  &  &  &  &  &  &  &  &  &  &  &  &  &  &  &  &  &  &  \\
\hline
 &  &  &  &  &  &  &  &  &  &  &  &  &  &  &  &  &  &  &  &  &  \\
\hline
 &  &  &  &  &  &  &  &  &  &  &  &  &  &  &  &  &  &  &  &  &  \\
\hline
 &  &  &  &  &  &  &  &  &  &  &  &  &  &  &  &  &  &  &  &  &  \\
\hline
 &  &  &  &  &  &  &  &  &  &  &  &  &  &  &  &  &  &  &  &  &  \\
\hline
 &  &  &  &  &  &  &  &  &  &  &  &  &  &  &  &  &  &  &  &  &  \\
\hline
 &  &  &  &  &  &  &  &  &  &  &  &  &  &  &  &  &  &  &  &  &  \\
\hline
 &  &  &  &  &  &  &  &  &  &  &  &  &  &  &  &  &  &  &  &  &  \\
\hline
 &  &  &  &  &  &  &  &  &  &  &  &  &  &  &  &  &  &  &  &  &  \\
\hline
 &  &  &  &  &  &  &  &  &  &  &  &  &  &  &  &  &  &  &  &  &  \\
\hline
 &  &  &  &  &  &  &  &  &  &  &  &  &  &  &  &  &  &  &  &  &  \\
\hline
 &  &  &  &  &  &  &  &  &  &  &  &  &  &  &  &  &  &  &  &  &  \\
\hline
 &  &  &  &  &  &  &  &  &  &  &  &  &  &  &  &  &  &  &  &  &  \\
\hline
 &  &  &  &  &  &  &  &  &  &  &  &  &  &  &  &  &  &  &  &  &  \\
\hline
 &  &  &  &  &  &  &  &  &  &  &  &  &  &  &  &  &  &  &  &  &  \\
\hline
 &  &  &  &  &  &  &  &  &  &  &  &  &  &  &  &  &  &  &  &  &  \\
\hline
 &  &  &  &  &  &  &  &  &  &  &  &  &  &  &  &  &  &  &  &  &  \\
\hline
 &  &  &  &  &  &  &  &  &  &  &  &  &  &  &  &  &  &  &  &  &  \\
\hline
 &  &  &  &  &  &  &  &  &  &  &  &  &  &  &  &  &  &  &  &  &  \\
\hline
 &  &  &  &  &  &  &  &  &  &  &  &  &  &  &  &  &  &  &  &  &  \\
\hline
 &  &  &  &  &  &  &  &  &  &  &  &  &  &  &  &  &  &  &  &  &  \\
\hline
 &  &  &  &  &  &  &  &  &  &  &  &  &  &  &  &  &  &  &  &  &  \\
\hline
 &  &  &  &  &  &  &  &  &  &  &  &  &  &  &  &  &  &  &  &  &  \\
\hline
\end{tabular}
\end{center}

Odpowiedź:

\section*{Zadanie 27. (0-2)}
Rozwiąż równanie \(x\left(x^{2}-2 x+3\right)=0\).\\
\includegraphics[max width=\textwidth, center]{2025_02_10_691d6ddac4a5a8149712g-13}

Odpowiedź:

\section*{Zadanie 28. (0-2)}
Czworokąt \(A B C D\) wpisano w okrąg tak, że bok \(A B\) jest średnicą tego okręgu (zobacz rysunek). Udowodnij, że \(|A D|^{2}+|B D|^{2}=|B C|^{2}+|A C|^{2}\).\\
\includegraphics[max width=\textwidth, center]{2025_02_10_691d6ddac4a5a8149712g-14}\\
\includegraphics[max width=\textwidth, center]{2025_02_10_691d6ddac4a5a8149712g-14(1)}

\section*{Zadanie 29. (0-2)}
Udowodnij, że dla dowolnych liczb rzeczywistych \(x, y\) prawdziwa jest nierówność \(3 x^{2}+5 y^{2}-4 x y \geq 0\).\\
\includegraphics[max width=\textwidth, center]{2025_02_10_691d6ddac4a5a8149712g-15}

Zadanie 30. (0-2)\\
Funkcja kwadratowa, \(f\) dla \(x=-3\) przyjmuje wartość największą równą 4. Do wykresu funkcji \(f\) należy punkt \(A=(-1,3)\). Zapisz wzór funkcji kwadratowej \(f\).

\begin{center}
\begin{tabular}{|c|c|c|c|c|c|c|c|c|c|c|c|c|c|c|c|c|c|c|c|c|c|c|}
\hline
 &  &  &  &  &  &  &  &  &  &  &  &  &  &  &  &  &  &  &  &  &  &  \\
\hline
 &  &  &  &  &  &  &  &  &  &  &  &  &  &  &  &  &  &  &  &  &  &  \\
\hline
 &  &  &  &  &  &  &  &  &  &  &  &  &  &  &  &  &  &  &  &  &  &  \\
\hline
 &  &  &  &  &  &  &  &  &  &  &  &  &  &  &  &  &  &  &  &  &  &  \\
\hline
 &  &  &  &  &  &  &  &  &  &  &  &  &  &  &  &  &  &  &  &  &  &  \\
\hline
 &  &  &  &  &  &  &  &  &  &  &  &  &  &  &  &  &  &  &  &  &  &  \\
\hline
 &  &  &  &  &  &  &  &  &  &  &  &  &  &  &  &  &  &  &  &  &  &  \\
\hline
 &  &  &  &  &  &  &  &  &  &  &  &  &  &  &  &  &  &  &  &  &  &  \\
\hline
 &  &  &  &  &  &  &  &  &  &  &  &  &  &  &  &  &  &  &  &  &  &  \\
\hline
 &  &  &  &  &  &  &  &  &  &  &  &  &  &  &  &  &  &  &  &  &  &  \\
\hline
 &  &  &  &  &  &  &  &  &  &  &  &  &  &  &  &  &  &  &  &  &  &  \\
\hline
 &  &  &  &  &  &  &  &  &  &  &  &  &  &  &  &  &  &  &  &  &  &  \\
\hline
 &  &  &  &  &  &  &  &  &  &  &  &  &  &  &  &  &  &  &  &  &  &  \\
\hline
 &  &  &  &  &  &  &  &  &  &  &  &  &  &  &  &  &  &  &  &  &  &  \\
\hline
 &  &  &  &  &  &  &  &  &  &  &  &  &  &  &  &  &  &  &  &  &  &  \\
\hline
 &  &  &  &  &  &  &  &  &  &  &  &  &  &  &  &  &  &  &  &  &  &  \\
\hline
 &  &  &  &  &  &  &  &  &  &  &  &  &  &  &  &  &  &  &  &  &  &  \\
\hline
 &  &  &  &  &  &  &  &  &  &  &  &  &  &  &  &  &  &  &  &  &  &  \\
\hline
 &  &  &  &  &  &  &  &  &  &  &  &  &  &  &  &  &  &  &  &  &  &  \\
\hline
 &  &  &  &  &  &  &  &  &  &  &  &  &  &  &  &  &  &  &  &  &  &  \\
\hline
 &  &  &  &  &  &  &  &  &  &  &  &  &  &  &  &  &  &  &  &  &  &  \\
\hline
 &  &  &  &  &  &  &  &  &  &  &  &  &  &  &  &  &  &  &  &  &  &  \\
\hline
 &  &  &  &  &  &  &  &  &  &  &  &  &  &  &  &  &  &  &  &  &  &  \\
\hline
 &  &  &  &  &  &  &  &  &  &  &  &  &  &  &  &  &  &  &  &  &  &  \\
\hline
 &  &  &  &  &  &  &  &  &  &  &  &  &  &  &  &  &  &  &  &  &  &  \\
\hline
 &  &  &  &  &  &  &  &  &  &  &  &  &  &  &  &  &  &  &  &  &  &  \\
\hline
 &  &  &  &  &  &  &  &  &  &  &  &  &  &  &  &  &  &  &  &  &  &  \\
\hline
 &  &  &  &  &  &  &  &  &  &  &  &  &  &  &  &  &  &  &  &  &  &  \\
\hline
 &  &  &  &  &  &  &  &  &  &  &  &  &  &  &  &  &  &  &  &  &  &  \\
\hline
 &  &  &  &  &  &  &  &  &  &  &  &  &  &  &  &  &  &  &  &  &  &  \\
\hline
 &  &  &  &  &  &  &  &  &  &  &  &  &  &  &  &  &  &  &  &  &  &  \\
\hline
 &  &  &  &  &  &  &  &  &  &  &  &  &  &  &  &  &  &  &  &  &  &  \\
\hline
 &  &  &  &  &  &  &  &  &  &  &  &  &  &  &  &  &  &  &  &  &  &  \\
\hline
 &  &  &  &  &  &  &  &  &  &  &  &  &  &  &  &  &  &  &  &  &  &  \\
\hline
 &  &  &  &  &  &  &  &  &  &  &  &  &  &  &  &  &  &  &  &  &  &  \\
\hline
 &  &  &  &  &  &  &  &  &  &  &  &  &  &  &  &  &  &  &  &  &  &  \\
\hline
 &  &  &  &  &  &  &  &  &  &  &  &  &  &  &  &  &  &  &  &  &  &  \\
\hline
 &  &  &  &  &  &  &  &  &  &  &  &  &  &  &  &  &  &  &  &  &  &  \\
\hline
 &  &  &  &  &  &  &  &  &  &  &  &  &  &  &  &  &  &  &  &  &  &  \\
\hline
 &  &  &  &  &  &  &  &  &  &  &  &  &  &  &  &  &  &  &  &  &  &  \\
\hline
 &  &  &  &  &  &  &  &  &  &  &  &  &  &  &  &  &  &  &  &  &  &  \\
\hline
 &  &  &  &  &  &  &  &  &  &  &  &  &  &  &  &  &  &  &  &  &  &  \\
\hline
\end{tabular}
\end{center}

Odpowiedź:

\section*{Zadanie 31. (0-2)}
Ze zbioru liczb naturalnych dwucyfrowych losowo wybieramy jedną liczbę. Oblicz prawdopodobieństwo zdarzenia \(A\) polegającego na tym, że otrzymamy liczbę podzielną przez 8 lub liczbę podzielną przez 12.\\
\includegraphics[max width=\textwidth, center]{2025_02_10_691d6ddac4a5a8149712g-17}

Odpowiedź: \(\qquad\)

\section*{Zadanie 32. (0-4)}
Dany jest nieskończony rosnący ciąg arytmetyczny \(\left(a_{n}\right)\), dla \(n \geq 1\) taki, że \(a_{5}=18\). Wyrazy \(a_{1}, a_{3}\) oraz \(a_{13}\) tego ciągu są odpowiednio pierwszym, drugim i trzecim wyrazem pewnego ciągu geometrycznego. Wyznacz wzór na \(n\)-ty wyraz ciągu \(\left(a_{n}\right)\).\\
\includegraphics[max width=\textwidth, center]{2025_02_10_691d6ddac4a5a8149712g-18}

Odpowiedź: \(\qquad\)

\section*{Zadanie 33. (0-4)}
Dany jest trójkąt równoramienny \(A B C\), w którym \(|A C|=|B C|\). Ponadto wiadomo, że \(A=(-2,4)\) i \(B=(6,-2)\). Wierzchołek \(C\) należy do osi \(O y\). Oblicz współrzędne wierzchołka \(C\).\\
\includegraphics[max width=\textwidth, center]{2025_02_10_691d6ddac4a5a8149712g-19}

Odpowiedź:

\section*{Zadanie 34. (0-5)}
Objętość ostrosłupa prawidłowego trójkątnego \(A B C S\) jest równa \(27 \sqrt{3}\). Długość krawędzi \(A B\) podstawy ostrosłupa jest równa 6 (zobacz rysunek). Oblicz pole powierzchni całkowitej tego ostrosłupa.\\
\includegraphics[max width=\textwidth, center]{2025_02_10_691d6ddac4a5a8149712g-20}\\
\includegraphics[max width=\textwidth, center]{2025_02_10_691d6ddac4a5a8149712g-20(1)}\\
\includegraphics[max width=\textwidth, center]{2025_02_10_691d6ddac4a5a8149712g-21}

Odpowiedź:

\section*{BRUDNOPIS (nie podlega ocenie)}

\end{document}