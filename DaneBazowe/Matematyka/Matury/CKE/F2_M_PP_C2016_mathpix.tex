\documentclass[10pt]{article}
\usepackage[polish]{babel}
\usepackage[utf8]{inputenc}
\usepackage[T1]{fontenc}
\usepackage{graphicx}
\usepackage[export]{adjustbox}
\graphicspath{ {./images/} }
\usepackage{amsmath}
\usepackage{amsfonts}
\usepackage{amssymb}
\usepackage[version=4]{mhchem}
\usepackage{stmaryrd}

\author{DATA: 3 czerwca 2016 r.\\
Godzina rozpoczęcia: 9:00\\
CZAS PRACY: 170 minut\\
LicZba punktów do uzyskania: 50}
\date{}


\newcommand\Varangle{\mathop{{<\!\!\!\!\!\text{\small)}}\:}\nolimits}

\begin{document}
\maketitle
\begin{center}
\includegraphics[max width=\textwidth]{2025_02_10_0dbe9173aa4ecdfeee91g-01(1)}
\end{center}



\section*{Instrukcja dla zdającego}
\begin{enumerate}
  \item Sprawdź, czy arkusz egzaminacyjny zawiera 21 stron (zadania 1-33). Ewentualny brak zgłoś przewodniczącemu zespołu nadzorującego egzamin.
  \item Rozwiązania zadań i odpowiedzi wpisuj w miejscu na to przeznaczonym.
  \item Odpowiedzi do zadań zamkniętych (1-25) zaznacz na karcie odpowiedzi, w części karty przeznaczonej dla zdającego. Zamaluj \(\quad\) pola do tego przeznaczone. Błędne zaznaczenie otocz kółkiem \(\square_{\text {i zaznacz właściwe }}\)
  \item Pamiętaj, że pominięcie argumentacji lub istotnych obliczeń w rozwiązaniu zadania otwartego (26-33) może spowodować, że za to rozwiązanie nie otrzymasz pełnej liczby punktów.
  \item Pisz czytelnie i używaj tylko długopisu lub pióra z czarnym tuszem lub atramentem.
  \item Nie używaj korektora, a błędne zapisy wyraźnie przekreśl.
  \item Pamiętaj, że zapisy w brudnopisie nie będą oceniane.
  \item Możesz korzystać z zestawu wzorów matematycznych, cyrkla i linijki, a także z kalkulatora prostego.
  \item Na tej stronie oraz na karcie odpowiedzi wpisz swój numer PESEL i przyklej naklejkę z kodem.
  \item Nie wpisuj żadnych znaków w części przeznaczonej dla egzaminatora.\\
\includegraphics[max width=\textwidth, center]{2025_02_10_0dbe9173aa4ecdfeee91g-01}
\end{enumerate}

W zadaniach od 1. do 25. wybierz i zaznacz na karcie odpowiedzi poprawna odpowied́́.

\section*{Zadanie 1. (0-1)}
Liczba \(\frac{7^{6} \cdot 6^{7}}{42^{6}}\) jest równa\\
A. \(42^{36}\)\\
B. \(42^{7}\)\\
C. 6\\
D. 1

\section*{Zadanie 2. (0-1)}
Cenę pewnego towaru podwyższono o \(20 \%\), a następnie nową cenę tego towaru podwyższono o \(30 \%\). Takie dwie podwyżki ceny tego towaru można zastąpić równoważną im jedną podwyżką\\
A. o \(50 \%\)\\
B. \(\mathrm{o} 56 \%\)\\
C. \(\mathrm{o} 60 \%\)\\
D. \(066 \%\)

\section*{Zadanie 3. (0-1)}
Liczba \(\sqrt[3]{3 \sqrt{3}}\) jest równa\\
A. \(\sqrt[6]{3}\)\\
B. \(\sqrt[4]{3}\)\\
C. \(\sqrt[3]{3}\)\\
D. \(\sqrt{3}\)

\section*{Zadanie 4. (0-1)}
Różnica \(50001^{2}-49999^{2}\) jest równa\\
A. 2000000\\
B. 200000\\
C. 20000\\
D. 4

\section*{Zadanie 5. (0-1)}
Najmniejsza wartość wyrażenia \((x-y)(x+y)\) dla \(x, y \in\{2,3,4\}\) jest równa\\
A. 2\\
B. -24\\
C. 0\\
D. -12

\section*{Zadanie 6. (0-1)}
Wartość wyrażenia \(\log _{3} \frac{3}{2}+\log _{3} \frac{2}{9}\) jest równa\\
A. -1\\
B. -2\\
C. \(\log _{3} \frac{5}{11}\)\\
D. \(\log _{3} \frac{31}{18}\)

\section*{Zadanie 7. (0-1)}
Spośród liczb, które są rozwiązaniami równania \((x-8)\left(x^{2}-4\right)\left(x^{2}+16\right)=0\), wybrano największą i najmniejszą. Suma tych dwóch liczb jest równa\\
A. 12\\
B. 10\\
C. 6\\
D. 4

BRUDNOPIS (nie podlega ocenie)\\
\(\qquad\)

\section*{Zadanie 8. (0-1)}
Rozwiązaniem równania \(\frac{x-7}{x}=5\), gdzie \(x \neq 0\), jest liczba należąca do przedziału\\
A. \((-\infty,-2)\)\\
B. \(\langle-2,-1)\)\\
C. \(\langle-1,0)\)\\
D. \((0,+\infty)\)

\section*{Zadanie 9. (0-1)}
Funkcja \(f\) określona jest wzorem \(f(x)=\frac{2 x^{3}}{x^{4}+1}\) dla każdej liczby rzeczywistej \(x\). Wtedy liczba \(f(-\sqrt{2})\) jest równa\\
A. \(-\frac{8}{5}\)\\
B. \(-\frac{4 \sqrt{2}}{3}\)\\
C. \(-\frac{4 \sqrt{2}}{5}\)\\
D. \(-\frac{4}{3}\)

\section*{Zadanie 10. (0-1)}
Dana jest funkcja kwadratowa \(f(x)=-2(x+5)(x-11)\). Wskaż maksymalny przedział, w którym funkcja \(f\) jest rosnąca.\\
A. \((-\infty, 3\rangle\)\\
B. \((-\infty, 5\rangle\)\\
C. \((-\infty, 11\rangle\)\\
D. \(\langle 6,+\infty)\)

\section*{Zadanie 11. (0-1)}
Ciąg \(\left(a_{n}\right)\) jest określony wzorem \(a_{n}=6(n-16)\) dla \(n \geq 1\). Suma dziesięciu początkowych wyrazów tego ciągu jest równa\\
A. -54\\
B. -126\\
C. -630\\
D. -270

\section*{Zadanie 12. (0-1)}
Dany jest ciąg geometryczny \(\left(a_{n}\right)\), w którym \(a_{1}=72\) i \(a_{4}=9\). Iloraz \(q\) tego ciągu jest równy\\
A. \(q=\frac{1}{2}\)\\
B. \(q=\frac{1}{6}\)\\
C. \(q=\frac{1}{4}\)\\
D. \(q=\frac{1}{8}\)

\section*{Zadanie 13. (0-1)}
Dany jest trapez \(A B C D\), w którym przekątna \(A C\) jest prostopadła do ramienia \(B C\), \(|A D|=|D C|\) oraz \(|\Varangle A B C|=50^{\circ}\) (zobacz rysunek).\\
\includegraphics[max width=\textwidth, center]{2025_02_10_0dbe9173aa4ecdfeee91g-04}

Stąd wynika, że\\
A. \(\beta=100^{\circ}\)\\
B. \(\beta=120^{\circ}\)\\
C. \(\beta=110^{\circ}\)\\
D. \(\beta=130^{\circ}\)

BRUDNOPIS (nie podlega ocenie)\\
\includegraphics[max width=\textwidth, center]{2025_02_10_0dbe9173aa4ecdfeee91g-05}

\section*{Zadanie 14. (0-1)}
Punkty \(A, B, C\) i \(D\) leżą na okręgu o środku \(O\) (zobacz rysunek). Miary zaznaczonych kątów \(\alpha\) i \(\beta\) są odpowiednio równe\\
\includegraphics[max width=\textwidth, center]{2025_02_10_0dbe9173aa4ecdfeee91g-06}\\
A. \(\alpha=36^{\circ}, \beta=72^{\circ}\)\\
B. \(\alpha=54^{\circ}, \beta=72^{\circ}\)\\
C. \(\alpha=36^{\circ}, \beta=108^{\circ}\)\\
D. \(\alpha=72^{\circ}, \beta=72^{\circ}\)

\section*{Zadanie 15. (0-1)}
Słoń waży 5 ton, a waga mrówki jest równa 0,5 grama. Ile razy słoń jest cięższy od mrówki?\\
A. \(10^{6}\)\\
B. \(10^{7}\)\\
C. 10\\
D. \(10^{8}\)

\section*{Zadanie 16. (0-1)}
Każde z ramion trójkąta równoramiennego ma długość 20 . Kąt zawarty między ramionami tego trójkąta ma miarę \(150^{\circ}\). Pole tego trójkąta jest równe\\
A. 100\\
B. 200\\
C. \(100 \sqrt{3}\)\\
D. \(100 \sqrt{2}\)

\section*{Zadanie 17. (0-1)}
Prosta określona wzorem \(y=a x+1\) jest symetralną odcinka \(A B\), gdzie \(A=(-3,2)\) i \(B=(1,4)\). Wynika stąd, że\\
A. \(a=-\frac{1}{2}\)\\
B. \(\quad a=\frac{1}{2}\)\\
C. \(a=-2\)\\
D. \(a=2\)

BRUDNOPIS (nie podlega ocenie)\\
\includegraphics[max width=\textwidth, center]{2025_02_10_0dbe9173aa4ecdfeee91g-07}

\section*{Zadanie 18. (0-1)}
Układ równań \(\left\{\begin{array}{l}y=-a x+2 a \\ y=\frac{b}{3} x-2\end{array}\right.\) nie ma rozwiązań dla\\
A. \(a=-1\) i \(b=-3\)\\
B. \(\quad a=1\) i \(b=3\)\\
C. \(a=1\) i \(b=-3\)\\
D. \(a=-1\) i \(b=3\)

\section*{Zadanie 19. (0-1)}
Do pewnej liczby a dodano 54. Otrzymaną sumę podzielono przez 2. W wyniku tego działania otrzymano liczbę dwa razy większą od liczby \(a\). Zatem\\
A. \(a=27\)\\
B. \(a=18\)\\
C. \(a=24\)\\
D. \(a=36\)

\section*{Zadanie 20. (0-1)}
Podstawą ostrosłupa prawidłowego czworokątnego \(A B C D S\) jest kwadrat \(A B C D\). Wszystkie ściany boczne tego ostrosłupa są trójkątami równobocznymi. Miara kąta \(A S C\) jest równa\\
A. \(45^{\circ}\)\\
B. \(30^{\circ}\)\\
C. \(75^{\circ}\)\\
D. \(90^{\circ}\)

\section*{Zadanie 21. (0-1)}
Rzucamy trzy razy symetryczną monetą. Niech \(p\) oznacza prawdopodobieństwo otrzymania dokładnie jednego orła w tych trzech rzutach. Wtedy\\
A. \(0 \leq p<0,25\)\\
B. \(0,25 \leq p \leq 0,4\)\\
C. \(0,4<p \leq 0,5\)\\
D. \(p>0,5\)

\section*{Zadanie 22. (0-1)}
Średnia arytmetyczna czterech liczb: \(x-1,3 x, 5 x+1\) i \(7 x\) jest równa 72 . Wynika stąd, że\\
A. \(x=9\)\\
B. \(x=10\)\\
C. \(x=17\)\\
D. \(x=18\)

BRUDNOPIS (nie podlega ocenie)\\
\includegraphics[max width=\textwidth, center]{2025_02_10_0dbe9173aa4ecdfeee91g-09}

\section*{Zadanie 23. (0-1)}
Na rysunku przedstawione są dwie proste równoległe \(k\) i \(l\) o równaniach \(y=a x+b\) oraz \(y=m x+n\). Początek układu współrzędnych leży między tymi prostymi.\\
\includegraphics[max width=\textwidth, center]{2025_02_10_0dbe9173aa4ecdfeee91g-10(1)}

Zatem\\
A. \(a \cdot m>0\) i \(b \cdot n>0\)\\
B. \(a \cdot m>0\) i \(b \cdot n<0\)\\
C. \(a \cdot m<0\) i \(b \cdot n>0\)\\
D. \(a \cdot m<0\) i \(b \cdot n<0\)

\section*{Zadanie 24. (0-1)}
Dane są dwie sumy algebraiczne \(3 x^{3}-2 x\) oraz \(-3 x^{2}-2\). Iloczyn tych sum jest równy\\
A. \(-9 x^{5}+4 x\)\\
B. \(-9 x^{6}+6 x^{3}-6 x^{2}+4 x\)\\
C. \(-9 x^{5}+6 x^{3}-6 x^{2}+4 x\)\\
D. \(-9 x^{6}+4 x\)

\section*{Zadanie 25. (0-1)}
Punkty \(D\) i \(E\) są środkami przyprostokątnych \(A C\) i \(B C\) trójkąta prostokątnego \(A B C\). Punkty \(F\) i G leżą na przeciwprostokątnej \(A B\) tak, że odcinki \(D F\) i \(E G\) są do niej prostopadłe (zobacz rysunek). Pole trójkąta \(B G E\) jest równe 1, a pole trójkąta \(A F D\) jest równe 4.\\
\includegraphics[max width=\textwidth, center]{2025_02_10_0dbe9173aa4ecdfeee91g-10}

Zatem pole trójkąta \(A B C\) jest równe\\
A. 12\\
B. 16\\
C. 18\\
D. 20

BRUDNOPIS (nie podlega ocenie)\\
\includegraphics[max width=\textwidth, center]{2025_02_10_0dbe9173aa4ecdfeee91g-11}

\section*{Zadanie 26. (0-2)}
Rozwiąż równanie \(\frac{2 x+1}{2 x}=\frac{2 x+1}{x+1}\), gdzie \(x \neq-1\) i \(x \neq 0\).

\begin{center}
\begin{tabular}{|c|c|c|c|c|c|c|c|c|c|c|c|c|c|c|c|c|c|c|c|c|c|}
\hline
 &  &  &  &  &  &  &  &  &  &  &  &  &  &  &  &  &  &  &  &  &  \\
\hline
 &  &  &  &  &  &  &  &  &  &  &  &  &  &  &  &  &  &  &  &  &  \\
\hline
 &  &  &  &  &  &  &  &  &  &  &  &  &  &  &  &  &  &  &  &  &  \\
\hline
 &  &  &  &  &  &  &  &  &  &  &  &  &  &  &  &  &  &  &  &  &  \\
\hline
 &  &  &  &  &  &  &  &  &  &  &  &  &  &  &  &  &  &  &  &  &  \\
\hline
 &  &  &  &  &  &  &  &  &  &  &  &  &  &  &  &  &  &  &  &  &  \\
\hline
 &  &  &  &  &  &  &  &  &  &  &  &  &  &  &  &  &  &  &  &  &  \\
\hline
 &  &  &  &  &  &  &  &  &  &  &  &  &  &  &  &  &  &  &  &  &  \\
\hline
 &  &  &  &  &  &  &  &  &  &  &  &  &  &  &  &  &  &  &  &  &  \\
\hline
 &  &  &  &  &  &  &  &  &  &  &  &  &  &  &  & - &  &  &  &  &  \\
\hline
 &  &  &  &  &  &  &  &  &  &  &  &  &  &  &  &  &  &  &  &  &  \\
\hline
 &  &  &  &  &  &  &  &  &  &  &  &  &  &  &  &  &  &  &  &  &  \\
\hline
 &  &  &  &  &  &  &  &  &  &  &  &  &  &  &  &  &  &  &  &  &  \\
\hline
 &  &  &  &  &  &  &  &  &  &  &  &  &  &  &  &  &  &  &  &  &  \\
\hline
 &  &  &  &  &  &  &  &  &  &  &  &  &  &  &  &  &  &  &  &  &  \\
\hline
 &  &  &  &  &  &  &  &  &  &  &  &  &  &  &  &  &  &  &  &  &  \\
\hline
 &  &  &  &  &  &  &  &  &  &  &  &  &  &  &  &  &  &  &  &  &  \\
\hline
 &  &  &  &  &  &  &  &  &  &  &  &  &  &  &  &  &  &  &  &  &  \\
\hline
 &  &  &  &  &  &  &  &  &  &  &  &  &  &  &  &  &  &  &  &  &  \\
\hline
 &  &  &  &  &  &  &  &  &  &  &  &  &  &  &  &  &  &  &  &  &  \\
\hline
 &  &  &  &  &  &  &  &  &  &  &  &  &  &  &  &  &  &  &  &  &  \\
\hline
 &  &  &  &  &  &  &  &  &  &  &  &  &  &  &  &  &  &  &  &  &  \\
\hline
 &  &  &  &  &  &  &  &  &  &  &  &  &  &  &  &  &  &  &  &  &  \\
\hline
 &  &  &  &  &  &  &  &  &  &  &  &  &  &  &  &  &  &  &  &  &  \\
\hline
 &  &  &  &  &  &  &  &  &  &  &  &  &  &  &  &  &  &  &  &  &  \\
\hline
 &  &  &  &  &  &  &  &  &  &  &  &  &  &  &  &  &  &  &  &  &  \\
\hline
 &  &  &  &  &  &  &  &  &  &  &  &  &  &  &  &  &  &  &  &  &  \\
\hline
 &  &  &  &  &  &  &  &  &  &  &  &  &  &  &  &  &  &  &  &  &  \\
\hline
 &  &  &  &  &  &  &  &  &  &  &  &  &  &  &  &  &  &  &  &  &  \\
\hline
 &  &  &  &  &  &  &  &  &  &  &  &  &  &  &  &  &  &  &  &  &  \\
\hline
 &  &  &  &  &  &  &  &  &  &  &  &  &  &  &  &  &  &  &  &  &  \\
\hline
 &  &  &  &  &  &  &  &  &  &  &  &  &  &  &  &  &  &  &  &  &  \\
\hline
 &  &  &  &  &  &  &  &  &  &  &  &  &  &  &  &  &  &  &  &  &  \\
\hline
 &  &  &  &  &  &  &  &  &  &  &  &  &  &  &  &  &  &  &  &  &  \\
\hline
 &  &  &  &  &  &  &  &  &  &  &  &  &  &  &  &  &  &  &  &  &  \\
\hline
 &  &  &  &  &  &  &  &  &  &  &  &  &  &  &  &  &  &  &  &  &  \\
\hline
 &  &  &  &  &  &  &  &  &  &  &  &  &  &  &  &  &  &  &  &  &  \\
\hline
 &  &  &  &  &  &  &  &  &  &  &  &  &  &  &  &  &  &  &  &  &  \\
\hline
 &  &  &  &  &  &  &  &  &  &  &  &  &  &  &  &  &  &  &  &  &  \\
\hline
 &  &  &  &  &  &  &  &  &  &  &  &  &  &  &  &  &  &  &  &  &  \\
\hline
 &  &  &  &  &  &  &  &  &  &  &  &  &  &  &  &  &  &  &  &  &  \\
\hline
 &  &  &  &  &  &  &  &  &  &  &  &  &  &  &  &  &  &  &  &  &  \\
\hline
 &  &  &  &  &  &  &  &  &  &  &  &  &  &  &  &  &  &  &  &  &  \\
\hline
\end{tabular}
\end{center}

Odpowiedź:

\section*{Zadanie 27. (0-2)}
Dane są proste o równaniach \(y=x+2\) oraz \(y=-3 x+b\), które przecinają się w punkcie leżącym na osi \(O y\) układu współrzędnych. Oblicz pole trójkąta, którego dwa boki zawierają się w danych prostych, a trzeci jest zawarty w osi \(O x\).

\begin{center}
\begin{tabular}{|c|c|c|c|c|c|c|c|c|c|c|c|c|c|c|c|c|c|c|c|c|c|c|c|c|c|c|c|c|c|}
\hline
 &  &  &  &  &  &  &  &  &  &  &  &  &  &  &  &  &  &  &  &  &  &  &  &  &  &  &  &  &  \\
\hline
 &  &  &  &  &  &  &  &  &  &  &  &  &  &  &  &  &  &  &  &  &  &  &  &  &  &  &  &  &  \\
\hline
 &  &  &  &  &  &  &  &  &  &  &  &  &  &  &  &  &  &  &  &  &  &  &  &  &  &  &  &  &  \\
\hline
 &  &  &  &  &  &  &  &  &  &  &  &  &  &  &  &  &  &  &  &  &  &  &  &  &  &  &  &  &  \\
\hline
 &  &  &  &  &  &  &  &  &  &  &  &  &  &  &  &  &  &  &  &  &  &  &  &  &  &  &  &  &  \\
\hline
 &  &  &  &  &  &  &  &  &  &  &  &  &  &  &  &  &  &  &  &  &  &  &  &  &  &  &  &  &  \\
\hline
 &  &  &  &  &  &  &  &  &  &  &  &  &  &  &  &  &  &  &  &  &  &  &  &  &  &  &  &  &  \\
\hline
 &  &  &  &  &  &  &  &  &  &  &  &  &  &  &  &  &  &  &  &  &  &  &  &  &  &  &  &  &  \\
\hline
 &  &  &  &  &  &  &  &  &  &  &  &  &  &  &  &  &  &  &  &  &  &  &  &  &  &  &  &  &  \\
\hline
 &  &  &  &  &  &  &  &  &  &  &  &  &  &  &  &  &  &  &  &  &  &  &  &  &  &  &  &  &  \\
\hline
 &  &  &  &  &  &  &  &  &  &  &  &  &  &  &  &  &  &  &  &  &  &  &  &  &  &  &  &  &  \\
\hline
 &  &  &  &  &  &  &  &  &  &  &  &  &  &  &  &  &  &  &  &  &  &  &  &  &  &  &  &  &  \\
\hline
 &  &  &  &  &  &  &  &  &  &  &  &  &  &  &  &  &  &  &  &  &  &  &  &  &  &  &  &  &  \\
\hline
 &  &  &  &  &  &  &  &  &  &  &  &  &  &  &  &  &  &  &  &  &  &  &  &  &  &  &  &  &  \\
\hline
 &  &  &  &  &  &  &  &  &  &  &  &  &  &  &  &  &  &  &  &  &  &  &  &  &  &  &  &  &  \\
\hline
 &  &  &  &  &  &  &  &  &  &  &  &  &  &  &  &  &  &  &  &  &  &  &  &  &  &  &  &  &  \\
\hline
 &  &  &  &  &  &  &  &  &  &  &  &  &  &  &  &  &  &  &  &  &  &  &  &  &  &  &  &  &  \\
\hline
 &  &  &  &  &  &  &  &  &  &  &  &  &  &  &  &  &  &  &  &  &  &  &  &  &  &  &  &  &  \\
\hline
 &  &  &  &  &  &  &  &  &  &  &  &  &  &  &  &  &  &  &  &  &  &  &  &  &  &  &  &  &  \\
\hline
 &  &  &  &  &  &  &  &  &  &  &  &  &  &  &  &  &  &  &  &  &  &  &  &  &  &  &  &  &  \\
\hline
 &  &  &  &  &  &  &  &  &  &  &  &  &  &  &  &  &  &  &  &  &  &  &  &  &  &  &  &  &  \\
\hline
 &  &  &  &  &  &  &  &  &  &  &  &  &  &  &  &  &  &  &  &  &  &  &  &  &  &  &  &  &  \\
\hline
 &  &  &  &  &  &  &  &  &  &  &  &  &  &  &  &  &  &  &  &  &  &  &  &  &  &  &  &  &  \\
\hline
 &  &  &  &  &  &  &  &  &  &  &  &  &  &  &  &  &  &  &  &  &  &  &  &  &  &  &  &  &  \\
\hline
 &  &  &  &  &  &  &  &  &  &  &  &  &  &  &  &  &  &  &  &  &  &  &  &  &  &  &  &  &  \\
\hline
 &  &  &  &  &  &  &  &  &  &  &  &  &  &  &  &  &  &  &  &  &  &  &  &  &  &  &  &  &  \\
\hline
 &  &  &  &  &  & - &  &  &  &  &  &  &  &  &  &  &  &  &  &  &  &  &  &  &  &  &  &  &  \\
\hline
 &  &  &  &  &  &  &  &  &  &  &  &  &  &  &  &  &  &  &  &  &  &  &  &  &  &  &  &  &  \\
\hline
 & \(\square\) &  &  &  &  &  &  &  &  &  &  &  &  &  &  &  &  &  &  &  &  &  &  &  &  &  &  &  &  \\
\hline
 & \(\bigcirc\) &  &  &  &  &  &  &  &  &  &  &  &  &  &  &  &  &  &  &  &  &  &  &  &  &  &  &  &  \\
\hline
 &  &  &  &  &  &  &  &  &  &  &  &  &  &  &  &  &  &  &  &  &  &  &  &  &  &  &  &  &  \\
\hline
 &  &  &  &  &  &  &  &  &  &  &  &  &  &  &  &  &  &  &  &  &  &  &  &  &  &  &  &  &  \\
\hline
 & - &  &  &  &  &  &  &  &  &  &  &  &  &  &  &  &  &  &  &  &  &  &  &  &  &  &  &  &  \\
\hline
 & - &  &  &  &  &  &  &  &  &  &  &  &  &  &  &  &  &  &  &  &  &  &  &  &  &  &  &  &  \\
\hline
 &  &  &  &  &  &  &  &  &  &  &  &  &  &  &  &  &  &  &  &  &  &  &  &  &  &  &  &  &  \\
\hline
 & \(\square\) &  &  &  &  &  &  &  &  &  &  &  &  &  &  &  &  &  &  &  &  &  &  &  &  &  &  &  &  \\
\hline
 & - &  &  &  &  &  &  &  &  &  &  &  &  &  &  &  &  &  &  &  &  &  &  &  &  &  &  &  &  \\
\hline
 &  &  &  &  &  &  &  &  &  &  &  &  &  &  &  &  &  &  &  &  &  &  &  &  &  &  &  &  &  \\
\hline
 &  &  &  &  &  &  &  &  &  &  &  &  &  &  &  &  &  &  &  & \includegraphics[max width=\textwidth]{2025_02_10_0dbe9173aa4ecdfeee91g-13}
 &  &  &  &  &  &  &  &  &  \\
\hline
 &  &  &  &  &  &  &  &  &  &  &  &  &  &  &  &  &  &  &  &  &  &  &  &  &  &  &  &  &  \\
\hline
\end{tabular}
\end{center}

Odpowiedź: \(\qquad\)

\section*{Zadanie 28. (0-2)}
Wykaż, że dla dowolnych liczb rzeczywistych \(x, y\) prawdziwa jest nierówność

\[
x^{4}+y^{4}+x^{2}+y^{2} \geq 2\left(x^{3}+y^{3}\right)
\]

\begin{center}
\includegraphics[max width=\textwidth]{2025_02_10_0dbe9173aa4ecdfeee91g-14}
\end{center}

\section*{Zadanie 29. (0-2)}
Dany jest trapez prostokątny \(A B C D\) o podstawach \(A B\) i \(C D\) oraz wysokości \(A D\). Dwusieczna kąta \(A B C\) przecina ramię \(A D\) w punkcie \(E\) oraz dwusieczną kąta \(B C D\) w punkcie \(F\) (zobacz rysunek).\\
\includegraphics[max width=\textwidth, center]{2025_02_10_0dbe9173aa4ecdfeee91g-15}

Wykaż, że w czworokącie \(C D E F\) sumy miar przeciwległych kątów są sobie równe.\\
\includegraphics[max width=\textwidth, center]{2025_02_10_0dbe9173aa4ecdfeee91g-15(1)}

\section*{Zadanie 30. (0-4)}
W trójkącie \(A B C\) dane są długości boków \(|A B|=15\) i \(|A C|=12\) oraz \(\cos \alpha=\frac{4}{5}\), gdzie \(\alpha=\Varangle B A C\). Na bokach \(A B\) i \(A C\) tego trójkąta obrano punkty odpowiednio \(D\) i \(E\) takie, że \(|B D|=2|A D|\) i \(|A E|=2|C E|\) (zobacz rysunek).\\
\includegraphics[max width=\textwidth, center]{2025_02_10_0dbe9173aa4ecdfeee91g-16}

Oblicz pole\\
a) trójkąta \(A D E\).\\
b) czworokąta \(B C E D\).\\
\includegraphics[max width=\textwidth, center]{2025_02_10_0dbe9173aa4ecdfeee91g-16(1)}\\
\includegraphics[max width=\textwidth, center]{2025_02_10_0dbe9173aa4ecdfeee91g-17}

Odpowiedź:

\section*{Zadanie 31. (0-5)}
Dany jest ciąg arytmetyczny \(\left(a_{n}\right)\) określony dla każdej liczby naturalnej \(n \geq 1\), w którym \(a_{1}+a_{2}+a_{3}+a_{4}=2016\) oraz \(a_{5}+a_{6}+a_{7}+\ldots+a_{12}=2016\). Oblicz pierwszy wyraz, różnicę oraz najmniejszy dodatni wyraz ciągu \(\left(a_{n}\right)\).

\begin{center}
\begin{tabular}{|c|c|c|c|c|c|c|c|c|c|c|c|c|c|c|c|c|c|c|c|c|c|}
\hline
 &  &  &  &  &  &  &  &  &  &  &  &  &  &  &  &  &  &  &  &  &  \\
\hline
 &  &  &  &  &  &  &  &  &  &  &  &  &  &  &  &  &  &  &  &  &  \\
\hline
 &  &  &  &  &  &  &  &  &  &  &  &  &  &  &  &  &  &  &  &  &  \\
\hline
 &  &  &  &  &  &  &  &  &  &  &  &  &  &  &  &  &  &  &  &  &  \\
\hline
 &  &  &  &  &  &  &  &  &  &  &  &  &  &  &  &  &  &  &  &  &  \\
\hline
 &  &  &  &  &  &  &  &  &  &  &  &  &  &  &  &  &  &  &  &  &  \\
\hline
 &  &  &  &  &  &  &  &  &  &  &  &  &  &  &  &  &  &  &  &  &  \\
\hline
 &  &  &  &  &  &  &  &  &  &  &  &  &  &  &  &  &  &  &  &  &  \\
\hline
 &  &  &  &  &  &  &  &  &  &  &  &  &  &  &  &  &  &  &  &  &  \\
\hline
 &  &  &  &  &  &  &  &  &  &  &  &  &  &  &  &  &  &  &  &  &  \\
\hline
 &  &  &  &  &  &  &  &  &  &  &  &  &  &  &  &  &  &  &  &  &  \\
\hline
 &  &  &  &  &  &  &  &  &  &  &  &  &  &  &  &  &  &  &  &  &  \\
\hline
 &  &  &  &  &  &  &  &  &  &  &  &  &  &  &  &  &  &  &  &  &  \\
\hline
 &  &  &  &  &  &  &  &  &  &  &  &  &  &  &  &  &  &  &  &  &  \\
\hline
 &  &  &  &  &  &  &  &  &  &  &  &  &  &  &  &  &  &  &  &  &  \\
\hline
 &  &  &  &  &  &  &  &  &  &  &  &  &  &  &  &  &  &  &  &  &  \\
\hline
 &  &  &  &  &  &  &  &  &  &  &  &  &  &  &  &  &  &  &  &  &  \\
\hline
 &  &  &  &  &  &  &  &  &  &  &  &  &  &  &  &  &  &  &  &  &  \\
\hline
 &  &  &  &  &  &  &  &  &  &  &  &  &  &  &  &  &  &  &  &  &  \\
\hline
 &  &  &  &  &  &  &  &  &  &  &  &  &  &  &  &  &  &  &  &  &  \\
\hline
 &  &  &  &  &  &  &  &  &  &  &  &  &  &  &  &  &  &  &  &  &  \\
\hline
 &  &  &  &  &  &  &  &  &  &  &  &  &  &  &  &  &  &  &  &  &  \\
\hline
 &  &  &  &  &  &  &  &  &  &  &  &  &  &  &  &  &  &  &  &  &  \\
\hline
 &  &  &  &  &  &  &  &  &  &  &  &  &  &  &  &  &  &  &  &  &  \\
\hline
 &  &  &  &  &  &  &  &  &  &  &  &  &  &  &  &  &  &  &  &  &  \\
\hline
 &  &  &  &  &  &  &  &  &  &  &  &  &  &  &  &  &  &  &  &  &  \\
\hline
 &  &  &  &  &  &  &  &  &  &  &  &  &  &  &  &  &  &  &  &  &  \\
\hline
 &  &  &  &  &  &  &  &  &  &  &  &  &  &  &  &  &  &  &  &  &  \\
\hline
 &  &  &  &  &  &  &  &  &  &  &  &  &  &  &  &  &  &  &  &  &  \\
\hline
 &  &  &  &  &  &  &  &  &  &  &  &  &  &  &  &  &  &  &  &  &  \\
\hline
 &  &  &  &  &  &  &  &  &  &  &  &  &  &  &  &  &  &  &  &  &  \\
\hline
 &  &  &  &  &  &  &  &  &  &  &  &  &  &  &  &  &  &  &  &  &  \\
\hline
 &  &  &  &  &  &  &  &  &  &  &  &  &  &  &  &  &  &  &  &  &  \\
\hline
 &  &  &  &  &  &  &  &  &  &  &  &  &  &  &  &  &  &  &  &  &  \\
\hline
 &  &  &  &  &  &  &  &  &  &  &  &  &  &  &  &  &  &  &  &  &  \\
\hline
 &  &  &  &  &  &  &  &  &  &  &  &  &  &  &  &  &  &  &  &  &  \\
\hline
 &  &  &  &  &  &  &  &  &  &  &  &  &  &  &  &  &  &  &  &  &  \\
\hline
 &  &  &  &  &  &  &  &  &  &  &  &  &  &  &  &  &  &  &  &  &  \\
\hline
 &  &  &  &  &  &  &  &  &  &  &  &  &  &  &  &  &  &  &  &  &  \\
\hline
 &  &  &  &  &  &  &  &  &  &  &  &  &  &  &  &  &  &  &  &  &  \\
\hline
 &  &  &  &  &  &  &  &  &  &  &  &  &  &  &  &  &  &  &  &  &  \\
\hline
 &  &  &  &  &  &  &  &  &  &  &  &  &  &  &  &  &  &  &  &  &  \\
\hline
\end{tabular}
\end{center}

Odpowiedź: .

\section*{Zadanie 32. (0-4)}
Dany jest stożek o objętości \(8 \pi\), w którym stosunek wysokości do promienia podstawy jest równy \(3: 8\). Oblicz pole powierzchni bocznej tego stożka.\\
\includegraphics[max width=\textwidth, center]{2025_02_10_0dbe9173aa4ecdfeee91g-19}

Odpowiedź:

\section*{Zadanie 33. (0-4)}
Rejsowy samolot z Warszawy do Rzymu przelatuje nad Austrią każdorazowo tą samą trasą z taką samą zakładaną prędkością przelotową. We wtorek jego średnia prędkość była o \(10 \%\) większa niż prędkość przelotowa, a w czwartek średnia prędkość była o \(10 \%\) mniejsza od zakładanej prędkości przelotowej. Czas przelotu nad Austrią w czwartek różnił się od wtorkowego o 12 minut. Jak długo trwał przelot tego samolotu nad Austrią we wtorek?

\begin{center}
\begin{tabular}{|c|c|c|c|c|c|c|c|c|c|c|c|c|c|c|c|c|c|c|c|c|c|c|c|}
\hline
 &  &  &  &  &  &  &  &  &  &  &  &  &  &  &  &  &  &  &  &  &  &  &  \\
\hline
 &  &  &  &  &  &  &  &  &  &  &  &  &  &  &  &  &  &  &  &  &  &  &  \\
\hline
 &  &  &  &  &  &  &  &  &  &  &  &  &  &  &  &  &  &  &  &  &  &  &  \\
\hline
 &  &  &  &  &  &  &  &  &  &  &  &  &  &  &  &  &  &  &  &  &  &  &  \\
\hline
 &  &  &  &  &  &  &  &  &  &  &  &  &  &  &  &  &  &  &  &  &  &  &  \\
\hline
 &  &  &  &  &  &  &  &  &  &  &  &  &  &  &  &  &  &  &  &  &  &  &  \\
\hline
 &  &  &  &  &  &  &  &  &  &  &  &  &  &  &  &  &  &  &  &  &  &  &  \\
\hline
 &  &  &  &  &  &  &  &  &  &  &  &  &  &  &  &  &  &  &  &  &  &  &  \\
\hline
 &  &  &  &  &  &  &  &  &  &  &  &  &  &  &  &  &  &  &  &  &  &  &  \\
\hline
 &  &  &  &  &  &  &  &  &  &  &  &  &  &  &  &  &  &  &  &  &  &  &  \\
\hline
 &  &  &  &  &  &  &  &  &  &  &  &  &  &  &  &  &  &  &  &  &  &  &  \\
\hline
 &  &  &  &  &  &  &  &  &  &  &  &  &  &  &  &  &  &  &  &  &  &  &  \\
\hline
 &  &  &  &  &  &  &  &  &  &  &  &  &  &  &  &  &  &  &  &  &  &  &  \\
\hline
 &  &  &  &  &  &  &  &  &  &  &  &  &  &  &  &  &  &  &  &  &  &  &  \\
\hline
 &  &  &  &  &  &  &  &  &  &  &  &  &  &  &  &  &  &  &  &  &  &  &  \\
\hline
 &  &  &  &  &  &  &  &  &  &  &  &  &  &  &  &  &  &  &  &  &  &  &  \\
\hline
 &  &  &  &  &  &  &  &  &  &  &  &  &  &  &  &  &  &  &  &  &  &  &  \\
\hline
 &  &  &  &  &  &  &  &  &  &  &  &  &  &  &  &  &  &  &  &  &  &  &  \\
\hline
 &  &  &  &  &  &  &  &  &  &  &  &  &  &  &  &  &  &  &  &  &  &  &  \\
\hline
 &  &  &  &  &  &  &  &  &  &  &  &  &  &  &  &  &  &  &  &  &  &  &  \\
\hline
 &  &  &  &  &  &  &  &  &  &  &  &  &  &  &  &  &  &  &  &  &  &  &  \\
\hline
 &  &  &  &  &  &  &  &  &  &  &  &  &  &  &  &  &  &  &  &  &  &  &  \\
\hline
 &  &  &  &  &  &  &  &  &  &  &  &  &  &  &  &  &  &  &  &  &  &  &  \\
\hline
 &  &  &  &  &  &  &  &  &  &  &  &  &  &  &  &  &  &  &  &  &  &  &  \\
\hline
 &  &  &  &  &  &  &  &  &  &  &  &  &  &  &  &  &  &  &  &  &  &  &  \\
\hline
 &  &  &  &  &  &  &  &  &  &  &  &  &  &  &  &  &  &  &  &  &  &  &  \\
\hline
 &  &  &  &  &  &  &  &  &  &  &  &  &  &  &  &  &  &  &  &  &  &  &  \\
\hline
 &  &  &  &  &  &  &  &  &  &  &  &  &  &  &  &  &  &  &  &  &  &  &  \\
\hline
 &  &  &  &  &  &  &  &  &  &  &  &  &  &  &  &  &  &  &  &  &  &  &  \\
\hline
 &  &  &  &  &  &  &  &  &  &  &  &  &  &  &  &  &  &  &  &  &  &  &  \\
\hline
 &  &  &  &  &  &  &  &  &  &  &  &  &  &  &  &  &  &  &  &  &  &  &  \\
\hline
 &  &  &  &  &  &  &  &  &  &  &  &  &  &  &  &  &  &  &  &  &  &  &  \\
\hline
 &  &  &  &  &  &  &  &  &  &  &  &  &  &  &  &  &  &  &  &  &  &  &  \\
\hline
 &  &  &  &  &  &  &  &  &  &  &  &  &  &  &  &  &  &  &  &  &  &  &  \\
\hline
 &  &  &  &  &  &  &  &  &  &  &  &  &  &  &  &  &  &  &  &  &  &  &  \\
\hline
 &  &  &  &  &  &  &  &  &  &  &  &  &  &  &  &  &  &  &  &  &  &  &  \\
\hline
 &  &  &  &  &  &  &  &  &  &  &  &  &  &  &  &  &  &  &  &  &  &  &  \\
\hline
 &  &  &  &  &  &  &  &  &  &  &  &  &  &  &  &  &  &  &  &  &  &  &  \\
\hline
 &  &  &  &  &  &  &  &  &  &  &  &  &  &  &  &  &  &  &  &  &  &  &  \\
\hline
 &  &  &  &  &  &  &  &  &  &  &  &  &  &  &  &  &  &  &  &  &  &  &  \\
\hline
\end{tabular}
\end{center}

Odpowiedź: \(\qquad\)

\section*{BRUDNOPIS (nie podlega ocenie)}

\end{document}