\documentclass[10pt]{article}
\usepackage[polish]{babel}
\usepackage[utf8]{inputenc}
\usepackage[T1]{fontenc}
\usepackage{graphicx}
\usepackage[export]{adjustbox}
\graphicspath{ {./images/} }
\usepackage{amsmath}
\usepackage{amsfonts}
\usepackage{amssymb}
\usepackage[version=4]{mhchem}
\usepackage{stmaryrd}

\title{EGZAMIN MATURALNY Z MATEMATYKI }

\author{}
\date{}


\begin{document}
\maketitle
\begin{center}
\includegraphics[max width=\textwidth]{2025_02_10_233d5a95a741500f2176g-01}
\end{center}

 Poziom podstawowy\section*{Data: \(\mathbf{2}\) czerwca 2017 r.}
Godzina rozpoczecia: 9:00\\
CZAS PRACY: \(\mathbf{1 7 0}\) minut

\begin{center}
\begin{tabular}{|c|}
\hline
\begin{tabular}{l}
UZUPELNIA ZESPÓ \\
NADZORUJACY \\
\end{tabular} \\
\hline
Uprawnienia zdającego do: \(\square\) dostosowania kryteriów oceniania \(\qquad\) nieprzenoszenia zaznaczeń na kartę \(\qquad\) dostosowania w zw. z dyskalkulią \\
\hline
\end{tabular}
\end{center}

Liczba punktów do uzyskania: \(\mathbf{5 0}\)

\section*{Instrukcja dla zdającego}
\begin{enumerate}
  \item Sprawdź, czy arkusz egzaminacyjny zawiera 24 strony (zadania 1-34). Ewentualny brak zgłoś przewodniczącemu zespołu nadzorującego egzamin.
  \item Rozwiązania zadań i odpowiedzi wpisuj w miejscu na to przeznaczonym.
  \item Odpowiedzi do zadań zamkniętych (1-25) zaznacz na karcie odpowiedzi, w części karty przeznaczonej dla zdającego. Zamaluj \(\quad\) pola do tego przeznaczone. Błędne zaznaczenie otocz kółkiem \(\square_{\text {i zaznacz właściwe }}\)
  \item Pamiętaj, że pominięcie argumentacji lub istotnych obliczeń w rozwiązaniu zadania otwartego (26-34) może spowodować, że za to rozwiązanie nie otrzymasz pełnej liczby punktów.
  \item Pisz czytelnie i używaj tylko długopisu lub pióra z czarnym tuszem lub atramentem.
  \item Nie używaj korektora, a błędne zapisy wyraźnie przekreśl.
  \item Pamiętaj, że zapisy w brudnopisie nie będą oceniane.
  \item Możesz korzystać z zestawu wzorów matematycznych, cyrkla i linijki, a także z kalkulatora prostego.
  \item Na tej stronie oraz na karcie odpowiedzi wpisz swój numer PESEL i przyklej naklejkę z kodem.
  \item Nie wpisuj żadnych znaków w części przeznaczonej dla egzaminatora.
\end{enumerate}

W zadaniach od 1. do 25. wybierz i zaznacz na karcie odpowiedzi poprawna odpowiedź.

\section*{Zadanie 1. (0-1)}
Liczba \(|9-2|-|4-7|\) jest równa\\
A. 4\\
B. 10\\
C. -10\\
D. -4

\section*{Zadanie 2. (0-1)}
Iloczyn dodatnich liczb \(a\) i \(b\) jest równy 1350. Ponadto \(15 \%\) liczby \(a\) jest równe \(10 \%\) liczby \(b\). Stąd wynika, że \(b\) jest równe\\
A. 9\\
B. 18\\
C. 45\\
D. 50

\section*{Zadanie 3. (0-1)}
Suma \(16^{24}+16^{24}+16^{24}+16^{24}\) jest równa\\
A. \(4^{24}\)\\
B. \(4^{25}\)\\
C. \(4^{48}\)\\
D. \(4^{49}\)

\section*{Zadanie 4. (0-1)}
Liczba \(\log _{3} 27-\log _{3} 1\) jest równa\\
A. 0\\
B. 1\\
C. 2\\
D. 3

\section*{Zadanie 5. (0-1)}
Dla każdej liczby rzeczywistej \(x\) wyrażenie \(x^{6}-2 x^{3}-3\) jest równe\\
A. \(\left(x^{3}+1\right)\left(x^{2}-3\right)\)\\
B. \(\left(x^{3}-3\right)\left(x^{3}+1\right)\)\\
C. \(\left(x^{2}+3\right)\left(x^{4}-1\right)\)\\
D. \(\left(x^{4}+1\right)\left(x^{2}-3\right)\)

\section*{Zadanie 6. (0-1)}
Wartość wyrażenia \((b-a)^{2}\) dla \(a=2 \sqrt{3}\) i \(b=\sqrt{75}\) jest równa\\
A. 9\\
B. 27\\
C. 63\\
D. 147

BRUDNOPIS (nie podlega ocenie)\\
\(\qquad\)\\
\includegraphics[max width=\textwidth, center]{2025_02_10_233d5a95a741500f2176g-03(2)}\\
\(\qquad\)\\
\includegraphics[max width=\textwidth, center]{2025_02_10_233d5a95a741500f2176g-03(6)}\\
\includegraphics[max width=\textwidth, center]{2025_02_10_233d5a95a741500f2176g-03(7)}\\
\includegraphics[max width=\textwidth, center]{2025_02_10_233d5a95a741500f2176g-03(8)}\\
\(\qquad\)\\
\includegraphics[max width=\textwidth, center]{2025_02_10_233d5a95a741500f2176g-03(9)}\\
\(\qquad\)\\
\(\qquad\)\\
\includegraphics[max width=\textwidth, center]{2025_02_10_233d5a95a741500f2176g-03(5)}\\
\includegraphics[max width=\textwidth, center]{2025_02_10_233d5a95a741500f2176g-03(12)}\\
\includegraphics[max width=\textwidth, center]{2025_02_10_233d5a95a741500f2176g-03(1)}\\
\includegraphics[max width=\textwidth, center]{2025_02_10_233d5a95a741500f2176g-03(16)}\\
\includegraphics[max width=\textwidth, center]{2025_02_10_233d5a95a741500f2176g-03(13)}\\
\(\qquad\)\\
\includegraphics[max width=\textwidth, center]{2025_02_10_233d5a95a741500f2176g-03(10)}\\
\includegraphics[max width=\textwidth, center]{2025_02_10_233d5a95a741500f2176g-03(11)}\\
\includegraphics[max width=\textwidth, center]{2025_02_10_233d5a95a741500f2176g-03(15)}\\
\includegraphics[max width=\textwidth, center]{2025_02_10_233d5a95a741500f2176g-03(3)}\\
\(\qquad\)\\
\includegraphics[max width=\textwidth, center]{2025_02_10_233d5a95a741500f2176g-03(4)}\\
\includegraphics[max width=\textwidth, center]{2025_02_10_233d5a95a741500f2176g-03(14)}\\
\includegraphics[max width=\textwidth, center]{2025_02_10_233d5a95a741500f2176g-03}

\section*{Zadanie 7. (0-1)}
Funkcja liniowa \(f\) jest określona wzorem \(f(x)=21-\frac{7}{3} x\). Miejscem zerowym funkcji \(f\) jest\\
A. -9\\
B. \(-\frac{7}{3}\)\\
C. 9\\
D. 21

\section*{Zadanie 8. (0-1)}
Rozwiązaniem układu równań \(\left\{\begin{array}{l}x+y=1 \\ x-y=b\end{array}\right.\) z niewiadomymi \(x\) i \(y\) jest para liczb dodatnich. Wynika stąd, że\\
A. \(b<-1\)\\
B. \(b=-1\)\\
C. \(-1<b<1\)\\
D. \(b \geq 1\)

\section*{Zadanie 9. (0-1)}
Funkcja kwadratowa \(f\) jest określona wzorem \(f(x)=x^{2}+b x+c\) oraz \(f(-1)=f(3)=1\). Współczynnik \(b\) jest równy\\
A. -2\\
B. -1\\
C. 0\\
D. 3

\section*{Zadanie 10. (0-1)}
Równanie \(x(x-3)\left(x^{2}+25\right)=0\) ma dokładnie\\
A. cztery rozwiązania: \(x=0, x=3, x=5, x=-5\)\\
B. trzy rozwiązania: \(x=3, x=5, x=-5\)\\
C. dwa rozwiązania: \(x=0, x=3\)\\
D. jedno rozwiązanie: \(x=3\)

\section*{Zadanie 11. (0-1)}
Funkcja kwadratowa \(f\) jest określona wzorem \(f(x)=(x-3)(7-x)\). Wierzchołek paraboli będącej wykresem funkcji \(f\) należy do prostej o równaniu\\
A. \(y=-5\)\\
B. \(y=5\)\\
C. \(y=-4\)\\
D. \(y=4\)

BRUDNOPIS (nie podlega ocenie)\\
\(\qquad\)\\
\includegraphics[max width=\textwidth, center]{2025_02_10_233d5a95a741500f2176g-05(5)}\\
\(\qquad\)\\
\includegraphics[max width=\textwidth, center]{2025_02_10_233d5a95a741500f2176g-05(3)}\\
\includegraphics[max width=\textwidth, center]{2025_02_10_233d5a95a741500f2176g-05(16)}\\
\includegraphics[max width=\textwidth, center]{2025_02_10_233d5a95a741500f2176g-05(12)}\\
\(\qquad\)\\
\includegraphics[max width=\textwidth, center]{2025_02_10_233d5a95a741500f2176g-05(10)}\\
\(\qquad\)\\
\(\qquad\)\\
\includegraphics[max width=\textwidth, center]{2025_02_10_233d5a95a741500f2176g-05(14)}\\
\includegraphics[max width=\textwidth, center]{2025_02_10_233d5a95a741500f2176g-05(15)}\\
\includegraphics[max width=\textwidth, center]{2025_02_10_233d5a95a741500f2176g-05(7)}\\
\includegraphics[max width=\textwidth, center]{2025_02_10_233d5a95a741500f2176g-05(11)}\\
\includegraphics[max width=\textwidth, center]{2025_02_10_233d5a95a741500f2176g-05(9)}\\
\(\qquad\)\\
\includegraphics[max width=\textwidth, center]{2025_02_10_233d5a95a741500f2176g-05(8)}\\
\includegraphics[max width=\textwidth, center]{2025_02_10_233d5a95a741500f2176g-05}\\
\includegraphics[max width=\textwidth, center]{2025_02_10_233d5a95a741500f2176g-05(4)}\\
\includegraphics[max width=\textwidth, center]{2025_02_10_233d5a95a741500f2176g-05(6)}\\
\(\qquad\)\\
\includegraphics[max width=\textwidth, center]{2025_02_10_233d5a95a741500f2176g-05(13)}\\
\includegraphics[max width=\textwidth, center]{2025_02_10_233d5a95a741500f2176g-05(1)}\\
\includegraphics[max width=\textwidth, center]{2025_02_10_233d5a95a741500f2176g-05(2)}

\section*{Zadanie 12. (0-1)}
Punkt \(A=(2017,0)\) należy do wykresu funkcji \(f\) określonej wzorem\\
A. \(f(x)=(x+2017)^{2}\)\\
B. \(f(x)=x^{2}-2017\)\\
C. \(f(x)=(x+2017)(x-2017)\)\\
D. \(f(x)=x^{2}+2017\)

\section*{Zadanie 13. (0-1)}
W ciągu arytmetycznym \(\left(a_{n}\right)\), określonym dla \(n \geq 1\), spełniony jest warunek \(2 a_{3}=a_{2}+a_{1}+1\). Różnica \(r\) tego ciągu jest równa\\
A. 0\\
B. \(\frac{1}{3}\)\\
C. \(\frac{1}{2}\)\\
D. 1

\section*{Zadanie 14. (0-1)}
Dany jest ciąg geometryczny \(\left(x, 2 x^{2}, 4 x^{3}, 8\right)\) o wyrazach nieujemnych. Wtedy\\
A. \(x=0\)\\
B. \(x=1\)\\
C. \(x=2\)\\
D. \(x=4\)

\section*{Zadanie 15. (0-1)}
Kąt \(\alpha\) jest ostry i \(\operatorname{tg} \alpha=\frac{12}{5}\). Wówczas \(\sin \alpha\) jest równy\\
A. \(\frac{5}{17}\)\\
B. \(\frac{12}{17}\)\\
C. \(\frac{5}{13}\)\\
D. \(\frac{12}{13}\)

BRUDNOPIS (nie podlega ocenie)\\
\(\qquad\)\\
\includegraphics[max width=\textwidth, center]{2025_02_10_233d5a95a741500f2176g-07(13)}\\
\(\qquad\)\\
\includegraphics[max width=\textwidth, center]{2025_02_10_233d5a95a741500f2176g-07(4)}\\
\includegraphics[max width=\textwidth, center]{2025_02_10_233d5a95a741500f2176g-07(7)}\\
\includegraphics[max width=\textwidth, center]{2025_02_10_233d5a95a741500f2176g-07(11)}\\
\(\qquad\)\\
\includegraphics[max width=\textwidth, center]{2025_02_10_233d5a95a741500f2176g-07(16)}\\
\(\qquad\)\\
\(\qquad\)\\
\includegraphics[max width=\textwidth, center]{2025_02_10_233d5a95a741500f2176g-07(12)}\\
\includegraphics[max width=\textwidth, center]{2025_02_10_233d5a95a741500f2176g-07(14)}\\
\includegraphics[max width=\textwidth, center]{2025_02_10_233d5a95a741500f2176g-07(10)}\\
\includegraphics[max width=\textwidth, center]{2025_02_10_233d5a95a741500f2176g-07(5)}\\
\includegraphics[max width=\textwidth, center]{2025_02_10_233d5a95a741500f2176g-07(9)}\\
\(\qquad\)\\
\includegraphics[max width=\textwidth, center]{2025_02_10_233d5a95a741500f2176g-07}\\
\includegraphics[max width=\textwidth, center]{2025_02_10_233d5a95a741500f2176g-07(2)}\\
\includegraphics[max width=\textwidth, center]{2025_02_10_233d5a95a741500f2176g-07(8)}\\
\includegraphics[max width=\textwidth, center]{2025_02_10_233d5a95a741500f2176g-07(15)}\\
\(\qquad\)\\
\includegraphics[max width=\textwidth, center]{2025_02_10_233d5a95a741500f2176g-07(3)}\\
\includegraphics[max width=\textwidth, center]{2025_02_10_233d5a95a741500f2176g-07(1)}\\
\includegraphics[max width=\textwidth, center]{2025_02_10_233d5a95a741500f2176g-07(6)}

\section*{Zadanie 16. (0-1)}
W okręgu o środku \(O\) dany jest kąt wpisany \(A B C\) o mierze \(20^{\circ}\) (patrz rysunek).\\
\includegraphics[max width=\textwidth, center]{2025_02_10_233d5a95a741500f2176g-08(1)}

Miara kąta \(C A O\) jest równa\\
A. \(85^{\circ}\)\\
B. \(70^{\circ}\)\\
C. \(80^{\circ}\)\\
D. \(75^{\circ}\)

\section*{Zadanie 17. (0-1)}
Odcinek \(B D\) jest zawarty w dwusiecznej kąta ostrego \(A B C\) trójkąta prostokątnego, w którym przyprostokątne \(A C\) i \(B C\) mają długości odpowiednio 5 i 3.\\
\includegraphics[max width=\textwidth, center]{2025_02_10_233d5a95a741500f2176g-08}

Wówczas miara \(\varphi\) kąta \(D B C\) spełnia warunek\\
A. \(20^{\circ}<\varphi<25^{\circ}\)\\
B. \(25^{\circ}<\varphi<30^{\circ}\)\\
C. \(30^{\circ}<\varphi<35^{\circ}\)\\
D. \(35^{\circ}<\varphi<40^{\circ}\)

BRUDNOPIS (nie podlega ocenie)\\
\(\qquad\)

\section*{Zadanie 18. (0-1)}
Prosta przechodząca przez punkt \(A=(-10,5)\) i początek układu wspórrzędnych jest prostopadła do prostej o równaniu\\
A. \(y=-2 x+4\)\\
B. \(y=\frac{1}{2} x\)\\
C. \(y=-\frac{1}{2} x+1\)\\
D. \(y=2 x-4\)

\section*{Zadanie 19. (0-1)}
Punkty \(A=(-21,11)\) i \(B=(3,17)\) są końcami odcinka \(A B\). Obrazem tego odcinka w symetrii względem osi \(O x\) układu współrzędnych jest odcinek \(A^{\prime} B^{\prime}\). Środkiem odcinka \(A^{\prime} B^{\prime}\) jest punkt o współrzędnych\\
A. \((-9,-14)\)\\
B. \((-9,14)\)\\
C. \((9,-14)\)\\
D. \((9,14)\)

\section*{Zadanie 20. (0-1)}
Trójkąt \(A B C\) jest podobny do trójkąta \(A^{\prime} B^{\prime} C^{\prime}\) w skali \(\frac{5}{2}\), przy czym \(|A B|=\frac{5}{2}\left|A^{\prime} B^{\prime}\right|\). Stosunek pola trójkąta \(A B C\) do pola trójkąta \(A^{\prime} B^{\prime} C^{\prime}\) jest równy\\
A. \(\frac{4}{25}\)\\
B. \(\frac{2}{5}\)\\
C. \(\frac{5}{2}\)\\
D. \(\frac{25}{4}\)

\section*{Zadanie 21. (0-1)}
Pole koła opisanego na trójkącie równobocznym jest równe \(\frac{1}{3} \pi^{3}\). Długość boku tego trójkąta jest równa\\
A. \(\frac{\pi}{3}\)\\
B. \(\pi\)\\
C. \(\sqrt{3} \pi\)\\
D. \(3 \pi\)

BRUDNOPIS (nie podlega ocenie)\\
\(\qquad\)\\
\includegraphics[max width=\textwidth, center]{2025_02_10_233d5a95a741500f2176g-11(5)}\\
\(\qquad\)\\
\includegraphics[max width=\textwidth, center]{2025_02_10_233d5a95a741500f2176g-11(12)}\\
\includegraphics[max width=\textwidth, center]{2025_02_10_233d5a95a741500f2176g-11(4)}\\
\includegraphics[max width=\textwidth, center]{2025_02_10_233d5a95a741500f2176g-11(9)}\\
\(\qquad\)\\
\includegraphics[max width=\textwidth, center]{2025_02_10_233d5a95a741500f2176g-11(6)}\\
\(\qquad\)\\
\(\qquad\)\\
\includegraphics[max width=\textwidth, center]{2025_02_10_233d5a95a741500f2176g-11(15)}\\
\includegraphics[max width=\textwidth, center]{2025_02_10_233d5a95a741500f2176g-11(7)}\\
\includegraphics[max width=\textwidth, center]{2025_02_10_233d5a95a741500f2176g-11(13)}\\
\includegraphics[max width=\textwidth, center]{2025_02_10_233d5a95a741500f2176g-11(3)}\\
\includegraphics[max width=\textwidth, center]{2025_02_10_233d5a95a741500f2176g-11(2)}\\
\(\qquad\)\\
\includegraphics[max width=\textwidth, center]{2025_02_10_233d5a95a741500f2176g-11(14)}\\
\includegraphics[max width=\textwidth, center]{2025_02_10_233d5a95a741500f2176g-11(11)}\\
\includegraphics[max width=\textwidth, center]{2025_02_10_233d5a95a741500f2176g-11(16)}\\
\includegraphics[max width=\textwidth, center]{2025_02_10_233d5a95a741500f2176g-11}\\
\(\qquad\)\\
\includegraphics[max width=\textwidth, center]{2025_02_10_233d5a95a741500f2176g-11(8)}\\
\includegraphics[max width=\textwidth, center]{2025_02_10_233d5a95a741500f2176g-11(10)}\\
\includegraphics[max width=\textwidth, center]{2025_02_10_233d5a95a741500f2176g-11(1)}

\section*{Zadanie 22. (0-1)}
Pole trójkąta prostokątnego \(A B C\), przedstawionego na rysunku, jest równe\\
\includegraphics[max width=\textwidth, center]{2025_02_10_233d5a95a741500f2176g-12}\\
A. \(\frac{32 \sqrt{3}}{6}\)\\
B. \(\frac{16 \sqrt{3}}{6}\)\\
C. \(\frac{8 \sqrt{3}}{3}\)\\
D. \(\frac{4 \sqrt{3}}{3}\)

\section*{Zadanie 23. (0-1)}
Długość przekątnej sześcianu jest równa 6 . Stąd wynika, że pole powierzchni całkowitej tego sześcianu jest równe\\
A. 72\\
B. 48\\
C. 152\\
D. 108

\section*{Zadanie 24. (0-1)}
Pole powierzchni bocznej walca jest równe \(16 \pi\), a promień jego podstawy ma długość 2 . Wysokość tego walca jest równa\\
A. 4\\
B. 8\\
C. \(4 \pi\)\\
D. \(8 \pi\)

\section*{Zadanie 25. (0-1)}
Rzucamy dwa razy symetryczną sześcienną kostką do gry. Prawdopodobieństwo otrzymania pary liczb, których iloczyn jest większy od 20 , jest równe\\
A. \(\frac{1}{6}\)\\
B. \(\frac{5}{36}\)\\
C. \(\frac{1}{9}\)\\
D. \(\frac{2}{9}\)

BRUDNOPIS (nie podlega ocenie)\\
\(\qquad\)

\section*{Zadanie 26. (0-2)}
Rozwiąż nierówność \(\left(x-\frac{1}{2}\right) x>3\left(x-\frac{1}{2}\right)\left(x+\frac{1}{3}\right)\).

\begin{center}
\begin{tabular}{|c|c|c|c|c|c|c|c|c|c|c|c|c|c|c|c|c|c|c|c|c|c|c|}
\hline
 &  &  &  &  &  &  &  &  &  &  &  &  &  &  &  &  &  &  &  &  &  &  \\
\hline
 &  &  &  &  &  &  &  &  &  &  &  &  &  &  &  &  &  &  &  &  &  &  \\
\hline
 &  &  &  &  &  &  &  &  &  &  &  &  &  &  &  &  &  &  &  &  &  &  \\
\hline
 &  &  &  &  &  &  &  &  &  &  &  &  &  &  &  &  &  &  &  &  &  &  \\
\hline
 &  &  &  &  &  &  &  &  &  &  &  &  &  &  &  &  &  &  &  &  &  &  \\
\hline
 &  &  &  &  &  &  &  &  &  &  &  &  &  &  &  &  &  &  &  &  &  &  \\
\hline
 &  &  &  &  &  &  &  &  &  &  &  &  &  &  &  &  &  &  &  &  &  &  \\
\hline
 &  &  &  &  &  &  &  &  &  &  &  &  &  &  &  &  &  &  &  &  &  &  \\
\hline
 &  &  &  &  &  &  &  &  &  &  &  &  &  &  &  &  &  &  &  &  &  &  \\
\hline
 &  &  &  &  &  &  &  &  &  &  &  &  &  &  &  &  &  &  &  &  &  &  \\
\hline
 &  &  &  &  &  &  &  &  &  &  &  &  &  &  &  &  &  &  &  &  &  &  \\
\hline
 &  &  &  &  &  &  &  &  &  &  &  &  &  &  &  &  &  &  &  &  &  &  \\
\hline
 &  &  &  &  &  &  &  &  &  &  &  &  &  &  &  &  &  &  &  &  &  &  \\
\hline
 &  &  &  &  &  &  &  &  &  &  &  &  &  &  &  &  &  &  &  &  &  &  \\
\hline
 &  &  &  &  &  &  &  &  &  &  &  &  &  &  &  &  &  &  &  &  &  &  \\
\hline
 &  &  &  &  &  &  &  &  &  &  &  &  &  &  &  &  &  &  &  &  &  &  \\
\hline
 &  &  &  &  &  &  &  &  &  &  &  &  &  &  &  &  &  &  &  &  &  &  \\
\hline
 &  &  &  &  &  &  &  &  &  &  &  &  &  &  &  &  &  &  &  &  &  &  \\
\hline
 &  &  &  &  &  &  &  &  &  &  &  &  &  &  &  &  &  &  &  &  &  &  \\
\hline
 &  &  &  &  &  &  &  &  &  &  &  &  &  &  &  &  &  &  &  &  &  &  \\
\hline
 &  &  &  &  &  &  &  &  &  &  &  &  &  &  &  &  &  &  &  &  &  &  \\
\hline
 &  &  &  &  &  &  &  &  &  &  &  &  &  &  &  &  &  &  &  &  &  &  \\
\hline
 &  &  &  &  &  &  &  &  &  &  &  &  &  &  &  &  &  &  &  &  &  &  \\
\hline
 &  &  &  &  &  &  &  &  &  &  &  &  &  &  &  &  &  &  &  &  &  &  \\
\hline
 &  &  &  &  &  &  &  &  &  &  &  &  &  &  &  &  &  &  &  &  &  &  \\
\hline
 &  &  &  &  &  &  &  &  &  &  &  &  &  &  &  &  &  &  &  &  &  &  \\
\hline
 &  &  &  &  &  &  &  &  &  &  &  &  &  &  &  &  &  &  &  &  &  &  \\
\hline
 &  &  &  &  &  &  &  &  &  &  &  &  &  &  &  &  &  &  &  &  &  &  \\
\hline
 &  &  &  &  &  &  &  &  &  &  &  &  &  &  &  &  &  &  &  &  &  &  \\
\hline
 &  &  &  &  &  &  &  &  &  &  &  &  &  &  &  &  &  &  &  &  &  &  \\
\hline
 &  &  &  &  &  &  &  &  &  &  &  &  &  &  &  &  &  &  &  &  &  &  \\
\hline
 &  &  &  &  &  &  &  &  &  &  &  &  &  &  &  &  &  &  &  &  &  &  \\
\hline
 &  &  &  &  &  &  &  &  &  &  &  &  &  &  &  &  &  &  &  &  &  &  \\
\hline
 &  &  &  &  &  &  &  &  &  &  &  &  &  &  &  &  &  &  &  &  &  &  \\
\hline
 &  &  &  &  &  &  &  &  &  &  &  &  &  &  &  &  &  &  &  &  &  &  \\
\hline
 &  &  &  &  &  &  &  &  &  &  &  &  &  &  &  &  &  &  &  &  &  &  \\
\hline
 &  &  &  &  &  &  &  &  &  &  &  &  &  &  &  &  &  &  &  &  &  &  \\
\hline
 &  &  &  &  &  &  &  &  &  &  &  &  &  &  &  &  &  &  &  &  &  &  \\
\hline
 &  &  &  &  &  &  &  &  &  &  &  &  &  &  &  &  &  &  &  &  &  &  \\
\hline
 &  &  &  &  &  &  &  &  &  &  &  &  &  &  &  &  &  &  &  &  &  &  \\
\hline
 &  &  &  &  &  &  &  &  &  &  &  &  &  &  &  &  &  &  &  &  &  &  \\
\hline
 &  &  &  &  &  &  &  &  &  &  &  &  &  &  &  &  &  &  &  &  &  &  \\
\hline
\end{tabular}
\end{center}

Odpowiedź:

\section*{Zadanie 27. (0-2)}
Kąt \(\alpha\) jest ostry i spełniona jest równość \(\sin \alpha+\cos \alpha=\frac{\sqrt{7}}{2}\). Oblicz wartość wyrażenia \((\sin \alpha-\cos \alpha)^{2}\).

\begin{center}
\begin{tabular}{|c|c|c|c|c|c|c|c|c|c|c|c|c|c|c|c|c|c|c|c|c|c|c|c|c|c|}
\hline
 &  &  &  &  &  &  &  &  &  &  &  &  &  &  &  &  &  &  &  &  &  &  &  &  &  \\
\hline
 &  &  &  &  &  &  &  &  &  &  &  &  &  &  &  &  &  &  &  &  &  &  &  &  &  \\
\hline
 &  &  &  &  &  &  &  &  &  &  &  &  &  &  &  &  &  &  &  &  &  &  &  &  &  \\
\hline
 &  &  &  &  &  &  &  &  &  &  &  &  &  &  &  &  &  &  &  &  &  &  &  &  &  \\
\hline
 &  &  &  &  &  &  &  &  &  &  &  &  &  &  &  &  &  &  &  &  &  &  &  &  &  \\
\hline
 &  &  &  &  &  &  &  &  &  &  &  &  &  &  &  &  &  &  &  &  &  &  &  &  &  \\
\hline
 &  &  &  &  &  &  &  &  &  &  &  &  &  &  &  &  &  &  &  &  &  &  &  &  &  \\
\hline
 &  &  &  &  &  &  &  &  &  &  &  &  &  &  &  &  &  &  &  &  &  &  &  &  &  \\
\hline
 &  &  &  &  &  &  &  &  &  &  &  &  &  &  &  &  &  &  &  &  &  &  &  &  &  \\
\hline
 &  &  &  &  &  &  &  &  &  &  &  &  &  &  &  &  &  &  &  &  &  &  &  &  &  \\
\hline
 &  &  &  &  &  &  &  &  &  &  &  &  &  &  &  &  &  &  &  &  &  &  &  &  &  \\
\hline
 &  &  &  &  &  &  &  &  &  &  &  &  &  &  &  &  &  &  &  &  &  &  &  &  &  \\
\hline
 &  &  &  &  &  &  &  &  &  &  &  &  &  &  &  &  &  &  &  &  &  &  &  &  &  \\
\hline
 &  &  &  &  &  &  &  &  &  &  &  &  &  &  &  &  &  &  &  &  &  &  &  &  &  \\
\hline
 &  &  &  &  &  &  &  &  &  &  &  &  &  &  &  &  &  &  &  &  &  &  &  &  &  \\
\hline
 &  &  &  &  &  &  &  &  &  &  &  &  &  &  &  &  &  &  &  &  &  &  &  &  &  \\
\hline
 &  &  &  &  &  &  &  &  &  &  &  &  &  &  &  &  &  &  &  &  &  &  &  &  &  \\
\hline
 &  &  &  &  &  &  &  &  &  &  &  &  &  &  &  &  &  &  &  &  &  &  &  &  &  \\
\hline
 &  &  &  &  &  &  &  &  &  &  &  &  &  &  &  &  &  &  &  &  &  &  &  &  &  \\
\hline
 &  &  &  &  &  &  &  &  &  &  &  &  &  &  &  &  &  &  &  &  &  &  &  &  &  \\
\hline
 &  &  &  &  &  &  &  &  &  &  &  &  &  &  &  &  &  &  &  &  &  &  &  &  &  \\
\hline
 &  &  &  &  &  &  &  &  &  &  &  &  &  &  &  &  &  &  &  &  &  &  &  &  &  \\
\hline
 &  &  &  &  &  &  &  &  &  &  &  &  &  &  &  &  &  &  &  &  &  &  &  &  &  \\
\hline
 &  &  &  &  &  &  &  &  &  &  &  &  &  &  &  &  &  &  &  &  &  &  &  &  &  \\
\hline
 &  &  &  &  &  &  &  &  &  &  &  &  &  &  &  &  &  &  &  &  &  &  &  & \includegraphics[max width=\textwidth]{2025_02_10_233d5a95a741500f2176g-15}
 &  \\
\hline
 &  &  &  &  &  &  &  &  &  &  &  &  &  &  &  &  &  &  &  &  &  &  &  &  &  \\
\hline
 &  &  &  &  &  &  &  &  &  &  &  &  &  &  &  &  &  &  &  &  &  &  &  &  &  \\
\hline
 &  &  &  &  &  &  &  &  &  &  &  &  &  &  &  &  &  &  &  &  &  &  &  &  &  \\
\hline
 &  &  &  &  &  &  &  &  &  &  &  &  &  &  &  &  &  &  &  &  &  &  &  & \( 7 \) &  \\
\hline
 &  &  &  &  &  &  &  &  &  &  &  &  &  &  &  &  &  &  &  &  &  &  &  &  &  \\
\hline
 &  &  &  &  &  &  &  &  &  &  &  &  &  &  &  &  &  &  &  &  &  &  &  &  &  \\
\hline
 &  &  &  &  &  &  &  &  &  &  &  &  &  &  &  &  &  &  &  &  &  &  &  &  &  \\
\hline
\includegraphics[max width=\textwidth]{2025_02_10_233d5a95a741500f2176g-15(1)}
 & - &  &  &  &  &  &  &  &  &  &  &  &  &  &  &  &  &  &  &  &  &  &  &  &  \\
\hline
 & - &  &  &  &  &  &  &  &  &  &  &  &  &  &  &  &  &  &  &  &  &  &  &  &  \\
\hline
 & - & - &  &  &  &  &  &  &  &  &  &  &  &  &  &  &  &  &  &  &  &  &  &  &  \\
\hline
 &  &  &  &  &  &  &  &  &  &  &  &  &  &  &  &  &  &  &  &  &  &  &  &  &  \\
\hline
 &  &  &  &  &  &  &  &  &  &  &  &  &  &  &  &  &  &  &  &  &  &  &  &  &  \\
\hline
 &  &  &  &  &  &  &  &  &  &  &  &  &  &  &  &  &  &  &  &  &  & \includegraphics[max width=\textwidth]{2025_02_10_233d5a95a741500f2176g-15(2)}
 &  &  &  \\
\hline
 &  &  &  &  &  &  &  &  &  &  &  &  &  &  &  &  &  &  &  &  &  &  &  &  &  \\
\hline
 &  &  &  &  &  &  &  &  &  &  &  &  &  &  &  &  &  &  &  &  &  &  &  &  &  \\
\hline
\end{tabular}
\end{center}

Odpowiedź:

Zadanie 28. (0-2)\\
Dwusieczna kąta ostrego \(A B C\) przecina przyprostokątną \(A C\) trójkąta prostokątnego \(A B C\) w punkcie \(D\).\\
\includegraphics[max width=\textwidth, center]{2025_02_10_233d5a95a741500f2176g-16}

Udowodnij, że jeżeli \(|A D|=|B D|\), to \(|C D|=\frac{1}{2} \cdot|B D|\).\\
\includegraphics[max width=\textwidth, center]{2025_02_10_233d5a95a741500f2176g-16(1)}

Zadanie 29. (0-2)\\
Wykaż, że prawdziwa jest nierówność \((1,5)^{100}<6^{25}\).\\
\(\qquad\)\\
\includegraphics[max width=\textwidth, center]{2025_02_10_233d5a95a741500f2176g-17(20)}\\
\includegraphics[max width=\textwidth, center]{2025_02_10_233d5a95a741500f2176g-17(15)}\\
\includegraphics[max width=\textwidth, center]{2025_02_10_233d5a95a741500f2176g-17(17)}\\
\includegraphics[max width=\textwidth, center]{2025_02_10_233d5a95a741500f2176g-17(13)}\\
\includegraphics[max width=\textwidth, center]{2025_02_10_233d5a95a741500f2176g-17(7)}\\
\includegraphics[max width=\textwidth, center]{2025_02_10_233d5a95a741500f2176g-17(14)}\\
\includegraphics[max width=\textwidth, center]{2025_02_10_233d5a95a741500f2176g-17(6)}\\
\includegraphics[max width=\textwidth, center]{2025_02_10_233d5a95a741500f2176g-17(1)}\\
\includegraphics[max width=\textwidth, center]{2025_02_10_233d5a95a741500f2176g-17(3)}\\
\includegraphics[max width=\textwidth, center]{2025_02_10_233d5a95a741500f2176g-17(10)}\\
\(\qquad\)\\
\includegraphics[max width=\textwidth, center]{2025_02_10_233d5a95a741500f2176g-17(2)}\\
\includegraphics[max width=\textwidth, center]{2025_02_10_233d5a95a741500f2176g-17(11)}\\
\includegraphics[max width=\textwidth, center]{2025_02_10_233d5a95a741500f2176g-17}\\
\includegraphics[max width=\textwidth, center]{2025_02_10_233d5a95a741500f2176g-17(9)}\\
\includegraphics[max width=\textwidth, center]{2025_02_10_233d5a95a741500f2176g-17(18)}\\
\includegraphics[max width=\textwidth, center]{2025_02_10_233d5a95a741500f2176g-17(19)}\\
\includegraphics[max width=\textwidth, center]{2025_02_10_233d5a95a741500f2176g-17(21)}\\
\includegraphics[max width=\textwidth, center]{2025_02_10_233d5a95a741500f2176g-17(16)}\\
\includegraphics[max width=\textwidth, center]{2025_02_10_233d5a95a741500f2176g-17(5)}\\
\(\qquad\)\\
\includegraphics[max width=\textwidth, center]{2025_02_10_233d5a95a741500f2176g-17(4)}\\
\includegraphics[max width=\textwidth, center]{2025_02_10_233d5a95a741500f2176g-17(12)}\\
\includegraphics[max width=\textwidth, center]{2025_02_10_233d5a95a741500f2176g-17(8)}

\section*{Zadanie 30. (0-2)}
Suma trzydziestu początkowych wyrazów ciągu arytmetycznego \(\left(a_{n}\right)\), określonego dla \(n \geq 1\), jest równa 30. Ponadto \(a_{30}=30\). Oblicz różnicę tego ciągu.\\
\includegraphics[max width=\textwidth, center]{2025_02_10_233d5a95a741500f2176g-18}

Odpowiedź:

\section*{Zadanie 31. (0-2)}
Ze zbioru liczb \(\{1,2,3,4,5,6,7,8,9,10,11,12,13,14,15\}\) losujemy bez zwracania dwa razy po jednej liczbie. Wylosowane liczby tworzą parę \((a, b)\), gdzie \(a\) jest wynikiem pierwszego losowania, \(b\) jest wynikiem drugiego losowania. Oblicz, ile jest wszystkich par \((a, b)\) takich, że iloczyn \(a \cdot b\) jest liczbą parzystą.\\
\includegraphics[max width=\textwidth, center]{2025_02_10_233d5a95a741500f2176g-19}

Odpowiedź: \(\qquad\)

\section*{Zadanie 32. (0-4)}
Ramię trapezu równoramiennego \(A B C D\) ma długość \(\sqrt{26}\). Przekątne w tym trapezie są prostopadłe, a punkt ich przecięcia dzieli je w stosunku \(2: 3\). Oblicz pole tego trapezu.\\
\includegraphics[max width=\textwidth, center]{2025_02_10_233d5a95a741500f2176g-20}

Odpowiedź:

\section*{Zadanie 33. (0-4)}
Punkty \(A=(-2,-8)\) i \(B=(14,-8)\) są wierzchołkami trójkąta równoramiennego \(A B C\), w którym \(|A B|=|A C|\). Wysokość \(A D\) tego trójkąta jest zawarta w prostej o równaniu \(y=\frac{1}{2} x-7\). Oblicz wspórrzędne wierzchołka \(C\) tego trójkąta.

\begin{center}
\begin{tabular}{|c|c|c|c|c|c|c|c|c|c|c|c|c|c|c|c|c|c|c|c|c|c|c|c|c|c|c|c|c|}
\hline
 &  &  &  &  &  &  &  &  &  &  &  &  &  &  &  &  &  &  &  &  &  &  &  &  &  &  &  &  \\
\hline
 &  &  &  &  &  &  &  &  &  &  &  &  &  &  &  &  &  &  &  &  &  &  &  &  &  &  &  &  \\
\hline
 &  &  &  &  &  &  &  &  &  &  &  &  &  &  &  &  &  &  &  &  &  &  &  &  &  &  &  &  \\
\hline
 &  &  &  &  &  &  &  &  &  &  &  &  &  &  &  &  &  &  &  &  &  &  &  &  &  &  &  &  \\
\hline
 &  &  &  &  &  &  &  &  &  &  &  &  &  &  &  &  &  &  &  &  &  &  &  &  &  &  &  &  \\
\hline
 &  &  &  &  &  &  &  &  &  &  &  &  &  &  &  &  &  &  &  &  &  &  &  &  &  &  &  &  \\
\hline
 &  &  &  &  &  &  &  &  &  &  &  &  &  &  &  &  &  &  &  &  &  &  &  &  &  &  &  &  \\
\hline
 &  &  &  &  &  &  &  &  &  &  &  &  &  &  &  &  &  &  &  &  &  &  &  &  &  &  &  &  \\
\hline
 &  &  &  &  &  &  &  &  &  &  &  &  &  &  &  &  &  &  &  &  &  &  &  &  &  &  &  &  \\
\hline
 &  &  &  &  &  &  &  &  &  &  &  &  &  &  &  &  &  &  &  &  &  &  &  &  &  &  &  &  \\
\hline
 &  &  &  &  &  &  &  &  &  &  &  &  &  &  &  &  &  &  &  &  &  &  &  &  &  &  &  &  \\
\hline
 &  &  &  &  &  &  &  &  &  &  &  &  &  &  &  &  &  &  &  &  &  &  &  &  &  &  &  &  \\
\hline
 &  &  &  &  &  &  &  &  &  &  &  &  &  &  &  &  &  &  &  &  &  &  &  &  &  &  &  &  \\
\hline
 & \includegraphics[max width=\textwidth]{2025_02_10_233d5a95a741500f2176g-21(3)}
 &  &  &  &  &  &  &  &  &  &  &  &  &  &  &  &  &  &  &  &  &  &  &  &  &  &  &  \\
\hline
 &  &  &  &  &  &  &  &  &  &  &  &  &  &  &  &  &  &  &  &  &  &  &  &  &  &  &  &  \\
\hline
 & \includegraphics[max width=\textwidth]{2025_02_10_233d5a95a741500f2176g-21(2)}
 &  &  &  &  &  &  &  &  &  &  &  &  &  &  &  &  &  &  &  &  &  &  &  &  &  &  &  \\
\hline
 &  &  &  &  &  &  &  &  &  &  &  &  &  &  &  &  &  &  &  &  &  &  &  &  &  &  &  &  \\
\hline
 &  &  &  &  &  &  &  &  &  &  &  &  &  &  &  &  &  &  &  &  &  &  &  &  &  &  &  &  \\
\hline
 & \includegraphics[max width=\textwidth]{2025_02_10_233d5a95a741500f2176g-21}
 &  &  &  &  &  &  &  &  &  &  &  &  &  &  &  &  &  &  &  &  &  &  &  &  &  &  &  \\
\hline
 &  &  &  &  &  &  &  &  &  &  &  &  &  &  &  &  &  &  &  &  &  &  &  &  &  &  &  &  \\
\hline
 & \includegraphics[max width=\textwidth]{2025_02_10_233d5a95a741500f2176g-21(5)}
 &  &  &  &  &  &  &  &  &  &  &  &  &  &  &  &  &  &  &  &  &  &  &  &  &  &  &  \\
\hline
 & \includegraphics[max width=\textwidth]{2025_02_10_233d5a95a741500f2176g-21(1)}
 &  &  &  &  &  &  &  &  &  &  &  &  &  &  &  &  &  &  &  &  &  &  &  &  &  &  &  \\
\hline
 & - &  &  &  &  &  &  &  &  &  &  &  &  &  &  &  &  &  &  &  &  &  &  &  &  &  &  &  \\
\hline
 & \includegraphics[max width=\textwidth]{2025_02_10_233d5a95a741500f2176g-21(4)}
 &  &  &  &  &  &  &  &  &  &  &  &  &  &  &  &  &  &  &  &  &  &  &  &  &  &  &  \\
\hline
 & \(\square\) &  &  &  &  &  &  &  &  &  &  &  &  &  &  &  &  &  &  &  &  &  &  &  &  &  &  &  \\
\hline
 & , &  &  &  &  &  &  &  &  &  &  &  &  &  &  &  &  &  &  &  &  &  &  &  &  &  &  &  \\
\hline
 & , &  &  &  &  & - &  &  &  &  &  &  &  &  & \includegraphics[max width=\textwidth]{2025_02_10_233d5a95a741500f2176g-21(6)}
 &  &  &  &  &  &  &  &  &  &  &  &  &  \\
\hline
 & \(\square\) &  &  &  &  &  &  &  &  &  &  &  &  &  &  &  &  &  &  &  &  &  &  &  &  &  &  &  \\
\hline
 & , &  &  &  &  &  &  &  &  &  &  &  &  &  &  &  &  &  &  &  &  &  &  &  &  &  &  &  \\
\hline
 & - &  &  &  &  &  &  &  &  &  &  &  &  &  &  &  &  &  &  &  &  &  &  &  &  &  &  &  \\
\hline
 & - &  &  &  &  &  &  &  &  &  &  &  &  &  &  &  &  &  &  &  &  &  &  &  &  &  &  &  \\
\hline
 & , &  &  &  &  &  &  &  &  &  &  &  &  &  &  &  &  &  &  &  &  &  &  &  &  &  &  &  \\
\hline
 & - &  &  &  &  &  &  &  &  &  &  &  &  &  &  &  &  &  &  &  &  &  &  &  &  &  &  &  \\
\hline
 & - & \(\square\) &  &  &  &  &  &  &  &  &  &  &  &  & - &  &  &  &  &  &  &  &  &  &  &  &  &  \\
\hline
 & , &  &  &  &  &  &  &  &  &  &  &  &  &  &  &  &  &  &  &  &  &  &  &  &  &  &  &  \\
\hline
 & - &  &  &  &  &  &  &  &  &  &  &  &  &  &  &  &  &  &  &  &  &  &  & \(\square\) &  &  &  &  \\
\hline
 &  &  &  &  &  &  &  &  &  &  &  &  &  &  &  &  &  &  &  &  &  &  &  &  &  &  &  &  \\
\hline
 &  &  &  &  &  &  &  &  &  &  &  &  &  &  &  &  &  &  &  &  &  &  &  &  &  &  &  &  \\
\hline
\end{tabular}
\end{center}

Odpowiedź: \(\qquad\)

\section*{Zadanie 34. (0-5)}
Podstawą graniastosłupa prostego \(A B C D A^{\prime} B^{\prime} C^{\prime} D^{\prime}\) jest romb \(A B C D\). Przekątna \(A C^{\prime}\) tego graniastosłupa ma długość 8 i jest nachylona do płaszczyzny podstawy pod kątem \(30^{\circ}\), a przekątna \(B D^{\prime}\) jest nachylona do tej płaszczyzny pod kątem \(45^{\circ}\). Oblicz pole powierzchni całkowitej tego graniastosłupa.\\
\includegraphics[max width=\textwidth, center]{2025_02_10_233d5a95a741500f2176g-22}\\
\includegraphics[max width=\textwidth, center]{2025_02_10_233d5a95a741500f2176g-22(1)}\\
\includegraphics[max width=\textwidth, center]{2025_02_10_233d5a95a741500f2176g-23}

Odpowiedź:

\section*{BRUDNOPIS (nie podlega ocenie)}

\end{document}