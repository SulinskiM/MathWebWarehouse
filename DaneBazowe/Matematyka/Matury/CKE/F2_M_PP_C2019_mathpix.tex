\documentclass[10pt]{article}
\usepackage[polish]{babel}
\usepackage[utf8]{inputenc}
\usepackage[T1]{fontenc}
\usepackage{graphicx}
\usepackage[export]{adjustbox}
\graphicspath{ {./images/} }
\usepackage{amsmath}
\usepackage{amsfonts}
\usepackage{amssymb}
\usepackage[version=4]{mhchem}
\usepackage{stmaryrd}

\newcommand\Varangle{\mathop{{<\!\!\!\!\!\text{\small)}}\:}\nolimits}

\begin{document}
KOD\\
\includegraphics[max width=\textwidth, center]{2025_02_10_4569edf99f1a2b54835bg-01}

PESEL\\
\includegraphics[max width=\textwidth, center]{2025_02_10_4569edf99f1a2b54835bg-01(1)}\\
miejsce na naklejke

EGZAMIN MATURALNY Z MATEMATYKI Poziom podstawowy

\section*{Data: \(\mathbf{4}\) czerwca 2019 r.}
Godzina rozpoczecia: 9:00\\
CZAS PRACY: \(\mathbf{1 7 0}\) minut

\begin{center}
\begin{tabular}{|c|}
\hline
UZUPELNIA ZESPÓL \\
NADZORUJACY \\
\end{tabular}
\end{center}\(|\)\begin{tabular}{|c|l|}
\hline
Uprawnienia zdająego do: &  \\
\(\square \square\) & \begin{tabular}{l}
dostosowania \\
kryteriów oceniania \\
nieprzenoszenia \\
zaznaczeń na karte \\
\end{tabular} \\
\(\square \square\) & \begin{tabular}{l}
dostosowania \\
w zw. z dyskalkulią \\
\end{tabular} \\
\hline
\end{tabular}

Liczba punktów do uzyskania: \(\mathbf{5 0}\)

\section*{Instrukcja dla zdającego}
\begin{enumerate}
  \item Sprawdź, czy arkusz egzaminacyjny zawiera 26 stron (zadania 1-34). Ewentualny brak zgłoś przewodniczącemu zespołu nadzorującego egzamin.
  \item Rozwiązania zadań i odpowiedzi wpisuj w miejscu na to przeznaczonym.
  \item Odpowiedzi do zadań zamkniętych (1-25) zaznacz na karcie odpowiedzi, w części karty przeznaczonej dla zdającego. Zamaluj \(\square\) pola do tego przeznaczone. Błędne zaznaczenie otocz kółkiem \(\square_{\text {i zaznacz właściwe }}\)
  \item Pamiętaj, że pominięcie argumentacji lub istotnych obliczeń w rozwiązaniu zadania otwartego (26-34) może spowodować, że za to rozwiązanie nie otrzymasz pełnej liczby punktów.
  \item Pisz czytelnie i używaj tylko długopisu lub pióra z czarnym tuszem lub atramentem.
  \item Nie używaj korektora, a błędne zapisy wyraźnie przekreśl.
  \item Pamiętaj, że zapisy w brudnopisie nie będą oceniane.
  \item Możesz korzystać z zestawu wzorów matematycznych, cyrkla i linijki, a także z kalkulatora prostego.
  \item Na tej stronie oraz na karcie odpowiedzi wpisz swój numer PESEL i przyklej naklejkę z kodem.
  \item Nie wpisuj żadnych znaków w części przeznaczonej dla egzaminatora.\\
\includegraphics[max width=\textwidth, center]{2025_02_10_4569edf99f1a2b54835bg-01(2)}
\end{enumerate}

W zadaniach od 1. do 25. wybierz i zaznacz na karcie odpowiedzi poprawna odpowiedź.

\section*{Zadanie 1. (0-1)}
Rozwiązaniem równania \(\frac{\left(x^{2}-2 x-3\right) \cdot\left(x^{2}-9\right)}{x-1}=0 \underline{\text { nie jest }}\) liczba\\
A. -3\\
B. -1\\
C. 1\\
D. 3

\section*{Zadanie 2. (0-1)}
Liczba \(\frac{\log _{3} 27}{\log _{3} \sqrt{27}}\) jest równa\\
A. \(-\frac{1}{2}\)\\
B. 2\\
C. -2\\
D. \(\frac{1}{2}\)

\section*{Zadanie 3. (0-1)}
Jedną z liczb spełniających nierówność \((x-6) \cdot(x-2)^{2} \cdot(x+4) \cdot(x+10)>0\) jest\\
A. -5\\
B. 0\\
C. 3\\
D. 5

\section*{Zadanie 4. (0-1)}
Liczba dodatnia a jest zapisana w postaci ułamka zwykłego. Jeżeli licznik tego ułamka zmniejszymy o \(50 \%\), a jego mianownik zwiększymy o \(50 \%\), to otrzymamy liczbę \(b\) taką, że\\
A. \(b=\frac{1}{4} a\)\\
B. \(b=\frac{1}{3} a\)\\
C. \(b=\frac{1}{2} a\)\\
D. \(b=\frac{2}{3} a\)

\section*{Zadanie 5. (0-1)}
Funkcja liniowa \(f\) jest określona wzorem \(f(x)=(a+1) x+11\), gdzie \(a\) to pewna liczba rzeczywista, ma miejsce zerowe równe \(x=\frac{3}{4}\). Stąd wynika, że\\
A. \(a=-\frac{41}{3}\)\\
B. \(a=\frac{41}{3}\)\\
C. \(a=-\frac{47}{3}\)\\
D. \(a=\frac{47}{3}\)

\section*{BRUDNOPIS (nie podlega ocenie)}
\begin{center}
\includegraphics[max width=\textwidth]{2025_02_10_4569edf99f1a2b54835bg-03}
\end{center}

\section*{Zadanie 6. (0-1)}
Funkcja \(f\) jest określona dla każdej liczby rzeczywistej \(x\) wzorem \(f(x)=(m \sqrt{5}-1) x+3\).\\
Ta funkcja jest rosnąca dla każdej liczby \(m\) spełniającej warunek\\
A. \(m>\frac{1}{\sqrt{5}}\)\\
B. \(m>1-\sqrt{5}\)\\
C. \(m<\sqrt{5}-1\)\\
D. \(m<\frac{1}{\sqrt{5}}\)

\section*{Zadanie 7. (0-1)}
Układ równań \(\left\{\begin{array}{l}2 x-y=2 \\ x+m y=1\end{array}\right.\) ma nieskończenie wiele rozwiązań dla\\
A. \(m=-1\)\\
B. \(m=1\)\\
C. \(m=\frac{1}{2}\)\\
D. \(m=-\frac{1}{2}\)

\section*{Zadanie 8. (0-1)}
Rysunek przedstawia wykres funkcji \(f\) zbudowany z 6 odcinków, przy czym punkty \(B=(2,-1)\) i \(C=(4,-1)\) należą do wykresu funkcji.\\
\includegraphics[max width=\textwidth, center]{2025_02_10_4569edf99f1a2b54835bg-04}

Równanie \(f(x)=-1 \mathrm{ma}\)\\
A. dokładnie jedno rozwiązanie.\\
B. dokładnie dwa rozwiązania.\\
C. dokładnie trzy rozwiązania.\\
D. nieskończenie wiele rozwiązań.

\section*{BRUDNOPIS (nie podlega ocenie)}
\begin{center}
\includegraphics[max width=\textwidth]{2025_02_10_4569edf99f1a2b54835bg-05}
\end{center}

\section*{Zadanie 9. (0-1)}
Dany jest rosnący ciąg arytmetyczny \(\left(a_{n}\right)\), określony dla liczb naturalnych \(n \geq 1\), o wyrazach dodatnich. Jeśli \(a_{2}+a_{9}=a_{4}+a_{k}\), to \(k\) jest równe\\
A. 8\\
B. 7\\
C. 6\\
D. 5

\section*{Zadanie 10. (0-1)}
W ciągu \(\left(a_{n}\right)\) określonym dla każdej liczby \(n \geq 1\) jest spełniony warunek \(a_{n+3}=-2 \cdot 3^{n+1}\). Wtedy\\
A. \(a_{5}=-54\)\\
B. \(a_{5}=-27\)\\
C. \(a_{5}=27\)\\
D. \(a_{5}=54\)

\section*{Zadanie 11. (0-1)}
Dla każdej liczby rzeczywistej \(x\) wyrażenie \((3 x-2)^{2}-(2 x-3)(2 x+3)\) jest po uproszczeniu równe\\
A. \(5 x^{2}-12 x-5\)\\
B. \(5 x^{2}-13\)\\
C. \(5 x^{2}-12 x+13\)\\
D. \(5 x^{2}+5\)

\section*{Zadanie 12. (0-1)}
Kąt \(\alpha \in\left(0^{\circ}, 180^{\circ}\right)\) oraz wiadomo, że \(\sin \alpha \cdot \cos \alpha=-\frac{3}{8}\). Wartość wyrażenia \((\cos \alpha-\sin \alpha)^{2}+2\) jest równa\\
A. \(\frac{15}{4}\)\\
B. \(\frac{9}{4}\)\\
C. \(\frac{27}{8}\)\\
D. \(\frac{21}{8}\)

\section*{Zadanie 13. (0-1)}
Wartość wyrażenia \(2 \sin ^{2} 18^{\circ}+\sin ^{2} 72^{\circ}+\cos ^{2} 18^{\circ}\) jest równa\\
A. 0\\
B. 1\\
C. 2\\
D. 4

\section*{BRUDNOPIS (nie podlega ocenie)}
\begin{center}
\begin{tabular}{|c|c|c|c|c|c|c|c|c|c|c|c|c|c|c|c|c|c|c|c|c|c|c|c|}
\hline
 &  &  &  &  &  &  &  &  &  &  &  &  &  &  &  &  &  &  &  &  &  &  &  \\
\hline
 &  &  &  &  &  &  &  &  &  &  &  &  &  &  &  &  &  &  &  &  &  &  &  \\
\hline
 &  &  &  &  &  &  &  &  &  &  &  &  &  &  &  &  &  &  &  &  &  &  &  \\
\hline
 &  &  &  &  &  &  &  &  &  &  &  &  &  &  &  &  &  &  &  &  &  &  &  \\
\hline
 &  &  &  &  &  &  &  &  &  &  &  &  &  &  &  &  &  &  &  &  &  &  &  \\
\hline
 &  &  &  &  &  &  &  &  &  &  &  &  &  &  &  &  &  &  &  &  &  &  &  \\
\hline
 &  &  &  &  &  &  &  &  &  &  &  &  &  &  &  &  &  &  &  &  &  &  &  \\
\hline
 &  &  &  &  &  &  &  &  &  &  &  &  &  &  &  &  &  &  &  &  &  &  &  \\
\hline
 &  &  &  &  &  &  &  &  &  &  &  &  &  &  &  &  &  &  &  &  &  &  &  \\
\hline
 &  &  &  &  &  &  &  &  &  &  &  &  &  &  &  &  &  &  &  &  &  &  &  \\
\hline
 &  &  &  &  &  &  &  &  &  &  &  &  &  &  &  &  &  &  &  &  &  &  &  \\
\hline
 &  &  &  &  &  &  &  &  &  &  &  &  &  &  &  &  &  &  &  &  &  &  &  \\
\hline
 &  &  &  &  &  &  &  &  &  &  &  &  &  &  &  &  &  &  &  &  &  &  &  \\
\hline
 &  &  &  &  &  &  &  &  &  &  &  &  &  &  &  &  &  &  &  &  &  &  &  \\
\hline
 &  &  &  &  &  &  &  &  &  &  &  &  &  &  &  &  &  &  &  &  &  &  &  \\
\hline
 &  &  &  &  &  &  &  &  &  &  &  &  &  &  &  &  &  &  &  &  &  &  &  \\
\hline
 &  &  &  &  &  &  &  &  &  &  &  &  &  &  &  &  &  &  &  &  &  &  &  \\
\hline
 &  &  &  &  &  &  &  &  &  &  &  &  &  &  &  &  &  &  &  &  &  &  &  \\
\hline
 &  &  &  &  &  &  &  &  &  &  &  &  &  &  &  &  &  &  &  &  &  &  &  \\
\hline
 &  &  &  &  &  &  &  &  &  &  &  &  &  &  &  &  &  &  &  &  &  &  &  \\
\hline
 &  &  &  &  &  &  &  &  &  &  &  &  &  &  &  &  &  &  &  &  &  &  &  \\
\hline
 &  &  &  &  &  &  &  &  &  &  &  &  &  &  &  &  &  &  &  &  &  &  &  \\
\hline
 &  &  &  &  &  &  &  &  &  &  &  &  &  &  &  &  &  &  &  &  &  &  &  \\
\hline
 &  &  &  &  &  &  &  &  &  &  &  &  &  &  &  &  &  &  &  &  &  &  &  \\
\hline
 &  &  &  &  &  &  &  &  &  &  &  &  &  &  &  &  &  &  &  &  &  &  &  \\
\hline
 &  &  &  &  &  &  &  &  &  &  &  &  &  &  &  &  &  &  &  &  &  &  &  \\
\hline
 &  &  &  &  &  &  &  &  &  &  &  &  &  &  &  &  &  &  &  &  &  &  &  \\
\hline
 &  &  &  &  &  &  &  &  &  &  &  &  &  &  &  &  &  &  &  &  &  &  &  \\
\hline
 &  &  &  &  &  &  &  &  &  &  &  &  &  &  &  &  &  &  &  &  &  &  &  \\
\hline
 &  &  &  &  &  &  &  &  &  &  &  &  &  &  &  &  &  &  &  &  &  &  &  \\
\hline
 &  &  &  &  &  &  &  &  &  &  &  &  &  &  &  &  &  &  &  &  &  &  &  \\
\hline
 &  &  &  &  &  &  &  &  &  &  &  &  &  &  &  &  &  &  &  &  &  &  &  \\
\hline
 &  &  &  &  &  &  &  &  &  &  &  &  &  &  &  &  &  &  &  &  &  &  &  \\
\hline
 &  &  &  &  &  &  &  &  &  &  &  &  &  &  &  &  &  &  &  &  &  &  &  \\
\hline
 &  &  &  &  &  &  &  &  &  &  &  &  &  &  &  &  &  &  &  &  &  &  &  \\
\hline
 &  &  &  &  &  &  &  &  &  &  &  &  &  &  &  &  &  &  &  &  &  &  &  \\
\hline
 &  &  &  &  &  &  &  &  &  &  &  &  &  &  &  &  &  &  &  &  &  &  &  \\
\hline
 &  &  &  &  &  &  &  &  &  &  &  &  &  &  &  &  &  &  &  &  &  &  &  \\
\hline
 &  &  &  &  &  &  &  &  &  &  &  &  &  &  &  &  &  &  &  &  &  &  &  \\
\hline
 &  &  &  &  &  &  &  &  &  &  &  &  &  &  &  &  &  &  &  &  &  &  &  \\
\hline
 &  &  &  &  &  &  &  &  &  &  &  &  &  &  &  &  &  &  &  &  &  &  &  \\
\hline
 &  &  &  &  &  &  &  &  &  &  &  &  &  &  &  &  &  &  &  &  &  &  &  \\
\hline
 &  &  &  &  &  &  &  &  &  &  &  &  &  &  &  &  &  &  &  &  &  &  &  \\
\hline
 &  &  &  &  &  &  &  &  &  &  &  &  &  &  &  &  &  &  &  &  &  &  &  \\
\hline
 &  &  &  &  &  &  &  &  &  &  &  &  &  &  &  &  &  &  &  &  &  &  &  \\
\hline
 &  &  &  &  &  &  &  &  &  &  &  &  &  &  &  &  &  &  &  &  &  &  &  \\
\hline
 &  &  &  &  &  &  &  &  &  &  &  &  &  &  &  &  &  &  &  &  &  &  &  \\
\hline
 &  &  &  &  &  &  &  &  &  &  &  &  &  &  &  &  &  &  &  &  &  &  &  \\
\hline
\end{tabular}
\end{center}

\section*{Zadanie 14. (0-1)}
Punkty \(B\), \(C\) i \(D\) leżą na okręgu o środku \(S\) i promieniu \(r\). Punkt \(A\) jest punktem wspólnym prostych \(B C\) i \(S D\), a odcinki \(A B\) i \(S C\) są równej długości. Miara kąta \(B C S\) jest równa \(34^{\circ}\) (zobacz rysunek). Wtedy\\
A. \(\alpha=12^{\circ}\)\\
B. \(\alpha=17^{\circ}\)\\
C. \(\alpha=22^{\circ}\)\\
D. \(\alpha=34^{\circ}\)\\
\includegraphics[max width=\textwidth, center]{2025_02_10_4569edf99f1a2b54835bg-08(1)}

\section*{Zadanie 15. (0-1)}
Pole trójkąta \(A B C\) o wierzchołkach \(A=(0,0), B=(4,2), C=(2,6)\) jest równe\\
A. 5\\
B. 10\\
C. 15\\
D. 20

\section*{Zadanie 16. (0-1)}
Na okręgu o środku w punkcie \(O\) wybrano trzy punkty \(A, B, C\) tak, że \(|\Varangle A O B|=70^{\circ}\), \(|\Varangle O A C|=25^{\circ}\). Cięciwa \(A C\) przecina promień \(O B\) (zobacz rysunek). Wtedy miara \(\Varangle O B C\) jest równa\\
A. \(\alpha=25^{\circ}\)\\
B. \(\alpha=60^{\circ}\)\\
C. \(\alpha=70^{\circ}\)\\
D. \(\alpha=85^{\circ}\)\\
\includegraphics[max width=\textwidth, center]{2025_02_10_4569edf99f1a2b54835bg-08}

\section*{Zadanie 17. (0-1)}
W układzie współrzędnych na płaszczyźnie dany jest odcinek \(A B\) o końcach w punktach \(A=(7,4), B=(11,12)\). Punkt \(S\) leży wewnątrz odcinka \(A B\) oraz \(|A S|=3 \cdot|B S|\). Wówczas\\
A. \(S=(8,6)\)\\
B. \(S=(9,8)\)\\
C. \(S=(10,10)\)\\
D. \(S=(13,16)\)

\section*{BRUDNOPIS (nie podlega ocenie)}
\begin{center}
\begin{tabular}{|c|c|c|c|c|c|c|c|c|c|c|c|c|c|c|c|c|c|c|c|c|c|c|}
\hline
 &  &  &  &  &  &  &  &  &  &  &  &  &  &  &  &  &  &  &  &  &  &  \\
\hline
 &  &  &  &  &  &  &  &  &  &  &  &  &  &  &  &  &  &  &  &  &  &  \\
\hline
 &  &  &  &  &  &  &  &  &  &  &  &  &  &  &  &  &  &  &  &  &  &  \\
\hline
 &  &  &  &  &  &  &  &  &  &  &  &  &  &  &  &  &  &  &  &  &  &  \\
\hline
 &  &  &  &  &  &  &  &  &  &  &  &  &  &  &  &  &  &  &  &  &  &  \\
\hline
 &  &  &  &  &  &  &  &  &  &  &  &  &  &  &  &  &  &  &  &  &  &  \\
\hline
 &  &  &  &  &  &  &  &  &  &  &  &  &  &  &  &  &  &  &  &  &  &  \\
\hline
 &  &  &  &  &  &  &  &  &  &  &  &  &  &  &  &  &  &  &  &  &  &  \\
\hline
 &  &  &  &  &  &  &  &  &  &  &  &  &  &  &  &  &  &  &  &  &  &  \\
\hline
 &  &  &  &  &  &  &  &  &  &  &  &  &  &  &  &  &  &  &  &  &  &  \\
\hline
 &  &  &  &  &  &  &  &  &  &  &  &  &  &  &  &  &  &  &  &  &  &  \\
\hline
 &  &  &  &  &  &  &  &  &  &  &  &  &  &  &  &  &  &  &  &  &  &  \\
\hline
 &  &  &  &  &  &  &  &  &  &  &  &  &  &  &  &  &  &  &  &  &  &  \\
\hline
 &  &  &  &  &  &  &  &  &  &  &  &  &  &  &  &  &  &  &  &  &  &  \\
\hline
 &  &  &  &  &  &  &  &  &  &  &  &  &  &  &  &  &  &  &  &  &  &  \\
\hline
 &  &  &  &  &  &  &  &  &  &  &  &  &  &  &  &  &  &  &  &  &  &  \\
\hline
 &  &  &  &  &  &  &  &  &  &  &  &  &  &  &  &  &  &  &  &  &  &  \\
\hline
 &  &  &  &  &  &  &  &  &  &  &  &  &  &  &  &  &  &  &  &  &  &  \\
\hline
 &  &  &  &  &  &  &  &  &  &  &  &  &  &  &  &  &  &  &  &  &  &  \\
\hline
 &  &  &  &  &  &  &  &  &  &  &  &  &  &  &  &  &  &  &  &  &  &  \\
\hline
 &  &  &  &  &  &  &  &  &  &  &  &  &  &  &  &  &  &  &  &  &  &  \\
\hline
 &  &  &  &  &  &  &  &  &  &  &  &  &  &  &  &  &  &  &  &  &  &  \\
\hline
 &  &  &  &  &  &  &  &  &  &  &  &  &  &  &  &  &  &  &  &  &  &  \\
\hline
 &  &  &  &  &  &  &  &  &  &  &  &  &  &  &  &  &  &  &  &  &  &  \\
\hline
 &  &  &  &  &  &  &  &  &  &  &  &  &  &  &  &  &  &  &  &  &  &  \\
\hline
 &  &  &  &  &  &  &  &  &  &  &  &  &  &  &  &  &  &  &  &  &  &  \\
\hline
 &  &  &  &  &  &  &  &  &  &  &  &  &  &  &  &  &  &  &  &  &  &  \\
\hline
 &  &  &  &  &  &  &  &  &  &  &  &  &  &  &  &  &  &  &  &  &  &  \\
\hline
 &  &  &  &  &  &  &  &  &  &  &  &  &  &  &  &  &  &  &  &  &  &  \\
\hline
 &  &  &  &  &  &  &  &  &  &  &  &  &  &  &  &  &  &  &  &  &  &  \\
\hline
 &  &  &  &  &  &  &  &  &  &  &  &  &  &  &  &  &  &  &  &  &  &  \\
\hline
 &  &  &  &  &  &  &  &  &  &  &  &  &  &  &  &  &  &  &  &  &  &  \\
\hline
 &  &  &  &  &  &  &  &  &  &  &  &  &  &  &  &  &  &  &  &  &  &  \\
\hline
 &  &  &  &  &  &  &  &  &  &  &  &  &  &  &  &  &  &  &  &  &  &  \\
\hline
 &  &  &  &  &  &  &  &  &  &  &  &  &  &  &  &  &  &  &  &  &  &  \\
\hline
 &  &  &  &  &  &  &  &  &  &  &  &  &  &  &  &  &  &  &  &  &  &  \\
\hline
 &  &  &  &  &  &  &  &  &  &  &  &  &  &  &  &  &  &  &  &  &  &  \\
\hline
 &  &  &  &  &  &  &  &  &  &  &  &  &  &  &  &  &  &  &  &  &  &  \\
\hline
 &  &  &  &  &  &  &  &  &  &  &  &  &  &  &  &  &  &  &  &  &  &  \\
\hline
 &  &  &  &  &  &  &  &  &  &  &  &  &  &  &  &  &  &  &  &  &  &  \\
\hline
 &  &  &  &  &  &  &  &  &  &  &  &  &  &  &  &  &  &  &  &  &  &  \\
\hline
 &  &  &  &  &  &  &  &  &  &  &  &  &  &  &  &  &  &  &  &  &  &  \\
\hline
 &  &  &  &  &  &  &  &  &  &  &  &  &  &  &  &  &  &  &  &  &  &  \\
\hline
 &  &  &  &  &  &  &  &  &  &  &  &  &  &  &  &  &  &  &  &  &  &  \\
\hline
 &  &  &  &  &  &  &  &  &  &  &  &  &  &  &  &  &  &  &  &  &  &  \\
\hline
 &  &  &  &  &  &  &  &  &  &  &  &  &  &  &  &  &  &  &  &  &  &  \\
\hline
 &  &  &  &  &  &  &  &  &  &  &  &  &  &  &  &  &  &  &  &  &  &  \\
\hline
 &  &  &  &  &  &  &  &  &  &  &  &  &  &  &  &  &  &  &  &  &  &  \\
\hline
\end{tabular}
\end{center}

\section*{Zadanie 18. (0-1)}
Suma odległości punktu \(A=(-4,2)\) od prostych o równaniach \(x=4\) i \(y=-4\) jest równa\\
A. 14\\
B. 12\\
C. 10\\
D. 8

\section*{Zadanie 19. (0-1)}
Suma długości wszystkich krawędzi sześcianu jest równa 96 cm . Pole powierzchni całkowitej tego sześcianu jest równe\\
A. \(48 \mathrm{~cm}^{2}\)\\
B. \(64 \mathrm{~cm}^{2}\)\\
C. \(384 \mathrm{~cm}^{2}\)\\
D. \(512 \mathrm{~cm}^{2}\)

\section*{Zadanie 20. (0-1)}
Dany jest trójkąt równoramienny \(A B C\), w którym \(|A C|=|B C|\). Kąt między ramionami tego trójkąta ma miarę \(44^{\circ}\). Dwusieczna kąta poprowadzona z wierzchołka \(A\) przecina bok \(B C\) tego trójkąta w punkcie \(D\). Kąt \(A D C\) ma miarę\\
A. \(78^{\circ}\)\\
B. \(34^{\circ}\)\\
C. \(68^{\circ}\)\\
D. \(102^{\circ}\)

\section*{Zadanie 21. (0-1)}
Liczb naturalnych dwucyfrowych podzielnych przez 6 jest\\
A. 60\\
B. 45\\
C. 30\\
D. 15

\section*{Zadanie 22. (0-1)}
Podstawą ostrosłupa jest kwadrat \(A B C D\) o boku długości 4. Krawędź boczna \(D S\) jest prostopadła do podstawy i ma długość 3 (zobacz rysunek).\\
\includegraphics[max width=\textwidth, center]{2025_02_10_4569edf99f1a2b54835bg-10}

Pole ściany \(B C S\) tego ostrosłupa jest równe\\
A. 20\\
B. 10\\
C. 16\\
D. 12

\section*{BRUDNOPIS (nie podlega ocenie)}
\section*{Zadanie 23. (0-1)}
Dany jest sześcian \(A B C D E F G H\). Przekątne \(A C\) i \(B D\) ściany \(A B C D\) sześcianu przecinają się w punkcie \(P\) (zobacz rysunek).\\
\includegraphics[max width=\textwidth, center]{2025_02_10_4569edf99f1a2b54835bg-12(1)}

Tangens kąta, jaki odcinek \(P H\) tworzy z płaszczyzną \(A B C D\), jest równy\\
A. \(\frac{\sqrt{2}}{2}\)\\
B. \(\frac{1}{2}\)\\
C. 1\\
D. \(\sqrt{2}\)

\section*{Zadanie 24. (0-1)}
Przekrojem osiowym walca jest kwadrat o przekątnej długości 12. Objętość tego walca jest zatem równa\\
A. \(36 \pi \sqrt{2}\)\\
B. \(108 \pi \sqrt{2}\)\\
C. \(54 \pi\)\\
D. \(108 \pi\)

\section*{Zadanie 25. (0-1)}
Ze zbioru kolejnych liczb naturalnych \(\{20,21,22, \ldots, 39,40\}\) losujemy jedną liczbę. Prawdopodobieństwo wylosowania liczby podzielnej przez 4 jest równe\\
A. \(\frac{1}{4}\)\\
B. \(\frac{2}{7}\)\\
C. \(\frac{6}{19}\)\\
D. \(\frac{3}{10}\)

\section*{BRUDNOPIS (nie podlega ocenie)}
\includegraphics[max width=\textwidth, center]{2025_02_10_4569edf99f1a2b54835bg-12}\\
\includegraphics[max width=\textwidth, center]{2025_02_10_4569edf99f1a2b54835bg-13}

\section*{Zadanie 26. (0-2)}
Rozwiąż nierówność \(x(7 x+2)>7 x+2\).

\begin{center}
\begin{tabular}{|c|c|c|c|c|c|c|c|c|c|c|c|c|c|c|c|c|c|c|c|c|c|c|}
\hline
 &  &  &  &  &  &  &  &  &  &  &  &  &  &  &  &  &  &  &  &  &  &  \\
\hline
 &  &  &  &  &  &  &  &  &  &  &  &  &  &  &  &  &  &  &  &  &  &  \\
\hline
 &  &  &  &  &  &  &  &  &  &  &  &  &  &  &  &  &  &  &  &  &  &  \\
\hline
 &  &  &  &  &  &  &  &  &  &  &  &  &  &  &  &  &  &  &  &  &  &  \\
\hline
 &  &  &  &  &  &  &  &  &  &  &  &  &  &  &  &  &  &  &  &  &  &  \\
\hline
 &  &  &  &  &  &  &  &  &  &  &  &  &  &  &  &  &  &  &  &  &  &  \\
\hline
 &  &  &  &  &  &  &  &  &  &  &  &  &  &  &  &  &  &  &  &  &  &  \\
\hline
 &  &  &  &  &  &  &  &  &  &  &  &  &  &  &  &  &  &  &  &  &  &  \\
\hline
 &  &  &  &  &  &  &  &  &  &  &  &  &  &  &  &  &  &  &  &  &  &  \\
\hline
 &  &  &  &  &  &  &  &  &  &  &  &  &  &  &  &  &  &  &  &  &  &  \\
\hline
 &  &  &  &  &  &  &  &  &  &  &  &  &  &  &  &  &  &  &  &  &  &  \\
\hline
 &  &  &  &  &  &  &  &  &  &  &  &  &  &  &  &  &  &  &  &  &  &  \\
\hline
 &  &  &  &  &  &  &  &  &  &  &  &  &  &  &  &  &  &  &  &  &  &  \\
\hline
 &  &  &  &  &  &  &  &  &  &  &  &  &  &  &  &  &  &  &  &  &  &  \\
\hline
 &  &  &  &  &  &  &  &  &  &  &  &  &  &  &  &  &  &  &  &  &  &  \\
\hline
 &  &  &  &  &  &  &  &  &  &  &  &  &  &  &  &  &  &  &  &  &  &  \\
\hline
 &  &  &  &  &  &  &  &  &  &  &  &  &  &  &  &  &  &  &  &  &  &  \\
\hline
 &  &  &  &  &  &  &  &  &  &  &  &  &  &  &  &  &  &  &  &  &  &  \\
\hline
 &  &  &  &  &  &  &  &  &  &  &  &  &  &  &  &  &  &  &  &  &  &  \\
\hline
 &  &  &  &  &  &  &  &  &  &  &  &  &  &  &  &  &  &  &  &  &  &  \\
\hline
 &  &  &  &  &  &  &  &  &  &  &  &  &  &  &  &  &  &  &  &  &  &  \\
\hline
 &  &  &  &  &  &  &  &  &  &  &  &  &  &  &  &  &  &  &  &  &  &  \\
\hline
 &  &  &  &  &  &  &  &  &  &  &  &  &  &  &  &  &  &  &  &  &  &  \\
\hline
 &  &  &  &  &  &  &  &  &  &  &  &  &  &  &  &  &  &  &  &  &  &  \\
\hline
 &  &  &  &  &  &  &  &  &  &  &  &  &  &  &  &  &  &  &  &  &  &  \\
\hline
 &  &  &  &  &  &  &  &  &  &  &  &  &  &  &  &  &  &  &  &  &  &  \\
\hline
 &  &  &  &  &  &  &  &  &  &  &  &  &  &  &  &  &  &  &  &  &  &  \\
\hline
 &  &  &  &  &  &  &  &  &  &  &  &  &  &  &  &  &  &  &  &  &  &  \\
\hline
 &  &  &  &  &  &  &  &  &  &  &  &  &  &  &  &  &  &  &  &  &  &  \\
\hline
 &  &  &  &  &  &  &  &  &  &  &  &  &  &  &  &  &  &  &  &  &  &  \\
\hline
 &  &  &  &  &  &  &  &  &  &  &  &  &  &  &  &  &  &  &  &  &  &  \\
\hline
 &  &  &  &  &  &  &  &  &  &  &  &  &  &  &  &  &  &  &  &  &  &  \\
\hline
 &  &  &  &  &  &  &  &  &  &  &  &  &  &  &  &  &  &  &  &  &  &  \\
\hline
 &  &  &  &  &  &  &  &  &  &  &  &  &  &  &  &  &  &  &  &  &  &  \\
\hline
 &  &  &  &  &  &  &  &  &  &  &  &  &  &  &  &  &  &  &  &  &  &  \\
\hline
 &  &  &  &  &  &  &  &  &  &  &  &  &  &  &  &  &  &  &  &  &  &  \\
\hline
 &  &  &  &  &  &  &  &  &  &  &  &  &  &  &  &  &  &  &  &  &  &  \\
\hline
 &  &  &  &  &  &  &  &  &  &  &  &  &  &  &  &  &  &  &  &  &  &  \\
\hline
 &  &  &  &  &  &  &  &  &  &  &  &  &  &  &  &  &  &  &  &  &  &  \\
\hline
 &  &  &  &  &  &  &  &  &  &  &  &  &  &  &  &  &  &  &  &  &  &  \\
\hline
 &  &  &  &  &  &  &  &  &  &  &  &  &  &  &  &  &  &  &  &  &  &  \\
\hline
 &  &  &  &  &  &  &  &  &  &  &  &  &  &  &  &  &  &  &  &  &  &  \\
\hline
 &  &  &  &  &  &  &  &  &  &  &  &  &  &  &  &  &  &  &  &  &  &  \\
\hline
 &  &  &  &  &  &  &  &  &  &  &  &  &  &  &  &  &  &  &  &  &  &  \\
\hline
 &  &  &  &  &  &  &  &  &  &  &  &  &  &  &  &  &  &  &  &  &  &  \\
\hline
\end{tabular}
\end{center}

Odpowiedź:

\section*{Zadanie 27. (0-2)}
Wyznacz wszystkie liczby rzeczywiste \(x\), które spełniają warunek: \(\frac{3 x^{2}-8 x-3}{x-3}=x-3\).

\begin{center}
\begin{tabular}{|c|c|c|c|c|c|c|c|c|c|c|c|c|c|c|c|c|c|c|c|c|c|c|}
\hline
 &  &  &  &  &  &  &  &  &  &  &  &  &  &  &  &  &  &  &  &  &  &  \\
\hline
 &  &  &  &  &  &  &  &  &  &  &  &  &  &  &  &  &  &  &  &  &  &  \\
\hline
 &  &  &  &  &  &  &  &  &  &  &  &  &  &  &  &  &  &  &  &  &  &  \\
\hline
 &  &  &  &  &  &  &  &  &  &  &  &  &  &  &  &  &  &  &  &  &  &  \\
\hline
 &  &  &  &  &  &  &  &  &  &  &  &  &  &  &  &  &  &  &  &  &  &  \\
\hline
 &  &  &  &  &  &  &  &  &  &  &  &  &  &  &  &  &  &  &  &  &  &  \\
\hline
 &  &  &  &  &  &  &  &  &  &  &  &  &  &  &  &  &  &  &  &  &  &  \\
\hline
 &  &  &  &  &  &  &  &  &  &  &  &  &  &  &  &  &  &  &  &  &  &  \\
\hline
 &  &  &  &  &  &  &  &  &  &  &  &  &  &  &  &  &  &  &  &  &  &  \\
\hline
 &  &  &  &  &  &  &  &  &  &  &  &  &  &  &  &  &  &  &  &  &  &  \\
\hline
 &  &  &  &  &  &  &  &  &  &  &  &  &  &  &  &  &  &  &  &  &  &  \\
\hline
 &  &  &  &  &  &  &  &  &  &  &  &  &  &  &  &  &  &  &  &  &  &  \\
\hline
 &  &  &  &  &  &  &  &  &  &  &  &  &  &  &  &  &  &  &  &  &  &  \\
\hline
 &  &  &  &  &  &  &  &  &  &  &  &  &  &  &  &  &  &  &  &  &  &  \\
\hline
 &  &  &  &  &  &  &  &  &  &  &  &  &  &  &  &  &  &  &  &  &  &  \\
\hline
 &  &  &  &  &  &  &  &  &  &  &  &  &  &  &  &  &  &  &  &  &  &  \\
\hline
 &  &  &  &  &  &  &  &  &  &  &  &  &  &  &  &  &  &  &  &  &  &  \\
\hline
 &  &  &  &  &  &  &  &  &  &  &  &  &  &  &  &  &  &  &  &  &  &  \\
\hline
 &  &  &  &  &  &  &  &  &  &  &  &  &  &  &  &  &  &  &  &  &  &  \\
\hline
 &  &  &  &  &  &  &  &  &  &  &  &  &  &  &  &  &  &  &  &  &  &  \\
\hline
 &  &  &  &  &  &  &  &  &  &  &  &  &  &  &  &  &  &  &  &  &  &  \\
\hline
 &  &  &  &  &  &  &  &  &  &  &  &  &  &  &  &  &  &  &  &  &  &  \\
\hline
 &  &  &  &  &  &  &  &  &  &  &  &  &  &  &  &  &  &  &  &  &  &  \\
\hline
 &  &  &  &  &  &  &  &  &  &  &  &  &  &  &  &  &  &  &  &  &  &  \\
\hline
 &  &  &  &  &  &  &  &  &  &  &  &  &  &  &  &  &  &  &  &  &  &  \\
\hline
 &  &  &  &  &  &  &  &  &  &  &  &  &  &  &  &  &  &  &  &  &  &  \\
\hline
 &  &  &  &  &  &  &  &  &  &  &  &  &  &  &  &  &  &  &  &  &  &  \\
\hline
 &  &  &  &  &  &  &  &  &  &  &  &  &  &  &  &  &  &  &  &  &  &  \\
\hline
 &  &  &  &  &  &  &  &  &  &  &  &  &  &  &  &  &  &  &  &  &  &  \\
\hline
 &  &  &  &  &  &  &  &  &  &  &  &  &  &  &  &  &  &  &  &  &  &  \\
\hline
 &  &  &  &  &  &  &  &  &  &  &  &  &  &  &  &  &  &  &  &  &  &  \\
\hline
 &  &  &  &  &  &  &  &  &  &  &  &  &  &  &  &  &  &  &  &  &  &  \\
\hline
 &  &  &  &  &  &  &  &  &  &  &  &  &  &  &  &  &  &  &  &  &  &  \\
\hline
 &  &  &  &  &  &  &  &  &  &  &  &  &  &  &  &  &  &  &  &  &  &  \\
\hline
 &  &  &  &  &  &  &  &  &  &  &  &  &  &  &  &  &  &  &  &  &  &  \\
\hline
 &  &  &  &  &  &  &  &  &  &  &  &  &  &  &  &  &  &  &  &  &  &  \\
\hline
 &  &  &  &  &  &  &  &  &  &  &  &  &  &  &  &  &  &  &  &  &  &  \\
\hline
 &  &  &  &  &  &  &  &  &  &  &  &  &  &  &  &  &  &  &  &  &  &  \\
\hline
 &  &  &  &  &  &  &  &  &  &  &  &  &  &  &  &  &  &  &  &  &  &  \\
\hline
 &  &  &  &  &  &  &  &  &  &  &  &  &  &  &  &  &  &  &  &  &  &  \\
\hline
 &  &  &  &  &  &  &  &  &  &  &  &  &  &  &  &  &  &  &  &  &  &  \\
\hline
 &  &  &  &  &  &  &  &  &  &  &  &  &  &  &  &  &  &  &  &  &  &  \\
\hline
 &  &  &  &  &  &  &  &  &  &  &  &  &  &  &  &  &  &  &  &  &  &  \\
\hline
 &  &  &  &  &  &  &  &  &  &  &  &  &  &  &  &  &  &  &  &  &  &  \\
\hline
\end{tabular}
\end{center}

Odpowiedź:

\section*{Zadanie 28. (0-2)}
Dany jest trójkąt \(A B C\). Punkt \(S\) jest środkiem boku \(A B\) tego trójkąta (zobacz rysunek). Wykaż, że odległości punktów \(A\) i \(B\) od prostej \(C S\) są równe.\\
\includegraphics[max width=\textwidth, center]{2025_02_10_4569edf99f1a2b54835bg-16}\\
\includegraphics[max width=\textwidth, center]{2025_02_10_4569edf99f1a2b54835bg-16(1)}

\section*{Zadanie 29. (0-2)}
Wykaż, że dla każdej liczby \(a>0\) i dla każdej liczby \(b>0\) prawdziwa jest nierówność

\[
\frac{1}{a}+\frac{1}{b} \geq \frac{4}{a+b} .
\]

\begin{center}
\begin{tabular}{|c|c|c|c|c|c|c|c|c|c|c|c|c|c|c|c|c|c|c|c|c|c|c|}
\hline
 &  &  &  &  &  &  &  &  &  &  &  &  &  &  &  &  &  &  &  &  &  &  \\
\hline
 &  &  &  &  &  &  &  &  &  &  &  &  &  &  &  &  &  &  &  &  &  &  \\
\hline
 &  &  &  &  &  &  &  &  &  &  &  &  &  &  &  &  &  &  &  &  &  &  \\
\hline
 &  &  &  &  &  &  &  &  &  &  &  &  &  &  &  &  &  &  &  &  &  &  \\
\hline
 &  &  &  &  &  &  &  &  &  &  &  &  &  &  &  &  &  &  &  &  &  &  \\
\hline
 &  &  &  &  &  &  &  &  &  &  &  &  &  &  &  &  &  &  &  &  &  &  \\
\hline
 &  &  &  &  &  &  &  &  &  &  &  &  &  &  &  &  &  &  &  &  &  &  \\
\hline
 &  &  &  &  &  &  &  &  &  &  &  &  &  &  &  &  &  &  &  &  &  &  \\
\hline
 &  &  &  &  &  &  &  &  &  &  &  &  &  &  &  &  &  &  &  &  &  &  \\
\hline
 &  &  &  &  &  &  &  &  &  &  &  &  &  &  &  &  &  &  &  &  &  &  \\
\hline
 &  &  &  &  &  &  &  &  &  &  &  &  &  &  &  &  &  &  &  &  &  &  \\
\hline
 &  &  &  &  &  &  &  &  &  &  &  &  &  &  &  &  &  &  &  &  &  &  \\
\hline
 &  &  &  &  &  &  &  &  &  &  &  &  &  &  &  &  &  &  &  &  &  &  \\
\hline
 &  &  &  &  &  &  &  &  &  &  &  &  &  &  &  &  &  &  &  &  &  &  \\
\hline
 &  &  &  &  &  &  &  &  &  &  &  &  &  &  &  &  &  &  &  &  &  &  \\
\hline
 &  &  &  &  &  &  &  &  &  &  &  &  &  &  &  &  &  &  &  &  &  &  \\
\hline
 &  &  &  &  &  &  &  &  &  &  &  &  &  &  &  &  &  &  &  &  &  &  \\
\hline
 &  &  &  &  &  &  &  &  &  &  &  &  &  &  &  &  &  &  &  &  &  &  \\
\hline
 &  &  &  &  &  &  &  &  &  &  &  &  &  &  &  &  &  &  &  &  &  &  \\
\hline
 &  &  &  &  &  &  &  &  &  &  &  &  &  &  &  &  &  &  &  &  &  &  \\
\hline
 &  &  &  &  &  &  &  &  &  &  &  &  &  &  &  &  &  &  &  &  &  &  \\
\hline
 &  &  &  &  &  &  &  &  &  &  &  &  &  &  &  &  &  &  &  &  &  &  \\
\hline
 &  &  &  &  &  &  &  &  &  &  &  &  &  &  &  &  &  &  &  &  &  &  \\
\hline
 &  &  &  &  &  &  &  &  &  &  &  &  &  &  &  &  &  &  &  &  &  &  \\
\hline
 &  &  &  &  &  &  &  &  &  &  &  &  &  &  &  &  &  &  &  &  &  &  \\
\hline
 &  &  &  &  &  &  &  &  &  &  &  &  &  &  &  &  &  &  &  &  &  &  \\
\hline
 &  &  &  &  &  &  &  &  &  &  &  &  &  &  &  &  &  &  &  &  &  &  \\
\hline
 &  &  &  &  &  &  &  &  &  &  &  &  &  &  &  &  &  &  &  &  &  &  \\
\hline
 &  &  &  &  &  &  &  &  &  &  &  &  &  &  &  &  &  &  &  &  &  &  \\
\hline
 &  &  &  &  &  &  &  &  &  &  &  &  &  &  &  &  &  &  &  &  &  &  \\
\hline
 &  &  &  &  &  &  &  &  &  &  &  &  &  &  &  &  &  &  &  &  &  &  \\
\hline
 &  &  &  &  &  &  &  &  &  &  &  &  &  &  &  &  &  &  &  &  &  &  \\
\hline
 &  &  &  &  &  &  &  &  &  &  &  &  &  &  &  &  &  &  &  &  &  &  \\
\hline
 &  &  &  &  &  &  &  &  &  &  &  &  &  &  &  &  &  &  &  &  &  &  \\
\hline
 &  &  &  &  &  &  &  &  &  &  &  &  &  &  &  &  &  &  &  &  &  &  \\
\hline
 &  &  &  &  &  &  &  &  &  &  &  &  &  &  &  &  &  &  &  &  &  &  \\
\hline
 &  &  &  &  &  &  &  &  &  &  &  &  &  &  &  &  &  &  &  &  &  &  \\
\hline
 &  &  &  &  &  &  &  &  &  &  &  &  &  &  &  &  &  &  &  &  &  &  \\
\hline
 &  &  &  &  &  &  &  &  &  &  &  &  &  &  &  &  &  &  &  &  &  &  \\
\hline
 &  &  &  &  &  &  &  &  &  &  &  &  &  &  &  &  &  &  &  &  &  &  \\
\hline
 &  &  &  &  &  &  &  &  &  &  &  &  &  &  &  &  &  &  &  &  &  &  \\
\hline
 &  &  &  &  &  &  &  &  &  &  &  &  &  &  &  &  &  &  &  &  &  &  \\
\hline
 &  &  &  &  &  &  &  &  &  &  &  &  &  &  &  &  &  &  &  &  &  &  \\
\hline
 &  &  &  &  &  &  &  &  &  &  &  &  &  &  &  &  &  &  &  &  &  &  \\
\hline
 &  &  &  &  &  &  &  &  &  &  &  &  &  &  &  &  &  &  &  &  &  &  \\
\hline
\end{tabular}
\end{center}

Zadanie 30. (0-2)\\
W ciągu geometrycznym przez \(S_{n}\) oznaczamy sumę \(n\) początkowych wyrazów tego ciągu, dla liczb naturalnych \(n \geq 1\). Wiadomo, że dla pewnego ciągu geometrycznego: \(S_{1}=2\) i \(S_{2}=12\). Wyznacz iloraz i piąty wyraz tego ciągu.\\
\includegraphics[max width=\textwidth, center]{2025_02_10_4569edf99f1a2b54835bg-18}

Odpowiedź: \(\qquad\)

\section*{Zadanie 31. (0-2)}
Doświadczenie losowe polega na trzykrotnym rzucie symetryczną sześcienną kostką do gry. Oblicz prawdopodobieństwo zdarzenia polegającego na tym, że otrzymamy sumę oczek równą 16.\\
\includegraphics[max width=\textwidth, center]{2025_02_10_4569edf99f1a2b54835bg-19}

Odpowiedź:

\section*{Zadanie 32. (0-5)}
Podstawą ostrosłupa \(A B C D S\) jest prostokąt o polu równym 432, a stosunek długości boków tego prostokąta jest równy \(3: 4\). Przekątne podstawy \(A B C D\) przecinają się w punkcie \(O\). Odcinek \(S O\) jest wysokością ostrosłupa (zobacz rysunek). Kąt \(S A O\) ma miarę \(60^{\circ}\). Oblicz objętość tego ostrosłupa.\\
\includegraphics[max width=\textwidth, center]{2025_02_10_4569edf99f1a2b54835bg-20}\\
\includegraphics[max width=\textwidth, center]{2025_02_10_4569edf99f1a2b54835bg-20(1)}\\
\includegraphics[max width=\textwidth, center]{2025_02_10_4569edf99f1a2b54835bg-21}

Odpowiedź

\section*{Zadanie 33. (0-4)}
Liczby rzeczywiste \(x\) i \(z\) spełniają warunek \(2 x+z=1\). Wyznacz takie wartości \(x \mathrm{i} z\), dla których wyrażenie \(x^{2}+z^{2}+7 x z\) przyjmuje największą wartość. Podaj tę największą wartość.\\
\includegraphics[max width=\textwidth, center]{2025_02_10_4569edf99f1a2b54835bg-22}\\
\includegraphics[max width=\textwidth, center]{2025_02_10_4569edf99f1a2b54835bg-23}

Odpowiedź:

\section*{Zadanie 34. (0-4)}
Dany jest trójkąt rozwartokątny \(A B C\), w którym \(\Varangle A C B\) ma miarę \(120^{\circ}\). Ponadto wiadomo, że \(|B C|=10\) i \(|A B|=10 \sqrt{7}\) (zobacz rysunek). Oblicz długość trzeciego boku trójkąta \(A B C\).\\
\includegraphics[max width=\textwidth, center]{2025_02_10_4569edf99f1a2b54835bg-24}\\
\includegraphics[max width=\textwidth, center]{2025_02_10_4569edf99f1a2b54835bg-25}

Odpowiedź:

\section*{BRUDNOPIS (nie podlega ocenie)}

\end{document}