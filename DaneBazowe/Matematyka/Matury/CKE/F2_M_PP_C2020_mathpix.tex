\documentclass[10pt]{article}
\usepackage[polish]{babel}
\usepackage[utf8]{inputenc}
\usepackage[T1]{fontenc}
\usepackage{graphicx}
\usepackage[export]{adjustbox}
\graphicspath{ {./images/} }
\usepackage{amsmath}
\usepackage{amsfonts}
\usepackage{amssymb}
\usepackage[version=4]{mhchem}
\usepackage{stmaryrd}

\begin{document}
\section*{WYPELNIA ZDAJĄCY}
\section*{KOD}
PESEL\\
\includegraphics[max width=\textwidth, center]{2025_02_10_357d9e78eca0e59d194ag-01}

\begin{center}
\begin{tabular}{|l|l|l|l|l|l|l|l|l|l|l|}
\hline
 &  &  &  &  &  &  &  &  &  &  \\
\hline
\end{tabular}
\end{center}

miejsce na naklejke

\section*{EGZAMIN MATURALNY}
 Z MATEMATYKILiczba punktów do uzyskania: 50

\begin{center}
\begin{tabular}{|c|}
\hline
WYPELNIA ZESPÓL \\
NADZORUJĄCY \\
\end{tabular}
\end{center}\(|\)\begin{tabular}{|ll|}
\hline
\(\square\) & \begin{tabular}{l}
dostosowania \\
kryteriów oceniania \\
nieprzenoszenia \\
zaznaczeń na kartę \\
dostosowania \\
w zw. z dyskalkulią \\
\end{tabular} \\
\hline
\end{tabular}

\section*{Instrukcja dla zdającego}
\begin{enumerate}
  \item Sprawdź, czy arkusz egzaminacyjny zawiera 24 strony (zadania 1-34). Ewentualny brak zgłoś przewodniczącemu zespołu nadzorującego egzamin.
  \item Rozwiązania zadań i odpowiedzi wpisuj w miejscu na to przeznaczonym.
  \item Odpowiedzi do zadań zamkniętych (1-25) zaznacz na karcie odpowiedzi, w części karty przeznaczonej dla zdającego. Zamaluj \(\quad\) pola do tego przeznaczone. Błędne zaznaczenie otocz kółkiem \({ }^{\text {i zaznacz właściwe }}\).
  \item Pamiętaj, że pominięcie argumentacji lub istotnych obliczeń w rozwiązaniu zadania otwartego (26-34) może spowodować, że za to rozwiązanie nie otrzymasz pełnej liczby punktów.
  \item Pisz czytelnie i używaj tylko długopisu lub pióra z czarnym tuszem lub atramentem.
  \item Nie używaj korektora, a błędne zapisy wyraźnie przekreśl.
  \item Pamiętaj, że zapisy w brudnopisie nie będą oceniane.
  \item Możesz korzystać z zestawu wzorów matematycznych, cyrkla i linijki, a także z kalkulatora prostego.
  \item Na tej stronie oraz na karcie odpowiedzi wpisz swój numer PESEL i przyklej naklejkę z kodem.
  \item Nie wpisuj żadnych znaków w części przeznaczonej dla egzaminatora.
\end{enumerate}

W każdym z zadań od 1. do 25. wybierz i zaznacz na karcie odpowiedzi poprawna odpowiedź.

\section*{Zadanie 1. (0-1)}
Równość \(2+a=\frac{9 a}{2 a+1}\) jest prawdziwa, gdy\\
A. \(a=-2\)\\
B. \(a=-1\)\\
C. \(a=1\)\\
D. \(a=2\)

\section*{Zadanie 2. (0-1)}
Liczba \(1-\left(2^{7}-1\right)^{2}\) jest równa\\
A. \(-2^{14}\)\\
B. \(2^{8}-2^{14}\)\\
C. \(2-2^{14}\)\\
D. \(-2^{14}-2 \cdot 2^{7}+2\)

\section*{Zadanie 3. (0-1)}
Liczba \(\log _{\sqrt{2}} 4^{8}\) jest równa\\
A. 2\\
B. 4\\
C. 32\\
D. 16

\section*{Zadanie 4. (0-1)}
Masę Słońca równą \(1,989 \cdot 10^{30} \mathrm{~kg}\) przybliżono do \(2 \cdot 10^{30} \mathrm{~kg}\). Błąd bezwzględny tego przybliżenia jest równy\\
A. \(0,0011 \cdot 10^{30} \mathrm{~kg}\)\\
B. \(1,1 \cdot 10^{30} \mathrm{~kg}\)\\
C. \(0,11 \cdot 10^{30} \mathrm{~kg}\)\\
D. \(0,011 \cdot 10^{30} \mathrm{~kg}\)

\section*{Zadanie 5. (0-1)}
Największą liczbą całkowitą spełniającą nierówność \(\frac{1}{6}-x \geq \frac{2}{3} x+4\) jest\\
A. -3\\
B. -2\\
C. 2\\
D. 3

\section*{Zadanie 6. (0-1)}
Równanie \(\frac{1-x}{x}=2 x\) w zbiorze liczb całkowitych\\
A. nie ma żadnego rozwiązania.\\
B. ma dokładnie jedno rozwiązanie.\\
C. ma dokładnie dwa rozwiązania.\\
D. ma więcej niż dwa rozwiązania.

\section*{BRUDNOPIS (nie podlega ocenie)}
\(\square\)

\section*{Zadanie 7. (0-1)}
Boki trójkąta \(A B C\) są zawarte w prostych o równaniach \(y=\frac{2}{3} x+2\) i \(y=-x+2\) oraz osi \(O x\) układu współrzędnych (zobacz rysunek).\\
\includegraphics[max width=\textwidth, center]{2025_02_10_357d9e78eca0e59d194ag-04}

Pole trójkąta \(A B C\) jest równe\\
A. 10\\
B. \(\frac{5}{2}\)\\
C. 5\\
D. \(\frac{3}{2}\)

\section*{Zadanie 8. (0-1)}
Punkt \(P=(-3,7)\) leży na wykresie funkcji liniowej \(f\) określonej wzorem \(f(x)=(2 m-1) x+5\).\\
Zatem\\
A. \(m=\frac{1}{6}\)\\
B. \(m=-\frac{1}{6}\)\\
C. \(m=\frac{5}{6}\)\\
D. \(m=-\frac{5}{6}\)

\section*{Zadanie 9. (0-1)}
Wykresem funkcji kwadratowej \(f\) określonej wzorem \(f(x)=-x^{2}+6 x+4\) jest parabola o wierzchołku w punkcie \((3, q)\). Liczba \(q\) jest równa\\
A. 4\\
B. 7\\
C. 9\\
D. 13

\section*{Zadanie 10. (0-1)}
Funkcja \(f\) każdej liczbie naturalnej \(n \geq 1\) przyporządkowuje resztę z dzielenia tej liczby przez 4. Zbiorem wartości funkcji \(f\) jest\\
A. \(\{0,1,2,3\}\)\\
B. \(\{1,2,3,4\}\)\\
C. \(\{0,1,2,3,4\}\)\\
D. \(\{0,2\}\)

\section*{BRUDNOPIS (nie podlega ocenie)}
\section*{Zadanie 11. (0-1)}
Na rysunku poniżej przedstawiono fragment wykresu funkcji kwadratowej \(f\) określonej wzorem \(f(x)=a x^{2}+b x+c\).\\
\includegraphics[max width=\textwidth, center]{2025_02_10_357d9e78eca0e59d194ag-06}

Stąd wynika, że\\
A. \(\left\{\begin{array}{l}a<0 \\ b<0\end{array}\right.\)\\
B. \(\left\{\begin{array}{l}a<0 \\ b>0\end{array}\right.\)\\
C. \(\left\{\begin{array}{l}a>0 \\ b<0\end{array}\right.\)\\
D. \(\left\{\begin{array}{l}a>0 \\ b>0\end{array}\right.\)

\section*{Zadanie 12. (0-1)}
Proste o równaniach \(y=(m-2) x\) oraz \(y=\frac{3}{4} x+7\) są prostopadłe. Wtedy\\
A. \(m=-\frac{5}{4}\)\\
B. \(m=\frac{2}{3}\)\\
C. \(m=\frac{11}{4}\)\\
D. \(m=\frac{10}{3}\)

\section*{Zadanie 13. (0-1)}
Ciąg arytmetyczny \(\left(a_{n}\right)\) jest określony dla każdej liczby naturalnej \(n \geq 1\). Czwarty wyraz tego ciągu jest równy \(a_{4}=2020\). Suma \(a_{2}+a_{6}\) jest równa\\
A. 505\\
B. 1010\\
C. 2020\\
D. 4040

\section*{Zadanie 14. (0-1)}
Ciąg geometryczny \(\left(a_{n}\right)\) jest określony dla każdej liczby naturalnej \(n \geq 1\) oraz \(a_{2}=6\) i \(a_{5}=-48\). Wynika stąd, że\\
A. \(a_{7}>0\)\\
B. \(a_{7}<0\)\\
C. \(a_{7}>a_{6}\)\\
D. \(a_{7}>a_{8}\)

\section*{BRUDNOPIS (nie podlega ocenie)}
\section*{Zadanie 15. (0-1)}
Punkty \(A=(80,-1)\) i \(B=(-6,-19)\) są wierzchołkami trójkąta prostokątnego \(A B C\). W tym trójkącie kąt przy wierzchołku \(C\) jest prosty. Środkiem okręgu opisanego na tym trójkącie jest punkt o wspórrzędnych\\
A. \((43,-10)\)\\
B. \((37,10)\)\\
C. \((43,10)\)\\
D. \((37,-10)\)

\section*{Zadanie 16. (0-1)}
W trapezie prostokątnym \(A B C D\) są dane długości boków: \(|A B|=8,|B C|=5,|D C|=5,|A D|=4\) (zobacz rysunek).\\
\includegraphics[max width=\textwidth, center]{2025_02_10_357d9e78eca0e59d194ag-08}

Tangens kąta ostrego \(A B C\) w tym trapezie jest równy\\
A. \(\frac{4}{3}\)\\
B. \(\frac{4}{5}\)\\
C. \(\frac{3}{4}\)\\
D. \(\frac{3}{5}\)

\section*{Zadanie 17. (0-1)}
Punkty \(A=(1,-2)\) i \(C=(0,5)\) są końcami przekątnej kwadratu \(A B C D\). Obwód tego kwadratu jest równy\\
A. 12\\
B. 20\\
C. 28\\
D. 48

\section*{Zadanie 18. (0-1)}
Pole trójkąta równoramiennego jest równe \(25 \sqrt{2}\). Miara kąta między ramionami tego trójkąta jest równa \(45^{\circ}\). Każde z ramion tego trójkąta ma długość\\
A. \(10 \sqrt{2}\)\\
B. \(5 \sqrt{2}\)\\
C. 5\\
D. 10

\section*{Zadanie 19. (0-1)}
Dany jest trójkąt prostokątny \(A B C\), w którym przyprostokątna \(B C\) ma długość 250 cm , a przyprostokątna \(A C\) ma długość 91 cm . Miara \(\beta\) kąta \(A B C\) spełnia warunek\\
A. \(19^{\circ}<\beta<21^{\circ}\)\\
B. \(21^{\circ}<\beta<23^{\circ}\)\\
C. \(67^{\circ}<\beta<69^{\circ}\)\\
D. \(69^{\circ}<\beta<71^{\circ}\)

\section*{Zadanie 20. (0-1)}
Tworząca stożka jest o 2 dłuższa od promienia jego podstawy, a pole powierzchni bocznej jest o \(2 \pi\) większe od pola podstawy. Promień podstawy tego stożka jest równy\\
A. 3\\
B. \(2 \pi\)\\
C. 1\\
D. \(\pi\)

\section*{Zadanie 21. (0-1)}
Objętość ostrosłupa prawidłowego czworokątnego, w którym wysokość jest dwa razy dłuższa od krawędzi podstawy, jest równa 144. Długość krawędzi podstawy tego ostrosłupa jest równa\\
A. 18\\
B. 36\\
C. 3\\
D. 6

\section*{Zadanie 22. (0-1)}
Podstawą graniastosłupa prawidłowego jest kwadrat o boku 2. Przekątna graniastosłupa tworzy z jego podstawą kąt o mierze \(60^{\circ}\) (zobacz rysunek).\\
\includegraphics[max width=\textwidth, center]{2025_02_10_357d9e78eca0e59d194ag-10}

Wysokość tego graniastosłupa jest równa\\
A. \(\frac{\sqrt{2}}{2}\)\\
B. \(4 \sqrt{2}\)\\
C. \(\frac{\sqrt{6}}{4}\)\\
D. \(2 \sqrt{6}\)

\section*{Zadanie 23. (0-1)}
Wszystkich czterocyfrowych liczb naturalnych, w których cyfra tysięcy i cyfra setek są większe od 4 , a każda z pozostałych cyfr jest mniejsza od 6, jest\\
A. \(4 \cdot 4 \cdot 5 \cdot 5\)\\
B. \(5 \cdot 4 \cdot 6 \cdot 5\)\\
C. \(5 \cdot 5 \cdot 6 \cdot 6\)\\
D. \(4 \cdot 3 \cdot 5 \cdot 4\)

\section*{Zadanie 24. (0-1)}
Wariancją zestawu czterech ocen z matematyki: 1, 3, 5, 3, jest liczba\\
A. 1\\
B. 2\\
C. 3\\
D. 5

\section*{Zadanie 25. (0-1)}
W urnie jest 9 kul, w tym cztery kule czerwone, trzy zielone i dwie kule białe. Losujemy jedną kulę. Prawdopodobieństwo, że nie wylosowano ani kuli zielonej, ani białej, jest równe\\
A. \(\frac{4}{5}\)\\
B. \(\frac{4}{9}\)\\
C. \(\frac{5}{9}\)\\
D. \(\frac{6}{9}\)

\section*{BRUDNOPIS (nie podlega ocenie)}
\(\qquad\)

Zadanie 26. (0-2)\\
Rozwiąż nierówność \((2 x+5)(3 x-1) \geq 0\).

\begin{center}
\begin{tabular}{|c|c|c|c|c|c|c|c|c|c|c|c|c|c|c|c|c|c|c|c|c|c|c|}
\hline
 &  &  &  &  &  &  &  &  &  &  &  &  &  &  &  &  &  &  &  &  &  &  \\
\hline
 &  &  &  &  &  &  &  &  &  &  &  &  &  &  &  &  &  &  &  &  &  &  \\
\hline
 &  &  &  &  &  &  &  &  &  &  &  &  &  &  &  &  &  &  &  &  &  &  \\
\hline
 &  &  &  &  &  &  &  &  &  &  &  &  &  &  &  &  &  &  &  &  &  &  \\
\hline
 &  &  &  &  &  &  &  &  &  &  &  &  &  &  &  &  &  &  &  &  &  &  \\
\hline
 &  &  &  &  &  &  &  &  &  &  &  &  &  &  &  &  &  &  &  &  &  &  \\
\hline
 &  &  &  &  &  &  &  &  &  &  &  &  &  &  &  &  &  &  &  &  &  &  \\
\hline
 &  &  &  &  &  &  &  &  &  &  &  &  &  &  &  &  &  &  &  &  &  &  \\
\hline
 &  &  &  &  &  &  &  &  &  &  &  &  &  &  &  &  &  &  &  &  &  &  \\
\hline
 &  &  &  &  &  &  &  &  &  &  &  &  &  &  &  &  &  &  &  &  &  &  \\
\hline
 &  &  &  &  &  &  &  &  &  &  &  &  &  &  &  &  &  &  &  &  &  &  \\
\hline
 &  &  &  &  &  &  &  &  &  &  &  &  &  &  &  &  &  &  &  &  &  &  \\
\hline
 &  &  &  &  &  &  &  &  &  &  &  &  &  &  &  &  &  &  &  &  &  &  \\
\hline
 &  &  &  &  &  &  &  &  &  &  &  &  &  &  &  &  &  &  &  &  &  &  \\
\hline
 &  &  &  &  &  &  &  &  &  &  &  &  &  &  &  &  &  &  &  &  &  &  \\
\hline
 &  &  &  &  &  &  &  &  &  &  &  &  &  &  &  &  &  &  &  &  &  &  \\
\hline
 &  &  &  &  &  &  &  &  &  &  &  &  &  &  &  &  &  &  &  &  &  &  \\
\hline
 &  &  &  &  &  &  &  &  &  &  &  &  &  &  &  &  &  &  &  &  &  &  \\
\hline
 &  &  &  &  &  &  &  &  &  &  &  &  &  &  &  &  &  &  &  &  &  &  \\
\hline
 &  &  &  &  &  &  &  &  &  &  &  &  &  &  &  &  &  &  &  &  &  &  \\
\hline
 &  &  &  &  &  &  &  &  &  &  &  &  &  &  &  &  &  &  &  &  &  &  \\
\hline
 &  &  &  &  &  &  &  &  &  &  &  &  &  &  &  &  &  &  &  &  &  &  \\
\hline
 &  &  &  &  &  &  &  &  &  &  &  &  &  &  &  &  &  &  &  &  &  &  \\
\hline
 &  &  &  &  &  &  &  &  &  &  &  &  &  &  &  &  &  &  &  &  &  &  \\
\hline
 &  &  &  &  &  &  &  &  &  &  &  &  &  &  &  &  &  &  &  &  &  &  \\
\hline
 &  &  &  &  &  &  &  &  &  &  &  &  &  &  &  &  &  &  &  &  &  &  \\
\hline
 &  &  &  &  &  &  &  &  &  &  &  &  &  &  &  &  &  &  &  &  &  &  \\
\hline
 &  &  &  &  &  &  &  &  &  &  &  &  &  &  &  &  &  &  &  &  &  &  \\
\hline
 &  &  &  &  &  &  &  &  &  &  &  &  &  &  &  &  &  &  &  &  &  &  \\
\hline
 &  &  &  &  &  &  &  &  &  &  &  &  &  &  &  &  &  &  &  &  &  &  \\
\hline
 &  &  &  &  &  &  &  &  &  &  &  &  &  &  &  &  &  &  &  &  &  &  \\
\hline
 &  &  &  &  &  &  &  &  &  &  &  &  &  &  &  &  &  &  &  &  &  &  \\
\hline
 &  &  &  &  &  &  &  &  &  &  &  &  &  &  &  &  &  &  &  &  &  &  \\
\hline
 &  &  &  &  &  &  &  &  &  &  &  &  &  &  &  &  &  &  &  &  &  &  \\
\hline
 &  &  &  &  &  &  &  &  &  &  &  &  &  &  &  &  &  &  &  &  &  &  \\
\hline
 &  &  &  &  &  &  &  &  &  &  &  &  &  &  &  &  &  &  &  &  &  &  \\
\hline
 &  &  &  &  &  &  &  &  &  &  &  &  &  &  &  &  &  &  &  &  &  &  \\
\hline
 &  &  &  &  &  &  &  &  &  &  &  &  &  &  &  &  &  &  &  &  &  &  \\
\hline
 &  &  &  &  &  &  &  &  &  &  &  &  &  &  &  &  &  &  &  &  &  &  \\
\hline
- &  &  &  &  &  &  &  &  &  &  &  &  &  &  &  &  &  &  &  &  &  &  \\
\hline
 &  &  &  &  &  &  &  &  &  &  &  &  &  &  &  &  &  &  &  &  &  &  \\
\hline
- &  &  &  &  &  &  &  &  &  &  &  &  & - &  &  &  &  &  &  &  &  &  \\
\hline
\(\square\) &  &  &  &  &  &  &  &  &  &  &  &  &  &  &  &  &  &  &  &  &  &  \\
\hline
\(\bigcirc\) &  &  &  &  &  &  &  &  &  &  &  &  &  &  &  &  &  &  &  &  &  &  \\
\hline
\(\square\) &  &  &  &  &  &  &  &  &  &  &  &  &  &  &  &  &  &  &  &  &  &  \\
\hline
- &  &  &  &  &  &  &  &  &  &  &  &  &  &  &  &  &  &  &  &  &  &  \\
\hline
\end{tabular}
\end{center}

Odpowiedź:

\section*{Zadanie 27. (0-2)}
Dane są liczby \(a=3 \log _{2} 12-\log _{2} 27\) i \(b=(\sqrt{6}-\sqrt{7})(3 \sqrt{6}+3 \sqrt{7})\). Wartością \(a-b\) jest liczba całkowita. Oblicz tę liczbee.\\
\includegraphics[max width=\textwidth, center]{2025_02_10_357d9e78eca0e59d194ag-13}

Odpowiedź:

\section*{Zadanie 28. (0-2)}
Wykaż, że jeśli liczby rzeczywiste \(a\) i \(b\) spełniają warunek \(a<4\) i \(b<4\), to \(a b+16>4 a+4 b\).\\
\includegraphics[max width=\textwidth, center]{2025_02_10_357d9e78eca0e59d194ag-14}

\section*{Zadanie 29. (0-2)}
Bok \(A B\) jest średnicą, a punkt \(S\) jest środkiem okręgu opisanego na trójkącie \(A B C\). Punkt \(D\) leży na tym okręgu, a odcinek \(S D\) zawarty jest w symetralnej boku \(B C\) trójkąta (zobacz rysunek).\\
\includegraphics[max width=\textwidth, center]{2025_02_10_357d9e78eca0e59d194ag-15}

Wykaż, że odcinek \(A D\) jest zawarty w dwusiecznej kąta \(C A B\).\\
\includegraphics[max width=\textwidth, center]{2025_02_10_357d9e78eca0e59d194ag-15(1)}

\section*{Zadanie 30. (0-2)}
Dany jest trzywyrazowy ciąg \((x+2,4 x+2, x+11)\). Oblicz te wszystkie wartości \(x\), dla których ten ciąg jest geometryczny.\\
\includegraphics[max width=\textwidth, center]{2025_02_10_357d9e78eca0e59d194ag-16}

Odpowiedź:

\section*{Zadanie 31. (0-2)}
Prosta \(k\) jest nachylona do osi Ox pod kątem ostrym \(\alpha\), takim, że \(\cos \alpha=\frac{\sqrt{3}}{3}\). Wyznacz współczynnik kierunkowy prostej \(k\).\\
\includegraphics[max width=\textwidth, center]{2025_02_10_357d9e78eca0e59d194ag-17}

Odpowiedź:

\section*{Zadanie 32. (0-4)}
Punkty \(A=(1,-1), B=(6,1), C=(7,5)\) i \(D=(2,4)\) są wierzchołkami czworokąta \(A B C D\). Oblicz współrzędne punktu przecięcia przekątnych tego czworokąta.\\
\includegraphics[max width=\textwidth, center]{2025_02_10_357d9e78eca0e59d194ag-18}\\
(T|

Odpowiedź:

\section*{Zadanie 33. (0-4)}
Rzucamy cztery razy symetryczną monetą. Oblicz prawdopodobieństwo zdarzenia \(A\), polegającego na tym, że liczba otrzymanych orłów będzie różna od liczby otrzymanych reszek.\\
\includegraphics[max width=\textwidth, center]{2025_02_10_357d9e78eca0e59d194ag-20}\\
\includegraphics[max width=\textwidth, center]{2025_02_10_357d9e78eca0e59d194ag-21}

Odpowiedź:

\section*{Zadanie 34. (0-5)}
W ostrosłupie prawidłowym sześciokątnym ABCDEFS, którego krawędź podstawy a ma długość 8 (zobacz rysunek), ściana boczna jest nachylona do płaszczyzny podstawy pod kątem \(\alpha=60^{\circ}\). Oblicz cosinus kąta między krawędzią boczną a płaszczyzną podstawy tego ostrosłupa.\\
\includegraphics[max width=\textwidth, center]{2025_02_10_357d9e78eca0e59d194ag-22}\\
\includegraphics[max width=\textwidth, center]{2025_02_10_357d9e78eca0e59d194ag-22(1)}\\
\includegraphics[max width=\textwidth, center]{2025_02_10_357d9e78eca0e59d194ag-23}

Odpowiedź:

\section*{BRUDNOPIS (nie podlega ocenie)}

\end{document}