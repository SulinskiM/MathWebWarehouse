\documentclass[10pt]{article}
\usepackage[polish]{babel}
\usepackage[utf8]{inputenc}
\usepackage[T1]{fontenc}
\usepackage{graphicx}
\usepackage[export]{adjustbox}
\graphicspath{ {./images/} }
\usepackage{amsmath}
\usepackage{amsfonts}
\usepackage{amssymb}
\usepackage[version=4]{mhchem}
\usepackage{stmaryrd}

\begin{document}
\section*{WYPEŁNIA ZDAJĄCY}
\section*{KOD}
\begin{center}
\includegraphics[max width=\textwidth]{2025_02_10_7383028d0e04e8f03b7cg-01(2)}
\end{center}

\section*{PESEL}
\begin{center}
\includegraphics[max width=\textwidth]{2025_02_10_7383028d0e04e8f03b7cg-01}
\end{center}

\section*{Miejsce na naklejkę.}
Sprawdż, czy kod na naklejce to E-100.\\
Jeżeli tak - przyklej naklejke. Jeżeli nie - zgłoś to nauczycielowi.

\section*{EGZAMIN MATURALNY Z MATEMATYKI Poziom Podstawowy}
\section*{DAtA: 2 czerwca 2021 r.}
Godzina rozpoczęcia: 9:00\\
CZAS PRACY: \(\mathbf{1 7 0}\) minut\\
LicZBA PUNKTÓW DO UZYSKANIA: 45

\section*{WYPEŁNIA ZESPÓŁ NADZORUJĄCY}
Uprawnienia zdającego do:\\
\includegraphics[max width=\textwidth, center]{2025_02_10_7383028d0e04e8f03b7cg-01(1)}

\section*{Instrukcja dla zdającego}
\begin{enumerate}
  \item Sprawdź, czy arkusz egzaminacyjny zawiera 27 stron (zadania \(1-35\) ). Ewentualny brak zgłoś przewodniczącemu zespołu nadzorującego egzamin.
  \item Na tej stronie oraz na karcie odpowiedzi wpisz swój numer PESEL i przyklej naklejkę z kodem.
  \item Nie wpisuj żadnych znaków w części przeznaczonej dla egzaminatora.
  \item Rozwiązania zadań i odpowiedzi wpisuj w miejscu na to przeznaczonym.
  \item Odpowiedzi do zadań zamkniętych (1-28) zaznacz na karcie odpowiedzi w części karty przeznaczonej dla zdającego. Zamaluj \(\square\) pola do tego przeznaczone. Błędne zaznaczenie otocz kółkiem i i zaznacz właściwe.
  \item Pamiętaj, że pominięcie argumentacji lub istotnych obliczeń w rozwiązaniu zadania otwartego (29-35) może spowodować, że za to rozwiązanie nie otrzymasz pełnej liczby punktów.
  \item Pisz czytelnie i używaj tylko długopisu lub pióra z czarnym tuszem lub atramentem.
  \item Nie używaj korektora, a błędne zapisy wyraźnie przekreśl.
  \item Pamiętaj, że zapisy w brudnopisie nie będą oceniane.
  \item Możesz korzystać z zestawu wzorów matematycznych, cyrkla i linijki oraz kalkulatora prostego.
\end{enumerate}

W każdym z zadań od 1. do 28. wybierz i zaznacz na karcie odpowiedzi poprawną odpowiedź.

\section*{Zadanie 1. (0-1)}
Wartość wyrażenia \(\sqrt{2} \cdot(\sqrt{2}-\sqrt{3})+\sqrt{3} \cdot(\sqrt{2}-\sqrt{3})\) jest równa\\
A. \(5-2 \sqrt{6}\)\\
B. 5\\
C. \(5+2 \sqrt{6}\)\\
D. -1

\section*{Zadanie 2. (0-1)}
Liczba \(\left(7^{\frac{5}{4}} \cdot 7^{\frac{1}{4}}\right)^{\frac{2}{3}}\) jest równa\\
A. \(7^{\frac{5}{3}}\)\\
B. \(7^{1}\)\\
C. \(7^{\frac{3}{2}}\)\\
D. \(7^{\frac{10}{3}}\)

\section*{Zadanie 3. (0-1)}
Niech \(\log _{3} 18=c\). Wtedy \(\log _{3} 54\) jest równy\\
A. \(c-1\)\\
B. \(c\)\\
C. \(c+1\)\\
D. \(c+2\)

\section*{Zadanie 4. (0-1)}
Cenę drukarki obniżono o \(20 \%\), a następnie nową cenę obniżono o \(10 \%\). W wyniku obu tych zmian cena drukarki zmniejszyła się w stosunku do ceny sprzed obu obniżek o\\
A. \(18 \%\)\\
B. \(28 \%\)\\
C. \(30 \%\)\\
D. \(72 \%\)

\section*{Zadanie 5. (0-1)}
Dla każdej liczby rzeczywistej \(x\) wyrażenie \((x-1)^{2}-(2-x)^{2}\) jest równe\\
A. \(2 x-3\)\\
B. \(2 x^{2}-6 x-3\)\\
C. \((2 x-3)^{2}\)\\
D. 9

BRUDNOPIS (nie podlega ocenie)\\
\includegraphics[max width=\textwidth, center]{2025_02_10_7383028d0e04e8f03b7cg-03}

\section*{Zadanie 6. (0-1)}
Wskaż rysunek, na którym przedstawiony jest zbiór wszystkich liczb rzeczywistych \(x\), spełniających jednocześnie nierówności \(0<7-3 x\) oraz \(7-3 x \leq 5 x-3\).\\
A.\\
\includegraphics[max width=\textwidth, center]{2025_02_10_7383028d0e04e8f03b7cg-04(2)}\\
B.\\
\includegraphics[max width=\textwidth, center]{2025_02_10_7383028d0e04e8f03b7cg-04(3)}\\
C.\\
\includegraphics[max width=\textwidth, center]{2025_02_10_7383028d0e04e8f03b7cg-04(1)}\\
D.\\
\includegraphics[max width=\textwidth, center]{2025_02_10_7383028d0e04e8f03b7cg-04}

\section*{Zadanie 7. (0-1)}
Rozwiązaniem równania \(x \sqrt{3}+2=2 x-8\) jest liczba\\
A. \(10(2+\sqrt{3})\)\\
B. \(\frac{10}{\sqrt{3}-2}\)\\
C. \(10(\sqrt{3}-2)\)\\
D. \(\frac{\sqrt{3}+10}{2}\)

\section*{Zadanie 8. (0-1)}
Równanie \(\frac{x^{2}-7 x}{x^{2}-49}=0\) ma \(w\) zbiorze liczb rzeczywistych dokładnie\\
A. jedno rozwiązanie.\\
B. dwa rozwiązania.\\
C. trzy rozwiązania.\\
D. cztery rozwiązania.

BRUDNOPIS (nie podlega ocenie)\\
\includegraphics[max width=\textwidth, center]{2025_02_10_7383028d0e04e8f03b7cg-05}

\section*{Zadanie 9. (0-1)}
Na rysunku przedstawiono wykres funkcji \(f\) określonej w zbiorze \((-1,7)\).\\
\includegraphics[max width=\textwidth, center]{2025_02_10_7383028d0e04e8f03b7cg-06}

Wskaż zdanie prawdziwe.\\
A. Funkcja \(f\) ma trzy miejsca zerowe.\\
B. Zbiorem wartości funkcji \(f\) jest \(\langle-1,1\) ).\\
C. Funkcja \(f\) osiąga wartość największą równą 1 .\\
D. Funkcja \(f\) osiąga wartości ujemne dla argumentów ze zbioru \((-1,0)\).

\section*{Zadanie 10. (0-1)}
Wykresem funkcji kwadratowej \(f\) określonej wzorem \(f(x)=-3(x+4)(x-2)\) jest parabola o wierzchołku \(W=(p, q)\). Współrzędne wierzchołka \(W\) spełniają warunki\\
A. \(p>0\) i \(q>0\)\\
B. \(p<0\) i \(q>0\)\\
C. \(p<0\) i \(q<0\)\\
D. \(p>0\) i \(q<0\)

BRUDNOPIS (nie podlega ocenie)\\
\includegraphics[max width=\textwidth, center]{2025_02_10_7383028d0e04e8f03b7cg-07}

\section*{Informacja do zadań 11. i 12.}
Na rysunku przedstawiono fragment wykresu funkcji kwadratowej \(f\). Jednym z miejsc zerowych tej funkcji jest liczba 2. Do wykresu funkcji \(f\) należy punkt ( 0,3 ). Prosta o równaniu \(x=-2\) jest osią symetrii paraboli, będącej wykresem funkcji \(f\).\\
\includegraphics[max width=\textwidth, center]{2025_02_10_7383028d0e04e8f03b7cg-08}

\section*{Zadanie 11. (0-1)}
Drugim miejscem zerowym funkcji \(f\) jest liczba\\
A. -2\\
B. -3\\
C. -4\\
D. -6

\section*{Zadanie 12. (0-1)}
Wartość funkcji \(f\) dla argumentu (-4) jest równa\\
A. -2\\
B. 0\\
C. 3\\
D. 4

\section*{Zadanie 13. (0-1)}
Dane są ciągi \(\left(a_{n}\right),\left(b_{n}\right),\left(c_{n}\right),\left(d_{n}\right)\), określone dla każdej liczby naturalnej \(n \geq 1\) wzorami: \(a_{n}=20 n+3, \quad b_{n}=2 n^{2}-3, \quad c_{n}=n^{2}+10 n-2, \quad d_{n}=\frac{n+187}{n}\). Liczba 197 jest dziesiątym wyrazem ciągu\\
A. \(\left(a_{n}\right)\)\\
B. \(\left(b_{n}\right)\)\\
C. \(\left(c_{n}\right)\)\\
D. \(\left(d_{n}\right)\)

BRUDNOPIS (nie podlega ocenie)\\
\includegraphics[max width=\textwidth, center]{2025_02_10_7383028d0e04e8f03b7cg-09}

\section*{Zadanie 14. (0-1)}
Ciąg geometryczny \(\left(a_{n}\right)\), określony dla każdej liczby naturalnej \(n \geq 1\), jest rosnący i wszystkie jego wyrazy są dodatnie. Ponadto spełniony jest warunek \(a_{3}=a_{1} \cdot a_{2}\). Niech \(q\) oznacza iloraz ciągu \(\left(a_{n}\right)\). Wtedy\\
A. \(a_{1}=\frac{1}{q}\)\\
B. \(a_{1}=q\)\\
C. \(a_{1}=q^{2}\)\\
D. \(a_{1}=q^{3}\)

\section*{Zadanie 15. (0-1)}
Kąt o mierze \(\alpha\) jest ostry i \(\operatorname{tg} \alpha=\sqrt{5}\). Wtedy\\
A. \(\cos ^{2} \alpha=\frac{1}{6}\)\\
B. \(\cos ^{2} \alpha=\frac{1}{5}\)\\
C. \(\cos ^{2} \alpha=\frac{\sqrt{5}}{5}\)\\
D. \(\cos ^{2} \alpha=\frac{5}{6}\)

\section*{Zadanie 16. (0-1)}
Na okręgu o środku w punkcie \(O\) leżą punkty \(A, B\) oraz \(C\). Odcinek \(A C\) jest średnicą tego okręgu, a kąt środkowy \(A O B\) ma miarę \(82^{\circ}\) (zobacz rysunek).\\
\includegraphics[max width=\textwidth, center]{2025_02_10_7383028d0e04e8f03b7cg-10}

Miara kąta \(O B C\) jest równa\\
A. \(41^{\circ}\)\\
B. \(45^{\circ}\)\\
C. \(49^{\circ}\)\\
D. \(51^{\circ}\)

BRUDNOPIS (nie podlega ocenie)\\
\includegraphics[max width=\textwidth, center]{2025_02_10_7383028d0e04e8f03b7cg-11}

\section*{Zadanie 17. (0-1)}
Dane są okrąg i prosta styczna do tego okręgu w punkcie \(A\). Punkty \(B\) i \(C\) są położone na okręgu tak, że \(B C\) jest jego średnicą. Cięciwa \(A B\) tworzy ze styczną kąt o mierze \(40^{\circ}\) (zobacz rysunek).\\
\includegraphics[max width=\textwidth, center]{2025_02_10_7383028d0e04e8f03b7cg-12}

Miara kąta \(A B C\) jest równa\\
A. \(20^{\circ}\)\\
B. \(40^{\circ}\)\\
C. \(45^{\circ}\)\\
D. \(50^{\circ}\)

\section*{Zadanie 18. (0-1)}
Dany jest trójkąt prostokątny \(A B C\) o bokach \(|A C|=24,|B C|=10,|A B|=26\). Dwusieczne kątów tego trójkąta przecinają się w punkcie \(P\) (zobacz rysunek).\\
\includegraphics[max width=\textwidth, center]{2025_02_10_7383028d0e04e8f03b7cg-12(1)}

Odległość \(x\) punktu \(P\) od przeciwprostokątnej \(A B\) jest równa\\
A. 2\\
B. 4\\
C. \(\frac{5}{2}\)\\
D. \(\frac{13}{3}\)

BRUDNOPIS (nie podlega ocenie)\\
\includegraphics[max width=\textwidth, center]{2025_02_10_7383028d0e04e8f03b7cg-13}

\section*{Zadanie 19. (0-1)}
Jeden z boków równoległoboku ma długość równą 5. Przekątne tego równoległoboku mogą mieć długości\\
A. 4 i 6\\
B. 4 i 3\\
C. 10 i 10\\
D. 5 i 5

\section*{Zadanie 20. (0-1)}
W pewnym trójkącie równoramiennym największy kąt ma miarę \(120^{\circ}\), a najdłuższy bok ma długość 12 (zobacz rysunek).\\
\includegraphics[max width=\textwidth, center]{2025_02_10_7383028d0e04e8f03b7cg-14}

12

Najkrótsza wysokość tego trójkąta ma długość równą\\
A. 6\\
B. \(2 \sqrt{3}\)\\
C. \(4 \sqrt{3}\)\\
D. \(6 \sqrt{3}\)

\section*{Zadanie 21. (0-1)}
Prosta przechodząca przez punkty \((-4,-1)\) oraz \((5,5)\) ma równanie\\
A. \(y=x+3\)\\
B. \(y=\frac{2}{3} x+\frac{5}{3}\)\\
C. \(y=x-3\)\\
D. \(y=\frac{2}{3} x+\frac{11}{3}\)

\section*{Zadanie 22. (0-1)}
Proste o równaniach \(y=-\frac{1}{m-2} x-1\) i \(y=\frac{1}{3} x+1\) są równoległe. Wynika stąd, że\\
A. \(m=\frac{5}{3}\)\\
B. \(m=-1\)\\
C. \(m=\frac{7}{3}\)\\
D. \(m=5\)

\section*{Zadanie 23. (0-1)}
W prostokącie \(A B C D\) dane są wierzchołki \(C=(-3,1)\) oraz \(D=(2,1)\). Bok \(A D\) ma długość 6. Pole tego prostokąta jest równe\\
A. \(6 \sqrt{29}\)\\
B. \(12 \sqrt{2}\)\\
C. 24\\
D. 30

BRUDNOPIS (nie podlega ocenie)\\
\includegraphics[max width=\textwidth, center]{2025_02_10_7383028d0e04e8f03b7cg-15}

\section*{Zadanie 24. (0-1)}
Obrazem prostej o równaniu \(x-2 y+3=0\) w symetrii osiowej względem osi \(O y\) jest prosta o równaniu\\
A. \(-x+2 y+3=0\)\\
B. \(-x+2 y-3=0\)\\
C. \(x+2 y-3=0\)\\
D. \(x+2 y+3=0\)

\section*{Zadanie 25. (0-1)}
Graniastosłup prawidłowy ma 36 krawędzi. Długość każdej z tych krawędzi jest równa 4. Pole powierzchni bocznej tego graniastosłupa jest równe\\
A. 176\\
B. 192\\
C. 224\\
D. 288

\section*{Zadanie 26. (0-1)}
Wysokość ściany bocznej ostrosłupa prawidłowego sześciokątnego jest 2 razy dłuższa od krawędzi jego podstawy. Stosunek pola powierzchni bocznej tego ostrosłupa do pola jego podstawy jest równy\\
A. \(\frac{1}{2}\)\\
B. \(\frac{4 \sqrt{3}}{3}\)\\
C. 1\\
D. \(\frac{\sqrt{3}}{4}\)

\section*{Zadanie 27. (0-1)}
W pudełku znajdują się płytki z literami. Na każdej płytce jest wydrukowana jedna litera spółgłoskowa albo samogłoskowa. Płytek z literami spółgłoskowymi jest o \(25 \%\) więcej niż płytek z literami samogłoskowymi. Losujemy jedną płytkę. Prawdopodobieństwo wylosowania płytki z literą samogłoskową jest równe\\
A. 0,75\\
B. 0,25\\
C. \(\frac{4}{9}\)\\
D. \(\frac{5}{9}\)

\section*{Zadanie 28. (0-1)}
Średnia arytmetyczna czterech liczb dodatnich: \(2,3 x, 3 x+2,3 x+4\) jest równa \(\frac{13}{2}\). Wynika stąd, że\\
A. \(x=9\)\\
B. \(x=\frac{13}{2}\)\\
C. \(x=\frac{5}{9}\)\\
D. \(x=2\)

BRUDNOPIS (nie podlega ocenie)\\
\includegraphics[max width=\textwidth, center]{2025_02_10_7383028d0e04e8f03b7cg-17}

Zadanie 29. (0-2)\\
Rozwiąż nierówność:

\[
2(x+1)(x-3)<x^{2}-9
\]

\begin{center}
\includegraphics[max width=\textwidth]{2025_02_10_7383028d0e04e8f03b7cg-18}
\end{center}

Odpowiedź:

Zadanie 30. (0-2)\\
Wykaż, że dla wszystkich liczb rzeczywistych \(a, b\) i \(c\) takich, że \(\frac{a+b}{2}>c\) i \(\frac{b+c}{2}>a\), prawdziwa jest nierówność

\[
\frac{a+c}{2}<b
\]

\begin{center}
\includegraphics[max width=\textwidth]{2025_02_10_7383028d0e04e8f03b7cg-19}
\end{center}

Zadanie 31. (0-2)\\
Dany jest ciąg arytmetyczny \(\left(a_{n}\right)\), określony dla wszystkich liczb naturalnych \(n \geq 1\). Suma dwudziestu początkowych wyrazów tego ciągu jest równa \(20 a_{21}+62\). Oblicz różnicę ciągu \(\left(a_{n}\right)\).\\
\includegraphics[max width=\textwidth, center]{2025_02_10_7383028d0e04e8f03b7cg-20}

Odpowiedź:

Zadanie 32. (0-2)\\
Dany jest trapez o podstawach długości \(a\) oraz \(b\) i wysokości \(h\). Każdą z podstaw tego trapezu wydłużono o \(25 \%\), a wysokość skrócono tak, że powstał nowy trapez o takim samym polu. Oblicz, o ile procent skrócono wysokość \(h\) trapezu.\\
\includegraphics[max width=\textwidth, center]{2025_02_10_7383028d0e04e8f03b7cg-21}

Odpowiedź:

Zadanie 33. (0-2)\\
W trójkącie \(A B C\) boki \(B C\) i \(A C\) są równej długości. Prosta \(k\) jest prostopadła do podstawy \(A B\) tego trójkąta i przecina boki \(A B\) oraz \(B C\) w punktach - odpowiednio - \(D\) i \(E\). Pole czworokąta \(A D E C\) jest 17 razy większe od pola trójkąta \(B E D\). Oblicz \(\frac{|C E|}{|E B|}\).\\
\includegraphics[max width=\textwidth, center]{2025_02_10_7383028d0e04e8f03b7cg-22}

Odpowiedź:

Zadanie 34. (0-2)\\
Ze zbioru wszystkich liczb naturalnych dwucyfrowych, których cyfra dziesiątek należy do zbioru \(\{3,4,5,6,7,8\}\), a cyfra jedności należy do zbioru \(\{0,1,2,3,4\}\), losujemy jedną liczbę. Oblicz prawdopodobieństwo zdarzenia polegającego na tym, że wylosujemy liczbę dwucyfrową, która jest podzielna przez 4.\\
\includegraphics[max width=\textwidth, center]{2025_02_10_7383028d0e04e8f03b7cg-23}

Odpowiedż:

Zadanie 35. (0-5)\\
Podstawa \(A B\) trójkąta równoramiennego \(A B C\) jest zawarta w prostej o równaniu \(y=-2 x+16\). Wierzchołki \(B\) i \(C\) mają współrzędne \(B=(3,10)\) i \(C=(-2,3)\). Oblicz współrzędne wierzchołka \(A\) i pole trójkąta \(A B C\).\\
\includegraphics[max width=\textwidth, center]{2025_02_10_7383028d0e04e8f03b7cg-24}\\
\includegraphics[max width=\textwidth, center]{2025_02_10_7383028d0e04e8f03b7cg-25}

Odpowiedź:

BRUDNOPIS (nie podlega ocenie)\\
\includegraphics[max width=\textwidth, center]{2025_02_10_7383028d0e04e8f03b7cg-26}\\
\includegraphics[max width=\textwidth, center]{2025_02_10_7383028d0e04e8f03b7cg-27}


\end{document}