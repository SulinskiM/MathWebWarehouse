\documentclass[10pt]{article}
\usepackage[polish]{babel}
\usepackage[utf8]{inputenc}
\usepackage[T1]{fontenc}
\usepackage{graphicx}
\usepackage[export]{adjustbox}
\graphicspath{ {./images/} }
\usepackage{amsmath}
\usepackage{amsfonts}
\usepackage{amssymb}
\usepackage[version=4]{mhchem}
\usepackage{stmaryrd}

\begin{document}
\section*{WYPEŁNIA ZDAJĄCY}
\section*{KOD}
PESEL\\
\includegraphics[max width=\textwidth, center]{2025_02_10_dbaef3a5b1e9442b6008g-01(1)}

\section*{Miejsce na naklejke.}
Sprawdż, czy kod na naklejce to E-100.\\
Jeżeli tak - przyklej naklejkę. Jeżeli nie - zgłoś to nauczycielowi.

\section*{EGZAMIN MATURALNY Z MATEMATYKI Poziom podstawowy}
\section*{Data: 2 czerwca 2022 r.}
Godzina rozpoczecia: 9:00\\
Czas pracy: \(\mathbf{1 7 0}\) minut\\
LICZBA PUNKTÓW DO UZYSKANIA: 45

\section*{WYPEŁNIA ZESPÓŁ NADZORUJACY}
Uprawnienia zdającego do:\\
nieprzenoszenia zaznaczeń na kartę dostosowania zasad oceniania dostosowania w zw. z dyskalkulią.\\
\includegraphics[max width=\textwidth, center]{2025_02_10_dbaef3a5b1e9442b6008g-01}

\section*{Instrukcja dla zdającego}
\begin{enumerate}
  \item Sprawdź, czy arkusz egzaminacyjny zawiera 26 stron (zadania 1-35). Ewentualny brak zgłoś przewodniczącemu zespołu nadzorującego egzamin.
  \item Na tej stronie oraz na karcie odpowiedzi wpisz swój numer PESEL i przyklej naklejkę z kodem.
  \item Nie wpisuj żadnych znaków w części przeznaczonej dla egzaminatora.
  \item Rozwiązania zadań i odpowiedzi wpisuj w miejscu na to przeznaczonym.
  \item Odpowiedzi do zadań zamkniętych (1-28) zaznacz na karcie odpowiedzi w części karty przeznaczonej dla zdającego. Zamaluj \(\square\) pola do tego przeznaczone. Błędne zaznaczenie otocz kółkiem © i zaznacz właściwe.
  \item Pamiętaj, że pominięcie argumentacji lub istotnych obliczeń w rozwiązaniu zadania otwartego (29-35) może spowodować, że za to rozwiązanie nie otrzymasz pełnej liczby punktów.
  \item Pisz czytelnie i używaj tylko długopisu lub pióra z czarnym tuszem lub atramentem.
  \item Nie używaj korektora, a błędne zapisy wyraźnie przekreśl.
  \item Pamiętaj, że zapisy w brudnopisie nie będą oceniane.
  \item Możesz korzystać z zestawu wzorów matematycznych, cyrkla i linijki oraz kalkulatora prostego.
\end{enumerate}

W każdym z zadań od 1. do 28. wybierz i zaznacz na karcie odpowiedzi poprawną odpowiedź.

\section*{Zadanie 1. (0-1)}
Liczba \(\sqrt{128}: \sqrt[3]{64}\) jest równa\\
A. \(\frac{1}{2} \sqrt{2}\)\\
B. 2\\
C. \(\sqrt{2}\)\\
D. \(2 \sqrt{2}\)

\section*{Zadanie 2. (0-1)}
Liczba \(\frac{2^{-3} \cdot 3^{-3} \cdot 4^{0}}{2^{-1} \cdot 3^{-4} \cdot 4^{-1}}\) jest równa\\
A. 1\\
B. 3\\
C. 24\\
D. 48

\section*{Zadanie 3. (0-1)}
Liczba dwukrotnie większa od \(\log 3+\log 2\) jest równa\\
A. \(\log 12\)\\
B. \(\log 36\)\\
C. \(\log 10\)\\
D. \(\log 25\)

\section*{Zadanie 4. (0-1)}
30\% liczby \(x\) jest o 2730 mniejsze od liczby \(x\). Liczba \(x\) jest równa\\
A. 3900\\
B. 1911\\
C. 9100\\
D. 2100

\section*{Zadanie 5. (0-1)}
Dla każdej liczby rzeczywistej \(a\) wyrażenie \(5-(4+2 a)(4-2 a)\) jest równe\\
A. \(-4 a^{2}-16 a-11\)\\
B. \(4 a^{2}-11\)\\
C. \(-4 a^{2}-11\)\\
D. \(4 a^{2}+16 a-11\)

BRUDNOPIS (nie podlega ocenie)\\
\includegraphics[max width=\textwidth, center]{2025_02_10_dbaef3a5b1e9442b6008g-03}

\section*{Zadanie 6. (0-1)}
Jedną z liczb spełniających nierówność \(x^{4}-3 x^{3}+3<0\) jest\\
A. 1\\
B. \((-1)\)\\
C. 2\\
D. \((-2)\)

\section*{Informacja do zadań 7. i 8.}
Na rysunku przedstawiono fragment wykresu funkcji kwadratowej \(f\) określonej wzorem \(f(x)=2 x^{2}+5 x\).\\
\includegraphics[max width=\textwidth, center]{2025_02_10_dbaef3a5b1e9442b6008g-04}

\section*{Zadanie 7. (0-1)}
Osią symetrii wykresu funkcji \(f\) jest prosta o równaniu\\
A. \(x=-\frac{5}{4}\)\\
B. \(x=\frac{5}{4}\)\\
C. \(y=-\frac{5}{4}\)\\
D. \(y=-\frac{25}{16}\)

\section*{Zadanie 8. (0-1)}
Funkcja kwadratowa \(g\) jest określona wzorem \(g(x)=2 x^{2}-5 x\). Wykres funkcji \(g\) jest\\
A. symetryczny do wykresu funkcji \(f\) względem osi \(O x\).\\
B. symetryczny do wykresu funkcji \(f\) względem osi \(O y\).\\
C. symetryczny do wykresu funkcji \(f\) względem początku układu współrzędnych.\\
D. przesunięty względem wykresu funkcji \(f\) o 10 jednostek w kierunku przeciwnym do zwrotu osi \(O x\).

BRUDNOPIS (nie podlega ocenie)\\
\includegraphics[max width=\textwidth, center]{2025_02_10_dbaef3a5b1e9442b6008g-05}

\section*{Zadanie 9. (0-1)}
Równanie \(\left(x^{2}-27\right)\left(x^{2}+16\right)=0\) ma dokładnie\\
A. jedno rozwiązanie rzeczywiste.\\
B. dwa rozwiązania rzeczywiste.\\
C. trzy rozwiązania rzeczywiste.\\
D. cztery rozwiązania rzeczywiste.

\section*{Zadanie 10. (0-1)}
Funkcja \(f\) jest określona wzorem \(f(x)=\frac{4}{x}-4\) dla każdej liczby rzeczywistej \(x \neq 0\).\\
Liczba \(f(2)-f(-2)\) jest równa\\
A. \((-8)\)\\
B. \((-4)\)\\
C. 4\\
D. 0

\section*{Zadanie 11. (0-1)}
Punkt \(M=(3,-2)\) należy do wykresu funkcji liniowej \(f\) określonej wzorem \(f(x)=5 x+b-4\). Wynika stąd, że \(b\) jest równe\\
A. \((-17)\)\\
B. \((-13)\)\\
C. 13\\
D. 17

\section*{Zadanie 12. (0-1)}
Funkcja kwadratowa \(f\) określona wzorem \(f(x)=-2(x-1)^{2}+3\) jest rosnąca w przedziale\\
A. \((-\infty, 1)\)\\
B. \((-2,+\infty)\)\\
C. \((-\infty, 3\rangle\)\\
D. \((1,+\infty)\)

BRUDNOPIS (nie podlega ocenie)\\
\includegraphics[max width=\textwidth, center]{2025_02_10_dbaef3a5b1e9442b6008g-07}

\section*{Zadanie 13. (0-1)}
Na rysunku jest przedstawiony fragment wykresu funkcji \(y=f(x)\).\\
\includegraphics[max width=\textwidth, center]{2025_02_10_dbaef3a5b1e9442b6008g-08}

W przedziale \((-4,6)\) równanie \(f(x)=-1\)\\
A. nie ma rozwiązań.\\
B. ma dokładnie jedno rozwiązanie.\\
C. ma dokładnie dwa rozwiązania.\\
D. ma dokładnie trzy rozwiązania.

\section*{Zadanie 14. (0-1)}
Ciąg \(\left(a_{n}\right)\) jest określony wzorem \(a_{n}=\frac{n-2}{2 n^{2}}\) dla każdej liczby naturalnej \(n \geq 1\). Piąty wyraz tego ciągu jest równy\\
A. \(\left(-\frac{1}{10}\right)\)\\
B. \(\frac{3}{50}\)\\
C. \(\frac{3}{100}\)\\
D. \(\left(-\frac{1}{5}\right)\)

\section*{Zadanie 15. (0-1)}
Ciąg \(\left(a_{n}\right)\), określony dla każdej liczby naturalnej \(n \geq 1\), jest arytmetyczny. Różnica tego ciągu jest równa 2 oraz \(a_{8}=48\). Czwarty wyraz tego ciągu jest równy\\
A. 2\\
B. 24\\
C. 3\\
D. 40

BRUDNOPIS (nie podlega ocenie)\\
\includegraphics[max width=\textwidth, center]{2025_02_10_dbaef3a5b1e9442b6008g-09}

\section*{Zadanie 16. (0-1)}
Kąt \(\alpha\) jest ostry i \(\sin \alpha=\frac{2}{3}\). Wtedy \(\cos ^{2}\left(90^{\circ}-\alpha\right)\) jest równy\\
A. \(\frac{1}{9}\)\\
B. \(\frac{2}{9}\)\\
C. \(\frac{4}{9}\)\\
D. \(\frac{5}{9}\)

\section*{Zadanie 17. (0-1)}
Na trójkącie ostrokątnym \(A B C\) opisano okrąg o środku \(O\). Miara kąta \(A B C\) jest równa \(65^{\circ}\). Miara kąta \(A C O\) jest równa\\
A. \(130^{\circ}\)\\
B. \(25^{\circ}\)\\
C. \(65^{\circ}\)\\
D. \(50^{\circ}\)

\section*{Zadanie 18. (0-1)}
Trójkąt \(A B C\) jest prostokątny. Odcinek \(A D\) jest wysokością tego trójkąta poprowadzoną z wierzchołka \(A\) na przeciwprostokątną \(B C\). Wtedy\\
A. \(\frac{|A D|}{|A B|}=\frac{|C D|}{|A C|}\)\\
B. \(\frac{|A D|}{|A B|}=\frac{|C D|}{|A D|}\)\\
c. \(\frac{|A D|}{|A B|}=\frac{|A C|}{|A B|}\)\\
D. \(\frac{|A D|}{|A B|}=\frac{|B C|}{|B D|}\)

\section*{Zadanie 19. (0-1)}
Pole rombu o obwodzie 20 i kącie rozwartym \(120^{\circ}\) jest równe\\
A. \(\frac{25 \sqrt{3}}{2}\)\\
B. \(\frac{5 \sqrt{3}}{2}\)\\
C. \(\frac{25}{2}\)\\
D. \(\frac{25 \sqrt{3}}{4}\)

\section*{Zadanie 20. (0-1)}
W trójkącie miary kątów są równe: \(\alpha, 4 \alpha, \alpha+30^{\circ}\). Miara największego kąta tego trójkąta jest równa\\
A. \(55^{\circ}\)\\
B. \(90^{\circ}\)\\
C. \(100^{\circ}\)\\
D. \(120^{\circ}\)

BRUDNOPIS (nie podlega ocenie)\\
\includegraphics[max width=\textwidth, center]{2025_02_10_dbaef3a5b1e9442b6008g-11}

\section*{Zadanie 21. (0-1)}
Na boku \(B C\) kwadratu \(A B C D\) (na zewnątrz) zbudowano trójkąt równoboczny \(B E C\) (zobacz rysunek).\\
\includegraphics[max width=\textwidth, center]{2025_02_10_dbaef3a5b1e9442b6008g-12}

Miara kąta \(D E C\) jest równa\\
A. \(10^{\circ}\)\\
B. \(20^{\circ}\)\\
C. \(15^{\circ}\)\\
D. \(30^{\circ}\)

\section*{Zadanie 22. (0-1)}
Proste o równaniach \(y=-\frac{5}{4} x-2\) oraz \(y=\frac{4}{2 m-1} x+1\) są prostopadłe. Wynika stąd, że\\
A. \(m=\frac{21}{10}\)\\
B. \(m=-\frac{11}{10}\)\\
C. \(m=-2\)\\
D. \(m=3\)

\section*{Zadanie 23. (0-1)}
Proste o równaniach \(y=-3 x+\frac{1}{3}\) oraz \(y=\frac{1}{3} x-3\) przecinają się w punkcie \(P=\left(x_{0}, y_{0}\right)\).\\
Wynika stąd, że\\
A. \(x_{0}>0\) i \(y_{0}>0\).\\
B. \(x_{0}>0\) i \(y_{0}<0\).\\
C. \(x_{0}<0\) i \(y_{0}>0\).\\
D. \(x_{0}<0\) i \(y_{0}<0\).

\section*{Zadanie 24. (0-1)}
Liczba wszystkich krawędzi graniastosłupa jest równa 42. Liczba wszystkich wierzchołków tego graniastosłupa jest równa\\
A. 14\\
B. 28\\
C. 15\\
D. 42

BRUDNOPIS (nie podlega ocenie)\\
\includegraphics[max width=\textwidth, center]{2025_02_10_dbaef3a5b1e9442b6008g-13}

\section*{Zadanie 25. (0-1)}
W ostrosłupie prawidłowym trójkątnym wszystkie krawędzie mają długość 8. Pole powierzchni całkowitej tego ostrosłupa jest równe\\
A. \(64 \sqrt{3}\)\\
B. \(64 \sqrt{2}\)\\
C. \(16 \sqrt{3}\)\\
D. \(16 \sqrt{2}\)

\section*{Zadanie 26. (0-1)}
Rozważamy wszystkie liczby naturalne czterocyfrowe, których suma cyfr jest równa 3. Wszystkich takich liczb jest\\
A. 13\\
B. 10\\
C. 7\\
D. 9

\section*{Zadanie 27. (0-1)}
W pudełku są tylko kule białe, czarne i zielone. Kul białych jest dwa razy więcej niż czarnych, a czarnych jest trzy razy więcej niż zielonych. Z pudełka losujemy jedną kulę. Prawdopodobieństwo wylosowania kuli białej jest równe\\
A. \(\frac{2}{3}\)\\
B. \(\frac{2}{9}\)\\
C. \(\frac{1}{6}\)\\
D. \(\frac{3}{5}\)

\section*{Zadanie 28. (0-1)}
W pewnej grupie uczniów przeprowadzono ankietę na temat liczby odsłuchanych audiobooków w lutym 2022 roku. Wyniki ankiety przedstawiono w tabeli.

\begin{center}
\begin{tabular}{|l|c|c|c|c|c|c|}
\hline
Liczba odsłuchanych audiobooków & 0 & 1 & 2 & 3 & 4 & 7 \\
\hline
Liczba uczniów & 9 & 5 & 3 & 4 & 1 & 3 \\
\hline
\end{tabular}
\end{center}

Mediana liczby odsłuchanych audiobooków w tej grupie uczniów jest równa\\
A. 3\\
B. 2\\
C. 1\\
D. \(\frac{3}{2}\)

BRUDNOPIS (nie podlega ocenie)\\
\includegraphics[max width=\textwidth, center]{2025_02_10_dbaef3a5b1e9442b6008g-15}

Zadanie 29. (0-2)\\
Rozwiąż nierówność

\[
-3 x^{2}+8 \geq 10 x
\]

\begin{center}
\begin{tabular}{|c|c|c|c|c|c|c|c|c|c|c|c|c|c|c|c|c|c|c|c|c|c|c|}
\hline
 &  &  &  &  &  &  &  &  &  &  &  &  &  &  &  &  &  &  &  &  &  &  \\
\hline
 &  &  &  &  &  &  &  &  &  &  &  &  &  &  &  &  &  &  &  &  &  &  \\
\hline
 &  &  &  &  &  &  &  &  &  &  &  &  &  &  &  &  &  &  &  &  &  &  \\
\hline
 &  &  &  &  &  &  &  &  &  &  &  &  &  &  &  &  &  &  &  &  &  &  \\
\hline
 &  &  &  &  &  &  &  &  &  &  &  &  &  &  &  &  &  &  &  &  &  &  \\
\hline
 &  &  &  &  &  &  &  &  &  &  &  &  &  &  &  &  &  &  &  &  &  &  \\
\hline
 &  &  &  &  &  &  &  &  &  &  &  &  &  &  &  &  &  &  &  &  &  &  \\
\hline
 &  &  &  &  &  & \includegraphics[max width=\textwidth]{2025_02_10_dbaef3a5b1e9442b6008g-16}
 &  &  &  &  &  &  &  &  &  &  &  &  &  &  &  &  \\
\hline
 &  &  &  &  &  &  &  &  &  &  &  &  &  &  &  &  &  &  &  &  &  &  \\
\hline
 &  &  &  &  &  &  &  &  &  &  &  &  &  &  &  &  &  &  &  &  &  &  \\
\hline
 &  &  &  &  &  &  &  &  &  &  &  &  &  &  &  &  &  &  &  &  &  &  \\
\hline
 &  &  &  &  &  &  &  &  &  &  &  &  &  &  &  &  &  &  &  &  &  &  \\
\hline
 &  &  &  &  &  &  &  &  &  &  &  &  &  &  &  &  &  &  &  &  &  &  \\
\hline
 &  &  &  &  &  &  &  &  &  &  &  &  &  &  &  &  &  &  &  &  &  &  \\
\hline
 &  &  &  &  &  &  &  &  &  &  &  &  &  &  &  &  &  &  &  &  &  &  \\
\hline
 &  &  &  &  &  &  &  &  &  &  &  &  &  &  &  &  &  &  &  &  &  &  \\
\hline
 &  &  &  &  &  &  &  &  &  &  &  &  &  &  &  &  &  &  &  &  &  &  \\
\hline
 &  &  &  &  &  &  &  &  &  &  &  &  &  &  &  &  &  &  &  &  &  &  \\
\hline
 &  &  &  &  &  &  &  &  &  &  &  &  &  &  &  &  &  &  &  &  &  &  \\
\hline
 &  &  &  &  &  &  &  &  &  &  &  &  &  &  &  &  &  &  &  &  &  &  \\
\hline
 &  &  &  &  &  &  &  &  &  &  &  &  &  &  &  &  &  &  &  &  &  &  \\
\hline
 &  &  &  &  &  &  &  &  &  &  &  &  &  &  &  &  &  &  &  &  &  &  \\
\hline
 &  &  &  &  &  &  &  &  &  &  &  &  &  &  &  &  &  &  &  &  &  &  \\
\hline
 &  &  &  &  &  &  &  &  &  &  &  &  &  &  &  &  &  &  &  &  &  &  \\
\hline
 &  &  &  &  &  &  &  &  &  &  &  &  &  &  &  &  &  &  &  &  &  &  \\
\hline
 &  &  &  &  &  &  &  &  &  &  &  &  &  &  &  &  &  &  &  &  &  &  \\
\hline
 &  &  &  &  &  &  &  &  &  &  &  &  &  &  &  &  &  &  &  &  &  &  \\
\hline
 &  &  &  &  &  &  &  &  &  &  &  &  &  &  &  &  &  &  &  &  &  &  \\
\hline
 &  &  &  &  &  &  &  &  &  &  &  &  &  &  &  &  &  &  &  &  &  &  \\
\hline
 &  &  &  &  &  &  &  &  &  &  &  &  &  &  &  &  &  &  &  &  &  &  \\
\hline
 &  &  &  &  &  &  &  &  &  &  &  &  &  &  &  &  &  &  &  &  &  &  \\
\hline
 &  &  &  &  &  &  &  &  &  &  &  &  &  &  &  &  &  &  &  &  &  &  \\
\hline
 &  &  &  &  &  &  &  &  &  &  &  &  &  &  &  &  &  &  &  &  &  &  \\
\hline
 &  &  &  &  &  &  &  &  &  &  &  &  &  &  &  &  &  &  &  &  &  &  \\
\hline
 &  &  &  &  &  &  &  &  &  &  &  &  &  &  &  &  &  &  &  &  &  &  \\
\hline
 &  &  &  &  &  &  &  &  &  &  &  &  &  &  &  &  &  &  &  &  &  &  \\
\hline
 &  &  &  &  &  &  &  &  &  &  &  &  &  &  &  &  &  &  &  &  &  &  \\
\hline
 &  &  &  &  &  &  &  &  &  &  &  &  &  &  &  &  &  &  &  &  &  &  \\
\hline
 &  &  &  &  &  &  &  &  &  &  &  &  &  &  &  &  &  &  &  &  &  &  \\
\hline
 &  &  &  &  &  &  &  &  &  &  &  &  &  &  &  &  &  &  &  &  &  &  \\
\hline
 &  &  &  &  &  &  &  &  &  &  &  &  &  &  &  &  &  &  &  &  &  &  \\
\hline
 &  &  &  &  &  &  &  &  &  &  &  &  &  &  &  &  &  &  &  &  &  &  \\
\hline
 &  &  &  &  &  &  &  &  &  &  &  &  &  &  &  &  &  &  &  &  &  &  \\
\hline
 &  &  &  &  &  &  &  &  &  &  &  &  &  &  &  &  &  &  &  &  &  &  \\
\hline
\end{tabular}
\end{center}

Zadanie 30. (0-2)\\
Wykaż, że dla każdej liczby rzeczywistej \(x\) i każdej liczby rzeczywistej \(y\) takich, że \(x \neq y\) prawdziwa jest nierówność

\[
\left(\frac{1}{5} x+\frac{4}{5} y\right)^{2}<\frac{x^{2}+4 y^{2}}{5}
\]

\begin{center}
\includegraphics[max width=\textwidth]{2025_02_10_dbaef3a5b1e9442b6008g-17}
\end{center}

Zadanie 31. (0-2)\\
Funkcja kwadratowa \(f\) ma dokładnie jedno miejsce zerowe równe 2. Ponadto \(f(0)=8\). Wyznacz wzór funkcji \(f\).

\begin{center}
\begin{tabular}{|c|c|c|c|c|c|c|c|c|c|c|c|c|c|c|c|c|c|c|c|c|c|c|c|c|}
\hline
 &  &  &  &  &  &  &  &  &  &  &  &  &  &  &  &  &  &  &  &  &  &  &  &  \\
\hline
 &  &  &  &  &  &  &  &  &  &  &  &  &  &  &  &  &  &  &  &  &  &  &  &  \\
\hline
 &  &  &  &  &  &  &  &  &  &  &  &  &  &  &  &  &  &  &  &  &  &  &  &  \\
\hline
 &  &  &  &  &  &  &  &  &  &  &  &  &  &  &  &  &  &  &  &  &  &  &  &  \\
\hline
 &  &  &  &  &  &  &  &  &  &  &  &  &  &  &  &  &  &  &  &  &  &  &  &  \\
\hline
 &  &  &  &  &  &  &  &  &  &  &  &  &  &  &  &  &  &  &  &  &  &  &  &  \\
\hline
 &  &  &  &  &  &  &  &  &  &  &  &  &  &  &  &  &  &  &  &  &  &  &  &  \\
\hline
- &  &  &  &  &  &  &  &  &  &  &  &  &  &  &  &  &  &  &  &  &  &  &  &  \\
\hline
 &  &  &  &  &  &  &  &  &  &  &  &  &  &  &  &  &  &  &  &  &  &  &  &  \\
\hline
 &  &  &  &  &  &  &  &  &  &  &  &  &  &  &  &  &  &  &  &  &  &  &  &  \\
\hline
 &  &  &  &  &  &  &  &  &  &  &  &  &  &  &  &  &  &  &  &  &  &  &  &  \\
\hline
 &  &  &  &  &  &  &  &  &  &  &  &  &  &  &  &  &  &  &  &  &  &  &  &  \\
\hline
 &  &  &  &  &  &  &  &  &  &  &  &  &  &  &  &  &  &  &  &  &  &  &  &  \\
\hline
 &  &  &  &  &  &  &  &  &  &  &  &  &  &  &  &  &  &  &  &  &  &  &  &  \\
\hline
 &  &  &  &  &  &  &  &  &  &  &  &  &  &  &  &  &  &  &  &  &  &  &  &  \\
\hline
 &  &  &  &  &  &  &  &  &  &  &  &  &  &  &  &  &  &  &  &  &  &  &  &  \\
\hline
 &  &  &  &  &  &  &  &  &  &  &  &  &  &  &  &  &  &  &  &  &  &  &  &  \\
\hline
- &  &  &  &  &  &  &  &  &  &  &  &  &  &  &  &  &  &  &  &  &  &  &  &  \\
\hline
 &  &  &  &  &  &  &  &  &  &  &  &  &  &  &  &  &  &  &  &  &  &  &  &  \\
\hline
 &  &  &  &  &  &  &  &  &  &  &  &  &  &  &  &  &  &  &  &  &  &  &  &  \\
\hline
 &  &  &  &  &  &  &  &  &  &  &  &  &  &  &  &  &  &  &  &  &  &  &  &  \\
\hline
 &  &  &  &  &  &  &  &  &  &  &  &  &  &  &  &  &  &  &  &  &  &  &  &  \\
\hline
 &  &  &  &  &  &  &  &  &  &  &  &  &  &  &  &  &  &  &  &  &  &  &  &  \\
\hline
 &  &  &  &  &  &  &  &  &  &  &  &  &  &  &  &  &  &  &  &  &  &  &  &  \\
\hline
 &  &  &  &  &  &  &  &  &  &  &  &  &  &  &  &  &  &  &  &  &  &  &  &  \\
\hline
 &  &  &  &  &  &  &  &  &  &  &  &  &  &  &  &  &  &  &  &  &  &  &  &  \\
\hline
 &  &  &  &  &  &  &  &  &  &  &  &  &  &  &  &  &  &  &  &  &  &  &  &  \\
\hline
 &  &  &  &  &  &  &  &  &  &  &  &  &  &  &  &  &  &  &  &  &  &  &  &  \\
\hline
 &  &  &  &  &  &  &  &  &  &  &  &  &  &  &  &  &  &  &  &  &  &  &  &  \\
\hline
 &  &  &  &  &  &  &  &  &  &  &  &  &  &  &  &  &  &  &  &  &  &  &  &  \\
\hline
 &  &  &  &  &  &  &  &  &  &  &  &  &  &  &  &  &  &  &  &  &  &  &  &  \\
\hline
 &  &  &  &  &  &  &  &  &  &  &  &  &  &  &  &  &  &  &  &  &  &  &  &  \\
\hline
 &  &  &  &  &  &  &  &  &  &  &  &  &  &  &  &  &  &  &  &  &  &  &  &  \\
\hline
 &  &  &  &  &  &  &  &  &  &  &  &  &  &  &  &  &  &  &  &  &  &  &  &  \\
\hline
 &  &  &  &  &  &  &  &  &  &  &  &  &  &  &  &  &  &  &  &  &  &  &  &  \\
\hline
 &  &  &  &  &  &  &  &  &  &  &  &  &  &  &  &  &  &  &  &  &  &  &  &  \\
\hline
 &  &  &  &  &  &  &  &  &  &  &  &  &  &  &  &  &  &  &  &  &  &  &  &  \\
\hline
 &  &  &  &  &  &  &  &  &  &  &  &  &  &  &  &  &  &  &  &  &  &  &  &  \\
\hline
 & - &  &  &  &  &  &  &  &  &  &  &  &  &  &  &  &  &  &  &  &  &  &  &  \\
\hline
 &  &  &  &  &  &  &  &  &  &  &  &  &  &  &  &  &  &  &  &  &  &  &  &  \\
\hline
 & \includegraphics[max width=\textwidth]{2025_02_10_dbaef3a5b1e9442b6008g-18(1)}
 &  &  &  &  &  &  &  &  &  &  &  &  &  &  &  &  &  &  &  &  &  &  &  \\
\hline
 &  &  &  &  &  &  &  &  &  &  &  &  &  &  &  &  &  &  &  &  & - &  &  &  \\
\hline
 &  &  &  &  &  &  &  &  &  &  &  &  &  &  &  &  &  &  &  &  &  &  &  &  \\
\hline
 & \includegraphics[max width=\textwidth]{2025_02_10_dbaef3a5b1e9442b6008g-18}
 &  &  &  &  &  &  &  &  &  &  &  &  &  &  &  &  &  &  &  &  &  &  &  \\
\hline
 &  &  &  &  &  &  &  &  &  &  &  &  &  &  &  &  &  &  &  &  &  &  &  &  \\
\hline
\end{tabular}
\end{center}

Zadanie 32. (0-2)\\
Trójwyrazowy ciąg \((x, 3 x+2,9 x+16)\) jest geometryczny. Oblicz \(x\).\\
\includegraphics[max width=\textwidth, center]{2025_02_10_dbaef3a5b1e9442b6008g-19}

Zadanie 33. (0-2)\\
Dany jest trapez prostokątny \(A B C D\). Podstawa \(A B\) tego trapezu jest równa 26 , a ramię \(B C\) ma długość 24. Przekątna \(A C\) tego trapezu jest prostopadła do ramienia \(B C\) (zobacz rysunek). Oblicz długość ramienia \(A D\).\\
\includegraphics[max width=\textwidth, center]{2025_02_10_dbaef3a5b1e9442b6008g-20(1)}\\
\includegraphics[max width=\textwidth, center]{2025_02_10_dbaef3a5b1e9442b6008g-20}

Zadanie 34. (0-2)\\
Ze zbioru wszystkich liczb naturalnych dwucyfrowych większych od 53 losujemy jedną liczbę. Niech A oznacza zdarzenie polegające na wylosowaniu liczby podzielnej przez 7. Oblicz prawdopodobieństwo zdarzenia \(A\).

\begin{center}
\begin{tabular}{|c|c|c|c|c|c|c|c|c|c|c|c|c|c|c|c|c|c|c|c|c|c|c|c|c|c|c|}
\hline
 &  &  &  &  &  &  &  &  &  &  &  &  &  &  &  &  &  &  &  &  &  &  &  &  &  &  \\
\hline
 &  &  &  &  &  &  &  &  &  &  &  &  &  &  &  &  &  &  &  &  &  &  &  &  &  &  \\
\hline
 &  &  &  &  &  &  &  &  &  &  &  &  &  &  &  &  &  &  &  &  &  &  &  &  &  &  \\
\hline
 &  &  &  &  &  &  &  &  &  &  &  &  &  &  &  &  &  &  &  &  &  &  &  &  &  &  \\
\hline
 &  &  &  &  &  &  &  &  &  &  &  &  &  &  &  &  &  &  &  &  &  &  &  &  &  &  \\
\hline
 &  &  &  &  &  &  &  &  &  &  &  &  &  &  &  &  &  &  &  &  &  &  &  &  &  &  \\
\hline
 &  &  &  &  &  &  &  &  &  &  &  &  &  &  &  &  &  &  &  &  &  & \includegraphics[max width=\textwidth]{2025_02_10_dbaef3a5b1e9442b6008g-21}
 &  &  &  &  \\
\hline
 &  &  &  &  &  &  &  &  &  &  &  &  &  &  &  &  &  &  &  &  &  &  &  &  &  &  \\
\hline
 &  &  &  &  &  &  &  &  &  &  &  &  &  &  &  &  &  &  &  &  &  & \includegraphics[max width=\textwidth]{2025_02_10_dbaef3a5b1e9442b6008g-21(1)}
 &  &  &  &  \\
\hline
 &  &  &  &  &  &  &  &  &  &  &  &  &  &  &  &  &  &  &  &  &  &  &  &  &  &  \\
\hline
 &  &  &  &  &  &  &  &  &  &  &  &  &  &  &  &  &  &  &  &  &  &  &  &  &  &  \\
\hline
 &  &  &  &  &  &  &  &  &  &  &  &  &  &  &  &  &  &  &  &  &  &  &  &  &  &  \\
\hline
 &  &  &  &  &  &  &  &  &  &  &  &  &  &  &  &  &  &  &  &  &  &  &  &  &  &  \\
\hline
 &  &  &  &  &  &  &  &  &  &  &  &  &  &  &  &  &  &  &  &  &  &  &  &  &  &  \\
\hline
 &  &  &  &  &  &  &  &  &  &  &  &  &  &  &  &  &  &  &  &  &  &  &  &  &  &  \\
\hline
 &  &  &  &  &  &  &  &  &  &  &  &  &  &  &  &  &  &  &  &  &  &  &  &  &  &  \\
\hline
 &  &  &  &  &  &  &  &  &  &  &  &  &  &  &  &  &  &  &  &  &  &  &  &  &  &  \\
\hline
 &  &  &  &  &  &  &  &  &  &  &  &  &  &  &  &  &  &  &  &  &  &  &  &  &  &  \\
\hline
 &  &  &  &  &  &  &  &  &  &  &  &  &  &  &  &  &  &  &  &  &  &  &  &  &  &  \\
\hline
 &  &  &  &  &  &  &  &  &  &  &  &  &  &  &  &  &  &  &  &  &  &  &  &  &  &  \\
\hline
 &  &  &  &  &  &  &  &  &  &  &  &  &  &  &  &  &  &  &  &  &  &  &  &  &  &  \\
\hline
 &  &  &  &  &  &  &  &  &  &  &  &  &  &  &  &  &  &  &  &  &  &  &  &  &  &  \\
\hline
 &  &  &  &  &  &  &  &  &  &  &  &  &  &  &  &  &  &  &  &  &  &  &  &  &  &  \\
\hline
 &  &  &  &  &  &  &  &  &  &  &  &  &  &  &  &  &  &  &  &  &  &  &  &  &  &  \\
\hline
 &  &  &  &  &  &  &  &  &  &  &  &  &  &  &  &  &  &  &  &  &  &  &  &  &  &  \\
\hline
 &  &  &  &  &  &  &  &  &  &  &  &  &  &  &  &  &  &  &  &  &  &  &  &  &  &  \\
\hline
 &  &  &  &  &  &  &  &  &  &  &  &  &  &  &  &  &  &  &  &  &  &  &  &  &  &  \\
\hline
 &  &  &  &  &  &  &  &  &  &  &  &  &  &  &  &  &  &  &  &  &  &  &  &  &  &  \\
\hline
 &  &  &  &  &  &  &  &  &  &  &  &  &  &  &  &  &  &  &  &  &  &  &  &  &  &  \\
\hline
 &  &  &  &  &  &  &  &  &  &  &  &  &  &  &  &  &  &  &  &  &  &  &  &  &  &  \\
\hline
 & - &  &  &  &  &  &  &  &  &  &  &  &  &  &  &  &  &  &  &  &  &  &  &  &  &  \\
\hline
 & \(\square\) &  &  &  &  &  &  &  &  &  &  &  &  &  &  &  &  &  &  &  &  &  &  &  &  &  \\
\hline
 &  &  &  &  &  &  &  &  &  &  &  &  &  &  &  &  &  &  &  &  &  &  &  &  &  &  \\
\hline
 &  &  &  &  &  &  &  &  &  &  &  &  &  &  &  &  &  &  &  &  &  &  &  &  &  &  \\
\hline
 & - &  &  &  &  &  &  &  &  &  &  &  &  &  &  &  &  &  &  &  &  &  &  &  &  &  \\
\hline
 &  &  &  &  &  &  &  &  &  &  &  &  &  &  &  &  &  &  &  &  &  &  &  &  &  &  \\
\hline
 &  &  &  &  &  &  &  &  &  &  &  &  &  &  &  &  &  &  &  &  &  &  &  &  &  &  \\
\hline
 &  &  &  &  &  &  &  &  &  &  &  &  &  &  &  &  &  &  &  &  &  &  &  &  &  &  \\
\hline
 & - &  &  &  &  &  &  &  &  &  &  &  &  &  &  &  &  &  &  &  &  &  &  &  &  &  \\
\hline
 &  &  &  &  &  &  &  &  &  &  &  &  &  &  &  &  &  &  &  &  &  &  &  &  &  &  \\
\hline
 &  &  &  &  &  &  &  &  &  &  &  &  &  &  &  &  &  &  &  &  &  &  &  &  &  &  \\
\hline
 &  &  &  &  &  &  &  &  &  &  &  &  &  &  &  &  &  &  &  &  &  &  &  &  &  &  \\
\hline
 &  &  &  &  &  &  &  &  &  &  &  &  &  &  &  &  &  &  &  &  &  &  &  &  &  &  \\
\hline
\end{tabular}
\end{center}

Zadanie 35. (0-5)\\
Punkt \(A=(1,-3)\) jest wierzchołkiem trójkąta \(A B C\), w którym \(|A C|=|B C|\).\\
Punkt \(S=(5,-1)\) jest środkiem odcinka \(A B\). Wierzchołek \(C\) tego trójkąta leży na prostej o równaniu \(y=x+10\). Oblicz współrzędne wierzchołków \(B\) i \(C\) tego trójkąta.\\
\includegraphics[max width=\textwidth, center]{2025_02_10_dbaef3a5b1e9442b6008g-22}\\
\includegraphics[max width=\textwidth, center]{2025_02_10_dbaef3a5b1e9442b6008g-23}\\
\includegraphics[max width=\textwidth, center]{2025_02_10_dbaef3a5b1e9442b6008g-24}

BRUDNOPIS (nie podlega ocenie)\\
\includegraphics[max width=\textwidth, center]{2025_02_10_dbaef3a5b1e9442b6008g-25}\\
\includegraphics[max width=\textwidth, center]{2025_02_10_dbaef3a5b1e9442b6008g-26}


\end{document}