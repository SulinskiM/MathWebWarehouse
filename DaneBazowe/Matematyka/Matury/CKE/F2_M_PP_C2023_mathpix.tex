\documentclass[10pt]{article}
\usepackage[polish]{babel}
\usepackage[utf8]{inputenc}
\usepackage[T1]{fontenc}
\usepackage{graphicx}
\usepackage[export]{adjustbox}
\graphicspath{ {./images/} }
\usepackage{amsmath}
\usepackage{amsfonts}
\usepackage{amssymb}
\usepackage[version=4]{mhchem}
\usepackage{stmaryrd}

\author{LiczBa punktów do uzyskania: 46}
\date{}


\newcommand\Varangle{\mathop{{<\!\!\!\!\!\text{\small)}}\:}\nolimits}

\begin{document}
\maketitle
CENTRALNA\\
KOMISJA\\
EGZAMINACYJNA

\section*{WYPEŁNIA ZDAJĄCY}
\section*{KOD}
\begin{center}
\includegraphics[max width=\textwidth]{2025_02_10_e5283a420a8565521494g-01}
\end{center}

PESEL\\
\includegraphics[max width=\textwidth, center]{2025_02_10_e5283a420a8565521494g-01(1)}

\section*{Miejsce na naklejkę.}
Sprawdź, czy kod na naklejce to E-100.

Jeżeli tak - przyklej naklejke. Jeżeli nie - zgłoś to nauczycielowi.

\section*{Egzamin maturalny}
\section*{MATEMATYKA}
\section*{Poziom podstawowy}
\section*{Symbol arkusza}
EMAP-Po-100-2306

\section*{Data: 2 czerwca 2023 r. Godzina rozpoczęcia: 9:00}
 Czas trwania: 170 minut

\section*{Przed rozpoczęciem pracy z arkuszem egzaminacyjnym}
\begin{enumerate}
  \item Sprawdź, czy nauczyciel przekazał Ci właściwy arkusz egzaminacyjny, tj. arkusz we właściwej formule, z właściwego przedmiotu na właściwym poziomie.
  \item Jeżeli przekazano Ci niewłaściwy arkusz - natychmiast zgłoś to nauczycielowi. Nie rozrywaj banderol.
  \item Jeżeli przekazano Ci właściwy arkusz - rozerwij banderole po otrzymaniu takiego polecenia od nauczyciela. Zapoznaj się z instrukcją na stronie 2.
\end{enumerate}

\section*{Instrukcja dla zdającego}
\begin{enumerate}
  \item Sprawdź, czy arkusz egzaminacyjny zawiera 31 stron (zadania 1-36). Ewentualny brak zgłoś przewodniczącemu zespołu nadzorującego egzamin.
  \item Na pierwszej stronie arkusza oraz na karcie odpowiedzi wpisz swój numer PESEL i przyklej naklejkę z kodem.
  \item Odpowiedzi do zadań zamkniętych (1-29) zaznacz na karcie odpowiedzi w części przeznaczonej dla zdającego. Zamaluj \(\square\) pola do tego przeznaczone. Błędne zaznaczenie otocz kółkiem © i zaznacz właściwe.
  \item Pamiętaj, że pominięcie argumentacji lub istotnych obliczeń w rozwiązaniu zadania otwartego (30-36) może spowodować, że za to rozwiązanie nie otrzymasz pełnej liczby punktów.
  \item Rozwiązania zadań i odpowiedzi wpisuj w miejscu na to przeznaczonym.
  \item Pisz czytelnie i używaj tylko długopisu lub pióra z czarnym tuszem lub atramentem.
  \item Nie używaj korektora, a błędne zapisy wyraźnie przekreśl.
  \item Nie wpisuj żadnych znaków w części przeznaczonej dla egzaminatora.
  \item Pamiętaj, że zapisy w brudnopisie nie będą oceniane.
  \item Możesz korzystać z Wybranych wzorów matematycznych, cyrkla i linijki oraz kalkulatora prostego. Upewnij się, czy przekazano Ci broszurę z okładką taką jak widoczna poniżej.\\
\includegraphics[max width=\textwidth, center]{2025_02_10_e5283a420a8565521494g-02}
\end{enumerate}

\section*{Zadania egzaminacyjne są wydrukowane na następnych stronach.}
W każdym z zadań od 1. do 29. wybierz i zaznacz na karcie odpowiedzi poprawną odpowiedź.

\section*{Zadanie 1. (0-1)}
Liczba \(6^{30}: 4^{15}\) jest równa\\
A. \((1,5)^{15}\)\\
B. \((1,5)^{2}\)\\
C. \(3^{30}\)\\
D. \(3^{0}\)

\section*{Zadanie 2. (0-1)}
Dla każdej dodatniej liczby rzeczywistej \(x\) iloczyn \(\sqrt{x} \cdot \sqrt[3]{x} \cdot \sqrt[6]{x}\) jest równy\\
A. \(x\)\\
B. \(\sqrt[10]{x}\)\\
C. \(\sqrt[18]{x}\)\\
D. \(x^{2}\)

\section*{Zadanie 3. (0-1)}
Klient wpłacił do banku 30000 zł na lokatę dwuletnią. Po każdym rocznym okresie oszczędzania bank dolicza odsetki w wysokości 7\% od kwoty bieżącego kapitału znajdującego się na lokacie.\\
Po dwóch latach oszczędzania łączna wartość doliczonych odsetek na tej lokacie (bez uwzględniania podatków) jest równa\\
A. 2100 zt\\
B. \(2247 \mathrm{zł}\)\\
C. \(4200 \mathrm{zł}\)\\
D. \(4347 \mathrm{zł}\)

\section*{Zadanie 4. (0-1)}
Liczba \(\log _{2} \frac{1}{8}+\log _{2} 4\) jest równa\\
A. \((-1)\)\\
B. \(\frac{1}{2}\)\\
C. 2\\
D. 5

\section*{Zadanie 5. (0-1)}
Liczba \((1+\sqrt{5})^{2}-(1-\sqrt{5})^{2}\) jest równa\\
A. 0\\
B. \((-10)\)\\
C. \(4 \sqrt{5}\)\\
D. \(2+2 \sqrt{5}\)

BRUDNOPIS (nie podlega ocenie)\\
\includegraphics[max width=\textwidth, center]{2025_02_10_e5283a420a8565521494g-05}

\section*{Zadanie 6. (0-1)}
Do zbioru rozwiązań nierówności \((x-3)(x-2)(x+20)<0\) należy liczba\\
A. \((-20)\)\\
B. \((-23)\)\\
C. 20\\
D. 23

\section*{Informacja do zadań 7.-8.}
Na rysunku przedstawiono wykres funkcji \(f\).\\
\includegraphics[max width=\textwidth, center]{2025_02_10_e5283a420a8565521494g-06}

\section*{Zadanie 7. (0-1)}
Dziedziną funkcji \(f\) jest zbiór\\
A. \((-3,-1) \cup(1,3)\)\\
B. \((-3,3)\)\\
C. \((-5,-1) \cup(1,5)\)\\
D. \((-5,5)\)

\section*{Zadanie 8. (0-1)}
Zbiorem wartości funkcji \(f\) jest zbiór\\
A. \((-3,-1) \cup(1,3)\)\\
B. \(\langle-3,-1\rangle \cup\langle 1,3\rangle\)\\
C. \((-5,-1) \cup(1,5)\)\\
D. \(\langle-5,-1\rangle \cup\langle 1,5\rangle\)

BRUDNOPIS (nie podlega ocenie)\\
\includegraphics[max width=\textwidth, center]{2025_02_10_e5283a420a8565521494g-07}

\section*{Zadanie 9. (0-1)}
Funkcja \(f\) jest określona wzorem \(f(x)=\frac{x^{2}+4}{x-2}\) dla każdej liczby rzeczywistej \(x \neq 2\).\\
Wartość funkcji \(f\) dla argumentu 4 jest równa\\
A. 6\\
B. 2\\
C. 10\\
D. 8

\section*{Zadanie 10. (0-1)}
Prosta o równaniu \(y=a x+b\) przechodzi przez punkty \(A=(-3,-1)\) oraz \(B=(4,3)\). Współczynnik \(a\) w równaniu tej prostej jest równy\\
A. \((-4)\)\\
B. \(\left(-\frac{1}{2}\right)\)\\
C. 2\\
D. \(\frac{4}{7}\)

\section*{Zadanie 11. (0-1)}
Wykresy funkcji liniowych

\[
f(x)=(2 m+3) x+5 \quad \text { oraz } \quad g(x)=-x
\]

nie mają punktów wspólnych dla\\
A. \(m=-2\)\\
B. \(m=-1\)\\
C. \(m=1\)\\
D. \(m=2\)

BRUDNOPIS (nie podlega ocenie)\\
\includegraphics[max width=\textwidth, center]{2025_02_10_e5283a420a8565521494g-09}

\section*{Zadanie 12. (0-1)}
Funkcja kwadratowa \(f\) jest określona wzorem \(f(x)=a x^{2}+b x+1\), gdzie \(a\) oraz \(b\) są pewnymi liczbami rzeczywistymi, takimi, że \(a<0\) i \(b>0\). Na jednym z rysunków A-D przedstawiono fragment wykresu tej funkcji.\\
Fragment wykresu funkcji \(f\) przedstawiono na rysunku\\
A.\\
\includegraphics[max width=\textwidth, center]{2025_02_10_e5283a420a8565521494g-10}\\
B.\\
\includegraphics[max width=\textwidth, center]{2025_02_10_e5283a420a8565521494g-10(1)}\\
C.\\
\includegraphics[max width=\textwidth, center]{2025_02_10_e5283a420a8565521494g-10(2)}\\
D.\\
\includegraphics[max width=\textwidth, center]{2025_02_10_e5283a420a8565521494g-10(3)}

\section*{Zadanie 13. (0-1)}
Ciąg \(\left(a_{n}\right)\) jest określony wzorem \(a_{n}=\frac{n-2}{3}\) dla każdej liczby naturalnej \(n \geq 1\).\\
Liczba wyrazów tego ciągu mniejszych od 10 jest równa\\
A. 28\\
B. 31\\
C. 32\\
D. 27

BRUDNOPIS (nie podlega ocenie)\\
\includegraphics[max width=\textwidth, center]{2025_02_10_e5283a420a8565521494g-11}

\section*{Zadanie 14. (0-1)}
Ciąg \(\left(a_{n}\right)\), określony wzorem \(a_{n}=-2^{n}\) dla każdej liczby naturalnej \(n \geq 1\), jest\\
A. ciągiem arytmetycznym o różnicy 2.\\
B. ciągiem arytmetycznym o różnicy ( -2 ).\\
C. ciągiem geometrycznym o ilorazie 2.\\
D. ciągiem geometrycznym o ilorazie (-2).

\section*{Zadanie 15. (0-1)}
Trzywyrazowy ciąg (1,4, \(a+5\) ) jest arytmetyczny.\\
Liczba a jest równa\\
A. 0\\
B. 7\\
C. 2\\
D. 11

\section*{Zadanie 16. (0-1)}
Ciąg geometryczny \(\left(a_{n}\right)\) jest określony dla każdej liczby naturalnej \(n \geq 1\). W tym ciągu \(a_{1}=3,75\) oraz \(a_{2}=-7,5\).\\
Suma trzech początkowych wyrazów ciągu \(\left(a_{n}\right)\) jest równa\\
A. 11,25\\
B. \((-18,75)\)\\
C. 15\\
D. \((-15)\)

\section*{Zadanie 17. (0-1)}
Dla każdego kąta ostrego \(\alpha\) wyrażenie \(\cos \alpha-\cos \alpha \cdot \sin ^{2} \alpha\) jest równe\\
A. \(\cos ^{3} \alpha\)\\
B. \(\sin ^{2} \alpha\)\\
C. \(1-\sin ^{2} \alpha\)\\
D. \(\cos \alpha\)

\section*{Zadanie 18. (0-1)}
Cosinus kąta ostrego \(\alpha\) jest równy \(\frac{2}{3}\). Wtedy \(\operatorname{tg} \alpha\) jest równy\\
A. \(\frac{2 \sqrt{5}}{5}\)\\
B. \(\frac{\sqrt{5}}{2}\)\\
C. 2\\
D. \(\frac{1}{2}\)

BRUDNOPIS (nie podlega ocenie)\\
\includegraphics[max width=\textwidth, center]{2025_02_10_e5283a420a8565521494g-13}

\section*{Zadanie 19. (0-1)}
Na łukach \(A B\) i \(C D\) okręgu są oparte kąty wpisane \(A D B\) i \(D B C\), takie że \(|\Varangle A D B|=20^{\circ}\) i \(|\Varangle D B C|=40^{\circ}\) (zobacz rysunek). Cięciwy \(A C\) i \(B D\) przecinają się w punkcie \(K\).\\
\includegraphics[max width=\textwidth, center]{2025_02_10_e5283a420a8565521494g-14(1)}

Miara kąta \(D K C\) jest równa\\
A. \(80^{\circ}\)\\
B. \(60^{\circ}\)\\
C. \(50^{\circ}\)\\
D. \(40^{\circ}\)

\section*{Zadanie 20. (0-1)}
Pole równoległoboku \(A B C D\) jest równe \(40 \sqrt{6}\). Bok \(A D\) tego równoległoboku ma długość 10, a kąt \(A B C\) równoległoboku ma miarę \(135^{\circ}\) (zobacz rysunek).\\
\includegraphics[max width=\textwidth, center]{2025_02_10_e5283a420a8565521494g-14}

Długość boku \(A B\) jest równa\\
A. \(8 \sqrt{3}\)\\
B. \(8 \sqrt{2}\)\\
C. \(16 \sqrt{2}\)\\
D. \(16 \sqrt{3}\)

BRUDNOPIS (nie podlega ocenie)\\
\includegraphics[max width=\textwidth, center]{2025_02_10_e5283a420a8565521494g-15}

\section*{Zadanie 21. (0-1)}
Odcinek \(A B\) jest średnicą okręgu o środku \(S\). Prosta \(k\) jest styczna do tego okręgu w punkcie \(A\). Prosta \(l\) przecina ten okrąg w punktach \(B\) i \(C\). Proste \(k\) i \(l\) przecinają się w punkcie \(D\), przy czym \(|B C|=4\) i \(|C D|=3\) (zobacz rysunek).\\
\includegraphics[max width=\textwidth, center]{2025_02_10_e5283a420a8565521494g-16}

Odległość punktu \(A\) od prostej \(l\) jest równa\\
A. \(\frac{7}{2}\)\\
B. 5\\
C. \(\sqrt{12}\)\\
D. \(\sqrt{3}+2\)

\section*{Zadanie 22. (0-1)}
Funkcja liniowa \(f\) jest określona wzorem \(f(x)=-x+1\). Funkcja \(g\) jest liniowa. W prostokątnym układzie współrzędnych wykres funkcji \(g\) przechodzi przez punkt \(P=(0,-1)\) i jest prostopadły do wykresu funkcji \(f\).\\
Wzorem funkcji \(g\) jest\\
A. \(g(x)=x+1\)\\
B. \(g(x)=-x-1\)\\
C. \(g(x)=-x+1\)\\
D. \(g(x)=x-1\)

BRUDNOPIS (nie podlega ocenie)\\
\includegraphics[max width=\textwidth, center]{2025_02_10_e5283a420a8565521494g-17}

\section*{Zadanie 23. (0-1)}
Dane są punkty \(A=(1,7)\) oraz \(P=(3,1)\). Punkt \(P\) dzieli odcinek \(A B\) tak, że \(|A P|:|P B|=1: 3\).\\
Punkt \(B\) ma współrzędne\\
A. \((9,-5)\)\\
B. \((9,-17)\)\\
C. \((7,-11)\)\\
D. \((5,-5)\)

\section*{Zadanie 24. (0-1)}
Punkty \(A=(-1,5)\) oraz \(C=(3,-3)\) są przeciwległymi wierzchołkami kwadratu \(A B C D\). Pole kwadratu \(A B C D\) jest równe\\
A. \(8 \sqrt{10}\)\\
B. \(16 \sqrt{5}\)\\
C. 40\\
D. 80

\section*{Zadanie 25. (0-1)}
Punkt \(S^{\prime}=(3,7)\) jest obrazem punktu \(S=(3 a-1, b+7)\) w symetrii osiowej względem osi \(O x\) układu współrzędnych, gdy\\
A. \(a=\frac{4}{3}\) oraz \(b=0\).\\
B. \(a=\frac{4}{3}\) oraz \(b=-14\).\\
C. \(a=-\frac{2}{3}\) oraz \(b=-14\).\\
D. \(a=-\frac{2}{3}\) oraz \(b=0\).

\section*{Zadanie 26. (0-1)}
Objętość ostrosłupa prawidłowego trójkątnego o wysokości 8 jest równa \(2 \sqrt{3}\). Długość krawędzi podstawy tego ostrosłupa jest równa\\
A. 3\\
B. \(\frac{\sqrt{6}}{2}\)\\
C. 1\\
D. \(\sqrt{3}\)

BRUDNOPIS (nie podlega ocenie)\\
\includegraphics[max width=\textwidth, center]{2025_02_10_e5283a420a8565521494g-19}

\section*{Zadanie 27. (0-1)}
Dany jest graniastosłup prawidłowy sześciokątny \(A B C D E F A^{\prime} B^{\prime} C^{\prime} D^{\prime} E^{\prime} F^{\prime}\), w którym krawędź podstawy ma długość 5. Przekątna \(A D^{\prime}\) tego graniastosłupa jest nachylona do płaszczyzny podstawy pod kątem \(45^{\circ}\) (zobacz rysunek).\\
\includegraphics[max width=\textwidth, center]{2025_02_10_e5283a420a8565521494g-20}

Pole ściany bocznej tego graniastosłupa jest równe\\
A. 12,5\\
B. 25\\
C. 50\\
D. 100

\section*{Zadanie 28. (0-1)}
Średnia arytmetyczna zestawu pewnych stu liczb całkowitych dodatnich jest równa s. Każdą z liczb tego zestawu zwiększamy o 4, w wyniku czego otrzymujemy nowy zestaw stu liczb. Średnia arytmetyczna nowego zestawu stu liczb jest równa\\
A. \(s+4\)\\
B. \(s+\frac{4}{100}\)\\
C. \(\frac{s+4}{100}\)\\
D. \(4 s\)

\section*{Zadanie 29. (0-1)}
Wszystkich liczb naturalnych trzycyfrowych o sumie cyff równej 3 jest\\
A. 8\\
B. 4\\
C. 5\\
D. 6

BRUDNOPIS (nie podlega ocenie)\\
\includegraphics[max width=\textwidth, center]{2025_02_10_e5283a420a8565521494g-21}

Zadanie 30. (0-2)\\
Rozwiąż nierówność

\[
x(2 x-1)<2 x
\]

\begin{center}
\begin{tabular}{|c|c|c|c|c|c|c|c|c|c|c|c|c|c|c|c|c|c|c|c|c|c|c|}
\hline
 &  &  &  &  &  &  &  &  &  &  &  &  &  &  &  &  &  &  &  &  &  &  \\
\hline
 &  &  &  &  &  &  &  &  &  &  &  &  &  &  &  &  &  &  &  &  &  &  \\
\hline
 &  &  &  &  &  &  &  &  &  &  &  &  &  &  &  &  &  &  &  &  &  &  \\
\hline
 &  &  &  &  &  &  &  &  &  &  &  &  &  &  &  &  &  &  &  &  &  &  \\
\hline
 &  &  &  &  &  &  &  &  &  &  &  &  &  &  &  &  &  &  &  &  &  &  \\
\hline
 &  &  &  &  &  &  &  &  &  &  &  &  &  &  &  &  &  &  &  &  &  &  \\
\hline
 &  &  &  &  &  &  &  &  &  &  &  &  &  &  &  &  &  &  &  &  &  &  \\
\hline
 &  &  &  &  &  &  &  &  &  &  &  &  &  &  &  &  &  &  &  &  &  &  \\
\hline
 &  &  &  &  &  &  &  &  &  &  &  &  &  &  &  &  &  &  &  &  &  &  \\
\hline
 &  &  &  &  &  &  &  &  &  &  &  &  &  &  &  &  &  &  &  &  &  &  \\
\hline
 &  &  &  &  &  &  &  &  &  &  &  &  &  &  &  &  &  &  &  &  &  &  \\
\hline
 &  &  &  &  &  &  &  &  &  &  &  &  &  &  &  &  &  &  &  &  &  &  \\
\hline
 &  &  &  &  &  &  &  &  &  &  &  &  &  &  &  &  &  &  &  &  &  &  \\
\hline
 &  &  &  &  &  &  &  &  &  &  &  &  &  &  &  &  &  &  &  &  &  &  \\
\hline
 &  &  &  &  &  &  &  &  &  &  &  &  &  &  &  &  &  &  &  &  &  &  \\
\hline
 &  &  &  &  &  &  &  &  &  &  &  &  &  &  &  &  &  &  &  &  &  &  \\
\hline
 &  &  &  &  &  &  &  &  &  &  &  &  &  &  &  &  &  &  &  &  &  &  \\
\hline
 &  &  &  &  &  &  &  &  &  &  &  &  &  &  &  &  &  &  &  &  &  &  \\
\hline
 &  &  &  &  &  &  &  &  &  &  &  &  &  &  &  &  &  &  &  &  &  &  \\
\hline
 &  &  &  &  &  &  &  &  &  &  &  &  &  &  &  &  &  &  &  &  &  &  \\
\hline
 &  &  &  &  &  &  &  &  &  &  &  &  &  &  &  &  &  &  &  &  &  &  \\
\hline
 &  &  &  &  &  &  &  &  &  &  &  &  &  &  &  &  &  &  &  &  &  &  \\
\hline
 &  &  &  &  &  &  &  &  &  &  &  &  &  &  &  &  &  &  &  &  &  &  \\
\hline
 &  &  &  &  &  &  &  &  &  &  &  &  &  &  &  &  &  &  &  &  &  &  \\
\hline
 &  &  &  &  &  &  &  &  &  &  &  &  &  &  &  &  &  &  &  &  &  &  \\
\hline
 &  &  &  &  &  &  &  &  &  &  &  &  &  &  &  &  &  &  &  &  &  &  \\
\hline
 &  &  &  &  &  &  &  &  &  &  &  &  &  &  &  &  &  &  &  &  &  &  \\
\hline
 &  &  &  &  &  &  &  &  &  &  &  &  &  &  &  &  &  &  &  &  &  &  \\
\hline
 &  &  &  &  &  &  &  &  &  &  &  &  &  &  &  &  &  &  &  &  &  &  \\
\hline
 &  &  &  &  &  &  &  &  &  &  &  &  &  &  &  &  &  &  &  &  &  &  \\
\hline
 &  &  &  &  &  &  &  &  &  &  &  &  &  &  &  &  &  &  &  &  &  &  \\
\hline
 &  &  &  &  &  &  &  &  &  &  &  &  &  &  &  &  &  &  &  &  &  &  \\
\hline
 &  &  &  &  &  &  &  &  &  &  &  &  &  &  &  &  &  &  &  &  &  &  \\
\hline
 &  &  &  &  &  &  &  &  &  &  &  &  &  &  &  &  &  &  &  &  &  &  \\
\hline
 &  &  &  &  &  &  &  &  &  &  &  &  &  &  &  &  &  &  &  &  &  &  \\
\hline
 &  &  &  &  &  &  &  &  &  &  &  &  &  &  &  &  &  &  &  &  &  &  \\
\hline
 &  &  &  &  &  &  &  &  &  &  &  &  &  &  &  &  &  &  &  &  &  &  \\
\hline
 &  &  &  &  &  &  &  &  &  &  &  &  &  &  &  &  &  &  &  &  &  &  \\
\hline
 &  &  &  &  &  &  &  &  &  &  &  &  &  &  &  &  &  &  &  &  &  &  \\
\hline
 &  &  &  &  &  &  &  &  &  &  &  &  &  &  &  &  &  &  &  &  &  &  \\
\hline
 &  &  &  &  &  &  &  &  &  &  &  &  &  &  &  &  &  &  &  &  &  &  \\
\hline
 &  &  &  &  &  &  &  &  &  &  &  &  &  &  &  &  &  &  &  &  &  &  \\
\hline
 &  &  &  &  &  &  &  &  &  &  &  &  &  &  &  &  &  &  &  &  &  &  \\
\hline
 &  &  &  &  &  &  &  &  &  &  &  &  &  &  &  &  &  &  &  &  &  &  \\
\hline
 &  &  &  &  &  &  &  &  &  &  &  &  &  &  &  &  &  &  &  &  &  &  \\
\hline
\end{tabular}
\end{center}

Zadanie 31. (0-2)\\
Rozwiąż równanie

\[
\left(2 x^{2}+3 x\right)\left(x^{2}-7\right)=0
\]

\begin{center}
\includegraphics[max width=\textwidth]{2025_02_10_e5283a420a8565521494g-23}
\end{center}

Zadanie 32. (0-2)\\
Wykaż, że dla każdej liczby rzeczywistej \(a\) i dla każdej liczby rzeczywistej \(b\) takiej, że \(b \neq a\), prawdziwa jest nierówność

\[
a^{2}+3 b^{2}+4>2 a+6 b
\]

\begin{center}
\begin{tabular}{|c|c|c|c|c|c|c|c|c|c|c|c|c|c|c|c|c|c|c|c|c|c|}
\hline
 &  &  &  &  &  &  &  &  &  &  &  &  &  &  &  &  &  &  &  &  &  \\
\hline
 &  &  &  &  &  &  &  &  &  &  &  &  &  &  &  &  &  &  &  &  &  \\
\hline
 &  &  &  &  &  &  &  &  &  &  &  &  &  &  &  &  &  &  &  &  &  \\
\hline
 &  &  &  &  &  &  &  &  &  &  &  &  &  &  &  &  &  &  &  &  &  \\
\hline
 &  &  &  &  &  &  &  &  &  &  &  &  &  &  &  &  &  &  &  &  &  \\
\hline
 &  &  &  &  &  &  &  &  &  &  &  &  &  &  &  &  &  &  &  &  &  \\
\hline
 &  &  &  &  &  &  &  &  &  &  &  &  &  &  &  &  &  &  &  &  &  \\
\hline
 &  &  &  &  &  &  &  &  &  &  &  &  &  &  &  &  &  &  &  &  &  \\
\hline
 &  &  &  &  &  &  &  &  &  &  &  &  &  &  &  &  &  &  &  &  &  \\
\hline
 &  &  &  &  &  &  &  &  &  &  &  &  &  &  &  &  &  &  &  &  &  \\
\hline
 &  &  &  &  &  &  &  &  &  &  &  &  &  &  &  &  &  &  &  &  &  \\
\hline
 &  &  &  &  &  &  &  &  &  &  &  &  &  &  &  &  &  &  &  &  &  \\
\hline
 &  &  &  &  &  &  &  &  &  &  &  &  &  &  &  &  &  &  &  &  &  \\
\hline
 &  &  &  &  &  &  &  &  &  &  &  &  &  &  &  &  &  &  &  &  &  \\
\hline
 &  &  &  &  &  &  &  &  &  &  &  &  &  &  &  &  &  &  &  &  &  \\
\hline
 &  &  &  &  &  &  &  &  &  &  &  &  &  &  &  &  &  &  &  &  &  \\
\hline
 &  &  &  &  &  &  &  &  &  &  &  &  &  &  &  &  &  &  &  &  &  \\
\hline
 &  &  &  &  &  &  &  &  &  &  &  &  &  &  &  &  &  &  &  &  &  \\
\hline
 &  &  &  &  &  &  &  &  &  &  &  &  &  &  &  &  &  &  &  &  &  \\
\hline
 &  &  &  &  &  &  &  &  &  &  &  &  &  &  &  &  &  &  &  &  &  \\
\hline
 &  &  &  &  &  &  &  &  &  &  &  &  &  &  &  &  &  &  &  &  &  \\
\hline
 &  &  &  &  &  &  &  &  &  &  &  &  &  &  &  &  &  &  &  &  &  \\
\hline
 &  &  &  &  &  &  &  &  &  &  &  &  &  &  &  &  &  &  &  &  &  \\
\hline
 &  &  &  &  &  &  &  &  &  &  &  &  &  &  &  &  &  &  &  &  &  \\
\hline
 &  &  &  &  &  &  &  &  &  &  &  &  &  &  &  &  &  &  &  &  &  \\
\hline
 &  &  &  &  &  &  &  &  &  &  &  &  &  &  &  &  &  &  &  &  &  \\
\hline
 &  &  &  &  &  &  &  &  &  &  &  &  &  &  &  &  &  &  &  &  &  \\
\hline
 &  &  &  &  &  &  &  &  &  &  &  &  &  &  &  &  &  &  &  &  &  \\
\hline
 &  &  &  &  &  &  &  &  &  &  &  &  &  &  &  &  &  &  &  &  &  \\
\hline
 &  &  &  &  &  &  &  &  &  &  &  &  &  &  &  &  &  &  &  &  &  \\
\hline
 &  &  &  &  &  &  &  &  &  &  &  &  &  &  &  &  &  &  &  &  &  \\
\hline
 &  &  &  &  &  &  &  &  &  &  &  &  &  &  &  &  &  &  &  &  &  \\
\hline
 &  &  &  &  &  &  &  &  &  &  &  &  &  &  &  &  &  &  &  &  &  \\
\hline
 &  &  &  &  &  &  &  &  &  &  &  &  &  &  &  &  &  &  &  &  &  \\
\hline
 &  &  &  &  &  &  &  &  &  &  &  &  &  &  &  &  &  &  &  &  &  \\
\hline
 &  &  &  &  &  &  &  &  &  &  &  &  &  &  &  &  &  &  &  &  &  \\
\hline
 &  &  &  &  &  &  &  &  &  &  &  &  &  &  &  &  &  &  &  &  &  \\
\hline
 &  &  &  &  &  &  &  &  &  &  &  &  &  &  &  &  &  &  &  &  &  \\
\hline
 &  &  &  &  &  &  &  &  &  &  &  &  &  &  &  &  &  &  &  &  &  \\
\hline
 &  &  &  &  &  &  &  &  &  &  &  &  &  &  &  &  &  &  &  &  &  \\
\hline
 &  &  &  &  &  &  &  &  &  &  &  &  &  &  &  &  &  &  &  &  &  \\
\hline
 &  &  &  &  &  &  &  &  &  &  &  &  &  &  &  &  &  &  &  &  &  \\
\hline
 &  &  &  &  &  &  &  &  &  &  &  &  &  &  &  &  &  &  &  &  &  \\
\hline
\end{tabular}
\end{center}

Zadanie 33. (0-2)\\
Wykresem funkcji kwadratowej \(f\) jest parabola o wierzchołku w punkcie \(A=(0,3)\). Punkt \(B=(2,0)\) leży na wykresie funkcji \(f\). Wyznacz wzór funkcji \(f\).

\begin{center}
\begin{tabular}{|c|c|c|c|c|c|c|c|c|c|c|c|c|c|c|c|c|c|c|c|c|c|}
\hline
 &  &  &  &  &  &  &  &  &  &  &  &  &  &  &  &  &  &  &  &  &  \\
\hline
 &  &  &  &  &  &  &  &  &  &  &  &  &  &  &  &  &  &  &  &  &  \\
\hline
 &  &  &  &  &  &  &  &  &  &  &  &  &  &  &  &  &  &  &  &  &  \\
\hline
 &  &  &  &  &  &  &  &  &  &  &  &  &  &  &  &  &  &  &  &  &  \\
\hline
 &  &  &  &  &  &  &  &  &  &  &  &  &  &  &  &  &  &  &  &  &  \\
\hline
 &  &  &  &  &  &  &  &  &  &  &  &  &  &  &  &  &  &  &  &  &  \\
\hline
 &  &  &  &  &  &  &  &  &  &  &  &  &  &  &  &  &  &  &  &  &  \\
\hline
 &  &  &  &  &  &  &  &  &  &  &  &  &  &  &  &  &  &  &  &  &  \\
\hline
 &  &  &  &  &  &  &  &  &  &  &  &  &  &  &  &  &  &  &  &  &  \\
\hline
 &  &  &  &  &  &  &  &  &  &  &  &  &  &  &  &  &  &  &  &  &  \\
\hline
 &  &  &  &  &  &  &  &  &  &  &  &  &  &  &  &  &  &  &  &  &  \\
\hline
 &  &  &  &  &  &  &  &  &  &  &  &  &  &  &  &  &  &  &  &  &  \\
\hline
 &  &  &  &  &  &  &  &  &  &  &  &  &  &  &  &  &  &  &  &  &  \\
\hline
 &  &  &  &  &  &  &  &  &  &  &  &  &  &  &  &  &  &  &  &  &  \\
\hline
 &  &  &  &  &  &  &  &  &  &  &  &  &  &  &  &  &  &  &  &  &  \\
\hline
 &  &  &  &  &  &  &  &  &  &  &  &  &  &  &  &  &  &  &  &  &  \\
\hline
 &  &  &  &  &  &  &  &  &  &  &  &  &  &  &  &  &  &  &  &  &  \\
\hline
 &  &  &  &  &  &  &  &  &  &  &  &  &  &  &  &  &  &  &  &  &  \\
\hline
 &  &  &  &  &  &  &  &  &  &  &  &  &  &  &  &  &  &  &  &  &  \\
\hline
 &  &  &  &  &  &  &  &  &  &  &  &  &  &  &  &  &  &  &  &  &  \\
\hline
 &  &  &  &  &  &  &  &  &  &  &  &  &  &  &  &  &  &  &  &  &  \\
\hline
 &  &  &  &  &  &  &  &  &  &  &  &  &  &  &  &  &  &  &  &  &  \\
\hline
 &  &  &  &  &  &  &  &  &  &  &  &  &  &  &  &  &  &  &  &  &  \\
\hline
 &  &  &  &  &  &  &  &  &  &  &  &  &  &  &  &  &  &  &  &  &  \\
\hline
 &  &  &  &  &  &  &  &  &  &  &  &  &  &  &  &  &  &  &  &  &  \\
\hline
 &  &  &  &  &  &  &  &  &  &  &  &  &  &  &  &  &  &  &  &  &  \\
\hline
 &  &  &  &  &  &  &  &  &  &  &  &  &  &  &  &  &  &  &  &  &  \\
\hline
 &  &  &  &  &  &  &  &  &  &  &  &  &  &  &  &  &  &  &  &  &  \\
\hline
 &  &  &  &  &  &  &  &  &  &  &  &  &  &  &  &  &  &  &  &  &  \\
\hline
 &  &  &  &  &  &  &  &  &  &  &  &  &  &  &  &  &  &  &  &  &  \\
\hline
 &  &  &  &  &  &  &  &  &  &  &  &  &  &  &  &  &  &  &  &  &  \\
\hline
 &  &  &  &  &  &  &  &  &  &  &  &  &  &  &  &  &  &  &  &  &  \\
\hline
 &  &  &  &  &  &  &  &  &  &  &  &  &  &  &  &  &  &  &  &  &  \\
\hline
 &  &  &  &  &  &  &  &  &  &  &  &  &  &  &  &  &  &  &  &  &  \\
\hline
 &  &  &  &  &  &  &  &  &  &  &  &  &  &  &  &  &  &  &  &  &  \\
\hline
 &  &  &  &  &  &  &  &  &  &  &  &  &  &  &  &  &  &  &  &  &  \\
\hline
 &  &  &  &  &  &  &  &  &  &  &  &  &  &  &  &  &  &  &  &  &  \\
\hline
 &  &  &  &  &  &  &  &  &  &  &  &  &  &  &  &  &  &  &  &  &  \\
\hline
 &  &  &  &  &  &  &  &  &  &  &  &  &  &  &  &  &  &  &  &  &  \\
\hline
 &  &  &  &  &  &  &  &  &  &  &  &  &  &  &  &  &  &  &  &  &  \\
\hline
 &  &  &  &  &  &  &  &  &  &  &  &  &  &  &  &  &  &  &  &  &  \\
\hline
 &  &  &  &  &  &  &  &  &  &  &  &  &  &  &  &  &  &  &  &  &  \\
\hline
 &  &  &  &  &  &  &  &  &  &  &  &  &  &  &  &  &  &  &  &  &  \\
\hline
\end{tabular}
\end{center}

Zadanie 34. (0-2)\\
W trójkącie prostokątnym równoramiennym \(A B C\) o przeciwprostokątnej \(B C\) punkt \(D\) jest środkiem ramienia \(A B\). Odcinek \(C D\) ma długość 5 (zobacz rysunek). Oblicz obwód trójkąta \(A B C\).\\
\includegraphics[max width=\textwidth, center]{2025_02_10_e5283a420a8565521494g-26}

\begin{center}
\begin{tabular}{|c|c|c|c|c|c|c|c|c|c|c|c|c|c|c|c|c|c|c|c|c|c|c|c|c|c|c|c|c|c|c|}
\hline
 &  &  &  &  &  &  &  &  &  &  &  &  &  &  &  &  &  &  &  &  &  &  &  &  &  &  &  &  &  &  \\
\hline
 &  &  &  &  &  &  &  &  &  &  &  &  &  &  &  &  &  &  &  &  &  &  &  &  &  &  &  &  &  &  \\
\hline
 &  &  &  &  &  &  &  &  &  &  &  &  &  &  &  &  &  &  &  &  &  &  &  &  &  &  &  &  &  &  \\
\hline
 &  &  &  &  &  &  &  &  &  &  &  &  &  &  &  &  &  &  &  &  &  &  &  &  &  &  &  &  &  &  \\
\hline
 &  &  &  &  &  &  &  &  &  &  &  &  &  &  &  &  &  &  &  &  &  &  &  &  &  &  &  &  &  &  \\
\hline
 &  &  &  &  &  &  &  &  &  &  &  &  &  &  &  &  &  &  &  &  &  &  &  &  &  &  &  &  &  &  \\
\hline
 &  &  &  &  &  &  &  &  &  &  &  &  &  &  &  &  &  &  &  &  &  &  &  &  &  &  &  &  &  &  \\
\hline
 &  &  &  &  &  &  &  &  &  &  &  &  &  &  &  &  &  &  &  &  &  &  &  &  &  &  &  &  &  &  \\
\hline
 &  &  &  &  &  &  &  &  &  &  &  &  &  &  &  &  &  &  &  &  &  &  &  &  &  &  &  &  &  &  \\
\hline
 &  &  &  &  &  &  &  &  &  &  &  &  &  &  &  &  &  &  &  &  &  &  &  &  &  &  &  &  &  &  \\
\hline
 &  &  &  &  &  &  &  &  &  &  &  &  &  &  &  &  &  &  &  &  &  &  &  &  &  &  &  &  &  &  \\
\hline
 &  &  &  &  &  &  &  &  &  &  &  &  &  &  &  &  &  &  &  &  &  &  &  &  &  &  &  &  &  &  \\
\hline
 &  &  &  &  &  &  &  &  &  &  &  &  &  &  &  &  &  &  &  &  &  &  &  &  &  &  &  &  &  &  \\
\hline
 &  &  &  &  &  &  &  &  &  &  &  &  &  &  &  &  &  &  &  &  &  &  &  &  &  &  &  &  &  &  \\
\hline
 &  &  &  &  &  &  &  &  &  &  &  &  &  &  &  &  &  &  &  &  &  &  &  &  &  &  &  &  &  &  \\
\hline
 &  &  &  &  &  &  &  &  &  &  &  &  &  &  &  &  &  &  &  &  &  &  &  &  &  &  &  &  &  &  \\
\hline
 &  &  &  &  &  &  &  &  &  &  &  &  &  &  &  &  &  &  &  &  &  &  &  &  &  &  &  &  &  &  \\
\hline
 &  &  &  &  &  &  &  &  &  &  &  &  &  &  &  &  &  &  &  &  &  &  &  &  &  &  &  &  &  &  \\
\hline
 &  &  &  &  &  &  &  &  &  &  &  &  &  &  &  &  &  &  &  &  &  &  &  &  &  &  &  &  &  &  \\
\hline
 &  &  &  &  &  &  &  &  &  &  &  &  &  &  &  &  &  &  &  &  &  &  &  &  &  &  &  &  &  &  \\
\hline
 &  &  &  &  &  &  &  &  &  &  &  &  &  &  &  &  &  &  &  &  &  &  &  &  &  &  &  &  &  &  \\
\hline
 &  &  &  &  &  &  &  &  &  &  &  &  &  &  &  &  &  &  &  &  &  &  &  &  &  &  &  &  &  &  \\
\hline
 &  &  &  &  &  &  &  &  &  &  &  &  &  &  &  &  &  &  &  &  &  &  &  &  &  &  &  &  &  &  \\
\hline
 &  &  &  &  &  &  &  &  &  &  &  &  &  &  &  &  &  &  &  &  &  &  &  &  &  &  &  &  &  &  \\
\hline
 &  &  &  &  &  &  &  &  &  &  &  &  &  &  &  &  &  &  &  &  &  &  &  &  &  &  &  &  &  &  \\
\hline
 &  &  &  &  &  &  &  &  &  &  &  &  &  &  &  &  &  &  &  &  &  &  &  &  &  &  &  &  &  &  \\
\hline
 &  &  &  &  &  &  &  &  &  &  &  &  &  &  &  &  &  &  &  &  &  &  &  &  &  &  &  &  &  &  \\
\hline
 &  &  &  &  &  &  &  &  &  &  &  &  &  &  &  &  &  &  &  &  &  &  &  &  &  &  &  &  &  &  \\
\hline
 &  &  &  &  &  &  &  &  &  &  &  &  &  &  &  &  &  &  &  &  &  &  &  &  &  &  &  &  &  &  \\
\hline
 &  &  &  &  &  &  &  &  &  &  &  &  &  &  &  &  &  &  &  &  &  &  &  &  &  &  &  &  &  &  \\
\hline
 &  &  &  &  &  &  &  &  &  &  &  &  &  &  &  &  &  &  &  &  &  &  &  &  &  &  &  &  &  &  \\
\hline
 &  &  &  &  &  &  &  &  &  &  &  &  &  &  &  &  &  &  &  &  &  &  &  &  &  &  &  &  &  &  \\
\hline
 &  &  &  &  &  &  &  &  &  &  &  &  &  &  &  &  &  &  &  &  &  &  &  &  &  &  &  &  &  &  \\
\hline
\end{tabular}
\end{center}

Zadanie 35. (0-2)\\
Ze zbioru ośmiu kolejnych liczb naturalnych - od 1 do 8 - losujemy kolejno bez zwracania dwa razy po jednej liczbie.\\
Niech \(A\) oznacza zdarzenie polegające na tym, że suma wylosowanych liczb jest dzielnikiem liczby 8.\\
Oblicz prawdopodobieństwo zdarzenia \(A\).\\
\includegraphics[max width=\textwidth, center]{2025_02_10_e5283a420a8565521494g-27}

Zadanie 36. (0-5)\\
W trapezie równoramiennym \(A B C D\) podstawa \(C D\) ma długość 5 . Punkt \(F=(3,11)\) jest środkiem odcinka \(C D\). Prosta o równaniu \(y=-\frac{4}{3} x+15\) jest osią symetrii tego trapezu oraz \(B=\left(\frac{23}{2}, 8\right)\).\\
Oblicz współrzędne wierzchołka \(A\) oraz pole tego trapezu.\\
\includegraphics[max width=\textwidth, center]{2025_02_10_e5283a420a8565521494g-28}\\
\includegraphics[max width=\textwidth, center]{2025_02_10_e5283a420a8565521494g-29}

BRUDNOPIS (nie podlega ocenie)\\
\(\qquad\)\\
\includegraphics[max width=\textwidth, center]{2025_02_10_e5283a420a8565521494g-30(11)}\\
\includegraphics[max width=\textwidth, center]{2025_02_10_e5283a420a8565521494g-30(12)}\\
\includegraphics[max width=\textwidth, center]{2025_02_10_e5283a420a8565521494g-30(9)}\\
\includegraphics[max width=\textwidth, center]{2025_02_10_e5283a420a8565521494g-30(6)}\\
\includegraphics[max width=\textwidth, center]{2025_02_10_e5283a420a8565521494g-30(10)}\\
\includegraphics[max width=\textwidth, center]{2025_02_10_e5283a420a8565521494g-30(8)}\\
\includegraphics[max width=\textwidth, center]{2025_02_10_e5283a420a8565521494g-30(4)}\\
\includegraphics[max width=\textwidth, center]{2025_02_10_e5283a420a8565521494g-30(3)}\\
\includegraphics[max width=\textwidth, center]{2025_02_10_e5283a420a8565521494g-30(1)}\\
\includegraphics[max width=\textwidth, center]{2025_02_10_e5283a420a8565521494g-30}\\
\includegraphics[max width=\textwidth, center]{2025_02_10_e5283a420a8565521494g-30(5)}\\
\(\qquad\)\\
\includegraphics[max width=\textwidth, center]{2025_02_10_e5283a420a8565521494g-30(7)}\\
\includegraphics[max width=\textwidth, center]{2025_02_10_e5283a420a8565521494g-30(2)}\\
\(\qquad\)\\
\includegraphics[max width=\textwidth, center]{2025_02_10_e5283a420a8565521494g-31}

\section*{MATEMATYKA}
\section*{Poziom podstawowy}
\section*{Formuła 2015}
\section*{MATEMATYKA}
\section*{Poziom podstawowy}
Formuła 2015

\section*{MATEMATYKA}
\section*{Poziom podstawowy}
Formuła 2015


\end{document}