\documentclass[10pt]{article}
\usepackage[polish]{babel}
\usepackage[utf8]{inputenc}
\usepackage[T1]{fontenc}
\usepackage{graphicx}
\usepackage[export]{adjustbox}
\graphicspath{ {./images/} }
\usepackage{amsmath}
\usepackage{amsfonts}
\usepackage{amssymb}
\usepackage[version=4]{mhchem}
\usepackage{stmaryrd}

\begin{document}
CENTRALNA\\
KOMISJA\\
EGZAMINACYJNA

WYPEŁNIA ZDAJĄCY\\
KOD\\
\includegraphics[max width=\textwidth, center]{2025_02_10_aae24856b02a99f0df04g-01(1)}

\section*{PESEL}
\begin{center}
\includegraphics[max width=\textwidth]{2025_02_10_aae24856b02a99f0df04g-01}
\end{center}

\section*{Miejsce na naklejke.}
Sprawdż, czy kod na naklejce to\\
E-100.

Jeżeli tak - przyklej naklejkę. Jeżeli nie - zgłoś to nauczycielowi.

\section*{Egzamin maturalny}
\section*{MATEMATYKA}
\section*{Poziom podstawowy}
Symbol arkusza EMAP-PO-100-2406

\section*{DAtA: 4 czerwca 2024 r. Godzina rozpoczęcia: 9:00 Czas trwania: 170 minut}
\section*{WYPEŁNIA ZESPÓŁ NADZORUJĄCY}
Uprawnienia zdającego do: dostosowania zasad oceniania dostosowania wzw z dyskalkulią nieprzenoszenia odpowiedzi na kartę.

\section*{LICZBA PUNKTÓW DO UZYSKANIA: 46}
\section*{Przed rozpoczęciem pracy z arkuszem egzaminacyjnym}
\begin{enumerate}
  \item Sprawdź, czy nauczyciel przekazał Ci właściwy arkusz egzaminacyjny, tj. arkusz we właściwej formule, z właściwego przedmiotu na właściwym poziomie.
  \item Jeżeli przekazano Ci niewłaściwy arkusz - natychmiast zgłoś to nauczycielowi. Nie rozrywaj banderol.
  \item Jeżeli przekazano Ci właściwy arkusz - rozerwij banderole po otrzymaniu takiego polecenia od nauczyciela. Zapoznaj się z instrukcją na stronie 2.
\end{enumerate}

\section*{Instrukcja dla zdającego}
\begin{enumerate}
  \item Sprawdź, czy arkusz egzaminacyjny zawiera 33 strony (zadania 1-36).
\end{enumerate}

Ewentualny brak zgłoś przewodniczącemu zespołu nadzorującego egzamin.\\
2. Na pierwszej stronie arkusza oraz na karcie odpowiedzi wpisz swój numer PESEL i przyklej naklejkę z kodem.\\
3. Odpowiedzi do zadań zamkniętych (1-29) zaznacz na karcie odpowiedzi w części karty przeznaczonej dla zdającego. Zamaluj \(\square\) pola do tego przeznaczone. Błędne zaznaczenie otocz kółkiem i zaznacz właściwe.\\
4. Pamiętaj, że pominięcie argumentacji lub istotnych obliczeń w rozwiązaniu zadania otwartego (30-36) może spowodować, że za to rozwiązanie nie otrzymasz pełnej liczby punktów.\\
5. Rozwiązania zadań i odpowiedzi wpisuj w miejscu na to przeznaczonym.\\
6. Pisz czytelnie i używaj tylko długopisu lub pióra z czarnym tuszem lub atramentem.\\
7. Nie używaj korektora, a błędne zapisy wyraźnie przekreśl.\\
8. Nie wpisuj żadnych znaków w części przeznaczonej dla egzaminatora.\\
9. Pamiętaj, że zapisy w brudnopisie nie będą oceniane.\\
10. Możesz korzystać z Wybranych wzorów matematycznych, cyrkla i linijki oraz kalkulatora prostego. Upewnij się, czy przekazano Ci broszurę z okładką taką jak widoczna poniżej.\\
\includegraphics[max width=\textwidth, center]{2025_02_10_aae24856b02a99f0df04g-02}

\section*{Zadania egzaminacyjne są wydrukowane na następnych stronach.}
W każdym z zadań od 1. do 29. wybierz i zaznacz na karcie odpowiedzi poprawną odpowiedź.

\section*{Zadanie 1. (0-1)}
Liczba \(2^{-1} \cdot 32^{\frac{3}{5}}\) jest równa\\
A. \((-16)\)\\
B. \((-4)\)\\
C. 2\\
D. 4

\section*{Zadanie 2. (0-1)}
Liczba \(\log _{3}\left(\frac{3}{2}\right)+\log _{3}\left(\frac{2}{9}\right)\) jest równa\\
A. \(\log _{3} \frac{31}{18}\)\\
B. \(\log _{3} \frac{5}{11}\)\\
C. \((-1)\)\\
D. \(\frac{1}{3}\)

\section*{Zadanie 3. (0-1)}
Liczba \((2 \sqrt{10}+\sqrt{2})^{2}\) jest równa\\
A. 22\\
B. 42\\
C. \(42+4 \sqrt{5}\)\\
D. \(42+8 \sqrt{5}\)

\section*{Zadanie 4. (0-1)}
Dane są dwa prostokąty: \(\mathcal{P}_{1}\) oraz \(\mathcal{P}_{2}\).\\
Długości boków prostokąta \(\mathcal{P}_{1}\) są równe \(a\) oraz \(b\).\\
Długości boków prostokąta \(\mathcal{P}_{2}\) są równe \(0,2 a\) oraz \(8 b\).\\
Pole prostokąta \(\mathcal{P}_{1}\) stanowi\\
A. \(60 \%\) pola prostokąta \(\mathcal{P}_{2}\).\\
B. \(62,5 \%\) pola prostokąta \(\mathcal{P}_{2}\).\\
C. \(160 \%\) pola prostokąta \(\mathcal{P}_{2}\).\\
D. \(162,5 \%\) pola prostokąta \(\mathcal{P}_{2}\).

BRUDNOPIS (nie podlega ocenie)\\
\includegraphics[max width=\textwidth, center]{2025_02_10_aae24856b02a99f0df04g-05}

\section*{Zadanie 5. (0-1)}
Klient wpłacił do banku na trzyletnią lokatę kwotę w wysokości \(K_{0}\) zł. Po każdym rocznym okresie oszczędzania bank dolicza odsetki w wysokości \(6 \%\) od kwoty bieżącego kapitału znajdującego się na lokacie - zgodnie z procentem składanym.

Po trzech latach oszczędzania w tym banku kwota na lokacie (bez uwzględniania podatków) jest równa\\
A. \(K_{0} \cdot(1,06)^{3}\)\\
B. \(K_{0} \cdot(1,02)^{3}\)\\
C. \(K_{0} \cdot(1,03)^{6}\)\\
D. \(K_{0} \cdot 1,18\)

\section*{Zadanie 6. (0-1)}
Liczba wszystkich całkowitych dodatnich rozwiązań nierówności

\[
\frac{3 x-5}{12}<\frac{1}{3}
\]

jest równa\\
A. 2\\
B. 3\\
C. 5\\
D. 6

\section*{Zadanie 7. (0-1)}
Układ równań \(\left\{\begin{aligned} x-2 y & =3 \\ -4 x+8 y & =-12\end{aligned}\right.\)\\
A. nie ma rozwiązań.\\
B. ma dokładnie jedno rozwiązanie.\\
C. ma dokładnie dwa rozwiązania.\\
D. ma nieskończenie wiele rozwiązań.

\section*{Zadanie 8. (0-1)}
Jednym z rozwiązań równania \(\frac{3 x \cdot(2 x+8)}{x-2}=0\) jest liczba\\
A. \((-8)\)\\
B. \((-4)\)\\
C. 2\\
D. 3

BRUDNOPIS (nie podlega ocenie)\\
\includegraphics[max width=\textwidth, center]{2025_02_10_aae24856b02a99f0df04g-07}

\section*{Zadanie 9. (0-1)}
Funkcja \(y=f(x)\) jest określona za pomocą tabeli

\begin{center}
\begin{tabular}{|c|c|c|c|c|c|}
\hline
\(\boldsymbol{x}\) & -2 & -1 & 0 & 1 & 2 \\
\hline
\(\boldsymbol{y}\) & -1 & 0 & 1 & 0 & 3 \\
\hline
\end{tabular}
\end{center}

Wskaż zdanie prawdziwe.\\
A. Funkcja \(f\) ma dokładnie jedno miejsce zerowe.\\
B. W układzie współrzędnych \((x, y)\) wykres funkcji \(f\) jest symetryczny względem osi \(O y\).\\
C. Największa wartość funkcji \(f\) jest równa 3 .\\
D. Najmniejsza wartość funkcji \(f\) jest równa (-2).

\section*{Zadanie 10. (0-1)}
Liczba 2 jest miejscem zerowym funkcji liniowej \(f(x)=(3-m) x+4\).\\
Liczba \(m\) jest równa\\
A. 0\\
B. 3\\
C. 4\\
D. 5

BRUDNOPIS (nie podlega ocenie)\\
\includegraphics[max width=\textwidth, center]{2025_02_10_aae24856b02a99f0df04g-09}

\section*{Informacja do zadań 11.-13.}
Na rysunku 1., w układzie współrzędnych \((x, y)\), przedstawiono wykres funkcji \(f\).\\
Rysunek 1.\\
\includegraphics[max width=\textwidth, center]{2025_02_10_aae24856b02a99f0df04g-10(1)}

Zadanie 11. (0-1)\\
Największa wartość funkcji \(f\) jest równa\\
A. 2\\
B. 4\\
C. 6\\
D. 7

\section*{Zadanie 12. (0-1)}
Funkcja \(f\) jest malejąca w zbiorze\\
A. \(\langle-1,1\rangle\)\\
B. \(\langle 0,4\rangle\)\\
C. \(\langle 1,7\rangle\)\\
D. \(\langle 4,7\rangle\)

BRUDNOPIS (nie podlega ocenie)\\
\includegraphics[max width=\textwidth, center]{2025_02_10_aae24856b02a99f0df04g-10}

\section*{Zadanie 13. (0-1)}
Na rysunku 2., w układzie współłrzędnych \((x, y)\), przedstawiono wykres funkcji \(g\), powstałej w wyniku przesunięcia równoległego wykresu funkcji \(f\) wzdłuż osi \(O x\) o 4 jednostki w lewo.

Rysunek 2.\\
\includegraphics[max width=\textwidth, center]{2025_02_10_aae24856b02a99f0df04g-11}

Funkcje \(f\) i \(g\) są powiązane zależnością\\
A. \(g(x)=f(x+4)\)\\
B. \(g(x)=f(x-4)\)\\
C. \(g(x)=f(x)+4\)\\
D. \(g(x)=f(x)-4\)

BRUDNOPIS (nie podlega ocenie)\\
\includegraphics[max width=\textwidth, center]{2025_02_10_aae24856b02a99f0df04g-11(1)}

Zadanie 14. (0-1)\\
Funkcja kwadratowa \(f\) jest określona wzorem \(f(x)=-(x+1)^{2}+4\).\\
Fragment wykresu funkcji \(y=f(x)\) przedstawiono na rysunku\\
A.\\
\includegraphics[max width=\textwidth, center]{2025_02_10_aae24856b02a99f0df04g-12(2)}\\
B.\\
\includegraphics[max width=\textwidth, center]{2025_02_10_aae24856b02a99f0df04g-12(3)}\\
C.\\
\includegraphics[max width=\textwidth, center]{2025_02_10_aae24856b02a99f0df04g-12(1)}\\
D.\\
\includegraphics[max width=\textwidth, center]{2025_02_10_aae24856b02a99f0df04g-12}

BRUDNOPIS (nie podlega ocenie)\\
\includegraphics[max width=\textwidth, center]{2025_02_10_aae24856b02a99f0df04g-13}

\section*{Zadanie 15. (0-1)}
Ciąg \(\left(a_{n}\right)\) jest określony wzorem \(a_{n}=2 \cdot(-1)^{n+1}+5\) dla każdej liczby naturalnej \(n \geq 1\). Suma dziesięciu początkowych kolejnych wyrazów tego ciągu jest równa\\
A. 3\\
B. 7\\
C. 50\\
D. 100

\section*{Zadanie 16. (0-1)}
W ciągu arytmetycznym \(\left(a_{n}\right)\), określonym dla każdej liczby naturalnej \(n \geq 1\), dane są wyrazy: \(a_{1}=7\) oraz \(a_{2}=13\).

Wyraz \(a_{10}\) jest równy\\
A. \((-47)\)\\
B. 52\\
C. 61\\
D. 67

\section*{Zadanie 17. (0-1)}
Trzywyrazowy ciąg ( \(-1,2, x\) ) jest arytmetyczny.\\
Trzywyrazowy ciąg ( \(-1,2, y\) ) jest geometryczny.\\
Liczby \(x\) oraz \(y\) spełniają warunki\\
A. \(x>0\) i \(y>0\)\\
B. \(x>0\) i \(y<0\)\\
C. \(x<0\) i \(y>0\)\\
D. \(x<0\) i \(y<0\)

\section*{Zadanie 18. (0-1)}
Liczba \(1+\cos ^{2} 27^{\circ}\) jest równa\\
A. \(2-\sin ^{2} 27^{\circ}\)\\
B. \(\sin ^{2} 27^{\circ}\)\\
C. \(2+\sin ^{2} 27^{\circ}\)\\
D. 2

BRUDNOPIS (nie podlega ocenie)\\
\includegraphics[max width=\textwidth, center]{2025_02_10_aae24856b02a99f0df04g-15}

\section*{Zadanie 19. (0-1)}
Podstawy trapezu prostokątnego \(A B C D\) mają długości: \(|A B|=8\) oraz \(|C D|=5\).\\
Wysokość \(A D\) tego trapezu ma długość \(\sqrt{3}\) (zobacz rysunek).\\
\includegraphics[max width=\textwidth, center]{2025_02_10_aae24856b02a99f0df04g-16(1)}

Miara kąta ostrego \(A B C\) jest równa\\
A. \(15^{\circ}\)\\
B. \(30^{\circ}\)\\
C. \(45^{\circ}\)\\
D. \(60^{\circ}\)

\section*{Zadanie 20. (0-1)}
Punkty \(A, B\) oraz \(C\) leżą na okręgu o środku w punkcie \(S\). Długość łuku \(A B\), na którym jest oparty kąt wpisany \(A C B\), jest równa \(\frac{1}{5}\) długości okręgu (zobacz rysunek).\\
\includegraphics[max width=\textwidth, center]{2025_02_10_aae24856b02a99f0df04g-16}

Miara kąta ostrego \(A C B\) jest równa\\
A. \(18^{\circ}\)\\
B. \(30^{\circ}\)\\
C. \(36^{\circ}\)\\
D. \(72^{\circ}\)

BRUDNOPIS (nie podlega ocenie)\\
\includegraphics[max width=\textwidth, center]{2025_02_10_aae24856b02a99f0df04g-17}

\section*{Zadanie 21. (0-1)}
Proste \(k\) oraz \(l\) są określone równaniami

\[
\begin{aligned}
& k: y=(3 m+1) x+2 \\
& l: y=-4 x+(2 m+5)
\end{aligned}
\]

Proste \(k\) oraz \(l\) są równoległe, gdy liczba \(m\) jest równa\\
A. \((-4)\)\\
B. \(\left(-\frac{5}{3}\right)\)\\
C. \(\left(-\frac{3}{2}\right)\)\\
D. \((-1)\)

\section*{Zadanie 22. (0-1)}
Dana jest prosta o równaniu \(y=-3 x+1\).\\
Obrazem tej prostej w symetrii środkowej względem początku układu wspórrzędnych jest prosta o równaniu\\
A. \(y=3 x+1\)\\
B. \(y=3 x-1\)\\
C. \(y=-3 x+1\)\\
D. \(y=-3 x-1\)

\section*{Zadanie 23. (0-1)}
Przekątna ściany sześcianu ma długość \(2 \sqrt{2}\).\\
Objętość tego sześcianu jest równa\\
A. 8\\
B. 24\\
C. \(\frac{16 \sqrt{6}}{9}\)\\
D. \(16 \sqrt{2}\)

BRUDNOPIS (nie podlega ocenie)\\
\includegraphics[max width=\textwidth, center]{2025_02_10_aae24856b02a99f0df04g-19}

\section*{Zadanie 24. (0-1)}
Podstawą graniastosłupa prawidłowego czworokątnego jest kwadrat o boku długości 4.\\
Przekątna tego graniastosłupa jest nachylona do płaszczyzny podstawy pod kątem \(\alpha\) takim, że \(\operatorname{tg} \alpha=2\) (zobacz rysunek).\\
\includegraphics[max width=\textwidth, center]{2025_02_10_aae24856b02a99f0df04g-20}

Wysokość tego graniastosłupa jest równa\\
A. 2\\
B. 8\\
C. \(8 \sqrt{2}\)\\
D. \(16 \sqrt{2}\)

\section*{Zadanie 25. (0-1)}
Ostrosłup prawidłowy ma 2024 ściany boczne.\\
Liczba wszystkich krawędzi tego ostrosłupa jest równa\\
A. 2025\\
B. 2026\\
C. 4048\\
D. 4052

\section*{Zadanie 26. (0-1)}
Dany jest ostrosłup prawidłowy czworokątny \(A B C D S\) o podstawie \(A B C D\).\\
Długość krawędzi podstawy tego ostrosłupa jest równa 4.\\
Pole powierzchni całkowitej tego ostrosłupa jest równe 56.\\
Wysokość ściany bocznej poprowadzona z wierzchołka \(S\) do krawędzi podstawy \(A B\) tego ostrosłupa jest równa\\
A. 3\\
B. \(\frac{5}{2}\)\\
C. \(\frac{21}{2}\)\\
D. 5

BRUDNOPIS (nie podlega ocenie)\\
\includegraphics[max width=\textwidth, center]{2025_02_10_aae24856b02a99f0df04g-21}

\section*{Zadanie 27. (0-1)}
Na diagramie przedstawiono wyniki sprawdzianu z matematyki w pewnej klasie maturalnej. Na osi poziomej podano oceny, które uzyskali uczniowie tej klasy, a na osi pionowej podano liczbę uczniów, którzy otrzymali daną ocenę.\\
\includegraphics[max width=\textwidth, center]{2025_02_10_aae24856b02a99f0df04g-22}

Średnia arytmetyczna ocen uzyskanych z tego sprawdzianu przez uczniów tej klasy jest równa\\
A. 3\\
B. 3,12\\
C. 3,5\\
D. \(4,1(6)\)

\section*{Zadanie 28. (0-1)}
Wszystkich liczb naturalnych czterocyfrowych parzystych, w których zapisie dziesiętnym występują tylko cyfry 2, 4, 7 (np. 7272, 2222, 7244), jest\\
A. 16\\
B. 27\\
C. 54\\
D. 81

\section*{Zadanie 29. (0-1)}
W pudełku znajdują się wyłącznie kule białe i czarne. Kul czarnych jest 18. Z tego pudełka w sposób losowy wyciągamy jedną kulę.\\
Prawdopodobieństwo zdarzenia polegającego na tym, że wyciągniemy kulę czarną, jest równe \(\frac{3}{5}\).

Liczba kul białych w pudełku, przed wyciągnięciem jednej kuli, była równa\\
A. 9\\
B. 12\\
C. 15\\
D. 30

BRUDNOPIS (nie podlega ocenie)\\
\includegraphics[max width=\textwidth, center]{2025_02_10_aae24856b02a99f0df04g-23}

Zadanie 30. (0-2)\\
Rozwiąż nierówność

\[
x(3 x-1)+4<7 x
\]

\begin{center}
\begin{tabular}{|c|c|c|c|c|c|c|c|c|c|c|c|c|c|c|c|c|c|c|c|c|c|c|}
\hline
 &  &  &  &  &  &  &  &  &  &  &  &  &  &  &  &  &  &  &  &  &  &  \\
\hline
 &  &  &  &  &  &  &  &  &  &  &  &  &  &  &  &  &  &  &  &  &  &  \\
\hline
 &  &  &  &  &  &  &  &  &  &  &  &  &  &  &  &  &  &  &  &  &  &  \\
\hline
 &  &  &  &  &  &  &  &  &  &  &  &  &  &  &  &  &  &  &  &  &  &  \\
\hline
 &  &  &  &  &  &  &  &  &  &  &  &  &  &  &  &  &  &  &  &  &  &  \\
\hline
 &  &  &  &  &  &  &  &  &  &  &  &  &  &  &  &  &  &  &  &  &  &  \\
\hline
 &  &  &  &  &  &  &  &  &  &  &  &  &  &  &  &  &  &  &  &  &  &  \\
\hline
 &  &  &  &  &  &  &  &  &  &  &  &  &  &  &  &  &  &  &  &  &  &  \\
\hline
 &  &  &  &  &  &  &  &  &  &  &  &  &  &  &  &  &  &  &  &  &  &  \\
\hline
- &  &  &  &  &  &  &  &  &  &  &  &  &  &  &  &  &  &  &  &  &  &  \\
\hline
 &  &  &  &  &  &  &  &  &  &  &  &  &  &  &  &  &  &  &  &  &  &  \\
\hline
 &  &  &  &  &  &  &  &  &  &  &  &  &  &  &  &  &  &  &  &  &  &  \\
\hline
 &  &  &  &  &  &  &  &  &  &  &  &  &  &  &  &  &  &  &  &  &  &  \\
\hline
 &  &  &  &  &  &  &  &  &  &  &  &  &  &  &  &  &  &  &  &  &  &  \\
\hline
 &  &  &  &  &  &  &  &  &  &  &  &  &  &  &  &  &  &  &  &  &  &  \\
\hline
\includegraphics[max width=\textwidth]{2025_02_10_aae24856b02a99f0df04g-24}
 &  &  &  &  &  &  &  &  &  &  &  &  &  &  &  &  &  &  &  &  &  &  \\
\hline
 &  &  &  &  &  &  &  &  &  &  &  &  &  &  &  &  &  &  &  &  &  &  \\
\hline
 &  &  &  &  &  &  &  &  &  &  &  &  &  &  &  &  &  &  &  &  &  &  \\
\hline
 &  &  &  &  &  &  &  &  &  &  &  &  &  &  &  &  &  &  &  &  &  &  \\
\hline
- &  &  &  &  &  &  &  &  &  &  &  &  &  &  &  &  &  &  &  &  &  &  \\
\hline
 &  &  &  &  &  &  &  &  &  &  &  &  &  &  &  &  &  &  &  &  &  &  \\
\hline
- &  &  &  &  &  &  &  &  &  &  &  &  &  &  &  &  &  &  &  &  &  &  \\
\hline
- &  &  &  &  &  &  &  &  &  &  &  &  &  &  &  &  &  &  &  &  &  &  \\
\hline
- &  &  &  &  &  &  &  &  &  &  &  &  &  &  &  &  &  &  &  &  &  &  \\
\hline
 &  &  &  &  &  &  &  &  &  &  &  &  &  &  &  &  &  &  &  &  &  &  \\
\hline
 &  &  &  &  &  &  &  &  &  &  &  &  &  &  &  &  &  &  &  &  &  &  \\
\hline
- &  &  &  &  &  &  &  &  &  &  &  &  &  &  &  &  &  &  &  &  &  &  \\
\hline
 &  &  &  &  &  &  &  &  &  &  &  &  &  &  &  &  &  &  &  &  &  &  \\
\hline
 &  &  &  &  &  &  &  &  &  &  &  &  &  &  &  &  &  &  &  &  &  &  \\
\hline
 &  &  &  &  &  &  &  &  &  &  &  &  &  &  &  &  &  &  &  &  &  &  \\
\hline
 &  &  &  &  &  &  &  &  &  &  &  &  &  &  &  &  &  &  &  &  &  &  \\
\hline
 &  &  &  &  &  &  &  &  &  &  &  &  &  &  &  &  &  &  &  &  &  &  \\
\hline
 &  &  &  &  &  &  &  &  &  &  &  &  &  &  &  &  &  &  &  &  &  &  \\
\hline
 &  &  &  &  &  &  &  &  &  &  &  &  &  &  &  &  &  &  &  &  &  &  \\
\hline
 &  &  &  &  &  &  &  &  &  &  &  &  &  &  &  &  &  &  &  &  &  &  \\
\hline
 &  &  &  &  &  &  &  &  &  &  &  &  &  &  &  &  &  &  &  &  &  &  \\
\hline
 &  &  &  &  &  &  &  &  &  &  &  &  &  &  &  &  &  &  &  &  &  &  \\
\hline
 &  &  &  &  &  &  &  &  &  &  &  &  &  &  &  &  &  &  &  &  &  &  \\
\hline
 &  &  &  &  &  &  &  &  &  &  &  &  &  &  &  &  &  &  &  &  &  &  \\
\hline
 &  &  &  &  &  &  &  &  &  &  &  &  &  &  &  &  &  &  &  &  &  &  \\
\hline
 &  &  &  &  &  &  &  &  &  &  &  &  &  &  &  &  &  &  &  &  &  &  \\
\hline
 &  &  &  &  &  &  &  &  &  &  &  &  &  &  &  &  &  &  &  &  &  &  \\
\hline
 &  &  &  &  &  &  &  &  &  &  &  &  &  &  &  &  &  &  &  &  &  &  \\
\hline
\end{tabular}
\end{center}

Zadanie 31. (0-2)\\
Parabola, która jest wykresem funkcji kwadratowej \(f\), ma z osiami układu wspótzędnych \((x, y)\) dokładnie dwa punkty wspólne: \(M=(0,18)\) oraz \(N=(3,0)\).

Wyznacz wzór funkcji kwadratowej \(f\).

\begin{center}
\begin{tabular}{|c|c|c|c|c|c|c|c|c|c|c|c|c|c|c|c|c|c|c|c|c|c|c|}
\hline
 &  &  &  &  &  &  &  &  &  &  &  &  &  &  &  &  &  &  &  &  &  &  \\
\hline
 &  &  &  &  &  &  &  &  &  &  &  &  &  &  &  &  &  &  &  &  &  &  \\
\hline
 &  &  &  &  &  &  &  &  &  &  &  &  &  &  &  &  &  &  &  &  &  &  \\
\hline
 &  &  &  &  &  &  &  &  &  &  &  &  &  &  &  &  &  &  &  &  &  &  \\
\hline
 &  &  &  &  &  &  &  &  &  &  &  &  &  &  &  &  &  &  &  &  &  &  \\
\hline
 &  &  &  &  &  &  &  &  &  &  &  &  &  &  &  &  &  &  &  &  &  &  \\
\hline
 &  &  &  &  &  &  &  &  &  &  &  &  &  &  &  &  &  &  &  &  &  &  \\
\hline
 &  &  &  &  &  &  &  &  &  &  &  &  &  &  &  &  &  &  &  &  &  &  \\
\hline
 &  &  &  &  &  &  &  &  &  &  &  &  &  &  &  &  &  &  &  &  &  &  \\
\hline
 &  &  &  &  &  &  &  &  &  &  &  &  &  &  &  &  &  &  &  &  &  &  \\
\hline
 &  &  &  &  &  &  &  &  &  &  &  &  &  &  &  &  &  &  &  &  &  &  \\
\hline
 &  &  &  &  &  &  &  &  &  &  &  &  &  &  &  &  &  &  &  &  &  &  \\
\hline
 &  &  &  &  &  &  &  &  &  &  &  &  &  &  &  &  &  &  &  &  &  &  \\
\hline
 &  &  &  &  &  &  &  &  &  &  &  &  &  &  &  &  &  &  &  &  &  &  \\
\hline
 &  &  &  &  &  &  &  &  &  &  &  &  &  &  &  &  &  &  &  &  &  &  \\
\hline
 &  &  &  &  &  &  &  &  &  &  &  &  &  &  &  &  &  &  &  &  &  &  \\
\hline
 &  &  &  &  &  &  &  &  &  &  &  &  &  &  &  &  &  &  &  &  &  &  \\
\hline
 &  &  &  &  &  &  &  &  &  &  &  &  &  &  &  &  &  &  &  &  &  &  \\
\hline
 &  &  &  &  &  &  &  &  &  &  &  &  &  &  &  &  &  &  &  &  &  &  \\
\hline
 &  &  &  &  &  &  &  &  &  &  &  &  &  &  &  &  &  &  &  &  &  &  \\
\hline
 &  &  &  &  &  &  &  &  &  &  &  &  &  &  &  &  &  &  &  &  &  &  \\
\hline
 &  &  &  &  &  &  &  &  &  &  &  &  &  &  &  &  &  &  &  &  &  &  \\
\hline
 &  &  &  &  &  &  &  &  &  &  &  &  &  &  &  &  &  &  &  &  &  &  \\
\hline
 &  &  &  &  &  &  &  &  &  &  &  &  &  &  &  &  &  &  &  &  &  &  \\
\hline
 &  &  &  &  &  &  &  &  &  &  &  &  &  &  &  &  &  &  &  &  &  &  \\
\hline
 &  &  &  &  &  &  &  &  &  &  &  &  &  &  &  &  &  &  &  &  &  &  \\
\hline
 &  &  &  &  &  &  &  &  &  &  &  &  &  &  &  &  &  &  &  &  &  &  \\
\hline
 &  &  &  &  &  &  &  &  &  &  &  &  &  &  &  &  &  &  &  &  &  &  \\
\hline
 &  &  &  &  &  &  &  &  &  &  &  &  &  &  &  &  &  &  &  &  &  &  \\
\hline
 &  &  &  &  &  &  &  &  &  &  &  &  &  &  &  &  &  &  &  &  &  &  \\
\hline
 &  &  &  &  &  &  &  &  &  &  &  &  &  &  &  &  &  &  &  &  &  &  \\
\hline
 &  &  &  &  &  &  &  &  &  &  &  &  &  &  &  &  &  &  &  &  &  &  \\
\hline
 &  &  &  &  &  &  &  &  &  &  &  &  &  &  &  &  &  &  &  &  &  &  \\
\hline
 &  &  &  &  &  &  &  &  &  &  &  &  &  &  &  &  &  &  &  &  &  &  \\
\hline
 &  &  &  &  &  &  &  &  &  &  &  &  &  &  &  &  &  &  &  &  &  &  \\
\hline
 &  &  &  &  &  &  &  &  &  &  &  &  &  &  &  &  &  &  &  &  &  &  \\
\hline
 &  &  &  &  &  &  &  &  &  &  &  &  &  &  &  &  &  &  &  &  &  &  \\
\hline
 &  &  &  &  &  &  &  &  &  &  &  &  &  &  &  &  &  &  &  &  &  &  \\
\hline
 &  &  &  &  &  &  &  &  &  &  &  &  &  &  &  &  &  &  &  &  &  &  \\
\hline
 &  &  &  &  &  &  &  &  &  &  &  &  &  &  &  &  &  &  &  &  &  &  \\
\hline
 &  &  &  &  &  &  &  &  &  &  &  &  &  &  &  &  &  &  &  &  &  &  \\
\hline
 &  &  &  &  &  &  &  &  &  &  &  &  &  &  &  &  &  &  &  &  &  &  \\
\hline
 &  &  &  &  &  &  &  &  &  &  &  &  &  &  &  &  &  &  &  &  &  &  \\
\hline
\end{tabular}
\end{center}

Zadanie 32. (0-2)\\
Wykaż, że dla każdej liczby rzeczywistej \(x \neq 1\) i dla każdej liczby rzeczywistej \(y\) prawdziwa jest nierówność

\[
x^{2}+49 y^{2}>2(x+7 y-1)
\]

\begin{center}
\begin{tabular}{|c|c|c|c|c|c|c|c|c|c|c|c|c|c|c|c|c|c|c|c|c|c|c|c|c|}
\hline
 &  &  &  &  &  &  &  &  &  &  &  &  &  &  &  &  &  &  &  &  &  &  &  &  \\
\hline
 &  &  &  &  &  &  &  &  &  &  &  &  &  &  &  &  &  &  &  &  &  &  &  &  \\
\hline
 &  &  &  &  &  &  &  &  &  &  &  &  &  &  &  &  &  &  &  &  &  &  &  &  \\
\hline
 &  &  &  &  &  &  &  &  &  &  &  &  &  &  &  &  &  &  &  &  &  &  &  &  \\
\hline
 &  &  &  &  &  &  &  &  &  &  &  &  &  &  &  &  &  &  &  &  &  &  &  &  \\
\hline
 &  &  &  &  &  &  &  &  &  &  &  &  &  &  &  &  &  &  &  &  &  &  &  &  \\
\hline
 &  &  &  &  &  &  &  &  &  &  &  &  &  &  &  &  &  &  &  &  &  &  &  &  \\
\hline
 &  &  &  &  &  &  &  &  &  &  &  &  &  &  &  &  &  &  &  &  &  &  &  &  \\
\hline
 &  &  &  &  &  &  &  &  &  &  &  &  &  &  &  &  &  &  &  &  &  &  &  &  \\
\hline
 &  &  &  &  &  &  &  &  &  &  &  &  &  &  &  &  &  &  &  &  &  &  &  &  \\
\hline
 &  &  &  &  &  &  &  &  &  &  &  &  &  &  &  &  &  &  &  &  &  &  &  &  \\
\hline
 &  &  &  &  &  &  &  &  &  &  &  &  &  &  &  &  &  &  &  &  &  &  &  &  \\
\hline
 &  &  &  &  &  &  &  &  &  &  &  &  &  &  &  &  &  &  &  &  &  &  &  &  \\
\hline
 &  &  &  &  &  &  &  &  &  &  &  &  &  &  &  &  &  &  &  &  &  &  &  &  \\
\hline
 &  &  &  &  &  &  &  &  &  &  &  &  &  &  &  &  &  &  &  &  &  &  &  &  \\
\hline
 &  &  &  &  &  &  &  &  &  &  &  &  &  &  &  &  &  &  &  &  &  &  &  &  \\
\hline
 &  &  &  &  &  &  &  &  &  &  &  &  &  &  &  &  &  &  &  &  &  &  &  &  \\
\hline
 &  &  &  &  &  &  &  &  &  &  &  &  &  &  &  &  &  &  &  &  &  &  &  &  \\
\hline
 &  &  &  &  &  &  &  &  &  &  &  &  &  &  &  &  &  &  &  &  &  &  &  &  \\
\hline
 &  &  &  &  &  &  &  &  &  &  &  &  &  &  &  &  &  &  &  &  &  &  &  &  \\
\hline
 &  &  &  &  &  &  &  &  &  &  &  &  &  &  &  &  &  &  &  &  &  &  &  &  \\
\hline
 &  &  &  &  &  &  &  &  &  &  &  &  &  &  &  &  &  &  &  &  &  &  &  &  \\
\hline
 &  &  &  &  &  &  &  &  &  &  &  &  &  &  &  &  &  &  &  &  &  &  &  &  \\
\hline
 &  &  &  &  &  &  &  &  &  &  &  &  &  &  &  &  &  &  &  &  &  &  &  &  \\
\hline
 &  &  &  &  &  &  &  &  &  &  &  &  &  &  &  &  &  &  &  &  &  &  &  &  \\
\hline
 &  &  &  &  &  &  &  &  &  &  &  &  &  &  &  &  &  &  &  &  &  &  &  &  \\
\hline
 &  &  &  &  &  &  &  &  &  &  &  &  &  &  &  &  &  &  &  &  &  &  &  &  \\
\hline
 &  &  &  &  &  &  &  &  &  &  &  &  &  &  &  &  &  &  &  &  &  &  &  &  \\
\hline
 &  &  &  &  &  &  &  &  &  &  &  &  &  &  &  &  &  &  &  &  &  &  &  &  \\
\hline
 &  &  &  &  &  &  &  &  &  &  &  &  &  &  &  &  &  &  &  &  &  &  &  &  \\
\hline
 &  &  &  &  &  &  &  &  &  &  &  &  &  &  &  &  &  &  &  &  &  &  &  &  \\
\hline
 &  &  &  &  &  &  &  &  &  &  &  &  &  &  &  &  &  &  &  &  &  &  &  &  \\
\hline
 &  &  &  &  &  &  &  &  &  &  &  &  &  &  &  &  &  &  &  &  &  &  &  &  \\
\hline
 &  &  &  &  &  &  &  &  &  &  &  &  &  &  &  &  &  &  &  &  &  &  &  &  \\
\hline
 &  &  &  &  &  &  &  &  &  &  &  &  &  &  &  &  &  &  &  &  &  &  &  &  \\
\hline
 &  &  &  &  &  &  &  &  &  &  &  &  &  &  &  &  &  &  &  &  &  &  &  &  \\
\hline
 &  &  &  &  &  &  &  &  &  &  &  &  &  &  &  &  &  &  &  &  &  &  &  &  \\
\hline
 &  &  &  &  &  &  &  &  &  &  &  &  &  &  &  &  &  &  &  &  &  &  &  &  \\
\hline
 &  &  &  &  &  &  &  &  &  &  &  &  &  &  &  &  &  &  &  &  &  &  &  &  \\
\hline
 &  &  &  &  &  &  &  &  &  &  &  &  &  &  &  &  &  &  &  &  &  &  &  &  \\
\hline
 &  &  &  &  &  &  &  &  &  &  &  &  &  &  &  &  &  &  &  &  &  &  &  &  \\
\hline
 &  &  &  &  &  &  &  &  &  &  &  &  &  &  &  &  &  &  &  &  &  &  &  &  \\
\hline
 & - &  &  &  &  &  &  & - &  &  & - &  &  &  &  &  &  &  & - &  &  &  &  &  \\
\hline
 &  &  &  &  &  &  &  &  &  &  &  &  &  &  &  &  &  &  &  &  &  &  &  &  \\
\hline
\end{tabular}
\end{center}

\section*{Zadanie 33. (0-2)}
Bok kwadratu \(A B C D\) ma długość równą 12. Punkt \(S\) jest środkiem boku \(B C\) tego kwadratu. Na odcinku \(A S\) leży punkt \(P\) taki, że odcinek \(B P\) jest prostopadły do odcinka \(A S\). Oblicz długość odcinka \(B P\).\\
\includegraphics[max width=\textwidth, center]{2025_02_10_aae24856b02a99f0df04g-27}

Zadanie 34. (0-2)\\
Trzywyrazowy ciąg \(\left(4 x^{2}-1,2 x^{2}+1,1-x\right)\) jest arytmetyczny.\\
Oblicz \(x\).

\begin{center}
\begin{tabular}{|c|c|c|c|c|c|c|c|c|c|c|c|c|c|c|c|c|c|c|c|c|c|c|c|c|c|}
\hline
 &  &  &  &  &  &  &  &  &  &  &  &  &  &  &  &  &  &  &  &  &  &  &  &  &  \\
\hline
 &  &  &  &  &  &  &  &  &  &  &  &  &  &  &  &  &  &  &  &  &  &  &  &  &  \\
\hline
 &  &  &  &  &  &  &  &  &  &  &  &  &  &  &  &  &  &  &  &  &  &  &  &  &  \\
\hline
 &  &  &  &  &  &  &  &  &  &  &  &  &  &  &  &  &  &  &  &  &  &  &  &  &  \\
\hline
 &  &  &  &  &  &  &  &  &  &  &  &  &  &  &  &  &  &  &  &  &  &  &  &  &  \\
\hline
 &  &  &  &  &  &  &  &  &  &  &  &  &  &  &  &  &  &  &  &  &  &  &  &  &  \\
\hline
 &  &  &  &  &  &  &  &  &  &  &  &  &  &  &  &  &  &  &  &  &  &  &  &  &  \\
\hline
 &  &  &  &  &  &  &  &  &  &  &  &  &  &  &  &  &  &  &  &  &  &  &  &  &  \\
\hline
 &  &  &  &  &  &  &  &  &  &  &  &  &  &  &  &  &  &  &  &  &  &  &  &  &  \\
\hline
 &  &  &  &  &  &  &  &  &  &  &  &  &  &  &  &  &  &  &  &  &  &  &  &  &  \\
\hline
 &  &  &  &  &  &  &  &  &  &  &  &  &  &  &  &  &  &  &  &  &  &  &  &  &  \\
\hline
 &  &  &  &  &  &  &  &  &  &  &  &  &  &  &  &  &  &  &  &  &  &  &  &  &  \\
\hline
 &  &  &  &  &  &  &  &  &  &  &  &  &  &  &  &  &  &  &  &  &  &  &  &  &  \\
\hline
 &  &  &  &  &  &  &  &  &  &  &  &  &  &  &  &  &  &  &  &  &  &  &  &  &  \\
\hline
 &  &  &  &  &  &  &  &  &  &  &  &  &  &  &  &  &  &  &  &  &  &  &  &  &  \\
\hline
 &  &  &  &  &  &  &  &  &  &  &  &  &  &  &  &  &  &  &  &  &  &  &  &  &  \\
\hline
 &  &  &  &  &  &  &  &  &  &  &  &  &  &  &  &  &  &  &  &  &  &  &  &  &  \\
\hline
 &  &  &  &  &  &  &  &  &  &  &  &  &  &  &  &  &  &  &  &  &  &  &  &  &  \\
\hline
 &  &  &  &  &  &  &  &  &  &  &  &  &  &  &  &  &  &  &  &  &  &  &  &  &  \\
\hline
 &  &  &  &  &  &  &  &  &  &  &  &  &  &  &  &  &  &  &  &  &  &  &  &  &  \\
\hline
 &  &  &  &  &  &  &  &  &  &  &  &  &  &  &  &  &  &  &  &  &  &  &  &  &  \\
\hline
 &  &  &  &  &  &  &  &  &  &  &  &  &  &  &  &  &  &  &  &  &  &  &  &  &  \\
\hline
 &  &  &  &  &  &  &  &  &  &  &  &  &  &  &  &  &  &  &  &  &  &  &  &  &  \\
\hline
 &  &  &  &  &  &  &  &  &  &  &  &  &  &  &  &  &  &  &  &  &  &  &  &  &  \\
\hline
 &  &  &  &  &  &  &  &  &  &  &  &  &  &  &  &  &  &  &  &  &  &  &  &  &  \\
\hline
 &  &  &  &  &  &  &  &  &  &  &  &  &  &  &  &  &  &  &  &  &  &  &  &  &  \\
\hline
 &  &  &  &  &  &  &  &  &  &  &  &  &  &  &  &  &  &  &  &  &  &  &  &  &  \\
\hline
 &  &  &  &  &  &  &  &  &  &  &  &  &  &  &  &  &  &  &  &  &  &  &  &  &  \\
\hline
 &  &  &  &  &  &  &  &  &  &  &  &  &  &  &  &  &  &  &  &  &  &  &  &  &  \\
\hline
 &  &  &  &  &  &  &  &  &  &  &  &  &  &  &  &  &  &  &  &  &  &  &  &  &  \\
\hline
 &  &  &  &  &  &  &  &  &  &  &  &  &  &  &  &  &  &  &  &  &  &  &  &  &  \\
\hline
 &  &  &  &  &  &  &  &  &  &  &  &  &  &  &  &  &  &  &  &  &  &  &  &  &  \\
\hline
 &  &  &  &  &  &  &  &  &  &  &  &  &  &  &  &  &  &  &  &  &  &  &  &  &  \\
\hline
 &  &  &  &  &  &  &  &  &  &  &  &  &  &  &  &  &  &  &  &  &  &  &  &  &  \\
\hline
 &  &  &  &  &  &  &  &  &  &  &  &  &  &  &  &  &  &  &  &  &  &  &  &  &  \\
\hline
 &  &  &  &  &  &  &  &  &  &  &  &  &  &  &  &  &  &  &  &  &  &  &  &  &  \\
\hline
 &  &  &  &  &  &  &  &  &  &  &  &  &  &  &  &  &  &  &  &  &  &  &  &  & - \\
\hline
 &  &  &  &  &  &  &  &  &  &  &  &  &  &  &  &  &  &  &  &  &  &  &  &  &  \\
\hline
 &  &  &  &  &  &  &  &  &  &  &  &  &  &  &  &  &  &  &  &  &  &  &  &  &  \\
\hline
 &  &  &  &  &  &  &  &  &  &  &  &  &  &  &  &  &  &  &  &  &  &  &  &  &  \\
\hline
 &  &  &  &  &  &  &  &  &  &  & - &  &  &  & - &  &  &  &  &  &  &  &  &  &  \\
\hline
 &  &  &  &  &  &  &  &  &  &  &  &  &  &  &  &  &  &  &  &  &  &  &  &  &  \\
\hline
 &  &  &  &  &  &  &  &  &  &  &  &  &  &  &  &  &  &  &  &  &  &  &  &  &  \\
\hline
 &  &  &  &  &  &  &  &  &  &  & - &  &  & - & - &  &  &  &  & - &  &  &  &  &  \\
\hline
L &  &  &  &  &  &  &  &  &  &  &  &  &  &  &  &  &  &  &  &  &  &  &  &  &  \\
\hline
\end{tabular}
\end{center}

\section*{Zadanie 35. (0-2)}
Doświadczenie losowe polega na dwukrotnym rzucie symetryczną sześcienną kostką do gry, która na każdej ściance ma inną liczbę oczek - od jednego oczka do sześciu oczek.

Oblicz prawdopodobieństwo zdarzenia \(A\) polegającego na tym, że w pierwszym rzucie wypadnie większa liczba oczek niż w drugim rzucie.

\begin{center}
\begin{tabular}{|c|c|c|c|c|c|c|c|c|c|c|c|c|c|c|c|c|c|c|c|c|c|c|}
\hline
 &  &  &  &  &  &  &  &  &  &  &  &  &  &  &  &  &  &  &  &  &  &  \\
\hline
 &  &  &  &  &  &  &  &  &  &  &  &  &  &  &  &  &  &  &  &  &  &  \\
\hline
 &  &  &  &  &  &  &  &  &  &  &  &  &  &  &  &  &  &  &  &  &  &  \\
\hline
 &  &  &  &  &  &  &  &  &  &  &  &  &  &  &  &  &  &  &  &  &  &  \\
\hline
 &  &  &  &  &  &  &  &  &  &  &  &  &  &  &  &  &  &  &  &  &  &  \\
\hline
 &  &  &  &  &  &  &  &  &  &  &  &  &  &  &  &  &  &  &  &  &  &  \\
\hline
 &  &  &  &  &  &  &  &  &  &  &  &  &  &  &  &  &  &  &  &  &  &  \\
\hline
 &  &  &  &  &  &  &  &  &  &  &  &  &  &  &  &  &  &  &  &  &  &  \\
\hline
 &  &  &  &  &  &  &  &  &  &  &  &  &  &  &  &  &  &  &  &  &  &  \\
\hline
 &  &  &  &  &  &  &  &  &  &  &  &  &  &  &  &  &  &  &  &  &  &  \\
\hline
 &  &  &  &  &  &  &  &  &  &  &  &  &  &  &  &  &  &  &  &  &  &  \\
\hline
 &  &  &  &  &  &  &  &  &  &  &  &  &  &  &  &  &  &  &  &  &  &  \\
\hline
 &  &  &  &  &  &  &  &  &  &  &  &  &  &  &  &  &  &  &  &  &  &  \\
\hline
 &  &  &  &  &  &  &  &  &  &  &  &  &  &  &  &  &  &  &  &  &  &  \\
\hline
 &  &  &  &  &  &  &  &  &  &  &  &  &  &  &  &  &  &  &  &  &  &  \\
\hline
 &  &  &  &  &  &  &  &  &  &  &  &  &  &  &  &  &  &  &  &  &  &  \\
\hline
 &  &  &  &  &  &  &  &  &  &  &  &  &  &  &  &  &  &  &  &  &  &  \\
\hline
 &  &  &  &  &  &  &  &  &  &  &  &  &  &  &  &  &  &  &  &  &  &  \\
\hline
 &  &  &  &  &  &  &  &  &  &  &  &  &  &  &  &  &  &  &  &  &  &  \\
\hline
 &  &  &  &  &  &  &  &  &  &  &  &  &  &  &  &  &  &  &  &  &  &  \\
\hline
 &  &  &  &  &  &  &  &  &  &  &  &  &  &  &  &  &  &  &  &  &  &  \\
\hline
 &  &  &  &  &  &  &  &  &  &  &  &  &  &  &  &  &  &  &  &  &  &  \\
\hline
 &  &  &  &  &  &  &  &  &  &  &  &  &  &  &  &  &  &  &  &  &  &  \\
\hline
 &  &  &  &  &  &  &  &  &  &  &  &  &  &  &  &  &  &  &  &  &  &  \\
\hline
 &  &  &  &  &  &  &  &  &  &  &  &  &  &  &  &  &  &  &  &  &  &  \\
\hline
 &  &  &  &  &  &  &  &  &  &  &  &  &  &  &  &  &  &  &  &  &  &  \\
\hline
 &  &  &  &  &  &  &  &  &  &  &  &  &  &  &  &  &  &  &  &  &  &  \\
\hline
 &  &  &  &  &  &  &  &  &  &  &  &  &  &  &  &  &  &  &  &  &  &  \\
\hline
 &  &  &  &  &  &  &  &  &  &  &  &  &  &  &  &  &  &  &  &  &  &  \\
\hline
 &  &  &  &  &  &  &  &  &  &  &  &  &  &  &  &  &  &  &  &  &  &  \\
\hline
 &  &  &  &  &  &  &  &  &  &  &  &  &  &  &  &  &  &  &  &  &  &  \\
\hline
 &  &  &  &  &  &  &  &  &  &  &  &  &  &  &  &  &  &  &  &  &  &  \\
\hline
 &  &  &  &  &  &  &  &  &  &  &  &  &  &  &  &  &  &  &  &  &  &  \\
\hline
 &  &  &  &  &  &  &  &  &  &  &  &  &  &  &  &  &  &  &  &  &  &  \\
\hline
 &  &  &  &  &  &  &  &  &  &  &  &  &  &  &  &  &  &  &  &  &  &  \\
\hline
 &  &  &  &  &  &  &  &  &  &  &  &  &  &  &  &  &  &  &  &  &  &  \\
\hline
 &  &  &  &  &  &  &  &  &  &  &  &  &  &  &  &  &  &  &  &  &  &  \\
\hline
 &  &  &  &  &  &  &  &  &  &  &  &  &  &  &  &  &  &  &  &  &  &  \\
\hline
 &  &  &  &  &  &  &  &  &  &  &  &  &  &  &  &  &  &  &  &  &  &  \\
\hline
 &  &  &  &  &  &  &  &  &  &  &  &  &  &  &  &  &  &  &  &  &  &  \\
\hline
 &  &  &  &  &  &  &  &  &  &  &  &  &  &  &  &  &  &  &  &  &  &  \\
\hline
 &  &  &  &  &  &  &  &  &  &  &  &  &  &  &  &  &  &  &  &  &  &  \\
\hline
\end{tabular}
\end{center}

Zadanie 36. (0-5)\\
W układzie współrzędnych \((x, y)\) dane są punkty \(A=(2,8)\) oraz \(B=(10,2)\).\\
Symetralna odcinka \(A B\) przecina oś \(O x\) układu współrzędnych w punkcie \(P\).\\
Oblicz współrzędne punktu \(P\) oraz obwód trójkąta \(A B P\).

\begin{center}
\begin{tabular}{|c|c|c|c|c|c|c|c|c|c|c|c|c|c|c|c|c|c|c|c|c|c|c|c|c|c|c|}
\hline
 &  &  &  &  &  &  &  &  &  &  &  &  &  &  &  &  &  &  &  &  &  &  &  &  &  &  \\
\hline
 &  &  &  &  &  &  &  &  &  &  &  &  &  &  &  &  &  &  &  &  &  &  &  &  &  &  \\
\hline
 &  &  &  &  &  &  &  &  &  &  &  &  &  &  &  &  &  &  &  &  &  &  &  &  &  &  \\
\hline
 &  &  &  &  &  &  &  &  &  &  &  &  &  &  &  &  &  &  &  &  &  &  &  &  &  &  \\
\hline
 &  &  &  &  &  &  &  &  &  &  &  &  &  &  &  &  &  &  &  &  &  &  &  &  &  &  \\
\hline
 &  &  &  &  &  &  &  &  &  &  &  &  &  &  &  &  &  &  &  &  &  &  &  &  &  &  \\
\hline
 &  &  &  &  &  &  &  &  &  &  &  &  &  &  &  &  &  &  &  &  &  &  &  &  &  &  \\
\hline
 &  &  &  &  &  &  &  &  &  &  &  &  &  &  &  &  &  &  &  &  &  &  &  &  &  &  \\
\hline
 &  &  &  &  &  &  &  &  &  &  &  &  &  &  &  &  &  &  &  &  &  &  &  &  &  &  \\
\hline
 &  &  &  &  &  &  &  &  &  &  &  &  &  &  &  &  &  &  &  &  &  &  &  &  &  &  \\
\hline
 &  &  &  &  &  &  &  &  &  &  &  &  &  &  &  &  &  &  &  &  &  &  &  &  &  &  \\
\hline
 &  &  &  &  &  &  &  &  &  &  &  &  &  &  &  &  &  &  &  &  &  &  &  &  &  &  \\
\hline
 &  &  &  &  &  &  &  &  &  &  &  &  &  &  &  &  &  &  &  &  &  &  &  &  &  &  \\
\hline
 &  &  &  &  &  &  &  &  &  &  &  &  &  &  &  &  &  &  &  &  &  &  &  &  &  &  \\
\hline
 &  &  &  &  &  &  &  &  &  &  &  &  &  &  &  &  &  &  &  &  &  &  &  &  &  &  \\
\hline
 &  &  &  &  &  &  &  &  &  &  &  &  &  &  &  &  &  &  &  &  &  &  &  &  &  &  \\
\hline
 &  &  &  &  &  &  &  &  &  &  &  &  &  &  &  &  &  &  &  &  &  &  &  &  &  &  \\
\hline
 &  &  &  &  &  &  &  &  &  &  &  &  &  &  &  &  &  &  &  &  &  &  &  &  &  &  \\
\hline
 &  &  &  &  &  &  &  &  &  &  &  &  &  &  &  &  &  &  &  &  &  &  &  &  &  &  \\
\hline
 &  &  &  &  &  &  &  &  &  &  &  &  &  &  &  &  &  &  &  &  &  &  &  &  &  &  \\
\hline
 &  &  &  &  &  &  &  &  &  &  &  &  &  &  &  &  &  &  &  &  &  &  &  &  &  &  \\
\hline
 &  &  &  &  &  &  &  &  &  &  &  &  &  &  &  &  &  &  &  &  &  &  &  &  &  &  \\
\hline
 &  &  &  &  &  &  &  &  &  &  &  &  &  &  &  &  &  &  &  &  &  &  &  &  &  &  \\
\hline
 &  &  &  &  &  &  &  &  &  &  &  &  &  &  &  &  &  &  &  &  &  &  &  &  &  &  \\
\hline
 &  &  &  &  &  &  &  &  &  &  &  &  &  &  &  &  &  &  &  &  &  &  &  &  &  &  \\
\hline
 &  &  &  &  &  &  &  &  &  &  &  &  &  &  &  &  &  &  &  &  &  &  &  &  &  &  \\
\hline
 &  &  &  &  &  &  &  &  &  &  &  &  &  &  &  &  &  &  &  &  &  &  &  &  &  &  \\
\hline
 &  &  &  &  &  &  &  &  &  &  &  &  &  &  &  &  &  &  &  &  &  &  &  &  &  &  \\
\hline
 &  &  &  &  &  &  &  &  &  &  &  &  &  &  &  &  &  &  &  &  &  &  &  &  &  &  \\
\hline
 &  &  &  &  &  &  &  &  &  &  &  &  &  &  &  &  &  &  &  &  &  &  &  &  &  &  \\
\hline
 &  &  &  &  &  &  &  &  &  &  &  &  &  &  &  &  &  &  &  &  &  &  &  &  &  &  \\
\hline
 &  &  &  &  &  &  &  &  &  &  &  &  &  &  &  &  &  &  &  &  &  &  &  &  &  &  \\
\hline
 &  &  &  &  &  &  &  &  &  &  &  &  &  &  &  &  &  &  &  &  &  &  &  &  &  &  \\
\hline
 &  &  &  &  &  &  &  &  &  &  &  &  &  &  &  &  &  &  &  &  &  &  &  &  &  &  \\
\hline
 &  &  &  &  &  &  &  &  &  &  &  &  &  &  &  &  &  &  &  &  &  &  &  &  &  &  \\
\hline
 &  &  &  &  &  &  &  &  &  &  &  &  &  &  &  &  &  &  &  &  &  &  &  &  &  &  \\
\hline
 &  &  &  &  &  &  &  &  &  &  &  &  &  &  &  &  &  &  &  &  &  &  &  &  &  &  \\
\hline
 &  &  &  &  &  &  &  &  &  &  &  &  &  &  &  &  &  &  &  &  &  &  &  &  &  &  \\
\hline
 &  &  &  &  &  &  &  &  &  &  &  &  &  &  &  &  &  &  &  &  &  &  &  &  &  &  \\
\hline
 &  &  &  &  &  &  &  &  &  &  &  &  &  &  &  &  &  &  &  &  &  &  &  &  &  &  \\
\hline
 &  &  &  &  &  &  &  &  &  &  &  &  &  &  &  &  &  &  &  &  &  &  &  &  &  &  \\
\hline
\(\square\) &  &  &  &  &  &  &  &  &  &  &  &  &  &  &  &  &  &  &  &  &  &  &  &  &  &  \\
\hline
 &  &  &  &  &  &  &  &  &  &  &  &  &  &  &  &  &  &  &  &  &  &  &  &  &  &  \\
\hline
\end{tabular}
\end{center}

\begin{center}
\includegraphics[max width=\textwidth]{2025_02_10_aae24856b02a99f0df04g-31}
\end{center}

BRUDNOPIS (nie podlega ocenie)

\begin{center}
\begin{tabular}{|c|c|c|c|c|c|c|c|c|c|c|c|c|c|c|c|c|c|c|c|c|c|c|}
\hline
 &  &  &  &  &  &  &  &  &  &  &  &  &  &  &  &  &  &  &  &  &  &  \\
\hline
 &  &  &  &  &  &  &  &  &  &  &  &  &  &  &  &  &  &  &  &  &  &  \\
\hline
 &  &  &  &  &  &  &  &  &  &  &  &  &  &  &  &  &  &  &  &  &  &  \\
\hline
 &  &  &  &  &  &  &  &  &  &  &  &  &  &  &  &  &  &  &  &  &  &  \\
\hline
 &  &  &  &  &  &  &  &  &  &  &  &  &  &  &  &  &  &  &  &  &  &  \\
\hline
 &  &  &  &  &  &  &  &  &  &  &  &  &  &  &  &  &  &  &  &  &  &  \\
\hline
 &  &  &  &  &  &  &  &  &  &  &  &  &  &  &  &  &  &  &  &  &  &  \\
\hline
 &  &  &  &  &  &  &  &  &  &  &  &  &  &  &  &  &  &  &  &  &  &  \\
\hline
 &  &  &  &  &  &  &  &  &  &  &  &  &  &  &  &  &  &  &  &  &  &  \\
\hline
 &  &  &  &  &  &  &  &  &  &  &  &  &  &  &  &  &  &  &  &  &  &  \\
\hline
 &  &  &  &  &  &  &  &  &  &  &  &  &  &  &  &  &  &  &  &  &  &  \\
\hline
 &  &  &  &  &  &  &  &  &  &  &  &  &  &  &  &  &  &  &  &  &  &  \\
\hline
 &  &  &  &  &  &  &  &  &  &  &  &  &  &  &  &  &  &  &  &  &  &  \\
\hline
 &  &  &  &  &  &  &  &  &  &  &  &  &  &  &  &  &  &  &  &  &  &  \\
\hline
 &  &  &  &  &  &  &  &  &  &  &  &  &  &  &  &  &  &  &  &  &  &  \\
\hline
 &  &  &  &  &  &  &  &  &  &  &  &  &  &  &  &  &  &  &  &  &  &  \\
\hline
 &  &  &  &  &  &  &  &  &  &  &  &  &  &  &  &  &  &  &  &  &  &  \\
\hline
 &  &  &  &  &  &  &  &  &  &  &  &  &  &  &  &  &  &  &  &  &  &  \\
\hline
 &  &  &  &  &  &  &  &  &  &  &  &  &  &  &  &  &  &  &  &  &  &  \\
\hline
 &  &  &  &  &  &  &  &  &  &  &  &  &  &  &  &  &  &  &  &  &  &  \\
\hline
 &  &  &  &  &  &  &  &  &  &  &  &  &  &  &  &  &  &  &  &  &  &  \\
\hline
 &  &  &  &  &  &  &  &  &  &  &  &  &  &  &  &  &  &  &  &  &  &  \\
\hline
 &  &  &  &  &  &  &  &  &  &  &  &  &  &  &  &  &  &  &  &  &  &  \\
\hline
 &  &  &  &  &  &  &  &  &  &  &  &  &  &  &  &  &  &  &  &  &  &  \\
\hline
 &  &  &  &  &  &  &  &  &  &  &  &  &  &  &  &  &  &  &  &  &  &  \\
\hline
 &  &  &  &  &  &  &  &  &  &  &  &  &  &  &  &  &  &  &  &  &  &  \\
\hline
 &  &  &  &  &  &  &  &  &  &  &  &  &  &  &  &  &  &  &  &  &  &  \\
\hline
 &  &  &  &  &  &  &  &  &  &  &  &  &  &  &  &  &  &  &  &  &  &  \\
\hline
 &  &  &  &  &  &  &  &  &  &  &  &  &  &  &  &  &  &  &  &  &  &  \\
\hline
 &  &  &  &  &  &  &  &  &  &  &  &  &  &  &  &  &  &  &  &  &  &  \\
\hline
 &  &  &  &  &  &  &  &  &  &  &  &  &  &  &  &  &  &  &  &  &  &  \\
\hline
 &  &  &  &  &  &  &  &  &  &  &  &  &  &  &  &  &  &  &  &  &  &  \\
\hline
 &  &  &  &  &  &  &  &  &  &  &  &  &  &  &  &  &  &  &  &  &  &  \\
\hline
 &  &  &  &  &  &  &  &  &  &  &  &  &  &  &  &  &  &  &  &  &  &  \\
\hline
\includegraphics[max width=\textwidth]{2025_02_10_aae24856b02a99f0df04g-32(2)}
 &  &  &  &  &  &  &  &  &  &  &  &  &  &  &  &  &  &  &  &  &  &  \\
\hline
 &  &  &  &  &  &  &  &  &  &  &  &  &  &  &  &  &  &  &  &  &  &  \\
\hline
\includegraphics[max width=\textwidth]{2025_02_10_aae24856b02a99f0df04g-32(1)}
 & \includegraphics[max width=\textwidth]{2025_02_10_aae24856b02a99f0df04g-32(3)}
 &  &  &  &  &  &  &  &  &  &  &  &  &  &  &  &  &  &  &  &  &  \\
\hline
 &  &  &  &  &  &  &  &  &  &  &  &  &  &  &  &  &  &  &  &  &  &  \\
\hline
\includegraphics[max width=\textwidth]{2025_02_10_aae24856b02a99f0df04g-32}
 &  &  &  &  &  &  &  &  &  &  &  &  &  &  &  &  &  &  &  &  &  &  \\
\hline
 &  &  &  &  &  &  &  &  &  &  &  &  &  &  &  &  &  &  &  &  &  &  \\
\hline
\(\square\) &  &  &  &  &  &  &  &  &  &  &  &  &  &  &  &  &  &  &  &  &  &  \\
\hline
 &  &  &  &  &  &  &  &  &  &  &  &  &  &  &  &  &  &  &  &  &  &  \\
\hline
 &  &  &  &  &  &  &  &  &  &  &  &  &  &  &  &  &  &  &  &  &  &  \\
\hline
 &  &  &  &  &  &  &  &  &  &  &  &  &  &  &  &  &  &  &  &  &  &  \\
\hline
 &  &  &  &  &  &  &  &  &  &  &  &  &  &  &  &  &  &  &  &  &  &  \\
\hline
 &  &  &  &  &  &  &  &  &  &  &  &  &  &  &  &  &  &  &  &  &  &  \\
\hline
 &  &  &  &  &  &  &  &  &  &  &  &  &  &  &  &  &  &  &  &  &  &  \\
\hline
 &  &  &  &  &  &  &  &  &  &  &  &  &  &  &  &  &  &  &  &  &  &  \\
\hline
\end{tabular}
\end{center}

\begin{center}
\includegraphics[max width=\textwidth]{2025_02_10_aae24856b02a99f0df04g-33}
\end{center}

\section*{MATEMATYKA}
\section*{Poziom podstawowy}
\section*{Formuła 2015}
\section*{MATEMATYKA}
\section*{Poziom podstawowy}
Formuła 2015

\section*{MATEMATYKA}
\section*{Poziom podstawowy}
Formuła 2015


\end{document}