\documentclass[10pt]{article}
\usepackage[polish]{babel}
\usepackage[utf8]{inputenc}
\usepackage[T1]{fontenc}
\usepackage{amsmath}
\usepackage{amsfonts}
\usepackage{amssymb}
\usepackage[version=4]{mhchem}
\usepackage{stmaryrd}
\usepackage{graphicx}
\usepackage[export]{adjustbox}
\graphicspath{ {./images/} }

\title{EGZAMIN MATURALNY OD ROKU SZKOLNEGO 2014/2015 }

\author{}
\date{}


\begin{document}
\maketitle
\section*{MATEMATYKA POZIOM PODSTAWOWY}
\section*{PRZYKŁADOWY ZESTAW ZADAŃ (A1)}
W czasie trwania egzaminu zdający może korzystać z zestawu wzorów matematycznych, linijki i cyrkla oraz kalkulatora.

Czas pracy: 170 minut

\section*{ZADANIA ZAMKNIĘTE}
W zadaniach 1-23 wybierz i zaznacz poprawnq odpowiedź.

\section*{Zadanie 1. (0-1)}
Liczba 15 jest przybliżeniem z niedomiarem liczby \(x\). Błąd bezwzględny tego przybliżenia jest równy 0,24 . Liczba \(x\) to\\
A. 14,76\\
B. 14,80\\
C. 15,20\\
D. 15,24

\section*{Zadanie 2. (0-1)}
Punkty \(E=(7,1)\) i \(F=(9,7)\) to środki boków, odpowiednio \(A B\) i \(B C\) kwadratu \(A B C D\). Przekątna tego kwadratu ma długość\\
A. \(4 \sqrt{5}\)\\
B. 10\\
C. \(4 \sqrt{10}\)\\
D. 20

Zadanie 3. (0-1)\\
Liczba \(\left(\frac{3+\sqrt{3}}{\sqrt{3}}\right)^{2}\) jest równa\\
A. 4\\
B. 9\\
C. \(\frac{3+\sqrt{3}}{3}\)\\
D. \(4+2 \sqrt{3}\)

\section*{Zadanie 4. (0-1)}
Liczba \(3^{\frac{9}{4}}\) jest równa\\
A. \(3 \cdot \sqrt[4]{3}\)\\
B. \(9 \cdot \sqrt[4]{3}\)\\
C. \(27 \cdot \sqrt[4]{3}\)\\
D. \(3^{9} \cdot 3^{\frac{1}{4}}\)

Zadanie 5. (0-1)\\
Funkcja wykładnicza określona wzorem \(f(x)=3^{x}\) przyjmuje wartość 6 dla argumentu\\
A. \(x=2\)\\
B. \(x=\log _{3} 2\)\\
C. \(x=\log _{3} 6\)\\
D. \(x=\log _{6} 3\)

Zadanie 6. (0-1)\\
Wyrażenie \(16-(3 x+1)^{2}\) jest równe\\
A. \((3-3 x) \cdot(5+3 x)\)\\
B. \((15-3 x)^{2}\)\\
C. \((5-3 x) \cdot(5+3 x)\)\\
D. \(15-9 x^{2}\)\\
\includegraphics[max width=\textwidth, center]{2025_02_09_a1ef28b8910f95e806dcg-03}

\section*{Zadanie 7. (0-1)}
Wskaż równość prawdziwą.\\
A. \(-256^{2}=(-256)^{2}\)\\
B. \(256^{3}=(-256)^{3}\)\\
C. \(\sqrt{(-256)^{2}}=-256\)\\
D. \(\sqrt[3]{-256}=-\sqrt[3]{256}\)

Zadanie 8. (0-1)\\
Zbiorem rozwiązań nierówności \(\frac{2-x}{3}-\frac{2 x-1}{2}<x\) jest przedział\\
A. \(\left(-\infty, \frac{1}{2}\right)\)\\
B. \(\left(-\infty, \frac{1}{14}\right)\)\\
C. \(\left(\frac{1}{14},+\infty\right)\)\\
D. \(\left(\frac{1}{2},+\infty\right)\)

Zadanie 9. (0-1)\\
W klasie jest cztery razy więcej chłopców niż dziewcząt. Ile procent wszystkich uczniów tej klasy stanowią dziewczęta?\\
A. \(4 \%\)\\
B. 5\%\\
C. 20\%\\
D. \(25 \%\)

Zadanie 10. (0-1)\\
Reszta z dzielenia liczby 55 przez 8 jest równa\\
A. 4\\
B. 5\\
C. 6\\
D. 7

\section*{Zadanie 11. (0-1)}
Funkcja \(f\) przyporządkowuje każdej liczbie naturalnej większej od 1 jej największy dzielnik będący liczbą pierwszą. Spośród liczb: \(f(42), f(44), f(45), f(48)\) największa to\\
A. \(f(42)\)\\
B. \(f(44)\)\\
C. \(f(45)\)\\
D. \(f(48)\)

\section*{Zadanie 12. (0-1)}
Rysunek przedstawia ostrosłup prawidłowy czworokątny \(A B C D S\).\\
\includegraphics[max width=\textwidth, center]{2025_02_09_a1ef28b8910f95e806dcg-04}

Kątem między krawędzią \(C S\) a płaszczyzną podstawy tego ostrosłupa jest kąt\\
A. \(D C S\)\\
B. \(A C S\)\\
C. OSC\\
D. \(S C B\)\\
\includegraphics[max width=\textwidth, center]{2025_02_09_a1ef28b8910f95e806dcg-05}

Zadanie 13. (0-1)\\
Wykresem funkcji kwadratowej \(f\) jest parabola o wierzchołku \(W=(5,7)\). Wówczas prawdziwa jest równość\\
A. \(f(1)=f(9)\)\\
B. \(f(1)=f(11)\)\\
C. \(f(1)=f(13)\)\\
D. \(f(1)=f(15)\)

Zadanie 14. (0-1)\\
Jeżeli kąt \(\alpha\) jest ostry i \(\operatorname{tg} \alpha=\frac{3}{4}\), to \(\frac{2-\cos \alpha}{2+\cos \alpha}\) równa się\\
A. -1\\
B. \(-\frac{1}{3}\)\\
C. \(\frac{3}{7}\)\\
D. \(\frac{84}{25}\)

Zadanie 15. (0-1)\\
Równanie \((2 x-1) \cdot(x-2)=(1-2 x) \cdot(x+2)\) ma dwa rozwiązania. Są to liczby\\
A. -2 oraz \(\frac{1}{2}\)\\
B. 0 oraz \(\frac{1}{2}\)\\
C. \(\frac{1}{2}\) oraz 2\\
D. -2 oraz 2

\section*{Zadanie 16. (0-1)}
Dane jest równanie \(3 x+4 y-5=0\). Z którym z poniższych równań tworzy ono układ sprzeczny?\\
A. \(6 x+8 y-10=0\)\\
B. \(4 x-3 y+5=0\)\\
C. \(9 x+12 y-10=0\)\\
D. \(5 x+4 y-3=0\)

Zadanie 17. (0-1)\\
W trójkącie, przedstawionym na rysunku poniżej, sinus kąta ostrego \(\alpha\) jest równy\\
\includegraphics[max width=\textwidth, center]{2025_02_09_a1ef28b8910f95e806dcg-06}\\
A. \(\frac{1}{5}\)\\
B. \(\frac{\sqrt{6}}{12}\)\\
C. \(\frac{5}{24}\)\\
D. \(\frac{2 \sqrt{6}}{5}\)

\section*{BRUDNOPIS}
\begin{center}
\includegraphics[max width=\textwidth]{2025_02_09_a1ef28b8910f95e806dcg-07}
\end{center}

\section*{Zadanie 18. (0-1)}
Tworząca stożka ma długość \(l\), a promień jego podstawy jest równy \(r\) (zobacz rysunek).\\
\includegraphics[max width=\textwidth, center]{2025_02_09_a1ef28b8910f95e806dcg-08}

Powierzchnia boczna tego stożka jest 2 razy większa od pola jego podstawy. Wówczas\\
A. \(r=\frac{1}{6} l\)\\
B. \(r=\frac{1}{4} l\)\\
C. \(r=\frac{1}{3} l\)\\
D. \(r=\frac{1}{2} l\)

Zadanie 19. (0-1)\\
Dane są dwa okręgi o promieniach 10 i 15. Mniejszy okrag przechodzi przez środek większego okręgu. Odległość między środkami tych okręgów jest równa\\
A. 2,5\\
B. 5\\
C. 10\\
D. 12,5

Zadanie 20. (0-1)\\
Każdy uczestnik spotkania dwunastoosobowej grupy przyjaciół uścisnął dłoń każdemu z pozostałych członków tej grupy. Liczba wszystkich uścisków dłoni była równa\\
A. 66\\
B. 72\\
C. 132\\
D. 144

Zadanie 21. (0-1)\\
W dziewięciowyrazowym ciągu geometrycznym o wyrazach dodatnich pierwszy wyraz jest równy 3, a ostatni wyraz jest równy 12 . Piąty wyraz tego ciągu jest równy\\
A. \(3 \sqrt[4]{2}\)\\
B. 6\\
C. \(7 \frac{1}{2}\)\\
D. \(8 \frac{1}{7}\)

\section*{Zadanie 22. (0-1)}
Ciag \(\left(a_{n}\right)\) jest określony wzorem \(a_{n}=(n+3)(n-5)\) dla \(n \geq 1\). Liczba ujemnych wyrazów tego ciagu jest równa\\
A. 3\\
B. 4\\
C. 7\\
D. 9

Zadanie 23. (0-1)\\
Rzucamy jeden raz symetryczną sześcienną kostką do gry. Niech \(p_{i}\) oznacza prawdopodobieństwo wyrzucenia liczby oczek podzielnej przez \(i\). Wtedy\\
A. \(2 p_{4}=p_{2}\)\\
B. \(2 p_{6}=p_{3}\)\\
C. \(2 p_{3}=p_{6}\)\\
D. \(2 p_{2}=p_{4}\)\\
\includegraphics[max width=\textwidth, center]{2025_02_09_a1ef28b8910f95e806dcg-09}

\section*{ZADANIA OTWARTE}
Rozwiazania zadań 24-33 należy zapisać w wyznaczonych miejscach pod treścia zadania.

\section*{Zadanie 24. (0-2)}
Zbiorem rozwiązań nierówności \(a x+4 \geq 0\) z niewiadomą \(x\) jest przedział ( \(-\infty, 2\rangle\). Wyznacz \(a\).

\begin{center}
\begin{tabular}{|c|c|c|c|c|c|c|c|c|c|c|c|c|c|c|c|c|c|c|c|c|c|c|c|c|c|c|c|c|c|c|c|}
\hline
 &  &  &  &  &  &  &  &  &  &  &  &  &  &  &  &  &  &  &  &  &  &  &  &  &  &  &  &  &  &  &  \\
\hline
 &  &  &  &  &  &  &  &  &  &  &  &  &  &  &  &  &  &  &  &  &  &  &  &  &  &  &  &  &  &  &  \\
\hline
 &  &  &  &  &  &  &  &  &  &  &  &  &  &  &  &  &  &  &  &  &  &  &  &  &  &  &  &  &  &  &  \\
\hline
 &  &  &  &  &  &  &  &  &  &  &  &  &  &  &  &  &  &  &  &  &  &  &  &  &  &  &  &  &  &  &  \\
\hline
 &  &  &  &  &  &  &  &  &  &  &  &  &  &  &  &  &  &  &  &  &  &  &  &  &  &  &  &  &  &  &  \\
\hline
 &  &  &  &  &  &  &  &  &  &  &  &  &  &  &  &  &  &  &  &  &  &  &  &  &  &  &  &  &  &  &  \\
\hline
 &  &  &  &  &  &  &  &  &  &  &  &  &  &  &  &  &  &  &  &  &  &  &  &  &  &  &  &  &  &  &  \\
\hline
 &  &  &  &  &  &  &  &  &  &  &  &  &  &  &  &  &  &  &  &  &  &  &  &  &  &  &  &  &  &  &  \\
\hline
 &  &  &  &  &  &  &  &  &  &  &  &  &  &  &  &  &  &  &  &  &  &  &  &  &  &  &  &  &  &  &  \\
\hline
 &  &  &  &  &  &  &  &  &  &  &  &  &  &  &  &  &  &  &  &  &  &  &  &  &  &  &  &  &  &  &  \\
\hline
 &  &  &  &  &  &  &  &  &  &  &  &  &  &  &  &  &  &  &  &  &  &  &  &  &  &  &  &  &  &  &  \\
\hline
 &  &  &  &  &  &  &  &  &  &  &  &  &  &  &  &  &  &  &  &  &  &  &  &  &  &  &  &  &  &  &  \\
\hline
 &  &  &  &  &  &  &  &  &  &  &  &  &  &  &  &  &  &  &  &  &  &  &  &  &  &  &  &  &  &  &  \\
\hline
 &  &  &  &  &  &  &  &  &  &  &  &  &  &  &  &  &  &  &  &  &  &  &  &  &  &  &  &  &  &  &  \\
\hline
 &  &  &  &  &  &  &  &  &  &  &  &  &  &  &  &  &  &  &  &  &  &  &  &  &  &  &  &  &  &  &  \\
\hline
 &  &  &  &  &  &  &  &  &  &  &  &  &  &  &  &  &  &  &  &  &  &  &  &  &  &  &  &  &  &  &  \\
\hline
 &  &  &  &  &  &  &  &  &  &  &  &  &  &  &  &  &  &  &  &  &  &  &  &  &  &  &  &  &  &  &  \\
\hline
 &  &  &  &  &  &  &  &  &  &  &  &  &  &  &  &  &  &  &  &  &  &  &  &  &  &  &  &  &  &  &  \\
\hline
\end{tabular}
\end{center}

\section*{Zadanie 25. (0-2)}
Rozwiąż równanie \(\frac{x(x+1)}{x-1}=5 x-4\), dla \(x \neq 1\).\\
\includegraphics[max width=\textwidth, center]{2025_02_09_a1ef28b8910f95e806dcg-10}

Zadanie 26. (0-2)\\
Kwadrat \(K_{1}\) ma bok długości \(a\). Obok niego rysujemy kolejno kwadraty \(K_{2}, K_{3}, K_{4}, \ldots\) takie, że kolejny kwadrat ma bok o połowę mniejszy od boku poprzedniego kwadratu (zobacz rysunek).\\
\includegraphics[max width=\textwidth, center]{2025_02_09_a1ef28b8910f95e806dcg-11}

Wyznacz pole kwadratu \(K_{12}\).\\
\includegraphics[max width=\textwidth, center]{2025_02_09_a1ef28b8910f95e806dcg-11(1)}

\section*{Zadanie 27. (0-2)}
W pierścieniu kołowym cięciwa zewnętrznego okręgu ma długość 10 i jest styczna do wewnętrznego okręgu (zobacz rysunek).\\
\includegraphics[max width=\textwidth, center]{2025_02_09_a1ef28b8910f95e806dcg-12}

Wykaż, że pole tego pierścienia można wyrazić wzorem, w którym nie występują promienie wyznaczających go okręgów.\\
\includegraphics[max width=\textwidth, center]{2025_02_09_a1ef28b8910f95e806dcg-12(2)}

\section*{Zadanie 28. (0-2)}
Uzasadnij, że liczba \(4^{12}+4^{13}+4^{14}\) jest podzielna przez 42 .

\begin{center}
\begin{tabular}{|c|c|c|c|c|c|c|c|c|c|c|c|c|c|c|c|c|c|c|c|c|c|c|c|c|c|c|c|c|c|c|}
\hline
 &  &  &  &  &  &  &  &  &  &  &  &  &  &  &  &  &  &  &  &  &  &  &  &  &  &  &  &  &  &  \\
\hline
 &  &  &  &  &  &  &  &  &  &  &  &  &  &  &  &  &  &  &  &  &  &  &  &  &  &  &  &  &  &  \\
\hline
 &  &  &  &  &  &  &  &  &  &  &  &  &  &  &  &  &  &  &  &  &  &  &  &  &  &  &  &  &  &  \\
\hline
 &  &  &  &  &  &  &  &  &  &  &  &  &  &  &  &  &  &  &  &  &  &  &  &  &  &  &  &  &  &  \\
\hline
 &  &  &  &  &  &  &  &  &  &  &  &  &  &  &  &  &  &  &  &  &  &  &  &  &  &  &  &  &  &  \\
\hline
 &  &  &  &  &  &  &  &  &  &  &  &  &  &  &  &  &  &  &  &  &  &  &  &  &  &  &  &  &  &  \\
\hline
 &  &  &  &  &  &  &  &  &  &  &  &  &  &  &  &  &  &  &  &  &  &  &  &  &  &  &  &  &  &  \\
\hline
 &  &  &  &  &  &  &  &  &  &  &  &  &  &  &  &  &  &  &  &  &  &  &  &  &  &  &  &  &  &  \\
\hline
 &  &  &  &  &  &  &  &  &  &  &  &  &  &  &  &  &  &  &  &  &  &  &  &  &  &  &  &  &  &  \\
\hline
 &  &  &  &  &  &  &  &  &  &  &  &  &  &  &  &  &  &  &  &  &  &  &  &  &  &  &  &  &  &  \\
\hline
 &  &  &  &  &  &  &  &  &  &  &  &  &  &  &  &  &  &  &  &  &  &  &  &  &  &  &  &  &  &  \\
\hline
 &  &  &  &  &  &  &  &  &  &  &  &  &  &  &  &  &  &  &  &  &  &  &  &  &  &  &  &  &  &  \\
\hline
 &  &  &  &  &  &  &  &  &  &  &  &  &  &  &  &  &  &  &  &  &  &  &  &  &  &  &  &  &  &  \\
\hline
 &  &  &  &  &  &  &  &  &  &  &  &  &  &  &  &  &  &  &  &  &  &  &  &  &  &  &  & \includegraphics[max width=\textwidth]{2025_02_09_a1ef28b8910f95e806dcg-12(1)}
 &  &  \\
\hline
 &  &  &  &  &  &  &  &  &  &  &  &  &  &  &  &  &  &  &  &  &  &  &  &  &  &  &  &  &  &  \\
\hline
 &  &  &  &  &  &  &  &  &  &  &  &  &  &  &  &  &  &  &  &  &  &  &  &  &  &  &  &  &  &  \\
\hline
\end{tabular}
\end{center}

Zadanie 29. (0-2)\\
Na trójkącie o bokach długości \(\sqrt{7}, \sqrt{8}, \sqrt{15}\) opisano okrag. Oblicz promień tego okreguu.

\begin{center}
\begin{tabular}{|c|c|c|c|c|c|c|c|c|c|c|c|c|c|c|c|c|c|c|c|c|c|c|c|c|c|c|c|c|c|c|}
\hline
 &  &  &  &  &  &  &  &  &  &  &  &  &  &  &  &  &  &  &  &  &  &  &  &  &  &  &  &  &  &  \\
\hline
 &  &  &  &  &  &  &  &  &  &  &  &  &  &  &  &  &  &  &  &  &  &  &  &  &  &  &  &  &  &  \\
\hline
 &  &  &  &  &  &  &  &  &  &  &  &  &  &  &  &  &  &  &  &  &  &  &  &  &  &  &  &  &  &  \\
\hline
 &  &  &  &  &  &  &  &  &  &  &  &  &  &  &  &  &  &  &  &  &  &  &  &  &  &  &  &  &  &  \\
\hline
 &  &  &  &  &  &  &  &  &  &  &  &  &  &  &  &  &  &  &  &  &  &  &  &  &  &  &  &  &  &  \\
\hline
 &  &  &  &  &  &  &  &  &  &  &  &  &  &  &  &  &  &  &  &  &  &  &  &  &  &  &  &  &  &  \\
\hline
 &  &  &  &  &  &  &  &  &  &  &  &  &  &  &  &  &  &  &  &  &  &  &  &  &  &  &  &  &  &  \\
\hline
 &  &  &  &  &  &  &  &  &  &  &  &  &  &  &  &  &  &  &  &  &  &  &  &  &  &  &  &  &  &  \\
\hline
 &  &  &  &  &  &  &  &  &  &  &  &  &  &  &  &  &  &  &  &  &  &  &  &  &  &  &  &  &  &  \\
\hline
 &  &  &  &  &  &  &  &  &  &  &  &  &  &  &  &  &  &  &  &  &  &  &  &  &  &  &  &  &  &  \\
\hline
 &  &  &  &  &  &  &  &  &  &  &  &  &  &  &  &  &  &  &  &  &  &  &  &  &  &  &  &  &  &  \\
\hline
 &  &  &  &  &  &  &  &  &  &  &  &  &  &  &  &  &  &  &  &  &  &  &  &  &  &  &  &  &  &  \\
\hline
 &  &  &  &  &  &  &  &  &  &  &  &  &  &  &  &  &  &  &  &  &  &  &  &  &  &  &  &  &  &  \\
\hline
 &  &  &  &  &  &  &  &  &  &  &  &  &  &  &  &  &  &  &  &  &  &  &  &  &  &  &  &  &  &  \\
\hline
 &  &  &  &  &  &  &  &  &  &  &  &  &  &  &  &  &  &  &  &  &  &  &  &  &  &  &  &  &  &  \\
\hline
 &  &  &  &  &  &  &  &  &  &  &  &  &  &  &  &  &  &  &  &  &  &  &  &  &  &  &  &  &  &  \\
\hline
 &  &  &  &  &  &  &  &  &  &  &  &  &  &  &  &  &  &  &  &  &  &  &  &  &  &  & \includegraphics[max width=\textwidth]{2025_02_09_a1ef28b8910f95e806dcg-13(1)}
 &  &  &  \\
\hline
 &  &  &  &  &  &  &  &  &  &  &  &  &  &  &  &  &  &  &  &  &  &  &  &  &  &  & \includegraphics[max width=\textwidth]{2025_02_09_a1ef28b8910f95e806dcg-13}
 &  &  &  \\
\hline
 &  &  &  &  &  &  &  &  &  &  &  &  &  &  &  &  &  &  &  &  &  &  &  &  &  &  &  &  &  &  \\
\hline
\end{tabular}
\end{center}

Zadanie 30. (0-2)\\
Proste \(l\) i \(k\) przecinają się w punkcie \(A=(0,4)\). Prosta \(l\) wyznacza wraz z dodatnimi półosiami układu współrzędnych trójkąt o polu 8 , zaś prosta \(k\) - trójkąt o polu 10 . Oblicz pole trójkąta, którego wierzchołkami są: punkt \(A\) oraz punkty przecięcia prostych \(l\) i \(k\) z osią \(O x\).\\
\includegraphics[max width=\textwidth, center]{2025_02_09_a1ef28b8910f95e806dcg-13(2)}

\section*{Zadanie 31. (0-4)}
Ala jeździ do szkoły rowerem, a Ola skuterem. Obie pokonują tę samą drogę. Ala wyjechała do szkoły o godzinie 7:00 i pokonała całą drogę w ciągu 40 minut. Ola wyjechała 10 minut później niż Ala, a pokonanie całej drogi zajęło jej tylko 20 minut. Oblicz, o której godzinie Ola wyprzedziła Alę.\\
\includegraphics[max width=\textwidth, center]{2025_02_09_a1ef28b8910f95e806dcg-14}

Zadanie 32. (0-5)\\
Dane są wierzchołki trójkąta \(A B C: A=(2,2), B=(9,5)\) i \(C=(3,9)\). Z wierzchołka \(C\) poprowadzono wysokość tego trójkąta, która przecina bok \(A B\) w punkcie \(D\). Wyznacz równanie prostej przechodzącej przez punkt \(D\) i równoległej do boku \(B C\).\\
\includegraphics[max width=\textwidth, center]{2025_02_09_a1ef28b8910f95e806dcg-15}

\section*{Zadanie 33. (0-4)}
Jacek bawi się sześciennymi klockami o krawędzi 2 cm . Zbudował z nich jeden duży sześcian o krawędzi 8 cm i wykorzystał do tego wszystkie swoje klocki. Następnie zburzył budowlę i ułożył z tych klocków drugą bryłe - graniastosłup prawidłowy czworokątny. Wtedy okazało się, że został mu dokładnie jeden klocek, którego nie było gdzie dołożyć. Oblicz stosunek pola powierzchni całkowitej pierwszej ułożonej bryły do pola powierzchni całkowitej drugiej bryły i wynik podaj w postaci ułamka nieskracalnego.\\
\includegraphics[max width=\textwidth, center]{2025_02_09_a1ef28b8910f95e806dcg-16}

\section*{BRUDNOPIS}

\end{document}