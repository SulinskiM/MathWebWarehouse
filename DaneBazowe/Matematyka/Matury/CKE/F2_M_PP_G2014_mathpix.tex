\documentclass[10pt]{article}
\usepackage[polish]{babel}
\usepackage[utf8]{inputenc}
\usepackage[T1]{fontenc}
\usepackage{amsmath}
\usepackage{amsfonts}
\usepackage{amssymb}
\usepackage[version=4]{mhchem}
\usepackage{stmaryrd}
\usepackage{graphicx}
\usepackage[export]{adjustbox}
\graphicspath{ {./images/} }
\usepackage{multirow}

\author{DATA: 16 grudnia 2014 r. Czas pracy: \(\mathbf{1 7 0}\) minut\\
Liczba punktów do uzyskania: 50}
\date{}


\begin{document}
\maketitle
\section*{EGZAMIN MATURALNY Z MATEMATYKI PoZIOM PODSTAWOWY}
\section*{PrZYKLADOWY ARKUSZ EGZAMINACYJNY}


\section*{Instrukcja dla zdającego}
\begin{enumerate}
  \item Sprawdź, czy arkusz egzaminacyjny zawiera 23 strony (zadania 1-33). Ewentualny brak zgłoś przewodniczącemu zespołu nadzorującego egzamin.
  \item Rozwiązania zadań i odpowiedzi wpisuj w miejscu na to przeznaczonym.
  \item Odpowiedzi do zadań zamkniętych (1-24) przenieś na kartę odpowiedzi, zaznaczając je w części karty przeznaczonej dla zdającego. Zamaluj pola do tego przeznaczone. Błędne zaznaczenie otocz kółkiem i zaznacz właściwe.
  \item Pamiętaj, że pominięcie argumentacji lub istotnych obliczeń w rozwiązaniu zadania otwartego (25-33) może spowodować, że za to rozwiązanie nie otrzymasz pełnej liczby punktów.
  \item Pisz czytelnie i używaj tylko długopisu lub pióra z czarnym tuszem lub atramentem.
  \item Nie używaj korektora, a błędne zapisy wyraźnie przekreśl.
  \item Pamiętaj, że zapisy w brudnopisie nie będą oceniane.
  \item Możesz korzystać z zestawu wzorów matematycznych, cyrkla i linijki oraz kalkulatora prostego.
  \item Na tej stronie oraz na karcie odpowiedzi wpisz swój numer PESEL i przyklej naklejkę z kodem.
  \item Nie wpisuj żadnych znaków w części przeznaczonej dla egzaminatora.
\end{enumerate}

W zadaniach 1.-24. wybierz i zaznacz na karcie odpowiedzi jedna poprawna odpowiedź.

\section*{Zadanie 1. (0-1)}
Liczba 0,6 jest jednym z przybliżeń liczby \(\frac{5}{8}\). Błąd względny tego przybliżenia, wyrażony w procentach, jest równy\\
A. \(0,025 \%\)\\
B. \(2,5 \%\)\\
C. \(0,04 \%\)\\
D. \(4 \%\)

\section*{Zadanie 2. (0-1)}
Dany jest okrąg o środku \(S=(-6,-8)\) i promieniu 2014. Obrazem tego okręgu w symetrii osiowej względem osi \(O y\) jest okrąg o środku w punkcie \(S_{1}\). Odległość między punktami \(S\) i \(S_{1}\) jest równa\\
A. 12\\
B. 16\\
C. 2014\\
D. 4028

\section*{Zadanie 3. (0-1)}
Rozwiązaniami równania \(\left(x^{3}-8\right)(x-5)(2 x+1)=0\) są liczby\\
A. \(-8 ;-5 ; 1\)\\
B. \(-1 ; 5 ; 8\)\\
C. \(-\frac{1}{2} ; 2 ; 5\)\\
D. \(-\frac{1}{2} ; 5 ; 8\)

\section*{Zadanie 4. (0-1)}
Cena towaru została podwyższona o \(30 \%\), a po pewnym czasie nową, wyższą cenę ponownie podwyższono, tym razem o \(10 \%\). W rezultacie obu podwyżek wyjściowa cena towaru zwiększyła się o\\
A. \(15 \%\)\\
B. \(20 \%\)\\
C. \(40 \%\)\\
D. \(43 \%\)

\section*{Zadanie 5. (0-1)}
Dane są dwie funkcje określone dla wszystkich liczb rzeczywistych \(x\) wzorami \(f(x)=-5 x+1\) oraz \(g(x)=5^{x}\). Liczba punktów wspólnych wykresów tych funkcji jest równa\\
A. 3\\
B. 2\\
C. 1\\
D. 0

\section*{Zadanie 6. (0-1)}
Wyrażenie \((3 x+1+y)^{2}\) jest równe\\
A. \(3 x^{2}+y^{2}+1\)\\
B. \(9 x^{2}+6 x+y^{2}+1\)\\
C. \(3 x^{2}+y^{2}+6 x y+6 x+1\)\\
D. \(9 x^{2}+y^{2}+6 x y+6 x+2 y+1\)

\section*{Zadanie 7. (0-1)}
Połowa sumy \(4^{28}+4^{28}+4^{28}+4^{28}\) jest równa\\
A. \(2^{30}\)\\
B. \(2^{57}\)\\
C. \(2^{63}\)\\
D. \(2^{112}\)

\section*{Zadanie 8. (0-1)}
Równania \(y=-\frac{3}{4} x+\frac{5}{4}\) oraz \(y=-\frac{4}{3}\) opisują dwie proste\\
A. przecinające się pod kątem o mierze \(90^{\circ}\).\\
B. pokrywające się.\\
C. przecinające się pod kątem różnym od \(90^{\circ}\).\\
D. równoległe i różne.

\section*{Zadanie 9. (0-1)}
Na płaszczyźnie dane są punkty: \(A=(\sqrt{2}, \sqrt{6}), B=(0,0)\) i \(C=(\sqrt{2}, 0)\). Kąt \(B A C\) jest równy\\
A. \(30^{\circ}\)\\
B. \(45^{\circ}\)\\
C. \(60^{\circ}\)\\
D. \(75^{\circ}\)

\section*{Zadanie 10. (0-1)}
Funkcja \(f\), określona dla wszystkich liczb całkowitych dodatnich, przyporządkowuje liczbie \(x\) ostatnią cyfrę jej kwadratu. Zbiór wartości funkcji \(f\) zawiera dokładnie\\
A. 5 elementów.\\
B. 6 elementów.\\
C. 9 elementów.\\
D. 10 elementów.

\section*{Zadanie 11. (0-1)}
Ekipa złożona z 25 pracowników wymieniła tory kolejowe na pewnym odcinku w ciągu 156 dni. Jeśli wymianę torów kolejowych na kolejnym odcinku o tej samej długości trzeba przeprowadzić w ciągu 100 dni, to, przy założeniu takiej samej wydajności, należy zatrudnić do pracy o\\
A. 14 osób więcej.\\
B. 17 osób więcej.\\
C. 25 osób więcej.\\
D. 39 osób więcej.

\section*{Zadanie 12. (0-1)}
Z sześcianu \(A B C D E F G H\) o krawędzi długości \(a\) odcięto ostrosłup \(A B D E\) (zobacz rysunek).\\
\includegraphics[max width=\textwidth, center]{2025_02_09_e5777ff444f0f25ff8deg-06(1)}

Ile razy objętość tego ostrosłupa jest mniejsza od objętości pozostałej części sześcianu?\\
A. 2 razy.\\
B. 3 razy.\\
C. 4 razy.\\
D. 5 razy.

\section*{Zadanie 13. (0-1)}
W układzie współrzędnych narysowano część paraboli o wierzchołku w punkcie \(A=(2,4)\), która jest wykresem funkcji kwadratowej \(f\).\\
\includegraphics[max width=\textwidth, center]{2025_02_09_e5777ff444f0f25ff8deg-06}

Funkcja \(f\) może być opisana wzorem\\
A. \(f(x)=(x-2)^{2}+4\)\\
B. \(f(x)=(x+2)^{2}+4\)\\
C. \(f(x)=-(x-2)^{2}+4\)\\
D. \(f(x)=-(x+2)^{2}+4\)

\section*{Zadanie 14. (0-1)}
Punkty \(A=(-6-2 \sqrt{2}, 4-2 \sqrt{2}), \quad B=(2+4 \sqrt{2},-6 \sqrt{2}), \quad C=(2+6 \sqrt{2}, 6-2 \sqrt{2})\) są kolejnymi wierzchołkami równoległoboku \(A B C D\). Przekątne tego równoległoboku przecinają się w punkcie\\
A. \(S=(-1+4 \sqrt{2}, 5-5 \sqrt{2})\)\\
B. \(S=(-2+\sqrt{2}, 2-4 \sqrt{2})\)\\
C. \(S=(2+5 \sqrt{2}, 3-4 \sqrt{2})\)\\
D. \(S=(-2+2 \sqrt{2}, 5-2 \sqrt{2})\)

\section*{Zadanie 15. (0-1)}
Liczba \(\sin 150^{\circ}\) jest równa liczbie\\
A. \(\cos 60^{\circ}\)\\
B. \(\cos 120^{\circ}\)\\
C. \(\operatorname{tg} 120^{\circ}\)\\
D. \(\operatorname{tg} 60^{\circ}\)

\section*{Zadanie 16. (0-1)}
Na ścianie kamienicy zaprojektowano mural utworzony z szeregu trójkątów równobocznych różnej wielkości. Najmniejszy trójkąt ma bok długości 1 m , a bok każdego z następnych trójkątów jest o 10 cm dłuższy niż bok poprzedzającego go trójkąta. Ostatni trójkąt ma bok długości 5,9 m. Ile trójkątów przedstawia mural?\\
A. 49\\
B. 50\\
C. 59\\
D. 60

\section*{Zadanie 17. (0-1)}
Dany jest trójkąt równoramienny, w którym ramię o długości 20 tworzy z podstawą kąt \(67,5^{\circ}\). Pole tego trójkąta jest równe\\
A. \(100 \sqrt{3}\)\\
B. \(100 \sqrt{2}\)\\
C. \(200 \sqrt{3}\)\\
D. \(200 \sqrt{2}\)

\section*{Zadanie 18. (0-1)}
Na rysunkach poniżej przedstawiono siatki dwóch ostrosłupów.\\
\includegraphics[max width=\textwidth, center]{2025_02_09_e5777ff444f0f25ff8deg-10(1)}

Pole powierzchni całkowitej ostrosłupa o krawędzi \(a\) jest dwa razy większe od pola powierzchni całkowitej ostrosłupa o krawędzi \(b\). Ile razy objętość ostrosłupa o krawędzi \(a\) jest większa od objętości ostrosłupa o krawędzi \(b\) ?\\
A. \(\sqrt{2}\)\\
B. 2\\
C. \(2 \sqrt{2}\)\\
D. 4

\section*{Zadanie 19. (0-1)}
Na okręgu o środku \(S\) leżą punkty \(A, B\), \(C\) i \(D\). Odcinek \(A B\) jest średnicą tego okręgu. Kąt między tą średnicą a cięciwą \(A C\) jest równy \(21^{\circ}\) (zobacz rysunek).\\
\includegraphics[max width=\textwidth, center]{2025_02_09_e5777ff444f0f25ff8deg-10}

Kąt \(\alpha\) między cięciwami \(A D\) i \(C D\) jest równy\\
A. \(21^{\circ}\)\\
B. \(42^{\circ}\)\\
C. \(48^{\circ}\)\\
D. \(69^{\circ}\)

\section*{Zadanie 20. (0-1)}
Średnia arytmetyczna zestawu danych: \(3,8,3,11,3,10,3, x\) jest równa 6 . Mediana tego zestawu jest równa\\
A. 5\\
B. 6\\
C. 7\\
D. 8

\section*{Zadanie 21. (0-1)}
Dany jest ciąg geometryczny \(\left(a_{n}\right)\), w którym \(a_{1}=-\sqrt{2}, a_{2}=2, a_{3}=-2 \sqrt{2}\). Dziesiąty wyraz tego ciągu, czyli \(a_{10}\), jest równy\\
A. 32\\
B. -32\\
C. \(16 \sqrt{2}\)\\
D. \(-16 \sqrt{2}\)

\section*{Zadanie 22. (0-1)}
Ciąg \(\left(a_{n}\right)\) jest określony wzorem \(a_{n}=\frac{24-4 n}{n}\) dla \(n \geq 1\). Liczba wszystkich całkowitych nieujemnych wyrazów tego ciągu jest równa\\
A. 7\\
B. 6\\
C. 5\\
D. 4

\section*{Zadanie 23. (0-1)}
Rzucamy sześć razy symetryczną sześcienną kostką do gry. Niech \(p_{i}\) oznacza prawdopodobieństwo wyrzucenia \(i\) oczek w \(i\)-tym rzucie. Wtedy\\
A. \(p_{6}=1\)\\
B. \(p_{6}=\frac{1}{6}\)\\
C. \(p_{3}=0\)\\
D. \(p_{3}=\frac{1}{3}\)

\section*{Zadanie 24. (0-1)}
Wskaż liczbę, która spełnia równanie \(4^{x}=9\).\\
A. \(\log 9-\log 4\)\\
B. \(\frac{\log 2}{\log 3}\)\\
C. \(2 \log _{9} 2\)\\
D. \(2 \log _{4} 3\)

Rozwiązania zadań 25.-33. należy zapisać \(w\) wyznaczonych miejscach pod treścia zadania.

\section*{Zadanie 25. (0-2)}
Rozwiąż nierówność: \(-x^{2}-4 x+21<0\).\\
\includegraphics[max width=\textwidth, center]{2025_02_09_e5777ff444f0f25ff8deg-14}

Odpowiedź:

Zadanie 26. (0-2)\\
Uzasadnij, że żadna liczba całkowita nie jest rozwiązaniem równania \(\frac{2 x+4}{x-2}=2 x+1\).

\begin{center}
\begin{tabular}{|c|c|c|c|c|c|c|c|c|c|c|c|c|c|c|c|c|c|c|c|c|c|c|c|c|c|c|c|c|}
\hline
 &  &  &  &  &  &  &  &  &  &  &  &  &  &  &  &  &  &  &  &  &  &  &  &  &  &  &  &  \\
\hline
 &  &  &  &  &  &  &  &  &  &  &  &  &  &  &  &  &  &  &  &  &  &  &  &  &  &  &  &  \\
\hline
 &  &  &  &  &  &  &  &  &  &  &  &  &  &  &  &  &  &  &  &  &  &  &  &  &  &  &  &  \\
\hline
 &  &  &  &  &  &  &  &  &  &  &  &  &  &  &  &  &  &  &  &  &  &  &  &  &  &  &  &  \\
\hline
 &  &  &  &  &  &  &  &  &  &  &  &  &  &  &  &  &  &  &  &  &  &  &  &  &  &  &  &  \\
\hline
 &  &  &  &  &  &  &  &  &  &  &  &  &  &  &  &  &  &  &  &  &  &  &  &  &  &  &  &  \\
\hline
 &  &  &  &  &  &  &  &  &  &  &  &  &  &  &  &  &  &  &  &  &  &  &  &  &  &  &  &  \\
\hline
 &  &  &  &  &  &  &  &  &  &  &  &  &  &  &  &  &  &  &  &  &  &  &  &  &  &  &  &  \\
\hline
 &  &  &  &  &  &  &  &  &  &  &  &  &  &  &  &  &  &  &  &  &  &  &  &  &  &  &  &  \\
\hline
 &  &  &  &  &  &  &  &  &  &  &  &  &  &  &  &  &  &  &  &  &  &  &  &  &  &  &  &  \\
\hline
 &  &  &  &  &  &  &  &  &  &  &  &  &  &  &  &  &  &  &  &  &  &  &  &  &  &  &  &  \\
\hline
 &  &  &  &  &  &  &  &  &  &  &  &  &  &  &  &  &  &  &  &  &  &  &  &  &  &  &  &  \\
\hline
 &  &  &  &  &  &  &  &  &  &  &  &  &  &  &  &  &  &  &  &  &  &  &  &  &  &  &  &  \\
\hline
 &  &  &  &  &  &  &  &  &  &  &  &  &  &  &  &  &  &  &  &  &  &  &  &  &  &  &  &  \\
\hline
 &  &  &  &  &  &  &  &  &  &  &  &  &  &  &  &  &  &  &  &  &  &  &  &  &  &  &  &  \\
\hline
 &  &  &  &  &  &  &  &  &  &  &  &  &  &  &  &  &  &  &  &  &  &  &  &  &  &  &  &  \\
\hline
 &  &  &  &  &  &  &  &  &  &  &  &  &  &  &  &  &  &  &  &  &  &  &  &  &  &  &  &  \\
\hline
 &  &  &  &  &  &  &  &  &  &  &  &  &  &  &  &  &  &  &  &  &  &  &  &  &  &  &  &  \\
\hline
 &  &  &  &  &  &  &  &  &  &  &  &  &  &  &  &  &  &  &  &  &  &  &  &  &  &  &  &  \\
\hline
 &  &  &  &  &  &  &  &  &  &  &  &  &  &  &  &  &  &  &  &  &  &  &  &  &  &  &  &  \\
\hline
 &  &  &  &  &  &  &  &  &  &  &  &  &  &  &  &  &  &  &  &  &  &  &  &  &  &  &  &  \\
\hline
 &  &  &  &  &  &  &  &  &  &  &  &  &  &  &  &  &  &  &  &  &  &  &  &  &  &  &  &  \\
\hline
 &  &  &  &  &  &  &  &  &  &  &  &  &  &  &  &  &  &  &  &  &  &  &  &  &  &  &  &  \\
\hline
 &  &  &  &  &  &  &  &  &  &  &  &  &  &  &  &  &  &  &  &  &  &  &  &  &  &  &  &  \\
\hline
 &  &  &  &  &  &  &  &  &  &  &  &  &  &  &  &  &  &  &  &  &  &  &  &  &  &  &  &  \\
\hline
 &  &  &  &  &  &  &  &  &  &  &  &  &  &  &  &  &  &  &  &  &  &  &  &  &  &  &  &  \\
\hline
 &  &  &  &  &  &  &  &  &  &  &  &  &  &  &  &  &  &  &  &  &  &  &  &  &  &  &  &  \\
\hline
 &  &  &  &  &  &  &  &  &  &  &  &  &  &  &  &  &  &  &  &  &  &  &  &  &  &  &  &  \\
\hline
 &  &  &  &  &  &  &  &  &  & - &  &  &  &  &  &  &  &  &  &  &  &  &  &  &  &  &  &  \\
\hline
 &  &  &  &  &  &  &  &  &  &  &  &  &  &  &  &  &  &  &  &  &  &  &  &  &  &  &  &  \\
\hline
 &  &  &  &  &  &  &  &  &  &  &  &  &  &  &  &  &  &  &  &  &  &  &  &  &  &  &  &  \\
\hline
 &  &  &  &  &  &  &  &  &  &  &  &  &  &  &  &  &  &  &  &  &  &  &  &  &  &  &  &  \\
\hline
 &  &  &  &  &  &  &  &  &  &  &  &  &  &  &  &  &  &  &  &  &  &  &  &  &  &  &  &  \\
\hline
 &  &  &  &  &  &  &  &  &  &  &  &  &  &  &  &  &  &  &  &  &  &  &  &  &  &  &  &  \\
\hline
 &  &  &  &  &  &  &  &  &  &  &  &  &  &  &  &  &  &  &  &  &  &  &  &  &  &  &  &  \\
\hline
 &  &  &  &  &  &  &  &  &  &  &  &  &  &  &  &  &  &  &  &  &  &  &  &  &  &  &  &  \\
\hline
 &  &  &  &  &  &  &  &  &  &  &  &  &  &  &  &  &  &  &  &  &  &  &  &  &  &  &  &  \\
\hline
 &  &  &  &  &  &  &  &  &  &  &  &  &  &  &  &  &  &  &  &  &  &  &  &  &  &  &  &  \\
\hline
 &  &  &  &  &  &  &  &  &  &  &  &  &  &  &  &  &  &  &  &  &  &  &  &  &  &  &  &  \\
\hline
 &  &  &  &  &  &  &  &  &  &  &  &  &  &  &  &  &  &  &  &  &  &  &  &  &  &  &  &  \\
\hline
\end{tabular}
\end{center}

\begin{center}
\begin{tabular}{|c|l|c|c|}
\hline
\multirow{3}{*}{\begin{tabular}{c}
Wypełnia \\
egzaminator \\
\end{tabular}} & Nr zadania & 25. & 26. \\
\cline { 2 - 4 }
 & Maks. liczba pkt & 2 & 2 \\
\cline { 2 - 4 }
 & Uzyskana liczba pkt &  &  \\
\hline
\end{tabular}
\end{center}

Strona 15 z 23

\section*{Zadanie 27. (0-2)}
Czas połowicznego rozpadu pierwiastka to okres, jaki jest potrzebny, by ze \(100 \%\) pierwiastka pozostało \(50 \%\) tego pierwiastka. Oznacza to, że ilość pierwiastka pozostała z każdego grama pierwiastka po \(x\) okresach rozpadu połowicznego wyraża się wzorem \(y=\left(\frac{1}{2}\right)^{x}\).\\
W przypadku izotopu jodu \({ }^{131}\) I czas połowicznego rozpadu jest równy 8 dni. Wyznacz najmniejszą liczbę dni, po upływie których pozostanie z \(1 \mathrm{~g}{ }^{131} \mathrm{I}\) nie więcej niż \(0,125 \mathrm{~g}\) tego pierwiastka.

\begin{center}
\begin{tabular}{|c|c|c|c|c|c|c|c|c|c|c|c|c|c|c|c|c|c|c|c|c|c|}
\hline
 &  &  &  &  &  &  &  &  &  &  &  &  &  &  &  &  &  &  &  &  &  \\
\hline
 &  &  &  &  &  &  &  &  &  &  &  &  &  &  &  &  &  &  &  &  &  \\
\hline
 &  &  &  &  &  &  &  &  &  &  &  &  &  &  &  &  &  &  &  &  &  \\
\hline
 &  &  &  &  &  &  &  &  &  &  &  &  &  &  &  &  &  &  &  &  &  \\
\hline
 &  &  &  &  &  &  &  &  &  &  &  &  &  &  &  &  &  &  &  &  &  \\
\hline
 &  &  &  &  &  &  &  &  &  &  &  &  &  &  &  &  &  &  &  &  &  \\
\hline
 &  &  &  &  &  &  &  &  &  &  &  &  &  &  &  &  &  &  &  &  &  \\
\hline
 &  &  &  &  &  &  &  &  &  &  &  &  &  &  &  &  &  &  &  &  &  \\
\hline
 &  &  &  &  &  &  &  &  &  &  &  &  &  &  &  &  &  &  &  &  &  \\
\hline
 &  &  &  &  &  &  &  &  &  &  &  &  &  &  &  &  &  &  &  &  &  \\
\hline
 &  &  &  &  &  &  &  &  &  &  &  &  &  &  &  &  &  &  &  &  &  \\
\hline
 &  &  &  &  &  &  &  &  &  &  &  &  &  &  &  &  &  &  &  &  &  \\
\hline
 &  &  &  &  &  &  &  &  &  &  &  &  &  &  &  &  &  &  &  &  &  \\
\hline
 &  &  &  &  &  &  &  &  &  &  &  &  &  &  &  &  &  &  &  &  &  \\
\hline
 &  &  &  &  &  &  &  &  &  &  &  &  &  &  &  &  &  &  &  &  &  \\
\hline
 &  &  &  &  &  &  &  &  &  &  &  &  &  &  &  &  &  &  &  &  &  \\
\hline
 &  &  &  &  &  &  &  &  &  &  &  &  &  &  &  &  &  &  &  &  &  \\
\hline
 &  &  &  &  &  &  &  &  &  &  &  &  &  &  &  &  &  &  &  &  &  \\
\hline
 &  &  &  &  &  &  &  &  &  &  &  &  &  &  &  &  &  &  &  &  &  \\
\hline
 &  &  &  &  &  &  &  &  &  &  &  &  &  &  &  &  &  &  &  &  &  \\
\hline
 &  &  &  &  &  &  &  &  &  &  &  &  &  &  &  &  &  &  &  &  &  \\
\hline
 &  &  &  &  &  &  &  &  &  &  &  &  &  &  &  &  &  &  &  &  &  \\
\hline
 &  &  &  &  &  &  &  &  &  &  &  &  &  &  &  &  &  &  &  &  &  \\
\hline
 &  &  &  &  &  &  &  &  &  &  &  &  &  &  &  &  &  &  &  &  &  \\
\hline
 &  &  &  &  &  &  &  &  &  &  &  &  &  &  &  &  &  &  &  &  &  \\
\hline
 &  &  &  &  &  &  &  &  &  &  &  &  &  &  &  &  &  &  &  &  &  \\
\hline
 &  &  &  &  &  &  &  &  &  &  &  &  &  &  &  &  &  &  &  &  &  \\
\hline
 &  &  &  &  &  &  &  &  &  &  &  &  &  &  &  &  &  &  &  &  &  \\
\hline
 &  &  &  &  &  &  &  &  &  &  &  &  &  &  &  &  &  &  &  &  &  \\
\hline
 &  &  &  &  &  &  &  &  &  &  &  &  &  &  &  &  &  &  &  &  &  \\
\hline
 &  &  &  &  &  &  &  &  &  &  &  &  &  &  &  &  &  &  &  &  &  \\
\hline
 &  &  &  &  &  &  &  &  &  &  &  &  &  &  &  &  &  &  &  &  &  \\
\hline
 &  &  &  &  &  &  &  &  &  &  &  &  &  &  &  &  &  &  &  &  &  \\
\hline
 &  &  &  &  &  &  &  &  &  &  &  &  &  &  &  &  &  &  &  &  &  \\
\hline
 &  &  &  &  &  &  &  &  &  &  &  &  &  &  &  &  &  &  &  &  &  \\
\hline
 &  &  &  &  &  &  &  &  &  &  &  &  &  &  &  &  &  &  &  &  &  \\
\hline
 &  &  &  &  &  &  &  &  &  &  &  &  &  &  &  &  &  &  &  &  &  \\
\hline
\end{tabular}
\end{center}

Odpowiedź: \(\qquad\)

Zadanie 28. (0-2)\\
Uzasadnij, że jeżeli liczba całkowita nie dzieli się przez 3, to jej kwadrat przy dzieleniu przez 3 daje resztę 1.\\
\includegraphics[max width=\textwidth, center]{2025_02_09_e5777ff444f0f25ff8deg-17}

\begin{center}
\begin{tabular}{|c|l|c|c|}
\hline
\multirow{3}{*}{\begin{tabular}{c}
Wypełnia \\
egzaminator \\
\end{tabular}} & Nr zadania & 27. & 28. \\
\cline { 2 - 4 }
 & Maks. liczba pkt & 2 & 2 \\
\cline { 2 - 4 }
 & Uzyskana liczba pkt &  &  \\
\hline
\end{tabular}
\end{center}

\section*{Zadanie 29. (0-2)}
Wartość prędkości średniej obliczamy jako iloraz drogi i czasu, w którym ta droga została przebyta. Samochód przejechał z miejscowości \(A\) do miejscowości \(C\) przez miejscowość \(B\), która znajduje się w połowie drogi z \(A\) do \(C\). Wartość prędkości średniej samochodu na trasie z \(A\) do \(B\) była równa \(40 \mathrm{~km} / \mathrm{h}\), a na trasie z \(B\) do \(C-60 \mathrm{~km} / \mathrm{h}\). Oblicz wartość prędkości średniej samochodu na całej trasie z \(A\) do \(C\).\\
\includegraphics[max width=\textwidth, center]{2025_02_09_e5777ff444f0f25ff8deg-18}

Odpowiedź:

\section*{Zadanie 30. (0-4)}
Zakupiono 16 biletów do teatru, w tym 10 biletów na miejsca od 1 . do 10 . w pierwszym rzędzie i 6 biletów na miejsca od 11. do 16. w szesnastym rzędzie. Jakie jest prawdopodobieństwo zdarzenia polegającego na tym, że 2 wylosowane bilety, spośród szesnastu, będą biletami na sąsiadujące miejsca?\\
\includegraphics[max width=\textwidth, center]{2025_02_09_e5777ff444f0f25ff8deg-19}

Odpowiedź: \(\qquad\)

\begin{center}
\begin{tabular}{|c|l|c|c|}
\hline
\multirow{3}{*}{\begin{tabular}{l}
Wypełnia \\
egzaminator \\
\end{tabular}} & Nr zadania & 29. & 30. \\
\cline { 2 - 4 }
 & Maks. liczba pkt & 2 & 4 \\
\cline { 2 - 4 }
 & Uzyskana liczba pkt &  &  \\
\hline
\end{tabular}
\end{center}

\section*{Zadanie 31. (0-4)}
W trapezie \(A B C D(A B \| C D)\) przekątne \(A C\) i \(B D\) przecinają się w punkcie \(O\) takim, że \(|A O|:|O C|=5: 1\). Pole trójkąta \(A O D\) jest równe 10. Uzasadnij, że pole trapezu \(A B C D\) jest równe 72.\\
\includegraphics[max width=\textwidth, center]{2025_02_09_e5777ff444f0f25ff8deg-20}

\section*{Zadanie 32. (0-4)}
Punkty \(A=(3,3)\) i \(B=(9,1)\) są wierzchołkami trójkąta \(A B C\), a punkt \(M=(1,6)\) jest środkiem boku \(A C\). Oblicz współrzędne punktu przecięcia prostej \(A B\) z wysokością tego trójkąta, poprowadzoną z wierzchołka \(C\).\\
\includegraphics[max width=\textwidth, center]{2025_02_09_e5777ff444f0f25ff8deg-21}

Odpowiedź:

\begin{center}
\begin{tabular}{|c|l|c|c|}
\hline
\multirow{2}{*}{\begin{tabular}{c}
Wypełnia \\
egzaminator \\
\end{tabular}} & Nr zadania & 31. & 32. \\
\cline { 2 - 4 }
 & Maks. liczba pkt & 4 & 4 \\
\cline { 2 - 4 }
 & Uzyskana liczba pkt &  &  \\
\hline
\end{tabular}
\end{center}

\section*{Zadanie 33. (0-4)}
Tworząca stożka ma długość 17, a wysokość stożka jest krótsza od średnicy jego podstawy o 22. Oblicz pole powierzchni całkowitej i objętość tego stożka.\\
\includegraphics[max width=\textwidth, center]{2025_02_09_e5777ff444f0f25ff8deg-22}

Odpowiedź:

\begin{center}
\begin{tabular}{|c|l|c|}
\hline
\multirow{2}{*}{\begin{tabular}{l}
Wypelnia \\
egzaminator \\
\end{tabular}} & Nr zadania & 33. \\
\cline { 2 - 3 }
 & Maks. liczba pkt & 4 \\
\cline { 2 - 3 }
 & Uzyskana liczba pkt &  \\
\hline
\end{tabular}
\end{center}

\section*{BRUDNOPIS (nie podlega ocenie)}

\end{document}