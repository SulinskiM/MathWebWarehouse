\documentclass[a4paper,12pt]{article}
\usepackage{latexsym}
\usepackage{amsmath}
\usepackage{amssymb}
\usepackage{graphicx}
\usepackage{wrapfig}
\pagestyle{plain}
\usepackage{fancybox}
\usepackage{bm}

\begin{document}
\begin{center}
\includegraphics[width=181.440mm,height=312.000mm]{./F2_M_PP_M2015_page0_images/image001.eps}
\end{center}
Arkusz zawiera info acje

prawnie chronione do momentu

rozpoczęcia egzaminu.

1

UZUPEL  A ZDAJACY

KOD  PESEL

{\it miejsce}

{\it na naklejkę}

dysleksja

EGZAMIN MATU  LNY Z MATEMATYKI

POZIOM PODSTAWOWY

DATA: 5 maja 2015 r.

CZAS P CY: 170 minut

LICZBA P  KTÓW DO UZYS NIA: 50

Instrukcja dla zdającego

1.

2.

3.

Sprawd $\acute{\mathrm{z}}$, czy arkusz egzaminacyjny zawiera 24 strony (zadania $1-34$).

Ewentualny brak zgłoś przewodniczącemu zespo nadzorującego

egzamin.

Rozwiązania zadań i odpowiedzi wpisuj w miejscu na to przeznaczonym.

Odpowiedzi do zadań zamkniętych $(1-25)$ przenieś na ka ę odpowiedzi,

zaznaczając je w części ka $\mathrm{y}$ przeznaczonej dla zdającego. Zamaluj $\blacksquare$

pola do tego przeznaczone. Błędne zaznaczenie otocz kółkiem \fcircle$\bullet$

i zaznacz właściwe.

4.

5.

Pamiętaj, $\dot{\mathrm{z}}\mathrm{e}$ pominięcie argumentacji lub istotnych obliczeń

w rozwiązaniu zadania otwa ego (26-34) $\mathrm{m}\mathrm{o}\dot{\mathrm{z}}\mathrm{e}$ spowodować, $\dot{\mathrm{z}}\mathrm{e}$ za to

rozwiązanie nie otrzymasz pełnej liczby punktów.

Pisz czytelnie i $\mathrm{u}\dot{\mathrm{z}}$ aj tvlko $\mathrm{d}$ gopisu lub -Dióra z czamym tuszem lub

atramentem.

6. Nie uzywaj korektora, a błędne zapisy wyrazínie prze eśl.

7. Pamiętaj, $\dot{\mathrm{z}}\mathrm{e}$ zapisy w brudnopisie nie będą oceniane.

8. $\mathrm{M}\mathrm{o}\dot{\mathrm{z}}$ esz korzystać z zesta wzorów matematycznych, cyrkla i linijki oraz

kalkulatora prostego.

9. Na tej stronie oraz na karcie odpowiedzi wpisz swój numer PESEL

i przyklej naklejkę z kodem.

10. Nie wpisuj $\dot{\mathrm{z}}$ adnych znaków w części przeznaczonej dla egzaminatora.

$\Vert\Vert\Vert\Vert\Vert\Vert\Vert\Vert\Vert\Vert\Vert\Vert\Vert\Vert\Vert\Vert\Vert\Vert\Vert\Vert\Vert\Vert\Vert\Vert|$

$\mathrm{M}\mathrm{M}\mathrm{A}-\mathrm{P}1_{-}1\mathrm{P}-152$

Układ graficzny

\copyright CKE 2015

1




{\it Wzadaniach od l. do 25. wybierz i zaznacz na karcie odpowiedzi poprawnq odpowiedzí}.

Zadanie $*.(0-l\rangle$

Wskaz rysunek, na którym przedstawiono przedział, będący zbiorem wszystkich rozwiązań

nierówności $-4\leq x-1\leq 4.$
\begin{center}
\includegraphics[width=173.376mm,height=13.512mm]{./F2_M_PP_M2015_page1_images/image001.eps}
\end{center}
$-5$  3  {\it x}

A.
\begin{center}
\includegraphics[width=173.328mm,height=15.192mm]{./F2_M_PP_M2015_page1_images/image002.eps}
\end{center}
$-3$  5  {\it x}

B.
\begin{center}
\includegraphics[width=173.376mm,height=14.220mm]{./F2_M_PP_M2015_page1_images/image003.eps}
\end{center}
$-3$  {\it x}

5

C.
\begin{center}
\includegraphics[width=173.328mm,height=14.280mm]{./F2_M_PP_M2015_page1_images/image004.eps}
\end{center}
$-5$  {\it x}

3

D.

Zadanie 2. (0-1)

Dane są liczby $a=-\displaystyle \frac{1}{27}, b=\log_{\frac{1}{4}}64, c=\log_{\frac{1}{3}}27$. Iloczyn $abc$ jest równy

A. -9 B. --31 C. -31 D. 3

Zadanie 3. (0-1)

Kwotę 1000 zł u1okowano w banku na roczną 1okatę oprocentowaną w wysokości 4\%

w stosunku rocznym. Po zakończeniu lokaty od naliczonych odsetek odprowadzany jest

podatek w wysokości 19\%. Maksyma1na kwota, jaką po upływie roku będzie mozna wypłacić

z banku, jest równa

A.

1000 $(1-\displaystyle \frac{81}{100}\cdot\frac{4}{100})$

B.

1000 $(1+\displaystyle \frac{19}{100}\cdot\frac{4}{100})$

C.

1000 $(1+\displaystyle \frac{81}{100}\cdot\frac{4}{100})$

D.

1000 $(1-\displaystyle \frac{19}{100}\cdot\frac{4}{100})$

Zadam$\mathrm{e}4.(0-1)$

Równość $\displaystyle \frac{m}{5-\sqrt{5}}=\frac{5+\sqrt{5}}{5}$ zachodzi dla

A. $m=5$

B. $m=4$

C. $m=1$

D. $m=-5$

Strona 2 z24

MMA-IP





{\it BRUDNOPIS} ({\it nie podlega ocenie})

Strona ll z24





Zadanie 26. $(0-2\rangle$

Rozwiąz nierówność $2x^{2}-4x>(x+3)(x-2).$

Odpowiedzí:

Strona 12 z24

MD





Zadanie 27. (0-2)

Wykaz, $\dot{\mathrm{z}}\mathrm{e}$ dla $\mathrm{k}\mathrm{a}\dot{\mathrm{z}}$ dej liczby rzeczywistej $x$ i dla $\mathrm{k}\mathrm{a}\dot{\mathrm{z}}$ dej liczby rzeczywistej $y$ prawdziwa jest

nierówność $4x^{2}-8xy+5y^{2}\geq 0.$
\begin{center}
\includegraphics[width=96.012mm,height=17.832mm]{./F2_M_PP_M2015_page12_images/image001.eps}
\end{center}
Wypelnia

egzaminator

Nr zadania

Maks. liczba kt

2

27.

2

Uzyskana liczba pkt

IMA-IP

Strona 13 z24





Zadanie 28. (0-2)

Dany jest kwadrat ABCD. Przekątne $AC\mathrm{i}BD$ przecinają się w punkcie $E$. Punkty $K\mathrm{i}M$ są

środkami odcinków- odpowiednio -$AE\mathrm{i}EC$. Punkty $L\mathrm{i}N$ lez$\cdot$ą na przekątnej $BD$ tak, $\dot{\mathrm{z}}\mathrm{e}$

$|BL|=\displaystyle \frac{1}{3}|BE| \mathrm{i} |DN|=\displaystyle \frac{1}{3}|DE|$ (zobacz rysunek). Wykaz, $\dot{\mathrm{z}}\mathrm{e}$ stosunek pola czworokąta KLMN

do pola kwadratu ABCD jest równy 1: 3.
\begin{center}
\includegraphics[width=60.864mm,height=60.252mm]{./F2_M_PP_M2015_page13_images/image001.eps}
\end{center}
{\it D  C}

{\it N}

{\it M}

{\it K  L}

{\it A  B}

Strona 14 z24

MMA-IP





Zadanie 29. $(0-2\rangle$

Oblicz najmniejszą

w przedziale $\langle 0, 4\rangle.$

i

największą wartość

funkcji kwadratowej

$f(x)=x^{2}-6x+3$

Odpowied $\acute{\mathrm{z}}$:
\begin{center}
\includegraphics[width=96.012mm,height=17.784mm]{./F2_M_PP_M2015_page14_images/image001.eps}
\end{center}
Wypelnia

egzaminator

Nr zadania

Maks. liczba kt

28.

2

2

Uzyskana liczba pkt

IMA-IP

Strona 15 z24





Zadanie 30. (0-2)

$\mathrm{W}$ układzie współrzędnych są dane punkty $A=(-43,-12), B=(50,19)$. Prosta $AB$ przecina

oś $Ox$ w punkcie $P$. Oblicz pierwszą współrzędną punktu $P.$

Odpowiedzí:

Strona 16 z24

MMA-IP





Zadanie $3l. (0-2)$

$\mathrm{J}\mathrm{e}\dot{\mathrm{z}}$ eli do licznika i do mianownika nieskracalnego dodatniego ułamka dodamy połowę jego

licznika, to otrzymamy $\displaystyle \frac{4}{7}$, ajezeli do licznika i do mianownika dodamy l, to otrzymamy $\displaystyle \frac{1}{2}.$

Wyznacz ten ułamek.

Odpowied $\acute{\mathrm{z}}$:
\begin{center}
\includegraphics[width=96.012mm,height=17.784mm]{./F2_M_PP_M2015_page16_images/image001.eps}
\end{center}
Wypelnia

egzaminator

Nr zadania

Maks. liczba kt

30.

2

31.

2

Uzyskana liczba pkt

IMA-IP

Strona 17 z24





Zadanie 32. (0-4)

Wysokość graniastosłupa prawidłowego czworokątnego jest równa 16. Przekątna graniastosłupa

jest nachylona do płaszczyzny jego podstawy pod kątem, którego cosinus jest równy $\displaystyle \frac{3}{5}$. Oblicz

pole powierzchni całkowitej tego graniastosłupa.

Strona 18 z24

MMA-IP





Odpowiedzí :
\begin{center}
\includegraphics[width=82.044mm,height=17.784mm]{./F2_M_PP_M2015_page18_images/image001.eps}
\end{center}
Wypelnia

egzamÍnator

Nr zadania

Maks. liczba kt

32.

4

Uzyskana liczba pkt

IMA-IP

Strona 19 z24





Zadanie 33. (0-4)

Wśród 115 osób przeprowadzono badania ankietowe, związane z zakupami w pewnym

kiosku. W ponizszej tabeli przedstawiono informacje o tym, ile osób kupiło bilety

tramwajowe ulgowe oraz ile osób kupiło bilety tramwajowe normalne.
\begin{center}
\begin{tabular}{|l|l|}
\hline
\multicolumn{1}{|l|}{$\begin{array}{l}\mbox{Rodzaj kupionych}	\\	\mbox{biletów}	\end{array}$}&	\multicolumn{1}{|l|}{Liczba osób}	\\
\hline
\multicolumn{1}{|l|}{ulgowe}&	\multicolumn{1}{|l|}{$76$}	\\
\hline
\multicolumn{1}{|l|}{normalne}&	\multicolumn{1}{|l|}{$41$}	\\
\hline
\end{tabular}

\end{center}
Uwaga! 27 osób spośród ankietowanych kupiło oba rodzaje bi1etów.

Oblicz prawdopodobieństwo zdarzenia polegającego na tym, $\dot{\mathrm{z}}\mathrm{e}$ osoba losowo wybrana

spośród ankietowanych nie kupiła $\dot{\mathrm{z}}$ adnego biletu. Wynik przedstaw w formie nieskracalnego

ułamka.

Strona 20 z24

MMA-IP





{\it BRUDNOPIS} ({\it nie podlega ocenie})

Strona 3 z24





Odpowiedzí :
\begin{center}
\includegraphics[width=82.044mm,height=17.784mm]{./F2_M_PP_M2015_page20_images/image001.eps}
\end{center}
Wypelnia

egzamÍnator

Nr zadania

Maks. liczba kt

33.

4

Uzyskana liczba pkt

IMA-IP

Strona 21 z24





Zadanie 34. $\zeta 0-5\rangle$

$\mathrm{W}$ nieskończonym ciągu arytmetycznym $(a_{n})$, określonym dla $n\geq 1$, suma jedenastu

początkowych wyrazów tego ciągu jest równa 187. Średnia arytmetyczna pierwszego,

trzeciego i dziewiątego wyrazu tego ciągu, jest równa 12. Wyrazy $a_{1}, a_{3}, a_{k}$ ciągu $(a_{n}),$

w podanej kolejności, tworzą nowy ciąg- trzywyrazowy ciąg geometryczny $(b_{n})$. Oblicz $k.$

Strona 22 z24

MMA-IP





Odpowiedzí :
\begin{center}
\includegraphics[width=82.044mm,height=17.784mm]{./F2_M_PP_M2015_page22_images/image001.eps}
\end{center}
Wypelnia

egzamÍnator

Nr zadania

Maks. liczba kt

34.

5

Uzyskana liczba pkt

IMA-IP

Strona 23 z24





{\it BRUDNOPIS} ({\it nie podlega ocenie})

Strona 24 z24

MD





Zadanie 5. (0-1)

Układ równań 

A. zbiór pusty.

B. dokładnie jeden punkt.

C. dokładnie dwa rózne punkty.

D. zbiór nieskończony.

Zadanie 6. (0-1)

Suma wszystkich pierwiastków równania $(x+3)(x+7)(x-11)=0$ jest równa

A. $-1$

B. 21

C. l

D. $-21$

Zadanie 7. $(0-1\rangle$

Równanie $\displaystyle \frac{x-1}{x+1}=x-1$

A. ma dokładniejedno rozwiązanie: $x=1.$

B. ma dokładniejedno rozwiązanie: $x=0.$

C. ma dokładniejedno rozwiązanie: $x=-1.$

D. ma dokładnie dwa rozwiązania: $x=0, x=1.$

Zadanie 8. (0-1)

Na rysunku przedstawiono wykres funkcjif
\begin{center}
\includegraphics[width=120.492mm,height=75.996mm]{./F2_M_PP_M2015_page3_images/image001.eps}
\end{center}
{\it y}

3

2

1

$-4 -3  -2 -1$  {\it x}

0  1 2 3 4  5

$-1$

$-2$

$-3$

Zbiorem wartości ffinkcji $f$ jest

A. $(-2,2)$ B. $\langle-2$, 2$)$

C. $\langle-2,  2\rangle$

D. $(-2,2\rangle$

Zadanie $g. (0-1)$

Na wykresie funkcji liniowej określonej wzorem $f(x)=(m-1)x+3$ lezy punkt $S=(5,-2).$

Zatem

A. $m=-1$

B. $m=0$

C. $m=1$

D. $m=2$

Strona 4 z24

MMA-IP





{\it BRUDNOPIS} ({\it nie podlega ocenie})

Strona 5 z24





Zadanie $l0. (0-1\rangle$

Funkcja liniowa $f$ określona wzorem $f(x)=2x+b$ ma takie samo miejsce zerowe, jakie ma

funkcja liniowa $g(x)=-3x+4$. Stąd wynika, $\dot{\mathrm{z}}\mathrm{e}$

A. $b=4$

B.

{\it b}$=$- -23

C.

{\it b}$=$- -38

D.

{\it b}$=$ -43

Zadanie ll. $(0-l\rangle$

Funkcja kwadratowa określonajest wzorem $f(x)=x^{2}+x+c. \mathrm{J}\mathrm{e}\dot{\mathrm{z}}$ eli $f(3)=4$, to

A. $f(1)=-6$

B. $f(1)=0$

C. $f(1)=6$

D. $f(1)=18$

$\mathrm{Z}\mathrm{a}\mathrm{d}\mathrm{a}\mathrm{n}\ddagger \mathrm{e}12. (0-1\rangle$

Ile liczb całkowitych $x$ spełnia nierówność $\displaystyle \frac{2}{7}<\frac{x}{14}<\frac{4}{3}$ ?

A. 14

B. 15

C. 16

D. 17

$\mathrm{Z}\mathrm{a}\mathrm{d}\mathrm{a}\mathrm{n}\ddagger \mathrm{e}13. (0-1)$

$\mathrm{W}$ rosnącym ciągu geometrycznym $(a_{n})$, określonym dla $n\geq 1$, spełniony jest warunek

$a_{4}=3a_{1}$. Iloraz $q$ tego ciągu jest równy

A.

{\it q}$=$ -31

B.

{\it q}$=$ -$\sqrt{}$313

C. $q=\sqrt[3]{3}$

D. $q=3$

Zadanie 14. (0-1)

Tangens kąta a zaznaczonego na

su ujest równy

A. -

$\sqrt{3}$

3

B. --45
\begin{center}
\includegraphics[width=83.364mm,height=58.776mm]{./F2_M_PP_M2015_page5_images/image001.eps}
\end{center}
{\it y}

6

{\it P}

5

4

3

2

{\it x}

$-5$

1

$a$

$-3-2-1 0$ 1

$-1$

2 3 4 5

D. - -45

C. $-1$

$P=(-4,5)$

Zadanie 15. $(0-1\rangle$

$\mathrm{J}\mathrm{e}\dot{\mathrm{z}}$ eli $0^{\mathrm{o}}<\alpha<90^{\mathrm{o}}$ oraz $\mathrm{t}\mathrm{g}\alpha=2\sin\alpha$, to

A.

$\displaystyle \cos\alpha=\frac{1}{2}$

B.

$\displaystyle \cos\alpha=\frac{\sqrt{2}}{2}$

C.

$\displaystyle \cos\alpha=\frac{\sqrt{3}}{2}$

D. $\cos\alpha=1$

Strona 6 z24

MMA-IP





{\it BRUDNOPIS} ({\it nie podlega ocenie})

Strona 7 z24





Zadanie $1\epsilon. (0-1\rangle$

Miara kąta wpisanego w okrąg jest o $20^{\mathrm{o}}$ mniejsza od miary kąta środkowego opartego na

tym samym łuku. Wynika stąd, $\dot{\mathrm{z}}\mathrm{e}$ miara kąta wpisanegojest równa

A. $5^{\mathrm{o}}$

B. $10^{\mathrm{o}}$

C. $20^{\mathrm{o}}$

D. $30^{\mathrm{o}}$

$\mathrm{Z}\mathrm{a}\mathrm{d}\mathrm{a}\mathrm{n}\ddagger \mathrm{e}17. (0-1\rangle$

Pole rombu o obwodzie $8$jest równe l. Kąt ostry tego rombu ma miarę $\alpha$. Wtedy

A. $14^{\mathrm{o}}<\alpha<15^{\mathrm{o}}$

B. $29^{\mathrm{o}}<\alpha<30^{\mathrm{o}}$

C. $60^{\mathrm{o}}<\alpha<61^{\mathrm{o}}$

D. $75^{\mathrm{o}}<\alpha<76^{\mathrm{o}}$

Zadanie 18. (0-1)

Prosta $l$ o równaniu $y=m^{2}x+3$ jest równoległa do prostej $k$ o równaniu $y=(4m-4)x-3.$

Zatem

A. $m=2$ B. $m=-2$ C. $m=-2-2\sqrt{2}$ D. $m=2+2\sqrt{2}$

$\mathrm{Z}\mathrm{a}\mathrm{d}\mathrm{a}\mathrm{n}\ddagger \mathrm{e}l9. (0-1\rangle$

Proste o równaniach: $y=2mx-m^{2}-1$ oraz $y=4m^{2}x+m^{2}+1$ są prostopadłe dla

A. {\it m}$=$--21 B. {\it m}$=$-21 C. {\it m}$=$1 D. {\it m}$=$2

Zadanie 20. (0-1)

Dane są punkty $M=(-2,1) \mathrm{i} N=(-1,3)$. Punkt $K$ jest środkiem odcinka $MN$. Obrazem

punktu $K$ w symetrii względem początku układu współrzędnychjest punkt

A.

{\it K}$\prime =$(2, - -23)

B.

{\it K}$\prime =$(2, -23)

C.

{\it K}$\prime =$(-23 , 2)

D. {\it K}$\prime =$(-23 , -2)

Zadanie 21. (0-1)

W graniastosłupie prawidłowym czworokątnym EFGHIJKL wierzchołki E, G, L połączono

odcinkami (takjak na rysunku).
\begin{center}
\includegraphics[width=54.912mm,height=78.432mm]{./F2_M_PP_M2015_page7_images/image001.eps}
\end{center}
{\it L K}

{\it I J}

{\it H G}

{\it O}

{\it E F}

Wskaz kąt między wysokością OL trójkąta EGL i płaszczyzną podstawy tego graniastosłupa.

A.

$\neq HOL$

B.

$\neq OGL$

C. $\neq HLO$

D.

$\neq OHL$

Strona 8 z24

MMA-IP





{\it BRUDNOPIS} ({\it nie podlega ocenie})

Strona 9 z24





Zadanie 22. $(0-1\rangle$

Przekrojem osiowym stozka jest trójkąt równoboczny o boku długości

stozkajest równa

6. Objętość tego

A. $27\pi\sqrt{3}$

B. $9\pi\sqrt{3}$

C. $ 18\pi$

D. $ 6\pi$

Zadanie 23. (0-1)

$\mathrm{K}\mathrm{a}\dot{\mathrm{z}}$ da krawędz graniastosłupa prawidłowego trójkątnego ma długość

powierzchni całkowitej tego graniastosłupajest równe

równą 8. Po1e

A.

$\displaystyle \frac{8^{2}}{3}(\frac{\sqrt{3}}{2}+3)$

B. $8^{2}\cdot\sqrt{3}$

C.

$\displaystyle \frac{8^{2}\sqrt{6}}{3}$

D.

$8^{2}(\displaystyle \frac{\sqrt{3}}{2}+3)$

Zadanie 24. $(0-1\rangle$

Średnia arytmetyczna zestawu danych:

2, 4, 7, 8, 9

jest taka samajak średnia arytmetyczna zestawu danych:

2, 4, 7, 8, 9, $x.$

Wynika stąd, $\dot{\mathrm{z}}\mathrm{e}$

A. $x=0$

B. $x=3$

C. $x=5$

D. $x=6$

Zadanie 25. (0-1)

$\mathrm{W} \mathrm{k}\mathrm{a}\dot{\mathrm{z}}$ dym z trzech pojemników znajduje się para kul, z których jedna jest czerwona,

a druga - niebieska. $\mathrm{Z} \mathrm{k}\mathrm{a}\dot{\mathrm{z}}$ dego pojemnika losujemy jedną kulę. Niech $p$ oznacza

prawdopodobieństwo zdarzenia polegającego na tym, $\dot{\mathrm{z}}\mathrm{e}$ dokładnie dwie z trzech

wylosowanych kul będą czerwone. Wtedy

A.

{\it p}$=$ -41

B.

{\it p}$=$ -83

C.

{\it p}$=$ -21

D.

{\it p}$=$ -23

Strona 10 z24

MMA-IP



\end{document}