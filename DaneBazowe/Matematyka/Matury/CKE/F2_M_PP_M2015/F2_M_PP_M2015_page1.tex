\documentclass[a4paper,12pt]{article}
\usepackage{latexsym}
\usepackage{amsmath}
\usepackage{amssymb}
\usepackage{graphicx}
\usepackage{wrapfig}
\pagestyle{plain}
\usepackage{fancybox}
\usepackage{bm}

\begin{document}

{\it Wzadaniach od l. do 25. wybierz i zaznacz na karcie odpowiedzi poprawnq odpowiedzí}.

Zadanie $*.(0-l\rangle$

Wskaz rysunek, na którym przedstawiono przedział, będący zbiorem wszystkich rozwiązań

nierówności $-4\leq x-1\leq 4.$
\begin{center}
\includegraphics[width=173.376mm,height=13.512mm]{./F2_M_PP_M2015_page1_images/image001.eps}
\end{center}
$-5$  3  {\it x}

A.
\begin{center}
\includegraphics[width=173.328mm,height=15.192mm]{./F2_M_PP_M2015_page1_images/image002.eps}
\end{center}
$-3$  5  {\it x}

B.
\begin{center}
\includegraphics[width=173.376mm,height=14.220mm]{./F2_M_PP_M2015_page1_images/image003.eps}
\end{center}
$-3$  {\it x}

5

C.
\begin{center}
\includegraphics[width=173.328mm,height=14.280mm]{./F2_M_PP_M2015_page1_images/image004.eps}
\end{center}
$-5$  {\it x}

3

D.

Zadanie 2. (0-1)

Dane są liczby $a=-\displaystyle \frac{1}{27}, b=\log_{\frac{1}{4}}64, c=\log_{\frac{1}{3}}27$. Iloczyn $abc$ jest równy

A. -9 B. --31 C. -31 D. 3

Zadanie 3. (0-1)

Kwotę 1000 zł u1okowano w banku na roczną 1okatę oprocentowaną w wysokości 4\%

w stosunku rocznym. Po zakończeniu lokaty od naliczonych odsetek odprowadzany jest

podatek w wysokości 19\%. Maksyma1na kwota, jaką po upływie roku będzie mozna wypłacić

z banku, jest równa

A.

1000 $(1-\displaystyle \frac{81}{100}\cdot\frac{4}{100})$

B.

1000 $(1+\displaystyle \frac{19}{100}\cdot\frac{4}{100})$

C.

1000 $(1+\displaystyle \frac{81}{100}\cdot\frac{4}{100})$

D.

1000 $(1-\displaystyle \frac{19}{100}\cdot\frac{4}{100})$

Zadam$\mathrm{e}4.(0-1)$

Równość $\displaystyle \frac{m}{5-\sqrt{5}}=\frac{5+\sqrt{5}}{5}$ zachodzi dla

A. $m=5$

B. $m=4$

C. $m=1$

D. $m=-5$

Strona 2 z24

MMA-IP
\end{document}
