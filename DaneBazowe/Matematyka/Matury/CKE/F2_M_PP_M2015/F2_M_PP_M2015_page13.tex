\documentclass[a4paper,12pt]{article}
\usepackage{latexsym}
\usepackage{amsmath}
\usepackage{amssymb}
\usepackage{graphicx}
\usepackage{wrapfig}
\pagestyle{plain}
\usepackage{fancybox}
\usepackage{bm}

\begin{document}

Zadanie 28. (0-2)

Dany jest kwadrat ABCD. Przekątne $AC\mathrm{i}BD$ przecinają się w punkcie $E$. Punkty $K\mathrm{i}M$ są

środkami odcinków- odpowiednio -$AE\mathrm{i}EC$. Punkty $L\mathrm{i}N$ lez$\cdot$ą na przekątnej $BD$ tak, $\dot{\mathrm{z}}\mathrm{e}$

$|BL|=\displaystyle \frac{1}{3}|BE| \mathrm{i} |DN|=\displaystyle \frac{1}{3}|DE|$ (zobacz rysunek). Wykaz, $\dot{\mathrm{z}}\mathrm{e}$ stosunek pola czworokąta KLMN

do pola kwadratu ABCD jest równy 1: 3.
\begin{center}
\includegraphics[width=60.864mm,height=60.252mm]{./F2_M_PP_M2015_page13_images/image001.eps}
\end{center}
{\it D  C}

{\it N}

{\it M}

{\it K  L}

{\it A  B}

Strona 14 z24

MMA-IP
\end{document}
