\documentclass[a4paper,12pt]{article}
\usepackage{latexsym}
\usepackage{amsmath}
\usepackage{amssymb}
\usepackage{graphicx}
\usepackage{wrapfig}
\pagestyle{plain}
\usepackage{fancybox}
\usepackage{bm}

\begin{document}

Zadanie $3l. (0-2)$

$\mathrm{J}\mathrm{e}\dot{\mathrm{z}}$ eli do licznika i do mianownika nieskracalnego dodatniego ułamka dodamy połowę jego

licznika, to otrzymamy $\displaystyle \frac{4}{7}$, ajezeli do licznika i do mianownika dodamy l, to otrzymamy $\displaystyle \frac{1}{2}.$

Wyznacz ten ułamek.

Odpowied $\acute{\mathrm{z}}$:
\begin{center}
\includegraphics[width=96.012mm,height=17.784mm]{./F2_M_PP_M2015_page16_images/image001.eps}
\end{center}
Wypelnia

egzaminator

Nr zadania

Maks. liczba kt

30.

2

31.

2

Uzyskana liczba pkt

IMA-IP

Strona 17 z24
\end{document}
