\documentclass[a4paper,12pt]{article}
\usepackage{latexsym}
\usepackage{amsmath}
\usepackage{amssymb}
\usepackage{graphicx}
\usepackage{wrapfig}
\pagestyle{plain}
\usepackage{fancybox}
\usepackage{bm}

\begin{document}

Zadanie 33. (0-4)

Wśród 115 osób przeprowadzono badania ankietowe, związane z zakupami w pewnym

kiosku. W ponizszej tabeli przedstawiono informacje o tym, ile osób kupiło bilety

tramwajowe ulgowe oraz ile osób kupiło bilety tramwajowe normalne.
\begin{center}
\begin{tabular}{|l|l|}
\hline
\multicolumn{1}{|l|}{$\begin{array}{l}\mbox{Rodzaj kupionych}	\\	\mbox{biletów}	\end{array}$}&	\multicolumn{1}{|l|}{Liczba osób}	\\
\hline
\multicolumn{1}{|l|}{ulgowe}&	\multicolumn{1}{|l|}{$76$}	\\
\hline
\multicolumn{1}{|l|}{normalne}&	\multicolumn{1}{|l|}{$41$}	\\
\hline
\end{tabular}

\end{center}
Uwaga! 27 osób spośród ankietowanych kupiło oba rodzaje bi1etów.

Oblicz prawdopodobieństwo zdarzenia polegającego na tym, $\dot{\mathrm{z}}\mathrm{e}$ osoba losowo wybrana

spośród ankietowanych nie kupiła $\dot{\mathrm{z}}$ adnego biletu. Wynik przedstaw w formie nieskracalnego

ułamka.

Strona 20 z24

MMA-IP
\end{document}
