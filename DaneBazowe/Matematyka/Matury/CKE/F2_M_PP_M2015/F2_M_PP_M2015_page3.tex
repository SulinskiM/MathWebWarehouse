\documentclass[a4paper,12pt]{article}
\usepackage{latexsym}
\usepackage{amsmath}
\usepackage{amssymb}
\usepackage{graphicx}
\usepackage{wrapfig}
\pagestyle{plain}
\usepackage{fancybox}
\usepackage{bm}

\begin{document}

Zadanie 5. (0-1)

Układ równań 

A. zbiór pusty.

B. dokładnie jeden punkt.

C. dokładnie dwa rózne punkty.

D. zbiór nieskończony.

Zadanie 6. (0-1)

Suma wszystkich pierwiastków równania $(x+3)(x+7)(x-11)=0$ jest równa

A. $-1$

B. 21

C. l

D. $-21$

Zadanie 7. $(0-1\rangle$

Równanie $\displaystyle \frac{x-1}{x+1}=x-1$

A. ma dokładniejedno rozwiązanie: $x=1.$

B. ma dokładniejedno rozwiązanie: $x=0.$

C. ma dokładniejedno rozwiązanie: $x=-1.$

D. ma dokładnie dwa rozwiązania: $x=0, x=1.$

Zadanie 8. (0-1)

Na rysunku przedstawiono wykres funkcjif
\begin{center}
\includegraphics[width=120.492mm,height=75.996mm]{./F2_M_PP_M2015_page3_images/image001.eps}
\end{center}
{\it y}

3

2

1

$-4 -3  -2 -1$  {\it x}

0  1 2 3 4  5

$-1$

$-2$

$-3$

Zbiorem wartości ffinkcji $f$ jest

A. $(-2,2)$ B. $\langle-2$, 2$)$

C. $\langle-2,  2\rangle$

D. $(-2,2\rangle$

Zadanie $g. (0-1)$

Na wykresie funkcji liniowej określonej wzorem $f(x)=(m-1)x+3$ lezy punkt $S=(5,-2).$

Zatem

A. $m=-1$

B. $m=0$

C. $m=1$

D. $m=2$

Strona 4 z24

MMA-IP
\end{document}
