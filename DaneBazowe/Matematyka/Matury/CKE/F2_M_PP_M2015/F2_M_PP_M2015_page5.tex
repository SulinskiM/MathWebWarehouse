\documentclass[a4paper,12pt]{article}
\usepackage{latexsym}
\usepackage{amsmath}
\usepackage{amssymb}
\usepackage{graphicx}
\usepackage{wrapfig}
\pagestyle{plain}
\usepackage{fancybox}
\usepackage{bm}

\begin{document}

Zadanie $l0. (0-1\rangle$

Funkcja liniowa $f$ określona wzorem $f(x)=2x+b$ ma takie samo miejsce zerowe, jakie ma

funkcja liniowa $g(x)=-3x+4$. Stąd wynika, $\dot{\mathrm{z}}\mathrm{e}$

A. $b=4$

B.

{\it b}$=$- -23

C.

{\it b}$=$- -38

D.

{\it b}$=$ -43

Zadanie ll. $(0-l\rangle$

Funkcja kwadratowa określonajest wzorem $f(x)=x^{2}+x+c. \mathrm{J}\mathrm{e}\dot{\mathrm{z}}$ eli $f(3)=4$, to

A. $f(1)=-6$

B. $f(1)=0$

C. $f(1)=6$

D. $f(1)=18$

$\mathrm{Z}\mathrm{a}\mathrm{d}\mathrm{a}\mathrm{n}\ddagger \mathrm{e}12. (0-1\rangle$

Ile liczb całkowitych $x$ spełnia nierówność $\displaystyle \frac{2}{7}<\frac{x}{14}<\frac{4}{3}$ ?

A. 14

B. 15

C. 16

D. 17

$\mathrm{Z}\mathrm{a}\mathrm{d}\mathrm{a}\mathrm{n}\ddagger \mathrm{e}13. (0-1)$

$\mathrm{W}$ rosnącym ciągu geometrycznym $(a_{n})$, określonym dla $n\geq 1$, spełniony jest warunek

$a_{4}=3a_{1}$. Iloraz $q$ tego ciągu jest równy

A.

{\it q}$=$ -31

B.

{\it q}$=$ -$\sqrt{}$313

C. $q=\sqrt[3]{3}$

D. $q=3$

Zadanie 14. (0-1)

Tangens kąta a zaznaczonego na

su ujest równy

A. -

$\sqrt{3}$

3

B. --45
\begin{center}
\includegraphics[width=83.364mm,height=58.776mm]{./F2_M_PP_M2015_page5_images/image001.eps}
\end{center}
{\it y}

6

{\it P}

5

4

3

2

{\it x}

$-5$

1

$a$

$-3-2-1 0$ 1

$-1$

2 3 4 5

D. - -45

C. $-1$

$P=(-4,5)$

Zadanie 15. $(0-1\rangle$

$\mathrm{J}\mathrm{e}\dot{\mathrm{z}}$ eli $0^{\mathrm{o}}<\alpha<90^{\mathrm{o}}$ oraz $\mathrm{t}\mathrm{g}\alpha=2\sin\alpha$, to

A.

$\displaystyle \cos\alpha=\frac{1}{2}$

B.

$\displaystyle \cos\alpha=\frac{\sqrt{2}}{2}$

C.

$\displaystyle \cos\alpha=\frac{\sqrt{3}}{2}$

D. $\cos\alpha=1$

Strona 6 z24

MMA-IP
\end{document}
