\documentclass[a4paper,12pt]{article}
\usepackage{latexsym}
\usepackage{amsmath}
\usepackage{amssymb}
\usepackage{graphicx}
\usepackage{wrapfig}
\pagestyle{plain}
\usepackage{fancybox}
\usepackage{bm}

\begin{document}

Zadanie $1\epsilon. (0-1\rangle$

Miara kąta wpisanego w okrąg jest o $20^{\mathrm{o}}$ mniejsza od miary kąta środkowego opartego na

tym samym łuku. Wynika stąd, $\dot{\mathrm{z}}\mathrm{e}$ miara kąta wpisanegojest równa

A. $5^{\mathrm{o}}$

B. $10^{\mathrm{o}}$

C. $20^{\mathrm{o}}$

D. $30^{\mathrm{o}}$

$\mathrm{Z}\mathrm{a}\mathrm{d}\mathrm{a}\mathrm{n}\ddagger \mathrm{e}17. (0-1\rangle$

Pole rombu o obwodzie $8$jest równe l. Kąt ostry tego rombu ma miarę $\alpha$. Wtedy

A. $14^{\mathrm{o}}<\alpha<15^{\mathrm{o}}$

B. $29^{\mathrm{o}}<\alpha<30^{\mathrm{o}}$

C. $60^{\mathrm{o}}<\alpha<61^{\mathrm{o}}$

D. $75^{\mathrm{o}}<\alpha<76^{\mathrm{o}}$

Zadanie 18. (0-1)

Prosta $l$ o równaniu $y=m^{2}x+3$ jest równoległa do prostej $k$ o równaniu $y=(4m-4)x-3.$

Zatem

A. $m=2$ B. $m=-2$ C. $m=-2-2\sqrt{2}$ D. $m=2+2\sqrt{2}$

$\mathrm{Z}\mathrm{a}\mathrm{d}\mathrm{a}\mathrm{n}\ddagger \mathrm{e}l9. (0-1\rangle$

Proste o równaniach: $y=2mx-m^{2}-1$ oraz $y=4m^{2}x+m^{2}+1$ są prostopadłe dla

A. {\it m}$=$--21 B. {\it m}$=$-21 C. {\it m}$=$1 D. {\it m}$=$2

Zadanie 20. (0-1)

Dane są punkty $M=(-2,1) \mathrm{i} N=(-1,3)$. Punkt $K$ jest środkiem odcinka $MN$. Obrazem

punktu $K$ w symetrii względem początku układu współrzędnychjest punkt

A.

{\it K}$\prime =$(2, - -23)

B.

{\it K}$\prime =$(2, -23)

C.

{\it K}$\prime =$(-23 , 2)

D. {\it K}$\prime =$(-23 , -2)

Zadanie 21. (0-1)

W graniastosłupie prawidłowym czworokątnym EFGHIJKL wierzchołki E, G, L połączono

odcinkami (takjak na rysunku).
\begin{center}
\includegraphics[width=54.912mm,height=78.432mm]{./F2_M_PP_M2015_page7_images/image001.eps}
\end{center}
{\it L K}

{\it I J}

{\it H G}

{\it O}

{\it E F}

Wskaz kąt między wysokością OL trójkąta EGL i płaszczyzną podstawy tego graniastosłupa.

A.

$\neq HOL$

B.

$\neq OGL$

C. $\neq HLO$

D.

$\neq OHL$

Strona 8 z24

MMA-IP
\end{document}
