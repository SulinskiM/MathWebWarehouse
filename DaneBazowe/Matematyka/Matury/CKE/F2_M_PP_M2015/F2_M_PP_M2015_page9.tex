\documentclass[a4paper,12pt]{article}
\usepackage{latexsym}
\usepackage{amsmath}
\usepackage{amssymb}
\usepackage{graphicx}
\usepackage{wrapfig}
\pagestyle{plain}
\usepackage{fancybox}
\usepackage{bm}

\begin{document}

Zadanie 22. $(0-1\rangle$

Przekrojem osiowym stozka jest trójkąt równoboczny o boku długości

stozkajest równa

6. Objętość tego

A. $27\pi\sqrt{3}$

B. $9\pi\sqrt{3}$

C. $ 18\pi$

D. $ 6\pi$

Zadanie 23. (0-1)

$\mathrm{K}\mathrm{a}\dot{\mathrm{z}}$ da krawędz graniastosłupa prawidłowego trójkątnego ma długość

powierzchni całkowitej tego graniastosłupajest równe

równą 8. Po1e

A.

$\displaystyle \frac{8^{2}}{3}(\frac{\sqrt{3}}{2}+3)$

B. $8^{2}\cdot\sqrt{3}$

C.

$\displaystyle \frac{8^{2}\sqrt{6}}{3}$

D.

$8^{2}(\displaystyle \frac{\sqrt{3}}{2}+3)$

Zadanie 24. $(0-1\rangle$

Średnia arytmetyczna zestawu danych:

2, 4, 7, 8, 9

jest taka samajak średnia arytmetyczna zestawu danych:

2, 4, 7, 8, 9, $x.$

Wynika stąd, $\dot{\mathrm{z}}\mathrm{e}$

A. $x=0$

B. $x=3$

C. $x=5$

D. $x=6$

Zadanie 25. (0-1)

$\mathrm{W} \mathrm{k}\mathrm{a}\dot{\mathrm{z}}$ dym z trzech pojemników znajduje się para kul, z których jedna jest czerwona,

a druga - niebieska. $\mathrm{Z} \mathrm{k}\mathrm{a}\dot{\mathrm{z}}$ dego pojemnika losujemy jedną kulę. Niech $p$ oznacza

prawdopodobieństwo zdarzenia polegającego na tym, $\dot{\mathrm{z}}\mathrm{e}$ dokładnie dwie z trzech

wylosowanych kul będą czerwone. Wtedy

A.

{\it p}$=$ -41

B.

{\it p}$=$ -83

C.

{\it p}$=$ -21

D.

{\it p}$=$ -23

Strona 10 z24

MMA-IP
\end{document}
