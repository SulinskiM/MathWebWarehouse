\documentclass[a4paper,12pt]{article}
\usepackage{latexsym}
\usepackage{amsmath}
\usepackage{amssymb}
\usepackage{graphicx}
\usepackage{wrapfig}
\pagestyle{plain}
\usepackage{fancybox}
\usepackage{bm}

\begin{document}

gCK$\epsilon$6OENZffl$\tau$AfRS{\it m}JA$\xi$ANLNAACY{\it S}NA

Arkusz zawiera info acje

prawnie chronione do momentu

rozpoczęcia egzaminu.
\begin{center}
\includegraphics[width=22.908mm,height=19.248mm]{./F2_M_PP_M2016_page0_images/image001.eps}
\end{center}
1  $\iota$

UZUPELNIA ZDAJACY

{\it miejsce}

{\it na naklejkę}
\begin{center}
\includegraphics[width=21.900mm,height=14.736mm]{./F2_M_PP_M2016_page0_images/image002.eps}
\end{center}
KOD
\begin{center}
\includegraphics[width=79.656mm,height=14.736mm]{./F2_M_PP_M2016_page0_images/image003.eps}
\end{center}
PESEL
\begin{center}
\includegraphics[width=194.616mm,height=246.432mm]{./F2_M_PP_M2016_page0_images/image004.eps}
\end{center}
dyskalkulia  dysleksja

EGZAMIN MATU  LNY Z MATEMATY

POZIOM PODSTAWOWY

LICZBA P  KTÓW DO UZYS NIA: 50

Instrukcja dla zdającego

1.

2.

3.

4.

5.

Sprawdzí, czy arkusz egzaminacyjny zawiera 24 strony (zadania $1-34$).

Ewentualny brak zgłoś przewodniczącemu zespo nadzorującego

egzamin.

Rozwiązania zadań i odpowiedzi wpisuj w miejscu na to przeznaczonym.

Odpowiedzi do zadań za ię ch $(1-25)$ zaznacz na karcie odpowiedzi,

w części ka $\mathrm{y}$ przeznaczonej dla zdającego. Zamaluj $\blacksquare$ pola do tego

przeznaczone. Błędne zaznaczenie otocz kólkiem \copyright i zaznacz wlaściwe.

Pamiętaj, $\dot{\mathrm{z}}\mathrm{e}$ pominięcie argumentacji lub istotnych obliczeń

w rozwiązaniu zadania otwa ego (26-34) $\mathrm{m}\mathrm{o}\dot{\mathrm{z}}\mathrm{e}$ spowodować, $\dot{\mathrm{z}}\mathrm{e}$ za to

rozwiązanie nie otrzymasz pelnej liczby pu tów.

Pisz cz elnie i $\mathrm{u}\dot{\mathrm{z}}$ aj lko $\mathrm{d}$ gopisu lub pióra z czarnym tuszem lub

atramentem.

6. Nie $\mathrm{u}\dot{\mathrm{z}}$ aj korektora, a błędne zapisy $\mathrm{r}\mathrm{a}\acute{\mathrm{z}}\mathrm{n}\mathrm{i}\mathrm{e}$ prze eśl.

7. Pamiętaj, $\dot{\mathrm{z}}\mathrm{e}$ zapisy w brudnopisie nie będą oceniane.

8. $\mathrm{M}\mathrm{o}\dot{\mathrm{z}}$ esz korzystać z zesta wzorów matema cznych, cyrkla i linijki,

a ta $\mathrm{e}$ z kalkulatora prostego.

9. Na tej stronie oraz na karcie odpowiedzi wpisz swój numer PESEL

i przyklej naklejkę z kodem.

10. Nie wpisuj $\dot{\mathrm{z}}$ adnych znaków w części przeznaczonej dla egzaminatora.

$\Vert\Vert\Vert\Vert\Vert\Vert\Vert\Vert\Vert\Vert\Vert\Vert\Vert\Vert\Vert\Vert\Vert\Vert\Vert\Vert\Vert\Vert\Vert\Vert|$

$\mathrm{M}\mathrm{M}\mathrm{A}-\mathrm{P}1_{-}1\mathrm{P}-162$

Układ graficzny

\copyright CKE 2015




{\it Wzadaniach od l. do 25. wybierz i zaznacz na karcie odpowiedzi poprawnq odpowiedzí}.

Zadanie $*.(0-l\rangle$

Dla $\mathrm{k}\mathrm{a}\dot{\mathrm{z}}$ dej dodatniej liczby $a$ iloraz $\displaystyle \frac{a^{-2,6}}{a^{1,3}}$ jest równy

A.

$a^{-3,9}$

B.

$a^{-2}$

C.

$a^{-1,3}$

D.

$a^{1,3}$

Zadam$\mathrm{e}2. (0-1)$

Liczba $\log_{\sqrt{2}}(2\sqrt{2})$ jest równa

A.

-23

B. 2

C.

-25

D. 3

Zadanie 3. (0-1)

Liczby $a\mathrm{i}c$ są dodatnie. Liczba $b$ stanowi 48\% 1iczby $a$ oraz 32\% 1iczby $c$. Wynika stąd, $\dot{\mathrm{z}}\mathrm{e}$

A. $c=1,5a$

B. $c=1,6a$

C. $c=0,8a$

D. $c=0,16a$

Zadam$\mathrm{e}4.(0-1)$

Równość $(2\sqrt{2}-a)^{2}=17-12\sqrt{2}$ jest prawdziwa dla

A. $a=3$

B. $a=1$

C. $a=-2$

D. $a=-3$

Zadanie 5. $(0-1\rangle$

Jedną z liczb, które spełniają nierówność $-x^{5}+x^{3}-x<-2$, jest

A. l

B. $-1$

C. 2

D. $-2$

Zadam$\mathrm{e}6.(0-1)$

Proste o równaniach $2x-3y=4\mathrm{i}5x-6y=7$ przecinają się w punkcie $P$. Stąd wynika, $\dot{\mathrm{z}}\mathrm{e}$

A. $P=(1,2)$

B. $P=(-1,2)$

C. $P=(-1,-2)$

D. $P=(1,-2)$

Zadanie 7. (0-1)

Punkty ABCD $\mathrm{l}\mathrm{e}\dot{\mathrm{z}}$ ą na o ęgu o środku $\mathrm{S}$ (zobacz

Miara kąta $BDC$ jest równa

A. $91^{\mathrm{o}}$

B. $72,5^{\mathrm{o}}$

C. $18^{\mathrm{o}}$
\begin{center}
\includegraphics[width=90.828mm,height=95.352mm]{./F2_M_PP_M2016_page1_images/image001.eps}
\end{center}
sunek).

{\it D}

{\it C}

$27^{\mathrm{o}}$

{\it S}

$118^{\mathrm{o}}$

{\it B}

{\it A}

Strona 2 z24

D. $32^{\mathrm{o}}$

MMA-IP





{\it BRUDNOPIS} ({\it nie podlega ocenie})

Strona ll z24





Zadanie 26. $(0-2\rangle$

$\mathrm{W}$ tabeli przedstawiono roczne przyrosty wysokości pewnej sosny w ciągu sześciu kolejnych

lat.
\begin{center}
\begin{tabular}{|l|l|l|l|l|l|l|}
\hline
\multicolumn{1}{|l|}{kolejne lata}&	\multicolumn{1}{|l|}{$1$}&	\multicolumn{1}{|l|}{ $2$}&	\multicolumn{1}{|l|}{ $3$}&	\multicolumn{1}{|l|}{ $4$}&	\multicolumn{1}{|l|}{ $5$}&	\multicolumn{1}{|l|}{ $6$}	\\
\hline
\multicolumn{1}{|l|}{przyrost (w cm)}&	\multicolumn{1}{|l|}{$10$}&	\multicolumn{1}{|l|}{ $10$}&	\multicolumn{1}{|l|}{ $7$}&	\multicolumn{1}{|l|}{ $8$}&	\multicolumn{1}{|l|}{ $8$}&	\multicolumn{1}{|l|}{ $7$}	\\
\hline
\end{tabular}

\end{center}
Oblicz średni roczny przyrost wysokości tej sosny w badanym okresie sześciu lat. Otrzymany

wynik zaokrąglij do l cm. Oblicz błąd względny otrzymanego przyblizenia. Podaj ten błąd

w procentach.

Odpowiedzí:

Strona 12 z24

MMA-IP





Zadanie 27. (0-2)

Rozwiąz nierówność $2x^{2}-4x>3x^{2}-6x.$

Odpowied $\acute{\mathrm{z}}$:
\begin{center}
\includegraphics[width=96.012mm,height=17.784mm]{./F2_M_PP_M2016_page12_images/image001.eps}
\end{center}
Wypelnia

egzaminator

Nr zadania

Maks. liczba kt

2

27.

2

Uzyskana liczba pkt

IMA-IP

Strona 13 z24





Zadanie 28. (0-2)

Rozwiąz równanie $(4-x)(x^{2}+2x-15)=0.$

Odpowiedzí:

Strona 14 z24

MD





Zadanie 29. $(0-2\rangle$

Dany jest trójkąt prostokątny $ABC$. Na przyprostokątnych $AC\mathrm{i}$ AB tego trójkąta obrano

odpowiednio punkty $D\mathrm{i}G$. Na przeciwprostokątnej $BC$ wyznaczono punkty $E\mathrm{i}F$ takie, $\dot{\mathrm{z}}\mathrm{e}$

$|\triangleleft DEC|=|\wedge BGF|=90^{\mathrm{o}}$ (zobacz rysunek). Wykaz, $\dot{\mathrm{z}}\mathrm{e}$ trójkąt $CDE$ jest podobny do

trójkąta FBG.
\begin{center}
\includegraphics[width=87.780mm,height=55.728mm]{./F2_M_PP_M2016_page14_images/image001.eps}
\end{center}
{\it C}

{\it E}

{\it F}

{\it D}

{\it A  G B}
\begin{center}
\includegraphics[width=96.012mm,height=17.784mm]{./F2_M_PP_M2016_page14_images/image002.eps}
\end{center}
Wypelnia

egzaminator

Nr zadania

Maks. liczba kt

28.

2

2

Uzyskana liczba pkt

IMA-IP

Strona 15 z24





Zadanie 30. (0-2)

Ciąg $(a_{n})$ jest określony wzorem $a_{n}=2n^{2}+2n$ dla $n\geq 1$. Wykaz, $\dot{\mathrm{z}}\mathrm{e}$ suma $\mathrm{k}\mathrm{a}\dot{\mathrm{z}}$ dych dwóch

kolejnych wyrazów tego ciągu jest kwadratem liczby naturalnej.

Strona 16 z24

MMA-IP





Zadanie $3l. (0-2)$

Skala Richtera słuz$\mathrm{y}$ do określania siły trzęsień ziemi. Siła ta opisana jest wzorem

$R=\displaystyle \log\frac{A}{4_{\mathfrak{c}}}$, gdzie $A$ oznacza amplitudę trzęsienia wyrazoną w centymetrach, $A_{0}=10^{\rightarrow\iota}$ cm

jest stałą, nazywaną amplitudą wzorcową. 5 maja 2014 roku w Taj1andii miało miejsce

trzęsienie ziemi o sile 6,2 w ska1i Richtera. Ob1icz amp1itudę trzęsienia ziemi w Taj1andii

i rozstrzygnij, czyjest ona większa, czy- mniejsza od 100 cm.

Odpowied $\acute{\mathrm{z}}$:
\begin{center}
\includegraphics[width=96.012mm,height=17.784mm]{./F2_M_PP_M2016_page16_images/image001.eps}
\end{center}
Wypelnia

egzaminator

Nr zadania

Maks. liczba kt

30.

2

31.

2

Uzyskana liczba pkt

IMA-IP

Strona 17 z24





Zadanie 32. $(0-4$

Jeden z kątów trójkąta jest trzy razy większy od mniejszego z dwóch pozostałych kątów,

które róznią się o $50^{\mathrm{o}}$. Oblicz kąty tego trójkąta.

Strona 18 z24

MMA-IP





Odpowiedzí :
\begin{center}
\includegraphics[width=82.044mm,height=17.784mm]{./F2_M_PP_M2016_page18_images/image001.eps}
\end{center}
Wypelnia

egzamÍnator

Nr zadania

Maks. liczba kt

32.

4

Uzyskana liczba pkt

IMA-IP

Strona 19 z24





Zadanie 33. $(0-5\rangle$

Podstawą ostrosłupa prawidłowego trójkątnego ABCS jest trójkąt równoboczny $ABC.$

Wysokość SO tego ostrosłupajest równa wysokościjego podstawy. Objętość tego ostrosiupa

jest równa 27. Ob1icz po1e powierzchni bocznej ostrosłupa ABCS oraz cosinus kąta, jaki

tworzą wysokość ściany bocznej i płaszczyzna podstawy ostrosłupa.

Strona 20 z24

MMA-IP





{\it BRUDNOPIS} ({\it nie podlega ocenie})

Strona 3 z24





Odpowiedzí :
\begin{center}
\includegraphics[width=82.044mm,height=17.784mm]{./F2_M_PP_M2016_page20_images/image001.eps}
\end{center}
Wypelnia

egzamÍnator

Nr zadania

Maks. liczba kt

33.

5

Uzyskana liczba pkt

IMA-IP

Strona 21 z24





Zadanie 34. $\zeta 0-4$)

Ze zbioru wszystkich liczb naturalnych dwucyfrowych losujemy kolejno dwa razy po jednej

liczbie bez zwracania. Oblicz prawdopodobieństwo zdarzenia polegającego na tym, $\dot{\mathrm{z}}\mathrm{e}$ suma

wylosowanych liczb będzie równa 30. Wynik zapisz w postaci ułamka zwykłego

nieskracalnego.

Strona 22 z24

MMA-IP





Odpowiedzí :
\begin{center}
\includegraphics[width=82.044mm,height=17.784mm]{./F2_M_PP_M2016_page22_images/image001.eps}
\end{center}
Wypelnia

egzamÍnator

Nr zadania

Maks. liczba kt

34.

4

Uzyskana liczba pkt

IMA-IP

Strona 23 z24





{\it BRUDNOPIS} ({\it nie podlega ocenie})

Strona 24 z24

MD





Zadanie 8. (0-1)

Danajest funkcja liniowa $f(x)=\displaystyle \frac{3}{4}x+6$. Miejscem zerowym tej funkcjijest liczba

A. 8

B. 6

C. $-6$

D. $-8$

Zadam$\mathrm{e}9.(0-1)$

Równanie wymietne $\displaystyle \frac{3x-1}{x+5}=3$, gdzie $x\neq-5,$

A.

B.

C.

D.

nie ma rozwiązań rzeczywistych.

ma dokładniejedno rozwiązanie rzeczywiste.

ma dokładnie dwa rozwiązania rzeczywiste.

ma dokładnie trzy rozwiązania rzeczywiste.

InformaCja do zadat 10. $i11.$

Na rysunku przedstawiony jest fragment paraboli będącej wykresem funkcji kwadratowej $f$

Wierzchołkiem tej parabolijest punkt $W=(1,9)$. Liczby $-2\mathrm{i}4$ to miejsca zerowe funkcji $f.$
\begin{center}
\includegraphics[width=192.228mm,height=118.164mm]{./F2_M_PP_M2016_page3_images/image001.eps}
\end{center}
Zadanie NO. (0-1)

Zbiorem wartości funkcji f jest przedział

A.

$(-\infty'-2\rangle$

B. $\langle-2,  4\rangle$

C.

$\langle 4,+\infty)$

D. $(-\infty$' $ 9\rangle$

Zadanie ll. $(0-1\rangle$

Najmniejsza wartość funkcji $f$ w przedziale $\langle-1,2\rangle$ jest równa

A. 2

B. 5

C. 8

D. 9

Strona 4 z24

MMA-IP





{\it BRUDNOPIS} ({\it nie podlega ocenie})

Strona 5 z24





Zadanie 12. $(0-1\rangle$

Funkcja $f$ określona jest wzorem $f(x)=\displaystyle \frac{2x^{3}}{x^{6}+1}$ dla $\mathrm{k}\mathrm{a}\dot{\mathrm{z}}$ dej liczby rzeczywistej $x$. Wtedy

$f(-\sqrt[3]{3})$ jest równa

A.

$-\displaystyle \frac{\sqrt[3]{9}}{2}$

B.

- -53

C.

-53

D.

$\displaystyle \frac{\sqrt[3]{3}}{2}$

Zadanie 13. $(0-\mathrm{f}\rangle$

$\mathrm{W}$ okręgu o środku w punkcie $S$ poprowadzono cięciwę AB, która utworzyła z promieniem

$AS$ kąt o mierze $31^{\mathrm{o}}$ (zobacz rysunek). Promień tego okręgu ma długość 10. Od1egłość punktu

$S$ od cięciwy $AB$ jest liczbą z przedziału

A. $\displaystyle \{\frac{9}{2},\frac{11}{2}\}$

B. $\displaystyle \frac{11}{2}, \displaystyle \frac{13}{2}$

C. $\displaystyle \frac{13}{2}, \displaystyle \frac{19}{2}$
\begin{center}
\includegraphics[width=72.588mm,height=76.200mm]{./F2_M_PP_M2016_page5_images/image001.eps}
\end{center}
$B$

{\it K}

{\it S}

31

{\it A}

$\displaystyle \frac{19}{2}, \displaystyle \frac{37}{2}\}$

D.

Zadanie 14. $(0-1\rangle$

Cztetnasty wyraz ciągu arytmetycznegojest równy 8, a róznica tego ciągujest równa $(-\displaystyle \frac{3}{2}).$

Siódmy wyraz tego ciągujest równy

A.

$\displaystyle \frac{37}{2}$

B.

$-\displaystyle \frac{37}{2}$

C.

- -25

D.

-25

Zadanie 15. (0-1)

Ciąg $(x,2x+3,4x+3)$ jest geometryczny. Pierwszy wyraz tego ciągu jest równy

A. $-4$

B. l

C. 0

D. $-1$

Zadanie $l6. (0-1\rangle$

Przedstawione na rysunku trójkąty $ABC\mathrm{i}PQR$ są podobne. Bok $AB$ trójkąta $ABC$ ma długość

A. 8

B. 8,5

C. 9,5
\begin{center}
\includegraphics[width=105.660mm,height=60.456mm]{./F2_M_PP_M2016_page5_images/image002.eps}
\end{center}
18

{\it Q} $62^{\mathrm{o}} R$

{\it C}

17

9

$70^{\mathrm{o}}$

$70^{\mathrm{o}}  48^{\mathrm{o}}$

{\it A B}

{\it x  P}

D. 10

Strona 6 z24

MMA-IP





{\it BRUDNOPIS} ({\it nie podlega ocenie})

Strona 7 z24





Zadanie 17. $(0-1\rangle$

Kąt $\alpha$ jest ostry i $\displaystyle \mathrm{t}\mathrm{g}\alpha=\frac{2}{3}$. Wtedy

A.

$\mathrm{s}$i$\displaystyle \mathrm{n}\alpha=\frac{3\sqrt{13}}{26}$

B.

$\mathrm{s}$i$\displaystyle \mathrm{n}\alpha=\frac{\sqrt{13}}{13}$

C.

$\displaystyle \sin\alpha=\frac{2\sqrt{13}}{13}$

D.

$\mathrm{s}$i$\displaystyle \mathrm{n}\alpha=\frac{3\sqrt{13}}{13}$

Zadanie 18. (0-1)

$\mathrm{Z}$ odcinków o długościach: 5, $2a+1, a-1$ mozna zbudować trójkąt równoramienny. Wynika

stąd, $\dot{\mathrm{z}}\mathrm{e}$

A. $a=6$

B. $a=4$

C. $a=3$

D. $a=2$

ZadanÎe $l9. (0-1)$

Okręgi o promieniach 3 $\mathrm{i} 4$ są styczne zewnętrznie. Prosta styczna do okręgu

o promieniu 4 w punkcie $P$ przechodzi przez środek okręgu o promieniu 3 (zobacz rysunek).
\begin{center}
\includegraphics[width=171.504mm,height=116.184mm]{./F2_M_PP_M2016_page7_images/image001.eps}
\end{center}
{\it P}

$O_{1}$  3 4  $O_{2}$

Pole trójkąta, którego wierzchołkami są środki okręgów i punkt styczności P, jest równe

A. 14

B. $2\sqrt{33}$

C. $4\sqrt{33}$

D. 12

Zadanie 20. $(0-1\rangle$

Proste opisane równaniami $y=\displaystyle \frac{2}{m-1}x+m-2$ oraz $y=mx+\displaystyle \frac{1}{m+1}$ są prostopadłe, gdy

A. $m=2$

B.

{\it m}$=$ -21

C.

{\it m}$=$ -31

D. $m=-2$

Strona 8 z 24

MMA-IP





{\it BRUDNOPIS} ({\it nie podlega ocenie})

Strona 9 z24





Zadanie 21. $(0-\mathrm{f}\rangle$

$\mathrm{W}$ układzie współrzędnych dane są punkty $A=(a,6)$ oraz $B=(7,b)$. Środkiem odcinka $AB$

jest punkt $M=(3,4)$. Wynika stąd, $\dot{\mathrm{z}}\mathrm{e}$

A. $a=5 \mathrm{i}b=5$

B. $a=-1 \mathrm{i}b=2$

C. $a=4\mathrm{i}b=10$

D. $a=-4 \mathrm{i}b=-2$

Zadanie 32. (0-1)

Rzucamy trzy razy symetryczną monetą. Niech p oznacza prawdopodobieństwo otrzymania

dokładnie dwóch orłów w tych trzech rzutach. Wtedy

A. $0\leq p<0,2$

B. $0,2\leq p\leq 0,35$

C. $0,35<p\leq 0,5$

D. $0,5<p\leq 1$

Zadanie 23. $(0-1\rangle$

Kąt rozwarcia stozka ma miarę $120^{\mathrm{o}}$, a tworząca tego stozka ma długość 4. Objętość tego

stozkajest równa

A. $ 36\pi$

B. $ 18\pi$

C. $ 24\pi$

D. $ 8\pi$

Zadanie 24. (0-1)

Przekątna podstawy graniastosłupa prawidłowego czworokątnego jest dwa razy dłuzsza od

wysokości graniastosłupa. Graniastosłup przecięto płaszczyzną przechodzącą przez przekątną

podstawy ijeden wierzchołek drugiej podstawy (patrz rysunek).

Płaszczyzna przekroju tworzy z podstawą graniastosłupa kąt $\alpha$ o mierze

A. $30^{\mathrm{o}}$

B. $45^{\mathrm{o}}$

C. $60^{\mathrm{o}}$

D. $75^{\mathrm{o}}$

Zadanie 25. $(0-1\rangle$

Średnia arytmetyczna szeŚciu liczb naturalnych: 31, 16, 25, 29, 27, $x$, jest równa $\displaystyle \frac{x}{2}$. Mediana

tych liczb jest równa

A. 26

B. 27

C. 28

D. 29

Strona 10 z24

MMA-IP



\end{document}