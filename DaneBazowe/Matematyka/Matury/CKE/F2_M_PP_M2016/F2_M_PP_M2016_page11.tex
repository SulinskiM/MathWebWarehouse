\documentclass[a4paper,12pt]{article}
\usepackage{latexsym}
\usepackage{amsmath}
\usepackage{amssymb}
\usepackage{graphicx}
\usepackage{wrapfig}
\pagestyle{plain}
\usepackage{fancybox}
\usepackage{bm}

\begin{document}

Zadanie 26. $(0-2\rangle$

$\mathrm{W}$ tabeli przedstawiono roczne przyrosty wysokości pewnej sosny w ciągu sześciu kolejnych

lat.
\begin{center}
\begin{tabular}{|l|l|l|l|l|l|l|}
\hline
\multicolumn{1}{|l|}{kolejne lata}&	\multicolumn{1}{|l|}{$1$}&	\multicolumn{1}{|l|}{ $2$}&	\multicolumn{1}{|l|}{ $3$}&	\multicolumn{1}{|l|}{ $4$}&	\multicolumn{1}{|l|}{ $5$}&	\multicolumn{1}{|l|}{ $6$}	\\
\hline
\multicolumn{1}{|l|}{przyrost (w cm)}&	\multicolumn{1}{|l|}{$10$}&	\multicolumn{1}{|l|}{ $10$}&	\multicolumn{1}{|l|}{ $7$}&	\multicolumn{1}{|l|}{ $8$}&	\multicolumn{1}{|l|}{ $8$}&	\multicolumn{1}{|l|}{ $7$}	\\
\hline
\end{tabular}

\end{center}
Oblicz średni roczny przyrost wysokości tej sosny w badanym okresie sześciu lat. Otrzymany

wynik zaokrąglij do l cm. Oblicz błąd względny otrzymanego przyblizenia. Podaj ten błąd

w procentach.

Odpowiedzí:

Strona 12 z24

MMA-IP
\end{document}
