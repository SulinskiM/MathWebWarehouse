\documentclass[a4paper,12pt]{article}
\usepackage{latexsym}
\usepackage{amsmath}
\usepackage{amssymb}
\usepackage{graphicx}
\usepackage{wrapfig}
\pagestyle{plain}
\usepackage{fancybox}
\usepackage{bm}

\begin{document}

Zadanie $3l. (0-2)$

Skala Richtera słuz$\mathrm{y}$ do określania siły trzęsień ziemi. Siła ta opisana jest wzorem

$R=\displaystyle \log\frac{A}{4_{\mathfrak{c}}}$, gdzie $A$ oznacza amplitudę trzęsienia wyrazoną w centymetrach, $A_{0}=10^{\rightarrow\iota}$ cm

jest stałą, nazywaną amplitudą wzorcową. 5 maja 2014 roku w Taj1andii miało miejsce

trzęsienie ziemi o sile 6,2 w ska1i Richtera. Ob1icz amp1itudę trzęsienia ziemi w Taj1andii

i rozstrzygnij, czyjest ona większa, czy- mniejsza od 100 cm.

Odpowied $\acute{\mathrm{z}}$:
\begin{center}
\includegraphics[width=96.012mm,height=17.784mm]{./F2_M_PP_M2016_page16_images/image001.eps}
\end{center}
Wypelnia

egzaminator

Nr zadania

Maks. liczba kt

30.

2

31.

2

Uzyskana liczba pkt

IMA-IP

Strona 17 z24
\end{document}
