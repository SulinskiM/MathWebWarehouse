\documentclass[a4paper,12pt]{article}
\usepackage{latexsym}
\usepackage{amsmath}
\usepackage{amssymb}
\usepackage{graphicx}
\usepackage{wrapfig}
\pagestyle{plain}
\usepackage{fancybox}
\usepackage{bm}

\begin{document}

Zadanie 8. (0-1)

Danajest funkcja liniowa $f(x)=\displaystyle \frac{3}{4}x+6$. Miejscem zerowym tej funkcjijest liczba

A. 8

B. 6

C. $-6$

D. $-8$

Zadam$\mathrm{e}9.(0-1)$

Równanie wymietne $\displaystyle \frac{3x-1}{x+5}=3$, gdzie $x\neq-5,$

A.

B.

C.

D.

nie ma rozwiązań rzeczywistych.

ma dokładniejedno rozwiązanie rzeczywiste.

ma dokładnie dwa rozwiązania rzeczywiste.

ma dokładnie trzy rozwiązania rzeczywiste.

InformaCja do zadat 10. $i11.$

Na rysunku przedstawiony jest fragment paraboli będącej wykresem funkcji kwadratowej $f$

Wierzchołkiem tej parabolijest punkt $W=(1,9)$. Liczby $-2\mathrm{i}4$ to miejsca zerowe funkcji $f.$
\begin{center}
\includegraphics[width=192.228mm,height=118.164mm]{./F2_M_PP_M2016_page3_images/image001.eps}
\end{center}
Zadanie NO. (0-1)

Zbiorem wartości funkcji f jest przedział

A.

$(-\infty'-2\rangle$

B. $\langle-2,  4\rangle$

C.

$\langle 4,+\infty)$

D. $(-\infty$' $ 9\rangle$

Zadanie ll. $(0-1\rangle$

Najmniejsza wartość funkcji $f$ w przedziale $\langle-1,2\rangle$ jest równa

A. 2

B. 5

C. 8

D. 9

Strona 4 z24

MMA-IP
\end{document}
