\documentclass[a4paper,12pt]{article}
\usepackage{latexsym}
\usepackage{amsmath}
\usepackage{amssymb}
\usepackage{graphicx}
\usepackage{wrapfig}
\pagestyle{plain}
\usepackage{fancybox}
\usepackage{bm}

\begin{document}

Zadanie 12. $(0-1\rangle$

Funkcja $f$ określona jest wzorem $f(x)=\displaystyle \frac{2x^{3}}{x^{6}+1}$ dla $\mathrm{k}\mathrm{a}\dot{\mathrm{z}}$ dej liczby rzeczywistej $x$. Wtedy

$f(-\sqrt[3]{3})$ jest równa

A.

$-\displaystyle \frac{\sqrt[3]{9}}{2}$

B.

- -53

C.

-53

D.

$\displaystyle \frac{\sqrt[3]{3}}{2}$

Zadanie 13. $(0-\mathrm{f}\rangle$

$\mathrm{W}$ okręgu o środku w punkcie $S$ poprowadzono cięciwę AB, która utworzyła z promieniem

$AS$ kąt o mierze $31^{\mathrm{o}}$ (zobacz rysunek). Promień tego okręgu ma długość 10. Od1egłość punktu

$S$ od cięciwy $AB$ jest liczbą z przedziału

A. $\displaystyle \{\frac{9}{2},\frac{11}{2}\}$

B. $\displaystyle \frac{11}{2}, \displaystyle \frac{13}{2}$

C. $\displaystyle \frac{13}{2}, \displaystyle \frac{19}{2}$
\begin{center}
\includegraphics[width=72.588mm,height=76.200mm]{./F2_M_PP_M2016_page5_images/image001.eps}
\end{center}
$B$

{\it K}

{\it S}

31

{\it A}

$\displaystyle \frac{19}{2}, \displaystyle \frac{37}{2}\}$

D.

Zadanie 14. $(0-1\rangle$

Cztetnasty wyraz ciągu arytmetycznegojest równy 8, a róznica tego ciągujest równa $(-\displaystyle \frac{3}{2}).$

Siódmy wyraz tego ciągujest równy

A.

$\displaystyle \frac{37}{2}$

B.

$-\displaystyle \frac{37}{2}$

C.

- -25

D.

-25

Zadanie 15. (0-1)

Ciąg $(x,2x+3,4x+3)$ jest geometryczny. Pierwszy wyraz tego ciągu jest równy

A. $-4$

B. l

C. 0

D. $-1$

Zadanie $l6. (0-1\rangle$

Przedstawione na rysunku trójkąty $ABC\mathrm{i}PQR$ są podobne. Bok $AB$ trójkąta $ABC$ ma długość

A. 8

B. 8,5

C. 9,5
\begin{center}
\includegraphics[width=105.660mm,height=60.456mm]{./F2_M_PP_M2016_page5_images/image002.eps}
\end{center}
18

{\it Q} $62^{\mathrm{o}} R$

{\it C}

17

9

$70^{\mathrm{o}}$

$70^{\mathrm{o}}  48^{\mathrm{o}}$

{\it A B}

{\it x  P}

D. 10

Strona 6 z24

MMA-IP
\end{document}
