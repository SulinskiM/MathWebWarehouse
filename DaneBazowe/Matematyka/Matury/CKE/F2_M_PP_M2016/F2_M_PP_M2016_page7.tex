\documentclass[a4paper,12pt]{article}
\usepackage{latexsym}
\usepackage{amsmath}
\usepackage{amssymb}
\usepackage{graphicx}
\usepackage{wrapfig}
\pagestyle{plain}
\usepackage{fancybox}
\usepackage{bm}

\begin{document}

Zadanie 17. $(0-1\rangle$

Kąt $\alpha$ jest ostry i $\displaystyle \mathrm{t}\mathrm{g}\alpha=\frac{2}{3}$. Wtedy

A.

$\mathrm{s}$i$\displaystyle \mathrm{n}\alpha=\frac{3\sqrt{13}}{26}$

B.

$\mathrm{s}$i$\displaystyle \mathrm{n}\alpha=\frac{\sqrt{13}}{13}$

C.

$\displaystyle \sin\alpha=\frac{2\sqrt{13}}{13}$

D.

$\mathrm{s}$i$\displaystyle \mathrm{n}\alpha=\frac{3\sqrt{13}}{13}$

Zadanie 18. (0-1)

$\mathrm{Z}$ odcinków o długościach: 5, $2a+1, a-1$ mozna zbudować trójkąt równoramienny. Wynika

stąd, $\dot{\mathrm{z}}\mathrm{e}$

A. $a=6$

B. $a=4$

C. $a=3$

D. $a=2$

ZadanÎe $l9. (0-1)$

Okręgi o promieniach 3 $\mathrm{i} 4$ są styczne zewnętrznie. Prosta styczna do okręgu

o promieniu 4 w punkcie $P$ przechodzi przez środek okręgu o promieniu 3 (zobacz rysunek).
\begin{center}
\includegraphics[width=171.504mm,height=116.184mm]{./F2_M_PP_M2016_page7_images/image001.eps}
\end{center}
{\it P}

$O_{1}$  3 4  $O_{2}$

Pole trójkąta, którego wierzchołkami są środki okręgów i punkt styczności P, jest równe

A. 14

B. $2\sqrt{33}$

C. $4\sqrt{33}$

D. 12

Zadanie 20. $(0-1\rangle$

Proste opisane równaniami $y=\displaystyle \frac{2}{m-1}x+m-2$ oraz $y=mx+\displaystyle \frac{1}{m+1}$ są prostopadłe, gdy

A. $m=2$

B.

{\it m}$=$ -21

C.

{\it m}$=$ -31

D. $m=-2$

Strona 8 z 24

MMA-IP
\end{document}
