\documentclass[a4paper,12pt]{article}
\usepackage{latexsym}
\usepackage{amsmath}
\usepackage{amssymb}
\usepackage{graphicx}
\usepackage{wrapfig}
\pagestyle{plain}
\usepackage{fancybox}
\usepackage{bm}

\begin{document}

{\it Wzadaniach od l. do 25. wybierz i zaznacz na karcie odpowiedzi poprawnq odpowiedzí}.

Zadanie l. $(0-l)$

Liczba $5^{8}\cdot 16^{-2}$ jest równa

A. $(\displaystyle \frac{5}{2})^{8}$ B.

-25

Zadanie 2. (0-1)

Liczba $\sqrt[3]{54}-\sqrt[3]{2}$ jest równa

A. $\sqrt[3]{52}$ B. 3

Zadanie 3. $(0-l\rangle$

Liczba 2 $\log_{2}3-2\log_{2}5$ jest równa

A.

$\displaystyle \log_{2}\frac{9}{25}$

B.

$\log_{2} \displaystyle \frac{3}{5}$

C. $10^{8}$

D. 10

C. $2\sqrt[3]{2}$

D. 2

C.

$\log_{2} \displaystyle \frac{9}{5}$

D.

$\displaystyle \log_{2}\frac{6}{25}$

Zadanie 4. (0-1)

Liczba osobników pewnego zagrozonego wyginięciem gatunku zwierząt wzrosła w stosunku

do liczby tych zwierząt z 31 grudnia 2011 r. 0120\% i obecnie jest równa 8910. I1e zwierząt

liczyła populacja tego gatunku w ostatnim dniu 2011 roku?

A. 4050

B. 1782

C. 7425

D. 7128

Zadanie 5. (0-1)

Równość $(x\sqrt{2}-2)^{2}=(2+\sqrt{2})^{2}$ jest

A. prawdziwa dla $x=-\sqrt{2}.$

B. prawdziwa dla $x=\sqrt{2}.$

C. prawdziwa dla $x=-1.$

D. fałszywa dla $\mathrm{k}\mathrm{a}\dot{\mathrm{z}}$ dej liczby $x.$

Strona 2 z 26

MMA-II
\end{document}
