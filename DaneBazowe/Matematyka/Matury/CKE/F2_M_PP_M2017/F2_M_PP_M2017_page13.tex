\documentclass[a4paper,12pt]{article}
\usepackage{latexsym}
\usepackage{amsmath}
\usepackage{amssymb}
\usepackage{graphicx}
\usepackage{wrapfig}
\pagestyle{plain}
\usepackage{fancybox}
\usepackage{bm}

\begin{document}

Zadanie 22. $(0-1\rangle$

Promień AS podstawy walca jest równy wysokości $OS$ tego walca. Sinus kąta $OAS$ (zobacz

rysunek) jest równy
\begin{center}
\includegraphics[width=49.332mm,height=39.924mm]{./F2_M_PP_M2017_page13_images/image001.eps}
\end{center}
{\it O}

$\nearrow$

{\it S}

{\it A}

A.

-$\sqrt{}$23

B.

-$\sqrt{}$22

C.

-21

D. l

Zadanie 23. $(0-l\rangle$

Dany jest stozek o wysokości 4 i średnicy podstawy 12. Objętość tego stozkajest równa

A. $ 576\pi$

B. $ 192\pi$

C. $ 144\pi$

D. $ 48\pi$

Zadanie 24. $(0-1\rangle$

Średnia arytmetyczna oŚmiu liczb: 3, 5, 7, 9, $x$, 15, 17, $19$jest równa ll. Wtedy

A. $x=1$

B. $x=2$

C. $x=11$

D. $x=13$

Zadanie 25. (0-1)

Ze zbioru dwudziestu czterech kolejnych liczb naturalnych od l do 241osujemy jedną 1iczbę.

Niech $A$ oznacza zdarzenie, $\dot{\mathrm{z}}\mathrm{e}$ wylosowana liczba będzie dzielnikiem liczby 24. Wtedy

prawdopodobieństwo zdarzenia $A$ jest równe

A.

-41

B.

-31

C.

-81

D.

-61

Strona 14 z26

MMA-II
\end{document}
