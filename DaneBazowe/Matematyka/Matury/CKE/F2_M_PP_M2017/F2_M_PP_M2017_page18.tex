\documentclass[a4paper,12pt]{article}
\usepackage{latexsym}
\usepackage{amsmath}
\usepackage{amssymb}
\usepackage{graphicx}
\usepackage{wrapfig}
\pagestyle{plain}
\usepackage{fancybox}
\usepackage{bm}

\begin{document}

Zadanie 29. (0-4)

Funkcja kwadratowa $f$ jest określona dla wszystkich liczb rzeczywistych $x$ wzorem

$f(x)=ax^{2}+bx+c$. Największa wartość funkcji $f$ jest równa 6 oraz $f(-6)=f(0)=\displaystyle \frac{3}{2}.$

Oblicz wartość współczynnika $a.$

Odpowied $\acute{\mathrm{z}}.$
\begin{center}
\includegraphics[width=96.012mm,height=17.832mm]{./F2_M_PP_M2017_page18_images/image001.eps}
\end{center}
Wypelnia

egzaminator

Nr zadania

Maks. liczba kt

28.

2

4

Uzyskana liczba pkt

MMA-IP

Strona 19 z26
\end{document}
