\documentclass[a4paper,12pt]{article}
\usepackage{latexsym}
\usepackage{amsmath}
\usepackage{amssymb}
\usepackage{graphicx}
\usepackage{wrapfig}
\pagestyle{plain}
\usepackage{fancybox}
\usepackage{bm}

\begin{document}

Zadanie 10. $(0-1\rangle$

Na rysunku przedstawiono fragment wykresu

której miejsca zerowe to: $-3 \mathrm{i}1.$

funkcji kwadratowej $f(x)=ax^{2}+bx+c,$
\begin{center}
\includegraphics[width=86.004mm,height=100.380mm]{./F2_M_PP_M2017_page5_images/image001.eps}
\end{center}
{\it 5y}

)4

3

2

1

{\it x}

$-5$ -$4  -3 -2  0 \xi$

$-2$

$\rightarrow 3$

$-4$

Współczynnik c we wzorze funkcji f jest równy

A. l

B. 2

C. 3

D. 4

Zadanie 11. (0-1)

Na rysunku przedstawiono fragment wykresu funkcji wykładniczej $f$ określonej wzorem

$f(x)=a^{x}$. Punkt $A=(1,2)$ nalezy do tego wykresu ffinkcji.
\begin{center}
\includegraphics[width=143.460mm,height=75.588mm]{./F2_M_PP_M2017_page5_images/image002.eps}
\end{center}
Podstawa a potęgijest równa

A.

- -21

B.

-21

C. $-2$

D. 2

Strona 6 z26

MMA-II
\end{document}
