\documentclass[a4paper,12pt]{article}
\usepackage{latexsym}
\usepackage{amsmath}
\usepackage{amssymb}
\usepackage{graphicx}
\usepackage{wrapfig}
\pagestyle{plain}
\usepackage{fancybox}
\usepackage{bm}

\begin{document}

Zadanie 29. (0-2)

Okręgi o środkach odpowiednio $A\mathrm{i}B$ są styczne zewnętrznie i $\mathrm{k}\mathrm{a}\dot{\mathrm{z}}\mathrm{d}\mathrm{y}$ z nich jest styczny do

obu ramion danego kąta prostego (zobacz rysunek). Promień okręgu o środku $A$ jest równy 2.

{\it A}.

{\it B}.

Uzasadnij, $\dot{\mathrm{z}}\mathrm{e}$ promień okręgu o środku $B$ jest mniejszy od $\sqrt{2}-1.$
\begin{center}
\includegraphics[width=96.012mm,height=17.832mm]{./F2_M_PP_M2018_page16_images/image001.eps}
\end{center}
Wypelnia

egzaminator

Nr zadania

Maks. liczba kt

28.

2

2

Uzyskana liczba pkt

MMA-IP

Strona 17 z26
\end{document}
