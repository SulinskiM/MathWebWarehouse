\documentclass[a4paper,12pt]{article}
\usepackage{latexsym}
\usepackage{amsmath}
\usepackage{amssymb}
\usepackage{graphicx}
\usepackage{wrapfig}
\pagestyle{plain}
\usepackage{fancybox}
\usepackage{bm}

\begin{document}

Zadanie 16. $(0-l)$

Dany jest okrąg o środku $S$. Punkty $K, L\mathrm{i}M$ lez$\cdot$ą na tym okręgu. Na łuku $KL$ tego okręgu są

oparte kąty $KSL \mathrm{i} KML$ (zobacz rysunek), których miary $a \mathrm{i} \beta$ spełniają warunek

$\alpha+\beta=111^{\mathrm{o}}$. Wynika stąd, $\dot{\mathrm{z}}\mathrm{e}$
\begin{center}
\includegraphics[width=64.368mm,height=61.620mm]{./F2_M_PP_M2018_page7_images/image001.eps}
\end{center}
{\it M}

$\alpha$

{\it K  L}

A. $\alpha=74^{\mathrm{o}}$

B. $\alpha=76^{\mathrm{o}}$

C. $\alpha=70^{\mathrm{o}}$

D. $\alpha=72^{\mathrm{o}}$

Zadanie $17_{\alpha}(0-1)$

Dany jest trapez prostokątny KLMN, którego podstawy mają długości $|KL|=a, |MN|=b,$

$a>b$. Kąt $KLM$ ma miarę $60^{\mathrm{o}}$. Długość ramienia $LM$ tego trapezujest równa

{\it b}
\begin{center}
\includegraphics[width=89.868mm,height=45.516mm]{./F2_M_PP_M2018_page7_images/image002.eps}
\end{center}
{\it N  M}

{\it K  a  L}

A. $a-b$

B. $2(a-b)$

C.

$a+\displaystyle \frac{1}{2}b$

D.

-{\it a} $+$2 {\it b}

Zadanie 18. $(0-l)$

Punkt $K=(2,2)$ jest wierzchołkiem trójkąta równoramiennego $KLM$, w którym $|KM|=|LM|$

Odcinek MNjest wysokością trójkąta i $N=(4,3)$. Zatem

A. $L=(5,3)$

B. $L=(6,4)$

C. $L=(3,5)$

D. $L=(4,6)$

Zadanie $l9. (0\rightarrow l)$

Proste o równaniach $y=(m+2)x+3$ oraz $y=(2m-1)x-3$ są równoległe, gdy

A. $m=2$

B. $m=3$

C. $m=0$

D. $m=1$

Strona 8 z26

MMA-IP
\end{document}
