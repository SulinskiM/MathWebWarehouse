\documentclass[a4paper,12pt]{article}
\usepackage{latexsym}
\usepackage{amsmath}
\usepackage{amssymb}
\usepackage{graphicx}
\usepackage{wrapfig}
\pagestyle{plain}
\usepackage{fancybox}
\usepackage{bm}

\begin{document}

Zadanie 20. $(0-l)$

Podstawą ostrosłupa jest kwadrat KLMN o boku długości 4. Wysokością tego ostrosłupajest

krawędzí $NS$, ajej długość $\mathrm{t}\mathrm{e}\dot{\mathrm{z}}$ jest równa 4 (zobacz rysunek).

Kąt $\alpha$, jaki tworzą krawędzie $KS\mathrm{i}MS$, spełnia warunek

A. $\alpha=45^{\mathrm{o}}$

B. $45^{\mathrm{o}}<\alpha<60^{\mathrm{o}}$

C. $\alpha>60^{\mathrm{o}}$

D. $\alpha=60^{\mathrm{o}}$

Zadanie $2l. (0-l)$

Podstawą graniastosłupa prostegojest prostokąt o bokach długości 3 $\mathrm{i}4$. Kąt $cx$, jaki przekątna

tego graniastosłupa tworzy zjego podstawą, jest równy $45^{\mathrm{o}}$ (zobacz rysunek).

Wysokość graniastosłupajest równa

A. 5

B. $3\sqrt{2}$

C. $5\sqrt{2}$

D.

$\displaystyle \frac{5\sqrt{3}}{3}$

Zadanie 22. (0-1)

Na rysunku przedstawiono bryłę zbudowaną z walca i półkuli. Wysokość walcajest równa r

ijest taka samajak promień półkuli oraz taka samajak promień podstawy walca.

Objętość tej bryłyjest równa

A.

$\displaystyle \frac{5}{3}\pi r^{3}$

B.

$\displaystyle \frac{4}{3}\pi r^{3}$

C.

$\displaystyle \frac{2}{3}\pi r^{3}$

D.

$\displaystyle \frac{1}{3}\pi r^{3}$

Strona 10 z26

MMA-IP
\end{document}
