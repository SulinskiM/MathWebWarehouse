\documentclass[a4paper,12pt]{article}
\usepackage{latexsym}
\usepackage{amsmath}
\usepackage{amssymb}
\usepackage{graphicx}
\usepackage{wrapfig}
\pagestyle{plain}
\usepackage{fancybox}
\usepackage{bm}

\begin{document}

{\it W kazdym z zadań od l. do 25. wybierz i zaznacz na karcie odpowiedzi poprawnq odpowiedzí}.

Zadanie l. $(0-1\rangle$

Liczba $\log_{\sqrt{2}}2$ jest równa

A. 2

B. 4

C.

$\sqrt{2}$

D.

-21

Zadanie 2. (0-1)

Liczba naturalna $n=2^{14}\cdot 5^{15}$ w zapisie dziesiętnym ma

A. 14 cyfr

B. 15 cyfr

C. 16 cyfr

D. 30 cyfr

Zadanie 3. (0-1)

$\mathrm{W}$ pewnym banku prowizja od udzielanych kredytów hipotecznych przez cały styczeń była

równa 4\%. Na początku 1utego ten bank obnizył wysokość prowizji od wszystkich kredytów

$0 1$ punkt procentowy. Oznacza to, $\dot{\mathrm{z}}\mathrm{e}$ prowizja od kredytów hipotecznych w tym banku

zmniejszyła się o

A. l\%

B. 25\%

C. 33\%

D. 75\%

Zadanie 4. $(0-l)$

Równość $\displaystyle \frac{1}{4}+\frac{1}{5}+\frac{1}{a}=1$ jest prawdziwa dla

A.

$a=\displaystyle \frac{11}{20}$

B.

{\it a}$=$ -98

C.

{\it a}$=$ -98

D.

{\it a}$=$ -2101

Zadanie 5. $(0-l)$

Para liczb $x=2 \mathrm{i}y=2$ jest rozwiązaniem układu równań 

A. $a=-1$

B. $a=1$

C. $a=-2$

D. $a=2$

Zadanie 6. $(0-l)$

Równanie $\displaystyle \frac{(x-1)(x+2)}{x-3}=0$

A. ma trzy rózne rozwiązania: $x=1, x=3, x=-2.$

B. ma trzy rózne rozwiązania: $x=-1, x=-3, x=2.$

C. ma dwa rózne rozwiązania: $x=1, x=-2.$

D. ma dwa rózne rozwiązania: $x=-1, x=2.$

Strona 2 z26

MMA-IP
\end{document}
