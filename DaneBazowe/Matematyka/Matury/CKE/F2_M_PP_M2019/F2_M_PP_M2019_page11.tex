\documentclass[a4paper,12pt]{article}
\usepackage{latexsym}
\usepackage{amsmath}
\usepackage{amssymb}
\usepackage{graphicx}
\usepackage{wrapfig}
\pagestyle{plain}
\usepackage{fancybox}
\usepackage{bm}

\begin{document}

Zadanie 22. $(0\rightarrow 1)$

Promień kuli i promień podstawy stozka są równe 4. Po1e powierzchni ku1i jest równe po1u

powierzchni całkowitej stozka. Długość tworzącej stozka jest równa

A. 8

B. 4

C. 16

D. 12

Zadanie 23. $(0-l)$

Mediana zestawu sześciu danych liczb: 4, 8, 21, $a$, 16, 25, jest równa l4. Zatem

A. $a=7$

B. $a=12$

C. $a=14$

D. $a=20$

Zadanie 24, (0-1)

Wszystkich liczb pięciocyfrowych, w których występują wyłącznie cyfry 0, 2, 5, jest

A.

12

B. 36

C. 162

D. 243

Zadanie 25. (0-1)

$\mathrm{W}$ pudełku jest 40 ku1. Wśród nich jest 35 ku1 białych, a pozostałe to ku1e czerwone.

Prawdopodobieństwo wylosowania kazdej kuli jest takie samo. $\mathrm{Z}$ pudełka losujemy jedną

kulę. Prawdopodobieństwo zdarzenia polegającego na tym, $\dot{\mathrm{z}}\mathrm{e}$ otrzymamy kulę czerwoną, jest

równe

A.

-81

B.

-51

C.

$\displaystyle \frac{1}{40}$

D.

$\displaystyle \frac{1}{35}$

Strona 12 z 26

MMA-IP
\end{document}
