\documentclass[a4paper,12pt]{article}
\usepackage{latexsym}
\usepackage{amsmath}
\usepackage{amssymb}
\usepackage{graphicx}
\usepackage{wrapfig}
\pagestyle{plain}
\usepackage{fancybox}
\usepackage{bm}

\begin{document}

Zadanie 29. $(0\rightarrow 2\rangle$

Dany jest okrąg o Środku w punkcie $S$ i promieniu $r$. Na przedłuzeniu cięciwy $AB$ poza

punkt $B$ odłozono odcinek $BC$ równy promieniowi danego okręgu. Przez punkty $C \mathrm{i} S$

poprowadzono prostą. Prosta $CS$ przecina dany okrąg w punktach $D\mathrm{i}E$ (zobacz rysunek).

Wykaz, $\dot{\mathrm{z}}$ ejezeli miara kąta ACSjest równa $a$, to miara kąta $ASD$ jest równa $3a.$
\begin{center}
\includegraphics[width=96.924mm,height=62.580mm]{./F2_M_PP_M2019_page16_images/image001.eps}
\end{center}
{\it D  r}

{\it S}

{\it r}

{\it E}

{\it r  r}

{\it C}

{\it A  B}
\begin{center}
\includegraphics[width=96.012mm,height=17.832mm]{./F2_M_PP_M2019_page16_images/image002.eps}
\end{center}
Wypelnia

egzaminator

Nr zadania

Maks. liczba kt

28.

2

2

Uzyskana liczba pkt

MMA-IP

Strona 17 z 26
\end{document}
