\documentclass[a4paper,12pt]{article}
\usepackage{latexsym}
\usepackage{amsmath}
\usepackage{amssymb}
\usepackage{graphicx}
\usepackage{wrapfig}
\pagestyle{plain}
\usepackage{fancybox}
\usepackage{bm}

\begin{document}

Zadanie ll. $(0-l)$

$\mathrm{W}$ ciągu arytmetycznym $(a_{n})$, określonym dla $n\geq 1$, dane są dwa wyrazy: $a_{1}=7\mathrm{i}a_{8}=-49.$

Suma ośmiu początkowych wyrazów tego ciągujest równa

A. $-168$

B. $-189$

C. $-21$

D. $-42$

Zadanie 12. (0-1)

Dany jest ciąg geometryczny $(a_{n})$, określony dla $n\geq 1$. Wszystkie wyrazy tego ciągu są

dodatnie i spełnionyjest watunek $\displaystyle \frac{a_{5}}{a_{3}}=\frac{1}{9}$. Iloraz tego ciągu jest równy

A.

-31

B.

-$\sqrt{}$13

C. 3

D. $\sqrt{3}$

Zadanie 13. (0-1)

Sinus kąta ostrego $a$ jest równy $\displaystyle \frac{4}{5}$. Wtedy

A.

$\displaystyle \cos\alpha=\frac{5}{4}$

B.

$\displaystyle \cos\alpha=\frac{1}{5}$

C.

$\displaystyle \cos\alpha=\frac{9}{25}$

D.

$\displaystyle \cos\alpha=\frac{3}{5}$

Zadanie 14, $(0-l)$

Punkty $D\mathrm{i}E$ lez$\cdot$ą na okręgu opisanym na trójkącie równobocznym $ABC$ (zobacz rysunek).

Odcinek $CD$ jest średnicą tego okręgu. Kąt wpisany $DEB$ ma miarę $\alpha.$
\begin{center}
\includegraphics[width=46.788mm,height=52.680mm]{./F2_M_PP_M2019_page5_images/image001.eps}
\end{center}
{\it C}

{\it E}

$\alpha$

{\it A  B}

{\it D}

Zatem

A. $\alpha=30^{\mathrm{o}}$

B. $\alpha<30^{\mathrm{o}}$

C. $\alpha>45^{\mathrm{o}}$

D. $\alpha=45^{\mathrm{o}}$

Strona 6 z26

MMA-IP
\end{document}
