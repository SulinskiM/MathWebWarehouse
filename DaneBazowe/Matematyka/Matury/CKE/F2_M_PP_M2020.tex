\documentclass[a4paper,12pt]{article}
\usepackage{latexsym}
\usepackage{amsmath}
\usepackage{amssymb}
\usepackage{graphicx}
\usepackage{wrapfig}
\pagestyle{plain}
\usepackage{fancybox}
\usepackage{bm}

\begin{document}

$\mathrm{g}_{\mathrm{E}\mathrm{G}\mathrm{Z}\mathrm{A}\mathrm{M}\mathrm{I}\mathrm{N}\mathrm{A}\mathrm{C}\mathrm{Y}\mathrm{J}\mathrm{N}\mathrm{A}}^{\mathrm{C}\mathrm{E}\mathrm{N}\mathrm{T}\mathrm{R}\mathrm{A}\mathrm{L}\mathrm{N}\mathrm{A}}$KOMISJA

Arkusz zawiera informacje

prawnie chronione do momentu

rozpoczęcia egzaminu.

WYPELNIA ZDAJACY

{\it miejsce}

{\it na naklejkę}
\begin{center}
\includegraphics[width=21.900mm,height=16.104mm]{./F2_M_PP_M2020_page0_images/image001.eps}
\end{center}
KOD
\begin{center}
\includegraphics[width=79.608mm,height=16.104mm]{./F2_M_PP_M2020_page0_images/image002.eps}
\end{center}
PESEL
\begin{center}
\includegraphics[width=193.644mm,height=264.720mm]{./F2_M_PP_M2020_page0_images/image003.eps}
\end{center}
EGZAMIN MATU  LNY

Z MATEMATY

POZIOM PODSTAWOWY

DATA: 5 maja 2020 $\mathrm{r}.$

LICZBA P KTÓW DO UZYS NIA: 50

Instrukcja dla zdającego

1.

2.

3.

4.

5.

Sprawdzí, czy ar sz egzaminacyjny zawiera 26 stron (zadania $1-34$).

Ewentualny brak zgłoś przewodniczącemu zespo nadzorującego

egzamin.

Rozwiązania zadań i odpowiedzi wpisuj w miejscu na to przeznaczonym.

Odpowiedzi do zadań za ię ch $(1-25)$ zaznacz na karcie odpowiedzi,

w części ka przeznaczonej dla zdającego. Zamaluj $\blacksquare$ pola do tego

przeznaczone. Blędne zaznaczenie otocz kólkiem \copyright i zaznacz wlaściwe.

Pamiętaj, $\dot{\mathrm{z}}\mathrm{e}$ pominięcie argumentacji lub istotnych obliczeń

w rozwiązaniu zadania otwa ego (26-34) $\mathrm{m}\mathrm{o}\dot{\mathrm{z}}\mathrm{e}$ spowodować, $\dot{\mathrm{z}}\mathrm{e}$ za to

rozwiązanie nie otr masz pełnej liczby pu tów.

Pisz cz elnie i $\mathrm{u}\dot{\mathrm{z}}$ aj lko $\mathrm{d}$ gopisu lub pióra z czamym tuszem lub

atramentem.

6. Nie $\mathrm{u}\dot{\mathrm{z}}$ aj korektora, a błędne zapisy razínie prze eśl.

7. Pamiętaj, $\dot{\mathrm{z}}\mathrm{e}$ zapisy w brudnopisie nie będą oceniane.

8. $\mathrm{M}\mathrm{o}\dot{\mathrm{z}}$ esz korzystać z zesta wzorów matema cznych, cyrkla i linijki,

a ta $\mathrm{e}$ z kalkulatora prostego.

9. Na tej stronie oraz na karcie odpowiedzi wpisz swój numer PESEL

i przyklej naklejkę z kodem.

10. Nie wpisuj $\dot{\mathrm{z}}$ adnych znaków w części przeznaczonej dla egzaminatora.

$\Vert\Vert\Vert\Vert\Vert\Vert\Vert\Vert\Vert\Vert\Vert\Vert\Vert\Vert\Vert\Vert\Vert\Vert\Vert\Vert\Vert\Vert\Vert\Vert|$

$\mathrm{M}\mathrm{M}\mathrm{A}-\mathrm{P}1_{-}1\mathrm{P}-202$

Układ graficzny

\copyright CKE 2015

$| 1$




{\it W kazdym z zadań od l. do 25. wybierz i zaznacz na karcie odpowiedzi poprawnq odpowiedzí}.

Zadanie 1. (0-1)

Wartość wyrazenia $x^{2}-6x+9$ dla $x=\sqrt{3}+3$

A. l

B. 3

Zadanie2. (0-1)

Liczba $\displaystyle \frac{2^{50}\cdot 3^{40}}{36^{10}}$ jest równa

A.

$6^{70}$

B. $6^{45}$

Zadanie 3. $(0-1\rangle$

Liczba $\log_{5}\sqrt{125}$ jest równa

A.

-23

B. 2

est równa

C. $1+2\sqrt{3}$

D. $1-2\sqrt{3}$

C. $2^{30}\cdot 3^{20}$

D. $2^{10}\cdot 3^{20}$

C. 3

D.

-23

Zadanie 4. $(0-1\rangle$

Cenę $x$ pewnego towaru obnizono o 20\% i otrzymano cenę $y$. Aby przywrócić cenę $x$, nową

cenę $y$ nalezy podnieść o

A. 25\%

B. 20\%

C. 15\%

D. 12\%

Zadanie 5, $(0-1\rangle$

Zbiorem wszystkich rozwiązań nierówności 3 $(1-x)>2(3x-1)-12x$ jest przedział

A.

$(-\displaystyle \frac{5}{3},+\infty)$

B.

(-$\infty$, -35)

C.

$(\displaystyle \frac{5}{3},+\infty)$

D.

(-$\infty$'- -35)

Zadanie 6. (0-1)

Suma wszystkich rozwiązań równania $x(x-3)(x+2)=0$ jest równa

A. 0

B. l

C. 2

D. 3

Strona 2 z26

MMA-IP





{\it BRUDNOPIS} ({\it nie podlega ocenie})

$\mathrm{A}_{-}1\mathrm{P}$

Strona ll z26





Zadanie 23. (0-1)

Cztery liczby: 2, 3, a, 8, tworzące zestaw danych, są uporządkowane rosnąco. Mediana tego

zestawu czterech danychjest równa medianie zestawu pięciu danych: 5, 3, 6, 8, 2. Zatem

A. $a=7$

B. $a=6$

C. $a=5$

D. $a=4$

ZadanIe 24. $(0\rightarrow 1$\}

Przekątna sześcianu ma długość $4\sqrt{3}$. Pole powierzchni tego sześcianujest równe

A. 96

B. $24\sqrt{3}$

C. 192

D. $16\sqrt{3}$

Zadanie 25. $(0\rightarrow 1)$

Dwa stozki o takich samych podstawach połączono podstawami w taki sposób jak na rysunku.

Stosunek wysokości tych stozkówjest równy 3: 2. Objętość stozka o krótszej wysokościjest

równa 12 $\mathrm{c}\mathrm{m}^{3}$

Objętość bryły utworzonej z połączonych stozkówjest równa

A.

20 $\mathrm{c}\mathrm{m}^{3}$

B.

$30\mathrm{c}\mathrm{m}^{3}$

C.

$39\mathrm{c}\mathrm{m}^{3}$

D. 52, $5\mathrm{c}\mathrm{m}^{3}$

Strona 12 z26

MMA-IP





{\it BRUDNOPIS} ({\it nie podlega ocenie})

$\mathrm{A}_{-}1\mathrm{P}$

Strona 13 z26





Zadanie $2\epsilon. (0-2)$

Rozwiąz nierówność 2 $(x-1)(x+3)>x-1.$

Odpowiedzí:

Strona 14 z26

MMA-I]





Zadanie 27. (0-2)

Rozwiąz równanie $(x^{2}-1)(x^{2}-2x)=0.$

Odpowiedzí:
\begin{center}
\includegraphics[width=96.012mm,height=17.784mm]{./F2_M_PP_M2020_page14_images/image001.eps}
\end{center}
WypelnÍa

egzaminator

Nr zadanÍa

Maks. lÍczba kt

2

27.

2

Uzyskana liczba pkt

MMA-IP

Strona 15 z26





Zadanie 2@. (0-2)

Wykaz, $\dot{\mathrm{z}}\mathrm{e}$ dlakazdych dwóch róznych liczb rzeczywistych $a\mathrm{i}b$ prawdziwajest nierówność

$a(a-2b)+2b^{2}>0.$

Strona 16 z26

MMA-IP





Zadanie 29. (0-2)

Trójkąt ABCjest równoboczny. Punkt $E$ lezy na wysokości $CD$ tego trójkąta oraz $|CE|=\displaystyle \frac{3}{4}|CD|.$

Punkt $F$ lezy na boku $BC$ i odcinek $EF$ jest prostopadły do $BC$ (zobacz rysunek).
\begin{center}
\includegraphics[width=82.140mm,height=73.968mm]{./F2_M_PP_M2020_page16_images/image001.eps}
\end{center}
{\it C}

{\it F}

{\it A  D  B}

Wykaz, $\displaystyle \dot{\mathrm{z}}\mathrm{e}|CF|=\frac{9}{16}|CB|.$
\begin{center}
\includegraphics[width=96.012mm,height=17.784mm]{./F2_M_PP_M2020_page16_images/image002.eps}
\end{center}
WypelnÍa

egzaminator

Nr zadanÍa

Maks. lÍczba kt

28.

2

2

Uzyskana liczba pkt

MMA-IP

Strona 17 z26





Zadani\S 30. (0-2)

Rzucamy dwa razy symetryczną sześcienną kostką do gry, która na $\mathrm{k}\mathrm{a}\dot{\mathrm{z}}$ dej ściance ma inną

liczbę oczek-odjednego oczka do sześciu oczek. Oblicz prawdopodobieństwo zdarzenia $A$

polegającego na tym, ze co najmniej jeden raz wypadnie ścianka z pięcioma oczkami.

Odpowiedzí:

Strona 18 z26

MMA-IP





Zadani\S $3l. (0-2)$

Kąt $\alpha$ jest ostry i spełnia warunek $\displaystyle \frac{2\sin\alpha+3\cos\alpha}{\cos\alpha}=4$. Oblicz tangens kąta $\alpha.$

Odpowiedzí:
\begin{center}
\includegraphics[width=96.012mm,height=17.832mm]{./F2_M_PP_M2020_page18_images/image001.eps}
\end{center}
Wypelnia

egzaminator

Nr zadania

Maks. liczba kt

30.

2

31.

2

Uzyskana liczba pkt

MMA-IP

Strona 19 z26





Zadani\S 32. $(0-4\rangle$

Dany jest kwadrat ABCD, w którym $A=(5,-\displaystyle \frac{5}{3})$. Przekątna $BD$ tego kwadratu jest zawarta

w prostej o równaniu $y=\displaystyle \frac{4}{3}x$. Oblicz współrzędne punktu przecięcia przekątnych $AC\mathrm{i}BD$ oraz

pole kwadratu ABCD.

Strona 20 z26

MMA-IP





{\it BRUDNOPIS} ({\it nie podlega ocenie})

$\mathrm{A}_{-}1\mathrm{P}$

Strona 3 z 26





Odpowied $\acute{\mathrm{z}}$:
\begin{center}
\includegraphics[width=82.044mm,height=17.784mm]{./F2_M_PP_M2020_page20_images/image001.eps}
\end{center}
Nr zadania

Wypelnia Maks. liczba kt

egzamÍnator

Uzyskana liczba pkt

32.

4

MMA-IP

Strona 21 z26





Zadani\S 33. $(0-4\rangle$

Wszystkie wyrazy ciągu geometrycznego $(a_{n})$, określonego dla $n\geq 1$, są dodatnie. Wyrazy tego

ciągu spełniają warunek $6a_{1}-5a_{2}+a_{3}=0$. Oblicz iloraz

$\langle 2\sqrt{2}, 3\sqrt{2}\rangle.$

q tego ciągu nalezący do przedziatu

Strona 22 z26

MMA-IP





Odpowiedzí:
\begin{center}
\includegraphics[width=82.044mm,height=17.832mm]{./F2_M_PP_M2020_page22_images/image001.eps}
\end{center}
Wypelnia

egzaminator

Nr zadania

Maks. liczba kt

33.

4

Uzyskana liczba pkt

MMA-IP

Strona 23 z26





Zadani\S 34. $(0-5\rangle$

Dany jest ostrosłup prawidłowy czworokątny ABCDS, którego krawędzí boczna ma długość 6

(zobacz rysunek). Ściana boczna tego ostrosłupajest nachylona do płaszczyzny podstawy pod

kątem, którego tangensjest równy $\sqrt{7}$. Oblicz objętość tego ostrosłupa.

Strona 24 z26

MMA-IP





Odpowiedzí:
\begin{center}
\includegraphics[width=82.044mm,height=17.832mm]{./F2_M_PP_M2020_page24_images/image001.eps}
\end{center}
Wypelnia

egzaminator

Nr zadania

Maks. liczba kt

34.

5

Uzyskana liczba pkt

MMA-IP

Strona 25 z26





{\it BRUDNOPIS} ({\it nie podlega ocenie})

Strona 26 z26

MMA-I]





Imformacja do zadań 7.$-9.$

Funkcja

kwadratowa f jest

określona

wzorem

$f(x)=a(x-1)(x-3)$. Na rysunku

przedstawiono fragment paraboli będącej wykresem tej ffinkcji. Wierzchołkiem tej parabolijest

punkt $W=(2,1).$
\begin{center}
\includegraphics[width=117.912mm,height=97.128mm]{./F2_M_PP_M2020_page3_images/image001.eps}
\end{center}
4  {\it y}

3

2

{\it W}

1

$| 1  | 1$  1

$-4 -3  -2 -1$  0  1 2 3 4  {\it 5 x}

$-1$

$-2$

$-3$

$-4$

Zadanie 7. (0-1)

Współczynnik a we wzorze funkcji f jest równy

A. l

B. 2

C. $-2$

D. $-1$

Zadaqie @. (0-1)

Największa wartość funkcji $f$ w przedziale $\langle$1, $ 4\rangle$ jest równa

A. $-3$

B. 0

C. l

D. 2

Zadanie 9. $(0-1\rangle$

Osią symetrii paraboli będącej wykresem ffinkcji $f$ jest prosta o równaniu

A. $x=1$

B. $x=2$

C. $y=1$

D. $y=2$

Strona 4 z 26

MMA-IP





{\it BRUDNOPIS} ({\it nie podlega ocenie})

$\mathrm{A}_{-}1\mathrm{P}$

Strona 5 z 26





Zadanie $l0. (0\rightarrow 1)$

Równanie $x(x-2)=(x-2)^{2}$ w zbiorze liczb rzeczywistych

A. nie ma rozwiązań.

B. ma dokładniejedno rozwiązanie: $x=2.$

C. ma dokładniejedno rozwiązanie: $x=0.$

D. ma dwa rózne rozwiązania: $x=1 \mathrm{i}x=2.$

Zadanie ll, $(0-1\rangle$

Na iysunku przedstawiono fiiagment wykresu funkcji liniowej $f$ określonej wzorem $f(x)=ax+b.$
\begin{center}
\includegraphics[width=118.008mm,height=97.788mm]{./F2_M_PP_M2020_page5_images/image001.eps}
\end{center}
4  {\it y}

3

1

$-4 -3  -2$

$-1 0$

$-1$

1 2 3 4  5  {\it x}

$-2$

$-3$

$-4$

Współczynniki a oraz b we wzorze funkcji f spełniają zalezność

A. $a+b>0$

B. $a+b=0$

C. $a\cdot b>0$

D. $a\cdot b<0$

ZadanIe 12. $(0\rightarrow 1$\}

Funkcja $f$ jest określona wzorem $f(x)=4^{-x}+1$ dla kazdej liczby rzeczywistej $x$. Liczba $f(\displaystyle \frac{1}{2})$

jest równa

A.

-21

B.

-23

C. 3

D. 17

Zadanie 13. $(0-1\rangle$

Proste o równaniach $y=(m-2)x$ oraz $y=\displaystyle \frac{3}{4}x+7$ są równoległe. Wtedy

A.

{\it m}$=$- -45

B.

{\it m}$=$ -23

C.

$m=\displaystyle \frac{11}{4}$

D.

$m=\displaystyle \frac{10}{3}$

Strona 6 z 26

MMA-IP





{\it BRUDNOPIS} ({\it nie podlega ocenie})

$\mathrm{A}_{-}1\mathrm{P}$

Strona 7 z26





Zadanie 14. $(0\rightarrow 1)$

Ciąg $(a_{n})$ jest określony wzorem $a_{n}=2n^{2}$ dla $n\geq 1$. Róz$\cdot$nica $a_{5}-a_{4}$ jest równa

A. 4

B. 20

C. 36

D. 18

Zadanie 15. $(0\rightarrow 1)$

$\mathrm{W}$ ciągu arytmetycznym $(a_{n})$, określonym dla $n\geq 1$, czwarty wyraz jest równy 3, a róznica

tego ciągujest równa 5. Suma $a_{1}+a_{2}+a_{3}+a_{4}$ jest równa

A. $-42$

B. $-36$

C. $-18$

D. 6

Zadanie $l6. (0-1\rangle$

Punkt $A=(\displaystyle \frac{1}{3},-1)$ nalezy do wykresu funkcji liniowej $f$ określonej wzorem $f(x)=3x+b.$

Wynika stąd, $\dot{\mathrm{z}}\mathrm{e}$

A. $b=2$

B. $b=1$

C. $b=-1$

D. $b=-2$

Zadanie $l7. (0-1\rangle$

Punkty $A, B, C, D$ lez$\cdot$ą na okręgu o środku w punkcie $O$. Kąt środkowy DOC ma miarę $118^{\mathrm{o}}$

(zobacz rysunek).
\begin{center}
\includegraphics[width=57.612mm,height=60.804mm]{./F2_M_PP_M2020_page7_images/image001.eps}
\end{center}
{\it B D}

{\it O}  $118^{\mathrm{o}}$

{\it A C}

Miara kąta ABC jest równa

A. $59^{\mathrm{o}}$

B. $48^{\mathrm{o}}$

C. $62^{\mathrm{o}}$

D. $31^{\mathrm{o}}$

Zadanie $l8. (0-1)$

Prosta przechodząca przez punkty $A=(3,-2)\mathrm{i}B=(-1,6)$ jest określona równaniem

A. $y=-2x+4$

B. $y=-2x-8$

C. $y=2x+8$

D. $y=2x-4$

Strona 8 z26

MMA-IP





{\it BRUDNOPIS} ({\it nie podlega ocenie})

$\mathrm{A}_{-}1\mathrm{P}$

Strona 9 z26





Zadanie $l9. (0\rightarrow 1)$

Danyjest trójkąt prostokątny o kątach ostrych $\alpha \mathrm{i}\beta$ (zobacz rysunek).

Wyrazenie $ 2\cos\alpha-\sin\beta$ jest równe

A. $ 2\sin\beta$

B. $\cos\alpha$

C. 0

D. 2

Zadanie 20. $(0-1\rangle$

Punkt $B$ jest obrazem punktu $A=(-3,5) \mathrm{w}$

współrzędnych. DługoŚć odcinka $AB$ jest równa

symetrii względem

początku układu

A. $2\sqrt{34}$

B. 8

C. $\sqrt{34}$

D. 12

Zadanie 21. $(0-1\rangle$

Ilejest wszystkich dwucyfrowych liczb naturalnych utworzonych z cyfr: 1, 3, 5, 7, 9, w których

cyfry się nie powtarzają?

A. 10

B. 15

C. 20

D. 25

Zadanie 22. $(0-1\rangle$

Pole prostokąta ABCD jest równe 90. Na bokachAB $\mathrm{i}$ {\it CD} wybrano -odpowiednio -punkty {\it P}$\mathrm{i}R,$

takie, $\displaystyle \dot{\mathrm{z}}\mathrm{e}\frac{|AP|}{|PB|}=\frac{|CR|}{|RD|}=\frac{3}{2}$ (zobacz rysunek).
\begin{center}
\includegraphics[width=78.180mm,height=48.672mm]{./F2_M_PP_M2020_page9_images/image001.eps}
\end{center}
{\it D R  C}

{\it A  P B}

Pole czworokąta APCR jest równe

A. 36

B. 40

C. 54

D. 60

Strona 10 z26

MMA-IP



\end{document}