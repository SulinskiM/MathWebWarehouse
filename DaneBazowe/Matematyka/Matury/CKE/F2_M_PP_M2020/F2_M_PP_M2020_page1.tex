\documentclass[a4paper,12pt]{article}
\usepackage{latexsym}
\usepackage{amsmath}
\usepackage{amssymb}
\usepackage{graphicx}
\usepackage{wrapfig}
\pagestyle{plain}
\usepackage{fancybox}
\usepackage{bm}

\begin{document}

{\it W kazdym z zadań od l. do 25. wybierz i zaznacz na karcie odpowiedzi poprawnq odpowiedzí}.

Zadanie 1. (0-1)

Wartość wyrazenia $x^{2}-6x+9$ dla $x=\sqrt{3}+3$

A. l

B. 3

Zadanie2. (0-1)

Liczba $\displaystyle \frac{2^{50}\cdot 3^{40}}{36^{10}}$ jest równa

A.

$6^{70}$

B. $6^{45}$

Zadanie 3. $(0-1\rangle$

Liczba $\log_{5}\sqrt{125}$ jest równa

A.

-23

B. 2

est równa

C. $1+2\sqrt{3}$

D. $1-2\sqrt{3}$

C. $2^{30}\cdot 3^{20}$

D. $2^{10}\cdot 3^{20}$

C. 3

D.

-23

Zadanie 4. $(0-1\rangle$

Cenę $x$ pewnego towaru obnizono o 20\% i otrzymano cenę $y$. Aby przywrócić cenę $x$, nową

cenę $y$ nalezy podnieść o

A. 25\%

B. 20\%

C. 15\%

D. 12\%

Zadanie 5, $(0-1\rangle$

Zbiorem wszystkich rozwiązań nierówności 3 $(1-x)>2(3x-1)-12x$ jest przedział

A.

$(-\displaystyle \frac{5}{3},+\infty)$

B.

(-$\infty$, -35)

C.

$(\displaystyle \frac{5}{3},+\infty)$

D.

(-$\infty$'- -35)

Zadanie 6. (0-1)

Suma wszystkich rozwiązań równania $x(x-3)(x+2)=0$ jest równa

A. 0

B. l

C. 2

D. 3

Strona 2 z26

MMA-IP
\end{document}
