\documentclass[a4paper,12pt]{article}
\usepackage{latexsym}
\usepackage{amsmath}
\usepackage{amssymb}
\usepackage{graphicx}
\usepackage{wrapfig}
\pagestyle{plain}
\usepackage{fancybox}
\usepackage{bm}

\begin{document}

Imformacja do zadań 7.$-9.$

Funkcja

kwadratowa f jest

określona

wzorem

$f(x)=a(x-1)(x-3)$. Na rysunku

przedstawiono fragment paraboli będącej wykresem tej ffinkcji. Wierzchołkiem tej parabolijest

punkt $W=(2,1).$
\begin{center}
\includegraphics[width=117.912mm,height=97.128mm]{./F2_M_PP_M2020_page3_images/image001.eps}
\end{center}
4  {\it y}

3

2

{\it W}

1

$| 1  | 1$  1

$-4 -3  -2 -1$  0  1 2 3 4  {\it 5 x}

$-1$

$-2$

$-3$

$-4$

Zadanie 7. (0-1)

Współczynnik a we wzorze funkcji f jest równy

A. l

B. 2

C. $-2$

D. $-1$

Zadaqie @. (0-1)

Największa wartość funkcji $f$ w przedziale $\langle$1, $ 4\rangle$ jest równa

A. $-3$

B. 0

C. l

D. 2

Zadanie 9. $(0-1\rangle$

Osią symetrii paraboli będącej wykresem ffinkcji $f$ jest prosta o równaniu

A. $x=1$

B. $x=2$

C. $y=1$

D. $y=2$

Strona 4 z 26

MMA-IP
\end{document}
