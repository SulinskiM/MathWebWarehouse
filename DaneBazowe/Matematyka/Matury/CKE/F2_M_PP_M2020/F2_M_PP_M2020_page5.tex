\documentclass[a4paper,12pt]{article}
\usepackage{latexsym}
\usepackage{amsmath}
\usepackage{amssymb}
\usepackage{graphicx}
\usepackage{wrapfig}
\pagestyle{plain}
\usepackage{fancybox}
\usepackage{bm}

\begin{document}

Zadanie $l0. (0\rightarrow 1)$

Równanie $x(x-2)=(x-2)^{2}$ w zbiorze liczb rzeczywistych

A. nie ma rozwiązań.

B. ma dokładniejedno rozwiązanie: $x=2.$

C. ma dokładniejedno rozwiązanie: $x=0.$

D. ma dwa rózne rozwiązania: $x=1 \mathrm{i}x=2.$

Zadanie ll, $(0-1\rangle$

Na iysunku przedstawiono fiiagment wykresu funkcji liniowej $f$ określonej wzorem $f(x)=ax+b.$
\begin{center}
\includegraphics[width=118.008mm,height=97.788mm]{./F2_M_PP_M2020_page5_images/image001.eps}
\end{center}
4  {\it y}

3

1

$-4 -3  -2$

$-1 0$

$-1$

1 2 3 4  5  {\it x}

$-2$

$-3$

$-4$

Współczynniki a oraz b we wzorze funkcji f spełniają zalezność

A. $a+b>0$

B. $a+b=0$

C. $a\cdot b>0$

D. $a\cdot b<0$

ZadanIe 12. $(0\rightarrow 1$\}

Funkcja $f$ jest określona wzorem $f(x)=4^{-x}+1$ dla kazdej liczby rzeczywistej $x$. Liczba $f(\displaystyle \frac{1}{2})$

jest równa

A.

-21

B.

-23

C. 3

D. 17

Zadanie 13. $(0-1\rangle$

Proste o równaniach $y=(m-2)x$ oraz $y=\displaystyle \frac{3}{4}x+7$ są równoległe. Wtedy

A.

{\it m}$=$- -45

B.

{\it m}$=$ -23

C.

$m=\displaystyle \frac{11}{4}$

D.

$m=\displaystyle \frac{10}{3}$

Strona 6 z 26

MMA-IP
\end{document}
