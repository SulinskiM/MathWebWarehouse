\documentclass[a4paper,12pt]{article}
\usepackage{latexsym}
\usepackage{amsmath}
\usepackage{amssymb}
\usepackage{graphicx}
\usepackage{wrapfig}
\pagestyle{plain}
\usepackage{fancybox}
\usepackage{bm}

\begin{document}

Zadanie $l9. (0\rightarrow 1)$

Danyjest trójkąt prostokątny o kątach ostrych $\alpha \mathrm{i}\beta$ (zobacz rysunek).

Wyrazenie $ 2\cos\alpha-\sin\beta$ jest równe

A. $ 2\sin\beta$

B. $\cos\alpha$

C. 0

D. 2

Zadanie 20. $(0-1\rangle$

Punkt $B$ jest obrazem punktu $A=(-3,5) \mathrm{w}$

współrzędnych. DługoŚć odcinka $AB$ jest równa

symetrii względem

początku układu

A. $2\sqrt{34}$

B. 8

C. $\sqrt{34}$

D. 12

Zadanie 21. $(0-1\rangle$

Ilejest wszystkich dwucyfrowych liczb naturalnych utworzonych z cyfr: 1, 3, 5, 7, 9, w których

cyfry się nie powtarzają?

A. 10

B. 15

C. 20

D. 25

Zadanie 22. $(0-1\rangle$

Pole prostokąta ABCD jest równe 90. Na bokachAB $\mathrm{i}$ {\it CD} wybrano -odpowiednio -punkty {\it P}$\mathrm{i}R,$

takie, $\displaystyle \dot{\mathrm{z}}\mathrm{e}\frac{|AP|}{|PB|}=\frac{|CR|}{|RD|}=\frac{3}{2}$ (zobacz rysunek).
\begin{center}
\includegraphics[width=78.180mm,height=48.672mm]{./F2_M_PP_M2020_page9_images/image001.eps}
\end{center}
{\it D R  C}

{\it A  P B}

Pole czworokąta APCR jest równe

A. 36

B. 40

C. 54

D. 60

Strona 10 z26

MMA-IP
\end{document}
