\documentclass[a4paper,12pt]{article}
\usepackage{latexsym}
\usepackage{amsmath}
\usepackage{amssymb}
\usepackage{graphicx}
\usepackage{wrapfig}
\pagestyle{plain}
\usepackage{fancybox}
\usepackage{bm}

\begin{document}

CENTRALNA

KOMISJA

EGZAMINACYJNA

Arkusz zawiera informacje prawnie chronione

do momentu rozpoczecia egzaminu.

KOD

WYPELNIA ZDAJACY

PESEL

{\it Miejsce na naklejke}.

{\it Sprawdz}', {\it czy kod na naklejce to}

E-100.
\begin{center}
\includegraphics[width=21.900mm,height=10.212mm]{./F2_M_PP_M2021_page0_images/image001.eps}

\includegraphics[width=79.608mm,height=10.212mm]{./F2_M_PP_M2021_page0_images/image002.eps}
\end{center}
$J\mathrm{e}\dot{\mathrm{z}}$ {\it eli tak}- {\it przyklej naklejkq}.

{\it lezeli nie}- {\it zgtoś to nauczycielowi}.

EGZAMIN MATURALNY Z MATEMATYKI

POZIOM PODSTAWOWY

WYPELN[A ZESPÓL NADZORUJACY

DAT$\mathrm{A}^{\cdot}$ 5 maja 202l $\mathrm{r}.$

GODZINA $\mathrm{R}\mathrm{O}\mathrm{Z}\mathrm{P}\mathrm{O}\mathrm{C}\mathrm{Z}\xi \mathrm{C}\mathrm{l}\mathrm{A}:9:00$

CZAS PRACY: $\{70 \displaystyle \min \mathrm{u}\mathrm{t}$

LICZBA PUNKTÓW DO UZYSKANIA 45

Uprawnienia zdajqcego do:

\fbox{} dostosowania zasad oceniania

\fbox{} dostosowania w zw. z dyskalkulia

\fbox{} nieprzenoszenia zaznaczeń na karte.

$\Vert\Vert\Vert\Vert\Vert\Vert\Vert\Vert\Vert\Vert\Vert\Vert\Vert\Vert\Vert\Vert\Vert\Vert\Vert\Vert\Vert\Vert\Vert\Vert\Vert\Vert\Vert\Vert\Vert\Vert|$

EMAP-P0-100-2105

lnstrukcja dla zdajqcego

l. Sprawdz', czy arkusz egzaminacyjny zawiera 25 stron (zadania $1-35$).

Ewentualny brak zgloś przewodniczqcemu zespolu nadzorujqcego egzamin.

2. Na tej stronie oraz na karcie odpowiedzi wpisz swój numer PESEL i przyklej naklejk9

z kodem.

3. Nie wpisuj $\dot{\mathrm{z}}$ adnych znaków w cześci przeznaczonej dla egzaminatora.

4. Rozwiqzania zadań i odpowiedzi wpisuj w miejscu na to przeznaczonym.

5. Odpowiedzi do zadań $\mathrm{z}\mathrm{a}\mathrm{m}\mathrm{k}\mathrm{n}\mathrm{i}_{9}$tych ($1-28)$ zaznacz na karcie odpowiedzi w cz9ści

karty przeznaczonej dla zdajqcego. Zamaluj $\blacksquare$ pola do tego przeznaczone. Bledne

zaznaczenie otocz kólkiem @ i zaznacz wlaściwe.

6. Pamietaj, $\dot{\mathrm{z}}\mathrm{e}$ pominiecie argumentacji lub istotnych obliczeń w rozwiqzaniu zadania

otwartego (29-35) $\mathrm{m}\mathrm{o}\dot{\mathrm{z}}\mathrm{e}$ spowodowač, $\dot{\mathrm{z}}\mathrm{e}$ za to rozwiqzanie nie otrzymasz pelnej

liczby punktów.

7. Pisz czytelnie i $\mathrm{u}\dot{\mathrm{z}}$ ywaj tylko dlugopisu lub pióra z czarnym tuszem lub atramentem.

8. Nie $\mathrm{u}\dot{\mathrm{z}}$ ywaj korektora, a bledne zapisy wyraz'nie przekreśl.

9. Pamiptaj, $\dot{\mathrm{z}}\mathrm{e}$ zapisy w brudnopisie nie $\mathrm{b}9\mathrm{d}\mathrm{q}$ oceniane.

10. $\mathrm{M}\mathrm{o}\dot{\mathrm{z}}$ esz korzystač z zestawu wzorów matematycznych, cyrkla i linijki oraz kalkulatora

prostego.

Uklad graficzny

\copyright CKE 2021




{\it Wkazdym z zadań od f. do 28. wybierz izaznacz na karcie odpowiedzi poprawnq odpowiedz}'.

Zadanie 1. (0-1)

Liczba $100^{5}\cdot(0,1)^{-6}$ jest równa

A. $10^{13}$

B. $10^{16}$

C. $10^{-1}$

D. $10^{-30}$

Zadanie 2. $\{0-l\mathrm{I}$

Liczba 78 stanowi 150\% 1iczby $c$. Wtedy liczba $c$ jest równa

A. 60

B. 52

C. 48

D. 39

Zadanie 3. $\langle 0-ll$

Rozwazamy przedzialy liczbowe $(-\infty,5) \mathrm{i} \langle-1, +\infty$). lle jest wszystkich liczb calkowitych,

które nalezq jednocześnie do obu rozwazanych przedzialów?

A. 6

B. 5

C. 4

D. 7

Zadanie 4. $\{0-l\mathrm{I}$

Suma 2 $\log\sqrt{10}+\log 10^{\mathrm{s}}$ jest równa

A. 2

B. 3

C. 4

D. 5

Zadänie 5. $\langle 0-ll$

Róznica $0,(3)-\displaystyle \frac{23}{33}$ jest równa

A. $-0,(39)$

B. $-\displaystyle \frac{39}{100}$

C. $-0,36$

D. $-\displaystyle \frac{4}{11}$

Zadänie 6. ćO-1)

Zbiorem wszystkich rozwiqzań nierówności $\displaystyle \frac{2-x}{2}-2x\geq 1$ jest przedzial

A. $\langle 0, +\infty)$

B. $(-\infty,  0\rangle$

C. $(-\infty,  5\rangle$

D.(-$\infty$,-31\}

Strona 2 z25

$\mathrm{E}\mathrm{M}\mathrm{A}\mathrm{P}-\mathrm{P}0_{-}100$





-{\it RUDNOPIS} \{{\it nie podlega ocenie}\}

$-\mathrm{P}0_{-}100$

Strona ll z 25





Zadänie 22. (0-1)

$\mathrm{W}$ równolegloboku ABCD, przedstawionym na rysunku, kqt $\alpha$ ma miar9 $70^{\mathrm{o}}$
\begin{center}
\includegraphics[width=94.392mm,height=45.720mm]{./F2_M_PP_M2021_page11_images/image001.eps}
\end{center}
{\it D  C}

$\alpha  \beta$

{\it A  B}

Wtedy kqt $\beta$ ma miar9

A. $80^{\mathrm{o}}$

B. $70^{\mathrm{o}}$

C. $60^{\mathrm{o}}$

D. $50^{\mathrm{o}}$

Zadanie 23. $\langle 0-1$)

$\mathrm{W}\mathrm{k}\mathrm{a}\dot{\mathrm{z}}$ dym $n$-kqcie wypuklym $(n\geq 3)$ liczba przekqtnych jest równa $\displaystyle \frac{n(n-3)}{2}$ Wielokqtem

wypuklym, w którym liczba przekqtnych jest o 25 wieksza od 1iczby boków, jest

A. siedmiokqt.

B. dziesieciokqt.

C. dwunastokqt.

D. pietnastokqt.

Zadänie 24. (0-1)

Pole figury $F_{1}$ zlozonej z dwóch stycznych zewnetrznie kól o promieniach l $\mathrm{i} 3$ jest równe

polu figury $F_{2}$ zlozonej z dwóch stycznych zewnptrznie kól o promieniach dlugości $r$ (zobacz

rysunek).

Figura $F_{1}$

Figura $F_{2}$
\begin{center}
\includegraphics[width=60.456mm,height=45.768mm]{./F2_M_PP_M2021_page11_images/image002.eps}

\includegraphics[width=75.588mm,height=38.148mm]{./F2_M_PP_M2021_page11_images/image003.eps}
\end{center}
{\it r r}

D\}ugośč r promieniajest równa

A. $\sqrt{3}$

B. 2

C. $\sqrt{5}$

D. 3

Strona 12 z25

$\mathrm{E}\mathrm{M}\mathrm{A}\mathrm{P}-\mathrm{P}0_{-}100$





-{\it RUDNOPIS} \{{\it nie podlega ocenie}\}

$-\mathrm{P}0_{-}100$

Strona 13 z25





Zadänie 25. (0-1)

Punkt $A=(3,-5)$ jest wierzcholkiem kwadratu ABCD, a punkt $M=(1,3)$ jest punktem

$\mathrm{p}\mathrm{r}\mathrm{z}\mathrm{e}\mathrm{c}\mathrm{i}_{9}\mathrm{c}\mathrm{i}\mathrm{a}$ si9 przekatnych tego kwadratu. Wynika stqd, $\dot{\mathrm{z}}\mathrm{e}$ pole kwadratu ABCD jest równe

A. 68

B. 136

C. $2\sqrt{34}$

D. $8\sqrt{34}$

Zadanie 26. (0-1)

$\mathrm{Z}$ wierzcholków sześcianu ABCDEFGH losujemy jednocześnie dwa rózne wierzcholki.

Prawdopodobieństwo tego, $\dot{\mathrm{z}}\mathrm{e}$ wierzcholki te bedq końcami przekqtnej sześcianu

{\it ABCDEFGH, jest równe}

A. -71

B. -47

C. $\displaystyle \frac{1}{14}$

D. -73

Zadanie 27. $\{0-1$)

Wszystkich liczb naturalnych trzycyfrowych, wiekszych od 700, w których $\mathrm{k}\mathrm{a}\dot{\mathrm{z}}$ da cyfra nalez $\mathrm{y}$

do zbioru \{1, 2, 3, 7, 8, 9\} i $\dot{\mathrm{z}}$ adna cyfra $\mathrm{s}\mathrm{i}\mathrm{e}$ nie powtarza, jest

A. 108

B. 60

C. 40

D. 299

Zadanie 28. $\{0-1\}$

Sześciowyrazowy ciqg liczbowy $(1,2,2x,x+2,5,6)$

tego ciqgu jest równa 4. Wynika stqd, $\dot{\mathrm{z}}\mathrm{e}$

jest niemalejqcy. Mediana wyrazów

A. $x=1$

B. $\chi=$ -23

$-. x=2$

D. $\chi=$ -83

Strona 14 z25

$\mathrm{E}\mathrm{M}\mathrm{A}\mathrm{P}-\mathrm{P}0_{-}100$





-{\it RUDNOPIS} \{{\it nie podlega ocenie}\}

$-\mathrm{P}0_{-}100$

Strona 15z 25





Zadänie 29. $(0-2)$

Rozwiqz nierównośč:

$x^{2}-5x\leq 14$

Odpowiedz':

Strona 16 z25

$\mathrm{E}\mathrm{M}\mathrm{A}\mathrm{P}-\mathrm{P}0_{-}100$





Zadänie 30. $(0-2)$

Wykaz, $\dot{\mathrm{z}}\mathrm{e}$ dla $\mathrm{k}\mathrm{a}\dot{\mathrm{z}}$ dych trzech dodatnich liczb $a, b$

nierównośč

-{\it ab}$<$--{\it ba}$++${\it cc}

i

$c$ takich, $\dot{\mathrm{z}}\mathrm{e} a<b$, spelniona jest
\begin{center}
\begin{tabular}{|l|l|l|l|}
\cline{2-4}
&	\multicolumn{1}{|l|}{Nr zadania}&	\multicolumn{1}{|l|}{$29.$}&	\multicolumn{1}{|l|}{ $30.$}	\\
\cline{2-4}
&	\multicolumn{1}{|l|}{Maks. liczba pkt}&	\multicolumn{1}{|l|}{$2$}&	\multicolumn{1}{|l|}{ $2$}	\\
\cline{2-4}
\multicolumn{1}{|l|}{egzaminator}&	\multicolumn{1}{|l|}{Uzyskana liczba pkt}&	\multicolumn{1}{|l|}{}&	\multicolumn{1}{|l|}{}	\\
\hline
\end{tabular}

\end{center}
$\mathrm{E}\mathrm{M}\mathrm{A}\mathrm{P}-\mathrm{P}0_{-}100$

Strona 17 z25





Zadänie 31. $(0-2l$

Funkcja liniowa $f$ przyjmuje wartośč

Wyznacz wzór funkcji $f.$

2 dla argumentu

0, a ponadto $f(4)-f(2)=6.$

Odpowiedz':

Strona 18 z25

$\mathrm{E}\mathrm{M}\mathrm{A}\mathrm{P}-\mathrm{P}0_{-}100$





Zadänie 32. $(0-2)$

Rozwiqz równanie:

$\displaystyle \frac{3x+2}{3x-2}=4-x$

Odpowiedz':
\begin{center}
\begin{tabular}{|l|l|l|l|}
\cline{2-4}
&	\multicolumn{1}{|l|}{Nr zadania}&	\multicolumn{1}{|l|}{$31.$}&	\multicolumn{1}{|l|}{ $32.$}	\\
\cline{2-4}
&	\multicolumn{1}{|l|}{Maks. liczba pkt}&	\multicolumn{1}{|l|}{$2$}&	\multicolumn{1}{|l|}{ $2$}	\\
\cline{2-4}
\multicolumn{1}{|l|}{egzaminator}&	\multicolumn{1}{|l|}{Uzyskana liczba pkt}&	\multicolumn{1}{|l|}{}&	\multicolumn{1}{|l|}{}	\\
\hline
\end{tabular}

\end{center}
$\mathrm{E}\mathrm{M}\mathrm{A}\mathrm{P}-\mathrm{P}0_{-}100$

Strona 19 z25





Zadänie 33. $(0-2)$

Trójkqt równoboczny $ABC$ ma pole równe $9\sqrt{3}$. Prosta równolegla do boku $BC$ przecina

boki AB $\mathrm{i} AC -$ odpowiednio-w punktach $K \mathrm{i} L$. Trójkqty $ABC \mathrm{i} AKL$ sq podobne,

a stosunek dlugości boków tych trójkqtówjest równy $\displaystyle \frac{3}{2}$. Oblicz dlugośč boku trójkqta $AKL.$

Odpowiedz':

Strona 20 z25

$\mathrm{E}\mathrm{M}\mathrm{A}\mathrm{P}-\mathrm{P}0_{-}100$





-{\it RUDNOPIS} \{{\it nie podlega ocenie}\}

$-\mathrm{P}0_{-}100$

Strona 3 z25





Zadänie 34. $(0-2)$

Gracz rzuca dwukrotnie symetrycznq sześciennq kostkq do gry i oblicza sum9 1iczb

wyrzuconych oczek. Oblicz prawdopodobieństwo zdarzenia $A$ polegajqcego na tym, $\dot{\mathrm{z}}\mathrm{e}$ suma

liczb wyrzuconych oczekjest równa 4 1ub 5, 1ub 6.

Odpowiedz':
\begin{center}
\begin{tabular}{|l|l|l|l|}
\cline{2-4}
&	\multicolumn{1}{|l|}{Nr zadania}&	\multicolumn{1}{|l|}{$33.$}&	\multicolumn{1}{|l|}{ $34.$}	\\
\cline{2-4}
&	\multicolumn{1}{|l|}{Maks. liczba pkt}&	\multicolumn{1}{|l|}{$2$}&	\multicolumn{1}{|l|}{ $2$}	\\
\cline{2-4}
\multicolumn{1}{|l|}{egzaminator}&	\multicolumn{1}{|l|}{Uzyskana liczba pkt}&	\multicolumn{1}{|l|}{}&	\multicolumn{1}{|l|}{}	\\
\hline
\end{tabular}

\end{center}
$\mathrm{E}\mathrm{M}\mathrm{A}\mathrm{P}-\mathrm{P}0_{-}100$

Strona 21 z25





Zadänie 35. $(0-5$\}

Punkty $A=(-20,12) \mathrm{i} B=(7,3)$ sq wierzcholkami trójkqta równoramiennego $ABC,$

w którym $|AC|=|BC|$. Wierzcholek $C \mathrm{l}\mathrm{e}\dot{\mathrm{z}}\mathrm{y}$ na osi $0\mathrm{y}$ ukladu wspólrzednych. Oblicz

wspólrz9dne wierzcho1ka $C$ oraz obwód tego trójkqta.

Strona 22 z25

$\mathrm{E}\mathrm{M}\mathrm{A}\mathrm{P}-\mathrm{P}0_{-}100$





Odpowiedz':

Wypelnia

egzaminator

Nr zadania

Maks. liczba pkt

Uzyskana liczba pkt

35.

5

$\mathrm{E}\mathrm{M}\mathrm{A}\mathrm{P}-\mathrm{P}0_{-}100$

Strona 23 z25





-{\it RUDNOPIS} \{{\it nie podlega ocenie}\}

Strona 24z 25

$\mathrm{E}\mathrm{M}\mathrm{A}\mathrm{P}-\mathrm{P}0_{-}1$





$|-100$

Strona 25 z25




















Zadänie $7_{1}. (0-1)$

Na ponizszym rysunku przedstawiono wykres funkcji $f$ określonej w zbiorze $\langle-6, 5\rangle.$

Funkcja g

prawdziwe.

jest określona wzorem

$g(x)=f(x)-2$ dla

$\chi\in\langle-6,5\rangle$. Wskaz zdanie

A. Liczba $f(2)+g(2)$ jest równa $(-2).$

B. Zbiory wartości funkcji $f \mathrm{i} g$ sq równe.

C. Funkcje $f \mathrm{i} g$ majq te same miejsca zerowe.

D. Punkt $P=(0,-2)$ nalez $\mathrm{y}$ do wykresów funkcji $f \mathrm{i} g.$

Zadanie S. $\langle 0-1$)

Na rysunku obok przedstawiono geometrycznq

interpretacje jednego z $\mathrm{n}\mathrm{i}\dot{\mathrm{z}}$ ej zapisanych ukladów

równań. Wska $\dot{\mathrm{z}}$ ten uklad, którego geometrycznq

interpretacje przedstawiono na rysunku.

A. 

B. 

C. 

D. 

Strona 4 z25

$\mathrm{E}\mathrm{M}\mathrm{A}\mathrm{P}-\mathrm{P}0_{-}100$





-{\it RUDNOPIS} \{{\it nie podlega ocenie}\}

$-\mathrm{P}0_{-}100$

Strona 5 z25





Zadänie 9. (0-1)

Proste o równaniach $y=3x-5$ oraz $y=\displaystyle \frac{m-3}{2}x+\frac{9}{2}$ sq równolegle, gdy

A. $m=1$

B. $m=3$

C. $m=6$

D. $m=9$

Zadanie 10. (0-1)

Funkcja $f$ jest określona wzorem $f(x)=\displaystyle \frac{\chi^{2}}{2x-2}$ dla $\mathrm{k}\mathrm{a}\dot{\mathrm{z}}$ dej liczby rzeczywistej $\chi\neq 1$. Wtedy

dla argumentu $x=\sqrt{3}-1$ wartośč funkcji $f$ jest równa

A. $\displaystyle \frac{1}{\sqrt{3}-1}$

B. $-1$

C. l

D. $\displaystyle \frac{1}{\sqrt{3}-2}$

Zadanie $l1_{\varepsilon}(0-1)$

Do wykresu funkcji $f$ określonej dla $\mathrm{k}\mathrm{a}\dot{\mathrm{z}}$ dej liczby rzeczywistej $x$

nalez $\mathrm{y}$ punkt o wspólrzednych

wzorem $f(x)=3^{\chi}-2$

A. $(-1,-5)$

B. $(0,-2)$

C. $(0,-1)$

D. (2, 4)

Zadanie $12_{s}(0-1)$

Funkcja kwadratowa

w przedziale

f określona wzorem

$f(x)=-2(x+1)(x-3)$

jest malejqca

A. $\langle 1, +\infty)$

B. $(-\infty,  1\rangle$

C. $(-\infty, -8\rangle$

D. $\langle-8, +\infty)$

Zadanie 13. $(0-1$\}

Trzywyrazowy ciag $(15,3x,\displaystyle \frac{5}{3})$ jest geometryczny i wszystkiejego wyrazy sq dodatnie. Stqd

wynika, $\dot{\mathrm{z}}\mathrm{e}$

A. $\chi=$ -53

B. $\chi=$ -45

C. $x=1$

D. $\chi=$ -35

Zadanie 14. (0-1)

Ciqg $(b_{n})$ jest określonywzorem $b_{n}=3n^{2}-25n$ dla $\mathrm{k}\mathrm{a}\dot{\mathrm{z}}$ dej liczby naturalnej $n\geq 1$. Liczba

niedodatnich wyrazów ciqgu $(b_{n})$ jest równa

A. 14

B. 13

C. 9

D. 8

Strona 6 z25

$\mathrm{E}\mathrm{M}\mathrm{A}\mathrm{P}-\mathrm{P}0_{-}100$





-{\it RUDNOPIS} \{{\it nie podlega ocenie}\}

$-\mathrm{P}0_{-}100$

Strona 7 z25





Zadänie 15. (0-1)

Ciqg arytmetyczny $(a_{n})$ jest określony dla $\mathrm{k}\mathrm{a}\dot{\mathrm{z}}$ dej liczby naturalnej $n\geq 1$. Trzeci i piqty wyraz

ciqgu spelniajq warunek $a_{3}+a_{5}=58$. Wtedy czwarty wyraz tego ciqgu jest równy

A. 28

B. 29

C. 33

D. 40

Zadanie 16. (0-1)

Dla $\mathrm{k}\mathrm{a}\dot{\mathrm{z}}$ dego kqta ostrego $\alpha$ iloczyn $\displaystyle \frac{\cos\alpha}{1-\sin^{2}\alpha} \displaystyle \frac{1-\cos^{2}\alpha}{\sin\alpha}$ jest równy

A. $\sin\alpha$

B. $\mathrm{t}\mathrm{g}\alpha$

C. $\cos\alpha$

D. $\sin^{2}\alpha$

Zadanie 17. (0-1)

Prosta $k$ jest styczna w punkcie $A$ do okregu o środku 0. Punkt $B \mathrm{l}\mathrm{e}\dot{\mathrm{z}}\mathrm{y}$ na tym okregu

i miara kqta $A0B$ jest równa $80^{\mathrm{o}}$. Przez punkty 0 $\mathrm{i} B$ poprowadzono prostq, która przecina

prostq $k$ w punkcie $C$ (zobacz rysunek).
\begin{center}
\includegraphics[width=143.460mm,height=35.208mm]{./F2_M_PP_M2021_page7_images/image001.eps}
\end{center}
{\it 0}

{\it B}

$80^{\mathrm{o}}$

{\it C  k}

{\it A}

B. $\displaystyle \frac{37}{3}$

A. 12

Pole tego trójkqta jest równe

Zadänie $l8. (0-1)$

Przyprostokqtna $AC$ trójkata prostokqtnego

rysunek).

B. $30^{\mathrm{o}}$

A. $10^{\mathrm{o}}$

Miara kqta BAC jest równa

C. $40^{\mathrm{o}}$

D. $50^{\mathrm{o}}$

$ABC$ ma dlugośč 8 oraz $\displaystyle \mathrm{t}\mathrm{g}\alpha=\frac{2}{5}$

(zobacz
\begin{center}
\includegraphics[width=66.852mm,height=33.684mm]{./F2_M_PP_M2021_page7_images/image002.eps}
\end{center}
{\it B}

$\alpha$

{\it C} 8  {\it A}

C. $\displaystyle \frac{62}{5}$

D. $\displaystyle \frac{64}{5}$

Strona 8 z25

$\mathrm{E}\mathrm{M}\mathrm{A}\mathrm{P}-\mathrm{P}0_{-}100$





-{\it RUDNOPIS} \{{\it nie podlega ocenie}\}

$-\mathrm{P}0_{-}100$

Strona 9 z25





Zadänie $l9. (0-1)$

Pole pewnego trójkqta równobocznego jest równe $\displaystyle \frac{4\sqrt{3}}{9}$. Obwód tego trójkqta jest równy

A. 4

B. 2

C. -43

D. -23

Zadanie 20. $(0-1$\}

$\mathrm{W}$ trójkqcie $ABC$ bok $BC$ ma dlugość 13, a wysokośč

$CD$ tego trójkqta dzieli bok $AB$ na odcinki o dtugościach

$|AD|=3 \mathrm{i} |BD|=12$ (zobacz rysunek obok). Dlugośč

boku $AC$ jest równa
\begin{center}
\includegraphics[width=64.572mm,height=28.452mm]{./F2_M_PP_M2021_page9_images/image001.eps}
\end{center}
{\it C}

13

{\it A} 3  {\it D}  12  {\it B}

A. $\sqrt{34}$

B. $\displaystyle \frac{13}{4}$

C. $2\sqrt{14}$

D. $3\sqrt{45}$

Zadänie 21. (0-1)

Punkty $A, B, C \mathrm{i} D \mathrm{l}\mathrm{e}\dot{\mathrm{z}}\mathrm{q}$ na okregu o środku $S$. Miary kqtów $SBC, BCD, CDA \mathrm{s}_{\mathrm{c}}1$ równe

odpowiednio: $|4SBC|=60^{\mathrm{o}}, |4BCD|=110^{\mathrm{o}}, |4CDA|=90^{\mathrm{o}}$ (zobacz rysunek).
\begin{center}
\includegraphics[width=60.192mm,height=64.464mm]{./F2_M_PP_M2021_page9_images/image002.eps}
\end{center}
{\it C}

$110^{\mathrm{o}}$

$60^{\mathrm{o}}$  {\it B}

{\it D  S}

$\alpha$

{\it A}

Wynika stqd, $\dot{\mathrm{z}}\mathrm{e}$ miara $\alpha$ kqta DAS jest równa

A. $25^{\mathrm{o}}$

B. $30^{\mathrm{o}}$

C. $35^{\mathrm{o}}$

D. $40^{\mathrm{o}}$

Strona 10 z25

$\mathrm{E}\mathrm{M}\mathrm{A}\mathrm{P}-\mathrm{P}0_{-}100$



\end{document}