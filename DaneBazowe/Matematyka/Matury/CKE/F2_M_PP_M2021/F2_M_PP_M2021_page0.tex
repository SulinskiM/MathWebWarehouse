\documentclass[a4paper,12pt]{article}
\usepackage{latexsym}
\usepackage{amsmath}
\usepackage{amssymb}
\usepackage{graphicx}
\usepackage{wrapfig}
\pagestyle{plain}
\usepackage{fancybox}
\usepackage{bm}

\begin{document}

CENTRALNA

KOMISJA

EGZAMINACYJNA

Arkusz zawiera informacje prawnie chronione

do momentu rozpoczecia egzaminu.

KOD

WYPELNIA ZDAJACY

PESEL

{\it Miejsce na naklejke}.

{\it Sprawdz}', {\it czy kod na naklejce to}

E-100.
\begin{center}
\includegraphics[width=21.900mm,height=10.212mm]{./F2_M_PP_M2021_page0_images/image001.eps}

\includegraphics[width=79.608mm,height=10.212mm]{./F2_M_PP_M2021_page0_images/image002.eps}
\end{center}
$J\mathrm{e}\dot{\mathrm{z}}$ {\it eli tak}- {\it przyklej naklejkq}.

{\it lezeli nie}- {\it zgtoś to nauczycielowi}.

EGZAMIN MATURALNY Z MATEMATYKI

POZIOM PODSTAWOWY

WYPELN[A ZESPÓL NADZORUJACY

DAT$\mathrm{A}^{\cdot}$ 5 maja 202l $\mathrm{r}.$

GODZINA $\mathrm{R}\mathrm{O}\mathrm{Z}\mathrm{P}\mathrm{O}\mathrm{C}\mathrm{Z}\xi \mathrm{C}\mathrm{l}\mathrm{A}:9:00$

CZAS PRACY: $\{70 \displaystyle \min \mathrm{u}\mathrm{t}$

LICZBA PUNKTÓW DO UZYSKANIA 45

Uprawnienia zdajqcego do:

\fbox{} dostosowania zasad oceniania

\fbox{} dostosowania w zw. z dyskalkulia

\fbox{} nieprzenoszenia zaznaczeń na karte.

$\Vert\Vert\Vert\Vert\Vert\Vert\Vert\Vert\Vert\Vert\Vert\Vert\Vert\Vert\Vert\Vert\Vert\Vert\Vert\Vert\Vert\Vert\Vert\Vert\Vert\Vert\Vert\Vert\Vert\Vert|$

EMAP-P0-100-2105

lnstrukcja dla zdajqcego

l. Sprawdz', czy arkusz egzaminacyjny zawiera 25 stron (zadania $1-35$).

Ewentualny brak zgloś przewodniczqcemu zespolu nadzorujqcego egzamin.

2. Na tej stronie oraz na karcie odpowiedzi wpisz swój numer PESEL i przyklej naklejk9

z kodem.

3. Nie wpisuj $\dot{\mathrm{z}}$ adnych znaków w cześci przeznaczonej dla egzaminatora.

4. Rozwiqzania zadań i odpowiedzi wpisuj w miejscu na to przeznaczonym.

5. Odpowiedzi do zadań $\mathrm{z}\mathrm{a}\mathrm{m}\mathrm{k}\mathrm{n}\mathrm{i}_{9}$tych ($1-28)$ zaznacz na karcie odpowiedzi w cz9ści

karty przeznaczonej dla zdajqcego. Zamaluj $\blacksquare$ pola do tego przeznaczone. Bledne

zaznaczenie otocz kólkiem @ i zaznacz wlaściwe.

6. Pamietaj, $\dot{\mathrm{z}}\mathrm{e}$ pominiecie argumentacji lub istotnych obliczeń w rozwiqzaniu zadania

otwartego (29-35) $\mathrm{m}\mathrm{o}\dot{\mathrm{z}}\mathrm{e}$ spowodowač, $\dot{\mathrm{z}}\mathrm{e}$ za to rozwiqzanie nie otrzymasz pelnej

liczby punktów.

7. Pisz czytelnie i $\mathrm{u}\dot{\mathrm{z}}$ ywaj tylko dlugopisu lub pióra z czarnym tuszem lub atramentem.

8. Nie $\mathrm{u}\dot{\mathrm{z}}$ ywaj korektora, a bledne zapisy wyraz'nie przekreśl.

9. Pamiptaj, $\dot{\mathrm{z}}\mathrm{e}$ zapisy w brudnopisie nie $\mathrm{b}9\mathrm{d}\mathrm{q}$ oceniane.

10. $\mathrm{M}\mathrm{o}\dot{\mathrm{z}}$ esz korzystač z zestawu wzorów matematycznych, cyrkla i linijki oraz kalkulatora

prostego.

Uklad graficzny

\copyright CKE 2021
\end{document}
