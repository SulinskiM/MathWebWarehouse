\documentclass[a4paper,12pt]{article}
\usepackage{latexsym}
\usepackage{amsmath}
\usepackage{amssymb}
\usepackage{graphicx}
\usepackage{wrapfig}
\pagestyle{plain}
\usepackage{fancybox}
\usepackage{bm}

\begin{document}

{\it Wkazdym z zadań od f. do 28. wybierz izaznacz na karcie odpowiedzi poprawnq odpowiedz}'.

Zadanie 1. (0-1)

Liczba $100^{5}\cdot(0,1)^{-6}$ jest równa

A. $10^{13}$

B. $10^{16}$

C. $10^{-1}$

D. $10^{-30}$

Zadanie 2. $\{0-l\mathrm{I}$

Liczba 78 stanowi 150\% 1iczby $c$. Wtedy liczba $c$ jest równa

A. 60

B. 52

C. 48

D. 39

Zadanie 3. $\langle 0-ll$

Rozwazamy przedzialy liczbowe $(-\infty,5) \mathrm{i} \langle-1, +\infty$). lle jest wszystkich liczb calkowitych,

które nalezq jednocześnie do obu rozwazanych przedzialów?

A. 6

B. 5

C. 4

D. 7

Zadanie 4. $\{0-l\mathrm{I}$

Suma 2 $\log\sqrt{10}+\log 10^{\mathrm{s}}$ jest równa

A. 2

B. 3

C. 4

D. 5

Zadänie 5. $\langle 0-ll$

Róznica $0,(3)-\displaystyle \frac{23}{33}$ jest równa

A. $-0,(39)$

B. $-\displaystyle \frac{39}{100}$

C. $-0,36$

D. $-\displaystyle \frac{4}{11}$

Zadänie 6. ćO-1)

Zbiorem wszystkich rozwiqzań nierówności $\displaystyle \frac{2-x}{2}-2x\geq 1$ jest przedzial

A. $\langle 0, +\infty)$

B. $(-\infty,  0\rangle$

C. $(-\infty,  5\rangle$

D.(-$\infty$,-31\}

Strona 2 z25

$\mathrm{E}\mathrm{M}\mathrm{A}\mathrm{P}-\mathrm{P}0_{-}100$
\end{document}
