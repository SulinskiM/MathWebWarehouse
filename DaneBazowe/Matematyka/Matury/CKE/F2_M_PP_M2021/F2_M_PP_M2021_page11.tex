\documentclass[a4paper,12pt]{article}
\usepackage{latexsym}
\usepackage{amsmath}
\usepackage{amssymb}
\usepackage{graphicx}
\usepackage{wrapfig}
\pagestyle{plain}
\usepackage{fancybox}
\usepackage{bm}

\begin{document}

Zadänie 22. (0-1)

$\mathrm{W}$ równolegloboku ABCD, przedstawionym na rysunku, kqt $\alpha$ ma miar9 $70^{\mathrm{o}}$
\begin{center}
\includegraphics[width=94.392mm,height=45.720mm]{./F2_M_PP_M2021_page11_images/image001.eps}
\end{center}
{\it D  C}

$\alpha  \beta$

{\it A  B}

Wtedy kqt $\beta$ ma miar9

A. $80^{\mathrm{o}}$

B. $70^{\mathrm{o}}$

C. $60^{\mathrm{o}}$

D. $50^{\mathrm{o}}$

Zadanie 23. $\langle 0-1$)

$\mathrm{W}\mathrm{k}\mathrm{a}\dot{\mathrm{z}}$ dym $n$-kqcie wypuklym $(n\geq 3)$ liczba przekqtnych jest równa $\displaystyle \frac{n(n-3)}{2}$ Wielokqtem

wypuklym, w którym liczba przekqtnych jest o 25 wieksza od 1iczby boków, jest

A. siedmiokqt.

B. dziesieciokqt.

C. dwunastokqt.

D. pietnastokqt.

Zadänie 24. (0-1)

Pole figury $F_{1}$ zlozonej z dwóch stycznych zewnetrznie kól o promieniach l $\mathrm{i} 3$ jest równe

polu figury $F_{2}$ zlozonej z dwóch stycznych zewnptrznie kól o promieniach dlugości $r$ (zobacz

rysunek).

Figura $F_{1}$

Figura $F_{2}$
\begin{center}
\includegraphics[width=60.456mm,height=45.768mm]{./F2_M_PP_M2021_page11_images/image002.eps}

\includegraphics[width=75.588mm,height=38.148mm]{./F2_M_PP_M2021_page11_images/image003.eps}
\end{center}
{\it r r}

D\}ugośč r promieniajest równa

A. $\sqrt{3}$

B. 2

C. $\sqrt{5}$

D. 3

Strona 12 z25

$\mathrm{E}\mathrm{M}\mathrm{A}\mathrm{P}-\mathrm{P}0_{-}100$
\end{document}
