\documentclass[a4paper,12pt]{article}
\usepackage{latexsym}
\usepackage{amsmath}
\usepackage{amssymb}
\usepackage{graphicx}
\usepackage{wrapfig}
\pagestyle{plain}
\usepackage{fancybox}
\usepackage{bm}

\begin{document}

Zadänie 25. (0-1)

Punkt $A=(3,-5)$ jest wierzcholkiem kwadratu ABCD, a punkt $M=(1,3)$ jest punktem

$\mathrm{p}\mathrm{r}\mathrm{z}\mathrm{e}\mathrm{c}\mathrm{i}_{9}\mathrm{c}\mathrm{i}\mathrm{a}$ si9 przekatnych tego kwadratu. Wynika stqd, $\dot{\mathrm{z}}\mathrm{e}$ pole kwadratu ABCD jest równe

A. 68

B. 136

C. $2\sqrt{34}$

D. $8\sqrt{34}$

Zadanie 26. (0-1)

$\mathrm{Z}$ wierzcholków sześcianu ABCDEFGH losujemy jednocześnie dwa rózne wierzcholki.

Prawdopodobieństwo tego, $\dot{\mathrm{z}}\mathrm{e}$ wierzcholki te bedq końcami przekqtnej sześcianu

{\it ABCDEFGH, jest równe}

A. -71

B. -47

C. $\displaystyle \frac{1}{14}$

D. -73

Zadanie 27. $\{0-1$)

Wszystkich liczb naturalnych trzycyfrowych, wiekszych od 700, w których $\mathrm{k}\mathrm{a}\dot{\mathrm{z}}$ da cyfra nalez $\mathrm{y}$

do zbioru \{1, 2, 3, 7, 8, 9\} i $\dot{\mathrm{z}}$ adna cyfra $\mathrm{s}\mathrm{i}\mathrm{e}$ nie powtarza, jest

A. 108

B. 60

C. 40

D. 299

Zadanie 28. $\{0-1\}$

Sześciowyrazowy ciqg liczbowy $(1,2,2x,x+2,5,6)$

tego ciqgu jest równa 4. Wynika stqd, $\dot{\mathrm{z}}\mathrm{e}$

jest niemalejqcy. Mediana wyrazów

A. $x=1$

B. $\chi=$ -23

$-. x=2$

D. $\chi=$ -83

Strona 14 z25

$\mathrm{E}\mathrm{M}\mathrm{A}\mathrm{P}-\mathrm{P}0_{-}100$
\end{document}
