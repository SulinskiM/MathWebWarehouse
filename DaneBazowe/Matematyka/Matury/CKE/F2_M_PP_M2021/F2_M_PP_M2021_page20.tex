\documentclass[a4paper,12pt]{article}
\usepackage{latexsym}
\usepackage{amsmath}
\usepackage{amssymb}
\usepackage{graphicx}
\usepackage{wrapfig}
\pagestyle{plain}
\usepackage{fancybox}
\usepackage{bm}

\begin{document}

Zadänie 34. $(0-2)$

Gracz rzuca dwukrotnie symetrycznq sześciennq kostkq do gry i oblicza sum9 1iczb

wyrzuconych oczek. Oblicz prawdopodobieństwo zdarzenia $A$ polegajqcego na tym, $\dot{\mathrm{z}}\mathrm{e}$ suma

liczb wyrzuconych oczekjest równa 4 1ub 5, 1ub 6.

Odpowiedz':
\begin{center}
\begin{tabular}{|l|l|l|l|}
\cline{2-4}
&	\multicolumn{1}{|l|}{Nr zadania}&	\multicolumn{1}{|l|}{$33.$}&	\multicolumn{1}{|l|}{ $34.$}	\\
\cline{2-4}
&	\multicolumn{1}{|l|}{Maks. liczba pkt}&	\multicolumn{1}{|l|}{$2$}&	\multicolumn{1}{|l|}{ $2$}	\\
\cline{2-4}
\multicolumn{1}{|l|}{egzaminator}&	\multicolumn{1}{|l|}{Uzyskana liczba pkt}&	\multicolumn{1}{|l|}{}&	\multicolumn{1}{|l|}{}	\\
\hline
\end{tabular}

\end{center}
$\mathrm{E}\mathrm{M}\mathrm{A}\mathrm{P}-\mathrm{P}0_{-}100$

Strona 21 z25
\end{document}
