\documentclass[a4paper,12pt]{article}
\usepackage{latexsym}
\usepackage{amsmath}
\usepackage{amssymb}
\usepackage{graphicx}
\usepackage{wrapfig}
\pagestyle{plain}
\usepackage{fancybox}
\usepackage{bm}

\begin{document}

Zadänie $7_{1}. (0-1)$

Na ponizszym rysunku przedstawiono wykres funkcji $f$ określonej w zbiorze $\langle-6, 5\rangle.$

Funkcja g

prawdziwe.

jest określona wzorem

$g(x)=f(x)-2$ dla

$\chi\in\langle-6,5\rangle$. Wskaz zdanie

A. Liczba $f(2)+g(2)$ jest równa $(-2).$

B. Zbiory wartości funkcji $f \mathrm{i} g$ sq równe.

C. Funkcje $f \mathrm{i} g$ majq te same miejsca zerowe.

D. Punkt $P=(0,-2)$ nalez $\mathrm{y}$ do wykresów funkcji $f \mathrm{i} g.$

Zadanie S. $\langle 0-1$)

Na rysunku obok przedstawiono geometrycznq

interpretacje jednego z $\mathrm{n}\mathrm{i}\dot{\mathrm{z}}$ ej zapisanych ukladów

równań. Wska $\dot{\mathrm{z}}$ ten uklad, którego geometrycznq

interpretacje przedstawiono na rysunku.

A. 

B. 

C. 

D. 

Strona 4 z25

$\mathrm{E}\mathrm{M}\mathrm{A}\mathrm{P}-\mathrm{P}0_{-}100$
\end{document}
