\documentclass[a4paper,12pt]{article}
\usepackage{latexsym}
\usepackage{amsmath}
\usepackage{amssymb}
\usepackage{graphicx}
\usepackage{wrapfig}
\pagestyle{plain}
\usepackage{fancybox}
\usepackage{bm}

\begin{document}

Zadänie 9. (0-1)

Proste o równaniach $y=3x-5$ oraz $y=\displaystyle \frac{m-3}{2}x+\frac{9}{2}$ sq równolegle, gdy

A. $m=1$

B. $m=3$

C. $m=6$

D. $m=9$

Zadanie 10. (0-1)

Funkcja $f$ jest określona wzorem $f(x)=\displaystyle \frac{\chi^{2}}{2x-2}$ dla $\mathrm{k}\mathrm{a}\dot{\mathrm{z}}$ dej liczby rzeczywistej $\chi\neq 1$. Wtedy

dla argumentu $x=\sqrt{3}-1$ wartośč funkcji $f$ jest równa

A. $\displaystyle \frac{1}{\sqrt{3}-1}$

B. $-1$

C. l

D. $\displaystyle \frac{1}{\sqrt{3}-2}$

Zadanie $l1_{\varepsilon}(0-1)$

Do wykresu funkcji $f$ określonej dla $\mathrm{k}\mathrm{a}\dot{\mathrm{z}}$ dej liczby rzeczywistej $x$

nalez $\mathrm{y}$ punkt o wspólrzednych

wzorem $f(x)=3^{\chi}-2$

A. $(-1,-5)$

B. $(0,-2)$

C. $(0,-1)$

D. (2, 4)

Zadanie $12_{s}(0-1)$

Funkcja kwadratowa

w przedziale

f określona wzorem

$f(x)=-2(x+1)(x-3)$

jest malejqca

A. $\langle 1, +\infty)$

B. $(-\infty,  1\rangle$

C. $(-\infty, -8\rangle$

D. $\langle-8, +\infty)$

Zadanie 13. $(0-1$\}

Trzywyrazowy ciag $(15,3x,\displaystyle \frac{5}{3})$ jest geometryczny i wszystkiejego wyrazy sq dodatnie. Stqd

wynika, $\dot{\mathrm{z}}\mathrm{e}$

A. $\chi=$ -53

B. $\chi=$ -45

C. $x=1$

D. $\chi=$ -35

Zadanie 14. (0-1)

Ciqg $(b_{n})$ jest określonywzorem $b_{n}=3n^{2}-25n$ dla $\mathrm{k}\mathrm{a}\dot{\mathrm{z}}$ dej liczby naturalnej $n\geq 1$. Liczba

niedodatnich wyrazów ciqgu $(b_{n})$ jest równa

A. 14

B. 13

C. 9

D. 8

Strona 6 z25

$\mathrm{E}\mathrm{M}\mathrm{A}\mathrm{P}-\mathrm{P}0_{-}100$
\end{document}
