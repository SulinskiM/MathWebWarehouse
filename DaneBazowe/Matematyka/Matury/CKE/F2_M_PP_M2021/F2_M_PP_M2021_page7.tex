\documentclass[a4paper,12pt]{article}
\usepackage{latexsym}
\usepackage{amsmath}
\usepackage{amssymb}
\usepackage{graphicx}
\usepackage{wrapfig}
\pagestyle{plain}
\usepackage{fancybox}
\usepackage{bm}

\begin{document}

Zadänie 15. (0-1)

Ciqg arytmetyczny $(a_{n})$ jest określony dla $\mathrm{k}\mathrm{a}\dot{\mathrm{z}}$ dej liczby naturalnej $n\geq 1$. Trzeci i piqty wyraz

ciqgu spelniajq warunek $a_{3}+a_{5}=58$. Wtedy czwarty wyraz tego ciqgu jest równy

A. 28

B. 29

C. 33

D. 40

Zadanie 16. (0-1)

Dla $\mathrm{k}\mathrm{a}\dot{\mathrm{z}}$ dego kqta ostrego $\alpha$ iloczyn $\displaystyle \frac{\cos\alpha}{1-\sin^{2}\alpha} \displaystyle \frac{1-\cos^{2}\alpha}{\sin\alpha}$ jest równy

A. $\sin\alpha$

B. $\mathrm{t}\mathrm{g}\alpha$

C. $\cos\alpha$

D. $\sin^{2}\alpha$

Zadanie 17. (0-1)

Prosta $k$ jest styczna w punkcie $A$ do okregu o środku 0. Punkt $B \mathrm{l}\mathrm{e}\dot{\mathrm{z}}\mathrm{y}$ na tym okregu

i miara kqta $A0B$ jest równa $80^{\mathrm{o}}$. Przez punkty 0 $\mathrm{i} B$ poprowadzono prostq, która przecina

prostq $k$ w punkcie $C$ (zobacz rysunek).
\begin{center}
\includegraphics[width=143.460mm,height=35.208mm]{./F2_M_PP_M2021_page7_images/image001.eps}
\end{center}
{\it 0}

{\it B}

$80^{\mathrm{o}}$

{\it C  k}

{\it A}

B. $\displaystyle \frac{37}{3}$

A. 12

Pole tego trójkqta jest równe

Zadänie $l8. (0-1)$

Przyprostokqtna $AC$ trójkata prostokqtnego

rysunek).

B. $30^{\mathrm{o}}$

A. $10^{\mathrm{o}}$

Miara kqta BAC jest równa

C. $40^{\mathrm{o}}$

D. $50^{\mathrm{o}}$

$ABC$ ma dlugośč 8 oraz $\displaystyle \mathrm{t}\mathrm{g}\alpha=\frac{2}{5}$

(zobacz
\begin{center}
\includegraphics[width=66.852mm,height=33.684mm]{./F2_M_PP_M2021_page7_images/image002.eps}
\end{center}
{\it B}

$\alpha$

{\it C} 8  {\it A}

C. $\displaystyle \frac{62}{5}$

D. $\displaystyle \frac{64}{5}$

Strona 8 z25

$\mathrm{E}\mathrm{M}\mathrm{A}\mathrm{P}-\mathrm{P}0_{-}100$
\end{document}
