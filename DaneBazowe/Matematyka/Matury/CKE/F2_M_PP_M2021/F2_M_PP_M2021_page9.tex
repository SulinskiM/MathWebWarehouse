\documentclass[a4paper,12pt]{article}
\usepackage{latexsym}
\usepackage{amsmath}
\usepackage{amssymb}
\usepackage{graphicx}
\usepackage{wrapfig}
\pagestyle{plain}
\usepackage{fancybox}
\usepackage{bm}

\begin{document}

Zadänie $l9. (0-1)$

Pole pewnego trójkqta równobocznego jest równe $\displaystyle \frac{4\sqrt{3}}{9}$. Obwód tego trójkqta jest równy

A. 4

B. 2

C. -43

D. -23

Zadanie 20. $(0-1$\}

$\mathrm{W}$ trójkqcie $ABC$ bok $BC$ ma dlugość 13, a wysokośč

$CD$ tego trójkqta dzieli bok $AB$ na odcinki o dtugościach

$|AD|=3 \mathrm{i} |BD|=12$ (zobacz rysunek obok). Dlugośč

boku $AC$ jest równa
\begin{center}
\includegraphics[width=64.572mm,height=28.452mm]{./F2_M_PP_M2021_page9_images/image001.eps}
\end{center}
{\it C}

13

{\it A} 3  {\it D}  12  {\it B}

A. $\sqrt{34}$

B. $\displaystyle \frac{13}{4}$

C. $2\sqrt{14}$

D. $3\sqrt{45}$

Zadänie 21. (0-1)

Punkty $A, B, C \mathrm{i} D \mathrm{l}\mathrm{e}\dot{\mathrm{z}}\mathrm{q}$ na okregu o środku $S$. Miary kqtów $SBC, BCD, CDA \mathrm{s}_{\mathrm{c}}1$ równe

odpowiednio: $|4SBC|=60^{\mathrm{o}}, |4BCD|=110^{\mathrm{o}}, |4CDA|=90^{\mathrm{o}}$ (zobacz rysunek).
\begin{center}
\includegraphics[width=60.192mm,height=64.464mm]{./F2_M_PP_M2021_page9_images/image002.eps}
\end{center}
{\it C}

$110^{\mathrm{o}}$

$60^{\mathrm{o}}$  {\it B}

{\it D  S}

$\alpha$

{\it A}

Wynika stqd, $\dot{\mathrm{z}}\mathrm{e}$ miara $\alpha$ kqta DAS jest równa

A. $25^{\mathrm{o}}$

B. $30^{\mathrm{o}}$

C. $35^{\mathrm{o}}$

D. $40^{\mathrm{o}}$

Strona 10 z25

$\mathrm{E}\mathrm{M}\mathrm{A}\mathrm{P}-\mathrm{P}0_{-}100$
\end{document}
