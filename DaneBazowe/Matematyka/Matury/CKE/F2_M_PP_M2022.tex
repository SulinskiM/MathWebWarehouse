\documentclass[a4paper,12pt]{article}
\usepackage{latexsym}
\usepackage{amsmath}
\usepackage{amssymb}
\usepackage{graphicx}
\usepackage{wrapfig}
\pagestyle{plain}
\usepackage{fancybox}
\usepackage{bm}

\begin{document}

CENTRALNA

KOMISJA

EGZAMINACYJNA

Arkusz zawiera informacje prawnie chronione

do momentu rozpoczecia egzaminu.

WYPELNIA ZDAJACY

{\it Miejsce na naklejke}.

{\it Sprawdz}', {\it czy kod na naklejce to}

e-100.
\begin{center}
\includegraphics[width=21.900mm,height=16.260mm]{./F2_M_PP_M2022_page0_images/image001.eps}
\end{center}
KOD
\begin{center}
\includegraphics[width=79.656mm,height=16.260mm]{./F2_M_PP_M2022_page0_images/image002.eps}
\end{center}
PESEL

{\it Jezeli tak}- {\it przyklej naklejkq}.

{\it Jezeli nie}- {\it zgtoś to nauczycielowi}.

EGZAMIN MATURALNY Z MATEMATYKI

POZIOM PODSTAWOWY

WYPELNIA ZESPÓt NADZORUJACY

DATA: 5 maja 2022 $\mathrm{r}.$

GODZINA ROZPOCZeClA: 9: 00

CZAS PRACY: $\{70 \displaystyle \min \mathrm{u}\mathrm{t}$

LICZBA PUNKTÓW DO UZYSKANIA: 45

Uprawnienia zdaj\S cego do:

\fbox{} nieprzenoszenia zaznaczeń na karte

\fbox{} dostosowania zasad oceniania

\fbox{} dostosowania w zw. z dyskalkuliq.

$\Vert\Vert\Vert\Vert\Vert\Vert\Vert\Vert\Vert\Vert\Vert\Vert\Vert\Vert\Vert\Vert\Vert\Vert\Vert\Vert\Vert\Vert\Vert\Vert\Vert\Vert\Vert\Vert\Vert\Vert|$

EMAP-P0-100-2205

lnstrukcja dla zdajqcego

l. Sprawdz', czy arkusz egzaminacyjny zawiera 25 stron (zadania $1-35$).

Ewentualny brak zgloś przewodniczacemu zespolu nadzorujacego egzamin.

2. Na tej stronie oraz na karcie odpowiedzi wpisz swój numer PESEL i przyklej naklejke

z kodem.

3. Nie wpisuj $\dot{\mathrm{z}}$ adnych znaków w cz9ści przeznaczonej d1a egzaminatora.

4. Rozwiqzania zadań i odpowiedzi wpisuj w miejscu na to przeznaczonym.

5. Odpowiedzi do zadań $\mathrm{z}\mathrm{a}\mathrm{m}\mathrm{k}\mathrm{n}\mathrm{i}_{9}$tych ($1-28)$ zaznacz na karcie odpowiedzi w cześci

karty przeznaczonej dla zdajacego. Zamaluj $\blacksquare$ pola do tego przeznaczone. $\mathrm{B}$pdne

zaznaczenie otocz kólkiem @ i zaznacz wlaściwe.

6. Pamietaj, $\dot{\mathrm{z}}\mathrm{e}$ pominiecie argumentacji lub istotnych obliczeń w rozwiqzaniu zadania

otwartego (29-35) $\mathrm{m}\mathrm{o}\dot{\mathrm{z}}\mathrm{e}$ spowodowač, $\dot{\mathrm{z}}\mathrm{e}$ za to rozwiqzanie nie otrzymasz pelnej

liczby punktów.

7. Pisz czytelnie i $\mathrm{u}\dot{\mathrm{z}}$ ywaj tylko dlugopisu lub pióra z czarnym tuszem lub atramentem.

8. Nie $\mathrm{u}\dot{\mathrm{z}}$ ywaj korektora, a bledne zapisy wyraz'nie przekreśl.

9. Pamietaj, $\dot{\mathrm{z}}\mathrm{e}$ zapisy w brudnopisie nie bedq oceniane.

10. $\mathrm{M}\mathrm{o}\dot{\mathrm{z}}$ esz korzystač z zestawu wzorów matematycznych, cyrkla i linijki oraz kalkulatora

prostego.

Uk\}ad graficzny

\copyright CKE 2021




{\it Wkazdym z zadań od} $f.$ {\it do 28. wybierz izaznacz na karcie odpowiedzi poprawna} $od\sqrt{}owi\mathrm{e}d\acute{z}.$

Zadanie $\mathrm{f}. (0-1$\}

Liczba $(2\sqrt{8}-3\sqrt{2})^{2}$ jest równa

A. 2

B. l

C. 26

D. 14

ZadanIe 2. $(0-1$\}

Dodatnie liczby $x \mathrm{i} y \mathrm{s}\mathrm{p}\mathrm{e}$niajq warunek $2x=3y$. Wynika stqd, $\dot{\mathrm{z}}\mathrm{e}$ wartośč wyrazenia

$\displaystyle \frac{x^{2}+y^{2}}{x\cdot y}$ jest równa

A. -23

B. $\displaystyle \frac{13}{6}$

C. $\displaystyle \frac{6}{13}$

D. -23

Zadanie 3. (0-1)

Liczba $4\log_{4}2+2\log_{4}8$ jest równa

A. 61og410

B. 16

C. 5

D. 61og416

Zädanie 4. (0-1)

Cena dzialki po kolejnych dwóch obnizkach, za $\mathrm{k}\mathrm{a}\dot{\mathrm{z}}$ dym razem o 10\% w odniesieniu do ceny

obowiqzujqcej w danym momencie, jest równa 78732 z1. Cena tej dzia1ki przed obiema

obnizkami byla, w zaokragleniu do l zl, równa

A. 98732 z1

B. 97200 z1

C. 95266 z1

D. 94478 z1

Zadanie 5. $(0-1\rangle$

Liczba 3 $2+\displaystyle \frac{1}{4}$ jest równa

A. $3^{2} \sqrt[4]{3}$

B. $\sqrt[4]{3^{3}}$

C. $3^{2}+\sqrt[4]{3}$

D. $3^{2}+ \sqrt{3^{4}}$

Strona 2 z25

$\mathrm{E}\mathrm{M}\mathrm{A}\mathrm{P}-\mathrm{P}0_{-}100$





: {\it RU DNOPIS} \{{\it nie podlega ocenie}\}

$\mathrm{h}\mathrm{P}-\mathrm{P}0_{-}100$

Strona ll z25





$\mathrm{Z}\mathrm{a}\mathrm{d}\mathrm{a}*\mathrm{i}\mathrm{e}19. \langle 0-1\}$

Wysokośč trójkqta równobocznego jest równa $6\sqrt{3}$. Pole tego trójkqta jest równe

A. $3\sqrt{3}$

B. $4\sqrt{3}$

C. $27\sqrt{3}$

D. $36\sqrt{3}$

Zadanie 20. (0-1)

Boki równolegloboku maja dlugości 6 $\mathrm{i} 10$, a kqt rozwarty mipdzy tymi bokami ma

miar9 $120^{\mathrm{o}}$ Pole tego równolegloboku jest równe

A. $30\sqrt{3}$

B. 30

C. $60\sqrt{3}$

D. 60

Zadanie 21. $\langle 0\rightarrow 1$)

Punkty $A=(-2,6)$ oraz $B=(3,b) \mathrm{l}\mathrm{e}\dot{\mathrm{z}}$ a na prostej, która przechodzi przez poczatek

ukladu wspólrzednych. Wtedy $b$ jest równe

A. 9

B. $(-9)$

C. $(-4)$

D. 4

Zadanie $22_{r}(0-1)$

Dane sq cztery proste $k, l, m, n$ o równaniach:

$k$: $\mathrm{y}=-x+1$

$l$: $y=\displaystyle \frac{2}{3}x+1$

$m$: $\displaystyle \mathrm{y}=-\frac{3}{2}x+4$

$n$: $y=-\displaystyle \frac{2}{3}x-1$

Wśród tych prostych prostopadle sa

A. proste k oraz l.

B. proste k oraz n.

C. proste l oraz m.

D. proste m oraz n.

Zadanie 23. (0-1)

Punkty $K=(4,-10) \mathrm{i} L=(b,2)$ sa końcami odcinka $KL$. Pierwsza wspólrz9dna środka

odcinka $KL$ jest równa $(-12)$. Wynika stqd, $\dot{\mathrm{z}}\mathrm{e}$

A. $b=-28$

B. $b=-14$

C. $b=-24$

D. $b=-10$

Strona 12 z25

$\mathrm{E}\mathrm{M}\mathrm{A}\mathrm{P}-\mathrm{P}0_{-}100$





: {\it RU DNOPIS} \{{\it nie podlega ocenie}\}

$\mathrm{h}\mathrm{P}-\mathrm{P}0_{-}100$

Strona 13 z25





Zadarie 24. $(0-1$\}

Punkty $A=(-4,4) \mathrm{i} B=(4,0)$ sq sqsiednimi wierzcholkami kwadratu ABCD. Przekqtna

tego kwadratu ma dlugośč

A. $4\sqrt{10}$

B. $4\sqrt{2}$

C. $4\sqrt{5}$

D. $4\sqrt{7}$

Zadanie 25. (0-1)

Podstawq graniastoslupa prostego jest romb o przekqtnych dlugości 7 cm i 10 cm.

Wysokośč tego graniastoslupa jest krótsza od dluzszej przekqtnej rombu o 2 cm. Wtedy

obj9tośč graniastos1upa jest równa

A. 560 $\mathrm{c}\mathrm{m}^{3}$

B. 280 $\mathrm{c}\mathrm{m}^{3}$

C. $\displaystyle \frac{280}{3}\mathrm{c}\mathrm{m}^{3}$

D. $\displaystyle \frac{560}{3}\mathrm{c}\mathrm{m}^{3}$

Zadanie $26_{*}(0-1)$

Danyjest sześcian ABCDEFGH o krawedzi dlugości $a.$

Punkty $E, F, G, B$ sa wierzcholkami ostroslupa EFGB

(zobacz rysunek).
\begin{center}
\includegraphics[width=57.864mm,height=56.484mm]{./F2_M_PP_M2022_page13_images/image001.eps}
\end{center}
{\it H}

II

{\it G}

{\it E}  IIII

{\it F}

III

I

I

I

I

I

I

I

-- --- $C$

{\it A  B}

{\it a}

Pole powierzchni calkowitej ostroslupa EFGB jest równe

A. $a^{2}$

B. $\displaystyle \frac{3\sqrt{3}}{2}\cdot a^{2}$

C. -23 {\it a}2

D. $\displaystyle \frac{3+\sqrt{3}}{2}\cdot a^{2}$

Zadanie 27. (0-1)

Wszystkich róznych liczb naturalnych czterocyfrowych nieparzystych podzielnych przez 5

jest

A. $9\cdot 8\cdot 7\cdot 2$

B. $9\cdot 10\cdot 10\cdot 1$

C. $9\cdot 10\cdot 10\cdot 2$

D. $9\cdot 9\cdot 8\cdot 1$

Zadanie 28. (0-1)

$\acute{\mathrm{S}}$ rednia arytmetyczna zestawu sześciu liczb: $2x$, 4, 6, 8, 11, 13, jest równa 5. Wynika stqd, $\dot{\mathrm{z}}\mathrm{e}$

A. $x=-1$

B. $x=7$

C. $x=-6$

D. $x=6$

Strona 14 z25

$\mathrm{E}\mathrm{M}\mathrm{A}\mathrm{P}-\mathrm{P}0_{-}100$





: {\it RU DNOPIS} \{{\it nie podlega ocenie}\}

$\mathrm{h}\mathrm{P}-\mathrm{P}0_{-}100$

Strona 15 z25





Zadarie 29. (0-2)

Rozwiqz nierównośč:

$3x^{2}-2x-9\geq 7$

Strona 16 z25

$\mathrm{E}\mathrm{M}\mathrm{A}\mathrm{P}-\mathrm{P}0_{-}10$





Zadarie 30. (0-2)

$\mathrm{W}$ ciqgu arytmetycznym $(a_{n})$, określonym dla $\mathrm{k}\mathrm{a}\dot{\mathrm{z}}$ dej liczby naturalnej $n\geq 1,$

$a_{1}=-1 \mathrm{i} a_{4}=8$. Oblicz sum9 stu poczqtkowych ko1ejnych wyrazów tego ciagu.
\begin{center}
\begin{tabular}{|l|l|l|l|}
\cline{2-4}
&	\multicolumn{1}{|l|}{Nr zadania}&	\multicolumn{1}{|l|}{$29.$}&	\multicolumn{1}{|l|}{ $30.$}	\\
\cline{2-4}
&	\multicolumn{1}{|l|}{Maks. liczba pkt}&	\multicolumn{1}{|l|}{$2$}&	\multicolumn{1}{|l|}{ $2$}	\\
\cline{2-4}
\multicolumn{1}{|l|}{egzaminator}&	\multicolumn{1}{|l|}{Uzyskana liczba pkt}&	\multicolumn{1}{|l|}{}&	\multicolumn{1}{|l|}{}	\\
\hline
\end{tabular}

\end{center}
$\mathrm{E}\mathrm{M}\mathrm{A}\mathrm{P}-\mathrm{P}0_{-}100$

Strona 17 z25





Zadanie 31. (0-2)

Wykaz, $\dot{\mathrm{z}}\mathrm{e}$ dla $\mathrm{k}\mathrm{a}\dot{\mathrm{z}}$ dej liczby rzeczywistej $a$ i $\mathrm{k}\mathrm{a}\dot{\mathrm{z}}$ dej liczby rzeczywistej $b$ takich, $\dot{\mathrm{z}}\mathrm{e} b\neq a,$

spelniona jest nierównośč

-{\it a}2 $+$2 {\it b}2 $>$ (-{\it a} $+$2 {\it b})2

Strona 18 z25

$\mathrm{E}\mathrm{M}\mathrm{A}\mathrm{P}-\mathrm{P}0_{-}100$





Zadanie 32. (0-2)

Kqt $\alpha$ jest ostry i tg $\alpha=2$. Oblicz wartośč wyrazenia $\sin^{2}\alpha.$
\begin{center}
\begin{tabular}{|l|l|l|l|}
\cline{2-4}
&	\multicolumn{1}{|l|}{Nr zadania}&	\multicolumn{1}{|l|}{$31.$}&	\multicolumn{1}{|l|}{ $32.$}	\\
\cline{2-4}
&	\multicolumn{1}{|l|}{Maks. liczba pkt}&	\multicolumn{1}{|l|}{$2$}&	\multicolumn{1}{|l|}{ $2$}	\\
\cline{2-4}
\multicolumn{1}{|l|}{egzaminator}&	\multicolumn{1}{|l|}{Uzyskana liczba pkt}&	\multicolumn{1}{|l|}{}&	\multicolumn{1}{|l|}{}	\\
\hline
\end{tabular}

\end{center}
$\mathrm{E}\mathrm{M}\mathrm{A}\mathrm{P}-\mathrm{P}0_{-}100$

Strona 19 z25





Zadanie 33. $\{0-2\}$

Danyjest trójkqt równoramienny $ABC$, w którym $|AC|=|BC|$. Dwusieczna kqta $BAC$

przecina bok $BC$ w takim punkcie $D, \dot{\mathrm{z}}\mathrm{e}$ trójkqty $ABC \mathrm{i} BDA$ sa podobne (zobacz

rysunek). Oblicz miare kqta $BAC.$

{\it C}
\begin{center}
\includegraphics[width=35.412mm,height=51.456mm]{./F2_M_PP_M2022_page19_images/image001.eps}
\end{center}
{\it D}

{\it A B}

Strona 20 z25

$\mathrm{E}\mathrm{M}\mathrm{A}\mathrm{P}-\mathrm{P}0_{-}100$





: {\it RU DNOPIS} \{{\it nie podlega ocenie}\}

$\mathrm{h}\mathrm{P}-\mathrm{P}0_{-}100$

Strona 3 z25





Zadarie 34. (0-2)

Ze zbioru dziewiecioelementowego $M=\{1$, 2, 3, 4, 5, 6, 7, 8, 9$\}$ losujemy kolejno ze

zwracaniem dwa razy po jednej liczbie. Zdarzenie $A$ polega na wylosowaniu dwóch liczb ze

zbioru $M$, których iloczyn jest równy 24. Ob1icz prawdopodobieństwo zdarzenia $A.$
\begin{center}
\begin{tabular}{|l|l|l|l|}
\cline{2-4}
&	\multicolumn{1}{|l|}{Nr zadania}&	\multicolumn{1}{|l|}{$33.$}&	\multicolumn{1}{|l|}{ $34.$}	\\
\cline{2-4}
&	\multicolumn{1}{|l|}{Maks. liczba pkt}&	\multicolumn{1}{|l|}{$2$}&	\multicolumn{1}{|l|}{ $2$}	\\
\cline{2-4}
\multicolumn{1}{|l|}{egzaminator}&	\multicolumn{1}{|l|}{Uzyskana liczba pkt}&	\multicolumn{1}{|l|}{}&	\multicolumn{1}{|l|}{}	\\
\hline
\end{tabular}

\end{center}
$\mathrm{E}\mathrm{M}\mathrm{A}\mathrm{P}-\mathrm{P}0_{-}100$

Strona 21 z25





Zadanie 35. (0-5)

Wykres funkcji kwadratowej $f$ określonej wzorem $f(x)=ax^{2}+bx+c$ ma z prostq

o równaniu $\mathrm{y}=6$ dokladniejeden punkt wspólny. Punkty $A=(-5,0) \mathrm{i} B=(3,0)$

nalez $\mathrm{c}$] do wykresu funkcji $f$. Oblicz wartości wspólczynników $a, b$ oraz $c.$

Strona 22 z25

$\mathrm{E}\mathrm{M}\mathrm{A}\mathrm{P}-\mathrm{P}0_{-}100$





Wypelnia

egzaminator

Nr zadania

Maks. liczba pkt

Uzyskana liczba pkt

35.

5

-PO-100

Strona 23 z25





: {\it RU DNOPIS} \{{\it nie podlega ocenie}\}

Strona 24z 25

$\mathrm{E}\mathrm{M}\mathrm{A}\mathrm{P}-\mathrm{P}0_{-}10$





$0_{-}100$

Strona 25 z25




















Zadarie 6. $(0-1$\}

Rozwiqzaniem ukladu równań 

A. $\chi_{0}>0 \mathrm{i}$

$\mathrm{y}_{0}>0$

B. $\chi_{0}>0 \mathrm{i}$

$y_{0}<0$

C. $\chi_{0}<0 \mathrm{i}$

$\mathrm{y}_{0}>0$

D. $\chi_{0}<0 \mathrm{i}$

$y_{0}<0$

Zadanie 7. $(0-1$\}

Zbiorem wszystkich rozwiqzań nierówności $\displaystyle \frac{2}{5}-\frac{\chi}{3}>\frac{\chi}{5}$ jest przedzial

A. $(-\infty,0)$

B. $(0,+\infty)$

C.(-$\infty$,-43)

D. $(\displaystyle \frac{3}{4},+\infty)$

Zadanie 8. $\langle 0-1$)

lloczyn wszystkich rozwiazań równania $2x(x^{2}-9)(x+1)=0$ jest równy

A. $(-3)$

B. 3

C. 0

D. 9

Zadanie 9. (0-1)

Na rysunku przedstawiono wykres funkcji f.
\begin{center}
\includegraphics[width=161.640mm,height=89.148mm]{./F2_M_PP_M2022_page3_images/image001.eps}
\end{center}
{\it y}

1 0  1  10 $\chi$

B. $(-8)$

A. $(-12)$

lloczyn $f(-3)\cdot f(0)\cdot f(4)$ jest równy

C. 0

D. 16

Strona 4 z25

$\mathrm{E}\mathrm{M}\mathrm{A}\mathrm{P}-\mathrm{P}0_{-}100$





: {\it RU DNOPIS} \{{\it nie podlega ocenie}\}

$\mathrm{h}\mathrm{P}-\mathrm{P}0_{-}100$

Strona 5 z25





Zadanie 10. $\{0-1\}$

Na rysunku l. przedstawiono wykres funkcji $f$ określonej na zbiorze $\langle-4, 5\rangle.$

Rysunek l.

Funkcje g określono za pomocq funkcji f. Wykres funkcji g przedstawiono na rysunku 2.

Rysunek 2.

Wynika stqd, $\dot{\mathrm{z}}\mathrm{e}$

A. $g(x)=f(x)-2$

C. $g(x)=f(x)+2$

B. $g(x)=f(x-2)$

D. $g(x)=f(x+2)$

Strona 6 z25

$\mathrm{E}\mathrm{M}\mathrm{A}\mathrm{P}-\mathrm{P}0_{-}100$





: {\it RU DNOPIS} \{{\it nie podlega ocenie}\}

$\mathrm{h}\mathrm{P}-\mathrm{P}0_{-}100$

Strona 7 z25





Zadanie ll. $\langle 0-1$\}

Miejscem zerowym funkcji liniowej $f$ określonej wzorem $f(x)=-\displaystyle \frac{1}{3}(x+3)+5$ jest liczba

A. $(-3)$

B. -92

C. 5

D. 12

Zadan$\mathrm{e}12. \langle 0-1$)

Wykresem funkcji kwadratowej $f(x)=3x^{2}+bx+c$ jest parabola o wierzcholku w punkcie

$W=(-3,2)$. Wzór tej funkcji w postaci kanonicznej to

A. $f(x)=3(x-3)^{2}+2$

B. $f(x)=3(x+3)^{2}+2$

C. $f(x)=(x-3)^{2}+2$

D. $f(x)=(x+3)^{2}+2$

Zadanie 13. (0-1)

Ciqg $(a_{n})$ jest określony wzorem $a_{n}=\displaystyle \frac{2n^{2}-30n}{n}$ dla $\mathrm{k}\mathrm{a}\dot{\mathrm{z}}$ dej liczby naturalnej $n\geq 1.$

Wtedy $a_{7}$ jest równy

A. $(-196)$

B. $(-32)$

C. $(-26)$

D. $(-16)$

Zadanie 14. $\langle 0-1$)

$\mathrm{W}$ ciqgu arytmetycznym $(a_{n})$, określonym dla $\mathrm{k}\mathrm{a}\dot{\mathrm{z}}$ dej liczby naturalnej $n\geq 1,$

$a_{5}=-31$ oraz $a_{10}=-66$. Róznica tego ciagu jest równa

A. $(-7)$

B. $(-19,4)$

C. 7

D. 19,4

Zadanie \{5. (0-1)

Wszystkie wyrazy nieskończonego ciqgu geometrycznego $(a_{n})$, określonego dla $\mathrm{k}\mathrm{a}\dot{\mathrm{z}}$ dej

liczby naturalnej $n\geq 1$, sa dodatnie i $9a_{5}=4a_{3}$. Wtedy iloraz tego ciqgu jest równy

A. -23

B. -23

C. -92

D. -92

Zadanie 16. $(0-1$\}

Liczba $\cos 12^{\mathrm{o}}\cdot\sin 78^{\mathrm{o}}+\sin 12^{\mathrm{o}}\cdot\cos 78^{\mathrm{o}}$ jest równa

A. -21

B. $\displaystyle \frac{\sqrt{2}}{2}$

C. $\displaystyle \frac{\sqrt{3}}{2}$

D. l

Strona 8 z25

$\mathrm{E}\mathrm{M}\mathrm{A}\mathrm{P}-\mathrm{P}0_{-}100$





: {\it RU DNOPIS} \{{\it nie podlega ocenie}\}

$\mathrm{h}\mathrm{P}-\mathrm{P}0_{-}100$

Strona 9 z25





$\mathrm{Z}\mathrm{a}\mathrm{d}\mathrm{a}*\mathrm{i}\mathrm{e}17, \langle 0-1\}$

Punkty $A, B, C \mathrm{l}\mathrm{e}\dot{\mathrm{z}}\mathrm{q}$ na okregu o środku $S$. Punkt $D$ jest punktem przeciecia $\mathrm{c}\mathrm{i}_{9}$ciwy $AC$

i średnicy okregu poprowadzonej z punktu $B$. Miara kqta $BSC$ jest równa $\alpha$, a miara kqta

$ADB$ jest równa $\gamma$ (zobacz rysunek).
\begin{center}
\includegraphics[width=64.668mm,height=60.456mm]{./F2_M_PP_M2022_page9_images/image001.eps}
\end{center}
{\it A}

{\it S}

{\it D}

$\gamma$

{\it C}

$\alpha$

{\it B}

B. $ 180^{\mathrm{o}}-\displaystyle \frac{\alpha}{2}-\gamma$

A. $\displaystyle \frac{\alpha}{2}+\gamma-180^{\mathrm{o}}$

Wtedy kqt ABD ma miare

C. $ 180^{\mathrm{o}}-\alpha-\gamma$

D. $\alpha+\gamma-180^{\mathrm{o}}$

Zadanie 18. (0-1)

Punkty $A, B, P \mathrm{l}\mathrm{e}\dot{\mathrm{z}}\mathrm{q}$ na okregu o środku $S$ i promieniu 6. Czworokat ASBP jest rombem,

w którym $\mathrm{k}\mathrm{a}\mathrm{t}$ ostry PAS ma miare $60^{\mathrm{o}}$ (zobacz rysunek).
\begin{center}
\includegraphics[width=76.296mm,height=79.296mm]{./F2_M_PP_M2022_page9_images/image002.eps}
\end{center}
{\it P}

{\it A}

{\it B}

{\it S}

Pole zakreskowanej na rysunku figury jest równe

A. $ 6\pi$

B. $ 9\pi$

C. $ 10\pi$

D. $ 12\pi$

Strona 10 z25

$\mathrm{E}\mathrm{M}\mathrm{A}\mathrm{P}-\mathrm{P}0_{-}100$



\end{document}