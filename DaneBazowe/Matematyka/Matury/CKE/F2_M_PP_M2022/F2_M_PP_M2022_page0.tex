\documentclass[a4paper,12pt]{article}
\usepackage{latexsym}
\usepackage{amsmath}
\usepackage{amssymb}
\usepackage{graphicx}
\usepackage{wrapfig}
\pagestyle{plain}
\usepackage{fancybox}
\usepackage{bm}

\begin{document}

CENTRALNA

KOMISJA

EGZAMINACYJNA

Arkusz zawiera informacje prawnie chronione

do momentu rozpoczecia egzaminu.

WYPELNIA ZDAJACY

{\it Miejsce na naklejke}.

{\it Sprawdz}', {\it czy kod na naklejce to}

e-100.
\begin{center}
\includegraphics[width=21.900mm,height=16.260mm]{./F2_M_PP_M2022_page0_images/image001.eps}
\end{center}
KOD
\begin{center}
\includegraphics[width=79.656mm,height=16.260mm]{./F2_M_PP_M2022_page0_images/image002.eps}
\end{center}
PESEL

{\it Jezeli tak}- {\it przyklej naklejkq}.

{\it Jezeli nie}- {\it zgtoś to nauczycielowi}.

EGZAMIN MATURALNY Z MATEMATYKI

POZIOM PODSTAWOWY

WYPELNIA ZESPÓt NADZORUJACY

DATA: 5 maja 2022 $\mathrm{r}.$

GODZINA ROZPOCZeClA: 9: 00

CZAS PRACY: $\{70 \displaystyle \min \mathrm{u}\mathrm{t}$

LICZBA PUNKTÓW DO UZYSKANIA: 45

Uprawnienia zdaj\S cego do:

\fbox{} nieprzenoszenia zaznaczeń na karte

\fbox{} dostosowania zasad oceniania

\fbox{} dostosowania w zw. z dyskalkuliq.

$\Vert\Vert\Vert\Vert\Vert\Vert\Vert\Vert\Vert\Vert\Vert\Vert\Vert\Vert\Vert\Vert\Vert\Vert\Vert\Vert\Vert\Vert\Vert\Vert\Vert\Vert\Vert\Vert\Vert\Vert|$

EMAP-P0-100-2205

lnstrukcja dla zdajqcego

l. Sprawdz', czy arkusz egzaminacyjny zawiera 25 stron (zadania $1-35$).

Ewentualny brak zgloś przewodniczacemu zespolu nadzorujacego egzamin.

2. Na tej stronie oraz na karcie odpowiedzi wpisz swój numer PESEL i przyklej naklejke

z kodem.

3. Nie wpisuj $\dot{\mathrm{z}}$ adnych znaków w cz9ści przeznaczonej d1a egzaminatora.

4. Rozwiqzania zadań i odpowiedzi wpisuj w miejscu na to przeznaczonym.

5. Odpowiedzi do zadań $\mathrm{z}\mathrm{a}\mathrm{m}\mathrm{k}\mathrm{n}\mathrm{i}_{9}$tych ($1-28)$ zaznacz na karcie odpowiedzi w cześci

karty przeznaczonej dla zdajacego. Zamaluj $\blacksquare$ pola do tego przeznaczone. $\mathrm{B}$pdne

zaznaczenie otocz kólkiem @ i zaznacz wlaściwe.

6. Pamietaj, $\dot{\mathrm{z}}\mathrm{e}$ pominiecie argumentacji lub istotnych obliczeń w rozwiqzaniu zadania

otwartego (29-35) $\mathrm{m}\mathrm{o}\dot{\mathrm{z}}\mathrm{e}$ spowodowač, $\dot{\mathrm{z}}\mathrm{e}$ za to rozwiqzanie nie otrzymasz pelnej

liczby punktów.

7. Pisz czytelnie i $\mathrm{u}\dot{\mathrm{z}}$ ywaj tylko dlugopisu lub pióra z czarnym tuszem lub atramentem.

8. Nie $\mathrm{u}\dot{\mathrm{z}}$ ywaj korektora, a bledne zapisy wyraz'nie przekreśl.

9. Pamietaj, $\dot{\mathrm{z}}\mathrm{e}$ zapisy w brudnopisie nie bedq oceniane.

10. $\mathrm{M}\mathrm{o}\dot{\mathrm{z}}$ esz korzystač z zestawu wzorów matematycznych, cyrkla i linijki oraz kalkulatora

prostego.

Uk\}ad graficzny

\copyright CKE 2021
\end{document}
