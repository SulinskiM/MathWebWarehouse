\documentclass[a4paper,12pt]{article}
\usepackage{latexsym}
\usepackage{amsmath}
\usepackage{amssymb}
\usepackage{graphicx}
\usepackage{wrapfig}
\pagestyle{plain}
\usepackage{fancybox}
\usepackage{bm}

\begin{document}

{\it Wkazdym z zadań od} $f.$ {\it do 28. wybierz izaznacz na karcie odpowiedzi poprawna} $od\sqrt{}owi\mathrm{e}d\acute{z}.$

Zadanie $\mathrm{f}. (0-1$\}

Liczba $(2\sqrt{8}-3\sqrt{2})^{2}$ jest równa

A. 2

B. l

C. 26

D. 14

ZadanIe 2. $(0-1$\}

Dodatnie liczby $x \mathrm{i} y \mathrm{s}\mathrm{p}\mathrm{e}$niajq warunek $2x=3y$. Wynika stqd, $\dot{\mathrm{z}}\mathrm{e}$ wartośč wyrazenia

$\displaystyle \frac{x^{2}+y^{2}}{x\cdot y}$ jest równa

A. -23

B. $\displaystyle \frac{13}{6}$

C. $\displaystyle \frac{6}{13}$

D. -23

Zadanie 3. (0-1)

Liczba $4\log_{4}2+2\log_{4}8$ jest równa

A. 61og410

B. 16

C. 5

D. 61og416

Zädanie 4. (0-1)

Cena dzialki po kolejnych dwóch obnizkach, za $\mathrm{k}\mathrm{a}\dot{\mathrm{z}}$ dym razem o 10\% w odniesieniu do ceny

obowiqzujqcej w danym momencie, jest równa 78732 z1. Cena tej dzia1ki przed obiema

obnizkami byla, w zaokragleniu do l zl, równa

A. 98732 z1

B. 97200 z1

C. 95266 z1

D. 94478 z1

Zadanie 5. $(0-1\rangle$

Liczba 3 $2+\displaystyle \frac{1}{4}$ jest równa

A. $3^{2} \sqrt[4]{3}$

B. $\sqrt[4]{3^{3}}$

C. $3^{2}+\sqrt[4]{3}$

D. $3^{2}+ \sqrt{3^{4}}$

Strona 2 z25

$\mathrm{E}\mathrm{M}\mathrm{A}\mathrm{P}-\mathrm{P}0_{-}100$
\end{document}
