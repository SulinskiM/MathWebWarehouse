\documentclass[a4paper,12pt]{article}
\usepackage{latexsym}
\usepackage{amsmath}
\usepackage{amssymb}
\usepackage{graphicx}
\usepackage{wrapfig}
\pagestyle{plain}
\usepackage{fancybox}
\usepackage{bm}

\begin{document}

$\mathrm{Z}\mathrm{a}\mathrm{d}\mathrm{a}*\mathrm{i}\mathrm{e}19. \langle 0-1\}$

Wysokośč trójkqta równobocznego jest równa $6\sqrt{3}$. Pole tego trójkqta jest równe

A. $3\sqrt{3}$

B. $4\sqrt{3}$

C. $27\sqrt{3}$

D. $36\sqrt{3}$

Zadanie 20. (0-1)

Boki równolegloboku maja dlugości 6 $\mathrm{i} 10$, a kqt rozwarty mipdzy tymi bokami ma

miar9 $120^{\mathrm{o}}$ Pole tego równolegloboku jest równe

A. $30\sqrt{3}$

B. 30

C. $60\sqrt{3}$

D. 60

Zadanie 21. $\langle 0\rightarrow 1$)

Punkty $A=(-2,6)$ oraz $B=(3,b) \mathrm{l}\mathrm{e}\dot{\mathrm{z}}$ a na prostej, która przechodzi przez poczatek

ukladu wspólrzednych. Wtedy $b$ jest równe

A. 9

B. $(-9)$

C. $(-4)$

D. 4

Zadanie $22_{r}(0-1)$

Dane sq cztery proste $k, l, m, n$ o równaniach:

$k$: $\mathrm{y}=-x+1$

$l$: $y=\displaystyle \frac{2}{3}x+1$

$m$: $\displaystyle \mathrm{y}=-\frac{3}{2}x+4$

$n$: $y=-\displaystyle \frac{2}{3}x-1$

Wśród tych prostych prostopadle sa

A. proste k oraz l.

B. proste k oraz n.

C. proste l oraz m.

D. proste m oraz n.

Zadanie 23. (0-1)

Punkty $K=(4,-10) \mathrm{i} L=(b,2)$ sa końcami odcinka $KL$. Pierwsza wspólrz9dna środka

odcinka $KL$ jest równa $(-12)$. Wynika stqd, $\dot{\mathrm{z}}\mathrm{e}$

A. $b=-28$

B. $b=-14$

C. $b=-24$

D. $b=-10$

Strona 12 z25

$\mathrm{E}\mathrm{M}\mathrm{A}\mathrm{P}-\mathrm{P}0_{-}100$
\end{document}
