\documentclass[a4paper,12pt]{article}
\usepackage{latexsym}
\usepackage{amsmath}
\usepackage{amssymb}
\usepackage{graphicx}
\usepackage{wrapfig}
\pagestyle{plain}
\usepackage{fancybox}
\usepackage{bm}

\begin{document}

Zadarie 24. $(0-1$\}

Punkty $A=(-4,4) \mathrm{i} B=(4,0)$ sq sqsiednimi wierzcholkami kwadratu ABCD. Przekqtna

tego kwadratu ma dlugośč

A. $4\sqrt{10}$

B. $4\sqrt{2}$

C. $4\sqrt{5}$

D. $4\sqrt{7}$

Zadanie 25. (0-1)

Podstawq graniastoslupa prostego jest romb o przekqtnych dlugości 7 cm i 10 cm.

Wysokośč tego graniastoslupa jest krótsza od dluzszej przekqtnej rombu o 2 cm. Wtedy

obj9tośč graniastos1upa jest równa

A. 560 $\mathrm{c}\mathrm{m}^{3}$

B. 280 $\mathrm{c}\mathrm{m}^{3}$

C. $\displaystyle \frac{280}{3}\mathrm{c}\mathrm{m}^{3}$

D. $\displaystyle \frac{560}{3}\mathrm{c}\mathrm{m}^{3}$

Zadanie $26_{*}(0-1)$

Danyjest sześcian ABCDEFGH o krawedzi dlugości $a.$

Punkty $E, F, G, B$ sa wierzcholkami ostroslupa EFGB

(zobacz rysunek).
\begin{center}
\includegraphics[width=57.864mm,height=56.484mm]{./F2_M_PP_M2022_page13_images/image001.eps}
\end{center}
{\it H}

II

{\it G}

{\it E}  IIII

{\it F}

III

I

I

I

I

I

I

I

-- --- $C$

{\it A  B}

{\it a}

Pole powierzchni calkowitej ostroslupa EFGB jest równe

A. $a^{2}$

B. $\displaystyle \frac{3\sqrt{3}}{2}\cdot a^{2}$

C. -23 {\it a}2

D. $\displaystyle \frac{3+\sqrt{3}}{2}\cdot a^{2}$

Zadanie 27. (0-1)

Wszystkich róznych liczb naturalnych czterocyfrowych nieparzystych podzielnych przez 5

jest

A. $9\cdot 8\cdot 7\cdot 2$

B. $9\cdot 10\cdot 10\cdot 1$

C. $9\cdot 10\cdot 10\cdot 2$

D. $9\cdot 9\cdot 8\cdot 1$

Zadanie 28. (0-1)

$\acute{\mathrm{S}}$ rednia arytmetyczna zestawu sześciu liczb: $2x$, 4, 6, 8, 11, 13, jest równa 5. Wynika stqd, $\dot{\mathrm{z}}\mathrm{e}$

A. $x=-1$

B. $x=7$

C. $x=-6$

D. $x=6$

Strona 14 z25

$\mathrm{E}\mathrm{M}\mathrm{A}\mathrm{P}-\mathrm{P}0_{-}100$
\end{document}
