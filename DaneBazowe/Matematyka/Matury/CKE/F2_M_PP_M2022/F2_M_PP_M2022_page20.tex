\documentclass[a4paper,12pt]{article}
\usepackage{latexsym}
\usepackage{amsmath}
\usepackage{amssymb}
\usepackage{graphicx}
\usepackage{wrapfig}
\pagestyle{plain}
\usepackage{fancybox}
\usepackage{bm}

\begin{document}

Zadarie 34. (0-2)

Ze zbioru dziewiecioelementowego $M=\{1$, 2, 3, 4, 5, 6, 7, 8, 9$\}$ losujemy kolejno ze

zwracaniem dwa razy po jednej liczbie. Zdarzenie $A$ polega na wylosowaniu dwóch liczb ze

zbioru $M$, których iloczyn jest równy 24. Ob1icz prawdopodobieństwo zdarzenia $A.$
\begin{center}
\begin{tabular}{|l|l|l|l|}
\cline{2-4}
&	\multicolumn{1}{|l|}{Nr zadania}&	\multicolumn{1}{|l|}{$33.$}&	\multicolumn{1}{|l|}{ $34.$}	\\
\cline{2-4}
&	\multicolumn{1}{|l|}{Maks. liczba pkt}&	\multicolumn{1}{|l|}{$2$}&	\multicolumn{1}{|l|}{ $2$}	\\
\cline{2-4}
\multicolumn{1}{|l|}{egzaminator}&	\multicolumn{1}{|l|}{Uzyskana liczba pkt}&	\multicolumn{1}{|l|}{}&	\multicolumn{1}{|l|}{}	\\
\hline
\end{tabular}

\end{center}
$\mathrm{E}\mathrm{M}\mathrm{A}\mathrm{P}-\mathrm{P}0_{-}100$

Strona 21 z25
\end{document}
