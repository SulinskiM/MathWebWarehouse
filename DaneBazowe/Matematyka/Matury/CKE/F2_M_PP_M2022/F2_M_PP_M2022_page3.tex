\documentclass[a4paper,12pt]{article}
\usepackage{latexsym}
\usepackage{amsmath}
\usepackage{amssymb}
\usepackage{graphicx}
\usepackage{wrapfig}
\pagestyle{plain}
\usepackage{fancybox}
\usepackage{bm}

\begin{document}

Zadarie 6. $(0-1$\}

Rozwiqzaniem ukladu równań 

A. $\chi_{0}>0 \mathrm{i}$

$\mathrm{y}_{0}>0$

B. $\chi_{0}>0 \mathrm{i}$

$y_{0}<0$

C. $\chi_{0}<0 \mathrm{i}$

$\mathrm{y}_{0}>0$

D. $\chi_{0}<0 \mathrm{i}$

$y_{0}<0$

Zadanie 7. $(0-1$\}

Zbiorem wszystkich rozwiqzań nierówności $\displaystyle \frac{2}{5}-\frac{\chi}{3}>\frac{\chi}{5}$ jest przedzial

A. $(-\infty,0)$

B. $(0,+\infty)$

C.(-$\infty$,-43)

D. $(\displaystyle \frac{3}{4},+\infty)$

Zadanie 8. $\langle 0-1$)

lloczyn wszystkich rozwiazań równania $2x(x^{2}-9)(x+1)=0$ jest równy

A. $(-3)$

B. 3

C. 0

D. 9

Zadanie 9. (0-1)

Na rysunku przedstawiono wykres funkcji f.
\begin{center}
\includegraphics[width=161.640mm,height=89.148mm]{./F2_M_PP_M2022_page3_images/image001.eps}
\end{center}
{\it y}

1 0  1  10 $\chi$

B. $(-8)$

A. $(-12)$

lloczyn $f(-3)\cdot f(0)\cdot f(4)$ jest równy

C. 0

D. 16

Strona 4 z25

$\mathrm{E}\mathrm{M}\mathrm{A}\mathrm{P}-\mathrm{P}0_{-}100$
\end{document}
