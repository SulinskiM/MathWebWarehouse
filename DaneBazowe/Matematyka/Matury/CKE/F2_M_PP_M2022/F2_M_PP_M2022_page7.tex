\documentclass[a4paper,12pt]{article}
\usepackage{latexsym}
\usepackage{amsmath}
\usepackage{amssymb}
\usepackage{graphicx}
\usepackage{wrapfig}
\pagestyle{plain}
\usepackage{fancybox}
\usepackage{bm}

\begin{document}

Zadanie ll. $\langle 0-1$\}

Miejscem zerowym funkcji liniowej $f$ określonej wzorem $f(x)=-\displaystyle \frac{1}{3}(x+3)+5$ jest liczba

A. $(-3)$

B. -92

C. 5

D. 12

Zadan$\mathrm{e}12. \langle 0-1$)

Wykresem funkcji kwadratowej $f(x)=3x^{2}+bx+c$ jest parabola o wierzcholku w punkcie

$W=(-3,2)$. Wzór tej funkcji w postaci kanonicznej to

A. $f(x)=3(x-3)^{2}+2$

B. $f(x)=3(x+3)^{2}+2$

C. $f(x)=(x-3)^{2}+2$

D. $f(x)=(x+3)^{2}+2$

Zadanie 13. (0-1)

Ciqg $(a_{n})$ jest określony wzorem $a_{n}=\displaystyle \frac{2n^{2}-30n}{n}$ dla $\mathrm{k}\mathrm{a}\dot{\mathrm{z}}$ dej liczby naturalnej $n\geq 1.$

Wtedy $a_{7}$ jest równy

A. $(-196)$

B. $(-32)$

C. $(-26)$

D. $(-16)$

Zadanie 14. $\langle 0-1$)

$\mathrm{W}$ ciqgu arytmetycznym $(a_{n})$, określonym dla $\mathrm{k}\mathrm{a}\dot{\mathrm{z}}$ dej liczby naturalnej $n\geq 1,$

$a_{5}=-31$ oraz $a_{10}=-66$. Róznica tego ciagu jest równa

A. $(-7)$

B. $(-19,4)$

C. 7

D. 19,4

Zadanie \{5. (0-1)

Wszystkie wyrazy nieskończonego ciqgu geometrycznego $(a_{n})$, określonego dla $\mathrm{k}\mathrm{a}\dot{\mathrm{z}}$ dej

liczby naturalnej $n\geq 1$, sa dodatnie i $9a_{5}=4a_{3}$. Wtedy iloraz tego ciqgu jest równy

A. -23

B. -23

C. -92

D. -92

Zadanie 16. $(0-1$\}

Liczba $\cos 12^{\mathrm{o}}\cdot\sin 78^{\mathrm{o}}+\sin 12^{\mathrm{o}}\cdot\cos 78^{\mathrm{o}}$ jest równa

A. -21

B. $\displaystyle \frac{\sqrt{2}}{2}$

C. $\displaystyle \frac{\sqrt{3}}{2}$

D. l

Strona 8 z25

$\mathrm{E}\mathrm{M}\mathrm{A}\mathrm{P}-\mathrm{P}0_{-}100$
\end{document}
