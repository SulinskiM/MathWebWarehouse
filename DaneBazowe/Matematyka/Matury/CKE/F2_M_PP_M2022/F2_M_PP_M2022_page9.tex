\documentclass[a4paper,12pt]{article}
\usepackage{latexsym}
\usepackage{amsmath}
\usepackage{amssymb}
\usepackage{graphicx}
\usepackage{wrapfig}
\pagestyle{plain}
\usepackage{fancybox}
\usepackage{bm}

\begin{document}

$\mathrm{Z}\mathrm{a}\mathrm{d}\mathrm{a}*\mathrm{i}\mathrm{e}17, \langle 0-1\}$

Punkty $A, B, C \mathrm{l}\mathrm{e}\dot{\mathrm{z}}\mathrm{q}$ na okregu o środku $S$. Punkt $D$ jest punktem przeciecia $\mathrm{c}\mathrm{i}_{9}$ciwy $AC$

i średnicy okregu poprowadzonej z punktu $B$. Miara kqta $BSC$ jest równa $\alpha$, a miara kqta

$ADB$ jest równa $\gamma$ (zobacz rysunek).
\begin{center}
\includegraphics[width=64.668mm,height=60.456mm]{./F2_M_PP_M2022_page9_images/image001.eps}
\end{center}
{\it A}

{\it S}

{\it D}

$\gamma$

{\it C}

$\alpha$

{\it B}

B. $ 180^{\mathrm{o}}-\displaystyle \frac{\alpha}{2}-\gamma$

A. $\displaystyle \frac{\alpha}{2}+\gamma-180^{\mathrm{o}}$

Wtedy kqt ABD ma miare

C. $ 180^{\mathrm{o}}-\alpha-\gamma$

D. $\alpha+\gamma-180^{\mathrm{o}}$

Zadanie 18. (0-1)

Punkty $A, B, P \mathrm{l}\mathrm{e}\dot{\mathrm{z}}\mathrm{q}$ na okregu o środku $S$ i promieniu 6. Czworokat ASBP jest rombem,

w którym $\mathrm{k}\mathrm{a}\mathrm{t}$ ostry PAS ma miare $60^{\mathrm{o}}$ (zobacz rysunek).
\begin{center}
\includegraphics[width=76.296mm,height=79.296mm]{./F2_M_PP_M2022_page9_images/image002.eps}
\end{center}
{\it P}

{\it A}

{\it B}

{\it S}

Pole zakreskowanej na rysunku figury jest równe

A. $ 6\pi$

B. $ 9\pi$

C. $ 10\pi$

D. $ 12\pi$

Strona 10 z25

$\mathrm{E}\mathrm{M}\mathrm{A}\mathrm{P}-\mathrm{P}0_{-}100$
\end{document}
