\documentclass[a4paper,12pt]{article}
\usepackage{latexsym}
\usepackage{amsmath}
\usepackage{amssymb}
\usepackage{graphicx}
\usepackage{wrapfig}
\pagestyle{plain}
\usepackage{fancybox}
\usepackage{bm}

\begin{document}

CENTRALNA

KOMISJA

EGZAMINACYJNA

Arkusz zawiera informacje prawnie chronione

do momentu rozpoczecia egzaminu.

KOD

WYPELNIA ZDAJACY

PESEL

{\it Miejsce na naklejke}.

{\it Sprawdz}', {\it czy kod na naklejce to}

e-100.
\begin{center}
\includegraphics[width=21.900mm,height=10.164mm]{./F2_M_PP_M2023_page0_images/image001.eps}

\includegraphics[width=79.656mm,height=10.164mm]{./F2_M_PP_M2023_page0_images/image002.eps}
\end{center}
/{\it ezeli tak}- {\it przyklej naklejkq}.

/{\it ezeli nie}- {\it zgtoś to nauczycielowi}.

Egzamin maturalny

DATA: 8 maja 2023 r.

GODZINA R0ZP0CZECIA: 9:00

CZAS TRWANIA: $170 \displaystyle \min$ ut

MAP-P0-100-2305

WyPEtNlA ZESPÓL NADZORUJACY

Uprawnienia $\mathrm{z}\mathrm{d}\mathrm{a}\mathrm{j}_{8}$cego do:

\fbox{} dostosowania zasad oceniania

\fbox{} dostosowania w zw. z dyskalkuliq

\fbox{} nieprzenoszenia zaznaczeń na karte.

LICZBA PUNKTÓW DO UZYSKANIA 46

Przed rozpoczeciem pracy z arkuszem egzaminacyjnym

1.

Sprawd $\acute{\mathrm{z}}$, czy nauczyciel przekazal Ci wlaściwy arkusz egzaminacyjny,

tj. arkusz we wlaściwej formule, z w[aściwego przedmiotu na wlaściwym

poziomie.

2.

$\mathrm{J}\mathrm{e}\dot{\mathrm{z}}$ eli przekazano Ci niew[aściwy arkusz- natychmiast zgloś to nauczycielowi.

Nie rozrywaj banderol.

3.

$\mathrm{J}\mathrm{e}\dot{\mathrm{z}}$ eli przekazano Ci w[aściwy arkusz- rozerwij banderole po otrzymaniu

takiego polecenia od nauczyciela. Zapoznaj $\mathrm{s}\mathrm{i}\mathrm{e}$ z instrukcjq na stronie 2.

Uk\}ad graficzny

\copyright CKE 2022

$\Vert\Vert\Vert\Vert\Vert\Vert\Vert\Vert\Vert\Vert\Vert\Vert\Vert\Vert\Vert\Vert\Vert\Vert\Vert\Vert\Vert\Vert\Vert\Vert\Vert\Vert\Vert\Vert\Vert\Vert|$




lnstrukcja dla zdajqcego

l. Sprawdz', czy arkusz egzaminacyjny zawiera 30 stron (zadania $1-36$).

Ewentualny brak zgloś przewodniczqcemu zespolu nadzorujacego egzamin.

2. Na pierwszej stronie arkusza oraz na karcie odpowiedzi wpisz swój numer PESEL

i przyklej naklejke z kodem.

3. Odpowiedzi do zadań zamknietych $(1-29)$ zaznacz na karcie odpowiedzi w cz9ści karty

przeznaczonej dla zdajacego. Zamaluj $\blacksquare$ pola do tego przeznaczone. $\mathrm{B}_{9}\mathrm{d}\mathrm{n}\mathrm{e}$

zaznaczenie otocz kólkiem \copyright i zaznacz wlaściwe.

4. Pamiptaj, $\dot{\mathrm{z}}\mathrm{e}$ pominiecie argumentacji lub istotnych obliczeń w rozwiqzaniu zadania

otwartego (30-36) $\mathrm{m}\mathrm{o}\dot{\mathrm{z}}\mathrm{e}$ spowodowač, $\dot{\mathrm{z}}\mathrm{e}$ za to rozwiqzanie nie otrzymasz pelnej liczby

punktów.

5. Rozwiqzania zadań i odpowiedzi wpisuj w miejscu na to przeznaczonym.

6. Pisz czytelnie i $\mathrm{u}\dot{\mathrm{z}}$ ywaj tylko dlugopisu lub pióra z czarnym tuszem lub atramentem.

7. Nie $\mathrm{u}\dot{\mathrm{z}}$ ywaj korektora, a bledne zapisy wyra $\acute{\mathrm{z}}\mathrm{n}\mathrm{i}\mathrm{e}$ przekreśl.

8. Nie wpisuj $\dot{\mathrm{z}}$ adnych znaków w cześci przeznaczonej dla egzaminatora.

9. $\mathrm{P}\mathrm{a}\mathrm{m}\mathrm{i}_{9}\mathrm{t}\mathrm{a}\mathrm{j}, \dot{\mathrm{z}}\mathrm{e}$ zapisy w brudnopisie nie bedq oceniane.

10. $\mathrm{M}\mathrm{o}\dot{\mathrm{z}}$ esz korzystač z Wybranych wzorów matematycznych, cyrkla i linijki oraz kalkulatora

prostego. Upewnij $\mathrm{s}\mathrm{i}\mathrm{e}$, czy przekazano Ci broszur9 z ok1adka taka jak widoczna ponizej.

Wybrane wzo y

matematyczne

$a\wedge=.\dot{f}^{-\prime}(x_{(},)$

$q\cdot\underline{\wedge n\tau\wedge}\iota\omega \mathrm{o}\mathrm{n}$

Strona 2 z30

$\mathrm{E}\mathrm{M}\mathrm{A}\mathrm{P}-\mathrm{P}0_{-}100$





: {\it RU DNOPIS} \{{\it nie podlega ocenie}\}

$-\mathrm{P}0_{-}100$

Strona llz30





Zadanie 15. $\langle 0-1$\}

Ciqg $(a_{n})$ jest określony wzorem $a_{n}=2^{n}$

Wyraz $a_{4}$ jest równy

$(n+1)$ dla $\mathrm{k}\mathrm{a}\dot{\mathrm{z}}$ dej liczby naturalnej $n\geq 1.$

A. 64

B. 40

C. 48

D. 80

Zadanie 16. $(0-1\mathrm{J}$

Trzywyrazowy ciag $($27, 9, $a-1)$ jest geometryczny.

Liczba $a$ jest równa

A. 3

B. 0

Zadqnie 17. $(0-1$\}

$\mathrm{W}$ ukladzie wspólrzednych zaznaczono $\mathrm{k}\mathrm{a}\mathrm{t} 0$

o wierzcholku w punkcie $0=(0,0)$. Jedno

z ramion tego kqta pokrywa $\mathrm{s}\mathrm{i}\mathrm{e}$ z dodatnia

póosiq $0x$, a drugie przechodzi przez punkt

$P=(-3,1)$ (zobacz rysunek).

Tangens kqta $\alpha$ jest równy

A. -$\sqrt{}$110

B. $(-\displaystyle \frac{3}{\sqrt{10}})$

C. 4

D. 2
\begin{center}
\includegraphics[width=78.288mm,height=48.768mm]{./F2_M_PP_M2023_page11_images/image001.eps}
\end{center}
{\it y}

$P=(-3,1)$

$\alpha$

{\it 0}

$-1$

1 2 3  $\chi$

C. $(-\displaystyle \frac{3}{1})$

D. $(-\displaystyle \frac{1}{3})$

Zädanie $l8. (0-1$\}

Dla $\mathrm{k}\mathrm{a}\dot{\mathrm{z}}$ dego kqta ostrego $\alpha$ wyrazenie $\sin^{4}\alpha+\sin^{2}\alpha\cdot\cos^{2}\alpha$ jest równe

A. $\sin^{2}\alpha$

B. $\sin^{6}\alpha\cdot\cos^{2}\alpha$

C. $\sin^{4}\alpha+1$

D. $\sin^{2}\alpha\cdot(\sin\alpha+\cos\alpha)\cdot(\sin\alpha-\cos\alpha)$

Strona 12 z30

$\mathrm{E}\mathrm{M}\mathrm{A}\mathrm{P}-\mathrm{P}0_{-}100$





: {\it RU DNOPIS} \{{\it nie podlega ocenie}\}

$-\mathrm{P}0_{-}100$

Strona 13z30





Zadanie 19. $\langle 0-1$\}

Punkty $A, B, C \mathrm{l}\mathrm{e}\dot{\mathrm{z}}\mathrm{q}$ na okregu o środku w punkcie 0.

Kqt $AC0$ ma miar9 $70^{\mathrm{o}}$ (zobacz rysunek).
\begin{center}
\includegraphics[width=68.172mm,height=72.384mm]{./F2_M_PP_M2023_page13_images/image001.eps}
\end{center}
{\it B}

{\it 0}

$70^{\mathrm{o}}$

{\it C}

{\it A}

Miara kata ostrego ABC jest równa

A. $10^{\mathrm{o}}$

B. $20^{\mathrm{o}}$

C. $35^{\mathrm{o}}$

D. $40^{\mathrm{o}}$

Zadanie 20. $\langle 0-1$\}

$\mathrm{W}$ rombie o boku dlugości $6\sqrt{2} \mathrm{k}\mathrm{a}\mathrm{t}$ rozwarty ma miare $150^{\mathrm{o}}$

lloczyn dlugości przekqtnych tego rombu jest równy

A. 24

B. 72

Zadanie 21. (0-1)

Przez punkty A $\mathrm{i} B, \mathrm{l}\mathrm{e}\dot{\mathrm{z}}$ ace na okregu

o środku 0, poprowadzono proste styczne

do tego okrpgu, przecinajace $\mathrm{s}\mathrm{i}\mathrm{e}$

w punkcie $C$ (zobacz rysunek).

Miara kata ACB jest równa

A. $20^{\mathrm{o}}$

B. $35^{\mathrm{o}}$

C. 36

D. $36\sqrt{2}$
\begin{center}
\includegraphics[width=95.604mm,height=62.736mm]{./F2_M_PP_M2023_page13_images/image002.eps}
\end{center}
{\it B}

$0140^{\mathrm{o}}$

{\it A  C}

C. $40^{\mathrm{o}}$

D. $70^{\mathrm{o}}$

Strona 14 z30

$\mathrm{E}\mathrm{M}\mathrm{A}\mathrm{P}-\mathrm{P}0_{-}100$





: {\it RU DNOPIS} \{{\it nie podlega ocenie}\}

$-\mathrm{P}0_{-}100$

Strona 15z30





Zadanie 22. $\{0-1\}$

Danyjest trójkqt $ABC$, w którym

$|BC|=6$. Miara kqta $ACB$ jest

równa $150^{\mathrm{o}}$ (zobacz rysunek).
\begin{center}
\includegraphics[width=97.536mm,height=39.372mm]{./F2_M_PP_M2023_page15_images/image001.eps}
\end{center}
{\it B}

6

$150^{\mathrm{o}}$

{\it C  A}

Wysokośč trójkata ABC opuszczona z wierzcholka B jest równa

A. 3

B. 4

C. $3\sqrt{3}$

D. $4\sqrt{3}$

Zadanie 23. $[0-1$\}

Dana jest prosta $k$ o równaniu $y=-\displaystyle \frac{1}{3}x+2.$

Prosta o równaniu $y=ax+b$ jest równolegla do prostej $k$ i przechodzi przez

punkt $P=(3,5)$, gdy

A. $a=3 \mathrm{i} b=4.$

B. $a=-\displaystyle \frac{1}{3} \mathrm{i} b=4.$

C. $a=3 \mathrm{i} b=-4.$

D. $a=-\displaystyle \frac{1}{3} \mathrm{i} b=6.$

Zadanie 24. $(0-1$\}

Dane sa punkty $K=(-3,-7)$ oraz $S=(5,3)$. Punkt $S$ jest środkiem odcinka $KL$. Wtedy

punkt $L$ ma wspólrz9dne

A. (13, 10)

B. (13, 13)

C. $(1,-2)$

D. $(7,-1)$

Zadanie 25. (0-1)

Dana jest prosta o równaniu $y=2x-3$. Obrazem tej prostej w symetrii środkowej

wzgledem poczqtku ukladu wspólrzędnych jest prosta o równaniu

A. $y=2x+3$

B. $y=-2x-3$

C. $y=-2x+3$

D. $y=2x-3$

Strona 16 z30

$\mathrm{E}\mathrm{M}\mathrm{A}\mathrm{P}-\mathrm{P}0_{-}100$





: {\it RU DNOPIS} \{{\it nie podlega ocenie}\}

$-\mathrm{P}0_{-}100$

Strona 17 z30





Zadarie $26_{d}(0-1$\}

Dany jest graniastoslup prawidlowy czworokqtny, w którym krawpd $\acute{\mathrm{z}}$ podstawy ma

dlugośč 15. Przekatna graniastos1upa jest nachy1ona do p1aszczyzny podstawy pod

kqtem $\alpha$ takim, $\dot{\mathrm{z}}\mathrm{e} \displaystyle \cos\alpha=\frac{\sqrt{2}}{3}$

Dlugośč przekqtnej tego graniastoslupa jest równa

A. $15\sqrt{2}$

B. 45

C. $5\sqrt{2}$

D. 10

ZadanIe 27. $\zeta 0-1$\}

$\acute{\mathrm{S}}$ rednia arytmetyczna liczb $x, y, z$ jest równa 4.

$\acute{\mathrm{S}}$ rednia arytmetyczna czterech liczb: $1+x, 2+y, 3+z$, 14, jest równa

A. 6

B. 9

C. 8

D. 13

Zadanie28. $\langle 0-1$\}

Wszystkich liczb naturalnych pieciocyfrowych, w których zapisie dziesietnym wystepujq tylko

cyfry 0, 5, 7 (np. 57075, 55555), jest

A. $5^{3}$

B. $2\cdot 4^{3}$

C. $2\cdot 3^{4}$

D. $3^{5}$

Zadanie 29. $(0-1$\}

$\mathrm{W}$ pewnym ostroslupie prawidlowym stosunek liczby $W$ wszystkich wierzcholków do

liczby $K$ wszystkich krawedzi jest równy $\displaystyle \frac{W}{K}=\frac{3}{5}.$

Podstawa tego ostroslupa jest

A. kwadrat.

B. piciokt foremny.

C. sześciokat foremny.

D. siedmiokat foremny.

Strona 18 z30

$\mathrm{E}\mathrm{M}\mathrm{A}\mathrm{P}-\mathrm{P}0_{-}100$





: {\it RU DNOPIS} \{{\it nie podlega ocenie}\}

$-\mathrm{P}0_{-}100$

Strona 19z30





Zadarie 30. (0-2)

Rozwiqz nierównośč

$x(x-2)>2x^{2}-3$

Strona 20 z30

$\mathrm{E}\mathrm{M}\mathrm{A}\mathrm{P}-\mathrm{P}0_{-}10$





Zadania egzaminacyine sq wydrukowane

na nastepnych stronach.

$\mathrm{E}\mathrm{M}\mathrm{A}\mathrm{P}-\mathrm{P}0_{-}100$

Strona 3 z30





Zadarie 31. (0-2)

Pan Stanislaw splacil $\mathrm{p}\mathrm{o}\dot{\mathrm{z}}$ yczkę w wysokości

rata byla mniejsza od poprzedniej o 30 z1.

Oblicz kwot9 pierwszej raty.

8910 zl w osiemnastu ratach. $\mathrm{K}\mathrm{a}\dot{\mathrm{z}}$ da kolejna
\begin{center}
\begin{tabular}{|l|l|l|l|}
\cline{2-4}
&	\multicolumn{1}{|l|}{Nr zadania}&	\multicolumn{1}{|l|}{$30.$}&	\multicolumn{1}{|l|}{ $31.$}	\\
\cline{2-4}
&	\multicolumn{1}{|l|}{Maks. liczba pkt}&	\multicolumn{1}{|l|}{$2$}&	\multicolumn{1}{|l|}{ $2$}	\\
\cline{2-4}
\multicolumn{1}{|l|}{egzaminator}&	\multicolumn{1}{|l|}{Uzyskana liczba pkt}&	\multicolumn{1}{|l|}{}&	\multicolumn{1}{|l|}{}	\\
\hline
\end{tabular}

\end{center}
$\mathrm{E}\mathrm{M}\mathrm{A}\mathrm{P}-\mathrm{P}0_{-}100$

Strona 21 z30





Zadanie 32. (0-2)

Wykaz, $\dot{\mathrm{z}}\mathrm{e}$ dla $\mathrm{k}\mathrm{a}\dot{\mathrm{z}}$ dej liczby rzeczywistej $x\neq 1$ i dla $\mathrm{k}\mathrm{a}\dot{\mathrm{z}}$ dej liczby rzeczywistej

prawdziwa jest nierównośč

$x^{2}+y^{2}+5>2x+4y$

{\it y}

Strona 22 z30

$\mathrm{E}\mathrm{M}\mathrm{A}\mathrm{P}-\mathrm{P}0_{-}100$





Zadanie 33. (0-2)

Trójkqty prostokatne $T_{1}$ i $T_{2}$ sq podobne. Przyprostokqtne trójkata

5 $\mathrm{i} 12$. Przeciwprostokqtna trójkqta $T_{2}$ ma dlugośč 26.

Oblicz pole trójkqta $T_{2}.$

$T_{1}$ maja dlugości
\begin{center}
\begin{tabular}{|l|l|l|l|}
\cline{2-4}
&	\multicolumn{1}{|l|}{Nr zadania}&	\multicolumn{1}{|l|}{$32.$}&	\multicolumn{1}{|l|}{ $33.$}	\\
\cline{2-4}
&	\multicolumn{1}{|l|}{Maks. liczba pkt}&	\multicolumn{1}{|l|}{$2$}&	\multicolumn{1}{|l|}{ $2$}	\\
\cline{2-4}
\multicolumn{1}{|l|}{egzaminator}&	\multicolumn{1}{|l|}{Uzyskana liczba pkt}&	\multicolumn{1}{|l|}{}&	\multicolumn{1}{|l|}{}	\\
\hline
\end{tabular}

\end{center}
$\mathrm{E}\mathrm{M}\mathrm{A}\mathrm{P}-\mathrm{P}0_{-}100$

Strona 23 z30





Zadarie 34. (0-2)

$\mathrm{W}$ kwadracie ABCD punkty $A=(-8,-2)$ oraz $C=(0,4)$ sa końcami przekqtnej.

Wyznacz równanie prostej zawierajacej przekqtnq $BD$ tego kwadratu.

Strona 24 z30

$\mathrm{E}\mathrm{M}\mathrm{A}\mathrm{P}-\mathrm{P}0_{-}100$





Zadarie 35. (0-2)

Ze zbioru ośmiu liczb \{2, 3, 4, 5, 6, 7, 8, 9\} 1osujemy ze zwracaniem ko1ejno dwa razy po

jednej liczbie.

Oblicz prawdopodobieństwo zdarzenia $A$ polegajqcego na tym, $\dot{\mathrm{z}}\mathrm{e}$ iloczyn wylosowanych

liczb jest podzielny przez 15.
\begin{center}
\begin{tabular}{|l|l|l|l|}
\cline{2-4}
&	\multicolumn{1}{|l|}{Nr zadania}&	\multicolumn{1}{|l|}{$34.$}&	\multicolumn{1}{|l|}{ $35.$}	\\
\cline{2-4}
&	\multicolumn{1}{|l|}{Maks. liczba pkt}&	\multicolumn{1}{|l|}{$2$}&	\multicolumn{1}{|l|}{ $2$}	\\
\cline{2-4}
\multicolumn{1}{|l|}{egzaminator}&	\multicolumn{1}{|l|}{Uzyskana liczba pkt}&	\multicolumn{1}{|l|}{}&	\multicolumn{1}{|l|}{}	\\
\hline
\end{tabular}

\end{center}
$\mathrm{E}\mathrm{M}\mathrm{A}\mathrm{P}-\mathrm{P}0_{-}100$

Strona 25 z30





Zadarie 36. $\langle 0-5$)

Podstawq graniastoslupa prostego ABCDEF jest trójkqt

równoramienny $ABC$, w którym $|AC|=|BC|, |AB|=8.$

Wysokośč trójkata $ABC$, poprowadzona z wierzcholka $C,$

ma dlugośč 3. Przekqtna CE ściany bocznej tworzy

z krawpdziq $CB$ podstawy $ABC \triangleright_{\iota}\mathrm{q}\mathrm{t} 60^{\mathrm{o}}$ (zobacz

rysunek).

Oblicz pole powierzchni calkowitej oraz objetośč tego graniastoslupa.

Strona 26 z30

$\mathrm{E}\mathrm{M}\mathrm{A}\mathrm{P}-\mathrm{P}0_{-}100$





Wypelnia

egzaminator

Nr zadania

Maks. liczba pkt

Uzyskana liczba pkt

36.

5

-PO-100

Strona 27 z30





: {\it RU DNOPIS} \{{\it nie podlega ocenie}\}

Strona 28z 30

$\mathrm{E}\mathrm{M}\mathrm{A}\mathrm{P}-\mathrm{P}0_{-}10$





$0_{-}100$

Strona 29 z30





Strona 30 z30

$\mathrm{E}\mathrm{M}\mathrm{A}\mathrm{P}-\mathrm{P}0_{-}10$





{\it Wkazdym z zadań od} $f.$ {\it do 29. wybierz izaznacz na karcie odpowiedzi poprawna} $od\sqrt{}owi\mathrm{e}d\acute{z}.$

Zadanie $\mathrm{f}. (0-1$\}

Liczba $\log_{9}27+\log_{9}3$ jest równa

A. 81

B. 9

C. 4

D. 2

Zadan$\mathrm{e}2. (0-1$\}

Liczba $\sqrt[3]{-\frac{27}{16}}\cdot\sqrt[3]{2}$ jest równa

A. $(-\displaystyle \frac{3}{2})$

B. -23

C. -32

D. $(-\displaystyle \frac{2}{3})$

Zadanie $3_{r}(0-4)$

Cene aparatu fotograficznego obnizono o 15\%, a nastepnie-o 20\% w odniesieniu do

ceny obowiqzujacej w danym momencie. Po tych dwóch obnizkach aparat kosztuje 340 z1.

Przed obiema obnizkami cena tego aparatu byla równa

A. 500 z1

B. 425 z1

C. 400 z1

D. 375 z1

Zadanie 4. $(0-1\rangle$

Dla $\mathrm{k}\mathrm{a}\dot{\mathrm{z}}$ dej liczby rzeczywistej $a$ wyrazenie $(2a-3)^{2}-(2a+3)^{2}$ jest równe

A. $-24a$

B. 0

C. 18

D. $16a^{2}-24a$

Strona 4 z30

$\mathrm{E}\mathrm{M}\mathrm{A}\mathrm{P}-\mathrm{P}0_{-}100$















: {\it RU DNOPIS} \{{\it nie podlega ocenie}\}

$-\mathrm{P}0_{-}100$

Strona 5z30





Zadanie 5. $(0-1$\}

Na rysunku przedstawiono interpretacj9 $\displaystyle \mathrm{g}\mathrm{e}\mathrm{o}\mathrm{m}\mathrm{e}\mathrm{t}\mathrm{r}\mathrm{y}\mathrm{c}\mathrm{z}\bigcap_{\mathrm{c}1}$ jednego z $\mathrm{n}\mathrm{i}\dot{\mathrm{z}}$ ej zapisanych ukladów

równań.

Wskaz ten uklad równań, którego interpretacje geometryczna przedstawiono na rysunku.

A. 

B. 

C. 

D. 

Zädanie 6. $\{0-1\}$

Zbiorem wszystkich rozwiqzań nierówności

$-2(x+3)\displaystyle \leq\frac{2-x}{3}$

jest przedzial

A. $(-\infty, -4\rangle$

B. $(-\infty,  4\rangle$

C. $\langle-4, \infty)$

D. $\langle$4, $\infty)$

Zadanie 7. (0-1)

Jednym z rozwiqzań równania $\sqrt{3}(x^{2}-2)(x+3)=0$ jest liczba

A. 3

B. 2

C. $\sqrt{3}$

D. $\sqrt{2}$

Strona 6 z30

$\mathrm{E}\mathrm{M}\mathrm{A}\mathrm{P}-\mathrm{P}0_{-}100$





: {\it RU DNOPIS} \{{\it nie podlega ocenie}\}

$-\mathrm{P}0_{-}100$

Strona 7 z30





Zadanie @. $(0-\mathrm{t}\rangle$

Równanie $\displaystyle \frac{(x+1)(x-1)^{2}}{(x-1)(x+1)^{2}}=0$ w zbiorze liczb rzeczywistych

A. nie ma rozwiqzania.

B. ma dokladnie jedno rozwiqzanie: $-1.$

C. ma dokladnie jedno rozwiqzanie: l.

D. ma dokladnie dwa rozwiqzania: $-1$ oraz l.

Zadanie 9. (0-4)

Miejscem zerowym funkcji liniowej $f(x)=(2p-1)x+p$ jest liczba $(-4)$. Wtedy

A. {\it p}$=$ -49

B. {\it p} $=$ -47

C. $p=-4$

D. {\it p}$=$ - -47

Zadanie 10. $\langle 0-1$\}

Funkcja liniowa $f$ jest określona wzorem

$f(x)=ax+b$, gdzie $a \mathrm{i} b$ sa pewnymi

liczbami rzeczywistymi. Na rysunku obok

przedstawiono fragment wykresu funkcji $f$

w ukladzie wspólrzednych $(x,y).$
\begin{center}
\includegraphics[width=82.956mm,height=69.540mm]{./F2_M_PP_M2023_page7_images/image001.eps}
\end{center}
{\it y}

1

0 1  $\chi$

$y=f(x)$

Liczba $a$ oraz liczba $b$ we wzorze funkcji $f \mathrm{s}\mathrm{p}\mathrm{e}$niajq warunki:

A. $a>0 \mathrm{i} b>0.$

B. $a>0 \mathrm{i} b<0.$

C. $a<0 \mathrm{i} b>0.$

D. $a<0 \mathrm{i} b<0.$

Strona 8 z30

$\mathrm{E}\mathrm{M}\mathrm{A}\mathrm{P}-\mathrm{P}0_{-}100$





: {\it RU DNOPIS} \{{\it nie podlega ocenie}\}

$-\mathrm{P}0_{-}100$

Strona 9z30





lnformacja do zadań ll.$-13.$

$\mathrm{W}$ ukladzie wspólrzednych $(x,y)$

narysowano wykres funkcji $y=f(x)$

(zobacz sunek).

Zadanie ll. $\langle 0-1$\}

Dziedzinq funkcji $f$ jest zbiór

A. $\langle-6,  5\rangle$

B. $(-6,5)$

Zadanie 12. $\langle 0-1$)

Funkcja $f$ jest malejqca w zbiorze

A. $\langle-6, -3)$

B. $\langle-3,1\rangle$
\begin{center}
\includegraphics[width=100.680mm,height=83.820mm]{./F2_M_PP_M2023_page9_images/image001.eps}
\end{center}
{\it y}

0  $\chi$

C. $(-3,5\rangle$

D. $\langle-3,  5\rangle$

C. (l, $ 2\rangle$

D. $\langle$2, $ 5\rangle$

Zadanie 43. (0-1)

$\mathrm{N}\mathrm{a}\mathrm{j}\mathrm{w}\mathrm{i}_{9}$ksza wartośč funkcji $f$ w przedziale $\langle-4,  1\rangle$ jest równa

A. 0

B. l

C. 2

D. 5

Zädanie $l4. (0-1)$

Jednym z miejsc zerowych funkcji kwadratowej $f$ jest liczba $(-5)$. Pierwsza wspólrzedna

wierzcholka paraboli, $\mathrm{b}_{9}$dqcej wykresem funkcji $f$, jest równa 3.

Drugim miejscem zerowym funkcji $f$ jest liczba

A. ll

B. l

C. $(-1)$

D. $(-13)$

Strona 10 z30

$\mathrm{E}\mathrm{M}\mathrm{A}\mathrm{P}-\mathrm{P}0_{-}100$



\end{document}