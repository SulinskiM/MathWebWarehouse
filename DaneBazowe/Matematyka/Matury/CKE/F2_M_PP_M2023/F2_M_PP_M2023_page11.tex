\documentclass[a4paper,12pt]{article}
\usepackage{latexsym}
\usepackage{amsmath}
\usepackage{amssymb}
\usepackage{graphicx}
\usepackage{wrapfig}
\pagestyle{plain}
\usepackage{fancybox}
\usepackage{bm}

\begin{document}

Zadanie 15. $\langle 0-1$\}

Ciqg $(a_{n})$ jest określony wzorem $a_{n}=2^{n}$

Wyraz $a_{4}$ jest równy

$(n+1)$ dla $\mathrm{k}\mathrm{a}\dot{\mathrm{z}}$ dej liczby naturalnej $n\geq 1.$

A. 64

B. 40

C. 48

D. 80

Zadanie 16. $(0-1\mathrm{J}$

Trzywyrazowy ciag $($27, 9, $a-1)$ jest geometryczny.

Liczba $a$ jest równa

A. 3

B. 0

Zadqnie 17. $(0-1$\}

$\mathrm{W}$ ukladzie wspólrzednych zaznaczono $\mathrm{k}\mathrm{a}\mathrm{t} 0$

o wierzcholku w punkcie $0=(0,0)$. Jedno

z ramion tego kqta pokrywa $\mathrm{s}\mathrm{i}\mathrm{e}$ z dodatnia

póosiq $0x$, a drugie przechodzi przez punkt

$P=(-3,1)$ (zobacz rysunek).

Tangens kqta $\alpha$ jest równy

A. -$\sqrt{}$110

B. $(-\displaystyle \frac{3}{\sqrt{10}})$

C. 4

D. 2
\begin{center}
\includegraphics[width=78.288mm,height=48.768mm]{./F2_M_PP_M2023_page11_images/image001.eps}
\end{center}
{\it y}

$P=(-3,1)$

$\alpha$

{\it 0}

$-1$

1 2 3  $\chi$

C. $(-\displaystyle \frac{3}{1})$

D. $(-\displaystyle \frac{1}{3})$

Zädanie $l8. (0-1$\}

Dla $\mathrm{k}\mathrm{a}\dot{\mathrm{z}}$ dego kqta ostrego $\alpha$ wyrazenie $\sin^{4}\alpha+\sin^{2}\alpha\cdot\cos^{2}\alpha$ jest równe

A. $\sin^{2}\alpha$

B. $\sin^{6}\alpha\cdot\cos^{2}\alpha$

C. $\sin^{4}\alpha+1$

D. $\sin^{2}\alpha\cdot(\sin\alpha+\cos\alpha)\cdot(\sin\alpha-\cos\alpha)$

Strona 12 z30

$\mathrm{E}\mathrm{M}\mathrm{A}\mathrm{P}-\mathrm{P}0_{-}100$
\end{document}
