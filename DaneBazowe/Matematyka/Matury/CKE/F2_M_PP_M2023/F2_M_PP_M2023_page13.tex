\documentclass[a4paper,12pt]{article}
\usepackage{latexsym}
\usepackage{amsmath}
\usepackage{amssymb}
\usepackage{graphicx}
\usepackage{wrapfig}
\pagestyle{plain}
\usepackage{fancybox}
\usepackage{bm}

\begin{document}

Zadanie 19. $\langle 0-1$\}

Punkty $A, B, C \mathrm{l}\mathrm{e}\dot{\mathrm{z}}\mathrm{q}$ na okregu o środku w punkcie 0.

Kqt $AC0$ ma miar9 $70^{\mathrm{o}}$ (zobacz rysunek).
\begin{center}
\includegraphics[width=68.172mm,height=72.384mm]{./F2_M_PP_M2023_page13_images/image001.eps}
\end{center}
{\it B}

{\it 0}

$70^{\mathrm{o}}$

{\it C}

{\it A}

Miara kata ostrego ABC jest równa

A. $10^{\mathrm{o}}$

B. $20^{\mathrm{o}}$

C. $35^{\mathrm{o}}$

D. $40^{\mathrm{o}}$

Zadanie 20. $\langle 0-1$\}

$\mathrm{W}$ rombie o boku dlugości $6\sqrt{2} \mathrm{k}\mathrm{a}\mathrm{t}$ rozwarty ma miare $150^{\mathrm{o}}$

lloczyn dlugości przekqtnych tego rombu jest równy

A. 24

B. 72

Zadanie 21. (0-1)

Przez punkty A $\mathrm{i} B, \mathrm{l}\mathrm{e}\dot{\mathrm{z}}$ ace na okregu

o środku 0, poprowadzono proste styczne

do tego okrpgu, przecinajace $\mathrm{s}\mathrm{i}\mathrm{e}$

w punkcie $C$ (zobacz rysunek).

Miara kata ACB jest równa

A. $20^{\mathrm{o}}$

B. $35^{\mathrm{o}}$

C. 36

D. $36\sqrt{2}$
\begin{center}
\includegraphics[width=95.604mm,height=62.736mm]{./F2_M_PP_M2023_page13_images/image002.eps}
\end{center}
{\it B}

$0140^{\mathrm{o}}$

{\it A  C}

C. $40^{\mathrm{o}}$

D. $70^{\mathrm{o}}$

Strona 14 z30

$\mathrm{E}\mathrm{M}\mathrm{A}\mathrm{P}-\mathrm{P}0_{-}100$
\end{document}
