\documentclass[a4paper,12pt]{article}
\usepackage{latexsym}
\usepackage{amsmath}
\usepackage{amssymb}
\usepackage{graphicx}
\usepackage{wrapfig}
\pagestyle{plain}
\usepackage{fancybox}
\usepackage{bm}

\begin{document}

Zadanie 22. $\{0-1\}$

Danyjest trójkqt $ABC$, w którym

$|BC|=6$. Miara kqta $ACB$ jest

równa $150^{\mathrm{o}}$ (zobacz rysunek).
\begin{center}
\includegraphics[width=97.536mm,height=39.372mm]{./F2_M_PP_M2023_page15_images/image001.eps}
\end{center}
{\it B}

6

$150^{\mathrm{o}}$

{\it C  A}

Wysokośč trójkata ABC opuszczona z wierzcholka B jest równa

A. 3

B. 4

C. $3\sqrt{3}$

D. $4\sqrt{3}$

Zadanie 23. $[0-1$\}

Dana jest prosta $k$ o równaniu $y=-\displaystyle \frac{1}{3}x+2.$

Prosta o równaniu $y=ax+b$ jest równolegla do prostej $k$ i przechodzi przez

punkt $P=(3,5)$, gdy

A. $a=3 \mathrm{i} b=4.$

B. $a=-\displaystyle \frac{1}{3} \mathrm{i} b=4.$

C. $a=3 \mathrm{i} b=-4.$

D. $a=-\displaystyle \frac{1}{3} \mathrm{i} b=6.$

Zadanie 24. $(0-1$\}

Dane sa punkty $K=(-3,-7)$ oraz $S=(5,3)$. Punkt $S$ jest środkiem odcinka $KL$. Wtedy

punkt $L$ ma wspólrz9dne

A. (13, 10)

B. (13, 13)

C. $(1,-2)$

D. $(7,-1)$

Zadanie 25. (0-1)

Dana jest prosta o równaniu $y=2x-3$. Obrazem tej prostej w symetrii środkowej

wzgledem poczqtku ukladu wspólrzędnych jest prosta o równaniu

A. $y=2x+3$

B. $y=-2x-3$

C. $y=-2x+3$

D. $y=2x-3$

Strona 16 z30

$\mathrm{E}\mathrm{M}\mathrm{A}\mathrm{P}-\mathrm{P}0_{-}100$
\end{document}
