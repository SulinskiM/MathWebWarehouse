\documentclass[a4paper,12pt]{article}
\usepackage{latexsym}
\usepackage{amsmath}
\usepackage{amssymb}
\usepackage{graphicx}
\usepackage{wrapfig}
\pagestyle{plain}
\usepackage{fancybox}
\usepackage{bm}

\begin{document}

Zadarie $26_{d}(0-1$\}

Dany jest graniastoslup prawidlowy czworokqtny, w którym krawpd $\acute{\mathrm{z}}$ podstawy ma

dlugośč 15. Przekatna graniastos1upa jest nachy1ona do p1aszczyzny podstawy pod

kqtem $\alpha$ takim, $\dot{\mathrm{z}}\mathrm{e} \displaystyle \cos\alpha=\frac{\sqrt{2}}{3}$

Dlugośč przekqtnej tego graniastoslupa jest równa

A. $15\sqrt{2}$

B. 45

C. $5\sqrt{2}$

D. 10

ZadanIe 27. $\zeta 0-1$\}

$\acute{\mathrm{S}}$ rednia arytmetyczna liczb $x, y, z$ jest równa 4.

$\acute{\mathrm{S}}$ rednia arytmetyczna czterech liczb: $1+x, 2+y, 3+z$, 14, jest równa

A. 6

B. 9

C. 8

D. 13

Zadanie28. $\langle 0-1$\}

Wszystkich liczb naturalnych pieciocyfrowych, w których zapisie dziesietnym wystepujq tylko

cyfry 0, 5, 7 (np. 57075, 55555), jest

A. $5^{3}$

B. $2\cdot 4^{3}$

C. $2\cdot 3^{4}$

D. $3^{5}$

Zadanie 29. $(0-1$\}

$\mathrm{W}$ pewnym ostroslupie prawidlowym stosunek liczby $W$ wszystkich wierzcholków do

liczby $K$ wszystkich krawedzi jest równy $\displaystyle \frac{W}{K}=\frac{3}{5}.$

Podstawa tego ostroslupa jest

A. kwadrat.

B. piciokt foremny.

C. sześciokat foremny.

D. siedmiokat foremny.

Strona 18 z30

$\mathrm{E}\mathrm{M}\mathrm{A}\mathrm{P}-\mathrm{P}0_{-}100$
\end{document}
