\documentclass[a4paper,12pt]{article}
\usepackage{latexsym}
\usepackage{amsmath}
\usepackage{amssymb}
\usepackage{graphicx}
\usepackage{wrapfig}
\pagestyle{plain}
\usepackage{fancybox}
\usepackage{bm}

\begin{document}

Zadarie 36. $\langle 0-5$)

Podstawq graniastoslupa prostego ABCDEF jest trójkqt

równoramienny $ABC$, w którym $|AC|=|BC|, |AB|=8.$

Wysokośč trójkata $ABC$, poprowadzona z wierzcholka $C,$

ma dlugośč 3. Przekqtna CE ściany bocznej tworzy

z krawpdziq $CB$ podstawy $ABC \triangleright_{\iota}\mathrm{q}\mathrm{t} 60^{\mathrm{o}}$ (zobacz

rysunek).

Oblicz pole powierzchni calkowitej oraz objetośč tego graniastoslupa.

Strona 26 z30

$\mathrm{E}\mathrm{M}\mathrm{A}\mathrm{P}-\mathrm{P}0_{-}100$
\end{document}
