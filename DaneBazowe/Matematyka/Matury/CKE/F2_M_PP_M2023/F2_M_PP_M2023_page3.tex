\documentclass[a4paper,12pt]{article}
\usepackage{latexsym}
\usepackage{amsmath}
\usepackage{amssymb}
\usepackage{graphicx}
\usepackage{wrapfig}
\pagestyle{plain}
\usepackage{fancybox}
\usepackage{bm}

\begin{document}

{\it Wkazdym z zadań od} $f.$ {\it do 29. wybierz izaznacz na karcie odpowiedzi poprawna} $od\sqrt{}owi\mathrm{e}d\acute{z}.$

Zadanie $\mathrm{f}. (0-1$\}

Liczba $\log_{9}27+\log_{9}3$ jest równa

A. 81

B. 9

C. 4

D. 2

Zadan$\mathrm{e}2. (0-1$\}

Liczba $\sqrt[3]{-\frac{27}{16}}\cdot\sqrt[3]{2}$ jest równa

A. $(-\displaystyle \frac{3}{2})$

B. -23

C. -32

D. $(-\displaystyle \frac{2}{3})$

Zadanie $3_{r}(0-4)$

Cene aparatu fotograficznego obnizono o 15\%, a nastepnie-o 20\% w odniesieniu do

ceny obowiqzujacej w danym momencie. Po tych dwóch obnizkach aparat kosztuje 340 z1.

Przed obiema obnizkami cena tego aparatu byla równa

A. 500 z1

B. 425 z1

C. 400 z1

D. 375 z1

Zadanie 4. $(0-1\rangle$

Dla $\mathrm{k}\mathrm{a}\dot{\mathrm{z}}$ dej liczby rzeczywistej $a$ wyrazenie $(2a-3)^{2}-(2a+3)^{2}$ jest równe

A. $-24a$

B. 0

C. 18

D. $16a^{2}-24a$

Strona 4 z30

$\mathrm{E}\mathrm{M}\mathrm{A}\mathrm{P}-\mathrm{P}0_{-}100$
\end{document}
