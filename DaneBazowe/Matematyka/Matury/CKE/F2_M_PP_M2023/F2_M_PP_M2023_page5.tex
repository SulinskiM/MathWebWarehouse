\documentclass[a4paper,12pt]{article}
\usepackage{latexsym}
\usepackage{amsmath}
\usepackage{amssymb}
\usepackage{graphicx}
\usepackage{wrapfig}
\pagestyle{plain}
\usepackage{fancybox}
\usepackage{bm}

\begin{document}

Zadanie 5. $(0-1$\}

Na rysunku przedstawiono interpretacj9 $\displaystyle \mathrm{g}\mathrm{e}\mathrm{o}\mathrm{m}\mathrm{e}\mathrm{t}\mathrm{r}\mathrm{y}\mathrm{c}\mathrm{z}\bigcap_{\mathrm{c}1}$ jednego z $\mathrm{n}\mathrm{i}\dot{\mathrm{z}}$ ej zapisanych ukladów

równań.

Wskaz ten uklad równań, którego interpretacje geometryczna przedstawiono na rysunku.

A. 

B. 

C. 

D. 

Zädanie 6. $\{0-1\}$

Zbiorem wszystkich rozwiqzań nierówności

$-2(x+3)\displaystyle \leq\frac{2-x}{3}$

jest przedzial

A. $(-\infty, -4\rangle$

B. $(-\infty,  4\rangle$

C. $\langle-4, \infty)$

D. $\langle$4, $\infty)$

Zadanie 7. (0-1)

Jednym z rozwiqzań równania $\sqrt{3}(x^{2}-2)(x+3)=0$ jest liczba

A. 3

B. 2

C. $\sqrt{3}$

D. $\sqrt{2}$

Strona 6 z30

$\mathrm{E}\mathrm{M}\mathrm{A}\mathrm{P}-\mathrm{P}0_{-}100$
\end{document}
