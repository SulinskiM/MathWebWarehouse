\documentclass[a4paper,12pt]{article}
\usepackage{latexsym}
\usepackage{amsmath}
\usepackage{amssymb}
\usepackage{graphicx}
\usepackage{wrapfig}
\pagestyle{plain}
\usepackage{fancybox}
\usepackage{bm}

\begin{document}

Zadanie @. $(0-\mathrm{t}\rangle$

Równanie $\displaystyle \frac{(x+1)(x-1)^{2}}{(x-1)(x+1)^{2}}=0$ w zbiorze liczb rzeczywistych

A. nie ma rozwiqzania.

B. ma dokladnie jedno rozwiqzanie: $-1.$

C. ma dokladnie jedno rozwiqzanie: l.

D. ma dokladnie dwa rozwiqzania: $-1$ oraz l.

Zadanie 9. (0-4)

Miejscem zerowym funkcji liniowej $f(x)=(2p-1)x+p$ jest liczba $(-4)$. Wtedy

A. {\it p}$=$ -49

B. {\it p} $=$ -47

C. $p=-4$

D. {\it p}$=$ - -47

Zadanie 10. $\langle 0-1$\}

Funkcja liniowa $f$ jest określona wzorem

$f(x)=ax+b$, gdzie $a \mathrm{i} b$ sa pewnymi

liczbami rzeczywistymi. Na rysunku obok

przedstawiono fragment wykresu funkcji $f$

w ukladzie wspólrzednych $(x,y).$
\begin{center}
\includegraphics[width=82.956mm,height=69.540mm]{./F2_M_PP_M2023_page7_images/image001.eps}
\end{center}
{\it y}

1

0 1  $\chi$

$y=f(x)$

Liczba $a$ oraz liczba $b$ we wzorze funkcji $f \mathrm{s}\mathrm{p}\mathrm{e}$niajq warunki:

A. $a>0 \mathrm{i} b>0.$

B. $a>0 \mathrm{i} b<0.$

C. $a<0 \mathrm{i} b>0.$

D. $a<0 \mathrm{i} b<0.$

Strona 8 z30

$\mathrm{E}\mathrm{M}\mathrm{A}\mathrm{P}-\mathrm{P}0_{-}100$
\end{document}
