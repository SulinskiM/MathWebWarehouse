\documentclass[a4paper,12pt]{article}
\usepackage{latexsym}
\usepackage{amsmath}
\usepackage{amssymb}
\usepackage{graphicx}
\usepackage{wrapfig}
\pagestyle{plain}
\usepackage{fancybox}
\usepackage{bm}

\begin{document}

lnformacja do zadań ll.$-13.$

$\mathrm{W}$ ukladzie wspólrzednych $(x,y)$

narysowano wykres funkcji $y=f(x)$

(zobacz sunek).

Zadanie ll. $\langle 0-1$\}

Dziedzinq funkcji $f$ jest zbiór

A. $\langle-6,  5\rangle$

B. $(-6,5)$

Zadanie 12. $\langle 0-1$)

Funkcja $f$ jest malejqca w zbiorze

A. $\langle-6, -3)$

B. $\langle-3,1\rangle$
\begin{center}
\includegraphics[width=100.680mm,height=83.820mm]{./F2_M_PP_M2023_page9_images/image001.eps}
\end{center}
{\it y}

0  $\chi$

C. $(-3,5\rangle$

D. $\langle-3,  5\rangle$

C. (l, $ 2\rangle$

D. $\langle$2, $ 5\rangle$

Zadanie 43. (0-1)

$\mathrm{N}\mathrm{a}\mathrm{j}\mathrm{w}\mathrm{i}_{9}$ksza wartośč funkcji $f$ w przedziale $\langle-4,  1\rangle$ jest równa

A. 0

B. l

C. 2

D. 5

Zädanie $l4. (0-1)$

Jednym z miejsc zerowych funkcji kwadratowej $f$ jest liczba $(-5)$. Pierwsza wspólrzedna

wierzcholka paraboli, $\mathrm{b}_{9}$dqcej wykresem funkcji $f$, jest równa 3.

Drugim miejscem zerowym funkcji $f$ jest liczba

A. ll

B. l

C. $(-1)$

D. $(-13)$

Strona 10 z30

$\mathrm{E}\mathrm{M}\mathrm{A}\mathrm{P}-\mathrm{P}0_{-}100$
\end{document}
