\documentclass[a4paper,12pt]{article}
\usepackage{latexsym}
\usepackage{amsmath}
\usepackage{amssymb}
\usepackage{graphicx}
\usepackage{wrapfig}
\pagestyle{plain}
\usepackage{fancybox}
\usepackage{bm}

\begin{document}

CENTRALNA

KOMISJA

EGZAMINACYJNA

Arkusz zawiera informacje prawnie chronione

do momentu rozpoczecia egzaminu.

KOD

WYPELNIA ZDAJACY

PESEL

{\it Miejsce na naklejke}.

{\it Sprawdz}', {\it czy kod na naklejce to}

e-100.
\begin{center}
\includegraphics[width=21.900mm,height=10.164mm]{./F2_M_PP_M2024_page0_images/image001.eps}

\includegraphics[width=79.656mm,height=10.164mm]{./F2_M_PP_M2024_page0_images/image002.eps}
\end{center}
/{\it ezeli tak}- {\it przyklej naklejkq}.

/{\it ezeli nie}- {\it zgtoś to nauczycielowi}.

Egzamin maturalny

DATA: 8 maja 2024 r.

GODZINA R0ZP0CZECIA: 9:00

CZAS TRWANIA: $170 \displaystyle \min$ ut

MAP-P0-100-2405

WYPELNIA ZESPÓt NADZORUJACY

Uprawnienia zdajqcego do:

\fbox{} dostosowania zasad oceniania

\fbox{} dostosowania w zw. z dyskalkuliq

\fbox{} nieprzenoszenia odpowiedzi na karte.

LICZBA PUNKTÓW DO UZYSKANIA 46

Przed rozpoczeciem pracy z arkuszem egzaminacyjnym

1.

Sprawd $\acute{\mathrm{z}}$, czy nauczyciel przekazal Ci wlaściwy arkusz egzaminacyjny,

tj. arkusz we wlaściwej formule, z w[aściwego przedmiotu na wlaściwym

poziomie.

2.

$\mathrm{J}\mathrm{e}\dot{\mathrm{z}}$ eli przekazano Ci niew[aściwy arkusz- natychmiast zgloś to nauczycielowi.

Nie rozrywaj banderol.

3.

$\mathrm{J}\mathrm{e}\dot{\mathrm{z}}$ eli przekazano Ci w[aściwy arkusz- rozerwij banderole po otrzymaniu

takiego polecenia od nauczyciela. Zapoznaj $\mathrm{s}\mathrm{i}\mathrm{e}$ z instrukcjq na stronie 2.

Uk\}ad graficzny

\copyright CKE 2022

$\Vert\Vert\Vert\Vert\Vert\Vert\Vert\Vert\Vert\Vert\Vert\Vert\Vert\Vert\Vert\Vert\Vert\Vert\Vert\Vert\Vert\Vert\Vert\Vert\Vert\Vert\Vert\Vert\Vert\Vert|$




lnstrukcja dla zdajqcego

l. Sprawdz', czy arkusz egzaminacyjny zawiera 31 stron (zadania $1-36$).

Ewentualny brak zg\}oś przewodniczqcemu zespolu nadzorujqcego egzamin.

2. Na pierwszej stronie arkusza oraz na karcie odpowiedzi wpisz swój numer PESEL

i przyklej naklejke z kodem.

3. Odpowiedzi do zadań zamknietych $(1-29)$ zaznacz na karcie odpowiedzi w cz9ści karty

przeznaczonej dla zdajacego. Zamaluj $\blacksquare$ pola do tego przeznaczone. $\mathrm{B}_{9}\mathrm{d}\mathrm{n}\mathrm{e}$

zaznaczenie otocz kólkiem \copyright i zaznacz wlaściwe.

4. Pamiptaj, $\dot{\mathrm{z}}\mathrm{e}$ pominiecie argumentacji lub istotnych obliczeń w rozwiqzaniu zadania

otwartego (30-36) $\mathrm{m}\mathrm{o}\dot{\mathrm{z}}\mathrm{e}$ spowodowač, $\dot{\mathrm{z}}\mathrm{e}$ za to rozwiazanie nie otrzymasz pelnej liczby

punktów.

5. Rozwiqzania zadań i odpowiedzi wpisuj w miejscu na to przeznaczonym.

6. Pisz czytelnie i $\mathrm{u}\dot{\mathrm{z}}$ ywaj tylko dlugopisu lub pióra z czarnym tuszem lub atramentem.

7. Nie $\mathrm{u}\dot{\mathrm{z}}$ ywaj korektora, a bledne zapisy wyra $\acute{\mathrm{z}}\mathrm{n}\mathrm{i}\mathrm{e}$ przekreśl.

8. Nie wpisuj $\dot{\mathrm{z}}$ adnych znaków w cześci przeznaczonej dla egzaminatora.

9. $\mathrm{P}\mathrm{a}\mathrm{m}\mathrm{i}_{9}\mathrm{t}\mathrm{a}\mathrm{j}, \dot{\mathrm{z}}\mathrm{e}$ zapisy w brudnopisie nie bedq oceniane.

10. $\mathrm{M}\mathrm{o}\dot{\mathrm{z}}$ esz korzystač z Wybranych wzorów matematycznych, cyrkla i linijki oraz kalkulatora

prostego. Upewnij $\mathrm{s}\mathrm{i}\mathrm{e}$, czy przekazano Ci broszure z okladka taka jak widoczna ponizej.

Wybrane wzory

matematyczne

Strona 2 z31

$\mathrm{E}\mathrm{M}\mathrm{A}\mathrm{P}-\mathrm{P}0_{-}100$





: {\it RU DNOPIS} \{{\it nie podlega ocenie}\}

$\mathrm{h}\mathrm{P}-\mathrm{P}0_{-}100$

Strona llz31





lnformacja do zadań 14.$-15.$

Na rysunku przedstawiono fragment paraboli, która jest wykresem funkcji kwadratowej $f$

(zobacz rysunek). Wierzcholek tej paraboli oraz punkty przeciecia paraboli z osiami ukladu

wspólrz9dnych maja obie wspó1rz9dne ca1kowite.

Zadanie 14. $\langle 0-1$\}

Funkcja kwadratowa $f$ jest określona wzorem

A. $f(x)=-(x+1)^{2}-9$

B. $f(x)=-(x-1)^{2}+9$

C. $f(x)=-(x-1)^{2}-9$

D. $f(x)=-(x+1)^{2}+9$

Zädanie i5. (0-1)

Dla funkcji f prawdziwa jest równośč

A. $f(-4)=f(6)$

B. $f(-4)=f(4)$

C. $f(-4)=f(5)$

D. $f(-4)=f(7)$

Strona 12 z31

$\mathrm{E}\mathrm{M}\mathrm{A}\mathrm{P}-\mathrm{P}0_{-}100$





: {\it RU DNOPIS} \{{\it nie podlega ocenie}\}

$\mathrm{h}\mathrm{P}-\mathrm{P}0_{-}100$

Strona 13z31





Zadanie 16. $\langle 0-1$)

$\mathrm{W}$ ciqgu arytmetycznym $(a_{n})$, określonym dla $\mathrm{k}\mathrm{a}\dot{\mathrm{z}}$ dej liczby naturalnej $n\geq 1$, dane sa

wyrazy $a_{4}=-2$ oraz $a_{6}=16.$

Piqty wyraz tego ciqgu jest równy

A. -27

B. -92

C. 7

D. 9

Zadanie 17. $(0-1$\}

Ciqg geometryczny $(a_{n})$ jest określony wzorem $a_{n}=2^{n-1}$, dla $\mathrm{k}\mathrm{a}\dot{\mathrm{z}}$ dej liczby naturalnej $n\geq 1.$

lloraz tego ciqgu jest równy

A. -21

B. $(-2)$

C. 2

D. l

Zadanie 18. $(0-1$\}

Ciqg $(b_{n})$ jest określony wzorem $b_{n}=(n+2)(7-n)$, dla $\mathrm{k}\mathrm{a}\dot{\mathrm{z}}$ dej liczby naturalnej $n\geq 1.$

Liczba dodatnich wyrazów ciqgu $(b_{n})$ jest równa

A. 6

B. 7

C. 8

D. 9

Zädanie $l9. (0-1)$

Liczba $\sin^{3}20^{\mathrm{o}}+\cos^{2}20^{\mathrm{o}}\cdot\sin 20^{\mathrm{o}}$ jest równa

A. $\cos 20^{\mathrm{o}}$

B. $\sin 20^{\mathrm{o}}$

C. $\mathrm{t}\mathrm{g}20^{\mathrm{o}}$

D. $\sin 20^{\mathrm{o}}\cdot\cos 20^{\mathrm{o}}$

Zadanie 20. (0-1)

$\mathrm{K}\mathrm{a}\mathrm{t} \alpha$ jest ostry oraz $\cos\alpha= \displaystyle \frac{5}{13}$. Wtedy

A. $\displaystyle \mathrm{t}\mathrm{g}\alpha=\frac{12}{13}$

B. $\displaystyle \mathrm{t}\mathrm{g}\alpha=\frac{12}{5}$

C. $\displaystyle \mathrm{t}\mathrm{g}\alpha=\frac{5}{12}$

D. $\displaystyle \mathrm{t}\mathrm{g}\alpha=\frac{13}{12}$

Strona 14 z31

$\mathrm{E}\mathrm{M}\mathrm{A}\mathrm{P}-\mathrm{P}0_{-}100$





: {\it RU DNOPIS} \{{\it nie podlega ocenie}\}

$\mathrm{h}\mathrm{P}-\mathrm{P}0_{-}100$

Strona 15z31





Zadarie 21. $(0-1$\}

Danyjest równoleglobok o bokach dlugości 3 $\mathrm{i} 4$ oraz o kqcie mipdzy nimi o mierze $120^{\mathrm{o}}$

Pole tego równolegloboku jest równe

A. 6

B. $6\sqrt{3}$

C. 12

D. $12\sqrt{3}$

Zadanie 22. $\langle 0-1$\}

$\mathrm{W}$ trójkacie $MKC$ bok $MK$ ma d\}ugośč 24. Prosta równo1eg$\dagger$a do boku $MK$ przecina boki

$MC \mathrm{i} KC -$ odpowiednio-w punktach $A$ oraz $B$ takich, $\dot{\mathrm{z}}\mathrm{e} |AB|=6 \mathrm{i} |AC|=3$

(zobacz rysunek).

{\it C}
\begin{center}
\includegraphics[width=123.084mm,height=56.232mm]{./F2_M_PP_M2024_page15_images/image001.eps}
\end{center}
3

{\it A B}

6

{\it M}  24  {\it K}

Dlugośč odcinka MA jest równa

A. 18

B. 15

C. 9

D. 12

Zadanie 23. $\{0-1\}$

$\mathrm{W}$ trójkqcie $ABC$, wpisanym w $\mathrm{o}\mathrm{k}\mathrm{r}_{\mathrm{c}}\mathrm{l}\mathrm{g}$ o środku w punkcie $S, \mathrm{k}\mathrm{a}\mathrm{t} ACB$ ma miare $42^{\mathrm{o}}$

(zobacz rysunek).
\begin{center}
\includegraphics[width=68.016mm,height=65.172mm]{./F2_M_PP_M2024_page15_images/image002.eps}
\end{center}
{\it C}

$42^{\mathrm{o}}$  {\it S}

{\it A}

{\it B}

Miara kqta ostrego BAS jest równa

A. $42^{\mathrm{o}}$

B. $45^{\mathrm{o}}$

C. $48^{\mathrm{o}}$

D. $69^{\mathrm{o}}$

Strona 16 z31

$\mathrm{E}\mathrm{M}\mathrm{A}\mathrm{P}-\mathrm{P}0_{-}100$





: {\it RU DNOPIS} \{{\it nie podlega ocenie}\}

$\mathrm{h}\mathrm{P}-\mathrm{P}0_{-}100$

Strona 17 z31





Zadanie 24. (0-1)

Proste k oraz l sq określone równaniami

{\it k}:

$y=(m+1)x+7$

{\it l}:

$y=-2x+7$

Proste k oraz l sq prostopadle, gdy liczba m jest równa

A. $(-\displaystyle \frac{1}{2})$

B. -21

C. $(-3)$

D. l

Zadanie 25. $\langle 0-1$\}

Na prostej $l$ o wspólczynniku kierunkowym $\displaystyle \frac{1}{2}\mathrm{l}\mathrm{e}\dot{\mathrm{z}}$ a punkty $A=(2,-4)$ oraz $B=(0,b).$

Wtedy liczba $b$ jest równa

A. $(-5)$

B. 10

C. $(-2)$

D. 0

Zadqnie 26. (0-1)

Wysokośč graniastoslupa prawidlowego sześciokatnego jest równa 6 (zobacz rysunek).

Pole podstawy tego graniastoslupa jest równe $15\sqrt{3}.$
\begin{center}
\includegraphics[width=62.328mm,height=70.968mm]{./F2_M_PP_M2024_page17_images/image001.eps}
\end{center}
I

I

I

I

I

I

I

I

I

I

I

I

I

$\underline{\mathrm{I}}$

I

I

I

I

I

I

I

I

I

I

I

I

6

Pole lednel ściany bocznej tego graniastoslupa jest równe

A. $36\sqrt{10}$

B. 60

C. $6\sqrt{10}$

D. 360

Strona 18 z31

$\mathrm{E}\mathrm{M}\mathrm{A}\mathrm{P}-\mathrm{P}0_{-}100$





: {\it RU DNOPIS} \{{\it nie podlega ocenie}\}

$\mathrm{h}\mathrm{P}-\mathrm{P}0_{-}100$

Strona 19z31





Zadanie 27. $\langle 0-1$)

Kqt nachylenia najdluzszej przekqtnej graniastoslupa prawidlowego sześciokqtnego do

plaszczyzny podstawy jest zaznaczony na rysunku

A.
\begin{center}
\includegraphics[width=52.176mm,height=63.852mm]{./F2_M_PP_M2024_page19_images/image001.eps}
\end{center}
I

I

I

I

I

I

I

I

I

I

I

I

I

I

I

I

I

I

I

I

I

I

I

C.
\begin{center}
\includegraphics[width=52.068mm,height=63.804mm]{./F2_M_PP_M2024_page19_images/image002.eps}
\end{center}
I

I

I

I

I

I

I

I

I

I

I

I

I

I

I

I

I

I

I

I

I

I

I

B.
\begin{center}
\includegraphics[width=52.176mm,height=63.852mm]{./F2_M_PP_M2024_page19_images/image003.eps}
\end{center}
I

I

I

I

I

I

I

I

I

I

I

I

I

I

I

I

I

I

I

I

I

I

D.
\begin{center}
\includegraphics[width=52.176mm,height=63.852mm]{./F2_M_PP_M2024_page19_images/image004.eps}
\end{center}
I

I

I

I

I

I

I

I

I

I

I

I

I

I

I

I

I

I

I

I

I

I

I

Zadanie 28. $(0-1$\}

Obj9tośč ostros1upa prawid1owego czworokatnego jest równa 64. Wysokośč tego ostros1upa

jest równa 12.

Dlugośč krawedzi podstawy tego ostroslupa jest równa

A. 2

B. 4

C. 6

D. 8

Zadanie 29. (0-1)

Rozwazamy wszystkie kody czterocyfrowe utworzone tylko z cyfr 1, 3, 6, 8, przy czym

w $\mathrm{k}\mathrm{a}\dot{\mathrm{z}}$ dym kodzie $\mathrm{k}\mathrm{a}\dot{\mathrm{z}}$ da z tych cyfr wystepuje dokladnie jeden raz.

Liczba wszystkich takich kodów jest równa

A. 4

B. 10

C. 24

D. 16

Strona 20 z31

$\mathrm{E}\mathrm{M}\mathrm{A}\mathrm{P}-\mathrm{P}0_{-}100$





Zadania egzaminacyine sq wydrukowane

na nastepnych stronach.

$\mathrm{E}\mathrm{M}\mathrm{A}\mathrm{P}-\mathrm{P}0_{-}100$

Strona 3 z31





: {\it RU DNOPIS} \{{\it nie podlega ocenie}\}

$\mathrm{h}\mathrm{P}-\mathrm{P}0_{-}100$

Strona 21 z 31





Zadarie 30. (0-2)

Rozwiqz nierównośč

$x^{2}-4\leq 3x$

Strona 22 z31

$\mathrm{E}\mathrm{M}\mathrm{A}\mathrm{P}-\mathrm{P}0_{-}10$





Zadarie 31. (0-2)

Wykaz, $\dot{\mathrm{z}}\mathrm{e}$ dla $\mathrm{k}\mathrm{a}\dot{\mathrm{z}}$ dej liczby rzeczywistej $x$ i dla $\mathrm{k}\mathrm{a}\dot{\mathrm{z}}$ dej liczby rzeczywistej $y$ takich,

$\dot{\mathrm{z}}\mathrm{e} x\neq \mathrm{y}$, prawdziwa jest nierównośč

$(3x+y)(x+3y)>16xy$
\begin{center}
\begin{tabular}{|l|l|l|l|}
\cline{2-4}
&	\multicolumn{1}{|l|}{Nr zadania}&	\multicolumn{1}{|l|}{$30.$}&	\multicolumn{1}{|l|}{ $31.$}	\\
\cline{2-4}
&	\multicolumn{1}{|l|}{Maks. liczba pkt}&	\multicolumn{1}{|l|}{$2$}&	\multicolumn{1}{|l|}{ $2$}	\\
\cline{2-4}
\multicolumn{1}{|l|}{egzaminator}&	\multicolumn{1}{|l|}{Uzyskana liczba pkt}&	\multicolumn{1}{|l|}{}&	\multicolumn{1}{|l|}{}	\\
\hline
\end{tabular}

\end{center}
$\mathrm{E}\mathrm{M}\mathrm{A}\mathrm{P}-\mathrm{P}0_{-}100$

Strona 23 z31





Zadanie 32. (0-2)

Osiq symetrii wykresu funkcji kwadratowej $f(x)=x^{2}+bx+c$ jest prosta o równaniu

$\chi=-2$. Jednym z miejsc zerowych funkcji $f$ jest liczba l.

Oblicz wspólczynniki $b$ oraz $c.$

Strona 24 z31

$\mathrm{E}\mathrm{M}\mathrm{A}\mathrm{P}-\mathrm{P}0_{-}100$





Zadanie 33. (0-2)

Ciqg arytmetyczny $(a_{n})$ jest określony dla $\mathrm{k}\mathrm{a}\dot{\mathrm{z}}$ dej liczby naturalnej $n\geq 1$. Trzeci wyraz

tego ciqgu jest równy $(-1)$, a suma piptnastu poczqtkowych kolejnych wyrazów tego ciqgu

jest równa $(-165).$

Oblicz róznic9 tego ciagu.
\begin{center}
\begin{tabular}{|l|l|l|l|}
\cline{2-4}
&	\multicolumn{1}{|l|}{Nr zadania}&	\multicolumn{1}{|l|}{$32.$}&	\multicolumn{1}{|l|}{ $33.$}	\\
\cline{2-4}
&	\multicolumn{1}{|l|}{Maks. liczba pkt}&	\multicolumn{1}{|l|}{$2$}&	\multicolumn{1}{|l|}{ $2$}	\\
\cline{2-4}
\multicolumn{1}{|l|}{egzaminator}&	\multicolumn{1}{|l|}{Uzyskana liczba pkt}&	\multicolumn{1}{|l|}{}&	\multicolumn{1}{|l|}{}	\\
\hline
\end{tabular}

\end{center}
$\mathrm{E}\mathrm{M}\mathrm{A}\mathrm{P}-\mathrm{P}0_{-}100$

Strona 25 z31





Zadarie 34. (0-2)

Danyjest równoleglobok ABCD, w którym $A=(-2,6)$ oraz $B=(10,2)$. Przekqtne $AC$

oraz $BD$ tego równolegloboku przecinajq $\mathrm{s}\mathrm{i}\mathrm{e}$ w punkcie $P=(6,7).$

Oblicz dlugośč boku $BC$ tego równolegloboku.

Strona 26 z31

$\mathrm{E}\mathrm{M}\mathrm{A}\mathrm{P}-\mathrm{P}0_{-}100$





Zadarie 35. (0-2)

Dany jest piecioelementowy zbiór $K=\{5$, 6, 7, 8, 9$\}$. Wylosowanie $\mathrm{k}\mathrm{a}\dot{\mathrm{z}}$ dej liczby z tego

zbioru jestjednakowo prawdopodobne. Ze zbioru $K$ losujemy ze zwracaniem kolejno dwa

razy po jednej liczbie i zapisujemy je w kolejności losowania.

Oblicz prawdopodobieństwo zdarzenia $A$ polegajqcego na tym, $\dot{\mathrm{z}}\mathrm{e}$ suma wylosowanych

liczb jest liczbq parzystq.
\begin{center}
\begin{tabular}{|l|l|l|l|}
\cline{2-4}
&	\multicolumn{1}{|l|}{Nr zadania}&	\multicolumn{1}{|l|}{$34.$}&	\multicolumn{1}{|l|}{ $35.$}	\\
\cline{2-4}
&	\multicolumn{1}{|l|}{Maks. liczba pkt}&	\multicolumn{1}{|l|}{$2$}&	\multicolumn{1}{|l|}{ $2$}	\\
\cline{2-4}
\multicolumn{1}{|l|}{egzaminator}&	\multicolumn{1}{|l|}{Uzyskana liczba pkt}&	\multicolumn{1}{|l|}{}&	\multicolumn{1}{|l|}{}	\\
\hline
\end{tabular}

\end{center}
$\mathrm{E}\mathrm{M}\mathrm{A}\mathrm{P}-\mathrm{P}0_{-}100$

Strona 27 z31





Zadarie 36. $\langle 0-5$)

$\mathrm{W}$ graniastoslupie prawidlowym czworokqtnym o objętości równej 108 stosunek d1ugości

krawedzi podstawy do wysokości graniastoslupa jest równy $\displaystyle \frac{1}{4}.$

Przekqtna tego graniastoslupa jest nachylona do plaszczyzny jego podstawy pod kqtem $\alpha$

(zobacz rysunek).

Oblicz cosinus kqta $\alpha$ oraz pole powierzchni calkowitej tego graniastoslupa.

Strona 28 z31

$\mathrm{E}\mathrm{M}\mathrm{A}\mathrm{P}-\mathrm{P}0_{-}100$





Wypelnia

egzaminator

Nr zadania

Maks. liczba pkt

Uzyskana liczba pkt

36.

5

-PO-100

Strona 29 z31





: {\it RU DNOPIS} \{{\it nie podlega ocenie}\}

Strona 30z31

$\mathrm{E}\mathrm{M}\mathrm{A}\mathrm{P}-\mathrm{P}0_{-}10$





{\it Wkazdym z zadań od} $f.$ {\it do 29. wybierz izaznacz na karcie odpowiedzi poprawna} $od\sqrt{}owi\mathrm{e}d\acute{z}.$

Zadanie $\mathrm{f}. (0-1$\}

Na poczqtku sezonu letniego cen9 $x$ pary sandalów podwyzszono o 20\%. Po miesiqcu

nowq cenę obnizono o 10\%. Po obu tych zmianach ta para sanda1ów kosztowa1a 81 z1.

Poczqtkowa cena $x$ pary sandalów byta równa

A. 45 z1

B. 73,63 z1

Zadanie 2. (0-1)

Liczba $(\displaystyle \frac{1}{16})^{8}\cdot 8^{16}$ jest równa

A. $2^{24}$

B. $2^{16}$

Zadanie 3, (0-1)

Liczba $\log_{\sqrt{3}}9$ jest równa

A. 2

B. 3

C. 75 z1

D. 87,48 z1

C. $2^{12}$

D. $2^{8}$

C. 4

D. 9

Zädanie 4. (0-1)

Dla $\mathrm{k}\mathrm{a}\dot{\mathrm{z}}$ dej liczby rzeczywistej $a$ i dla $\mathrm{k}\mathrm{a}\dot{\mathrm{z}}$ dej liczby rzeczywistej $b$ wartość wyrazenia

$(2a+b)^{2}-(2a-b)^{2}$ jest równa wartości wyra $\dot{\mathrm{z}}$ enia

A. $8a^{2}$

B. 8ab

C. $-8ab$

D. $2b^{2}$

Zadanie 5. (0-1)

Zbiorem wszystkich rozwiqzań nierówności

1- -23 $\chi<$ -32-$\chi$

jest przedzial

A. $(-\displaystyle \infty,-\frac{2}{3})$

B.(-$\infty$,-23)

C. $(-\displaystyle \frac{2}{3}r+\infty)$

D. $(\displaystyle \frac{2}{3},+\infty)$

Strona 4 z31

$\mathrm{E}\mathrm{M}\mathrm{A}\mathrm{P}-\mathrm{P}0_{-}100$





$0_{-}100$

Strona 31 z31










: {\it RU DNOPIS} \{{\it nie podlega ocenie}\}

$-\mathrm{P}0_{-}100$

Strona 5z31





Zadarie 6. $(0-1$\}

$\mathrm{N}\mathrm{a}\mathrm{j}\mathrm{w}\mathrm{i}_{9}$ksza liczbq bedqcq rozwiazaniem rzeczywistym równania $x(x+2)(x^{2}+9)=0$ jest

A. $(-2)$

B. 0

C. 2

D. 3

Zadanie 7. (0-1)

Równanie $\displaystyle \frac{x+1}{(x+2)(x-3)}=0$ w zbiorze liczb rzeczywistych

A. nie ma rozwiqzania.

B. ma dokladnie jedno rozwiqzanie: $(-1).$

C. ma dokladnie dwa rozwiqzania: $(-2)$ oraz 3.

D. ma dokladnie trzy rozwiazania: $(-1), (-2)$ oraz 3.

Zadqnie 8. $\langle 0-1$)

$\mathrm{W}$ paz'dzierniku 2022 roku za1ozono dwa sady, w których posadzono 1acznie 1960 drzew.

Po roku stwierdzono, $\dot{\mathrm{z}}\mathrm{e}$ uschlo 5\% drzew w pierwszym sadzie i 10\% drzew w drugim

sadzie. Uschniete drzewa usunieto, a nowych nie dosadzano.

Liczba drzew, które pozostaly w drugim sadzie, stanowila 60\% 1iczby drzew, które

pozostaly w pierwszym sadzie.

Niech $x$ oraz $y$ oznaczajq liczby drzew posadzonych- odpowiednio-w pierwszym

i drugim sadzie.

Ukladem równań, którego poprawne rozwiqzanie prowadzi do obliczenia liczby $x$ drzew

posadzonych w pierwszym sadzie oraz liczby $y$ drzew posadzonych w drugim sadzie, jest

A. 

B. 

C. 

D. 

Strona 6 z31

$\mathrm{E}\mathrm{M}\mathrm{A}\mathrm{P}-\mathrm{P}0_{-}100$





: {\it RU DNOPIS} \{{\it nie podlega ocenie}\}

$\mathrm{h}\mathrm{P}-\mathrm{P}0_{-}100$

Strona 7 z31





Zadanie 9. $(0-1$\}

$\acute{\mathrm{S}}$ rednia arytmetyczna trzech liczb: $a, b, c$, jest równa 9.

$\acute{\mathrm{S}}$ rednia arytmetyczna sześciu liczb: $a, a, b, b, c, c$, jest równa

A. 9

B. 6

C. 4,5

D. 18

Zadanie 10. (0-1)

Na rysunku przedstawiono dwie proste równolegle, które sq interpretacjq geometrycznq

jednego z ponizszych ukladów równań A-D.
\begin{center}
\includegraphics[width=97.176mm,height=83.520mm]{./F2_M_PP_M2024_page7_images/image001.eps}
\end{center}
{\it y}

1

0  1  $\chi$

Ukladem równań, którego interpretacje geometryczna przedstawiono na rysunku, jest

A. 

B. 

C. 

D. 

Strona 8 z31

$\mathrm{E}\mathrm{M}\mathrm{A}\mathrm{P}-\mathrm{P}0_{-}100$





: {\it RU DNOPIS} \{{\it nie podlega ocenie}\}

$\mathrm{h}\mathrm{P}-\mathrm{P}0_{-}100$

Strona 9z31





Zadanie ll. $\langle 0-1$)

Na rysunku przedstawiono wykres funkcji $f.$

Zbiorem wartości tej funkcji jest

A. $(-6,6\rangle$

B. $\langle$1, 4$)$

C. $\langle$1, $ 4\rangle$

D. $\langle-6,  6\rangle$

Zadanie 12. $\langle 0-1$\}

Funkcja liniowa $f$ jest określona wzorem $f(x)=(-2k+3)x+k-1$, gdzie $k\in \mathbb{R}.$

Funkcja $f$ jest malejqca dla $\mathrm{k}\mathrm{a}\dot{\mathrm{z}}$ dej liczby $k$ nalezacej do przedzialu

A. $(-\infty,1)$

B. $(-\displaystyle \infty,-\frac{3}{2})$

C. $(1,+\infty)$

D. $(\displaystyle \frac{3}{2},+\infty)$

Zädanie $l3. (0-1$\}

Funkcje liniowe $f$ oraz $g$, określone wzorami $f(x)=3x+6$ oraz $g(x)=ax+7$, maja

to samo miejsce zerowe.

Wspólczynnik $a$ we wzorze funkcji $g$ jest równy

A. $(-\displaystyle \frac{7}{2})$

B. $(-\displaystyle \frac{2}{7})$

C. -72

D. -27

Strona 10 z31

$\mathrm{E}\mathrm{M}\mathrm{A}\mathrm{P}-\mathrm{P}0_{-}100$



\end{document}