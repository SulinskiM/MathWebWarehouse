\documentclass[a4paper,12pt]{article}
\usepackage{latexsym}
\usepackage{amsmath}
\usepackage{amssymb}
\usepackage{graphicx}
\usepackage{wrapfig}
\pagestyle{plain}
\usepackage{fancybox}
\usepackage{bm}

\begin{document}

lnformacja do zadań 14.$-15.$

Na rysunku przedstawiono fragment paraboli, która jest wykresem funkcji kwadratowej $f$

(zobacz rysunek). Wierzcholek tej paraboli oraz punkty przeciecia paraboli z osiami ukladu

wspólrz9dnych maja obie wspó1rz9dne ca1kowite.

Zadanie 14. $\langle 0-1$\}

Funkcja kwadratowa $f$ jest określona wzorem

A. $f(x)=-(x+1)^{2}-9$

B. $f(x)=-(x-1)^{2}+9$

C. $f(x)=-(x-1)^{2}-9$

D. $f(x)=-(x+1)^{2}+9$

Zädanie i5. (0-1)

Dla funkcji f prawdziwa jest równośč

A. $f(-4)=f(6)$

B. $f(-4)=f(4)$

C. $f(-4)=f(5)$

D. $f(-4)=f(7)$

Strona 12 z31

$\mathrm{E}\mathrm{M}\mathrm{A}\mathrm{P}-\mathrm{P}0_{-}100$
\end{document}
