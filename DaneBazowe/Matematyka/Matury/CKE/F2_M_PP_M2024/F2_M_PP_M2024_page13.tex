\documentclass[a4paper,12pt]{article}
\usepackage{latexsym}
\usepackage{amsmath}
\usepackage{amssymb}
\usepackage{graphicx}
\usepackage{wrapfig}
\pagestyle{plain}
\usepackage{fancybox}
\usepackage{bm}

\begin{document}

Zadanie 16. $\langle 0-1$)

$\mathrm{W}$ ciqgu arytmetycznym $(a_{n})$, określonym dla $\mathrm{k}\mathrm{a}\dot{\mathrm{z}}$ dej liczby naturalnej $n\geq 1$, dane sa

wyrazy $a_{4}=-2$ oraz $a_{6}=16.$

Piqty wyraz tego ciqgu jest równy

A. -27

B. -92

C. 7

D. 9

Zadanie 17. $(0-1$\}

Ciqg geometryczny $(a_{n})$ jest określony wzorem $a_{n}=2^{n-1}$, dla $\mathrm{k}\mathrm{a}\dot{\mathrm{z}}$ dej liczby naturalnej $n\geq 1.$

lloraz tego ciqgu jest równy

A. -21

B. $(-2)$

C. 2

D. l

Zadanie 18. $(0-1$\}

Ciqg $(b_{n})$ jest określony wzorem $b_{n}=(n+2)(7-n)$, dla $\mathrm{k}\mathrm{a}\dot{\mathrm{z}}$ dej liczby naturalnej $n\geq 1.$

Liczba dodatnich wyrazów ciqgu $(b_{n})$ jest równa

A. 6

B. 7

C. 8

D. 9

Zädanie $l9. (0-1)$

Liczba $\sin^{3}20^{\mathrm{o}}+\cos^{2}20^{\mathrm{o}}\cdot\sin 20^{\mathrm{o}}$ jest równa

A. $\cos 20^{\mathrm{o}}$

B. $\sin 20^{\mathrm{o}}$

C. $\mathrm{t}\mathrm{g}20^{\mathrm{o}}$

D. $\sin 20^{\mathrm{o}}\cdot\cos 20^{\mathrm{o}}$

Zadanie 20. (0-1)

$\mathrm{K}\mathrm{a}\mathrm{t} \alpha$ jest ostry oraz $\cos\alpha= \displaystyle \frac{5}{13}$. Wtedy

A. $\displaystyle \mathrm{t}\mathrm{g}\alpha=\frac{12}{13}$

B. $\displaystyle \mathrm{t}\mathrm{g}\alpha=\frac{12}{5}$

C. $\displaystyle \mathrm{t}\mathrm{g}\alpha=\frac{5}{12}$

D. $\displaystyle \mathrm{t}\mathrm{g}\alpha=\frac{13}{12}$

Strona 14 z31

$\mathrm{E}\mathrm{M}\mathrm{A}\mathrm{P}-\mathrm{P}0_{-}100$
\end{document}
