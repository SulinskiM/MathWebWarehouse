\documentclass[a4paper,12pt]{article}
\usepackage{latexsym}
\usepackage{amsmath}
\usepackage{amssymb}
\usepackage{graphicx}
\usepackage{wrapfig}
\pagestyle{plain}
\usepackage{fancybox}
\usepackage{bm}

\begin{document}

Zadarie 21. $(0-1$\}

Danyjest równoleglobok o bokach dlugości 3 $\mathrm{i} 4$ oraz o kqcie mipdzy nimi o mierze $120^{\mathrm{o}}$

Pole tego równolegloboku jest równe

A. 6

B. $6\sqrt{3}$

C. 12

D. $12\sqrt{3}$

Zadanie 22. $\langle 0-1$\}

$\mathrm{W}$ trójkacie $MKC$ bok $MK$ ma d\}ugośč 24. Prosta równo1eg$\dagger$a do boku $MK$ przecina boki

$MC \mathrm{i} KC -$ odpowiednio-w punktach $A$ oraz $B$ takich, $\dot{\mathrm{z}}\mathrm{e} |AB|=6 \mathrm{i} |AC|=3$

(zobacz rysunek).

{\it C}
\begin{center}
\includegraphics[width=123.084mm,height=56.232mm]{./F2_M_PP_M2024_page15_images/image001.eps}
\end{center}
3

{\it A B}

6

{\it M}  24  {\it K}

Dlugośč odcinka MA jest równa

A. 18

B. 15

C. 9

D. 12

Zadanie 23. $\{0-1\}$

$\mathrm{W}$ trójkqcie $ABC$, wpisanym w $\mathrm{o}\mathrm{k}\mathrm{r}_{\mathrm{c}}\mathrm{l}\mathrm{g}$ o środku w punkcie $S, \mathrm{k}\mathrm{a}\mathrm{t} ACB$ ma miare $42^{\mathrm{o}}$

(zobacz rysunek).
\begin{center}
\includegraphics[width=68.016mm,height=65.172mm]{./F2_M_PP_M2024_page15_images/image002.eps}
\end{center}
{\it C}

$42^{\mathrm{o}}$  {\it S}

{\it A}

{\it B}

Miara kqta ostrego BAS jest równa

A. $42^{\mathrm{o}}$

B. $45^{\mathrm{o}}$

C. $48^{\mathrm{o}}$

D. $69^{\mathrm{o}}$

Strona 16 z31

$\mathrm{E}\mathrm{M}\mathrm{A}\mathrm{P}-\mathrm{P}0_{-}100$
\end{document}
