\documentclass[a4paper,12pt]{article}
\usepackage{latexsym}
\usepackage{amsmath}
\usepackage{amssymb}
\usepackage{graphicx}
\usepackage{wrapfig}
\pagestyle{plain}
\usepackage{fancybox}
\usepackage{bm}

\begin{document}

Zadanie 24. (0-1)

Proste k oraz l sq określone równaniami

{\it k}:

$y=(m+1)x+7$

{\it l}:

$y=-2x+7$

Proste k oraz l sq prostopadle, gdy liczba m jest równa

A. $(-\displaystyle \frac{1}{2})$

B. -21

C. $(-3)$

D. l

Zadanie 25. $\langle 0-1$\}

Na prostej $l$ o wspólczynniku kierunkowym $\displaystyle \frac{1}{2}\mathrm{l}\mathrm{e}\dot{\mathrm{z}}$ a punkty $A=(2,-4)$ oraz $B=(0,b).$

Wtedy liczba $b$ jest równa

A. $(-5)$

B. 10

C. $(-2)$

D. 0

Zadqnie 26. (0-1)

Wysokośč graniastoslupa prawidlowego sześciokatnego jest równa 6 (zobacz rysunek).

Pole podstawy tego graniastoslupa jest równe $15\sqrt{3}.$
\begin{center}
\includegraphics[width=62.328mm,height=70.968mm]{./F2_M_PP_M2024_page17_images/image001.eps}
\end{center}
I

I

I

I

I

I

I

I

I

I

I

I

I

$\underline{\mathrm{I}}$

I

I

I

I

I

I

I

I

I

I

I

I

6

Pole lednel ściany bocznej tego graniastoslupa jest równe

A. $36\sqrt{10}$

B. 60

C. $6\sqrt{10}$

D. 360

Strona 18 z31

$\mathrm{E}\mathrm{M}\mathrm{A}\mathrm{P}-\mathrm{P}0_{-}100$
\end{document}
