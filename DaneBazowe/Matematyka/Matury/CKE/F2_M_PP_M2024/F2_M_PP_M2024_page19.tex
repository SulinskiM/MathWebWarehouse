\documentclass[a4paper,12pt]{article}
\usepackage{latexsym}
\usepackage{amsmath}
\usepackage{amssymb}
\usepackage{graphicx}
\usepackage{wrapfig}
\pagestyle{plain}
\usepackage{fancybox}
\usepackage{bm}

\begin{document}

Zadanie 27. $\langle 0-1$)

Kqt nachylenia najdluzszej przekqtnej graniastoslupa prawidlowego sześciokqtnego do

plaszczyzny podstawy jest zaznaczony na rysunku

A.
\begin{center}
\includegraphics[width=52.176mm,height=63.852mm]{./F2_M_PP_M2024_page19_images/image001.eps}
\end{center}
I

I

I

I

I

I

I

I

I

I

I

I

I

I

I

I

I

I

I

I

I

I

I

C.
\begin{center}
\includegraphics[width=52.068mm,height=63.804mm]{./F2_M_PP_M2024_page19_images/image002.eps}
\end{center}
I

I

I

I

I

I

I

I

I

I

I

I

I

I

I

I

I

I

I

I

I

I

I

B.
\begin{center}
\includegraphics[width=52.176mm,height=63.852mm]{./F2_M_PP_M2024_page19_images/image003.eps}
\end{center}
I

I

I

I

I

I

I

I

I

I

I

I

I

I

I

I

I

I

I

I

I

I

D.
\begin{center}
\includegraphics[width=52.176mm,height=63.852mm]{./F2_M_PP_M2024_page19_images/image004.eps}
\end{center}
I

I

I

I

I

I

I

I

I

I

I

I

I

I

I

I

I

I

I

I

I

I

I

Zadanie 28. $(0-1$\}

Obj9tośč ostros1upa prawid1owego czworokatnego jest równa 64. Wysokośč tego ostros1upa

jest równa 12.

Dlugośč krawedzi podstawy tego ostroslupa jest równa

A. 2

B. 4

C. 6

D. 8

Zadanie 29. (0-1)

Rozwazamy wszystkie kody czterocyfrowe utworzone tylko z cyfr 1, 3, 6, 8, przy czym

w $\mathrm{k}\mathrm{a}\dot{\mathrm{z}}$ dym kodzie $\mathrm{k}\mathrm{a}\dot{\mathrm{z}}$ da z tych cyfr wystepuje dokladnie jeden raz.

Liczba wszystkich takich kodów jest równa

A. 4

B. 10

C. 24

D. 16

Strona 20 z31

$\mathrm{E}\mathrm{M}\mathrm{A}\mathrm{P}-\mathrm{P}0_{-}100$
\end{document}
