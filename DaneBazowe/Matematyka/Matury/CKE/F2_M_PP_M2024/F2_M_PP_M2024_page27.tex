\documentclass[a4paper,12pt]{article}
\usepackage{latexsym}
\usepackage{amsmath}
\usepackage{amssymb}
\usepackage{graphicx}
\usepackage{wrapfig}
\pagestyle{plain}
\usepackage{fancybox}
\usepackage{bm}

\begin{document}

Zadarie 36. $\langle 0-5$)

$\mathrm{W}$ graniastoslupie prawidlowym czworokqtnym o objętości równej 108 stosunek d1ugości

krawedzi podstawy do wysokości graniastoslupa jest równy $\displaystyle \frac{1}{4}.$

Przekqtna tego graniastoslupa jest nachylona do plaszczyzny jego podstawy pod kqtem $\alpha$

(zobacz rysunek).

Oblicz cosinus kqta $\alpha$ oraz pole powierzchni calkowitej tego graniastoslupa.

Strona 28 z31

$\mathrm{E}\mathrm{M}\mathrm{A}\mathrm{P}-\mathrm{P}0_{-}100$
\end{document}
