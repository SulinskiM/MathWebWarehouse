\documentclass[a4paper,12pt]{article}
\usepackage{latexsym}
\usepackage{amsmath}
\usepackage{amssymb}
\usepackage{graphicx}
\usepackage{wrapfig}
\pagestyle{plain}
\usepackage{fancybox}
\usepackage{bm}

\begin{document}

{\it Wkazdym z zadań od} $f.$ {\it do 29. wybierz izaznacz na karcie odpowiedzi poprawna} $od\sqrt{}owi\mathrm{e}d\acute{z}.$

Zadanie $\mathrm{f}. (0-1$\}

Na poczqtku sezonu letniego cen9 $x$ pary sandalów podwyzszono o 20\%. Po miesiqcu

nowq cenę obnizono o 10\%. Po obu tych zmianach ta para sanda1ów kosztowa1a 81 z1.

Poczqtkowa cena $x$ pary sandalów byta równa

A. 45 z1

B. 73,63 z1

Zadanie 2. (0-1)

Liczba $(\displaystyle \frac{1}{16})^{8}\cdot 8^{16}$ jest równa

A. $2^{24}$

B. $2^{16}$

Zadanie 3, (0-1)

Liczba $\log_{\sqrt{3}}9$ jest równa

A. 2

B. 3

C. 75 z1

D. 87,48 z1

C. $2^{12}$

D. $2^{8}$

C. 4

D. 9

Zädanie 4. (0-1)

Dla $\mathrm{k}\mathrm{a}\dot{\mathrm{z}}$ dej liczby rzeczywistej $a$ i dla $\mathrm{k}\mathrm{a}\dot{\mathrm{z}}$ dej liczby rzeczywistej $b$ wartość wyrazenia

$(2a+b)^{2}-(2a-b)^{2}$ jest równa wartości wyra $\dot{\mathrm{z}}$ enia

A. $8a^{2}$

B. 8ab

C. $-8ab$

D. $2b^{2}$

Zadanie 5. (0-1)

Zbiorem wszystkich rozwiqzań nierówności

1- -23 $\chi<$ -32-$\chi$

jest przedzial

A. $(-\displaystyle \infty,-\frac{2}{3})$

B.(-$\infty$,-23)

C. $(-\displaystyle \frac{2}{3}r+\infty)$

D. $(\displaystyle \frac{2}{3},+\infty)$

Strona 4 z31

$\mathrm{E}\mathrm{M}\mathrm{A}\mathrm{P}-\mathrm{P}0_{-}100$
\end{document}
