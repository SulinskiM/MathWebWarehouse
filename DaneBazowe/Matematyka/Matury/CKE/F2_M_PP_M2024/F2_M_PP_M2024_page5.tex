\documentclass[a4paper,12pt]{article}
\usepackage{latexsym}
\usepackage{amsmath}
\usepackage{amssymb}
\usepackage{graphicx}
\usepackage{wrapfig}
\pagestyle{plain}
\usepackage{fancybox}
\usepackage{bm}

\begin{document}

Zadarie 6. $(0-1$\}

$\mathrm{N}\mathrm{a}\mathrm{j}\mathrm{w}\mathrm{i}_{9}$ksza liczbq bedqcq rozwiazaniem rzeczywistym równania $x(x+2)(x^{2}+9)=0$ jest

A. $(-2)$

B. 0

C. 2

D. 3

Zadanie 7. (0-1)

Równanie $\displaystyle \frac{x+1}{(x+2)(x-3)}=0$ w zbiorze liczb rzeczywistych

A. nie ma rozwiqzania.

B. ma dokladnie jedno rozwiqzanie: $(-1).$

C. ma dokladnie dwa rozwiqzania: $(-2)$ oraz 3.

D. ma dokladnie trzy rozwiazania: $(-1), (-2)$ oraz 3.

Zadqnie 8. $\langle 0-1$)

$\mathrm{W}$ paz'dzierniku 2022 roku za1ozono dwa sady, w których posadzono 1acznie 1960 drzew.

Po roku stwierdzono, $\dot{\mathrm{z}}\mathrm{e}$ uschlo 5\% drzew w pierwszym sadzie i 10\% drzew w drugim

sadzie. Uschniete drzewa usunieto, a nowych nie dosadzano.

Liczba drzew, które pozostaly w drugim sadzie, stanowila 60\% 1iczby drzew, które

pozostaly w pierwszym sadzie.

Niech $x$ oraz $y$ oznaczajq liczby drzew posadzonych- odpowiednio-w pierwszym

i drugim sadzie.

Ukladem równań, którego poprawne rozwiqzanie prowadzi do obliczenia liczby $x$ drzew

posadzonych w pierwszym sadzie oraz liczby $y$ drzew posadzonych w drugim sadzie, jest

A. 

B. 

C. 

D. 

Strona 6 z31

$\mathrm{E}\mathrm{M}\mathrm{A}\mathrm{P}-\mathrm{P}0_{-}100$
\end{document}
