\documentclass[a4paper,12pt]{article}
\usepackage{latexsym}
\usepackage{amsmath}
\usepackage{amssymb}
\usepackage{graphicx}
\usepackage{wrapfig}
\pagestyle{plain}
\usepackage{fancybox}
\usepackage{bm}

\begin{document}

Zadanie 9. $(0-1$\}

$\acute{\mathrm{S}}$ rednia arytmetyczna trzech liczb: $a, b, c$, jest równa 9.

$\acute{\mathrm{S}}$ rednia arytmetyczna sześciu liczb: $a, a, b, b, c, c$, jest równa

A. 9

B. 6

C. 4,5

D. 18

Zadanie 10. (0-1)

Na rysunku przedstawiono dwie proste równolegle, które sq interpretacjq geometrycznq

jednego z ponizszych ukladów równań A-D.
\begin{center}
\includegraphics[width=97.176mm,height=83.520mm]{./F2_M_PP_M2024_page7_images/image001.eps}
\end{center}
{\it y}

1

0  1  $\chi$

Ukladem równań, którego interpretacje geometryczna przedstawiono na rysunku, jest

A. 

B. 

C. 

D. 

Strona 8 z31

$\mathrm{E}\mathrm{M}\mathrm{A}\mathrm{P}-\mathrm{P}0_{-}100$
\end{document}
