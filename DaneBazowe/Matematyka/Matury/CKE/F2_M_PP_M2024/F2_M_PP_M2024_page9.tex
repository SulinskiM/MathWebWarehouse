\documentclass[a4paper,12pt]{article}
\usepackage{latexsym}
\usepackage{amsmath}
\usepackage{amssymb}
\usepackage{graphicx}
\usepackage{wrapfig}
\pagestyle{plain}
\usepackage{fancybox}
\usepackage{bm}

\begin{document}

Zadanie ll. $\langle 0-1$)

Na rysunku przedstawiono wykres funkcji $f.$

Zbiorem wartości tej funkcji jest

A. $(-6,6\rangle$

B. $\langle$1, 4$)$

C. $\langle$1, $ 4\rangle$

D. $\langle-6,  6\rangle$

Zadanie 12. $\langle 0-1$\}

Funkcja liniowa $f$ jest określona wzorem $f(x)=(-2k+3)x+k-1$, gdzie $k\in \mathbb{R}.$

Funkcja $f$ jest malejqca dla $\mathrm{k}\mathrm{a}\dot{\mathrm{z}}$ dej liczby $k$ nalezacej do przedzialu

A. $(-\infty,1)$

B. $(-\displaystyle \infty,-\frac{3}{2})$

C. $(1,+\infty)$

D. $(\displaystyle \frac{3}{2},+\infty)$

Zädanie $l3. (0-1$\}

Funkcje liniowe $f$ oraz $g$, określone wzorami $f(x)=3x+6$ oraz $g(x)=ax+7$, maja

to samo miejsce zerowe.

Wspólczynnik $a$ we wzorze funkcji $g$ jest równy

A. $(-\displaystyle \frac{7}{2})$

B. $(-\displaystyle \frac{2}{7})$

C. -72

D. -27

Strona 10 z31

$\mathrm{E}\mathrm{M}\mathrm{A}\mathrm{P}-\mathrm{P}0_{-}100$
\end{document}
