\documentclass[10pt]{article}
\usepackage[polish]{babel}
\usepackage[utf8]{inputenc}
\usepackage[T1]{fontenc}
\usepackage{graphicx}
\usepackage[export]{adjustbox}
\graphicspath{ {./images/} }
\usepackage{amsmath}
\usepackage{amsfonts}
\usepackage{amssymb}
\usepackage[version=4]{mhchem}
\usepackage{stmaryrd}
\usepackage{multirow}

\title{Sinus kąta ostrego \(\alpha\) jest równy \(\frac{3}{4}\). Wówczas\\
A. \(\cos \alpha=\frac{1}{4}\)\\
B. \(\cos \alpha=\frac{\sqrt{7}}{4}\)\\
C. \(\cos \alpha=\frac{7}{16}\)\\
D. \(\cos \alpha=\frac{\sqrt{13}}{16}\)\\
\(\qquad\)\\
\includegraphics[max width=\textwidth, center]{2025_02_10_74e27f4b95bc3ce8e9b7g-07(4)}\\
\(\qquad\)\\
\includegraphics[max width=\textwidth, center]{2025_02_10_74e27f4b95bc3ce8e9b7g-07(11)}}

\author{Data: 25 sierpnia 2015 r.\\
Godzina rozpoczęcia: 9:00\\
CZAS PRACY: \(\mathbf{1 7 0}\) minut\\
LicZba punktów do uzyskania: 50}
\date{}


\begin{document}
\maketitle
\begin{center}
\includegraphics[max width=\textwidth]{2025_02_10_74e27f4b95bc3ce8e9b7g-01}
\end{center}

\section*{EGZAMIN MATURALNY Z MATEMATYKI POZIOM PODSTAWOWY}


\section*{Instrukcja dla zdającego}
\begin{enumerate}
  \item Sprawdź, czy arkusz egzaminacyjny zawiera 24 strony (zadania 1-34). Ewentualny brak zgłoś przewodniczącemu zespołu nadzorującego egzamin.
  \item Rozwiązania zadań i odpowiedzi wpisuj w miejscu na to przeznaczonym.
  \item Odpowiedzi do zadań zamkniętych (1-25) przenieś na kartę odpowiedzi, zaznaczając je w części karty przeznaczonej dla zdającego. Zamaluj pola do tego przeznaczone. Błędne zaznaczenie otocz kółkiem ( ) i zaznacz właściwe.
  \item Pamiętaj, że pominięcie argumentacji lub istotnych obliczeń w rozwiązaniu zadania otwartego (26-34) może spowodować, że za to rozwiązanie nie otrzymasz pełnej liczby punktów.
  \item Pisz czytelnie i używaj tylko długopisu lub pióra z czarnym tuszem lub atramentem.
  \item Nie używaj korektora, a błędne zapisy wyraźnie przekreśl.
  \item Pamiętaj, że zapisy w brudnopisie nie będą oceniane.
  \item Możesz korzystać z zestawu wzorów matematycznych, cyrkla i linijki oraz kalkulatora prostego.
  \item Na tej stronie oraz na karcie odpowiedzi wpisz swój numer PESEL i przyklej naklejkę z kodem.
  \item Nie wpisuj żadnych znaków w części przeznaczonej dla egzaminatora.\\
\includegraphics[max width=\textwidth, center]{2025_02_10_74e27f4b95bc3ce8e9b7g-01(1)}
\end{enumerate}

W zadaniach od 1. do 25. wybierz i zaznacz na karcie odpowiedzi poprawna odpowied́́.

\section*{Zadanie 1. (0-1)}
Jeśli \(a=\frac{3}{2}\) i \(b=2\), to wartość wyrażenia \(\frac{a \cdot b}{a+b}\) jest równa\\
A. \(\frac{2}{3}\)\\
B. 1\\
C. \(\frac{6}{7}\)\\
D. \(\frac{27}{6}\)

\section*{Zadanie 2. (0-1)}
Dany jest prostokąt o wymiarach \(40 \mathrm{~cm} \times 100 \mathrm{~cm}\). Jeżeli każdy z dłuższych boków tego prostokąta wydłużymy o \(20 \%\), a każdy z krótszych boków skrócimy o \(20 \%\), to w wyniku obu przekształceń pole tego prostokąta\\
A. zwiększy się o \(8 \%\).\\
B. zwiększy się o \(4 \%\).\\
C. zmniejszy się o \(8 \%\).\\
D. zmniejszy się o \(4 \%\).

\section*{Zadanie 3. (0-1)}
Liczba \(\frac{9^{5} \cdot 5^{9}}{45^{5}}\) jest równa\\
A. \(45^{40}\)\\
B. \(45^{9}\)\\
C. \(9^{4}\)\\
D. \(5^{4}\)

\section*{Zadanie 4. (0-1)}
Liczba \(\sqrt{\frac{9}{7}}+\sqrt{\frac{7}{9}}\) jest równa\\
A. \(\sqrt{\frac{16}{63}}\)\\
B. \(\frac{16}{3 \sqrt{7}}\)\\
C. 1\\
D. \(\frac{3+\sqrt{7}}{3 \sqrt{7}}\)

\section*{Zadanie 5. (0-1)}
Wartość wyrażenia \(\log _{5} 0,04-\frac{1}{2} \log _{25} 5 \cdot \log _{25} 1\) jest równa\\
A. -3\\
B. \(-2 \frac{1}{4}\)\\
C. -2\\
D. 0

\section*{Zadanie 6. (0-1)}
Wartość wyrażenia \((a+5)^{2}\) jest większa od wartości wyrażenia \(\left(a^{2}+10 a\right)\) o\\
A. 50\\
B. 10\\
C. 5\\
D. 25

BRUDNOPIS (nie podlega ocenie)\\
\includegraphics[max width=\textwidth, center]{2025_02_10_74e27f4b95bc3ce8e9b7g-03}

\section*{Zadanie 7. (0-1)}
Na jednym z poniższych rysunków przedstawiono interpretację geometryczną układu równań \(\left\{\begin{array}{l}x+3 y=-5 \\ 3 x-2 y=-4\end{array}\right.\)\\
Wskaż ten rysunek.\\
A.\\
\includegraphics[max width=\textwidth, center]{2025_02_10_74e27f4b95bc3ce8e9b7g-04}\\
B.\\
\includegraphics[max width=\textwidth, center]{2025_02_10_74e27f4b95bc3ce8e9b7g-04(1)}\\
C.\\
\includegraphics[max width=\textwidth, center]{2025_02_10_74e27f4b95bc3ce8e9b7g-04(3)}\\
D.\\
\includegraphics[max width=\textwidth, center]{2025_02_10_74e27f4b95bc3ce8e9b7g-04(2)}

\section*{Zadanie 8. (0-1)}
Najmniejszą liczbą całkowitą spełniającą nierówność \(2(x-2) \leq 4(x-1)+1\) jest\\
A. -2\\
B. -1\\
C. 0\\
D. 1

\section*{Zadanie 9. (0-1)}
Rozwiązaniem równania \(x^{2}(x+1)=x^{2}-8\) jest\\
A. -9\\
B. -2\\
C. 2\\
D. 7

\section*{Zadanie 10. (0-1)}
Funkcja \(f\) jest określona wzorem \(f(x)=\frac{2 x-8}{x}\) dla każdej liczby rzeczywistej \(x \neq 0\).\\
Wówczas wartość funkcji \(f(\sqrt{2})\) jest równa\\
A. \(2-4 \sqrt{2}\)\\
B. \(1-2 \sqrt{2}\)\\
C. \(1+2 \sqrt{2}\)\\
D. \(2+4 \sqrt{2}\)

BRUDNOPIS (nie podlega ocenie)\\
\includegraphics[max width=\textwidth, center]{2025_02_10_74e27f4b95bc3ce8e9b7g-05}

\section*{Zadanie 11. (0-1)}
Parabola o wierzchołku \(W=(-3,5)\) i ramionach skierowanych w dół może być wykresem funkcji określonej wzorem\\
A. \(y=2 \cdot(x+3)^{2}+5\)\\
B. \(y=-2 \cdot(x-3)^{2}+5\)\\
C. \(y=-2 \cdot(x+3)^{2}+5\)\\
D. \(y=-2 \cdot(x-3)^{2}-5\)

\section*{Zadanie 12. (0-1)}
Wykres funkcji liniowej \(y=2 x-3\) przecina oś \(O y\) w punkcie o współrzędnych\\
A. \((0,-3)\)\\
B. \((-3,0)\)\\
C. \((0,2)\)\\
D. \((0,3)\)

\section*{Zadanie 13. (0-1)}
Wierzchołek paraboli będącej wykresem funkcji kwadratowej \(y=f(x)\) ma współrzędne \((2,2)\). Wówczas wierzchołek paraboli będącej wykresem funkcji \(g(x)=f(x+2)\) ma współrzędne\\
A. \((4,2)\)\\
B. \((0,2)\)\\
C. \((2,0)\)\\
D. \((2,4)\)

\section*{Zadanie 14. (0-1)}
Wszystkie dwucyfrowe liczby naturalne podzielne przez 7 tworzą rosnący ciąg arytmetyczny. Dwunastym wyrazem tego ciągu jest liczba\\
A. 77\\
B. 84\\
C. 91\\
D. 98

\section*{Zadanie 15. (0-1)}
Ciạg liczbowy określony jest wzorem \(a_{n}=\frac{2^{n}-1}{2^{n}+1}\), dla \(n \geq 1\). Piąty wyraz tego ciągu jest równy\\
A. -1\\
B. \(\frac{31}{33}\)\\
C. \(\frac{9}{11}\)\\
D. 1

\section*{Zadanie 16. (0-1)}
\(\qquad\)\\
\(\qquad\)\\
\includegraphics[max width=\textwidth, center]{2025_02_10_74e27f4b95bc3ce8e9b7g-07(3)}\\
\(\qquad\)\\
\includegraphics[max width=\textwidth, center]{2025_02_10_74e27f4b95bc3ce8e9b7g-07(2)}\\
\(\qquad\)\\
\includegraphics[max width=\textwidth, center]{2025_02_10_74e27f4b95bc3ce8e9b7g-07}\\
\includegraphics[max width=\textwidth, center]{2025_02_10_74e27f4b95bc3ce8e9b7g-07(10)}\\
\(\qquad\)

\includegraphics[max width=\textwidth]{2025_02_10_74e27f4b95bc3ce8e9b7g-07(8)} \begin{tabular}{l}
P \\
\hline
\end{tabular}

\includegraphics[max width=\textwidth, center]{2025_02_10_74e27f4b95bc3ce8e9b7g-07(9)}\\
\(\qquad\)\\
\includegraphics[max width=\textwidth, center]{2025_02_10_74e27f4b95bc3ce8e9b7g-07(7)}\\
\includegraphics[max width=\textwidth, center]{2025_02_10_74e27f4b95bc3ce8e9b7g-07(6)}\\
\includegraphics[max width=\textwidth, center]{2025_02_10_74e27f4b95bc3ce8e9b7g-07(5)}\\
\includegraphics[max width=\textwidth, center]{2025_02_10_74e27f4b95bc3ce8e9b7g-07(1)}

\section*{Zadanie 17. (0-1)}
W trójkącie prostokątnym o długościach przyprostokątnych 2 i 5 cosinus większego z kątów ostrych jest równy\\
A. \(\frac{5}{2}\)\\
B. \(\frac{2}{5}\)\\
C. \(\frac{2}{\sqrt{29}}\)\\
D. \(\frac{5}{\sqrt{29}}\)

\section*{Zadanie 18. (0-1)}
Pole rombu o boku 6 i kącie rozwartym \(150^{\circ}\) jest równe\\
A. \(18 \sqrt{2}\)\\
B. 18\\
C. \(36 \sqrt{2}\)\\
D. 36

\section*{Zadanie 19. (0-1)}
W okręgu o środku \(O\) dany jest kąt o mierze \(50^{\circ}\), zaznaczony na rysunku.\\
\includegraphics[max width=\textwidth, center]{2025_02_10_74e27f4b95bc3ce8e9b7g-08}

Miara kąta oznaczonego na rysunku literą \(\alpha\) jest równa\\
A. \(40^{\circ}\)\\
B. \(50^{\circ}\)\\
C. \(20^{\circ}\)\\
D. \(25^{\circ}\)

\section*{Zadanie 20. (0-1)}
Współczynnik kierunkowy prostej, na której leżą punkty \(A=(-4,3)\) oraz \(B=(8,7)\), jest równy\\
A. \(a=3\)\\
B. \(\quad a=-1\)\\
C. \(a=\frac{5}{6}\)\\
D. \(\quad a=\frac{1}{3}\)

BRUDNOPIS (nie podlega ocenie)\\
\includegraphics[max width=\textwidth, center]{2025_02_10_74e27f4b95bc3ce8e9b7g-09}

\section*{Zadanie 21. (0-1)}
Punkt \(S=(2,-5)\) jest środkiem odcinka \(A B\), gdzie \(A=(-4,3)\) i \(B=(8, b)\). Wtedy\\
A. \(b=-13\)\\
B. \(b=-2\)\\
C. \(b=-1\)\\
D. \(b=6\)

\section*{Zadanie 22. (0-1)}
Dany jest trójkąt prostokątny o długościach boków \(a, b, c\), gdzie \(a<b<c\). Obracając ten trójkąt, wokół prostej zawierającej dłuższą przyprostokątną o kąt \(360^{\circ}\), otrzymujemy bryłę, której objętość jest równa\\
A. \(\quad V=\frac{1}{3} a^{2} b \pi\)\\
B. \(\quad V=a^{2} b \pi\)\\
C. \(V=\frac{1}{3} b^{2} a \pi\)\\
D. \(V=a^{2} \pi+\pi a c\)

\section*{Zadanie 23. (0-1)}
Przekątna przekroju osiowego walca, którego promień podstawy jest równy 4 i wysokość jest równa 6, ma długość\\
A. \(\sqrt{10}\)\\
B. \(\sqrt{20}\)\\
C. \(\sqrt{52}\)\\
D. 10

\section*{Zadanie 24. (0-1)}
W grupie jest 15 kobiet i 18 mężczyzn. Losujemy jedną osobę z tej grupy. Prawdopodobieństwo tego, że będzie to kobieta, jest równe\\
A. \(\frac{1}{15}\)\\
B. \(\frac{1}{33}\)\\
C. \(\frac{15}{33}\)\\
D. \(\frac{15}{18}\)

\section*{Zadanie 25. (0-1)}
Ile jest wszystkich liczb czterocyfrowych, większych od 3000, utworzonych wyłącznie z cyfr \(1,2,3\), przy założeniu, że cyfry mogą się powtarzać, ale nie wszystkie \(z\) tych cyfr muszą być wykorzystane?\\
A. 3\\
B. 6\\
C. 9\\
D. 27

BRUDNOPIS (nie podlega ocenie)\\
\includegraphics[max width=\textwidth, center]{2025_02_10_74e27f4b95bc3ce8e9b7g-11}

Rozwiąż równanie \(\frac{2 x-4}{x}=\frac{x}{2 x-4}\), gdzie \(x \neq 0\) i \(x \neq 2\).

\begin{center}
\begin{tabular}{|c|c|c|c|c|c|c|c|c|c|c|c|c|c|c|c|c|c|c|c|c|c|}
\hline
 &  &  &  &  &  &  &  &  &  &  &  &  &  &  &  &  &  &  &  &  &  \\
\hline
 &  &  &  &  &  &  &  &  &  &  &  &  &  &  &  &  &  &  &  &  &  \\
\hline
 &  &  &  &  &  &  &  &  &  &  &  &  &  &  &  &  &  &  &  &  &  \\
\hline
 &  &  &  &  &  &  &  &  &  &  &  &  &  &  &  &  &  &  &  &  &  \\
\hline
 &  &  &  &  &  &  &  &  &  &  &  &  &  &  &  &  &  &  &  &  &  \\
\hline
 &  &  &  &  &  &  &  &  &  &  &  &  &  &  &  &  &  &  &  &  &  \\
\hline
 &  &  &  &  &  &  &  &  &  &  &  &  &  &  &  &  &  &  &  &  &  \\
\hline
 &  &  &  &  &  &  &  &  &  &  &  &  &  &  &  &  &  &  &  &  &  \\
\hline
 &  &  &  &  &  &  &  &  &  &  &  &  &  &  &  &  &  &  &  &  &  \\
\hline
 &  &  &  &  &  &  &  &  &  &  &  &  &  &  &  &  &  &  &  &  &  \\
\hline
 &  &  &  &  &  &  &  &  &  &  &  &  &  &  &  &  &  &  &  &  &  \\
\hline
 &  &  &  &  &  &  &  &  &  &  &  &  &  &  &  &  &  &  &  &  &  \\
\hline
 &  &  &  &  &  &  &  &  &  &  &  &  &  &  &  &  &  &  &  &  &  \\
\hline
 &  &  &  &  &  &  &  &  &  &  &  &  &  &  &  &  &  &  &  &  &  \\
\hline
 &  &  &  &  &  &  &  &  &  &  &  &  &  &  &  &  &  &  &  &  &  \\
\hline
 &  &  &  &  &  &  &  &  &  &  &  &  &  &  &  &  &  &  &  &  &  \\
\hline
 &  &  &  &  &  &  &  &  &  &  &  &  &  &  &  &  &  &  &  &  &  \\
\hline
 &  &  &  &  &  &  &  &  &  &  &  &  &  &  &  &  &  &  &  &  &  \\
\hline
 &  &  &  &  &  &  &  &  &  &  &  &  &  &  &  &  &  &  &  &  &  \\
\hline
 &  &  &  &  &  &  &  &  &  &  &  &  &  &  &  &  &  &  &  &  &  \\
\hline
 &  &  &  &  &  &  &  &  &  &  &  &  &  &  &  &  &  &  &  &  &  \\
\hline
 &  &  &  &  &  &  &  &  &  &  &  &  &  &  &  &  &  &  &  &  &  \\
\hline
 &  &  &  &  &  &  &  &  &  &  &  &  &  &  &  &  &  &  &  &  &  \\
\hline
 &  &  &  &  &  &  &  &  &  &  &  &  &  &  &  &  &  &  &  &  &  \\
\hline
 &  &  &  &  &  &  &  &  &  &  &  &  &  &  &  &  &  &  &  &  &  \\
\hline
 &  &  &  &  &  &  &  &  &  &  &  &  &  &  &  &  &  &  &  &  &  \\
\hline
 &  &  &  &  &  &  &  &  &  &  &  &  &  &  &  &  &  &  &  &  &  \\
\hline
 &  &  &  &  &  &  &  &  &  &  &  &  &  &  &  &  &  &  &  &  &  \\
\hline
 &  &  &  &  &  &  &  &  &  &  &  &  &  &  &  &  &  &  &  &  &  \\
\hline
 &  &  &  &  &  &  &  &  &  &  &  &  &  &  &  &  &  &  &  &  &  \\
\hline
 &  &  &  &  &  &  &  &  &  &  &  &  &  &  &  &  &  &  &  &  &  \\
\hline
 &  &  &  &  &  &  &  &  &  &  &  &  &  &  &  &  &  &  &  &  &  \\
\hline
 &  &  &  &  &  &  &  &  &  &  &  &  &  &  &  &  &  &  &  &  &  \\
\hline
 &  &  &  &  &  &  &  &  &  &  &  &  &  &  &  &  &  &  &  &  &  \\
\hline
 &  &  &  &  &  &  &  &  &  &  &  &  &  &  &  &  &  &  &  &  &  \\
\hline
 &  &  &  &  &  &  &  &  &  &  &  &  &  &  &  &  &  &  &  &  &  \\
\hline
 &  &  &  &  &  &  &  &  &  &  &  &  &  &  &  &  &  &  &  &  &  \\
\hline
 &  &  &  &  &  &  &  &  &  &  &  &  &  &  &  &  &  &  &  &  &  \\
\hline
 &  &  &  &  &  &  &  &  &  &  &  &  &  &  &  &  &  &  &  &  &  \\
\hline
 &  &  &  &  &  &  &  &  &  &  &  &  &  &  &  &  &  &  &  &  &  \\
\hline
 &  &  &  &  &  &  &  &  &  &  &  &  &  &  &  &  &  &  &  &  &  \\
\hline
\end{tabular}
\end{center}

Odpowiedź:

\section*{Zadanie 27. (0-2)}
Mamy dwa pudełka: w pierwszym znajduje się 6 kul ponumerowanych kolejnymi liczbami od 1 do 6 , a w drugim - 8 kul ponumerowanych kolejnymi liczbami od 1 do 8 . Losujemy po jednej kuli z każdego pudełka i tworzymy liczbę dwucyfrową w ten sposób, że numer kuli wylosowanej z pierwszego pudełka jest cyfrą dziesiątek, a numer kuli wylosowanej z drugiego - cyfrą jedności tej liczby. Oblicz prawdopodobieństwo, że utworzona liczba jest podzielna przez 11.\\
\includegraphics[max width=\textwidth, center]{2025_02_10_74e27f4b95bc3ce8e9b7g-13}

\begin{center}
\begin{tabular}{|c|l|c|c|}
\hline
\multirow{3}{*}{\begin{tabular}{l}
Wypetnia \\
egzaminator \\
\end{tabular}} & Nr zadania & 26. & 27. \\
\cline { 2 - 4 }
 & Maks. liczba pkt & 2 & 2 \\
\cline { 2 - 4 }
 & Uzyskana liczba pkt &  &  \\
\hline
\end{tabular}
\end{center}

\section*{Zadanie 28. (0-2)}
Rozwiąż nierówność \(20 x \geq 4 x^{2}+24\).

\begin{center}
\begin{tabular}{|c|c|c|c|c|c|c|c|c|c|c|c|c|c|c|c|c|c|c|c|c|c|c|}
\hline
 &  &  &  &  &  &  &  &  &  &  &  &  &  &  &  &  &  &  &  &  &  &  \\
\hline
 &  &  &  &  &  &  &  &  &  &  &  &  &  &  &  &  &  &  &  &  &  &  \\
\hline
 &  &  &  &  &  &  &  &  &  &  &  &  &  &  &  &  &  &  &  &  &  &  \\
\hline
 &  &  &  &  &  &  &  &  &  &  &  &  &  &  &  &  &  &  &  &  &  &  \\
\hline
 &  &  &  &  &  &  &  &  &  &  &  &  &  &  &  &  &  &  &  &  &  &  \\
\hline
 &  &  &  &  &  &  &  &  &  &  &  &  &  &  &  &  &  &  &  &  &  &  \\
\hline
 &  &  &  &  &  &  &  &  &  &  &  &  &  &  &  &  &  &  &  &  &  &  \\
\hline
 &  &  &  &  &  &  &  &  &  &  &  &  &  &  &  &  &  &  &  &  &  &  \\
\hline
 &  &  &  &  &  &  &  &  &  &  &  &  &  &  &  &  &  &  &  &  &  &  \\
\hline
 &  &  &  &  &  &  &  &  &  &  &  &  &  &  &  &  &  &  &  &  &  &  \\
\hline
 &  &  &  &  &  &  &  &  &  &  &  &  &  &  &  &  &  &  &  &  &  &  \\
\hline
 &  &  &  &  &  &  &  &  &  &  &  &  &  &  &  &  &  &  &  &  &  &  \\
\hline
 &  &  &  &  &  &  &  &  &  &  &  &  &  &  &  &  &  &  &  &  &  &  \\
\hline
 &  &  &  &  &  &  &  &  &  &  &  &  &  &  &  &  &  &  &  &  &  &  \\
\hline
 &  &  &  &  &  &  &  &  &  &  &  &  &  &  &  &  &  &  &  &  &  &  \\
\hline
 &  &  &  &  &  &  &  &  &  &  &  &  &  &  &  &  &  &  &  &  &  &  \\
\hline
 &  &  &  &  &  &  &  &  &  &  &  &  &  &  &  &  &  &  &  &  &  &  \\
\hline
 &  &  &  &  &  &  &  &  &  &  &  &  &  &  &  &  &  &  &  &  &  &  \\
\hline
 &  &  &  &  &  &  &  &  &  &  &  &  &  &  &  &  &  &  &  &  &  &  \\
\hline
 &  &  &  &  &  &  &  &  &  &  &  &  &  &  &  &  &  &  &  &  &  &  \\
\hline
 &  &  &  &  &  &  &  &  &  &  &  &  &  &  &  &  &  &  &  &  &  &  \\
\hline
 &  &  &  &  &  &  &  &  &  &  &  &  &  &  &  &  &  &  &  &  &  &  \\
\hline
 &  &  &  &  &  &  &  &  &  &  &  &  &  &  &  &  &  &  &  &  &  &  \\
\hline
 &  &  &  &  &  &  &  &  &  &  &  &  &  &  &  &  &  &  &  &  &  &  \\
\hline
 &  &  &  &  &  &  &  &  &  &  &  &  &  &  &  &  &  &  &  &  &  &  \\
\hline
 &  &  &  &  &  &  &  &  &  &  &  &  &  &  &  &  &  &  &  &  &  &  \\
\hline
 &  &  &  &  &  &  &  &  &  &  &  &  &  &  &  &  &  &  &  &  &  &  \\
\hline
 &  &  &  &  &  &  &  &  &  &  &  &  &  &  &  &  &  &  &  &  &  &  \\
\hline
 &  &  &  &  &  &  &  &  &  &  &  &  &  &  &  &  &  &  &  &  &  &  \\
\hline
 &  &  &  &  &  &  &  &  &  &  &  &  &  &  &  &  &  &  &  &  &  &  \\
\hline
 &  &  &  &  &  &  &  &  &  &  &  &  &  &  &  &  &  &  &  &  &  &  \\
\hline
 &  &  &  &  &  &  &  &  &  &  &  &  &  &  &  &  &  &  &  &  &  &  \\
\hline
 &  &  &  &  &  &  &  &  &  &  &  &  &  &  &  &  &  &  &  &  &  &  \\
\hline
 &  &  &  &  &  &  &  &  &  &  &  &  &  &  &  &  &  &  &  &  &  &  \\
\hline
 &  &  &  &  &  &  &  &  &  &  &  &  &  &  &  &  &  &  &  &  &  &  \\
\hline
 &  &  &  &  &  &  &  &  &  &  &  &  &  &  &  &  &  &  &  &  &  &  \\
\hline
 &  &  &  &  &  &  &  &  &  &  &  &  &  &  &  &  &  &  &  &  &  &  \\
\hline
 &  &  &  &  &  &  &  &  &  &  &  &  &  &  &  &  &  &  &  &  &  &  \\
\hline
 &  &  &  &  &  &  &  &  &  &  &  &  &  &  &  &  &  &  &  &  &  &  \\
\hline
 &  &  &  &  &  &  &  &  &  &  &  &  &  &  &  &  &  &  &  &  &  &  \\
\hline
 &  &  &  &  &  &  &  &  &  &  &  &  &  &  &  &  &  &  &  &  &  &  \\
\hline
 &  &  &  &  &  &  &  &  &  &  &  &  &  &  &  &  &  &  &  &  &  &  \\
\hline
 &  &  &  &  &  &  &  &  &  &  &  &  &  &  &  &  &  &  &  &  &  &  \\
\hline
 &  &  &  &  &  &  &  &  &  &  &  &  &  &  &  &  &  &  &  &  &  &  \\
\hline
 &  &  &  &  &  &  &  &  &  &  &  &  &  &  &  &  &  &  &  &  &  &  \\
\hline
\end{tabular}
\end{center}

Odpowiedź:

\section*{Zadanie 29. (0-2)}
Kąt \(\alpha\) jest ostry i spełnia równość \(\operatorname{tg} \alpha+\frac{1}{\operatorname{tg} \alpha}=\frac{7}{2}\). Oblicz wartość wyrażenia \(\sin \alpha \cdot \cos \alpha\).

\begin{center}
\begin{tabular}{|c|c|c|c|c|c|c|c|c|c|c|c|c|c|c|c|c|c|c|c|c|c|c|c|c|c|c|c|c|}
\hline
 &  &  &  &  &  &  &  &  &  &  &  &  &  &  &  &  &  &  &  &  &  &  &  &  &  &  &  &  \\
\hline
 &  &  &  &  &  &  &  &  &  &  &  &  &  &  &  &  &  &  &  &  &  &  &  &  &  &  &  &  \\
\hline
 &  &  &  &  &  &  &  &  &  &  &  &  &  &  &  &  &  &  &  &  &  &  &  &  &  &  &  &  \\
\hline
 &  &  &  &  &  &  &  &  &  &  &  &  &  &  &  &  &  &  &  &  &  &  &  &  &  &  &  &  \\
\hline
 &  &  &  &  &  &  &  &  &  &  &  &  &  &  &  &  &  &  &  &  &  &  &  &  &  &  &  &  \\
\hline
 &  &  &  &  &  &  &  &  &  &  &  &  &  &  &  &  &  &  &  &  &  &  &  &  &  &  &  &  \\
\hline
 &  &  &  &  &  &  &  &  &  &  &  &  &  &  &  &  &  &  &  &  &  &  &  &  &  &  &  &  \\
\hline
 &  &  &  &  &  &  &  &  &  &  &  &  &  &  &  &  &  &  &  &  &  &  &  &  &  &  &  &  \\
\hline
 &  &  &  &  &  &  &  &  &  &  &  &  &  &  &  &  &  &  &  &  &  &  &  &  &  &  &  &  \\
\hline
 &  &  &  &  &  &  &  &  &  &  &  &  &  &  &  &  &  &  &  &  &  &  &  &  &  &  &  &  \\
\hline
 &  &  &  &  &  &  &  &  &  &  &  &  &  &  &  &  &  &  &  &  &  &  &  &  &  &  &  &  \\
\hline
 &  &  &  &  &  &  &  &  &  &  &  &  &  &  &  &  &  &  &  &  &  &  &  &  &  &  &  &  \\
\hline
 &  &  &  &  &  &  &  &  &  &  &  &  &  &  &  &  &  &  &  &  &  &  &  &  &  &  &  &  \\
\hline
 &  &  &  &  &  &  &  &  &  &  &  &  &  &  &  &  &  &  &  &  &  &  &  &  &  &  &  &  \\
\hline
 & - &  &  &  &  &  &  &  &  &  &  &  &  &  &  &  &  &  &  &  &  &  &  &  &  &  &  &  \\
\hline
 &  &  &  &  &  &  &  &  &  &  &  &  &  &  &  &  &  &  &  &  &  &  &  &  &  &  &  &  \\
\hline
 & \includegraphics[max width=\textwidth]{2025_02_10_74e27f4b95bc3ce8e9b7g-15(7)}
 &  &  &  &  &  &  &  &  &  &  &  &  &  &  &  &  &  &  &  &  &  &  &  &  &  &  &  \\
\hline
 &  &  &  &  &  &  &  &  &  &  &  &  &  &  &  &  &  &  &  &  &  &  &  &  &  &  &  &  \\
\hline
 &  &  &  &  &  &  &  &  &  &  &  &  &  &  &  &  &  &  &  &  &  &  &  &  &  &  &  &  \\
\hline
 & \includegraphics[max width=\textwidth]{2025_02_10_74e27f4b95bc3ce8e9b7g-15(6)}
 &  &  &  &  &  &  &  &  &  &  &  &  &  &  &  &  &  &  &  &  &  &  &  &  &  &  &  \\
\hline
 &  &  &  &  &  &  &  &  &  &  &  &  &  &  &  &  &  &  &  &  &  &  &  &  &  &  &  &  \\
\hline
 & \includegraphics[max width=\textwidth]{2025_02_10_74e27f4b95bc3ce8e9b7g-15(4)}
 &  &  &  &  &  &  &  &  &  &  &  &  &  &  &  &  &  &  &  &  &  &  &  &  &  &  &  \\
\hline
 & \includegraphics[max width=\textwidth]{2025_02_10_74e27f4b95bc3ce8e9b7g-15}
 &  &  &  &  &  &  &  &  &  &  &  &  &  &  &  &  &  &  &  &  &  &  &  &  &  &  &  \\
\hline
 & - &  &  &  &  &  &  &  &  &  &  &  &  &  &  &  &  &  &  &  &  &  &  &  &  &  &  &  \\
\hline
 & \includegraphics[max width=\textwidth]{2025_02_10_74e27f4b95bc3ce8e9b7g-15(2)}
 &  &  &  &  &  &  &  & \includegraphics[max width=\textwidth]{2025_02_10_74e27f4b95bc3ce8e9b7g-15(1)}
 &  &  &  &  &  &  &  &  &  &  &  &  &  &  &  &  &  &  &  \\
\hline
 & - &  &  &  &  &  &  &  &  &  &  &  &  &  &  &  &  &  &  &  &  &  &  &  &  &  &  &  \\
\hline
 & - &  &  &  &  &  &  &  & - &  &  & \includegraphics[max width=\textwidth]{2025_02_10_74e27f4b95bc3ce8e9b7g-15(5)}
 & \includegraphics[max width=\textwidth]{2025_02_10_74e27f4b95bc3ce8e9b7g-15(8)}
 &  &  &  &  &  &  &  &  &  &  &  &  &  &  &  \\
\hline
 & \includegraphics[max width=\textwidth]{2025_02_10_74e27f4b95bc3ce8e9b7g-15(3)}
 &  &  &  &  &  &  &  &  &  &  &  &  &  &  &  &  &  &  &  &  &  &  &  &  &  &  &  \\
\hline
 & - &  &  &  &  &  &  &  &  &  &  &  &  &  &  &  &  &  &  &  &  &  &  &  &  &  &  &  \\
\hline
 & . &  &  &  &  &  &  &  &  &  &  &  &  &  &  &  &  &  &  &  &  &  &  &  &  &  &  &  \\
\hline
 & , &  &  &  &  &  &  &  &  &  &  &  &  &  &  &  &  &  &  &  &  &  &  &  &  &  &  &  \\
\hline
 &  &  &  &  &  &  &  &  &  &  &  &  &  &  &  &  &  &  &  &  &  &  &  &  &  &  &  &  \\
\hline
 & - &  &  &  &  &  &  &  &  &  &  &  &  &  &  &  &  &  &  &  &  &  &  &  &  &  &  &  \\
\hline
 & - &  &  &  &  &  &  &  &  &  &  &  &  &  &  &  &  &  &  &  &  &  &  &  &  &  &  &  \\
\hline
 & - &  &  &  &  &  &  &  &  &  &  &  &  &  &  &  &  &  &  &  &  &  &  &  &  &  &  &  \\
\hline
 & - &  &  &  &  &  &  &  &  &  &  &  &  &  &  &  &  &  &  &  &  &  &  &  &  &  &  &  \\
\hline
 &  &  &  &  &  &  &  &  &  &  &  &  &  &  &  &  &  &  &  &  &  &  &  &  &  &  &  &  \\
\hline
 &  &  &  &  &  &  &  &  &  &  &  &  &  &  &  &  &  &  &  &  &  &  &  &  &  &  &  &  \\
\hline
\end{tabular}
\end{center}

Odpowiedź:

\begin{center}
\begin{tabular}{|c|l|c|c|}
\hline
\multirow{3}{*}{\begin{tabular}{l}
Wypelnia \\
egzaminator \\
\end{tabular}} & Nr zadania & 28. & 29. \\
\cline { 2 - 4 }
 & Maks. liczba pkt & 2 & 2 \\
\cline { 2 - 4 }
 & Uzyskana liczba pkt &  &  \\
\hline
\end{tabular}
\end{center}

Zadanie 30. (0-2)\\
Wykaż, że dla wszystkich nieujemnych liczb rzeczywistych \(x, y\) prawdziwa jest nierówność \(x^{3}+y^{3} \geq x^{2} y+x y^{2}\).

\begin{center}
\begin{tabular}{|c|c|c|c|c|c|c|c|c|c|c|c|c|c|c|c|c|c|c|c|c|c|}
\hline
 &  &  &  &  &  &  &  &  &  &  &  &  &  &  &  &  &  &  &  &  &  \\
\hline
 &  &  &  &  &  &  &  &  &  &  &  &  &  &  &  &  &  &  &  &  &  \\
\hline
 &  &  &  &  &  &  &  &  &  &  &  &  &  &  &  &  &  &  &  &  &  \\
\hline
 &  &  &  &  &  &  &  &  &  &  &  &  &  &  &  &  &  &  &  &  &  \\
\hline
 &  &  &  &  &  &  &  &  &  &  &  &  &  &  &  &  &  &  &  &  &  \\
\hline
 &  &  &  &  &  &  &  &  &  &  &  &  &  &  &  &  &  &  &  &  &  \\
\hline
 &  &  &  &  &  &  &  &  &  &  &  &  &  &  &  &  &  &  &  &  &  \\
\hline
 &  &  &  &  &  &  &  &  &  &  &  &  &  &  &  &  &  &  &  &  &  \\
\hline
 &  &  &  &  &  &  &  &  &  &  &  &  &  &  &  &  &  &  &  &  &  \\
\hline
 &  &  &  &  &  &  &  &  &  &  &  &  &  &  &  &  &  &  &  &  &  \\
\hline
 &  &  &  &  &  &  &  &  &  &  &  &  &  &  &  &  &  &  &  &  &  \\
\hline
 &  &  &  &  &  &  &  &  &  &  &  &  &  &  &  &  &  &  &  &  &  \\
\hline
 &  &  &  &  &  &  &  &  &  &  &  &  &  &  &  &  &  &  &  &  &  \\
\hline
 &  &  &  &  &  &  &  &  &  &  &  &  &  &  &  &  &  &  &  &  &  \\
\hline
 &  &  &  &  &  &  &  &  &  &  &  &  &  &  &  &  &  &  &  &  &  \\
\hline
 &  &  &  &  &  &  &  &  &  &  &  &  &  &  &  &  &  &  &  &  &  \\
\hline
 &  &  &  &  &  &  &  &  &  &  &  &  &  &  &  &  &  &  &  &  &  \\
\hline
 &  &  &  &  &  &  &  &  &  &  &  &  &  &  &  &  &  &  &  &  &  \\
\hline
 &  &  &  &  &  &  &  &  &  &  &  &  &  &  &  &  &  &  &  &  &  \\
\hline
 &  &  &  &  &  &  &  &  &  &  &  &  &  &  &  &  &  &  &  &  &  \\
\hline
 &  &  &  &  &  &  &  &  &  &  &  &  &  &  &  &  &  &  &  &  &  \\
\hline
 &  &  &  &  &  &  &  &  &  &  &  &  &  &  &  &  &  &  &  &  &  \\
\hline
 &  &  &  &  &  &  &  &  &  &  &  &  &  &  &  &  &  &  &  &  &  \\
\hline
 &  &  &  &  &  &  &  &  &  &  &  &  &  &  &  &  &  &  &  &  &  \\
\hline
 &  &  &  &  &  &  &  &  &  &  &  &  &  &  &  &  &  &  &  &  &  \\
\hline
 &  &  &  &  &  &  &  &  &  &  &  &  &  &  &  &  &  &  &  &  &  \\
\hline
 &  &  &  &  &  &  &  &  &  &  &  &  &  &  &  &  &  &  &  &  &  \\
\hline
 &  &  &  &  &  &  &  &  &  &  &  &  &  &  &  &  &  &  &  &  &  \\
\hline
 &  &  &  &  &  &  &  &  &  &  &  &  &  &  &  &  &  &  &  &  &  \\
\hline
 &  &  &  &  &  &  &  &  &  &  &  &  &  &  &  &  &  &  &  &  &  \\
\hline
 &  &  &  &  &  &  &  &  &  &  &  &  &  &  &  &  &  &  &  &  &  \\
\hline
 &  &  &  &  &  &  &  &  &  &  &  &  &  &  &  &  &  &  &  &  &  \\
\hline
 &  &  &  &  &  &  &  &  &  &  &  &  &  &  &  &  &  &  &  &  &  \\
\hline
 &  &  &  &  &  &  &  &  &  &  &  &  &  &  &  &  &  &  &  &  &  \\
\hline
 &  &  &  &  &  &  &  &  &  &  &  &  &  &  &  &  &  &  &  &  &  \\
\hline
 &  &  &  &  &  &  &  &  &  &  &  &  &  &  &  &  &  &  &  &  &  \\
\hline
 &  &  &  &  &  &  &  &  &  &  &  &  &  &  &  &  &  &  &  &  &  \\
\hline
 &  &  &  &  &  &  &  &  &  &  &  &  &  &  &  &  &  &  &  &  &  \\
\hline
 &  &  &  &  &  &  &  &  &  &  &  &  &  &  &  &  &  &  &  &  &  \\
\hline
 &  &  &  &  &  &  &  &  &  &  &  &  &  &  &  &  &  &  &  &  &  \\
\hline
 &  &  &  &  &  &  &  &  &  &  &  &  &  &  &  &  &  &  &  &  &  \\
\hline
 &  &  &  &  &  &  &  &  &  &  &  &  &  &  &  &  &  &  &  &  &  \\
\hline
 &  &  &  &  &  &  &  &  &  &  &  &  &  &  &  &  &  &  &  &  &  \\
\hline
 &  &  &  &  &  &  &  &  &  &  &  &  &  &  &  &  &  &  &  &  &  \\
\hline
\end{tabular}
\end{center}

\section*{Zadanie 31. (0-2)}
W prostokącie \(A B C D\) punkt \(P\) jest środkiem boku \(B C\), a punkt \(R\) jest środkiem boku \(C D\). Wykaż, że pole trójkąta \(A P R\) jest równe sumie pól trójkątów \(A D R\) oraz \(P C R\).\\
\includegraphics[max width=\textwidth, center]{2025_02_10_74e27f4b95bc3ce8e9b7g-17}

\begin{center}
\begin{tabular}{|c|l|c|c|}
\hline
\multirow{3}{*}{\begin{tabular}{l}
Wypetnia \\
egzaminator \\
\end{tabular}} & Nr zadania & \(\mathbf{3 0 .}\) & 31. \\
\cline { 2 - 4 }
 & Maks. liczba pkt & 2 & 2 \\
\cline { 2 - 4 }
 & Uzyskana liczba pkt &  &  \\
\hline
\end{tabular}
\end{center}

\section*{Zadanie 32. (0-4)}
Wyznacz równanie osi symetrii trójkąta o wierzchołkach \(A=(-2,2), \quad B=(6,-2)\), \(C=(10,6)\).

\begin{center}
\begin{tabular}{|c|c|c|c|c|c|c|c|c|c|c|c|c|c|c|c|c|c|c|c|c|c|c|}
\hline
 &  &  &  &  &  &  &  &  &  &  &  &  &  &  &  &  &  &  &  &  &  &  \\
\hline
 &  &  &  &  &  &  &  &  &  &  &  &  &  &  &  &  &  &  &  &  &  &  \\
\hline
 &  &  &  &  &  &  &  &  &  &  &  &  &  &  &  &  &  &  &  &  &  &  \\
\hline
 &  &  &  &  &  &  &  &  &  &  &  &  &  &  &  &  &  &  &  &  &  &  \\
\hline
 &  &  &  &  &  &  &  &  &  &  &  &  &  &  &  &  &  &  &  &  &  &  \\
\hline
 &  &  &  &  &  &  &  &  &  &  &  &  &  &  &  &  &  &  &  &  &  &  \\
\hline
 &  &  &  &  &  &  &  &  &  &  &  &  &  &  &  &  &  &  &  &  &  &  \\
\hline
 &  &  &  &  &  &  &  &  &  &  &  &  &  &  &  &  &  &  &  &  &  &  \\
\hline
 &  &  &  &  &  &  &  &  &  &  &  &  &  &  &  &  &  &  &  &  &  &  \\
\hline
 &  &  &  &  &  &  &  &  &  &  &  &  &  &  &  &  &  &  &  &  &  &  \\
\hline
 &  &  &  &  &  &  &  &  &  &  &  &  &  &  &  &  &  &  &  &  &  &  \\
\hline
 &  &  &  &  &  &  &  &  &  &  &  &  &  &  &  &  &  &  &  &  &  &  \\
\hline
 &  &  &  &  &  &  &  &  &  &  &  &  &  &  &  &  &  &  &  &  &  &  \\
\hline
 &  &  &  &  &  &  &  &  &  &  &  &  &  &  &  &  &  &  &  &  &  &  \\
\hline
 &  &  &  &  &  &  &  &  &  &  &  &  &  &  &  &  &  &  &  &  &  &  \\
\hline
 &  &  &  &  &  &  &  &  &  &  &  &  &  &  &  &  &  &  &  &  &  &  \\
\hline
 &  &  &  &  &  &  &  &  &  &  &  &  &  &  &  &  &  &  &  &  &  &  \\
\hline
 &  &  &  &  &  &  &  &  &  &  &  &  &  &  &  &  &  &  &  &  &  &  \\
\hline
 &  &  &  &  &  &  &  &  &  &  &  &  &  &  &  &  &  &  &  &  &  &  \\
\hline
 &  &  &  &  &  &  &  &  &  &  &  &  &  &  &  &  &  &  &  &  &  &  \\
\hline
 &  &  &  &  &  &  &  &  &  &  &  &  &  &  &  &  &  &  &  &  &  &  \\
\hline
 &  &  &  &  &  &  &  &  &  &  &  &  &  &  &  &  &  &  &  &  &  &  \\
\hline
 &  &  &  &  &  &  &  &  &  &  &  &  &  &  &  &  &  &  &  &  &  &  \\
\hline
 &  &  &  &  &  &  &  &  &  &  &  &  &  &  &  &  &  &  &  &  &  &  \\
\hline
 &  &  &  &  &  &  &  &  &  &  &  &  &  &  &  &  &  &  &  &  &  &  \\
\hline
 &  &  &  &  &  &  &  &  &  &  &  &  &  &  &  &  &  &  &  &  &  &  \\
\hline
 &  &  &  &  &  &  &  &  &  &  &  &  &  &  &  &  &  &  &  &  &  &  \\
\hline
 &  &  &  &  &  &  &  &  &  &  &  &  &  &  &  &  &  &  &  &  &  &  \\
\hline
 &  &  &  &  &  &  &  &  &  &  &  &  &  &  &  &  &  &  &  &  &  &  \\
\hline
 &  &  &  &  &  &  &  &  &  &  &  &  &  &  &  &  &  &  &  &  &  &  \\
\hline
 &  &  &  &  &  &  &  &  &  &  &  &  &  &  &  &  &  &  &  &  &  &  \\
\hline
 &  &  &  &  &  &  &  &  &  &  &  &  &  &  &  &  &  &  &  &  &  &  \\
\hline
 &  &  &  &  &  &  &  &  &  &  &  &  &  &  &  &  &  &  &  &  &  &  \\
\hline
 &  &  &  &  &  &  &  &  &  &  &  &  &  &  &  &  &  &  &  &  &  &  \\
\hline
 &  &  &  &  &  &  &  &  &  &  &  &  &  &  &  &  &  &  &  &  &  &  \\
\hline
 &  &  &  &  &  &  &  &  &  &  &  &  &  &  &  &  &  &  &  &  &  &  \\
\hline
 &  &  &  &  &  &  &  &  &  &  &  &  &  &  &  &  &  &  &  &  &  &  \\
\hline
 &  &  &  &  &  &  &  &  &  &  &  &  &  &  &  &  &  &  &  &  &  &  \\
\hline
 &  &  &  &  &  &  &  &  &  &  &  &  &  &  &  &  &  &  &  &  &  &  \\
\hline
 &  &  &  &  &  &  &  &  &  &  &  &  &  &  &  &  &  &  &  &  &  &  \\
\hline
 &  &  &  &  &  &  &  &  &  &  &  &  &  &  &  &  &  &  &  &  &  &  \\
\hline
 &  &  &  &  &  &  &  &  &  &  &  &  &  &  &  &  &  &  &  &  &  &  \\
\hline
 &  &  &  &  &  &  &  &  &  &  &  &  &  &  &  &  &  &  &  &  &  &  \\
\hline
 &  &  &  &  &  &  &  &  &  &  &  &  &  &  &  &  &  &  &  &  &  &  \\
\hline
\end{tabular}
\end{center}

\begin{center}
\includegraphics[max width=\textwidth]{2025_02_10_74e27f4b95bc3ce8e9b7g-19}
\end{center}

Odpowiedź: \(\qquad\)

\begin{center}
\begin{tabular}{|c|l|c|}
\hline
\multirow{3}{*}{\begin{tabular}{l}
Wypelnia \\
egzaminator \\
\end{tabular}} & Nr zadania & 32. \\
\cline { 2 - 3 }
 & Maks. liczba pkt & 4 \\
\cline { 2 - 3 }
 & Uzyskana liczba pkt &  \\
\hline
\end{tabular}
\end{center}

\section*{Zadanie 33. (0-4)}
Podstawą ostrosłupa \(A B C D S\) jest prostokąt, którego boki pozostają w stosunku 3:4, a pole jest równe 192 (zobacz rysunek). Punkt \(E\) jest wyznaczony przez przecinające się przekątne podstawy, a odcinek \(S E\) jest wysokością ostrosłupa. Każda krawędź boczna tego ostrosłupa jest nachylona do płaszczyzny podstawy pod kątem \(30^{\circ}\). Oblicz objętość ostrosłupa.\\
\includegraphics[max width=\textwidth, center]{2025_02_10_74e27f4b95bc3ce8e9b7g-20}\\
\includegraphics[max width=\textwidth, center]{2025_02_10_74e27f4b95bc3ce8e9b7g-21}

Odpowiedź: \(\qquad\)

\begin{center}
\begin{tabular}{|c|l|c|}
\hline
\multirow{3}{*}{\begin{tabular}{l}
Wypelnia \\
egzaminator \\
\end{tabular}} & Nr zadania & 33. \\
\cline { 2 - 3 }
 & Maks. liczba pkt & 4 \\
\cline { 2 - 3 }
 & Uzyskana liczba pkt &  \\
\hline
\end{tabular}
\end{center}

\section*{Zadanie 34. (0-5)}
Funkcja kwadratowa \(f\) określona jest wzorem \(f(x)=a x^{2}+b x+c\). Zbiorem rozwiązań nierówności \(f(x)>0\) jest przedział \((0,12)\). Największa wartość funkcji \(f\) jest równa 9 . Oblicz współczynniki \(a, b\) i \(c\) funkcji \(f\).

\begin{center}
\begin{tabular}{|c|c|c|c|c|c|c|c|c|c|c|c|c|c|c|c|c|c|c|c|c|c|}
\hline
 &  &  &  &  &  &  &  &  &  &  &  &  &  &  &  &  &  &  &  &  &  \\
\hline
 &  &  &  &  &  &  &  &  &  &  &  &  &  &  &  &  &  &  &  &  &  \\
\hline
 &  &  &  &  &  &  &  &  &  &  &  &  &  &  &  &  &  &  &  &  &  \\
\hline
 &  &  &  &  &  &  &  &  &  &  &  &  &  &  &  &  &  &  &  &  &  \\
\hline
 &  &  &  &  &  &  &  &  &  &  &  &  &  &  &  &  &  &  &  &  &  \\
\hline
 &  &  &  &  &  &  &  &  &  &  &  &  &  &  &  &  &  &  &  &  &  \\
\hline
 &  &  &  &  &  &  &  &  &  &  &  &  &  &  &  &  &  &  &  &  &  \\
\hline
 &  &  &  &  &  &  &  &  &  &  &  &  &  &  &  &  &  &  &  &  &  \\
\hline
 &  &  &  &  &  &  &  &  &  &  &  &  &  &  &  &  &  &  &  &  &  \\
\hline
 &  &  &  &  &  &  &  &  &  &  &  &  &  &  &  &  &  &  &  &  &  \\
\hline
 &  &  &  &  &  &  &  &  &  &  &  &  &  &  &  &  &  &  &  &  &  \\
\hline
 &  &  &  &  &  &  &  &  &  &  &  &  &  &  &  &  &  &  &  &  &  \\
\hline
 &  &  &  &  &  &  &  &  &  &  &  &  &  &  &  &  &  &  &  &  &  \\
\hline
 &  &  &  &  &  &  &  &  &  &  &  &  &  &  &  &  &  &  &  &  &  \\
\hline
 &  &  &  &  &  &  &  &  &  &  &  &  &  &  &  &  &  &  &  &  &  \\
\hline
 &  &  &  &  &  &  &  &  &  &  &  &  &  &  &  &  &  &  &  &  &  \\
\hline
 &  &  &  &  &  &  &  &  &  &  &  &  &  &  &  &  &  &  &  &  &  \\
\hline
 &  &  &  &  &  &  &  &  &  &  &  &  &  &  &  &  &  &  &  &  &  \\
\hline
 &  &  &  &  &  &  &  &  &  &  &  &  &  &  &  &  &  &  &  &  &  \\
\hline
 &  &  &  &  &  &  &  &  &  &  &  &  &  &  &  &  &  &  &  &  &  \\
\hline
 &  &  &  &  &  &  &  &  &  &  &  &  &  &  &  &  &  &  &  &  &  \\
\hline
 &  &  &  &  &  &  &  &  &  &  &  &  &  &  &  &  &  &  &  &  &  \\
\hline
 &  &  &  &  &  &  &  &  &  &  &  &  &  &  &  &  &  &  &  &  &  \\
\hline
 &  &  &  &  &  &  &  &  &  &  &  &  &  &  &  &  &  &  &  &  &  \\
\hline
 &  &  &  &  &  &  &  &  &  &  &  &  &  &  &  &  &  &  &  &  &  \\
\hline
 &  &  &  &  &  &  &  &  &  &  &  &  &  &  &  &  &  &  &  &  &  \\
\hline
 &  &  &  &  &  &  &  &  &  &  &  &  &  &  &  &  &  &  &  &  &  \\
\hline
 &  &  &  &  &  &  &  &  &  &  &  &  &  &  &  &  &  &  &  &  &  \\
\hline
 &  &  &  &  &  &  &  &  &  &  &  &  &  &  &  &  &  &  &  &  &  \\
\hline
 &  &  &  &  &  &  &  &  &  &  &  &  &  &  &  &  &  &  &  &  &  \\
\hline
 &  &  &  &  &  &  &  &  &  &  &  &  &  &  &  &  &  &  &  &  &  \\
\hline
 &  &  &  &  &  &  &  &  &  &  &  &  &  &  &  &  &  &  &  &  &  \\
\hline
 &  &  &  &  &  &  &  &  &  &  &  &  &  &  &  &  &  &  &  &  &  \\
\hline
 &  &  &  &  &  &  &  &  &  &  &  &  &  &  &  &  &  &  &  &  &  \\
\hline
 &  &  &  &  &  &  &  &  &  &  &  &  &  &  &  &  &  &  &  &  &  \\
\hline
 &  &  &  &  &  &  &  &  &  &  &  &  &  &  &  &  &  &  &  &  &  \\
\hline
 &  &  &  &  &  &  &  &  &  &  &  &  &  &  &  &  &  &  &  &  &  \\
\hline
 &  &  &  &  &  &  &  &  &  &  &  &  &  &  &  &  &  &  &  &  &  \\
\hline
 &  &  &  &  &  &  &  &  &  &  &  &  &  &  &  &  &  &  &  &  &  \\
\hline
 &  &  &  &  &  &  &  &  &  &  &  &  &  &  &  &  &  &  &  &  &  \\
\hline
 &  &  &  &  &  &  &  &  &  &  &  &  &  &  &  &  &  &  &  &  &  \\
\hline
 &  &  &  &  &  &  &  &  &  &  &  &  &  &  &  &  &  &  &  &  &  \\
\hline
 &  &  &  &  &  &  &  &  &  &  &  &  &  &  &  &  &  &  &  &  &  \\
\hline
\end{tabular}
\end{center}

\begin{center}
\includegraphics[max width=\textwidth]{2025_02_10_74e27f4b95bc3ce8e9b7g-23}
\end{center}

Odpowiedź: \(\qquad\)

\begin{center}
\begin{tabular}{|c|l|c|}
\hline
\multirow{3}{*}{\begin{tabular}{l}
Wypetnia \\
egzaminator \\
\end{tabular}} & Nr zadania & 34. \\
\cline { 2 - 3 }
 & Maks. liczba pkt & 5 \\
\cline { 2 - 3 }
 & Uzyskana liczba pkt &  \\
\hline
\end{tabular}
\end{center}

\section*{BRUDNOPIS (nie podlega ocenie)}

\end{document}