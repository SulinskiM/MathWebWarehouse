\documentclass[10pt]{article}
\usepackage[polish]{babel}
\usepackage[utf8]{inputenc}
\usepackage[T1]{fontenc}
\usepackage{graphicx}
\usepackage[export]{adjustbox}
\graphicspath{ {./images/} }
\usepackage{amsmath}
\usepackage{amsfonts}
\usepackage{amssymb}
\usepackage[version=4]{mhchem}
\usepackage{stmaryrd}

\author{DATA: 23 sierpnia 2016 r.\\
Godzina rozpoczęcia: 9:00\\
Czas pracy: 170 minut\\
LicZba punktów do uzyskania: 50}
\date{}


\newcommand\Varangle{\mathop{{<\!\!\!\!\!\text{\small)}}\:}\nolimits}

\begin{document}
\maketitle
\begin{center}
\includegraphics[max width=\textwidth]{2025_02_10_932b254328dc20198cfdg-01}
\end{center}



\section*{Instrukcja dla zdającego}
\begin{enumerate}
  \item Sprawdź, czy arkusz egzaminacyjny zawiera 23 strony (zadania 1-34). Ewentualny brak zgłoś przewodniczącemu zespołu nadzorującego egzamin.
  \item Rozwiązania zadań i odpowiedzi wpisuj w miejscu na to przeznaczonym.
  \item Odpowiedzi do zadań zamkniętych (1-25) zaznacz na karcie odpowiedzi, w części karty przeznaczonej dla zdającego. Zamaluj \(\square\) pola do tego przeznaczone. Błędne zaznaczenie otocz kółkiem \({ }^{\text {i zaznacz właściwe }}\).
  \item Pamiętaj, że pominięcie argumentacji lub istotnych obliczeń w rozwiązaniu zadania otwartego (26-34) może spowodować, że za to rozwiązanie nie otrzymasz pełnej liczby punktów.
  \item Pisz czytelnie i używaj tylko długopisu lub pióra z czarnym tuszem lub atramentem.
  \item Nie używaj korektora, a błędne zapisy wyraźnie przekreśl.
  \item Pamiętaj, że zapisy w brudnopisie nie będą oceniane.
  \item Możesz korzystać z zestawu wzorów matematycznych, cyrkla i linijki, a także z kalkulatora prostego.
  \item Na tej stronie oraz na karcie odpowiedzi wpisz swój numer PESEL i przyklej naklejkę z kodem.
  \item Nie wpisuj żadnych znaków w częşci przeznaczonej dla egzaminatora.\\
\includegraphics[max width=\textwidth, center]{2025_02_10_932b254328dc20198cfdg-01(1)}
\end{enumerate}

W zadaniach od 1. do 25. wybierz i zaznacz na karcie odpowiedzi poprawna odpowiedź.

\section*{Zadanie 1. (0-1)}
Suma pięciu kolejnych liczb całkowitych jest równa 195. Najmniejszą z tych liczb jest\\
A. 37\\
B. 38\\
C. 39\\
D. 40

\section*{Zadanie 2. (0-1)}
Buty, które kosztowały 220 złotych, przeceniono i sprzedano za 176 złotych. O ile procent obniżono cenę butów?\\
A. 80\\
B. 20\\
C. 22\\
D. 44

\section*{Zadanie 3. (0-1)}
Liczba \(\frac{4^{5} \cdot 5^{4}}{20^{4}}\) jest równa\\
A. \(4^{4}\)\\
B. \(20^{16}\)\\
C. \(20^{5}\)\\
D. 4

\section*{Zadanie 4. (0-1)}
Liczba \(\frac{\log _{3} 729}{\log _{6} 36}\) jest równa\\
A. \(\log _{6} 693\)\\
B. 3\\
C. \(\log _{\frac{1}{2}} \frac{81}{4}\)\\
D. 4

\section*{Zadanie 5. (0-1)}
Najmniejszą liczbą całkowitą spełniającą nierówność \(\frac{x}{5}+\sqrt{7}>0\) jest\\
A. -14\\
B. -13\\
C. 13\\
D. 14

\section*{Zadanie 6. (0-1)}
Funkcja kwadratowa jest określona wzorem \(f(x)=(x-1)(x-9)\). Wynika stąd, że funkcja \(f\) jest rosnąca w przedziale\\
A. \(\langle 5,+\infty)\)\\
B. \((-\infty, 5\rangle\)\\
C. \((-\infty,-5\rangle\)\\
D. \((-5,+\infty)\)

BRUDNOPIS (nie podlega ocenie)\\
\includegraphics[max width=\textwidth, center]{2025_02_10_932b254328dc20198cfdg-03}

\section*{Zadanie 7. (0-1)}
Na rysunku przedstawiony jest fragment wykresu funkcji liniowej \(f\), przy czym \(f(0)=-2\) i \(f(1)=0\).\\
\includegraphics[max width=\textwidth, center]{2025_02_10_932b254328dc20198cfdg-04}

Wykres funkcji \(g\) jest symetryczny do wykresu funkcji \(f\) względem początku układu współrzędnych. Funkcja \(g\) jest określona wzorem\\
A. \(g(x)=2 x+2\)\\
B. \(g(x)=2 x-2\)\\
C. \(g(x)=-2 x+2\)\\
D. \(g(x)=-2 x-2\)

\section*{Zadanie 8. (0-1)}
Pierwszy wyraz ciągu geometrycznego jest równy 8, a czwarty wyraz tego ciągu jest równy (-216). Iloraz tego ciągu jest równy\\
A. \(-\frac{224}{3}\)\\
B. -3\\
C. -9\\
D. -27

\section*{Zadanie 9. (0-1)}
Kąt \(\alpha\) jest ostry i \(\sin \alpha=\frac{4}{5}\). Wtedy wartość wyrażenia \(\sin \alpha-\cos \alpha\) jest równa\\
A. \(\frac{1}{5}\)\\
B. \(\frac{3}{5}\)\\
C. \(\frac{17}{25}\)\\
D. \(\frac{1}{25}\)

\section*{Zadanie 10. (0-1)}
Jeśli funkcja kwadratowa \(f(x)=x^{2}+2 x+3 a\) nie ma ani jednego miejsca zerowego, to liczba \(a\) spełnia warunek\\
A. \(a<-1\)\\
B. \(-1 \leq a<0\)\\
C. \(0 \leq a<\frac{1}{3}\)\\
D. \(\quad a>\frac{1}{3}\)

BRUDNOPIS (nie podlega ocenie)\\
\includegraphics[max width=\textwidth, center]{2025_02_10_932b254328dc20198cfdg-05}

\section*{Zadanie 11. (0-1)}
Dla każdej liczby całkowitej dodatniej \(n\) suma \(n\) początkowych wyrazów ciągu arytmetycznego \(\left(a_{n}\right)\) jest określona wzorem \(S_{n}=2 n^{2}+n\). Wtedy wyraz \(a_{2}\) jest równy\\
A. 3\\
B. 6\\
C. 7\\
D. 10

\section*{Zadanie 12. (0-1)}
Układ równań \(\left\{\begin{aligned} 2 x-3 y & =5 \\ -4 x+6 y & =-10\end{aligned}\right.\)\\
A. nie ma rozwiązań.\\
B. ma dokładnie jedno rozwiązanie.\\
C. ma dokładnie dwa rozwiązania.\\
D. ma nieskończenie wiele rozwiązań.

\section*{Zadanie 13. (0-1)}
Liczba \(\frac{|3-9|}{-3}\) jest równa\\
A. 2\\
B. -2\\
C. 0\\
D. -4

\section*{Zadanie 14. (0-1)}
Na której z podanych prostych leżą wszystkie punkty o współrzędnych \((m-1,2 m+5)\), gdzie \(m\) jest dowolną liczbą rzeczywistą?\\
A. \(y=2 x+5\)\\
B. \(y=2 x+6\)\\
C. \(y=2 x+7\)\\
D. \(y=2 x+8\)

\section*{Zadanie 15. (0-1)}
Kąt rozwarcia stożka ma miarę \(120^{\circ}\), a tworząca tego stożka ma długość 6. Promień podstawy stożka jest równy\\
A. 3\\
B. 6\\
C. \(3 \sqrt{3}\)\\
D. \(6 \sqrt{3}\)

\section*{Zadanie 16. (0-1)}
Wartość wyrażenia \(\left(\operatorname{tg} 60^{\circ}+\operatorname{tg} 45^{\circ}\right)^{2}-\sin 60^{\circ}\) jest równa\\
A. \(2-\frac{3 \sqrt{3}}{2}\)\\
B. \(2+\frac{\sqrt{3}}{2}\)\\
C. \(4-\frac{\sqrt{3}}{2}\)\\
D. \(4+\frac{3 \sqrt{3}}{2}\)

BRUDNOPIS (nie podlega ocenie)\\
\includegraphics[max width=\textwidth, center]{2025_02_10_932b254328dc20198cfdg-07}

\section*{Zadanie 17. (0-1)}
Dany jest walec, w którym promień podstawy jest równy \(r\), a wysokość walca jest od tego promienia dwa razy większa. Objętość tego walca jest równa\\
A. \(2 \pi r^{3}\)\\
B. \(4 \pi r^{3}\)\\
C. \(\pi r^{2}(r+2)\)\\
D. \(\pi r^{2}(r-2)\)

\section*{Zadanie 18. (0-1)}
Przekątne równoległoboku mają długości 4 i 8, a kąt między tymi przekątnymi ma miarę \(30^{\circ}\). Pole tego równoległoboku jest równe\\
A. 32\\
B. 16\\
C. 12\\
D. 8

\section*{Zadanie 19. (0-1)}
Punkty \(A, B, C\) i \(D\) leżą na okręgu o środku \(S\). Cięciwa \(C D\) przecina średnicę \(A B\) tego okręgu w punkcie \(E\) tak, że \(|\Varangle B E C|=100^{\circ}\). Kąt środkowy \(A S C\) ma miarę \(110^{\circ}\) (zobacz rysunek).\\
\includegraphics[max width=\textwidth, center]{2025_02_10_932b254328dc20198cfdg-08}

Kąt wpisany \(B A D\) ma miarę\\
A. \(15^{\circ}\)\\
B. \(20^{\circ}\)\\
C. \(25^{\circ}\)\\
D. \(30^{\circ}\)

\section*{Zadanie 20. (0-1)}
Okręgi o środkach \(S_{1}=(3,4)\) oraz \(S_{2}=(9,-4)\) i równych promieniach są styczne zewnętrznie. Promień każdego z tych okręgów jest równy\\
A. 8\\
B. 6\\
C. 5\\
D. \(\frac{5}{2}\)

BRUDNOPIS (nie podlega ocenie)\\
\includegraphics[max width=\textwidth, center]{2025_02_10_932b254328dc20198cfdg-09}

\section*{Zadanie 21. (0-1)}
Podstawą graniastosłupa prawidłowego czworokątnego jest kwadrat o boku długości 2, a przekątna ściany bocznej ma długość 3 (zobacz rysunek). Kąt, jaki tworzą przekątne ścian bocznych tego graniastosłupa wychodzące z jednego wierzchołka, ma miarę \(\alpha\).\\
\includegraphics[max width=\textwidth, center]{2025_02_10_932b254328dc20198cfdg-10}

Wtedy wartość \(\sin \frac{\alpha}{2}\) jest równa\\
A. \(\frac{2}{3}\)\\
B. \(\frac{\sqrt{7}}{3}\)\\
C. \(\frac{\sqrt{7}}{7}\)\\
D. \(\frac{\sqrt{2}}{3}\)

\section*{Zadanie 22. (0-1)}
Różnica liczby krawędzi i liczby wierzchołków ostrosłupa jest równa 11. Podstawą tego ostrosłupa jest\\
A. dziesięciokąt.\\
B. jedenastokąt.\\
C. dwunastokąt.\\
D. trzynastokąt.

\section*{Zadanie 23. (0-1)}
Jeżeli do zestawu czterech danych: 4, 7, 8, \(x\) dołączymy liczbę 2 , to średnia arytmetyczna wzrośnie o 2. Zatem\\
A. \(x=-51\)\\
B. \(x=-6\)\\
C. \(x=10\)\\
D. \(x=29\)

\section*{Zadanie 24. (0-1)}
Ile jest wszystkich dwucyfrowych liczb naturalnych podzielnych przez 3?\\
A. 12\\
B. 24\\
C. 29\\
D. 30

\section*{Zadanie 25. (0-1)}
Doświadczenie losowe polega na rzucie dwiema symetrycznymi monetami i sześcienną kostką do gry. Prawdopodobieństwo zdarzenia polegającego na tym, że wynikiem rzutu są dwa orły i sześć oczek na kostce, jest równe\\
A. \(\frac{1}{48}\)\\
B. \(\frac{1}{24}\)\\
C. \(\frac{1}{12}\)\\
D. \(\frac{1}{3}\)

BRUDNOPIS (nie podlega ocenie)\\
\includegraphics[max width=\textwidth, center]{2025_02_10_932b254328dc20198cfdg-11}

Zadanie 26. (0-2)\\
Rozwiąż nierówność \(3 x^{2}-6 x \geq(x-2)(x-8)\).\\
\includegraphics[max width=\textwidth, center]{2025_02_10_932b254328dc20198cfdg-12}

Odpowiedź:

\section*{Zadanie 27. (0-2)}
Jeżeli do licznika pewnego nieskracalnego ułamka dodamy 32, a mianownik pozostawimy niezmieniony, to otrzymamy liczbę 2 . Jeżeli natomiast od licznika i od mianownika tego ułamka odejmiemy 6 , to otrzymamy liczbe \(\frac{8}{17}\). Wyznacz ten ułamek.\\
\includegraphics[max width=\textwidth, center]{2025_02_10_932b254328dc20198cfdg-13}

Odpowiedź: \(\qquad\)

Zadanie 28. (0-2)\\
Wykaż, że jeżeli liczby rzeczywiste \(a, b, c\) spełniają warunek \(a b c=1\), to\\
\(a^{-1}+b^{-1}+c^{-1}=a b+a c+b c\).

\begin{center}
\begin{tabular}{|c|c|c|c|c|c|c|c|c|c|c|c|c|c|c|c|c|c|c|c|c|c|c|c|c|c|c|c|}
\hline
 &  &  &  &  &  &  &  &  &  &  &  &  &  &  &  &  &  &  &  &  &  &  &  &  &  &  &  \\
\hline
 &  &  &  &  &  &  &  &  &  &  &  &  &  &  &  &  &  &  &  &  &  &  &  &  &  &  &  \\
\hline
 &  &  &  &  &  &  &  &  &  &  &  &  &  &  &  &  &  &  &  &  &  &  &  &  &  &  &  \\
\hline
 &  &  &  &  &  &  &  &  &  &  &  &  &  &  &  &  &  &  &  &  &  &  &  &  &  &  &  \\
\hline
 &  &  &  &  &  &  &  &  &  &  &  &  &  &  &  &  &  &  &  &  &  &  &  &  &  &  &  \\
\hline
 &  &  &  &  &  &  &  &  &  &  &  &  &  &  &  &  &  &  &  &  &  &  &  &  &  &  &  \\
\hline
 &  &  &  &  &  &  &  &  &  &  &  &  &  &  &  &  &  &  &  &  &  &  &  &  &  &  &  \\
\hline
 &  &  &  &  &  &  &  &  &  &  &  &  &  &  &  &  &  &  &  &  &  &  &  &  &  &  &  \\
\hline
 &  &  &  &  &  &  &  &  &  &  &  &  &  &  &  &  &  &  &  &  &  &  &  &  &  &  &  \\
\hline
 &  &  &  &  &  &  &  &  &  &  &  &  &  &  &  &  &  &  &  &  &  &  &  &  &  &  &  \\
\hline
 &  &  &  &  &  &  &  &  &  &  &  &  &  &  &  &  &  &  &  &  &  &  &  &  &  &  &  \\
\hline
 &  &  &  &  &  &  &  &  &  &  &  &  &  &  &  &  &  &  &  &  &  &  &  &  &  &  &  \\
\hline
 &  &  &  &  &  &  &  &  &  &  &  &  &  &  &  &  &  &  &  &  &  &  &  &  &  &  &  \\
\hline
 &  &  &  &  &  &  &  &  &  &  &  &  &  &  &  &  &  &  &  &  &  &  &  &  &  &  &  \\
\hline
 &  &  &  &  &  &  &  &  &  &  &  &  &  &  &  &  &  &  &  &  &  &  &  &  &  &  &  \\
\hline
 &  &  &  &  &  &  &  &  &  &  &  &  &  &  &  &  &  &  &  &  &  &  &  &  &  &  &  \\
\hline
 &  &  &  &  &  &  &  &  &  &  &  &  &  &  &  &  &  &  &  &  &  &  &  &  &  &  &  \\
\hline
 &  &  &  &  &  &  &  &  &  &  &  &  &  &  &  &  &  &  &  &  &  &  &  &  &  &  &  \\
\hline
 &  &  &  &  &  &  &  &  &  &  &  &  &  &  &  &  &  &  &  &  &  &  &  &  &  &  &  \\
\hline
 &  &  &  &  &  &  &  &  &  &  &  &  &  &  &  &  &  &  &  &  &  &  &  &  &  &  &  \\
\hline
 &  &  &  &  &  &  &  &  &  &  &  &  &  &  &  &  &  &  &  &  &  &  &  &  &  &  &  \\
\hline
 &  &  &  &  &  &  &  &  &  &  &  &  &  &  &  &  &  &  &  &  &  &  &  &  &  &  &  \\
\hline
 &  &  &  &  &  &  &  &  &  &  &  &  &  &  &  &  &  &  &  &  &  &  &  &  &  &  &  \\
\hline
 &  &  &  &  &  &  &  &  &  &  &  &  &  &  &  &  &  &  &  &  &  &  &  &  &  &  &  \\
\hline
 &  &  &  &  &  &  &  &  &  &  &  &  &  &  &  &  &  &  &  &  &  &  &  &  &  &  &  \\
\hline
 &  &  &  &  &  &  &  &  &  &  &  &  &  &  &  &  &  &  &  &  &  &  &  &  &  &  &  \\
\hline
 &  &  &  &  &  &  &  &  &  &  &  &  &  &  &  &  &  &  &  &  &  &  &  &  &  &  &  \\
\hline
 &  &  &  &  &  &  &  &  &  &  &  &  &  &  &  &  &  &  &  &  &  &  &  &  &  &  &  \\
\hline
 &  &  &  &  &  &  &  &  &  &  &  &  &  &  &  &  &  &  &  &  &  &  &  &  &  &  &  \\
\hline
 &  &  &  &  &  &  &  & - &  &  &  &  &  &  &  &  &  &  &  &  &  &  &  &  &  &  &  \\
\hline
 &  &  &  &  &  &  &  &  &  &  &  &  &  &  &  &  &  &  &  &  &  &  &  &  &  &  &  \\
\hline
 &  &  &  &  &  &  &  &  &  &  &  &  &  &  &  &  &  &  &  &  &  &  &  &  &  &  &  \\
\hline
 &  &  &  &  &  &  &  &  &  &  &  &  &  &  &  &  &  &  &  &  &  &  &  &  &  &  &  \\
\hline
 &  &  &  &  &  &  &  &  &  &  &  &  &  &  &  &  &  &  &  &  &  &  &  &  &  &  &  \\
\hline
 &  &  &  &  &  &  &  &  &  &  &  &  &  &  &  &  &  &  &  &  &  &  &  &  &  &  &  \\
\hline
 &  &  &  &  &  &  &  &  &  &  &  &  &  &  &  &  &  &  &  &  &  &  &  &  &  &  &  \\
\hline
 &  &  &  &  &  &  &  &  &  &  &  &  &  &  &  &  &  &  &  &  &  &  &  &  &  &  &  \\
\hline
 &  &  &  &  &  &  &  &  &  &  &  &  &  &  &  &  &  &  &  &  &  &  &  &  &  &  &  \\
\hline
 &  &  &  &  &  &  &  &  &  &  &  &  &  &  &  &  &  &  &  &  &  &  &  &  &  &  &  \\
\hline
 &  &  &  &  &  &  &  &  &  &  &  &  &  &  &  &  &  &  &  &  &  &  &  &  &  &  &  \\
\hline
 &  &  &  &  &  &  &  &  &  &  &  &  &  &  &  &  &  &  &  &  &  &  &  &  &  &  &  \\
\hline
 &  &  &  &  &  &  &  &  &  &  &  &  &  &  &  &  &  &  &  &  &  &  &  &  &  &  &  \\
\hline
 &  &  &  &  &  &  &  &  &  &  &  &  &  &  &  &  &  &  &  &  &  &  &  &  &  &  &  \\
\hline
 &  &  &  &  &  &  &  &  &  &  &  &  &  &  &  &  &  &  &  &  &  &  &  &  &  &  &  \\
\hline
 &  &  &  &  &  &  &  &  &  &  &  &  &  &  &  &  &  &  &  &  &  &  &  &  &  &  &  \\
\hline
\end{tabular}
\end{center}

\section*{Zadanie 29. (0-2)}
Funkcja kwadratowa jest określona wzorem \(f(x)=x^{2}-11 x\). Oblicz najmniejszą wartość funkcji \(f\) w przedziale \(\langle-6,6\rangle\).\\
\includegraphics[max width=\textwidth, center]{2025_02_10_932b254328dc20198cfdg-15}

Odpowiedź:

Zadanie 30. (0-2)\\
W trapezie \(A B C D\) o podstawach \(A B\) i \(C D\) przekątne \(A C\) oraz \(B D\) przecinają się w punkcie \(S\). Wykaż, że jeżeli \(|A S|=\frac{5}{6}|A C|\), to pole trójkąta \(A B S\) jest 25 razy większe od pola trójkąta \(D C S\).\\
\(\qquad\)\\
\includegraphics[max width=\textwidth, center]{2025_02_10_932b254328dc20198cfdg-16(1)}\\
\(\qquad\)\\
\includegraphics[max width=\textwidth, center]{2025_02_10_932b254328dc20198cfdg-16}\\
\(\qquad\)\\
\includegraphics[max width=\textwidth, center]{2025_02_10_932b254328dc20198cfdg-16(2)}\\
\(\qquad\)

\section*{Zadanie 31. (0-4)}
Ciąg arytmetyczny ( \(a_{n}\) ) określony jest wzorem \(a_{n}=2016-3 n\), dla \(n \geq 1\). Oblicz sumę wszystkich dodatnich wyrazów tego ciągu.\\
\includegraphics[max width=\textwidth, center]{2025_02_10_932b254328dc20198cfdg-17}

Odpowiedź: \(\qquad\)

\section*{Zadanie 32. (0-4)}
Na rysunku przedstawione są dwa wierzchołki trójkąta prostokątnego \(A B C\) : \(A=(-3,-3)\) i \(C=(2,7)\) oraz prosta o równaniu \(y=\frac{3}{4} x-\frac{3}{4}\), zawierająca przeciwprostokątną \(A B\) tego trójkąta.\\
\includegraphics[max width=\textwidth, center]{2025_02_10_932b254328dc20198cfdg-18}

Oblicz współrzędne wierzchołka \(B\) tego trójkąta i długość odcinka \(A B\).

\begin{center}
\begin{tabular}{|c|c|c|c|c|c|c|c|c|c|c|c|c|c|c|c|c|c|c|c|c|c|}
\hline
 &  &  &  &  &  &  &  &  &  &  &  &  &  &  &  &  &  &  &  &  &  \\
\hline
 &  &  &  &  &  &  &  &  &  &  &  &  &  &  &  &  &  &  &  &  &  \\
\hline
 &  &  &  &  &  &  &  &  &  &  &  &  &  &  &  &  &  &  &  &  &  \\
\hline
 &  &  &  &  &  &  &  &  &  &  &  &  &  &  &  &  &  &  &  &  &  \\
\hline
 &  &  &  &  &  &  &  &  &  &  &  &  &  &  &  &  &  &  &  &  &  \\
\hline
 &  &  &  &  &  &  &  &  &  &  &  &  &  &  &  &  &  &  &  &  &  \\
\hline
 &  &  &  &  &  &  &  &  &  &  &  &  &  &  &  &  &  &  &  &  &  \\
\hline
 &  &  &  &  &  &  &  &  &  &  &  &  &  &  &  &  &  &  &  &  &  \\
\hline
 &  &  &  &  &  &  &  &  &  &  &  &  &  &  &  &  &  &  &  &  &  \\
\hline
 &  &  &  &  &  &  &  &  &  &  &  &  &  &  &  &  &  &  &  &  &  \\
\hline
 &  &  &  &  &  &  &  &  &  &  &  &  &  &  &  &  &  &  &  &  &  \\
\hline
 &  &  &  &  &  &  &  &  &  &  &  &  &  &  &  &  &  &  &  &  &  \\
\hline
 &  &  &  &  &  &  &  &  &  &  &  &  &  &  &  &  &  &  &  &  &  \\
\hline
 &  &  &  &  &  &  &  &  &  &  &  &  &  &  &  &  &  &  &  &  &  \\
\hline
 &  &  &  &  &  &  &  &  &  &  &  &  &  &  &  &  &  &  &  &  &  \\
\hline
 &  &  &  &  &  &  &  &  &  &  &  &  &  &  &  &  &  &  &  &  &  \\
\hline
 &  &  &  &  &  &  &  &  &  &  &  &  &  &  &  &  &  &  &  &  &  \\
\hline
 &  &  &  &  &  &  &  &  &  &  &  &  &  &  &  &  &  &  &  &  &  \\
\hline
 &  &  &  &  &  &  &  &  &  &  &  &  &  &  &  &  &  &  &  &  &  \\
\hline
 &  &  &  &  &  &  &  &  &  &  &  &  &  &  &  &  &  &  &  &  &  \\
\hline
 &  &  &  &  &  &  &  &  &  &  &  &  &  &  &  &  &  &  &  &  &  \\
\hline
 &  &  &  &  &  &  &  &  &  &  &  &  &  &  &  &  &  &  &  &  &  \\
\hline
 &  &  &  &  &  &  &  &  &  &  &  &  &  &  &  &  &  &  &  &  &  \\
\hline
 &  &  &  &  &  &  &  &  &  &  &  &  &  &  &  &  &  &  &  &  &  \\
\hline
 & - &  &  &  &  &  &  &  &  &  &  &  &  &  &  &  &  &  &  &  &  \\
\hline
 &  &  &  &  &  &  &  &  &  &  &  &  &  &  &  &  &  &  &  &  &  \\
\hline
\end{tabular}
\end{center}

\begin{center}
\includegraphics[max width=\textwidth]{2025_02_10_932b254328dc20198cfdg-19}
\end{center}

Odpowiedź:

\section*{Zadanie 33. (0-5)}
Trójkąt równoboczny \(A B C\) jest podstawą ostrosłupa prawidłowego \(A B C S\), w którym ściana boczna jest nachylona do płaszczyzny podstawy pod kątem \(60^{\circ}\), a krawędź boczna ma długość 7 (zobacz rysunek). Oblicz objętość tego ostrosłupa.\\
\includegraphics[max width=\textwidth, center]{2025_02_10_932b254328dc20198cfdg-20}\\
\includegraphics[max width=\textwidth, center]{2025_02_10_932b254328dc20198cfdg-21}

Odpowiedź:

\section*{Zadanie 34. (0-2)}
Ze zbioru siedmiu liczb naturalnych \(\{1,2,3,4,5,6,7\}\) losujemy dwie różne liczby. Oblicz prawdopodobieństwo zdarzenia polegającego na tym, że większą z wylosowanych liczb będzie liczba 5.\\
\includegraphics[max width=\textwidth, center]{2025_02_10_932b254328dc20198cfdg-22}

\section*{BRUDNOPIS (nie podlega ocenie)}

\end{document}