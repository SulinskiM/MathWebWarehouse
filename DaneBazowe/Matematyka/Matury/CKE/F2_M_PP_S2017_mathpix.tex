\documentclass[10pt]{article}
\usepackage[polish]{babel}
\usepackage[utf8]{inputenc}
\usepackage[T1]{fontenc}
\usepackage{graphicx}
\usepackage[export]{adjustbox}
\graphicspath{ {./images/} }
\usepackage{amsmath}
\usepackage{amsfonts}
\usepackage{amssymb}
\usepackage[version=4]{mhchem}
\usepackage{stmaryrd}

\newcommand\Varangle{\mathop{{<\!\!\!\!\!\text{\small)}}\:}\nolimits}

\begin{document}
\begin{center}
\includegraphics[max width=\textwidth]{2025_02_10_f4632bba2e9581e2f027g-01}
\end{center}

\section*{EGZAMIN MATURALNY Z MATEMATYKI Poziom podstawowy}
\section*{Data: 22 sierpnia 2017 r.}
Godzina rozpoczecia: 9:00\\
CZAS PRACY: \(\mathbf{1 7 0}\) minut\\
Liczba punktów do uzyskania: \(\mathbf{5 0}\)

\section*{Instrukcja dla zdającego}
\begin{enumerate}
  \item Sprawdź, czy arkusz egzaminacyjny zawiera 26 stron (zadania 1-34). Ewentualny brak zgłoś przewodniczącemu zespołu nadzorującego egzamin.
  \item Rozwiązania zadań i odpowiedzi wpisuj w miejscu na to przeznaczonym.
  \item Odpowiedzi do zadań zamkniętych (1-25) zaznacz na karcie odpowiedzi, w części karty przeznaczonej dla zdającego. Zamaluj \(\quad\) pola do tego przeznaczone. Błędne zaznaczenie otocz kółkiem \({ }_{\text {i zaznacz właściwe }}\)
  \item Pamiętaj, że pominięcie argumentacji lub istotnych obliczeń w rozwiązaniu zadania otwartego (26-34) może spowodować, że za to rozwiązanie nie otrzymasz pełnej liczby punktów.
  \item Pisz czytelnie i używaj tylko długopisu lub pióra z czarnym tuszem lub atramentem.
  \item Nie używaj korektora, a błędne zapisy wyraźnie przekreśl.
  \item Pamiętaj, że zapisy w brudnopisie nie będą oceniane.
  \item Możesz korzystać z zestawu wzorów matematycznych, cyrkla i linijki, a także \(z\) kalkulatora prostego.
  \item Na tej stronie oraz na karcie odpowiedzi wpisz swój numer PESEL i przyklej naklejkę z kodem.
  \item Nie wpisuj żadnych znaków w częşci przeznaczonej dla egzaminatora.\\
\includegraphics[max width=\textwidth, center]{2025_02_10_f4632bba2e9581e2f027g-01(1)}
\end{enumerate}

W zadaniach od 1. do 25. wybierz i zaznacz na karcie odpowiedzi poprawna odpowiedź.

\section*{Zadanie 1. (0-1)}
Niech \(a=-2, b=3\). Wartość wyrażenia \(a^{b}-b^{a}\) jest równa\\
A. \(\frac{73}{9}\)\\
B. \(\frac{71}{9}\)\\
C. \(-\frac{73}{9}\)\\
D. \(-\frac{71}{9}\)

\section*{Zadanie 2. (0-1)}
Liczba \(9^{9} \cdot 81^{2}\) jest równa\\
A. \(81^{4}\)\\
B. 81\\
C. \(9^{13}\)\\
D. \(9^{36}\)

\section*{Zadanie 3. (0-1)}
Wartość wyrażenia \(\log _{4} 8+5 \log _{4} 2\) jest równa\\
A. 2\\
B. 4\\
C. \(2+\log _{4} 5\)\\
D. \(1+\log _{4} 10\)

\section*{Zadanie 4. (0-1)}
Dane są dwa koła. Promień pierwszego koła jest większy od promienia drugiego koła o \(30 \%\). Wynika stąd, że pole pierwszego koła jest większe od pola drugiego koła\\
A. o mniej niż \(50 \%\), ale więcej niż \(40 \%\).\\
B. o mniej niż \(60 \%\), ale więcej niż \(50 \%\).\\
C. dokładnie o \(60 \%\).\\
D. o więcej niż \(60 \%\).

\section*{Zadanie 5. (0-1)}
Liczba \((2 \sqrt{7}-5)^{2} \cdot(2 \sqrt{7}+5)^{2}\) jest równa\\
A. 9\\
B. 3\\
C. 2809\\
D. \(28-20 \sqrt{7}\)

BRUDNOPIS (nie podlega ocenie)\\
\includegraphics[max width=\textwidth, center]{2025_02_10_f4632bba2e9581e2f027g-03}

\section*{Zadanie 6. (0-1)}
Wskaż rysunek, na którym jest przedstawiony zbiór wszystkich liczb \(x\) spełniających warunek: \(11 \leq 2 x-7 \leq 15\).\\
A.\\
\includegraphics[max width=\textwidth, center]{2025_02_10_f4632bba2e9581e2f027g-04}\\
B.\\
\includegraphics[max width=\textwidth, center]{2025_02_10_f4632bba2e9581e2f027g-04(1)}\\
C.\\
\includegraphics[max width=\textwidth, center]{2025_02_10_f4632bba2e9581e2f027g-04(3)}\\
D.\\
\includegraphics[max width=\textwidth, center]{2025_02_10_f4632bba2e9581e2f027g-04(2)}

\section*{Zadanie 7. (0-1)}
Rozważmy treść następującego zadania:\\
Obwód prostokąta o bokach dtugości a i b jest równy 60. Jeden z boków tego prostokąta jest o 10 dtuższy od drugiego. Oblicz długości boków tego prostokąta.

Który układ równań opisuje zależności między długościami boków tego prostokąta?\\
A. \(\left\{\begin{array}{l}2(a+b)=60 \\ a+10=b\end{array}\right.\)\\
B. \(\left\{\begin{array}{l}2 a+b=60 \\ 10 b=a\end{array}\right.\)\\
C. \(\left\{\begin{array}{l}2 a b=60 \\ a-b=10\end{array}\right.\)\\
D. \(\left\{\begin{array}{l}2(a+b)=60 \\ 10 a=b\end{array}\right.\)

\section*{Zadanie 8. (0-1)}
Rozwiązaniem równania \(\frac{x+1}{x+2}=3\), gdzie \(x \neq-2\), jest liczba należąca do przedziału\\
A. \((-2,1)\)\\
B. \(\langle 1,+\infty)\)\\
C. \((-\infty,-5)\)\\
D. \(\langle-5,-2)\)

\section*{Zadanie 9. (0-1)}
Linę o długości 100 metrów rozcięto na trzy części, których długości pozostają w stosunku 3:4:5. Stąd wynika, że najdłuższa z tych części ma długość\\
A. \(41 \frac{2}{3}\) metra.\\
B. \(33 \frac{1}{3}\) metra.\\
C. 60 metrów.\\
D. 25 metrów.

BRUDNOPIS (nie podlega ocenie)\\
\includegraphics[max width=\textwidth, center]{2025_02_10_f4632bba2e9581e2f027g-05}

Zadanie 10. (0-1)\\
Na rysunku przedstawiono fragment wykresu funkcji kwadratowej \(f\) określonej wzorem \(f(x)=x^{2}+b x+c\).\\
\includegraphics[max width=\textwidth, center]{2025_02_10_f4632bba2e9581e2f027g-06}

Współczynniki \(b\) i \(c\) spełniają warunki:\\
A. \(b<0, c>0\)\\
B. \(b<0, c<0\)\\
C. \(b>0, c>0\)\\
D. \(b>0, c<0\)

\section*{Zadanie 11. (0-1)}
Dany jest ciąg arytmetyczny \(\left(a_{n}\right)\), określony dla \(n \geq 1\), o którym wiemy, że: \(a_{1}=2\) i \(a_{2}=9\).\\
Wtedy \(a_{n}=79 \mathrm{dla}\)\\
A. \(n=10\)\\
B. \(n=11\)\\
C. \(n=12\)\\
D. \(n=13\)

\section*{Zadanie 12. (0-1)}
Dany jest trzywyrazowy ciąg geometryczny o wyrazach dodatnich: (81,3x,4). Stąd wynika, że\\
A. \(x=18\)\\
B. \(x=6\)\\
C. \(x=\frac{85}{6}\)\\
D. \(x=\frac{6}{85}\)

\section*{Zadanie 13. (0-1)}
Kąt \(\alpha\) jest ostry i spełniona jest równość \(\sin \alpha=\frac{2 \sqrt{6}}{7}\). Stąd wynika, że\\
A. \(\cos \alpha=\frac{24}{49}\)\\
B. \(\cos \alpha=\frac{5}{7}\)\\
C. \(\cos \alpha=\frac{25}{49}\)\\
D. \(\cos \alpha=\frac{5 \sqrt{6}}{7}\)

BRUDNOPIS (nie podlega ocenie)\\
\includegraphics[max width=\textwidth, center]{2025_02_10_f4632bba2e9581e2f027g-07}

\section*{Zadanie 14. (0-1)}
Na okręgu o środku w punkcie \(O\) leżą punkty \(A, B\) i \(C\) (zobacz rysunek). Kąt \(A B C\) ma miarę \(121^{\circ}\), a kąt \(B O C\) ma miarę \(40^{\circ}\).\\
\includegraphics[max width=\textwidth, center]{2025_02_10_f4632bba2e9581e2f027g-08}

Kąt \(A O B\) ma miarę\\
A. \(59^{\circ}\)\\
B. \(50^{\circ}\)\\
C. \(81^{\circ}\)\\
D. \(78^{\circ}\)

\section*{Zadanie 15. (0-1)}
W trójkącie \(A B C\) punkt \(D\) leży na boku \(B C\), a punkt \(E\) leży na boku \(A C\). Odcinek \(D E\) jest równoległy do boku \(A B\), a ponadto \(|A E|=|D E|=4,|A B|=6\) (zobacz rysunek).\\
\includegraphics[max width=\textwidth, center]{2025_02_10_f4632bba2e9581e2f027g-08(1)}

Odcinek CE ma długość\\
A. \(\frac{16}{3}\)\\
B. \(\frac{8}{3}\)\\
C. 8\\
D. 6

\section*{Zadanie 16. (0-1)}
Dany jest trójkąt równoboczny, którego pole jest równe \(6 \sqrt{3}\). Bok tego trójkąta ma długość\\
A. \(3 \sqrt{2}\)\\
B. \(2 \sqrt{3}\)\\
C. \(2 \sqrt{6}\)\\
D. \(6 \sqrt{2}\)

BRUDNOPIS (nie podlega ocenie)\\
\includegraphics[max width=\textwidth, center]{2025_02_10_f4632bba2e9581e2f027g-09}

\section*{Zadanie 17. (0-1)}
Punkty \(B=(-2,4)\) i \(C=(5,1)\) są sąsiednimi wierzchołkami kwadratu \(A B C D\). Pole tego kwadratu jest równe\\
A. 29\\
B. 40\\
C. 58\\
D. 74

\section*{Zadanie 18. (0-1)}
Na rysunku przedstawiono ostrosłup prawidłowy czworokątny \(A B C D S\) o podstawie \(A B C D\).\\
\includegraphics[max width=\textwidth, center]{2025_02_10_f4632bba2e9581e2f027g-10}

Kąt nachylenia krawędzi bocznej \(S A\) ostrosłupa do płaszczyzny podstawy \(A B C D\) to\\
A. \(\Varangle S A O\)\\
B. \(\Varangle S A B\)\\
C. \(\Varangle S O A\)\\
D. \(\Varangle A S B\)

\section*{Zadanie 19. (0-1)}
Graniastosłup ma 14 wierzchołków. Liczba wszystkich krawędzi tego graniastosłupa jest równa\\
A. 14\\
B. 21\\
C. 28\\
D. 26

\section*{Zadanie 20. (0-1)}
Prosta \(k\) przechodzi przez punkt \(A=(4,-4)\) i jest prostopadła do osi \(O x\). Prosta \(k\) ma równanie\\
A. \(x-4=0\)\\
B. \(x-y=0\)\\
C. \(y+4=0\)\\
D. \(x+y=0\)

BRUDNOPIS (nie podlega ocenie)\\
\(\qquad\)

Zadanie 21. (0-1)\\
Prosta \(l\) jest nachylona do osi \(O x\) pod kątem \(30^{\circ}\) i przecina oś \(O y\) w punkcie \((0,-\sqrt{3})\) (zobacz rysunek).\\
\includegraphics[max width=\textwidth, center]{2025_02_10_f4632bba2e9581e2f027g-12}

Prosta \(l\) ma równanie\\
A. \(y=\frac{\sqrt{3}}{3} x-\sqrt{3}\)\\
B. \(y=\frac{\sqrt{3}}{3} x+\sqrt{3}\)\\
C. \(y=\frac{1}{2} x-\sqrt{3}\)\\
D. \(y=\frac{1}{2} x+\sqrt{3}\)

\section*{Zadanie 22. (0-1)}
Dany jest stożek o wysokości 6 i tworzącej \(3 \sqrt{5}\). Objętość tego stożka jest równa\\
A. \(36 \pi\)\\
B. \(18 \pi\)\\
C. \(108 \pi\)\\
D. \(54 \pi\)

\section*{Zadanie 23. (0-1)}
Średnia arytmetyczna zestawu danych: \(x, 2,4,6,8,10,12,14\) jest równa 9 . Wtedy mediana tego zestawu danych jest równa\\
A. 8\\
B. 9\\
C. 10\\
D. 16

\section*{Zadanie 24. (0-1)}
Ile jest wszystkich czterocyfrowych liczb naturalnych mniejszych niż 2017 ?\\
A. 2016\\
B. 2017\\
C. 1016\\
D. 1017

\section*{Zadanie 25. (0-1)}
Z pudełka, w którym jest tylko 6 kul białych i \(n\) kul czarnych, losujemy jedną kulę. Prawdopodobieństwo wylosowania kuli białej jest równe \(\frac{1}{3}\). Liczba kul czarnych jest równa\\
A. \(n=9\)\\
B. \(n=2\)\\
C. \(n=18\)\\
D. \(n=12\)

BRUDNOPIS (nie podlega ocenie)\\
\includegraphics[max width=\textwidth, center]{2025_02_10_f4632bba2e9581e2f027g-13}

\section*{Zadanie 26. (0-2)}
Rozwiąż nierówność \(2 x^{2}+x-6 \leq 0\).

\begin{center}
\begin{tabular}{|c|c|c|c|c|c|c|c|c|c|c|c|c|c|c|c|c|c|c|c|c|c|c|c|}
\hline
 &  &  &  &  &  &  &  &  &  &  &  &  &  &  &  &  &  &  &  &  &  &  &  \\
\hline
 &  &  &  &  &  &  &  &  &  &  &  &  &  &  &  &  &  &  &  &  &  &  &  \\
\hline
 &  &  &  &  &  &  &  &  &  &  &  &  &  &  &  &  &  &  &  &  &  &  &  \\
\hline
 &  &  &  &  &  &  &  &  &  &  &  &  &  &  &  &  &  &  &  &  &  &  &  \\
\hline
 &  &  &  &  &  &  &  &  &  &  &  &  &  &  &  &  &  &  &  &  &  &  &  \\
\hline
 &  &  &  &  &  &  &  &  &  &  &  &  &  &  &  &  &  &  &  &  &  &  &  \\
\hline
 &  &  &  &  &  &  &  &  &  &  &  &  &  &  &  &  &  &  &  &  &  &  &  \\
\hline
 &  &  &  &  &  &  &  &  &  &  &  &  &  &  &  &  &  &  &  &  &  &  &  \\
\hline
 &  &  &  &  &  &  &  &  &  &  &  &  &  &  &  &  &  &  &  &  &  &  &  \\
\hline
 &  &  &  &  &  &  &  &  &  &  &  &  &  &  &  &  &  &  &  &  &  &  &  \\
\hline
 &  &  &  &  &  &  &  &  &  &  &  &  &  &  &  &  &  &  &  &  &  &  &  \\
\hline
 &  &  &  &  &  &  &  &  &  &  &  &  &  &  &  &  &  &  &  &  &  &  &  \\
\hline
 &  &  &  &  &  &  &  &  &  &  &  &  &  &  &  &  &  &  &  &  &  &  &  \\
\hline
 &  &  &  &  &  &  &  &  &  &  &  &  &  &  &  &  &  &  &  &  &  &  &  \\
\hline
 &  &  &  &  &  &  &  &  &  &  &  &  &  &  &  &  &  &  &  &  &  &  &  \\
\hline
 &  &  &  &  &  &  &  &  &  &  &  &  &  &  &  &  &  &  &  &  &  &  &  \\
\hline
 &  &  &  &  &  &  &  &  &  &  &  &  &  &  &  &  &  &  &  &  &  &  &  \\
\hline
 &  &  &  &  &  &  &  &  &  &  &  &  &  &  &  &  &  &  &  &  &  &  &  \\
\hline
 &  &  &  &  &  &  &  &  &  &  &  &  &  &  &  &  &  &  &  &  &  &  &  \\
\hline
 & , &  &  &  &  &  &  &  &  &  &  &  &  &  &  &  &  &  &  &  &  &  &  \\
\hline
 &  &  &  &  &  &  &  &  &  &  &  &  &  &  &  &  &  &  &  &  &  &  &  \\
\hline
 & \( \pm \) &  &  &  &  &  &  &  &  &  &  &  &  &  &  &  &  &  &  &  &  &  &  \\
\hline
 &  &  &  &  &  &  &  &  &  &  &  &  &  &  &  &  &  &  &  &  &  &  &  \\
\hline
 &  &  &  &  &  &  &  &  &  &  &  &  &  &  &  &  &  &  &  &  &  &  &  \\
\hline
 &  &  &  &  &  &  &  &  &  &  &  &  &  &  &  &  &  &  &  &  &  &  &  \\
\hline
 &  &  &  &  &  &  &  &  &  &  &  &  &  &  &  &  &  &  &  &  &  &  &  \\
\hline
 &  &  &  &  &  &  &  &  &  &  &  &  &  &  &  &  &  &  &  &  &  &  &  \\
\hline
 &  &  &  &  &  &  &  &  &  &  &  &  &  &  &  &  &  &  &  &  &  &  &  \\
\hline
 &  &  &  &  &  &  &  &  &  &  &  &  &  &  &  &  &  &  &  &  &  &  &  \\
\hline
 &  &  &  &  &  &  &  &  &  &  &  &  &  &  &  &  &  &  &  &  &  &  &  \\
\hline
 &  &  &  &  &  &  &  &  &  &  &  &  &  &  &  &  &  &  &  &  &  &  &  \\
\hline
 &  &  &  &  &  &  &  &  &  &  &  &  &  &  &  &  &  &  &  &  &  &  &  \\
\hline
 &  &  &  &  &  &  &  &  &  &  &  &  &  &  &  &  &  &  &  &  &  &  &  \\
\hline
 &  &  &  &  &  &  &  &  &  &  &  &  &  &  &  &  &  &  &  &  &  &  &  \\
\hline
 &  &  &  &  &  &  &  &  &  &  &  &  &  &  &  &  &  &  &  &  &  &  &  \\
\hline
 &  &  &  &  &  &  &  &  &  &  &  &  &  &  &  &  &  &  &  &  &  &  &  \\
\hline
 &  &  &  &  &  &  &  &  &  &  &  &  &  &  &  &  &  &  &  &  &  &  &  \\
\hline
 &  &  &  &  &  &  &  &  &  &  &  &  &  &  &  &  &  &  &  &  &  &  &  \\
\hline
 &  &  &  &  &  &  &  &  &  &  &  &  &  &  &  &  &  &  &  &  &  &  &  \\
\hline
 & - &  &  &  &  &  &  &  &  &  &  &  &  &  &  &  &  &  &  &  &  &  &  \\
\hline
 & - &  &  &  &  &  &  &  &  &  &  &  &  &  &  &  &  &  &  &  &  &  &  \\
\hline
 & - &  &  &  &  &  &  &  &  &  &  &  &  &  &  &  &  &  &  &  &  &  &  \\
\hline
 &  & - &  &  &  &  &  &  &  &  &  &  &  &  &  &  &  &  &  &  &  &  &  \\
\hline
 &  &  &  &  &  &  &  &  &  &  &  &  &  &  &  &  &  &  &  &  &  &  &  \\
\hline
\end{tabular}
\end{center}

Zadanie 27. (0-2)\\
Rozwiąż równanie \(\left(x^{2}-6\right)(3 x+2)=0\).\\
\includegraphics[max width=\textwidth, center]{2025_02_10_f4632bba2e9581e2f027g-15}

Odpowiedź:

Zadanie 28. (0-2)\\
Udowodnij, że dla dowolnej dodatniej liczby rzeczywistej \(x\) prawdziwa jest nierówność

\[
4 x+\frac{1}{x} \geq 4
\]

\begin{center}
\includegraphics[max width=\textwidth]{2025_02_10_f4632bba2e9581e2f027g-16}
\end{center}

\section*{Zadanie 29. (0-2)}
Dany jest trójkąt prostokątny \(A B C\), w którym \(|\Varangle A C B|=90^{\circ}\) i \(|\Varangle A B C|=60^{\circ}\). Niech \(D\) oznacza punkt wspólny wysokości poprowadzonej z wierzchołka \(C\) kąta prostego i przeciwprostokątnej \(A B\) tego trójkąta. Wykaż, że \(|A D|:|D B|=3: 1\).\\
\includegraphics[max width=\textwidth, center]{2025_02_10_f4632bba2e9581e2f027g-17}

Ze zbioru liczb \(\{1,2,4,5,10\}\) losujemy dwa razy po jednej liczbie ze zwracaniem. Oblicz prawdopodobieństwo zdarzenia \(A\) polegającego na tym, że iloraz pierwszej wylosowanej liczby przez drugą wylosowaną liczbę jest liczbą całkowitą.\\
\includegraphics[max width=\textwidth, center]{2025_02_10_f4632bba2e9581e2f027g-18}

Odpowiedź:

\section*{Zadanie 31. (0-2)}
Dany jest ciąg arytmetyczny \(\left(a_{n}\right)\), określony dla \(n \geq 1\), w którym spełniona jest równość \(a_{21}+a_{24}+a_{27}+a_{30}=100\). Oblicz sumę \(a_{25}+a_{26}\).\\
\includegraphics[max width=\textwidth, center]{2025_02_10_f4632bba2e9581e2f027g-19}

Odpowiedź:

\section*{Zadanie 32. (0-4)}
Funkcja kwadratowa \(f(x)=a x^{2}+b x+c\) ma dwa miejsca zerowe \(x_{1}=-2\) i \(x_{2}=6\). Wykres funkcji \(f\) przechodzi przez punkt \(A=(1,-5)\). Oblicz najmniejszą wartość funkcji \(f\).\\
\(\qquad\)\\
\includegraphics[max width=\textwidth, center]{2025_02_10_f4632bba2e9581e2f027g-21}

Odpowiedź:

\section*{Zadanie 33. (0-4)}
Punkt \(C=(0,0)\) jest wierzchołkiem trójkąta prostokątnego \(A B C\), którego wierzchołek \(A\) leży na osi \(O x\), a wierzchołek \(B\) na osi \(O y\) układu współrzędnych. Prosta zawierająca wysokość tego trójkąta opuszczoną z wierzchołka \(C\) przecina przeciwprostokątną \(A B\) w punkcie \(D=(3,4)\).\\
\includegraphics[max width=\textwidth, center]{2025_02_10_f4632bba2e9581e2f027g-22}

Oblicz współrzędne wierzchołków \(A\) i \(B\) tego trójkąta oraz długość przeciwprostokątnej \(A B\).\\
\includegraphics[max width=\textwidth, center]{2025_02_10_f4632bba2e9581e2f027g-22(1)}\\
\includegraphics[max width=\textwidth, center]{2025_02_10_f4632bba2e9581e2f027g-23}

Odpowiedź:

\section*{Zadanie 34. (0-5)}
Podstawą graniastosłupa prostego \(A B C D E F\) jest trójkąt prostokątny \(A B C\), w którym \(|\Varangle A C B|=90^{\circ}\) (zobacz rysunek). Stosunek długości przyprostokątnej \(A C\) tego trójkąta do długości przyprostokątnej \(B C\) jest równy \(4: 3\). Punkt \(S\) jest środkiem okręgu opisanego na trójkącie \(A B C\), a długość odcinka \(S C\) jest równa 5 . Pole ściany bocznej \(B E F C\) graniastosłupa jest równe 48. Oblicz objętość tego graniastosłupa.\\
\includegraphics[max width=\textwidth, center]{2025_02_10_f4632bba2e9581e2f027g-24}\\
\includegraphics[max width=\textwidth, center]{2025_02_10_f4632bba2e9581e2f027g-25}

Odpowiedź:

\section*{BRUDNOPIS (nie podlega ocenie)}

\end{document}