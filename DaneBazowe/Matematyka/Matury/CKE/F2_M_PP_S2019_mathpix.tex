\documentclass[10pt]{article}
\usepackage[polish]{babel}
\usepackage[utf8]{inputenc}
\usepackage[T1]{fontenc}
\usepackage{graphicx}
\usepackage[export]{adjustbox}
\graphicspath{ {./images/} }
\usepackage{amsmath}
\usepackage{amsfonts}
\usepackage{amssymb}
\usepackage[version=4]{mhchem}
\usepackage{stmaryrd}
\usepackage{multirow}

\newcommand\Varangle{\mathop{{<\!\!\!\!\!\text{\small)}}\:}\nolimits}

\begin{document}
Arkusz zawiera informacje prawnie chronione do momentu\\
\includegraphics[max width=\textwidth, center]{2025_02_10_0a55cf6f17fce59a7358g-01(1)}

EGZAMIN MATURALNY Z MATEMATYKI Poziom podstawowy

DATA: 20 sierpnia 2019 r.\\
Godzina rozpoczecia: 9:00\\
CZAS PRACY: \(\mathbf{1 7 0}\) minut

\begin{center}
\begin{tabular}{|c|}
\hline
UZUPELNIA ZESPÓL \\
NADZORUJACY \\
\end{tabular}
\end{center}\(|\)\begin{tabular}{|c|}
\hline
\(\square\)\begin{tabular}{l}
Uprawnienia zdającego do: \\
dostosowania \\
kryteriów oceniania \\
nieprzenoszenia \\
zaznaczeń na karte \\
dostosowania \\
w zw. z dyskalkulią \\
\end{tabular} \\
\hline
\end{tabular}

\section*{Liczba punktów do uzyskania: 50}
\section*{Instrukcja dla zdającego}
\begin{enumerate}
  \item Sprawdź, czy arkusz egzaminacyjny zawiera 24 strony (zadania 1-34). Ewentualny brak zgłoś przewodniczącemu zespołu nadzorującego egzamin.
  \item Rozwiązania zadań i odpowiedzi wpisuj w miejscu na to przeznaczonym.
  \item Odpowiedzi do zadań zamkniętych (1-25) zaznacz na karcie odpowiedzi, w części karty przeznaczonej dla zdającego. Zamaluj \({ }^{\square}\) pola do tego przeznaczone. Błędne zaznaczenie otocz kółkiem \({ }^{\text {i zaznacz właściwe. }}\)
  \item Pamiętaj, że pominięcie argumentacji lub istotnych obliczeń w rozwiązaniu zadania otwartego (26-34) może spowodować, że za to rozwiązanie nie otrzymasz pełnej liczby punktów.
  \item Pisz czytelnie i używaj tylko długopisu lub pióra z czarnym tuszem lub atramentem.
  \item Nie używaj korektora, a błędne zapisy wyraźnie przekreśl.
  \item Pamiętaj, że zapisy w brudnopisie nie będą oceniane.
  \item Możesz korzystać z zestawu wzorów matematycznych, cyrkla i linijki, a także z kalkulatora prostego.
  \item Na tej stronie oraz na karcie odpowiedzi wpisz swój numer PESEL i przyklej naklejkę z kodem.
  \item Nie wpisuj żadnych znaków w części przeznaczonej dla egzaminatora.\\
\includegraphics[max width=\textwidth, center]{2025_02_10_0a55cf6f17fce59a7358g-01}
\end{enumerate}

W każdym z zadań od 1. do 25. wybierz i zaznacz na karcie odpowiedzi poprawną odpowiedź.

\section*{Zadanie 1. (0-1)}
Liczba \(\log _{\sqrt{7}} 7\) jest równa\\
A. 2\\
B. 7\\
C. \(\sqrt{7}\)\\
D. \(\frac{1}{2}\)

\section*{Zadanie 2. (0-1)}
Kwadrat liczby \(8-3 \sqrt{7}\) jest równy\\
A. \(127+48 \sqrt{7}\)\\
B. \(127-48 \sqrt{7}\)\\
C. \(1-48 \sqrt{7}\)\\
D. \(1+48 \sqrt{7}\)

\section*{Zadanie 3. (0-1)}
Jeżeli \(75 \%\) liczby \(a\) jest równe 177 i \(59 \%\) liczby \(b\) jest równe 177 , to\\
A. \(b-a=26\)\\
B. \(b-a=64\)\\
C. \(a-b=26\)\\
D. \(a-b=64\)

\section*{Zadanie 4. (0-1)}
Równanie \(x(5 x+1)=5 x+1\) ma dokładnie\\
A. jedno rozwiązanie: \(x=1\).\\
B. dwa rozwiązania: \(x=1\) i \(x=-1\).\\
C. dwa rozwiązania: \(x=-\frac{1}{5}\) i \(x=1\).\\
D. dwa rozwiązania: \(x=\frac{1}{5}\) i \(x=-1\).

\section*{Zadanie 5. (0-1)}
Para liczb \(x=3\) i \(y=1\) jest rozwiązaniem układu równań \(\left\{\begin{array}{c}-x+12 y=a^{2} \\ 2 x+a y=9\end{array}\right.\) dla\\
A. \(a=\frac{7}{3}\)\\
B. \(\quad a=-3\)\\
C. \(a=3\)\\
D. \(a=-\frac{7}{3}\)

\section*{Zadanie 6. (0-1)}
Równanie \(\frac{(x-2)(x+4)}{(x-4)^{2}}=0\) ma dokładnie\\
A. jedno rozwiązanie: \(x=2\).\\
B. jedno rozwiązanie: \(x=-2\).\\
C. dwa rozwiązania: \(x=2, x=-4\).\\
D. dwa rozwiązania: \(x=-2, x=4\).

\section*{BRUDNOPIS (nie podlega ocenie)}
\begin{center}
\includegraphics[max width=\textwidth]{2025_02_10_0a55cf6f17fce59a7358g-03}
\end{center}

\section*{Zadanie 7. (0-1)}
Miejscami zerowymi funkcji kwadratowej \(f\) określonej wzorem \(f(x)=9-(3-x)^{2}\) są liczby\\
A. 0 oraz 3\\
B. -6 oraz 6\\
C. 0 oraz - 6\\
D. 0 oraz 6

\section*{Zadanie 8. (0-1)}
Na rysunku przedstawiono fragment paraboli będącej wykresem funkcji kwadratowej \(g\). Wierzchołkiem tej paraboli jest punkt \(W=(1,1)\).\\
\includegraphics[max width=\textwidth, center]{2025_02_10_0a55cf6f17fce59a7358g-04}

Zbiorem wartości funkcji \(g\) jest przedział\\
A. \((-\infty, 0\rangle\)\\
B. \(\langle 0,2\rangle\)\\
C. \(\langle 1,+\infty)\)\\
D. \((-\infty, 1\rangle\)

\section*{Zadanie 9. (0-1)}
Liczbą większą od 5 jest\\
A. \(\left(\frac{1}{25}\right)^{-\frac{1}{2}}\)\\
B. \(\left(\frac{1}{25}\right)^{-\frac{1}{5}}\)\\
C. \(125^{\frac{2}{3}}\)\\
D. \(125^{\frac{1}{3}}\)

\section*{Zadanie 10. (0-1)}
Punkt \(A=(a, 3)\) leży na prostej określonej równaniem \(y=\frac{3}{4} x+6\). Stąd wynika, że\\
A. \(a=-4\)\\
B. \(a=4\)\\
C. \(a=\frac{33}{4}\)\\
D. \(a=\frac{39}{4}\)

\section*{BRUDNOPIS (nie podlega ocenie)}
\begin{center}
\begin{tabular}{|c|c|c|c|c|c|c|c|c|c|c|c|c|c|c|c|c|c|c|c|c|c|c|c|c|}
\hline
 &  &  &  &  &  &  &  &  &  &  &  &  &  &  &  &  &  &  &  &  &  &  &  &  \\
\hline
 &  &  &  &  &  &  &  &  &  &  &  &  &  &  &  &  &  &  &  &  &  &  &  &  \\
\hline
 &  &  &  &  &  &  &  &  &  &  &  &  &  &  &  &  &  &  &  &  &  &  &  &  \\
\hline
 &  &  &  &  &  &  &  &  &  &  &  &  &  &  &  &  &  &  &  &  &  &  &  &  \\
\hline
 &  &  &  &  &  &  &  &  &  &  &  &  &  &  &  &  &  &  &  &  &  &  &  &  \\
\hline
 &  &  &  &  &  &  &  &  &  &  &  &  &  &  &  &  &  &  &  &  &  &  &  &  \\
\hline
 &  &  &  &  &  &  &  &  &  &  &  &  &  &  &  &  &  &  &  &  &  &  &  &  \\
\hline
 &  &  &  &  &  &  &  &  &  &  &  &  &  &  &  &  &  &  &  &  &  &  &  &  \\
\hline
 &  &  &  &  &  &  &  &  &  &  &  &  &  &  &  &  &  &  &  &  &  &  &  &  \\
\hline
 &  &  &  &  &  &  &  &  &  &  &  &  &  &  &  &  &  &  &  &  &  &  &  &  \\
\hline
 &  &  &  &  &  &  &  &  &  &  &  &  &  &  &  &  &  &  &  &  &  &  &  &  \\
\hline
 &  &  &  &  &  &  &  &  &  &  &  &  &  &  &  &  &  &  &  &  &  &  &  &  \\
\hline
 &  &  &  &  &  &  &  &  &  &  &  &  &  &  &  &  &  &  &  &  &  &  &  &  \\
\hline
 &  &  &  &  &  &  &  &  &  &  &  &  &  &  &  &  &  &  &  &  &  &  &  &  \\
\hline
 &  &  &  &  &  &  &  &  &  &  &  &  &  &  &  &  &  &  &  &  &  &  &  &  \\
\hline
 &  &  &  &  &  &  &  &  &  &  &  &  &  &  &  &  &  &  &  &  &  &  &  &  \\
\hline
 &  &  &  &  &  &  &  &  &  &  &  &  &  &  &  &  &  &  &  &  &  &  &  &  \\
\hline
 &  &  &  &  &  &  &  &  &  &  &  &  &  &  &  &  &  &  &  &  &  &  &  &  \\
\hline
 &  &  &  &  &  &  &  &  &  &  &  &  &  &  &  &  &  &  &  &  &  &  &  &  \\
\hline
 &  &  &  &  &  &  &  &  &  &  &  &  &  &  &  &  &  &  &  &  &  &  &  &  \\
\hline
 &  &  &  &  &  &  &  &  &  &  &  &  &  &  &  &  &  &  &  &  &  &  &  &  \\
\hline
 &  &  &  &  &  &  &  &  &  &  &  &  &  &  &  &  &  &  &  &  &  &  &  &  \\
\hline
 &  &  &  &  &  &  &  &  &  &  &  &  &  &  &  &  &  &  &  &  &  &  &  &  \\
\hline
 &  &  &  &  &  &  &  &  &  &  &  &  &  &  &  &  &  &  &  &  &  &  &  &  \\
\hline
 &  &  &  &  &  &  &  &  &  &  &  &  &  &  &  &  &  &  &  &  &  &  &  &  \\
\hline
 &  &  &  &  &  &  &  &  &  &  &  &  &  &  &  &  &  &  &  &  &  &  &  &  \\
\hline
 &  &  &  &  &  &  &  &  &  &  &  &  &  &  &  &  &  &  &  &  &  &  &  &  \\
\hline
 &  &  &  &  &  &  &  &  &  &  &  &  &  &  &  &  &  &  &  &  &  &  &  &  \\
\hline
 &  &  &  &  &  &  &  &  &  &  &  &  &  &  &  &  &  &  &  &  &  &  &  &  \\
\hline
 &  &  &  &  &  &  &  &  &  &  &  &  &  &  &  &  &  &  &  &  &  &  &  &  \\
\hline
 &  &  &  &  &  &  &  &  &  &  &  &  &  &  &  &  &  &  &  &  &  &  &  &  \\
\hline
 &  &  &  &  &  &  &  &  &  &  &  &  &  &  &  &  &  &  &  &  &  &  &  &  \\
\hline
 &  &  &  &  &  &  &  &  &  &  &  &  &  &  &  &  &  &  &  &  &  &  &  &  \\
\hline
 &  &  &  &  &  &  &  &  &  &  &  &  &  &  &  &  &  &  &  &  &  &  &  &  \\
\hline
 &  &  &  &  &  &  &  &  &  &  &  &  &  &  &  &  &  &  &  &  &  &  &  &  \\
\hline
 & \includegraphics[max width=\textwidth]{2025_02_10_0a55cf6f17fce59a7358g-05}
 &  &  &  &  &  &  &  &  &  &  &  &  &  &  &  &  &  &  &  &  &  &  &  \\
\hline
 & - &  &  &  &  &  &  &  &  &  &  &  &  &  &  &  &  &  &  &  &  &  &  &  \\
\hline
 & - &  &  &  &  &  &  &  &  &  &  &  &  &  &  &  &  &  &  &  &  &  &  &  \\
\hline
 & - &  &  &  &  &  &  &  &  &  &  &  &  &  &  &  &  &  &  &  &  &  &  &  \\
\hline
 & - &  &  &  &  &  &  &  &  &  &  &  &  &  &  &  &  &  &  &  &  &  &  &  \\
\hline
 & - &  &  &  &  &  &  &  &  &  &  &  &  &  &  &  &  &  &  &  &  &  &  &  \\
\hline
 & - &  &  &  &  &  &  &  &  &  &  &  &  &  &  & - &  &  &  &  &  &  &  &  \\
\hline
 & - &  &  &  &  &  &  &  &  &  &  &  &  &  &  &  &  &  &  &  &  &  &  &  \\
\hline
 & - &  &  &  &  &  &  &  &  &  &  &  &  &  &  &  &  &  &  &  &  &  &  &  \\
\hline
 & - &  &  &  &  &  &  &  &  &  &  &  &  &  &  &  &  &  &  &  &  &  &  &  \\
\hline
 &  &  &  &  &  &  &  &  &  &  &  &  &  &  &  &  &  &  &  &  &  &  &  &  \\
\hline
 &  &  &  &  &  &  &  &  &  &  &  &  &  &  &  &  &  &  &  &  &  &  &  &  \\
\hline
 &  &  &  &  &  &  &  &  &  &  &  &  &  &  &  &  &  &  &  &  &  &  &  &  \\
\hline
 &  &  &  &  &  &  &  &  &  &  &  &  &  &  &  &  &  &  &  &  &  &  &  &  \\
\hline
\end{tabular}
\end{center}

\section*{Zadanie 11. (0-1)}
W ciągu arytmetycznym \(\left(a_{n}\right)\), określonym dla \(n \geq 1\), dane są dwa wyrazy: \(a_{1}=-11\) i \(a_{9}=5\). Suma dziewięciu początkowych wyrazów tego ciągu jest równa\\
A. -24\\
B. -27\\
C. -16\\
D. -18

\section*{Zadanie 12. (0-1)}
Wszystkie wyrazy ciągu geometrycznego \(\left(a_{n}\right)\), określonego dla \(n \geq 1\), są liczbami dodatnimi.\\
Drugi wyraz tego ciągu jest równy 162, a piąty wyraz jest równy 48 . Oznacza to, że iloraz tego ciągu jest równy\\
A. \(\frac{2}{3}\)\\
B. \(\frac{3}{4}\)\\
C. \(\frac{1}{3}\)\\
D. \(\frac{1}{2}\)

\section*{Zadanie 13. (0-1)}
Cosinus kąta ostrego \(\alpha\) jest równy \(\frac{12}{13}\). Wtedy\\
A. \(\sin \alpha=\frac{13}{12}\)\\
B. \(\sin \alpha=\frac{1}{13}\)\\
C. \(\sin \alpha=\frac{5}{13}\)\\
D. \(\sin \alpha=\frac{25}{169}\)

\section*{Zadanie 14. (0-1)}
Dany jest trójkąt równoramienny \(A B C\), w którym \(|A C|=|B C|\). Na podstawie \(A B\) tego trójkąta leży punkt \(D\), taki że \(|A D|=|C D|,|B C|=|B D|\) oraz \(\Varangle B C D=72^{\circ}\) (zobacz rysunek). Wynika stąd, że kąt \(A C D\) ma miarę\\
A. \(\quad 38^{\circ}\)\\
B. \(36^{\circ}\)\\
C. \(42^{\circ}\)\\
D. \(40^{\circ}\)\\
\includegraphics[max width=\textwidth, center]{2025_02_10_0a55cf6f17fce59a7358g-06}

\section*{Zadanie 15. (0-1)}
Okrąg, którego środkiem jest punkt \(S=(a, 5)\), jest styczny do osi \(O y\) i do prostej o równaniu \(y=2\). Promień tego okręgu jest równy\\
A. 3\\
B. 5\\
C. 2\\
D. 4

\section*{BRUDNOPIS (nie podlega ocenie)}
\begin{center}
\includegraphics[max width=\textwidth]{2025_02_10_0a55cf6f17fce59a7358g-07}
\end{center}

\section*{Zadanie 16. (0-1)}
Podstawą ostrosłupa prawidłowego czworokątnego \(A B C D S\) jest kwadrat \(A B C D\) (zobacz rysunek). Wszystkie ściany boczne tego ostrosłupa są trójkątami równobocznymi. Miara kąta SAC jest równa\\
A. \(60^{\circ}\)\\
B. \(45^{\circ}\)\\
C. \(90^{\circ}\)\\
D. \(75^{\circ}\)\\
\includegraphics[max width=\textwidth, center]{2025_02_10_0a55cf6f17fce59a7358g-08}

\section*{Zadanie 17. (0-1)}
Proste o równaniach \(y=(4 m+1) x-19\) oraz \(y=(5 m-4) x+20\) są równoległe, gdy\\
A. \(m=5\)\\
B. \(m=-\frac{1}{4}\)\\
C. \(m=\frac{5}{4}\)\\
D. \(m=-5\)

\section*{Zadanie 18. (0-1)}
W układzie współrzędnych punkt \(S=(40,40)\) jest środkiem odcinka \(K L\), którego jednym z końców jest punkt \(K=(0,8)\). Zatem\\
A. \(L=(20,24)\)\\
B. \(L=(-80,-72)\)\\
C. \(L=(-40,-24)\)\\
D. \(L=(80,72)\)

\section*{Zadanie 19. (0-1)}
Punkt \(P=(-6,-8)\), przekształcono najpierw w symetrii względem osi \(O x\), a potem w symetrii względem osi \(O y\). W wyniku tych przekształceń otrzymano punkt \(Q\). Zatem\\
A. \(Q=(6,8)\)\\
B. \(Q=(-6,-8)\)\\
C. \(Q=(8,6)\)\\
D. \(Q=(-8,-6)\)

\section*{Zadanie 20. (0-1)}
W układzie współrzędnych na płaszczyźnie danych jest 5 punktów: \(A=(1,4), B=(-5,-1)\), \(C=(-5,3), D=(6,-4), P=(-30,-76)\).

Punkt \(P\) należy do tej samej ćwiartki układu współrzędnych co punkt\\
A. \(A\)\\
B. \(B\)\\
C. \(C\)\\
D. \(D\)

\section*{BRUDNOPIS (nie podlega ocenie)}
\begin{center}
\includegraphics[max width=\textwidth]{2025_02_10_0a55cf6f17fce59a7358g-09}
\end{center}

\section*{Zadanie 21. (0-1)}
Dany jest prostopadłościan o wymiarach \(30 \mathrm{~cm} \times 40 \mathrm{~cm} \times 120 \mathrm{~cm}\) (zobacz rysunek), a ponadto dane są cztery odcinki \(a, b, c, d\), o długościach - odpowiednio - \(119 \mathrm{~cm}, 121 \mathrm{~cm}\), 129 cm i 131 cm .\\
\includegraphics[max width=\textwidth, center]{2025_02_10_0a55cf6f17fce59a7358g-10}

Przekątna tego prostopadłościanu jest dłuższa\\
A. tylko od odcinka \(a\).\\
B. tylko od odcinków \(a\) i \(b\).\\
C. tylko od odcinków \(a, b\) i \(c\).\\
D. od wszystkich czterech danych odcinków.

\section*{Zadanie 22. (0-1)}
Pole powierzchni całkowitej pewnego stożka jest 3 razy większe od pola powierzchni pewnej kuli. Promień tej kuli jest równy 2 i jest taki sam jak promień podstawy tego stożka. Tworząca tego stożka ma długość równą\\
A. 12\\
B. 11\\
C. 24\\
D. 22

\section*{Zadanie 23. (0-1)}
Srednia arytmetyczna dziesięciu liczb naturalnych \(3,10,5, x, x, x, x, 12,19,7\) jest równa 12.\\
Mediana tych liczb jest równa\\
A. 14\\
B. 12\\
C. 16\\
D. \(x\)

\section*{Zadanie 24. (0-1)}
Wszystkich liczb naturalnych czterocyfrowych parzystych, w których występują wyłącznie cyfry \(1,2,3\), jest\\
A. 54\\
B. 81\\
C. 8\\
D. 27

\section*{Zadanie 25. (0-1)}
W grupie 60 osób (kobiet i mężczyzn) jest 35 kobiet. Z tej grupy losujemy jedną osobę. Prawdopodobieństwo wylosowania każdej osoby jest takie samo. Prawdopodobieństwo zdarzenia polegającego na tym, że wylosujemy mężczyznę, jest równe\\
A. \(\frac{1}{60}\)\\
B. \(\frac{1}{25}\)\\
C. \(\frac{7}{12}\)\\
D. \(\frac{5}{12}\)

\section*{BRUDNOPIS (nie podlega ocenie)}
\begin{center}
\includegraphics[max width=\textwidth]{2025_02_10_0a55cf6f17fce59a7358g-11}
\end{center}

\section*{Zadanie 26. (0-2)}
Rozwiąż równanie \(\left(x^{2}-16\right)\left(x^{3}-1\right)=0\).\\
\includegraphics[max width=\textwidth, center]{2025_02_10_0a55cf6f17fce59a7358g-12}

Odpowiedź:

\section*{Zadanie 27. (0-2)}
Rozwiąż nierówność \(2 x^{2}-5 x+3 \leq 0\).\\
\includegraphics[max width=\textwidth, center]{2025_02_10_0a55cf6f17fce59a7358g-13}

Odpowiedź: \(\qquad\)

\begin{center}
\begin{tabular}{|c|l|c|c|}
\hline
\multirow{3}{*}{\begin{tabular}{l}
Wypełnia \\
egzaminator \\
\end{tabular}} & Nr zadania & 26. & 27. \\
\cline { 2 - 4 }
 & Maks. liczba pkt & 2 & 2 \\
\cline { 2 - 4 }
 & Uzyskana liczba pkt &  &  \\
\hline
\end{tabular}
\end{center}

\section*{Zadanie 28. (0-2)}
Wykaż, że dla każdej liczby dodatniej \(x\) prawdziwa jest nierównośc \(x+\frac{1-x}{x} \geq 1\).

\begin{center}
\begin{tabular}{|c|c|c|c|c|c|c|c|c|c|c|c|c|c|c|c|c|c|c|c|c|c|}
\hline
 &  &  &  &  &  &  &  &  &  &  &  &  &  &  &  &  &  &  &  &  &  \\
\hline
 &  &  &  &  &  &  &  &  &  &  &  &  &  &  &  &  &  &  &  &  &  \\
\hline
 &  &  &  &  &  &  &  &  &  &  &  &  &  &  &  &  &  &  &  &  &  \\
\hline
 &  &  &  &  &  &  &  &  &  &  &  &  &  &  &  &  &  &  &  &  &  \\
\hline
 &  &  &  &  &  &  &  &  &  &  &  &  &  &  &  &  &  &  &  &  &  \\
\hline
 &  &  &  &  &  &  &  &  &  &  &  &  &  &  &  &  &  &  &  &  &  \\
\hline
 &  &  &  &  &  &  &  &  &  &  &  &  &  &  &  &  &  &  &  &  &  \\
\hline
 &  &  &  &  &  &  &  &  &  &  &  &  &  &  &  &  &  &  &  &  &  \\
\hline
 &  &  &  &  &  &  &  &  &  &  &  &  &  &  &  &  &  &  &  &  &  \\
\hline
 &  &  &  &  &  &  &  &  &  &  &  &  &  &  &  &  &  &  &  &  &  \\
\hline
 &  &  &  &  &  &  &  &  &  &  &  &  &  &  &  &  &  &  &  &  &  \\
\hline
 &  &  &  &  &  &  &  &  &  &  &  &  &  &  &  &  &  &  &  &  &  \\
\hline
 &  &  &  &  &  &  &  &  &  &  &  &  &  &  &  &  &  &  &  &  &  \\
\hline
 &  &  &  &  &  &  &  &  &  &  &  &  &  &  &  &  &  &  &  &  &  \\
\hline
 &  &  &  &  &  &  &  &  &  &  &  &  &  &  &  &  &  &  &  &  &  \\
\hline
 &  &  &  &  &  &  &  &  &  &  &  &  &  &  &  &  &  &  &  &  &  \\
\hline
 &  &  &  &  &  &  &  &  &  &  &  &  &  &  &  &  &  &  &  &  &  \\
\hline
 &  &  &  &  &  &  &  &  &  &  &  &  &  &  &  &  &  &  &  &  &  \\
\hline
 &  &  &  &  &  &  &  &  &  &  &  &  &  &  &  &  &  &  &  &  &  \\
\hline
 &  &  &  &  &  &  &  &  &  &  &  &  &  &  &  &  &  &  &  &  &  \\
\hline
 &  &  &  &  &  &  &  &  &  &  &  &  &  &  &  &  &  &  &  &  &  \\
\hline
 &  &  &  &  &  &  &  &  &  &  &  &  &  &  &  &  &  &  &  &  &  \\
\hline
 &  &  &  &  &  &  &  &  &  &  &  &  &  &  &  &  &  &  &  &  &  \\
\hline
 &  &  &  &  &  &  &  &  &  &  &  &  &  &  &  &  &  &  &  &  &  \\
\hline
 &  &  &  &  &  &  &  &  &  &  &  &  &  &  &  &  &  &  &  &  &  \\
\hline
 &  &  &  &  &  &  &  &  &  &  &  &  &  &  &  &  &  &  &  &  &  \\
\hline
 &  &  &  &  &  &  &  &  &  &  &  &  &  &  &  &  &  &  &  &  &  \\
\hline
 &  &  &  &  &  &  &  &  &  &  &  &  &  &  &  &  &  &  &  &  &  \\
\hline
 &  &  &  &  &  &  &  &  &  &  &  &  &  &  &  &  &  &  &  &  &  \\
\hline
 &  &  &  &  &  &  &  &  &  &  &  &  &  &  &  &  &  &  &  &  &  \\
\hline
 &  &  &  &  &  &  &  &  &  &  &  &  &  &  &  &  &  &  &  &  &  \\
\hline
 &  &  &  &  &  &  &  &  &  &  &  &  &  &  &  &  &  &  &  &  &  \\
\hline
 &  &  &  &  &  &  &  &  &  &  &  &  &  &  &  &  &  &  &  &  &  \\
\hline
 &  &  &  &  &  &  &  &  &  &  &  &  &  &  &  &  &  &  &  &  &  \\
\hline
 &  &  &  &  &  &  &  &  &  &  &  &  &  &  &  &  &  &  &  &  &  \\
\hline
 &  &  &  &  &  &  &  &  &  &  &  &  &  &  &  &  &  &  &  &  &  \\
\hline
 &  &  &  &  &  &  &  &  &  &  &  &  &  &  &  &  &  &  &  &  &  \\
\hline
 &  &  &  &  &  &  &  &  &  &  &  &  &  &  &  &  &  &  &  &  &  \\
\hline
 &  &  &  &  &  &  &  &  &  &  &  &  &  &  &  &  &  &  &  &  &  \\
\hline
 &  &  &  &  &  &  &  &  &  &  &  &  &  &  &  &  &  &  &  &  &  \\
\hline
 &  &  &  &  &  &  &  &  &  &  &  &  &  &  &  &  &  &  &  &  &  \\
\hline
 &  &  &  &  &  &  &  &  &  &  &  &  &  &  &  &  &  &  &  &  &  \\
\hline
 &  &  &  &  &  &  &  &  &  &  &  &  &  &  &  &  &  &  &  &  &  \\
\hline
 &  &  &  &  &  &  &  &  &  &  &  &  &  &  &  &  &  &  &  &  &  \\
\hline
 &  &  &  &  &  &  &  &  &  &  &  &  &  &  &  &  &  &  &  &  &  \\
\hline
 &  &  &  &  &  &  &  &  &  &  &  &  &  &  &  &  &  &  &  &  &  \\
\hline
\end{tabular}
\end{center}

\section*{Zadanie 29. (0-2)}
Wierzchołki \(A\) i \(C\) trójkąta \(A B C\) leżą na okręgu o promieniu \(r\), a środek \(S\) tego okręgu leży na boku \(A B\) trójkąta (zobacz rysunek). Prosta \(B C\) jest styczna do tego okręgu w punkcie \(C\), a ponadto \(|A C|=r \sqrt{3}\). Wykaż, że kąt \(A C B\) ma miarę \(120^{\circ}\).\\
\includegraphics[max width=\textwidth, center]{2025_02_10_0a55cf6f17fce59a7358g-15}

\begin{center}
\begin{tabular}{|c|l|c|c|}
\hline
\multirow{3}{*}{\begin{tabular}{l}
Wypetnia \\
egzaminator \\
\end{tabular}} & Nr zadania & 28. & 29. \\
\cline { 2 - 4 }
 & Maks. liczba pkt & 2 & 2 \\
\cline { 2 - 4 }
 & Uzyskana liczba pkt &  &  \\
\hline
\end{tabular}
\end{center}

\section*{Zadanie 30. (0-2)}
Ze zbioru wszystkich liczb naturalnych dwucyfrowych losujemy jedną liczbę. Oblicz prawdopodobieństwo zdarzenia \(A\) polegającego na tym, że wylosowana liczba ma w zapisie dziesiętnym cyfrę dziesiątek, która należy do zbioru \(\{1,3,5,7,9\}\), i jednocześnie cyfrę jedności, która należy do zbioru \(\{0,2,4,6,8\}\).

\begin{center}
\begin{tabular}{|c|c|c|c|c|c|c|c|c|c|c|c|c|c|c|c|c|c|c|c|c|c|c|c|c|c|c|c|c|}
\hline
 &  &  &  &  &  &  &  &  &  &  &  &  &  &  &  &  &  &  &  &  &  &  &  &  &  &  &  &  \\
\hline
 &  &  &  &  &  &  &  &  &  &  &  &  &  &  &  &  &  &  &  &  &  &  &  &  &  &  &  &  \\
\hline
 &  &  &  &  &  &  &  &  &  &  &  &  &  &  &  &  &  &  &  &  &  &  &  &  &  &  &  &  \\
\hline
 &  &  &  &  &  &  &  &  &  &  &  &  &  &  &  &  &  &  &  &  &  &  &  &  &  &  &  &  \\
\hline
 &  &  &  &  &  &  &  &  &  &  &  &  &  &  &  &  &  &  &  &  &  &  &  &  &  &  &  &  \\
\hline
 &  &  &  &  &  &  &  &  &  &  &  &  &  &  &  &  &  &  &  &  &  &  &  &  &  &  &  &  \\
\hline
 &  &  &  &  &  &  &  &  &  &  &  &  &  &  &  &  &  &  &  &  &  &  &  &  &  &  &  &  \\
\hline
 &  &  &  &  &  &  &  &  &  &  &  &  &  &  &  &  &  &  &  &  &  &  &  &  &  &  &  &  \\
\hline
 &  &  &  &  &  &  &  &  &  &  &  &  &  &  &  &  &  &  &  &  &  &  &  &  &  &  &  &  \\
\hline
 &  &  &  &  &  &  &  &  &  &  &  &  &  &  &  &  &  &  &  &  &  &  &  &  &  &  &  &  \\
\hline
 &  &  &  &  &  &  &  &  &  &  &  &  &  &  &  &  &  &  &  &  &  &  &  &  &  &  &  &  \\
\hline
 &  &  &  &  &  &  &  &  &  &  &  &  &  &  &  &  &  &  &  &  &  &  &  &  &  &  &  &  \\
\hline
 &  &  &  &  &  &  &  &  &  &  &  &  &  &  &  &  &  &  &  &  &  &  &  &  &  &  &  &  \\
\hline
 &  &  &  &  &  &  &  &  &  &  &  &  &  &  &  &  &  &  &  &  &  &  &  &  &  &  &  &  \\
\hline
 &  &  &  &  &  &  &  &  &  &  &  &  &  &  &  &  &  &  &  &  &  &  &  &  &  &  &  &  \\
\hline
 &  &  &  &  &  &  &  &  &  &  &  &  &  &  &  &  &  &  &  &  &  &  &  &  &  &  &  &  \\
\hline
 &  &  &  &  &  &  &  &  &  &  &  &  &  &  &  &  &  &  &  &  &  &  &  &  &  &  &  &  \\
\hline
 &  &  &  &  &  &  &  &  &  &  &  &  &  &  &  &  &  &  &  &  &  &  &  &  &  &  &  &  \\
\hline
 &  &  &  &  &  &  &  &  &  &  &  &  &  &  &  &  &  &  &  &  &  &  &  &  &  &  &  &  \\
\hline
 &  &  &  &  &  &  &  &  &  &  &  &  &  &  &  &  &  &  &  &  &  &  &  &  &  &  &  &  \\
\hline
 &  &  &  &  &  &  &  &  &  &  &  &  &  &  &  &  &  &  &  &  &  &  &  &  &  &  &  &  \\
\hline
 &  &  &  &  &  &  &  &  &  &  &  &  &  &  &  &  &  &  &  &  &  &  &  &  &  &  &  &  \\
\hline
 &  &  &  &  &  &  &  &  &  &  &  &  &  &  &  &  &  &  &  &  &  &  &  &  &  &  &  &  \\
\hline
 &  &  &  &  &  &  &  &  &  &  &  &  &  &  &  &  &  &  &  &  &  &  &  &  &  &  &  &  \\
\hline
 &  &  &  &  &  &  &  &  &  &  &  &  &  &  &  &  &  &  &  &  &  &  &  &  &  &  &  &  \\
\hline
 &  &  &  &  &  &  &  &  &  &  &  &  &  &  &  &  &  &  &  &  &  &  &  &  &  &  &  &  \\
\hline
 &  &  &  &  &  &  &  &  &  &  &  &  & - &  &  &  &  &  &  &  &  &  &  &  &  &  &  &  \\
\hline
 &  &  &  &  &  &  &  &  &  &  &  &  &  &  &  &  &  &  &  &  &  &  &  &  &  &  &  &  \\
\hline
 &  &  &  &  &  &  &  &  &  &  &  &  & - &  &  &  &  &  &  &  &  &  &  &  &  &  &  &  \\
\hline
 &  &  &  &  &  &  &  &  &  &  &  &  &  &  &  &  &  &  &  &  &  &  &  &  &  &  &  &  \\
\hline
 &  &  &  &  &  &  &  &  &  &  &  &  &  &  &  &  &  &  &  &  &  &  &  &  &  &  &  &  \\
\hline
 &  &  &  &  &  &  &  &  &  &  &  &  &  &  &  &  &  &  &  &  &  &  &  &  &  &  &  &  \\
\hline
 &  &  &  &  &  &  &  &  &  &  &  &  &  &  &  &  &  &  &  &  &  &  &  &  &  &  &  &  \\
\hline
 & - &  &  &  &  &  &  &  &  &  &  &  &  &  &  &  &  &  &  &  &  &  &  &  &  &  &  &  \\
\hline
 & - &  &  &  &  &  &  &  &  &  &  &  &  &  &  &  &  &  &  &  &  &  &  &  &  &  &  &  \\
\hline
 & - &  &  &  &  &  &  &  &  &  &  &  &  &  &  &  &  &  &  &  &  &  &  &  &  &  &  &  \\
\hline
 & - &  &  &  &  &  &  &  &  &  &  &  &  &  &  &  &  &  &  &  &  &  &  &  &  &  &  &  \\
\hline
 & - &  &  &  &  &  &  &  &  &  &  &  &  &  &  &  &  &  &  &  &  &  &  &  &  &  &  &  \\
\hline
 &  &  &  &  &  &  &  &  &  &  &  &  &  &  &  &  &  &  &  &  &  &  &  &  &  &  &  &  \\
\hline
 &  &  &  &  &  &  &  &  &  &  &  &  &  &  &  &  &  &  &  &  &  &  &  &  &  &  &  &  \\
\hline
\end{tabular}
\end{center}

Odpowiedź:

\section*{Zadanie 31. (0-2)}
Przekątne rombu \(A B C D\) przecinają się w punkcie \(S=\left(-\frac{21}{2},-1\right)\). Punkty \(A\) i \(C\) leżą na prostej o równaniu \(y=\frac{1}{3} x+\frac{5}{2}\). Wyznacz równanie prostej \(B D\).\\
\includegraphics[max width=\textwidth, center]{2025_02_10_0a55cf6f17fce59a7358g-17}

Odpowiedź: \(\qquad\)

\begin{center}
\begin{tabular}{|c|l|c|c|}
\hline
\multirow{3}{*}{\begin{tabular}{c}
Wypelnia \\
egzaminator \\
\end{tabular}} & Nr zadania & 30. & 31. \\
\cline { 2 - 4 }
 & Maks. liczba pkt & 2 & 2 \\
\cline { 2 - 4 }
 & Uzyskana liczba pkt &  &  \\
\hline
\end{tabular}
\end{center}

\section*{Zadanie 32. (0-4)}
W ciągu arytmetycznym \(\left(a_{1}, a_{2}, \ldots, a_{39}, a_{40}\right)\) suma wyrazów tego ciągu o numerach parzystych jest równa 1340, a suma wyrazów ciągu o numerach nieparzystych jest równa 1400 . Wyznacz ostatni wyraz tego ciągu arytmetycznego.\\
\includegraphics[max width=\textwidth, center]{2025_02_10_0a55cf6f17fce59a7358g-18}\\
\includegraphics[max width=\textwidth, center]{2025_02_10_0a55cf6f17fce59a7358g-19}

Odpowiedź:

\begin{center}
\begin{tabular}{|c|l|c|}
\hline
\multirow{3}{*}{\begin{tabular}{l}
Wypelnia \\
egzaminator \\
\end{tabular}} & Nr zadania & 32. \\
\cline { 2 - 3 }
 & Maks. liczba pkt & 4 \\
\cline { 2 - 3 }
 & Uzyskana liczba pkt &  \\
\hline
\end{tabular}
\end{center}

\section*{Zadanie 33. (0-4)}
Srodek okręgu leży w odległości 10 cm od cięciwy tego okręgu. Długość tej cięciwy jest o 22 cm większa od promienia tego okręgu. Oblicz promień tego okręgu.\\
\includegraphics[max width=\textwidth, center]{2025_02_10_0a55cf6f17fce59a7358g-20}\\
\includegraphics[max width=\textwidth, center]{2025_02_10_0a55cf6f17fce59a7358g-21}

Odpowiedź:

\begin{center}
\begin{tabular}{|c|l|c|}
\hline
\multirow{3}{*}{\begin{tabular}{l}
Wypelnia \\
egzaminator \\
\end{tabular}} & Nr zadania & 33. \\
\cline { 2 - 3 }
 & Maks. liczba pkt & 4 \\
\cline { 2 - 3 }
 & Uzyskana liczba pkt &  \\
\hline
\end{tabular}
\end{center}

\section*{Zadanie 34. (0-5)}
Długość krawędzi bocznej ostrosłupa prawidłowego czworokątnego \(A B C D S\) jest równa 12 . (zobacz rysunek). Krawędź boczna tworzy z wysokością tego ostrosłupa kąt \(\alpha\) taki, że \(\operatorname{tg} \alpha=\frac{2}{\sqrt{5}}\). Oblicz objętość tego ostrosłupa.\\
\includegraphics[max width=\textwidth, center]{2025_02_10_0a55cf6f17fce59a7358g-22}\\
\includegraphics[max width=\textwidth, center]{2025_02_10_0a55cf6f17fce59a7358g-22(1)}

\begin{center}
\begin{tabular}{|c|c|c|c|c|c|c|c|c|c|c|c|c|c|c|c|c|c|c|c|c|c|c|}
\hline
 &  &  &  &  &  &  &  &  &  &  &  &  &  &  &  &  &  &  &  &  &  &  \\
\hline
 &  &  &  &  &  &  &  &  &  &  &  &  &  &  &  &  &  &  &  &  &  &  \\
\hline
 &  &  &  &  &  &  &  &  &  &  &  &  &  &  &  &  &  &  &  &  &  &  \\
\hline
 &  &  &  &  &  &  &  &  &  &  &  &  &  &  &  &  &  &  &  &  &  &  \\
\hline
 &  &  &  &  &  &  &  &  &  &  &  &  &  &  &  &  &  &  &  &  &  &  \\
\hline
 &  &  &  &  &  &  &  &  &  &  &  &  &  &  &  &  &  &  &  &  &  &  \\
\hline
 &  &  &  &  &  &  &  &  &  &  &  &  &  &  &  &  &  &  &  &  &  &  \\
\hline
 &  &  &  &  &  &  &  &  &  &  &  &  &  &  &  &  &  &  &  &  &  &  \\
\hline
 &  &  &  &  &  &  &  &  &  &  &  &  &  &  &  &  &  &  &  &  &  &  \\
\hline
 &  &  &  &  &  &  &  &  &  &  &  &  &  &  &  &  &  &  &  &  &  &  \\
\hline
 &  &  &  &  &  &  &  &  &  &  &  &  &  &  &  &  &  &  &  &  &  &  \\
\hline
 &  &  &  &  &  &  &  &  &  &  &  &  &  &  &  &  &  &  &  &  &  &  \\
\hline
 &  &  &  &  &  &  &  &  &  &  &  &  &  &  &  &  &  &  &  &  &  &  \\
\hline
 &  &  &  &  &  &  &  &  &  &  &  &  &  &  &  &  &  &  &  &  &  &  \\
\hline
 &  &  &  &  &  &  &  &  &  &  &  &  &  &  &  &  &  &  &  &  &  &  \\
\hline
 &  &  &  &  &  &  &  &  &  &  &  &  &  &  &  &  &  &  &  &  &  &  \\
\hline
 &  &  &  &  &  &  &  &  &  &  &  &  &  &  &  &  &  &  &  &  &  &  \\
\hline
 &  &  &  &  &  &  &  &  &  &  &  &  &  &  &  &  &  &  &  &  &  &  \\
\hline
 &  &  &  &  &  &  &  &  &  &  &  &  &  &  &  &  &  &  &  &  &  &  \\
\hline
 &  &  &  &  &  &  &  &  &  &  &  &  &  &  &  &  &  &  &  &  &  &  \\
\hline
 &  &  &  &  &  &  &  &  &  &  &  &  &  &  &  &  &  &  &  &  &  &  \\
\hline
 &  &  &  &  &  &  &  &  &  &  &  &  &  &  &  &  &  &  &  &  &  &  \\
\hline
 &  &  &  &  &  &  &  &  &  &  &  &  &  &  &  &  &  &  &  &  &  &  \\
\hline
 &  &  &  &  &  &  &  &  &  &  &  &  &  &  &  &  &  &  &  &  &  &  \\
\hline
 &  &  &  &  &  &  &  &  &  &  &  &  &  &  &  &  &  &  &  &  &  &  \\
\hline
 &  &  &  &  &  &  &  &  &  &  &  &  &  &  &  &  &  &  &  &  &  &  \\
\hline
 &  &  &  &  &  &  &  &  &  &  &  &  &  &  &  &  &  &  &  &  &  &  \\
\hline
 &  &  &  &  &  &  &  &  &  &  &  &  &  &  &  &  &  &  &  &  &  &  \\
\hline
 &  &  &  &  &  &  &  &  &  &  &  &  &  &  &  &  &  &  &  &  &  &  \\
\hline
 &  &  &  &  &  &  &  &  &  &  &  &  &  &  &  &  &  &  &  &  &  &  \\
\hline
 &  &  &  &  &  &  &  &  &  &  &  &  &  &  &  &  &  &  &  &  &  &  \\
\hline
 &  &  &  &  &  &  &  &  &  &  &  &  &  &  &  &  &  &  &  &  &  &  \\
\hline
 &  &  &  &  &  &  &  &  &  &  &  &  &  &  &  &  &  &  &  &  &  &  \\
\hline
 &  &  &  &  &  &  &  &  &  &  &  &  &  &  &  &  &  &  &  &  &  &  \\
\hline
 &  &  &  &  &  &  &  &  &  &  &  &  &  &  &  &  &  &  &  &  &  &  \\
\hline
 &  &  &  &  &  &  &  &  &  &  &  &  &  &  &  &  &  &  &  &  &  &  \\
\hline
 &  &  &  &  &  &  &  &  &  &  &  &  &  &  &  &  &  &  &  &  &  &  \\
\hline
 &  &  &  &  &  &  &  &  &  &  &  &  &  &  &  &  &  &  &  &  &  &  \\
\hline
 &  &  &  &  &  &  &  &  &  &  &  &  &  &  &  &  &  &  &  &  &  &  \\
\hline
 &  &  &  &  &  &  &  &  &  &  &  &  &  &  &  &  &  &  &  &  &  &  \\
\hline
 &  &  &  &  &  &  &  &  &  &  &  &  &  &  &  &  &  &  &  &  &  &  \\
\hline
 &  &  &  &  &  &  &  &  &  &  &  &  &  &  &  &  &  &  &  &  &  &  \\
\hline
 &  &  &  &  &  &  &  &  &  &  &  &  &  &  &  &  &  &  &  &  &  &  \\
\hline
\end{tabular}
\end{center}

Odpowiedź:

\begin{center}
\begin{tabular}{|c|l|c|}
\hline
\multirow{3}{*}{\begin{tabular}{l}
Wypelnia \\
egzaminator \\
\end{tabular}} & Nr zadania & 34. \\
\cline { 2 - 3 }
 & Maks. liczba pkt & 5 \\
\cline { 2 - 3 }
 & Uzyskana liczba pkt &  \\
\hline
\end{tabular}
\end{center}

\section*{BRUDNOPIS (nie podlega ocenie)}

\end{document}