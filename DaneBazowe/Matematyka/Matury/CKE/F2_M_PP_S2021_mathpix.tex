\documentclass[10pt]{article}
\usepackage[polish]{babel}
\usepackage[utf8]{inputenc}
\usepackage[T1]{fontenc}
\usepackage{graphicx}
\usepackage[export]{adjustbox}
\graphicspath{ {./images/} }
\usepackage{amsmath}
\usepackage{amsfonts}
\usepackage{amssymb}
\usepackage[version=4]{mhchem}
\usepackage{stmaryrd}
\usepackage{multirow}

\newcommand\Varangle{\mathop{{<\!\!\!\!\!\text{\small)}}\:}\nolimits}

\begin{document}
\section*{WYPEŁNIA ZDAJĄCY}
\section*{KOD}
PESEL\\
\includegraphics[max width=\textwidth, center]{2025_02_10_1c593321e13381c19341g-01(1)}

\section*{Miejsce na naklejke.}
 Sprawdż, czy kod na naklejce to E-100.Jeżeli tak - przyklej naklejkę. Jeżeli nie - zgłoś to nauczycielowi.

\section*{EGZAMIN MATURALNY Z MATEMATYKI \\
 Poziom podstawowy}
DATA: 24 sierpnia 2021 r.\\
GODZINA ROZPOCZĘCIA: 9:00\\
CZAS PRACY: \(\mathbf{1 7 0}\) minut\\
LICZBA PUNKTÓW DO UZYSKANIA: 45

WYPEŁNIA ZESPÓŁ NADZORUJACY\\
Uprawnienia zdającego do: dostosowania zasad oceniania dostosowania w zw. z dyskalkulią nieprzenoszenia zaznaczeń na kartę.\\
\includegraphics[max width=\textwidth, center]{2025_02_10_1c593321e13381c19341g-01}

EMAP-P0-100-2108

\section*{Instrukcja dla zdającego}
\begin{enumerate}
  \item Sprawdź, czy arkusz egzaminacyjny zawiera 25 stron (zadania 1-35). Ewentualny brak zgłoś przewodniczącemu zespołu nadzorującego egzamin.
  \item Na tej stronie oraz na karcie odpowiedzi wpisz swój numer PESEL i przyklej naklejkę z kodem.
  \item Nie wpisuj żadnych znaków w części przeznaczonej dla egzaminatora.
  \item Rozwiązania zadań i odpowiedzi wpisuj w miejscu na to przeznaczonym.
  \item Odpowiedzi do zadań zamkniętych (1-28) zaznacz na karcie odpowiedzi w części karty przeznaczonej dla zdajacego. Zamaluj \(\square\) pola do tego przeznaczone. Błędne zaznaczenie otocz kółkiem i i zaznacz właściwe.
  \item Pamiętaj, że pominięcie argumentacji lub istotnych obliczeń w rozwiązaniu zadania otwartego (29-35) może spowodować, że za to rozwiązanie nie otrzymasz pełnej liczby punktów.
  \item Pisz czytelnie i używaj tylko długopisu lub pióra z czarnym tuszem lub atramentem.
  \item Nie używaj korektora, a błędne zapisy wyraźnie przekreśl.
  \item Pamiętaj, że zapisy w brudnopisie nie będą oceniane.
  \item Możesz korzystać z zestawu wzorów matematycznych, cyrkla i linijki oraz kalkulatora prostego.
\end{enumerate}

W każdym z zadań od 1. do 28. wybierz i zaznacz na karcie odpowiedzi poprawną odpowiedź.

\section*{Zadanie 1. (0-1)}
Liczba \(9^{-10} \cdot 3^{19}\) jest równa\\
A. \(27^{9}\)\\
B. \(9^{-2}\)\\
C. \(3^{10}\)\\
D. \(3^{-1}\)

\section*{Zadanie 2. (0-1)}
Liczba \(\log _{6} 9+2 \log _{6} 2\) jest równa\\
A. \(\log _{6} \frac{9}{4}\)\\
B. 1\\
C. 2\\
D. \(\log _{6} \frac{81}{2}\)

\section*{Zadanie 3. (0-1)}
Liczba \(x\) stanowi \(80 \%\) liczby dodatniej \(y\). Wynika stąd, że liczba \(y\) to\\
A. \(125 \%\) liczby \(x\).\\
B. \(120 \%\) liczby \(x\).\\
C. \(25 \%\) liczby \(x\).\\
D. \(20 \%\) liczby \(x\).

\section*{Zadanie 4. (0-1)}
Dla każdej liczby rzeczywistej \(x\) i każdej liczby rzeczywistej \(y\) wyrażenie \((3 x+8 y)^{2}\) jest równe\\
A. \(9 x^{2}+48 x y+64 y^{2}\)\\
B. \(9 x^{2}+64 y^{2}\)\\
C. \(3 x^{2}+48 x y+8 y^{2}\)\\
D. \(3 x^{2}+8 y^{2}\)

\section*{Zadanie 5. (0-1)}
Liczba (-2) jest rozwiązaniem równania\\
A. \(x^{2}+4=0\)\\
B. \(\frac{x+2}{2}=1\)\\
C. \(\frac{x}{x+2}=0\)\\
D. \(x^{2}(x+2)+2(x+2)=0\)

BRUDNOPIS (nie podlega ocenie)\\
\includegraphics[max width=\textwidth, center]{2025_02_10_1c593321e13381c19341g-03}

\section*{Zadanie 6. (0-1)}
Zbiorem wszystkich rozwiązań nierówności \(5-\frac{2-6 x}{4} \geq 2 x+1\) jest przedział\\
A. \((-\infty, 1)\)\\
B. \(\langle 1,+\infty)\)\\
C. \((-\infty, 7)\)\\
D. \(\langle 7,+\infty)\)

\section*{Zadanie 7. (0-1)}
Funkcja liniowa \(f\) jest określona wzorem \(f(x)=-2 x+4\). Wykres funkcji \(f\) przesunięto wzdłuż osi \(O x\) o 2 jednostki w lewo (tzn. przeciwnie do zwrotu osi), w wyniku czego otrzymano wykres funkcji \(g\). Funkcja \(g\) jest określona wzorem\\
A. \(g(x)=-2 x+2\)\\
B. \(g(x)=-2 x\)\\
C. \(g(x)=-2 x+6\)\\
D. \(g(x)=-2 x+8\)

\section*{Zadanie 8. (0-1)}
Funkcja \(f\) jest określona wzorem \(f(x)=a x+4\) dla każdej liczby rzeczywistej \(x\). Miejscem zerowym tej funkcji jest liczba ( -1 ). Wtedy\\
A. \(a=-4\)\\
B. \(a=1\)\\
C. \(a=4\)\\
D. \(a=5\)

\section*{Zadanie 9. (0-1)}
Prosta \(k\) przechodzi przez punkt \(A=(2,-3)\) i jest nachylona do osi \(O x\) pod kątem \(45^{\circ}\) (zobacz rysunek). Prosta \(k\) ma równanie\\
A. \(y=x-5\)\\
B. \(y=-x-1\)\\
C. \(y=-x+5\)\\
D. \(y=x+5\)\\
\includegraphics[max width=\textwidth, center]{2025_02_10_1c593321e13381c19341g-04}

BRUDNOPIS (nie podlega ocenie)\\
\includegraphics[max width=\textwidth, center]{2025_02_10_1c593321e13381c19341g-05}

\section*{Zadanie 10. (0-1)}
Funkcja kwadratowa \(f\) jest określona wzorem \(f(x)=-2(x+3)(x-5)\). Wierzchołek paraboli, która jest wykresem funkcji \(f\), ma współrzędną \(x\) równą\\
A. \((-3)\)\\
B. \((-1)\)\\
C. 1\\
D. 5

\section*{Zadanie 11. (0-1)}
Funkcja \(f\) jest określona wzorem \(f(x)=-x^{2}+4\) dla każdej liczby rzeczywistej \(x\). Zbiorem wartości funkcji \(f\) jest przedział\\
A. \((-\infty,-2\rangle\)\\
B. \(\langle 2,+\infty)\)\\
C. \((-4,+\infty)\)\\
D. \((-\infty, 4)\)

\section*{Zadanie 12. (0-1)}
Na rysunku przedstawiono fragment wykresu funkcji kwadratowej \(f\).\\
\includegraphics[max width=\textwidth, center]{2025_02_10_1c593321e13381c19341g-06}

Jeden spośród podanych poniżej wzorów jest wzorem tej funkcji. Wskaż wzór funkcji \(f\).\\
A. \(f(x)=x^{2}-6 x+11\)\\
B. \(f(x)=-x^{2}+x+2\)\\
C. \(f(x)=x^{2}-6 x-7\)\\
D. \(f(x)=-x^{2}+6 x-7\)

\section*{Zadanie 13. (0-1)}
Ciąg arytmetyczny ( \(a_{n}\) ) jest określony dla każdej liczby naturalnej \(n \geq 1\). Różnica tego ciągu jest równa 2. Wtedy\\
A. \(a_{24}-a_{6}=18\)\\
B. \(a_{24}-a_{6}=20\)\\
C. \(a_{24}-a_{6}=36\)\\
D. \(a_{24}-a_{6}=38\)

BRUDNOPIS (nie podlega ocenie)\\
\includegraphics[max width=\textwidth, center]{2025_02_10_1c593321e13381c19341g-07}

\section*{Zadanie 14. (0-1)}
Suma wszystkich liczb całkowitych dodatnich parzystych i jednocześnie mniejszych od 1001 jest równa\\
A. \(\frac{2+998}{2} \cdot 499\)\\
B. \(\frac{2+1000}{2} \cdot 500\)\\
C. \(\frac{2+1001}{2} \cdot 500\)\\
D. \(\frac{1+1001}{2} \cdot 1001\)

\section*{Zadanie 15. (0-1)}
Trójwyrazowy ciąg ( \(2, x, 18\) ) jest rosnącym ciągiem geometrycznym. Wtedy\\
A. \(x=16\)\\
B. \(x=10\)\\
C. \(x=6\)\\
D. \(x=9\)

\section*{Zadanie 16. (0-1)}
Kąt \(\alpha\) jest ostry i \(\sin \alpha=\frac{7}{25}\). Wynika stąd, że\\
A. \(\cos \alpha=\frac{576}{625}\)\\
B. \(\cos \alpha=\frac{24}{25}\)\\
C. \(\cos \alpha=-\sqrt{\frac{24}{25}}\)\\
D. \(\cos \alpha=\frac{18}{25}\)

\section*{Zadanie 17. (0-1)}
Czworokąt \(A B C D\) jest wpisany w okrąg o środku \(S\). Bok \(A D\) jest średnicą tego okręgu, a miara kąta \(B D C\) jest równa \(20^{\circ}\) (zobacz rysunek).\\
\includegraphics[max width=\textwidth, center]{2025_02_10_1c593321e13381c19341g-08}

Wtedy miara kąta \(B S C\) jest równa\\
A. \(10^{\circ}\)\\
B. \(20^{\circ}\)\\
C. \(30^{\circ}\)\\
D. \(40^{\circ}\)

BRUDNOPIS (nie podlega ocenie)\\
\includegraphics[max width=\textwidth, center]{2025_02_10_1c593321e13381c19341g-09}

\section*{Zadanie 18. (0-1)}
Okrąg o środku w punkcie \(O\) jest wpisany w trójkąt \(A B C\). Wiadomo, że \(|A B|=|A C|\) i \(|\Varangle B O C|=100^{\circ}\) (zobacz rysunek).\\
\includegraphics[max width=\textwidth, center]{2025_02_10_1c593321e13381c19341g-10(1)}

Miara kąta \(B A C\) jest równa\\
A. \(20^{\circ}\)\\
B. \(30^{\circ}\)\\
C. \(40^{\circ}\)\\
D. \(50^{\circ}\)

\section*{Zadanie 19. (0-1)}
Punkty \(A, B, C\) i \(D\) leżą na okręgu o środku w punkcie \(O\). Cięciwy \(D B\) i \(A C\) przecinają się w punkcie \(E,|\Varangle A C B|=55^{\circ}\) oraz \(|\Varangle A E B|=140^{\circ}\) (zobacz rysunek).\\
\includegraphics[max width=\textwidth, center]{2025_02_10_1c593321e13381c19341g-10}

Miara kąta \(D A C\) jest równa\\
A. \(45^{\circ}\)\\
B. \(55^{\circ}\)\\
C. \(70^{\circ}\)\\
D. \(85^{\circ}\)

BRUDNOPIS (nie podlega ocenie)\\
\includegraphics[max width=\textwidth, center]{2025_02_10_1c593321e13381c19341g-11}

\section*{Zadanie 20. (0-1)}
Przekątna \(A C\) prostokąta \(A B C D\) ma długość 70 . Na boku \(A B\) obrano punkt \(E\), na przekątnej \(A C\) obrano punkt \(F\), a na boku \(A D\) obrano punkt \(G\) - tak, że czworokąt \(A E F G\) jest prostokątem (zobacz rysunek). Ponadto \(|E F|=30\) i \(|G F|=40\).\\
\includegraphics[max width=\textwidth, center]{2025_02_10_1c593321e13381c19341g-12}

Obwód prostokąta \(A B C D\) jest równy\\
A. 158\\
B. 196\\
C. 336\\
D. 490

\section*{Zadanie 21. (0-1)}
W układzie wspórzędnych dane są dwa punkty \(A=(1,-2)\) oraz \(B=(3,1)\). Współczynnik kierunkowy prostej \(A B\) jest równy\\
A. \(\left(-\frac{3}{2}\right)\)\\
B. \(\left(-\frac{2}{3}\right)\)\\
C. \(\frac{2}{3}\)\\
D. \(\frac{3}{2}\)

\section*{Zadanie 22. (0-1)}
Prosta \(k\) ma równanie \(y=-\frac{4}{7} x+24\). Współczynnik kierunkowy prostej prostopadłej do prostej \(k\) jest równy\\
A. \(\frac{7}{4}\)\\
B. \(\left(-\frac{7}{4}\right)\)\\
C. \(\left(-\frac{4}{7}\right)\)\\
D. \(\frac{4}{7}\)

\section*{Zadanie 23. (0-1)}
Punkty \(A=(3,7)\) i \(C=(-4,6)\) są końcami przekątnej kwadratu \(A B C D\). Promień okręgu opisanego na tym kwadracie jest równy\\
A. \(\frac{\sqrt{2}}{2}\)\\
B. \(\frac{5}{2}\)\\
C. \(\frac{5 \sqrt{2}}{2}\)\\
D. 5

BRUDNOPIS (nie podlega ocenie)\\
\includegraphics[max width=\textwidth, center]{2025_02_10_1c593321e13381c19341g-13}

\section*{Zadanie 24. (0-1)}
Każda krawędź graniastosłupa prawidłowego sześciokątnego ma długość równą 2 (zobacz rysunek).\\
\includegraphics[max width=\textwidth, center]{2025_02_10_1c593321e13381c19341g-14}

Pole powierzchni całkowitej tego graniastosłupa jest równe\\
A. \(24+2 \sqrt{3}\)\\
B. \(24+6 \sqrt{3}\)\\
C. \(24+12 \sqrt{3}\)\\
D. \(24+24 \sqrt{3}\)

\section*{Zadanie 25. (0-1)}
Przekątna sześcianu jest równa 6. Wynika stąd, że objętość tego sześcianu jest równa\\
A. \(24 \sqrt{3}\)\\
B. 72\\
C. \(54 \sqrt{2}\)\\
D. \(648 \sqrt{3}\)

\section*{Zadanie 26. (0-1)}
Wszystkich liczb naturalnych pięciocyfrowych parzystych jest\\
A. \(9 \cdot 2 \cdot 10^{3}\)\\
B. \(9 \cdot 5 \cdot 10^{3}\)\\
C. \(5 \cdot 10^{4}\)\\
D. \(4 \cdot 10^{5}\)

\section*{Zadanie 27. (0-1)}
W pudełku znajdują się tylko kule białe i kule czerwone. Stosunek liczby kul białych do liczby kul czerwonych jest równy 3:4. Wylosowanie każdej kuli z tego pudełka jest jednakowo prawdopodobne. Losujemy jedną kulę. Niech \(A\) oznacza zdarzenie polegające na tym, że wylosowana z pudełka kula będzie biała. Prawdopodobieństwo zdarzenia \(A\) jest równe\\
A. \(\frac{1}{4}\)\\
B. \(\frac{1}{3}\)\\
C. \(\frac{3}{7}\)\\
D. \(\frac{3}{4}\)

\section*{Zadanie 28. (0-1)}
Średnia arytmetyczna pięciu liczb: \(5 x+6,6 x+7,7 x+8,8 x+9,9 x+10\), jest równa 8 . Wtedy \(x\) jest równe\\
A. \((-35)\)\\
B. 0\\
C. 0,35\\
D. 35

BRUDNOPIS (nie podlega ocenie)\\
\includegraphics[max width=\textwidth, center]{2025_02_10_1c593321e13381c19341g-15}

Zadanie 29. (0-2)\\
Rozwiąż nierówność:

\[
x^{2}-5 \geq 4 x
\]

\begin{center}
\includegraphics[max width=\textwidth]{2025_02_10_1c593321e13381c19341g-16}
\end{center}

Odpowiedź:

Zadanie 30. (0-2)\\
Rozwiąż równanie:

\[
\frac{x+8}{x-7}=2 x
\]

\begin{center}
\begin{tabular}{|c|c|c|c|c|c|c|c|c|c|c|c|c|c|c|c|c|c|c|c|c|c|c|}
\hline
 &  &  &  &  &  &  &  &  &  &  &  &  &  &  &  &  &  &  &  &  &  &  \\
\hline
 &  &  &  &  &  &  &  &  &  &  &  &  &  &  &  &  &  &  &  &  &  &  \\
\hline
 &  &  &  &  &  &  &  &  &  &  &  &  &  &  &  &  &  &  &  &  &  &  \\
\hline
 &  &  &  &  &  &  &  &  &  &  &  &  &  &  &  &  &  &  &  &  &  &  \\
\hline
 &  &  &  &  &  &  &  &  &  &  &  &  &  &  &  &  &  &  &  &  &  &  \\
\hline
 &  &  &  &  &  &  &  &  &  &  &  &  &  &  &  &  &  &  &  &  &  &  \\
\hline
 &  &  &  &  &  &  &  &  &  &  &  &  &  &  &  &  &  &  &  &  &  &  \\
\hline
 &  &  &  &  &  &  &  &  &  &  &  &  &  &  &  &  &  &  &  &  &  &  \\
\hline
 &  &  &  &  &  &  &  &  &  &  &  &  &  &  &  &  &  &  &  &  &  &  \\
\hline
 &  &  &  &  &  &  &  &  &  &  &  &  &  &  &  &  &  &  &  &  &  &  \\
\hline
 &  &  &  &  &  &  &  &  &  &  &  &  &  &  &  &  &  &  &  &  &  &  \\
\hline
 &  &  &  &  &  &  &  &  &  &  &  &  &  &  &  &  &  &  &  &  &  &  \\
\hline
 &  &  &  &  &  &  &  &  &  &  &  &  &  &  &  &  &  &  &  &  &  &  \\
\hline
 &  &  &  &  &  &  &  &  &  &  &  &  &  &  &  &  &  &  &  &  &  &  \\
\hline
 &  &  &  &  &  &  &  &  &  &  &  &  &  &  &  &  &  &  &  &  &  &  \\
\hline
 &  &  &  &  &  &  &  &  &  &  &  &  &  &  &  &  &  &  &  &  &  &  \\
\hline
 &  &  &  &  &  &  &  &  &  &  &  &  &  &  &  &  &  &  &  &  &  &  \\
\hline
 &  &  &  &  &  &  &  &  &  &  &  &  &  &  &  &  &  &  &  &  &  &  \\
\hline
 &  &  &  &  &  &  &  &  &  &  &  &  &  &  &  &  &  &  &  &  &  &  \\
\hline
 &  &  &  &  &  &  &  &  &  &  &  &  &  &  &  &  &  &  &  &  &  &  \\
\hline
 &  &  &  &  &  &  &  &  &  &  &  &  &  &  &  &  &  &  &  &  &  &  \\
\hline
 &  &  &  &  &  &  &  &  &  &  &  &  &  &  &  &  &  &  &  &  &  &  \\
\hline
 &  &  &  &  &  &  &  &  &  &  &  &  &  &  &  &  &  &  &  &  &  &  \\
\hline
 &  &  &  &  &  &  &  &  &  &  &  &  &  &  &  &  &  &  &  &  &  &  \\
\hline
 &  &  &  &  &  &  &  &  &  &  &  &  &  &  &  &  &  &  &  &  &  &  \\
\hline
 &  &  &  &  &  &  &  &  &  &  &  &  &  &  &  &  &  &  &  &  &  &  \\
\hline
 &  &  &  &  &  &  &  &  &  &  &  &  &  &  &  &  &  &  &  &  &  &  \\
\hline
 &  &  &  &  &  &  &  &  &  &  &  &  &  &  &  &  &  &  &  &  &  &  \\
\hline
 &  &  &  &  &  &  &  &  &  &  &  &  &  &  &  &  &  &  &  &  &  &  \\
\hline
 &  &  &  &  &  &  &  &  &  &  &  &  &  &  &  &  &  &  &  &  &  &  \\
\hline
 &  &  &  &  &  &  &  &  &  &  &  &  &  &  &  &  &  &  &  &  &  &  \\
\hline
 &  &  &  &  &  &  &  &  &  &  &  &  &  &  &  &  &  &  &  &  &  &  \\
\hline
 &  &  &  &  &  &  &  &  &  &  &  &  &  &  &  &  &  &  &  &  &  &  \\
\hline
 &  &  &  &  &  &  &  &  &  &  &  &  &  &  &  &  &  &  &  &  &  &  \\
\hline
 &  &  &  &  &  &  &  &  &  &  &  &  &  &  &  &  &  &  &  &  &  &  \\
\hline
 &  &  &  &  &  &  &  &  &  &  &  &  &  &  &  &  &  &  &  &  &  &  \\
\hline
 &  &  &  &  &  &  &  &  &  &  &  &  &  &  &  &  &  &  &  &  &  &  \\
\hline
 &  &  &  &  &  &  &  &  &  &  &  &  &  &  &  &  &  &  &  &  &  &  \\
\hline
\end{tabular}
\end{center} Odpowiedż:

\begin{center}
\begin{tabular}{|c|c|c|c|}
\hline
\multirow{3}{*}{\begin{tabular}{c}
Wypełnia \\
egzaminator \\
\end{tabular}} & Nr zadania & 29. & 30. \\
\cline { 2 - 4 }
 & Maks. liczba pkt & 2 & 2 \\
\cline { 2 - 4 }
 & Uzyskana liczba pkt &  &  \\
\hline
\end{tabular}
\end{center}

Zadanie 31. (0-2)\\
Wykaż, że dla każdej liczby rzeczywistej \(a\) i każdej liczby rzeczywistej \(b\) spełniona jest nierówność

\[
b(5 b-4 a)+a^{2} \geq 0
\]

\begin{center}
\begin{tabular}{|c|c|c|c|c|c|c|c|c|c|c|c|c|c|c|c|c|c|c|c|c|}
\hline
 &  &  &  &  &  &  &  &  &  &  &  &  &  &  &  &  &  &  &  &  \\
\hline
 &  &  &  &  &  &  &  &  &  &  &  &  &  &  &  &  &  &  &  &  \\
\hline
 &  &  &  &  &  &  &  &  &  &  &  &  &  &  &  &  &  &  &  &  \\
\hline
 &  &  &  &  &  &  &  &  &  &  &  &  &  &  &  &  &  &  &  &  \\
\hline
 &  &  &  &  &  &  &  &  &  &  &  &  &  &  &  &  &  &  &  &  \\
\hline
 &  &  &  &  &  &  &  &  &  &  &  &  &  &  &  &  &  &  &  &  \\
\hline
 &  &  &  &  &  &  &  &  &  &  &  &  &  &  &  &  &  &  &  &  \\
\hline
 &  &  &  &  &  &  &  &  &  &  &  &  &  &  &  &  &  &  &  &  \\
\hline
 &  &  &  &  &  &  &  &  &  &  &  &  &  &  &  &  &  &  &  &  \\
\hline
 &  &  &  &  &  &  &  &  &  &  &  &  &  &  &  & - &  &  &  &  \\
\hline
 &  &  &  &  &  &  &  &  &  &  &  &  &  &  &  &  &  &  &  &  \\
\hline
 &  &  &  &  &  &  &  &  &  &  &  &  &  &  &  &  &  &  &  &  \\
\hline
 &  &  &  &  &  &  &  &  &  &  &  &  &  &  &  &  &  &  &  &  \\
\hline
 &  &  &  &  &  &  &  &  &  &  &  &  &  &  &  &  &  &  &  &  \\
\hline
 &  &  &  &  &  &  &  &  &  &  &  &  &  &  &  &  &  &  &  &  \\
\hline
 &  &  &  &  &  &  &  &  &  &  &  &  &  &  &  &  &  &  &  &  \\
\hline
 &  &  &  &  &  &  &  &  &  &  &  &  &  &  &  &  &  &  &  &  \\
\hline
 &  &  &  &  &  &  &  &  &  &  &  &  &  &  &  &  &  &  &  &  \\
\hline
 &  &  &  &  &  &  &  &  &  &  &  &  &  &  &  &  &  &  &  &  \\
\hline
 &  &  &  &  &  &  &  &  &  &  &  &  &  &  &  &  &  &  &  &  \\
\hline
 &  &  &  &  &  &  &  &  &  &  &  &  &  &  &  &  &  &  &  &  \\
\hline
 &  &  &  &  &  &  &  &  &  &  &  &  &  &  &  &  &  &  &  &  \\
\hline
 &  &  &  &  &  &  &  &  &  &  &  &  &  &  &  &  &  &  &  &  \\
\hline
 &  &  &  &  &  &  &  &  &  &  &  &  &  &  &  &  &  &  &  &  \\
\hline
 &  &  &  &  &  &  &  &  &  &  &  &  &  &  &  &  &  &  &  &  \\
\hline
 &  &  &  &  &  &  &  &  &  &  &  &  &  &  &  &  &  &  &  &  \\
\hline
 &  &  &  &  &  &  &  &  &  &  &  &  &  &  &  &  &  &  &  &  \\
\hline
 &  &  &  &  &  &  &  &  &  &  &  &  &  &  &  &  &  &  &  &  \\
\hline
 &  &  &  &  &  &  &  &  &  &  &  &  &  &  &  &  &  &  &  &  \\
\hline
 &  &  &  &  &  &  &  &  &  &  &  &  &  &  &  &  &  &  &  &  \\
\hline
 &  &  &  &  &  &  &  &  &  &  &  &  &  &  &  &  &  &  &  &  \\
\hline
 &  &  &  &  &  &  &  &  &  &  &  &  &  &  &  &  &  &  &  &  \\
\hline
 &  &  &  &  &  &  &  &  &  &  &  &  &  &  &  &  &  &  &  &  \\
\hline
 &  &  &  &  &  &  &  &  &  &  &  &  &  &  &  &  &  &  &  &  \\
\hline
 &  &  &  &  &  &  &  &  &  &  &  &  &  &  &  &  &  &  &  &  \\
\hline
 &  &  &  &  &  &  &  &  &  &  &  &  &  &  &  &  &  &  &  &  \\
\hline
 &  &  &  &  &  &  &  &  &  &  &  &  &  &  &  &  &  &  &  &  \\
\hline
 &  &  &  &  &  &  &  &  &  &  &  &  &  &  &  &  &  &  &  &  \\
\hline
 &  &  &  &  &  &  &  &  &  &  &  &  &  &  &  &  &  &  &  &  \\
\hline
 &  &  &  &  &  &  &  &  &  &  &  &  &  &  &  &  &  &  &  &  \\
\hline
 &  &  &  &  &  &  &  &  &  &  &  &  &  &  &  &  &  &  &  &  \\
\hline
 &  &  &  &  &  &  &  &  &  &  &  &  &  &  &  &  &  &  &  &  \\
\hline
\end{tabular}
\end{center}

\section*{Zadanie 32. (0-2)}
W trójkącie \(A B C\) kąt przy wierzchołku \(A\) jest prosty, a kąt przy wierzchołku \(B\) ma miarę \(30^{\circ}\). Na boku \(A B\) tego trójkąta obrano punkt \(D\) tak, że miara kąta \(C D A\) jest równa \(60^{\circ}\) oraz \(|A D|=6\) (zobacz rysunek). Oblicz \(|B D|\).\\
\includegraphics[max width=\textwidth, center]{2025_02_10_1c593321e13381c19341g-19}

Odpowiedź: \(\qquad\)

\begin{center}
\begin{tabular}{|c|c|c|c|}
\hline
\multirow{3}{*}{\begin{tabular}{c}
Wypełnia \\
egzaminator \\
\end{tabular}} & Nr zadania & 31. & 32. \\
\cline { 2 - 4 }
 & Maks. liczba pkt & 2 & 2 \\
\cline { 2 - 4 }
 & Uzyskana liczba pkt &  &  \\
\hline
\end{tabular}
\end{center}

Zadanie 33. (0-2)\\
Dany jest trapez \(A B C D\) o podstawach \(A B\) i \(C D\). Przekątne \(A C\) i \(B D\) tego trapezu przecinają się w punkcie \(S\) (zobacz rysunek) tak, że \(\frac{|A S|}{|S C|}=\frac{3}{2}\). Pole trójkąta \(A B S\) jest równe 12 . Oblicz pole trójkąta \(C D S\).\\
\includegraphics[max width=\textwidth, center]{2025_02_10_1c593321e13381c19341g-20}\\
\includegraphics[max width=\textwidth, center]{2025_02_10_1c593321e13381c19341g-20(1)}

Odpowiedź: .

\section*{Zadanie 34. (0-2)}
Doświadczenie losowe polega na dwukrotnym rzucie symetryczną sześcienną kostką do gry, która na każdej ściance ma inną liczbę oczek - od jednego do sześciu oczek. Niech \(A\) oznacza zdarzenie polegające na tym, że iloczyn liczb oczek wyrzuconych w dwóch rzutach jest równy 12. Oblicz prawdopodobieństwo zdarzenia \(A\).

\begin{center}
\begin{tabular}{|c|c|c|c|c|c|c|c|c|c|c|c|c|c|c|c|c|c|c|c|c|c|c|c|c|c|c|c|c|c|}
\hline
 &  &  &  &  &  &  &  &  &  &  &  &  &  &  &  &  &  &  &  &  &  &  &  &  &  &  &  &  &  \\
\hline
 &  &  &  &  &  &  &  &  &  &  &  &  &  &  &  &  &  &  &  &  &  &  &  &  &  &  &  &  &  \\
\hline
 &  &  &  &  &  &  &  &  &  &  &  &  &  &  &  &  &  &  &  &  &  &  &  &  &  &  &  &  &  \\
\hline
 &  &  &  &  &  &  &  &  &  &  &  &  &  &  &  &  &  &  &  &  &  &  &  &  &  &  &  &  &  \\
\hline
 &  &  &  &  &  &  &  &  &  &  &  &  &  &  &  &  &  &  &  &  &  &  &  &  &  &  &  &  &  \\
\hline
 &  &  &  &  &  &  &  &  &  &  &  &  &  &  &  &  &  &  &  &  &  &  &  &  &  &  &  &  &  \\
\hline
 &  &  &  &  &  &  &  &  &  &  &  &  &  &  &  &  &  &  &  &  &  &  &  &  &  &  &  &  &  \\
\hline
 &  &  &  &  &  &  &  &  &  &  &  &  &  &  &  &  &  &  &  &  &  &  &  &  &  &  &  &  &  \\
\hline
 &  &  &  &  &  &  &  &  &  &  &  &  &  &  &  &  &  &  &  &  &  &  &  &  &  &  &  &  &  \\
\hline
 &  &  &  &  &  &  &  &  &  &  &  &  &  &  &  &  &  &  &  &  &  &  &  &  &  &  &  &  &  \\
\hline
 &  &  &  &  &  &  &  &  &  &  &  &  &  &  &  &  &  &  &  &  &  &  &  &  &  &  &  &  &  \\
\hline
 &  &  &  &  &  &  &  &  &  &  &  &  &  &  &  &  &  &  &  &  &  &  &  &  &  &  &  &  &  \\
\hline
 &  &  &  &  &  &  &  &  &  &  &  &  &  &  &  &  &  &  &  &  &  &  &  &  &  &  &  &  &  \\
\hline
 &  &  &  &  &  &  &  &  &  &  &  &  &  &  &  &  &  &  &  &  &  &  &  &  &  &  &  &  &  \\
\hline
 &  &  &  &  &  &  &  &  &  &  &  &  &  &  &  &  &  &  &  &  &  &  &  &  &  &  &  &  &  \\
\hline
 &  &  &  &  &  &  &  &  &  &  &  &  &  &  &  &  &  &  &  &  &  &  &  &  &  &  &  &  &  \\
\hline
 &  &  &  &  &  &  &  &  &  &  &  &  &  &  &  &  &  &  &  &  &  &  &  &  &  &  &  &  &  \\
\hline
 &  &  &  &  &  &  &  &  &  &  &  &  &  &  &  &  &  &  &  &  &  &  &  &  &  &  &  &  &  \\
\hline
 &  &  &  &  &  &  &  &  &  &  &  &  &  &  &  &  &  &  &  &  &  &  &  &  &  &  &  &  &  \\
\hline
 &  &  &  &  &  &  &  &  &  &  &  &  &  &  &  &  &  &  &  &  &  &  &  &  &  &  &  &  &  \\
\hline
 &  &  &  &  &  &  &  &  &  &  &  &  &  &  &  &  &  &  &  &  &  &  &  &  &  &  &  &  &  \\
\hline
 &  &  &  &  &  &  &  &  &  &  &  &  &  &  &  &  &  &  &  &  &  &  &  &  &  &  &  &  &  \\
\hline
 &  &  &  &  &  &  &  &  &  &  &  &  &  &  &  &  &  &  &  &  &  &  &  &  &  &  &  &  &  \\
\hline
 &  &  &  &  &  &  &  &  &  &  &  &  &  &  &  &  &  &  &  &  &  &  &  &  &  &  &  &  &  \\
\hline
 &  &  &  &  &  &  &  &  &  &  &  &  &  &  &  &  &  &  &  &  &  &  &  &  &  &  &  &  &  \\
\hline
 &  &  &  &  &  &  &  &  &  &  &  &  &  &  &  &  &  &  &  &  &  &  &  &  &  &  &  &  &  \\
\hline
 &  &  &  &  &  &  &  &  &  &  &  &  &  &  &  &  &  &  &  &  &  &  &  &  &  &  &  &  &  \\
\hline
 &  &  &  &  &  &  &  &  &  &  &  &  &  &  &  &  &  &  &  &  &  &  &  &  &  &  &  &  &  \\
\hline
 & - &  &  &  &  &  &  &  &  &  &  &  &  &  &  &  &  &  &  &  &  &  &  &  &  &  &  &  &  \\
\hline
 & - &  &  &  &  &  &  &  &  &  &  &  &  &  &  &  &  &  &  &  &  &  &  &  &  &  &  &  &  \\
\hline
 &  &  &  &  &  &  &  &  &  &  &  &  &  &  &  &  &  &  &  &  &  &  &  &  &  &  &  &  &  \\
\hline
 &  &  &  &  &  &  &  &  &  &  &  &  &  &  &  &  &  &  &  &  &  &  &  &  &  &  &  &  &  \\
\hline
 &  &  &  &  &  &  &  &  &  &  &  &  &  &  &  &  &  &  &  &  &  &  &  &  &  &  &  &  &  \\
\hline
 & - &  &  &  &  &  &  &  &  &  &  &  &  &  &  &  &  &  &  &  &  &  &  &  &  &  &  &  &  \\
\hline
 & - &  &  &  &  &  &  &  & - &  &  & - &  &  &  &  &  &  &  &  &  &  &  &  &  &  &  &  &  \\
\hline
 &  &  &  &  &  &  &  &  &  &  &  &  &  &  &  &  &  &  &  &  &  &  &  &  &  &  &  &  &  \\
\hline
\end{tabular}
\end{center}

Odpowiedź: \(\qquad\)

\begin{center}
\begin{tabular}{|c|c|c|c|}
\hline
\multirow{3}{*}{\begin{tabular}{c}
Wypełnia \\
egzaminator \\
\end{tabular}} & Nr zadania & 33. & 34. \\
\cline { 2 - 4 }
 & Maks. liczba pkt & 2 & 2 \\
\cline { 2 - 4 }
 & Uzyskana liczba pkt &  &  \\
\hline
\end{tabular}
\end{center}

\section*{Zadanie 35. (0-5)}
Dany jest ciąg \(\left(a_{n}\right)\) określony wzorem \(a_{n}=\frac{5-3 n}{7}\) dla każdej liczby naturalnej \(n \geq 1\). Trójwyrazowy ciąg \(\left(a_{4}, x^{2}+2, a_{11}\right)\), gdzie \(x\) jest liczbą rzeczywistą, jest geometryczny. Oblicz \(x\) oraz iloraz tego ciągu geometrycznego.\\
\includegraphics[max width=\textwidth, center]{2025_02_10_1c593321e13381c19341g-22}\\
\includegraphics[max width=\textwidth, center]{2025_02_10_1c593321e13381c19341g-23}

Odpowiedź:

\begin{center}
\begin{tabular}{|c|c|c|}
\hline
\multirow{3}{*}{\begin{tabular}{c}
Wypełnia \\
egzaminator \\
\end{tabular}} & Nr zadania & 35. \\
\cline { 2 - 3 }
 & Maks. liczba pkt & 5 \\
\cline { 2 - 3 }
 & Uzyskana liczba pkt &  \\
\hline
\end{tabular}
\end{center}

BRUDNOPIS (nie podlega ocenie)\\
\includegraphics[max width=\textwidth, center]{2025_02_10_1c593321e13381c19341g-24}\\
\includegraphics[max width=\textwidth, center]{2025_02_10_1c593321e13381c19341g-25}


\end{document}