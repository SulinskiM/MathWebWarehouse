% This LaTeX document needs to be compiled with XeLaTeX.
\documentclass[10pt]{article}
\usepackage[utf8]{inputenc}
\usepackage{graphicx}
\usepackage[export]{adjustbox}
\graphicspath{ {./images/} }
\usepackage{amsmath}
\usepackage{amsfonts}
\usepackage{amssymb}
\usepackage[version=4]{mhchem}
\usepackage{stmaryrd}
\usepackage{multirow}
\usepackage[fallback]{xeCJK}
\usepackage{polyglossia}
\usepackage{fontspec}
\IfFontExistsTF{Noto Serif CJK JP}
{\setCJKmainfont{Noto Serif CJK JP}}
{\IfFontExistsTF{STSong}
  {\setCJKmainfont{STSong}}
  {\IfFontExistsTF{Droid Sans Fallback}
    {\setCJKmainfont{Droid Sans Fallback}}
    {\setCJKmainfont{SimSun}}
}}

\setmainlanguage{polish}
\IfFontExistsTF{CMU Serif}
{\setmainfont{CMU Serif}}
{\IfFontExistsTF{DejaVu Sans}
  {\setmainfont{DejaVu Sans}}
  {\setmainfont{Georgia}}
}

\author{Data: 23 sierpnia 2022 r.}
\date{}


\newcommand\Varangle{\mathop{{<\!\!\!\!\!\text{\small)}}\:}\nolimits}

\begin{document}
\maketitle
\section*{WYPEŁNIA ZDAJĄCY}
\section*{KOD}
PESEL\\
\includegraphics[max width=\textwidth, center]{2025_02_10_21a9f9e366fbb9d66090g-01(1)}

\section*{Miejsce na naklejke.}
Sprawdż, czy kod na naklejce to E-100.\\
Jeżeli tak - przyklej naklejkę. Jeżeli nie - zgłoś to nauczycielowi.

\section*{EGZAMIN MATURALNY Z MATEMATYKI Poziom podstawowy}
Godzina rozpoczecia: 9:00\\
CzAS PRACY: \(\mathbf{1 7 0}\) minut\\
LICZBA PUNKTÓW DO UZYSKANIA: 45

WYPEヒNIA ZESPÓŁ NADZORUJACY\\
Uprawnienia zdającego do:\\
nieprzenoszenia zaznaczeń na kartę dostosowania zasad oceniania dostosowania w zw. z dyskalkulią.\\
\includegraphics[max width=\textwidth, center]{2025_02_10_21a9f9e366fbb9d66090g-01}

EMAP-P0-100-2208

\section*{Instrukcja dla zdającego}
\begin{enumerate}
  \item Sprawdź, czy arkusz egzaminacyjny zawiera 25 stron (zadania 1-35).
\end{enumerate}

Ewentualny brak zgłoś przewodniczącemu zespołu nadzorującego egzamin.\\
2. Na tej stronie oraz na karcie odpowiedzi wpisz swój numer PESEL i przyklej naklejkę z kodem.\\
3. Nie wpisuj żadnych znaków w części przeznaczonej dla egzaminatora.\\
4. Rozwiązania zadań i odpowiedzi wpisuj w miejscu na to przeznaczonym.\\
5. Odpowiedzi do zadań zamkniętych (1-28) zaznacz na karcie odpowiedzi w części karty przeznaczonej dla zdającego. Zamaluj \(\square\) pola do tego przeznaczone. Błędne zaznaczenie otocz kółkiem i i zaznacz właściwe.\\
6. Pamiętaj, że pominięcie argumentacji lub istotnych obliczeń w rozwiązaniu zadania otwartego (29-35) może spowodować, że za to rozwiązanie nie otrzymasz pełnej liczby punktów.\\
7. Pisz czytelnie i używaj tylko długopisu lub pióra z czarnym tuszem lub atramentem.\\
8. Nie używaj korektora, a błędne zapisy wyraźnie przekreśl.\\
9. Pamiętaj, że zapisy w brudnopisie nie będą oceniane.\\
10. Możesz korzystać z zestawu wzorów matematycznych, cyrkla i linijki oraz kalkulatora prostego.

W każdym z zadań od 1. do 28. wybierz i zaznacz na karcie odpowiedzi poprawną odpowiedź.

\section*{Zadanie 1. (0-1)}
Liczba \(\frac{8^{-40}}{2^{10}}\) jest równa\\
A. \(4^{-4}\)\\
B. \(4^{-50}\)\\
C. \(2^{-47}\)\\
D. \(2^{-130}\)

\section*{Zadanie 2. (0-1)}
Liczba \(\log _{2} 32-\log _{2} 8\) jest równa\\
A. 2\\
B. 14\\
C. 16\\
D. 24

\section*{Zadanie 3. (0-1)}
Liczba \((5-2 \sqrt{3})^{2}\) jest równa\\
A. \(25+4 \sqrt{3}\)\\
B. \(25-4 \sqrt{3}\)\\
C. \(37+20 \sqrt{3}\)\\
D. \(37-20 \sqrt{3}\)

\section*{Zadanie 4. (0-1)}
Cenę \(x\) (w złotych) pewnego towaru obniżono najpierw o \(30 \%\), a następnie obniżono o \(20 \%\) w odniesieniu do ceny obowiązującej w danym momencie. Po obydwu tych obniżkach cena towaru jest równa\\
A. \(0,36 \cdot x\) złotych.\\
B. \(0,44 \cdot x\) złotych.\\
C. 0,50 \(\cdot x\) złotych.\\
D. 0,56 • \(x\) złotych.

\section*{Zadanie 5. (0-1)}
Jednym z rozwiązań równania \(5(x+1)-x^{2}(x+1)=0\) jest liczba\\
A. 1\\
B. \((-1)\)\\
C. 5\\
D. \((-5)\)

BRUDNOPIS (nie podlega ocenie)\\
\includegraphics[max width=\textwidth, center]{2025_02_10_21a9f9e366fbb9d66090g-03}

\section*{Zadanie 6. (0-1)}
Zbiorem wszystkich rozwiązań nierówności \(\frac{8 x-3}{4}>6 x\) jest przedział\\
A. \(\left(-\infty,-\frac{3}{4}\right)\)\\
B. \(\left(-\frac{3}{4},+\infty\right)\)\\
C. \(\left(-\infty,-\frac{3}{16}\right)\)\\
D. \(\left(-\frac{3}{16},+\infty\right)\)

\section*{Zadanie 7. (0-1)}
Suma wszystkich rozwiązań równania \((2 x-1)(2 x-2)(x+2)=0\) jest równa\\
A. \(\left(-\frac{7}{2}\right)\)\\
B. \(\left(-\frac{1}{2}\right)\)\\
C. \(\frac{1}{2}\)\\
D. 1

\section*{Zadanie 8. (0-1)}
Punkt \(A=(1,2)\) należy do wykresu funkcji \(f\), określonej wzorem \(f(x)=\left(m^{2}-3\right) x^{3}-m^{2}+m+1\) dla każdej liczby rzeczywistej \(x\). Wtedy\\
A. \(m=-4\)\\
B. \(m=-2\)\\
C. \(m=0\)\\
D. \(m=4\)

\section*{Zadanie 9. (0-1)}
Funkcja liniowa \(f\) określona wzorem \(f(x)=(2 m-5) x+22\) jest rosnąca dla\\
A. \(m>\frac{2}{5}\)\\
B. \(m>2,5\)\\
C. \(m>0\)\\
D. \(m>2\)

\section*{Zadanie 10. (0-1)}
Funkcja kwadratowa \(f\) określona wzorem \(f(x)=x^{2}+b x+c\) osiąga dla \(x=2\) wartość najmniejszą równą 4. Wtedy\\
A. \(b=-4, c=8\)\\
B. \(b=4, c=-8\)\\
C. \(b=-4, c=-8\)\\
D. \(b=4, c=8\)

BRUDNOPIS (nie podlega ocenie)\\
\includegraphics[max width=\textwidth, center]{2025_02_10_21a9f9e366fbb9d66090g-05}

\section*{Zadanie 11. (0-1)}
Dana jest funkcja kwadratowa \(f\) określona wzorem \(f(x)=-2(x-2)(x+1)\). Funkcja \(f\) jest rosnąca w zbiorze\\
A. \(\left(-\infty, \frac{1}{2}\right)\)\\
B. \((-1,2)\)\\
C. \(\left(0, \frac{5}{2}\right)\)\\
D. \(\left(\frac{5}{2},+\infty\right)\)

\section*{Zadanie 12. (0-1)}
Na rysunku przedstawiono wykres funkcji \(f\) określonej na zbiorze \(\langle-2,5\) ).\\
\includegraphics[max width=\textwidth, center]{2025_02_10_21a9f9e366fbb9d66090g-06}

Funkcja \(g\) jest określona za pomocą funkcji \(f\) następująco: \(g(x)=f(x-1)\). Wykres funkcji \(g\) można otrzymać poprzez odpowiednie przesunięcie wykresu funkcji \(f\). Dziedziną funkcji \(g\) jest zbiór\\
A. \(\langle 0,2)\)\\
B. \(\langle-1,6)\)\\
C. \((-3,4)\)\\
D. \(\langle 1,3)\)

\section*{Zadanie 13. (0-1)}
Dane są ciągi \(a_{n}=3 n\) oraz \(b_{n}=4 n-2\), określone dla każdej liczby naturalnej \(n \geq 1\).\\
Liczba 10\\
A. jest wyrazem ciągu \(\left(a_{n}\right)\) i jest wyrazem ciągu \(\left(b_{n}\right)\).\\
B. jest wyrazem ciągu ( \(a_{n}\) ) i nie jest wyrazem ciągu \(\left(b_{n}\right)\).\\
C. nie jest wyrazem ciągu ( \(a_{n}\) ) i jest wyrazem ciągu \(\left(b_{n}\right)\).\\
D. nie jest wyrazem ciągu \(\left(a_{n}\right)\) i nie jest wyrazem ciągu \(\left(b_{n}\right)\).

BRUDNOPIS (nie podlega ocenie)\\
\includegraphics[max width=\textwidth, center]{2025_02_10_21a9f9e366fbb9d66090g-07}

\section*{Zadanie 14. (0-1)}
Dany jest ciąg geometryczny \(\left(a_{n}\right)\), określony dla każdej liczby naturalnej \(n \geq 1\). Drugi wyraz tego ciągu oraz iloraz ciągu \(\left(a_{n}\right)\) są równe 2 . Suma pięciu początkowych kolejnych wyrazów tego ciągu jest równa\\
A. 1\\
B. 11\\
C. 21\\
D. 31

\section*{Zadanie 15. (0-1)}
W ciągu dwóch godzin trzy jednakowe maszyny produkują razem 1200 guzików. Ile guzików wyprodukuje pięć takich maszyn w ciągu jednej godziny? Przyjmij, że maszyny pracują z taką samą, stałą wydajnością.\\
A. 800\\
B. 900\\
C. 1000\\
D. 1500

\section*{Zadanie 16. (0-1)}
Przyprostokątna \(A C\) trójkąta prostokątnego \(A B C\) ma długość 6 , a przeciwprostokątna \(A B\) ma długość \(3 \sqrt{5}\). Wtedy tangens kąta ostrego \(C A B\) tego trójkąta jest równy\\
A. \(\frac{\sqrt{5}}{5}\)\\
B. \(\frac{2 \sqrt{5}}{5}\)\\
C. \(\frac{1}{2}\)\\
D. 2

\section*{Zadanie 17. (0-1)}
Nie istnieje kąt ostry \(\alpha\) taki, że\\
A. \(\sin \alpha=\frac{1}{3}\) i \(\quad \cos \alpha=\frac{2}{3}\)\\
B. \(\sin \alpha=\frac{5}{13} \quad\) i \(\quad \cos \alpha=\frac{12}{13}\)\\
C. \(\sin \alpha=\frac{3}{5} \quad\) i \(\quad \cos \alpha=\frac{4}{5}\)\\
D. \(\sin \alpha=\frac{9}{15} \quad\) i \(\quad \cos \alpha=\frac{12}{15}\)

BRUDNOPIS (nie podlega ocenie)\\
\includegraphics[max width=\textwidth, center]{2025_02_10_21a9f9e366fbb9d66090g-09}

\section*{Zadanie 18. (0-1)}
Wierzchołki \(A, B, C\) czworokąta \(A B S C\) leżą na okręgu o środku \(S\). Kąt \(A B S\) ma miarę \(40^{\circ}\) (zobacz rysunek), a przekątna \(B C\) jest dwusieczną tego kąta.\\
\includegraphics[max width=\textwidth, center]{2025_02_10_21a9f9e366fbb9d66090g-10}

Miara kąta \(A S C\) jest równa\\
A. \(30^{\circ}\)\\
B. \(40^{\circ}\)\\
C. \(50^{\circ}\)\\
D. \(60^{\circ}\)

\section*{Zadanie 19. (0-1)}
Punkty \(A\) oraz \(B\) leżą na okręgu o środku \(S\). Kąt środkowy \(A S B\) ma miarę \(100^{\circ}\). Prosta \(l\) jest styczna do tego okręgu w punkcie \(A\) itworzy z cięciwą \(A B\) okręgu kąt o mierze \(\alpha\) (zobacz rysunek).\\
\includegraphics[max width=\textwidth, center]{2025_02_10_21a9f9e366fbb9d66090g-10(1)}

Wtedy\\
A. \(\alpha=40^{\circ}\)\\
B. \(\alpha=45^{\circ}\)\\
C. \(\alpha=50^{\circ}\)\\
D. \(\alpha=60^{\circ}\)

BRUDNOPIS (nie podlega ocenie)\\
\includegraphics[max width=\textwidth, center]{2025_02_10_21a9f9e366fbb9d66090g-11}

\section*{Zadanie 20. (0-1)}
Pole prostokąta jest równe 16, a przekątne tego prostokąta przecinają się pod kątem ostrym \(\alpha\), takim, że \(\sin \alpha=0,2\). Długość przekątnej tego prostokąta jest równa\\
A. \(4 \sqrt{5}\)\\
B. \(4 \sqrt{10}\)\\
C. 80\\
D. 160

\section*{Zadanie 21. (0-1)}
Proste o równaniach \(y=\frac{2}{3} x-3\) oraz \(y=(2 m-1) x+1\) są prostopadłe, gdy\\
A. \(m=-\frac{5}{4}\)\\
B. \(m=-\frac{1}{4}\)\\
C. \(m=\frac{5}{6}\)\\
D. \(m=\frac{5}{4}\)

\section*{Zadanie 22. (0-1)}
Punkty \(A=(1,-3)\) oraz \(C=(-2,4)\) są końcami przekątnej \(A C\) rombu \(A B C D\). Środek przekątnej \(B D\) tego rombu ma współrzędne\\
A. \(\left(-\frac{1}{2}, \frac{1}{2}\right)\)\\
B. \(\left(\frac{1}{2},-\frac{3}{2}\right)\)\\
C. \((-1,2)\)\\
D. \((-1,1)\)

\section*{Zadanie 23. (0-1)}
Punkty \(A=(-6,5), B=(5,7), C=(10,-3)\) są wierzchołkami równoległoboku \(A B C D\). Długość przekątnej \(B D\) tego równoległoboku jest równa\\
A. \(3 \sqrt{5}\)\\
B. \(4 \sqrt{5}\)\\
C. \(6 \sqrt{5}\)\\
D. \(8 \sqrt{5}\)

\section*{Zadanie 24. (0-1)}
Obrazem prostej o równaniu \(y=2 x+5\) w symetrii osiowej względem osi \(O x\) jest prosta o równaniu\\
A. \(y=2 x-5\)\\
B. \(y=-2 x-5\)\\
C. \(y=-2 x+5\)\\
D. \(y=2 x+5\)

BRUDNOPIS (nie podlega ocenie)\\
\includegraphics[max width=\textwidth, center]{2025_02_10_21a9f9e366fbb9d66090g-13}

\section*{Zadanie 25. (0-1)}
W graniastosłupie prawidłowym stosunek liczby wszystkich krawędzi do liczby wszystkich ścian jest równy \(7: 3\). Podstawą tego graniastosłupa jest\\
A. trójkąt.\\
B. pięciokąt.\\
C. siedmiokąt.\\
D. ośmioką.

\section*{Zadanie 26. (0-1)}
Średnia arytmetyczna zestawu liczb \(a, b, c, d\) jest równa 20 . Wtedy średnia arytmetyczna zestawu liczb \(a-10, b+30, c, d\) jest równa\\
A. 10\\
B. 20\\
C. 25\\
D. 30

\section*{Zadanie 27. (0-1)}
Wszystkich trzycyfrowych liczb naturalnych większych od 300 o wszystkich cyfrach parzystych jest\\
A. \(6 \cdot 10 \cdot 10\)\\
B. \(3 \cdot 10 \cdot 10\)\\
C. \(6 \cdot 5 \cdot 5\)\\
D. \(3 \cdot 5 \cdot 5\)

\section*{Zadanie 28. (0-1)}
Doświadczenie losowe polega na dwukrotnym rzucie symetryczną sześcienną kostką do gry, która na każdej ściance ma inną liczbę oczek - od jednego do sześciu. Niech p oznacza prawdopodobieństwo otrzymania w drugim rzucie liczby oczek podzielnej przez 3. Wtedy\\
A. \(p=\frac{1}{18}\)\\
B. \(p=\frac{1}{6}\)\\
C. \(p=\frac{1}{3}\)\\
D. \(p=\frac{2}{3}\)

BRUDNOPIS (nie podlega ocenie)\\
\includegraphics[max width=\textwidth, center]{2025_02_10_21a9f9e366fbb9d66090g-15}

Zadanie 29. (0-2)\\
Rozwiąż nierówność

\[
3 x^{2}-8 x \geq 3
\]

\begin{center}
\begin{tabular}{|c|c|c|c|c|c|c|c|c|c|c|c|c|c|c|c|c|c|c|c|c|c|c|c|}
\hline
 &  &  &  &  &  &  &  &  &  &  &  &  &  &  &  &  &  &  &  &  &  &  &  \\
\hline
 &  &  &  &  &  &  &  &  &  &  &  &  &  &  &  &  &  &  &  &  &  &  &  \\
\hline
 &  &  &  &  &  &  &  &  &  &  &  &  &  &  &  &  &  &  &  &  &  &  &  \\
\hline
 &  &  &  &  &  &  &  &  &  &  &  &  &  &  &  &  &  &  &  &  &  &  &  \\
\hline
 &  &  &  &  &  &  &  &  &  &  &  &  &  &  &  &  &  &  &  &  &  &  &  \\
\hline
 &  &  &  &  &  &  &  &  &  &  &  &  &  &  &  &  &  &  &  &  &  &  &  \\
\hline
 &  &  &  &  &  &  &  &  &  &  &  &  &  &  &  &  &  &  &  &  &  &  &  \\
\hline
 &  &  &  &  &  &  &  &  &  &  &  &  &  &  &  &  &  &  &  &  &  &  &  \\
\hline
 &  &  &  &  &  &  &  &  &  &  &  &  &  &  &  &  &  &  &  &  &  &  &  \\
\hline
 &  &  &  &  &  &  &  &  &  &  &  &  &  &  &  &  &  &  &  &  &  &  &  \\
\hline
 &  &  &  &  &  &  &  &  &  &  &  &  &  &  &  &  &  &  &  &  &  &  &  \\
\hline
 &  &  &  &  &  &  &  &  &  &  &  &  &  &  &  &  &  &  &  &  &  &  &  \\
\hline
 &  &  &  &  &  &  &  &  &  &  &  &  &  &  &  &  &  &  &  &  &  &  &  \\
\hline
 &  &  &  &  &  &  &  &  &  &  &  &  &  &  &  &  &  &  &  &  &  &  &  \\
\hline
 &  &  &  &  &  &  &  &  &  &  &  &  &  &  &  &  &  &  &  &  &  &  &  \\
\hline
 &  &  &  &  &  &  &  &  &  &  &  &  &  &  &  &  &  &  &  &  &  &  &  \\
\hline
 &  &  &  &  &  &  &  &  &  &  &  &  &  &  &  &  &  &  &  &  &  &  &  \\
\hline
 &  &  &  &  &  &  &  &  &  &  &  &  &  &  &  &  &  &  &  &  &  &  &  \\
\hline
 &  &  &  &  &  &  &  &  &  &  &  &  &  &  &  &  &  &  &  &  &  &  &  \\
\hline
 &  &  &  &  &  &  &  &  &  &  &  &  &  &  &  &  &  &  &  &  &  &  &  \\
\hline
 &  &  &  &  &  &  &  &  &  &  &  &  &  &  &  &  &  &  &  &  &  &  &  \\
\hline
 &  &  &  &  &  &  &  &  &  &  &  &  &  &  &  &  &  &  &  &  &  &  &  \\
\hline
 &  &  &  &  &  &  &  &  &  &  &  &  &  &  &  &  &  &  &  &  &  &  &  \\
\hline
 &  &  &  &  &  &  &  &  &  &  &  &  &  &  &  &  &  &  &  &  &  &  &  \\
\hline
 &  &  &  &  &  &  &  &  &  &  &  &  &  &  &  &  &  &  &  &  &  &  &  \\
\hline
 &  &  &  &  &  &  &  &  &  &  &  &  &  &  &  &  &  &  &  &  &  &  &  \\
\hline
 &  &  &  &  &  &  &  &  &  &  &  &  &  &  &  &  &  &  &  &  &  &  &  \\
\hline
 &  &  &  &  &  &  &  &  &  &  &  &  &  &  &  &  &  &  &  &  &  &  &  \\
\hline
 &  &  &  &  &  &  &  &  &  &  &  &  &  &  &  &  &  &  &  &  &  &  &  \\
\hline
 &  &  &  &  &  &  &  &  &  &  &  &  &  &  &  &  &  &  &  &  &  &  &  \\
\hline
 &  &  &  &  &  &  &  &  &  &  &  &  &  &  &  &  &  &  &  &  &  &  &  \\
\hline
 &  &  &  &  &  &  &  &  &  &  &  &  &  &  &  &  &  &  &  &  &  &  &  \\
\hline
 &  &  &  &  &  &  &  &  &  &  &  &  &  &  &  &  &  &  &  &  &  &  &  \\
\hline
 &  &  &  &  &  &  &  &  &  &  &  &  &  &  &  &  &  &  &  &  &  &  &  \\
\hline
 &  &  &  &  &  &  &  &  &  &  &  &  &  &  &  &  &  &  &  &  &  &  &  \\
\hline
 &  &  &  &  &  &  &  &  &  &  &  &  &  &  &  &  &  &  &  &  &  &  &  \\
\hline
 &  &  &  &  &  &  &  &  &  &  &  &  &  &  &  &  &  &  &  &  &  &  &  \\
\hline
 &  &  &  &  &  &  &  &  &  &  &  &  &  &  &  &  &  &  &  &  &  &  &  \\
\hline
 &  &  &  &  &  &  &  &  &  &  &  &  &  &  &  &  &  &  &  &  &  &  &  \\
\hline
 &  &  &  &  &  &  &  &  &  &  &  &  &  &  &  &  &  &  &  &  &  &  &  \\
\hline
 &  &  &  &  &  &  &  &  &  &  &  &  &  &  &  &  &  &  &  &  &  &  &  \\
\hline
 &  &  &  &  &  &  &  &  &  &  &  &  &  &  &  &  &  &  &  &  &  &  &  \\
\hline
 &  &  &  &  &  &  &  &  &  &  &  &  &  &  & - &  & - & - & - & - &  &  &  \\
\hline
 &  &  &  &  &  &  &  &  &  &  &  &  &  &  &  &  &  &  &  &  &  &  &  \\
\hline
\end{tabular}
\end{center}

Zadanie 30. (0-2)\\
Trójwyrazowy ciąg \((x, y-4, y)\) jest arytmetyczny. Suma wszystkich wyrazów tego ciągu jest równa 6. Oblicz wszystkie wyrazy tego ciągu.\\
\includegraphics[max width=\textwidth, center]{2025_02_10_21a9f9e366fbb9d66090g-17}

\begin{center}
\begin{tabular}{|c|l|c|c|}
\hline
\multirow{3}{*}{\begin{tabular}{l}
Wypełnia \\
egzaminator \\
\end{tabular}} & Nr zadania & 29. & 30. \\
\cline { 2 - 4 }
 & Maks. liczba pkt & 2 & 2 \\
\cline { 2 - 4 }
 & Uzyskana liczba pkt &  &  \\
\hline
\end{tabular}
\end{center}

Zadanie 31. (0-2)\\
Wykaż, że dla każdej liczby rzeczywistej \(a\) różnej od 0 i każdej liczby rzeczywistej \(b\) różnej od 0 spełniona jest nierówność

\[
2 a^{2}-4 a b+5 b^{2}>0
\]

\begin{center}
\begin{tabular}{|c|c|c|c|c|c|c|c|c|c|c|c|c|c|c|c|c|c|c|c|c|c|}
\hline
 &  &  &  &  &  &  &  &  &  &  &  &  &  &  &  &  &  &  &  &  &  \\
\hline
 &  &  &  &  &  &  &  &  &  &  &  &  &  &  &  &  &  &  &  &  &  \\
\hline
 &  &  &  &  &  &  &  &  &  &  &  &  &  &  &  &  &  &  &  &  &  \\
\hline
 &  &  &  &  &  &  &  &  &  &  &  &  &  &  &  &  &  &  &  &  &  \\
\hline
 &  &  &  &  &  &  &  &  &  &  &  &  &  &  &  &  &  &  &  &  &  \\
\hline
 &  &  &  &  &  &  &  &  &  &  &  &  &  &  &  &  &  &  &  &  &  \\
\hline
 &  &  &  &  &  &  &  &  &  &  &  &  &  &  &  &  &  &  &  &  &  \\
\hline
 &  &  &  &  &  &  &  &  &  &  &  &  &  &  &  &  &  &  &  &  &  \\
\hline
 &  &  &  &  &  &  &  &  &  &  &  &  &  &  &  &  &  &  &  &  &  \\
\hline
 &  &  &  &  &  &  &  &  &  &  &  &  &  &  &  &  &  &  &  &  &  \\
\hline
 &  &  &  &  &  &  &  &  &  &  &  &  &  &  &  &  &  &  &  &  &  \\
\hline
 &  &  &  &  &  &  &  &  &  &  &  &  &  &  &  &  &  &  &  &  &  \\
\hline
 &  &  &  &  &  &  &  &  &  &  &  &  &  &  &  &  &  &  &  &  &  \\
\hline
 &  &  &  &  &  &  &  &  &  &  &  &  &  &  &  &  &  &  &  &  &  \\
\hline
 &  &  &  &  &  &  &  &  &  &  &  &  &  &  &  &  &  &  &  &  &  \\
\hline
 &  &  &  &  &  &  &  &  &  &  &  &  &  &  &  &  &  &  &  &  &  \\
\hline
 &  &  &  &  &  &  &  &  &  &  &  &  &  &  &  &  &  &  &  &  &  \\
\hline
 &  &  &  &  &  &  &  &  &  &  &  &  &  &  &  &  &  &  &  &  &  \\
\hline
 &  &  &  &  &  &  &  &  &  &  &  &  &  &  &  &  &  &  &  &  &  \\
\hline
 &  &  &  &  &  &  &  &  &  &  &  &  &  &  &  &  &  &  &  &  &  \\
\hline
 &  &  &  &  &  &  &  &  &  &  &  &  &  &  &  &  &  &  &  &  &  \\
\hline
 &  &  &  &  &  &  &  &  &  &  &  &  &  &  &  &  &  &  &  &  &  \\
\hline
 &  &  &  &  &  &  &  &  &  &  &  &  &  &  &  &  &  &  &  &  &  \\
\hline
 &  &  &  &  &  &  &  &  &  &  &  &  &  &  &  &  &  &  &  &  &  \\
\hline
 &  &  &  &  &  &  &  &  &  &  &  &  &  &  &  &  &  &  &  &  &  \\
\hline
 &  &  &  &  &  &  &  &  &  &  &  &  &  &  &  &  &  &  &  &  &  \\
\hline
 &  &  &  &  &  &  &  &  &  &  &  &  &  &  &  &  &  &  &  &  &  \\
\hline
 &  &  &  &  &  &  &  &  &  &  &  &  &  &  &  &  &  &  &  &  &  \\
\hline
 &  &  &  &  &  &  &  &  &  &  &  &  &  &  &  &  &  &  &  &  &  \\
\hline
 &  &  &  &  &  &  &  &  &  &  &  &  &  &  &  &  &  &  &  &  &  \\
\hline
 &  &  &  &  &  &  &  &  &  &  &  &  &  &  &  &  &  &  &  &  &  \\
\hline
 &  &  &  &  &  &  &  &  &  &  &  &  &  &  &  &  &  &  &  &  &  \\
\hline
 &  &  &  &  &  &  &  &  &  &  &  &  &  &  &  &  &  &  &  &  &  \\
\hline
 &  &  &  &  &  &  &  &  &  &  &  &  &  &  &  &  &  &  &  &  &  \\
\hline
 &  &  &  &  &  &  &  &  &  &  &  &  &  &  &  &  &  &  &  &  &  \\
\hline
 &  &  &  &  &  &  &  &  &  &  &  &  &  &  &  &  &  &  &  &  &  \\
\hline
 &  &  &  &  &  &  &  &  &  &  &  &  &  &  &  &  &  &  &  &  &  \\
\hline
 &  &  &  &  &  &  &  &  &  &  &  &  &  &  &  &  &  &  &  &  &  \\
\hline
 &  &  &  &  &  &  &  &  &  &  &  &  &  &  &  &  &  &  &  &  &  \\
\hline
 &  &  &  &  &  &  &  &  &  &  &  &  &  &  &  &  &  &  &  &  &  \\
\hline
 &  &  &  &  &  &  &  & - &  &  &  &  &  &  &  &  &  &  &  &  &  \\
\hline
 &  &  &  &  &  &  &  &  &  &  &  &  &  &  &  &  &  &  &  &  &  \\
\hline
 &  &  &  &  &  &  &  &  &  &  &  &  &  &  &  &  &  &  &  &  &  \\
\hline
\end{tabular}
\end{center}

Zadanie 32. (0-2)\\
Rozwiąż równanie

\[
\frac{4}{x+2}=x-1
\]

\begin{center}
\includegraphics[max width=\textwidth]{2025_02_10_21a9f9e366fbb9d66090g-19}
\end{center}

\begin{center}
\begin{tabular}{|c|l|c|c|}
\hline
\multirow{2}{*}{\begin{tabular}{l}
Wypełnia \\
egzaminator \\
\end{tabular}} & Nr zadania & 31. & 32. \\
\cline { 2 - 4 }
 & Maks. liczba pkt & 2 & 2 \\
\cline { 2 - 4 }
 & Uzyskana liczba pkt &  &  \\
\hline
\end{tabular}
\end{center}

\section*{Zadanie 33. (0-2)}
Dany jest trójkąt równoboczny \(A B C\) o boku długości 24 . Punkt \(E\) leży na boku \(A B\), a punkt \(F\) - na boku \(B C\) tego trójkąta. Odcinek \(E F\) jest równoległy do boku \(A C\) i przechodzi przez środek \(S\) wysokości \(C D\) trójkąta \(A B C\) (zobacz rysunek). Oblicz długość odcinka \(E F\).\\
\includegraphics[max width=\textwidth, center]{2025_02_10_21a9f9e366fbb9d66090g-20}

Zadanie 34. (0-2)\\
Ze zbioru pięciu liczb \(\{-5,-4,1,2,3\}\) losujemy kolejno ze zwracaniem dwa razy po jednej liczbie. Zdarzenie \(A\) polega na wylosowaniu dwóch liczb, których iloczyn jest ujemny. Oblicz prawdopodobieństwo zdarzenia \(A\).\\
\includegraphics[max width=\textwidth, center]{2025_02_10_21a9f9e366fbb9d66090g-21}

\section*{Zadanie 35. (0-5)}
Dany jest graniastosłup prosty \(A B C D E F G H\), którego podstawą jest prostokąt \(A B C D\). W tym graniastosłupie \(|B D|=15\), a ponadto \(|C D|=3+|B C|\) oraz \(|\Varangle C D G|=60^{\circ}\) (zobacz rysunek).\\
Oblicz objętość i pole powierzchni bocznej tego graniastosłupa.\\
\includegraphics[max width=\textwidth, center]{2025_02_10_21a9f9e366fbb9d66090g-22}\\
\includegraphics[max width=\textwidth, center]{2025_02_10_21a9f9e366fbb9d66090g-22(1)}\\
\(\square\)

\begin{center}
\begin{tabular}{|c|l|c|}
\hline
\multirow{3}{*}{\begin{tabular}{c}
Wypełnia \\
egzaminator \\
\end{tabular}} & Nr zadania & 35. \\
\cline { 2 - 3 }
 & Maks. liczba pkt & 5 \\
\cline { 2 - 3 }
 & Uzyskana liczba pkt &  \\
\hline
\end{tabular}
\end{center}

BRUDNOPIS (nie podlega ocenie)\\
\includegraphics[max width=\textwidth, center]{2025_02_10_21a9f9e366fbb9d66090g-24}\\
\includegraphics[max width=\textwidth, center]{2025_02_10_21a9f9e366fbb9d66090g-25}


\end{document}