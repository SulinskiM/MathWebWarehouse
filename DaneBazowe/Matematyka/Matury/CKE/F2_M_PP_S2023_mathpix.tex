\documentclass[10pt]{article}
\usepackage[polish]{babel}
\usepackage[utf8]{inputenc}
\usepackage[T1]{fontenc}
\usepackage{graphicx}
\usepackage[export]{adjustbox}
\graphicspath{ {./images/} }
\usepackage{amsmath}
\usepackage{amsfonts}
\usepackage{amssymb}
\usepackage[version=4]{mhchem}
\usepackage{stmaryrd}

\author{LICZBA PUNKTÓW DO UZYSKANIA: 46}
\date{}


\begin{document}
\maketitle
CENTRALNA\\
KOMISJA\\
EGZAMINACYJNA

\section*{WYPEŁNIA ZDAJĄCY}
\section*{KOD}
\begin{center}
\includegraphics[max width=\textwidth]{2025_02_10_d79e1fe6707fdd3e9decg-01}
\end{center}

PESEL\\
\includegraphics[max width=\textwidth, center]{2025_02_10_d79e1fe6707fdd3e9decg-01(1)}

\section*{Miejsce na naklejke.}
Sprawdż, czy kod na naklejce to E-100.\\
Jeżeli tak - przyklej naklejke. Jeżeli nie - zgłoś to nauczycielowi.

\section*{Egzamin maturalny}
\section*{MATEMATYKA}
\section*{Poziom podstawowy}
\section*{Symbol arkusza}
EMAP-Po-100-2308

\section*{DAta: \(\mathbf{2 2}\) sierpnia 2023 r.}
 GODZINA ROZPOCZĘCIA: 9:00 Czas trwania: 170 minut

\section*{Przed rozpoczęciem pracy z arkuszem egzaminacyjnym}
\begin{enumerate}
  \item Sprawdź, czy nauczyciel przekazał Ci właściwy arkusz egzaminacyjny, tj. arkusz we właściwej formule, z właściwego przedmiotu na właściwym poziomie.
  \item Jeżeli przekazano Ci niewłaściwy arkusz - natychmiast zgłoś to nauczycielowi. Nie rozrywaj banderol.
  \item Jeżeli przekazano Ci właściwy arkusz - rozerwij banderole po otrzymaniu takiego polecenia od nauczyciela. Zapoznaj się z instrukcją na stronie 2.
\end{enumerate}

\section*{Instrukcja dla zdającego}
\begin{enumerate}
  \item Sprawdź, czy arkusz egzaminacyjny zawiera 30 stron (zadania 1-36). Ewentualny brak zgłoś przewodniczącemu zespołu nadzorującego egzamin.
  \item Na pierwszej stronie arkusza oraz na karcie odpowiedzi wpisz swój numer PESEL i przyklej naklejkę z kodem.
  \item Odpowiedzi do zadań zamkniętych (1-29) zaznacz na karcie odpowiedzi w części karty przeznaczonej dla zdającego. Zamaluj \(\square\) pola do tego przeznaczone. Błędne zaznaczenie otocz kółkiem © i zaznacz właściwe.
  \item Pamiętaj, że pominięcie argumentacji lub istotnych obliczeń w rozwiązaniu zadania otwartego (30-36) może spowodować, że za to rozwiązanie nie otrzymasz pełnej liczby punktów.
  \item Rozwiązania zadań i odpowiedzi wpisuj w miejscu na to przeznaczonym.
  \item Pisz czytelnie i używaj tylko długopisu lub pióra z czarnym tuszem lub atramentem.
  \item Nie używaj korektora, a błędne zapisy wyraźnie przekreśl.
  \item Nie wpisuj żadnych znaków w części przeznaczonej dla egzaminatora.
  \item Pamiętaj, że zapisy w brudnopisie nie będą oceniane.
  \item Możesz korzystać z Wybranych wzorów matematycznych, cyrkla i linijki oraz kalkulatora prostego. Upewnij się, czy przekazano Ci broszurę z okładką taką jak widoczna poniżej.\\
\includegraphics[max width=\textwidth, center]{2025_02_10_d79e1fe6707fdd3e9decg-02}
\end{enumerate}

\section*{Zadania egzaminacyjne są wydrukowane na następnych stronach.}
W każdym z zadań od 1. do 29. wybierz i zaznacz na karcie odpowiedzi poprawną odpowiedź.

\section*{Zadanie 1. (0-1)}
Liczba \(\log _{25} 1-\frac{1}{2} \log _{25} 5\) jest równa\\
A. \(\left(-\frac{1}{4}\right)\)\\
B. \(\left(-\frac{1}{2}\right)\)\\
C. \(\frac{1}{4}\)\\
D. \(\frac{1}{2}\)

\section*{Zadanie 2. (0-1)}
Liczba \(3 \sqrt{45}-\sqrt{20}\) jest równa\\
A. \((7 \cdot 5)^{\frac{1}{2}}\)\\
B. \(5^{\frac{1}{2}}\)\\
C. 7\\
D. \(7 \cdot 5^{\frac{1}{2}}\)

\section*{Zadanie 3. (0-1)}
W ramach wyprzedaży sezonowej płaszcz o początkowej wartości 240 zł przeceniono na 200 zł. Zatem cenę tego płaszcza obniżono o\\
A. \(16 \frac{2}{3} \%\) jego początkowej wartości.\\
B. \(20 \%\) jego początkowej wartości.\\
C. \(40 \%\) jego początkowej wartości.\\
D. \(83 \frac{1}{3} \%\) jego początkowej wartości.

\section*{Zadanie 4. (0-1)}
Wartość wyrażenia \(\frac{3^{-1}}{\left(-\frac{1}{9}\right)^{-2}} \cdot 81\) jest równa\\
A. \(\frac{1}{3}\)\\
B. \(\left(-\frac{1}{3}\right)\)\\
C. 3\\
D. \((-3)\)

BRUDNOPIS (nie podlega ocenie)\\
\includegraphics[max width=\textwidth, center]{2025_02_10_d79e1fe6707fdd3e9decg-05}

\section*{Zadanie 5. (0-1)}
Wartość wyrażenia \((2-\sqrt{3})^{2}-(\sqrt{3}-2)^{2}\) jest równa\\
A. \((-2 \sqrt{3})\)\\
B. 0\\
C. 6\\
D. \(8 \sqrt{3}\)

\section*{Zadanie 6. (0-1)}
W układzie współrzędnych \((x, y)\), punkt \((-8,6)\) jest punktem przecięcia prostych o równaniach\\
A. \(2 x+3 y=2 \quad\) i \(\quad-x+y=-14\).\\
B. \(3 x+2 y=-12\) i \(2 x+y=10\).\\
C. \(x+y=-2 \quad\) i \(\quad x-2 y=4\).\\
D. \(x-y=-14 \quad\) i \(\quad-2 x+y=22\).

\section*{Zadanie 7. (0-1)}
Zbiorem wszystkich rozwiązań nierówności

\[
-3(x-1) \leq \frac{5-3 x}{3}
\]

jest przedział\\
A. \(\left(-\infty, \frac{2}{3}\right)\)\\
B. \(\left(-\infty,-\frac{2}{3}\right)\)\\
C. \(\left(\frac{2}{3}, \infty\right)\)\\
D. \(\left(-\frac{2}{3}, \infty\right)\)

\section*{Zadanie 8. (0-1)}
Równanie \(\left(x^{2}-3 x\right)\left(x^{2}+1\right)=0 \mathrm{w}\) zbiorze liczb rzeczywistych ma dokładnie\\
A. jedno rozwiązanie.\\
B. dwa rozwiązania.\\
C. trzy rozwiązania.\\
D. cztery rozwiązania.

BRUDNOPIS (nie podlega ocenie)\\
\includegraphics[max width=\textwidth, center]{2025_02_10_d79e1fe6707fdd3e9decg-07}

\section*{Zadanie 9. (0-1)}
Funkcja \(f\) jest określona dla każdej liczby rzeczywistej \(x\) wzorem \(f(x)=\frac{x-k}{x^{2}+1}\), gdzie \(k\) jest pewną liczbą rzeczywistą. Ta funkcja spełnia warunek \(f(1)=2\).\\
Wartość współczynnika \(k\) we wzorze tej funkcji jest równa\\
A. \((-3)\)\\
B. 3\\
C. \((-4)\)\\
D. 4

\section*{Zadanie 10. (0-1)}
Miejscem zerowym funkcji liniowej \(f\) jest liczba 1 . Wykres tej funkcji przechodzi przez punkt ( \(-1,4\) ). Wzór funkcji \(f\) ma postać\\
A. \(f(x)=-\frac{1}{2} x+1\)\\
B. \(f(x)=-\frac{1}{3} x+\frac{1}{3}\)\\
C. \(f(x)=-2 x+2\)\\
D. \(f(x)=-3 x+1\)

\section*{Zadanie 11. (0-1)}
Funkcja kwadratowa \(f\) jest określona wzorem \(f(x)=(x-13)^{2}-256\). Jednym z miejsc zerowych tej funkcji jest liczba ( -3 ).\\
Drugim miejscem zerowym funkcji \(f\) jest liczba\\
A. \((-29)\)\\
B. \((-23)\)\\
C. 23\\
D. 29

BRUDNOPIS (nie podlega ocenie)\\
\includegraphics[max width=\textwidth, center]{2025_02_10_d79e1fe6707fdd3e9decg-09}

\section*{Informacja do zadań 12.-13.}
W układzie współrzędnych \((x, y)\) narysowano wykres funkcji \(y=f(x)\) (zobacz rysunek).\\
\includegraphics[max width=\textwidth, center]{2025_02_10_d79e1fe6707fdd3e9decg-10(1)}

\section*{Zadanie 12. (0-1)}
Funkcja \(f\) jest rosnąca w przedziale\\
A. \(\langle-5,4\rangle\)\\
B. \(\langle 5,7\rangle\)\\
C. \(\langle 1,5\rangle\)\\
D. \(\langle-1,5\rangle\)

\section*{Zadanie 13. (0-1)}
Funkcja \(g\) jest określona za pomocą funkcji \(f\) następująco: \(g(x)=f(-x)\) dla każdego \(x \in\langle-7,-5\rangle \cup\langle-4,4\rangle \cup\langle 5,7\rangle\). Na jednym z rysunków A-D przedstawiono, w układzie współrzędnych \((x, y)\), wykres funkcji \(y=g(x)\).\\
Wykres funkcji \(y=g(x)\) przedstawiono na rysunku\\
A.\\
B.\\
\includegraphics[max width=\textwidth, center]{2025_02_10_d79e1fe6707fdd3e9decg-10}\\
\includegraphics[max width=\textwidth, center]{2025_02_10_d79e1fe6707fdd3e9decg-10(2)}\\
C.\\
\includegraphics[max width=\textwidth, center]{2025_02_10_d79e1fe6707fdd3e9decg-11}\\
D.\\
\includegraphics[max width=\textwidth, center]{2025_02_10_d79e1fe6707fdd3e9decg-11(1)}

BRUDNOPIS (nie podlega ocenie)

\begin{center}
\begin{tabular}{|c|c|c|c|c|c|c|c|c|c|c|c|c|c|c|c|c|c|c|c|c|c|c|}
\hline
 &  &  &  &  &  &  &  &  &  &  &  &  &  &  &  &  &  &  &  &  &  &  \\
\hline
 &  &  &  &  &  &  &  &  &  &  &  &  &  &  &  &  &  &  &  &  &  &  \\
\hline
 &  &  &  &  &  &  &  &  &  &  &  &  &  &  &  &  &  &  &  &  &  &  \\
\hline
 &  &  &  &  &  &  &  &  &  &  &  &  &  &  &  &  &  &  &  &  &  &  \\
\hline
 &  &  &  &  &  &  &  &  &  &  &  &  &  &  &  &  &  &  &  &  &  &  \\
\hline
 &  &  &  &  &  &  &  &  &  &  &  &  &  &  &  &  &  &  &  &  &  &  \\
\hline
 &  &  &  &  &  &  &  &  &  &  &  &  &  &  &  &  &  &  &  &  &  &  \\
\hline
 &  &  &  &  &  &  &  &  &  &  &  &  &  &  &  &  &  &  &  &  &  &  \\
\hline
 &  &  &  &  &  &  &  &  &  &  &  &  &  &  &  &  &  &  &  &  &  &  \\
\hline
 &  &  &  &  &  &  &  &  &  &  &  &  &  &  &  &  &  &  &  &  &  &  \\
\hline
 &  &  &  &  &  &  &  &  &  &  &  &  &  &  &  &  &  &  &  &  &  &  \\
\hline
 &  &  &  &  &  &  &  &  &  &  &  &  &  &  &  &  &  &  &  &  &  &  \\
\hline
 &  &  &  &  &  &  &  &  &  &  &  &  &  &  &  &  &  &  &  &  &  &  \\
\hline
 &  &  &  &  &  &  &  &  &  &  &  &  &  &  &  &  &  &  &  &  &  &  \\
\hline
 &  &  &  &  &  &  &  &  &  &  &  &  &  &  &  &  &  &  &  &  &  &  \\
\hline
 &  &  &  &  &  &  &  &  &  &  &  &  &  &  &  &  &  &  &  &  &  &  \\
\hline
 &  &  &  &  &  &  &  &  &  &  &  &  &  &  &  &  &  &  &  &  &  &  \\
\hline
 &  &  &  &  &  &  &  &  &  &  &  &  &  &  &  &  &  &  &  &  &  &  \\
\hline
 &  &  &  &  &  &  &  &  &  &  &  &  &  &  &  &  &  &  &  &  &  &  \\
\hline
 &  &  &  &  &  &  &  &  &  &  &  &  &  &  &  &  &  &  &  &  &  &  \\
\hline
 &  &  &  &  &  &  &  &  &  &  &  &  &  &  &  &  &  &  &  &  &  &  \\
\hline
 &  &  &  &  &  &  &  &  &  &  &  &  &  &  &  &  &  &  &  &  &  &  \\
\hline
 &  &  &  &  &  &  &  &  &  &  &  &  &  &  &  &  &  &  &  &  &  &  \\
\hline
 &  &  &  &  &  &  &  &  &  &  &  &  &  &  &  &  &  &  &  &  &  &  \\
\hline
 &  &  &  &  &  &  &  &  &  &  &  &  &  &  &  &  &  &  &  &  &  &  \\
\hline
 &  &  &  &  &  &  &  &  &  &  &  &  &  &  &  &  &  &  &  &  &  &  \\
\hline
 &  &  &  &  &  &  &  &  &  &  &  &  &  &  &  &  &  &  &  &  &  &  \\
\hline
 &  &  &  &  &  &  &  &  &  &  &  &  &  &  &  &  &  &  &  &  &  &  \\
\hline
 &  &  &  &  &  &  &  &  &  &  &  &  &  &  &  &  &  &  &  &  &  &  \\
\hline
\end{tabular}
\end{center}

\section*{Zadanie 14. (0-1)}
Funkcja kwadratowa \(f\), określona wzorem \(f(x)=-(x-1)(x-5)\), przyjmuje wartość\\
A. najmniejszą równą 3 .\\
B. najmniejszą równą 4 .\\
C. największą równą 3 .\\
D. największą równą 4.

\section*{Zadanie 15. (0-1)}
Ciąg \(\left(a_{n}\right)\) jest określony wzorem \(a_{n}=(-1)^{n} \cdot \frac{n+1}{2}\) dla każdej liczby naturalnej \(n \geq 1\). Trzeci wyraz tego ciągu jest równy\\
A. 2\\
B. \((-2)\)\\
C. 3\\
D. \((-1)\)

\section*{Zadanie 16. (0-1)}
Czterowyrazowy ciąg ( \(-2,1, x, y\) ) jest geometryczny. Suma wszystkich wyrazów tego ciągu jest równa\\
A. \(\left(-\frac{5}{4}\right)\)\\
B. \((-4)\)\\
C. \(\left(-\frac{1}{4}\right)\)\\
D. \(\left(-\frac{15}{4}\right)\)

\section*{Zadanie 17. (0-1)}
Koło ma promień równy 3. Obwód wycinka tego koła o kącie środkowym \(30^{\circ}\) jest równy\\
A. \(\frac{3}{4} \pi\)\\
B. \(\frac{1}{2} \pi\)\\
C. \(\frac{3}{4} \pi+6\)\\
D. \(\frac{1}{2} \pi+6\)

\section*{Zadanie 18. (0-1)}
Kąt \(\alpha\) jest ostry i \(\cos \alpha=\frac{2 \sqrt{6}}{7}\). Sinus kąta \(\alpha\) jest równy\\
A. \(\frac{24}{49}\)\\
B. \(\frac{5}{7}\)\\
C. \(\frac{25}{49}\)\\
D. \(\frac{\sqrt{6}}{7}\)

BRUDNOPIS (nie podlega ocenie)\\
\includegraphics[max width=\textwidth, center]{2025_02_10_d79e1fe6707fdd3e9decg-13}

\section*{Zadanie 19. (0-1)}
W okręgu \(\mathcal{O}\) kąt środkowy \(\beta\) oraz kąt wpisany \(\alpha\) są oparte na tym samym łuku. Kąt \(\beta\) ma miarę o \(40^{\circ}\) większą od kąta \(\alpha\). Miara kąta \(\beta\) jest równa\\
A. \(40^{\circ}\)\\
B. \(80^{\circ}\)\\
C. \(100^{\circ}\)\\
D. \(120^{\circ}\)

\section*{Zadanie 20. (0-1)}
Pole trójkąta równobocznego o wysokości 3 jest równe\\
A. \(\frac{3 \sqrt{3}}{4}\)\\
B. \(\frac{9 \sqrt{3}}{4}\)\\
C. \(3 \sqrt{3}\)\\
D. \(6 \sqrt{3}\)

\section*{Zadanie 21. (0-1)}
Każdy z kątów wewnętrznych dziesięciokąta foremnego ma miarę\\
A. \(120^{\circ}\)\\
B. \(135^{\circ}\)\\
C. \(144^{\circ}\)\\
D. \(150^{\circ}\)

\section*{Zadanie 22. (0-1)}
Obwód trójkąta prostokątnego \(A B C\) jest równy \(L\). Na boku \(C B\) tego trójkąta obrano punkt \(E\), a na boku \(A B\) obrano punkt \(D\) tak, że \(D E \| A C\) oraz \(|A D|:|D B|=3: 4\) (zobacz rysunek).\\
\includegraphics[max width=\textwidth, center]{2025_02_10_d79e1fe6707fdd3e9decg-14}

Obwód trójkąta \(B E D\) jest równy\\
A. \(\frac{3}{4} L\)\\
B. \(\frac{3}{7} L\)\\
C. \(\frac{4}{7} L\)\\
D. \(\frac{1}{4} L\)

BRUDNOPIS (nie podlega ocenie)\\
\includegraphics[max width=\textwidth, center]{2025_02_10_d79e1fe6707fdd3e9decg-15}

\section*{Zadanie 23. (0-1)}
W układzie współrzędnych \((x, y)\) dane są prosta \(k\) o równaniu \(y=\frac{3}{4} x-\frac{7}{4}\) oraz punkt \(P=(12,-1)\).\\
Prosta przechodząca przez punkt \(P\) i równoległa do prostej \(k\) ma równanie\\
A. \(y=-\frac{3}{4} x+8\)\\
B. \(y=\frac{3}{4} x-10\)\\
C. \(y=\frac{4}{3} x-17\)\\
D. \(y=-\frac{4}{3} x+15\)

\section*{Zadanie 24. (0-1)}
W układzie współrzędnych \((x, y)\) punkt \(A=(-1,-4)\) jest wierzchołkiem równoległoboku \(A B C D\). Punkt \(S=(2,2)\) jest środkiem symetrii tego równoległoboku.\\
Długość przekątnej \(A C\) równoległoboku \(A B C D\) jest równa\\
A. \(\sqrt{5}\)\\
B. \(2 \sqrt{5}\)\\
C. \(3 \sqrt{5}\)\\
D. \(6 \sqrt{5}\)

\section*{Informacja do zadań 25.-26.}
Każda krawędź graniastosłupa prawidłowego sześciokątnego ma długość równą 6.

\section*{Zadanie 25. (0-1)}
Pole powierzchni całkowitej tego graniastosłupa jest równe\\
A. \(216+18 \sqrt{3}\)\\
B. \(216+54 \sqrt{3}\)\\
C. \(216+216 \sqrt{3}\)\\
D. \(216+108 \sqrt{3}\)

\section*{Zadanie 26. (0-1)}
Cosinus kąta nachylenia dłuższej przekątnej tego graniastosłupa do płaszczyzny podstawy graniastosłupa jest równy\\
A. \(\frac{1}{2}\)\\
B. \(\frac{2}{\sqrt{5}}\)\\
C. \(\frac{1}{\sqrt{5}}\)\\
D. \(\frac{\sqrt{3}}{2}\)

BRUDNOPIS (nie podlega ocenie)\\
\includegraphics[max width=\textwidth, center]{2025_02_10_d79e1fe6707fdd3e9decg-17}

\section*{Zadanie 27. (0-1)}
W ostrosłupie prawidłowym czworokątnym stosunek pola powierzchni bocznej do pola podstawy jest równy 12 . Wynika stąd, że w tym ostrosłupie stosunek wysokości ściany bocznej do krawędzi podstawy jest równy\\
A. 24\\
B. 3\\
C. 6\\
D. 4

\section*{Zadanie 28. (0-1)}
Na diagramie przedstawiono rozkład wynagrodzenia brutto wszystkich stu pracowników pewnej firmy za styczeń 2023 roku.\\
\includegraphics[max width=\textwidth, center]{2025_02_10_d79e1fe6707fdd3e9decg-18}\\
wynagrodzenie brutto za styczeń 2023 (w zł)

Średnia wynagrodzenia brutto wszystkich pracowników tej firmy za styczeń 2023 roku jest równa\\
A. \(5690 \mathrm{zł}\)\\
B. 5280 zt\\
C. \(6257 \mathrm{zł}\)\\
D. \(5900 \mathrm{zł}\)

\section*{Zadanie 29. (0-1)}
Wszystkich liczb naturalnych czterocyfrowych, w których zapisie dziesiętnym cyfry się nie powtarzaja, jest\\
A. \(9 \cdot 10 \cdot 10 \cdot 10 \cdot 10\)\\
B. 9.9 .9 .9\\
C. \(10 \cdot 9 \cdot 8 \cdot 7\)\\
D. \(9 \cdot 9 \cdot 8 \cdot 7\)

BRUDNOPIS (nie podlega ocenie)\\
\includegraphics[max width=\textwidth, center]{2025_02_10_d79e1fe6707fdd3e9decg-19}

Zadanie 30. (0-2)\\
Rozwiąż nierówność

\[
5-x^{2}>3 x+1
\]

\begin{center}
\begin{tabular}{|c|c|c|c|c|c|c|c|c|c|c|c|c|c|c|c|c|c|c|c|c|c|c|}
\hline
 &  &  &  &  &  &  &  &  &  &  &  &  &  &  &  &  &  &  &  &  &  &  \\
\hline
 &  &  &  &  &  &  &  &  &  &  &  &  &  &  &  &  &  &  &  &  &  &  \\
\hline
 &  &  &  &  &  &  &  &  &  &  &  &  &  &  &  &  &  &  &  &  &  &  \\
\hline
 &  &  &  &  &  &  &  &  &  &  &  &  &  &  &  &  &  &  &  &  &  &  \\
\hline
 &  &  &  &  &  &  &  &  &  &  &  &  &  &  &  &  &  &  &  &  &  &  \\
\hline
 &  &  &  &  &  &  &  &  &  &  &  &  &  &  &  &  &  &  &  &  &  &  \\
\hline
 &  &  &  &  &  &  &  &  &  &  &  &  &  &  &  &  &  &  &  &  &  &  \\
\hline
 &  &  &  &  &  &  &  &  &  &  &  &  &  &  &  &  &  &  &  &  &  &  \\
\hline
 &  &  &  &  &  &  &  &  &  &  &  &  &  &  &  &  &  &  &  &  &  &  \\
\hline
 &  &  &  &  &  &  &  &  &  &  &  &  &  &  &  &  &  &  &  &  &  &  \\
\hline
 &  &  &  &  &  &  &  &  &  &  &  &  &  &  &  &  &  &  &  &  &  &  \\
\hline
 &  &  &  &  &  &  &  &  &  &  &  &  &  &  &  &  &  &  &  &  &  &  \\
\hline
 &  &  &  &  &  &  &  &  &  &  &  &  &  &  &  &  &  &  &  &  &  &  \\
\hline
 &  &  &  &  &  &  &  &  &  &  &  &  &  &  &  &  &  &  &  &  &  &  \\
\hline
 &  &  &  &  &  &  &  &  &  &  &  &  &  &  &  &  &  &  &  &  &  &  \\
\hline
 &  &  &  &  &  &  &  &  &  &  &  &  &  &  &  &  &  &  &  &  &  &  \\
\hline
 &  &  &  &  &  &  &  &  &  &  &  &  &  &  &  &  &  &  &  &  &  &  \\
\hline
 &  &  &  &  &  &  &  &  &  &  &  &  &  &  &  &  &  &  &  &  &  &  \\
\hline
 &  &  &  &  &  &  &  &  &  &  &  &  &  &  &  &  &  &  &  &  &  &  \\
\hline
 &  &  &  &  &  &  &  &  &  &  &  &  &  &  &  &  &  &  &  &  &  &  \\
\hline
 &  &  &  &  &  &  &  &  &  &  &  &  &  &  &  &  &  &  &  &  &  &  \\
\hline
 &  &  &  &  &  &  &  &  &  &  &  &  &  &  &  &  &  &  &  &  &  &  \\
\hline
 &  &  &  &  &  &  &  &  &  &  &  &  &  &  &  &  &  &  &  &  &  &  \\
\hline
 &  &  &  &  &  &  &  &  &  &  &  &  &  &  &  &  &  &  &  &  &  &  \\
\hline
 &  &  &  &  &  &  &  &  &  &  &  &  &  &  &  &  &  &  &  &  &  &  \\
\hline
 &  &  &  &  &  &  &  &  &  &  &  &  &  &  &  &  &  &  &  &  &  &  \\
\hline
 &  &  &  &  &  &  &  &  &  &  &  &  &  &  &  &  &  &  &  &  &  &  \\
\hline
 &  &  &  &  &  &  &  &  &  &  &  &  &  &  &  &  &  &  &  &  &  &  \\
\hline
 &  &  &  &  &  &  &  &  &  &  &  &  &  &  &  &  &  &  &  &  &  &  \\
\hline
 &  &  &  &  &  &  &  &  &  &  &  &  &  &  &  &  &  &  &  &  &  &  \\
\hline
 &  &  &  &  &  &  &  &  &  &  &  &  &  &  &  &  &  &  &  &  &  &  \\
\hline
 &  &  &  &  &  &  &  &  &  &  &  &  &  &  &  &  &  &  &  &  &  &  \\
\hline
 &  &  &  &  &  &  &  &  &  &  &  &  &  &  &  &  &  &  &  &  &  &  \\
\hline
 &  &  &  &  &  &  &  &  &  &  &  &  &  &  &  &  &  &  &  &  &  &  \\
\hline
 &  &  &  &  &  &  &  &  &  &  &  &  &  &  &  &  &  &  &  &  &  &  \\
\hline
 &  &  &  &  &  &  &  &  &  &  &  &  &  &  &  &  &  &  &  &  &  &  \\
\hline
 &  &  &  &  &  &  &  &  &  &  &  &  &  &  &  &  &  &  &  &  &  &  \\
\hline
 &  &  &  &  &  &  &  &  &  &  &  &  &  &  &  &  &  &  &  &  &  &  \\
\hline
 &  &  &  &  &  &  &  &  &  &  &  &  &  &  &  &  &  &  &  &  &  &  \\
\hline
 &  &  &  &  &  &  &  &  &  &  &  &  &  &  &  &  &  &  &  &  &  &  \\
\hline
 &  &  &  &  &  &  &  &  &  &  &  &  &  &  &  &  &  &  &  &  &  &  \\
\hline
 &  &  &  &  &  &  &  &  &  &  &  &  &  &  &  &  &  &  &  &  &  &  \\
\hline
 &  &  &  &  &  &  &  &  &  &  &  &  &  &  &  &  &  &  &  &  &  &  \\
\hline
 &  &  &  &  &  &  &  &  &  &  &  &  &  &  &  &  &  &  &  &  &  &  \\
\hline
\end{tabular}
\end{center}

Zadanie 31. (0-2)\\
Ciąg \(\left(3 x^{2}+5 x, x^{2}, 20-x^{2}\right)\) jest arytmetyczny. Oblicz \(x\).\\
\includegraphics[max width=\textwidth, center]{2025_02_10_d79e1fe6707fdd3e9decg-21}

Zadanie 32. (0-2)\\
Wykaż, że dla każdej liczby rzeczywistej dodatniej \(x\) i dla każdej liczby rzeczywistej dodatniej \(y\) takiej, że \(x>2 y\), prawdziwa jest nierówność

\[
x^{2}+3 x y-10 y^{2}>0
\]

\begin{center}
\begin{tabular}{|c|c|c|c|c|c|c|c|c|c|c|c|c|c|c|c|c|c|c|c|c|c|}
\hline
 &  &  &  &  &  &  &  &  &  &  &  &  &  &  &  &  &  &  &  &  &  \\
\hline
 &  &  &  &  &  &  &  &  &  &  &  &  &  &  &  &  &  &  &  &  &  \\
\hline
 &  &  &  &  &  &  &  &  &  &  &  &  &  &  &  &  &  &  &  &  &  \\
\hline
 &  &  &  &  &  &  &  &  &  &  &  &  &  &  &  &  &  &  &  &  &  \\
\hline
 &  &  &  &  &  &  &  &  &  &  &  &  &  &  &  &  &  &  &  &  &  \\
\hline
 &  &  &  &  &  &  &  &  &  &  &  &  &  &  &  &  &  &  &  &  &  \\
\hline
 &  &  &  &  &  &  &  &  &  &  &  &  &  &  &  &  &  &  &  &  &  \\
\hline
 &  &  &  &  &  &  &  &  &  &  &  &  &  &  &  &  &  &  &  &  &  \\
\hline
 &  &  &  &  &  &  &  &  &  &  &  &  &  &  &  &  &  &  &  &  &  \\
\hline
 &  &  &  &  &  &  &  &  &  &  &  &  &  &  &  &  &  &  &  &  &  \\
\hline
 &  &  &  &  &  &  &  &  &  &  &  &  &  &  &  &  &  &  &  &  &  \\
\hline
 &  &  &  &  &  &  &  &  &  &  &  &  &  &  &  &  &  &  &  &  &  \\
\hline
 &  &  &  &  &  &  &  &  &  &  &  &  &  &  &  &  &  &  &  &  &  \\
\hline
 &  &  &  &  &  &  &  &  &  &  &  &  &  &  &  &  &  &  &  &  &  \\
\hline
 &  &  &  &  &  &  &  &  &  &  &  &  &  &  &  &  &  &  &  &  &  \\
\hline
 &  &  &  &  &  &  &  &  &  &  &  &  &  &  &  &  &  &  &  &  &  \\
\hline
 &  &  &  &  &  &  &  &  &  &  &  &  &  &  &  &  &  &  &  &  &  \\
\hline
 &  &  &  &  &  &  &  &  &  &  &  &  &  &  &  &  &  &  &  &  &  \\
\hline
 &  &  &  &  &  &  &  &  &  &  &  &  &  &  &  &  &  &  &  &  &  \\
\hline
 &  &  &  &  &  &  &  &  &  &  &  &  &  &  &  &  &  &  &  &  &  \\
\hline
 &  &  &  &  &  &  &  &  &  &  &  &  &  &  &  &  &  &  &  &  &  \\
\hline
 &  &  &  &  &  &  &  &  &  &  &  &  &  &  &  &  &  &  &  &  &  \\
\hline
 &  &  &  &  &  &  &  &  &  &  &  &  &  &  &  &  &  &  &  &  &  \\
\hline
 &  &  &  &  &  &  &  &  &  &  &  &  &  &  &  &  &  &  &  &  &  \\
\hline
 &  &  &  &  &  &  &  &  &  &  &  &  &  &  &  &  &  &  &  &  &  \\
\hline
 &  &  &  &  &  &  &  &  &  &  &  &  &  &  &  &  &  &  &  &  &  \\
\hline
 &  &  &  &  &  &  &  &  &  &  &  &  &  &  &  &  &  &  &  &  &  \\
\hline
 &  &  &  &  &  &  &  &  &  &  &  &  &  &  &  &  &  &  &  &  &  \\
\hline
 &  &  &  &  &  &  &  &  &  &  &  &  &  &  &  &  &  &  &  &  &  \\
\hline
 &  &  &  &  &  &  &  &  &  &  &  &  &  &  &  &  &  &  &  &  &  \\
\hline
 &  &  &  &  &  &  &  &  &  &  &  &  &  &  &  &  &  &  &  &  &  \\
\hline
 &  &  &  &  &  &  &  &  &  &  &  &  &  &  &  &  &  &  &  &  &  \\
\hline
 &  &  &  &  &  &  &  &  &  &  &  &  &  &  &  &  &  &  &  &  &  \\
\hline
 &  &  &  &  &  &  &  &  &  &  &  &  &  &  &  &  &  &  &  &  &  \\
\hline
 &  &  &  &  &  &  &  &  &  &  &  &  &  &  &  &  &  &  &  &  &  \\
\hline
 &  &  &  &  &  &  &  &  &  &  &  &  &  &  &  &  &  &  &  &  &  \\
\hline
 &  &  &  &  &  &  &  &  &  &  &  &  &  &  &  &  &  &  &  &  &  \\
\hline
 &  &  &  &  &  &  &  &  &  &  &  &  &  &  &  &  &  &  &  &  &  \\
\hline
 &  &  &  &  &  &  &  &  &  &  &  &  &  &  &  &  &  &  &  &  &  \\
\hline
 &  &  &  &  &  &  &  &  &  &  &  &  &  &  &  &  &  &  &  &  &  \\
\hline
 &  &  &  &  &  &  &  &  &  &  &  &  &  &  &  &  &  &  &  &  &  \\
\hline
 &  &  &  &  &  &  &  &  &  &  &  &  &  &  &  &  &  &  &  &  &  \\
\hline
 &  &  &  &  &  &  &  &  &  &  &  &  &  &  &  &  &  &  &  &  &  \\
\hline
\end{tabular}
\end{center}

Zadanie 33. (0-2)\\
Dany jest trapez równoramienny \(A B C D\), w którym podstawa \(C D\) ma długość 6 , ramię \(A D\) ma długość 4, a kąty \(B A D\) oraz \(A B C\) mają miarę \(60^{\circ}\) (zobacz rysunek).\\
\includegraphics[max width=\textwidth, center]{2025_02_10_d79e1fe6707fdd3e9decg-23}

Oblicz pole tego trapezu.\\
\includegraphics[max width=\textwidth, center]{2025_02_10_d79e1fe6707fdd3e9decg-23(1)}

Zadanie 34. (0-2)\\
Rozwiąż równanie

\[
\frac{2 x-3}{3 x-2}=\frac{1}{2 x}
\]

\begin{center}
\begin{tabular}{|c|c|c|c|c|c|c|c|c|c|c|c|c|c|c|c|c|c|c|c|c|}
\hline
 &  &  &  &  &  &  &  &  &  &  &  &  &  &  &  &  &  &  &  &  \\
\hline
 &  &  &  &  &  &  &  &  &  &  &  &  &  &  &  &  &  &  &  &  \\
\hline
 &  &  &  &  &  &  &  &  &  &  &  &  &  &  &  &  &  &  &  &  \\
\hline
 &  &  &  &  &  &  &  &  &  &  &  &  &  &  &  &  &  &  &  &  \\
\hline
 &  &  &  &  &  &  &  &  &  &  &  &  &  &  &  &  &  &  &  &  \\
\hline
 &  &  &  &  &  &  &  &  &  &  &  &  &  &  &  &  &  &  &  &  \\
\hline
 &  &  &  &  &  &  &  &  &  &  &  &  &  &  &  &  &  &  &  &  \\
\hline
 &  &  &  &  &  &  &  &  &  &  &  &  &  &  &  &  &  &  &  &  \\
\hline
 &  &  &  &  &  &  &  &  &  &  &  &  &  &  &  &  &  &  &  &  \\
\hline
 &  &  &  &  &  &  &  &  &  &  &  &  &  &  &  &  &  &  &  &  \\
\hline
 &  &  &  &  &  &  &  &  &  &  &  &  &  &  &  &  &  &  &  &  \\
\hline
 &  &  &  &  &  &  &  &  &  &  &  &  &  &  &  &  &  &  &  &  \\
\hline
 &  &  &  &  &  &  &  &  &  &  &  &  &  &  &  &  &  &  &  &  \\
\hline
 &  &  &  &  &  &  &  &  &  &  &  &  &  &  &  &  &  &  &  &  \\
\hline
 &  &  &  &  &  &  &  &  &  &  &  &  &  &  &  &  &  &  &  &  \\
\hline
 &  &  &  &  &  &  &  &  &  &  &  &  &  &  &  &  &  &  &  &  \\
\hline
 &  &  &  &  &  &  &  &  &  &  &  &  &  &  &  &  &  &  &  &  \\
\hline
 &  &  &  &  &  &  &  &  &  &  &  &  &  &  &  &  &  &  &  &  \\
\hline
 &  &  &  &  &  &  &  &  &  &  &  &  &  &  &  &  &  &  &  &  \\
\hline
 &  &  &  &  &  &  &  &  &  &  &  &  &  &  &  &  &  &  &  &  \\
\hline
 &  &  &  &  &  &  &  &  &  &  &  &  &  &  &  &  &  &  &  &  \\
\hline
 &  &  &  &  &  &  &  &  &  &  &  &  &  &  &  &  &  &  &  &  \\
\hline
 &  &  &  &  &  &  &  &  &  &  &  &  &  &  &  &  &  &  &  &  \\
\hline
 &  &  &  &  &  &  &  &  &  &  &  &  &  &  &  &  &  &  &  &  \\
\hline
 &  &  &  &  &  &  &  &  &  &  &  &  &  &  &  &  &  &  &  &  \\
\hline
 &  &  &  &  &  &  &  &  &  &  &  &  &  &  &  &  &  &  &  &  \\
\hline
 &  &  &  &  &  &  &  &  &  &  &  &  &  &  &  &  &  &  &  &  \\
\hline
 &  &  &  &  &  &  &  &  &  &  &  &  &  &  &  &  &  &  &  &  \\
\hline
 &  &  &  &  &  &  &  &  &  &  &  &  &  &  &  &  &  &  &  &  \\
\hline
 &  &  &  &  &  &  &  &  &  &  &  &  &  &  &  &  &  &  &  &  \\
\hline
 &  &  &  &  &  &  &  &  &  &  &  &  &  &  &  &  &  &  &  &  \\
\hline
 &  &  &  &  &  &  &  &  &  &  &  &  &  &  &  &  &  &  &  &  \\
\hline
 &  &  &  &  &  &  &  &  &  &  &  &  &  &  &  &  &  &  &  &  \\
\hline
 &  &  &  &  &  &  &  &  &  &  &  &  &  &  &  &  &  &  &  &  \\
\hline
 &  &  &  &  &  &  &  &  &  &  &  &  &  &  &  &  &  &  &  &  \\
\hline
 &  &  &  &  &  &  &  &  &  &  &  &  &  &  &  &  &  &  &  &  \\
\hline
 &  &  &  &  &  &  &  &  &  &  &  &  &  &  &  &  &  &  &  &  \\
\hline
 &  &  &  &  &  &  &  &  &  &  &  &  &  &  &  &  &  &  &  &  \\
\hline
 &  &  &  &  &  &  &  &  &  &  &  &  &  &  &  &  &  &  &  &  \\
\hline
 &  &  &  &  &  &  &  &  &  &  &  &  &  &  &  &  &  &  &  &  \\
\hline
 &  &  &  &  &  &  &  &  &  &  &  &  &  &  &  &  &  &  &  &  \\
\hline
 &  &  &  &  &  &  &  &  &  &  &  &  &  &  &  &  &  &  &  &  \\
\hline
 &  &  &  &  &  &  &  &  &  &  &  &  &  &  &  &  &  &  &  &  \\
\hline
 &  &  &  &  &  &  &  &  &  &  &  &  &  &  &  &  &  &  &  &  \\
\hline
\end{tabular}
\end{center}

Zadanie 35. (0-2)\\
Ze zbioru pięciu liczb \(\{1,2,3,4,5\}\) losujemy bez zwracania kolejno dwa razy po jednej liczbie.

Oblicz prawdopodobieństwo zdarzenia \(A\) polegającego na tym, że obie wylosowane liczby są nieparzyste.\\
\includegraphics[max width=\textwidth, center]{2025_02_10_d79e1fe6707fdd3e9decg-25}

\section*{Zadanie 36. (0-5)}
Punkty \(A=\left(\frac{22}{5},-\frac{21}{5}\right), B=(6,7)\) oraz \(C=(-9,2)\) są wierzchołkami trójkąta \(A B C\). Symetralna boku \(A B\) tego trójkąta przecina bok \(B C\) w punkcie \(D\). Oblicz wspótrzędne punktu \(D\).

\begin{center}
\begin{tabular}{|c|c|c|c|c|c|c|c|c|c|c|c|c|c|c|c|c|c|c|c|c|c|c|c|c|}
\hline
 &  &  &  &  &  &  &  &  &  &  &  &  &  &  &  &  &  &  &  &  &  &  &  &  \\
\hline
 &  &  &  &  &  &  &  &  &  &  &  &  &  &  &  &  &  &  &  &  &  &  &  &  \\
\hline
 &  &  &  &  &  &  &  &  &  &  &  &  &  &  &  &  &  &  &  &  &  &  &  &  \\
\hline
 &  &  &  &  &  &  &  &  &  &  &  &  &  &  &  &  &  &  &  &  &  &  &  &  \\
\hline
 &  &  &  &  &  &  &  &  &  &  &  &  &  &  &  &  &  &  &  &  &  &  &  &  \\
\hline
 &  &  &  &  &  &  &  &  &  &  &  &  &  &  &  &  &  &  &  &  &  &  &  &  \\
\hline
 &  &  &  &  &  &  &  &  &  &  &  &  &  &  &  &  &  &  &  &  &  &  &  &  \\
\hline
 &  &  &  &  &  &  &  &  &  &  &  &  &  &  &  &  &  &  &  &  &  &  &  &  \\
\hline
 &  &  &  &  &  &  &  &  &  &  &  &  &  &  &  &  &  &  &  &  &  &  &  &  \\
\hline
 &  &  &  &  &  &  &  &  &  &  &  &  &  &  &  &  &  &  &  &  &  &  &  &  \\
\hline
 &  &  &  &  &  &  &  &  &  &  &  &  &  &  &  &  &  &  &  &  &  &  &  &  \\
\hline
 &  &  &  &  &  &  &  &  &  &  &  &  &  &  &  &  &  &  &  &  &  &  &  &  \\
\hline
 &  &  &  &  &  &  &  &  &  &  &  &  &  &  &  &  &  &  &  &  &  &  &  &  \\
\hline
 &  &  &  &  &  &  &  &  &  &  &  &  &  &  &  &  &  &  &  &  &  &  &  &  \\
\hline
 &  &  &  &  &  &  &  &  &  &  &  &  &  &  &  &  &  &  &  &  &  &  &  &  \\
\hline
 &  &  &  &  &  &  &  &  &  &  &  &  &  &  &  &  &  &  &  &  &  &  &  &  \\
\hline
 &  &  &  &  &  &  &  &  &  &  &  &  &  &  &  &  &  &  &  &  &  &  &  &  \\
\hline
 &  &  &  &  &  &  &  &  &  &  &  &  &  &  &  &  &  &  &  &  &  &  &  &  \\
\hline
 &  &  &  &  &  &  &  &  &  &  &  &  &  &  &  &  &  &  &  &  &  &  &  &  \\
\hline
 &  &  &  &  &  &  &  &  &  &  &  &  &  &  &  &  &  &  &  &  &  &  &  &  \\
\hline
 &  &  &  &  &  &  &  &  &  &  &  &  &  &  &  &  &  &  &  &  &  &  &  &  \\
\hline
 &  &  &  &  &  &  &  &  &  &  &  &  &  &  &  &  &  &  &  &  &  &  &  &  \\
\hline
 &  &  &  &  &  &  &  &  &  &  &  &  &  &  &  &  &  &  &  &  &  &  &  &  \\
\hline
 &  &  &  &  &  &  &  &  &  &  &  &  &  &  &  &  &  &  &  &  &  &  &  &  \\
\hline
 &  &  &  &  &  &  &  &  &  &  &  &  &  &  &  &  &  &  &  &  &  &  &  &  \\
\hline
 &  &  &  &  &  &  &  &  &  &  &  &  &  &  &  &  &  &  &  &  &  &  &  &  \\
\hline
 &  &  &  &  &  &  &  &  &  &  &  &  &  &  &  &  &  &  &  &  &  &  &  &  \\
\hline
 &  &  &  &  &  &  &  &  &  &  &  &  &  &  &  &  &  &  &  &  &  &  &  &  \\
\hline
 &  &  &  &  &  &  &  &  &  &  &  &  &  &  &  &  &  &  &  &  &  &  &  &  \\
\hline
 &  &  &  &  &  &  &  &  &  &  &  &  &  &  &  &  &  &  &  &  &  &  &  &  \\
\hline
 &  &  &  &  &  &  &  &  &  &  &  &  &  &  &  &  &  &  &  &  &  &  &  &  \\
\hline
 &  &  &  &  &  &  &  &  &  &  &  &  &  &  &  &  &  &  &  &  &  &  &  &  \\
\hline
 &  &  &  &  &  &  &  &  &  &  &  &  &  &  &  &  &  &  &  &  &  &  &  &  \\
\hline
 &  &  &  &  &  &  &  &  &  &  &  &  &  &  &  &  &  &  &  &  &  &  &  &  \\
\hline
 &  &  &  &  &  &  &  &  &  &  &  &  &  &  &  &  &  &  &  &  &  &  &  &  \\
\hline
 &  &  &  &  &  &  &  &  &  &  &  &  &  &  &  &  &  &  &  &  &  &  &  &  \\
\hline
 &  &  &  &  &  &  &  &  &  &  &  &  &  &  &  &  &  &  &  &  &  &  &  &  \\
\hline
 &  &  &  &  &  &  &  &  &  &  &  &  &  &  &  &  &  &  &  &  &  &  &  &  \\
\hline
 &  &  &  &  &  &  &  &  &  &  &  &  &  &  &  &  &  &  &  &  &  &  &  &  \\
\hline
 &  &  &  &  &  &  &  &  &  &  &  &  &  &  &  &  &  &  &  &  &  &  &  &  \\
\hline
 &  &  &  &  &  &  &  &  &  &  &  &  &  &  &  &  &  &  &  &  &  &  &  &  \\
\hline
 &  &  &  &  &  &  &  &  &  &  &  &  &  &  &  &  &  &  &  &  &  &  &  &  \\
\hline
 &  &  &  &  &  &  &  &  &  &  &  &  &  &  &  &  &  &  &  &  &  &  &  &  \\
\hline
 &  &  &  &  &  &  &  &  &  &  &  &  &  &  &  &  &  &  &  &  &  &  &  &  \\
\hline
\end{tabular}
\end{center}

\begin{center}
\includegraphics[max width=\textwidth]{2025_02_10_d79e1fe6707fdd3e9decg-27}
\end{center}

BRUDNOPIS (nie podlega ocenie)

\begin{center}
\begin{tabular}{|c|c|c|c|c|c|c|c|c|c|c|c|c|c|c|c|c|c|c|c|c|c|c|}
\hline
 &  &  &  &  &  &  &  &  &  &  &  &  &  &  &  &  &  &  &  &  &  &  \\
\hline
 &  &  &  &  &  &  &  &  &  &  &  &  &  &  &  &  &  &  &  &  &  &  \\
\hline
 &  &  &  &  &  &  &  &  &  &  &  &  &  &  &  &  &  &  &  &  &  &  \\
\hline
 &  &  &  &  &  &  &  &  &  &  &  &  &  &  &  &  &  &  &  &  &  &  \\
\hline
 &  &  &  &  &  &  &  &  &  &  &  &  &  &  &  &  &  &  &  &  &  &  \\
\hline
 &  &  &  &  &  &  &  &  &  &  &  &  &  &  &  &  &  &  &  &  &  &  \\
\hline
 &  &  &  &  &  &  &  &  &  &  &  &  &  &  &  &  &  &  &  &  &  &  \\
\hline
 &  &  &  &  &  &  &  &  &  &  &  &  &  &  &  &  &  &  &  &  &  &  \\
\hline
 &  &  &  &  &  &  &  &  &  &  &  &  &  &  &  &  &  &  &  &  &  &  \\
\hline
 &  &  &  &  &  &  &  &  &  &  &  &  &  &  &  &  &  &  &  &  &  &  \\
\hline
 &  &  &  &  &  &  &  &  &  &  &  &  &  &  &  &  &  &  &  &  &  &  \\
\hline
 &  &  &  &  &  &  &  &  &  &  &  &  &  &  &  &  &  &  &  &  &  &  \\
\hline
 &  &  &  &  &  &  &  &  &  &  &  &  &  &  &  &  &  &  &  &  &  &  \\
\hline
 &  &  &  &  &  &  &  &  &  &  &  &  &  &  &  &  &  &  &  &  &  &  \\
\hline
 &  &  &  &  &  &  &  &  &  &  &  &  &  &  &  &  &  &  &  &  &  &  \\
\hline
 &  &  &  &  &  &  &  &  &  &  &  &  &  &  &  &  &  &  &  &  &  &  \\
\hline
 &  &  &  &  &  &  &  &  &  &  &  &  &  &  &  &  &  &  &  &  &  &  \\
\hline
 &  &  &  &  &  &  &  &  &  &  &  &  &  &  &  &  &  &  &  &  &  &  \\
\hline
 &  &  &  &  &  &  &  &  &  &  &  &  &  &  &  &  &  &  &  &  &  &  \\
\hline
 &  &  &  &  &  &  &  &  &  &  &  &  &  &  &  &  &  &  &  &  &  &  \\
\hline
 &  &  &  &  &  &  &  &  &  &  &  &  &  &  &  &  &  &  &  &  &  &  \\
\hline
 &  &  &  &  &  &  &  &  &  &  &  &  &  &  &  &  &  &  &  &  &  &  \\
\hline
 &  &  &  &  &  &  &  &  &  &  &  &  &  &  &  &  &  &  &  &  &  &  \\
\hline
 &  &  &  &  &  &  &  &  &  &  &  &  &  &  &  &  &  &  &  &  &  &  \\
\hline
 &  &  &  &  &  &  &  &  &  &  &  &  &  &  &  &  &  &  &  &  &  &  \\
\hline
 &  &  &  &  &  &  &  &  &  &  &  &  &  &  &  &  &  &  &  &  &  &  \\
\hline
 &  &  &  &  &  &  &  &  &  &  &  &  &  &  &  &  &  &  &  &  &  &  \\
\hline
 &  &  &  &  &  &  &  &  &  &  &  &  &  &  &  &  &  &  &  &  &  &  \\
\hline
 &  &  &  &  &  &  &  &  &  &  &  &  &  &  &  &  &  &  &  &  &  &  \\
\hline
 &  &  &  &  &  &  &  &  &  &  &  &  &  &  &  &  &  &  &  &  &  &  \\
\hline
 &  &  &  &  &  &  &  &  &  &  &  &  &  &  &  &  &  &  &  &  &  &  \\
\hline
 &  &  &  &  &  &  &  &  &  &  &  &  &  &  &  &  &  &  &  &  &  &  \\
\hline
 &  &  &  &  &  &  &  &  &  &  &  &  &  &  &  &  &  &  &  &  &  &  \\
\hline
 &  &  &  &  &  &  &  &  &  &  &  &  &  &  &  &  &  &  &  &  &  &  \\
\hline
 &  &  &  &  &  &  &  &  &  &  &  &  &  &  &  &  &  &  &  &  &  &  \\
\hline
 &  &  &  &  &  &  &  &  &  &  &  &  &  &  &  &  &  &  &  &  &  &  \\
\hline
 &  &  &  &  &  &  &  &  &  &  &  &  &  &  &  &  &  &  &  &  &  &  \\
\hline
 &  &  &  &  &  &  &  &  &  &  &  &  &  &  &  &  &  &  &  &  &  &  \\
\hline
 &  &  &  &  &  &  &  &  &  &  &  &  &  &  &  &  &  &  &  &  &  &  \\
\hline
 &  &  &  &  &  &  &  &  &  &  &  &  &  &  &  &  &  &  &  &  &  &  \\
\hline
- &  &  &  &  &  &  &  &  &  &  &  &  &  &  &  &  &  &  &  &  &  &  \\
\hline
 &  &  &  &  &  &  &  &  &  &  &  &  &  &  &  &  &  &  &  &  &  &  \\
\hline
 &  &  &  &  &  &  &  &  &  &  &  &  &  &  &  &  &  &  &  &  &  &  \\
\hline
 &  &  &  &  &  &  &  &  &  &  &  &  &  &  &  &  &  &  &  &  &  &  \\
\hline
 &  &  &  &  &  &  &  &  &  &  &  &  &  &  &  &  &  &  &  &  &  &  \\
\hline
 &  &  &  &  &  &  &  &  &  &  &  &  &  &  &  &  &  &  &  &  &  &  \\
\hline
 &  &  &  &  &  &  &  &  &  &  &  &  &  &  &  &  &  &  &  &  &  &  \\
\hline
 &  &  &  &  &  &  &  &  &  &  &  &  &  &  &  &  &  &  &  &  &  &  \\
\hline
\end{tabular}
\end{center}

\includegraphics[max width=\textwidth, center]{2025_02_10_d79e1fe6707fdd3e9decg-29}\\
\includegraphics[max width=\textwidth, center]{2025_02_10_d79e1fe6707fdd3e9decg-30}

\section*{MATEMATYKA}
\section*{Poziom podstawowy}
\section*{Formuła 2015}
\section*{MATEMATYKA}
\section*{Poziom podstawowy}
Formuła 2015

\section*{MATEMATYKA}
\section*{Poziom podstawowy}
Formuła 2015


\end{document}