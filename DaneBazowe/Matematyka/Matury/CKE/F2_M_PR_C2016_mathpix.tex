\documentclass[10pt]{article}
\usepackage[polish]{babel}
\usepackage[utf8]{inputenc}
\usepackage[T1]{fontenc}
\usepackage{graphicx}
\usepackage[export]{adjustbox}
\graphicspath{ {./images/} }
\usepackage{amsmath}
\usepackage{amsfonts}
\usepackage{amssymb}
\usepackage[version=4]{mhchem}
\usepackage{stmaryrd}

\author{Data: \(\mathbf{3}\) czerwca 2016 r.\\
Godzina rozpoczeqia: 14:00\\
CZAS PRACY: \(\mathbf{1 8 0}\) minut\\
LicZba punktów do uzyskania: 50}
\date{}


\newcommand\Varangle{\mathop{{<\!\!\!\!\!\text{\small)}}\:}\nolimits}

\begin{document}
\maketitle
\includegraphics[max width=\textwidth, center]{2025_02_10_8c71b85d173a743a8718g-01(1)}\\
dysleksja

\section*{EGZAMIN MATURALNY Z MATEMATYKI POZIOM ROZSZERZONY}


\section*{Instrukcja dla zdającego}
\begin{enumerate}
  \item Sprawdź, czy arkusz egzaminacyjny zawiera 22 strony (zadania 1-17). Ewentualny brak zgłoś przewodniczącemu zespołu nadzorującego egzamin.
  \item Rozwiązania zadań i odpowiedzi wpisuj w miejscu na to przeznaczonym.
  \item Odpowiedzi do zadań zamkniętych (1-5) zaznacz na karcie odpowiedzi, w części karty przeznaczonej dla zdającego. Zamaluj \(\square\) pola do tego przeznaczone. Błędne zaznaczenie otocz kółkiem i zaznacz właściwe.
  \item W zadaniach kodowanych (6-7) wpisz właściwe cyfry w kratkach umieszczonych pod treścią zadania.
  \item Pamiętaj, że pominięcie argumentacji lub istotnych obliczeń w rozwiązaniu zadania otwartego (8-17) może spowodować, że za to rozwiązanie nie otrzymasz pełnej liczby punktów.
  \item Pisz czytelnie i używaj tylko długopisu lub pióra z czarnym tuszem lub atramentem.
  \item Nie używaj korektora, a błędne zapisy wyraźnie przekreśl.
  \item Pamiętaj, że zapisy w brudnopisie nie będą oceniane.
  \item Możesz korzystać z zestawu wzorów matematycznych, cyrkla i linijki oraz kalkulatora prostego.
  \item Na tej stronie oraz na karcie odpowiedzi wpisz swój numer PESEL i przyklej naklejkę z kodem.
  \item Nie wpisuj żadnych znaków w części przeznaczonej dla egzaminatora.\\
\includegraphics[max width=\textwidth, center]{2025_02_10_8c71b85d173a743a8718g-01}
\end{enumerate}

W zadaniach od 1. do 5. wybierz i zaznacz na karcie odpowiedzi poprawna odpowiedź.

\section*{Zadanie 1. (0-1)}
Funkcja \(f\) określona jest wzorem \(f(x)=\left|3+5^{3-x}\right|-1\) dla każdej liczby rzeczywistej. Zbiorem wartości funkcjif jest\\
A. \((2,+\infty)\)\\
B. \(\langle 1,3\rangle\)\\
C. \((-1,+\infty)\)\\
D. \((0,+\infty)\)

\section*{Zadanie 2. (0-1)}
Wartość wyrażenia \(\sin ^{2} 75^{\circ}-\cos ^{2} 75^{\circ}\) jest równa\\
A. \(-\frac{1}{2}\)\\
B. \(\frac{1}{2}\)\\
C. \(-\frac{\sqrt{3}}{2}\)\\
D. \(\frac{\sqrt{3}}{2}\)

\section*{Zadanie 3. (0-1)}
W trapezie \(A B C D\) o podstawach \(A B\) i \(C D\) dane są: \(|A D|=6,|B C|=12,|A C|=10\) oraz \(|\Varangle A B C|=|\Varangle C A D|\) (zobacz rysunek).\\
\includegraphics[max width=\textwidth, center]{2025_02_10_8c71b85d173a743a8718g-02}

Wówczas długość podstawy \(A B\) tego trapezu jest równa\\
A. \(|A B|=18\)\\
B. \(|A B|=20\)\\
C. \(|A B|=22\)\\
D. \(|A B|=24\)

\section*{Zadanie 4. (0-1)}
W ostrosłupie prawidłowym czworokątnym wszystkie krawędzie mają jednakową długość. Wynika stąd, że cosinus kąta nachylenia ściany bocznej do płaszczyzny podstawy tego ostrosłupa jest równy\\
A. \(\frac{\sqrt{3}}{3}\)\\
B. \(\frac{\sqrt{3}}{2}\)\\
C. \(\frac{1}{2}\)\\
D. \(\frac{1}{3}\)

\section*{Zadanie 5. (0-1)}
Granica \(\lim _{n \rightarrow \infty} \frac{-7 n^{3}+3 n}{1+2 n+3 n^{2}+4 n^{5}}\) jest równa\\
A. \(-\infty\)\\
B. \(-\frac{7}{4}\)\\
C. 0\\
D. \(+\infty\)

BRUDNOPIS (nie podlega ocenie)\\
\includegraphics[max width=\textwidth, center]{2025_02_10_8c71b85d173a743a8718g-03}

\section*{Zadanie 6. (0-2)}
Dany jest nieskończony ciąg geometryczny \(\left(a_{n}\right)\) określony dla \(n \geq 1\), w którym iloraz jest równy pierwszemu wyrazowi, a suma wszystkich wyrazów tego ciągu jest równa 12 . Oblicz pierwszy wyraz tego ciągu. Zakoduj kolejno pierwsze trzy cyfry po przecinku otrzymanego wyniku.\\
\includegraphics[max width=\textwidth, center]{2025_02_10_8c71b85d173a743a8718g-04(1)}

\section*{Zadanie 7. (0-2)}
Dane są zdarzenia losowe \(A, B \subset \Omega\) takie, że \(P(A)=\frac{2}{7}\) i \(P(A \cup B)=\frac{3}{5}\). Oblicz \(P(B \backslash A)\), gdzie zdarzenie \(B \backslash A\) oznacza różnicę zdarzeń \(B\) i \(A\). Zakoduj kolejno pierwsze trzy cyfry po przecinku rozwinięcia dziesiętnego otrzymanego wyniku.\\
\includegraphics[max width=\textwidth, center]{2025_02_10_8c71b85d173a743a8718g-04}

\section*{BRUDNOPIS (nie podlega ocenie)}
\begin{center}
\includegraphics[max width=\textwidth]{2025_02_10_8c71b85d173a743a8718g-04(2)}
\end{center}

Zadanie 8. (0-4)\\
Wykaż, że dla \(a, b, c, d>0\) prawdziwa jest nierówność \(\sqrt{a+b} \cdot \sqrt{c+d} \geq \sqrt{a c}+\sqrt{b d}\).\\
\(\qquad\)\\
\includegraphics[max width=\textwidth, center]{2025_02_10_8c71b85d173a743a8718g-05}\\
\(\qquad\)\\
\includegraphics[max width=\textwidth, center]{2025_02_10_8c71b85d173a743a8718g-05(4)}\\
\includegraphics[max width=\textwidth, center]{2025_02_10_8c71b85d173a743a8718g-05(3)}\\
\(\qquad\)\\
\includegraphics[max width=\textwidth, center]{2025_02_10_8c71b85d173a743a8718g-05(5)}\\
\includegraphics[max width=\textwidth, center]{2025_02_10_8c71b85d173a743a8718g-05(2)}\\
\(\qquad\)

\includegraphics[max width=\textwidth]{2025_02_10_8c71b85d173a743a8718g-05(1)} \begin{tabular}{|l|l|l|l|l|l|l|l|l|l|l|l|l|}
 &  &  &  &  &  &  &  &  &  &  &  &  \\
\hline
\end{tabular}

\(\qquad\)\\
\includegraphics[max width=\textwidth, center]{2025_02_10_8c71b85d173a743a8718g-05(6)}

\includegraphics[max width=\textwidth]{2025_02_10_8c71b85d173a743a8718g-05(7)} \begin{tabular}{|l|l|l|l|l|l|l|l|l|l|l|l|l|l|}
\hline
 &  &  &  &  &  &  &  &  &  &  &  &  &  \\
\hline
 &  &  &  &  &  &  &  &  &  &  &  &  &  \\
\hline
\end{tabular}

Zadanie 9. (0-4)\\
Rozwiąż nierówność \(\left|x^{2}-3 x+2\right| \geq|x-1|\).\\
\includegraphics[max width=\textwidth, center]{2025_02_10_8c71b85d173a743a8718g-06}\\
\includegraphics[max width=\textwidth, center]{2025_02_10_8c71b85d173a743a8718g-07}

Odpowiedź:

\section*{Zadanie 10. (0-3)}
Dany jest ciąg \(\left(a_{n}\right)\) określony dla każdej liczby całkowitej \(n \geq 1\), w którym \(a_{4}=4\) oraz dla każdej liczby \(n \geq 1\) prawdziwa jest równość \(a_{n+1}=a_{n}+n-4\). Oblicz pierwszy wyraz ciągu \(\left(a_{n}\right)\) i ustal, czy ciąg ten jest malejący.

\begin{center}
\begin{tabular}{|c|c|c|c|c|c|c|c|c|c|c|c|c|c|c|c|c|c|c|c|c|}
\hline
 &  &  &  &  &  &  &  &  &  &  &  &  &  &  &  &  &  &  &  &  \\
\hline
 &  &  &  &  &  &  &  &  &  &  &  &  &  &  &  &  &  &  &  &  \\
\hline
 &  &  &  &  &  &  &  &  &  &  &  &  &  &  &  &  &  &  &  &  \\
\hline
 &  &  &  &  &  &  &  &  &  &  &  &  &  &  &  &  &  &  &  &  \\
\hline
 &  &  &  &  &  &  &  &  &  &  &  &  &  &  &  &  &  &  &  &  \\
\hline
 &  &  &  &  &  &  &  &  &  &  &  &  &  &  &  &  &  &  &  &  \\
\hline
 &  &  &  &  &  &  &  &  &  &  &  &  &  &  &  &  &  &  &  &  \\
\hline
 &  &  &  &  &  &  &  &  &  &  &  &  &  &  &  &  &  &  &  &  \\
\hline
 &  &  &  &  &  &  &  &  &  &  &  &  &  &  &  &  &  &  &  &  \\
\hline
 &  &  &  &  &  &  &  &  &  &  &  &  &  &  &  &  &  &  &  &  \\
\hline
 &  &  &  &  &  &  &  &  &  &  &  &  &  &  &  &  &  &  &  &  \\
\hline
 &  &  &  &  &  &  &  &  &  &  &  &  &  &  &  &  &  &  &  &  \\
\hline
 &  &  &  &  &  &  &  &  &  &  &  &  &  &  &  &  &  &  &  &  \\
\hline
 &  &  &  &  &  &  &  &  &  &  &  &  &  &  &  &  &  &  &  &  \\
\hline
 &  &  &  &  &  &  &  &  &  &  &  &  &  &  &  &  &  &  &  &  \\
\hline
 &  &  &  &  &  &  &  &  &  &  &  &  &  &  &  &  &  &  &  &  \\
\hline
 &  &  &  &  &  &  &  &  &  &  &  &  &  &  &  &  &  &  &  &  \\
\hline
 &  &  &  &  &  &  &  &  &  &  &  &  &  &  &  &  &  &  &  &  \\
\hline
 &  &  &  &  &  &  &  &  &  &  &  &  &  &  &  &  &  &  &  &  \\
\hline
 &  &  &  &  &  &  &  &  &  &  &  &  &  &  &  &  &  &  &  &  \\
\hline
 &  &  &  &  &  &  &  &  &  &  &  &  &  &  &  &  &  &  &  &  \\
\hline
 &  &  &  &  &  &  &  &  &  &  &  &  &  &  &  &  &  &  &  &  \\
\hline
 &  &  &  &  &  &  &  &  &  &  &  &  &  &  &  &  &  &  &  &  \\
\hline
 &  &  &  &  &  &  &  &  &  &  &  &  &  &  &  &  &  &  &  &  \\
\hline
 &  &  &  &  &  &  &  &  &  &  &  &  &  &  &  &  &  &  &  &  \\
\hline
 &  &  &  &  &  &  &  &  &  &  &  &  &  &  &  &  &  &  &  &  \\
\hline
 &  &  &  &  &  &  &  &  &  &  &  &  &  &  &  &  &  &  &  &  \\
\hline
 &  &  &  &  &  &  &  &  &  &  &  &  &  &  &  &  &  &  &  &  \\
\hline
 &  &  &  &  &  &  &  &  &  &  &  &  &  &  &  &  &  &  &  &  \\
\hline
 &  &  &  &  &  &  &  &  &  &  &  &  &  &  &  &  &  &  &  &  \\
\hline
 &  &  &  &  &  &  &  &  &  &  &  &  &  &  &  &  &  &  &  &  \\
\hline
 &  &  &  &  &  &  &  &  &  &  &  &  &  &  &  &  &  &  &  &  \\
\hline
 &  &  &  &  &  &  &  &  &  &  &  &  &  &  &  &  &  &  &  &  \\
\hline
 &  &  &  &  &  &  &  &  &  &  &  &  &  &  &  &  &  &  &  &  \\
\hline
 &  &  &  &  &  &  &  &  &  &  &  &  &  &  &  &  &  &  &  &  \\
\hline
 &  &  &  &  &  &  &  &  &  &  &  &  &  &  &  &  &  &  &  &  \\
\hline
 &  &  &  &  &  &  &  &  &  &  &  &  &  &  &  &  &  &  &  &  \\
\hline
 &  &  &  &  &  &  &  &  &  &  &  &  &  &  &  &  &  &  &  &  \\
\hline
 &  &  &  &  &  &  &  &  &  &  &  &  &  &  &  &  &  &  &  &  \\
\hline
 &  &  &  &  &  &  &  &  &  &  &  &  &  &  &  &  &  &  &  &  \\
\hline
 &  &  &  &  &  &  &  &  &  &  &  &  &  &  &  &  &  &  &  &  \\
\hline
 &  &  &  &  &  &  &  &  &  &  &  &  &  &  &  &  &  &  &  &  \\
\hline
 &  &  &  &  &  &  &  &  &  &  &  &  &  &  &  &  &  &  &  &  \\
\hline
 &  &  &  &  &  &  &  &  &  &  &  &  &  &  &  &  &  &  &  &  \\
\hline
 &  &  &  &  &  &  &  &  &  &  &  &  &  &  &  &  &  &  &  &  \\
\hline
\end{tabular}
\end{center}

\begin{center}
\includegraphics[max width=\textwidth]{2025_02_10_8c71b85d173a743a8718g-09}
\end{center}

Odpowiedź:

\section*{Zadanie 11. (0-3)}
Dany jest sześcian \(A B C D E F G H\). Przez wierzchołki \(A\) i \(C\) oraz środek \(K\) krawędzi \(B F\) poprowadzono płaszczyznę, która przecina przekątną \(B H\) w punkcie \(P\) (zobacz rysunek).\\
\includegraphics[max width=\textwidth, center]{2025_02_10_8c71b85d173a743a8718g-10}

Wykaż, że \(|B P|:|H P|=1: 3\).\\
\includegraphics[max width=\textwidth, center]{2025_02_10_8c71b85d173a743a8718g-10(1)}

\section*{Zadanie 12. (0-4)}
Liczba \(m\) jest sumą odwrotności dwóch różnych pierwiastków równania

\[
k^{2} x^{2}+(k-1) x+1=0, \text { gdzie } k \neq 0 .
\]

Wyznacz zbiór wartości funkcji określonej wzorem \(f(x)=2^{m}\).\\
\includegraphics[max width=\textwidth, center]{2025_02_10_8c71b85d173a743a8718g-11}

Odpowiedź: \(\qquad\)

\section*{Zadanie 13. (0-3)}
Rozwiąż nierówność \((2 \sin x-3)(2 \sin x+1)>0\) w przedziale \(x \in(0,2 \pi)\).

\begin{center}
\begin{tabular}{|c|c|c|c|c|c|c|c|c|c|c|c|c|c|c|c|c|c|c|c|c|c|}
\hline
 &  &  &  &  &  &  &  &  &  &  &  &  &  &  &  &  &  &  &  &  &  \\
\hline
 &  &  &  &  &  &  &  &  &  &  &  &  &  &  &  &  &  &  &  &  &  \\
\hline
 &  &  &  &  &  &  &  &  &  &  &  &  &  &  &  &  &  &  &  &  &  \\
\hline
 &  &  &  &  &  &  &  &  &  &  &  &  &  &  &  &  &  &  &  &  &  \\
\hline
 &  &  &  &  &  &  &  &  &  &  &  &  &  &  &  &  &  &  &  &  &  \\
\hline
 &  &  &  &  &  &  &  &  &  &  &  &  &  &  &  &  &  &  &  &  &  \\
\hline
 &  &  &  &  &  &  &  &  &  &  &  &  &  &  &  &  &  &  &  &  &  \\
\hline
 &  &  &  &  &  &  &  &  &  &  &  &  &  &  &  &  &  &  &  &  &  \\
\hline
 &  &  &  &  &  &  &  &  &  &  &  &  &  &  &  &  &  &  &  &  &  \\
\hline
 &  &  &  &  &  &  &  &  &  &  &  &  &  &  &  &  &  &  &  &  &  \\
\hline
 &  &  &  &  &  &  &  &  &  &  &  &  &  &  &  &  &  &  &  &  &  \\
\hline
 &  &  &  &  &  &  &  &  &  &  &  &  &  &  &  &  &  &  &  &  &  \\
\hline
 &  &  &  &  &  &  &  &  &  &  &  &  &  &  &  &  &  &  &  &  &  \\
\hline
 &  &  &  &  &  &  &  &  &  &  &  &  &  &  &  &  &  &  &  &  &  \\
\hline
 &  &  &  &  &  &  &  &  &  &  &  &  &  &  &  &  &  &  &  &  &  \\
\hline
 &  &  &  &  &  &  &  &  &  &  &  &  &  &  &  &  &  &  &  &  &  \\
\hline
 &  &  &  &  &  &  &  &  &  &  &  &  &  &  &  &  &  &  &  &  &  \\
\hline
 &  &  &  &  &  &  &  &  &  &  &  &  &  &  &  &  &  &  &  &  &  \\
\hline
 &  &  &  &  &  &  &  &  &  &  &  &  &  &  &  &  &  &  &  &  &  \\
\hline
 &  &  &  &  &  &  &  &  &  &  &  &  &  &  &  &  &  &  &  &  &  \\
\hline
 &  &  &  &  &  &  &  &  &  &  &  &  &  &  &  &  &  &  &  &  &  \\
\hline
 &  &  &  &  &  &  &  &  &  &  &  &  &  &  &  &  &  &  &  &  &  \\
\hline
 &  &  &  &  &  &  &  &  &  &  &  &  &  &  &  &  &  &  &  &  &  \\
\hline
 &  &  &  &  &  &  &  &  &  &  &  &  &  &  &  &  &  &  &  &  &  \\
\hline
 &  &  &  &  &  &  &  &  &  &  &  &  &  &  &  &  &  &  &  &  &  \\
\hline
 &  &  &  &  &  &  &  &  &  &  &  &  &  &  &  &  &  &  &  &  &  \\
\hline
 &  &  &  &  &  &  &  &  &  &  &  &  &  &  &  &  &  &  &  &  &  \\
\hline
 &  &  &  &  &  &  &  &  &  &  &  &  &  &  &  &  &  &  &  &  &  \\
\hline
 &  &  &  &  &  &  &  &  &  &  &  &  &  &  &  &  &  &  &  &  &  \\
\hline
 &  &  &  &  &  &  &  &  &  &  &  &  &  &  &  &  &  &  &  &  &  \\
\hline
 &  &  &  &  &  &  &  &  &  &  &  &  &  &  &  &  &  &  &  &  &  \\
\hline
 &  &  &  &  &  &  &  &  &  &  &  &  &  &  &  &  &  &  &  &  &  \\
\hline
 &  &  &  &  &  &  &  &  &  &  &  &  &  &  &  &  &  &  &  &  &  \\
\hline
 &  &  &  &  &  &  &  &  &  &  &  &  &  &  &  &  &  &  &  &  &  \\
\hline
 &  &  &  &  &  &  &  &  &  &  &  &  &  &  &  &  &  &  &  &  &  \\
\hline
 &  &  &  &  &  &  &  &  &  &  &  &  &  &  &  &  &  &  &  &  &  \\
\hline
 &  &  &  &  &  &  &  &  &  &  &  &  &  &  &  &  &  &  &  &  &  \\
\hline
 &  &  &  &  &  &  &  &  &  &  &  &  &  &  &  &  &  &  &  &  &  \\
\hline
 &  &  &  &  &  &  &  &  &  &  &  &  &  &  &  &  &  &  &  &  &  \\
\hline
 &  &  &  &  &  &  &  &  &  &  &  &  &  &  &  &  &  &  &  &  &  \\
\hline
 &  &  &  &  &  &  &  &  &  &  &  &  &  &  &  &  &  &  &  &  &  \\
\hline
 &  &  &  &  &  &  &  &  &  &  &  &  &  &  &  &  &  &  &  &  &  \\
\hline
 &  &  &  &  &  &  &  &  &  &  &  &  &  &  &  &  &  &  &  &  &  \\
\hline
 &  &  &  &  &  &  &  &  &  &  &  &  &  &  &  &  &  &  &  &  &  \\
\hline
\end{tabular}
\end{center}

\begin{center}
\includegraphics[max width=\textwidth]{2025_02_10_8c71b85d173a743a8718g-13}
\end{center}

Odpowiedź:

\section*{Zadanie 14. (0-4)}
W trójkącie prostokątnym stosunek różnicy długości przyprostokątnych do długości przeciwprostokątnej jest równy \(\frac{1}{2}\). Oblicz cosinusy kątów ostrych tego trójkąta.\\
\includegraphics[max width=\textwidth, center]{2025_02_10_8c71b85d173a743a8718g-14}\\
\includegraphics[max width=\textwidth, center]{2025_02_10_8c71b85d173a743a8718g-15}

Odpowiedź:

\section*{Zadanie 15. (0-4)}
Oblicz, ile jest wszystkich liczb naturalnych pięciocyfrowych, w których zapisie występują dokładnie trzy cyfry nieparzyste.\\
\includegraphics[max width=\textwidth, center]{2025_02_10_8c71b85d173a743a8718g-16}\\
\includegraphics[max width=\textwidth, center]{2025_02_10_8c71b85d173a743a8718g-17}

Odpowiedź:

\section*{Zadanie 16. (0-5)}
Punkty \(A=(-7,-2)\) i \(B=(4,-7)\) są wierzchołkami podstawy trójkąta równoramiennego \(A B C\), a wysokość opuszczona z wierzchołka \(A\) tego trójkąta zawiera się w prostej o równaniu \(2 x+19 y+52=0\). Oblicz współrzędne wierzchołka \(C\).\\
\includegraphics[max width=\textwidth, center]{2025_02_10_8c71b85d173a743a8718g-18}\\
\includegraphics[max width=\textwidth, center]{2025_02_10_8c71b85d173a743a8718g-19}

Odpowiedź:

\section*{Zadanie 17. (0-7)}
Rozpatrujemy wszystkie walce, których pole powierzchni całkowitej jest równe \(2 \pi\). Oblicz promień podstawy tego walca, który ma największą objętość. Podaj tę największą objętość.\\
\includegraphics[max width=\textwidth, center]{2025_02_10_8c71b85d173a743a8718g-20}\\
\includegraphics[max width=\textwidth, center]{2025_02_10_8c71b85d173a743a8718g-21}

Odpowiedź:

\section*{BRUDNOPIS (nie podlega ocenie)}

\end{document}