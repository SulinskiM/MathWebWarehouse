\documentclass[10pt]{article}
\usepackage[polish]{babel}
\usepackage[utf8]{inputenc}
\usepackage[T1]{fontenc}
\usepackage{amsmath}
\usepackage{amsfonts}
\usepackage{amssymb}
\usepackage[version=4]{mhchem}
\usepackage{stmaryrd}
\usepackage{graphicx}
\usepackage[export]{adjustbox}
\graphicspath{ {./images/} }

\title{EGZAMIN MATURALNY OD ROKU SZKOLNEGO 2014/2015 }

\author{}
\date{}


\newcommand\Varangle{\mathop{{<\!\!\!\!\!\text{\small)}}\:}\nolimits}

\begin{document}
\maketitle
\section*{MATEMATYKA \\
 POZIOM ROZSZERZONY}
\section*{PRZYKŁADOWY ZESTAW ZADAŃ (A1)}
W czasie trwania egzaminu zdający może korzystać z zestawu wzorów matematycznych, linijki i cyrkla oraz kalkulatora.

Czas pracy: 180 minut

\section*{ZADANIA ZAMKNIETE}
\section*{W zadaniach 1-5 wybierz i zaznacz poprawnq odpowiedź}
\section*{Zadanie 1. (0-1)}
Dane są dwie urny z kulami, w każdej jest 5 kul. W pierwszej urnie jest jedna kula biała i 4 kule czarne. W drugiej urnie są 3 kule białe i 2 kule czarne. Rzucamy jeden raz symetryczną sześcienną kostką do gry. Jeśli wypadnie jedno lub dwa oczka, to losujemy jedną kulę z pierwszej urny, natomiast jeśli wypadną co najmniej trzy oczka, to losujemy jedną kulę z drugiej urny. Prawdopodobieństwo wylosowania kuli białej jest równe\\
A. \(\frac{1}{15}\)\\
B. \(\frac{2}{5}\)\\
C. \(\frac{7}{15}\)\\
D. \(\frac{3}{5}\)

\section*{Zadanie 2. (0-1)}
Dany jest nieskończony ciag geometryczny \(\left(a_{n}\right)\) określony wzorem

\[
a_{n}=\frac{3}{(\sqrt{2})^{n}} \quad \text { dla } n=1,2,3, \ldots
\]

Suma wszystkich wyrazów tego ciagu jest równa\\
A. \(\frac{1}{\sqrt{2}-1}\)\\
B. \(\frac{\sqrt{2}}{\sqrt{2}-1}\)\\
C. \(\frac{2}{\sqrt{2}-1}\)\\
D. \(\frac{3}{\sqrt{2}-1}\)

\section*{Zadanie 3. (0-1)}
Liczba \(\frac{27^{665} \cdot \sqrt[3]{3^{-92}}}{(1)^{\frac{152}{2}}}\) jest równa \(\left(\frac{1}{3}\right)^{\frac{152}{3}}\)\\
A. \(3^{725}\)\\
B. \(3^{1995}\)\\
C. \(3^{2015}\)\\
D. \(3^{2045}\)

Zadanie 4. (0-1)\\
Okrąg \(o_{1}\) ma równanie \(x^{2}+(y-1)^{2}=25\), a okrąg \(o_{2}\) ma równanie \((x-1)^{2}+y^{2}=9\). Określ wzajemne położenie tych okręgów.\\
A. Te okręgi przecinają się w dwóch punktach.\\
B. Te okręgi są styczne.\\
C. Te okręgi nie mają punktów wspólnych oraz okrag \(o_{1}\) leży w całości wewnątrz okręgu \(o_{2}\).\\
D. Te okręgi nie mają punktów wspólnych oraz okragg \(o_{2}\) leży w całości wewnątrz okręgu \(o_{1}\).

\section*{Zadanie 5. (0-1)}
Dla każdego \(\alpha\) suma \(\sin \alpha+\sin 3 \alpha\) jest równa\\
A. \(\sin 4 \alpha\).\\
B. \(2 \sin 4 \alpha\).\\
C. \(2 \sin 2 \alpha \cos \alpha\).\\
D. \(2 \sin \alpha \cos 2 \alpha\).

\section*{BRUDNOPIS}
\begin{center}
\includegraphics[max width=\textwidth]{2025_02_09_5280929d4519786c875eg-03}
\end{center}

\section*{ZADANIA OTWARTE}
W zadaniach 6-9 zakoduj wynik w kratkach zamieszczonych obok polecenia. W zadaniach 10-18 rozwiqzania nalė̇y zapisać w wyznaczonych miejscach pod treścia zadania.

\section*{Zadanie 6. (0-2)}
Liczba \(n\) jest najmniejszą liczbą całkowitą spełniającą równanie

\[
2 \cdot|x+57|=|x-39| .
\]

Zakoduj cyfry: setek, dziesiątek i jedności liczby \(|n|\).\\
\includegraphics[max width=\textwidth, center]{2025_02_09_5280929d4519786c875eg-04(1)}

\section*{Zadanie 7. (0-2)}
Oblicz granicę ciagu \(\lim _{n \rightarrow \infty} \frac{3 n^{2}-5 n+2}{(8 n+7)(n+4)}\).

Zakoduj trzy pierwsze cyfry po przecinku rozwinięcia dziesiętnego obliczonej granicy.\\
\includegraphics[max width=\textwidth, center]{2025_02_09_5280929d4519786c875eg-04}

Zadanie 8. (0-2)\\
Dana jest funkcja \(f\) określona wzorem

\[
f(x)=\frac{x-8}{x^{2}+6}
\]

dla każdej liczby rzeczywistej \(x\). Oblicz wartość pochodnej tej funkcji w punkcie \(x=\frac{1}{2}\). Zakoduj trzy pierwsze cyfry po przecinku rozwinięcia dziesiętnego otrzymanego wyniku.\\
\includegraphics[max width=\textwidth, center]{2025_02_09_5280929d4519786c875eg-05(1)}

Zadanie 9. (0-2)\\
Oblicz \(\log _{3} \sqrt[4]{27}-\log _{3}\left(\log _{3} \sqrt[3]{\sqrt[3]{3}}\right)\).\\
Zakoduj cyfrę jedności i dwie pierwsze cyfry po przecinku rozwinięcia dziesiętnego otrzymanego wyniku.\\
\(\square\) ——\\
\includegraphics[max width=\textwidth, center]{2025_02_09_5280929d4519786c875eg-05}

Zadanie 10. (0-3)\\
Punkty \(P_{1}, P_{2}, P_{3}, \ldots, P_{23}, P_{24}\) dzielą okrag na 24 równe łuki (zobacz rysunek). Punkt \(A\) jest punktem przecięcia cięciw \(P_{11} P_{22}\) i \(P_{1} P_{16}\).\\
\includegraphics[max width=\textwidth, center]{2025_02_09_5280929d4519786c875eg-06}

Udowodnij, że \(\left|\Varangle P_{16} A P_{11}\right|=60^{\circ}\).\\
\includegraphics[max width=\textwidth, center]{2025_02_09_5280929d4519786c875eg-06(1)}\\
\includegraphics[max width=\textwidth, center]{2025_02_09_5280929d4519786c875eg-07}

\section*{Zadanie 11. (0-3)}
Udowodnij, że dla każdej liczby rzeczywistej \(x\) i każdej liczby rzeczywistej \(m\) prawdziwa jest nierówność

\[
20 x^{2}-24 m x+18 m^{2} \geq 4 x+12 m-5 .
\]

\begin{center}
\begin{tabular}{|c|c|c|c|c|c|c|c|c|c|c|c|c|c|c|c|c|c|c|c|c|c|c|}
\hline
 &  &  &  &  &  &  &  &  &  &  &  &  &  &  &  &  &  &  &  &  &  &  \\
\hline
 &  &  &  &  &  &  &  &  &  &  &  &  &  &  &  &  &  &  &  &  &  &  \\
\hline
 &  &  &  &  &  &  &  &  &  &  &  &  &  &  &  &  &  &  &  &  &  &  \\
\hline
 &  &  &  &  &  &  &  &  &  &  &  &  &  &  &  &  &  &  &  &  &  &  \\
\hline
 &  &  &  &  &  &  &  &  &  &  &  &  &  &  &  &  &  &  &  &  &  &  \\
\hline
 &  &  &  &  &  &  &  &  &  &  &  &  &  &  &  &  &  &  &  &  &  &  \\
\hline
 &  &  &  &  &  &  &  &  &  &  &  &  &  &  &  &  &  &  &  &  &  &  \\
\hline
 &  &  &  &  &  &  &  &  &  &  &  &  &  &  &  &  &  &  &  &  &  &  \\
\hline
 &  &  &  &  &  &  &  &  &  &  &  &  &  &  &  &  &  &  &  &  &  &  \\
\hline
 &  &  &  &  &  &  &  &  &  &  &  &  &  &  &  &  &  &  &  &  &  &  \\
\hline
 &  &  &  &  &  &  &  &  &  &  &  &  &  &  &  &  &  &  &  &  &  &  \\
\hline
 &  &  &  &  &  &  &  &  &  &  &  &  &  &  &  &  &  &  &  &  &  &  \\
\hline
 &  &  &  &  &  &  &  &  &  &  &  &  &  &  &  &  &  &  &  &  &  &  \\
\hline
 &  &  &  &  &  &  &  &  &  &  &  &  &  &  &  &  &  &  &  &  &  &  \\
\hline
 &  &  &  &  &  &  &  &  &  &  &  &  &  &  &  &  &  &  &  &  &  &  \\
\hline
 &  &  &  &  &  &  &  &  &  &  &  &  &  &  &  &  &  &  &  &  &  &  \\
\hline
 &  &  &  &  &  &  &  &  &  &  &  &  &  &  &  &  &  &  &  &  &  &  \\
\hline
 &  &  &  &  &  &  &  &  &  &  &  &  &  &  &  &  &  &  &  &  &  &  \\
\hline
 &  &  &  &  &  &  &  &  &  &  &  &  &  &  &  &  &  &  &  &  &  &  \\
\hline
 &  &  &  &  &  &  &  &  &  &  &  &  &  &  &  &  &  &  &  &  &  &  \\
\hline
 &  &  &  &  &  &  &  &  &  &  &  &  &  &  &  &  &  &  &  &  &  &  \\
\hline
 &  &  &  &  &  &  &  &  &  &  &  &  &  &  &  &  &  &  &  &  &  &  \\
\hline
 &  &  &  &  &  &  &  &  &  &  &  &  &  &  &  &  &  &  &  &  &  &  \\
\hline
 &  &  &  &  &  &  &  &  &  &  &  &  &  &  &  &  &  &  &  &  &  &  \\
\hline
 &  &  &  &  &  &  &  &  &  &  &  &  &  &  &  &  &  &  &  &  &  &  \\
\hline
 &  &  &  &  &  &  &  &  &  &  &  &  &  &  &  &  &  &  &  &  &  &  \\
\hline
 &  &  &  &  &  &  &  &  &  &  &  &  &  &  &  &  &  &  &  &  &  &  \\
\hline
 &  &  &  &  &  &  &  &  &  &  &  &  &  &  &  &  &  &  &  &  &  &  \\
\hline
 &  &  &  &  &  &  &  &  &  &  &  &  &  &  &  &  &  &  &  &  &  &  \\
\hline
 &  &  &  &  &  &  &  &  &  &  &  &  &  &  &  &  &  &  &  &  &  &  \\
\hline
 &  &  &  &  &  &  &  &  &  &  &  &  &  &  &  &  &  &  &  &  &  &  \\
\hline
 &  &  &  &  &  &  &  &  &  &  &  &  &  &  &  &  &  &  &  &  &  &  \\
\hline
 &  &  &  &  &  &  &  &  &  &  &  &  &  &  &  &  &  &  &  &  &  &  \\
\hline
 &  &  &  &  &  &  &  &  &  &  &  &  &  &  &  &  &  &  &  &  &  &  \\
\hline
 &  &  &  &  &  &  &  &  &  &  &  &  &  &  &  &  &  &  &  &  &  &  \\
\hline
 &  &  &  &  &  &  &  &  &  &  &  &  &  &  &  &  &  &  &  &  &  &  \\
\hline
 &  &  &  &  &  &  &  &  &  &  &  &  &  &  &  &  &  &  &  &  &  &  \\
\hline
 &  &  &  &  &  &  &  &  &  &  &  &  &  &  &  &  &  &  &  &  &  &  \\
\hline
 &  &  &  &  &  &  &  &  &  &  &  &  &  &  &  &  &  &  &  &  &  &  \\
\hline
 &  &  &  &  &  &  &  &  &  &  &  &  &  &  &  &  &  &  &  &  &  &  \\
\hline
 &  &  &  &  &  &  &  &  &  &  &  &  &  &  &  &  &  &  &  &  &  &  \\
\hline
 &  &  &  &  &  &  &  &  &  &  &  &  &  &  &  &  &  &  &  &  &  &  \\
\hline
\end{tabular}
\end{center}

Zadanie 12. (0-3)\\
Janek przeprowadza doświadczenie losowe, w którym jako wynik może otrzymać jedną z liczb: \(0,1,2,3,4,5,6\). Prawdopodobieństwo \(p_{k}\) otrzymania liczby \(k\) jest dane wzorem: \(p_{k}=\frac{1}{64} \cdot\binom{6}{k}\).\\
Rozważamy dwa zdarzenia:

\begin{itemize}
  \item zdarzenie \(A\) polegajace na otrzymaniu liczby ze zbioru \(\{1,3,5\}\),
  \item zdarzenie \(B\) polegające na otrzymaniu liczby ze zbioru \(\{2,3,4,5,6\}\).
\end{itemize}

Oblicz prawdopodobieństwo warunkowe \(P(A \mid B)\).

\begin{center}
\begin{tabular}{|c|c|c|c|c|c|c|c|c|c|c|c|c|c|c|c|c|c|c|c|c|c|c|c|c|c|c|c|c|c|}
\hline
 &  &  &  &  &  &  &  &  &  &  &  &  &  &  &  &  &  &  &  &  &  &  &  &  &  &  &  &  &  \\
\hline
 &  &  &  &  &  &  &  &  &  &  &  &  &  &  &  &  &  &  &  &  &  &  &  &  &  &  &  &  &  \\
\hline
 &  &  &  &  &  &  &  &  &  &  &  &  &  &  &  &  &  &  &  &  &  &  &  &  &  &  &  &  &  \\
\hline
 &  &  &  &  &  &  &  &  &  &  &  &  &  &  &  &  &  &  &  &  &  &  &  &  &  &  &  &  &  \\
\hline
 &  &  &  &  &  &  &  &  &  &  &  &  &  &  &  &  &  &  &  &  &  &  &  &  &  &  &  &  &  \\
\hline
 &  &  &  &  &  &  &  &  &  &  &  &  &  &  &  &  &  &  &  &  &  &  &  &  &  &  &  &  &  \\
\hline
 &  &  &  &  &  &  &  &  &  &  &  &  &  &  &  &  &  &  &  &  &  &  &  &  &  &  &  &  &  \\
\hline
 &  &  &  &  &  &  &  &  &  &  &  &  &  &  &  &  &  &  &  &  &  &  &  &  &  &  &  &  &  \\
\hline
 &  &  &  &  &  &  &  &  &  &  &  &  &  &  &  &  &  &  &  &  &  &  &  &  &  &  &  &  &  \\
\hline
 &  &  &  &  &  &  &  &  &  &  &  &  &  &  &  &  &  &  &  &  &  &  &  &  &  &  &  &  &  \\
\hline
 &  &  &  &  &  &  &  &  &  &  &  &  &  &  &  &  &  &  &  &  &  &  &  &  &  &  &  &  &  \\
\hline
 &  &  &  &  &  &  &  &  &  &  &  &  &  &  &  &  &  &  &  &  &  &  &  &  &  &  &  &  &  \\
\hline
 &  &  &  &  &  &  &  &  &  &  &  &  &  &  &  &  &  &  &  &  &  &  &  &  &  &  &  &  &  \\
\hline
 &  &  &  &  &  &  &  &  &  &  &  &  &  &  &  &  &  &  &  &  &  &  &  &  &  &  &  &  &  \\
\hline
 &  &  &  &  &  &  &  &  &  &  &  &  &  &  &  &  &  &  &  &  &  &  &  &  &  &  &  &  &  \\
\hline
 &  &  &  &  &  &  &  &  &  &  &  &  &  &  &  &  &  &  &  &  &  &  &  &  &  &  &  &  &  \\
\hline
 &  &  &  &  &  &  &  &  &  &  &  &  &  &  &  &  &  &  &  &  &  &  &  &  &  &  &  &  &  \\
\hline
 &  &  &  &  &  &  &  &  &  &  &  &  &  &  &  &  &  &  &  &  &  &  &  &  &  &  &  &  &  \\
\hline
 &  &  &  &  &  &  &  &  &  &  &  &  &  &  &  &  &  &  &  &  &  &  &  &  &  &  &  &  &  \\
\hline
 &  &  &  &  &  &  &  &  &  &  &  &  &  &  &  &  &  &  &  &  &  &  &  &  &  &  &  &  &  \\
\hline
 &  &  &  &  &  &  &  &  &  &  &  &  &  &  &  &  &  &  &  &  &  &  &  &  &  &  &  &  &  \\
\hline
 &  &  &  &  &  &  &  &  &  &  &  &  &  &  &  &  &  &  &  &  &  &  &  &  &  &  &  &  &  \\
\hline
 &  &  &  &  &  &  &  &  &  &  &  &  &  &  &  &  &  &  &  &  &  &  &  &  &  &  &  &  &  \\
\hline
 &  &  &  &  &  &  &  &  &  &  &  &  &  &  &  &  &  &  &  &  &  &  &  &  &  &  &  &  &  \\
\hline
 &  &  &  &  &  &  &  &  &  &  &  &  &  &  &  &  &  &  &  &  &  &  &  &  &  &  &  &  &  \\
\hline
 &  &  &  &  &  &  &  &  &  &  &  &  &  &  &  &  &  &  &  &  &  &  &  &  &  &  &  &  &  \\
\hline
 &  &  &  &  &  &  &  &  &  &  &  &  &  &  &  &  &  &  &  &  &  &  &  &  &  &  &  &  &  \\
\hline
 &  &  &  &  &  &  &  &  &  &  &  &  &  &  &  &  &  &  &  &  &  &  &  &  &  &  &  &  &  \\
\hline
 &  &  &  &  &  &  &  &  &  &  &  &  &  &  &  &  &  &  &  &  &  &  &  &  &  &  &  &  &  \\
\hline
 &  &  &  &  &  &  &  &  &  &  &  &  &  &  &  &  &  &  &  &  &  &  &  &  &  &  &  &  &  \\
\hline
 &  &  &  &  &  &  &  &  &  &  &  &  &  &  &  &  &  &  &  &  &  &  &  &  &  &  &  &  &  \\
\hline
 &  &  &  &  &  &  &  &  &  &  &  &  &  &  &  &  &  &  &  &  &  &  &  &  &  &  &  &  &  \\
\hline
\end{tabular}
\end{center}

Odpowiedź: \(\qquad\)

\section*{Zadanie 13. (0-3)}
Wyznacz wszystkie wartości parametru \(m\), dla których prosta o równaniu \(y=m x+(2 m+3)\) ma dokładnie dwa punkty wspólne z okręgiem o środku w punkcie \(S=(0,0)\) i promieniu \(r=3\).

\begin{center}
\begin{tabular}{|c|c|c|c|c|c|c|c|c|c|c|c|c|c|c|c|c|c|c|c|c|c|c|c|}
\hline
 &  &  &  &  &  &  &  &  &  &  &  &  &  &  &  &  &  &  &  &  &  &  &  \\
\hline
 &  &  &  &  &  &  &  &  &  &  &  &  &  &  &  &  &  &  &  &  &  &  &  \\
\hline
 &  &  &  &  &  &  &  &  &  &  &  &  &  &  &  &  &  &  &  &  &  &  &  \\
\hline
 &  &  &  &  &  &  &  &  &  &  &  &  &  &  &  &  &  &  &  &  &  &  &  \\
\hline
 &  &  &  &  &  &  &  &  &  &  &  &  &  &  &  &  &  &  &  &  &  &  &  \\
\hline
 &  &  &  &  &  &  &  &  &  &  &  &  &  &  &  &  &  &  &  &  &  &  &  \\
\hline
 &  &  &  &  &  &  &  &  &  &  &  &  &  &  &  &  &  &  &  &  &  &  &  \\
\hline
 &  &  &  &  &  &  &  &  &  &  &  &  &  &  &  &  &  &  &  &  &  &  &  \\
\hline
 &  &  &  &  &  &  &  &  &  &  &  &  &  &  &  &  &  &  &  &  &  &  &  \\
\hline
 &  &  &  &  &  &  &  &  &  &  &  &  &  &  &  &  &  &  &  &  &  &  &  \\
\hline
 &  &  &  &  &  &  &  &  &  &  &  &  &  &  &  &  &  &  &  &  &  &  &  \\
\hline
 &  &  &  &  &  &  &  &  &  &  &  &  &  &  &  &  &  &  &  &  &  &  &  \\
\hline
 &  &  &  &  &  &  &  &  &  &  &  &  &  &  &  &  &  &  &  &  &  &  &  \\
\hline
 &  &  &  &  &  &  &  &  &  &  &  &  &  &  &  &  &  &  &  &  &  &  &  \\
\hline
 &  &  &  &  &  &  &  &  &  &  &  &  &  &  &  &  &  &  &  &  &  &  &  \\
\hline
 &  &  &  &  &  &  &  &  &  &  &  &  &  &  &  &  &  &  &  &  &  &  &  \\
\hline
 &  &  &  &  &  &  &  &  &  &  &  &  &  &  &  &  &  &  &  &  &  &  &  \\
\hline
 &  &  &  &  &  &  &  &  &  &  &  &  &  &  &  &  &  &  &  &  &  &  &  \\
\hline
 &  &  &  &  &  &  &  &  &  &  &  &  &  &  &  &  &  &  &  &  &  &  &  \\
\hline
 &  &  &  &  &  &  &  &  &  &  &  &  &  &  &  &  &  &  &  &  &  &  &  \\
\hline
 &  &  &  &  &  &  &  &  &  &  &  &  &  &  &  &  &  &  &  &  &  &  &  \\
\hline
 &  &  &  &  &  &  &  &  &  &  &  &  &  &  &  &  &  &  &  &  &  &  &  \\
\hline
 &  &  &  &  &  &  &  &  &  &  &  &  &  &  &  &  &  &  &  &  &  &  &  \\
\hline
 &  &  &  &  &  &  &  &  &  &  &  &  &  &  &  &  &  &  &  &  &  &  &  \\
\hline
 &  &  &  &  &  &  &  &  &  &  &  &  &  &  &  &  &  &  &  &  &  &  &  \\
\hline
 &  &  &  &  &  &  &  &  &  &  &  &  &  &  &  &  &  &  &  &  &  &  &  \\
\hline
 &  &  &  &  &  &  &  &  &  &  &  &  &  &  &  &  &  &  &  &  &  &  &  \\
\hline
 &  &  &  &  &  &  &  &  &  &  &  &  &  &  &  &  &  &  &  &  &  &  &  \\
\hline
 &  &  &  &  &  &  &  &  &  &  &  &  &  &  &  &  &  &  &  &  &  &  &  \\
\hline
 &  &  &  &  &  &  &  &  &  &  &  &  &  &  &  &  &  &  &  &  &  &  &  \\
\hline
 &  &  &  &  &  &  &  &  &  &  &  &  &  &  &  &  &  &  &  &  &  &  &  \\
\hline
 &  &  &  &  &  &  &  &  &  &  &  &  &  &  &  &  &  &  &  &  &  &  &  \\
\hline
 &  &  &  &  &  &  &  &  &  &  &  &  &  &  &  &  &  &  &  &  &  &  &  \\
\hline
 &  &  &  &  &  &  &  &  &  &  &  &  &  &  &  &  &  &  &  &  &  &  &  \\
\hline
 &  &  &  &  &  &  &  &  &  &  &  &  &  &  &  &  &  &  &  &  &  &  &  \\
\hline
 &  &  &  &  &  &  &  &  &  &  &  &  &  &  &  &  &  &  &  &  &  &  &  \\
\hline
 &  &  &  &  &  &  &  &  &  &  &  &  &  &  &  &  &  &  &  &  &  &  &  \\
\hline
 &  &  &  &  &  &  &  &  &  &  &  &  &  &  &  &  &  &  &  &  &  &  &  \\
\hline
 &  &  &  &  &  &  &  &  &  &  &  &  &  &  &  &  &  &  &  &  &  &  &  \\
\hline
 &  &  &  &  &  &  &  &  &  &  &  &  &  &  &  &  &  &  &  &  &  &  &  \\
\hline
\end{tabular}
\end{center}

Odpowiedź:

\section*{Zadanie 14. (0-3)}
Dana jest parabola o równaniu \(y=x^{2}+1\) i leżący na niej punkt \(A\) o współrzędnej \(x\) równej 3 . Wyznacz równanie stycznej do tej paraboli w punkcie \(A\).\\
\includegraphics[max width=\textwidth, center]{2025_02_09_5280929d4519786c875eg-11}

Odpowiedź:

\section*{Zadanie 15. (0-3)}
W ostrosłupie prawidłowym czworokątnym krawędź podstawy ma długość \(a\). Kąt między krawędzią boczną, a krawędzią podstawy ma miarę \(\alpha>45^{\circ}\) (zobacz rysunek). Oblicz objętość tego ostrosłupa.\\
\includegraphics[max width=\textwidth, center]{2025_02_09_5280929d4519786c875eg-12}\\
\includegraphics[max width=\textwidth, center]{2025_02_09_5280929d4519786c875eg-12(1)}\\
\includegraphics[max width=\textwidth, center]{2025_02_09_5280929d4519786c875eg-13}

Odpowiedź:

\section*{Zadanie 16. (0-6)}
Punkty \(M\) i \(L\) leżą odpowiednio na bokach \(A B\) i \(A C\) trójkąta \(A B C\), przy czym zachodzą równości \(|M B|=2 \cdot|A M|\) oraz \(|L C|=3 \cdot|A L|\). Punkt \(S\) jest punktem przecięcia odcinków \(B L\) i \(C M\). Punkt \(K\) jest punktem przecięcia półprostej \(A S\) z odcinkiem \(B C\) (zobacz rysunek).\\
\includegraphics[max width=\textwidth, center]{2025_02_09_5280929d4519786c875eg-14}

Pole trójkąta \(A B C\) jest równe 660 . Oblicz pola trójkątów: \(A M S, A L S, B M S\) i \(C L S\).\\
\includegraphics[max width=\textwidth, center]{2025_02_09_5280929d4519786c875eg-14(1)}\\
\includegraphics[max width=\textwidth, center]{2025_02_09_5280929d4519786c875eg-15}

Odpowiedź:

Zadanie 17. (0-6)\\
Oblicz, ile jest stucyfrowych liczb naturalnych o sumie cyfr równej 4.\\
\includegraphics[max width=\textwidth, center]{2025_02_09_5280929d4519786c875eg-16}

Odpowiedź:

\section*{Zadanie 18. (0-7)}
Dany jest prostokątny arkusz kartonu o długości 80 cm i szerokości 50 cm . W czterech rogach tego arkusza wycięto kwadratowe naroża (zobacz rysunek).\\
\includegraphics[max width=\textwidth, center]{2025_02_09_5280929d4519786c875eg-17}

Następnie zagięto karton wzdłuż linii przerywanych, tworząc w ten sposób prostopadłościenne pudełko (bez przykrywki). Oblicz długość boku każdego z wyciętych kwadratowych naroży, dla której objętość otrzymanego pudełka jest największa. Oblicz tę maksymalną objętość.\\
\includegraphics[max width=\textwidth, center]{2025_02_09_5280929d4519786c875eg-17(1)}\\
\includegraphics[max width=\textwidth, center]{2025_02_09_5280929d4519786c875eg-18}

Odpowiedź:\\
\includegraphics[max width=\textwidth, center]{2025_02_09_5280929d4519786c875eg-19}


\end{document}