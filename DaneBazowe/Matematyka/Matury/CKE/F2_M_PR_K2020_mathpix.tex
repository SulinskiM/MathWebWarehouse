\documentclass[10pt]{article}
\usepackage[polish]{babel}
\usepackage[utf8]{inputenc}
\usepackage[T1]{fontenc}
\usepackage{graphicx}
\usepackage[export]{adjustbox}
\graphicspath{ {./images/} }
\usepackage{amsmath}
\usepackage{amsfonts}
\usepackage{amssymb}
\usepackage[version=4]{mhchem}
\usepackage{stmaryrd}
\usepackage{multirow}

\begin{document}
\begin{center}
\includegraphics[max width=\textwidth]{2025_02_09_765256a0654ca524b15ag-01}
\end{center}

\section*{PRÓBNY EGZAMIN MATURALNY}
\section*{Z MATEMATYKI}
Poziom rozszerzony

\section*{Data: kwiecień 2020 r. \\
 Czas pracy: 180 minut \\
 Liczba punktów do uzyskania: \(\mathbf{5 0}\)}
\section*{Instrukcja dla zdającego}
\begin{enumerate}
  \item Sprawdź, czy arkusz egzaminacyjny zawiera 20 stron (zadania 1-15). Ewentualny brak zgłoś przewodniczącemu zespołu nadzorującego egzamin.
  \item Rozwiązania i odpowiedzi zapisz w miejscu na to przeznaczonym przy każdym zadaniu.
  \item W rozwiązaniach zadań rachunkowych przedstaw tok rozumowania prowadzący do ostatecznego wyniku.
  \item Pisz czytelnie. Używaj długopisu/pióra tylko z czarnym tuszem/atramentem.
  \item Nie używaj korektora, a błędne zapisy wyraźnie przekreśl.
  \item Pamiętaj, że zapisy w brudnopisie nie będą oceniane.
  \item Możesz korzystać z Wybranych wzorów matematycznych, linijki oraz kalkulatora prostego.
  \item Na tej stronie oraz na karcie odpowiedzi wpisz swój numer PESEL i przyklej naklejkę z kodem.
  \item Nie wpisuj żadnych znaków w części przeznaczonej dla egzaminatora.
\end{enumerate}

MMA-R1\_1P

W zadaniach od 1. do 4. wybierz i zaznacz na karcie odpowiedzi poprawna odpowiedź.

\section*{Zadanie 1. (0-1)}
Niech \(L=\log _{\sqrt{2}} 2 \cdot \log _{2} \sqrt{3} \cdot \log _{\sqrt{3}} 4\). Wtedy\\
A. \(L=1\)\\
B. \(L=2\)\\
C. \(L=3\)\\
D. \(L=4\)

\section*{Zadanie 2. (0-1)}
Okrąg o równaniu \((x-3)^{2}+(y+7)^{2}=625\) jest styczny do okręgu o środku \(S=(12,5)\) i promieniu \(r\). Wynika stąd, że\\
A. \(r=5\)\\
B. \(r=15\)\\
C. \(r=10\)\\
D. \(r=20\)

\section*{Zadanie 3. (0-1)}
Liczba \(\sqrt{(1-\sqrt{2})^{2}}+\sqrt{(2-\sqrt{2})^{2}}\) jest równa\\
A. 1\\
B. -1\\
C. \(3-2 \sqrt{2}\)\\
D. \(2 \sqrt{2}+1\)

\section*{Zadanie 4. (0-1)}
Spośród poniższych nierówności wskaż tę, którą spełniają dokładnie trzy liczby całkowite.\\
A. \(\left|\frac{3}{4} x+5\right|<2\)\\
B. \(\left|\frac{4}{3} x+5\right|<2\)\\
C. \(\left|\frac{3}{5} x+4\right|<2\)\\
D. \(\left|\frac{4}{5} x+3\right|<2\)\\
\includegraphics[max width=\textwidth, center]{2025_02_09_765256a0654ca524b15ag-03}

\section*{Zadanie 5. (0-2)}
Oblicz współczynnik kierunkowy stycznej do wykresu funkcji \(f(x)=\frac{x^{2}}{x-1}\), określonej dla każdej liczby rzeczywistej \(x \neq 1\), poprowadzonej w punkcie \(A=\left(6, \frac{36}{5}\right)\) tego wykresu.\\
W poniższe kratki wpisz kolejno cyfrę jedności, pierwszą i drugą cyfrę po przecinku skończonego rozwinięcia dziesiętnego otrzymanego wyniku.\\
\includegraphics[max width=\textwidth, center]{2025_02_09_765256a0654ca524b15ag-04}

\section*{BRUDNOPIS (nie podlega ocenie)}
\begin{center}
\includegraphics[max width=\textwidth]{2025_02_09_765256a0654ca524b15ag-04(1)}
\end{center}

\section*{Zadanie 6. (0-3)}
W trójkącie \(A B C\) kąt \(B A C\) jest dwa razy większy od kąta \(A B C\). Wykaż, że prawdziwa jest równość \(|B C|^{2}-|A C|^{2}=|A B| \cdot|A C|\).\\
\(\qquad\)\\
\includegraphics[max width=\textwidth, center]{2025_02_09_765256a0654ca524b15ag-05(2)}\\
\(\qquad\)\\
\includegraphics[max width=\textwidth, center]{2025_02_09_765256a0654ca524b15ag-05(4)}\\
\includegraphics[max width=\textwidth, center]{2025_02_09_765256a0654ca524b15ag-05(5)}\\
\(\qquad\)\\
\includegraphics[max width=\textwidth, center]{2025_02_09_765256a0654ca524b15ag-05}\\
\(\qquad\)\\
\(\qquad\)\\
\(\qquad\)\\
\includegraphics[max width=\textwidth, center]{2025_02_09_765256a0654ca524b15ag-05(10)}\\
\(\qquad\)\\
\(\qquad\)\\
\(\qquad\)\\
\includegraphics[max width=\textwidth, center]{2025_02_09_765256a0654ca524b15ag-05(1)}\\
\includegraphics[max width=\textwidth, center]{2025_02_09_765256a0654ca524b15ag-05(8)}\\
\(\qquad\)\\
\(\qquad\)\\
\includegraphics[max width=\textwidth, center]{2025_02_09_765256a0654ca524b15ag-05(11)}\\
\(\qquad\)\\
\includegraphics[max width=\textwidth, center]{2025_02_09_765256a0654ca524b15ag-05(6)}\\
\(\qquad\)\\
\includegraphics[max width=\textwidth, center]{2025_02_09_765256a0654ca524b15ag-05(9)}\\
\includegraphics[max width=\textwidth, center]{2025_02_09_765256a0654ca524b15ag-05(3)}\\
\includegraphics[max width=\textwidth, center]{2025_02_09_765256a0654ca524b15ag-05(7)}\\
\(\qquad\)

Zadanie 7. (0-3)\\
Udowodnij, że dla dowolnego kąta \(\alpha \in\left(0, \frac{\pi}{2}\right)\) prawdziwa jest nierówność

\[
\sin \left(\frac{\pi}{12}-\alpha\right) \cdot \cos \left(\frac{\pi}{12}+\alpha\right)<\frac{1}{4}
\]

\(\qquad\)\\
\includegraphics[max width=\textwidth, center]{2025_02_09_765256a0654ca524b15ag-06(4)}\\
\(\qquad\)\\
\includegraphics[max width=\textwidth, center]{2025_02_09_765256a0654ca524b15ag-06(5)}\\
\includegraphics[max width=\textwidth, center]{2025_02_09_765256a0654ca524b15ag-06(8)}\\
\(\qquad\)\\
\includegraphics[max width=\textwidth, center]{2025_02_09_765256a0654ca524b15ag-06(3)}\\
\includegraphics[max width=\textwidth, center]{2025_02_09_765256a0654ca524b15ag-06}\\
\(\qquad\)\\
\includegraphics[max width=\textwidth, center]{2025_02_09_765256a0654ca524b15ag-06(9)}\\
\includegraphics[max width=\textwidth, center]{2025_02_09_765256a0654ca524b15ag-06(2)}\\
\(\qquad\)\\
\(\qquad\)\\
\includegraphics[max width=\textwidth, center]{2025_02_09_765256a0654ca524b15ag-06(7)}\\
\includegraphics[max width=\textwidth, center]{2025_02_09_765256a0654ca524b15ag-06(6)}\\
\(\qquad\)\\
\(\qquad\)\\
\includegraphics[max width=\textwidth, center]{2025_02_09_765256a0654ca524b15ag-06(1)}\\
\includegraphics[max width=\textwidth, center]{2025_02_09_765256a0654ca524b15ag-06(10)}

\begin{center}
\begin{tabular}{|c|l|c|c|c|}
\hline
\multirow{3}{*}{\begin{tabular}{l}
Wypelnia \\
egzaminator \\
\end{tabular}} & Nr zadania & 5. & \(\mathbf{6 .}\) & 7. \\
\cline { 2 - 5 }
 & Maks. liczba pkt & 2 & 3 & 3 \\
\cline { 2 - 5 }
 & Uzyskana liczba pkt &  &  &  \\
\hline
\end{tabular}
\end{center}

\section*{Zadanie 8. (0-3)}
Wykaż, że równanie \(x^{8}+x^{2}=2\left(x^{4}+x-1\right)\) ma tylko jedno rozwiązanie rzeczywiste \(x=1\).\\
\includegraphics[max width=\textwidth, center]{2025_02_09_765256a0654ca524b15ag-07}

\section*{Zadanie 9. (0-4)}
Ze zbioru wszystkich liczb naturalnych ośmiocyfrowych, w których zapisie dziesiętnym występują tylko cyfry ze zbioru \(\{0,1,3,5,7,9\}\), losujemy jedną. Oblicz prawdopodobieństwo zdarzenia polegającego na tym, że suma cyfr wylosowanej liczby jest równa 3.\\
\includegraphics[max width=\textwidth, center]{2025_02_09_765256a0654ca524b15ag-08}

Odpowiedź: \(\qquad\)

\begin{center}
\begin{tabular}{|c|l|c|c|}
\hline
\multirow{2}{*}{\begin{tabular}{c}
Wypelnia \\
egzaminator \\
\end{tabular}} & Nr zadania & \(\mathbf{8 .}\) & \(\mathbf{9 .}\) \\
\cline { 2 - 4 }
 & Maks. liczba pkt & 3 & \(\mathbf{4}\) \\
\cline { 2 - 4 }
 & Uzyskana liczba pkt &  &  \\
\hline
\end{tabular}
\end{center}

\section*{Zadanie 10. (0-4)}
Dany jest rosnący ciąg geometryczny ( \(a, a q, a q^{2}\) ), którego wszystkie wyrazy i iloraz są liczbami całkowitymi nieparzystymi. Jeśli największy wyraz ciągu zmniejszymy o 4, to otrzymamy ciąg arytmetyczny. Oblicz wyraz aq tego ciągu.\\
\includegraphics[max width=\textwidth, center]{2025_02_09_765256a0654ca524b15ag-09}

Odpowiedź: . .

\section*{Zadanie 11. (0-4)}
Dany jest nieskończony ciąg okręgów \(\left(o_{n}\right)\) o równaniach \(x^{2}+y^{2}=2^{11-n}, n \geq 1\). Niech \(P_{k}\) będzie pierścieniem ograniczonym zewnętrznym okręgiem \(o_{2 k-1}\) i wewnętrznym okręgiem \(o_{2 k}\). Oblicz sumę pól wszystkich pierścieni \(P_{k}\), gdzie \(k \geq 1\).\\
\includegraphics[max width=\textwidth, center]{2025_02_09_765256a0654ca524b15ag-10}

Odpowiedź: \(\qquad\)

\begin{center}
\begin{tabular}{|c|l|c|c|}
\hline
\multirow{3}{*}{\begin{tabular}{c}
Wypetnia \\
egzaminator \\
\end{tabular}} & Nr zadania & 10. & 11. \\
\cline { 2 - 4 }
 & Maks. liczba pkt & 4 & 4 \\
\cline { 2 - 4 }
 & Uzyskana liczba pkt &  &  \\
\hline
\end{tabular}
\end{center}

\section*{Zadanie 12. (0-5)}
Trapez prostokątny \(A B C D\) o podstawach \(A B\) i \(C D\) jest opisany na okręgu. Ramię \(B C\) ma długość 10, a ramię \(A D\) jest wysokością trapezu. Podstawa \(A B\) jest 2 razy dłuższa od podstawy \(C D\). Oblicz pole tego trapezu.\\
\includegraphics[max width=\textwidth, center]{2025_02_09_765256a0654ca524b15ag-11}\\
\includegraphics[max width=\textwidth, center]{2025_02_09_765256a0654ca524b15ag-12}

Odpowiedź:

\begin{center}
\begin{tabular}{|c|l|c|}
\hline
\multirow{2}{*}{\begin{tabular}{l}
Wypelnia \\
egzaminator \\
\end{tabular}} & Nr zadania & 12. \\
\cline { 2 - 3 }
 & Maks. liczba pkt & 5 \\
\cline { 2 - 3 }
 & Uzyskana liczba pkt &  \\
\hline
\end{tabular}
\end{center}

\section*{Zadanie 13. (0-5)}
Wierzchołki \(A\) i \(B\) trójkąta prostokątnego \(A B C\) leżą na osi \(O y\) układu współrzędnych. Okrąg wpisany w ten trójkąt jest styczny do boków \(A B, B C\) i \(C A\) w punktach - odpowiednio \(P=(0,10), Q=(8,6)\) i \(R=(9,13)\). Oblicz współrzędne wierzchołków \(A, B\) i \(C\) tego trójkąta.\\
\includegraphics[max width=\textwidth, center]{2025_02_09_765256a0654ca524b15ag-13}\\
\includegraphics[max width=\textwidth, center]{2025_02_09_765256a0654ca524b15ag-14}

Odpowiedź:

\begin{center}
\begin{tabular}{|c|l|c|}
\hline
\multirow{2}{*}{\begin{tabular}{l}
Wypelnia \\
egzaminator \\
\end{tabular}} & Nr zadania & 13. \\
\cline { 2 - 3 }
 & Maks. liczba pkt & 5 \\
\cline { 2 - 3 }
 & Uzyskana liczba pkt &  \\
\hline
\end{tabular}
\end{center}

\section*{Zadanie 14. (0-6)}
Wyznacz wszystkie wartości parametru \(m\), dla których równanie

\[
x^{2}-3 m x+(m+1)(2 m-1)=0
\]

ma dwa różne rozwiązania \(x_{1}, x_{2}\) spełniające warunki: \(x_{1} \cdot x_{2} \neq 0\) oraz \(0<\frac{1}{x_{1}}+\frac{1}{x_{2}} \leq \frac{2}{3}\).

\begin{center}
\begin{tabular}{|c|c|c|c|c|c|c|c|c|c|c|c|c|c|c|c|c|c|c|c|c|c|c|}
\hline
 &  &  &  &  &  &  &  &  &  &  &  &  &  &  &  &  &  &  &  &  &  &  \\
\hline
 &  &  &  &  &  &  &  &  &  &  &  &  &  &  &  &  &  &  &  &  &  &  \\
\hline
 &  &  &  &  &  &  &  &  &  &  &  &  &  &  &  &  &  &  &  &  &  &  \\
\hline
 &  &  &  &  &  &  &  &  &  &  &  &  &  &  &  &  &  &  &  &  &  &  \\
\hline
 &  &  &  &  &  &  &  &  &  &  &  &  &  &  &  &  &  &  &  &  &  &  \\
\hline
 &  &  &  &  &  &  &  &  &  &  &  &  &  &  &  &  &  &  &  &  &  &  \\
\hline
 &  &  &  &  &  &  &  &  &  &  &  &  &  &  &  &  &  &  &  &  &  &  \\
\hline
 &  &  &  &  &  &  &  &  &  &  &  &  &  &  &  &  &  &  &  &  &  &  \\
\hline
 &  &  &  &  &  &  &  &  &  &  &  &  &  &  &  &  &  &  &  &  &  &  \\
\hline
 &  &  &  &  &  &  &  &  &  &  &  &  &  &  &  &  &  &  &  &  &  &  \\
\hline
 &  &  &  &  &  &  &  &  &  &  &  &  &  &  &  &  &  &  &  &  &  &  \\
\hline
 &  &  &  &  &  &  &  &  &  &  &  &  &  &  &  &  &  &  &  &  &  &  \\
\hline
 &  &  &  &  &  &  &  &  &  &  &  &  &  &  &  &  &  &  &  &  &  &  \\
\hline
 &  &  &  &  &  &  &  &  &  &  &  &  &  &  &  &  &  &  &  &  &  &  \\
\hline
 &  &  &  &  &  &  &  &  &  &  &  &  &  &  &  &  &  &  &  &  &  &  \\
\hline
 &  &  &  &  &  &  &  &  &  &  &  &  &  &  &  &  &  &  &  &  &  &  \\
\hline
 &  &  &  &  &  &  &  &  &  &  &  &  &  &  &  &  &  &  &  &  &  &  \\
\hline
 &  &  &  &  &  &  &  &  &  &  &  &  &  &  &  &  &  &  &  &  &  &  \\
\hline
 &  &  &  &  &  &  &  &  &  &  &  &  &  &  &  &  &  &  &  &  &  &  \\
\hline
 &  &  &  &  &  &  &  &  &  &  &  &  &  &  &  &  &  &  &  &  &  &  \\
\hline
 &  &  &  &  &  &  &  &  &  &  &  &  &  &  &  &  &  &  &  &  &  &  \\
\hline
 &  &  &  &  &  &  &  &  &  &  &  &  &  &  &  &  &  &  &  &  &  &  \\
\hline
 &  &  &  &  &  &  &  &  &  &  &  &  &  &  &  &  &  &  &  &  &  &  \\
\hline
 &  &  &  &  &  &  &  &  &  &  &  &  &  &  &  &  &  &  &  &  &  &  \\
\hline
 &  &  &  &  &  &  &  &  &  &  &  &  &  &  &  &  &  &  &  &  &  &  \\
\hline
 &  &  &  &  &  &  &  &  &  &  &  &  &  &  &  &  &  &  &  &  &  &  \\
\hline
 &  &  &  &  &  &  &  &  &  &  &  &  &  &  &  &  &  &  &  &  &  &  \\
\hline
 &  &  &  &  &  &  &  &  &  &  &  &  &  &  &  &  &  &  &  &  &  &  \\
\hline
 &  &  &  &  &  &  &  &  &  &  &  &  &  &  &  &  &  &  &  &  &  &  \\
\hline
 &  &  &  &  &  &  &  &  &  &  &  &  &  &  &  &  &  &  &  &  &  &  \\
\hline
 &  &  &  &  &  &  &  &  &  &  &  &  &  &  &  &  &  &  &  &  &  &  \\
\hline
 &  &  &  &  &  &  &  &  &  &  &  &  &  &  &  &  &  &  &  &  &  &  \\
\hline
 &  &  &  &  &  &  &  &  &  &  &  &  &  &  &  &  &  &  &  &  &  &  \\
\hline
 &  &  &  &  &  &  &  &  &  &  &  &  &  &  &  &  &  &  &  &  &  &  \\
\hline
 &  &  &  &  &  &  &  &  &  &  &  &  &  &  &  &  &  &  &  &  &  &  \\
\hline
 &  &  &  &  &  &  &  &  &  &  &  &  &  &  &  &  &  &  &  &  &  &  \\
\hline
 &  &  &  &  &  &  &  &  &  &  &  &  &  &  &  &  &  &  &  &  &  &  \\
\hline
 &  &  &  &  &  &  &  &  &  &  &  &  &  &  &  &  &  &  &  &  &  &  \\
\hline
 &  &  &  &  &  &  &  &  &  &  &  &  &  &  &  &  &  &  &  &  &  &  \\
\hline
 &  &  &  &  &  &  &  &  &  &  &  &  &  &  &  &  &  &  &  &  &  &  \\
\hline
 &  &  &  &  &  &  &  &  &  &  &  &  &  &  &  &  &  &  &  &  &  &  \\
\hline
\end{tabular}
\end{center}

\begin{center}
\includegraphics[max width=\textwidth]{2025_02_09_765256a0654ca524b15ag-16}
\end{center}

Odpowiedź:

\begin{center}
\begin{tabular}{|c|l|c|}
\hline
\multirow{3}{*}{\begin{tabular}{l}
Wypernia \\
egzaminator \\
\end{tabular}} & Nr zadania & 14. \\
\cline { 2 - 3 }
 & Maks. liczba pkt & 6 \\
\cline { 2 - 3 }
 & Uzyskana liczba pkt &  \\
\hline
\end{tabular}
\end{center}

\section*{Zadanie 15. (0-7)}
Rozpatrujemy wszystkie możliwe drewniane szkielety o kształcie przedstawionym na rysunku, wykonane z listewek. Każda z tych listewek ma kształt prostopadłościanu o podstawie kwadratu o boku długości \(x\). Wymiary szkieletu zaznaczono na rysunku.\\
\includegraphics[max width=\textwidth, center]{2025_02_09_765256a0654ca524b15ag-17}\\
a) Wyznacz objętość \(V\) drewna potrzebnego do budowy szkieletu jako funkcję zmiennej \(x\).\\
b) Wyznacz dziedzinę funkcji \(V\).\\
c) Oblicz tę wartość \(x\), dla której zbudowany szkielet jest możliwie najcięższy, czyli kiedy funkcja \(V\) osiąga wartość największą. Oblicz tę największą objętość.\\
\includegraphics[max width=\textwidth, center]{2025_02_09_765256a0654ca524b15ag-17(1)}\\
\includegraphics[max width=\textwidth, center]{2025_02_09_765256a0654ca524b15ag-18}

Odpowiedź:

\begin{center}
\begin{tabular}{|c|l|c|}
\hline
\multirow{3}{*}{\begin{tabular}{l}
Wypernia \\
egzaminator \\
\end{tabular}} & Nr zadania & 15. \\
\cline { 2 - 3 }
 & Maks. liczba pkt & 7 \\
\cline { 2 - 3 }
 & Uzyskana liczba pkt &  \\
\hline
\end{tabular}
\end{center}

\section*{BRUDNOPIS (nie podlega ocenie)}
\section*{BRUDNOPIS (nie podlega ocenie)}

\end{document}