\documentclass[a4paper,12pt]{article}
\usepackage{latexsym}
\usepackage{amsmath}
\usepackage{amssymb}
\usepackage{graphicx}
\usepackage{wrapfig}
\pagestyle{plain}
\usepackage{fancybox}
\usepackage{bm}

\begin{document}
\begin{center}
\includegraphics[width=181.368mm,height=312.000mm]{./F2_M_PR_M2015_page0_images/image001.eps}
\end{center}
Arkusz zawiera info acje

prawnie chronione do momentu

rozpoczęcia egzaminu.

1

UZUPELNIA ZDAJACY

KOD  PESEL

{\it miejsce}

{\it na naklejkę}

dysleksja

EGZAMIN MATU  LNY Z MATEMATYKI

POZIOM ROZSZERZONY

DATA: 8 maja 2015 r.

LICZBA P  KTÓW DO UZYS NIA: 50

Instrukcja dla zdającego

1.

2.

3.

Sprawdzí, czy arkusz egzaminacyjny zawiera 22 strony (zadania $1-16$).

Ewentualny brak zgłoś przewodniczącemu zespo nadzorującego

egzamin.

Rozwiązania zadań i odpowiedzi wpisuj w miejscu na to przeznaczonym.

Odpowiedzi do zadań zamkniętych $(1-5)$ przenieś na kartę odpowiedzi,

zaznaczając je w części ka $\mathrm{y}$ przeznaczonej dla zdającego. Zamaluj $\blacksquare$

pola do tego przeznaczone. Błędne zaznaczenie otocz kółkiem $\mathrm{O}\bullet$

i zaznacz właściwe.

4.

5.

Pamiętaj, $\dot{\mathrm{z}}\mathrm{e}$ pominięcie argumentacji lub istotnych obliczeń

w rozwiązaniu zadania otwa ego (7-16) $\mathrm{m}\mathrm{o}\dot{\mathrm{z}}\mathrm{e}$ spowodować, $\dot{\mathrm{z}}\mathrm{e}$ za to

rozwiązanie nie otrzymasz pełnej liczby punktów.

Pisz cz elnie i $\mathrm{u}\dot{\mathrm{z}}$ aj tvlko $\mathrm{d}$ gopisu lub -Dióra z czatnym tuszem lub

atramentem.

6. Nie uzywaj korektora, a błędne zapisy wyra $\acute{\mathrm{z}}\mathrm{n}\mathrm{i}\mathrm{e}$ prze eśl.

7. Pamiętaj, $\dot{\mathrm{z}}\mathrm{e}$ zapisy w brudnopisie nie będą oceniane.

8. $\mathrm{M}\mathrm{o}\dot{\mathrm{z}}$ esz korzystać z zestawu wzorów matematycznych, cyrkla i linijki oraz

kalkulatora prostego.

9. Na tej stronie oraz na karcie odpowiedzi wpisz swój numer PESEL

i przyklej naklejkę z kodem.

10. Nie wpisuj $\dot{\mathrm{z}}$ adnych znaków w części przeznaczonej dla egzaminatora.

$\Vert\Vert\Vert\Vert\Vert\Vert\Vert\Vert\Vert\Vert\Vert\Vert\Vert\Vert\Vert\Vert\Vert\Vert\Vert\Vert\Vert\Vert\Vert\Vert|$

$\mathrm{M}\mathrm{M}\mathrm{A}-\mathrm{R}1_{-}1\mathrm{P}-152$

Układ graficzny

\copyright CKE 2015

1




{\it Wzadaniach od l. do 5. wybierz i zaznacz na karcie odpowiedzi poprawnq odpowiedzí}.

ZadaOie $l.(0-1)$

Na rysunku przedstawiony

nierównoŚć $|2x-8|\leq 10.$

jest zbiór

wszystkich liczb

rzeczywistych

spełniających
\begin{center}
\includegraphics[width=165.504mm,height=11.988mm]{./F2_M_PR_M2015_page1_images/image001.eps}
\end{center}
$-1$  {\it k  x}

Stąd wynika, $\dot{\mathrm{z}}\mathrm{e}$

A. $k=2$

B. $k=4$

C. $k=5$

D. $k=9$

Zadanie 2. $(0-l\rangle$

Dana jest funkcja $f$ określona wzorem $f(x)=$

Równanie $f(x)=1$ ma dokładnie

A. jedno rozwiązanie.

B. dwa rozwiązania.

C. cztery rozwiązania.

D. pięć rozwiązań.

Zadanie 3. (0-1)

Liczba $(3-2\sqrt{3})^{3}$ jest równa

A. $27-24\sqrt{3}$ B. $27-30\sqrt{3}$

C. $135-78\sqrt{3}$

D. $135-30\sqrt{3}$

Zadanie 4. $(0-l\rangle$

Równanie 2 $\sin x+3\cos x=6$ w przedziale $(0,2\pi)$

A. nie ma rozwiązań rzeczywistych.

B. ma dokładniejedno rozwiązanie rzeczywiste.

C. ma dokładnie dwa rozwiązania rzeczywiste.

D. ma więcej $\mathrm{n}\mathrm{i}\dot{\mathrm{z}}$ dwa rozwiązania rzeczywiste.

Ządanie 5. $(0-1\rangle$

Odległość początku układu współrzędnych od prostej o równaniu $y=2x+4$ jest równa

A.

$\displaystyle \frac{\sqrt{5}}{5}$

B.

$\displaystyle \frac{4\sqrt{5}}{5}$

C.

-45

D. 4

Strona 2 z22

MMA-IR





Zadanie 11. (0-4)

$\mathrm{W}$ pierwszej utnie umieszczono 3 ku1e białe i 5 ku1 czamych, a w drugiej urnie 7 ku1 białych

$\mathrm{i}2$ kule czarne. Losujemy jedną kulę z pierwszej umy, przekładamy ją do urny drugiej

i dodatkowo dokładamy do umy drugiej jeszcze dwie kule tego samego koloru, co

wylosowana kula. Następnie losujemy dwie kule z umy drugiej. Oblicz prawdopodobieństwo

zdarzenia polegającego na tym, $\dot{\mathrm{z}}\mathrm{e}$ obie kule wylosowane z drugiej urny będą białe.

Odpowied $\acute{\mathrm{z}}$:
\begin{center}
\includegraphics[width=96.012mm,height=17.832mm]{./F2_M_PR_M2015_page10_images/image001.eps}
\end{center}
Wypelnia

egzaminator

Nr zadania

Maks. liczba kt

10.

4

11.

4

Uzyskana liczba pkt

IMA-IR

Strona ll z22





Zadanie $l2. (0-4)$

Funkcja $f$ określona jest wzorem $f(x)=x^{3}-2x^{2}+1$ dla $\mathrm{k}\mathrm{a}\dot{\mathrm{z}}$ dej liczby rzeczywistej $x.$

Wyznacz równania tych stycznych do wykresu funkcji $f$, które są równoległe do prostej

o równaniu $y=4x.$

Strona 12 z22

MMA-IR





Odpowiedzí :
\begin{center}
\includegraphics[width=82.044mm,height=17.784mm]{./F2_M_PR_M2015_page12_images/image001.eps}
\end{center}
Wypelnia

egzamÍnator

Nr zadania

Maks. liczba kt

12.

4

Uzyskana liczba pkt

IMA-IR

Strona 13 z22





Zadanie 13. $(0-5\rangle$

Dany jest trójmian kwadratowy $f(x)=(m+1)x^{2}+2(m-2)x-m+4$. Wyznacz wszystkie

wartości parametru $m$, dla których trójmian $f$ ma dwa rózne pierwiastki rzeczywiste $x_{1}, x_{2},$

spełniające warunek $x_{1}^{2}-x_{2}^{2}=x_{1}^{4}-x_{2}^{4}$

Strona 14 z22

MMA-IR





Odpowiedzí :
\begin{center}
\includegraphics[width=82.044mm,height=17.832mm]{./F2_M_PR_M2015_page14_images/image001.eps}
\end{center}
Wypelnia

egzaminator

Nr zadania

Maks. liczba kt

13.

5

Uzyskana liczba pkt

IMA-IR

Strona 15 z22





Zadanie $l4. \zeta 0-5\rangle$

Podstawą ostrosłupa ABCDS jest kwadrat ABCD. Krawędzí boczna $SD$ jest wysokością

ostrosłupa, ajej długość jest dwa razy większa od długości krawędzi podstawy. Oblicz sinus

kąta między ścianami bocznymi $ABS\mathrm{i}CBS$ tego ostrosłupa.

Strona 16 z22

MMA-IR





Odpowiedzí :
\begin{center}
\includegraphics[width=82.044mm,height=17.784mm]{./F2_M_PR_M2015_page16_images/image001.eps}
\end{center}
Wypelnia

egzamÍnator

Nr zadania

Maks. liczba kt

14.

5

Uzyskana liczba pkt

IMA-IR

Strona 17 z22





Zadanie 15. $(0-6)$

Suma wszystkich

czterech współczynników wielomianu

$W(x)=x^{3}+ax^{2}+bx+c$ jest

równa 0. Trzy pierwiastki tego wie1omianu tworzą ciąg arytmetyczny o róznicy równej 3.

Oblicz współczynniki $a, b \mathrm{i}c$. Rozwaz wszystkie $\mathrm{m}\mathrm{o}\dot{\mathrm{z}}$ liwe przypadki.

Strona 18 z22

MMA-IR





Odpowiedzí :
\begin{center}
\includegraphics[width=82.044mm,height=17.784mm]{./F2_M_PR_M2015_page18_images/image001.eps}
\end{center}
Wypelnia

egzamÍnator

Nr zadania

Maks. liczba kt

15.

Uzyskana liczba pkt

IMA-IR

Strona 19 z22





Zadanie $1\epsilon. (0-7)$

Rozpatrujemy wszystkie stozki, których przekrojem osiowym jest trójkąt o obwodzie 20.

Oblicz wysokość i promień podstawy tego stozka, którego objętość jest największa. Oblicz

objętość tego stozka.

Strona 20 z22

MMA-IR





{\it BRUDNOPIS} ({\it nie podlega ocenie})

Strona 3 z22





Odpowiedzí :
\begin{center}
\includegraphics[width=82.044mm,height=17.784mm]{./F2_M_PR_M2015_page20_images/image001.eps}
\end{center}
Wypelnia

egzamÍnator

Nr zadania

Maks. liczba kt

7

Uzyskana liczba pkt

IMA-IR

Strona 21 z22





{\it BRUDNOPIS} ({\it nie podlega ocenie})

Strona 22 z22

MD





Zadanie $\epsilon.(0-2)$

Oblicz granicę $\displaystyle \lim_{n\rightarrow\infty}(\frac{11n^{3}+6n+5}{6n^{3}+1}-\frac{2n^{2}+2n+1}{5n^{2}-4}). \mathrm{W}$ ponizsze kratki wpisz kolejno cyfrę

jedności i pierwsze dwie cyfry po przecinku rozwinięcia dziesiętnego otrzymanego wyniku.
\begin{center}
\includegraphics[width=22.500mm,height=10.920mm]{./F2_M_PR_M2015_page3_images/image001.eps}
\end{center}
Strona 4 z22

MMA-IR





Zadanie 7. (0-2)

Liczby $(-1) \mathrm{i}3$ są miejscami zerowymi funkcji kwadratowej $f$. Oblicz $\displaystyle \frac{f(6)}{f(12)}.$

Odpowied $\acute{\mathrm{z}}$:
\begin{center}
\includegraphics[width=96.012mm,height=17.832mm]{./F2_M_PR_M2015_page4_images/image001.eps}
\end{center}
Wypelnia

egzaminator

Nr zadania

Maks. liczba kt

2

7.

2

Uzyskana liczba pkt

IMA-IR

Strona 5 z22





Zadanie S. (0-3)

Udowodnij, $\dot{\mathrm{z}}\mathrm{e}$ dla $\mathrm{k}\mathrm{a}\dot{\mathrm{z}}$ dej liczby rzeczywistej $x$ prawdziwajest nierówność

$x^{4}-x^{2}-2x+3>0.$

Strona 6 z22

MD




\begin{center}
\includegraphics[width=82.044mm,height=17.784mm]{./F2_M_PR_M2015_page6_images/image001.eps}
\end{center}
Wypelnia

egzamÍnator

Nr zadania

Maks. liczba kt

8.

3

Uzyskana liczba pkt

Strona 7 z22





Zadanie 9. (0-3)

Dwusieczne czworokąta ABCD wpisanego w okrąg przecinają się w czterech róznych

punktach: P, Q, R, S (zobacz rysunek)
\begin{center}
\includegraphics[width=119.628mm,height=112.212mm]{./F2_M_PR_M2015_page7_images/image001.eps}
\end{center}
{\it D}

{\it A  S}

{\it P  R}

{\it Q}

{\it B}

{\it C}

Wykaz, $\dot{\mathrm{z}}\mathrm{e}$ na czworokącie PQRS mozna opisać okrąg.

Strona 8 z22

MMA-IR




\begin{center}
\includegraphics[width=82.044mm,height=17.784mm]{./F2_M_PR_M2015_page8_images/image001.eps}
\end{center}
Wypelnia

egzamÍnator

Nr zadania

Maks. liczba kt

3

Uzyskana liczba pkt

Strona 9 z22





Zadanie 10. (0-4)

Długości boków czworokąta ABCD są równe: $|AB|=2, |BC|=3, |CD|=4, |DA|=5.$

Na czworokącie ABCD opisano okrąg. Oblicz długość przekątnej $AC$ tego czworokąta.

Odpowiedzí:

Strona 10 z22

MMA-IR



\end{document}