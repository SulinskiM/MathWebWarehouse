\documentclass[a4paper,12pt]{article}
\usepackage{latexsym}
\usepackage{amsmath}
\usepackage{amssymb}
\usepackage{graphicx}
\usepackage{wrapfig}
\pagestyle{plain}
\usepackage{fancybox}
\usepackage{bm}

\begin{document}

{\it Wzadaniach od l. do 5. wybierz i zaznacz na karcie odpowiedzi poprawnq odpowiedzí}.

ZadaOie $l.(0-1)$

Na rysunku przedstawiony

nierównoŚć $|2x-8|\leq 10.$

jest zbiór

wszystkich liczb

rzeczywistych

spełniających
\begin{center}
\includegraphics[width=165.504mm,height=11.988mm]{./F2_M_PR_M2015_page1_images/image001.eps}
\end{center}
$-1$  {\it k  x}

Stąd wynika, $\dot{\mathrm{z}}\mathrm{e}$

A. $k=2$

B. $k=4$

C. $k=5$

D. $k=9$

Zadanie 2. $(0-l\rangle$

Dana jest funkcja $f$ określona wzorem $f(x)=$

Równanie $f(x)=1$ ma dokładnie

A. jedno rozwiązanie.

B. dwa rozwiązania.

C. cztery rozwiązania.

D. pięć rozwiązań.

Zadanie 3. (0-1)

Liczba $(3-2\sqrt{3})^{3}$ jest równa

A. $27-24\sqrt{3}$ B. $27-30\sqrt{3}$

C. $135-78\sqrt{3}$

D. $135-30\sqrt{3}$

Zadanie 4. $(0-l\rangle$

Równanie 2 $\sin x+3\cos x=6$ w przedziale $(0,2\pi)$

A. nie ma rozwiązań rzeczywistych.

B. ma dokładniejedno rozwiązanie rzeczywiste.

C. ma dokładnie dwa rozwiązania rzeczywiste.

D. ma więcej $\mathrm{n}\mathrm{i}\dot{\mathrm{z}}$ dwa rozwiązania rzeczywiste.

Ządanie 5. $(0-1\rangle$

Odległość początku układu współrzędnych od prostej o równaniu $y=2x+4$ jest równa

A.

$\displaystyle \frac{\sqrt{5}}{5}$

B.

$\displaystyle \frac{4\sqrt{5}}{5}$

C.

-45

D. 4

Strona 2 z22

MMA-IR
\end{document}
