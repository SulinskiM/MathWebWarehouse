\documentclass[a4paper,12pt]{article}
\usepackage{latexsym}
\usepackage{amsmath}
\usepackage{amssymb}
\usepackage{graphicx}
\usepackage{wrapfig}
\pagestyle{plain}
\usepackage{fancybox}
\usepackage{bm}

\begin{document}

Zadanie 11. (0-4)

$\mathrm{W}$ pierwszej utnie umieszczono 3 ku1e białe i 5 ku1 czamych, a w drugiej urnie 7 ku1 białych

$\mathrm{i}2$ kule czarne. Losujemy jedną kulę z pierwszej umy, przekładamy ją do urny drugiej

i dodatkowo dokładamy do umy drugiej jeszcze dwie kule tego samego koloru, co

wylosowana kula. Następnie losujemy dwie kule z umy drugiej. Oblicz prawdopodobieństwo

zdarzenia polegającego na tym, $\dot{\mathrm{z}}\mathrm{e}$ obie kule wylosowane z drugiej urny będą białe.

Odpowied $\acute{\mathrm{z}}$:
\begin{center}
\includegraphics[width=96.012mm,height=17.832mm]{./F2_M_PR_M2015_page10_images/image001.eps}
\end{center}
Wypelnia

egzaminator

Nr zadania

Maks. liczba kt

10.

4

11.

4

Uzyskana liczba pkt

IMA-IR

Strona ll z22
\end{document}
