\documentclass[a4paper,12pt]{article}
\usepackage{latexsym}
\usepackage{amsmath}
\usepackage{amssymb}
\usepackage{graphicx}
\usepackage{wrapfig}
\pagestyle{plain}
\usepackage{fancybox}
\usepackage{bm}

\begin{document}

Zadanie $1\epsilon. (0-7)$

Rozpatrujemy wszystkie stozki, których przekrojem osiowym jest trójkąt o obwodzie 20.

Oblicz wysokość i promień podstawy tego stozka, którego objętość jest największa. Oblicz

objętość tego stozka.

Strona 20 z22

MMA-IR
\end{document}
