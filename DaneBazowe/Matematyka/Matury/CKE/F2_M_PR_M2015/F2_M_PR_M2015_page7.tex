\documentclass[a4paper,12pt]{article}
\usepackage{latexsym}
\usepackage{amsmath}
\usepackage{amssymb}
\usepackage{graphicx}
\usepackage{wrapfig}
\pagestyle{plain}
\usepackage{fancybox}
\usepackage{bm}

\begin{document}

Zadanie 9. (0-3)

Dwusieczne czworokąta ABCD wpisanego w okrąg przecinają się w czterech róznych

punktach: P, Q, R, S (zobacz rysunek)
\begin{center}
\includegraphics[width=119.628mm,height=112.212mm]{./F2_M_PR_M2015_page7_images/image001.eps}
\end{center}
{\it D}

{\it A  S}

{\it P  R}

{\it Q}

{\it B}

{\it C}

Wykaz, $\dot{\mathrm{z}}\mathrm{e}$ na czworokącie PQRS mozna opisać okrąg.

Strona 8 z22

MMA-IR
\end{document}
