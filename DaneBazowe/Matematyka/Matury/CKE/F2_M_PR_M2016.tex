\documentclass[a4paper,12pt]{article}
\usepackage{latexsym}
\usepackage{amsmath}
\usepackage{amssymb}
\usepackage{graphicx}
\usepackage{wrapfig}
\pagestyle{plain}
\usepackage{fancybox}
\usepackage{bm}

\begin{document}

$\mathrm{c}\varepsilon \mathrm{N}\mathrm{T}\mathrm{R}\mathrm{A}\mathrm{l}\aleph$AKOMfSJAECZAMINACYJNA

Arkusz zawiera info acje

prawnie chronione do momentu

rozpoczęcia egzaminu.
\begin{center}
\includegraphics[width=22.908mm,height=19.248mm]{./F2_M_PR_M2016_page0_images/image001.eps}
\end{center}
1  $\iota$

UZUPELNIA ZDAJACY

{\it miejsce}

{\it na naklejkę}
\begin{center}
\includegraphics[width=21.900mm,height=14.736mm]{./F2_M_PR_M2016_page0_images/image002.eps}
\end{center}
KOD
\begin{center}
\includegraphics[width=79.656mm,height=14.736mm]{./F2_M_PR_M2016_page0_images/image003.eps}
\end{center}
PESEL
\begin{center}
\includegraphics[width=194.568mm,height=251.616mm]{./F2_M_PR_M2016_page0_images/image004.eps}
\end{center}
dysleksja

EGZAMIN MATU  LNY Z MATEMATY

POZIOM ROZSZE  ONY

LICZBA P  KTÓW DO UZYS NIA: 50

Instrukcja dla zdającego

1.

2.

3.

4.

5.

6.

Sprawdzí, czy arkusz egzaminacyjny zawiera 22 strony (zadania $1-16$).

Ewentualny brak zgloś przewodniczącemu zespo nadzo jącego

egzamin.

Rozwiązania zadań i odpowiedzi wpisuj w miejscu na to przeznaczonym.

Odpowiedzi do zadań za ię ch $(1-5)$ zaznacz na karcie odpowiedzi

w części karty przeznaczonej dla zdającego. Zamaluj $\blacksquare$ pola do tego

przeznaczone. Błędne zaznaczenie otocz kółkiem \copyright i zaznacz właściwe.

$\mathrm{W}$ zadaniu 6. wpisz odpowiednie cyf w atki pod treścią zadania.

Pamiętaj, $\dot{\mathrm{z}}\mathrm{e}$ pominięcie argumentacji lub istotnych obliczeń

w rozwiązaniu zadania otwa ego (7-16) $\mathrm{m}\mathrm{o}\dot{\mathrm{z}}\mathrm{e}$ spowodować, $\dot{\mathrm{z}}\mathrm{e}$ za to

rozwiązanie nie otr masz pełnej liczby pu tów.

Pisz cz elnie i $\mathrm{u}\dot{\mathrm{z}}$ aj lko $\mathrm{d}$ gopisu lub pióra z czarnym tuszem lub

atramentem.

7. Nie $\mathrm{u}\dot{\mathrm{z}}$ aj korektora, a błędne zapisy $\mathrm{r}\mathrm{a}\acute{\mathrm{z}}\mathrm{n}\mathrm{i}\mathrm{e}$ prze eśl.

8. Pamiętaj, $\dot{\mathrm{z}}\mathrm{e}$ zapisy w brudnopisie nie będą oceniane.

9. $\mathrm{M}\mathrm{o}\dot{\mathrm{z}}$ esz korzystać z zesta wzorów matematycznych, cyrkla i linijki oraz

kalkulatora prostego.

10. Na tej stronie oraz na karcie odpowiedzi wpisz swój numer PESEL

i przyklej naklejkę z kodem.

ll. Nie wpisuj $\dot{\mathrm{z}}$ adnych znaków w części przeznaczonej dla egzaminatora.

$\Vert\Vert\Vert\Vert\Vert\Vert\Vert\Vert\Vert\Vert\Vert\Vert\Vert\Vert\Vert\Vert\Vert\Vert\Vert\Vert\Vert\Vert\Vert\Vert|$

$\mathrm{M}\mathrm{M}\mathrm{A}-\mathrm{R}1_{-}1\mathrm{P}-162$
\begin{center}
\includegraphics[width=22.908mm,height=19.248mm]{./F2_M_PR_M2016_page0_images/image005.eps}
\end{center}
1  $\iota$

Układ graficzny

\copyright CKE 2015




{\it Wzadaniach od l. do 5. wybierz i zaznacz na karcie odpowiedzi poprawnq odpowiedzí}.

Zadaoie $l.(0-1)$

$\mathrm{W}$ rozwinięciu wyrazenia $(2\sqrt{3}x+4y)^{3}$ współczynnik przy iloczynie $xy^{2}$ jest równy

A. $32\sqrt{3}$

B. 48

C. $96\sqrt{3}$

D. 144

Zadanie 2. (0-1)

Wielomian $W(x)=6x^{3}+3x^{2}-5x+p$ jest podzielny przez dwumian $x-1$ dla $p$ równego

A. 4

B. $-2$

C. 2

D. $-4$

Zadanie 3. (0-1)

Na rysunku przedstawiono fragment wykresu

dziedziną jest zbiór $D=(-\infty,3)\cup(3,+\infty).$

funkcji homograficznej $y=f(x)$, której

Równanie $|f(x)|=p$ z niewiadomą $x$ ma dokładniejedno rozwiązanie

A.

C.

w dwóch przypadkach: $p=0$ lub $p=3.$

tylko wtedy, gdy $p=3.$

B.

D.

w dwóch przypadkach: $p=0$ lub $p=2.$

tylko wtedy, gdy $p=2.$

Zadanie 4. (0-1)

Funkcja $f(x)=\displaystyle \frac{3x-1}{x^{2}+4}$ jest określona dla $\mathrm{k}\mathrm{a}\dot{\mathrm{z}}$ dej liczby rzeczywistej $x$. Pochodna tej funkcji

jest określona wzorem

A.

$f'(x)=\displaystyle \frac{-3x^{2}+2x+12}{(x^{2}+4)^{2}}$

B.

$f'(x)=\displaystyle \frac{-9x^{2}+2x-12}{(x^{2}+4)^{2}}$

C.

$f'(x)=\displaystyle \frac{3x^{2}-2x-12}{(x^{2}+4)^{2}}$

D.

$f'(x)=\displaystyle \frac{9x^{2}-2x+12}{(x^{2}+4)^{2}}$

Strona 2 z22

MMA-IR





Zadanie 11. (0-4)

Rozwiąz nierówność $\displaystyle \frac{2\cos x-\sqrt{3}}{\cos^{2}x}<0$ w przedziale $\langle 0, 2\pi\rangle.$

Odpowiedzí :
\begin{center}
\includegraphics[width=96.012mm,height=17.832mm]{./F2_M_PR_M2016_page10_images/image001.eps}
\end{center}
Wypelnia

egzaminator

Nr zadania

Maks. liczba kt

10.

4

11.

4

Uzyskana liczba pkt

IMA-IR

Strona ll z22





Zadanie $l2\cdot(0-6)$

Dany jest trójmian

kwadratowy $f(x)=x^{2}+2(m+1)x+6m+1.$

Wyznacz wszystkie

rzeczywiste wartości parametru $m$, dla których ten trójmian ma dwa rózne pierwiastki $x_{1}, x_{2}$

tego samego znaku, spełniające warunek $|x_{1}-x_{2}|<3.$

Strona 12 z22

MMA-IR





Odpowiedzí :
\begin{center}
\includegraphics[width=82.044mm,height=17.832mm]{./F2_M_PR_M2016_page12_images/image001.eps}
\end{center}
Wypelnia

egzaminator

Nr zadania

Maks. liczba kt

12.

Uzyskana liczba pkt

IMA-IR

Strona 13 z22





Zadanie 13. $(0-5\rangle$

Punkty $A=(30,32) \mathrm{i} B=(0,8)$ są sąsiednimi wierzchołkami czworokąta ABCD wpisanego

w okrąg. Prosta o równaniu $x-y+2=0$ jest jedyną osią symetrii tego czworokąta i zawiera

przekątną $AC$. Oblicz współrzędne wierzchołków $C\mathrm{i}D$ tego czworokąta.

Strona 14 z22

MMA-IR





Odpowiedzí :
\begin{center}
\includegraphics[width=82.044mm,height=17.832mm]{./F2_M_PR_M2016_page14_images/image001.eps}
\end{center}
Wypelnia

egzaminator

Nr zadania

Maks. liczba kt

13.

5

Uzyskana liczba pkt

IMA-IR

Strona 15 z22





Zadanie $l4. \zeta 0-3$)

Rozpatrujemy wszystkie liczby naturalne dziesięciocyfrowe, w zapisie których mogą

występować wyłącznie cyfry 1, 2, 3, przy czym cyfra 1 występuje dokładnie trzy razy.

Uzasadnij, $\dot{\mathrm{z}}\mathrm{e}$ takich liczb jest 15360.

Strona 16 z22

MMA-IR





Odpowiedzí :
\begin{center}
\includegraphics[width=82.044mm,height=17.784mm]{./F2_M_PR_M2016_page16_images/image001.eps}
\end{center}
Wypelnia

egzamÍnator

Nr zadania

Maks. liczba kt

14.

3

Uzyskana liczba pkt

IMA-IR

Strona 17 z22





Zadanie 15. $(0-6)$

$\mathrm{W}$ ostrosłupie prawidłowym czworokątnym ABCDS o podstawie ABCD wysokość jest równa 5,

a kąt między sąsiednimi ścianami bocznymi ostrosłupa ma miarę $120^{\mathrm{o}}$ Oblicz objętość tego

ostrosłupa.

Strona 18 z22

MMA-IR





Odpowiedzí :
\begin{center}
\includegraphics[width=82.044mm,height=17.832mm]{./F2_M_PR_M2016_page18_images/image001.eps}
\end{center}
Wypelnia

egzaminator

Nr zadania

Maks. liczba kt

15.

Uzyskana liczba pkt

IMA-IR

Strona 19 z22





Zadanie $1\epsilon. (0-7)$

Parabola o równaniu $y=2-\displaystyle \frac{1}{2}x^{2}$ przecina oś $Ox$ układu współrzędnych w punktach

$A=(-2,0) \mathrm{i} B=(2,0)$. Rozpatrujemy wszystkie trapezy równoramienne ABCD, których

dłuzszą podstawą jest odcinek $AB$, a końce $C\mathrm{i}D$ krótszej podstawy lez$\cdot$ą na paraboli (zobacz

rysunek).
\begin{center}
\includegraphics[width=87.120mm,height=50.592mm]{./F2_M_PR_M2016_page19_images/image001.eps}
\end{center}
{\it D C}

1

{\it A}

2

{\it B}

$-1$  0 1

Wyznacz pole trapezu ABCD w zalezności od pierwszej współrzędnej wierzchołka C. Oblicz

współrzędne wierzchołka C tego z rozpatrywanych trapezów, którego polejest największe.

Strona 20 z22

MMA-IR





{\it BRUDNOPIS} ({\it nie podlega ocenie})

Strona 3 z22





Odpowiedzí :
\begin{center}
\includegraphics[width=82.044mm,height=17.832mm]{./F2_M_PR_M2016_page20_images/image001.eps}
\end{center}
Wypelnia

egzaminator

Nr zadania

Maks. liczba kt

7

Uzyskana liczba pkt

IMA-IR

Strona 21 z22





{\it BRUDNOPIS} ({\it nie podlega ocenie})

Strona 22 z22

MD





Zadanie 5. $(0-l)$

Granica $\displaystyle \lim_{n\rightarrow\infty}\frac{(pn^{2}+4n)^{3}}{5n^{6}-4}=-\frac{8}{5}$. Wynika stąd, $\dot{\mathrm{z}}\mathrm{e}$

A.

$p=-8$

B.

$p=4$

C.

$p=2$

D.

$p=-2$

Zadanie $\epsilon$, (0-2)

Wśród 10 tysięcy mieszkańców pewnego miasta przeprowadzono sondaz dotyczący budowy

przedszkola publicznego. Wyniki sondaz$\mathrm{u}$ przedstawiono w tabeli.
\begin{center}
\begin{tabular}{|l|l|l|}
\hline
\multicolumn{1}{|l|}{Badane grupy}&	\multicolumn{1}{|l|}{$\begin{array}{l}\mbox{Liczba osób popierających}	\\	\mbox{budowę przedszkola}	\end{array}$}&	\multicolumn{1}{|l|}{$\begin{array}{l}\mbox{Liczba osób niepopierających}	\\	\mbox{budowy przedszkola}	\end{array}$}	\\
\hline
\multicolumn{1}{|l|}{Kobiety}&	\multicolumn{1}{|l|}{$5140$}&	\multicolumn{1}{|l|}{ $1860$}	\\
\hline
\multicolumn{1}{|l|}{Męzczyzíni}&	\multicolumn{1}{|l|}{$2260$}&	\multicolumn{1}{|l|}{ $740$}	\\
\hline
\end{tabular}

\end{center}
Oblicz prawdopodobieństwo zdarzenia polegającego na tym, $\dot{\mathrm{z}}\mathrm{e}$ losowo wybrana osoba,

spośród ankietowanych, popiera budowę przedszkola, jeśli wiadomo, $\dot{\mathrm{z}}\mathrm{e}$ jest męzczyzną.

Zakoduj trzy pierwsze cyfry po przecinku nieskończonego rozwinięcia dziesiętnego

otrzymanego wyniku.
\begin{center}
\includegraphics[width=22.500mm,height=10.920mm]{./F2_M_PR_M2016_page3_images/image001.eps}
\end{center}
{\it BRUDNOPIS} ({\it nie podlega ocenie})

Strona 4 z22

MMA-IR





Zadanie 7. (0-2)

Dany jest ciąg geometryczny $(a_{n})$ określony wzorem $a_{n}=(\displaystyle \frac{1}{2x-371})^{n}$ dla $n\geq 1$. Wszystkie

wyrazy tego ciągu są dodatnie. Wyznacz najmniejszą liczbę całkowitą $x$, dla której

nieskończony szereg $a_{1}+a_{2}+a_{3}+$ jest zbiezny.

Odpowied $\acute{\mathrm{z}}$:
\begin{center}
\includegraphics[width=96.012mm,height=17.832mm]{./F2_M_PR_M2016_page4_images/image001.eps}
\end{center}
Wypelnia

egzaminator

Nr zadania

Maks. liczba kt

2

7.

2

Uzyskana liczba pkt

IMA-IR

Strona 5 z22





Zadanie S. (0-3)

Wykaz, $\dot{\mathrm{z}}\mathrm{e}$ dla dowolnych dodatnich liczb rzeczywistych $x \mathrm{i} y$ takich, $\dot{\mathrm{z}}\mathrm{e}$

prawdziwajest nierówność $x+y\leq 2.$

$x^{2}+y^{2}=2,$

Strona 6 z22

MMA-IR




\begin{center}
\includegraphics[width=82.044mm,height=17.784mm]{./F2_M_PR_M2016_page6_images/image001.eps}
\end{center}
Wypelnia

egzamÍnator

Nr zadania

Maks. liczba kt

8.

3

Uzyskana liczba pkt

Strona 7 z22





Zadanie 9. (0-3)

Dany jest prostokąt ABCD. Okrąg wpisany w trójkąt BCD jest styczny do przekątnej BD

w punkcie N. Okrąg wpisany w trójkąt ABD jest styczny do boku AD w punkcie M, a środek S

tego okręgu lezy na odcinku MN, jak na rysunku.
\begin{center}
\includegraphics[width=77.016mm,height=51.204mm]{./F2_M_PR_M2016_page7_images/image001.eps}
\end{center}
{\it D C}

{\it M  N  S}

{\it A  B}

Wykaz, $\dot{\mathrm{z}}\mathrm{e}|MN|=|AD|.$

Strona 8 z22

MMA-IR




\begin{center}
\includegraphics[width=82.044mm,height=17.832mm]{./F2_M_PR_M2016_page8_images/image001.eps}
\end{center}
Wypelnia

egzaminator

Nr zadania

Maks. liczba kt

3

Uzyskana liczba pkt

Strona 9 z22





Zadanie 10. (0-4)

Wyznacz wszystkie wartości parametru $a$, dla których wykresy funkcji $f\mathrm{i}g$, określonych

wzorami $f(x)=x-2$ oraz $g(x)=5-ax$, przecinają się w punkcie o obu współrzędnych

dodatnich.

Odpowiedzí:

Strona 10 z22

MMA-IR



\end{document}