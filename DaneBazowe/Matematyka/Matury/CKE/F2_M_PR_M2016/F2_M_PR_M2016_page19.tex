\documentclass[a4paper,12pt]{article}
\usepackage{latexsym}
\usepackage{amsmath}
\usepackage{amssymb}
\usepackage{graphicx}
\usepackage{wrapfig}
\pagestyle{plain}
\usepackage{fancybox}
\usepackage{bm}

\begin{document}

Zadanie $1\epsilon. (0-7)$

Parabola o równaniu $y=2-\displaystyle \frac{1}{2}x^{2}$ przecina oś $Ox$ układu współrzędnych w punktach

$A=(-2,0) \mathrm{i} B=(2,0)$. Rozpatrujemy wszystkie trapezy równoramienne ABCD, których

dłuzszą podstawą jest odcinek $AB$, a końce $C\mathrm{i}D$ krótszej podstawy lez$\cdot$ą na paraboli (zobacz

rysunek).
\begin{center}
\includegraphics[width=87.120mm,height=50.592mm]{./F2_M_PR_M2016_page19_images/image001.eps}
\end{center}
{\it D C}

1

{\it A}

2

{\it B}

$-1$  0 1

Wyznacz pole trapezu ABCD w zalezności od pierwszej współrzędnej wierzchołka C. Oblicz

współrzędne wierzchołka C tego z rozpatrywanych trapezów, którego polejest największe.

Strona 20 z22

MMA-IR
\end{document}
