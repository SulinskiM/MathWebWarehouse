\documentclass[a4paper,12pt]{article}
\usepackage{latexsym}
\usepackage{amsmath}
\usepackage{amssymb}
\usepackage{graphicx}
\usepackage{wrapfig}
\pagestyle{plain}
\usepackage{fancybox}
\usepackage{bm}

\begin{document}

Zadanie 5. $(0-l)$

Granica $\displaystyle \lim_{n\rightarrow\infty}\frac{(pn^{2}+4n)^{3}}{5n^{6}-4}=-\frac{8}{5}$. Wynika stąd, $\dot{\mathrm{z}}\mathrm{e}$

A.

$p=-8$

B.

$p=4$

C.

$p=2$

D.

$p=-2$

Zadanie $\epsilon$, (0-2)

Wśród 10 tysięcy mieszkańców pewnego miasta przeprowadzono sondaz dotyczący budowy

przedszkola publicznego. Wyniki sondaz$\mathrm{u}$ przedstawiono w tabeli.
\begin{center}
\begin{tabular}{|l|l|l|}
\hline
\multicolumn{1}{|l|}{Badane grupy}&	\multicolumn{1}{|l|}{$\begin{array}{l}\mbox{Liczba osób popierających}	\\	\mbox{budowę przedszkola}	\end{array}$}&	\multicolumn{1}{|l|}{$\begin{array}{l}\mbox{Liczba osób niepopierających}	\\	\mbox{budowy przedszkola}	\end{array}$}	\\
\hline
\multicolumn{1}{|l|}{Kobiety}&	\multicolumn{1}{|l|}{$5140$}&	\multicolumn{1}{|l|}{ $1860$}	\\
\hline
\multicolumn{1}{|l|}{Męzczyzíni}&	\multicolumn{1}{|l|}{$2260$}&	\multicolumn{1}{|l|}{ $740$}	\\
\hline
\end{tabular}

\end{center}
Oblicz prawdopodobieństwo zdarzenia polegającego na tym, $\dot{\mathrm{z}}\mathrm{e}$ losowo wybrana osoba,

spośród ankietowanych, popiera budowę przedszkola, jeśli wiadomo, $\dot{\mathrm{z}}\mathrm{e}$ jest męzczyzną.

Zakoduj trzy pierwsze cyfry po przecinku nieskończonego rozwinięcia dziesiętnego

otrzymanego wyniku.
\begin{center}
\includegraphics[width=22.500mm,height=10.920mm]{./F2_M_PR_M2016_page3_images/image001.eps}
\end{center}
{\it BRUDNOPIS} ({\it nie podlega ocenie})

Strona 4 z22

MMA-IR
\end{document}
