\documentclass[a4paper,12pt]{article}
\usepackage{latexsym}
\usepackage{amsmath}
\usepackage{amssymb}
\usepackage{graphicx}
\usepackage{wrapfig}
\pagestyle{plain}
\usepackage{fancybox}
\usepackage{bm}

\begin{document}

Zadanie 7. (0-2)

Dany jest ciąg geometryczny $(a_{n})$ określony wzorem $a_{n}=(\displaystyle \frac{1}{2x-371})^{n}$ dla $n\geq 1$. Wszystkie

wyrazy tego ciągu są dodatnie. Wyznacz najmniejszą liczbę całkowitą $x$, dla której

nieskończony szereg $a_{1}+a_{2}+a_{3}+$ jest zbiezny.

Odpowied $\acute{\mathrm{z}}$:
\begin{center}
\includegraphics[width=96.012mm,height=17.832mm]{./F2_M_PR_M2016_page4_images/image001.eps}
\end{center}
Wypelnia

egzaminator

Nr zadania

Maks. liczba kt

2

7.

2

Uzyskana liczba pkt

IMA-IR

Strona 5 z22
\end{document}
