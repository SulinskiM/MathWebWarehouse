\documentclass[a4paper,12pt]{article}
\usepackage{latexsym}
\usepackage{amsmath}
\usepackage{amssymb}
\usepackage{graphicx}
\usepackage{wrapfig}
\pagestyle{plain}
\usepackage{fancybox}
\usepackage{bm}

\begin{document}

$\mathrm{g}_{\mathrm{E}\mathrm{G}\mathrm{Z}\mathrm{A}\mathrm{M}\mathrm{I}\mathrm{N}\mathrm{A}\subset \mathrm{Y}\mathrm{J}\mathrm{N}\mathrm{A}}^{\mathrm{C}\mathrm{E}\mathrm{N}\mathrm{T}\mathrm{R}\mathrm{A}\mathrm{L}\mathrm{N}\mathrm{A}}\mathrm{K}\mathrm{O}\mathrm{M}1\mathrm{S}\mathrm{J}\mathrm{A}$

Arkusz zawiera informacje

prawnie chronione do momentu

rozpoczęcia egzaminu.

UZUPELNIA ZDAJACY

{\it miejsce}

{\it na naklejkę}
\begin{center}
\includegraphics[width=21.900mm,height=16.056mm]{./F2_M_PR_M2017_page0_images/image001.eps}
\end{center}
KOD
\begin{center}
\includegraphics[width=79.656mm,height=16.104mm]{./F2_M_PR_M2017_page0_images/image002.eps}
\end{center}
PESEL
\begin{center}
\includegraphics[width=195.984mm,height=236.676mm]{./F2_M_PR_M2017_page0_images/image003.eps}
\end{center}
EGZAMIN MATU  LNY

Z MATEMATY

POZIOM ROZSZE  ONY

DATA: 9 maja 2017 $\mathrm{r}.$

CZAS P CY: $ 18\Uparrow$ minut

LICZBA P KTÓW DO UZYS NIA: 50

Instrukcja dla zdającego

l. Sprawdzí, czy arkusz egzaminacyjny zawiera 18 stron (zadania $1-15$).

Ewentualny brak zgłoś przewodniczącemu zespo nadzorującego

egzamin.

2. Rozwiązania zadań i odpowiedzi wpisuj w miejscu na to przeznaczonym.

3. Odpowiedzi do zadań za ię ch (l ) zaznacz na karcie odpowiedzi

w części ka $\mathrm{y}$ przeznaczonej dla zdającego. Zamaluj $\blacksquare$ pola do tego

przeznaczone. Błędne zaznaczenie otocz kólkiem \copyright i zaznacz wlaściwe.

4. $\mathrm{W}$ zadaniu 5. wpisz odpowiednie cyf w atki pod treścią zadania.

5. Pamiętaj, $\dot{\mathrm{z}}\mathrm{e}$ pominięcie argumentacji lub istotnych obliczeń

w rozwiązaniu zadania otwa ego (6-15) $\mathrm{m}\mathrm{o}\dot{\mathrm{z}}\mathrm{e}$ spowodować, $\dot{\mathrm{z}}\mathrm{e}$ za to

rozwiązanie nie otrzymasz pelnej liczby pu tów.

6. Pisz cz elnie i $\mathrm{u}\dot{\mathrm{z}}$ aj lko $\mathrm{d}$ gopisu lub pióra z czamym tuszem lub

atramentem.

7. Nie $\mathrm{u}\dot{\mathrm{z}}$ aj korektora, a błędne zapisy razínie prze eśl.

8. Pamiętaj, $\dot{\mathrm{z}}\mathrm{e}$ zapisy w brudnopisie nie będą oceniane.

9. $\mathrm{M}\mathrm{o}\dot{\mathrm{z}}$ esz korzystać z zesta wzorów matematycznych, cyrkla i linijki oraz

kal latora prostego.

10. Na tej stronie oraz na karcie odpowiedzi wpisz swój numer PESEL

i przyklej naklejkę z kodem.

ll. Nie wpisuj $\dot{\mathrm{z}}$ adnych znaków w części przeznaczonej dla egzaminatora.

$\Vert\Vert\Vert\Vert\Vert\Vert\Vert\Vert\Vert\Vert\Vert\Vert\Vert\Vert\Vert\Vert\Vert\Vert\Vert\Vert\Vert\Vert\Vert\Vert|$

$\mathrm{M}\mathrm{M}\mathrm{A}-\mathrm{R}1_{-}1\mathrm{P}-172$

Układ graficzny

\copyright CKE 2015




{\it Wzadaniach od l. do 4. wybierz i zaznacz na karcie odpowiedzi poprawnq odpowiedzí}.

Zadaoie $l.(0-1)$

Liczba $(\sqrt{2-\sqrt{3}}-\sqrt{2+\sqrt{3}})^{2}$ jest równa

A. 2

B. 4

C. $\sqrt{3}$

D.

$2\sqrt{3}$

Zadanie 2. $(0-l)$

Nieskończony ciąg

Wtedy

liczbowy jest określony wzorem

{\it an}$=$ -({\it n}22-{\it n}130$+${\it nn}) (22$+$-33{\it n})

dla $n\geq 1.$

A.

$\displaystyle \lim_{n\rightarrow\infty}a_{n}=\frac{1}{2}$

B.

$\displaystyle \lim_{n\rightarrow\infty}a_{n}=0$

C.

$\displaystyle \lim_{n\rightarrow\infty}a_{n}=-\infty$

D.

$\displaystyle \lim_{n\rightarrow\infty}a_{n}=-\frac{3}{2}$

Zadanie 3. $(0-l)$

Odcinek $CD$ jest wysokością trójkąta $ABC$, w którym $|AD|=|CD|=\displaystyle \frac{1}{2}|BC|$ (zobacz rysunek).

Okrąg o środku $C$ i promieniu $CD$ jest styczny do prostej $AB$. Okrąg ten przecina boki

$AC\mathrm{i}BC$ trójkąta odpowiednio w punktach $K\mathrm{i}L.$
\begin{center}
\includegraphics[width=63.804mm,height=47.652mm]{./F2_M_PR_M2017_page1_images/image001.eps}
\end{center}
{\it M}

$\alpha$

{\it C}

{\it L}

{\it K}

{\it A  D  B}

Zaznaczony na rysunku kąt $\alpha$ wpisany w okrągjest równy

A. $37,5^{\mathrm{o}}$

B. $45^{\mathrm{o}}$

C. 52, $5^{\mathrm{o}}$

D. $60^{\mathrm{o}}$

Zadanie 4. (0-1)

Dane są punkt $B=(-4,7)$ i wektor $\vec{u}=[-3,5]$. Punkt $A$, taki, $\dot{\mathrm{z}}\mathrm{e}\vec{AB}=-3\vec{u}$, ma współrzędne

A. $A=(5,-8)$

B. $A=(-13,22)$

C. $A=(9,-15)$

D. $A=(12,24)$

Strona 2 z18

MMA-IR





Odpowied $\acute{\mathrm{z}}$:
\begin{center}
\includegraphics[width=82.044mm,height=17.784mm]{./F2_M_PR_M2017_page10_images/image001.eps}
\end{center}
Wypelnia

egzamÍnator

Nr zadania

Maks. liczba kt

12.

5

Uzyskana liczba pkt

IMA-IR

Strona ll z18





Zadanie 13. $(0-5\rangle$

Wyznacz równanie okręgu przechodzącego przez punkty $A=(-5,3) \mathrm{i} B=(0,6)$, którego

środek lezy na prostej o równaniu $x-3y+1=0.$

Strona 12 z18

MMA-IR





Odpowied $\acute{\mathrm{z}}$:
\begin{center}
\includegraphics[width=82.044mm,height=17.784mm]{./F2_M_PR_M2017_page12_images/image001.eps}
\end{center}
Wypelnia

egzamÍnator

Nr zadania

Maks. liczba kt

13.

5

Uzyskana liczba pkt

IMA-IR

Strona 13 z18





Zadanie $l4. (0-6)$

Liczby $a, b, c$ są - odpowiednio - pierwszym, drugim i trzecim wyrazem ciągu

arytmetycznego. Suma tych liczb jest równa 27. Ciąg $(a-2,b,2c+1)$ jest geometryczny.

Wyznacz liczby $a, b, c.$

Strona 14 z18

MMA-IR





Odpowied $\acute{\mathrm{z}}$:
\begin{center}
\includegraphics[width=82.044mm,height=17.784mm]{./F2_M_PR_M2017_page14_images/image001.eps}
\end{center}
Wypelnia

egzamÍnator

Nr zadania

Maks. liczba kt

14.

Uzyskana liczba pkt

IMA-IR

Strona 15 z18





Zadanie 15. (0-7)

Rozpatrujemy wszystkie walce o danym polu powierzchni całkowitej P. Oblicz wysokość

i promień podstawy tego walca, którego objętość jest największa. Oblicz tę największą

objętość.

Strona 16 z18

MMA-IR





Odpowied $\acute{\mathrm{z}}$:
\begin{center}
\includegraphics[width=82.044mm,height=17.784mm]{./F2_M_PR_M2017_page16_images/image001.eps}
\end{center}
Wypelnia

egzamÍnator

Nr zadania

Maks. liczba kt

15.

7

Uzyskana liczba pkt

IMA-IR

Strona 17 z18





{\it BRUDNOPIS} ({\it nie podlega ocenie})

Strona 18 z18

Nl





{\it BRUDNOPIS} ({\it nie podlega ocenie})

Strona 3 z18





Zadanie 5. (0-2)

Reszta z dzielenia wielomianu $W(x)=x^{3}-2x^{2}+ax+\displaystyle \frac{3}{4}$ przez dwumian $x-2$ jest równa l.

Oblicz wartość współczynnika $a.$

$\mathrm{W}$ ponizsze kratki wpisz kolejno trzy pierwsze cyfry po przecinku rozwinięcia dziesiętnego

otrzymanego wyniku.
\begin{center}
\includegraphics[width=22.500mm,height=10.920mm]{./F2_M_PR_M2017_page3_images/image001.eps}
\end{center}
{\it BRUDNOPIS} ({\it nie podlega ocenie})

Zadanie 6. (0-3)

Funkcja $f$ jest określona wzorem $f(x)=\displaystyle \frac{x-1}{x^{2}+1}$ dla $\mathrm{k}\mathrm{a}\dot{\mathrm{z}}$ dej liczby rzeczywistej $x$. Wyznacz

równanie stycznej do wykresu tej funkcji w punkcie $P=(1,0).$

Odpowiedzí:

Strona 4 z18

MMA-IR





Zadanie 7. (0-3)

Udowodnij, $\dot{\mathrm{z}}\mathrm{e}$ dla dowolnych róznych liczb rzeczywistych $x,y$ prawdziwajest nierówność

$x^{2}y^{2}+2x^{2}+2y^{2}-8xy+4>0.$
\begin{center}
\includegraphics[width=109.980mm,height=17.832mm]{./F2_M_PR_M2017_page4_images/image001.eps}
\end{center}
Nr zadania

Wypelnia Maks. liczba kt

egzaminator

Uzyskana liczba pkt

5.

2

3

7.

3

IMA-IR

Strona 5 z18





Zadanie 8. $(0\rightarrow 3)$

$\mathrm{W}$ trójkącie ostrokątnym $ABC$ bok $AB$ ma długość $c$, długość boku $BC$ jest równa $a$ oraz

$|\leq ABC|=\beta$. Dwusieczna kąta $ABC$ przecina bok $AC$ trójkąta w punkcie $E.$

Wykaz, $\dot{\mathrm{z}}\mathrm{e}$ długość odcinka $BE$ jest równa $\displaystyle \frac{2ac\cdot\cos\frac{\beta}{2}}{a+c}$

Strona 6 z18

MMA-IR





Zadanie 9. (0-4)

$\mathrm{W}$ czworościanie, którego wszystkie krawędzie mają taką samą długość 6, umieszczono ku1ę

tak, $\dot{\mathrm{z}}\mathrm{e}$ ma ona dokładniejeden punkt wspólny z $\mathrm{k}\mathrm{a}\dot{\mathrm{z}}$ dą ścianą czworościanu. Płaszczyzna $\pi,$

równoległa do podstawy tego czworościanu, dzieli go na dwie bryły: ostrosłup o objętości

równej $\displaystyle \frac{8}{27}$ objętości dzielonego czworościanu i ostrosłup ścięty. Oblicz odległość środka $S$

kuli od płaszczyzny $\pi$, tj. długość najkrótszego spośród odcinków $SP$, gdzie Pjest punktem

płaszczyzny $\pi.$

Odpowiedzí :
\begin{center}
\includegraphics[width=96.012mm,height=17.832mm]{./F2_M_PR_M2017_page6_images/image001.eps}
\end{center}
Wypelnia

egzaminator

Nr zadania

Maks. liczba kt

8.

3

4

Uzyskana liczba pkt

IMA-IR

Strona 7 z18





Zadanie 10. (0-4)

Rozwiąz równanie $\cos 2x+3\cos x=-2$ w przedziale $\langle 0, 2\pi\rangle.$

Odpowiedzí:

Strona 8 z18

M





Zadanie 11. (0-4)

$\mathrm{W}$ pudełku znajduje się 8 piłeczek oznaczonych ko1ejnymi 1iczbami natura1nymi od 1 do 8.

Losujemy jedną piłeczkę, zapisujemy liczbę na niej występującą, a następnie zwracamy

piłeczkę do umy. Tę procedurę wykonujemy jeszcze dwa razy i tym samym otrzymujemy

zapisane trzy liczby. Oblicz prawdopodobieństwo wylosowania takich piłeczek, $\dot{\mathrm{z}}\mathrm{e}$ iloczyn

trzech zapisanych liczb jest podzielny przez 4. 1Vynik podaj w postaci ułamka zwykłego.

Odpowiedzí :
\begin{center}
\includegraphics[width=96.012mm,height=17.832mm]{./F2_M_PR_M2017_page8_images/image001.eps}
\end{center}
Wypelnia

egzaminator

Nr zadania

Maks. liczba kt

10.

4

11.

4

Uzyskana liczba pkt

IMA-IR

Strona 9 z18





Zadanie $ l2.(0-5\rangle$

Wyznacz wszystkie wartości parametru $m$, dla których równanie

$4x^{2}-6mx+(2m+3)(m-3)=0$

ma dwa rózne rozwiązania rzeczywiste $x_{1}$ i $x_{2}$, przy czym $x_{1}<x_{2}$, spełniające warunek

$(4x_{1}-4x_{2}-1)(4x_{1}-4x_{2}+1)<0.$

Strona 10 z18

$\mathrm{M}\mathrm{M}\mathrm{A}_{-}$



\end{document}