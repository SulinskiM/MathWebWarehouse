\documentclass[a4paper,12pt]{article}
\usepackage{latexsym}
\usepackage{amsmath}
\usepackage{amssymb}
\usepackage{graphicx}
\usepackage{wrapfig}
\pagestyle{plain}
\usepackage{fancybox}
\usepackage{bm}

\begin{document}

$\mathrm{g}_{\mathrm{E}\mathrm{G}\mathrm{Z}\mathrm{A}\mathrm{M}\mathrm{I}\mathrm{N}\mathrm{A}\subset \mathrm{Y}\mathrm{J}\mathrm{N}\mathrm{A}}^{\mathrm{C}\mathrm{E}\mathrm{N}\mathrm{T}\mathrm{R}\mathrm{A}\mathrm{L}\mathrm{N}\mathrm{A}}\mathrm{K}\mathrm{O}\mathrm{M}1\mathrm{S}\mathrm{J}\mathrm{A}$

Arkusz zawiera informacje

prawnie chronione do momentu

rozpoczęcia egzaminu.

UZUPELNIA ZDAJACY

{\it miejsce}

{\it na naklejkę}
\begin{center}
\includegraphics[width=21.900mm,height=16.056mm]{./F2_M_PR_M2017_page0_images/image001.eps}
\end{center}
KOD
\begin{center}
\includegraphics[width=79.656mm,height=16.104mm]{./F2_M_PR_M2017_page0_images/image002.eps}
\end{center}
PESEL
\begin{center}
\includegraphics[width=195.984mm,height=236.676mm]{./F2_M_PR_M2017_page0_images/image003.eps}
\end{center}
EGZAMIN MATU  LNY

Z MATEMATY

POZIOM ROZSZE  ONY

DATA: 9 maja 2017 $\mathrm{r}.$

CZAS P CY: $ 18\Uparrow$ minut

LICZBA P KTÓW DO UZYS NIA: 50

Instrukcja dla zdającego

l. Sprawdzí, czy arkusz egzaminacyjny zawiera 18 stron (zadania $1-15$).

Ewentualny brak zgłoś przewodniczącemu zespo nadzorującego

egzamin.

2. Rozwiązania zadań i odpowiedzi wpisuj w miejscu na to przeznaczonym.

3. Odpowiedzi do zadań za ię ch (l ) zaznacz na karcie odpowiedzi

w części ka $\mathrm{y}$ przeznaczonej dla zdającego. Zamaluj $\blacksquare$ pola do tego

przeznaczone. Błędne zaznaczenie otocz kólkiem \copyright i zaznacz wlaściwe.

4. $\mathrm{W}$ zadaniu 5. wpisz odpowiednie cyf w atki pod treścią zadania.

5. Pamiętaj, $\dot{\mathrm{z}}\mathrm{e}$ pominięcie argumentacji lub istotnych obliczeń

w rozwiązaniu zadania otwa ego (6-15) $\mathrm{m}\mathrm{o}\dot{\mathrm{z}}\mathrm{e}$ spowodować, $\dot{\mathrm{z}}\mathrm{e}$ za to

rozwiązanie nie otrzymasz pelnej liczby pu tów.

6. Pisz cz elnie i $\mathrm{u}\dot{\mathrm{z}}$ aj lko $\mathrm{d}$ gopisu lub pióra z czamym tuszem lub

atramentem.

7. Nie $\mathrm{u}\dot{\mathrm{z}}$ aj korektora, a błędne zapisy razínie prze eśl.

8. Pamiętaj, $\dot{\mathrm{z}}\mathrm{e}$ zapisy w brudnopisie nie będą oceniane.

9. $\mathrm{M}\mathrm{o}\dot{\mathrm{z}}$ esz korzystać z zesta wzorów matematycznych, cyrkla i linijki oraz

kal latora prostego.

10. Na tej stronie oraz na karcie odpowiedzi wpisz swój numer PESEL

i przyklej naklejkę z kodem.

ll. Nie wpisuj $\dot{\mathrm{z}}$ adnych znaków w części przeznaczonej dla egzaminatora.

$\Vert\Vert\Vert\Vert\Vert\Vert\Vert\Vert\Vert\Vert\Vert\Vert\Vert\Vert\Vert\Vert\Vert\Vert\Vert\Vert\Vert\Vert\Vert\Vert|$

$\mathrm{M}\mathrm{M}\mathrm{A}-\mathrm{R}1_{-}1\mathrm{P}-172$

Układ graficzny

\copyright CKE 2015
\end{document}
