\documentclass[a4paper,12pt]{article}
\usepackage{latexsym}
\usepackage{amsmath}
\usepackage{amssymb}
\usepackage{graphicx}
\usepackage{wrapfig}
\pagestyle{plain}
\usepackage{fancybox}
\usepackage{bm}

\begin{document}

{\it Wzadaniach od l. do 4. wybierz i zaznacz na karcie odpowiedzi poprawnq odpowiedzí}.

Zadaoie $l.(0-1)$

Liczba $(\sqrt{2-\sqrt{3}}-\sqrt{2+\sqrt{3}})^{2}$ jest równa

A. 2

B. 4

C. $\sqrt{3}$

D.

$2\sqrt{3}$

Zadanie 2. $(0-l)$

Nieskończony ciąg

Wtedy

liczbowy jest określony wzorem

{\it an}$=$ -({\it n}22-{\it n}130$+${\it nn}) (22$+$-33{\it n})

dla $n\geq 1.$

A.

$\displaystyle \lim_{n\rightarrow\infty}a_{n}=\frac{1}{2}$

B.

$\displaystyle \lim_{n\rightarrow\infty}a_{n}=0$

C.

$\displaystyle \lim_{n\rightarrow\infty}a_{n}=-\infty$

D.

$\displaystyle \lim_{n\rightarrow\infty}a_{n}=-\frac{3}{2}$

Zadanie 3. $(0-l)$

Odcinek $CD$ jest wysokością trójkąta $ABC$, w którym $|AD|=|CD|=\displaystyle \frac{1}{2}|BC|$ (zobacz rysunek).

Okrąg o środku $C$ i promieniu $CD$ jest styczny do prostej $AB$. Okrąg ten przecina boki

$AC\mathrm{i}BC$ trójkąta odpowiednio w punktach $K\mathrm{i}L.$
\begin{center}
\includegraphics[width=63.804mm,height=47.652mm]{./F2_M_PR_M2017_page1_images/image001.eps}
\end{center}
{\it M}

$\alpha$

{\it C}

{\it L}

{\it K}

{\it A  D  B}

Zaznaczony na rysunku kąt $\alpha$ wpisany w okrągjest równy

A. $37,5^{\mathrm{o}}$

B. $45^{\mathrm{o}}$

C. 52, $5^{\mathrm{o}}$

D. $60^{\mathrm{o}}$

Zadanie 4. (0-1)

Dane są punkt $B=(-4,7)$ i wektor $\vec{u}=[-3,5]$. Punkt $A$, taki, $\dot{\mathrm{z}}\mathrm{e}\vec{AB}=-3\vec{u}$, ma współrzędne

A. $A=(5,-8)$

B. $A=(-13,22)$

C. $A=(9,-15)$

D. $A=(12,24)$

Strona 2 z18

MMA-IR
\end{document}
