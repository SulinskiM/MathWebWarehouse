\documentclass[a4paper,12pt]{article}
\usepackage{latexsym}
\usepackage{amsmath}
\usepackage{amssymb}
\usepackage{graphicx}
\usepackage{wrapfig}
\pagestyle{plain}
\usepackage{fancybox}
\usepackage{bm}

\begin{document}

Zadanie 5. (0-2)

Reszta z dzielenia wielomianu $W(x)=x^{3}-2x^{2}+ax+\displaystyle \frac{3}{4}$ przez dwumian $x-2$ jest równa l.

Oblicz wartość współczynnika $a.$

$\mathrm{W}$ ponizsze kratki wpisz kolejno trzy pierwsze cyfry po przecinku rozwinięcia dziesiętnego

otrzymanego wyniku.
\begin{center}
\includegraphics[width=22.500mm,height=10.920mm]{./F2_M_PR_M2017_page3_images/image001.eps}
\end{center}
{\it BRUDNOPIS} ({\it nie podlega ocenie})

Zadanie 6. (0-3)

Funkcja $f$ jest określona wzorem $f(x)=\displaystyle \frac{x-1}{x^{2}+1}$ dla $\mathrm{k}\mathrm{a}\dot{\mathrm{z}}$ dej liczby rzeczywistej $x$. Wyznacz

równanie stycznej do wykresu tej funkcji w punkcie $P=(1,0).$

Odpowiedzí:

Strona 4 z18

MMA-IR
\end{document}
