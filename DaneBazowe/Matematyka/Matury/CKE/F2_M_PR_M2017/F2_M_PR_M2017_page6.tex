\documentclass[a4paper,12pt]{article}
\usepackage{latexsym}
\usepackage{amsmath}
\usepackage{amssymb}
\usepackage{graphicx}
\usepackage{wrapfig}
\pagestyle{plain}
\usepackage{fancybox}
\usepackage{bm}

\begin{document}

Zadanie 9. (0-4)

$\mathrm{W}$ czworościanie, którego wszystkie krawędzie mają taką samą długość 6, umieszczono ku1ę

tak, $\dot{\mathrm{z}}\mathrm{e}$ ma ona dokładniejeden punkt wspólny z $\mathrm{k}\mathrm{a}\dot{\mathrm{z}}$ dą ścianą czworościanu. Płaszczyzna $\pi,$

równoległa do podstawy tego czworościanu, dzieli go na dwie bryły: ostrosłup o objętości

równej $\displaystyle \frac{8}{27}$ objętości dzielonego czworościanu i ostrosłup ścięty. Oblicz odległość środka $S$

kuli od płaszczyzny $\pi$, tj. długość najkrótszego spośród odcinków $SP$, gdzie Pjest punktem

płaszczyzny $\pi.$

Odpowiedzí :
\begin{center}
\includegraphics[width=96.012mm,height=17.832mm]{./F2_M_PR_M2017_page6_images/image001.eps}
\end{center}
Wypelnia

egzaminator

Nr zadania

Maks. liczba kt

8.

3

4

Uzyskana liczba pkt

IMA-IR

Strona 7 z18
\end{document}
