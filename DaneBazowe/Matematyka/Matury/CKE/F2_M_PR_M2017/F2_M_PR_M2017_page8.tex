\documentclass[a4paper,12pt]{article}
\usepackage{latexsym}
\usepackage{amsmath}
\usepackage{amssymb}
\usepackage{graphicx}
\usepackage{wrapfig}
\pagestyle{plain}
\usepackage{fancybox}
\usepackage{bm}

\begin{document}

Zadanie 11. (0-4)

$\mathrm{W}$ pudełku znajduje się 8 piłeczek oznaczonych ko1ejnymi 1iczbami natura1nymi od 1 do 8.

Losujemy jedną piłeczkę, zapisujemy liczbę na niej występującą, a następnie zwracamy

piłeczkę do umy. Tę procedurę wykonujemy jeszcze dwa razy i tym samym otrzymujemy

zapisane trzy liczby. Oblicz prawdopodobieństwo wylosowania takich piłeczek, $\dot{\mathrm{z}}\mathrm{e}$ iloczyn

trzech zapisanych liczb jest podzielny przez 4. 1Vynik podaj w postaci ułamka zwykłego.

Odpowiedzí :
\begin{center}
\includegraphics[width=96.012mm,height=17.832mm]{./F2_M_PR_M2017_page8_images/image001.eps}
\end{center}
Wypelnia

egzaminator

Nr zadania

Maks. liczba kt

10.

4

11.

4

Uzyskana liczba pkt

IMA-IR

Strona 9 z18
\end{document}
