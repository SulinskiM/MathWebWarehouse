\documentclass[a4paper,12pt]{article}
\usepackage{latexsym}
\usepackage{amsmath}
\usepackage{amssymb}
\usepackage{graphicx}
\usepackage{wrapfig}
\pagestyle{plain}
\usepackage{fancybox}
\usepackage{bm}

\begin{document}

$\mathrm{g}_{\mathrm{E}\mathrm{G}\mathrm{Z}\mathrm{A}\mathrm{M}\mathrm{I}\mathrm{N}\mathrm{A}\mathrm{C}\mathrm{Y}\mathrm{J}\mathrm{N}\mathrm{A}}^{\mathrm{C}\mathrm{E}\mathrm{N}\mathrm{T}\mathrm{R}\mathrm{A}\mathrm{L}\mathrm{N}\mathrm{A}}$KOMISJA

Arkusz zawiera informacje

prawnie chronione do momentu

rozpoczęcia egzaminu.

UZUPELNIA ZDAJACY

{\it miejsce}

{\it na naklejkę}
\begin{center}
\includegraphics[width=21.900mm,height=16.104mm]{./F2_M_PR_M2018_page0_images/image001.eps}
\end{center}
KOD
\begin{center}
\includegraphics[width=79.608mm,height=16.104mm]{./F2_M_PR_M2018_page0_images/image002.eps}
\end{center}
PESEL
\begin{center}
\includegraphics[width=193.752mm,height=268.884mm]{./F2_M_PR_M2018_page0_images/image003.eps}
\end{center}
EGZAMIN MATU  LNY

Z MATEMATY

POZIOM ROZSZE ONY

DATA: 9 maja 2018 r.

LICZBA P KTÓW DO UZYS NIA: 50

Instrukcja dla zdającego

1.

2.

3.

4.

5.

6.

Sprawdzí, czy arkusz egzaminacyjny zawiera 18 stron (zadania $1-15$).

Ewentualny brak zgłoś przewodniczącemu zespo nadzorującego

egzamin.

Rozwiązania zadań i odpowiedzi wpisuj w miejscu na to przeznaczonym.

Odpowiedzi do zadań za iętych (l ) zaznacz na karcie odpowiedzi

w części ka przeznaczonej dla zdającego. Zamaluj $\blacksquare$ pola do tego

przeznaczone. Błędne zaznaczenie otocz kółkiem \copyright i zaznacz właściwe.

$\mathrm{W}$ zadaniu 5. wpisz odpowiednie cyf w atki pod treścią zadania.

Pamiętaj, $\dot{\mathrm{z}}\mathrm{e}$ pominięcie argumentacji lub istotnych obliczeń

w rozwiązaniu zadania otwa ego (6-15) $\mathrm{m}\mathrm{o}\dot{\mathrm{z}}\mathrm{e}$ spowodować, $\dot{\mathrm{z}}\mathrm{e}$ za to

rozwiązanie nie otrzymasz pełnej liczby pu tów.

Pisz czytelnie i $\mathrm{u}\dot{\mathrm{z}}$ aj tylko $\mathrm{d}$ gopisu lub pióra z czatnym tuszem lub

atramentem.

7. Nie $\mathrm{u}\dot{\mathrm{z}}$ aj korektora, a błędne zapisy $\mathrm{r}\mathrm{a}\acute{\mathrm{z}}\mathrm{n}\mathrm{i}\mathrm{e}$ prze eśl.

8. Pamiętaj, $\dot{\mathrm{z}}\mathrm{e}$ zapisy w $\mathrm{b}$ dnopisie nie będą oceniane.

9. $\mathrm{M}\mathrm{o}\dot{\mathrm{z}}$ esz korzystać z zesta wzorów matema cznych, cyrkla i linijki oraz

kalkulatora prostego.

10. Na tej stronie oraz na karcie odpowiedzi wpisz swój numer PESEL

i przyklej naklejkę z kodem.

ll. Nie wpisuj $\dot{\mathrm{z}}$ adnych znaków w części przeznaczonej dla egzaminatora.

$\Vert\Vert\Vert\Vert\Vert\Vert\Vert\Vert\Vert\Vert\Vert\Vert\Vert\Vert\Vert\Vert\Vert\Vert\Vert\Vert\Vert\Vert\Vert\Vert|$

$\mathrm{M}\mathrm{M}\mathrm{A}-\mathrm{R}1_{-}1\mathrm{P}-1\mathrm{S}2$

Układ graficzny

\copyright CKE 2015

$1$ :




{\it Wzadaniach od l. do 4. wybierz i zaznacz na karcie odpowiedzi poprawnq odpowiedzí}.

Zadanie $l. (0-1)$

Dane są liczby: $a=\displaystyle \frac{4\sqrt{8}}{2}, b=\displaystyle \frac{1}{2^{4}\sqrt{8}}, c=^{4}\sqrt{8}, d=\displaystyle \frac{2}{4\sqrt{8}}$ oraz $k=2^{-\frac{1}{4}}$. Prawdziwajest równość

A. $k=a$

B. $k=b$

C. $k=c$

D. $k=d$

Zadanie 2. (0-1)

Równanie $||x|-2|=|x|+2$

A. nie ma rozwiązań.

B. ma dokładniejedno rozwiązanie.

C. ma dokładnie dwa rozwiązania.

D. ma dokładnie cztery rozwiązania.

Zadanie 3. $(0-1\rangle$

Wartość wyrazenia 21og510- $\displaystyle \frac{1}{\log_{20}5}$ jest równa

A. $-1$

B. 0

Zadanie 4. (0-1)

Granica $\displaystyle \lim_{x\rightarrow 3^{-}}\frac{-x+2}{x^{2}-5x+6}$ jest równa

A.

$-\infty$

B. $-1$

C. l

D. 2

C. 0

D. $+\infty$

Strona 2 z18

MMA-IR





Odpowied $\acute{\mathrm{z}}$:
\begin{center}
\includegraphics[width=82.044mm,height=17.784mm]{./F2_M_PR_M2018_page10_images/image001.eps}
\end{center}
Wypelnia

egzamÍnator

Nr zadania

Maks. liczba kt

12.

Uzyskana liczba pkt

MMA-IR

Strona ll z18





Zadanie 13. (0-4)

Wyrazy ciągu geometrycznego $(a_{n})$, określonego dla $n\geq 1$, spełniają układ równań

$\left\{\begin{array}{l}
a_{3}+a_{6}=-84\\
a_{4}+a_{7}=168
\end{array}\right.$

Wyznacz liczbę $n$ początkowych wyrazów tego ciągu, których suma $S_{n}$ jest równa 32769.

Strona 12 z18

MMA-IR





Odpowied $\acute{\mathrm{z}}$:
\begin{center}
\includegraphics[width=82.044mm,height=17.784mm]{./F2_M_PR_M2018_page12_images/image001.eps}
\end{center}
Wypelnia

egzamÍnator

Nr zadania

Maks. liczba kt

13.

4

Uzyskana liczba pkt

MMA-IR

Strona 13 z18





Zadanie 14. $(0-6)$

Punkt $A=(7,-1)$ jest wierzchołkiem trójkąta równoramiennego $ABC$, w którym $|AC|=|BC|.$

Obie współrzędne wierzchołka $C$ są liczbami ujemnymi. Okrąg wpisany w trójkąt $ABC$ ma

równanie $x^{2}+y^{2}=10$. Oblicz współrzędne wierzchołków $B\mathrm{i}C$ tego trójkąta.

Strona 14 z18

MMA-IR





Odpowied $\acute{\mathrm{z}}$:
\begin{center}
\includegraphics[width=82.044mm,height=17.832mm]{./F2_M_PR_M2018_page14_images/image001.eps}
\end{center}
Wypelnia

egzaminator

Nr zadania

Maks. liczba kt

14.

Uzyskana liczba pkt

MMA-IR

Strona 15 z18





Zadanie $l5.(0\rightarrow 7)$

Rozpatrujemy wszystkie trapezy równoramienne, w które mozna wpisać okrąg, spełniające

warunek: suma długości dłuzszej podstawy $a$ i wysokości trapezujest równa 2.

a) Wyznacz wszystkie wartości $a$, dla których istnieje trapez o podanych własnościach.

b) Wykaz, $\dot{\mathrm{z}}\mathrm{e}$ obwód $L$ takiego trapezu, jako funkcja długości $a$ dłuzszej podstawy trapezu,

wyraza się wzorem $L(a)=\displaystyle \frac{4a^{2}-8a+8}{a}$

c)

Oblicz tangens kąta ostrego tego spośród rozpatrywanych trapezów, którego obwódjest

najmniejszy.

Strona 16 z18

MMA-IR





Odpowied $\acute{\mathrm{z}}$:
\begin{center}
\includegraphics[width=82.044mm,height=17.784mm]{./F2_M_PR_M2018_page16_images/image001.eps}
\end{center}
Wypelnia

egzamÍnator

Nr zadania

Maks. liczba kt

15.

7

Uzyskana liczba pkt

MMA-IR

Strona 17 z18





{\it BRUDNOPIS} ({\it nie podlega ocenie})

Strona 18 z18

MMA-II





{\it BRUDNOPIS} ({\it nie podlega ocenie})

$\mathrm{A}_{-}1\mathrm{R}$

Strona 3 z18





Zadanie 5. $(0-2\rangle$

Punkt $A=(-5,3)$ jest środkiem symetrii wykresu ffinkcji homograficznej określonej wzorem

$f(x)=\displaystyle \frac{ax+7}{x+d}$, gdy $x\neq-d$. Oblicz iloraz $\displaystyle \frac{d}{a}$

$\mathrm{W}$ ponizsze kratki wpisz kolejno cyfrę jedności i pierwsze dwie cyfry po przecinku

nieskończonego rozwinięcia dziesiętnego otrzymanego wyniku.
\begin{center}
\includegraphics[width=22.500mm,height=10.872mm]{./F2_M_PR_M2018_page3_images/image001.eps}
\end{center}
{\it BRUDNOPIS} ({\it nie podlega ocenie})

Zadanie 6. $(0-3\rangle$

Styczna do paraboli o równaniu $y=\sqrt{3}x^{2}-1$ w punkcie $P=(x_{0},y_{0})$ jest nachylona do osi $ox$

pod kątem $30^{\mathrm{o}}$. Oblicz współrzędne punktu $P.$

Odpowied $\acute{\mathrm{z}}$:

Strona 4 z18

MMA-IR





Zadanie 7. (0-3)

Trójkąt $ABC$ jest ostrokątny oraz $|AC|>|BC|$. Dwusieczna $d_{c}$ kąta $ACB$ przecina bok $AB$

w punkcie $K$. Punkt $L$ jest obrazem punktu $K$ w symetrii osiowej względem dwusiecznej $d_{A}$

kąta $BAC$, punkt Mjest obrazem punktu $L$ w symetrii osiowej względem dwusiecznej $d_{c}$ kąta

$ACB$, a punkt $N$ jest obrazem punktu $M$ w symetrii osiowej względem dwusiecznej $d_{B}$ kąta

$ABC$ (zobacz rysunek).
\begin{center}
\includegraphics[width=88.392mm,height=84.072mm]{./F2_M_PR_M2018_page4_images/image001.eps}
\end{center}
{\it C}

{\it L}

{\it M}

{\it A  K N  B}

Udowodnij, $\dot{\mathrm{z}}\mathrm{e}$ na czworokącie KNML mozna opisać okrąg.
\begin{center}
\includegraphics[width=109.980mm,height=17.832mm]{./F2_M_PR_M2018_page4_images/image002.eps}
\end{center}
Wypelnia

egzaminator

Nr zadania

Maks. lÍczba kt

5.

2

3

7.

3

Uzyskana liczba pkt

MMA-IR

Strona 5 z18





Zadanie S. (0-3)

Udowodnij, $\dot{\mathrm{z}}\mathrm{e}$ dla $\mathrm{k}\mathrm{a}\dot{\mathrm{z}}$ dej liczby całkowitej $k$ i dla $\mathrm{k}\mathrm{a}\dot{\mathrm{z}}$ dej liczby całkowitej $m$ liczba $k^{3}m-km^{3}$

jest podzielna przez 6.

Strona 6 z18

MMA-IR





Zadanie $g. (0-4)$

$\mathrm{Z}$ liczb ośmioelementowego zbioru $Z=\{1$, 2, 3, 4, 5, 6, 7, 9$\}$ tworzymy ośmiowyrazowy ciąg,

którego wyrazy się nie powtarzają. Oblicz prawdopodobieństwo zdarzenia polegającego na

tym, $\dot{\mathrm{z}}\mathrm{e}\dot{\mathrm{z}}$ adne dwie liczby parzyste nie są sąsiednimi wyrazami utworzonego ciągu. Wynik

przedstaw w postaci ułamka zwykłego nieskracalnego.

Odpowied $\acute{\mathrm{z}}$:
\begin{center}
\includegraphics[width=96.012mm,height=17.784mm]{./F2_M_PR_M2018_page6_images/image001.eps}
\end{center}
Wypelnia

egzaminator

Nr zadania

Maks. liczba kt

8.

3

4

Uzyskana liczba pkt

MMA-IR

Strona 7 z18





Zadanie 10.(0-4)

Objętość stozka ściętego (przedstawionego na rysunku) mozna obliczyć ze wzoru

$V=\displaystyle \frac{1}{3}\pi H(r^{2}+rR+R^{2})$, gdzie $r\mathrm{i}R$ są promieniami podstaw $(r<R)$, a $H$ jest wysokością

bryły. Danyjest stozek ścięty, którego wysokośćjest równa 10, objętość $ 840\pi$, a $r=6$. Oblicz

cosinus kąta nachylenia przekątnej przekroju osiowego tej bryły do jednej zjej podstaw.

Odpowiedzí :

Strona 8 z18

MMA-IR





Zadanie $1l. (0-4)$

Rozwiąz równanie $\sin 6x+\cos 3x=2\sin 3x+1$ w przedziale $\langle 0, \pi\rangle.$

Odpowiedzí :
\begin{center}
\includegraphics[width=96.012mm,height=17.832mm]{./F2_M_PR_M2018_page8_images/image001.eps}
\end{center}
Wypelnia

egzaminator

Nr zadania

Maks. liczba kt

10.

4

11.

4

Uzyskana liczba pkt

MMA-IR

Strona 9 z18





Zadanie 12. $(0-6)$

Wyznacz wszystkie wartości parametru $m$, dla których równanie $x^{2}+(m+1)x-m^{2}+1=0$ ma

dwa rozwiązania rzeczywiste $x_{1}$ i $x_{2}(x_{1}\neq x_{2})$, spełniające warunek $x_{1}^{3}+x_{2}^{3}>-7x_{1}x_{2}.$

Strona 10 z18

MMA-IR



\end{document}