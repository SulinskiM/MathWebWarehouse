\documentclass[a4paper,12pt]{article}
\usepackage{latexsym}
\usepackage{amsmath}
\usepackage{amssymb}
\usepackage{graphicx}
\usepackage{wrapfig}
\pagestyle{plain}
\usepackage{fancybox}
\usepackage{bm}

\begin{document}

Zadanie $l5.(0\rightarrow 7)$

Rozpatrujemy wszystkie trapezy równoramienne, w które mozna wpisać okrąg, spełniające

warunek: suma długości dłuzszej podstawy $a$ i wysokości trapezujest równa 2.

a) Wyznacz wszystkie wartości $a$, dla których istnieje trapez o podanych własnościach.

b) Wykaz, $\dot{\mathrm{z}}\mathrm{e}$ obwód $L$ takiego trapezu, jako funkcja długości $a$ dłuzszej podstawy trapezu,

wyraza się wzorem $L(a)=\displaystyle \frac{4a^{2}-8a+8}{a}$

c)

Oblicz tangens kąta ostrego tego spośród rozpatrywanych trapezów, którego obwódjest

najmniejszy.

Strona 16 z18

MMA-IR
\end{document}
