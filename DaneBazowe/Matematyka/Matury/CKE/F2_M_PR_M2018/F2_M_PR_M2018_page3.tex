\documentclass[a4paper,12pt]{article}
\usepackage{latexsym}
\usepackage{amsmath}
\usepackage{amssymb}
\usepackage{graphicx}
\usepackage{wrapfig}
\pagestyle{plain}
\usepackage{fancybox}
\usepackage{bm}

\begin{document}

Zadanie 5. $(0-2\rangle$

Punkt $A=(-5,3)$ jest środkiem symetrii wykresu ffinkcji homograficznej określonej wzorem

$f(x)=\displaystyle \frac{ax+7}{x+d}$, gdy $x\neq-d$. Oblicz iloraz $\displaystyle \frac{d}{a}$

$\mathrm{W}$ ponizsze kratki wpisz kolejno cyfrę jedności i pierwsze dwie cyfry po przecinku

nieskończonego rozwinięcia dziesiętnego otrzymanego wyniku.
\begin{center}
\includegraphics[width=22.500mm,height=10.872mm]{./F2_M_PR_M2018_page3_images/image001.eps}
\end{center}
{\it BRUDNOPIS} ({\it nie podlega ocenie})

Zadanie 6. $(0-3\rangle$

Styczna do paraboli o równaniu $y=\sqrt{3}x^{2}-1$ w punkcie $P=(x_{0},y_{0})$ jest nachylona do osi $ox$

pod kątem $30^{\mathrm{o}}$. Oblicz współrzędne punktu $P.$

Odpowied $\acute{\mathrm{z}}$:

Strona 4 z18

MMA-IR
\end{document}
