\documentclass[a4paper,12pt]{article}
\usepackage{latexsym}
\usepackage{amsmath}
\usepackage{amssymb}
\usepackage{graphicx}
\usepackage{wrapfig}
\pagestyle{plain}
\usepackage{fancybox}
\usepackage{bm}

\begin{document}

Zadanie 7. (0-3)

Trójkąt $ABC$ jest ostrokątny oraz $|AC|>|BC|$. Dwusieczna $d_{c}$ kąta $ACB$ przecina bok $AB$

w punkcie $K$. Punkt $L$ jest obrazem punktu $K$ w symetrii osiowej względem dwusiecznej $d_{A}$

kąta $BAC$, punkt Mjest obrazem punktu $L$ w symetrii osiowej względem dwusiecznej $d_{c}$ kąta

$ACB$, a punkt $N$ jest obrazem punktu $M$ w symetrii osiowej względem dwusiecznej $d_{B}$ kąta

$ABC$ (zobacz rysunek).
\begin{center}
\includegraphics[width=88.392mm,height=84.072mm]{./F2_M_PR_M2018_page4_images/image001.eps}
\end{center}
{\it C}

{\it L}

{\it M}

{\it A  K N  B}

Udowodnij, $\dot{\mathrm{z}}\mathrm{e}$ na czworokącie KNML mozna opisać okrąg.
\begin{center}
\includegraphics[width=109.980mm,height=17.832mm]{./F2_M_PR_M2018_page4_images/image002.eps}
\end{center}
Wypelnia

egzaminator

Nr zadania

Maks. lÍczba kt

5.

2

3

7.

3

Uzyskana liczba pkt

MMA-IR

Strona 5 z18
\end{document}
