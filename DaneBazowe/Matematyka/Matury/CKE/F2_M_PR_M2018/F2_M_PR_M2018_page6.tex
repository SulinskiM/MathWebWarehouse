\documentclass[a4paper,12pt]{article}
\usepackage{latexsym}
\usepackage{amsmath}
\usepackage{amssymb}
\usepackage{graphicx}
\usepackage{wrapfig}
\pagestyle{plain}
\usepackage{fancybox}
\usepackage{bm}

\begin{document}

Zadanie $g. (0-4)$

$\mathrm{Z}$ liczb ośmioelementowego zbioru $Z=\{1$, 2, 3, 4, 5, 6, 7, 9$\}$ tworzymy ośmiowyrazowy ciąg,

którego wyrazy się nie powtarzają. Oblicz prawdopodobieństwo zdarzenia polegającego na

tym, $\dot{\mathrm{z}}\mathrm{e}\dot{\mathrm{z}}$ adne dwie liczby parzyste nie są sąsiednimi wyrazami utworzonego ciągu. Wynik

przedstaw w postaci ułamka zwykłego nieskracalnego.

Odpowied $\acute{\mathrm{z}}$:
\begin{center}
\includegraphics[width=96.012mm,height=17.784mm]{./F2_M_PR_M2018_page6_images/image001.eps}
\end{center}
Wypelnia

egzaminator

Nr zadania

Maks. liczba kt

8.

3

4

Uzyskana liczba pkt

MMA-IR

Strona 7 z18
\end{document}
