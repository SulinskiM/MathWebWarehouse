\documentclass[a4paper,12pt]{article}
\usepackage{latexsym}
\usepackage{amsmath}
\usepackage{amssymb}
\usepackage{graphicx}
\usepackage{wrapfig}
\pagestyle{plain}
\usepackage{fancybox}
\usepackage{bm}

\begin{document}

Zadanie 10.(0-4)

Objętość stozka ściętego (przedstawionego na rysunku) mozna obliczyć ze wzoru

$V=\displaystyle \frac{1}{3}\pi H(r^{2}+rR+R^{2})$, gdzie $r\mathrm{i}R$ są promieniami podstaw $(r<R)$, a $H$ jest wysokością

bryły. Danyjest stozek ścięty, którego wysokośćjest równa 10, objętość $ 840\pi$, a $r=6$. Oblicz

cosinus kąta nachylenia przekątnej przekroju osiowego tej bryły do jednej zjej podstaw.

Odpowiedzí :

Strona 8 z18

MMA-IR
\end{document}
