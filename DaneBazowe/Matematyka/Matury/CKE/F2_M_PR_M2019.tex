\documentclass[a4paper,12pt]{article}
\usepackage{latexsym}
\usepackage{amsmath}
\usepackage{amssymb}
\usepackage{graphicx}
\usepackage{wrapfig}
\pagestyle{plain}
\usepackage{fancybox}
\usepackage{bm}

\begin{document}

$\mathrm{g}_{\mathrm{E}\mathrm{G}\mathrm{Z}\mathrm{A}\mathrm{M}\mathrm{I}\mathrm{N}\mathrm{A}\mathrm{C}\mathrm{Y}\mathrm{J}\mathrm{N}\mathrm{A}}^{\mathrm{C}\mathrm{E}\mathrm{N}\mathrm{T}\mathrm{R}\mathrm{A}\mathrm{L}\mathrm{N}\mathrm{A}}$KOMISJA

Arkusz zawiera informacje

prawnie chronione do momentu

rozpoczęcia egzaminu.

UZUPELNIA ZDAJACY

{\it miejsce}

{\it na naklejkę}
\begin{center}
\includegraphics[width=21.900mm,height=16.104mm]{./F2_M_PR_M2019_page0_images/image001.eps}
\end{center}
KOD
\begin{center}
\includegraphics[width=79.608mm,height=16.104mm]{./F2_M_PR_M2019_page0_images/image002.eps}
\end{center}
PESEL
\begin{center}
\includegraphics[width=196.440mm,height=246.984mm]{./F2_M_PR_M2019_page0_images/image003.eps}
\end{center}
EGZAMIN MATU  LNY

Z MATEMATY

POZIOM ROZSZE ONY

DATA: 9 maja 2019 r.

LICZBA P KTÓW DO UZYS NIA: 50

Instrukcja dla zdającego

1.

2.

3.

4.

5.

6.

Sprawdzí, czy arkusz egzaminacyjny zawiera 22 strony (zadania $1-15$).

Ewentualny brak zgłoś przewodniczącemu zespo nadzorującego

egzamin.

Rozwiązania zadań i odpowiedzi wpisuj w miejscu na to przeznaczonym.

Odpowiedzi do zadań za iętych (l ) zaznacz na karcie odpowiedzi

w części ka przeznaczonej dla zdającego. Zamaluj $\blacksquare$ pola do tego

przeznaczone. Błędne zaznaczenie otocz kółkiem \copyright i zaznacz właściwe.

$\mathrm{W}$ zadaniu 5. wpisz odpowiednie cyf w atki pod treścią zadania.

Pamiętaj, $\dot{\mathrm{z}}\mathrm{e}$ pominięcie argumentacji lub istotnych obliczeń

w rozwiązaniu zadania otwa ego (6-15) $\mathrm{m}\mathrm{o}\dot{\mathrm{z}}\mathrm{e}$ spowodować, $\dot{\mathrm{z}}\mathrm{e}$ za to

rozwiązanie nie otrzymasz pełnej liczby pu tów.

Pisz czytelnie i $\mathrm{u}\dot{\mathrm{z}}$ aj tylko $\mathrm{d}$ gopisu lub pióra z czatnym tuszem lub

atramentem.

7. Nie $\mathrm{u}\dot{\mathrm{z}}$ aj korektora, a błędne zapisy $\mathrm{r}\mathrm{a}\acute{\mathrm{z}}\mathrm{n}\mathrm{i}\mathrm{e}$ prze eśl.

8. Pamiętaj, $\dot{\mathrm{z}}\mathrm{e}$ zapisy w brudnopisie nie będą oceniane.

9. $\mathrm{M}\mathrm{o}\dot{\mathrm{z}}$ esz korzystać z zesta wzorów matema cznych, cyrkla i linijki oraz

kalkulatora prostego.

10. Na tej stronie oraz na karcie odpowiedzi wpisz swój numer PESEL

i przyklej naklejkę z kodem.

ll. Nie wpisuj $\dot{\mathrm{z}}$ adnych znaków w części przeznaczonej dla egzaminatora.

$\Vert\Vert\Vert\Vert\Vert\Vert\Vert\Vert\Vert\Vert\Vert\Vert\Vert\Vert\Vert\Vert\Vert\Vert\Vert\Vert\Vert\Vert\Vert\Vert|$

$\mathrm{M}\mathrm{M}\mathrm{A}-\mathrm{R}1_{-}1\mathrm{P}-192$

Układ graficzny

\copyright CKE 2015




{\it Wkazdym z zadań od l. do 4. wybierz i zaznacz na karcie odpowiedzi poprawnq odpowiedzí}.

Zadanie 1. (0-1)

Dla dowolnych liczb $x>0, x\neq 1, y>0, y\neq 1$ wartość wyrazenia $(\log_{\frac{1}{x}}y)\cdot(\log_{y}\perp x)$ jest

równa

$\underline{1}$

A. $x\cdot y$ B. C. $-1$ D. l

$x\cdot y$

Zadanie 2. (0-1)

Liczba $\cos^{2}105^{\mathrm{o}}-\sin^{2}105^{\mathrm{o}}$ jest równa

A.

- -$\sqrt{}$23

B.

- -21

C.

-21

D.

-$\sqrt{}$23

Zadanie 3. (0-1)

Na rysunku przedstawiono fragment wykresu ffinkcji $y=f(x)$, który jest złozony z dwóch

półprostych AD $\mathrm{i}$ CE oraz dwóch odcinków AB $\mathrm{i} BC$, gdzie $A=(-1,0), B=(1,2),$

$C=(3,0), D=(-4,3), E=(6,3).$
\begin{center}
\includegraphics[width=91.284mm,height=62.940mm]{./F2_M_PR_M2019_page1_images/image001.eps}
\end{center}
{\it y}

5

{\it D}

4

3

2

$E_{1}$

{\it B}

{\it x}

$-5$ -$4  -3  -2A$ -$1$  0  1 2  $3C^{4}$  5 6  7

$-1$

Wzór funkcji $f$ to

A. $f(x)=|x+1|+|x-1|$

B. $f(x)=\Vert x-1|-2|$

C. $f(x)=\Vert x-1|+2|$

D. $f(x)=|x-1|+2$

Zadanie 4. (0-1)

Zdarzenia losowe $A \mathrm{i} B$ zawarte w $\Omega$

zdarzenia $B'$, przeciwnego do zdarzenia $B,$

warunkowe $P(A|B)=\displaystyle \frac{1}{5}$. Wynika stąd, $\dot{\mathrm{z}}\mathrm{e}$

A. $P(A\displaystyle \cap B)=\frac{1}{20}$ B. $P(A\displaystyle \cap B)=\frac{4}{15}$

są takie, ze prawdopodobieństwo $P(B')$

est równe $\displaystyle \frac{1}{4}$ Ponadto prawdopodobieństwo

C. $P(A\displaystyle \cap B)=\frac{3}{20}$ D. $P(A\displaystyle \cap B)=\frac{4}{5}$

Strona 2 z22

MMA-IR





Odpowiedzí :
\begin{center}
\includegraphics[width=82.044mm,height=17.784mm]{./F2_M_PR_M2019_page10_images/image001.eps}
\end{center}
Wypelnia

egzaminator

Nr zadania

Maks. liczba kt

10.

4

Uzyskana liczba pkt

MMA-IR

Strona ll z22





Zadanie 11. (0-6)

Dane są okręgi o równaniach $x^{2}+y^{2}-12x-8y+43=0 \mathrm{i} x^{2}+y^{2}-2ax+4y+a^{2}-77=0.$

Wyznacz wszystkie wartości parametru $a$, dla których te okręgi mają dokładnie jeden punkt

wspólny. Rozwaz wszystkie przypadki.

Strona 12 z22

MMA-IR





Odpowiedzí :
\begin{center}
\includegraphics[width=82.044mm,height=17.784mm]{./F2_M_PR_M2019_page12_images/image001.eps}
\end{center}
Wypelnia

egzaminator

Nr zadania

Maks. liczba kt

11.

Uzyskana liczba pkt

MMA-IR

Strona 13 z22





Zadanie 12. (0-6)

Trzywyrazowy ciąg $(a,b,c)$ o wyrazach dodatnich jest arytmetyczny, natomiast ciąg

$(\displaystyle \frac{1}{a},\frac{2}{3b},\frac{1}{2a+2b+c})$ jest geometryczny. Oblicz iloraz ciągu geometrycznego.

Strona 14 z22

MMA-IR





Odpowiedzí :
\begin{center}
\includegraphics[width=82.044mm,height=17.784mm]{./F2_M_PR_M2019_page14_images/image001.eps}
\end{center}
Wypelnia

egzaminator

Nr zadania

Maks. liczba kt

12.

Uzyskana liczba pkt

MMA-IR

Strona 15 z22





Zadanie 13. (0-6)

Wielomian określony

wzorem

$W(x)=2x^{3}+(m^{3}+2)x^{2}-11x-2(2m+1)$ jest podzielny

przez dwumian $(x-2)$ oraz przy dzieleniu przez dwumian $(x+1)$ daje resztę 6. Ob1icz $m$

i dla wyznaczonej wartości $m$ rozwiąz nierówność $W(x)\leq 0.$

Strona 16 z22

MMA-IR





Odpowied $\acute{\mathrm{z}}$:
\begin{center}
\includegraphics[width=82.044mm,height=17.832mm]{./F2_M_PR_M2019_page16_images/image001.eps}
\end{center}
Wypelnia

egzaminator

Nr zadania

Maks. liczba kt

13.

Uzyskana liczba pkt

MMA-IR

Strona 17 z22





Zadanie 14. (0-4)

Rozwiąz równanie $(\displaystyle \cos x)[\sin(x-\frac{\pi}{3})+\sin(x+\frac{\pi}{3})]=\frac{1}{2}\sin x.$

Strona 18 z22

$\mathrm{M}\mathrm{M}\mathrm{A}_{-}1l$





Odpowiedzí :
\begin{center}
\includegraphics[width=82.044mm,height=17.784mm]{./F2_M_PR_M2019_page18_images/image001.eps}
\end{center}
Wypelnia

egzaminator

Nr zadania

Maks. liczba kt

14.

4

Uzyskana liczba pkt

MMA-IR

Strona 19 z22





Zadanie 15. (0-7)

Rozwazmy wszystkie graniastosłupy prawidłowe trójkątne o objętości $V=2$. Wyznacz

długości krawędzi tego z rozwazanych graniastosłupów, którego pole powierzchni całkowitej

jest najmniejsze. Oblicz to najmniejsze pole.

Strona 20 z22

MMA-IR





BRUDNOPIS

1R

Strona 3 z22





Odpowiedzí :
\begin{center}
\includegraphics[width=82.044mm,height=17.784mm]{./F2_M_PR_M2019_page20_images/image001.eps}
\end{center}
Wypelnia

egzaminator

Nr zadania

Maks. liczba kt

15.

7

Uzyskana liczba pkt

MMA-IR

Strona 21 z22





{\it BRUDNOPIS} ({\it nie podlega ocenie})

Strona 22 z22

$\mathrm{M}\mathrm{M}\mathrm{A}_{-}1l$





Zadanie 5. (0-2)

Oblicz granicę

$\displaystyle \lim_{n\rightarrow\infty}(\frac{9n^{3}+11n^{2}}{7n^{3}+5n^{2}+3n+1}-\frac{n^{2}}{3n^{2}+1})$

Wpisz w ponizsze kratki-od lewej do prawej- trzy kolejne cyfry po przecinku rozwinięcia

dziesiętnego otrzymanego wyniku.
\begin{center}
\includegraphics[width=22.500mm,height=10.872mm]{./F2_M_PR_M2019_page3_images/image001.eps}
\end{center}
Strona 4 z22

MMA-IR





Zadanie 6. (0-3)

Rozwazamy wszystkie liczby naturalne pięciocyfrowe zapisane przy uzyciu cyfr 1, 3, 5, 7, 9,

bez powtarzaniajakiejkolwiek cyfry. Oblicz sumę wszystkich takich liczb.

Odpowiedzí:
\begin{center}
\includegraphics[width=96.012mm,height=17.784mm]{./F2_M_PR_M2019_page4_images/image001.eps}
\end{center}
Wypelnia

egzamÍnator

Nr zadania

Maks. liczba kt

5.

2

3

Uzyskana liczba pkt

MMA-IR

Strona 5 z22





Zadanie 7. (0-2)

Punkt $P=(10,$ 2429$)$ lezy na paraboli o równaniu $y=2x^{2}+x+2219$. Prosta o równaniu

kierunkowym $y=ax+b$ jest styczna do tej paraboli w punkcie $P$. Oblicz współczynnik $b.$

Odpowied $\acute{\mathrm{z}}$:

Strona 6 z22

MMA-IR





Zadanie 8. (0-3)

Udowodnij, $\dot{\mathrm{z}}\mathrm{e}$ dla dowolnych dodatnich liczb rzeczywistych $x\mathrm{i}y$, takich $\dot{\mathrm{z}}\mathrm{e}x<y$, i dowolnej

dodatniej liczby rzeczywistej $a$, prawdziwajest nierówność $\displaystyle \frac{x+a}{y+a}+\frac{y}{x}>2.$
\begin{center}
\includegraphics[width=96.012mm,height=17.832mm]{./F2_M_PR_M2019_page6_images/image001.eps}
\end{center}
Wypelnia

egzaminator

Nr zadania

Maks. liczba kt

7.

2

8.

3

Uzyskana liczba pkt

MMA-IR

Strona 7 z22





Zadanie 9. (0-3)

Dany jest trójkąt równoramienny $ABC$, w którym $|AC|=|BC|$. Na ramieniu $AC$ tego trójkąta

wybrano punkt $M(M\neq A\mathrm{i}M\neq C)$, a na ramieniu $BC$ wybrano punkt $N$, w taki sposób, $\dot{\mathrm{z}}\mathrm{e}$

$|AM|=|CN|$. Przez punkty $M\mathrm{i}N$ poprowadzono proste prostopadłe do podstawy $AB$ tego

trójkąta, które wyznaczają na niej punkty $S\mathrm{i}T$. Udowodnij, $\displaystyle \dot{\mathrm{z}}\mathrm{e}|ST|=\frac{1}{2}|AB|.$

Strona 8 z22

MMA-IR




\begin{center}
\includegraphics[width=82.044mm,height=17.784mm]{./F2_M_PR_M2019_page8_images/image001.eps}
\end{center}
Wypelnia

egzaminator

Nr zadania

Maks. liczba kt

3

Uzyskana liczba pkt

1R

Strona 9 z22





Zadanie 10. (0-4)

Punkt $D$ lezy na boku $AB$ trójkąta $ABC$ oraz $|AC|=16, |AD|=6, |CD|=14 \mathrm{i} |BC|=|BD|.$

Oblicz obwód trójkąta $ABC.$

Strona 10 z22

MMA-IR



\end{document}