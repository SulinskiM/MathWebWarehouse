\documentclass[a4paper,12pt]{article}
\usepackage{latexsym}
\usepackage{amsmath}
\usepackage{amssymb}
\usepackage{graphicx}
\usepackage{wrapfig}
\pagestyle{plain}
\usepackage{fancybox}
\usepackage{bm}

\begin{document}

{\it Wkazdym z zadań od l. do 4. wybierz i zaznacz na karcie odpowiedzi poprawnq odpowiedzí}.

Zadanie 1. (0-1)

Dla dowolnych liczb $x>0, x\neq 1, y>0, y\neq 1$ wartość wyrazenia $(\log_{\frac{1}{x}}y)\cdot(\log_{y}\perp x)$ jest

równa

$\underline{1}$

A. $x\cdot y$ B. C. $-1$ D. l

$x\cdot y$

Zadanie 2. (0-1)

Liczba $\cos^{2}105^{\mathrm{o}}-\sin^{2}105^{\mathrm{o}}$ jest równa

A.

- -$\sqrt{}$23

B.

- -21

C.

-21

D.

-$\sqrt{}$23

Zadanie 3. (0-1)

Na rysunku przedstawiono fragment wykresu ffinkcji $y=f(x)$, który jest złozony z dwóch

półprostych AD $\mathrm{i}$ CE oraz dwóch odcinków AB $\mathrm{i} BC$, gdzie $A=(-1,0), B=(1,2),$

$C=(3,0), D=(-4,3), E=(6,3).$
\begin{center}
\includegraphics[width=91.284mm,height=62.940mm]{./F2_M_PR_M2019_page1_images/image001.eps}
\end{center}
{\it y}

5

{\it D}

4

3

2

$E_{1}$

{\it B}

{\it x}

$-5$ -$4  -3  -2A$ -$1$  0  1 2  $3C^{4}$  5 6  7

$-1$

Wzór funkcji $f$ to

A. $f(x)=|x+1|+|x-1|$

B. $f(x)=\Vert x-1|-2|$

C. $f(x)=\Vert x-1|+2|$

D. $f(x)=|x-1|+2$

Zadanie 4. (0-1)

Zdarzenia losowe $A \mathrm{i} B$ zawarte w $\Omega$

zdarzenia $B'$, przeciwnego do zdarzenia $B,$

warunkowe $P(A|B)=\displaystyle \frac{1}{5}$. Wynika stąd, $\dot{\mathrm{z}}\mathrm{e}$

A. $P(A\displaystyle \cap B)=\frac{1}{20}$ B. $P(A\displaystyle \cap B)=\frac{4}{15}$

są takie, ze prawdopodobieństwo $P(B')$

est równe $\displaystyle \frac{1}{4}$ Ponadto prawdopodobieństwo

C. $P(A\displaystyle \cap B)=\frac{3}{20}$ D. $P(A\displaystyle \cap B)=\frac{4}{5}$

Strona 2 z22

MMA-IR
\end{document}
