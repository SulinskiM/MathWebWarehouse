\documentclass[a4paper,12pt]{article}
\usepackage{latexsym}
\usepackage{amsmath}
\usepackage{amssymb}
\usepackage{graphicx}
\usepackage{wrapfig}
\pagestyle{plain}
\usepackage{fancybox}
\usepackage{bm}

\begin{document}

$\mathrm{g}_{\mathrm{E}\mathrm{G}\mathrm{Z}\mathrm{A}\mathrm{M}\mathrm{I}\mathrm{N}\mathrm{A}\mathrm{C}\mathrm{Y}\mathrm{J}\mathrm{N}\mathrm{A}}^{\mathrm{C}\mathrm{E}\mathrm{N}\mathrm{T}\mathrm{R}\mathrm{A}\mathrm{L}\mathrm{N}\mathrm{A}}$KOMISJA

Arkusz zawiera informacje

prawnie chronione do momentu

rozpoczęcia egzaminu.

WYPELNIA ZDAJACY

{\it miejsce}

{\it na naklejkę}
\begin{center}
\includegraphics[width=21.900mm,height=16.104mm]{./F2_M_PR_M2020_page0_images/image001.eps}
\end{center}
KOD
\begin{center}
\includegraphics[width=79.608mm,height=16.104mm]{./F2_M_PR_M2020_page0_images/image002.eps}
\end{center}
PESEL
\begin{center}
\includegraphics[width=193.548mm,height=268.584mm]{./F2_M_PR_M2020_page0_images/image003.eps}
\end{center}
EGZAMIN MATU  LNY

Z MATEMATY

POZIOM ROZSZE ONY

DATA: 7 maja 2020 r.

CZAS P CY:180 minut

LICZBA P KTÓW DO UZYS NIA: 50

Instrukcja dla zdającego

1.

2.

3.

Sprawdzí, czy arkusz egzaminacyjny zawiera 22 strony (zadania $1-15$).

Ewentualny brak zgłoś przewodniczącemu zespo nadzorującego

egzamin.

Rozwiązania zadań i odpowiedzi wpisuj w miejscu na to przeznaczonym.

Odpowiedzi do zadań za iętych (l ) zaznacz na karcie odpowiedzi

w części ka przeznaczonej dla zdającego. Zamaluj $\blacksquare$ pola do tego

4.

5.

6.

przeznaczone. Błędne zaznaczenie otocz kółkiem i zaznacz właściwe.

$\mathrm{W}$ zadaniu 5. wpisz odpowiednie cyf w atki pod treścią zadania.

Pamiętaj, $\dot{\mathrm{z}}\mathrm{e}$ pominięcie argumentacji lub istotnych obliczeń

w rozwiązaniu zadania o a ego (6-15) $\mathrm{m}\mathrm{o}\dot{\mathrm{z}}\mathrm{e}$ spowodować, $\dot{\mathrm{z}}\mathrm{e}$ za to

rozwiązanie nie otr masz pełnej liczby pu tów.

Pisz czytelnie i $\mathrm{u}\dot{\mathrm{z}}$ aj tylko $\mathrm{d}$ gopisu lub pióra z czatnym tuszem lub

atramentem.

7. Nie $\mathrm{u}\dot{\mathrm{z}}$ aj korektora, a błędne zapisy $\mathrm{r}\mathrm{a}\acute{\mathrm{z}}\mathrm{n}\mathrm{i}\mathrm{e}$ prze eśl.

8. Pamiętaj, $\dot{\mathrm{z}}\mathrm{e}$ zapisy w brudnopisie nie będą oceniane.

9. $\mathrm{M}\mathrm{o}\dot{\mathrm{z}}$ esz korzystać z zesta wzorów matema cznych, cyrkla i linijki oraz

kalkulatora prostego.

10. Na tej stronie oraz na karcie odpowiedzi wpisz swój numer PESEL

i przyklej naklejkę z kodem.

ll. Nie wpisuj $\dot{\mathrm{z}}$ adnych znaków w części przeznaczonej dla egzaminatora.

$\Vert\Vert\Vert\Vert\Vert\Vert\Vert\Vert\Vert\Vert\Vert\Vert\Vert\Vert\Vert\Vert\Vert\Vert\Vert\Vert\Vert\Vert\Vert\Vert|$

$\mathrm{M}\mathrm{M}\mathrm{A}-\mathrm{R}1_{-}1\mathrm{P}-202$

Układ graficzny

\copyright CKE 2015

$| 1$




{\it Wkazdym z zadań od l. do 4. wybierz i zaznacz na karcie odpowiedzi poprawnq odpowiedzí}.

Zadanie 1. (0-1)

Wielomian $W$ określony wzorem $W(x)=x^{2019}-3x^{2000}+2x+6$

A. jest podzielny przez $(x-1)$ i z dzielenia przez $(x+1)$ daje resztę równą 6.

B. jest podzielny przez $(x+1)$ i z dzielenia przez $(x-1)$ daje resztę równą 6.

C. jest podzielny przez $(x-1)$ ijest podzielny przez $(x+1).$

D. niejest podzielny ani przez $(x-1)$, ani przez $(x+1).$

Zadanie 2. (0-1)

Ciąg $(a_{n})$ jest określony wzorem $a_{n}=\displaystyle \frac{3n^{2}+7n-5}{11-5n+5n^{2}}$ dla $\mathrm{k}\mathrm{a}\dot{\mathrm{z}}$ dej liczby naturalnej $n\geq 1.$

Granica tego ciągu jest równa

A. 3

B.

-51

C.

-53

D.

$-\displaystyle \frac{5}{11}$

Zadanie 3. (0-1)

Mamy dwie urny. $\mathrm{W}$ pierwszej są 3 ku1e białe i 7 ku1 czamych, w drugiej jestjedna ku1a biała

$\mathrm{i}9$ kul czarnych. Rzucamy symetryczną sześcienną kostką do gry, która na $\mathrm{k}\mathrm{a}\dot{\mathrm{z}}$ dej ściance ma

inną liczbę oczek, odjednego oczka do sześciu oczek. Jeśli w wyniku rzutu otrzymamy ściankę

z jednym oczkiem, to losujemy jedną kulę z pierwszej umy, w przeciwnym przypadku-

losujemy jedną kulę z drugiej umy. Wtedy prawdopodobieństwo wylosowania kuli białej jest

równe

A.

$\displaystyle \frac{2}{15}$

B.

-51

C.

-45

D.

$\displaystyle \frac{13}{15}$

Zadanie 4. (0-1)

Po przekształceniu wyrazenia algebraicznego

$ax^{4}+bx^{3}y+cx^{2}y^{2}+dxy^{3}+ey^{4}$ współczynnik $c$ jest równy

$(\sqrt{2}\sqrt{3})^{4}$

do postaci

A. 6

B. 36

C. $8\sqrt{6}$

D. $12\sqrt{6}$

Strona 2 z22

MMA-IR





Odpowiedzí:
\begin{center}
\includegraphics[width=82.044mm,height=17.832mm]{./F2_M_PR_M2020_page10_images/image001.eps}
\end{center}
Wypelnia

egzaminator

Nr zadania

Maks. liczba kt

10.

5

Uzyskana liczba pkt

MMA-IR

Strona ll z22





Zadanie 11. (0-4)

Dane jest równanie kwadratowe $x^{2}-(3m+2)x+2m^{2}+7m-15=0$ z niewiadomą $x$. Wyznacz

wszystkie wartości parametru $m$, dla których rózne rozwiązania

i spełniają warunek

$2x_{1}^{2}+5x_{1}x_{2}+2x_{2}^{2}=2.$

$x_{1}$ i $x_{2}$ tego równania istnieją

Strona 12 z22

MMA-IR





Odpowiedzí:
\begin{center}
\includegraphics[width=82.044mm,height=17.784mm]{./F2_M_PR_M2020_page12_images/image001.eps}
\end{center}
Wypelnia

egzamÍnator

Nr zadania

Maks. liczba kt

11.

4

Uzyskana liczba pkt

MMA-IR

Strona 13 z22





Zadanie 12. (0-5)

Prosta o równaniu

$x+y-10=0$ przecina

okrąg o

równaniu $x^{2}+y^{2}-8x-6y+8=0$

wpunktach $K\mathrm{i}L$. Punkt $S$ jest środkiem cięciwy $KL$. Wyznacz równanie obrazu tego okręgu

wjednokładności o środku $S$ i skali $k=-3.$

Strona 14 z22

MMA-IR





Odpowiedzí:
\begin{center}
\includegraphics[width=82.044mm,height=17.784mm]{./F2_M_PR_M2020_page14_images/image001.eps}
\end{center}
Wypelnia

egzamÍnator

Nr zadania

Maks. liczba kt

12.

5

Uzyskana liczba pkt

MMA-IR

Strona 15 z22





Zadanie 13. (0-4)

Oblicz, ilejest wszystkich siedmiocyfrowych liczb naturalnych, w których zapisie dziesiętnym

występują dokładnie trzy cyfry l i dokładnie dwie cyfry 2.

Strona 16 z22

MMA-IR





Odpowiedzí:
\begin{center}
\includegraphics[width=82.044mm,height=17.784mm]{./F2_M_PR_M2020_page16_images/image001.eps}
\end{center}
Wypelnia

egzamÍnator

Nr zadania

Maks. liczba kt

13.

4

Uzyskana liczba pkt

MMA-IR

Strona 17 z22





Zadanie 14. (0-6)

Podstawą ostrosłupa czworokątnego ABCDS jest trapez ABCD (AB $||$ CD). Ramiona tego

trapezu mają długości $|AD|=10 \mathrm{i}|BC|=16$, a miara kąta $ABC$ jest równa $30^{\mathrm{o}}. \mathrm{K}\mathrm{a}\dot{\mathrm{z}}$ da ściana

boczna tego ostrosłupa tworzy z płaszczyzną podstawy kąt $\alpha$, taki, $\dot{\mathrm{z}}\mathrm{e}$ tg $\displaystyle \alpha=\frac{9}{2}$. Oblicz objętość

tego ostrosłupa.

Strona 18 z22

MMA-IR





Odpowiedzí:
\begin{center}
\includegraphics[width=82.044mm,height=17.784mm]{./F2_M_PR_M2020_page18_images/image001.eps}
\end{center}
Wypelnia

egzamÍnator

Nr zadania

Maks. liczba kt

14.

Uzyskana liczba pkt

MMA-IR

Strona 19 z22





Zadanie 15. (0-7)

Nalez$\mathrm{y}$ zaprojektować wymiary prostokątnego ekranu smartfona, tak aby odległości tego

ekranu od krótszych brzegów smartfona były równe 0,5 cm $\mathrm{k}\mathrm{a}\dot{\mathrm{z}}$ da, a odległości tego ekranu

od dłuzszych brzegów smartfona były równe 0,3 cm $\mathrm{k}\mathrm{a}\dot{\mathrm{z}}$ da (zobacz rysunek- ekran zaznaczono

kolorem szarym). Sam ekran ma mieć powierzchnię 60 $\mathrm{c}\mathrm{m}^{2}$. Wyznacz takie wymiary ekranu

smartfona, przy których powierzchnia ekranu wraz z obramowaniemjest najmniejsza.
\begin{center}
\includegraphics[width=103.584mm,height=116.388mm]{./F2_M_PR_M2020_page19_images/image001.eps}
\end{center}
0,5 cm

e an

0,5 cm

obramowanie

brzeg

Strona 20 z22

MMA-IR





BRUDNOPIS

1R

Strona 3 z22





Odpowiedzí:
\begin{center}
\includegraphics[width=82.044mm,height=17.784mm]{./F2_M_PR_M2020_page20_images/image001.eps}
\end{center}
Wypelnia

egzamÍnator

Nr zadania

Maks. liczba kt

15.

7

Uzyskana liczba pkt

MMA-IR

Strona 21 z22





{\it BRUDNOPIS} ({\it nie podlega ocenie})

Strona 22 z22

$\mathrm{M}\mathrm{M}\mathrm{A}_{-}1l$















Zadanie 5. (0-2)

$\mathrm{W}$ trójkącie $ABC$ bok $AB$ jest 3 razy dłuzszy od boku $AC$, a długość boku $BC$ stanowi $\displaystyle \frac{4}{5}$

długości boku $AB$. Oblicz cosinus najmniejszego kąta trójkąta $ABC.$

$\mathrm{W}$ kratki ponizej wpisz kolejno-od lewej do prawej -pierwszą, drugą oraz trzecią cyfrę

po przecinku nieskończonego rozwinięcia dziesiętnego otrzymanego wyniku.
\begin{center}
\includegraphics[width=22.500mm,height=10.872mm]{./F2_M_PR_M2020_page3_images/image001.eps}
\end{center}
Strona 4 z22

MMA-IR





Zadanie 6. (0-3)

Wyznacz wszystkie wartości parametru $a$, dla których równanie $|x-5|=(a-1)^{2}-4$ ma dwa

rózne rozwiązania dodatnie.

Odpowiedzí:
\begin{center}
\includegraphics[width=96.012mm,height=17.832mm]{./F2_M_PR_M2020_page4_images/image001.eps}
\end{center}
Wypelnia

egzaminator

Nr zadania

Maks. liczba kt

5.

2

3

Uzyskana liczba pkt

MMA-IR

Strona 5 z22





Zadanie 7. (0-3)

Dany jest trójkąt równoramienny $ABC$, w którym $|AC|=|BC|=6$, a punkt $D$ jest środkiem

podstawy $AB$. Okrąg o środku $D$ jest styczny do prostej $AC$ w punkcie $M$. Punkt $K$ lezy na boku

$AC$, punkt $L$ lezy na boku $BC$, odcinek $KL$ jest styczny do rozwazanego okręgu oraz $|KC|=|LC|=2$

(zobacz rysunek).
\begin{center}
\includegraphics[width=101.448mm,height=64.260mm]{./F2_M_PR_M2020_page5_images/image001.eps}
\end{center}
{\it C}

{\it K  L}

{\it M}

{\it A  D  B}

Wykaz, $\displaystyle \dot{\mathrm{z}}\mathrm{e}\frac{|AM|}{|MC|}=\frac{4}{5}$

Strona 6 z22

MMA-IR




\begin{center}
\includegraphics[width=82.044mm,height=17.784mm]{./F2_M_PR_M2020_page6_images/image001.eps}
\end{center}
Wypelnia

egzamÍnator

Nr zadania

Maks. liczba kt

7.

3

Uzyskana liczba pkt

1R

Strona 7 z22





Zadanie 8. (0-3)

Liczby dodatnie $a\mathrm{i}b$ spełniają równość $a^{2}+2a=4b^{2}+4b$. Wykaz, $\dot{\mathrm{z}}\mathrm{e}a=2b.$

Strona 8 z22

MMA-I]





Zadanie 9. (0-4)

Rozwiąz równanie 3 $\cos 2x+10\cos^{2}x=24\sin x-3$ dla $x\in\langle 0, 2\pi\rangle.$

Odpowiedzí:
\begin{center}
\includegraphics[width=96.012mm,height=17.784mm]{./F2_M_PR_M2020_page8_images/image001.eps}
\end{center}
WypelnÍa

egzaminator

Nr zadania

Maks. lÍczba kt

8.

3

4

Uzyskana liczba pkt

MMA-IR

Strona 9 z22





Zadanie 10. (0-5)

$\mathrm{W}$ trzywyrazowym ciągu geometrycznym $(a_{1},a_{2},a_{3})$ spełnionajest równość $a_{1}+a_{2}+a_{3}=\displaystyle \frac{21}{4}.$

Wyrazy $a_{1}, a_{2}, a_{3}$ są- odpowiednio-czwartym, drugim i pierwszym wyrazem rosnącego

ciągu arytmetycznego. Oblicz $a_{1}.$

Strona 10 z22

MMA-IR



\end{document}