\documentclass[a4paper,12pt]{article}
\usepackage{latexsym}
\usepackage{amsmath}
\usepackage{amssymb}
\usepackage{graphicx}
\usepackage{wrapfig}
\pagestyle{plain}
\usepackage{fancybox}
\usepackage{bm}

\begin{document}

$\mathrm{g}_{\mathrm{E}\mathrm{G}\mathrm{Z}\mathrm{A}\mathrm{M}\mathrm{I}\mathrm{N}\mathrm{A}\mathrm{C}\mathrm{Y}\mathrm{J}\mathrm{N}\mathrm{A}}^{\mathrm{C}\mathrm{E}\mathrm{N}\mathrm{T}\mathrm{R}\mathrm{A}\mathrm{L}\mathrm{N}\mathrm{A}}$KOMISJA

Arkusz zawiera informacje

prawnie chronione do momentu

rozpoczęcia egzaminu.

WYPELNIA ZDAJACY

{\it miejsce}

{\it na naklejkę}
\begin{center}
\includegraphics[width=21.900mm,height=16.104mm]{./F2_M_PR_M2020_page0_images/image001.eps}
\end{center}
KOD
\begin{center}
\includegraphics[width=79.608mm,height=16.104mm]{./F2_M_PR_M2020_page0_images/image002.eps}
\end{center}
PESEL
\begin{center}
\includegraphics[width=193.548mm,height=268.584mm]{./F2_M_PR_M2020_page0_images/image003.eps}
\end{center}
EGZAMIN MATU  LNY

Z MATEMATY

POZIOM ROZSZE ONY

DATA: 7 maja 2020 r.

CZAS P CY:180 minut

LICZBA P KTÓW DO UZYS NIA: 50

Instrukcja dla zdającego

1.

2.

3.

Sprawdzí, czy arkusz egzaminacyjny zawiera 22 strony (zadania $1-15$).

Ewentualny brak zgłoś przewodniczącemu zespo nadzorującego

egzamin.

Rozwiązania zadań i odpowiedzi wpisuj w miejscu na to przeznaczonym.

Odpowiedzi do zadań za iętych (l ) zaznacz na karcie odpowiedzi

w części ka przeznaczonej dla zdającego. Zamaluj $\blacksquare$ pola do tego

4.

5.

6.

przeznaczone. Błędne zaznaczenie otocz kółkiem i zaznacz właściwe.

$\mathrm{W}$ zadaniu 5. wpisz odpowiednie cyf w atki pod treścią zadania.

Pamiętaj, $\dot{\mathrm{z}}\mathrm{e}$ pominięcie argumentacji lub istotnych obliczeń

w rozwiązaniu zadania o a ego (6-15) $\mathrm{m}\mathrm{o}\dot{\mathrm{z}}\mathrm{e}$ spowodować, $\dot{\mathrm{z}}\mathrm{e}$ za to

rozwiązanie nie otr masz pełnej liczby pu tów.

Pisz czytelnie i $\mathrm{u}\dot{\mathrm{z}}$ aj tylko $\mathrm{d}$ gopisu lub pióra z czatnym tuszem lub

atramentem.

7. Nie $\mathrm{u}\dot{\mathrm{z}}$ aj korektora, a błędne zapisy $\mathrm{r}\mathrm{a}\acute{\mathrm{z}}\mathrm{n}\mathrm{i}\mathrm{e}$ prze eśl.

8. Pamiętaj, $\dot{\mathrm{z}}\mathrm{e}$ zapisy w brudnopisie nie będą oceniane.

9. $\mathrm{M}\mathrm{o}\dot{\mathrm{z}}$ esz korzystać z zesta wzorów matema cznych, cyrkla i linijki oraz

kalkulatora prostego.

10. Na tej stronie oraz na karcie odpowiedzi wpisz swój numer PESEL

i przyklej naklejkę z kodem.

ll. Nie wpisuj $\dot{\mathrm{z}}$ adnych znaków w części przeznaczonej dla egzaminatora.

$\Vert\Vert\Vert\Vert\Vert\Vert\Vert\Vert\Vert\Vert\Vert\Vert\Vert\Vert\Vert\Vert\Vert\Vert\Vert\Vert\Vert\Vert\Vert\Vert|$

$\mathrm{M}\mathrm{M}\mathrm{A}-\mathrm{R}1_{-}1\mathrm{P}-202$

Układ graficzny

\copyright CKE 2015

$| 1$
\end{document}
