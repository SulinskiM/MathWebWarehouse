\documentclass[a4paper,12pt]{article}
\usepackage{latexsym}
\usepackage{amsmath}
\usepackage{amssymb}
\usepackage{graphicx}
\usepackage{wrapfig}
\pagestyle{plain}
\usepackage{fancybox}
\usepackage{bm}

\begin{document}

{\it Wkazdym z zadań od l. do 4. wybierz i zaznacz na karcie odpowiedzi poprawnq odpowiedzí}.

Zadanie 1. (0-1)

Wielomian $W$ określony wzorem $W(x)=x^{2019}-3x^{2000}+2x+6$

A. jest podzielny przez $(x-1)$ i z dzielenia przez $(x+1)$ daje resztę równą 6.

B. jest podzielny przez $(x+1)$ i z dzielenia przez $(x-1)$ daje resztę równą 6.

C. jest podzielny przez $(x-1)$ ijest podzielny przez $(x+1).$

D. niejest podzielny ani przez $(x-1)$, ani przez $(x+1).$

Zadanie 2. (0-1)

Ciąg $(a_{n})$ jest określony wzorem $a_{n}=\displaystyle \frac{3n^{2}+7n-5}{11-5n+5n^{2}}$ dla $\mathrm{k}\mathrm{a}\dot{\mathrm{z}}$ dej liczby naturalnej $n\geq 1.$

Granica tego ciągu jest równa

A. 3

B.

-51

C.

-53

D.

$-\displaystyle \frac{5}{11}$

Zadanie 3. (0-1)

Mamy dwie urny. $\mathrm{W}$ pierwszej są 3 ku1e białe i 7 ku1 czamych, w drugiej jestjedna ku1a biała

$\mathrm{i}9$ kul czarnych. Rzucamy symetryczną sześcienną kostką do gry, która na $\mathrm{k}\mathrm{a}\dot{\mathrm{z}}$ dej ściance ma

inną liczbę oczek, odjednego oczka do sześciu oczek. Jeśli w wyniku rzutu otrzymamy ściankę

z jednym oczkiem, to losujemy jedną kulę z pierwszej umy, w przeciwnym przypadku-

losujemy jedną kulę z drugiej umy. Wtedy prawdopodobieństwo wylosowania kuli białej jest

równe

A.

$\displaystyle \frac{2}{15}$

B.

-51

C.

-45

D.

$\displaystyle \frac{13}{15}$

Zadanie 4. (0-1)

Po przekształceniu wyrazenia algebraicznego

$ax^{4}+bx^{3}y+cx^{2}y^{2}+dxy^{3}+ey^{4}$ współczynnik $c$ jest równy

$(\sqrt{2}\sqrt{3})^{4}$

do postaci

A. 6

B. 36

C. $8\sqrt{6}$

D. $12\sqrt{6}$

Strona 2 z22

MMA-IR
\end{document}
