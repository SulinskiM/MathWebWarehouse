\documentclass[a4paper,12pt]{article}
\usepackage{latexsym}
\usepackage{amsmath}
\usepackage{amssymb}
\usepackage{graphicx}
\usepackage{wrapfig}
\pagestyle{plain}
\usepackage{fancybox}
\usepackage{bm}

\begin{document}

Zadanie 15. (0-7)

Nalez$\mathrm{y}$ zaprojektować wymiary prostokątnego ekranu smartfona, tak aby odległości tego

ekranu od krótszych brzegów smartfona były równe 0,5 cm $\mathrm{k}\mathrm{a}\dot{\mathrm{z}}$ da, a odległości tego ekranu

od dłuzszych brzegów smartfona były równe 0,3 cm $\mathrm{k}\mathrm{a}\dot{\mathrm{z}}$ da (zobacz rysunek- ekran zaznaczono

kolorem szarym). Sam ekran ma mieć powierzchnię 60 $\mathrm{c}\mathrm{m}^{2}$. Wyznacz takie wymiary ekranu

smartfona, przy których powierzchnia ekranu wraz z obramowaniemjest najmniejsza.
\begin{center}
\includegraphics[width=103.584mm,height=116.388mm]{./F2_M_PR_M2020_page19_images/image001.eps}
\end{center}
0,5 cm

e an

0,5 cm

obramowanie

brzeg

Strona 20 z22

MMA-IR
\end{document}
