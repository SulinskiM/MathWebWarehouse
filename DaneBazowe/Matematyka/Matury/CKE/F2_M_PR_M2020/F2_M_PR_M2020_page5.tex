\documentclass[a4paper,12pt]{article}
\usepackage{latexsym}
\usepackage{amsmath}
\usepackage{amssymb}
\usepackage{graphicx}
\usepackage{wrapfig}
\pagestyle{plain}
\usepackage{fancybox}
\usepackage{bm}

\begin{document}

Zadanie 7. (0-3)

Dany jest trójkąt równoramienny $ABC$, w którym $|AC|=|BC|=6$, a punkt $D$ jest środkiem

podstawy $AB$. Okrąg o środku $D$ jest styczny do prostej $AC$ w punkcie $M$. Punkt $K$ lezy na boku

$AC$, punkt $L$ lezy na boku $BC$, odcinek $KL$ jest styczny do rozwazanego okręgu oraz $|KC|=|LC|=2$

(zobacz rysunek).
\begin{center}
\includegraphics[width=101.448mm,height=64.260mm]{./F2_M_PR_M2020_page5_images/image001.eps}
\end{center}
{\it C}

{\it K  L}

{\it M}

{\it A  D  B}

Wykaz, $\displaystyle \dot{\mathrm{z}}\mathrm{e}\frac{|AM|}{|MC|}=\frac{4}{5}$

Strona 6 z22

MMA-IR
\end{document}
