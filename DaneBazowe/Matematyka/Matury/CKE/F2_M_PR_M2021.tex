\documentclass[a4paper,12pt]{article}
\usepackage{latexsym}
\usepackage{amsmath}
\usepackage{amssymb}
\usepackage{graphicx}
\usepackage{wrapfig}
\pagestyle{plain}
\usepackage{fancybox}
\usepackage{bm}

\begin{document}

CENTRALNA

KOMISJA

EGZAMINACYJNA

Arkusz zawiera informacje prawnie chronione

do momentu rozpoczecia egzaminu.

KOD

WYPELNIA ZDAJACY

PESEL

{\it Miejsce na naklejke}.

{\it Sprawdz}', {\it czy kod na naklejce to}

E-100.
\begin{center}
\includegraphics[width=21.900mm,height=10.212mm]{./F2_M_PR_M2021_page0_images/image001.eps}

\includegraphics[width=79.608mm,height=10.212mm]{./F2_M_PR_M2021_page0_images/image002.eps}
\end{center}
$J\mathrm{e}\dot{\mathrm{z}}$ {\it eli tak}- {\it przyklej naklejkq}.

{\it lezeli nie}- {\it zgtoś to nauczycielowi}.

EGZAMIN MATURALNY Z MATEMATYKI

POZIOM ROZSZERZONY

WYPELNIA ZESP6L NADZORUJACY

DAT$\mathrm{A}^{\cdot}$ ll maja 2021 $\mathrm{r}.$

GODZINA ROZPOCZeClA: 9:00

CZAS PRACY: $180 \displaystyle \min \mathrm{u}\mathrm{t}$

LICZBA PUNKTÓW DO UZYSKANIA 50

Uprawnienia zdajacego do:

\fbox{} dostosowania zasad oceniania

\fbox{} dostosowania w zw. z dyskalkuliq

\fbox{} nieprzenoszenia zaznaczeń na kart9.

$\Vert\Vert\Vert\Vert\Vert\Vert\Vert\Vert\Vert\Vert\Vert\Vert\Vert\Vert\Vert\Vert\Vert\Vert\Vert\Vert\Vert\Vert\Vert\Vert\Vert\Vert\Vert\Vert\Vert\Vert|$

EMAP-R0-100-2105

lnstrukcja dla zdajqcego

l. Sprawdz', czy arkusz egzaminacyjny zawiera 27 stron (zadania $1-15$).

Ewentualny brak zgloś przewodniczqcemu zespolu nadzorujqcego egzamin.

2. Na tej stronie oraz na karcie odpowiedzi wpisz swój numer PESEL i przyklej naklejk9

z kodem.

3. Nie wpisuj $\dot{\mathrm{z}}$ adnych znaków w cześci przeznaczonej dla egzaminatora.

4. Rozwiqzania zadań i odpowiedzi wpisuj w miejscu na to przeznaczonym.

5. Odpowiedzi do zadań zamknietych $(1-4)$ zaznacz na karcie odpowiedzi w części karty

przeznaczonej dla zdajqcego. Zamaluj $\blacksquare$ pola do tego przeznaczone. $\mathrm{B}_{9}\mathrm{d}\mathrm{n}\mathrm{e}$

zaznaczenie otocz kólkiem @ i zaznacz wlaściwe.

6. $\mathrm{W}$ zadaniu 5. wpisz odpowiednie cyfry w kratki pod treściq zadania.

7. Pamiptaj, $\dot{\mathrm{z}}\mathrm{e}$ pominipcie argumentacji lub istotnych obliczeń w rozwiqzaniu zadania

otwartego (6-15) $\mathrm{m}\mathrm{o}\dot{\mathrm{z}}\mathrm{e}$ spowodować, $\dot{\mathrm{z}}\mathrm{e}$ za to rozwiqzanie nie otrzymasz pelnej liczby

punktów.

8. Pisz czytelnie i $\mathrm{u}\dot{\mathrm{z}}$ ywaj tylko dlugopisu lub pióra z czarnym tuszem lub atramentem.

9. Nie $\mathrm{u}\dot{\mathrm{z}}$ ywaj korektora, a bledne zapisy wyra $\acute{\mathrm{z}}$ nie przekreśl.

10. $\mathrm{P}\mathrm{a}\mathrm{m}\mathrm{i}_{9}\mathrm{t}\mathrm{a}\mathrm{j}, \dot{\mathrm{z}}\mathrm{e}$ zapisy w brudnopisie nie bedq oceniane.

11. $\mathrm{M}\mathrm{o}\dot{\mathrm{z}}$ esz korzystač z zestawu wzorów matematycznych, cyrkla i linijki oraz kalkulatora

prostego.

Uklad graficzny

\copyright CKE 2021




{\it W kazdym z zadań od f. do 4. wybierz i zaznacz na karcie odpowiedzi poprawnq odpowiedz}'.

Zadanie 1. (0-1)

Róznica $\cos^{2}165^{\mathrm{o}}-\sin^{2}165^{\mathrm{o}}$ jest równa

A. $-1$

B. $-\displaystyle \frac{\sqrt{3}}{2}$

C. - -21

D. $\displaystyle \frac{\sqrt{3}}{2}$

Zadanie 2. $\{0-l\mathrm{I}$

Na rysunku przedstawiono fragment

rzeczywistej $x.$

wykresu funkcji

f określonej

dla $\mathrm{k}\mathrm{a}\dot{\mathrm{z}}$ dej liczby

Jeden spośród podanych ponizej wzorów jest wzorem tej funkcji. Wskaz wzór funkcji f.

A. $f(x)=\displaystyle \frac{\cos x+1}{|\cos x|+1}$

B. $f(x)=\displaystyle \frac{\sin x+1}{|\sin x|+1}$

C. $f(x)=\displaystyle \frac{|\cos x|-2}{\cos x-2}$

D. $f(x)=\displaystyle \frac{|\sin x|-2}{\sin x-2}$

Zadanie 3. (0-1)

Wielomian $W(x)=x^{4}+81$ jest podzielny przez

A. $x-3$

B. $x^{2}+9$

C. $x^{2}-3\sqrt{2}x+9$

D. $x^{2}+3\sqrt{2}x-9$

Zadanie 4. (0-1)

Liczba róznych pierwiastków równania $3x+|x-4|=0$ jest równa

A. 0

B. l

C. 2

D. 3

Strona 2 z27

$\mathrm{E}\mathrm{M}\mathrm{A}\mathrm{P}-\mathrm{R}0_{-}100$




\begin{center}
\includegraphics[width=192.840mm,height=258.876mm]{./F2_M_PR_M2021_page10_images/image001.eps}
\end{center}
$\mathrm{O}\mathrm{d}\mathrm{p}\mathrm{o}\mathrm{w}\mathrm{i}\mathrm{e}\mathrm{d}\acute{\mathrm{z}}$:$\ldots\ldots\ldots\ldots\ldots\ldots\ldots\ldots\ldots\ldots\ldots\ldots\ldots\ldots\ldots\ldots\ldots\ldots\ldots\ldots\ldots\ldots\ldots\ldots\ldots\ldots\ldots\ldots\ldots\ldots\ldots\ldots\ldots\ldots\ldots\ldots\ldots\ldots\ldots\ldots\ldots\ldots$

Wypelnia

egzaminator

Nr zadania

Maks. liczba pkt

Uzyskana liczba pkt

9.

4

$\mathrm{E}\mathrm{M}\mathrm{A}\mathrm{P}-\mathrm{R}0_{-}100$

Strona ll z27





$\mathrm{Z}\text{à} \mathrm{d}^{\backslash }\cdot \mathrm{a}\mathfrak{n}1\mathrm{e}10. 1(0-4l1$

Prosta przechodzqca przez

punkty $A = (8,-6)$

i

$B = (5,15)$ jest styczna

do

okregu

o środku w punkcie $0 =$

(0,0). Oblicz promień tego okregu i wspólrzedne punktu styczności tego

okrpgu z prostq

{\it AB}.
\begin{center}
\includegraphics[width=192.840mm,height=258.876mm]{./F2_M_PR_M2021_page11_images/image001.eps}
\end{center}
Strona 12 z27

$\mathrm{E}\mathrm{M}\mathrm{A}\mathrm{P}-\mathrm{R}0_{-}100$




\begin{center}
\includegraphics[width=192.840mm,height=258.876mm]{./F2_M_PR_M2021_page12_images/image001.eps}
\end{center}
$\mathrm{O}\mathrm{d}\mathrm{p}\mathrm{o}\mathrm{w}\mathrm{i}\mathrm{e}\mathrm{d}\acute{\mathrm{z}}$:$\ldots\ldots\ldots\ldots\ldots\ldots\ldots\ldots\ldots\ldots\ldots\ldots\ldots\ldots\ldots\ldots\ldots\ldots\ldots\ldots\ldots\ldots\ldots\ldots\ldots\ldots\ldots\ldots\ldots\ldots\ldots\ldots\ldots\ldots\ldots\ldots\ldots\ldots\ldots\ldots\ldots\ldots$

Wypelnia

egzaminator

Nr zadania

Maks. liczba pkt

Uzyskana liczba pkt

10.

4

$\mathrm{E}\mathrm{M}\mathrm{A}\mathrm{P}-\mathrm{R}0_{-}100$

Strona 13 z27





$\mathrm{Z}\text{à} \mathrm{d}^{\backslash }\cdot \mathrm{a}\mathfrak{n}1\mathrm{e}11^{\backslash }1.(0-5\}$

Wyznacz wszystkie wartości parametru $m$, dla których trójmian kwadratowy

$4x^{2}-2(m+1)x+m$

ma dwa rózne pierwiastki rzeczywiste

$\chi_{1}$

oraz $x_{2}$, spelniajqce warunki

$\chi_{1} \neq 0,$

$\chi_{2}$

$\neq 0$

oraz

$\chi_{1} +x_{2} \leq_{\chi_{1}\chi_{2}}^{1_{+}1}$
\begin{center}
\includegraphics[width=192.840mm,height=234.852mm]{./F2_M_PR_M2021_page13_images/image001.eps}
\end{center}
Strona 14 z27

$\mathrm{E}\mathrm{M}\mathrm{A}\mathrm{P}-\mathrm{R}0_{-}100$




\begin{center}
\includegraphics[width=192.840mm,height=258.876mm]{./F2_M_PR_M2021_page14_images/image001.eps}
\end{center}
$\mathrm{O}\mathrm{d}\mathrm{p}\mathrm{o}\mathrm{w}\mathrm{i}\mathrm{e}\mathrm{d}\acute{\mathrm{z}}$:$\ldots\ldots\ldots\ldots\ldots\ldots\ldots\ldots\ldots\ldots\ldots\ldots\ldots\ldots\ldots\ldots\ldots\ldots\ldots\ldots\ldots\ldots\ldots\ldots\ldots\ldots\ldots\ldots\ldots\ldots\ldots\ldots\ldots\ldots\ldots\ldots\ldots\ldots\ldots\ldots\ldots\ldots$

Wypelnia

egzaminator

Nr zadania

Maks. liczba pkt

Uzyskana liczba pkt

11.

5

$\mathrm{E}\mathrm{M}\mathrm{A}\mathrm{P}-\mathrm{R}0_{-}100$

Strona 15 z27





$\mathrm{Z}\text{à} \mathrm{d}^{\backslash }\cdot \mathrm{a}\mathfrak{n}1\mathrm{e}12^{\backslash }1.(0-5\}$

Rozwiqz równanie

$\cos 2x =\displaystyle \frac{\sqrt{2}}{2}(\cos x-\sin x)$

w przedziale

$\langle 0, \pi\rangle.$
\begin{center}
\includegraphics[width=192.840mm,height=270.912mm]{./F2_M_PR_M2021_page15_images/image001.eps}
\end{center}
Strona 16 z27

$\mathrm{E}\mathrm{M}\mathrm{A}\mathrm{P}-\mathrm{R}0_{-}100$




\begin{center}
\includegraphics[width=192.840mm,height=258.876mm]{./F2_M_PR_M2021_page16_images/image001.eps}
\end{center}
$\mathrm{O}\mathrm{d}\mathrm{p}\mathrm{o}\mathrm{w}\mathrm{i}\mathrm{e}\mathrm{d}\acute{\mathrm{z}}$:$\ldots\ldots\ldots\ldots\ldots\ldots\ldots\ldots\ldots\ldots\ldots\ldots\ldots\ldots\ldots\ldots\ldots\ldots\ldots\ldots\ldots\ldots\ldots\ldots\ldots\ldots\ldots\ldots\ldots\ldots\ldots\ldots\ldots\ldots\ldots\ldots\ldots\ldots\ldots\ldots\ldots\ldots$

Wypelnia

egzaminator

Nr zadania

Maks. liczba pkt

Uzyskana liczba pkt

12.

5

$\mathrm{E}\mathrm{M}\mathrm{A}\mathrm{P}-\mathrm{R}0_{-}100$

Strona

17 z27





$\mathrm{Z}\text{à} \mathrm{d}^{\backslash }\cdot \mathrm{a}\mathrm{n}1\mathrm{e}1 13^{\backslash }. (0-4$

Dany jest trójkqt prostokqtny

{\it ABC}.

Promień okregu wpisanego w ten trójkqt jest pieć razy

krótszy od przeciwprostokqtnej tego trójkqta. Oblicz sinus tego z kqtów ostrych trójkqta

{\it ABC},

który ma wipkszq miar9.
\begin{center}
\includegraphics[width=192.840mm,height=258.924mm]{./F2_M_PR_M2021_page17_images/image001.eps}
\end{center}
Strona 18 z27

$\mathrm{E}\mathrm{M}\mathrm{A}\mathrm{P}-\mathrm{R}0_{-}100$




\begin{center}
\includegraphics[width=192.840mm,height=258.876mm]{./F2_M_PR_M2021_page18_images/image001.eps}
\end{center}
$\mathrm{O}\mathrm{d}\mathrm{p}\mathrm{o}\mathrm{w}\mathrm{i}\mathrm{e}\mathrm{d}\acute{\mathrm{z}}$:$\ldots\ldots\ldots\ldots\ldots\ldots\ldots\ldots\ldots\ldots\ldots\ldots\ldots\ldots\ldots\ldots\ldots\ldots\ldots\ldots\ldots\ldots\ldots\ldots\ldots\ldots\ldots\ldots\ldots\ldots\ldots\ldots\ldots\ldots\ldots\ldots\ldots\ldots\ldots\ldots\ldots\ldots$

Wypelnia

egzaminator

Nr zadania

Maks. liczba pkt

Uzyskana liczba pkt

13.

4

$\mathrm{E}\mathrm{M}\mathrm{A}\mathrm{P}-\mathrm{R}0_{-}100$

Strona 19 z27





Zadänie $l4. (0-6)$

Dane sq parabola o równaniu $y=x^{2}$ oraz punkty $A=(0,2) \mathrm{i} B=(1,3)$ (zobacz rysunek).

Rozpatrujemy wszystkie trójkqty $ABC$, których wierzcholek $C \mathrm{l}\mathrm{e}\dot{\mathrm{z}}\mathrm{y}$ na tej paraboli. Niech $m$

oznacza pierwszq wspólrzedna punktu $C.$

a) Wyznacz pole $P$ trójkqta $ABC$ jako funkcj9 zmiennej $m.$

b) Wyznacz wszystkie wartości $m$, dla których trójkqt $ABC$ jest ostrokqtny.
\begin{center}
\includegraphics[width=192.840mm,height=132.636mm]{./F2_M_PR_M2021_page19_images/image001.eps}
\end{center}
1

1

$1$

$\Gamma^{\neg}111$

$111\overline{1}$

1

$-|$

$1111$

$-|$

1

$1^{1} \mathrm{T}$

$1^{-}1$

I

$\mathrm{i}$

1

1

11

111

Jll

$\rfloor 1$

$\lceil$ tl i$|$- $| \dagger|$1l

$\mapsto_{1}\mathrm{I}1$

L

1

11-

1

$\leftarrow \mathrm{I}\mathrm{I}\mathrm{l}\mathrm{l}\mathrm{I}\mathrm{T}11$

ll

$-1$

$\lfloor 1$ 11

1

L$\lfloor+ +$Ill$| \iota|\lfloor$----$|$--lllll

11

11

11

Il

L

1

1

1

$- \lrcorner 1 \mathrm{L}\mathrm{l}1$

$\mathrm{I}1$

11

$-111 \dagger^{1} \rightarrow$

1

$11$ Il

-i $\dot{1}$

1

$1$ 1

$|$l.i

$1$

$111$

$\lrcorner 1$

$\leftarrow 1111$

$\downarrow$

$\dashv-1_{1}^{\mathrm{t}}1$

1

$1$

$1111$

1

1

L

1

$1$

$\mathrm{t}$

$|$ll $|$l--l

$1$

$1$

$1$

L

1

$\perp$

11

1

1!

$!$ : -$\vdash 11 1\mathrm{I}1$

$1$

T

I

1

$1$

$\lceil$

$11\mathrm{I}1$

$-1-$

$\mathrm{r}111$

$111$

$1111$

$11$

ll

1

$1_{-}1$

11

11

ll

$-1$

$1$

I

$111$

$+$

$\neg 1$ -T llll

$\tau \iota$l $|| | |$--l1lll

$\ulcorner 1$

1

$-1_{1}1^{-\mapsto}$

$11$

$+^{1}$

T

$-111$

1

11

Il $1 1$

Il

$\mathrm{L}\mathrm{l}1$

$\rfloor$

$1$

$1$

$1$

-l$|$

Strona 20 z27

$\mathrm{E}\mathrm{M}\mathrm{A}\mathrm{P}-\mathrm{R}0_{-}100$





- RUDNOPIS

$\{m^{\bullet}\mathrm{e}$ {\it podlega} $oc\mathrm{e}m^{\bullet}\mathrm{e}\}$
\begin{center}
\includegraphics[width=192.840mm,height=282.960mm]{./F2_M_PR_M2021_page2_images/image001.eps}
\end{center}
$\mathrm{E}\mathrm{M}\mathrm{A}\mathrm{P}-\mathrm{R}0_{-}100$

Strona 3 z27




\begin{center}
\includegraphics[width=192.840mm,height=294.948mm]{./F2_M_PR_M2021_page20_images/image001.eps}
\end{center}
$\mathrm{E}\mathrm{M}\mathrm{A}\mathrm{P}-\mathrm{R}0_{-}100$

Strona 21 z27




\begin{center}
\includegraphics[width=192.840mm,height=294.948mm]{./F2_M_PR_M2021_page21_images/image001.eps}
\end{center}
Strona 22 z27

$\mathrm{E}\mathrm{M}\mathrm{A}\mathrm{P}-\mathrm{R}0_{-}100$




\begin{center}
\includegraphics[width=192.840mm,height=258.876mm]{./F2_M_PR_M2021_page22_images/image001.eps}
\end{center}
$\mathrm{O}\mathrm{d}\mathrm{p}\mathrm{o}\mathrm{w}\mathrm{i}\mathrm{e}\mathrm{d}\acute{\mathrm{z}}$:$\ldots\ldots\ldots\ldots\ldots\ldots\ldots\ldots\ldots\ldots\ldots\ldots\ldots\ldots\ldots\ldots\ldots\ldots\ldots\ldots\ldots\ldots\ldots\ldots\ldots\ldots\ldots\ldots\ldots\ldots\ldots\ldots\ldots\ldots\ldots\ldots\ldots\ldots\ldots\ldots\ldots\ldots$

Wypelnia

egzaminator

Nr zadania

Maks. liczba pkt

Uzyskana liczba pkt

14.

6

$\mathrm{E}\mathrm{M}\mathrm{A}\mathrm{P}-\mathrm{R}0_{-}100$

Strona 23 z27





Zadänie 15. $(0-7\displaystyle \int$

Pewien zaklad otrzymal zamówienie na wykonanie prostopadlościennego zbiornika

(calkowicie otwartego od góry) o pojemności 144 $\mathrm{m}^{3}$ Dno zbiornika ma byč kwadratem.

$\dot{\mathrm{Z}}$ aden z wymiarów zbiornika (krawedzi prostopadlościanu) nie $\mathrm{m}\mathrm{o}\dot{\mathrm{z}}\mathrm{e}$ przekraczač 9 metrów.

Calkowity koszt wykonania zbiornika ustalono w nastepujqcy sposób:

- 100 zl za l $\mathrm{m}^{2}$ dna

- 75 zl za l $\mathrm{m}^{2}$ ściany bocznej.

Oblicz wymiary zbiornika, dla którego tak ustalony koszt wykonania będzie najmniejszy.
\begin{center}
\includegraphics[width=192.840mm,height=228.852mm]{./F2_M_PR_M2021_page23_images/image001.eps}
\end{center}
$1^{\cdot}$

1

11

$1$

$\underline{!}$

1

$1 \rfloor 1$

$1$

$1$

1111

$+$

$\mathrm{f}-1$

1

$\iota$

$1$

IlIl

1

$1$

$\mathrm{I}11$

$\iota^{1}$

$1$

$-\mathrm{I}1$

$1$

$1-\mathrm{I}\mathrm{I}1$

1

$!1$

$\rfloor$

l

$\dagger$lI

t

-ll

l

1

$1 |$

$\mathrm{r} !$

$1 |$

1

$+^{1}11$

$1^{-}11$

$\mathrm{r}\mathrm{l}1$

$\lrcorner\rfloor$

$1$

$\lceil$

$\mathrm{J}1$

$\rfloor|1$

1

1

.llI :$\neg\urcorner|$

$\mathrm{r}1$

$\vdash$IlllIl--]$+\dashv\neg$

11

I

$\mathrm{i}$

T

l

$|$

ll

$|$

$|$

I

1l$|$

11

1

$1$

1

$1_{-} \mathrm{T}$

$1$

$\downarrow$

$1$

11

$11$

- $\mathrm{r}$

ll

$\}$

1

1

1

I

1

1 $\mathrm{L} \mathrm{T}$

1

Il

$1$

$\mathrm{i}$

1

Tl$|$

ll

11

1

11

$1$

$\mathrm{I}\mathrm{l}\mathrm{I}|$

ll$|$l

$1\displaystyle \frac{!}{1}1|\mathrm{I}1$

$\rightarrow$ll

$11$

1

1

1

1

$\mapsto 11$

ll

$1$

$1$ Il

$\iota_{1}$ -

$1 \mathrm{i}$

1

1

$1$

$\mathrm{J}^{1}$

$\lrcorner 11\neg 1$

$1$

í $\neg$llLI L$\dagger$I--$| |$lll f $\ulcorner$ -$|$lllll $\perp+$lIl t

$\dagger 1$

-$\dagger||$ lll--l.

I

$\mathrm{l}\mathrm{r}\ulcorner 1$

1

$1$ 1

$-1$

$+$

1

$11111$

11

$1^{-}$

$\mathrm{L}\mathrm{I}\mathrm{I}$

$1$

ll

11

$\ulcorner$

$\dashv 11-11\mathrm{I}1\urcorner$

1

$1$ 11

11

$1$

$\wedge 1$

Il

$-11$

ll

$1$

$1$

ll

11

$\urcorner$

1

$11$

$\downarrow$-

1

$\mathrm{t} |^{-}1$

1

$!$

1

$\leftarrow 11$

il

1

--ll$|$

$1$

$1$

$|^{-}1$

$\dot{\mathrm{i}}|111$

$\dagger$1 $1^{-}1 \leftarrow$

$\dagger$

1

1

$11$

$\rfloor\iota_{\mathrm{I}}$

$1$

1

$1$

1

ll

- $11$

1

$11$

$\dagger 1$

1

$\dashv 1$

$-111^{-}1$

11

$11$

$1111$

ll

1 $|^{-}1$

$1$

$1$

$\mathrm{t}^{1}$

LlI

1

I

ll

1

\{1

l

$1 1$

$11$

1

$1$ 11

11

$\leftarrow 1$

T

$| +$llll$\lrcorner$1$|$

$\}1$

$\rfloor 1$

$1$

$\rfloor$

-$||$

-1

1

$\llcorner 1 \rfloor$

Strona 24 z27

$\mathrm{E}\mathrm{M}\mathrm{A}\mathrm{P}-\mathrm{R}0_{-}100$




\begin{center}
\includegraphics[width=192.840mm,height=294.948mm]{./F2_M_PR_M2021_page24_images/image001.eps}
\end{center}
$\mathrm{E}\mathrm{M}\mathrm{A}\mathrm{P}-\mathrm{R}0_{-}100$

Strona 25 z27




\begin{center}
\includegraphics[width=192.840mm,height=258.876mm]{./F2_M_PR_M2021_page25_images/image001.eps}
\end{center}
$\mathrm{O}\mathrm{d}\mathrm{p}\mathrm{o}\mathrm{w}\mathrm{i}\mathrm{e}\mathrm{d}\acute{\mathrm{z}}$:$\ldots\ldots\ldots\ldots\ldots\ldots\ldots\ldots\ldots\ldots\ldots\ldots\ldots\ldots\ldots\ldots\ldots\ldots\ldots\ldots\ldots\ldots\ldots\ldots\ldots\ldots\ldots\ldots\ldots\ldots\ldots\ldots\ldots\ldots\ldots\ldots\ldots\ldots\ldots\ldots\ldots\ldots$

Wypelnia

egzaminator

Nr zadania

Maks. liczba pkt

Uzyskana liczba pkt

15.

7

Strona 26 z27

$\mathrm{E}\mathrm{M}\mathrm{A}\mathrm{P}-\mathrm{R}0_{-}100$





: RU DNOPIS

\{$m^{\bullet}\mathrm{e}$ {\it podlega ocenie}\}
\begin{center}
\includegraphics[width=192.840mm,height=276.960mm]{./F2_M_PR_M2021_page26_images/image001.eps}
\end{center}
$\mathrm{E}\mathrm{M}\mathrm{A}\mathrm{P}-\mathrm{R}0_{-}100$

Strona 27 z27










Zadänie 5. $\langle 0-2l$

Oblicz granice $\displaystyle \lim_{n\rightarrow\infty}\frac{(3n+2)^{2}-(1-2n)^{2}}{(2n-1)^{2}}$

W ponizsze kratki wpisz kolejno-od lewej do prawej- cyfre jedności i pierwsze dwie cyfry po

przecinku skończonego rozwiniecia dziesietnego otrzymanego wyniku.
\begin{center}
\includegraphics[width=25.452mm,height=12.240mm]{./F2_M_PR_M2021_page3_images/image001.eps}

\includegraphics[width=192.840mm,height=222.864mm]{./F2_M_PR_M2021_page3_images/image002.eps}
\end{center}
Strona 4 z27

$\mathrm{E}\mathrm{M}\mathrm{A}\mathrm{P}-\mathrm{R}0_{-}100$





$\mathrm{Z}\text{à} \mathrm{d}^{\backslash }\cdot \mathrm{a}\mathfrak{n}1\mathrm{e}61. \langle 0-3)^{\backslash }$

Niech log218 $= \mathrm{c}$. Wykaz, $\dot{\mathrm{z}}\mathrm{e}$

$\log_{3}4 =\displaystyle \frac{4}{\mathrm{c}-1}$
\begin{center}
\includegraphics[width=192.840mm,height=246.888mm]{./F2_M_PR_M2021_page4_images/image001.eps}

\begin{tabular}{|l|l|l|l|}
\cline{2-4}
&	\multicolumn{1}{|l|}{Nr zadania}&	\multicolumn{1}{|l|}{$5.$}&	\multicolumn{1}{|l|}{ $6.$}	\\
\cline{2-4}
&	\multicolumn{1}{|l|}{Maks. liczba pkt}&	\multicolumn{1}{|l|}{$2$}&	\multicolumn{1}{|l|}{ $3$}	\\
\cline{2-4}
\multicolumn{1}{|l|}{egzaminator}&	\multicolumn{1}{|l|}{Uzyskana liczba pkt}&	\multicolumn{1}{|l|}{}&	\multicolumn{1}{|l|}{}	\\
\hline
\end{tabular}

\end{center}
$\mathrm{E}\mathrm{M}\mathrm{A}\mathrm{P}-\mathrm{R}0_{-}100$

Strona 5 z27





$\mathrm{Z}\text{à} \mathrm{d}^{\backslash }\cdot \mathrm{a}\mathfrak{n}1\mathrm{e}_{1}7_{1}1. (0-3)^{\backslash }$

Rozwiqz nierównośč:

$\displaystyle \frac{2x-1}{1-x}\leq\frac{2+2x}{5x}$
\begin{center}
\includegraphics[width=192.840mm,height=264.924mm]{./F2_M_PR_M2021_page5_images/image001.eps}
\end{center}
Strona 6 z27

$\mathrm{E}\mathrm{M}\mathrm{A}\mathrm{P}-\mathrm{R}0_{-}100$




\begin{center}
\includegraphics[width=192.840mm,height=258.876mm]{./F2_M_PR_M2021_page6_images/image001.eps}
\end{center}
$\mathrm{O}\mathrm{d}\mathrm{p}\mathrm{o}\mathrm{w}\mathrm{i}\mathrm{e}\mathrm{d}\acute{\mathrm{z}}$:$\ldots\ldots\ldots\ldots\ldots\ldots\ldots\ldots\ldots\ldots\ldots\ldots\ldots\ldots\ldots\ldots\ldots\ldots\ldots\ldots\ldots\ldots\ldots\ldots\ldots\ldots\ldots\ldots\ldots\ldots\ldots\ldots\ldots\ldots\ldots\ldots\ldots\ldots\ldots\ldots\ldots\ldots$

Wypelnia

egzaminator

Nr zadania

Maks. liczba pkt

Uzyskana liczba pkt

7.

3

$\mathrm{E}\mathrm{M}\mathrm{A}\mathrm{P}-\mathrm{R}0_{-}100$

Strona 7 z27





$\mathrm{Z}\text{à} \mathrm{d}^{\backslash }\cdot \mathrm{a}\mathfrak{n}1\mathrm{e}1$ @. $(0-3)^{\backslash }$

Danyjest trójkqt równoboczny $ABC$. Na bokach AB $\mathrm{i} AC$ wybrano punkty- odpowiednio-

{\it D} $\mathrm{i} E$ takie, $\dot{\mathrm{z}}\mathrm{e} |BD| = |AE| =\displaystyle \frac{1}{3}|AB|$. Odcinki CD $\mathrm{i}$ BE przecinajq si9 w punkcie $P$

(zobacz rysunek).

{\it C}
\begin{center}
\includegraphics[width=72.852mm,height=62.784mm]{./F2_M_PR_M2021_page7_images/image001.eps}
\end{center}
{\it E}

{\it P}

{\it A}

{\it D}

{\it B}

Wykaz, $\dot{\mathrm{z}}\mathrm{e}$ pole trójkata

{\it DBP}

jest 21

razy mniejsze od pola trójkqta

{\it ABC}.
\begin{center}
\includegraphics[width=192.840mm,height=162.708mm]{./F2_M_PR_M2021_page7_images/image002.eps}
\end{center}
Strona 8 z27

$\mathrm{E}\mathrm{M}\mathrm{A}\mathrm{P}-\mathrm{R}0_{-}100$




\begin{center}
\includegraphics[width=192.840mm,height=270.912mm]{./F2_M_PR_M2021_page8_images/image001.eps}
\end{center}
Wypelnia

egzaminator

Nr zadania

Maks. liczba pkt

Uzyskana liczba pkt

8.

3

$\mathrm{E}\mathrm{M}\mathrm{A}\mathrm{P}-\mathrm{R}0_{-}100$

Strona 9 z27





$\mathrm{Z}\text{à} \mathrm{d}^{\backslash }\cdot \mathrm{a}\mathfrak{n}\mathrm{i}_{1}\mathrm{e}1\mathrm{g}1. (0-\backslash 4)^{\backslash }\backslash \cdot$

Ze zbioru wszystkich

liczb naturalnych

czterocyfrowych losujemy

jednq liczb9.

Oblicz

prawdopodobieństwo zdarzenia polegajqcego na tym, $\dot{\mathrm{z}}\mathrm{e}$ wylosowana liczba jest podzielna

przez

15, jeśli wiadomo, $\dot{\mathrm{z}}\mathrm{e}$ jest ona podzielna przez 18.
\begin{center}
\includegraphics[width=192.840mm,height=264.924mm]{./F2_M_PR_M2021_page9_images/image001.eps}
\end{center}
Strona 10 z27

$\mathrm{E}\mathrm{M}\mathrm{A}\mathrm{P}-\mathrm{R}0_{-}100$



\end{document}