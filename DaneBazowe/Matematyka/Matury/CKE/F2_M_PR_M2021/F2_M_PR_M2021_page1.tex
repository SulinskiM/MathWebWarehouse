\documentclass[a4paper,12pt]{article}
\usepackage{latexsym}
\usepackage{amsmath}
\usepackage{amssymb}
\usepackage{graphicx}
\usepackage{wrapfig}
\pagestyle{plain}
\usepackage{fancybox}
\usepackage{bm}

\begin{document}

{\it W kazdym z zadań od f. do 4. wybierz i zaznacz na karcie odpowiedzi poprawnq odpowiedz}'.

Zadanie 1. (0-1)

Róznica $\cos^{2}165^{\mathrm{o}}-\sin^{2}165^{\mathrm{o}}$ jest równa

A. $-1$

B. $-\displaystyle \frac{\sqrt{3}}{2}$

C. - -21

D. $\displaystyle \frac{\sqrt{3}}{2}$

Zadanie 2. $\{0-l\mathrm{I}$

Na rysunku przedstawiono fragment

rzeczywistej $x.$

wykresu funkcji

f określonej

dla $\mathrm{k}\mathrm{a}\dot{\mathrm{z}}$ dej liczby

Jeden spośród podanych ponizej wzorów jest wzorem tej funkcji. Wskaz wzór funkcji f.

A. $f(x)=\displaystyle \frac{\cos x+1}{|\cos x|+1}$

B. $f(x)=\displaystyle \frac{\sin x+1}{|\sin x|+1}$

C. $f(x)=\displaystyle \frac{|\cos x|-2}{\cos x-2}$

D. $f(x)=\displaystyle \frac{|\sin x|-2}{\sin x-2}$

Zadanie 3. (0-1)

Wielomian $W(x)=x^{4}+81$ jest podzielny przez

A. $x-3$

B. $x^{2}+9$

C. $x^{2}-3\sqrt{2}x+9$

D. $x^{2}+3\sqrt{2}x-9$

Zadanie 4. (0-1)

Liczba róznych pierwiastków równania $3x+|x-4|=0$ jest równa

A. 0

B. l

C. 2

D. 3

Strona 2 z27

$\mathrm{E}\mathrm{M}\mathrm{A}\mathrm{P}-\mathrm{R}0_{-}100$
\end{document}
