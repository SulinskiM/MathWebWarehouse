\documentclass[a4paper,12pt]{article}
\usepackage{latexsym}
\usepackage{amsmath}
\usepackage{amssymb}
\usepackage{graphicx}
\usepackage{wrapfig}
\pagestyle{plain}
\usepackage{fancybox}
\usepackage{bm}

\begin{document}

$\mathrm{Z}\text{à} \mathrm{d}^{\backslash }\cdot \mathrm{a}\mathfrak{n}1\mathrm{e}10. 1(0-4l1$

Prosta przechodzqca przez

punkty $A = (8,-6)$

i

$B = (5,15)$ jest styczna

do

okregu

o środku w punkcie $0 =$

(0,0). Oblicz promień tego okregu i wspólrzedne punktu styczności tego

okrpgu z prostq

{\it AB}.
\begin{center}
\includegraphics[width=192.840mm,height=258.876mm]{./F2_M_PR_M2021_page11_images/image001.eps}
\end{center}
Strona 12 z27

$\mathrm{E}\mathrm{M}\mathrm{A}\mathrm{P}-\mathrm{R}0_{-}100$
\end{document}
