\documentclass[a4paper,12pt]{article}
\usepackage{latexsym}
\usepackage{amsmath}
\usepackage{amssymb}
\usepackage{graphicx}
\usepackage{wrapfig}
\pagestyle{plain}
\usepackage{fancybox}
\usepackage{bm}

\begin{document}

$\mathrm{Z}\text{à} \mathrm{d}^{\backslash }\cdot \mathrm{a}\mathrm{n}1\mathrm{e}1 13^{\backslash }. (0-4$

Dany jest trójkqt prostokqtny

{\it ABC}.

Promień okregu wpisanego w ten trójkqt jest pieć razy

krótszy od przeciwprostokqtnej tego trójkqta. Oblicz sinus tego z kqtów ostrych trójkqta

{\it ABC},

który ma wipkszq miar9.
\begin{center}
\includegraphics[width=192.840mm,height=258.924mm]{./F2_M_PR_M2021_page17_images/image001.eps}
\end{center}
Strona 18 z27

$\mathrm{E}\mathrm{M}\mathrm{A}\mathrm{P}-\mathrm{R}0_{-}100$
\end{document}
