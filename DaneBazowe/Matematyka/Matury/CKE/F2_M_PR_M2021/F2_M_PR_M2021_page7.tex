\documentclass[a4paper,12pt]{article}
\usepackage{latexsym}
\usepackage{amsmath}
\usepackage{amssymb}
\usepackage{graphicx}
\usepackage{wrapfig}
\pagestyle{plain}
\usepackage{fancybox}
\usepackage{bm}

\begin{document}

$\mathrm{Z}\text{à} \mathrm{d}^{\backslash }\cdot \mathrm{a}\mathfrak{n}1\mathrm{e}1$ @. $(0-3)^{\backslash }$

Danyjest trójkqt równoboczny $ABC$. Na bokach AB $\mathrm{i} AC$ wybrano punkty- odpowiednio-

{\it D} $\mathrm{i} E$ takie, $\dot{\mathrm{z}}\mathrm{e} |BD| = |AE| =\displaystyle \frac{1}{3}|AB|$. Odcinki CD $\mathrm{i}$ BE przecinajq si9 w punkcie $P$

(zobacz rysunek).

{\it C}
\begin{center}
\includegraphics[width=72.852mm,height=62.784mm]{./F2_M_PR_M2021_page7_images/image001.eps}
\end{center}
{\it E}

{\it P}

{\it A}

{\it D}

{\it B}

Wykaz, $\dot{\mathrm{z}}\mathrm{e}$ pole trójkata

{\it DBP}

jest 21

razy mniejsze od pola trójkqta

{\it ABC}.
\begin{center}
\includegraphics[width=192.840mm,height=162.708mm]{./F2_M_PR_M2021_page7_images/image002.eps}
\end{center}
Strona 8 z27

$\mathrm{E}\mathrm{M}\mathrm{A}\mathrm{P}-\mathrm{R}0_{-}100$
\end{document}
