\documentclass[a4paper,12pt]{article}
\usepackage{latexsym}
\usepackage{amsmath}
\usepackage{amssymb}
\usepackage{graphicx}
\usepackage{wrapfig}
\pagestyle{plain}
\usepackage{fancybox}
\usepackage{bm}

\begin{document}

$\mathrm{Z}\text{à} \mathrm{d}^{\backslash }\cdot \mathrm{a}\mathfrak{n}\mathrm{i}_{1}\mathrm{e}1\mathrm{g}1. (0-\backslash 4)^{\backslash }\backslash \cdot$

Ze zbioru wszystkich

liczb naturalnych

czterocyfrowych losujemy

jednq liczb9.

Oblicz

prawdopodobieństwo zdarzenia polegajqcego na tym, $\dot{\mathrm{z}}\mathrm{e}$ wylosowana liczba jest podzielna

przez

15, jeśli wiadomo, $\dot{\mathrm{z}}\mathrm{e}$ jest ona podzielna przez 18.
\begin{center}
\includegraphics[width=192.840mm,height=264.924mm]{./F2_M_PR_M2021_page9_images/image001.eps}
\end{center}
Strona 10 z27

$\mathrm{E}\mathrm{M}\mathrm{A}\mathrm{P}-\mathrm{R}0_{-}100$
\end{document}
