\documentclass[a4paper,12pt]{article}
\usepackage{latexsym}
\usepackage{amsmath}
\usepackage{amssymb}
\usepackage{graphicx}
\usepackage{wrapfig}
\pagestyle{plain}
\usepackage{fancybox}
\usepackage{bm}

\begin{document}

CENTRALNA

KOMISJA

EGZAMINACYJNA

Arkusz zawiera informacje prawnie chronione

do momentu rozpoczecia egzaminu.

WYPELNIA ZDAJACY

{\it Miejsce na naklejke}.

{\it Sprawdz}', {\it czy kod na naklejce to}

e-100.
\begin{center}
\includegraphics[width=21.900mm,height=16.260mm]{./F2_M_PR_M2022_page0_images/image001.eps}
\end{center}
KOD
\begin{center}
\includegraphics[width=79.656mm,height=16.260mm]{./F2_M_PR_M2022_page0_images/image002.eps}
\end{center}
PESEL

{\it Jezeli tak}- {\it przyklej naklejkq}.

{\it Jezeli nie}- {\it zgtoś to nauczycielowi}.

EGZAMIN MATURALNY Z MATEMATYKI

POZIOM ROZSZERZONY

{\it wr}PELNlA ZESPÓL $\mathrm{N}\mathrm{A}\mathrm{D}\mathrm{Z}\mathrm{O}\mathrm{R}\mathrm{U}\mathrm{J}\wedge \mathrm{C}Y$

DATA: \{$\{$ maja 2022 $\mathrm{r}.$

GODZINA ROZPOCZGCIA: 9: 00

CZAS PRACY: $\{80 \displaystyle \min \mathrm{u}\mathrm{t}$

LICZBA PUNKTÓW DO UZYSKANIA: 50

Uprawnienia zdaj\S cego do:

\fbox{} nieprzenoszenia zaznaczeń na karte

\fbox{} dostosowania zasad oceniania

\fbox{} dostosowania w zw. z dyskalkuliq.

$\Vert\Vert\Vert\Vert\Vert\Vert\Vert\Vert\Vert\Vert\Vert\Vert\Vert\Vert\Vert\Vert\Vert\Vert\Vert\Vert\Vert\Vert\Vert\Vert\Vert\Vert\Vert\Vert\Vert\Vert|$

EMAP-R0-100-2205

lnstrukcja dla zdajqcego

l. Sprawdz', czy arkusz egzaminacyjny zawiera 26 stron (zadania $1-15$).

Ewentualny brak zgloś przewodniczacemu zespolu nadzorujacego egzamin.

2. Na tej stronie oraz na karcie odpowiedzi wpisz swój numer PESEL i przyklej naklejke

z kodem.

3. Nie wpisuj $\dot{\mathrm{z}}$ adnych znaków w cz9ści przeznaczonej d1a egzaminatora.

4. Rozwiqzania zadań i odpowiedzi wpisuj w miejscu na to przeznaczonym.

5. Odpowiedzi do zadań zamknietych $(1-4)$ zaznacz na karcie odpowiedzi w cześci karty

przeznaczonej dla zdajqcego. Zamaluj $\blacksquare$ pola do tego przeznaczone. $\mathrm{B}_{9}\mathrm{d}\mathrm{n}\mathrm{e}$

zaznaczenie otocz kólkiem \copyright i zaznacz wlaściwe.

6. $\mathrm{W}$ zadaniu 5. wpisz odpowiednie cyfry w kratki pod treściq zadania.

7. Pamietaj, $\dot{\mathrm{z}}\mathrm{e}$ pominiecie argumentacji lub istotnych obliczeń w rozwiqzaniu zadania

otwartego (6-15) $\mathrm{m}\mathrm{o}\dot{\mathrm{z}}\mathrm{e}$ spowodowač, $\dot{\mathrm{z}}\mathrm{e}$ za to rozwiqzanie nie otrzymasz pelnej liczby

punktów.

8. Pisz czytelnie i $\mathrm{u}\dot{\mathrm{z}}$ ywaj tylko dlugopisu lub pióra z czarnym tuszem lub atramentem.

9. Nie $\mathrm{u}\dot{\mathrm{z}}$ ywaj korektora, a blędne zapisy wyraz'nie przekreśl.

10. Pamietaj, $\dot{\mathrm{z}}\mathrm{e}$ zapisy w brudnopisie nie bpdq oceniane.

11. $\mathrm{M}\mathrm{o}\dot{\mathrm{z}}$ esz korzystač z zestawu wzorów matematycznych, cyrkla i linijki oraz kalkulatora

prostego.

Uk\}ad graficzny

\copyright CKE 2021




{\it W kazdym z zadań od f. do 4. wybierz i zaznacz na karcie odpowiedzi poprawnq odpowiedz}'.

Zadanie l. $(0-1$\}

Liczba $\log_{3}\sqrt{27}-\log_{27}\sqrt{3}$ jest równa

A. -43

B. -21

C. $\displaystyle \frac{11}{12}$

D. 3

Zadanie 2. $\{0-1$)

Funkcja $f$ jest określona wzorem $f(x)=\displaystyle \frac{x^{3}-8}{x-2}$ dla $\mathrm{k}\mathrm{a}\dot{\mathrm{z}}$ dej liczby rzeczywistej $x\neq 2.$

Wartośč pochodnej tej funkcji dla argumentu $x=\displaystyle \frac{1}{2}$ jest równa

A. -43

B. -49

C. 3

D. $\displaystyle \frac{54}{8}$

Zadanie 3. $\langle 0-1$)

$\mathrm{J}\mathrm{e}\dot{\mathrm{z}}$ eli $\displaystyle \cos\beta=-\frac{1}{3} \mathrm{i} \displaystyle \beta\in(\pi,\frac{3}{2}\pi)$, to wartośč wyrazenia $\displaystyle \sin(\beta-\frac{1}{3}\pi)$ jest równa

A. $\displaystyle \frac{-2\sqrt{2}+\sqrt{3}}{6}$

B. $\displaystyle \frac{2\sqrt{6}+1}{6}$

C. $\displaystyle \frac{2\sqrt{2}+\sqrt{3}}{6}$

D. $\displaystyle \frac{1-2\sqrt{6}}{6}$

Zadanie 4. (0-1)

Dane sa dwie urny z kulami. $\mathrm{W}\mathrm{k}\mathrm{a}\dot{\mathrm{z}}$ dej z urn jest siedem kul. $\mathrm{W}$ pierwszej urnie sq jedna kula

biala i sześč kul czarnych, w drugiej urnie sa cztery kule biale i trzy kule czarne.

Rzucamyjeden raz symetryczna moneta. $\mathrm{J}\mathrm{e}\dot{\mathrm{z}}$ eli wypadnie reszka, to losujemyjedna kule

z pierwszej urny, w przeciwnym przypadku-jedna $\mathrm{k}\mathrm{u}\mathrm{l}9$ z drugiej urny.

Prawdopodobieństwo zdarzenia polegajqcego na tym, $\dot{\mathrm{z}}\mathrm{e}$ wylosujemy ku19 bia1a w tym

doświadczeniu, jest równe

A. $\displaystyle \frac{5}{14}$

B. $\displaystyle \frac{9}{14}$

C. -75

D. -67

Strona 2 z26

$\mathrm{E}\mathrm{M}\mathrm{A}\mathrm{P}-\mathrm{R}0_{-}100$





Wypelnia

egzaminator

Nr zadania

Maks. liczba pkt

Uzyskana liczba pkt

9.

4

-RO-100

Strona ll z26





Zadanie $10_{\mathrm{h}}\{0-4$)

Ciqg $(a_{n})$, określony dla $\mathrm{k}\mathrm{a}\dot{\mathrm{z}}$ dej liczby naturalnej $n\geq 1$, jest geometryczny i ma wszystkie

wyrazy dodatnie. Ponadto $a_{1}=675 \mathrm{i} a_{22}=\displaystyle \frac{5}{4}a_{23}+\frac{1}{5}a_{21}.$

Ciqg $(b_{n})$, określony dla $\mathrm{k}\mathrm{a}\dot{\mathrm{z}}$ dej liczby naturalnej $n\geq 1$, jest arytmetyczny.

Suma wszystkich wyrazów ciqgu $(a_{n})$ jest równa sumie dwudziestu pipciu poczqtkowych

kolejnych wyrazów ciqgu $(b_{n})$. Ponadto $a_{3}=b_{4}$. Oblicz $b_{1}.$

Strona 12 z26

$\mathrm{E}\mathrm{M}\mathrm{A}\mathrm{P}-\mathrm{R}0_{-}100$





Wypelnia

egzaminator

Nr zadania

Maks. liczba pkt

Uzyskana liczba pkt

10.

4

-RO-100

Strona 13 z26





Zadanie ll. $\{0-4$)

Rozwiqz równanie $\sin x+\sin 2x+\sin 3x=0$ w przedziale $\langle 0, \pi\rangle.$

Strona 14 z26

$\mathrm{E}\mathrm{M}\mathrm{A}\mathrm{P}-\mathrm{R}0_{-}10$





Wypelnia

egzaminator

Nr zadania

Maks. liczba pkt

Uzyskana liczba pkt

11.

4

-RO-100

Strona 15 z26





Zadarie 12. $\{0-5$)

Wyznacz wszystkie wartości parametru $m$, dla których równanie

$x^{2}-(m+1)x+m=0$

ma dwa rózne rozwiqzania rzeczywiste $x_{1}$ oraz $x_{2}, \mathrm{s}\mathrm{p}\mathrm{e}$niajqce warunki:

$\chi_{1}\neq 0, \chi_{2}\neq 0$

oraz

$\displaystyle \frac{1}{\chi_{1}}+\frac{1}{\chi_{2}}+2=\frac{1}{x_{1}^{2}}+\frac{1}{x_{2}^{2}}$

Strona 16 z26

$\mathrm{E}\mathrm{M}\mathrm{A}\mathrm{P}-\mathrm{R}0_{-}10$





Wypelnia

egzaminator

Nr zadania

Maks. liczba pkt

Uzyskana liczba pkt

12.

5

-RO-100

Strona 17 z26





Zadanie 13. $(0-5$\}

Danyjest $\mathrm{g}\mathrm{r}\mathrm{a}\mathrm{n}\mathrm{i}\mathrm{a}\mathrm{s}\mathrm{t}\mathrm{o}\mathrm{s}\dagger \mathrm{u}\mathrm{p}$ prosty ABCDEFGH o podstawie prostokqtnej ABCD. Przekatne

$AH \mathrm{i} AF$ ścian bocznych tworzq kqt ostry o mierze $\alpha$ takiej, $\dot{\mathrm{z}}\mathrm{e} \displaystyle \sin\alpha=\frac{12}{13}$ (zobacz

rysunek). Pole trójk ta $AFH$ jest równe 26,4

Oblicz wysokośč $h$ tego graniastoslupa.
\begin{center}
\includegraphics[width=60.660mm,height=83.316mm]{./F2_M_PR_M2022_page17_images/image001.eps}
\end{center}
{\it G}

I

I

I

I

{\it H} $1 E$

I

I

I

I

I

I

I

{\it F}

I

I

I

I

$--- B$

$\alpha$

{\it D  A}

{\it h}

Strona 18 z26

$\mathrm{E}\mathrm{M}\mathrm{A}\mathrm{P}-\mathrm{R}0_{-}100$





Wypelnia

egzaminator

Nr zadania

Maks. liczba pkt

Uzyskana liczba pkt

13.

5

-RO-100

Strona 19 z26





Zadarie 14. (0-6)

Punkt $A=(-3,2)$ jest wierzcholkiem trójkata równoramiennego $ABC$, w którym $|AC|=|BC|.$

Pole tego trójkqta jest równe 15. Bok $BC$ zawarty jest w prostej o równaniu $y=x-1.$

Oblicz wspólrzedne wierzcholków $B \mathrm{i} C$ tego trójkata.

Strona 20 z26

$\mathrm{E}\mathrm{M}\mathrm{A}\mathrm{P}-\mathrm{R}0_{-}100$





: {\it RU DNOPIS} \{{\it nie podlega ocenie}\}

$\mathrm{h}\mathrm{P}-\mathrm{R}0_{-}100$

Strona 3z 26





Wypelnia

egzaminator

Nr zadania

Maks. liczba pkt

Uzyskana liczba pkt

14.

6

-RO-100

Strona 21 z26





Zadarie 15. $\{0-7\}$

Rozpatrujemy wszystkie trójkqty równoramienne o obwodzie równym 18.

a) Wykaz, $\dot{\mathrm{z}}\mathrm{e}$ pole $P \mathrm{k}\mathrm{a}\dot{\mathrm{z}}$ dego z tych trójkqtów, jako funkcja dlugości $b$ ramienia, wyraza si9

wzorem $P(b)=\displaystyle \frac{(18-2b)\cdot\sqrt{18b-81}}{2}$

b) Wyznacz dziedzin9 funkcji P.

c) Oblicz dlugości boków tego z rozpatrywanych trójkatów, który ma najwipksze pole.

Strona 22 z26

$\mathrm{E}\mathrm{M}\mathrm{A}\mathrm{P}-\mathrm{R}0_{-}100$





$1)0_{-}100$

Strona 23 z26





Wypelnia

egzaminator

Nr zadania

Maks. liczba pkt

Uzyskana liczba pkt

15.

7

Strona 24 z26

$\mathrm{E}\mathrm{M}\mathrm{A}\mathrm{P}-\mathrm{R}0_{-}10$





: {\it RU DNOPIS} \{{\it nie podlega ocenie}\}

$\mathrm{h}\mathrm{P}-\mathrm{R}0_{-}100$

Strona 25 z26





Strona 26 z26

$\mathrm{E}\mathrm{M}\mathrm{A}\mathrm{P}-\mathrm{R}0_{-}10$















$\mathrm{Z}\mathrm{a}\mathrm{d}\mathrm{a}*\mathrm{i}\mathrm{e}5. (0-2\}$

Ciqg $(a_{n})$ jest określony dla $\mathrm{k}\mathrm{a}\dot{\mathrm{z}}\mathrm{d}\mathrm{e}\mathrm{j}$ liczby naturalnej $n\geq 1$ wzorem $a_{n}=\displaystyle \frac{(7p-1)n^{3}+5pn-3}{(p+1)n^{3}+n^{2}+p}$

gdzie $p$ jest liczbq rzeczywistq dodatniq.

Oblicz wartośč $p$, dla której granica ciagu $(a_{n})$ jest równa $\displaystyle \frac{4}{3}$

W ponizsze kratki wpisz kolejno-od lewej do prawej-pierwsza, drugq oraz trzeciq cyfre po

przecinku nieskończonego rozwiniecia dziesiptnego otrzymanego wyniku.
\begin{center}
\includegraphics[width=25.452mm,height=12.240mm]{./F2_M_PR_M2022_page3_images/image001.eps}
\end{center}
: {\it RU DNOPIS} \{{\it nie podlega ocenie}\}

Strona 4 z26

$\mathrm{E}\mathrm{M}\mathrm{A}\mathrm{P}-\mathrm{R}0_{-}100$





Zadarie 6. $\{0-3\}$

Wykaz, $\dot{\mathrm{z}}\mathrm{e}$ dla $\mathrm{k}\mathrm{a}\dot{\mathrm{z}}$ dej liczby rzeczywistej $x$ i dla $\mathrm{k}\mathrm{a}\dot{\mathrm{z}}$ dej liczby rzeczywistej $y$ takich, $\dot{\mathrm{z}}\mathrm{e}$

$2x>\mathrm{y}$, spelniona jest nierównośč

$7x^{3}+4x^{2}y\geq y^{3}+2xy^{2}-x^{3}$
\begin{center}
\begin{tabular}{|l|l|l|l|}
\cline{2-4}
&	\multicolumn{1}{|l|}{Nr zadania}&	\multicolumn{1}{|l|}{$5.$}&	\multicolumn{1}{|l|}{ $6.$}	\\
\cline{2-4}
&	\multicolumn{1}{|l|}{Maks. liczba pkt}&	\multicolumn{1}{|l|}{$2$}&	\multicolumn{1}{|l|}{ $3$}	\\
\cline{2-4}
\multicolumn{1}{|l|}{egzaminator}&	\multicolumn{1}{|l|}{Uzyskana liczba pkt}&	\multicolumn{1}{|l|}{}&	\multicolumn{1}{|l|}{}	\\
\hline
\end{tabular}

\end{center}
$\mathrm{E}\mathrm{M}\mathrm{A}\mathrm{P}-\mathrm{R}0_{-}100$

Strona 5 z26





Zadarie 7. $\{0-3\}$

Rozwiqz równanie:

$|x-3|=2x+11$

Strona 6 z26

$\mathrm{E}\mathrm{M}\mathrm{A}\mathrm{P}-\mathrm{R}0_{-}10$





Wypelnia

egzaminator

Nr zadania

Maks. liczba pkt

Uzyskana liczba pkt

7.

3

-RO-100

Strona 7 z26





Zadarie 8. $\{0-3\}$

Punkt $P$ jest punktem $\mathrm{p}\mathrm{r}\mathrm{z}\mathrm{e}\mathrm{c}\mathrm{i}_{9}\mathrm{c}\mathrm{i}\mathrm{a}$ przekqtnych trapezu ABCD. Dlugośč podstawy $CD$ jest

$0 2$ mniejsza od dlugości podstawy $AB$. Promień okregu opisanego na trójkacie

ostrokatnvm $CPD$ jest o 3 mniejszy od promienia okregu opisanego na trójkqcie $APB.$

Wykaz, $\dot{\mathrm{z}}\mathrm{e}$ spelnionyjest warunek $|DP|^{2}+|CP|^{2}-|CD|^{2}=\displaystyle \frac{4\sqrt{2}}{3}\cdot|DP| |CP|.$

Strona 8 z26

$\mathrm{E}\mathrm{M}\mathrm{A}\mathrm{P}-\mathrm{R}0_{-}100$





Wypelnia

egzaminator

Nr zadania

Maks. liczba pkt

Uzyskana liczba pkt

8.

3

-RO-100

Strona 9 z26





Zadanie $\mathrm{g}. \{0-4$)

Reszta z dzielenia wielomianu $W(x)=4x^{3}-6x^{2}-(5m+1)x-2m$ przez dwumian $x+2$

jest równa $(-30).$

Oblicz $m$ i dla wyznaczonej wartości $m$ rozwiqz nierównośč $W(x)\geq 0.$

Strona 10 z26

$\mathrm{E}\mathrm{M}\mathrm{A}\mathrm{P}-\mathrm{R}0_{-}100$



\end{document}