\documentclass[a4paper,12pt]{article}
\usepackage{latexsym}
\usepackage{amsmath}
\usepackage{amssymb}
\usepackage{graphicx}
\usepackage{wrapfig}
\pagestyle{plain}
\usepackage{fancybox}
\usepackage{bm}

\begin{document}

{\it W kazdym z zadań od f. do 4. wybierz i zaznacz na karcie odpowiedzi poprawnq odpowiedz}'.

Zadanie l. $(0-1$\}

Liczba $\log_{3}\sqrt{27}-\log_{27}\sqrt{3}$ jest równa

A. -43

B. -21

C. $\displaystyle \frac{11}{12}$

D. 3

Zadanie 2. $\{0-1$)

Funkcja $f$ jest określona wzorem $f(x)=\displaystyle \frac{x^{3}-8}{x-2}$ dla $\mathrm{k}\mathrm{a}\dot{\mathrm{z}}$ dej liczby rzeczywistej $x\neq 2.$

Wartośč pochodnej tej funkcji dla argumentu $x=\displaystyle \frac{1}{2}$ jest równa

A. -43

B. -49

C. 3

D. $\displaystyle \frac{54}{8}$

Zadanie 3. $\langle 0-1$)

$\mathrm{J}\mathrm{e}\dot{\mathrm{z}}$ eli $\displaystyle \cos\beta=-\frac{1}{3} \mathrm{i} \displaystyle \beta\in(\pi,\frac{3}{2}\pi)$, to wartośč wyrazenia $\displaystyle \sin(\beta-\frac{1}{3}\pi)$ jest równa

A. $\displaystyle \frac{-2\sqrt{2}+\sqrt{3}}{6}$

B. $\displaystyle \frac{2\sqrt{6}+1}{6}$

C. $\displaystyle \frac{2\sqrt{2}+\sqrt{3}}{6}$

D. $\displaystyle \frac{1-2\sqrt{6}}{6}$

Zadanie 4. (0-1)

Dane sa dwie urny z kulami. $\mathrm{W}\mathrm{k}\mathrm{a}\dot{\mathrm{z}}$ dej z urn jest siedem kul. $\mathrm{W}$ pierwszej urnie sq jedna kula

biala i sześč kul czarnych, w drugiej urnie sa cztery kule biale i trzy kule czarne.

Rzucamyjeden raz symetryczna moneta. $\mathrm{J}\mathrm{e}\dot{\mathrm{z}}$ eli wypadnie reszka, to losujemyjedna kule

z pierwszej urny, w przeciwnym przypadku-jedna $\mathrm{k}\mathrm{u}\mathrm{l}9$ z drugiej urny.

Prawdopodobieństwo zdarzenia polegajqcego na tym, $\dot{\mathrm{z}}\mathrm{e}$ wylosujemy ku19 bia1a w tym

doświadczeniu, jest równe

A. $\displaystyle \frac{5}{14}$

B. $\displaystyle \frac{9}{14}$

C. -75

D. -67

Strona 2 z26

$\mathrm{E}\mathrm{M}\mathrm{A}\mathrm{P}-\mathrm{R}0_{-}100$
\end{document}
