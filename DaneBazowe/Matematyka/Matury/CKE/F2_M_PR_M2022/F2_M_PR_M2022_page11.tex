\documentclass[a4paper,12pt]{article}
\usepackage{latexsym}
\usepackage{amsmath}
\usepackage{amssymb}
\usepackage{graphicx}
\usepackage{wrapfig}
\pagestyle{plain}
\usepackage{fancybox}
\usepackage{bm}

\begin{document}

Zadanie $10_{\mathrm{h}}\{0-4$)

Ciqg $(a_{n})$, określony dla $\mathrm{k}\mathrm{a}\dot{\mathrm{z}}$ dej liczby naturalnej $n\geq 1$, jest geometryczny i ma wszystkie

wyrazy dodatnie. Ponadto $a_{1}=675 \mathrm{i} a_{22}=\displaystyle \frac{5}{4}a_{23}+\frac{1}{5}a_{21}.$

Ciqg $(b_{n})$, określony dla $\mathrm{k}\mathrm{a}\dot{\mathrm{z}}$ dej liczby naturalnej $n\geq 1$, jest arytmetyczny.

Suma wszystkich wyrazów ciqgu $(a_{n})$ jest równa sumie dwudziestu pipciu poczqtkowych

kolejnych wyrazów ciqgu $(b_{n})$. Ponadto $a_{3}=b_{4}$. Oblicz $b_{1}.$

Strona 12 z26

$\mathrm{E}\mathrm{M}\mathrm{A}\mathrm{P}-\mathrm{R}0_{-}100$
\end{document}
