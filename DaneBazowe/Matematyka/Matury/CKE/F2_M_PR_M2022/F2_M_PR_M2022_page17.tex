\documentclass[a4paper,12pt]{article}
\usepackage{latexsym}
\usepackage{amsmath}
\usepackage{amssymb}
\usepackage{graphicx}
\usepackage{wrapfig}
\pagestyle{plain}
\usepackage{fancybox}
\usepackage{bm}

\begin{document}

Zadanie 13. $(0-5$\}

Danyjest $\mathrm{g}\mathrm{r}\mathrm{a}\mathrm{n}\mathrm{i}\mathrm{a}\mathrm{s}\mathrm{t}\mathrm{o}\mathrm{s}\dagger \mathrm{u}\mathrm{p}$ prosty ABCDEFGH o podstawie prostokqtnej ABCD. Przekatne

$AH \mathrm{i} AF$ ścian bocznych tworzq kqt ostry o mierze $\alpha$ takiej, $\dot{\mathrm{z}}\mathrm{e} \displaystyle \sin\alpha=\frac{12}{13}$ (zobacz

rysunek). Pole trójk ta $AFH$ jest równe 26,4

Oblicz wysokośč $h$ tego graniastoslupa.
\begin{center}
\includegraphics[width=60.660mm,height=83.316mm]{./F2_M_PR_M2022_page17_images/image001.eps}
\end{center}
{\it G}

I

I

I

I

{\it H} $1 E$

I

I

I

I

I

I

I

{\it F}

I

I

I

I

$--- B$

$\alpha$

{\it D  A}

{\it h}

Strona 18 z26

$\mathrm{E}\mathrm{M}\mathrm{A}\mathrm{P}-\mathrm{R}0_{-}100$
\end{document}
