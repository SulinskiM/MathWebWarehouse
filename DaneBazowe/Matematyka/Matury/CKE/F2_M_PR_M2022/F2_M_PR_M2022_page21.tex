\documentclass[a4paper,12pt]{article}
\usepackage{latexsym}
\usepackage{amsmath}
\usepackage{amssymb}
\usepackage{graphicx}
\usepackage{wrapfig}
\pagestyle{plain}
\usepackage{fancybox}
\usepackage{bm}

\begin{document}

Zadarie 15. $\{0-7\}$

Rozpatrujemy wszystkie trójkqty równoramienne o obwodzie równym 18.

a) Wykaz, $\dot{\mathrm{z}}\mathrm{e}$ pole $P \mathrm{k}\mathrm{a}\dot{\mathrm{z}}$ dego z tych trójkqtów, jako funkcja dlugości $b$ ramienia, wyraza si9

wzorem $P(b)=\displaystyle \frac{(18-2b)\cdot\sqrt{18b-81}}{2}$

b) Wyznacz dziedzin9 funkcji P.

c) Oblicz dlugości boków tego z rozpatrywanych trójkatów, który ma najwipksze pole.

Strona 22 z26

$\mathrm{E}\mathrm{M}\mathrm{A}\mathrm{P}-\mathrm{R}0_{-}100$
\end{document}
