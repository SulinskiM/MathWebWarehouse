\documentclass[a4paper,12pt]{article}
\usepackage{latexsym}
\usepackage{amsmath}
\usepackage{amssymb}
\usepackage{graphicx}
\usepackage{wrapfig}
\pagestyle{plain}
\usepackage{fancybox}
\usepackage{bm}

\begin{document}

$\mathrm{Z}\mathrm{a}\mathrm{d}\mathrm{a}*\mathrm{i}\mathrm{e}5. (0-2\}$

Ciqg $(a_{n})$ jest określony dla $\mathrm{k}\mathrm{a}\dot{\mathrm{z}}\mathrm{d}\mathrm{e}\mathrm{j}$ liczby naturalnej $n\geq 1$ wzorem $a_{n}=\displaystyle \frac{(7p-1)n^{3}+5pn-3}{(p+1)n^{3}+n^{2}+p}$

gdzie $p$ jest liczbq rzeczywistq dodatniq.

Oblicz wartośč $p$, dla której granica ciagu $(a_{n})$ jest równa $\displaystyle \frac{4}{3}$

W ponizsze kratki wpisz kolejno-od lewej do prawej-pierwsza, drugq oraz trzeciq cyfre po

przecinku nieskończonego rozwiniecia dziesiptnego otrzymanego wyniku.
\begin{center}
\includegraphics[width=25.452mm,height=12.240mm]{./F2_M_PR_M2022_page3_images/image001.eps}
\end{center}
: {\it RU DNOPIS} \{{\it nie podlega ocenie}\}

Strona 4 z26

$\mathrm{E}\mathrm{M}\mathrm{A}\mathrm{P}-\mathrm{R}0_{-}100$
\end{document}
