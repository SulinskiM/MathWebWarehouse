\documentclass[a4paper,12pt]{article}
\usepackage{latexsym}
\usepackage{amsmath}
\usepackage{amssymb}
\usepackage{graphicx}
\usepackage{wrapfig}
\pagestyle{plain}
\usepackage{fancybox}
\usepackage{bm}

\begin{document}

Zadarie 8. $\{0-3\}$

Punkt $P$ jest punktem $\mathrm{p}\mathrm{r}\mathrm{z}\mathrm{e}\mathrm{c}\mathrm{i}_{9}\mathrm{c}\mathrm{i}\mathrm{a}$ przekqtnych trapezu ABCD. Dlugośč podstawy $CD$ jest

$0 2$ mniejsza od dlugości podstawy $AB$. Promień okregu opisanego na trójkacie

ostrokatnvm $CPD$ jest o 3 mniejszy od promienia okregu opisanego na trójkqcie $APB.$

Wykaz, $\dot{\mathrm{z}}\mathrm{e}$ spelnionyjest warunek $|DP|^{2}+|CP|^{2}-|CD|^{2}=\displaystyle \frac{4\sqrt{2}}{3}\cdot|DP| |CP|.$

Strona 8 z26

$\mathrm{E}\mathrm{M}\mathrm{A}\mathrm{P}-\mathrm{R}0_{-}100$
\end{document}
