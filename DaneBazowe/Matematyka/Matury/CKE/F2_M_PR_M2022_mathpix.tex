\documentclass[10pt]{article}
\usepackage[polish]{babel}
\usepackage[utf8]{inputenc}
\usepackage[T1]{fontenc}
\usepackage{graphicx}
\usepackage[export]{adjustbox}
\graphicspath{ {./images/} }
\usepackage{amsmath}
\usepackage{amsfonts}
\usepackage{amssymb}
\usepackage[version=4]{mhchem}
\usepackage{stmaryrd}
\usepackage{multirow}

\title{KOD }

\author{}
\date{}


\begin{document}
\maketitle
PESEL\\
\includegraphics[max width=\textwidth, center]{2024_11_21_1b10b2a46207ca344bd2g-01(1)}

\section*{Miejsce na naklejkę.}
Sprawdź, czy kod na naklejce to\\
E-100.\\
Jeżeli tak - przyklej naklejkę. Jeżeli nie - zgłoś to nauczycielowi.

\section*{EGZAMIN MATURALNY Z MATEMATYKI PozIOM ROZSZERZONY}
DATA: 11 maja 2022 r. GODZINA ROZPOCZĘCIA: 9:00 CZAS PRACY: \(\mathbf{1 8 0}\) minut\\
LICZBA PUNKTÓW DO UZYSKANIA: 50

\section*{WYPEENIA ZESPÓŁ NADZORUJACY}
Uprawnienia zdającego do:\\
nieprzenoszenia zaznaczeń na kartę dostosowania zasad oceniania dostosowania w zw. z dyskalkulią.\\
\includegraphics[max width=\textwidth, center]{2024_11_21_1b10b2a46207ca344bd2g-01}

\section*{Instrukcja dla zdającego}
\begin{enumerate}
  \item Sprawdź, czy arkusz egzaminacyjny zawiera 26 stron (zadania 1-15).
\end{enumerate}

Ewentualny brak zgłoś przewodniczącemu zespołu nadzorującego egzamin.\\
2. Na tej stronie oraz na karcie odpowiedzi wpisz swój numer PESEL i przyklej naklejkę z kodem.\\
3. Nie wpisuj żadnych znaków w części przeznaczonej dla egzaminatora.\\
4. Rozwiązania zadań i odpowiedzi wpisuj w miejscu na to przeznaczonym.\\
5. Odpowiedzi do zadań zamkniętych (1-4) zaznacz na karcie odpowiedzi w części karty przeznaczonej dla zdającego. Zamaluj pola do tego przeznaczone. Błędne zaznaczenie otocz kółkiem i zaznacz właściwe.\\
6. W zadaniu 5. wpisz odpowiednie cyfry w kratki pod treścią zadania.\\
7. Pamiętaj, że pominięcie argumentacji lub istotnych obliczeń w rozwiązaniu zadania otwartego (6-15) może spowodować, że za to rozwiązanie nie otrzymasz pełnej liczby punktów.\\
8. Pisz czytelnie i używaj tylko długopisu lub pióra z czarnym tuszem lub atramentem.\\
9. Nie używaj korektora, a błędne zapisy wyraźnie przekreśl.\\
10. Pamiętaj, że zapisy w brudnopisie nie będą oceniane.\\
11. Możesz korzystać z zestawu wzorów matematycznych, cyrkla i linijki oraz kalkulatora prostego.

\section*{Zadanie 1. (0-1)}
Liczba \(\log _{3} \sqrt{27}-\log _{27} \sqrt{3}\) jest równa\\
A. \(\frac{4}{3}\)\\
B. \(\frac{1}{2}\)\\
C. \(\frac{11}{12}\)\\
D. 3

\section*{Zadanie 2. (0-1)}
Funkcja \(f\) jest określona wzorem \(f(x)=\frac{x^{3}-8}{x-2}\) dla każdej liczby rzeczywistej \(x \neq 2\).\\
Wartość pochodnej tej funkcji dla argumentu \(x=\frac{1}{2}\) jest równa\\
A. \(\frac{3}{4}\)\\
B. \(\frac{9}{4}\)\\
C. 3\\
D. \(\frac{54}{8}\)

\section*{Zadanie 3. (0-1)}
Jeżeli \(\cos \beta=-\frac{1}{3}\) i \(\beta \in\left(\pi, \frac{3}{2} \pi\right)\), to wartość wyrażenia \(\sin \left(\beta-\frac{1}{3} \pi\right)\) jest równa\\
A. \(\frac{-2 \sqrt{2}+\sqrt{3}}{6}\)\\
B. \(\frac{2 \sqrt{6}+1}{6}\)\\
C. \(\frac{2 \sqrt{2}+\sqrt{3}}{6}\)\\
D. \(\frac{1-2 \sqrt{6}}{6}\)

\section*{Zadanie 4. (0-1)}
Dane są dwie urny z kulami. W każdej z urn jest siedem kul. W pierwszej urnie są jedna kula biała i sześć kul czarnych, w drugiej urnie są cztery kule białe i trzy kule czarne.\\
Rzucamy jeden raz symetryczną monetą. Jeżeli wypadnie reszka, to losujemy jedną kulę z pierwszej urny, w przeciwnym przypadku - jedną kulę z drugiej urny.\\
Prawdopodobieństwo zdarzenia polegającego na tym, że wylosujemy kulę białą w tym doświadczeniu, jest równe\\
A. \(\frac{5}{14}\)\\
B. \(\frac{9}{14}\)\\
C. \(\frac{5}{7}\)\\
D. \(\frac{6}{7}\)

BRUDNOPIS (nie podlega ocenie)\\
\includegraphics[max width=\textwidth, center]{2024_11_21_1b10b2a46207ca344bd2g-03}

Zadanie 5. (0-2)\\
Ciąg \(\left(a_{n}\right)\) jest określony dla każdej liczby naturalnej \(n \geq 1\) wzorem \(a_{n}=\frac{(7 p-1) n^{3}+5 p n-3}{(p+1) n^{3}+n^{2}+p}\), gdzie \(p\) jest liczbą rzeczywistą dodatnią.\\
Oblicz wartość \(p\), dla której granica ciągu \(\left(a_{n}\right)\) jest równa \(\frac{4}{3}\).

W poniższe kratki wpisz kolejno - od lewej do prawej - pierwszą, drugą oraz trzecią cyfrę po przecinku nieskończonego rozwinięcia dziesiętnego otrzymanego wyniku.

\begin{center}
\begin{tabular}{|l|l|l|}
\hline
 &  &  \\
\hline
\end{tabular}
\end{center}

\section*{BRUDNOPIS (nie podlega ocenie)}
\begin{center}
\begin{tabular}{|c|c|c|c|c|c|c|c|c|c|c|c|c|c|c|c|c|c|c|c|c|c|}
\hline
 &  &  &  &  &  &  &  &  &  &  &  &  &  &  &  &  &  &  &  &  &  \\
\hline
 &  &  &  &  &  &  &  &  &  &  &  &  &  &  &  &  &  &  &  &  &  \\
\hline
 &  &  &  &  &  &  &  &  &  &  &  &  &  &  &  &  &  &  &  &  &  \\
\hline
 &  &  &  &  &  &  &  &  &  &  &  &  &  &  &  &  &  &  &  &  &  \\
\hline
 &  &  &  &  &  &  &  &  &  &  &  &  &  &  &  &  &  &  &  &  &  \\
\hline
 &  &  &  &  &  &  &  &  &  &  &  &  &  &  &  &  &  &  &  &  &  \\
\hline
 &  &  &  &  &  &  &  &  &  &  &  &  &  &  &  &  &  &  &  &  &  \\
\hline
 &  &  &  &  &  &  &  &  &  &  &  &  &  &  &  &  &  &  &  &  &  \\
\hline
 &  &  &  &  &  &  &  &  &  &  &  &  &  &  &  &  &  &  &  &  &  \\
\hline
 &  &  &  &  &  &  &  &  &  &  &  &  &  &  &  &  &  &  &  &  &  \\
\hline
 &  &  &  &  &  &  &  &  &  &  &  &  &  &  &  &  &  &  &  &  &  \\
\hline
 &  &  &  &  &  &  &  &  &  &  &  &  &  &  &  &  &  &  &  &  &  \\
\hline
 &  &  &  &  &  &  &  &  &  &  &  &  &  &  &  &  &  &  &  &  &  \\
\hline
 &  &  &  &  &  &  &  &  &  &  &  &  &  &  &  &  &  &  &  &  &  \\
\hline
 &  &  &  &  &  &  &  &  &  &  &  &  &  &  &  &  &  &  &  &  &  \\
\hline
 &  &  &  &  &  &  &  &  &  &  &  &  &  &  &  &  &  &  &  &  &  \\
\hline
 &  &  &  &  &  &  &  &  &  &  &  &  &  &  &  &  &  &  &  &  &  \\
\hline
 &  &  &  &  &  &  &  &  &  &  &  &  &  &  &  &  &  &  &  &  &  \\
\hline
 &  &  &  &  &  &  &  &  &  &  &  &  &  &  &  &  &  &  &  &  &  \\
\hline
 &  &  &  &  &  &  &  &  &  &  &  &  &  &  &  &  &  &  &  &  &  \\
\hline
 &  &  &  &  &  &  &  &  &  &  &  &  &  &  &  &  &  &  &  &  &  \\
\hline
 &  &  &  &  &  &  &  &  &  &  &  &  &  &  &  &  &  &  &  &  &  \\
\hline
 &  &  &  &  &  &  &  &  &  &  &  &  &  &  &  &  &  &  &  &  &  \\
\hline
 &  &  &  &  &  &  &  &  &  &  &  &  &  &  &  &  &  &  &  &  &  \\
\hline
 &  &  &  &  &  &  &  &  &  &  &  &  &  &  &  &  &  &  &  &  &  \\
\hline
 &  &  &  &  &  &  &  &  &  &  &  &  &  &  &  &  &  &  &  &  &  \\
\hline
 &  &  &  &  &  &  &  &  &  &  &  &  &  &  &  &  &  &  &  &  &  \\
\hline
 &  &  &  &  &  &  &  &  &  &  &  &  &  &  &  &  &  &  &  &  &  \\
\hline
 &  &  &  &  &  &  &  &  &  &  &  &  &  &  &  &  &  &  &  &  &  \\
\hline
 &  &  &  &  &  &  &  &  &  &  &  &  &  &  &  &  &  &  &  &  &  \\
\hline
 &  &  &  &  &  &  &  &  &  &  &  &  &  &  &  &  &  &  &  &  &  \\
\hline
 &  &  &  &  &  &  &  &  &  &  &  &  &  &  &  &  &  &  &  &  &  \\
\hline
 &  &  &  &  &  &  &  &  &  &  &  &  &  &  &  &  &  &  &  &  &  \\
\hline
\end{tabular}
\end{center}

Zadanie 6. (0-3)\\
Wykaż, że dla każdej liczby rzeczywistej \(x\) i dla każdej liczby rzeczywistej \(y\) takich, że \(2 x>y\), spełniona jest nierówność

\[
7 x^{3}+4 x^{2} y \geq y^{3}+2 x y^{2}-x^{3}
\]

\begin{center}
\begin{tabular}{|c|c|c|c|c|c|c|c|c|c|c|c|c|c|c|c|c|c|c|c|c|c|}
\hline
 &  &  &  &  &  &  &  &  &  &  &  &  &  &  &  &  &  &  &  &  &  \\
\hline
 &  &  &  &  &  &  &  &  &  &  &  &  &  &  &  &  &  &  &  &  &  \\
\hline
 &  &  &  &  &  &  &  &  &  &  &  &  &  &  &  &  &  &  &  &  &  \\
\hline
 &  &  &  &  &  &  &  &  &  &  &  &  &  &  &  &  &  &  &  &  &  \\
\hline
 &  &  &  &  &  &  &  &  &  &  &  &  &  &  &  &  &  &  &  &  &  \\
\hline
 &  &  &  &  &  &  &  &  &  &  &  &  &  &  &  &  &  &  &  &  &  \\
\hline
 &  &  &  &  &  &  &  &  &  &  &  &  &  &  &  &  &  &  &  &  &  \\
\hline
 &  &  &  &  &  &  &  &  &  &  &  &  &  &  &  &  &  &  &  &  &  \\
\hline
 &  &  &  &  &  &  &  &  &  &  &  &  &  &  &  &  &  &  &  &  &  \\
\hline
 &  &  &  &  &  &  &  &  &  &  &  &  &  &  &  &  &  &  &  &  &  \\
\hline
 &  &  &  &  &  &  &  &  &  &  &  &  &  &  &  &  &  &  &  &  &  \\
\hline
 &  &  &  &  &  &  &  &  &  &  &  &  &  &  &  &  &  &  &  &  &  \\
\hline
 &  &  &  &  &  &  &  &  &  &  &  &  &  &  &  &  &  &  &  &  &  \\
\hline
 &  &  &  &  &  &  &  &  &  &  &  &  &  &  &  &  &  &  &  &  &  \\
\hline
 &  &  &  &  &  &  &  &  &  &  &  &  &  &  &  &  &  &  &  &  &  \\
\hline
 &  &  &  &  &  &  &  &  &  &  &  &  &  &  &  &  &  &  &  &  &  \\
\hline
 &  &  &  &  &  &  &  &  &  &  &  &  &  &  &  &  &  &  &  &  &  \\
\hline
 &  &  &  &  &  &  &  &  &  &  &  &  &  &  &  &  &  &  &  &  &  \\
\hline
 &  &  &  &  &  &  &  &  &  &  &  &  &  &  &  &  &  &  &  &  &  \\
\hline
 &  &  &  &  &  &  &  &  &  &  &  &  &  &  &  &  &  &  &  &  &  \\
\hline
 &  &  &  &  &  &  &  &  &  &  &  &  &  &  &  &  &  &  &  &  &  \\
\hline
 &  &  &  &  &  &  &  &  &  &  &  &  &  &  &  &  &  &  &  &  &  \\
\hline
 &  &  &  &  &  &  &  &  &  &  &  &  &  &  &  &  &  &  &  &  &  \\
\hline
 &  &  &  &  &  &  &  &  &  &  &  &  &  &  &  &  &  &  &  &  &  \\
\hline
 &  &  &  &  &  &  &  &  &  &  &  &  &  &  &  &  &  &  &  &  &  \\
\hline
 &  &  &  &  &  &  &  &  &  &  &  &  &  &  &  &  &  &  &  &  &  \\
\hline
 &  &  &  &  &  &  &  &  &  &  &  &  &  &  &  &  &  &  &  &  &  \\
\hline
 &  &  &  &  &  &  &  &  &  &  &  &  &  &  &  &  &  &  &  &  &  \\
\hline
 &  &  &  &  &  &  &  &  &  &  &  &  &  &  &  &  &  &  &  &  &  \\
\hline
 &  &  &  &  &  &  &  &  &  &  &  &  &  &  &  &  &  &  &  &  &  \\
\hline
 &  &  &  &  &  &  &  &  &  &  &  &  &  &  &  &  &  &  &  &  &  \\
\hline
 &  &  &  &  &  &  &  &  &  &  &  &  &  &  &  &  &  &  &  &  &  \\
\hline
 &  &  &  &  &  &  &  &  &  &  &  &  &  &  &  &  &  &  &  &  &  \\
\hline
 &  &  &  &  &  &  &  &  &  &  &  &  &  &  &  &  &  &  &  &  &  \\
\hline
 &  &  &  &  &  &  &  &  &  &  &  &  &  &  &  &  &  &  &  &  &  \\
\hline
 &  &  &  &  &  &  &  &  &  &  &  &  &  &  &  &  &  &  &  &  &  \\
\hline
 &  &  &  &  &  &  &  &  &  &  &  &  &  &  &  &  &  &  &  &  &  \\
\hline
 &  &  &  &  &  &  &  &  &  &  &  &  &  &  &  &  &  &  &  &  &  \\
\hline
 &  &  &  &  &  &  &  &  &  &  &  &  &  &  &  &  &  &  &  &  &  \\
\hline
\end{tabular}
\end{center}

\begin{center}
\begin{tabular}{|c|l|c|c|}
\hline
\multirow{3}{*}{\begin{tabular}{c}
Wypełnia \\
egzaminator \\
\end{tabular}} & Nr zadania & 5. & 6. \\
\cline { 2 - 4 }
 & Maks. liczba pkt & 2 & 3 \\
\cline { 2 - 4 }
 & Uzyskana liczba pkt &  &  \\
\hline
\end{tabular}
\end{center}

Zadanie 7. (0-3)\\
Rozwiąż równanie:

\[
|x-3|=2 x+11
\]

\begin{center}
\begin{tabular}{|c|c|c|c|c|c|c|c|c|c|c|c|c|c|c|c|c|c|c|c|c|c|}
\hline
 &  &  &  &  &  &  &  &  &  &  &  &  &  &  &  &  &  &  &  &  &  \\
\hline
 &  &  &  &  &  &  &  &  &  &  &  &  &  &  &  &  &  &  &  &  &  \\
\hline
 &  &  &  &  &  &  &  &  &  &  &  &  &  &  &  &  &  &  &  &  &  \\
\hline
 &  &  &  &  &  &  &  &  &  &  &  &  &  &  &  &  &  &  &  &  &  \\
\hline
 &  &  &  &  &  &  &  &  &  &  &  &  &  &  &  &  &  &  &  &  &  \\
\hline
 &  &  &  &  &  &  &  &  &  &  &  &  &  &  &  &  &  &  &  &  &  \\
\hline
 &  &  &  &  &  &  &  &  &  &  &  &  &  &  &  &  &  &  &  &  &  \\
\hline
 &  &  &  &  &  &  &  &  &  &  &  &  &  &  &  &  &  &  &  &  &  \\
\hline
 &  &  &  &  &  &  &  &  &  &  &  &  &  &  &  &  &  &  &  &  &  \\
\hline
 &  &  &  &  &  &  &  &  &  &  &  &  &  &  &  &  &  &  &  &  &  \\
\hline
 &  &  &  &  &  &  &  &  &  &  &  &  &  &  &  &  &  &  &  &  &  \\
\hline
 &  &  &  &  &  &  &  &  &  &  &  &  &  &  &  &  &  &  &  &  &  \\
\hline
 &  &  &  &  &  &  &  &  &  &  &  &  &  &  &  &  &  &  &  &  &  \\
\hline
 &  &  &  &  &  &  &  &  &  &  &  &  &  &  &  &  &  &  &  &  &  \\
\hline
 &  &  &  &  &  &  &  &  &  &  &  &  &  &  &  &  &  &  &  &  &  \\
\hline
 &  &  &  &  &  &  &  &  &  &  &  &  &  &  &  &  &  &  &  &  &  \\
\hline
 &  &  &  &  &  &  &  &  &  &  &  &  &  &  &  &  &  &  &  &  &  \\
\hline
 &  &  &  &  &  &  &  &  &  &  &  &  &  &  &  &  &  &  &  &  &  \\
\hline
 &  &  &  &  &  &  &  &  &  &  &  &  &  &  &  &  &  &  &  &  &  \\
\hline
 &  &  &  &  &  &  &  &  &  &  &  &  &  &  &  &  &  &  &  &  &  \\
\hline
 &  &  &  &  &  &  &  &  &  &  &  &  &  &  &  &  &  &  &  &  &  \\
\hline
 &  &  &  &  &  &  &  &  &  &  &  &  &  &  &  &  &  &  &  &  &  \\
\hline
 &  &  &  &  &  &  &  &  &  &  &  &  &  &  &  &  &  &  &  &  &  \\
\hline
 &  &  &  &  &  &  &  &  &  &  &  &  &  &  &  &  &  &  &  &  &  \\
\hline
 &  &  &  &  &  &  &  &  &  &  &  &  &  &  &  &  &  &  &  &  &  \\
\hline
 &  &  &  &  &  &  &  &  &  &  &  &  &  &  &  &  &  &  &  &  &  \\
\hline
 &  &  &  &  &  &  &  &  &  &  &  &  &  &  &  &  &  &  &  &  &  \\
\hline
 &  &  &  &  &  &  &  &  &  &  &  &  &  &  &  &  &  &  &  &  &  \\
\hline
 &  &  &  &  &  &  &  &  &  &  &  &  &  &  &  &  &  &  &  &  &  \\
\hline
 &  &  &  &  &  &  &  &  &  &  &  &  &  &  &  &  &  &  &  &  &  \\
\hline
 &  &  &  &  &  &  &  &  &  &  &  &  &  &  &  &  &  &  &  &  &  \\
\hline
 &  &  &  &  &  &  &  &  &  &  &  &  &  &  &  &  &  &  &  &  &  \\
\hline
 &  &  &  &  &  &  &  &  &  &  &  &  &  &  &  &  &  &  &  &  &  \\
\hline
 &  &  &  &  &  &  &  &  &  &  &  &  &  &  &  &  &  &  &  &  &  \\
\hline
 &  &  &  &  &  &  &  &  &  &  &  &  &  &  &  &  &  &  &  &  &  \\
\hline
 &  &  &  &  &  &  &  &  &  &  &  &  &  &  &  &  &  &  &  &  &  \\
\hline
 &  &  &  &  &  &  &  &  &  &  &  &  &  &  &  &  &  &  &  &  &  \\
\hline
 &  &  &  &  &  &  &  &  &  &  &  &  &  &  &  &  &  &  &  &  &  \\
\hline
 &  &  &  &  &  &  &  &  &  &  &  &  &  &  &  &  &  &  &  &  &  \\
\hline
 &  &  &  &  &  &  &  &  &  &  &  &  &  &  &  &  &  &  &  &  &  \\
\hline
 &  &  &  &  &  &  &  &  &  &  &  &  &  &  &  &  &  &  &  &  &  \\
\hline
 &  &  &  &  &  &  &  &  &  &  &  &  &  &  &  &  &  &  &  &  &  \\
\hline
 &  &  &  &  &  &  &  &  &  &  &  &  &  &  &  &  &  &  &  &  &  \\
\hline
 &  &  &  &  &  &  &  &  &  &  &  &  &  &  &  &  &  &  &  &  &  \\
\hline
 &  &  &  &  &  &  &  &  &  &  &  &  &  &  &  &  &  &  &  &  &  \\
\hline
\end{tabular}
\end{center}

\begin{center}
\includegraphics[max width=\textwidth]{2024_11_21_1b10b2a46207ca344bd2g-07}
\end{center}

\begin{center}
\begin{tabular}{|c|l|c|}
\hline
\multirow{3}{*}{\begin{tabular}{c}
Wypełnia \\
egzaminator \\
\end{tabular}} & Nr zadania & 7. \\
\cline { 2 - 3 }
 & Maks. liczba pkt & 3 \\
\cline { 2 - 3 }
 & Uzyskana liczba pkt &  \\
\hline
\end{tabular}
\end{center}

Zadanie 8. (0-3)\\
Punkt \(P\) jest punktem przecięcia przekątnych trapezu \(A B C D\). Długość podstawy \(C D\) jest o 2 mniejsza od długości podstawy \(A B\). Promień okręgu opisanego na trójkącie ostrokatnym \(C P D\) jest o 3 mniejszy od promienia okręgu opisanego na trójkącie \(A P B\). Wykaż, że spełniony jest warunek \(|D P|^{2}+|C P|^{2}-|C D|^{2}=\frac{4 \sqrt{2}}{3} \cdot|D P| \cdot|C P|\).\\
\includegraphics[max width=\textwidth, center]{2024_11_21_1b10b2a46207ca344bd2g-08}\\
\includegraphics[max width=\textwidth, center]{2024_11_21_1b10b2a46207ca344bd2g-09}

\begin{center}
\begin{tabular}{|c|l|c|}
\hline
\multirow{2}{*}{\begin{tabular}{c}
Wypełnia \\
egzaminator \\
\end{tabular}} & Nr zadania & 8. \\
\cline { 2 - 3 }
 & Maks. liczba pkt & 3 \\
\cline { 2 - 3 }
 & Uzyskana liczba pkt &  \\
\hline
\end{tabular}
\end{center}

Zadanie 9. (0-4)\\
Reszta \(z\) dzielenia wielomianu \(W(x)=4 x^{3}-6 x^{2}-(5 m+1) x-2 m\) przez dwumian \(x+2\) jest równa ( -30 ).\\
Oblicz \(m\) i dla wyznaczonej wartości \(m\) rozwiąż nierówność \(W(x) \geq 0\).

\begin{center}
\begin{tabular}{|c|c|c|c|c|c|c|c|c|c|c|c|c|c|c|c|c|c|c|c|c|c|c|}
\hline
 &  &  &  &  &  &  &  &  &  &  &  &  &  &  &  &  &  &  &  &  &  &  \\
\hline
 &  &  &  &  &  &  &  &  &  &  &  &  &  &  &  &  &  &  &  &  &  &  \\
\hline
 &  &  &  &  &  &  &  &  &  &  &  &  &  &  &  &  &  &  &  &  &  &  \\
\hline
 &  &  &  &  &  &  &  &  &  &  &  &  &  &  &  &  &  &  &  &  &  &  \\
\hline
 &  &  &  &  &  &  &  &  &  &  &  &  &  &  &  &  &  &  &  &  &  &  \\
\hline
 &  &  &  &  &  &  &  &  &  &  &  &  &  &  &  &  &  &  &  &  &  &  \\
\hline
 &  &  &  &  &  &  &  &  &  &  &  &  &  &  &  &  &  &  &  &  &  &  \\
\hline
 &  &  &  &  &  &  &  &  &  &  &  &  &  &  &  &  &  &  &  &  &  &  \\
\hline
 &  &  &  &  &  &  &  &  &  &  &  &  &  &  &  &  &  &  &  &  &  &  \\
\hline
 &  &  &  &  &  &  &  &  &  &  &  &  &  &  &  &  &  &  &  &  &  &  \\
\hline
 &  &  &  &  &  &  &  &  &  &  &  &  &  &  &  &  &  &  &  &  &  &  \\
\hline
 &  &  &  &  &  &  &  &  &  &  &  &  &  &  &  &  &  &  &  &  &  &  \\
\hline
 &  &  &  &  &  &  &  &  &  &  &  &  &  &  &  &  &  &  &  &  &  &  \\
\hline
 &  &  &  &  &  &  &  &  &  &  &  &  &  &  &  &  &  &  &  &  &  &  \\
\hline
 &  &  &  &  &  &  &  &  &  &  &  &  &  &  &  &  &  &  &  &  &  &  \\
\hline
 &  &  &  &  &  &  &  &  &  &  &  &  &  &  &  &  &  &  &  &  &  &  \\
\hline
 &  &  &  &  &  &  &  &  &  &  &  &  &  &  &  &  &  &  &  &  &  &  \\
\hline
 &  &  &  &  &  &  &  &  &  &  &  &  &  &  &  &  &  &  &  &  &  &  \\
\hline
 &  &  &  &  &  &  &  &  &  &  &  &  &  &  &  &  &  &  &  &  &  &  \\
\hline
 &  &  &  &  &  &  &  &  &  &  &  &  &  &  &  &  &  &  &  &  &  &  \\
\hline
 &  &  &  &  &  &  &  &  &  &  &  &  &  &  &  &  &  &  &  &  &  &  \\
\hline
 &  &  &  &  &  &  &  &  &  &  &  &  &  &  &  &  &  &  &  &  &  &  \\
\hline
 &  &  &  &  &  &  &  &  &  &  &  &  &  &  &  &  &  &  &  &  &  &  \\
\hline
 &  &  &  &  &  &  &  &  &  &  &  &  &  &  &  &  &  &  &  &  &  &  \\
\hline
 &  &  &  &  &  &  &  &  &  &  &  &  &  &  &  &  &  &  &  &  &  &  \\
\hline
 &  &  &  &  &  &  &  &  &  &  &  &  &  &  &  &  &  &  &  &  &  &  \\
\hline
 &  &  &  &  &  &  &  &  &  &  &  &  &  &  &  &  &  &  &  &  &  &  \\
\hline
 &  &  &  &  &  &  &  &  &  &  &  &  &  &  &  &  &  &  &  &  &  &  \\
\hline
 &  &  &  &  &  &  &  &  &  &  &  &  &  &  &  &  &  &  &  &  &  &  \\
\hline
 &  &  &  &  &  &  &  &  &  &  &  &  &  &  &  &  &  &  &  &  &  &  \\
\hline
 &  &  &  &  &  &  &  &  &  &  &  &  &  &  &  &  &  &  &  &  &  &  \\
\hline
 &  &  &  &  &  &  &  &  &  &  &  &  &  &  &  &  &  &  &  &  &  &  \\
\hline
 &  &  &  &  &  &  &  &  &  &  &  &  &  &  &  &  &  &  &  &  &  &  \\
\hline
 &  &  &  &  &  &  &  &  &  &  &  &  &  &  &  &  &  &  &  &  &  &  \\
\hline
 &  &  &  &  &  &  &  &  &  &  &  &  &  &  &  &  &  &  &  &  &  &  \\
\hline
 &  &  &  &  &  &  &  &  &  &  &  &  &  &  &  &  &  &  &  &  &  &  \\
\hline
 &  &  &  &  &  &  &  &  &  &  &  &  &  &  &  &  &  &  &  &  &  &  \\
\hline
 &  &  &  &  &  &  &  &  &  &  &  &  &  &  &  &  &  &  &  &  &  &  \\
\hline
 &  &  &  &  &  &  &  &  &  &  &  &  &  &  &  &  &  &  &  &  &  &  \\
\hline
 &  &  &  &  &  &  &  &  &  &  &  &  &  &  &  &  &  &  &  &  &  &  \\
\hline
 &  &  &  &  &  &  &  &  &  &  &  &  &  &  &  &  &  &  &  &  &  &  \\
\hline
 &  &  &  &  &  &  &  &  &  &  &  &  &  &  &  &  &  &  &  &  &  &  \\
\hline
 &  &  &  &  &  &  &  &  &  &  &  &  &  &  &  &  &  &  &  &  &  &  \\
\hline
\end{tabular}
\end{center}

\begin{center}
\includegraphics[max width=\textwidth]{2024_11_21_1b10b2a46207ca344bd2g-11}
\end{center}

\begin{center}
\begin{tabular}{|c|l|c|}
\hline
\multirow{2}{*}{\begin{tabular}{c}
Wypełnia \\
egzaminator \\
\end{tabular}} & Nr zadania & 9. \\
\cline { 2 - 3 }
 & Maks. liczba pkt & 4 \\
\cline { 2 - 3 }
 & Uzyskana liczba pkt &  \\
\hline
\end{tabular}
\end{center}

Zadanie 10. (0-4)\\
Ciąg \(\left(a_{n}\right)\), określony dla każdej liczby naturalnej \(n \geq 1\), jest geometryczny i ma wszystkie wyrazy dodatnie. Ponadto \(a_{1}=675\) i \(a_{22}=\frac{5}{4} a_{23}+\frac{1}{5} a_{21}\).\\
Ciąg \(\left(b_{n}\right)\), określony dla każdej liczby naturalnej \(n \geq 1\), jest arytmetyczny.\\
Suma wszystkich wyrazów ciągu ( \(a_{n}\) ) jest równa sumie dwudziestu pięciu początkowych kolejnych wyrazów ciągu \(\left(b_{n}\right)\). Ponadto \(a_{3}=b_{4}\). Oblicz \(b_{1}\).

\begin{center}
\begin{tabular}{|c|c|c|c|c|c|c|c|c|c|c|c|c|c|c|c|c|c|c|c|c|c|}
\hline
 &  &  &  &  &  &  &  &  &  &  &  &  &  &  &  &  &  &  &  &  &  \\
\hline
 &  &  &  &  &  &  &  &  &  &  &  &  &  &  &  &  &  &  &  &  &  \\
\hline
 &  &  &  &  &  &  &  &  &  &  &  &  &  &  &  &  &  &  &  &  &  \\
\hline
 &  &  &  &  &  &  &  &  &  &  &  &  &  &  &  &  &  &  &  &  &  \\
\hline
 &  &  &  &  &  &  &  &  &  &  &  &  &  &  &  &  &  &  &  &  &  \\
\hline
 &  &  &  &  &  &  &  &  &  &  &  &  &  &  &  &  &  &  &  &  &  \\
\hline
 &  &  &  &  &  &  &  &  &  &  &  &  &  &  &  &  &  &  &  &  &  \\
\hline
 &  &  &  &  &  &  &  &  &  &  &  &  &  &  &  &  &  &  &  &  &  \\
\hline
 &  &  &  &  &  &  &  &  &  &  &  &  &  &  &  &  &  &  &  &  &  \\
\hline
 &  &  &  &  &  &  &  &  &  &  &  &  &  &  &  &  &  &  &  &  &  \\
\hline
 &  &  &  &  &  &  &  &  &  &  &  &  &  &  &  &  &  &  &  &  &  \\
\hline
 &  &  &  &  &  &  &  &  &  &  &  &  &  &  &  &  &  &  &  &  &  \\
\hline
 &  &  &  &  &  &  &  &  &  &  &  &  &  &  &  &  &  &  &  &  &  \\
\hline
 &  &  &  &  &  &  &  &  &  &  &  &  &  &  &  &  &  &  &  &  &  \\
\hline
 &  &  &  &  &  &  &  &  &  &  &  &  &  &  &  &  &  &  &  &  &  \\
\hline
 &  &  &  &  &  &  &  &  &  &  &  &  &  &  &  &  &  &  &  &  &  \\
\hline
 &  &  &  &  &  &  &  &  &  &  &  &  &  &  &  &  &  &  &  &  &  \\
\hline
 &  &  &  &  &  &  &  &  &  &  &  &  &  &  &  &  &  &  &  &  &  \\
\hline
 &  &  &  &  &  &  &  &  &  &  &  &  &  &  &  &  &  &  &  &  &  \\
\hline
 &  &  &  &  &  &  &  &  &  &  &  &  &  &  &  &  &  &  &  &  &  \\
\hline
 &  &  &  &  &  &  &  &  &  &  &  &  &  &  &  &  &  &  &  &  &  \\
\hline
 &  &  &  &  &  &  &  &  &  &  &  &  &  &  &  &  &  &  &  &  &  \\
\hline
 &  &  &  &  &  &  &  &  &  &  &  &  &  &  &  &  &  &  &  &  &  \\
\hline
 &  &  &  &  &  &  &  &  &  &  &  &  &  &  &  &  &  &  &  &  &  \\
\hline
 &  &  &  &  &  &  &  &  &  &  &  &  &  &  &  &  &  &  &  &  &  \\
\hline
 &  &  &  &  &  &  &  &  &  &  &  &  &  &  &  &  &  &  &  &  &  \\
\hline
 &  &  &  &  &  &  &  &  &  &  &  &  &  &  &  &  &  &  &  &  &  \\
\hline
 &  &  &  &  &  &  &  &  &  &  &  &  &  &  &  &  &  &  &  &  &  \\
\hline
 &  &  &  &  &  &  &  &  &  &  &  &  &  &  &  &  &  &  &  &  &  \\
\hline
 &  &  &  &  &  &  &  &  &  &  &  &  &  &  &  &  &  &  &  &  &  \\
\hline
 &  &  &  &  &  &  &  &  &  &  &  &  &  &  &  &  &  &  &  &  &  \\
\hline
 &  &  &  &  &  &  &  &  &  &  &  &  &  &  &  &  &  &  &  &  &  \\
\hline
 &  &  &  &  &  &  &  &  &  &  &  &  &  &  &  &  &  &  &  &  &  \\
\hline
 &  &  &  &  &  &  &  &  &  &  &  &  &  &  &  &  &  &  &  &  &  \\
\hline
 &  &  &  &  &  &  &  &  &  &  &  &  &  &  &  &  &  &  &  &  &  \\
\hline
 &  &  &  &  &  &  &  &  &  &  &  &  &  &  &  &  &  &  &  &  &  \\
\hline
 &  &  &  &  &  &  &  &  &  &  &  &  &  &  &  &  &  &  &  &  &  \\
\hline
 &  &  &  &  &  &  &  &  &  &  &  &  &  &  &  &  &  &  &  &  &  \\
\hline
 &  &  &  &  &  &  &  &  &  &  &  &  &  &  &  &  &  &  &  &  &  \\
\hline
 &  &  &  &  &  &  &  &  &  &  &  &  &  &  &  &  &  &  &  &  &  \\
\hline
\end{tabular}
\end{center}

\begin{center}
\includegraphics[max width=\textwidth]{2024_11_21_1b10b2a46207ca344bd2g-13}
\end{center}

\begin{center}
\begin{tabular}{|c|l|c|}
\hline
\multirow{2}{*}{\begin{tabular}{c}
Wypełnia \\
egzaminator \\
\end{tabular}} & Nr zadania & 10. \\
\cline { 2 - 3 }
 & Maks. liczba pkt & 4 \\
\cline { 2 - 3 }
 & Uzyskana liczba pkt &  \\
\hline
\end{tabular}
\end{center}

Zadanie 11. (0-4)\\
Rozwiąż równanie \(\sin x+\sin 2 x+\sin 3 x=0 w\) przedziale \(\langle 0, \pi\rangle\).

\begin{center}
\begin{tabular}{|c|c|c|c|c|c|c|c|c|c|c|c|c|c|c|c|c|c|c|c|c|c|c|c|}
\hline
 &  &  &  &  &  &  &  &  &  &  &  &  &  &  &  &  &  &  &  &  &  &  &  \\
\hline
 &  &  &  &  &  &  &  &  &  &  &  &  &  &  &  &  &  &  &  &  &  &  &  \\
\hline
 &  &  &  &  &  &  &  &  &  &  &  &  &  &  &  &  &  &  &  &  &  &  &  \\
\hline
 &  &  &  &  &  &  &  &  &  &  &  &  &  &  &  &  &  &  &  &  &  &  &  \\
\hline
 &  &  &  &  &  &  &  &  &  &  &  &  &  &  &  &  &  &  &  &  &  &  &  \\
\hline
 &  &  &  &  &  &  &  &  &  &  &  &  &  &  &  &  &  &  &  &  &  &  &  \\
\hline
 &  &  &  &  &  &  &  &  &  &  &  &  &  &  &  &  &  &  &  &  &  &  &  \\
\hline
 &  &  &  &  &  &  &  &  &  &  &  &  &  &  &  &  &  &  &  &  &  &  &  \\
\hline
 &  &  &  &  &  &  &  &  &  &  &  &  &  &  &  &  &  &  &  &  &  &  &  \\
\hline
 &  &  &  &  &  &  &  &  &  &  &  &  &  &  &  &  &  &  &  &  &  &  &  \\
\hline
 &  &  &  &  &  &  &  &  &  &  &  &  &  &  &  &  &  &  &  &  &  &  &  \\
\hline
 &  &  &  &  &  &  &  &  &  &  &  &  &  &  &  &  &  &  &  &  &  &  &  \\
\hline
 &  &  &  &  &  &  &  &  &  &  &  &  &  &  &  &  &  &  &  &  &  &  &  \\
\hline
 &  &  &  &  &  &  &  &  &  &  &  &  &  &  &  &  &  &  &  &  &  &  &  \\
\hline
 &  &  &  &  &  &  &  &  &  &  &  &  &  &  &  &  &  &  &  &  &  &  &  \\
\hline
 &  &  &  &  &  &  &  &  &  &  &  &  &  &  &  &  &  &  &  &  &  &  &  \\
\hline
 &  &  &  &  &  &  &  &  &  &  &  &  &  &  &  &  &  &  &  &  &  &  &  \\
\hline
 &  &  &  &  &  &  &  &  &  &  &  &  &  &  &  &  &  &  &  &  &  &  &  \\
\hline
 &  &  &  &  &  &  &  &  &  &  &  &  &  &  &  &  &  &  &  &  &  &  &  \\
\hline
 &  &  &  &  &  &  &  &  &  &  &  &  &  &  &  &  &  &  &  &  &  &  &  \\
\hline
 &  &  &  &  &  &  &  &  &  &  &  &  &  &  &  &  &  &  &  &  &  &  &  \\
\hline
 &  &  &  &  &  &  &  &  &  &  &  &  &  &  &  &  &  &  &  &  &  &  &  \\
\hline
 &  &  &  &  &  &  &  &  &  &  &  &  &  &  &  &  &  &  &  &  &  &  &  \\
\hline
 &  &  &  &  &  &  &  &  &  &  &  &  &  &  &  &  &  &  &  &  &  &  &  \\
\hline
 &  &  &  &  &  &  &  &  &  &  &  &  &  &  &  &  &  &  &  &  &  &  &  \\
\hline
 &  &  &  &  &  &  &  &  &  &  &  &  &  &  &  &  &  &  &  &  &  &  &  \\
\hline
 &  &  &  &  &  &  &  &  &  &  &  &  &  &  &  &  &  &  &  &  &  &  &  \\
\hline
 &  &  &  &  &  &  &  &  &  &  &  &  &  &  &  &  &  &  &  &  &  &  &  \\
\hline
 &  &  &  &  &  &  &  &  &  &  &  &  &  &  &  &  &  &  &  &  &  &  &  \\
\hline
 &  &  &  &  &  &  &  &  &  &  &  &  &  &  &  &  &  &  &  &  &  &  &  \\
\hline
 &  &  &  &  &  &  &  &  &  &  &  &  &  &  &  &  &  &  &  &  &  &  &  \\
\hline
 &  &  &  &  &  &  &  &  &  &  &  &  &  &  &  &  &  &  &  &  &  &  &  \\
\hline
 &  &  &  &  &  &  &  &  &  &  &  &  &  &  &  &  &  &  &  &  &  &  &  \\
\hline
 &  &  &  &  &  &  &  &  &  &  &  &  &  &  &  &  &  &  &  &  &  &  &  \\
\hline
 &  &  &  &  &  &  &  &  &  &  &  &  &  &  &  &  &  &  &  &  &  &  &  \\
\hline
 &  &  &  &  &  &  &  &  &  &  &  &  &  &  &  &  &  &  &  &  &  &  &  \\
\hline
 &  &  &  &  &  &  &  &  &  &  &  &  &  &  &  &  &  &  &  &  &  &  &  \\
\hline
 &  &  &  &  &  &  &  &  &  &  &  &  &  &  &  &  &  &  &  &  &  &  &  \\
\hline
 &  &  &  &  &  &  &  &  &  &  &  &  &  &  &  &  &  &  &  &  &  &  &  \\
\hline
 &  &  &  &  &  &  &  &  &  &  &  &  &  &  &  &  &  &  &  &  &  &  &  \\
\hline
 &  &  &  &  &  &  &  &  &  &  &  &  &  &  &  &  &  &  &  &  &  &  &  \\
\hline
 &  &  &  &  &  &  &  &  &  &  &  &  &  &  &  &  &  &  &  &  &  &  &  \\
\hline
 &  &  &  &  &  &  &  &  &  &  &  &  &  &  &  &  &  &  &  &  &  &  &  \\
\hline
 &  &  &  &  &  &  &  &  &  &  &  &  &  &  &  &  &  &  &  &  &  &  &  \\
\hline
 &  &  &  &  &  &  &  &  &  &  &  &  &  &  &  &  &  &  &  &  &  &  &  \\
\hline
 &  &  &  &  &  &  &  &  &  &  &  &  &  &  &  &  &  &  &  &  &  &  &  \\
\hline
\end{tabular}
\end{center}

\begin{center}
\includegraphics[max width=\textwidth]{2024_11_21_1b10b2a46207ca344bd2g-15}
\end{center}

\begin{center}
\begin{tabular}{|c|l|c|}
\hline
\multirow{2}{*}{\begin{tabular}{c}
Wypełnia \\
egzaminator \\
\end{tabular}} & Nr zadania & 11. \\
\cline { 2 - 3 }
 & Maks. liczba pkt & 4 \\
\cline { 2 - 3 }
 & Uzyskana liczba pkt &  \\
\hline
\end{tabular}
\end{center}

Zadanie 12. (0-5)\\
Wyznacz wszystkie wartości parametru \(m\), dla których równanie

\[
x^{2}-(m+1) x+m=0
\]

ma dwa różne rozwiązania rzeczywiste \(x_{1}\) oraz \(x_{2}\), spełniające warunki:

\[
x_{1} \neq 0, \quad x_{2} \neq 0 \quad \text { oraz } \quad \frac{1}{x_{1}}+\frac{1}{x_{2}}+2=\frac{1}{x_{1}^{2}}+\frac{1}{x_{2}^{2}}
\]

\includegraphics[max width=\textwidth, center]{2024_11_21_1b10b2a46207ca344bd2g-16}\\
\includegraphics[max width=\textwidth, center]{2024_11_21_1b10b2a46207ca344bd2g-17}

\begin{center}
\begin{tabular}{|c|l|c|}
\hline
\multirow{2}{*}{\begin{tabular}{c}
Wypełnia \\
egzaminator \\
\end{tabular}} & Nr zadania & 12. \\
\cline { 2 - 3 }
 & Maks. liczba pkt & 5 \\
\cline { 2 - 3 }
 & Uzyskana liczba pkt &  \\
\hline
\end{tabular}
\end{center}

Zadanie 13. (0-5)\\
Dany jest graniastosłup prosty \(A B C D E F G H\) o podstawie prostokątnej \(A B C D\). Przekątne\\
\(A H\) i \(A F\) ścian bocznych tworzą kąt ostry o mierze \(\alpha\) takiej, że \(\sin \alpha=\frac{12}{13}\) (zobacz rysunek). Pole trójkąta \(A F H\) jest równe 26,4.\\
Oblicz wysokość \(h\) tego graniastosłupa.\\
\includegraphics[max width=\textwidth, center]{2024_11_21_1b10b2a46207ca344bd2g-18}\\
\includegraphics[max width=\textwidth, center]{2024_11_21_1b10b2a46207ca344bd2g-19}

\begin{center}
\begin{tabular}{|c|l|c|}
\hline
\multirow{3}{*}{\begin{tabular}{c}
Wypełnia \\
egzaminator \\
\end{tabular}} & Nr zadania & 13. \\
\cline { 2 - 3 }
 & Maks. liczba pkt & 5 \\
\cline { 2 - 3 }
 & Uzyskana liczba pkt &  \\
\hline
\end{tabular}
\end{center}

Zadanie 14. (0-6)\\
Punkt \(A=(-3,2)\) jest wierzchołkiem trójkąta równoramiennego \(A B C\), w którym \(|A C|=|B C|\).\\
Pole tego trójkąta jest równe 15 . Bok \(B C\) zawarty jest w prostej o równaniu \(y=x-1\). Oblicz współrzędne wierzchołków \(B\) i \(C\) tego trójkąta.\\
\includegraphics[max width=\textwidth, center]{2024_11_21_1b10b2a46207ca344bd2g-20}\\
\includegraphics[max width=\textwidth, center]{2024_11_21_1b10b2a46207ca344bd2g-21}

\begin{center}
\begin{tabular}{|c|l|c|}
\hline
\multirow{3}{*}{\begin{tabular}{c}
Wypełnia \\
egzaminator \\
\end{tabular}} & Nr zadania & 14. \\
\cline { 2 - 3 }
 & Maks. liczba pkt & 6 \\
\cline { 2 - 3 }
 & Uzyskana liczba pkt &  \\
\hline
\end{tabular}
\end{center}

Zadanie 15. (0-7)\\
Rozpatrujemy wszystkie trójkąty równoramienne o obwodzie równym 18.\\
a) Wykaż, że pole \(P\) każdego z tych trójkątów, jako funkcja długości \(b\) ramienia, wyraża się wzorem \(P(b)=\frac{(18-2 b) \cdot \sqrt{18 b-81}}{2}\).\\
b) Wyznacz dziedzinę funkcji \(P\).\\
c) Oblicz długości boków tego z rozpatrywanych trójkątów, który ma największe pole.

\begin{center}
\begin{tabular}{|c|c|c|c|c|c|c|c|c|c|c|c|c|c|c|c|c|c|c|c|c|c|}
\hline
 &  &  &  &  &  &  &  &  &  &  &  &  &  &  &  &  &  &  &  &  &  \\
\hline
 &  &  &  &  &  &  &  &  &  &  &  &  &  &  &  &  &  &  &  &  &  \\
\hline
 &  &  &  &  &  &  &  &  &  &  &  &  &  &  &  &  &  &  &  &  &  \\
\hline
 &  &  &  &  &  &  &  &  &  &  &  &  &  &  &  &  &  &  &  &  &  \\
\hline
 &  &  &  &  &  &  &  &  &  &  &  &  &  &  &  &  &  &  &  &  &  \\
\hline
 &  &  &  &  &  &  &  &  &  &  &  &  &  &  &  &  &  &  &  &  &  \\
\hline
 &  &  &  &  &  &  &  &  &  &  &  &  &  &  &  &  &  &  &  &  &  \\
\hline
 &  &  &  &  &  &  &  &  &  &  &  &  &  &  &  &  &  &  &  &  &  \\
\hline
 &  &  &  &  &  &  &  &  &  &  &  &  &  &  &  &  &  &  &  &  &  \\
\hline
 &  &  &  &  &  &  &  &  &  &  &  &  &  &  &  &  &  &  &  &  &  \\
\hline
 &  &  &  &  &  &  &  &  &  &  &  &  &  &  &  &  &  &  &  &  &  \\
\hline
 &  &  &  &  &  &  &  &  &  &  &  &  &  &  &  &  &  &  &  &  &  \\
\hline
 &  &  &  &  &  &  &  &  &  &  &  &  &  &  &  &  &  &  &  &  &  \\
\hline
 &  &  &  &  &  &  &  &  &  &  &  &  &  &  &  &  &  &  &  &  &  \\
\hline
 &  &  &  &  &  &  &  &  &  &  &  &  &  &  &  &  &  &  &  &  &  \\
\hline
 &  &  &  &  &  &  &  &  &  &  &  &  &  &  &  &  &  &  &  &  &  \\
\hline
 &  &  &  &  &  &  &  &  &  &  &  &  &  &  &  &  &  &  &  &  &  \\
\hline
 &  &  &  &  &  &  &  &  &  &  &  &  &  &  &  &  &  &  &  &  &  \\
\hline
 &  &  &  &  &  &  &  &  &  &  &  &  &  &  &  &  &  &  &  &  &  \\
\hline
 &  &  &  &  &  &  &  &  &  &  &  &  &  &  &  &  &  &  &  &  &  \\
\hline
 &  &  &  &  &  &  &  &  &  &  &  &  &  &  &  &  &  &  &  &  &  \\
\hline
 &  &  &  &  &  &  &  &  &  &  &  &  &  &  &  &  &  &  &  &  &  \\
\hline
 &  &  &  &  &  &  &  &  &  &  &  &  &  &  &  &  &  &  &  &  &  \\
\hline
 &  &  &  &  &  &  &  &  &  &  &  &  &  &  &  &  &  &  &  &  &  \\
\hline
 &  &  &  &  &  &  &  &  &  &  &  &  &  &  &  &  &  &  &  &  &  \\
\hline
 &  &  &  &  &  &  &  &  &  &  &  &  &  &  &  &  &  &  &  &  &  \\
\hline
 &  &  &  &  &  &  &  &  &  &  &  &  &  &  &  &  &  &  &  &  &  \\
\hline
 &  &  &  &  &  &  &  &  &  &  &  &  &  &  &  &  &  &  &  &  &  \\
\hline
 &  &  &  &  &  &  &  &  &  &  &  &  &  &  &  &  &  &  &  &  &  \\
\hline
 &  &  &  &  &  &  &  &  &  &  &  &  &  &  &  &  &  &  &  &  &  \\
\hline
 &  &  &  &  &  &  &  &  &  &  &  &  &  &  &  &  &  &  &  &  &  \\
\hline
 &  &  &  &  &  &  &  &  &  &  &  &  &  &  &  &  &  &  &  &  &  \\
\hline
 &  &  &  &  &  &  &  &  &  &  &  &  &  &  &  &  &  &  &  &  &  \\
\hline
 &  &  &  &  &  &  &  &  &  &  &  &  &  &  &  &  &  &  &  &  &  \\
\hline
 &  &  &  &  &  &  &  &  &  &  &  &  &  &  &  &  &  &  &  &  &  \\
\hline
 &  &  &  &  &  &  &  &  &  &  &  &  &  &  &  &  &  &  &  &  &  \\
\hline
 &  &  &  &  &  &  &  &  &  &  &  &  &  &  &  &  &  &  &  &  &  \\
\hline
 &  &  &  &  &  &  &  &  &  &  &  &  &  &  &  &  &  &  &  &  &  \\
\hline
 &  &  &  &  &  &  &  &  &  &  &  &  &  &  &  &  &  &  &  &  &  \\
\hline
\end{tabular}
\end{center}

\includegraphics[max width=\textwidth, center]{2024_11_21_1b10b2a46207ca344bd2g-23}\\
\includegraphics[max width=\textwidth, center]{2024_11_21_1b10b2a46207ca344bd2g-24}

\begin{center}
\begin{tabular}{|c|l|c|}
\hline
\multirow{3}{*}{\begin{tabular}{c}
Wypełnia \\
egzaminator \\
\end{tabular}} & Nr zadania & 15. \\
\cline { 2 - 3 }
 & Maks. liczba pkt & 7 \\
\cline { 2 - 3 }
 & Uzyskana liczba pkt &  \\
\hline
\end{tabular}
\end{center}

BRUDNOPIS (nie podlega ocenie)\\
\includegraphics[max width=\textwidth, center]{2024_11_21_1b10b2a46207ca344bd2g-25}\\
\includegraphics[max width=\textwidth, center]{2024_11_21_1b10b2a46207ca344bd2g-26}


\end{document}