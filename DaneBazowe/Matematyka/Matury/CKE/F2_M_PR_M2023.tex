\documentclass[a4paper,12pt]{article}
\usepackage{latexsym}
\usepackage{amsmath}
\usepackage{amssymb}
\usepackage{graphicx}
\usepackage{wrapfig}
\pagestyle{plain}
\usepackage{fancybox}
\usepackage{bm}

\begin{document}

CENTRALNA

KOMISJA

EGZAMINACYJNA

KOD

WYPELNIA ZDAJACY

PESEL
\begin{center}
\includegraphics[width=21.900mm,height=10.164mm]{./F2_M_PR_M2023_page0_images/image001.eps}

\includegraphics[width=79.656mm,height=10.164mm]{./F2_M_PR_M2023_page0_images/image002.eps}
\end{center}
Egzamin maturalny

DATA: 12 maja 2023 r.

GODZINA R0ZP0CZECIA: 9:00

CZAS TRWANIA: $180 \displaystyle \min$ ut

Arkusz zawiera informacje prawnie chronione

do momentu rozpoczecia egzaminu.

{\it Miejsce na naklejke}.

{\it Sprawdz}', {\it czy kod na naklejce to}

e-100.

/{\it ezeli tak}- {\it przyklej naklejke}.

/{\it ezeli nie}- {\it zgtoś to nauczycielowi}.

MAP-R0-100-2305

$\Re \mathrm{V}\Psi\S \mathrm{L}\mathrm{N}\Re$ 2B@P A $\mathrm{N}\mathrm{A}\emptyset.\mathrm{Z}\otimes\Re \mathrm{U}\mathrm{d}.\mathrm{A}\otimes Y,$

Uprawnienia $\mathrm{z}\mathrm{d}\mathrm{a}\mathrm{j}_{8}$cego do:

\fbox{} dostosowania zasad oceniania

\fbox{} dostosowania w zw. z dyskalkuliq

\fbox{} nieprzenoszenia zaznaczeń na karte.

LICZBA PUNKTÓW DO UZYSKANIA 50

Przed rozpoczeciem pracy z arkuszem egzaminacyjnym

1.

Sprawd $\acute{\mathrm{z}}$, czy nauczyciel przekazal Ci wlaściwy arkusz egzaminacyjny,

tj. arkusz we wlaściwej formule, z w[aściwego przedmiotu na wlaściwym

poziomie.

2.

$\mathrm{J}\mathrm{e}\dot{\mathrm{z}}$ eli przekazano Ci niew[aściwy arkusz- natychmiast zgloś to nauczycielowi.

Nie rozrywaj banderol.

3. $\mathrm{J}\mathrm{e}\dot{\mathrm{z}}$ eli przekazano Ci w[aściwy arkusz- rozerwij banderole po otrzymaniu

takiego polecenia od nauczyciela. Zapoznaj $\mathrm{s}\mathrm{i}\mathrm{e}$ z instrukcjq na stronie 2.

Uk\}ad graficzny

\copyright CKE 2022

$\Vert\Vert\Vert\Vert\Vert\Vert\Vert\Vert\Vert\Vert\Vert\Vert\Vert\Vert\Vert\Vert\Vert\Vert\Vert\Vert\Vert\Vert\Vert\Vert\Vert\Vert\Vert\Vert\Vert\Vert|$




lnstrukcja dla zdajqcego

l. Sprawdz', czy arkusz egzaminacyjny zawiera 29 stron (zadania $1-16$).

Ewentualny brak zgloś przewodniczqcemu zespolu nadzorujqcego egzamin.

2. Na pierwszej stronie arkusza oraz na karcie odpowiedzi wpisz swój numer PESEL

i przyklej naklejke z kodem.

3. Odpowiedzi do zadań $\mathrm{z}\mathrm{a}\mathrm{m}\mathrm{k}\mathrm{n}\mathrm{i}9$tych ($1-4)$ zaznacz na karcie odpowiedzi w cześci karty

przeznaczonej dla zdajacego. Zamaluj $\blacksquare$ pola do tego przeznaczone. $\mathrm{B}9\mathrm{d}\mathrm{n}\mathrm{e}$

zaznaczenie otocz kólkiem\copyright izaznacz wlaściwe.

4. $\mathrm{W}$ zadaniu 5. wpisz odpowiednie cyfry w kratki pod treściq zadania.

5. Pamiptaj, $\dot{\mathrm{z}}\mathrm{e}$ pominiecie argumentacji lub istotnych obliczeń w rozwiqzaniu zadania

otwartego (6-16) $\mathrm{m}\mathrm{o}\dot{\mathrm{z}}\mathrm{e}$ spowodowač, $\dot{\mathrm{z}}\mathrm{e}$ za to rozwiqzanie nie otrzymasz pelnej liczby

punktów.

6. Rozwiqzania zadań i odpowiedzi wpisuj w miejscu na to przeznaczonym.

7. Pisz czytelnie i $\mathrm{u}\dot{\mathrm{z}}$ ywaj tylko dlugopisu lub pióra z czarnym tuszem lub atramentem.

8. Nie $\mathrm{u}\dot{\mathrm{z}}$ ywaj korektora, a bledne zapisy wyra $\acute{\mathrm{z}}$ nie przekreśl.

9. Nie wpisuj $\dot{\mathrm{z}}$ adnych znaków w cześci przeznaczonej dla egzaminatora.

10. Pamietaj, $\dot{\mathrm{z}}\mathrm{e}$ zapisy w brudnopisie nie beda oceniane.

11. $\mathrm{M}\mathrm{o}\dot{\mathrm{z}}$ esz korzystač z Wybranych wzoróvv matematycznych, cyrkla i linijki oraz kalkulatora

prostego. Upewnij $\mathrm{s}\mathrm{i}\mathrm{e}$, czy przekazano Ci broszur9 z ok1adka takq jak widoczna ponizej.

$mb\uparrow h_{a_{\Delta>0}}\ldots.,.$

[‡M]$\xi$

{\$} $\mathrm{r}..\iota \mathrm{u}\mathrm{p}z\mathrm{n}\backslash \sim-\wedge\cdot\cdot \backslash \mathrm{t}n\triangleright\tau \mathrm{r}\dot{\alpha}\mathrm{c}\backslash $

$\overline{\infty\epsilon \mathrm{w}\mathrm{r}}$'' $-\underline{\overline{.\backslash -}}\bullet$

Strona 2 z29

$\mathrm{E}\mathrm{M}\mathrm{A}\mathrm{P}-\mathrm{R}0_{-}100$





Zadanie 9. (0-3)

Funkcja $f$ jest określona wzorem $f(x)=\displaystyle \frac{3x^{2}-2x}{x^{2}+2x+8}$ dla $\mathrm{k}\mathrm{a}\dot{\mathrm{z}}$ dej liczby rzeczywistej $x.$

Punkt $P=(x_{0}$, 3$)$ nalez $\mathrm{y}$ do wykresu funkcji $f$. Oblicz $x_{0}$ oraz wyznacz równanie

stycznej do wykresu funkcji $f$ w punkcie $P.$
\begin{center}
\begin{tabular}{|l|l|l|l|}
\cline{2-4}
&	\multicolumn{1}{|l|}{Nr zadania}&	\multicolumn{1}{|l|}{$8.$}&	\multicolumn{1}{|l|}{ $9.$}	\\
\cline{2-4}
&	\multicolumn{1}{|l|}{Maks. liczba pkt}&	\multicolumn{1}{|l|}{$3$}&	\multicolumn{1}{|l|}{ $3$}	\\
\cline{2-4}
\multicolumn{1}{|l|}{egzaminator}&	\multicolumn{1}{|l|}{Uzyskana liczba pkt}&	\multicolumn{1}{|l|}{}&	\multicolumn{1}{|l|}{}	\\
\hline
\end{tabular}

\end{center}
$\mathrm{E}\mathrm{M}\mathrm{A}\mathrm{P}-\mathrm{R}0_{-}100$

Strona ll z29





Zadanie 10. $(0-4$\}

Rozwiqz nierównośč

$\displaystyle \sqrt{x^{2}+4x+4}<\frac{25}{3}-\sqrt{x^{2}-6x+9}$

{\it Wskazówka}: {\it skorzystaj z tego, ze} $\sqrt{a^{2}}=|a|$ {\it dla kazdei liczby rzeczywistei} $a.$

Strona 12 z29

$\mathrm{E}\mathrm{M}\mathrm{A}\mathrm{P}-\mathrm{R}0_{-}10$





Wypelnia

egzaminator

Nr zadania

Maks. liczba pkt

Uzyskana liczba pkt

10.

4

-RO-100

Strona 13 z29





Zadanie lt. $\langle 0-4$\}

Określamy kwadraty $K_{1}, K_{2}, K_{3}$, następujqco:

$\bullet K_{1}$ jest kwadratem o boku d\}ugości $a$

$\bullet K_{2}$ jest kwadratem, którego $\mathrm{k}\mathrm{a}\dot{\mathrm{z}}\mathrm{d}\mathrm{y}$ wierzcholek $\mathrm{l}\mathrm{e}\dot{\mathrm{z}}\mathrm{y}$ na innym boku kwadratu $K_{1}$

ten bok w stosunku 1 : 3

i dzieli

$\bullet K_{3}$ jest kwadratem, którego $\mathrm{k}\mathrm{a}\dot{\mathrm{z}}\mathrm{d}\mathrm{y}$ wierzcholek $\mathrm{l}\mathrm{e}\dot{\mathrm{z}}\mathrm{y}$ na innym boku kwadratu $K_{2}$ i dzieli

ten bok w stosunku 1 : 3

i ogólnie, dla $\mathrm{k}\mathrm{a}\dot{\mathrm{z}}$ dej liczby naturalnej $n\geq 2,$

$\bullet K_{n}$ jest kwadratem, którego $\mathrm{k}\mathrm{a}\dot{\mathrm{z}}\mathrm{d}\mathrm{y}$ wierzcholek $\mathrm{l}\mathrm{e}\dot{\mathrm{z}}\mathrm{y}$ na innym boku kwadratu $K_{n-1}$

i dzieli ten bok w stosunku 1 : 3.

Obwody wszystkich kwadratów określonych powyzej tworzq nieskończony ciqg

geometryczny.

Na rysunku przedstawiono kwadraty utworzone w sposób opisany powyzej.

{\it a}
\begin{center}
\includegraphics[width=58.824mm,height=58.872mm]{./F2_M_PR_M2023_page13_images/image001.eps}
\end{center}
{\it a}

Oblicz sume wszystkich wyrazów tego nieskończonego ciagu.

Strona 14 z29

$\mathrm{E}\mathrm{M}\mathrm{A}\mathrm{P}-\mathrm{R}0_{-}100$





Wypelnia

egzaminator

Nr zadania

Maks. liczba pkt

Uzyskana liczba pkt

11.

4

-RO-100

Strona 15 z29





Zadanie 12. $(0-4$\}

Rozwiqz równanie 3 $\sin^{2}x-\sin^{2}(2x)=0$ w przedziale $\langle\pi, 2\pi\rangle.$

Strona 16 z29

$\mathrm{E}\mathrm{M}\mathrm{A}\mathrm{P}-\mathrm{R}0_{-}10$





Wypelnia

egzaminator

Nr zadania

Maks. liczba pkt

Uzyskana liczba pkt

12.

4

-RO-100

Strona 17 z29





Zadanie 13. $(0-4$\}

Czworokqt ABCD, w którym $|BC|=4 \mathrm{i} |CD|=5$, jest opisany na okregu. Przekatna $AC$

tego czworokata tworzy z bokiem $BC$ kqt o mierze $60^{\mathrm{o}}$, natomiast z bokiem $AB-$ kqt ostry,

którego sinus jest równy $\displaystyle \frac{1}{4}$. Oblicz obwód $\mathrm{c}\mathrm{z}\mathrm{w}\mathrm{o}\mathrm{r}\mathrm{o}\mathrm{k}_{\mathrm{c}}$]$\mathrm{t}\mathrm{a}$ {\it ABCD}.

Strona 18 z29

$\mathrm{E}\mathrm{M}\mathrm{A}\mathrm{P}-\mathrm{R}0_{-}100$





Wypelnia

egzaminator

Nr zadania

Maks. liczba pkt

Uzyskana liczba pkt

13.

4

-RO-100

Strona 19 z29





Zadanie 14. $(0-4$\}

Danyjest sześcian ABCDEFGH o krawpdzi

dlugości 6. Punkt $S$ jest punktem przeciecia

przekqtnych $AH \mathrm{i}$ DE ściany bocznej ADHE

(zobacz rysunek).

Oblicz wysokośč trójkata SBH poprowadzona z punktu S na bok BH tego trójkata.

Strona 20 z29

$\mathrm{E}\mathrm{M}\mathrm{A}\mathrm{P}-\mathrm{R}0_{-}100$





Zadania egzaminacyine sq wydrukowane

na nastepnych stronach.

$\mathrm{E}\mathrm{M}\mathrm{A}\mathrm{P}-\mathrm{R}0_{-}100$

Strona 3 z29





Wypelnia

egzaminator

Nr zadania

Maks. liczba pkt

Uzyskana liczba pkt

14.

4

-RO-100

Strona 21 z29





Zadanie 15. $(0-5$\}

Wyznacz wszystkie wartości parametru $m\neq 2$, dla których równanie

$x^{2}+4x-\displaystyle \frac{m-3}{m-2}=0$

ma dwa rózne rozwiqzania rzeczywiste $x_{1}, x_{2}$ spelniajace warunek $x_{1}^{3}+x_{2}^{3}>-28.$

Strona 22 z29

$\mathrm{E}\mathrm{M}\mathrm{A}\mathrm{P}-\mathrm{R}0_{-}100$





Wypelnia

egzaminator

Nr zadania

Maks. liczba pkt

Uzyskana liczba pkt

15.

5

-RO-100

Strona 23 z29





Zadanie 16. (0-7)

Rozwazamy trójkqty $ABC$, w których $A=(0,0), B=(m,0)$, gdzie $m\in(4,+\infty),$

a wierzcholek $C \mathrm{l}\mathrm{e}\dot{\mathrm{z}}\mathrm{y}$ na prostej o równaniu $y=-2x$. Na boku $BC$ tego trójkqta $\mathrm{l}\mathrm{e}\dot{\mathrm{z}}\mathrm{y}$ punkt

$D=(3,2).$

a) Wykaz, $\dot{\mathrm{z}}\mathrm{e}$ dla $m\in(4,+\infty)$ pole $P$ trójkqta $ABC$, jako funkcja zmiennej $m$, wyraza $\mathrm{s}\mathrm{i}\mathrm{e}$

wzorem

$P(m)=\displaystyle \frac{m^{2}}{m-4}$

b) Oblicz t9 wartośč m, d1a której funkcja P osiaga wartośč najmniejszq. Wyznacz

równanie prostej BC, przy której funkcja F osiaga t9 najmniejszq wartośč.

Strona 24 z29

$\mathrm{E}\mathrm{M}\mathrm{A}\mathrm{P}-\mathrm{R}0_{-}100$





$1)0_{-}100$

Strona 25 z29





Wypelnia

egzaminator

Nr zadania

Maks. liczba pkt

Uzyskana liczba pkt

16.

7

Strona 26 z29

$\mathrm{E}\mathrm{M}\mathrm{A}\mathrm{P}-\mathrm{R}0_{-}10$





: {\it RU DNOPIS} \{{\it nie podlega ocenie}\}

$\mathrm{h}\mathrm{P}-\mathrm{R}0_{-}100$

Strona 27 z29





Strona 28 z29

$\mathrm{E}\mathrm{M}\mathrm{A}\mathrm{P}-\mathrm{R}0_{-}10$





$1)0_{-}100$

Strona 29 z29










$W$ {\it kazdym z zadań od} $f.$ {\it do 4. wybierz i zaznacz na karcie odpowiedzi poprawnq odpowiedz}'.

Zadanie $1_{p}(0-1)$

Granica $\displaystyle \lim_{x\rightarrow 1}\frac{x^{3}-1}{(x-1)(x+2)}$ jest równa

A. $(-1)$

B. 0

C. -31

D. l

Zadanie 2. (0-1)

Dane sq wektory $\vec{u}=[4,-5]$ oraz $\vec{v}=[-1,-5]$. Dlugośč wektora $\vec{u}-4\vec{v}$ jest równa

A. 7

B. 15

C. 17

D. 23

Zadanie 3. $(0-l\displaystyle \int$

Punkty $A, B, C, D, E \mathrm{l}\mathrm{e}\dot{\mathrm{z}}$ a na okregu o środku $S$. Miara $\ltimes \mathrm{a}\mathrm{t}\mathrm{a} BCD$ jest równa $110^{\mathrm{o}},$

a miara kqta $BDA$ jest równa $35^{\mathrm{o}}$ (zobacz rysunek).
\begin{center}
\includegraphics[width=77.316mm,height=77.160mm]{./F2_M_PR_M2023_page3_images/image001.eps}
\end{center}
{\it D  C}

$110^{\mathrm{o}}$

$35^{\mathrm{o}}$

$S_{\bullet}$

{\it E  B}

{\it A}

Wtedy kqt DEA ma miare równq

A. $100^{\mathrm{o}}$

B. $105^{\mathrm{o}}$

C. $110^{\mathrm{o}}$

D. $115^{\mathrm{o}}$

Zadanie 4. $\{0-1\}$

Dany jest zbiór trzynastu liczb \{1, 2, 3, 4, 5, 6, 7, 8, 9, 10, 11, 12, 13\}, z którego 1osujemy

jednocześnie dwie liczby. Wszystkich róznych sposobów wylosowania z tego zbioru dwóch

liczb, których iloczyn jest liczbq parzystq, jest

A. $\left(\begin{array}{l}
7\\
2
\end{array}\right)+49$

B. $\left(\begin{array}{l}
6\\
1
\end{array}\right)\cdot\left(\begin{array}{l}
7\\
1
\end{array}\right)+49$

C. $\left(\begin{array}{l}
13\\
2
\end{array}\right)-(_{2}^{7})$

D. $\left(\begin{array}{l}
13\\
2
\end{array}\right)-(_{2}^{6})$

Strona 4 z29

$\mathrm{E}\mathrm{M}\mathrm{A}\mathrm{P}-\mathrm{R}0_{-}100$















: {\it RU DNOPIS} \{{\it nie podlega ocenie}\}

$\mathrm{h}\mathrm{P}-\mathrm{R}0_{-}100$

Strona 5 z29





Zadanie 5. $(0-2$\}

Wielomian $W(x)=7x^{3}-9x^{2}+9x-2$ ma dokladnie jeden pierwiastek rzeczywisty.

Oblicz ten pierwiastek.

$\mathrm{W}$ ponizsze kratki wpisz kolejno-od lewej do prawej-pierwsza, drugq oraz trzeciq cyfr9 po

przecinku nieskończonego rozwiniecia dziesiptnego otrzymanego wyniku.
\begin{center}
\includegraphics[width=25.452mm,height=12.240mm]{./F2_M_PR_M2023_page5_images/image001.eps}
\end{center}
: {\it RU DNOPIS} \{{\it nie podlega ocenie}\}

Strona 6 z29

$\mathrm{E}\mathrm{M}\mathrm{A}\mathrm{P}-\mathrm{R}0_{-}100$





Zadanie 6. $\{0-3$)

Liczby rzeczywiste $x$ oraz $y$ spelniajqjednocześnie równanie $x+y=4$ i nierównośč

$x^{3}-x^{2}\mathrm{y}\leq x\mathrm{y}^{2}-y^{3}$. Wykaz, $\dot{\mathrm{z}}\mathrm{e} x=2$ oraz $y=2.$
\begin{center}
\begin{tabular}{|l|l|l|l|}
\cline{2-4}
&	\multicolumn{1}{|l|}{Nr zadania}&	\multicolumn{1}{|l|}{$5.$}&	\multicolumn{1}{|l|}{ $6.$}	\\
\cline{2-4}
&	\multicolumn{1}{|l|}{Maks. liczba pkt}&	\multicolumn{1}{|l|}{$2$}&	\multicolumn{1}{|l|}{ $3$}	\\
\cline{2-4}
\multicolumn{1}{|l|}{egzaminator}&	\multicolumn{1}{|l|}{Uzyskana liczba pkt}&	\multicolumn{1}{|l|}{}&	\multicolumn{1}{|l|}{}	\\
\hline
\end{tabular}

\end{center}
$\mathrm{E}\mathrm{M}\mathrm{A}\mathrm{P}-\mathrm{R}0_{-}100$

Strona 7 z29





Zadanie 7. (0-3)

Danyjest trójkqt prostokqtny $ABC$, w którym $|4ABC|=90^{\mathrm{o}}$ oraz $|4\mathrm{C}AB|=60^{\mathrm{o}}$ Punkty

$K \mathrm{i} L \mathrm{l}\mathrm{e}\dot{\mathrm{z}}$ a na bokach- odpowiednio -$AB \mathrm{i} BC$ tak, $\dot{\mathrm{z}}\mathrm{e} |BK|=|BL|=1$ (zobacz

rysunek). Odcinek $KL$ przecina wysokośč $BD$ tego trójkqta w punkcie $N$, a ponadto

$|AD|=2.$
\begin{center}
\includegraphics[width=133.704mm,height=81.432mm]{./F2_M_PR_M2023_page7_images/image001.eps}
\end{center}
{\it A}

$60^{\mathrm{o}}$  2

{\it D}

{\it K}

{\it N}

1

{\it C}

{\it B} l $L$

Wykaz, $\dot{\mathrm{z}}\mathrm{e} |ND|=\sqrt{3}+1.$

Strona 8 z29

$\mathrm{E}\mathrm{M}\mathrm{A}\mathrm{P}-\mathrm{R}0_{-}100$





Wypelnia

egzaminator

Nr zadania

Maks. liczba pkt

Uzyskana liczba pkt

7.

3

-RO-100

Strona 9 z29





Zadanie 8. (0-3)

$\mathrm{W}$ pojemniku jest siedem kul: pi9č ku1 bia1ych i dwie ku1e czarne. $\mathrm{Z}$ tego pojemnika losujemy

jednocześnie dwie kule bez zwracania. Nastppnie-z kul pozostalych w pojemniku-

losujemy jeszcze $\mathrm{j}\mathrm{e}\mathrm{d}\mathrm{h}_{\mathrm{c}1}$ ku19. Ob1icz prawdopodobieństwo wy1osowania ku1i czarnej w drugim

losowaniu.

Strona 10 z29

$\mathrm{E}\mathrm{M}\mathrm{A}\mathrm{P}-\mathrm{R}0_{-}100$



\end{document}