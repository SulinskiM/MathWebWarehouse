\documentclass[a4paper,12pt]{article}
\usepackage{latexsym}
\usepackage{amsmath}
\usepackage{amssymb}
\usepackage{graphicx}
\usepackage{wrapfig}
\pagestyle{plain}
\usepackage{fancybox}
\usepackage{bm}

\begin{document}

CENTRALNA

KOMISJA

EGZAMINACYJNA

KOD

WYPELNIA ZDAJACY

PESEL
\begin{center}
\includegraphics[width=21.900mm,height=10.164mm]{./F2_M_PR_M2023_page0_images/image001.eps}

\includegraphics[width=79.656mm,height=10.164mm]{./F2_M_PR_M2023_page0_images/image002.eps}
\end{center}
Egzamin maturalny

DATA: 12 maja 2023 r.

GODZINA R0ZP0CZECIA: 9:00

CZAS TRWANIA: $180 \displaystyle \min$ ut

Arkusz zawiera informacje prawnie chronione

do momentu rozpoczecia egzaminu.

{\it Miejsce na naklejke}.

{\it Sprawdz}', {\it czy kod na naklejce to}

e-100.

/{\it ezeli tak}- {\it przyklej naklejke}.

/{\it ezeli nie}- {\it zgtoś to nauczycielowi}.

MAP-R0-100-2305

$\Re \mathrm{V}\Psi\S \mathrm{L}\mathrm{N}\Re$ 2B@P A $\mathrm{N}\mathrm{A}\emptyset.\mathrm{Z}\otimes\Re \mathrm{U}\mathrm{d}.\mathrm{A}\otimes Y,$

Uprawnienia $\mathrm{z}\mathrm{d}\mathrm{a}\mathrm{j}_{8}$cego do:

\fbox{} dostosowania zasad oceniania

\fbox{} dostosowania w zw. z dyskalkuliq

\fbox{} nieprzenoszenia zaznaczeń na karte.

LICZBA PUNKTÓW DO UZYSKANIA 50

Przed rozpoczeciem pracy z arkuszem egzaminacyjnym

1.

Sprawd $\acute{\mathrm{z}}$, czy nauczyciel przekazal Ci wlaściwy arkusz egzaminacyjny,

tj. arkusz we wlaściwej formule, z w[aściwego przedmiotu na wlaściwym

poziomie.

2.

$\mathrm{J}\mathrm{e}\dot{\mathrm{z}}$ eli przekazano Ci niew[aściwy arkusz- natychmiast zgloś to nauczycielowi.

Nie rozrywaj banderol.

3. $\mathrm{J}\mathrm{e}\dot{\mathrm{z}}$ eli przekazano Ci w[aściwy arkusz- rozerwij banderole po otrzymaniu

takiego polecenia od nauczyciela. Zapoznaj $\mathrm{s}\mathrm{i}\mathrm{e}$ z instrukcjq na stronie 2.

Uk\}ad graficzny

\copyright CKE 2022

$\Vert\Vert\Vert\Vert\Vert\Vert\Vert\Vert\Vert\Vert\Vert\Vert\Vert\Vert\Vert\Vert\Vert\Vert\Vert\Vert\Vert\Vert\Vert\Vert\Vert\Vert\Vert\Vert\Vert\Vert|$
\end{document}
