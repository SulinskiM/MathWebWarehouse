\documentclass[a4paper,12pt]{article}
\usepackage{latexsym}
\usepackage{amsmath}
\usepackage{amssymb}
\usepackage{graphicx}
\usepackage{wrapfig}
\pagestyle{plain}
\usepackage{fancybox}
\usepackage{bm}

\begin{document}

Zadanie 16. (0-7)

Rozwazamy trójkqty $ABC$, w których $A=(0,0), B=(m,0)$, gdzie $m\in(4,+\infty),$

a wierzcholek $C \mathrm{l}\mathrm{e}\dot{\mathrm{z}}\mathrm{y}$ na prostej o równaniu $y=-2x$. Na boku $BC$ tego trójkqta $\mathrm{l}\mathrm{e}\dot{\mathrm{z}}\mathrm{y}$ punkt

$D=(3,2).$

a) Wykaz, $\dot{\mathrm{z}}\mathrm{e}$ dla $m\in(4,+\infty)$ pole $P$ trójkqta $ABC$, jako funkcja zmiennej $m$, wyraza $\mathrm{s}\mathrm{i}\mathrm{e}$

wzorem

$P(m)=\displaystyle \frac{m^{2}}{m-4}$

b) Oblicz t9 wartośč m, d1a której funkcja P osiaga wartośč najmniejszq. Wyznacz

równanie prostej BC, przy której funkcja F osiaga t9 najmniejszq wartośč.

Strona 24 z29

$\mathrm{E}\mathrm{M}\mathrm{A}\mathrm{P}-\mathrm{R}0_{-}100$
\end{document}
