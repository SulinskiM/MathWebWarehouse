\documentclass[a4paper,12pt]{article}
\usepackage{latexsym}
\usepackage{amsmath}
\usepackage{amssymb}
\usepackage{graphicx}
\usepackage{wrapfig}
\pagestyle{plain}
\usepackage{fancybox}
\usepackage{bm}

\begin{document}

$W$ {\it kazdym z zadań od} $f.$ {\it do 4. wybierz i zaznacz na karcie odpowiedzi poprawnq odpowiedz}'.

Zadanie $1_{p}(0-1)$

Granica $\displaystyle \lim_{x\rightarrow 1}\frac{x^{3}-1}{(x-1)(x+2)}$ jest równa

A. $(-1)$

B. 0

C. -31

D. l

Zadanie 2. (0-1)

Dane sq wektory $\vec{u}=[4,-5]$ oraz $\vec{v}=[-1,-5]$. Dlugośč wektora $\vec{u}-4\vec{v}$ jest równa

A. 7

B. 15

C. 17

D. 23

Zadanie 3. $(0-l\displaystyle \int$

Punkty $A, B, C, D, E \mathrm{l}\mathrm{e}\dot{\mathrm{z}}$ a na okregu o środku $S$. Miara $\ltimes \mathrm{a}\mathrm{t}\mathrm{a} BCD$ jest równa $110^{\mathrm{o}},$

a miara kqta $BDA$ jest równa $35^{\mathrm{o}}$ (zobacz rysunek).
\begin{center}
\includegraphics[width=77.316mm,height=77.160mm]{./F2_M_PR_M2023_page3_images/image001.eps}
\end{center}
{\it D  C}

$110^{\mathrm{o}}$

$35^{\mathrm{o}}$

$S_{\bullet}$

{\it E  B}

{\it A}

Wtedy kqt DEA ma miare równq

A. $100^{\mathrm{o}}$

B. $105^{\mathrm{o}}$

C. $110^{\mathrm{o}}$

D. $115^{\mathrm{o}}$

Zadanie 4. $\{0-1\}$

Dany jest zbiór trzynastu liczb \{1, 2, 3, 4, 5, 6, 7, 8, 9, 10, 11, 12, 13\}, z którego 1osujemy

jednocześnie dwie liczby. Wszystkich róznych sposobów wylosowania z tego zbioru dwóch

liczb, których iloczyn jest liczbq parzystq, jest

A. $\left(\begin{array}{l}
7\\
2
\end{array}\right)+49$

B. $\left(\begin{array}{l}
6\\
1
\end{array}\right)\cdot\left(\begin{array}{l}
7\\
1
\end{array}\right)+49$

C. $\left(\begin{array}{l}
13\\
2
\end{array}\right)-(_{2}^{7})$

D. $\left(\begin{array}{l}
13\\
2
\end{array}\right)-(_{2}^{6})$

Strona 4 z29

$\mathrm{E}\mathrm{M}\mathrm{A}\mathrm{P}-\mathrm{R}0_{-}100$
\end{document}
