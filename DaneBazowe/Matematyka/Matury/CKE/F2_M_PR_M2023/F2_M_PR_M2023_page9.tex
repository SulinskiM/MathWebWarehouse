\documentclass[a4paper,12pt]{article}
\usepackage{latexsym}
\usepackage{amsmath}
\usepackage{amssymb}
\usepackage{graphicx}
\usepackage{wrapfig}
\pagestyle{plain}
\usepackage{fancybox}
\usepackage{bm}

\begin{document}

Zadanie 8. (0-3)

$\mathrm{W}$ pojemniku jest siedem kul: pi9č ku1 bia1ych i dwie ku1e czarne. $\mathrm{Z}$ tego pojemnika losujemy

jednocześnie dwie kule bez zwracania. Nastppnie-z kul pozostalych w pojemniku-

losujemy jeszcze $\mathrm{j}\mathrm{e}\mathrm{d}\mathrm{h}_{\mathrm{c}1}$ ku19. Ob1icz prawdopodobieństwo wy1osowania ku1i czarnej w drugim

losowaniu.

Strona 10 z29

$\mathrm{E}\mathrm{M}\mathrm{A}\mathrm{P}-\mathrm{R}0_{-}100$
\end{document}
