\documentclass[a4paper,12pt]{article}
\usepackage{latexsym}
\usepackage{amsmath}
\usepackage{amssymb}
\usepackage{graphicx}
\usepackage{wrapfig}
\pagestyle{plain}
\usepackage{fancybox}
\usepackage{bm}

\begin{document}

CENTRALNA

KOMISJA

EGZAMINACYJNA

KOD

WYPELNIA ZDAJACY

PESEL
\begin{center}
\includegraphics[width=21.900mm,height=10.164mm]{./F2_M_PR_M2024_page0_images/image001.eps}

\includegraphics[width=79.656mm,height=10.164mm]{./F2_M_PR_M2024_page0_images/image002.eps}
\end{center}
Egzamin maturalny

DATA: 15 maja 2024 r.

GODZINA R0ZP0CZECIA: 9:00

CZAS TRWANIA: $180 \displaystyle \min$ ut

Arkusz zawiera informacje prawnie chronione

do momentu rozpoczecia egzaminu.

{\it Miejsce na naklejke}.

{\it Sprawdz}', {\it czy kod na naklejce to}

e-100.

/{\it ezeli tak}- {\it przyklej naklejkq}.

/{\it ezeli nie}- {\it zgtoś to nauczycielowi}.

MAP-R0-100-2405

$\mathrm{W}\mathrm{Y}\Re$gLMlA 8$\mathrm{E}8\mathrm{P}\mathfrak{H}^{\aleph}\mathrm{L}\mathrm{M}\mathrm{A}\mathrm{D}\mathrm{Z}\mathrm{Q}\Re \mathrm{U}\mathrm{J}h\vartheta Y$

Uprawnienia $\mathrm{z}\mathrm{d}\mathrm{a}1\varepsilon$cego do:

\fbox{} dostosowania zasad oceniania

\fbox{} nieprzenoszenia odpowiedzi na karte.

LICZBA PUNKTÓW DO UZYSKANIA 50

Przed rozpoczeciem pracy z arkuszem egzaminacyjnym

1.

Sprawd $\acute{\mathrm{z}}$, czy nauczyciel przekazal Ci wlaściwy arkusz egzaminacyjny,

tj. arkusz we wlaściwej formule, z w[aściwego przedmiotu na wlaściwym

poziomie.

2.

$\mathrm{J}\mathrm{e}\dot{\mathrm{z}}$ eli przekazano Ci niew[aściwy arkusz- natychmiast zgloś to nauczycielowi.

Nie rozrywaj banderol.

3. $\mathrm{J}\mathrm{e}\dot{\mathrm{z}}$ eli przekazano Ci w[aściwy arkusz- rozerwij banderole po otrzymaniu

takiego polecenia od nauczyciela. Zapoznaj $\mathrm{s}\mathrm{i}\mathrm{e}$ z instrukcjq na stronie 2.

Uk\}ad graficzny

\copyright CKE 2022

$\Vert\Vert\Vert\Vert\Vert\Vert\Vert\Vert\Vert\Vert\Vert\Vert\Vert\Vert\Vert\Vert\Vert\Vert\Vert\Vert\Vert\Vert\Vert\Vert\Vert\Vert\Vert\Vert\Vert\Vert|$




lnstrukcja dla zdajqcego

l. Sprawdz', czy arkusz egzaminacyjny zawiera 29 stron (zadania $1-16$).

Ewentualny brak zgloś przewodniczqcemu zespolu nadzorujqcego egzamin.

2. Na pierwszej stronie arkusza oraz na karcie odpowiedzi wpisz swój numer PESEL

i przyklej naklejke z kodem.

3. Odpowiedzi do zadań $\mathrm{z}\mathrm{a}\mathrm{m}\mathrm{k}\mathrm{n}\mathrm{i}9$tych ($1-4)$ zaznacz na karcie odpowiedzi w cześci karty

przeznaczonej dla zdajacego. Zamaluj $\blacksquare$ pola do tego przeznaczone. $\mathrm{B}9\mathrm{d}\mathrm{n}\mathrm{e}$

zaznaczenie otocz kólkiem\copyright izaznacz wlaściwe.

4. $\mathrm{W}$ zadaniu 5. wpisz odpowiednie cyfry w kratki pod treścia zadania.

5. $\mathrm{P}\mathrm{a}\mathrm{m}\mathrm{i}9\mathrm{t}\mathrm{a}\mathrm{j}, \dot{\mathrm{z}}\mathrm{e}$ pominiecie argumentacji lub istotnych obliczeń w rozwiqzaniu zadania

otwartego (6-16) $\mathrm{m}\mathrm{o}\dot{\mathrm{z}}\mathrm{e}$ spowodowač, $\dot{\mathrm{z}}\mathrm{e}$ za to rozwiqzanie nie otrzymasz pelnej liczby

punktów.

6. Rozwiqzania zadań i odpowiedzi wpisuj w miejscu na to przeznaczonym.

7. Pisz czytelnie i $\mathrm{u}\dot{\mathrm{z}}$ ywaj tylko dlugopisu lub pióra z czarnym tuszem lub atramentem.

8. Nie $\mathrm{u}\dot{\mathrm{z}}$ ywaj korektora, a bledne zapisy wyra $\acute{\mathrm{z}}$ nie przekreśl.

9. Nie wpisuj $\dot{\mathrm{z}}$ adnych znaków w cześci przeznaczonej dla egzaminatora.

10. Pamietaj, $\dot{\mathrm{z}}\mathrm{e}$ zapisy w brudnopisie nie bedq oceniane.

11. $\mathrm{M}\mathrm{o}\dot{\mathrm{z}}$ esz korzystač z Wybranych wzoróvv matematycznych, cyrkla i linijki oraz kalkulatora

prostego. Upewnij $\mathrm{s}\mathrm{i}\mathrm{e}$, czy przekazano Ci broszur9 z ok1adka takq jak widoczna ponizej.

$\text{{\it á}}_{-,\rightarrow f'(^{\wedge}x_{0})}^{n_{è\mathrm{A}\cdot\alpha}h}$

$\rightarrow$2'$|.(${\it ra}$\vartheta\eta\hat{}\tilde{}\hat{}${\it h}A$+$`{\it r}$\grave{}|${\it ua}.$\approx\acute{}${\it g}.`{\it u}..'$|\Delta${\it h}A$\sqrt{}>${\it u}$\acute{}$30-('

$-\rightarrow 3$

$\mathrm{q},1\cdots\cdot 1\cup \mathrm{R} \varsigma..\vee\prime:\tilde{\mathrm{v}}\mathrm{k}r.7k\cdot(\mathrm{n}\rightarrow\prime.$

$\overline{\mathrm{w}u\mathrm{r}}$[‡@]$\mathrm{r}\mathrm{w} --\overline{\underline{\mathrm{R}\infty-}},\bullet$

Strona 2 z29

$\mathrm{E}\mathrm{M}\mathrm{A}\mathrm{P}-\mathrm{R}0_{-}100$





Zadanie 9. (0-3)

Funkcja f jest określona wzorem

$f(x)=\displaystyle \frac{x^{3}-3x+2}{\chi}$

dla $\mathrm{k}\mathrm{a}\dot{\mathrm{z}}$ dej liczby rzeczywistej $x$ róznej od zera. Punkt $P$, o pierwszej wspólrz9dnej

równej 2, na1ez $\mathrm{y}$ do wykresu funkcji $f$. Prosta o równaniu $y=ax+b$ jest styczna do

wykresu funkcji $f$ w punkcie $P.$

Oblicz wspólczynniki $a$ oraz $b$ w równaniu tej stycznej.
\begin{center}
\begin{tabular}{|l|l|l|l|}
\cline{2-4}
&	\multicolumn{1}{|l|}{Nr zadania}&	\multicolumn{1}{|l|}{$8.$}&	\multicolumn{1}{|l|}{ $9.$}	\\
\cline{2-4}
&	\multicolumn{1}{|l|}{Maks. liczba pkt}&	\multicolumn{1}{|l|}{$3$}&	\multicolumn{1}{|l|}{ $3$}	\\
\cline{2-4}
\multicolumn{1}{|l|}{egzaminator}&	\multicolumn{1}{|l|}{Uzyskana liczba pkt}&	\multicolumn{1}{|l|}{}&	\multicolumn{1}{|l|}{}	\\
\hline
\end{tabular}

\end{center}
$\mathrm{E}\mathrm{M}\mathrm{A}\mathrm{P}-\mathrm{R}0_{-}100$

Strona ll z29





Zadanie 10. (0-3)

Spośród wszystkich liczb naturalnych sześciocyfrowych, których wszystkie cyfry naleza do

zbioru \{1, 2, 3, 4, 5, 6, 7, 8\}, 1osujemy jednq. Wy1osowanie $\mathrm{k}\mathrm{a}\dot{\mathrm{z}}$ dej z tych liczb jest jednakowo

prawdopodobne.

Oblicz prawdopodobieństwo zdarzenia polegajqcego na tym, $\dot{\mathrm{z}}\mathrm{e}$ wylosujemy liczbe, która

ma nastppujqca wlasnośč: kolejne cyfry tej liczby (liczqc od lewej strony) $\mathrm{t}\mathrm{w}\mathrm{o}\mathrm{r}\mathrm{Z}_{\mathrm{c}1}-\mathrm{w}$ podanej

kolejności- sześciowyrazowy ciqg malejqcy.

Strona 12 z29

$\mathrm{E}\mathrm{M}\mathrm{A}\mathrm{P}-\mathrm{R}0_{-}100$





Wypelnia

egzaminator

Nr zadania

Maks. liczba pkt

Uzyskana liczba pkt

10.

3

-RO-100

Strona 13 z29





Zadanie tl. $(0-4$\}

Trzywyrazowy ciag $(x,y,z)$ jest geometryczny i rosnqcy. Suma wyrazów tego ciqgu jest

równa 105. Liczby $x, y$ oraz $z$ sq- odpowiednio-pierwszym, drugim oraz szóstym

wyrazem ciqgu arytmetycznego $(a_{n})$, określonego dla $\mathrm{k}\mathrm{a}\dot{\mathrm{z}}$ dej liczby naturalnej $n\geq 1.$

Oblicz $x, \mathrm{y}$ oraz $z.$

Strona 14 z29

$\mathrm{E}\mathrm{M}\mathrm{A}\mathrm{P}-\mathrm{R}0_{-}100$





Wypelnia

egzaminator

Nr zadania

Maks. liczba pkt

Uzyskana liczba pkt

11.

4

-RO-100

Strona 15 z29





Zadanie 12. $(0-4$\}

Rozwiqz równanie

$\sin(2x)+\cos(2x)=1+\sin x-\cos x$

w zbiorze $\langle 0,2\pi\rangle.$

Strona 16 z29

$\mathrm{E}\mathrm{M}\mathrm{A}\mathrm{P}-\mathrm{R}0_{-}10$





Wypelnia

egzaminator

Nr zadania

Maks. liczba pkt

Uzyskana liczba pkt

12.

4

-RO-100

Strona 17 z29





Zadanie 13. $(0-4$\}

Promień okregu opisanego na trójkqcie $ABC$ jest równy 17. Najdlu $\dot{\mathrm{z}}$ szym bokiem tego

trójkata jest bok $AC$, a dlugości dwóch pozostalych boków sq równe $|AB|=30$ oraz

$|BC|=17$. Oblicz miar9 kqta $BAC$ oraz dlugośč boku $AC$ tego trójkqta.

Strona 18 z29

$\mathrm{E}\mathrm{M}\mathrm{A}\mathrm{P}-\mathrm{R}0_{-}100$





Wypelnia

egzaminator

Nr zadania

Maks. liczba pkt

Uzyskana liczba pkt

13.

4

-RO-100

Strona 19 z29





Zadanie 14. $(0-5$\}

$\acute{\mathrm{S}}$ rodek $S$ okregu o promieniu $\sqrt{5} \mathrm{l}\mathrm{e}\dot{\mathrm{z}}\mathrm{y}$ na prostej o równaniu $y=x+1$. Przez punkt

$A=(1,2)$, którego odleglośč od punktu $S$ jest wieksza od $\sqrt{5}$, poprowadzono dwie proste

styczne do tego okregu w punktach- odpowiednio -$B \mathrm{i} C$. Pole czworokata ABSC jest

równe 15.

Oblicz wspólrzedne punktu $S$. Rozwaz wszystkie przypadki.

Strona 20 z29

$\mathrm{E}\mathrm{M}\mathrm{A}\mathrm{P}-\mathrm{R}0_{-}100$





Zadania egzaminacyine sq wydrukowane

na nastepnych stronach.

$\mathrm{E}\mathrm{M}\mathrm{A}\mathrm{P}-\mathrm{R}0_{-}100$

Strona 3 z29





Wypelnia

egzaminator

Nr zadania

Maks. liczba pkt

Uzyskana liczba pkt

14.

5

-RO-100

Strona 21 z29





Zadanie 15. (0-6)

Wyznacz wszystkie wartości parametru $m$, dla których równanie

$x^{2}-(3m+1)\cdot x+2m^{2}+m+1=0$

ma dwa rózne rozwiazania rzeczywiste $x_{1}, x_{2}$ spelniajace warunek

$x_{1}^{3}+x_{2}^{3}+3\cdot x_{1}\cdot x_{2}\cdot(x_{1}+x_{2}-3)\leq 3m-7$

Strona 22 z29

$\mathrm{E}\mathrm{M}\mathrm{A}\mathrm{P}-\mathrm{R}0_{-}10$





$1)0_{-}100$

Strona 23 z29





Wypelnia

egzaminator

Nr zadania

Maks. liczba pkt

Uzyskana liczba pkt

15.

6

Strona 24 z29

$\mathrm{E}\mathrm{M}\mathrm{A}\mathrm{P}-\mathrm{R}0_{-}10$





Zadanie 16. (0-6)

Rozwazamy wszystkie graniastoslupy prawidlowe trójkqtne o objetości 3456, których

$\mathrm{k}\mathrm{r}\mathrm{a}\mathrm{w}9^{\mathrm{d}\acute{\mathrm{Z}}}$ podstawy ma dlugośč nie wipkszq $\mathrm{n}\mathrm{i}\dot{\mathrm{z}} 8\sqrt{3}.$

a)

Wykaz, $\dot{\mathrm{z}}\mathrm{e}$ pole $P$ powierzchni calkowitej graniastoslupa w zalezności od dlugości $a$

$\mathrm{k}\mathrm{r}\mathrm{a}\mathrm{w}9^{\mathrm{d}\mathrm{z}\mathrm{i}}$ podstawy $\mathrm{g}\mathrm{r}\mathrm{a}\mathrm{n}\mathrm{i}\mathrm{a}\mathrm{s}\mathrm{t}\mathrm{o}\mathrm{s}\mathrm{u}\mathrm{p}\mathrm{a}$ jest określone wzorem

$P(a)=\displaystyle \frac{a^{2}\cdot\sqrt{3}}{2}+\frac{13824\sqrt{3}}{a}$

b) Pole $P$ powierzchni calkowitej graniastoslupa w zalezności od d$\dagger$ugości $a$ krawedzi

podstawy graniastoslupa jest określone wzorem

$P(a)=\displaystyle \frac{a^{2}\cdot\sqrt{3}}{2}+\frac{13824\sqrt{3}}{a}$

dla $a\in(0,8\sqrt{3}\rangle.$

Wyznacz dlugość krawedzi podstawy tego z rozwazanych graniastoslupów, którego pole

powierzchni calkowitej jest najmniejsze. Oblicz to najmniejsze pole.

$\mathrm{E}\mathrm{M}\mathrm{A}\mathrm{P}-\mathrm{R}0_{-}100$

Strona 25 z29





Strona 26 z29

$\mathrm{E}\mathrm{M}\mathrm{A}\mathrm{P}-\mathrm{R}0_{-}10$





Wypelnia

egzaminator

Nr zadania

Maks. liczba pkt

Uzyskana liczba pkt

16.

6

-RO-100

Strona 27 z29





: {\it RU DNOPIS} \{{\it nie podlega ocenie}\}

Strona 28z 29

$\mathrm{E}\mathrm{M}\mathrm{A}\mathrm{P}-\mathrm{R}0_{-}10$





$1)0_{-}100$

Strona 29 z29










{\it W kazdym z zadań od f. do 4. wybierz i zaznacz na karcie odpowiedzi poprawnq odpowiedz}'.

Zadanie $1_{p}(0-1)$

Odleglośč punktu $A=(6,2)$ od prostej o równaniu $5x-12y+1=0$ jest równa

A. $\displaystyle \frac{7}{13}$

B. $\displaystyle \frac{7}{12}$

C. $\displaystyle \frac{5}{12}$

D. $\displaystyle \frac{12}{13}$

Zadanie 2. (0-1)

Równanie $|2x-4|=3x+1$ w zbiorze liczb rzeczywistych

A. nie ma rozwiazań.

B. ma dokladnie jedno rozwiazanie.

C. ma dokladnie dwa rozwiqzania.

D. ma dokladnie cztery rozwiazania.

Zadanie 3. $(0-l\displaystyle \int$

Funkcja $f$ jest określona wzorem $f(x)=|-(x+2)^{3}+5|$ dla $\mathrm{k}\mathrm{a}\dot{\mathrm{z}}$ dej liczby

rzeczywistej $x$. Zbiorem wartości funkcji $f$ jest przedzial

A. $\langle-2, +\infty)$

B. $\langle 0, +\infty)$

C. $\langle 3, +\infty)$

D. $\langle 5, +\infty)$

Zadanie 4. $\{0-1\}$

Granica $\displaystyle \lim_{\chi\rightarrow+\infty}\frac{1+3a+2ax+ax^{3}}{3+4x+5x^{2}+5x^{3}}$ jest równa 3. Wtedy

A. $a=3$

B. $a=9$

C. $a=15$

D. $a=21$

Strona 4 z29

$\mathrm{E}\mathrm{M}\mathrm{A}\mathrm{P}-\mathrm{R}0_{-}100$















: {\it RU DNOPIS} \{{\it nie podlega ocenie}\}

$\mathrm{h}\mathrm{P}-\mathrm{R}0_{-}100$

Strona 5 z29





Zadanie 5. $(0-2$\}

Wielomian $W(x)=8x^{3}+14x^{2}+5x+3$ jest iloczynem wielomianów $P(x)=2x+3$

oraz $Q(x)=ax^{2}+bx+c.$

$\mathrm{W}$ ponizsze kratki wpisz kolejno-od lewej do prawej- wartości wspólczynników: $a, b$

oraz $c.$
\begin{center}
\includegraphics[width=25.452mm,height=12.240mm]{./F2_M_PR_M2024_page5_images/image001.eps}
\end{center}
: {\it RU DNOPIS} \{{\it nie podlega ocenie}\}

Strona 6 z29

$\mathrm{E}\mathrm{M}\mathrm{A}\mathrm{P}-\mathrm{R}0_{-}100$





Zadanie 6. $\{0-3$)

Wykaz, $\dot{\mathrm{z}}\mathrm{e}\mathrm{j}\mathrm{e}\dot{\mathrm{z}}$ eli log54$=a$ oraz log43 $=b$, to log1280$=\displaystyle \frac{2a+1}{a\cdot(1+b)}$
\begin{center}
\begin{tabular}{|l|l|l|l|}
\cline{2-4}
&	\multicolumn{1}{|l|}{Nr zadania}&	\multicolumn{1}{|l|}{$5.$}&	\multicolumn{1}{|l|}{ $6.$}	\\
\cline{2-4}
&	\multicolumn{1}{|l|}{Maks. liczba pkt}&	\multicolumn{1}{|l|}{$2$}&	\multicolumn{1}{|l|}{ $3$}	\\
\cline{2-4}
\multicolumn{1}{|l|}{egzaminator}&	\multicolumn{1}{|l|}{Uzyskana liczba pkt}&	\multicolumn{1}{|l|}{}&	\multicolumn{1}{|l|}{}	\\
\hline
\end{tabular}

\end{center}
$\mathrm{E}\mathrm{M}\mathrm{A}\mathrm{P}-\mathrm{R}0_{-}100$

Strona 7 z29





Zadanie 7. $(0-3$\}

Danyjest $\mathrm{c}\mathrm{z}\mathrm{w}\mathrm{o}\mathrm{r}\mathrm{o}\mathrm{k}_{\mathrm{c}}$]$\mathrm{t}$ wypukly ABCD. $\mathrm{P}\mathrm{r}\mathrm{z}\mathrm{e}\mathrm{k}_{\mathrm{c}}$]$\mathrm{t}\mathrm{n}\mathrm{e} AC$ oraz $BD$ tego czworokqta przecinajq

si9 w punkcie $S.$

Wykaz, $\dot{\mathrm{z}}\mathrm{e}\mathrm{j}\mathrm{e}\dot{\mathrm{z}}$ eli $\displaystyle \frac{|AS|}{|DS|}=\frac{|BS|}{|CS|}$, to na czworokqcie ABCD $\mathrm{m}\mathrm{o}\dot{\mathrm{z}}$ na opisač okrqg.

Strona 8 z29

$\mathrm{E}\mathrm{M}\mathrm{A}\mathrm{P}-\mathrm{R}0_{-}100$





Wypelnia

egzaminator

Nr zadania

Maks. liczba pkt

Uzyskana liczba pkt

7.

3

-RO-100

Strona 9 z29





Zadanie 8. (0-3)

Rozwazamy wszystkie liczby naturalne, w których zapisie dziesietnym nie powtarza sie

jakakolwiek cyfra oraz dokladnie trzy cyfry sq nieparzyste i dokladnie dwie cyfry sq parzyste.

Oblicz, ile jest wszystkich takich liczb.

Strona 10 z29

$\mathrm{E}\mathrm{M}\mathrm{A}\mathrm{P}-\mathrm{R}0_{-}100$



\end{document}