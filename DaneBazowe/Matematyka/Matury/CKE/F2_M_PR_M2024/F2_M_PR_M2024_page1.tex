\documentclass[a4paper,12pt]{article}
\usepackage{latexsym}
\usepackage{amsmath}
\usepackage{amssymb}
\usepackage{graphicx}
\usepackage{wrapfig}
\pagestyle{plain}
\usepackage{fancybox}
\usepackage{bm}

\begin{document}

lnstrukcja dla zdajqcego

l. Sprawdz', czy arkusz egzaminacyjny zawiera 29 stron (zadania $1-16$).

Ewentualny brak zgloś przewodniczqcemu zespolu nadzorujqcego egzamin.

2. Na pierwszej stronie arkusza oraz na karcie odpowiedzi wpisz swój numer PESEL

i przyklej naklejke z kodem.

3. Odpowiedzi do zadań $\mathrm{z}\mathrm{a}\mathrm{m}\mathrm{k}\mathrm{n}\mathrm{i}9$tych ($1-4)$ zaznacz na karcie odpowiedzi w cześci karty

przeznaczonej dla zdajacego. Zamaluj $\blacksquare$ pola do tego przeznaczone. $\mathrm{B}9\mathrm{d}\mathrm{n}\mathrm{e}$

zaznaczenie otocz kólkiem\copyright izaznacz wlaściwe.

4. $\mathrm{W}$ zadaniu 5. wpisz odpowiednie cyfry w kratki pod treścia zadania.

5. $\mathrm{P}\mathrm{a}\mathrm{m}\mathrm{i}9\mathrm{t}\mathrm{a}\mathrm{j}, \dot{\mathrm{z}}\mathrm{e}$ pominiecie argumentacji lub istotnych obliczeń w rozwiqzaniu zadania

otwartego (6-16) $\mathrm{m}\mathrm{o}\dot{\mathrm{z}}\mathrm{e}$ spowodowač, $\dot{\mathrm{z}}\mathrm{e}$ za to rozwiqzanie nie otrzymasz pelnej liczby

punktów.

6. Rozwiqzania zadań i odpowiedzi wpisuj w miejscu na to przeznaczonym.

7. Pisz czytelnie i $\mathrm{u}\dot{\mathrm{z}}$ ywaj tylko dlugopisu lub pióra z czarnym tuszem lub atramentem.

8. Nie $\mathrm{u}\dot{\mathrm{z}}$ ywaj korektora, a bledne zapisy wyra $\acute{\mathrm{z}}$ nie przekreśl.

9. Nie wpisuj $\dot{\mathrm{z}}$ adnych znaków w cześci przeznaczonej dla egzaminatora.

10. Pamietaj, $\dot{\mathrm{z}}\mathrm{e}$ zapisy w brudnopisie nie bedq oceniane.

11. $\mathrm{M}\mathrm{o}\dot{\mathrm{z}}$ esz korzystač z Wybranych wzoróvv matematycznych, cyrkla i linijki oraz kalkulatora

prostego. Upewnij $\mathrm{s}\mathrm{i}\mathrm{e}$, czy przekazano Ci broszur9 z ok1adka takq jak widoczna ponizej.

$\text{{\it á}}_{-,\rightarrow f'(^{\wedge}x_{0})}^{n_{è\mathrm{A}\cdot\alpha}h}$

$\rightarrow$2'$|.(${\it ra}$\vartheta\eta\hat{}\tilde{}\hat{}${\it h}A$+$`{\it r}$\grave{}|${\it ua}.$\approx\acute{}${\it g}.`{\it u}..'$|\Delta${\it h}A$\sqrt{}>${\it u}$\acute{}$30-('

$-\rightarrow 3$

$\mathrm{q},1\cdots\cdot 1\cup \mathrm{R} \varsigma..\vee\prime:\tilde{\mathrm{v}}\mathrm{k}r.7k\cdot(\mathrm{n}\rightarrow\prime.$

$\overline{\mathrm{w}u\mathrm{r}}$[‡@]$\mathrm{r}\mathrm{w} --\overline{\underline{\mathrm{R}\infty-}},\bullet$

Strona 2 z29

$\mathrm{E}\mathrm{M}\mathrm{A}\mathrm{P}-\mathrm{R}0_{-}100$
\end{document}
