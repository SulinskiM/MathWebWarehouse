\documentclass[a4paper,12pt]{article}
\usepackage{latexsym}
\usepackage{amsmath}
\usepackage{amssymb}
\usepackage{graphicx}
\usepackage{wrapfig}
\pagestyle{plain}
\usepackage{fancybox}
\usepackage{bm}

\begin{document}

Zadanie 9. (0-3)

Funkcja f jest określona wzorem

$f(x)=\displaystyle \frac{x^{3}-3x+2}{\chi}$

dla $\mathrm{k}\mathrm{a}\dot{\mathrm{z}}$ dej liczby rzeczywistej $x$ róznej od zera. Punkt $P$, o pierwszej wspólrz9dnej

równej 2, na1ez $\mathrm{y}$ do wykresu funkcji $f$. Prosta o równaniu $y=ax+b$ jest styczna do

wykresu funkcji $f$ w punkcie $P.$

Oblicz wspólczynniki $a$ oraz $b$ w równaniu tej stycznej.
\begin{center}
\begin{tabular}{|l|l|l|l|}
\cline{2-4}
&	\multicolumn{1}{|l|}{Nr zadania}&	\multicolumn{1}{|l|}{$8.$}&	\multicolumn{1}{|l|}{ $9.$}	\\
\cline{2-4}
&	\multicolumn{1}{|l|}{Maks. liczba pkt}&	\multicolumn{1}{|l|}{$3$}&	\multicolumn{1}{|l|}{ $3$}	\\
\cline{2-4}
\multicolumn{1}{|l|}{egzaminator}&	\multicolumn{1}{|l|}{Uzyskana liczba pkt}&	\multicolumn{1}{|l|}{}&	\multicolumn{1}{|l|}{}	\\
\hline
\end{tabular}

\end{center}
$\mathrm{E}\mathrm{M}\mathrm{A}\mathrm{P}-\mathrm{R}0_{-}100$

Strona ll z29
\end{document}
