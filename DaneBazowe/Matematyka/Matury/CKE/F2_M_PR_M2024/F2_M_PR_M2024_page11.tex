\documentclass[a4paper,12pt]{article}
\usepackage{latexsym}
\usepackage{amsmath}
\usepackage{amssymb}
\usepackage{graphicx}
\usepackage{wrapfig}
\pagestyle{plain}
\usepackage{fancybox}
\usepackage{bm}

\begin{document}

Zadanie 10. (0-3)

Spośród wszystkich liczb naturalnych sześciocyfrowych, których wszystkie cyfry naleza do

zbioru \{1, 2, 3, 4, 5, 6, 7, 8\}, 1osujemy jednq. Wy1osowanie $\mathrm{k}\mathrm{a}\dot{\mathrm{z}}$ dej z tych liczb jest jednakowo

prawdopodobne.

Oblicz prawdopodobieństwo zdarzenia polegajqcego na tym, $\dot{\mathrm{z}}\mathrm{e}$ wylosujemy liczbe, która

ma nastppujqca wlasnośč: kolejne cyfry tej liczby (liczqc od lewej strony) $\mathrm{t}\mathrm{w}\mathrm{o}\mathrm{r}\mathrm{Z}_{\mathrm{c}1}-\mathrm{w}$ podanej

kolejności- sześciowyrazowy ciqg malejqcy.

Strona 12 z29

$\mathrm{E}\mathrm{M}\mathrm{A}\mathrm{P}-\mathrm{R}0_{-}100$
\end{document}
