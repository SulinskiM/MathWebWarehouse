\documentclass[a4paper,12pt]{article}
\usepackage{latexsym}
\usepackage{amsmath}
\usepackage{amssymb}
\usepackage{graphicx}
\usepackage{wrapfig}
\pagestyle{plain}
\usepackage{fancybox}
\usepackage{bm}

\begin{document}

Zadanie 14. $(0-5$\}

$\acute{\mathrm{S}}$ rodek $S$ okregu o promieniu $\sqrt{5} \mathrm{l}\mathrm{e}\dot{\mathrm{z}}\mathrm{y}$ na prostej o równaniu $y=x+1$. Przez punkt

$A=(1,2)$, którego odleglośč od punktu $S$ jest wieksza od $\sqrt{5}$, poprowadzono dwie proste

styczne do tego okregu w punktach- odpowiednio -$B \mathrm{i} C$. Pole czworokata ABSC jest

równe 15.

Oblicz wspólrzedne punktu $S$. Rozwaz wszystkie przypadki.

Strona 20 z29

$\mathrm{E}\mathrm{M}\mathrm{A}\mathrm{P}-\mathrm{R}0_{-}100$
\end{document}
