\documentclass[a4paper,12pt]{article}
\usepackage{latexsym}
\usepackage{amsmath}
\usepackage{amssymb}
\usepackage{graphicx}
\usepackage{wrapfig}
\pagestyle{plain}
\usepackage{fancybox}
\usepackage{bm}

\begin{document}

Zadanie 16. (0-6)

Rozwazamy wszystkie graniastoslupy prawidlowe trójkqtne o objetości 3456, których

$\mathrm{k}\mathrm{r}\mathrm{a}\mathrm{w}9^{\mathrm{d}\acute{\mathrm{Z}}}$ podstawy ma dlugośč nie wipkszq $\mathrm{n}\mathrm{i}\dot{\mathrm{z}} 8\sqrt{3}.$

a)

Wykaz, $\dot{\mathrm{z}}\mathrm{e}$ pole $P$ powierzchni calkowitej graniastoslupa w zalezności od dlugości $a$

$\mathrm{k}\mathrm{r}\mathrm{a}\mathrm{w}9^{\mathrm{d}\mathrm{z}\mathrm{i}}$ podstawy $\mathrm{g}\mathrm{r}\mathrm{a}\mathrm{n}\mathrm{i}\mathrm{a}\mathrm{s}\mathrm{t}\mathrm{o}\mathrm{s}\mathrm{u}\mathrm{p}\mathrm{a}$ jest określone wzorem

$P(a)=\displaystyle \frac{a^{2}\cdot\sqrt{3}}{2}+\frac{13824\sqrt{3}}{a}$

b) Pole $P$ powierzchni calkowitej graniastoslupa w zalezności od d$\dagger$ugości $a$ krawedzi

podstawy graniastoslupa jest określone wzorem

$P(a)=\displaystyle \frac{a^{2}\cdot\sqrt{3}}{2}+\frac{13824\sqrt{3}}{a}$

dla $a\in(0,8\sqrt{3}\rangle.$

Wyznacz dlugość krawedzi podstawy tego z rozwazanych graniastoslupów, którego pole

powierzchni calkowitej jest najmniejsze. Oblicz to najmniejsze pole.

$\mathrm{E}\mathrm{M}\mathrm{A}\mathrm{P}-\mathrm{R}0_{-}100$

Strona 25 z29
\end{document}
