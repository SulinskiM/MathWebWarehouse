\documentclass[a4paper,12pt]{article}
\usepackage{latexsym}
\usepackage{amsmath}
\usepackage{amssymb}
\usepackage{graphicx}
\usepackage{wrapfig}
\pagestyle{plain}
\usepackage{fancybox}
\usepackage{bm}

\begin{document}

{\it W kazdym z zadań od f. do 4. wybierz i zaznacz na karcie odpowiedzi poprawnq odpowiedz}'.

Zadanie $1_{p}(0-1)$

Odleglośč punktu $A=(6,2)$ od prostej o równaniu $5x-12y+1=0$ jest równa

A. $\displaystyle \frac{7}{13}$

B. $\displaystyle \frac{7}{12}$

C. $\displaystyle \frac{5}{12}$

D. $\displaystyle \frac{12}{13}$

Zadanie 2. (0-1)

Równanie $|2x-4|=3x+1$ w zbiorze liczb rzeczywistych

A. nie ma rozwiazań.

B. ma dokladnie jedno rozwiazanie.

C. ma dokladnie dwa rozwiqzania.

D. ma dokladnie cztery rozwiazania.

Zadanie 3. $(0-l\displaystyle \int$

Funkcja $f$ jest określona wzorem $f(x)=|-(x+2)^{3}+5|$ dla $\mathrm{k}\mathrm{a}\dot{\mathrm{z}}$ dej liczby

rzeczywistej $x$. Zbiorem wartości funkcji $f$ jest przedzial

A. $\langle-2, +\infty)$

B. $\langle 0, +\infty)$

C. $\langle 3, +\infty)$

D. $\langle 5, +\infty)$

Zadanie 4. $\{0-1\}$

Granica $\displaystyle \lim_{\chi\rightarrow+\infty}\frac{1+3a+2ax+ax^{3}}{3+4x+5x^{2}+5x^{3}}$ jest równa 3. Wtedy

A. $a=3$

B. $a=9$

C. $a=15$

D. $a=21$

Strona 4 z29

$\mathrm{E}\mathrm{M}\mathrm{A}\mathrm{P}-\mathrm{R}0_{-}100$
\end{document}
