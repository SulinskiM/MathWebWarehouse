% This LaTeX document needs to be compiled with XeLaTeX.
\documentclass[10pt]{article}
\usepackage[utf8]{inputenc}
\usepackage{ucharclasses}
\usepackage{graphicx}
\usepackage[export]{adjustbox}
\graphicspath{ {./images/} }
\usepackage{amsmath}
\usepackage{amsfonts}
\usepackage{amssymb}
\usepackage[version=4]{mhchem}
\usepackage{stmaryrd}
\usepackage{multirow}
\usepackage[fallback]{xeCJK}
\usepackage{polyglossia}
\usepackage{fontspec}
\IfFontExistsTF{Noto Serif CJK KR}
{\setCJKmainfont{Noto Serif CJK KR}}
{\IfFontExistsTF{Apple SD Gothic Neo}
  {\setCJKmainfont{Apple SD Gothic Neo}}
  {\IfFontExistsTF{UnDotum}
    {\setCJKmainfont{UnDotum}}
    {\setCJKmainfont{Malgun Gothic}}
}}

\setmainlanguage{polish}
\setotherlanguages{hindi}
\IfFontExistsTF{Noto Serif Devanagari}
{\newfontfamily\hindifont{Noto Serif Devanagari}}
{\IfFontExistsTF{Kohinoor Devanagari}
  {\newfontfamily\hindifont{Kohinoor Devanagari}}
  {\IfFontExistsTF{Devanagari MT}
    {\newfontfamily\hindifont{Devanagari MT}}
    {\IfFontExistsTF{Lohit Devanagari}
      {\newfontfamily\hindifont{Lohit Devanagari}}
      {\IfFontExistsTF{FreeSerif}
        {\newfontfamily\hindifont{FreeSerif}}
        {\newfontfamily\hindifont{Arial Unicode MS}}
}}}}
\IfFontExistsTF{CMU Serif}
{\newfontfamily\lgcfont{CMU Serif}}
{\IfFontExistsTF{DejaVu Sans}
  {\newfontfamily\lgcfont{DejaVu Sans}}
  {\newfontfamily\lgcfont{Georgia}}
}
\setDefaultTransitions{\lgcfont}{}
\setTransitionsForDevanagari{\hindifont}{\rmfamily}

\newcommand\Varangle{\mathop{{<\!\!\!\!\!\text{\small)}}\:}\nolimits}

\begin{document}
\begin{center}
\includegraphics[max width=\textwidth]{2025_02_09_654ac8557e15ed1a98aag-01}
\end{center}

\section*{EGZAMIN MATURALNY Z MATEMATYKI POZIOM PODSTAWOWY}
\section*{WYPELNIA ZESPÓŁ NADZORUJACY}
Uprawnienia zdającego do: dostosowania zasad oceniania dostosowania w zw. z dyskalkulią nieprzenoszenia zaznaczeń na kartę.

MMAP-P0-100-2203

\section*{Instrukcja dla zdającego}
\begin{enumerate}
  \item Sprawdź, czy arkusz egzaminacyjny zawiera 31 stron (zadania 1-30). Ewentualny brak zgłoś przewodniczącemu zespołu nadzorującego egzamin.
  \item Na tej stronie oraz na karcie odpowiedzi wpisz swój numer PESEL i przyklej naklejkę z kodem.
  \item Nie wpisuj żadnych znaków w części przeznaczonej dla egzaminatora.
  \item Rozwiązania zadań i odpowiedzi wpisuj w miejscu na to przeznaczonym.
  \item Symbol zamieszczony w nagłówku zadania oznacza, że rozwiązanie zadania zamkniętego musisz przenieść na kartę odpowiedzi.
  \item Odpowiedzi do zadań zamkniętych zaznacz na karcie odpowiedzi w części karty przeznaczonej dla zdającego. Zamaluj \(\square\) pola do tego przeznaczone. Błędne zaznaczenie otocz kółkiem © i zaznacz właściwe.
  \item Pamiętaj, że pominięcie argumentacji lub istotnych obliczeń w rozwiązaniu zadania otwartego może spowodować, że za to rozwiązanie nie otrzymasz pełnej liczby punktów.
  \item Pisz czytelnie i używaj tylko długopisu lub pióra z czarnym tuszem lub atramentem.
  \item Nie używaj korektora, a błędne zapisy wyraźnie przekreśl.
  \item Pamiętaj, że zapisy w brudnopisie nie będą oceniane.
  \item Możesz korzystać z zestawu wzorów matematycznych, cyrkla i linijki oraz kalkulatora prostego.
\end{enumerate}

Zadanie 1. (0-1) 뚱\\
Dokończ zdanie. Wybierz właściwą odpowiedź spośród podanych.\\
Wartość wyrażenia \(6^{100}+6^{100}+6^{100}+6^{100}+6^{100}+6^{100}\) jest równa\\
A. \(6^{600}\)\\
B. \(6^{101}\)\\
C. \(36^{100}\)\\
D. \(36^{600}\)

\begin{center}
\begin{tabular}{|c|c|c|c|c|c|c|c|c|c|c|c|c|c|c|c|c|c|c|c|c|c|c|c|c|}
\hline
 & Brudn & dnopi &  &  &  &  &  &  &  &  &  &  &  &  &  &  &  &  &  &  &  &  &  &  \\
\hline
 &  &  &  &  &  &  &  &  &  &  &  &  &  &  &  &  &  &  &  &  &  &  &  &  \\
\hline
 &  &  &  &  &  &  &  &  &  &  &  &  &  &  &  &  &  &  &  &  &  &  &  &  \\
\hline
 &  &  &  &  &  &  &  &  &  &  &  &  &  &  &  &  &  &  &  &  &  &  &  &  \\
\hline
 &  &  &  &  &  &  &  &  &  &  &  &  &  &  &  &  &  &  &  &  &  &  &  &  \\
\hline
 &  &  &  &  &  &  &  &  &  &  &  &  &  &  &  &  &  &  &  &  &  &  &  &  \\
\hline
 &  &  &  &  &  &  &  &  &  &  &  &  &  &  &  &  &  &  &  &  &  &  &  &  \\
\hline
 &  &  &  &  &  &  &  &  &  &  &  &  &  &  &  &  &  &  &  &  &  &  &  &  \\
\hline
 &  &  &  &  &  &  &  &  &  &  &  &  &  &  &  &  &  &  &  &  &  &  &  &  \\
\hline
 &  &  &  &  &  &  &  &  &  &  &  &  &  &  &  &  &  &  &  &  &  &  &  &  \\
\hline
 &  &  &  &  &  &  &  &  &  &  &  &  &  &  &  &  &  &  &  &  &  &  &  &  \\
\hline
 &  &  &  &  &  &  &  &  &  &  &  &  &  &  &  &  &  &  &  &  &  &  &  &  \\
\hline
 &  &  &  &  &  &  &  &  &  &  &  &  &  &  &  &  &  &  &  &  &  &  &  &  \\
\hline
 &  &  &  &  &  &  &  &  &  &  &  &  &  &  &  &  &  &  &  &  &  &  &  &  \\
\hline
 &  &  &  &  &  &  &  &  &  &  &  &  &  &  &  &  &  &  &  &  &  &  &  &  \\
\hline
 &  &  &  &  &  &  &  &  &  &  &  &  &  &  &  &  &  &  &  &  &  &  &  &  \\
\hline
\end{tabular}
\end{center}

Zadanie 2. (0-1)\\
Dokończ zdanie. Wybierz właściwą odpowiedź spośród podanych.\\
Wartość wyrażenia \(\log _{7} 98-\log _{7} 2\) jest równa\\
A. 7\\
B. 2\\
C. 1\\
D. \((-1)\)

\begin{center}
\begin{tabular}{|c|c|c|c|c|c|c|c|c|c|c|c|c|c|c|c|c|c|c|c|c|c|c|}
\hline
\multicolumn{5}{|l|}{Brudnopis} &  &  &  &  &  &  &  &  &  &  &  &  &  &  &  &  &  &  \\
\hline
 &  &  &  &  &  &  &  &  &  &  &  &  &  &  &  &  &  &  &  &  &  &  \\
\hline
 &  &  &  &  &  &  &  &  &  &  &  &  &  &  &  &  &  &  &  &  &  &  \\
\hline
 &  &  &  &  &  &  &  &  &  &  &  &  &  &  &  &  &  &  &  &  &  &  \\
\hline
 &  &  &  &  &  &  &  &  &  &  &  &  &  &  &  &  &  &  &  &  &  &  \\
\hline
 &  &  &  &  &  &  &  &  &  &  &  &  &  &  &  &  &  &  &  &  &  &  \\
\hline
 &  &  &  &  &  &  &  &  &  &  &  &  &  &  &  &  &  &  &  &  &  &  \\
\hline
 &  &  &  &  &  &  &  &  &  &  &  &  &  &  &  &  &  &  &  &  &  &  \\
\hline
 &  &  &  &  &  &  &  &  &  &  &  &  &  &  &  &  &  &  &  &  &  &  \\
\hline
 &  &  &  &  &  &  &  &  &  &  &  &  &  &  &  &  &  &  &  &  &  &  \\
\hline
 &  &  &  &  &  &  &  &  &  &  &  &  &  &  &  &  &  &  &  &  &  &  \\
\hline
 &  &  &  &  &  &  &  &  &  &  &  &  &  &  &  &  &  &  &  &  &  &  \\
\hline
 &  &  &  &  &  &  &  &  &  &  &  &  &  &  &  &  &  &  &  &  &  &  \\
\hline
 &  &  &  &  &  &  &  &  &  &  &  &  &  &  &  &  &  &  &  &  &  &  \\
\hline
 &  &  &  &  &  &  &  &  &  &  &  &  &  &  &  &  &  &  &  &  &  &  \\
\hline
 &  &  &  &  &  &  &  &  &  &  &  &  &  &  &  &  &  &  &  &  &  &  \\
\hline
\end{tabular}
\end{center}

Zadanie 3. (0-1) ax+\\
Dokończ zdanie. Wybierz właściwą odpowiedź spośród podanych.\\
Wszystkich liczb naturalnych trzycyfrowych, w których zapisie dziesiętnym nie występuje cyfra 2, jest\\
A. 900\\
B. 729\\
C. 648\\
D. 512

\begin{center}
\begin{tabular}{|c|c|c|c|c|c|c|c|c|c|c|c|c|c|c|c|c|c|c|c|c|c|c|c|c|}
\hline
 & rudn & opis &  &  &  &  &  &  &  &  &  &  &  &  &  &  &  &  &  &  &  &  &  &  \\
\hline
 &  &  &  &  &  &  &  &  &  &  &  &  &  &  &  &  &  &  &  &  &  &  &  &  \\
\hline
 &  &  &  &  &  &  &  &  &  &  &  &  &  &  &  &  &  &  &  &  &  &  &  &  \\
\hline
 &  &  &  &  &  &  &  &  &  &  &  &  &  &  &  &  &  &  &  &  &  &  &  &  \\
\hline
 &  &  &  &  &  &  &  &  &  &  &  &  &  &  &  &  &  &  &  &  &  &  &  &  \\
\hline
 &  &  &  &  &  &  &  &  &  &  &  &  &  &  &  &  &  &  &  &  &  &  &  &  \\
\hline
 &  &  &  &  &  &  &  &  &  &  &  &  &  &  &  &  &  &  &  &  &  &  &  &  \\
\hline
 &  &  &  &  &  &  &  &  &  &  &  &  &  &  &  &  &  &  &  &  &  &  &  &  \\
\hline
 &  &  &  &  &  &  &  &  &  &  &  &  &  &  &  &  &  &  &  &  &  &  &  &  \\
\hline
 &  &  &  &  &  &  &  &  &  &  &  &  &  &  &  &  &  &  &  &  &  &  &  &  \\
\hline
 &  &  &  &  &  &  &  &  &  &  &  &  &  &  &  &  &  &  &  &  &  &  &  &  \\
\hline
 &  &  &  &  &  &  &  &  &  &  &  &  &  &  &  &  &  &  &  &  &  &  &  &  \\
\hline
 &  &  &  &  &  &  &  &  &  &  &  &  &  &  &  &  &  &  &  &  &  &  &  &  \\
\hline
 &  &  &  &  &  &  &  &  &  &  &  &  &  &  &  &  &  &  &  &  &  &  &  &  \\
\hline
 &  &  &  &  &  &  &  &  &  &  &  &  &  &  &  &  &  &  &  &  &  &  &  &  \\
\hline
 &  &  &  &  &  &  &  &  &  &  &  &  &  &  &  &  &  &  &  &  &  &  &  &  \\
\hline
\end{tabular}
\end{center}

\section*{Zadanie 4. (0-1)}
Dokończ zdanie. Wybierz właściwą odpowiedź spośród podanych.\\
Dla każdej liczby rzeczywistej \(a\) wartość wyrażenia \((3+4 a)^{2}-(3-4 a)^{2}\) jest równa\\
A. \(32 a^{2}\)\\
B. 0\\
C. \(48 a\)\\
D. \(8 a^{2}\)\\
\includegraphics[max width=\textwidth, center]{2025_02_09_654ac8557e15ed1a98aag-03}

\section*{Zadanie 5. (0-2)}
Dane są dwie przecinające się proste. Miary kątów utworzonych przez te proste zapisano za pomocą wyrażeń algebraicznych (zobacz rysunek).\\
\includegraphics[max width=\textwidth, center]{2025_02_09_654ac8557e15ed1a98aag-04}

Dokończ zdanie. Wybierz dwie odpowiedzi, tak aby dla każdej z nich dokończenie poniższego zdania było prawdziwe.

Układem równań, w którym zapisano prawidłowe zależności między miarami kątów utworzonych przez te proste, jest układ\\
A. \(\left\{\begin{array}{l}(\alpha+\beta)+\beta=90^{\circ} \\ \alpha+\beta=2 \alpha-\beta\end{array}\right.\)\\
B. \(\left\{\begin{array}{l}(\alpha+\beta)+\beta=180^{\circ} \\ \alpha+\beta=2 \alpha-\beta\end{array}\right.\)\\
C. \(\left\{\begin{array}{l}(\alpha+\beta)+\beta=180^{\circ} \\ \beta=2 \alpha-\beta\end{array}\right.\)\\
D. \(\left\{\begin{array}{l}\alpha+\beta=90^{\circ} \\ \beta=2 \alpha-\beta\end{array}\right.\)\\
E. \(\left\{\begin{array}{l}\alpha+\beta=2 \alpha-\beta \\ 180^{\circ}-(2 \alpha-\beta)=\beta\end{array}\right.\)\\
F. \(\left\{\begin{array}{l}3 \alpha+2 \beta=360^{\circ} \\ 2 \alpha-\beta=2 \beta\end{array}\right.\)\\
\includegraphics[max width=\textwidth, center]{2025_02_09_654ac8557e15ed1a98aag-04(1)}

Zadanie 6. (0-1)\\
Dany jest wielomian

\[
W(x)=3 x^{3}+k x^{2}-12 x-7 k+12
\]

gdzie \(k\) jest pewną liczbą rzeczywistą. Wiadomo, że liczba (-2) jest pierwiastkiem tego wielomianu.

Dokończ zdanie. Wybierz właściwą odpowiedź spośród podanych.\\
Liczba \(k\) jest równa\\
A. 2\\
B. 4\\
C. 6\\
D. 8\\
\includegraphics[max width=\textwidth, center]{2025_02_09_654ac8557e15ed1a98aag-05}

\section*{Zadanie 7. (0-1) ब.}
Dokończ zdanie. Wybierz właściwą odpowiedź spośród podanych.\\
Równanie

\[
\frac{(4 x-6)(x-2)^{2}}{2 x(x-1,5)(x+6)}=0
\]

ma w zbiorze liczb rzeczywistych\\
A. dokładnie jedno rozwiązanie: \(x=2\).\\
B. dokładnie dwa rozwiązania: \(x=1,5, x=2\).\\
C. dokładnie trzy rozwiązania: \(x=-6, x=0, x=2\).\\
D. dokładnie cztery rozwiązania: \(x=-6, x=0, x=1,5, x=2\).\\
\includegraphics[max width=\textwidth, center]{2025_02_09_654ac8557e15ed1a98aag-05(1)}

Zadanie 8. (0-1)\\
Spośród rysunków A-D wybierz ten, na którym prawidłowo zaznaczono na osi liczbowej zbiór wszystkich liczb rzeczywistych spełniających nierówność:

\[
|x+1| \leq 2
\]

A.\\
\includegraphics[max width=\textwidth, center]{2025_02_09_654ac8557e15ed1a98aag-06(2)}\\
B.\\
\includegraphics[max width=\textwidth, center]{2025_02_09_654ac8557e15ed1a98aag-06}\\
C.\\
\includegraphics[max width=\textwidth, center]{2025_02_09_654ac8557e15ed1a98aag-06(1)}\\
D.\\
\includegraphics[max width=\textwidth, center]{2025_02_09_654ac8557e15ed1a98aag-06(3)}

Zadanie 9. (0-2)\\
Wykaż, że dla każdej liczby całkowitej nieparzystej \(n\) liczba \(n^{2}+2023\) jest podzielna przez 8.

\begin{center}
\begin{tabular}{|c|c|c|c|c|c|c|c|c|c|c|c|c|c|c|c|c|c|c|c|c|c|c|c|}
\hline
 &  &  &  &  &  &  &  &  &  &  &  &  &  &  &  &  &  &  &  &  &  &  &  \\
\hline
 &  &  &  &  &  &  &  &  &  &  &  &  &  &  &  &  &  &  &  &  &  &  &  \\
\hline
 &  &  &  &  &  &  &  &  &  &  &  &  &  &  &  &  &  &  &  &  &  &  &  \\
\hline
 &  &  &  &  &  &  &  &  &  &  &  &  &  &  &  &  &  &  &  &  &  &  &  \\
\hline
 &  &  &  &  &  &  &  &  &  &  &  &  &  &  &  &  &  &  &  &  &  &  &  \\
\hline
 &  &  &  &  &  &  &  &  &  &  &  &  &  &  &  &  &  &  &  &  &  &  &  \\
\hline
 &  &  &  &  &  &  &  &  &  &  &  &  &  &  &  &  &  &  &  &  &  &  &  \\
\hline
 &  &  &  &  &  &  &  &  &  &  &  &  &  &  &  &  &  &  &  &  &  &  &  \\
\hline
 &  &  &  &  &  &  &  &  &  &  &  &  &  &  &  &  &  &  &  &  &  &  &  \\
\hline
 &  &  &  &  &  &  &  &  &  &  &  &  &  &  &  &  &  &  &  &  &  &  &  \\
\hline
 &  &  &  &  &  &  &  &  &  &  &  &  &  &  &  &  &  &  &  &  &  &  &  \\
\hline
 &  &  &  &  &  &  &  &  &  &  &  &  &  &  &  &  &  &  &  &  &  &  &  \\
\hline
 &  &  &  &  &  &  &  &  &  &  &  &  &  &  &  &  &  &  &  &  &  &  &  \\
\hline
 &  &  &  &  &  &  &  &  &  &  &  &  &  &  &  &  &  &  &  &  &  &  &  \\
\hline
 &  &  &  &  &  &  &  &  &  &  &  &  &  &  &  &  &  &  &  &  &  &  &  \\
\hline
 &  &  &  &  &  &  &  &  &  &  &  &  &  &  &  &  &  &  &  &  &  &  &  \\
\hline
 &  &  &  &  &  &  &  &  &  &  &  &  &  &  &  &  &  &  &  &  &  &  &  \\
\hline
 &  &  &  &  &  &  &  &  &  &  &  &  &  &  &  &  &  &  &  &  &  &  &  \\
\hline
 &  &  &  &  &  &  &  &  &  &  &  &  &  &  &  &  &  &  &  &  &  &  &  \\
\hline
 &  &  &  &  &  &  &  &  &  &  &  &  &  &  &  &  &  &  &  &  &  &  &  \\
\hline
 &  &  &  &  &  &  &  &  &  &  &  &  &  &  &  &  &  &  &  &  &  &  &  \\
\hline
 &  &  &  &  &  &  &  &  &  &  &  &  &  &  &  &  &  &  &  &  &  &  &  \\
\hline
 &  &  &  &  &  &  &  &  &  &  &  &  &  &  &  &  &  &  &  &  &  &  &  \\
\hline
 &  &  &  &  &  &  &  &  &  &  &  &  &  &  &  &  &  &  &  &  &  &  &  \\
\hline
 &  &  &  &  &  &  &  &  &  &  &  &  &  &  &  &  &  &  &  &  &  &  &  \\
\hline
 &  &  &  &  &  &  &  &  &  &  &  &  &  &  &  &  &  &  &  &  &  &  &  \\
\hline
 &  &  &  &  &  &  &  &  &  &  &  &  &  &  &  &  &  &  &  &  &  &  &  \\
\hline
 &  &  &  &  &  &  &  &  &  &  &  &  &  &  &  &  &  &  &  &  &  &  &  \\
\hline
 &  &  &  &  &  &  &  &  &  &  &  &  &  &  &  &  &  &  &  &  &  &  &  \\
\hline
 &  &  &  &  &  &  &  &  &  &  &  &  &  &  &  &  &  &  &  &  &  &  &  \\
\hline
 &  &  &  &  &  &  &  &  &  &  &  &  &  &  &  &  &  &  &  &  &  &  &  \\
\hline
 &  &  &  &  &  &  &  &  &  &  &  &  &  &  &  &  &  &  &  &  &  &  &  \\
\hline
 &  &  &  &  &  &  &  &  &  &  &  &  &  &  &  &  &  &  &  &  &  &  &  \\
\hline
 &  &  &  &  &  &  &  &  &  &  &  &  &  &  &  &  &  &  &  &  &  &  &  \\
\hline
 &  &  &  &  &  &  &  &  &  &  &  &  &  &  &  &  &  &  &  &  &  &  &  \\
\hline
 &  &  &  &  &  &  &  &  &  &  &  &  &  &  &  &  &  &  &  &  &  &  &  \\
\hline
 &  &  &  &  &  &  &  &  &  &  &  &  &  &  &  &  &  &  &  &  &  &  &  \\
\hline
 &  &  &  &  &  &  &  &  &  &  &  &  &  &  &  &  &  &  &  &  &  &  &  \\
\hline
 &  &  &  &  &  &  &  &  &  &  &  &  &  &  &  &  &  &  &  &  &  &  &  \\
\hline
 &  &  &  &  &  &  &  &  &  &  &  &  &  &  &  &  &  &  &  &  &  &  &  \\
\hline
\end{tabular}
\end{center}

\begin{center}
\begin{tabular}{|c|l|c|}
\hline
\multirow{3}{*}{\begin{tabular}{l}
Wypełnia \\
egzaminator \\
\end{tabular}} & Nr zadania & 9. \\
\cline { 2 - 3 }
 & Maks. liczba pkt & 2 \\
\cline { 2 - 3 }
 & Uzyskana liczba pkt &  \\
\hline
\end{tabular}
\end{center}

\section*{Zadanie 10.}
Dana jest funkcja kwadratowa \(f\), której fragment wykresu przedstawiono w kartezjańskim układzie wspórzzędnych ( \(x, y\) ) na rysunku obok. Wierzchołek paraboli, która jest wykresem funkcji \(f\), oraz punkty przecięcia paraboli z osiami układu współrzędnych mają współrzędne całkowite.\\
\includegraphics[max width=\textwidth, center]{2025_02_09_654ac8557e15ed1a98aag-08(4)}

\section*{Zadanie 10.1. (0-1)}
Funkcja \(g\) jest określona za pomocą funkcji \(f\) następująco: \(g(x)=f(x-2)\).\\
Dokończ zdanie. Wybierz właściwą odpowiedź spośród podanych.\\
Wykres funkcji \(g\) przedstawiono na rysunku\\
A.\\
\includegraphics[max width=\textwidth, center]{2025_02_09_654ac8557e15ed1a98aag-08(3)}\\
B.\\
\includegraphics[max width=\textwidth, center]{2025_02_09_654ac8557e15ed1a98aag-08}\\
C.\\
\includegraphics[max width=\textwidth, center]{2025_02_09_654ac8557e15ed1a98aag-08(2)}\\
D.\\
\includegraphics[max width=\textwidth, center]{2025_02_09_654ac8557e15ed1a98aag-08(1)}

Zadanie 10.2. (0-1)\\
Wyznacz i zapisz w miejscu wykropkowanym poniżej zbiór wszystkich rozwiązań nierówności:

\[
f(x) \leq 0
\]

\section*{Zadanie 10.3. (0-3)}
Wyznacz wzór funkcji kwadratowej \(\boldsymbol{f}\) w postaci kanonicznej.\\
Zapisz obliczenia.

\begin{center}
\begin{tabular}{|c|c|c|c|c|c|c|c|c|c|c|c|c|c|c|c|c|c|c|c|c|c|c|c|c|c|c|c|}
\hline
 &  &  &  &  &  &  &  &  &  &  &  &  &  &  &  &  &  &  &  &  &  &  &  &  &  &  &  \\
\hline
 &  &  &  &  &  &  &  &  &  &  &  &  &  &  &  &  &  &  &  &  &  &  &  &  &  &  &  \\
\hline
 &  &  &  &  &  &  &  &  &  &  &  &  &  &  &  &  &  &  &  &  &  &  &  &  &  &  &  \\
\hline
 &  &  &  &  &  &  &  &  &  &  &  &  &  &  &  &  &  &  &  &  &  &  &  &  &  &  &  \\
\hline
 &  &  &  &  &  &  &  &  &  &  &  &  &  &  &  &  &  &  &  &  &  &  &  &  &  &  &  \\
\hline
 &  &  &  &  &  &  &  &  &  &  &  &  &  &  &  &  &  &  &  &  &  &  &  &  &  &  &  \\
\hline
 &  &  &  &  &  &  &  &  &  &  &  &  &  &  &  &  &  &  &  &  &  &  &  &  &  &  &  \\
\hline
 &  &  &  &  &  &  &  &  &  &  &  &  &  &  &  &  &  &  &  &  &  &  &  &  &  &  &  \\
\hline
 &  &  &  &  &  &  &  &  &  &  &  &  &  &  &  &  &  &  &  &  &  &  &  &  &  &  &  \\
\hline
 &  &  &  &  &  &  &  &  &  &  &  &  &  &  &  &  &  &  &  &  &  &  &  &  &  &  &  \\
\hline
 &  &  &  &  &  &  &  &  &  &  &  &  &  &  &  &  &  &  &  &  &  &  &  &  &  &  &  \\
\hline
 &  &  &  &  &  &  &  &  &  &  &  &  &  &  &  &  &  &  &  &  &  &  &  &  &  &  &  \\
\hline
 &  &  &  &  &  &  &  &  &  &  &  &  &  &  &  &  &  &  &  &  &  &  &  &  &  &  &  \\
\hline
 &  &  &  &  &  &  &  &  &  &  &  &  &  &  &  &  &  &  &  &  &  &  &  &  &  &  &  \\
\hline
 &  &  &  &  &  &  &  &  &  &  &  &  &  &  &  &  &  &  &  &  &  &  &  &  &  &  &  \\
\hline
 &  &  &  &  &  &  &  &  &  &  &  &  &  &  &  &  &  &  &  &  &  &  &  &  &  &  &  \\
\hline
 &  &  &  &  &  &  &  &  &  &  &  &  &  &  &  &  &  &  &  &  &  &  &  &  &  &  &  \\
\hline
 &  &  &  &  &  &  &  &  &  &  &  &  &  &  &  &  &  &  &  &  &  &  &  &  &  &  &  \\
\hline
 &  &  &  &  &  &  &  &  &  &  &  &  &  &  &  &  &  &  &  &  &  &  &  &  &  &  &  \\
\hline
 &  &  &  &  &  &  &  &  &  &  &  &  &  &  &  &  &  &  &  &  &  &  &  &  &  &  &  \\
\hline
 &  &  &  &  &  &  &  &  &  &  &  &  &  &  &  &  &  &  &  &  &  &  &  &  &  &  &  \\
\hline
 &  &  &  &  &  &  &  &  &  &  &  &  &  &  &  &  &  &  &  &  &  &  &  &  &  &  &  \\
\hline
 &  &  &  &  &  &  &  &  &  &  &  &  &  &  &  &  &  &  &  &  &  &  &  &  &  &  &  \\
\hline
 &  &  &  &  &  &  &  &  &  &  &  &  &  &  &  &  &  &  &  &  &  &  &  &  &  &  &  \\
\hline
 &  &  &  &  &  &  &  &  &  &  &  &  &  &  &  &  &  &  &  &  &  &  &  &  &  &  &  \\
\hline
 &  &  &  &  &  &  &  &  &  &  &  &  &  &  &  &  &  &  &  &  &  &  &  &  &  &  &  \\
\hline
 &  &  &  &  &  &  &  &  &  &  &  &  &  &  &  &  &  &  &  &  &  &  &  &  &  &  &  \\
\hline
 &  &  &  &  &  &  &  &  &  &  &  &  &  &  &  &  &  &  &  &  &  &  &  &  &  &  &  \\
\hline
 &  &  &  &  &  &  &  &  &  &  &  &  &  &  &  &  &  &  &  &  &  &  &  &  &  &  &  \\
\hline
 &  &  &  &  &  &  &  &  &  &  &  &  &  &  &  &  &  &  &  &  &  &  &  &  &  &  &  \\
\hline
 & | &  &  &  &  &  &  &  &  &  &  &  &  &  &  &  &  &  &  &  &  &  &  &  &  &  &  \\
\hline
 &  &  &  &  &  &  &  &  &  &  &  &  &  &  &  &  &  &  &  &  &  &  &  &  &  &  &  \\
\hline
\end{tabular}
\end{center}

\begin{center}
\begin{tabular}{|c|l|c|c|}
\hline
\multirow{3}{*}{\begin{tabular}{c}
Wypełnia \\
egzaminator \\
\end{tabular}} & Nr zadania & 10.2. & 10.3. \\
\cline { 2 - 4 }
 & Maks. liczba pkt & 1 & 3 \\
\cline { 2 - 4 }
 & Uzyskana liczba pkt &  &  \\
\hline
\end{tabular}
\end{center}

Zadanie 11. (0-1)\\
Dana jest funkcja liniowa \(f\) określona wzorem \(f(x)=a x+b\), gdzie \(a\) i \(b\) są liczbami rzeczywistymi. Wykres funkcji \(f\) przedstawiono w kartezjańskim układzie współrzędnych ( \(x, y\) ) na rysunku obok.\\
\includegraphics[max width=\textwidth, center]{2025_02_09_654ac8557e15ed1a98aag-10}

Dokończ zdanie. Wybierz właściwą odpowiedź spośród podanych.

Współczynniki \(a\) i \(b\) we wzorze funkcji \(f\) spełniają warunki\\
A. \(a>0\) i \(b>0\).\\
B. \(a>0\) i \(b<0\).\\
C. \(a<0\) i \(b>0\).\\
D. \(a<0\) i \(b<0\).\\
\includegraphics[max width=\textwidth, center]{2025_02_09_654ac8557e15ed1a98aag-10(1)}

\section*{Zadanie 12. (0-1) 뚱}
Firma przeprowadziła badania rynkowe dotyczące wpływu zmiany ceny \(P\) swojego produktu na liczbę \(Q\) kupujących ten produkt. Z badań wynika, że każdorazowe zwiększenie ceny o 1 jednostkę powoduje spadek liczby kupujących o 3 jednostki. Ponadto przy cenie równej 5 jednostek liczba kupujących jest równa 12 jednostek.

Dokończ zdanie. Wybierz właściwą odpowiedź spośród podanych.\\
Funkcja, która opisuje zależność liczby kupujących ten produkt od jego ceny, ma wzór\\
A. \(Q=-0,9 P^{2}+6,9\)\\
B. \(Q=-3 P+27\)\\
C. \(P=-0,9 Q^{2}+6,9\)\\
D. \(P=-3 Q+27\)\\
\includegraphics[max width=\textwidth, center]{2025_02_09_654ac8557e15ed1a98aag-11}

\section*{Zadanie 13.}
Czas \(T\) pótrwania leku w organizmie to czas, po którym masa leku worganizmie zmniejsza się o połowę - po przyjęciu jednorazowej dawki.\\
Przyjmij, że po przyjęciu jednej dawki masa \(m\) leku w organizmie zmienia się w czasie zgodnie z zależnością wykładniczą

\[
m(t)=m_{0} \cdot\left(\frac{1}{2}\right)^{\frac{t}{T}}
\]

gdzie:\\
\(m_{0}\) - masa przyjętej dawki leku\\
\(T\) - czas półtrwania leku\\
\(t\) - czas liczony od momentu przyjęcia dawki.\\
W przypadku przyjęcia kilku(nastu) dawek powyższa zależność pozwala obliczyć, ile leku pozostało w danym momencie w organizmie z każdej poprzednio przyjętej dawki. W ten sposób obliczone masy leku z przyjętych poprzednich dawek sumują się i dają informację̨ o całkowitej aktualnej masie leku w organizmie.

Pacjent otrzymuje co 4 dni o tej samej godzinie dawkę \(m_{0}=100 \mathrm{mg}\) leku L. Czas póttrwania tego leku w organizmie jest równy \(T=4\) doby.

\section*{Zadanie 13.1. (0-1) \\
 Dokończ zdanie. Wybierz właściwą odpowiedź spośród podanych.}
Wykres zależności masy \(M\) leku L w organizmie tego pacjenta od czasu \(t\), liczonego od momentu przyjęcia przez pacjenta pierwszej dawki, przedstawiono na rysunku\\
A.\\
\includegraphics[max width=\textwidth, center]{2025_02_09_654ac8557e15ed1a98aag-12(1)}\\
B.\\
\includegraphics[max width=\textwidth, center]{2025_02_09_654ac8557e15ed1a98aag-12}\\
C.\\
\includegraphics[max width=\textwidth, center]{2025_02_09_654ac8557e15ed1a98aag-13(1)}\\
D.\\
\includegraphics[max width=\textwidth, center]{2025_02_09_654ac8557e15ed1a98aag-13}

Zadanie 13.2. (0-3)\\
Oblicz masę leku L w organizmie tego pacjenta tuż przed przyjęciem jedenastej dawki tego leku. Wynik podaj w zaokrągleniu do \(0,1 \mathrm{mg}\).\\
Zapisz obliczenia.\\
\includegraphics[max width=\textwidth, center]{2025_02_09_654ac8557e15ed1a98aag-13(2)}\\
\includegraphics[max width=\textwidth, center]{2025_02_09_654ac8557e15ed1a98aag-14}

\section*{Zadanie 14. (0-1)}
Klient wpłacił do banku 20000 zł na lokatę dwuletnią. Po każdym rocznym okresie oszczędzania bank dolicza odsetki w wysokości 3\% od kwoty bieżącego kapitału znajdującego się na lokacie.

Dokończ zdanie. Wybierz właściwą odpowiedź spośród podanych.\\
Po 2 latach oszczędzania w tym banku kwota na lokacie (bez uwzględniania podatków) jest równa\\
A. \(20000 \cdot(1,12)^{2}\)\\
B. \(20000 \cdot 2 \cdot 1,03\)\\
C. \(20000 \cdot 1,06\)\\
D. \(20000 \cdot(1,03)^{2}\)\\
\includegraphics[max width=\textwidth, center]{2025_02_09_654ac8557e15ed1a98aag-15}

\section*{Zadanie 15. (0-1) 뚱}
Dany jest ciąg \(\left(a_{n}\right)\) określony wzorem \(a_{n}=-3 n+5\) dla każdej liczby naturalnej \(n \geq 1\).\\
Oceń prawdziwość poniższych stwierdzeń. Wybierz P, jeśli stwierdzenie jest prawdziwe, albo F - jeśli jest fałszywe.

\begin{center}
\begin{tabular}{|l|c|c|}
\hline
\begin{tabular}{l}
Liczby \(2,(-1),(-4)\) są trzema kolejnymi początkowymi wyrazami \\
ciągu \(\left(a_{n}\right)\). \\
\end{tabular} & \(\mathbf{P}\) & \(\mathbf{F}\) \\
\hline
\(\left(a_{n}\right)\) jest ciągiem arytmetycznym o różnicy równej 5. & \(\mathbf{P}\) & \(\mathbf{F}\) \\
\hline
\end{tabular}
\end{center}

\begin{center}
\includegraphics[max width=\textwidth]{2025_02_09_654ac8557e15ed1a98aag-15(1)}
\end{center}

\begin{center}
\begin{tabular}{|c|l|c|}
\hline
\multirow{3}{*}{\begin{tabular}{l}
Wypełnia \\
egzaminator \\
\end{tabular}} & Nr zadania & 13.2. \\
\cline { 2 - 3 }
 & Maks. liczba pkt & 3 \\
\cline { 2 - 3 }
 & Uzyskana liczba pkt &  \\
\hline
\end{tabular}
\end{center}

Zadanie 16. (0-1) птш\\
Dany jest trójkąt \(A B C\), w którym \(|A B|=6,|B C|=5,|A C|=10\).

Oceń prawdziwość poniższych stwierdzeń. Wybierz P, jeśli stwierdzenie jest prawdziwe, albo F - jeśli jest fałszywe.

\begin{center}
\begin{tabular}{|l|c|c|}
\hline
Cosinus kąta \(A B C\) jest równy (-0,65). & \(\mathbf{P}\) & \(\mathbf{F}\) \\
\hline
Trójkąt \(A B C\) jest rozwartokątny. & \(\mathbf{P}\) & \(\mathbf{F}\) \\
\hline
\end{tabular}
\end{center}

\begin{center}
\begin{tabular}{|c|c|c|c|c|c|c|c|c|c|c|c|c|c|c|c|c|c|c|c|c|c|c|c|c|c|}
\hline
\multicolumn{4}{|l|}{Brudnopis} &  &  &  &  & - & - & - & , & - & - &  &  &  & - & , &  &  &  &  &  &  &  \\
\hline
 &  &  &  &  &  &  &  &  &  &  &  &  &  &  &  &  &  &  &  &  &  &  &  &  &  \\
\hline
 &  &  &  &  &  &  &  &  &  &  &  &  &  &  &  &  &  &  &  &  &  &  &  &  &  \\
\hline
 &  &  &  &  &  &  &  &  &  &  &  &  &  &  &  &  &  &  &  &  &  &  &  &  &  \\
\hline
 &  &  &  &  &  &  &  &  &  &  &  &  &  &  &  &  &  &  &  &  &  &  &  &  &  \\
\hline
 &  &  &  &  &  &  &  &  &  &  &  &  &  &  &  &  &  &  &  &  &  &  &  &  &  \\
\hline
 &  &  &  &  &  &  &  &  &  &  &  &  &  &  &  &  &  &  &  &  &  &  &  &  &  \\
\hline
 &  &  &  &  &  &  &  &  &  &  &  &  &  &  &  &  &  &  &  &  &  &  &  &  &  \\
\hline
 &  &  &  &  &  &  &  &  &  &  &  &  &  &  &  &  &  &  &  &  &  &  &  &  &  \\
\hline
 &  &  &  &  &  &  &  &  &  &  &  &  &  &  &  &  &  &  &  &  &  &  &  &  &  \\
\hline
\end{tabular}
\end{center}

\section*{Zadanie 17. (0-1)}
Na płaszczyźnie, w kartezjańskim układzie współrzędnych \((x, y)\), dany jest okrąg o środku \(S=(2,-5)\) i promieniu \(r=3\).

Dokończ zdanie. Wybierz właściwą odpowiedź spośród podanych.\\
Równanie tego okręgu ma postać\\
A. \((x-2)^{2}+(y+5)^{2}=9\)\\
B. \((x+2)^{2}+(y-5)^{2}=3\)\\
C. \((x-2)^{2}+(y+5)^{2}=3\)\\
D. \((x+2)^{2}+(y-3)^{2}=9\)

\begin{center}
\begin{tabular}{|c|c|c|c|c|c|c|c|c|c|c|c|c|c|c|c|c|c|c|c|c|c|c|c|c|}
\hline
\multicolumn{4}{|l|}{Brudnopis} &  &  &  &  &  &  &  &  &  &  &  &  &  &  &  & - &  &  &  &  &  \\
\hline
 &  &  &  &  &  &  &  &  &  &  &  &  &  &  &  &  &  &  &  &  &  &  &  &  \\
\hline
 &  &  &  &  &  &  &  &  &  &  &  &  &  &  &  &  &  &  &  &  &  &  &  &  \\
\hline
 &  &  &  &  &  &  &  &  &  &  &  &  &  &  &  &  &  &  &  &  &  &  &  &  \\
\hline
 &  &  &  &  &  &  &  &  &  &  &  &  &  &  &  &  &  &  &  &  &  &  &  &  \\
\hline
 &  &  &  &  &  &  &  &  &  &  &  &  &  &  &  &  &  &  &  &  &  &  &  &  \\
\hline
 &  &  &  &  &  &  &  &  &  &  &  &  &  &  &  &  &  &  &  &  &  &  &  &  \\
\hline
 &  &  &  &  &  &  &  &  &  &  &  &  &  &  &  &  &  &  &  &  &  &  &  &  \\
\hline
 &  &  &  &  &  &  &  &  &  &  &  &  &  &  &  &  &  &  &  &  &  &  &  &  \\
\hline
 &  &  &  &  &  &  &  &  &  &  &  &  &  &  &  &  &  &  &  &  &  &  &  &  \\
\hline
\end{tabular}
\end{center}

Zadanie 18. (0-1)\\
Odcinki \(A D\) i \(B C\) przecinają się w punkcie \(O\). W trójkątach \(A B O\) i \(O D C\) zachodzą związki: \(|A O|=5,|B O|=3,|O C|=10,|\Varangle O A B|=|\Varangle O C D|\) (zobacz rysunek).\\
\includegraphics[max width=\textwidth, center]{2025_02_09_654ac8557e15ed1a98aag-17}

Oblicz długość boku \(O D\) trójkąta \(O D C\).\\
Zapisz obliczenia.\\
\includegraphics[max width=\textwidth, center]{2025_02_09_654ac8557e15ed1a98aag-17(1)}

\begin{center}
\begin{tabular}{|c|l|c|}
\hline
\multirow{3}{*}{\begin{tabular}{c}
Wypełnia \\
egzaminator \\
\end{tabular}} & Nr zadania & 18. \\
\cline { 2 - 3 }
 & Maks. liczba pkt & 1 \\
\cline { 2 - 3 }
 & Uzyskana liczba pkt &  \\
\hline
\end{tabular}
\end{center}

\section*{Zadanie 19. (0-2) 뚱}
Na płaszczyźnie, w kartezjańskim układzie współrzędnych \((x, y)\), dana jest prosta \(k\) o równaniu \(y=-3 x+1\).

Dokończ zdania. Wybierz odpowiedź spośród A-D oraz odpowiedź spośród E-H.\\
19.1. Jedną z prostych równoległych do prostej \(k\) jest prosta o równaniu\\
A. \(y=3 x+2\)\\
B. \(y=-3 x+2\)\\
C. \(y=\frac{1}{3} x+1\)\\
D. \(y=-\frac{1}{3} x+1\)\\
19.2. Jedną z prostych prostopadłych do prostej \(k\) jest prosta o równaniu\\
E. \(y=\frac{1}{3} x+2\)\\
F. \(y=-\frac{1}{3} x+2\)\\
G. \(y=3 x+1\)\\
H. \(y=-3 x+1\)\\
\includegraphics[max width=\textwidth, center]{2025_02_09_654ac8557e15ed1a98aag-18}

\section*{Zadanie 20. (0-1)}
W kartezjańskim układzie współrzędnych \((x, y)\) dany jest kwadrat \(A B C D\). Wierzchołki \(A=(-2,1)\) i \(C=(4,5)\) są końcami przekątnej tego kwadratu.

Dokończ zdanie. Wybierz właściwą odpowiedź spośród podanych.\\
Długość przekątnej kwadratu \(A B C D\) jest równa\\
A. 10\\
B. \(2 \sqrt{13}\)\\
C. \(2 \sqrt{10}\)\\
D. 8

\begin{center}
\begin{tabular}{|c|c|c|c|c|c|c|c|c|c|c|c|c|c|c|c|c|c|c|c|c|c|c|c|}
\hline
\multicolumn{4}{|l|}{Brudnopis} &  &  &  &  &  & - &  & - &  &  &  &  &  &  &  &  &  &  &  &  \\
\hline
 &  &  &  &  &  &  &  &  &  &  &  &  &  &  &  &  &  &  &  &  &  &  &  \\
\hline
 &  &  &  &  &  &  &  &  &  &  &  &  &  &  &  &  &  &  &  &  &  &  &  \\
\hline
 &  &  &  &  &  &  &  &  &  &  &  &  &  &  &  &  &  &  &  &  &  &  &  \\
\hline
 &  &  &  &  &  &  &  &  &  &  &  &  &  &  &  &  &  &  &  &  &  &  &  \\
\hline
 &  &  &  &  &  &  &  &  &  &  &  &  &  &  &  &  &  &  &  &  &  &  &  \\
\hline
 &  &  &  &  &  &  &  &  &  &  &  &  &  &  &  &  &  &  &  &  &  &  &  \\
\hline
 &  &  &  &  &  &  &  &  &  &  &  &  &  &  &  &  &  &  &  &  &  &  &  \\
\hline
 &  &  &  &  &  &  &  &  &  &  &  &  &  &  &  &  &  &  &  &  &  &  &  \\
\hline
 &  &  &  &  &  &  &  &  &  &  &  &  &  &  &  &  &  &  &  &  &  &  &  \\
\hline
\end{tabular}
\end{center}

Zadanie 21. (0-1) 떰\\
Odcinek \(A B\) jest średnicą okręgu o środku w punkcie \(O\) i promieniu \(r=8\) (zobacz rysunek). Cięciwa \(A C\) ma długośćć \(8 \sqrt{3}\).

\section*{Dokończ zdanie.}
Wybierz właściwą odpowiedź spośród podanych.\\
\includegraphics[max width=\textwidth, center]{2025_02_09_654ac8557e15ed1a98aag-19}

Miara kąta \(B A C\) jest równa\\
A. \(30^{\circ}\)\\
B. \(45^{\circ}\)\\
C. \(15^{\circ}\)\\
D. \(60^{\circ}\)

\begin{center}
\begin{tabular}{|c|c|c|c|c|c|c|c|c|c|c|c|c|c|c|c|c|c|c|c|c|c|c|c|c|}
\hline
 & Brudn & nopis &  &  &  &  &  &  &  &  &  &  &  &  &  &  &  &  &  &  &  &  &  &  \\
\hline
 &  &  &  &  &  &  &  &  &  &  &  &  &  &  &  &  &  &  &  &  &  &  &  &  \\
\hline
 &  &  &  &  &  &  &  &  &  &  &  &  &  &  &  &  &  &  &  &  &  &  &  &  \\
\hline
 &  &  &  &  &  &  &  &  &  &  &  &  &  &  &  &  &  &  &  &  &  &  &  &  \\
\hline
 &  &  &  &  &  &  &  &  &  &  &  &  &  &  &  &  &  &  &  &  &  &  &  &  \\
\hline
 &  &  &  &  &  &  &  &  &  &  &  &  &  &  &  &  &  &  &  &  &  &  &  &  \\
\hline
 &  &  &  &  &  &  &  &  &  &  &  &  &  &  &  &  &  &  &  &  &  &  &  &  \\
\hline
 &  &  &  &  &  &  &  &  &  &  &  &  &  &  &  &  &  &  &  &  &  &  &  &  \\
\hline
 &  &  &  &  &  &  &  &  &  &  &  &  &  &  &  &  &  &  &  &  &  &  &  &  \\
\hline
 &  &  &  &  &  &  &  &  &  &  &  &  &  &  &  &  &  &  &  &  &  &  &  &  \\
\hline
\end{tabular}
\end{center}

\section*{Zadanie 22. (0-1) 뚬}
Kąt \(\alpha\) jest ostry oraz \(4 \operatorname{tg} \alpha=3 \sin ^{2} \alpha+3 \cos ^{2} \alpha\).\\
Dokończ zdanie. Wybierz właściwą odpowiedź spośród podanych.\\
Tangens kąta \(\alpha\) jest równy\\
A. \(\frac{3}{4}\)\\
B. \(\frac{4}{3}\)\\
C. \(\frac{1}{4}\)\\
D. 4\\
\includegraphics[max width=\textwidth, center]{2025_02_09_654ac8557e15ed1a98aag-19(1)}

Zadanie 23. (0-1)\\
Dane są dwa trójkąty podobne \(A B C\) i \(K L M\) o polach równych - odpowiednio - \(P\) oraz \(2 P\). Obwód trójkąta \(A B C\) jest równy \(x\).

Dokończ zdanie tak, aby było prawdziwe. Wybierz odpowiedź A albo B oraz jej uzasadnienie 1., 2. albo 3.

Obwód trójkąta \(K L M\) jest równy

\begin{center}
\begin{tabular}{|c|c|c|c|c|}
\hline
\multirow[t]{2}{*}{A.} & \multirow[t]{2}{*}{\(\sqrt{2} \cdot x\),} & \multirow{3}{*}{ponieważ stosunek obwodów trójkątów podobnych jest równy} & 1. & kwadratowi stosunku pól tych trójkątów. \\
\hline
 &  &  & 2. & pierwiastkowi kwadratowemu ze stosunku pól tych trójkąów. \\
\hline
B. & \(2 x\), &  & 3. & stosunkowi pól tych trójkątów. \\
\hline
\end{tabular}
\end{center}

\begin{center}
\begin{tabular}{|c|c|c|c|c|c|c|c|c|c|c|c|c|c|c|c|c|c|c|c|c|c|}
\hline
 & Brudn & Inopis &  &  &  &  &  &  &  & - &  &  &  &  &  &  &  &  &  &  &  \\
\hline
 &  &  &  &  &  &  &  &  &  &  &  &  &  &  &  &  &  &  &  &  &  \\
\hline
 &  &  &  &  &  &  &  &  &  &  &  &  &  &  &  &  &  &  &  &  &  \\
\hline
 &  &  &  &  &  &  &  &  &  &  &  &  &  &  &  &  &  &  &  &  &  \\
\hline
 &  &  &  &  &  &  &  &  &  &  &  &  &  &  &  &  &  &  &  &  &  \\
\hline
 &  &  &  &  &  &  &  &  &  &  &  &  &  &  &  &  &  &  &  &  &  \\
\hline
 &  &  &  &  &  &  &  &  &  &  &  &  &  &  &  &  &  &  &  &  &  \\
\hline
 &  &  &  &  &  &  &  &  &  &  &  &  &  &  &  &  &  &  &  &  &  \\
\hline
 &  &  &  &  &  &  &  &  &  &  &  &  &  &  &  &  &  &  &  &  &  \\
\hline
 &  &  &  &  &  &  &  &  &  &  &  &  &  &  &  &  &  &  &  &  &  \\
\hline
 &  &  &  &  &  &  &  &  &  &  &  &  &  &  &  &  &  &  &  &  &  \\
\hline
 &  &  &  &  &  &  &  &  &  &  &  &  &  &  &  &  &  &  &  &  &  \\
\hline
 &  &  &  &  &  &  &  &  &  &  &  &  &  &  &  &  &  &  &  &  &  \\
\hline
 &  &  &  &  &  &  &  &  &  &  &  &  &  &  &  &  &  &  &  &  &  \\
\hline
 &  &  &  &  &  &  &  &  &  &  &  &  &  &  &  &  &  &  &  &  &  \\
\hline
 &  &  &  &  &  &  &  &  &  &  &  &  &  &  &  &  &  &  &  &  &  \\
\hline
 &  &  &  &  &  &  &  &  &  &  &  &  &  &  &  &  &  &  &  &  &  \\
\hline
 &  &  &  &  &  &  &  &  &  &  &  &  &  &  &  &  &  &  &  &  &  \\
\hline
 &  &  &  &  &  &  &  &  &  &  &  &  &  &  &  &  &  &  &  &  &  \\
\hline
 &  &  &  &  &  &  &  &  &  &  &  &  &  &  &  &  &  &  &  &  &  \\
\hline
 &  &  &  &  &  &  &  &  &  &  &  &  &  &  &  &  &  &  &  &  &  \\
\hline
 &  &  &  &  &  &  &  &  &  &  &  &  &  &  &  &  &  &  &  &  &  \\
\hline
 &  &  &  &  &  &  &  &  &  &  &  &  &  &  &  &  &  &  &  &  &  \\
\hline
 &  &  &  &  &  &  &  &  &  &  &  &  &  &  &  &  &  &  &  &  &  \\
\hline
 &  &  &  &  &  &  &  &  &  &  &  &  &  &  &  &  &  &  &  &  &  \\
\hline
 &  &  &  &  &  &  &  &  &  &  &  &  &  &  &  &  &  &  &  &  &  \\
\hline
 &  &  &  &  &  &  &  &  &  &  &  &  &  &  &  &  &  &  &  &  &  \\
\hline
 &  &  &  &  &  &  &  &  &  &  &  &  &  &  &  &  &  &  &  &  &  \\
\hline
 &  &  &  &  &  &  &  &  &  &  &  &  &  &  &  &  &  &  &  &  &  \\
\hline
 &  &  &  &  &  &  &  &  &  &  &  &  &  &  &  &  &  &  &  &  &  \\
\hline
 &  &  &  &  &  &  &  &  &  &  &  &  &  &  &  &  &  &  &  &  &  \\
\hline
\end{tabular}
\end{center}

\section*{Zadanie 24. (0-1) ㄸ..}
Punkty \(A\) oraz \(B\) leżą na okręgu o środku \(O\). Proste \(k\) i \(l\) są styczne do tego okręgu w punktach - odpowiednio - \(A\) i \(B\). Te proste przecinają się w punkcie \(S\) i tworzą kąt o mierze \(76^{\circ}\) (zobacz rysunek).\\
\includegraphics[max width=\textwidth, center]{2025_02_09_654ac8557e15ed1a98aag-21}

Dokończ zdanie. Wybierz właściwą odpowiedź spośród podanych.

Miara kąta \(O B A\) jest równa\\
A. \(52^{\circ}\)\\
B. \(26^{\circ}\)\\
C. \(14^{\circ}\)\\
D. \(38^{\circ}\)\\
\includegraphics[max width=\textwidth, center]{2025_02_09_654ac8557e15ed1a98aag-21(1)}

Zadanie 25. (0-1) шण्ण\\
Powierzchnię boczną graniastosłupa prawidłowego czworokątnego rozcięto wzdłuż krawędzi bocznej graniastosłupa i rozłożono na płaszczyźnie. Otrzymano w ten sposób prostokąt \(A B C D\), w którym bok \(B C\) odpowiada krawędzi rozcięcia (wysokości graniastosłupa).\\
Przekątna \(A C\) tego prostokąta ma długość 16 i tworzy z bokiem \(B C\) kąt o mierze \(30^{\circ}\) (zobacz rysunek).\\
\includegraphics[max width=\textwidth, center]{2025_02_09_654ac8557e15ed1a98aag-22}

Dokończ zdanie. Wybierz właściwą odpowiedź spośród podanych.\\
Długość krawędzi podstawy tego graniastosłupa jest równa\\
A. 8\\
B. \(8 \sqrt{3}\)\\
C. \(2 \sqrt{3}\)\\
D. 2\\
\includegraphics[max width=\textwidth, center]{2025_02_09_654ac8557e15ed1a98aag-22(1)}

Zadanie 26. (0-1)\\
Dany jest ostrosłup prawidłowy trójkątny \(A B C S\) o podstawie \(A B C\). Punkty \(D, E\) i \(F\) są środkami - odpowiednio - krawędzi bocznych \(A S, B S\) i \(C S\) (zobacz rysunek).\\
\includegraphics[max width=\textwidth, center]{2025_02_09_654ac8557e15ed1a98aag-23}

Dokończ zdanie. Wybierz właściwą odpowiedź spośród podanych.\\
Stosunek objętości ostrosłupa DEFS do objętości ostrosłupa \(A B C S\) jest równy\\
A. \(3: 4\)\\
B. \(1: 4\)\\
C. \(1: 8\)\\
D. \(3: 8\)\\
\includegraphics[max width=\textwidth, center]{2025_02_09_654ac8557e15ed1a98aag-23(1)}

\section*{Zadanie 27. (0-1) ㄸ.w}
Dany jest graniastosłup prawidłowy trójkątny \(A B C D E F\) (zobacz rysunek obok).

Na którym z rysunków prawidłowo narysowano, oznaczono i podpisano kąt \(\alpha\) pomiędzy ścianą boczną \(A C F D\) i przekątną \(A E\) ściany bocznej \(A B E D\) tego graniastosłupa? Wybierz właściwą odpowiedź spośród podanych.\\
\includegraphics[max width=\textwidth, center]{2025_02_09_654ac8557e15ed1a98aag-24}\\
A. \(\alpha=\Varangle E A G\)\\
B. \(\alpha=\Varangle E A D\)\\
\includegraphics[max width=\textwidth, center]{2025_02_09_654ac8557e15ed1a98aag-24(3)}\\
C. \(\alpha=\Varangle E A F\)\\
\includegraphics[max width=\textwidth, center]{2025_02_09_654ac8557e15ed1a98aag-24(2)}\\
D. \(\alpha=\Varangle E A C\)\\
\includegraphics[max width=\textwidth, center]{2025_02_09_654ac8557e15ed1a98aag-24(1)}

Zadanie 28. (0-3)\\
W pojemniku znajdują się losy loterii fantowej ponumerowane kolejnymi liczbami naturalnymi od 1000 do 9999 . Każdy los, którego numer jest liczbą o sumie cyfr równej 3, jest wygrywający. Uczestnicy loterii losują z pojemnika po jednym losie.

Oblicz prawdopodobieństwo zdarzenia polegającego na tym, że pierwszy los wyciągnięty z pojemnika był wygrywający. Zapisz obliczenia.\\
\includegraphics[max width=\textwidth, center]{2025_02_09_654ac8557e15ed1a98aag-25}

\begin{center}
\begin{tabular}{|c|l|c|}
\hline
\multirow{3}{*}{\begin{tabular}{l}
Wypełnia \\
egzaminator \\
\end{tabular}} & Nr zadania & 28. \\
\cline { 2 - 3 }
 & Maks. liczba pkt & 3 \\
\cline { 2 - 3 }
 & Uzyskana liczba pkt &  \\
\hline
\end{tabular}
\end{center}

Zadanie 29. (0-4)\\
Rozważamy wszystkie równoległoboki o obwodzie równym 200 i kącie ostrym o mierze \(30^{\circ}\).\\
Podaj wzór i dziedzinę funkcji opisującej zależność pola takiego równoległoboku od długości \(x\) boku równoległoboku.\\
Oblicz wymiary tego z rozważanych równoległoboków, który ma największe pole, i oblicz to największe pole.\\
Zapisz obliczenia.

\begin{center}
\begin{tabular}{|c|c|c|c|c|c|c|c|c|c|c|c|c|c|c|c|c|c|c|c|c|c|}
\hline
 &  &  &  &  &  &  &  &  &  &  &  &  &  &  &  &  &  &  &  &  &  \\
\hline
 &  &  &  &  &  &  &  &  &  &  &  &  &  &  &  &  &  &  &  &  &  \\
\hline
 &  &  &  &  &  &  &  &  &  &  &  &  &  &  &  &  &  &  &  &  &  \\
\hline
 &  &  &  &  &  &  &  &  &  &  &  &  &  &  &  &  &  &  &  &  &  \\
\hline
 &  &  &  &  &  &  &  &  &  &  &  &  &  &  &  &  &  &  &  &  &  \\
\hline
 &  &  &  &  &  &  &  &  &  &  &  &  &  &  &  &  &  &  &  &  &  \\
\hline
 &  &  &  &  &  &  &  &  &  &  &  &  &  &  &  &  &  &  &  &  &  \\
\hline
 &  &  &  &  &  &  &  &  &  &  &  &  &  &  &  &  &  &  &  &  &  \\
\hline
 &  &  &  &  &  &  &  &  &  &  &  &  &  &  &  &  &  &  &  &  &  \\
\hline
 &  &  &  &  &  &  &  &  &  &  &  &  &  &  &  &  &  &  &  &  &  \\
\hline
 &  &  &  &  &  &  &  &  &  &  &  &  &  &  &  &  &  &  &  &  &  \\
\hline
 &  &  &  &  &  &  &  &  &  &  &  &  &  &  &  &  &  &  &  &  &  \\
\hline
 &  &  &  &  &  &  &  &  &  &  &  &  &  &  &  &  &  &  &  &  &  \\
\hline
 &  &  &  &  &  &  &  &  &  &  &  &  &  &  &  &  &  &  &  &  &  \\
\hline
 &  &  &  &  &  &  &  &  &  &  &  &  &  &  &  &  &  &  &  &  &  \\
\hline
 &  &  &  &  &  &  &  &  &  &  &  &  &  &  &  &  &  &  &  &  &  \\
\hline
 &  &  &  &  &  &  &  &  &  &  &  &  &  &  &  &  &  &  &  &  &  \\
\hline
 &  &  &  &  &  &  &  &  &  &  &  &  &  &  &  &  &  &  &  &  &  \\
\hline
 &  &  &  &  &  &  &  &  &  &  &  &  &  &  &  &  &  &  &  &  &  \\
\hline
 &  &  &  &  &  &  &  &  &  &  &  &  &  &  &  &  &  &  &  &  &  \\
\hline
 &  &  &  &  &  &  &  &  &  &  &  &  &  &  &  &  &  &  &  &  &  \\
\hline
 &  &  &  &  &  &  &  &  &  &  &  &  &  &  &  &  &  &  &  &  &  \\
\hline
 &  &  &  &  &  &  &  &  &  &  &  &  &  &  &  &  &  &  &  &  &  \\
\hline
 &  &  &  &  &  &  &  &  &  &  &  &  &  &  &  &  &  &  &  &  &  \\
\hline
 &  &  &  &  &  &  &  &  &  &  &  &  &  &  &  &  &  &  &  &  &  \\
\hline
 &  &  &  &  &  &  &  &  &  &  &  &  &  &  &  &  &  &  &  &  &  \\
\hline
 &  &  &  &  &  &  &  &  &  &  &  &  &  &  &  &  &  &  &  &  &  \\
\hline
 &  &  &  &  &  &  &  &  &  &  &  &  &  &  &  &  &  &  &  &  &  \\
\hline
 &  &  &  &  &  &  &  &  &  &  &  &  &  &  &  &  &  &  &  &  &  \\
\hline
 &  &  &  &  &  &  &  &  &  &  &  &  &  &  &  &  &  &  &  &  &  \\
\hline
 &  &  &  &  &  &  &  &  &  &  &  &  &  &  &  &  &  &  &  &  &  \\
\hline
 &  &  &  &  &  &  &  &  &  &  &  &  &  &  &  &  &  &  &  &  &  \\
\hline
 &  &  &  &  &  &  &  &  &  &  &  &  &  &  &  &  &  &  &  &  &  \\
\hline
 &  &  &  &  &  &  &  &  &  &  &  &  &  &  &  &  &  &  &  &  &  \\
\hline
 &  &  &  &  &  &  &  &  &  &  &  &  &  &  &  &  &  &  &  &  &  \\
\hline
 &  &  &  &  &  &  &  &  &  &  &  &  &  &  &  &  &  &  &  &  &  \\
\hline
 &  &  &  &  &  &  &  &  &  &  &  &  &  &  &  &  &  &  &  &  &  \\
\hline
 &  &  &  &  &  &  &  &  &  &  &  &  &  &  &  &  &  &  &  &  &  \\
\hline
 &  &  &  &  &  &  &  &  &  &  &  &  &  &  &  &  &  &  &  &  &  \\
\hline
\end{tabular}
\end{center}

\begin{center}
\includegraphics[max width=\textwidth]{2025_02_09_654ac8557e15ed1a98aag-27}
\end{center}

\begin{center}
\begin{tabular}{|c|l|c|}
\hline
\multirow{3}{*}{\begin{tabular}{l}
Wypełnia \\
egzaminator \\
\end{tabular}} & Nr zadania & 29. \\
\cline { 2 - 3 }
 & Maks. liczba pkt & 4 \\
\cline { 2 - 3 }
 & Uzyskana liczba pkt &  \\
\hline
\end{tabular}
\end{center}

Strona 27 z 31

\section*{Zadanie 30.}
W pewnej grupie 100 uczniów przeprowadzono sondaż dotyczący dziennego czasu korzystania z komputera. Wyniki sondażu przedstawia poniższy diagram. Na osi poziomej podano - wyrażony w godzinach - dzienny czas korzystania przez ucznia z komputera. Na osi pionowej przedstawiono liczbę uczniów, którzy dziennie korzystają z komputera przez określony czas.

Liczba uczniów\\
\includegraphics[max width=\textwidth, center]{2025_02_09_654ac8557e15ed1a98aag-28}

Zadanie 30.1. (0-1)\\
Oceń prawdziwość poniższych stwierdzeń. Wybierz P, jeśli stwierdzenie jest prawdziwe, albo F - jeśli jest fałszywe.

\begin{center}
\begin{tabular}{|l|c|c|}
\hline
\begin{tabular}{l}
Mediana dziennego czasu korzystania przez ucznia \(z\) komputera jest równa \\
2,25 godziny. \\
\end{tabular} & P & F \\
\hline
\begin{tabular}{l}
Połowa \(z\) tej grupy uczniów korzysta dziennie \(z\) komputera przez mniej niż \\
2,5 godziny. \\
\end{tabular} & P & F \\
\hline
\end{tabular}
\end{center}

\begin{center}
\includegraphics[max width=\textwidth]{2025_02_09_654ac8557e15ed1a98aag-28(1)}
\end{center}

Zadanie 30.2. (0-1)\\
Dokończ zdanie. Wybierz właściwą odpowiedź spośród podanych.\\
Dominanta dziennego czasu korzystania przez ucznia z komputera jest równa\\
A. 2,25 godziny.\\
B. 2,50 godziny.\\
C. 2,75 godziny.\\
D. 1,50 godziny.

\begin{center}
\begin{tabular}{|c|c|c|c|c|c|c|c|c|c|c|c|c|c|c|c|c|c|c|c|c|c|c|c|}
\hline
\multicolumn{4}{|l|}{Brudnopis} &  &  &  &  &  &  &  &  & - &  &  &  &  &  &  &  & - &  &  &  \\
\hline
 &  &  &  &  &  &  &  &  &  &  &  &  &  &  &  &  &  &  &  &  &  &  &  \\
\hline
 &  &  &  &  &  &  &  &  &  &  &  &  &  &  &  &  &  &  &  &  &  &  &  \\
\hline
 &  &  &  &  &  &  &  &  &  &  &  &  &  &  &  &  &  &  &  &  &  &  &  \\
\hline
 &  &  &  &  &  &  &  &  &  &  &  &  &  &  &  &  &  &  &  &  &  &  &  \\
\hline
 &  &  &  &  &  &  &  &  &  &  &  &  &  &  &  &  &  &  &  &  &  &  &  \\
\hline
 &  &  &  &  &  &  &  &  &  &  &  &  &  &  &  &  &  &  &  &  &  &  &  \\
\hline
 &  &  &  &  &  &  &  &  &  &  &  &  &  &  &  &  &  &  &  &  &  &  &  \\
\hline
 &  &  &  &  &  &  &  &  &  &  &  &  &  &  &  &  &  &  &  &  &  &  &  \\
\hline
 &  &  &  &  &  &  &  &  &  &  &  &  &  &  &  &  &  &  &  &  &  &  &  \\
\hline
 &  &  &  &  &  &  &  &  &  &  &  &  &  &  &  &  &  &  &  &  &  &  &  \\
\hline
 &  &  &  &  &  &  &  &  &  &  &  &  &  &  &  &  &  &  &  &  &  &  &  \\
\hline
 &  &  &  &  &  &  &  &  &  &  &  &  &  &  &  &  &  &  &  &  &  &  &  \\
\hline
 &  &  &  &  &  &  &  &  &  &  &  &  &  &  &  &  &  &  &  &  &  &  &  \\
\hline
 &  &  &  &  &  &  &  &  &  &  &  &  &  &  &  &  &  &  &  &  &  &  &  \\
\hline
\end{tabular}
\end{center}

BRUDNOPIS (nie podlega ocenie)

\begin{center}
\begin{tabular}{|c|c|c|c|c|c|c|c|c|c|c|c|c|c|c|c|c|c|c|c|c|c|c|c|}
\hline
 &  &  &  &  &  &  &  &  &  &  &  &  &  &  &  &  &  &  &  &  &  &  &  \\
\hline
 &  &  &  &  &  &  &  &  &  &  &  &  &  &  &  &  &  &  &  &  &  &  &  \\
\hline
 &  &  &  &  &  &  &  &  &  &  &  &  &  &  &  &  &  &  &  &  &  &  &  \\
\hline
 &  &  &  &  &  &  &  &  &  &  &  &  &  &  &  &  &  &  &  &  &  &  &  \\
\hline
 &  &  &  &  &  &  &  &  &  &  &  &  &  &  &  &  &  &  &  &  &  &  &  \\
\hline
 &  &  &  &  &  &  &  &  &  &  &  &  &  &  &  &  &  &  &  &  &  &  &  \\
\hline
 &  &  &  &  &  &  &  &  &  &  &  &  &  &  &  &  &  &  &  &  &  &  &  \\
\hline
 &  &  &  &  &  &  &  &  &  &  &  &  &  &  &  &  &  &  &  &  &  &  &  \\
\hline
 &  &  &  &  &  &  &  &  &  &  &  &  &  &  &  &  &  &  &  &  &  &  &  \\
\hline
 &  &  &  &  &  &  &  &  &  &  &  &  &  &  &  &  &  &  &  &  &  &  &  \\
\hline
 &  &  &  &  &  &  &  &  &  &  &  &  &  &  &  &  &  &  &  &  &  &  &  \\
\hline
 &  &  &  &  &  &  &  &  &  &  &  &  &  &  &  &  &  &  &  &  &  &  &  \\
\hline
 &  &  &  &  &  &  &  &  &  &  &  &  &  &  &  &  &  &  &  &  &  &  &  \\
\hline
 &  &  &  &  &  &  &  &  &  &  &  &  &  &  &  &  &  &  &  &  &  &  &  \\
\hline
 &  &  &  &  &  &  &  &  &  &  &  &  &  &  &  &  &  &  &  &  &  &  &  \\
\hline
 &  &  &  &  &  &  &  &  &  &  &  &  &  &  &  &  &  &  &  &  &  &  &  \\
\hline
 &  &  &  &  &  &  &  &  &  &  &  &  &  &  &  &  &  &  &  &  &  &  &  \\
\hline
 &  &  &  &  &  &  &  &  &  &  &  &  &  &  &  &  &  &  &  &  &  &  &  \\
\hline
 &  &  &  &  &  &  &  &  &  &  &  &  &  &  &  &  &  &  &  &  &  &  &  \\
\hline
 &  &  &  &  &  &  &  &  &  &  &  &  &  &  &  &  &  &  &  &  &  &  &  \\
\hline
 &  &  &  &  &  &  &  &  &  &  &  &  &  &  &  &  &  &  &  &  &  &  &  \\
\hline
 &  &  &  &  &  &  &  &  &  &  &  &  &  &  &  &  &  &  &  &  &  &  &  \\
\hline
 &  &  &  &  &  &  &  &  &  &  &  &  &  &  &  &  &  &  &  &  &  &  &  \\
\hline
 &  &  &  &  &  &  &  &  &  &  &  &  &  &  &  &  &  &  &  &  &  &  &  \\
\hline
 &  &  &  &  &  &  &  &  &  &  &  &  &  &  &  &  &  &  &  &  &  &  &  \\
\hline
 &  &  &  &  &  &  &  &  &  &  &  &  &  &  &  &  &  &  &  &  &  &  &  \\
\hline
 &  &  &  &  &  &  &  &  &  &  &  &  &  &  &  &  &  &  &  &  &  &  &  \\
\hline
 &  &  &  &  &  &  &  &  &  &  &  &  &  &  &  &  &  &  &  &  &  &  &  \\
\hline
 &  &  &  &  &  &  &  &  &  &  &  &  &  &  &  &  &  &  &  &  &  &  &  \\
\hline
 &  &  &  &  &  &  &  &  &  &  &  &  &  &  &  &  &  &  &  &  &  &  &  \\
\hline
 &  &  &  &  &  &  &  &  &  &  &  &  &  &  &  &  &  &  &  &  &  &  &  \\
\hline
 &  &  &  &  &  &  &  &  &  &  &  &  &  &  &  &  &  &  &  &  &  &  &  \\
\hline
 &  &  &  &  &  &  &  &  &  &  &  &  &  &  &  &  &  &  &  &  &  &  &  \\
\hline
 &  &  &  &  &  &  &  &  &  &  &  &  &  &  &  &  &  &  &  &  &  &  &  \\
\hline
 &  &  &  &  &  &  &  &  &  &  &  &  &  &  &  &  &  &  &  &  &  &  &  \\
\hline
 &  &  &  &  &  &  &  &  &  &  &  &  &  &  &  &  &  &  &  &  &  &  &  \\
\hline
 &  &  &  &  &  &  &  &  &  &  &  &  &  &  &  &  &  &  &  &  &  &  &  \\
\hline
 &  &  &  &  &  &  &  &  &  &  &  &  &  &  &  &  &  &  &  &  &  &  &  \\
\hline
 &  &  &  &  &  &  &  &  &  &  &  &  &  &  &  &  &  &  &  &  &  &  &  \\
\hline
 &  &  &  &  &  &  &  &  &  &  &  &  &  &  &  &  &  &  &  &  &  &  &  \\
\hline
 &  &  &  &  &  &  &  &  &  &  &  &  &  &  &  &  &  &  &  &  &  &  &  \\
\hline
 &  &  &  &  &  &  &  &  &  &  &  &  &  &  &  &  &  &  &  &  &  &  &  \\
\hline
 &  &  &  &  &  &  &  &  &  &  &  &  &  &  &  &  &  &  &  &  &  &  &  \\
\hline
 &  &  &  &  &  &  &  &  &  &  &  &  &  &  &  &  &  &  &  &  &  &  &  \\
\hline
 &  &  &  &  &  &  &  &  &  &  &  &  &  &  &  &  &  &  &  &  &  &  &  \\
\hline
 &  &  &  &  &  &  &  &  &  &  &  &  &  &  &  &  &  &  &  &  &  &  &  \\
\hline
 &  &  &  &  &  &  &  &  &  &  &  &  &  &  &  &  &  &  &  &  &  &  &  \\
\hline
\end{tabular}
\end{center}

\begin{center}
\includegraphics[max width=\textwidth]{2025_02_09_654ac8557e15ed1a98aag-31}
\end{center}


\end{document}