% This LaTeX document needs to be compiled with XeLaTeX.
\documentclass[10pt]{article}
\usepackage[utf8]{inputenc}
\usepackage{amsmath}
\usepackage{amsfonts}
\usepackage{amssymb}
\usepackage[version=4]{mhchem}
\usepackage{stmaryrd}
\usepackage{graphicx}
\usepackage[export]{adjustbox}
\graphicspath{ {./images/} }
\usepackage{multirow}
\usepackage[fallback]{xeCJK}
\usepackage{polyglossia}
\usepackage{fontspec}
\IfFontExistsTF{Noto Serif CJK KR}
{\setCJKmainfont{Noto Serif CJK KR}}
{\IfFontExistsTF{Apple SD Gothic Neo}
  {\setCJKmainfont{Apple SD Gothic Neo}}
  {\IfFontExistsTF{UnDotum}
    {\setCJKmainfont{UnDotum}}
    {\setCJKmainfont{Malgun Gothic}}
}}
\IfFontExistsTF{Noto Serif CJK TC}
{\setCJKfallbackfamilyfont{\CJKrmdefault}{Noto Serif CJK TC}}
{\IfFontExistsTF{STSong}
  {\setCJKfallbackfamilyfont{\CJKrmdefault}{STSong}}
  {\IfFontExistsTF{Droid Sans Fallback}
    {\setCJKfallbackfamilyfont{\CJKrmdefault}{Droid Sans Fallback}}
    {\setCJKfallbackfamilyfont{\CJKrmdefault}{SimSun}}
}}

\setmainlanguage{polish}
\IfFontExistsTF{CMU Serif}
{\setmainfont{CMU Serif}}
{\IfFontExistsTF{DejaVu Sans}
  {\setmainfont{DejaVu Sans}}
  {\setmainfont{Georgia}}
}

\author{LICZBA PUNKTÓW DO UZYSKANIA: 46}
\date{}


\begin{document}
\maketitle
CENTRALNA\\
KOMISJA\\
EGZAMINACYJNA

Arkusz zawiera informacje prawnie chronione do momentu rozpoczęcia egzaminu.

\section*{Miejsce na naklejkę.}
 Sprawdż, czy kod na naklejce to M-100.Jeżeli tak - przyklej naklejkę. Jeżeli nie - zgłoś to nauczycielowi.

\section*{Egzamin maturalny}
\section*{MATEMATYKA}
\section*{Poziom podstawowy}
Symbol arkusza\\
MMAP-P0-100-2406

\section*{DAtA: 4 czerwca 2024 r. Godzina rozpoczęcia: 9:00 \\
 Czas trwania: \(\mathbf{1 8 0}\) minut}
\section*{WYPEŁNIA ZESPÓŁ NADZORUJACY}
Uprawnienia zdającego do:\\
\(\square\) dostosowania zasad oceniania dostosowania w zw. z dyskalkulią nieprzenoszenia odpowiedzi na kartę.

Przed rozpoczęciem pracy z arkuszem egzaminacyjnym

\begin{enumerate}
  \item Sprawdź, czy nauczyciel przekazał Ci właściwy arkusz egzaminacyjny, tj. arkusz we właściwej formule, z właściwego przedmiotu na właściwym poziomie.
  \item Jeżeli przekazano Ci niewłaściwy arkusz - natychmiast zgłoś to nauczycielowi. Nie rozrywaj banderol.
  \item Jeżeli przekazano Ci właściwy arkusz - rozerwij banderole po otrzymaniu takiego polecenia od nauczyciela. Zapoznaj się z instrukcją na stronie 2.\\
\includegraphics[max width=\textwidth, center]{2025_02_10_da789c2d54bb18c230ebg-02(1)}
\end{enumerate}

\section*{Instrukcja dla zdającego}
\begin{enumerate}
  \item Sprawdź, czy arkusz egzaminacyjny zawiera 31 stron (zadania 1-32). Ewentualny brak zgłoś przewodniczącemu zespołu nadzorującego egzamin.
  \item Na pierwszej stronie arkusza oraz na karcie odpowiedzi wpisz swój numer PESEL i przyklej naklejkę z kodem.
  \item Symbol 四四/zamieszczony w nagłówku zadania oznacza, że rozwiązanie zadania zamkniętego musisz przenieść na kartę odpowiedzi. Ocenie podlegają wyłącznie odpowiedzi zaznaczone na karcie odpowiedzi.
  \item Odpowiedzi do zadań zamkniętych zaznacz na karcie odpowiedzi w części karty przeznaczonej dla zdającego. Zamaluj pola do tego przeznaczone. Błędne zaznaczenie otocz kółkiem ( ) zaznacz właściwe.
  \item Pamiętaj, że pominięcie argumentacji lub istotnych obliczeń w rozwiązaniu zadania otwartego może spowodować, że za to rozwiązanie nie otrzymasz pełnej liczby punktów.
  \item Rozwiązania zadań i odpowiedzi wpisuj w miejscu na to przeznaczonym.
  \item Pisz czytelnie i używaj tylko długopisu lub pióra z czarnym tuszem lub atramentem.
  \item Nie używaj korektora, a błędne zapisy wyraźnie przekreśl.
  \item Pamiętaj, że zapisy w brudnopisie nie będą oceniane.
  \item Możesz korzystać z Wybranych wzorów matematycznych, cyrkla i linijki oraz kalkulatora prostego. Upewnij się, czy przekazano Ci broszurę z okładką taką jak widoczna poniżej.\\
\includegraphics[max width=\textwidth, center]{2025_02_10_da789c2d54bb18c230ebg-02}
\end{enumerate}

\section*{Zadania egzaminacyjne są wydrukowane na następnych stronach.}
\section*{Zadanie 1. (0-1)}
Dokończ zdanie. Wybierz właściwą odpowiedź spośród podanych.\\
Liczba \(2^{-1} \cdot 32^{\frac{3}{5}}\) jest równa\\
A. \((-16)\)\\
B. \((-4)\)\\
C. 2\\
D. 4

\begin{center}
\begin{tabular}{|c|c|c|c|c|c|c|c|c|c|c|c|c|c|c|c|c|c|c|c|c|c|c|c|}
\hline
\multicolumn{5}{|l|}{Brudnopis} &  &  &  &  &  &  &  & - &  &  &  & - & - &  &  & - &  &  &  \\
\hline
 &  &  &  &  &  &  &  &  &  &  &  &  &  &  &  &  &  &  &  &  &  &  &  \\
\hline
 &  &  &  &  &  &  &  &  &  &  &  &  &  &  &  &  &  &  &  &  &  &  &  \\
\hline
 &  &  &  &  &  &  &  &  &  &  &  &  &  &  &  &  &  &  &  &  &  &  &  \\
\hline
 &  &  &  &  &  &  &  &  &  &  &  &  &  &  &  &  &  &  &  &  &  &  &  \\
\hline
 &  &  &  &  &  &  &  &  &  &  &  &  &  &  &  &  &  &  &  &  &  &  &  \\
\hline
 &  &  &  &  &  &  &  &  &  &  &  &  &  &  &  &  &  &  &  &  &  &  &  \\
\hline
 &  &  &  &  &  &  &  &  &  &  &  &  &  &  &  &  &  &  &  &  &  &  &  \\
\hline
 &  &  &  &  &  &  &  &  &  &  &  &  &  &  &  &  &  &  &  &  &  &  &  \\
\hline
 &  &  &  &  &  &  &  &  &  &  &  &  &  &  &  &  &  &  &  &  &  &  &  \\
\hline
 &  &  &  &  &  &  &  &  &  &  &  &  &  &  &  &  &  &  &  &  &  &  &  \\
\hline
 &  &  &  &  &  &  &  &  &  &  &  &  &  &  &  &  &  &  &  &  &  &  &  \\
\hline
 &  &  &  &  &  &  &  &  &  &  &  &  &  &  &  &  &  &  &  &  &  &  &  \\
\hline
- &  &  &  &  &  &  &  &  &  &  &  &  &  &  &  &  &  &  &  &  &  &  &  \\
\hline
\end{tabular}
\end{center}

\section*{Zadanie 2. (0-1)}
Dokończ zdanie. Wybierz właściwą odpowiedź spośród podanych.\\
Liczba \(\log _{3}\left(\frac{3}{2}\right)+\log _{3}\left(\frac{2}{9}\right)\) jest równa\\
A. \(\log _{3} \frac{31}{18}\)\\
B. \(\log _{3} \frac{5}{11}\)\\
C. \((-1)\)\\
D. \(\frac{1}{3}\)\\
\includegraphics[max width=\textwidth, center]{2025_02_10_da789c2d54bb18c230ebg-04}

\section*{Zadanie 3.(0-1)}
Dokończ zdanie.Wybierz właściwą odpowiedź spośród podanych.\\
Liczba \((2 \sqrt{10}+\sqrt{2})^{2}\) jest równa\\
A. 22\\
B. 42\\
C. \(42+4 \sqrt{5}\)\\
D. \(42+8 \sqrt{5}\)\\
\includegraphics[max width=\textwidth, center]{2025_02_10_da789c2d54bb18c230ebg-05(1)}

\section*{Zadanie 4.(0-1)}
Klient wpłacił do banku na trzyletnią lokatę kwotę w wysokości \(K_{0}\) zł.Po każdym rocznym okresie oszczędzania bank dolicza odsetki w wysokości \(6 \%\) od kwoty bieżącego kapitału znajdującego się na lokacie-zgodnie z procentem składanym.

Dokończ zdanie.Wybierz właściwą odpowiedź spośród podanych.\\
Po trzech latach oszczędzania w tym banku kwota na lokacie(bez uwzględniania podatków) jest równa\\
A.\(K_{0} \cdot(1,06)^{3}\)\\
B.\(K_{0} \cdot(1,02)^{3}\)\\
C.\(K_{0} \cdot(1,03)^{6}\)\\
D.\(K_{0} \cdot 1,18\)\\
\includegraphics[max width=\textwidth, center]{2025_02_10_da789c2d54bb18c230ebg-05}

Zadanie 5. (0-2)\\
Wykaż, że dla każdej liczby naturalnej \(n \geq 1\) liczba \(5 n^{3}-5 n\) jest podzielna przez 30 .\\
\includegraphics[max width=\textwidth, center]{2025_02_10_da789c2d54bb18c230ebg-06}

\section*{Zadanie 6.(0-1)}
Dokończ zdanie.Wybierz właściwą odpowiedź spośród podanych.\\
Liczba wszystkich całkowitych dodatnich rozwiązań nierówności

\[
\frac{3 x-5}{12}<\frac{1}{3}
\]

jest równa\\
A. 2\\
B. 3\\
C. 5\\
D. 6

\begin{center}
\begin{tabular}{|c|c|c|c|c|c|c|c|c|c|c|c|c|c|c|c|c|c|c|c|c|c|c|}
\hline
\multicolumn{4}{|l|}{Brudnopis} &  &  &  &  &  &  &  &  &  &  &  & - & , &  &  & - &  &  &  \\
\hline
 &  &  &  &  &  &  &  &  &  &  &  &  &  &  &  &  &  &  &  &  &  &  \\
\hline
 &  &  &  &  &  &  &  &  &  &  &  &  &  &  &  &  &  &  &  &  &  &  \\
\hline
 &  &  &  &  &  &  &  &  &  &  &  &  &  &  &  &  &  &  &  &  &  &  \\
\hline
 &  &  &  &  &  &  &  &  &  &  &  &  &  &  &  &  &  &  &  &  &  &  \\
\hline
 &  &  &  &  &  &  &  &  &  &  &  &  &  &  &  &  &  &  &  &  &  &  \\
\hline
 &  &  &  &  &  &  &  &  &  &  &  &  &  &  &  &  &  &  &  &  &  &  \\
\hline
 &  &  &  &  &  &  &  &  &  &  &  &  &  &  &  &  &  &  &  &  &  &  \\
\hline
 &  &  &  &  &  &  &  &  &  &  &  &  &  &  &  &  &  &  &  &  &  &  \\
\hline
 &  &  &  &  &  &  &  &  &  &  &  &  &  &  &  &  &  &  &  &  &  &  \\
\hline
 &  &  &  &  &  &  &  &  &  &  &  &  &  &  &  &  &  &  &  &  &  &  \\
\hline
 &  &  &  &  &  &  &  &  &  &  &  &  &  &  &  &  &  &  &  &  &  &  \\
\hline
\end{tabular}
\end{center}

\section*{Zadanie 7.(0-1)}
Dokończ zdanie.Wybierz właściwą odpowiedź spośród podanych.\\
Układ równań \(\left\{\begin{aligned} x-2 y & =3 \\ -4 x+8 y & =-12\end{aligned}\right.\)\\
A.nie ma rozwiązań.\\
B.ma dokładnie jedno rozwiązanie.\\
C.ma dokładnie dwa rozwiązania.\\
D.ma nieskończenie wiele rozwiązań.\\
\includegraphics[max width=\textwidth, center]{2025_02_10_da789c2d54bb18c230ebg-07}

\section*{Zadanie 8. (0-1)}
Dokończ zdanie. Wybierz właściwą odpowiedź spośród podanych.\\
Dla każdej liczby rzeczywistej \(x\) różnej od: (-1), 0 i 1 , wartość wyrażenia \(\frac{2 x^{2}}{x^{2}-1} \cdot \frac{x+1}{x}\) jest równa wartości wyrażenia\\
A. \(2 x+2\)\\
B. \(\frac{2 x}{x-1}\)\\
C. \(\frac{2 x}{x^{2}-1}\)\\
D. \(\frac{2 x^{3}+1}{x^{3}-1}\)

\begin{center}
\begin{tabular}{|c|c|c|c|c|c|c|c|c|c|c|c|c|c|c|c|c|c|c|c|c|c|c|}
\hline
\multicolumn{5}{|l|}{Brudnopis} &  &  &  &  &  &  &  &  &  &  &  &  &  &  &  &  &  &  \\
\hline
 &  &  &  &  &  &  &  &  &  &  &  &  &  &  &  &  &  &  &  &  &  &  \\
\hline
 &  &  &  &  &  &  &  &  &  &  &  &  &  &  &  &  &  &  &  &  &  &  \\
\hline
 &  &  &  &  &  &  &  &  &  &  &  &  &  &  &  &  &  &  &  &  &  &  \\
\hline
 &  &  &  &  &  &  &  &  &  &  &  &  &  &  &  &  &  &  &  &  &  &  \\
\hline
 &  &  &  &  &  &  &  &  &  &  &  &  &  &  &  &  &  &  &  &  &  &  \\
\hline
 &  &  &  &  &  &  &  &  &  &  &  &  &  &  &  &  &  &  &  &  &  &  \\
\hline
 &  &  &  &  &  &  &  &  &  &  &  &  &  &  &  &  &  &  &  &  &  &  \\
\hline
 &  &  &  &  &  &  &  &  &  &  &  &  &  &  &  &  &  &  &  &  &  &  \\
\hline
 &  &  &  &  &  &  &  &  &  &  &  &  &  &  &  &  &  &  &  &  &  &  \\
\hline
 &  &  &  &  &  &  &  &  &  &  &  &  &  &  &  &  &  &  &  &  &  &  \\
\hline
 &  &  &  &  &  &  &  &  &  &  &  &  &  &  &  &  &  &  &  &  &  &  \\
\hline
 &  &  &  &  &  &  &  &  &  &  &  &  &  &  &  &  &  &  &  &  &  &  \\
\hline
\end{tabular}
\end{center}

\section*{Zadanie 9. (0-1)}
Wielomian \(W(x)=a x^{3}+b x^{2}+c x+d\) jest iloczynem wielomianów \(F(x)=(2-3 x)^{2}\) oraz \(G(x)=3 x-2\).

Uzupełnij poniższe zdanie. Wpisz odpowiednią liczbę w wykropkowanym miejscu tak, aby zdanie było prawdziwe.

Suma \(a+b+c+d\) współczynników wielomianu \(W\) jest równa \(\qquad\) .

\begin{center}
\begin{tabular}{|c|c|c|c|c|c|c|c|c|c|c|c|c|c|c|c|c|c|c|c|c|c|c|}
\hline
 & Brudno & nopis &  &  &  &  &  &  &  &  &  &  &  &  &  &  &  &  &  &  &  &  \\
\hline
 &  &  &  &  &  &  &  &  &  &  &  &  &  &  &  &  &  &  &  &  &  &  \\
\hline
 &  &  &  &  &  &  &  &  &  &  &  &  &  &  &  &  &  &  &  &  &  &  \\
\hline
 &  &  &  &  &  &  &  &  &  &  &  &  &  &  &  &  &  &  &  &  &  &  \\
\hline
 &  &  &  &  &  &  &  &  &  &  &  &  &  &  &  &  &  &  &  &  &  &  \\
\hline
 &  &  &  &  &  &  &  &  &  &  &  &  &  &  &  &  &  &  &  &  &  &  \\
\hline
 &  &  &  &  &  &  &  &  &  &  &  &  &  &  &  &  &  &  &  &  &  &  \\
\hline
 &  &  &  &  &  &  &  &  &  &  &  &  &  &  &  &  &  &  &  &  &  &  \\
\hline
 &  &  &  &  &  &  &  &  &  &  &  &  &  &  &  &  &  &  &  &  &  &  \\
\hline
 &  &  &  &  &  &  &  &  &  &  &  &  &  &  &  &  &  &  &  &  &  &  \\
\hline
 &  &  &  &  &  &  &  &  &  &  &  &  &  &  &  &  &  &  &  &  &  &  \\
\hline
 &  &  &  &  &  &  &  &  &  &  &  &  &  &  &  &  &  &  &  &  &  &  \\
\hline
 &  &  &  &  &  &  &  &  &  &  &  &  &  &  &  &  &  &  &  &  &  &  \\
\hline
\end{tabular}
\end{center}

Zadanie 10. (0-3)\\
Rozwiąż równanie

\[
4 x^{3}-12 x^{2}-x+3=0
\]

\section*{Zapisz obliczenia.}
\begin{center}
\begin{tabular}{|c|c|c|c|c|c|c|c|c|c|c|c|c|c|c|c|c|c|c|c|c|c|c|}
\hline
 &  &  &  &  &  &  &  &  &  &  &  &  &  &  &  &  &  &  &  &  &  &  \\
\hline
 &  &  &  &  &  &  &  &  &  &  &  &  &  &  &  &  &  &  &  &  &  &  \\
\hline
 &  &  &  &  &  &  &  &  &  &  &  &  &  &  &  &  &  &  &  &  &  &  \\
\hline
 &  &  &  &  &  &  &  &  &  &  &  &  &  &  &  &  &  &  &  &  &  &  \\
\hline
 &  &  &  &  &  &  &  &  &  &  &  &  &  &  &  &  &  &  &  &  &  &  \\
\hline
 &  &  &  &  &  &  &  &  &  &  &  &  &  &  &  &  &  &  &  &  &  &  \\
\hline
 &  &  &  &  &  &  &  &  &  &  &  &  &  &  &  &  &  &  &  &  &  &  \\
\hline
 &  &  &  &  &  &  &  &  &  &  &  &  &  &  &  &  &  &  &  &  &  &  \\
\hline
 &  &  &  &  &  &  &  &  &  &  &  &  &  &  &  &  &  &  &  &  &  &  \\
\hline
 &  &  &  &  &  &  &  &  &  &  &  &  &  &  &  &  &  &  &  &  &  &  \\
\hline
 &  &  &  &  &  &  &  &  &  &  &  &  &  &  &  &  &  &  &  &  &  &  \\
\hline
 &  &  &  &  &  &  &  &  &  &  &  &  &  &  &  &  &  &  &  &  &  &  \\
\hline
 &  &  &  &  &  &  &  &  &  &  &  &  &  &  &  &  &  &  &  &  &  &  \\
\hline
 &  &  &  &  &  &  &  &  &  &  &  &  &  &  &  &  &  &  &  &  &  &  \\
\hline
 &  &  &  &  &  &  &  &  &  &  &  &  &  &  &  &  &  &  &  &  &  &  \\
\hline
 &  &  &  &  &  &  &  &  &  &  &  &  &  &  &  &  &  &  &  &  &  &  \\
\hline
 &  &  &  &  &  &  &  &  &  &  &  &  &  &  &  &  &  &  &  &  &  &  \\
\hline
 &  &  &  &  &  &  &  &  &  &  &  &  &  &  &  &  &  &  &  &  &  &  \\
\hline
 &  &  &  &  &  &  &  &  &  &  &  &  &  &  &  &  &  &  &  &  &  &  \\
\hline
 &  &  &  &  &  &  &  &  &  &  &  &  &  &  &  &  &  &  &  &  &  &  \\
\hline
 &  &  &  &  &  &  &  &  &  &  &  &  &  &  &  &  &  &  &  &  &  &  \\
\hline
 &  &  &  &  &  &  &  &  &  &  &  &  &  &  &  &  &  &  &  &  &  &  \\
\hline
 &  &  &  &  &  &  &  &  &  &  &  &  &  &  &  &  &  &  &  &  &  &  \\
\hline
 &  &  &  &  &  &  &  &  &  &  &  &  &  &  &  &  &  &  &  &  &  &  \\
\hline
 &  &  &  &  &  &  &  &  &  &  &  &  &  &  &  &  &  &  &  &  &  &  \\
\hline
 &  &  &  &  &  &  &  &  &  &  &  &  &  &  &  &  &  &  &  &  &  &  \\
\hline
 &  &  &  &  &  &  &  &  &  &  &  &  &  &  &  &  &  &  &  &  &  &  \\
\hline
 &  &  &  &  &  &  &  &  &  &  &  &  &  &  &  &  &  &  &  &  &  &  \\
\hline
 &  &  &  &  &  &  &  &  &  &  &  &  &  &  &  &  &  &  &  &  &  &  \\
\hline
 &  &  &  &  &  &  &  &  &  &  &  &  &  &  &  &  &  &  &  &  &  &  \\
\hline
 &  &  &  &  &  &  &  &  &  &  &  &  &  &  &  &  &  &  &  &  &  &  \\
\hline
 &  &  &  &  &  &  &  &  &  &  &  &  &  &  &  &  &  &  &  &  &  &  \\
\hline
 &  &  &  &  &  &  &  &  &  &  &  &  &  &  &  &  &  &  &  &  &  &  \\
\hline
 &  &  &  &  &  &  &  &  &  &  &  &  &  &  &  &  &  &  &  &  &  &  \\
\hline
 &  &  &  &  &  &  &  &  &  &  &  &  &  &  &  &  &  &  &  &  &  &  \\
\hline
 &  &  &  &  &  &  &  &  &  &  &  &  &  &  &  &  &  &  &  &  &  &  \\
\hline
 &  &  &  &  &  &  &  &  &  &  &  &  &  &  &  &  &  &  &  &  &  &  \\
\hline
 &  &  &  &  &  &  &  &  &  &  &  &  &  &  &  &  &  &  &  &  &  &  \\
\hline
 &  &  &  &  &  &  &  &  &  &  &  &  &  &  &  &  &  &  &  &  &  &  \\
\hline
 &  &  &  &  &  &  &  &  &  &  &  &  &  &  &  &  &  &  &  &  &  &  \\
\hline
 &  &  &  &  &  &  &  &  &  &  &  &  &  &  &  &  &  &  &  &  &  &  \\
\hline
 &  &  &  &  &  &  &  &  &  &  &  &  &  &  &  &  &  &  &  &  &  &  \\
\hline
\end{tabular}
\end{center}

\section*{Zadanie 11.}
Na rysunku 1., w kartezjańskim układzie współrzędnych \((x, y)\), przedstawiono wykres funkcji \(f\). Każdy z punktów przecięcia wykresu funkcji \(f\) z prostą o równaniu \(y=2\) ma obie współrzędne całkowite.

\section*{Rysunek 1.}
\begin{center}
\includegraphics[max width=\textwidth]{2025_02_10_da789c2d54bb18c230ebg-10}
\end{center}

\section*{Zadanie 11.1. (0-1)}
Uzupełnij poniższe zdanie. Wpisz odpowiedni przedział w wykropkowanym miejscu tak, aby zdanie było prawdziwe.

Zbiorem wszystkich rozwiązań nierówności \(f(x) \leq 2\) jest przedział \(\qquad\) .\\
\includegraphics[max width=\textwidth, center]{2025_02_10_da789c2d54bb18c230ebg-10(1)}

Zadanie 11.2. (0-1)\\
Na rysunku 2., w kartezjańskim układzie współrzędnych \((x, y)\), przedstawiono wykres funkcji \(g\), powstałej w wyniku przesunięcia równoległego wykresu funkcji \(f\) wzdłuż osi \(O x\) o 4 jednostki w lewo.

\section*{Rysunek 2.}
\begin{center}
\includegraphics[max width=\textwidth]{2025_02_10_da789c2d54bb18c230ebg-11}
\end{center}

Dokończ zdanie. Wybierz odpowiedź A, B albo C oraz odpowiedź 1. albo 2.\\
Funkcje \(f\) i \(g\) są powiązane zależnością

\begin{center}
\begin{tabular}{|l|l|l|l|l|}
\hline
A. & \(g(x)=f(x+4)\) &  &  &  \\
\hline
B. & \(g(x)=f(x-4)\) & \multirow{3}{*}{oraz mają takie same} & dziedziny. &  \\
\cline { 1 - 1 }
 &  &  &  &  \\
\hline
C. & \(g(x)=f(x)-4\) &  & 2. & zbiory wartości. \\
\hline
\end{tabular}
\end{center}

\begin{center}
\includegraphics[max width=\textwidth]{2025_02_10_da789c2d54bb18c230ebg-11(1)}
\end{center}

\section*{Zadanie 12. (0-1)}
Funkcja \(y=f(x)\) jest określona za pomocą tabeli

\begin{center}
\begin{tabular}{|c|c|c|c|c|c|}
\hline
\(\boldsymbol{x}\) & -2 & -1 & 0 & 1 & 2 \\
\hline
\(\boldsymbol{y}\) & -1 & 0 & 1 & 0 & 3 \\
\hline
\end{tabular}
\end{center}

Oceń prawdziwość poniższych stwierdzeń. Wybierz \(P\), jeśli stwierdzenie jest prawdziwe, albo F - jeśli jest fałszywe.

\begin{center}
\begin{tabular}{|l|c|c|}
\hline
Funkcja \(f\) ma dokładnie jedno miejsce zerowe. & \(\mathbf{P}\) & \(\mathbf{F}\) \\
\hline
\begin{tabular}{l}
W kartezjańskim układzie współrzędnych \((x, y)\) wykres funkcji \(f\) jest \\
symetryczny względem osi \(O y\). \\
\end{tabular} & \(\mathbf{P}\) & \(\mathbf{F}\) \\
\hline
\end{tabular}
\end{center}

\begin{center}
\begin{tabular}{|c|c|c|c|c|c|c|c|c|c|c|c|c|c|c|c|c|c|c|c|c|}
\hline
 & Brudn & nopis &  &  & - &  &  & - & - &  &  &  &  & - & - &  &  &  &  &  \\
\hline
 &  &  &  &  &  &  &  &  &  &  &  &  &  &  &  &  &  &  &  &  \\
\hline
 &  &  &  &  &  &  &  &  &  &  &  &  &  &  &  &  &  &  &  &  \\
\hline
 &  &  &  &  &  &  &  &  &  &  &  &  &  &  &  &  &  &  &  &  \\
\hline
 &  &  &  &  &  &  &  &  &  &  &  &  &  &  &  &  &  &  &  &  \\
\hline
 &  &  &  &  &  &  &  &  &  &  &  &  &  &  &  &  &  &  &  &  \\
\hline
 &  &  &  &  &  &  &  &  &  &  &  &  &  &  &  &  &  &  &  &  \\
\hline
 &  &  &  &  &  &  &  &  &  &  &  &  &  &  &  &  &  &  &  &  \\
\hline
 &  &  &  &  &  &  &  &  &  &  &  &  &  &  &  &  &  &  &  &  \\
\hline
 &  &  &  &  &  &  &  &  &  &  &  &  &  &  &  &  &  &  &  &  \\
\hline
\end{tabular}
\end{center}

\section*{Zadanie 13. (0-1) 맘}
Liczba 2 jest miejscem zerowym funkcji liniowej \(f(x)=(3-m) x+4\).\\
Dokończ zdanie. Wybierz właściwą odpowiedź spośród podanych.\\
Liczba \(m\) jest równa\\
A. 0\\
B. 3\\
C. 4\\
D. 5\\
\includegraphics[max width=\textwidth, center]{2025_02_10_da789c2d54bb18c230ebg-12}

Zadanie 14. (0-2)\\
Parabola, która jest wykresem funkcji kwadratowej \(f\), ma z osiami kartezjańskiego układu współrzędnych \((x, y)\) dokładnie dwa punkty wspólne: \(M=(0,18)\) oraz \(N=(3,0)\).

\section*{Wyznacz wzór funkcji kwadratowej \(\boldsymbol{f}\). Zapisz obliczenia.}
\begin{center}
\includegraphics[max width=\textwidth]{2025_02_10_da789c2d54bb18c230ebg-13}
\end{center}

\section*{Zadanie 15.}
Funkcja kwadratowa \(f\) jest określona wzorem \(f(x)=-(x+1)^{2}+4\).

\section*{Zadanie 15.1. (0-1) \(\square\)}
Na jednym z rysunków A-D przedstawiono, w kartezjańskim układzie współrzędnych ( \(x, y\) ), fragment wykresu funkcji \(y=f(x)\).

Dokończ zdanie. Wybierz właściwą odpowiedź spośród podanych.\\
Fragment wykresu funkcji \(y=f(x)\) przedstawiono na rysunku\\
A.\\
\includegraphics[max width=\textwidth, center]{2025_02_10_da789c2d54bb18c230ebg-14(1)}\\
B.\\
\includegraphics[max width=\textwidth, center]{2025_02_10_da789c2d54bb18c230ebg-14(2)}\\
C.\\
\includegraphics[max width=\textwidth, center]{2025_02_10_da789c2d54bb18c230ebg-14}\\
D.\\
\includegraphics[max width=\textwidth, center]{2025_02_10_da789c2d54bb18c230ebg-14(3)}

Zadanie 15.2. (0-1) 띰\\
Oceń prawdziwość poniższych stwierdzeń. Wybierz \(P\), jeśli stwierdzenie jest prawdziwe, albo F - jeśli jest fałszywe.

\begin{center}
\begin{tabular}{|l|c|c|}
\hline
\begin{tabular}{l}
Wykres funkcji \(f\) przecina oś Oy kartezjańskiego układu współrzędnych \\
\((x, y)\) w punkcie o współrzędnych \((0,4)\). \\
\end{tabular} & \(\mathbf{P}\) & \(\mathbf{F}\) \\
\hline
Miejsca zerowe funkcji \(f\) są równe: \((-3)\) oraz 1. & \(\mathbf{P}\) & \(\mathbf{F}\) \\
\hline
\end{tabular}
\end{center}

\begin{center}
\includegraphics[max width=\textwidth]{2025_02_10_da789c2d54bb18c230ebg-15}
\end{center}

\section*{Zadanie 16.}
Ciąg \(\left(a_{n}\right)\) jest określony wzorem \(a_{n}=2 \cdot(-1)^{n+1}+5\) dla każdej liczby naturalnej \(n \geq 1\).

\section*{Zadanie 16.1. (0-1) 띠}
Dokończ zdanie. Wybierz właściwą odpowiedź spośród podanych.

Suma dziesięciu początkowych kolejnych wyrazów tego ciągu jest równa\\
A. 3\\
B. 7\\
C. 50\\
D. 100

\begin{center}
\begin{tabular}{|c|c|c|c|c|c|c|c|c|c|c|c|c|c|c|c|c|c|c|c|c|c|c|}
\hline
\multicolumn{4}{|l|}{Brudnopis} &  &  &  &  &  &  &  &  &  &  &  &  & - &  & - &  &  &  &  \\
\hline
 &  &  &  &  &  &  &  &  &  &  &  &  &  &  &  &  &  &  &  &  &  &  \\
\hline
 &  &  &  &  &  &  &  &  &  &  &  &  &  &  &  &  &  &  &  &  &  &  \\
\hline
 &  &  &  &  &  &  &  &  &  &  &  &  &  &  &  &  &  &  &  &  &  &  \\
\hline
 &  &  &  &  &  &  &  &  &  &  &  &  &  &  &  &  &  &  &  &  &  &  \\
\hline
 &  &  &  &  &  &  &  &  &  &  &  &  &  &  &  &  &  &  &  &  &  &  \\
\hline
 &  &  &  &  &  &  &  &  &  &  &  &  &  &  &  &  &  &  &  &  &  &  \\
\hline
 &  &  &  &  &  &  &  &  &  &  &  &  &  &  &  &  &  &  &  &  &  &  \\
\hline
 &  &  &  &  &  &  &  &  &  &  &  &  &  &  &  &  &  &  &  &  &  &  \\
\hline
 &  &  &  &  &  &  &  &  &  &  &  &  &  &  &  &  &  &  &  &  &  &  \\
\hline
 &  &  &  &  &  &  &  &  &  &  &  &  &  &  &  &  &  &  &  &  &  &  \\
\hline
 &  &  &  &  &  &  &  &  &  &  &  &  &  &  &  &  &  &  &  &  &  &  \\
\hline
\end{tabular}
\end{center}

Zadanie 16.2. (0-1) 띠\\
Oceń prawdziwość poniższych stwierdzeń. Wybierz \(P\), jeśli stwierdzenie jest prawdziwe, albo F - jeśli jest fałszywe.

\begin{center}
\begin{tabular}{|l|c|c|}
\hline
Ciąg \(\left(a_{n}\right)\) jest malejący. & \(\mathbf{P}\) & \(\mathbf{F}\) \\
\hline
Ciąg \(\left(a_{n}\right)\) jest geometryczny. & \(\mathbf{P}\) & \(\mathbf{F}\) \\
\hline
\end{tabular}
\end{center}

\begin{center}
\includegraphics[max width=\textwidth]{2025_02_10_da789c2d54bb18c230ebg-16}
\end{center}

\section*{Zadanie 17. (0-1) 띰}
W ciągu arytmetycznym \(\left(a_{n}\right)\), określonym dla każdej liczby naturalnej \(n \geq 1\), dane są wyrazy: \(a_{1}=7\) oraz \(a_{2}=13\).

Dokończ zdanie. Wybierz właściwą odpowiedź spośród podanych.\\
Wyraz \(a_{10}\) jest równy\\
A. \((-47)\)\\
B. 52\\
C. 61\\
D. 67\\
\includegraphics[max width=\textwidth, center]{2025_02_10_da789c2d54bb18c230ebg-17}

\section*{Zadanie 18. (0-1)}
Trzywyrazowy ciąg ( \(-1,2, x\) ) jest arytmetyczny.\\
Trzywyrazowy ciąg ( \(-1,2, y\) ) jest geometryczny.

Dokończ zdanie. Wybierz właściwą odpowiedź spośród podanych.\\
Liczby \(x\) oraz \(y\) spełniają warunki\\
A. \(x>0\) i \(y>0\)\\
B. \(x>0\) i \(y<0\)\\
C. \(x<0\) i \(y>0\)\\
D. \(x<0\) i \(y<0\)\\
\includegraphics[max width=\textwidth, center]{2025_02_10_da789c2d54bb18c230ebg-17(1)}

\section*{Zadanie 19. (0-1) 띰}
Dokończ zdanie. Wybierz właściwą odpowiedź spośród podanych.\\
Liczba \(1+\cos ^{2} 27^{\circ}\) jest równa\\
A. \(2-\sin ^{2} 27^{\circ}\)\\
B. \(\sin ^{2} 27^{\circ}\)\\
C. \(2+\sin ^{2} 27^{\circ}\)\\
D. 2

\begin{center}
\begin{tabular}{|c|c|c|c|c|c|c|c|c|c|c|c|c|c|c|c|c|c|c|c|c|c|c|c|}
\hline
\multicolumn{4}{|l|}{Brudnopis} &  &  &  &  &  &  &  &  &  &  &  &  &  &  &  &  &  &  &  &  \\
\hline
 &  &  &  &  &  &  &  &  &  &  &  &  &  &  &  &  &  &  &  &  &  &  &  \\
\hline
 &  &  &  &  &  &  &  &  &  &  &  &  &  &  &  &  &  &  &  &  &  &  &  \\
\hline
 &  &  &  &  &  &  &  &  &  &  &  &  &  &  &  &  &  &  &  &  &  &  &  \\
\hline
 &  &  &  &  &  &  &  &  &  &  &  &  &  &  &  &  &  &  &  &  &  &  &  \\
\hline
 &  &  &  &  &  &  &  &  &  &  &  &  &  &  &  &  &  &  &  &  &  &  &  \\
\hline
 &  &  &  &  &  &  &  &  &  &  &  &  &  &  &  &  &  &  &  &  &  &  &  \\
\hline
 &  &  &  &  &  &  &  &  &  &  &  &  &  &  &  &  &  &  &  &  &  &  &  \\
\hline
 &  &  &  &  &  &  &  &  &  &  &  &  &  &  &  &  &  &  &  &  &  &  &  \\
\hline
 &  &  &  &  &  &  &  &  &  &  &  &  &  &  &  &  &  &  &  &  &  &  &  \\
\hline
\end{tabular}
\end{center}

\section*{Zadanie 20. (0-1)}
Podstawy trapezu prostokątnego \(A B C D\) mają długości: \(|A B|=8\) oraz \(|C D|=5\).\\
Wysokość \(A D\) tego trapezu ma długość \(\sqrt{3}\) (zobacz rysunek).\\
\includegraphics[max width=\textwidth, center]{2025_02_10_da789c2d54bb18c230ebg-18}

Dokończ zdanie. Wybierz właściwą odpowiedź spośród podanych.

Miara kąta ostrego \(A B C\) jest równa\\
A. \(15^{\circ}\)\\
B. \(30^{\circ}\)\\
C. \(45^{\circ}\)\\
D. \(60^{\circ}\)

\begin{center}
\begin{tabular}{|c|c|c|c|c|c|c|c|c|c|c|c|c|c|c|c|c|c|c|c|c|c|c|c|c|c|c|c|c|c|c|}
\hline
 & rud & Inop &  &  &  &  &  &  &  &  &  &  &  &  &  &  &  &  &  &  &  &  &  &  &  &  &  &  &  &  \\
\hline
 &  &  &  &  &  &  &  &  &  &  &  &  &  &  &  &  &  &  &  &  &  &  &  &  &  &  &  &  &  &  \\
\hline
 &  &  &  &  &  &  &  &  &  &  &  &  &  &  &  &  &  &  &  &  &  &  &  &  &  &  &  &  &  &  \\
\hline
 &  &  &  &  &  &  &  &  &  &  &  &  &  &  &  &  &  &  &  &  &  &  &  &  &  &  &  &  &  &  \\
\hline
 &  &  &  &  &  &  &  &  &  &  &  &  &  &  &  &  &  &  &  &  &  &  &  &  &  &  &  &  &  &  \\
\hline
 &  &  &  &  &  &  &  &  &  &  &  &  &  &  &  &  &  &  &  &  &  &  &  &  &  &  &  &  &  &  \\
\hline
 &  &  &  &  &  &  &  &  &  &  &  &  &  &  &  &  &  &  &  &  &  &  &  &  &  &  &  &  &  &  \\
\hline
 &  &  &  &  &  &  &  &  &  &  &  &  &  &  &  &  &  &  &  &  &  &  &  &  &  &  &  &  &  &  \\
\hline
 &  &  &  &  &  &  &  &  &  &  &  &  &  &  &  &  &  &  &  &  &  &  &  &  &  &  &  &  &  &  \\
\hline
 &  &  &  &  &  &  &  &  &  &  &  &  &  &  &  &  &  &  &  &  &  &  &  &  &  &  &  &  &  &  \\
\hline
 &  &  &  &  &  &  &  &  &  &  &  &  &  &  &  &  &  &  &  &  &  &  &  &  &  &  &  &  &  &  \\
\hline
\end{tabular}
\end{center}

\section*{Zadanie 21. (0-1)}
Punkty \(A, B\) oraz \(C\) leżą na okręgu o środku w punkcie \(S\). Długość łuku \(A B\), na którym jest oparty kąt wpisany \(A C B\), jest równa \(\frac{1}{5}\) długości okręgu (zobacz rysunek).\\
\includegraphics[max width=\textwidth, center]{2025_02_10_da789c2d54bb18c230ebg-19}

Dokończ zdanie. Wybierz właściwą odpowiedź spośród podanych.

Miara kąta ostrego \(A C B\) jest równa\\
A. \(18^{\circ}\)\\
B. \(30^{\circ}\)\\
C. \(36^{\circ}\)\\
D. \(72^{\circ}\)\\
\includegraphics[max width=\textwidth, center]{2025_02_10_da789c2d54bb18c230ebg-19(1)}

Zadanie 22. (0-2)\\
Bok kwadratu \(A B C D\) ma długość równą 12. Punkt \(S\) jest środkiem boku \(B C\) tego kwadratu. Na odcinku \(A S\) leży punkt \(P\) taki, że odcinek \(B P\) jest prostopadły do odcinka \(A S\).

Oblicz długość odcinka BP. Zapisz obliczenia.\\
\includegraphics[max width=\textwidth, center]{2025_02_10_da789c2d54bb18c230ebg-20}

\section*{Zadanie 23.}
W kartezjańskim układzie współrzędnych \((x, y)\) dany jest okrąg \(\mathcal{O}\) o równaniu

\[
(x-1)^{2}+(y+2)^{2}=5
\]

\section*{Zadanie 23.1. (0-1)}
Oceń prawdziwość poniższych stwierdzeń. Wybierz P, jeśli stwierdzenie jest prawdziwe, albo F - jeśli jest fałszywe.

\begin{center}
\begin{tabular}{|l|c|c|}
\hline
Do okręgu \(\mathcal{O}\) należy punkt o współrzędnych \((-1,-3)\). & \(\mathbf{P}\) & \(\mathbf{F}\) \\
\hline
Promień okręgu \(\mathcal{O}\) jest równy 5. & \(\mathbf{P}\) & \(\mathbf{F}\) \\
\hline
\end{tabular}
\end{center}

\begin{center}
\begin{tabular}{|c|c|c|c|c|c|c|c|c|c|c|c|c|c|c|c|c|c|c|c|c|c|c|c|c|c|}
\hline
 & Brudn & dnopi &  &  &  &  &  &  &  &  &  &  &  &  &  &  &  &  &  &  &  &  &  &  &  \\
\hline
 &  &  &  &  &  &  &  &  &  &  &  &  &  &  &  &  &  &  &  &  &  &  &  &  &  \\
\hline
 &  &  &  &  &  &  &  &  &  &  &  &  &  &  & - &  &  &  &  &  &  &  &  &  &  \\
\hline
 &  &  &  &  &  &  &  &  &  &  &  &  &  &  &  &  &  &  &  &  &  &  &  &  &  \\
\hline
 &  &  &  &  &  &  &  &  &  &  &  &  &  &  &  &  &  &  &  &  &  &  &  &  &  \\
\hline
 &  &  &  &  &  &  &  &  &  &  &  &  &  &  &  &  &  &  &  &  &  &  &  &  &  \\
\hline
 &  &  &  &  &  &  &  &  &  &  &  &  &  &  &  &  &  &  &  &  &  &  &  &  &  \\
\hline
 &  &  &  &  &  &  &  &  &  &  &  &  &  &  &  &  &  &  &  &  &  &  &  &  &  \\
\hline
 &  &  &  &  &  &  &  &  &  &  &  &  &  &  &  &  &  &  &  &  &  &  &  &  &  \\
\hline
\end{tabular}
\end{center}

\section*{Zadanie 23.2. (0-1) 띠}
Okrąg \(\mathcal{K}\) jest obrazem okręgu \(\mathcal{O}\) w symetrii środkowej względem początku układu wspórzędnych.

Dokończ zdanie. Wybierz właściwą odpowiedź spośród podanych.\\
Okrąg \(\mathcal{K}\) jest określony równaniem\\
A. \((x-1)^{2}+(y+2)^{2}=5\)\\
B. \((x+1)^{2}+(y+2)^{2}=5\)\\
C. \((x-1)^{2}+(y-2)^{2}=5\)\\
D. \((x+1)^{2}+(y-2)^{2}=5\)\\
\includegraphics[max width=\textwidth, center]{2025_02_10_da789c2d54bb18c230ebg-21}

Zadanie 24. (0-4)\\
W kartezjańskim układzie współrzędnych \((x, y)\) dane są punkty \(A=(2,8)\) oraz \(B=(10,2)\). Symetralna odcinka \(A B\) przecina oś \(O x\) układu współrzędnych w punkcie \(P\).

Oblicz współrzędne punktu \(P\) oraz długość odcinka \(A P\). Zapisz obliczenia.\\
\includegraphics[max width=\textwidth, center]{2025_02_10_da789c2d54bb18c230ebg-22}\\
\includegraphics[max width=\textwidth, center]{2025_02_10_da789c2d54bb18c230ebg-23}

\section*{Zadanie 25. (0-1) 밈}
Ostrosłup prawidłowy ma 2024 ściany boczne.\\
Dokończ zdanie. Wybierz właściwą odpowiedź spośród podanych.\\
Liczba wszystkich krawędzi tego ostrosłupa jest równa\\
A. 2025\\
B. 2026\\
C. 4048\\
D. 4052

\begin{center}
\begin{tabular}{|c|c|c|c|c|c|c|c|c|c|c|c|c|c|c|c|c|c|c|c|c|c|}
\hline
\multicolumn{4}{|l|}{Brudnopis} &  &  &  &  &  &  &  &  &  &  &  &  &  &  &  &  &  &  \\
\hline
 &  &  &  &  &  &  &  &  &  &  &  &  &  &  &  &  &  &  &  &  &  \\
\hline
 &  &  &  &  &  &  &  &  &  &  &  &  &  &  &  &  &  &  &  &  &  \\
\hline
 &  &  &  &  &  &  &  &  &  &  &  &  &  &  &  &  &  &  &  &  &  \\
\hline
 &  &  &  &  &  &  &  &  &  &  &  &  &  &  &  &  &  &  &  &  &  \\
\hline
 &  &  &  &  &  &  &  &  &  &  &  &  &  &  &  &  &  &  &  &  &  \\
\hline
 &  &  &  &  &  &  &  &  &  &  &  &  &  &  &  &  &  &  &  &  &  \\
\hline
 &  &  &  &  &  &  &  &  &  &  &  &  &  &  &  &  &  &  &  &  &  \\
\hline
 &  &  &  &  &  &  &  &  &  &  &  &  &  &  &  &  &  &  &  &  &  \\
\hline
 &  &  &  &  &  &  &  &  &  &  &  &  &  &  &  &  &  &  &  &  &  \\
\hline
 &  &  &  &  &  &  &  &  &  &  &  &  &  &  &  &  &  &  &  &  &  \\
\hline
 &  &  &  &  &  &  &  &  &  &  &  &  &  &  &  &  &  &  &  &  &  \\
\hline
 &  &  &  &  &  &  &  &  &  &  &  &  &  &  &  &  &  &  &  &  &  \\
\hline
\end{tabular}
\end{center}

\section*{Zadanie 26. (0-1)}
Przekątna ściany sześcianu ma długośćc \(2 \sqrt{2}\).\\
Dokończ zdanie. Wybierz właściwą odpowiedź spośród podanych.\\
Objętość tego sześcianu jest równa\\
A. 8\\
B. 24\\
C. \(\frac{16 \sqrt{6}}{9}\)\\
D. \(16 \sqrt{2}\)

\begin{center}
\begin{tabular}{|c|c|c|c|c|c|c|c|c|c|c|c|c|c|c|c|c|c|c|c|c|}
\hline
 & Brudnop & nopis &  &  &  &  &  &  &  &  &  &  &  &  &  &  &  &  &  &  \\
\hline
 &  &  &  &  &  &  &  &  &  &  &  &  &  &  &  &  &  &  &  &  \\
\hline
 &  &  &  &  &  &  &  &  &  &  &  &  &  &  &  &  &  &  &  &  \\
\hline
 &  &  &  &  &  &  &  &  &  &  &  &  &  &  &  &  &  &  &  &  \\
\hline
 &  &  &  &  &  &  &  &  &  &  &  &  &  &  &  &  &  &  &  &  \\
\hline
 &  &  &  &  &  &  &  &  &  &  &  &  &  &  &  &  &  &  &  &  \\
\hline
 &  &  &  &  &  &  &  &  &  &  &  &  &  &  &  &  &  &  &  &  \\
\hline
 &  &  &  &  &  &  &  &  &  &  &  &  &  &  &  &  &  &  &  &  \\
\hline
 &  &  &  &  &  &  &  &  &  &  &  &  &  &  &  &  &  &  &  &  \\
\hline
 &  &  &  &  &  &  &  &  &  &  &  &  &  &  &  &  &  &  &  &  \\
\hline
 &  &  &  &  &  &  &  &  &  &  &  &  &  &  &  &  &  &  &  &  \\
\hline
 &  &  &  &  &  &  &  &  &  &  &  &  &  &  &  &  &  &  &  &  \\
\hline
- &  &  &  &  &  &  &  &  &  &  &  &  &  &  &  &  &  &  &  &  \\
\hline
 &  &  &  &  &  &  &  &  &  &  &  &  &  &  &  &  &  &  &  &  \\
\hline
\end{tabular}
\end{center}

\section*{Zadanie 27. (0-1) 回}
Podstawą graniastosłupa prawidłowego czworokątnego jest kwadrat o boku długości 4.\\
Przekątna tego graniastosłupa jest nachylona do płaszczyzny podstawy pod kątem \(\alpha\) takim, że \(\operatorname{tg} \alpha=2\) (zobacz rysunek).\\
\includegraphics[max width=\textwidth, center]{2025_02_10_da789c2d54bb18c230ebg-25}

Dokończ zdanie. Wybierz właściwą odpowiedź spośród podanych.\\
Wysokość tego graniastosłupa jest równa\\
A. 2\\
B. 8\\
C. \(8 \sqrt{2}\)\\
D. \(16 \sqrt{2}\)\\
\includegraphics[max width=\textwidth, center]{2025_02_10_da789c2d54bb18c230ebg-25(1)}

\section*{Zadanie 28. (0-1) \(\square\)}
Na diagramie przedstawiono wyniki sprawdzianu z matematyki w pewnej klasie maturalnej. Na osi poziomej podano oceny, które uzyskali uczniowie tej klasy, a na osi pionowej podano liczbę uczniów, którzy otrzymali daną ocenę.\\
\includegraphics[max width=\textwidth, center]{2025_02_10_da789c2d54bb18c230ebg-26}

Dokończ zdanie. Wybierz właściwą odpowiedź spośród podanych.

Średnia arytmetyczna ocen uzyskanych z tego sprawdzianu przez uczniów tej klasy jest równa\\
A. 3\\
B. 3,12\\
C. 3,5\\
D. \(4,1(6)\)\\
\includegraphics[max width=\textwidth, center]{2025_02_10_da789c2d54bb18c230ebg-26(1)}

\section*{Zadanie 29. (0-1)}
Dokończ zdanie. Wybierz właściwą odpowiedź spośród podanych.\\
Wszystkich liczb naturalnych czterocyfrowych parzystych, w których zapisie dziesiętnym występują tylko cyfry \(2,4,7\) (np.: 7272, 2222, 7244), jest\\
A. 16\\
B. 27\\
C. 54\\
D. 81

\begin{center}
\begin{tabular}{|c|c|c|c|c|c|c|c|c|c|c|c|c|c|c|c|c|c|c|c|c|c|c|c|c|c|}
\hline
 & Brud & dnop &  &  &  &  &  &  &  &  &  &  &  &  &  &  &  &  &  &  &  &  &  &  &  \\
\hline
 &  &  &  &  &  &  &  &  &  &  &  &  &  &  &  &  &  &  &  &  &  &  &  &  &  \\
\hline
 &  &  &  &  &  &  &  &  &  &  &  &  &  &  &  &  &  &  &  &  &  &  &  &  &  \\
\hline
 &  &  &  &  &  &  &  &  &  &  &  &  &  &  &  &  &  &  &  &  &  &  &  &  &  \\
\hline
 &  &  &  &  &  &  &  &  &  &  &  &  &  &  &  &  &  &  &  &  &  &  &  &  &  \\
\hline
 &  &  &  &  &  &  &  &  &  &  &  &  &  &  &  &  &  &  &  &  &  &  &  &  &  \\
\hline
 &  &  &  &  &  &  &  &  &  &  &  &  &  &  &  &  &  &  &  &  &  &  &  &  &  \\
\hline
 &  &  &  &  &  &  &  &  &  &  &  &  &  &  &  &  &  &  &  &  &  &  &  &  &  \\
\hline
 &  &  &  &  &  &  &  &  &  &  &  &  &  &  &  &  &  &  &  &  &  &  &  &  &  \\
\hline
 &  &  &  &  &  &  &  &  &  &  &  &  &  &  &  &  &  &  &  &  &  &  &  &  &  \\
\hline
 &  &  &  &  &  &  &  &  &  &  &  &  &  &  &  &  &  &  &  &  &  &  &  &  &  \\
\hline
 &  &  &  &  &  &  &  &  &  &  &  &  &  &  &  &  &  &  &  &  &  &  &  &  &  \\
\hline
 &  &  &  &  &  &  &  &  &  &  &  &  &  &  &  &  &  &  &  &  &  &  &  &  &  \\
\hline
\end{tabular}
\end{center}

\section*{Zadanie 30. (0-1) 밈}
W pudełku znajdują się wyłącznie kule białe i czarne. Kul czarnych jest 18.\\
\(Z\) tego pudełka w sposób losowy wyciągamy jedną kulę.\\
Prawdopodobieństwo zdarzenia polegającego na tym, że wyciągniemy kulę czarną, jest równe \(\frac{3}{5}\).

\section*{Dokończ zdanie. Wybierz właściwą odpowiedź spośród podanych.}
Liczba kul białych w pudełku, przed wyciągnięciem jednej kuli, była równa\\
A. 9\\
B. 12\\
C. 15\\
D. 30

\begin{center}
\begin{tabular}{|c|c|c|c|c|c|c|c|c|c|c|c|c|c|c|c|c|c|c|c|c|c|c|c|c|}
\hline
\multicolumn{4}{|l|}{Brudnopis} &  &  &  &  &  &  &  &  &  &  &  &  &  &  &  &  &  &  &  &  &  \\
\hline
 &  &  &  &  &  &  &  &  &  &  &  &  &  &  &  &  &  &  &  &  &  &  &  &  \\
\hline
 &  &  &  &  &  &  &  &  &  &  &  &  &  &  &  &  &  &  &  &  &  &  &  &  \\
\hline
 &  &  &  &  &  &  &  &  &  &  &  &  &  &  &  &  &  &  &  &  &  &  &  &  \\
\hline
 &  &  &  &  &  &  &  &  &  &  &  &  &  &  &  &  &  &  &  &  &  &  &  &  \\
\hline
 &  &  &  &  &  &  &  &  &  &  &  &  &  &  &  &  &  &  &  &  &  &  &  &  \\
\hline
 &  &  &  &  &  &  &  &  &  &  &  &  &  &  &  &  &  &  &  &  &  &  &  &  \\
\hline
 &  &  &  &  &  &  &  &  &  &  &  &  &  &  &  &  &  &  &  &  &  &  &  &  \\
\hline
 &  &  &  &  &  &  &  &  &  &  &  &  &  &  &  &  &  &  &  &  &  &  &  &  \\
\hline
 &  &  &  &  &  &  &  &  &  &  &  &  &  &  &  &  &  &  &  &  &  &  &  &  \\
\hline
 &  &  &  &  &  &  &  &  &  &  &  &  &  &  &  &  &  &  &  &  &  &  &  &  \\
\hline
 &  &  &  &  &  &  &  &  &  &  &  &  &  &  &  &  &  &  &  &  &  &  &  &  \\
\hline
\end{tabular}
\end{center}

\section*{Zadanie 31. (0-2)}
Doświadczenie losowe polega na dwukrotnym rzucie symetryczną sześcienną kostką do gry, która na każdej ściance ma inną liczbę oczek - od jednego oczka do sześciu oczek.

Oblicz prawdopodobieństwo zdarzenia \(A\) polegającego na tym, że w pierwszym rzucie wypadnie większa liczba oczek niż w drugim rzucie. Zapisz obliczenia.\\
\includegraphics[max width=\textwidth, center]{2025_02_10_da789c2d54bb18c230ebg-28}

\section*{Zadanie 32. (0-2)}
Właściciel sklepu z zabawkami przeprowadził lokalne badanie rynkowe dotyczące wpływu zmiany ceny zestawu klocków na liczbę kupujących ten produkt. Z badania wynika, że dzienny przychód \(P\) ze sprzedaży zestawów klocków, w zależności od kwoty obniżki ceny zestawu o \(x\) zł, wyraża się wzorem

\[
P(x)=(70-x)(20+x)
\]

gdzie \(x\) jest liczbą całkowitą spełniającą warunki \(x \geq 0\) i \(x \leq 60\).

\section*{Uzupełnij tabelę. Wpisz w każdą pustą komórkę tabeli właściwą odpowiedź, wybraną spośród oznaczonych literami A-E.}
\begin{center}
\begin{tabular}{|l|l|l|}
\hline
32.1. & \begin{tabular}{l}
Dzienny przychód ze sprzedaży zestawów klocków będzie największy, \\
gdy liczba \(x\) jest równa \\
\end{tabular} &  \\
\hline
32.2. & \begin{tabular}{l}
Dzienny przychód ze sprzedaży zestawów klocków będzie równy \(800 \mathrm{zł}\), \\
gdy liczba \(x\) jest równa \\
\end{tabular} &  \\
\hline
\end{tabular}
\end{center}

A. 25\\
B. 30\\
C. 45\\
D. 50\\
E. 60\\
\includegraphics[max width=\textwidth, center]{2025_02_10_da789c2d54bb18c230ebg-29}

BRUDNOPIS (nie podlega ocenie)\\
\includegraphics[max width=\textwidth, center]{2025_02_10_da789c2d54bb18c230ebg-30}\\
\includegraphics[max width=\textwidth, center]{2025_02_10_da789c2d54bb18c230ebg-31}

\section*{MATEMATYKA}
\section*{Poziom podstawowy}
Formuła 2023

\section*{MATEMATYKA}
\section*{Poziom podstawowy}
Formuła 2023

\section*{MATEMATYKA}
\section*{Poziom podstawowy}
Formuła 2023


\end{document}