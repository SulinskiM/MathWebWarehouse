% This LaTeX document needs to be compiled with XeLaTeX.
\documentclass[10pt]{article}
\usepackage[utf8]{inputenc}
\usepackage{ucharclasses}
\usepackage{amsmath}
\usepackage{amsfonts}
\usepackage{amssymb}
\usepackage[version=4]{mhchem}
\usepackage{stmaryrd}
\usepackage{graphicx}
\usepackage[export]{adjustbox}
\graphicspath{ {./images/} }
\usepackage{bbold}
\usepackage{multirow}
\usepackage[fallback]{xeCJK}
\usepackage{polyglossia}
\usepackage{fontspec}
\IfFontExistsTF{Noto Serif CJK KR}
{\setCJKmainfont{Noto Serif CJK KR}}
{\IfFontExistsTF{Apple SD Gothic Neo}
  {\setCJKmainfont{Apple SD Gothic Neo}}
  {\IfFontExistsTF{UnDotum}
    {\setCJKmainfont{UnDotum}}
    {\setCJKmainfont{Malgun Gothic}}
}}

\setmainlanguage{polish}
\setotherlanguages{hindi, thai}
\IfFontExistsTF{Noto Serif Devanagari}
{\newfontfamily\hindifont{Noto Serif Devanagari}}
{\IfFontExistsTF{Kohinoor Devanagari}
  {\newfontfamily\hindifont{Kohinoor Devanagari}}
  {\IfFontExistsTF{Devanagari MT}
    {\newfontfamily\hindifont{Devanagari MT}}
    {\IfFontExistsTF{Lohit Devanagari}
      {\newfontfamily\hindifont{Lohit Devanagari}}
      {\IfFontExistsTF{FreeSerif}
        {\newfontfamily\hindifont{FreeSerif}}
        {\newfontfamily\hindifont{Arial Unicode MS}}
}}}}
\IfFontExistsTF{Noto Serif Thai}
{\newfontfamily\thaifont{Noto Serif Thai}}
{\IfFontExistsTF{Thonburi}
  {\newfontfamily\thaifont{Thonburi}}
  {\IfFontExistsTF{FreeSerif}
    {\newfontfamily\thaifont{FreeSerif}}
    {\IfFontExistsTF{Tahoma}
      {\newfontfamily\thaifont{Tahoma}}
      {\newfontfamily\thaifont{Arial Unicode MS}}
}}}
\IfFontExistsTF{CMU Serif}
{\newfontfamily\lgcfont{CMU Serif}}
{\IfFontExistsTF{DejaVu Sans}
  {\newfontfamily\lgcfont{DejaVu Sans}}
  {\newfontfamily\lgcfont{Georgia}}
}
\setDefaultTransitions{\lgcfont}{}
\setTransitionsForDevanagari{\hindifont}{\rmfamily}
\setTransitionsFor{Thai}{\thaifont}{\lgcfont}

\title{Kolejne zadania egzaminacyjne są wydrukowane na następnych stronach. }

\author{Liczba punktów do uzyskania: 46}
\date{}


\newcommand\Varangle{\mathop{{<\!\!\!\!\!\text{\small)}}\:}\nolimits}

\begin{document}
\maketitle
CENTRALNA\\
KOMISJA\\
EGZAMINACYJNA

\section*{Miejsce na naklejkę.}
 Sprawdż, czy kod na naklejce to M-100.Jeżeli tak - przyklej naklejke. Jeżeli nie - zgłoś to nauczycielowi.

\section*{Egzamin maturalny}
Formuła 2023

\section*{MATEMATYKA}
\section*{Poziom podstawowy}
\section*{TEST DIAGNOSTYCZNY}
Symbol arkusza\\
MMAP-P0-100-2212

\section*{Data: \(\mathbf{1 4}\) grudnia 2022 r.}
Godzina rozpoczecia: 9:00\\
CZAS trwania: \(\mathbf{1 8 0}\) minut

\begin{verbatim}
WYPEŁNIA ZESPÓŁ NADZORUJACY
Uprawnienia zdającego do:
    dostosowania zasad oceniania
        dostosowania w zw. z dyskalkulią
    nieprzenoszenia zaznaczeń na kartę.
\end{verbatim}



\section*{Przed rozpoczęciem pracy z arkuszem egzaminacyjnym}
\begin{enumerate}
  \item Sprawdź, czy nauczyciel przekazał Ci właściwy arkusz egzaminacyjny, tj. arkusz we właściwej formule, z właściwego przedmiotu na właściwym poziomie.
  \item Jeżeli przekazano Ci niewłaściwy arkusz - natychmiast zgłoś to nauczycielowi. Nie rozrywaj banderol.
  \item Jeżeli przekazano Ci właściwy arkusz - rozerwij banderole po otrzymaniu takiego polecenia od nauczyciela. Zapoznaj się z instrukcją na stronie 2.\\
\includegraphics[max width=\textwidth, center]{2025_02_09_82ae64ed0062449be9d0g-02}
\end{enumerate}

\section*{Instrukcja dla zdającego}
\begin{enumerate}
  \item Sprawdź, czy arkusz egzaminacyjny zawiera 32 strony (zadania 1-33). Ewentualny brak zgłoś przewodniczącemu zespołu nadzorującego egzamin.
  \item Na pierwszej stronie arkusza oraz na karcie odpowiedzi wpisz swój numer PESEL i przyklej naklejkę z kodem.
  \item Pamiętaj, że pominięcie argumentacji lub istotnych obliczeń w rozwiązaniu zadania otwartego może spowodować, że za to rozwiązanie nie otrzymasz pełnej liczby punktów.
  \item Rozwiązania zadań i odpowiedzi wpisuj w miejscu na to przeznaczonym.
  \item Symbol zamieszczony w nagłówku zadania oznacza, że rozwiązanie zadania zamkniętego musisz przenieść na kartę odpowiedzi.
  \item Odpowiedzi do zadań zamkniętych zaznacz na karcie odpowiedzi w części karty przeznaczonej dla zdającego. Zamaluj \(\square\) pola do tego przeznaczone. Błędne zaznaczenie otocz kółkiem i i zaznacz właściwe.
  \item Nie wpisuj żadnych znaków w tabelkach przeznaczonych dla egzaminatora. Tabelki umieszczone są na marginesie przy odpowiednich zadaniach.
  \item Pisz czytelnie i używaj tylko długopisu lub pióra z czarnym tuszem lub atramentem.
  \item Nie używaj korektora, a błędne zapisy wyraźnie przekreśl.
  \item Pamiętaj, że zapisy w brudnopisie nie będą oceniane.
  \item Możesz korzystać z Wybranych wzorów matematycznych, cyrkla i linijki oraz kalkulatora prostego. Upewnij się, czy przekazano Ci broszurę z taką okładką, jak poniżej.\\
\includegraphics[max width=\textwidth, center]{2025_02_09_82ae64ed0062449be9d0g-02(1)}
\end{enumerate}

\section*{Zadania egzaminacyjne są wydrukowane na następnych stronach.}
\section*{Zadanie 1. (0-1) 뚱}
Dokończ zdanie. Wybierz właściwą odpowiedź spośród podanych.\\
Liczba \(\left(5 \cdot 5^{\frac{1}{2}}\right)^{\frac{1}{3}}\) jest równa\\
A. \(\sqrt[6]{5}\)\\
B. \(\sqrt[3]{25}\)\\
C. \(\sqrt{5}\)\\
D. \(\sqrt[3]{5}\)\\
\includegraphics[max width=\textwidth, center]{2025_02_09_82ae64ed0062449be9d0g-04}

\section*{Zadanie 2. (0-1)}
Pan Nowak kupił obligacje Skarbu Państwa za 40000 zł oprocentowane \(7 \%\) w skali roku. Odsetki są naliczane i kapitalizowane co rok.

Dokończ zdanie. Wybierz właściwą odpowiedź spośród podanych.\\
Wartość obligacji kupionych przez pana Nowaka będzie po dwóch latach równa\\
A. \(40000 \cdot(1,07)^{2} \mathrm{zł}\)\\
B. \(40000 \cdot(1,7)^{2} \mathrm{z}\)\\
C. \(40000 \cdot 1,14 \mathrm{zł}\)\\
D. \(40000 \cdot 1,49 \mathrm{zt}\)\\
\includegraphics[max width=\textwidth, center]{2025_02_09_82ae64ed0062449be9d0g-04(1)}

Zadanie 3. (0-1) वाष्ण\\
Właściciel sklepu kupił whurtowni 50 par identycznych spodni po \(x\) zł za parę\\
i 40 identycznych marynarek po \(y\) zł za sztukę. Za zakupy whurtowni zapłacił 8000 zł. Po doliczeniu marży \(50 \%\) na każdą parę spodni i \(20 \%\) na każdą marynarkę ceny detaliczne spodni i marynarki były jednakowe.

Dokończ zdanie. Wybierz właściwą odpowiedź spośród podanych.

Cenę pary spodni \(x\) oraz cenę marynarki \(y\), jakie trzeba zapłacić w hurtowni, można obliczyć z układu równań\\
A. \(\left\{\begin{array}{l}x+y=8000 \\ 0,5 x=0,2 y\end{array}\right.\)\\
B. \(\left\{\begin{array}{l}50 x+40 y=8000 \\ 0,5 x=0,2 y\end{array}\right.\)\\
C. \(\left\{\begin{array}{l}50 x+40 y=8000 \\ 1,5 x=1,2 y\end{array}\right.\)\\
D. \(\left\{\begin{array}{l}x+y=8000 \\ 1,5 x=1,2 y\end{array}\right.\)\\
\includegraphics[max width=\textwidth, center]{2025_02_09_82ae64ed0062449be9d0g-05}

Zadanie 4. (0-1) 뚬\\
Liczby rzeczywiste \(x\) i \(y\) są dodatnie oraz \(x \neq y\).\\
Dokończ zdanie. Wybierz właściwą odpowiedź spośród podanych.\\
Wyrażenie \(\frac{1}{x-y}+\frac{1}{x+y}\) można przekształcić do postaci\\
A. \(\frac{2}{x-y}\)\\
B. \(\frac{2}{x^{2}-y^{2}}\)\\
C. \(\frac{2 x}{x^{2}-y^{2}}\)\\
D. \(\frac{-2 x y}{x+y}\)

\begin{center}
\begin{tabular}{|c|c|c|c|c|c|c|c|c|c|c|c|c|c|c|c|c|c|c|c|c|c|c|c|c|c|c|c|c|c|c|}
\hline
\multicolumn{5}{|l|}{Brudnopis} &  &  &  &  &  &  &  &  &  &  &  &  &  &  &  &  &  &  &  &  &  &  &  &  &  &  \\
\hline
 &  &  &  &  &  &  &  &  &  &  &  &  &  &  &  &  &  &  &  &  &  &  &  &  &  &  &  &  &  &  \\
\hline
 &  &  &  &  &  &  &  &  &  &  &  &  &  &  &  &  &  &  &  &  &  &  &  &  &  &  &  &  &  &  \\
\hline
 &  &  &  &  &  &  &  &  &  &  &  &  &  &  &  &  &  &  &  &  &  &  &  &  &  &  &  &  &  &  \\
\hline
 &  &  &  &  &  &  &  &  &  &  &  &  &  &  &  &  &  &  &  &  &  &  &  &  &  &  &  &  &  &  \\
\hline
 &  &  &  &  &  &  &  &  &  &  &  &  &  &  &  &  &  &  &  &  &  &  &  &  &  &  &  &  &  &  \\
\hline
 &  &  &  &  &  &  &  &  &  &  &  &  &  &  &  &  &  &  &  &  &  &  &  &  &  &  &  &  &  &  \\
\hline
 &  &  &  &  &  &  &  &  &  &  &  &  &  &  &  &  &  &  &  &  &  &  &  &  &  &  &  &  &  &  \\
\hline
 &  &  &  &  &  &  &  &  &  &  &  &  &  &  &  &  &  &  &  &  &  &  &  &  &  &  &  &  &  &  \\
\hline
 &  &  &  &  &  &  &  &  &  &  &  &  &  &  &  &  &  &  &  &  &  &  &  &  &  &  &  &  &  &  \\
\hline
 &  &  &  &  &  &  &  &  &  &  &  &  &  &  &  &  &  &  &  &  &  &  &  &  &  &  &  &  &  &  \\
\hline
 &  &  &  &  &  &  &  &  &  &  &  &  &  &  &  &  &  &  &  &  &  &  &  &  &  &  &  &  &  &  \\
\hline
 &  &  &  &  &  &  &  &  &  &  &  &  &  &  &  &  &  &  &  &  &  &  &  &  &  &  &  &  &  &  \\
\hline
 &  &  &  &  &  &  &  &  &  &  &  &  &  &  &  &  &  &  &  &  &  &  &  &  &  &  &  &  &  &  \\
\hline
\end{tabular}
\end{center}

\section*{Zadanie 5. (0-1)}
Dokończ zdanie. Wybierz właściwą odpowiedź spośród podanych.\\
Wszystkich różnych liczb naturalnych czterocyfrowych, w których zapisie dziesiętnym wszystkie cyfry są różne, jest\\
A. \(9 \cdot 8 \cdot 7 \cdot 6\)\\
B. \(9 \cdot 9 \cdot 8 \cdot 7\)\\
C. \(10 \cdot 9 \cdot 8 \cdot 7\)\\
D. \(9 \cdot 10 \cdot 10 \cdot 10\)

\begin{center}
\begin{tabular}{|c|c|c|c|c|c|c|c|c|c|c|c|c|c|c|c|c|c|c|c|c|c|c|}
\hline
\multicolumn{5}{|l|}{Brudnopis} &  &  &  &  &  &  &  &  &  &  &  & - &  &  &  &  &  &  \\
\hline
 &  &  &  &  &  &  &  &  &  &  &  &  &  &  &  &  &  &  &  &  &  &  \\
\hline
 &  &  &  &  &  &  &  &  &  &  &  &  &  &  &  &  &  &  &  &  &  &  \\
\hline
 &  &  &  &  &  &  &  &  &  &  &  &  &  &  &  &  &  &  &  &  &  &  \\
\hline
 &  &  &  &  &  &  &  &  &  &  &  &  &  &  &  &  &  &  &  &  &  &  \\
\hline
 &  &  &  &  &  &  &  &  &  &  &  &  &  &  &  &  &  &  &  &  &  &  \\
\hline
 &  &  &  &  &  &  &  &  &  &  &  &  &  &  &  &  &  &  &  &  &  &  \\
\hline
 &  &  &  &  &  &  &  &  &  &  &  &  &  &  &  &  &  &  &  &  &  &  \\
\hline
 &  &  &  &  &  &  &  &  &  &  &  &  &  &  &  &  &  &  &  &  &  &  \\
\hline
 &  &  &  &  &  &  &  &  &  &  &  &  &  &  &  &  &  &  &  &  &  &  \\
\hline
 &  &  &  &  &  &  &  &  &  &  &  &  &  &  &  &  &  &  &  &  &  &  \\
\hline
 &  &  &  &  &  &  &  &  &  &  &  &  &  &  &  &  &  &  &  &  &  &  \\
\hline
 &  &  &  &  &  &  &  &  &  &  &  &  &  &  &  &  &  &  &  &  &  &  \\
\hline
\end{tabular}
\end{center}

Zadanie 6. (0-1)\\
Funkcja \(f\) jest określona wzorem \(f(x)=-\log x\) dla wszystkich liczb rzeczywistych dodatnich \(x\).

Dokończ zdanie. Wybierz właściwą odpowiedź spośród podanych.\\
Wartość funkcji \(f\) dla argumentu \(x=\sqrt{10}\) jest równa\\
A. 2\\
B. \(\left(-\frac{1}{2}\right)\)\\
C. \(\frac{1}{2}\)\\
D. \((-2)\)\\
\includegraphics[max width=\textwidth, center]{2025_02_09_82ae64ed0062449be9d0g-07}

\section*{Zadanie 7.}
W kartezjańskim układzie współrzędnych ( \(x, y\) ) przedstawiono fragment wykresu funkcji kwadratowej \(f(x)=a x^{2}+b x+c\). Wierzchołek paraboli, która jest wykresem funkcji \(f\), ma współrzędne \((5,-3)\). Jeden z punktów przecięcia paraboli z osią \(O x\) układu współrzędnych ma współrzędne \((4,0)\).\\
\includegraphics[max width=\textwidth, center]{2025_02_09_82ae64ed0062449be9d0g-08(1)}

Zadanie 7.1. (0-1)\\
Zapisz poniżej zbiór wszystkich wartości funkcji \(f\).\\
\(\qquad\)\\
\includegraphics[max width=\textwidth, center]{2025_02_09_82ae64ed0062449be9d0g-08}

Zadanie 7.2. (0-2)\\
Wyznacz wzór funkcji kwadratowej \(f\) w postaci kanonicznej.

Zapisz obliczenia.\\
\includegraphics[max width=\textwidth, center]{2025_02_09_82ae64ed0062449be9d0g-09}

Zadanie 8. (0-1) ㄸำ\\
Dana jest nierówność kwadratowa

\[
(3 x-9)(x+k)<0
\]

\(z\) niewiadomą \(x\) i parametrem \(k \in \mathbb{R}\). Rozwiązaniem tej nierówności jest przedział \((-2,3)\).\\
Dokończ zdanie. Wybierz właściwą odpowiedź spośród podanych.\\
Liczba \(k\) jest równa\\
A. \((-2)\)\\
B. 2\\
C. \((-3)\)\\
D. 3

\begin{center}
\begin{tabular}{|c|c|c|c|c|c|c|c|c|c|c|c|c|c|c|c|c|c|c|c|c|c|c|}
\hline
\multicolumn{5}{|l|}{Brudnopis} &  & - &  & - &  & - & - &  & - &  & - & - & - &  & - & - & - & - \\
\hline
 &  &  &  &  &  &  &  &  &  &  & - &  &  &  &  &  &  &  &  &  &  &  \\
\hline
 &  &  &  &  &  &  &  &  &  &  &  &  &  &  &  &  &  &  &  &  &  &  \\
\hline
 &  &  &  &  &  &  &  &  &  &  &  &  &  &  &  &  &  &  &  &  &  &  \\
\hline
 &  &  &  &  &  &  &  &  &  &  &  &  &  &  &  &  &  &  &  &  &  &  \\
\hline
 &  &  &  &  &  &  &  &  &  &  &  &  &  &  &  &  &  &  &  &  &  &  \\
\hline
 &  &  &  &  &  &  &  &  &  &  &  &  &  &  &  &  &  &  &  &  &  &  \\
\hline
 &  &  &  &  &  &  &  &  &  &  &  &  &  &  &  &  &  &  &  &  &  &  \\
\hline
 &  &  &  &  &  &  &  &  &  &  &  &  &  &  &  &  &  &  &  &  &  &  \\
\hline
\end{tabular}
\end{center}

\section*{Zadanie 9. (0-1) 띰}
Dana jest funkcja kwadratowa \(f(x)=a x^{2}+b x+c\), gdzie \(a, b\) i \(c\) są liczbami rzeczywistymi takimi, że \(a \neq 0\) oraz \(c<0\). Funkcja \(f\) nie ma miejsc zerowych.

Dokończ zdanie tak, aby było prawdziwe. Wybierz odpowiedź A albo B oraz jej uzasadnienie 1., 2. albo 3.

Wykres funkcji \(f\) leży w całości

\begin{center}
\begin{tabular}{|c|c|c|c|c|}
\hline
\multirow{2}{*}{A.} & \multirow{2}{*}{nad osią \(0 x\),} & \multirow{3}{*}{ponieważ} & 1. & \(a<0\) i \(b^{2}-4 a c<0\). \\
\hline
 &  &  & 2. & \(a>0\) i \(b^{2}-4 a c<0\). \\
\hline
B. & pod osią \(O x\), &  & 3. & \(a<0\) i \(b^{2}-4 a c=0\). \\
\hline
\end{tabular}
\end{center}

\begin{center}
\begin{tabular}{|c|c|c|c|c|c|c|c|c|c|c|c|c|c|c|c|c|c|c|c|c|c|c|c|}
\hline
\multicolumn{4}{|l|}{Brudnopis} &  &  &  &  &  &  &  &  &  &  &  &  &  &  &  &  &  &  &  &  \\
\hline
 &  &  &  &  &  &  &  &  &  &  &  &  &  &  &  &  &  &  &  &  &  &  &  \\
\hline
 &  &  &  &  &  &  &  &  &  &  &  &  &  &  &  &  &  &  &  &  &  &  &  \\
\hline
 &  &  &  &  &  &  &  &  &  &  &  &  &  &  &  &  &  &  &  &  &  &  &  \\
\hline
 &  &  &  &  &  &  &  &  &  &  &  &  &  &  &  &  &  &  &  &  &  &  &  \\
\hline
 &  &  &  &  &  &  &  &  &  &  &  &  &  &  &  &  &  &  &  &  &  &  &  \\
\hline
 &  &  &  &  &  &  &  &  &  &  &  &  &  &  &  &  &  &  &  &  &  &  &  \\
\hline
 &  &  &  &  &  &  &  &  &  &  &  &  &  &  &  &  &  &  &  &  &  &  &  \\
\hline
\end{tabular}
\end{center}

Zadanie 10. (0-1) 뚬\\
Dany jest układ równań

\[
\left\{\begin{array}{l}
y=x-1 \\
y=-x+1
\end{array}\right.
\]

Na którym z rysunków A-D przedstawiona jest interpretacja geometryczna tego układu równań? Wybierz właściwą odpowiedź spośród podanych.\\
A.\\
\includegraphics[max width=\textwidth, center]{2025_02_09_82ae64ed0062449be9d0g-11}\\
C.\\
\includegraphics[max width=\textwidth, center]{2025_02_09_82ae64ed0062449be9d0g-11(2)}\\
B.\\
\includegraphics[max width=\textwidth, center]{2025_02_09_82ae64ed0062449be9d0g-11(1)}\\
D.\\
\includegraphics[max width=\textwidth, center]{2025_02_09_82ae64ed0062449be9d0g-11(3)}

Zadanie 11. (0-1) 뚬\\
Dany jest wielomian \(W\) określony wzorem \(W(x)=x^{3}-2 x^{2}-3 x+6\) dla każdej liczby rzeczywistej \(x\).

Dokończ zdanie. Wybierz właściwą odpowiedź spośród podanych.\\
Wielomian \(W\) przy rozkładzie na czynniki ma postać\\
A. \(W(x)=(x+2)\left(x^{2}-3\right)\)\\
B. \(W(x)=(x-2)\left(x^{2}-3\right)\)\\
C. \(W(x)=(x+2)\left(x^{2}+3\right)\)\\
D. \(W(x)=(x-2)\left(x^{2}+3\right)\)\\
\includegraphics[max width=\textwidth, center]{2025_02_09_82ae64ed0062449be9d0g-12}

Zadanie 12. (0-1) 뚱\\
Dokończ zdanie. Wybierz właściwą odpowiedź spośród podanych.\\
Równanie \(\frac{(4-x)(2 x-3)}{(3 x-5)(3-2 x)}=0 \mathrm{w}\) zbiorze liczb rzeczywistych ma dokładnie\\
A. jedno rozwiązanie.\\
B. dwa rozwiązania.\\
C. trzy rozwiązania.\\
D. cztery rozwiązania.

\begin{center}
\begin{tabular}{|c|c|c|c|c|c|c|c|c|c|c|c|c|c|c|c|c|c|c|c|c|c|c|c|c|}
\hline
\multicolumn{4}{|l|}{Brudnopis} &  &  &  &  &  &  &  &  &  &  &  &  &  &  &  &  &  &  &  &  &  \\
\hline
 &  &  &  &  &  &  &  &  &  &  &  &  &  &  &  &  &  &  &  &  &  &  &  &  \\
\hline
 &  &  &  &  &  &  &  &  &  &  &  &  &  &  &  &  &  &  &  &  &  &  &  &  \\
\hline
 &  &  &  &  &  &  &  &  &  &  &  &  &  &  &  &  &  &  &  &  &  &  &  &  \\
\hline
 &  &  &  &  &  &  &  &  &  &  &  &  &  &  &  &  &  &  &  &  &  &  &  &  \\
\hline
 &  &  &  &  &  &  &  &  &  &  &  &  &  &  &  &  &  &  &  &  &  &  &  &  \\
\hline
\end{tabular}
\end{center}

Zadanie 13. (0-1)\\
Dana jest nierówność

\[
2-\frac{x}{2} \geq \frac{x}{3}-3
\]

Dokończ zdanie. Wybierz właściwą odpowiedź spośród podanych.\\
Największą liczbą całkowitą, która spełnia tę nierówność, jest\\
A. 6\\
B. 5\\
C. 7\\
D. \((-6)\)\\
\includegraphics[max width=\textwidth, center]{2025_02_09_82ae64ed0062449be9d0g-13}

\section*{Zadanie 14. (0-2)}
Wykaż, że dla każdej liczby naturalnej \(n\) liczba \(5 n^{2}+15 n\) jest podzielna przez 10 .\\
\includegraphics[max width=\textwidth, center]{2025_02_09_82ae64ed0062449be9d0g-13(1)}

Zadanie 15. (0-1) 뚜\\
Dany jest ciąg \(\left(a_{n}\right)\) określony wzorem \(a_{n}=2 n^{2}+n\) dla każdej liczby naturalnej \(n \geq 1\).\\
Oceń prawdziwość poniższych stwierdzeń. Wybierz \(P\), jeśli stwierdzenie jest prawdziwe, albo F - jeśli jest fałszywe.

\begin{center}
\begin{tabular}{|l|c|c|}
\hline
Ciąg \(\left(a_{n}\right)\) jest malejący. & \(\mathbf{P}\) & \(\mathbf{F}\) \\
\hline
Ósmy wyraz ciągu \(\left(a_{n}\right)\) jest równy 136. & \(\mathbf{P}\) & \(\mathbf{F}\) \\
\hline
\end{tabular}
\end{center}

\begin{center}
\begin{tabular}{|c|c|c|c|c|c|c|c|c|c|c|c|c|c|c|c|c|c|c|c|c|c|}
\hline
 & Brud & dnop &  &  &  &  &  &  &  & - &  &  &  &  & - &  &  &  &  &  &  \\
\hline
 &  &  &  &  &  &  &  &  &  &  &  &  & - &  &  &  &  &  &  &  &  \\
\hline
 &  &  &  &  &  &  &  &  &  &  &  &  &  &  &  &  &  &  &  &  &  \\
\hline
 &  &  &  &  &  &  &  &  &  &  &  &  &  &  &  &  &  &  &  &  &  \\
\hline
 &  &  &  &  &  &  &  &  &  &  &  &  &  &  &  &  &  &  &  &  &  \\
\hline
 &  &  &  &  &  &  &  &  &  &  &  &  &  &  &  &  &  &  &  &  &  \\
\hline
 &  &  &  &  &  &  &  &  &  &  &  &  &  &  &  &  &  &  &  &  &  \\
\hline
 &  &  &  &  &  &  &  &  &  &  &  &  &  &  &  &  &  &  &  &  &  \\
\hline
 &  &  &  &  &  &  &  &  &  &  &  &  &  &  &  &  &  &  &  &  &  \\
\hline
 &  &  &  &  &  &  &  &  &  &  &  &  &  &  &  &  &  &  &  &  &  \\
\hline
 &  &  &  &  &  &  &  &  &  &  &  &  &  &  &  &  &  &  &  &  &  \\
\hline
\end{tabular}
\end{center}

Zadanie 16. (0-1)\\
Pięciowyrazowy ciąg \(\left(-3, \frac{1}{2}, x, y, 11\right)\) jest arytmetyczny.\\
Dokończ zdanie. Wybierz właściwą odpowiedź spośród podanych.\\
Liczby \(x\) oraz \(y\) są równe\\
A. \(x=4\) oraz \(y=\frac{15}{2}\).\\
B. \(x=\frac{15}{2}\) oraz \(y=4\).\\
C. \(x=-4\) oraz \(y=\frac{15}{2}\).\\
D. \(x=-\frac{15}{2}\) oraz \(y=4\).\\
\includegraphics[max width=\textwidth, center]{2025_02_09_82ae64ed0062449be9d0g-14}

\section*{Zadanie 17. (0-2)}
Dany jest ciąg geometryczny \(\left(a_{n}\right)\), określony dla każdej liczby naturalnej \(n \geq 1\).\\
W tym ciągu \(a_{1}=-5, a_{2}=15, a_{3}=-45\).

Dokończ zdanie. Zaznacz dwie odpowiedzi tak, aby dla każdej z nich dokończenie poniższego zdania było prawdziwe.

Wzór ogólny ciągu ( \(a_{n}\) ) ma postać\\
A. \(a_{n}=-5 \cdot(-3)^{n-1}\)\\
B. \(a_{n}=-5 \cdot(-3)^{n}\)\\
C. \(a_{n}=-5 \cdot 3^{n-1}\)\\
D. \(a_{n}=-5 \cdot \frac{(-3)^{n}}{3}\)\\
E. \(a_{n}=5 \cdot \frac{(-3)^{n}}{3}\)\\
F. \(a_{n}=5 \cdot(-3)^{n} \cdot 3\)\\
\includegraphics[max width=\textwidth, center]{2025_02_09_82ae64ed0062449be9d0g-15}

Zadanie 18. (0-1) 뚬\\
Kąt \(\alpha\) jest ostry oraz \(\frac{1}{\sin ^{2} \alpha}+\frac{1}{\cos ^{2} \alpha}=\frac{64}{9}\).\\
Dokończ zdanie. Wybierz właściwą odpowiedź spośród podanych.\\
Wartość wyrażenia \(\sin \alpha \cdot \cos \alpha\) jest równa\\
A. \(\frac{8}{3}\)\\
B. \(\frac{3}{8}\)\\
C. \(\frac{64}{9}\)\\
D. \(\frac{9}{64}\)

\begin{center}
\begin{tabular}{|c|c|c|c|c|c|c|c|c|c|c|c|c|c|c|c|c|c|c|c|c|c|c|}
\hline
\multicolumn{4}{|l|}{Brudnopis} &  &  &  &  &  &  &  &  &  &  &  &  &  &  &  &  &  &  &  \\
\hline
 &  &  &  &  &  &  &  &  &  &  &  &  &  &  &  &  &  &  &  &  &  &  \\
\hline
 &  &  &  &  &  &  &  &  &  &  &  &  &  &  &  &  &  &  &  &  &  &  \\
\hline
 &  &  &  &  &  &  &  &  &  &  &  &  &  &  &  &  &  &  &  &  &  &  \\
\hline
 &  &  &  &  &  &  &  &  &  &  &  &  &  &  &  &  &  &  &  &  &  &  \\
\hline
 &  &  &  &  &  &  &  &  &  &  &  &  &  &  &  &  &  &  &  &  &  &  \\
\hline
 &  &  &  &  &  &  &  &  &  &  &  &  &  &  &  &  &  &  &  &  &  &  \\
\hline
 &  &  &  &  &  &  &  &  &  &  &  &  &  &  &  &  &  &  &  &  &  &  \\
\hline
\end{tabular}
\end{center}

\section*{Zadanie 19. (0-1) 푸}
Punkty \(A, B, C\) leżą na okręgu o środku \(O\) (zobacz rysunek).\\
Ponadto \(|\Varangle A O C|=130^{\circ}\) oraz \(|\Varangle B O A|=110^{\circ}\).

Dokończ zdanie. Wybierz właściwą odpowiedź spośród podanych.

Miara kąta wewnętrznego \(B A C\) trójkąta \(A B C\) jest równa\\
A. \(60^{\circ}\)\\
B. \(55^{\circ}\)\\
C. \(50^{\circ}\)\\
D. \(65^{\circ}\)\\
\includegraphics[max width=\textwidth, center]{2025_02_09_82ae64ed0062449be9d0g-16}

Zadanie 20. (0-4)\\
Do wyznaczenia trzech boków pewnego kąpieliska w kształcie prostokąta należy użyć liny o długości 200 m . Czwarty bok tego kąpieliska będzie pokrywał się z brzegiem plaży, który w tym miejscu jest linią prostą (zobacz rysunek).\\
\includegraphics[max width=\textwidth, center]{2025_02_09_82ae64ed0062449be9d0g-18}

Oblicz wymiary \(\boldsymbol{a}\) i \(\boldsymbol{b}\) kąpieliska tak, aby jego powierzchnia była największa.\\
Zapisz obliczenia.\\
\includegraphics[max width=\textwidth, center]{2025_02_09_82ae64ed0062449be9d0g-18(1)}\\
\includegraphics[max width=\textwidth, center]{2025_02_09_82ae64ed0062449be9d0g-19}

Zadanie 21. (0-1)\\
Dany jest kwadrat \(A B C D\) o boku długości 8.\\
Z wierzchołka \(A\) zakreślono koło o promieniu równym długości boku kwadratu (zobacz rysunek).\\
\includegraphics[max width=\textwidth, center]{2025_02_09_82ae64ed0062449be9d0g-20}

Dokończ zdanie. Wybierz właściwą odpowiedź spośród podanych.\\
Pole powierzchni części wspólnej koła i kwadratu jest równe\\
A. \(16 \pi\)\\
B. \(8 \pi\)\\
C. \(4 \sqrt{2} \pi\)\\
D. \(16 \sqrt{2} \pi\)

\begin{center}
\begin{tabular}{|c|c|c|c|c|c|c|c|c|c|c|c|c|c|c|c|c|c|c|c|c|c|c|}
\hline
 & Brudn & dnopis &  &  &  &  &  &  &  &  &  &  &  &  &  &  &  &  &  &  &  &  \\
\hline
 &  &  &  &  &  &  &  &  &  &  &  &  &  &  & - &  &  &  &  &  &  &  \\
\hline
 &  &  &  &  &  &  &  &  &  &  &  &  &  &  &  &  &  &  &  &  &  &  \\
\hline
 &  &  &  &  &  &  &  &  &  &  &  &  &  &  &  &  &  &  &  &  &  &  \\
\hline
 &  &  &  &  &  &  &  &  &  &  &  &  &  &  &  &  &  &  &  &  &  &  \\
\hline
 &  &  &  &  &  &  &  &  &  &  &  &  &  &  &  &  &  &  &  &  &  &  \\
\hline
 &  &  &  &  &  &  &  &  &  &  &  &  &  &  &  &  &  &  &  &  &  &  \\
\hline
 &  &  &  &  &  &  &  &  &  &  &  &  &  &  &  &  &  &  &  &  &  &  \\
\hline
 &  &  &  &  &  &  &  &  &  &  &  &  &  &  &  &  &  &  &  &  &  &  \\
\hline
 &  &  &  &  &  &  &  &  &  &  &  &  &  &  &  &  &  &  &  &  &  &  \\
\hline
 &  &  &  &  &  &  &  &  &  &  &  &  &  &  &  &  &  &  &  &  &  &  \\
\hline
 &  &  &  &  &  &  &  &  &  &  &  &  &  &  &  &  &  &  &  &  &  &  \\
\hline
 &  &  &  &  &  &  &  &  &  &  &  &  &  &  &  &  &  &  &  &  &  &  \\
\hline
 &  &  &  &  &  &  &  &  &  &  &  &  &  &  &  &  &  &  &  &  &  &  \\
\hline
 &  &  &  &  &  &  &  &  &  &  &  &  &  &  &  &  &  &  &  &  &  &  \\
\hline
 &  &  &  &  &  &  &  &  &  &  &  &  &  &  &  &  &  &  &  &  &  &  \\
\hline
 &  &  &  &  &  &  &  &  &  &  &  &  &  &  &  &  &  &  &  &  &  &  \\
\hline
 &  &  &  &  &  &  &  &  &  &  &  &  &  &  &  &  &  &  &  &  &  &  \\
\hline
 &  &  &  &  &  &  &  &  &  &  &  &  &  &  &  &  &  &  &  &  &  &  \\
\hline
 &  &  &  &  &  &  &  &  &  &  &  &  &  &  &  &  &  &  &  &  &  &  \\
\hline
 &  &  &  &  &  &  &  &  &  &  &  &  &  &  &  &  &  &  &  &  &  &  \\
\hline
 &  &  &  &  &  &  &  &  &  &  &  &  &  &  &  &  &  &  &  &  &  &  \\
\hline
 &  &  &  &  &  &  &  &  &  &  &  &  &  &  &  &  &  &  &  &  &  &  \\
\hline
 &  &  &  &  &  &  &  &  &  &  &  &  &  &  &  &  &  &  &  &  &  &  \\
\hline
 &  &  &  &  &  &  &  &  &  &  &  &  &  &  &  &  &  &  &  &  &  &  \\
\hline
 &  &  &  &  &  &  &  &  &  &  &  &  &  &  &  &  &  &  &  &  &  &  \\
\hline
 &  &  &  &  &  &  &  &  &  &  &  &  &  &  &  &  &  &  &  &  &  &  \\
\hline
\end{tabular}
\end{center}

Zadanie 22. (0-1)\\
Odcinki \(A C\) i \(B D\) przecinają się w punkcie \(O\). Ponadto \(|A D|=4\) i \(|O D|=|B C|=6\).\\
Kąty ODA i BCO są proste (zobacz rysunek).\\
\includegraphics[max width=\textwidth, center]{2025_02_09_82ae64ed0062449be9d0g-21}

\section*{Dokończ zdanie. Wybierz właściwą odpowiedź spośród podanych.}
Długość odcinka OC jest równa\\
A. 9\\
B. 8\\
C. \(2 \sqrt{13}\)\\
D. \(3 \sqrt{13}\)

\begin{center}
\begin{tabular}{|c|c|c|c|c|c|c|c|c|c|c|c|c|c|c|c|c|c|c|c|c|c|c|}
\hline
 & Brudn & nopis &  &  &  &  &  &  &  &  &  &  &  &  &  &  &  &  &  &  &  &  \\
\hline
 &  &  &  &  &  &  &  &  &  &  &  &  &  &  &  &  &  &  &  &  &  &  \\
\hline
 &  &  &  &  &  &  &  &  &  &  &  &  &  &  &  &  &  &  &  &  &  &  \\
\hline
 &  &  &  &  &  &  &  &  &  &  &  &  &  &  &  &  &  &  &  &  &  &  \\
\hline
 &  &  &  &  &  &  &  &  &  &  &  &  &  &  &  &  &  &  &  &  &  &  \\
\hline
 &  &  &  &  &  &  &  &  &  &  &  &  &  &  &  &  &  &  &  &  &  &  \\
\hline
 &  &  &  &  &  &  &  &  &  &  &  &  &  &  &  &  &  &  &  &  &  &  \\
\hline
 &  &  &  &  &  &  &  &  &  &  &  &  &  &  &  &  &  &  &  &  &  &  \\
\hline
 &  &  &  &  &  &  &  &  &  &  &  &  &  &  &  &  &  &  &  &  &  &  \\
\hline
 &  &  &  &  &  &  &  &  &  &  &  &  &  &  &  &  &  &  &  &  &  &  \\
\hline
 &  &  &  &  &  &  &  &  &  &  &  &  &  &  &  &  &  &  &  &  &  &  \\
\hline
 &  &  &  &  &  &  &  &  &  &  &  &  &  &  &  &  &  &  &  &  &  &  \\
\hline
 &  &  &  &  &  &  &  &  &  &  &  &  &  &  &  &  &  &  &  &  &  &  \\
\hline
 &  &  &  &  &  &  &  &  &  &  &  &  &  &  &  &  &  &  &  &  &  &  \\
\hline
 &  &  &  &  &  &  &  &  &  &  &  &  &  &  &  &  &  &  &  &  &  &  \\
\hline
 &  &  &  &  &  &  &  &  &  &  &  &  &  &  &  &  &  &  &  &  &  &  \\
\hline
 &  &  &  &  &  &  &  &  &  &  &  &  &  &  &  &  &  &  &  &  &  &  \\
\hline
 &  &  &  &  &  &  &  &  &  &  &  &  &  &  &  &  &  &  &  &  &  &  \\
\hline
 &  &  &  &  &  &  &  &  &  &  &  &  &  &  &  &  &  &  &  &  &  &  \\
\hline
 &  &  &  &  &  &  &  &  &  &  &  &  &  &  &  &  &  &  &  &  &  &  \\
\hline
 &  &  &  &  &  &  &  &  &  &  &  &  &  &  &  &  &  &  &  &  &  &  \\
\hline
 &  &  &  &  &  &  &  &  &  &  &  &  &  &  &  &  &  &  &  &  &  &  \\
\hline
 &  &  &  &  &  &  &  &  &  &  &  &  &  &  &  &  &  &  &  &  &  &  \\
\hline
 &  &  &  &  &  &  &  &  &  &  &  &  &  &  &  &  &  &  &  &  &  &  \\
\hline
 &  &  &  &  &  &  &  &  &  &  &  &  &  &  &  &  &  &  &  &  &  &  \\
\hline
 &  &  &  &  &  &  &  &  &  &  &  &  &  &  &  &  &  &  &  &  &  &  \\
\hline
 &  &  &  &  &  &  &  &  &  &  &  &  &  &  &  &  &  &  &  &  &  &  \\
\hline
\end{tabular}
\end{center}

\section*{Zadanie 23. (0-2)}
Przekątne równoległoboku \(A B C D\) mają długości: \(|A C|=16\) oraz \(|B D|=12\).\\
Wierzchołki \(E, F, G\) oraz \(H\) rombu \(E F G H\) leżą na bokach równoległoboku \(A B C D\) (zobacz rysunek).\\
Boki tego rombu są równoległe do przekątnych równoległoboku.

Oblicz długość boku rombu EFGH.\\
\includegraphics[max width=\textwidth, center]{2025_02_09_82ae64ed0062449be9d0g-22}

\section*{Zapisz obliczenia.}
\begin{center}
\includegraphics[max width=\textwidth]{2025_02_09_82ae64ed0062449be9d0g-22(1)}
\end{center}

Zadanie 24. (0-2)\\
Dany jest trójkąt \(A B C\), w którym \(|A C|=4,|A B|=3, \cos \Varangle B A C=\frac{4}{5}\).\\
Oblicz pole trójkąta \(A B C\).

Zapisz obliczenia.\\
\includegraphics[max width=\textwidth, center]{2025_02_09_82ae64ed0062449be9d0g-23}

\section*{Zadanie 25.}
Dany jest sześciokąt foremny \(A B C D E F\) o polu równym \(6 \sqrt{3}\) (zobacz rysunek).\\
\includegraphics[max width=\textwidth, center]{2025_02_09_82ae64ed0062449be9d0g-24(1)}

\section*{Zadanie 25.1. (0-1) ㄸ..}
Dokończ zdanie. Wybierz właściwą odpowiedź spośród podanych.\\
Pole trójkąta \(A B E\) jest równe\\
A. 6\\
B. \(4 \sqrt{3}\)\\
C. \(2 \sqrt{3}\)\\
D. 4\\
\includegraphics[max width=\textwidth, center]{2025_02_09_82ae64ed0062449be9d0g-24}

Zadanie 25.2. (0-1) बᄄ.\\
Dokończ zdanie. Wybierz właściwą odpowiedź spośród podanych.\\
Długość odcinka \(A E\) jest równa\\
A. 2\\
B. \(2 \sqrt{3}\)\\
C. \(4 \sqrt{3}\)\\
D. 4

\begin{center}
\begin{tabular}{|c|c|c|c|c|c|c|c|c|c|c|c|c|c|c|c|c|c|c|c|c|c|c|}
\hline
 & Brudnopis & opis &  &  &  &  &  &  &  &  &  &  &  &  &  &  &  &  &  &  &  &  \\
\hline
 &  &  &  &  &  &  &  &  &  &  &  &  &  &  &  &  &  &  &  &  &  &  \\
\hline
 &  &  &  &  &  &  &  &  &  &  &  &  &  &  &  &  &  &  &  &  &  &  \\
\hline
 &  &  &  &  &  &  &  &  &  &  &  &  &  &  &  &  &  &  &  &  &  &  \\
\hline
 &  &  &  &  &  &  &  &  &  &  &  &  &  &  &  &  &  &  &  &  &  &  \\
\hline
 &  &  &  &  &  &  &  &  &  &  &  &  &  &  &  &  &  &  &  &  &  &  \\
\hline
 &  &  &  &  &  &  &  &  &  &  &  &  &  &  &  &  &  &  &  &  &  &  \\
\hline
 &  &  &  &  &  &  &  &  &  &  &  &  &  &  &  &  &  &  &  &  &  &  \\
\hline
\end{tabular}
\end{center}

Zadanie 26. (0-1)\\
Dany jest trapez \(A B C D\), w którym \(A B \| C D\) oraz przekątne \(A C\) i \(B D\) przecinają się w punkcie \(O\) (zobacz rysunek). Wysokość tego trapezu jest równa 12. Obwód trójkąta \(A B O\) jest równy 39, a obwód trójkąta \(C D O\) jest równy 13.\\
\includegraphics[max width=\textwidth, center]{2025_02_09_82ae64ed0062449be9d0g-25}

Dokończ zdanie. Wybierz właściwą odpowiedź spośród podanych.\\
Wysokość trójkąta \(A B O\) poprowadzona z punktu \(O\) jest równa\\
A. 3\\
B. 4\\
C. 9\\
D. 6\\
\includegraphics[max width=\textwidth, center]{2025_02_09_82ae64ed0062449be9d0g-25(2)}

\section*{Zadanie 27. (0-1)}
Na płaszczyźnie, w kartezjańskim układzie współrzędnych \((x, y)\), dany jest okrąg \(\mathcal{O}\) o równaniu

\[
(x-3)^{2}+(y-3)^{2}=13
\]

Dokończ zdanie. Wybierz właściwą odpowiedź spośród podanych.\\
Okrąg \(\mathcal{O}\) przecina oś \(O y\) w punktach o współrzędnych\\
A. \((0,1) \mathrm{i}(0,5)\).\\
B. \((0,1)\) i \((0,-5)\).\\
C. \((1,0)\) i \((5,0)\).\\
D. \((0,-1)\) i \((0,5)\).\\
\includegraphics[max width=\textwidth, center]{2025_02_09_82ae64ed0062449be9d0g-25(1)}

Zadanie 28. (0-1) 뚬\\
Na płaszczyźnie, w kartezjańskim układzie współrzędnych \((x, y)\), dane są proste \(k\) oraz \(l\) o równaniach\\
\(k\) : \(y=\frac{1}{3} x-1\)\\
\(l: y=-3 x+6\)\\
Dokończ zdanie. Wybierz właściwą odpowiedź spośród podanych.\\
Proste \(k\) oraz \(l\)\\
A. nie mają punktów wspólnych.\\
B. są prostopadłe.\\
C. przecinają się w punkcie \(P=(0,-1)\).\\
D. się pokrywają.\\
\includegraphics[max width=\textwidth, center]{2025_02_09_82ae64ed0062449be9d0g-26}

\section*{Zadanie 29. (0-1) 뚜}
Na płaszczyźnie, w kartezjańskim układzie współrzędnych \((x, y)\), dane są punkty \(A=(1,2)\) i \(B=(2 m, m)\), gdzie \(m\) jest liczbą rzeczywistą, oraz prosta \(k\) o równaniu \(y=-x-1\).

Dokończ zdanie. Wybierz właściwą odpowiedź spośród podanych.\\
Prosta przechodząca przez punkty \(A\) i \(B\) jest równoległa do prostej \(k\), gdy\\
A. \(m=-1\)\\
B. \(m=1\)\\
C. \(m=\frac{1}{2}\)\\
D. \(m=2\)\\
\includegraphics[max width=\textwidth, center]{2025_02_09_82ae64ed0062449be9d0g-26(1)}

\section*{Zadanie 30.}
Dany jest sześcian \(A B C D E F G H\) o krawędzi długości 9. Wierzchołki podstawy \(A B C D\) sześcianu połączono odcinkami z punktem \(W\), który jest punktem przecięcia przekątnych podstawy \(E F G H\). Otrzymano w ten sposób ostrosłup prawidłowy czworokątny \(A B C D W\) (zobacz rysunek).\\
\includegraphics[max width=\textwidth, center]{2025_02_09_82ae64ed0062449be9d0g-27}

Zadanie 30.1. (0-1)\\
Dokończ zdanie. Wybierz właściwą odpowiedź spośród podanych.\\
Objętość \(V\) ostrosłupa \(A B C D W\) jest równa\\
A. 243\\
B. 364,5\\
C. 489\\
D. 729\\
\includegraphics[max width=\textwidth, center]{2025_02_09_82ae64ed0062449be9d0g-27(1)}

Zadanie 30.2. (0-2)\\
Oblicz cosinus kąta nachylenia krawędzi bocznej ostrosłupa do płaszczyzny podstawy.

Zapisz obliczenia.

\begin{center}
\begin{tabular}{|c|c|c|c|c|c|c|c|c|c|c|c|c|c|c|c|c|c|c|c|c|c|c|}
\hline
 &  &  &  &  &  &  &  &  &  &  &  &  &  &  &  &  &  &  &  &  &  &  \\
\hline
 &  &  &  &  &  &  &  &  &  &  &  &  &  &  &  &  &  &  &  &  &  &  \\
\hline
 &  &  &  &  &  &  &  &  &  &  &  &  &  &  &  &  &  &  &  &  &  &  \\
\hline
 &  &  &  &  &  &  &  &  &  &  &  &  &  &  &  &  &  &  &  &  &  &  \\
\hline
 &  &  &  &  &  &  &  &  &  &  &  &  &  &  &  &  &  &  &  &  &  &  \\
\hline
 &  &  &  &  &  &  &  &  &  &  &  &  &  &  &  &  &  &  &  &  &  &  \\
\hline
 &  &  &  &  &  &  &  &  &  &  &  &  &  &  &  &  &  &  &  &  &  &  \\
\hline
 &  &  &  &  &  &  &  &  &  &  &  &  &  &  &  &  &  &  &  &  &  &  \\
\hline
 &  &  &  &  &  &  &  &  &  &  &  &  &  &  &  &  &  &  &  &  &  &  \\
\hline
 &  &  &  &  &  &  &  &  &  &  &  &  &  &  &  &  &  &  &  &  &  &  \\
\hline
 &  &  &  &  &  &  &  &  &  &  &  &  &  &  &  &  &  &  &  &  &  &  \\
\hline
 &  &  &  &  &  &  &  &  &  &  &  &  &  &  &  &  &  &  &  &  &  &  \\
\hline
 &  &  &  &  &  &  &  &  &  &  &  &  &  &  &  &  &  &  &  &  &  &  \\
\hline
 &  &  &  &  &  &  &  &  &  &  &  &  &  &  &  &  &  &  &  &  &  &  \\
\hline
 &  &  &  &  &  &  &  &  &  &  &  &  &  &  &  &  &  &  &  &  &  &  \\
\hline
 &  &  &  &  &  &  &  &  &  &  &  &  &  &  &  &  &  &  &  &  &  &  \\
\hline
 &  &  &  &  &  &  &  &  &  &  &  &  &  &  &  &  &  &  &  &  &  &  \\
\hline
 &  &  &  &  &  &  &  &  &  &  &  &  &  &  &  &  &  &  &  &  &  &  \\
\hline
 &  &  &  &  &  &  &  &  &  &  &  &  &  &  &  &  &  &  &  &  &  &  \\
\hline
 &  &  &  &  &  &  &  &  &  &  &  &  &  &  &  &  &  &  &  &  &  &  \\
\hline
 &  &  &  &  &  &  &  &  &  &  &  &  &  &  &  &  &  &  &  &  &  &  \\
\hline
 &  &  &  &  &  &  &  &  &  &  &  &  &  &  &  &  &  &  &  &  &  &  \\
\hline
 &  &  &  &  &  &  &  &  &  &  &  &  &  &  &  &  &  &  &  &  &  &  \\
\hline
 &  &  &  &  &  &  &  &  &  &  &  &  &  &  &  &  &  &  &  &  &  &  \\
\hline
 &  &  &  &  &  &  &  &  &  &  &  &  &  &  &  &  &  &  &  &  &  &  \\
\hline
 &  &  &  &  &  &  &  &  &  &  &  &  &  &  &  &  &  &  &  &  &  &  \\
\hline
 &  &  &  &  &  &  &  &  &  &  &  &  &  &  &  &  &  &  &  &  &  &  \\
\hline
 &  &  &  &  &  &  &  &  &  &  &  &  &  &  &  &  &  &  &  &  &  &  \\
\hline
 &  &  &  &  &  &  &  &  &  &  &  &  &  &  &  &  &  &  &  &  &  &  \\
\hline
 &  &  &  &  &  &  &  &  &  &  &  &  &  &  &  &  &  &  &  &  &  &  \\
\hline
 &  &  &  &  &  &  &  &  &  &  &  &  &  &  &  &  &  &  &  &  &  &  \\
\hline
 &  &  &  &  &  &  &  &  &  &  &  &  &  &  &  &  &  &  &  &  &  &  \\
\hline
 &  &  &  &  &  &  &  &  &  &  &  &  &  &  &  &  &  &  &  &  &  &  \\
\hline
 &  &  &  &  &  &  &  &  &  &  &  &  &  &  &  &  &  &  &  &  &  &  \\
\hline
 &  &  &  &  &  &  &  &  &  &  &  &  &  &  &  &  &  &  &  &  &  &  \\
\hline
 &  &  &  &  &  &  &  &  &  &  &  &  &  &  &  &  &  &  &  &  &  &  \\
\hline
 &  &  &  &  &  &  &  &  &  &  &  &  &  &  &  &  &  &  &  &  &  &  \\
\hline
 &  &  &  &  &  &  &  &  &  &  &  &  &  &  &  &  &  &  &  &  &  &  \\
\hline
 &  &  &  &  &  &  &  &  &  &  &  &  &  &  &  &  &  &  &  &  &  &  \\
\hline
 &  &  &  &  &  &  &  &  &  &  &  &  &  &  &  &  &  &  &  &  &  &  \\
\hline
 &  &  &  &  &  &  &  &  &  &  &  &  &  &  &  &  &  &  &  &  &  &  \\
\hline
 &  &  &  &  &  &  &  &  &  &  &  &  &  &  &  &  &  &  &  &  &  &  \\
\hline
\end{tabular}
\end{center}

Zadanie 31. (0-1) 뚬\\
Dany jest sześcian \(\mathcal{F}\) o krawędzi długości \(a\) i objętości \(V\) oraz sześcian \(\mathcal{G}\) o krawędzi długości \(3 a\).

Dokończ zdanie. Wybierz właściwą odpowiedź spośród podanych.\\
Objętość sześcianu \(\mathcal{G}\) jest równa\\
A. 3 V\\
B. 9 V\\
C. 18 V\\
D. 27 V

\begin{center}
\begin{tabular}{|c|c|c|c|c|c|c|c|c|c|c|c|c|c|c|c|c|c|c|c|c|c|c|}
\hline
\multicolumn{4}{|l|}{Brudnopis} &  &  & - & - &  &  & - & - &  & - &  & - & - &  & - & - &  &  &  \\
\hline
 &  &  &  &  &  &  &  &  &  &  &  &  &  &  &  &  &  &  &  &  &  &  \\
\hline
 &  &  &  &  &  &  &  &  &  &  &  &  &  &  &  &  &  &  &  &  &  &  \\
\hline
 &  &  &  &  &  &  &  &  &  &  &  &  &  &  &  &  &  &  &  &  &  &  \\
\hline
 &  &  &  &  &  &  &  &  &  &  &  &  &  &  &  &  &  &  &  &  &  &  \\
\hline
 &  &  &  &  &  &  &  &  &  &  &  &  &  &  &  &  &  &  &  &  &  &  \\
\hline
 &  &  &  &  &  &  &  &  &  &  &  &  &  &  &  &  &  &  &  &  &  &  \\
\hline
 &  &  &  &  &  &  &  &  &  &  &  &  &  &  &  &  &  &  &  &  &  &  \\
\hline
 &  &  &  &  &  &  &  &  &  &  &  &  &  &  &  &  &  &  &  &  &  &  \\
\hline
 &  &  &  &  &  &  &  &  &  &  &  &  &  &  &  &  &  &  &  &  &  &  \\
\hline
 &  &  &  &  &  &  &  &  &  &  &  &  &  &  &  &  &  &  &  &  &  &  \\
\hline
\end{tabular}
\end{center}

\section*{Zadanie 32. (0-1) 다}
Na loterii stosunek liczby losów wygrywających do liczby losów przegrywających jest równy 2:7. Zakupiono jeden los z tej loterii.

Dokończ zdanie. Wybierz właściwą odpowiedź spośród podanych.\\
Prawdopodobieństwo zdarzenia polegającego na tym, że zakupiony los jest wygrywający, jest równe\\
A. \(\frac{1}{9}\)\\
B. \(\frac{1}{2}\)\\
C. \(\frac{2}{9}\)\\
D. \(\frac{2}{7}\)\\
\includegraphics[max width=\textwidth, center]{2025_02_09_82ae64ed0062449be9d0g-29}

\section*{Zadanie 33. (0-2)}
W eksperymencie badano kiełkowanie nasion w pięciu donicach. Na koniec eksperymentu policzono wykiełkowane nasiona w każdej z donic:

\begin{itemize}
  \item w I donicy - 133 nasiona
  \item w II donicy - 140 nasion
  \item w III donicy - 119 nasion
  \item w IV donicy - 147 nasion
  \item w V donicy - 161 nasion.
\end{itemize}

Odchylenie standardowe liczby wykiełkowanych nasion jest równe \(\sigma=14\).

Podaj numery donic, w których liczba wykiełkowanych nasion mieści się w przedziale określonym przez jedno odchylenie standardowe od średniej.

Zapisz obliczenia.\\
\includegraphics[max width=\textwidth, center]{2025_02_09_82ae64ed0062449be9d0g-30}

BRUDNOPIS (nie podlega ocenie)\\
\includegraphics[max width=\textwidth, center]{2025_02_09_82ae64ed0062449be9d0g-31}\\
\includegraphics[max width=\textwidth, center]{2025_02_09_82ae64ed0062449be9d0g-32}

\section*{MATEMATYKA}
\section*{Poziom podstawowy}
Formuła 2023

\section*{MATEMATYKA}
\section*{Poziom podstawowy}
Formuła 2023

\section*{MATEMATYKA}
\section*{Poziom podstawowy}
Formuła 2023


\end{document}