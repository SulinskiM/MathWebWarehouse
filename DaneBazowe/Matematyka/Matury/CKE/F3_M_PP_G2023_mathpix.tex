% This LaTeX document needs to be compiled with XeLaTeX.
\documentclass[10pt]{article}
\usepackage[utf8]{inputenc}
\usepackage{amsmath}
\usepackage{amsfonts}
\usepackage{amssymb}
\usepackage[version=4]{mhchem}
\usepackage{stmaryrd}
\usepackage{graphicx}
\usepackage[export]{adjustbox}
\graphicspath{ {./images/} }
\usepackage{multirow}
\usepackage[fallback]{xeCJK}
\usepackage{polyglossia}
\usepackage{fontspec}
\IfFontExistsTF{Noto Serif CJK SC}
{\setCJKmainfont{Noto Serif CJK SC}}
{\IfFontExistsTF{STSong}
  {\setCJKmainfont{STSong}}
  {\IfFontExistsTF{Droid Sans Fallback}
    {\setCJKmainfont{Droid Sans Fallback}}
    {\setCJKmainfont{SimSun}}
}}
\IfFontExistsTF{Noto Serif CJK KR}
{\setCJKfallbackfamilyfont{\CJKrmdefault}{Noto Serif CJK KR}}
{\IfFontExistsTF{Apple SD Gothic Neo}
  {\setCJKfallbackfamilyfont{\CJKrmdefault}{Apple SD Gothic Neo}}
  {\IfFontExistsTF{UnDotum}
    {\setCJKfallbackfamilyfont{\CJKrmdefault}{UnDotum}}
    {\setCJKfallbackfamilyfont{\CJKrmdefault}{Malgun Gothic}}
}}

\setmainlanguage{polish}
\IfFontExistsTF{CMU Serif}
{\setmainfont{CMU Serif}}
{\IfFontExistsTF{DejaVu Sans}
  {\setmainfont{DejaVu Sans}}
  {\setmainfont{Georgia}}
}

\author{Liczba punktów do uzyskania: 46}
\date{}


\begin{document}
\maketitle
CENTRALNA\\
KOMISJA\\
EGZAMINACYJNA

\section*{Miejsce na naklejkę.}
 Sprawdż, czy kod na naklejce to M-100.Jeżeli tak - przyklej naklejke. Jeżeli nie - zgłoś to nauczycielowi.

\section*{Egzamin maturalny}
Formuła 2023

\section*{MATEMATYKA}
\section*{Poziom podstawowy}
\section*{TEST DIAGNOSTYCZNY}
Symbol arkusza\\
MMAP-P0-100-2312

\section*{DATA: 7 grudnia 2023 r.}
 Godzina rozpoczecia: 9:00 CZAS trwania: \(\mathbf{1 8 0}\) minut\begin{verbatim}
WYPEŁNIA ZESPÓŁ NADZORUJACY
Uprawnienia zdającego do:
        dostosowania zasad oceniania
        dostosowania w zw. z dyskalkulia
        nieprzenoszenia zaznaczeń na kartę.
\end{verbatim}



\section*{Przed rozpoczęciem pracy z arkuszem egzaminacyjnym}
\begin{enumerate}
  \item Sprawdź, czy nauczyciel przekazał Ci właściwy arkusz egzaminacyjny, tj. arkusz we właściwej formule, z właściwego przedmiotu na właściwym poziomie.
  \item Jeżeli przekazano Ci niewłaściwy arkusz - natychmiast zgłoś to nauczycielowi. Nie rozrywaj banderol.
  \item Jeżeli przekazano Ci właściwy arkusz - rozerwij banderole po otrzymaniu takiego polecenia od nauczyciela. Zapoznaj się z instrukcją na stronie 2.\\
\includegraphics[max width=\textwidth, center]{2025_02_09_7ac42b6ab7ebca0d497ag-02}
\end{enumerate}

\section*{Instrukcja dla zdającego}
\begin{enumerate}
  \item Sprawdź, czy arkusz egzaminacyjny zawiera 33 strony (zadania \(1-30\) ). Ewentualny brak zgłoś przewodniczącemu zespołu nadzorującego egzamin.
  \item Na pierwszej stronie arkusza oraz na karcie odpowiedzi wpisz swój numer PESEL i przyklej naklejkę z kodem.
  \item Symbol zamkniętego musisz przenieść na kartę odpowiedzi. Ocenie podlegają wyłącznie odpowiedzi zaznaczone na karcie odpowiedzi.
  \item Odpowiedzi do zadań zamkniętych zaznacz na karcie odpowiedzi w częşci karty przeznaczonej dla zdającego. Zamaluj pola do tego przeznaczone. Błędne zaznaczenie otocz kółkiem ( i zaznacz właściwe.
  \item Pamiętaj, że pominięcie argumentacji lub istotnych obliczeń w rozwiązaniu zadania otwartego może spowodować, że za to rozwiązanie nie otrzymasz pełnej liczby punktów.
  \item Rozwiązania zadań i odpowiedzi wpisuj w miejscu na to przeznaczonym.
  \item Pisz czytelnie i używaj tylko długopisu lub pióra z czarnym tuszem lub atramentem.
  \item Nie używaj korektora, a błędne zapisy wyraźnie przekreśl.
  \item Nie wpisuj żadnych znaków w tabelkach przeznaczonych dla egzaminatora. Tabelki umieszczone są na marginesie przy odpowiednich zadaniach.
  \item Pamiętaj, że zapisy w brudnopisie nie będą oceniane.
  \item Możesz korzystać z Wybranych wzorów matematycznych, cyrkla i linijki oraz kalkulatora prostego. Upewnij się, czy przekazano Ci broszurę z okładką taką jak widoczna poniżej.\\
\includegraphics[max width=\textwidth, center]{2025_02_09_7ac42b6ab7ebca0d497ag-02(1)}
\end{enumerate}

\section*{Zadania egzaminacyjne są wydrukowane na następnych stronach.}
\section*{Zadanie 1. (0-1)}
Dokończ zdanie. Wybierz właściwą odpowiedź spośród podanych.\\
Liczba \(\left(3^{-2,4} \cdot 3^{\frac{2}{5}}\right)^{\frac{1}{2}}\) jest równa\\
A. \(\sqrt{3}\)\\
B. \(\frac{\sqrt{3}}{3}\)\\
C. \(\frac{1}{3}\)\\
D. 0,3

\begin{center}
\begin{tabular}{|c|c|c|c|c|c|c|c|c|c|c|c|c|c|c|c|c|c|c|c|c|c|c|}
\hline
 & Brudn & opis &  &  &  &  &  &  &  &  &  &  &  &  &  &  &  &  &  &  &  &  \\
\hline
 &  &  &  &  &  &  &  &  &  &  &  &  &  &  &  &  &  &  &  &  &  &  \\
\hline
 &  &  &  &  &  &  &  &  &  &  &  &  &  &  &  &  &  &  &  &  &  &  \\
\hline
 &  &  &  &  &  &  &  &  &  &  &  &  &  &  &  &  &  &  &  &  &  &  \\
\hline
 &  &  &  &  &  &  &  &  &  &  &  &  &  &  &  &  &  &  &  &  &  &  \\
\hline
 &  &  &  &  &  &  &  &  &  &  &  &  &  &  &  &  &  &  &  &  &  &  \\
\hline
 &  &  &  &  &  &  &  &  &  &  &  &  &  &  &  &  &  &  &  &  &  &  \\
\hline
 &  &  &  &  &  &  &  &  &  &  &  &  &  &  &  &  &  &  &  &  &  &  \\
\hline
 &  &  &  &  &  &  &  &  &  &  &  &  &  &  &  &  &  &  &  &  &  &  \\
\hline
 &  &  &  &  &  &  &  &  &  &  &  &  &  &  &  &  &  &  &  &  &  &  \\
\hline
 &  &  &  &  &  &  &  &  &  &  &  &  &  &  &  &  &  &  &  &  &  &  \\
\hline
 &  &  &  &  &  &  &  &  &  &  &  &  &  &  &  &  &  &  &  &  &  &  \\
\hline
 &  &  &  &  &  &  &  &  &  &  &  &  &  &  &  &  &  &  &  &  &  &  \\
\hline
 &  &  &  &  &  &  &  &  &  &  &  &  &  &  &  &  &  &  &  &  &  &  \\
\hline
\end{tabular}
\end{center}

\section*{Zadanie 2. (0-1) 밈}
Dokończ zdanie. Wybierz właściwą odpowiedź spośród podanych.\\
Liczba \(\log _{2} 96-\log _{2} 3\) jest równa\\
A. \(\log _{2} 93\)\\
B. \(\log _{2} 30\)\\
C. 4\\
D. 5

\begin{center}
\begin{tabular}{|c|c|c|c|c|c|c|c|c|c|c|c|c|c|c|c|c|c|c|c|c|c|c|c|c|c|c|c|c|c|c|}
\hline
\multicolumn{5}{|l|}{Brudnopis} &  &  &  &  &  &  &  &  &  &  &  &  &  &  &  &  &  &  &  &  &  &  &  &  &  &  \\
\hline
 &  &  &  &  &  &  &  &  &  &  &  &  &  &  &  &  &  &  &  &  &  &  &  &  &  &  &  &  &  &  \\
\hline
 &  &  &  &  &  &  &  &  &  &  &  &  &  &  &  &  &  &  &  &  &  &  &  &  &  &  &  &  &  &  \\
\hline
 &  &  &  &  &  &  &  &  &  &  &  &  &  &  &  &  &  &  &  &  &  &  &  &  &  &  &  &  &  &  \\
\hline
 &  &  &  &  &  &  &  &  &  &  &  &  &  &  &  &  &  &  &  &  &  &  &  &  &  &  &  &  &  &  \\
\hline
 &  &  &  &  &  &  &  &  &  &  &  &  &  &  &  &  &  &  &  &  &  &  &  &  &  &  &  &  &  &  \\
\hline
 &  &  &  &  &  &  &  &  &  &  &  &  &  &  &  &  &  &  &  &  &  &  &  &  &  &  &  &  &  &  \\
\hline
 &  &  &  &  &  &  &  &  &  &  &  &  &  &  &  &  &  &  &  &  &  &  &  &  &  &  &  &  &  &  \\
\hline
 &  &  &  &  &  &  &  &  &  &  &  &  &  &  &  &  &  &  &  &  &  &  &  &  &  &  &  &  &  &  \\
\hline
 &  &  &  &  &  &  &  &  &  &  &  &  &  &  &  &  &  &  &  &  &  &  &  &  &  &  &  &  &  &  \\
\hline
 &  &  &  &  &  &  &  &  &  &  &  &  &  &  &  &  &  &  &  &  &  &  &  &  &  &  &  &  &  &  \\
\hline
 &  &  &  &  &  &  &  &  &  &  &  &  &  &  &  &  &  &  &  &  &  &  &  &  &  &  &  &  &  &  \\
\hline
 &  &  &  &  &  &  &  &  &  &  &  &  &  &  &  &  &  &  &  &  &  &  &  &  &  &  &  &  &  &  \\
\hline
 &  &  &  &  &  &  &  &  &  &  &  &  &  &  &  &  &  &  &  &  &  &  &  &  &  &  &  &  &  &  \\
\hline
\end{tabular}
\end{center}

\section*{Zadanie 3. (0-1) 띰}
Pan Grzegorz wpłacił do banku pewną kwotę na lokatę dwuletnią. Po każdym rocznym okresie oszczędzania bank doliczał odsetki w wysokości 5\% od kwoty bieżącego kapitału znajdującego się na lokacie. Po dwóch latach oszczędzania pan Grzegorz odebrał z tego banku wraz z odsetkami kwotę 4851 zł (bez uwzględnienia podatków).

Dokończ zdanie. Wybierz właściwą odpowiedź spośród podanych.\\
Kwota wpłacona przez pana Grzegorza na tę lokatę była równa\\
A. 4300 zt\\
B. 4400 zt\\
C. 4500 zt\\
D. \(4600 \mathrm{zł}\)

\begin{center}
\begin{tabular}{|c|c|c|c|c|c|c|c|c|c|c|c|c|c|c|c|c|c|c|c|c|c|c|c|c|}
\hline
\multicolumn{5}{|l|}{Brudnopis} &  &  &  &  &  &  &  &  &  &  &  &  &  &  &  &  &  &  &  &  \\
\hline
 &  &  &  &  &  &  &  &  &  &  &  &  &  &  &  &  &  &  &  &  &  &  &  &  \\
\hline
 &  &  &  &  &  &  &  &  &  &  &  &  &  &  &  &  &  &  &  &  &  &  &  &  \\
\hline
 &  &  &  &  &  &  &  &  &  &  &  &  &  &  &  &  &  &  &  &  &  &  &  &  \\
\hline
 &  &  &  &  &  &  &  &  &  &  &  &  &  &  &  &  &  &  &  &  &  &  &  &  \\
\hline
 &  &  &  &  &  &  &  &  &  &  &  &  &  &  &  &  &  &  &  &  &  &  &  &  \\
\hline
 &  &  &  &  &  &  &  &  &  &  &  &  &  &  &  &  &  &  &  &  &  &  &  &  \\
\hline
 &  &  &  &  &  &  &  &  &  &  &  &  &  &  &  &  &  &  &  &  &  &  &  &  \\
\hline
 &  &  &  &  &  &  &  &  &  &  &  &  &  &  &  &  &  &  &  &  &  &  &  &  \\
\hline
 &  &  &  &  &  &  &  &  &  &  &  &  &  &  &  &  &  &  &  &  &  &  &  &  \\
\hline
\end{tabular}
\end{center}

\section*{Zadanie 4. (0-1)}
Na osi liczbowej zaznaczono przedział.\\
\includegraphics[max width=\textwidth, center]{2025_02_09_7ac42b6ab7ebca0d497ag-05}

Dokończ zdanie. Wybierz właściwą odpowiedź spośród podanych.\\
Zbiór zaznaczony na osi jest zbiorem wszystkich rozwiązań nierówności\\
A. \(|x-2|<5\)\\
B. \(|x-2|>5\)\\
C. \(|x-5|<2\)\\
D. \(|x-5|>2\)\\
\includegraphics[max width=\textwidth, center]{2025_02_09_7ac42b6ab7ebca0d497ag-05(1)}\\
\(5 . \quad\) Zadanie 5. (0-2)\\
Wykaż, że dla każdej liczby całkowitej nieparzystej \(n\) liczba \(3 n^{2}+4 n+1\) jest podzielna przez 4.\\
\includegraphics[max width=\textwidth, center]{2025_02_09_7ac42b6ab7ebca0d497ag-06}

\section*{Zadanie 6.(0-1)回}
Dany jest układ równań \(\left\{\begin{array}{l}x-3 y+5=0 \\ 2 x+y+3=0\end{array}\right.\)

Dokończ zdanie.Wybierz właściwą odpowiedź spośród podanych.

Rozwiązaniem tego układu równań jest para liczb\\
A.\(x=1\) i \(y=2\)\\
B.\(x=0\) i \(y=-3\)\\
C.\(x=-2\) i \(y=1\)\\
D.\(x=-1\) i \(y=-1\)\\
\includegraphics[max width=\textwidth, center]{2025_02_09_7ac42b6ab7ebca0d497ag-07(1)}

\section*{Zadanie 7.(0-1)}
Dokończ zdanie.Wybierz właściwą odpowiedź spośród podanych.

Dla każdej liczby rzeczywistej \(x\) różnej od(-3)i(-2)wartość wyrażenia \(\frac{x+3}{x^{2}+4 x+4} \cdot \frac{x^{2}+2 x}{2 x+6}\) jest równa wartości wyrażenia\\
A.\(\frac{x}{2}\)\\
B.\(\frac{x}{4}\)\\
C.\(\frac{x}{2 x+4}\)\\
D.\(\frac{x^{3}+3 x^{2}}{6 x^{2}+24 x+24}\)\\
\includegraphics[max width=\textwidth, center]{2025_02_09_7ac42b6ab7ebca0d497ag-07}

Zadanie 8. (0-1) 回\\
Dany jest wielomian \(W(x)=-3 x^{3}-x^{2}+k x+1\), gdzie \(k\) jest pewną liczbą rzeczywistą.\\
Wiadomo, że wielomian \(W\) można zapisać w postaci \(W(x)=(x+1) \cdot Q(x)\) dla pewnego wielomianu \(Q\).

Dokończ zdanie. Wybierz właściwą odpowiedź spośród podanych.\\
Liczba \(k\) jest równa\\
A. 29\\
B. \((-3)\)\\
C. 0\\
D. 3

\begin{center}
\begin{tabular}{|c|c|c|c|c|c|c|c|c|c|c|c|c|c|c|c|c|c|c|c|c|c|c|c|}
\hline
\multicolumn{4}{|l|}{Brudnopis} &  &  &  &  &  &  &  &  &  &  &  &  &  &  &  &  &  &  &  &  \\
\hline
 &  &  &  &  &  &  &  &  &  &  &  &  &  &  &  &  &  &  &  &  &  &  &  \\
\hline
 &  &  &  &  &  &  &  &  &  &  &  &  &  &  &  &  &  &  &  &  &  &  &  \\
\hline
 &  &  &  &  &  &  &  &  &  &  &  &  &  &  &  &  &  &  &  &  &  &  &  \\
\hline
 &  &  &  &  &  &  &  &  &  &  &  &  &  &  &  &  &  &  &  &  &  &  &  \\
\hline
 &  &  &  &  &  &  &  &  &  &  &  &  &  &  &  &  &  &  &  &  &  &  &  \\
\hline
 &  &  &  &  &  &  &  &  &  &  &  &  &  &  &  &  &  &  &  &  &  &  &  \\
\hline
 &  &  &  &  &  &  &  &  &  &  &  &  &  &  &  &  &  &  &  &  &  &  &  \\
\hline
 &  &  &  &  &  &  &  &  &  &  &  &  &  &  &  &  &  &  &  &  &  &  &  \\
\hline
 &  &  &  &  &  &  &  &  &  &  &  &  &  &  &  &  &  &  &  &  &  &  &  \\
\hline
 &  &  &  &  &  &  &  &  &  &  &  &  &  &  &  &  &  &  &  &  &  &  &  \\
\hline
 &  &  &  &  &  &  &  &  &  &  &  &  &  &  &  &  &  &  &  &  &  &  &  \\
\hline
 &  &  &  &  &  &  &  &  &  &  &  &  &  &  &  &  &  &  &  &  &  &  &  \\
\hline
\end{tabular}
\end{center}

Zadanie 9. (0-3)\\
Rozwiąż równanie

\[
2 x^{3}+3 x^{2}=10 x+15
\]

\section*{Zapisz obliczenia.}
\begin{center}
\begin{tabular}{|c|c|c|c|c|c|c|c|c|c|c|c|c|c|c|c|c|c|c|c|c|c|c|c|c|c|c|c|c|c|c|}
\hline
 &  &  &  &  &  &  &  &  &  &  &  &  &  &  &  &  &  &  &  &  &  &  &  &  &  &  &  &  &  &  \\
\hline
 &  &  &  &  &  &  &  &  &  &  &  &  &  &  &  &  &  &  &  &  &  &  &  &  &  &  &  &  &  &  \\
\hline
 &  &  &  &  &  &  &  &  &  &  &  &  &  &  &  &  &  &  &  &  &  &  &  &  &  &  &  &  &  &  \\
\hline
 &  &  &  &  &  &  &  &  &  &  &  &  &  &  &  &  &  &  &  &  &  &  &  &  &  &  &  &  &  &  \\
\hline
 &  &  &  &  &  &  &  &  &  &  &  &  &  &  &  &  &  &  &  &  &  &  &  &  &  &  &  &  &  &  \\
\hline
 &  &  &  &  &  &  &  &  &  &  &  &  &  &  &  &  &  &  &  &  &  &  &  &  &  &  &  &  &  &  \\
\hline
 &  &  &  &  &  &  &  &  &  &  &  &  &  &  &  &  &  &  &  &  &  &  &  &  &  &  &  &  &  &  \\
\hline
 &  &  &  &  &  &  &  &  &  &  &  &  &  &  &  &  &  &  &  &  &  &  &  &  &  &  &  &  &  &  \\
\hline
 &  &  &  &  &  &  &  &  &  &  &  &  &  &  &  &  &  &  &  &  &  &  &  &  &  &  &  &  &  &  \\
\hline
 &  &  &  &  &  &  &  &  &  &  &  &  &  &  &  &  &  &  &  &  &  &  &  &  &  &  &  &  &  &  \\
\hline
 &  &  &  &  &  &  &  &  &  &  &  &  &  &  &  &  &  &  &  &  &  &  &  &  &  &  &  &  &  &  \\
\hline
 &  &  &  &  &  &  &  &  &  &  &  &  &  &  &  &  &  &  &  &  &  &  &  &  &  &  &  &  &  &  \\
\hline
 &  &  &  &  &  &  &  &  &  &  &  &  & - &  &  &  &  &  &  &  &  &  &  &  &  &  &  &  &  &  \\
\hline
 &  &  &  &  &  &  &  &  &  &  &  &  &  &  &  &  &  &  &  &  &  &  &  &  &  &  &  &  &  &  \\
\hline
 &  &  &  &  &  &  &  &  &  &  &  &  &  &  &  &  &  &  &  &  &  &  &  &  &  &  &  &  &  &  \\
\hline
 &  &  &  &  &  &  &  &  &  &  &  &  &  &  &  &  &  &  &  &  &  &  &  &  &  &  &  &  &  &  \\
\hline
\end{tabular}
\end{center}

\begin{center}
\includegraphics[max width=\textwidth]{2025_02_09_7ac42b6ab7ebca0d497ag-09}
\end{center}

Zadanie 10. (0-1) 띰\\
Funkcja liniowa \(f\) jest określona wzorem \(f(x)=-\frac{1}{6} x+\frac{2}{3}\).\\
Oceń prawdziwość poniższych stwierdzeń. Wybierz \(P\), jeśli stwierdzenie jest prawdziwe, albo F - jeśli jest fałszywe.

\begin{center}
\begin{tabular}{|l|c|c|}
\hline
Miejscem zerowym funkcji \(f\) jest liczba 4. & \(\mathbf{P}\) & \(\mathbf{F}\) \\
\hline
Punkt przecięcia wykresu funkcji \(f\) z osią Oy ma współrzędne \(\left(0,-\frac{1}{6}\right)\). & \(\mathbf{P}\) & \(\mathbf{F}\) \\
\hline
\end{tabular}
\end{center}

\begin{center}
\begin{tabular}{|c|c|c|c|c|c|c|c|c|c|c|c|c|c|c|c|c|c|c|c|c|c|c|}
\hline
 & Brudn & nopis &  &  &  &  &  &  &  &  &  &  &  &  &  &  &  &  &  &  &  &  \\
\hline
 &  &  &  &  &  &  &  &  &  &  &  &  &  &  &  &  &  &  &  &  &  &  \\
\hline
 &  &  &  &  &  &  &  &  &  &  &  &  &  &  &  &  &  &  &  &  &  &  \\
\hline
 &  &  &  &  &  &  &  &  &  &  &  &  &  &  &  &  &  &  &  &  &  &  \\
\hline
 &  &  &  &  &  &  &  &  &  &  &  &  &  &  &  &  &  &  &  &  &  &  \\
\hline
 &  &  &  &  &  &  &  &  &  &  &  &  &  &  &  &  &  &  &  &  &  &  \\
\hline
 &  &  &  &  &  &  &  &  &  &  &  &  &  &  &  &  &  &  &  &  &  &  \\
\hline
 &  &  &  &  &  &  &  &  &  &  &  &  &  &  &  &  &  &  &  &  &  &  \\
\hline
 &  &  &  &  &  &  &  &  &  &  &  &  &  &  &  &  &  &  &  &  &  &  \\
\hline
 &  &  &  &  &  &  &  &  &  &  &  &  &  &  &  &  &  &  &  &  &  &  \\
\hline
 &  &  &  &  &  &  &  &  &  &  &  &  &  &  &  &  &  &  &  &  &  &  \\
\hline
 &  &  &  &  &  &  &  &  &  &  &  &  &  &  &  &  &  &  &  &  &  &  \\
\hline
 &  &  &  &  &  &  &  &  &  &  &  &  &  &  &  &  &  &  &  &  &  &  \\
\hline
 &  &  &  &  &  &  &  &  &  &  &  &  &  &  &  &  &  &  &  &  &  &  \\
\hline
 &  &  &  &  &  &  &  &  &  &  &  &  &  &  &  &  &  &  &  &  &  &  \\
\hline
 &  &  &  &  &  &  &  &  &  &  &  &  &  &  &  &  &  &  &  &  &  &  \\
\hline
 &  &  &  &  &  &  &  &  &  &  &  &  &  &  &  &  &  &  &  &  &  &  \\
\hline
 &  &  &  &  &  &  &  &  &  &  &  &  &  &  &  &  &  &  &  &  &  &  \\
\hline
 &  &  &  &  &  &  &  &  &  &  &  &  &  &  &  &  &  &  &  &  &  &  \\
\hline
 &  &  &  &  &  &  &  &  &  &  &  &  &  &  &  &  &  &  &  &  &  &  \\
\hline
 &  &  &  &  &  &  &  &  &  &  &  &  &  &  &  &  &  &  &  &  &  &  \\
\hline
 &  &  &  &  &  &  &  &  &  &  &  &  &  &  &  &  &  &  &  &  &  &  \\
\hline
 &  &  &  &  &  &  &  &  &  &  &  &  &  &  &  &  &  &  &  &  &  &  \\
\hline
 &  &  &  &  &  &  &  &  &  &  &  &  &  &  &  &  &  &  &  &  &  &  \\
\hline
 &  &  &  &  &  &  &  &  &  &  &  &  &  &  &  &  &  &  &  &  &  &  \\
\hline
 &  &  &  &  &  &  &  &  &  &  &  &  &  &  &  &  &  &  &  &  &  &  \\
\hline
 &  &  &  &  &  &  &  &  &  &  &  &  &  &  &  &  &  &  &  &  &  &  \\
\hline
 &  &  &  &  &  &  &  &  &  &  &  &  &  &  &  &  &  &  &  &  &  &  \\
\hline
 &  &  &  &  &  &  &  &  &  &  &  &  &  &  &  &  &  &  &  &  &  &  \\
\hline
 &  &  &  &  &  &  &  &  &  &  &  &  &  &  &  &  &  &  &  &  &  &  \\
\hline
 &  &  &  &  &  &  &  &  &  &  &  &  &  &  &  &  &  &  &  &  &  &  \\
\hline
 &  &  &  &  &  &  &  &  &  &  &  &  &  &  &  &  &  &  &  &  &  &  \\
\hline
 &  &  &  &  &  &  &  &  &  &  &  &  &  &  &  &  &  &  &  &  &  &  \\
\hline
 &  &  &  &  &  &  &  &  &  &  &  &  &  &  &  &  &  &  &  &  &  &  \\
\hline
 &  &  &  &  &  &  &  &  &  &  &  &  &  &  &  &  &  &  &  &  &  &  \\
\hline
 &  &  &  &  &  &  &  &  &  &  &  &  &  &  &  &  &  &  &  &  &  &  \\
\hline
\end{tabular}
\end{center}

\section*{Zadanie 11.}
W kartezjańskim układzie współrzędnych ( \(x, y\) ) przedstawiono fragment wykresu funkcji kwadratowej \(f\) (zobacz rysunek). Wierzchołek paraboli, która jest wykresem funkcji \(f\), oraz punkty przecięcia paraboli z osiami układu współrzędnych mają współrzędne całkowite.\\
\includegraphics[max width=\textwidth, center]{2025_02_09_7ac42b6ab7ebca0d497ag-11}

\section*{Zadanie 11.1. (0-1) 띰}
Dokończ zdanie. Wybierz właściwą odpowiedź spośród podanych.\\
Zbiorem wartości funkcji \(f\) jest przedział\\
A. \((-\infty,-2]\)\\
B. \((-\infty, 4]\)\\
C. \([-2,+\infty)\)\\
D. \([4,+\infty)\)\\
\includegraphics[max width=\textwidth, center]{2025_02_09_7ac42b6ab7ebca0d497ag-11(1)}

Zadanie 11.2. (0-1)\\
Zapisz poniżej w postaci przedziału zbiór wszystkich argumentów, dla których funkcja \(f\) przyjmuje wartości ujemne.

\begin{center}
\begin{tabular}{|c|c|c|c|c|c|c|c|c|c|c|c|c|c|c|c|c|c|c|c|c|c|c|c|c|c|}
\hline
 & Brudn & nopis &  &  &  &  &  &  &  &  &  &  &  &  &  &  &  &  &  &  &  &  &  &  &  \\
\hline
 &  &  &  &  &  &  &  &  &  &  &  &  &  &  &  &  &  &  &  &  &  &  &  &  &  \\
\hline
 &  &  &  &  &  &  &  &  &  &  &  &  &  &  &  &  &  &  &  &  &  &  &  &  &  \\
\hline
 &  &  &  &  &  &  &  &  &  &  &  &  &  &  &  &  &  &  &  &  &  &  &  &  &  \\
\hline
 &  &  &  &  &  &  &  &  &  &  &  &  &  &  &  &  &  &  &  &  &  &  &  &  &  \\
\hline
 &  &  &  &  &  &  &  &  &  &  &  &  &  &  &  &  &  &  &  &  &  &  &  &  &  \\
\hline
 &  &  &  &  &  &  &  &  &  &  &  &  &  &  &  &  &  &  &  &  &  &  &  &  &  \\
\hline
\end{tabular}
\end{center}

Zadanie 11.3. (0-2)\\
Uzupełnij zdanie. Wybierz dwie właściwe odpowiedzi spośród oznaczonych literami A-F i wpisz te litery w wykropkowanych miejscach.

Wzór funkcji \(f\) można przedstawić w postaci: \(\qquad\) oraz\\
A. \(f(x)=\frac{1}{2}(x-2)(x-6)\)\\
B. \(f(x)=\frac{1}{2}(x-4)^{2}-2\)\\
C. \(f(x)=2(x-2)(x-6)\)\\
D. \(f(x)=\frac{1}{2}(x+4)^{2}-2\)\\
E. \(f(x)=2(x+2)(x+6)\)\\
F. \(f(x)=2(x+4)^{2}-2\)

\begin{center}
\begin{tabular}{|c|c|c|c|c|c|c|c|c|c|c|c|c|c|c|c|c|c|c|c|c|c|c|c|}
\hline
\multicolumn{4}{|l|}{Brudnopis} &  &  &  &  &  &  &  &  &  &  &  &  &  &  &  &  &  &  &  &  \\
\hline
 &  &  &  &  &  &  &  &  &  &  &  &  &  &  &  &  &  &  &  &  &  &  &  \\
\hline
 &  &  &  &  &  &  &  &  &  &  &  &  &  &  &  &  &  &  &  &  &  &  &  \\
\hline
 &  &  &  &  &  &  &  &  &  &  &  &  &  &  &  &  &  &  &  &  &  &  &  \\
\hline
 &  &  &  &  &  &  &  &  &  &  &  &  &  &  &  &  &  &  &  &  &  &  &  \\
\hline
 &  &  &  &  &  &  &  &  &  &  &  &  &  &  &  &  &  &  &  &  &  &  &  \\
\hline
 &  &  &  &  &  &  &  &  &  &  &  &  &  &  &  &  &  &  &  &  &  &  &  \\
\hline
 &  &  &  &  &  &  &  &  &  &  &  &  &  &  &  &  &  &  &  &  &  &  &  \\
\hline
 &  &  &  &  &  &  &  &  &  &  &  &  &  &  &  &  &  &  &  &  &  &  &  \\
\hline
 &  &  &  &  &  &  &  &  &  &  &  &  &  &  &  &  &  &  &  &  &  &  &  \\
\hline
 &  &  &  &  &  &  &  &  &  &  &  &  &  &  &  &  &  &  &  &  &  &  &  \\
\hline
 &  &  &  &  &  &  &  &  &  &  &  &  &  &  &  &  &  &  &  &  &  &  &  \\
\hline
 &  &  &  &  &  &  &  &  &  &  &  &  &  &  &  &  &  &  &  &  &  &  &  \\
\hline
 &  &  &  &  &  &  &  &  &  &  &  &  &  &  &  &  &  &  &  &  &  &  &  \\
\hline
 &  &  &  &  &  &  &  &  &  &  &  &  &  &  &  &  &  &  &  &  &  &  &  \\
\hline
 &  &  &  &  &  &  &  &  &  &  &  &  &  &  &  &  &  &  &  &  &  &  &  \\
\hline
 &  &  &  &  &  &  &  &  &  &  &  &  &  &  &  &  &  &  &  &  &  &  &  \\
\hline
 &  &  &  &  &  &  &  &  &  &  &  &  &  &  &  &  &  &  &  &  &  &  &  \\
\hline
 &  &  &  &  &  &  &  &  &  &  &  &  &  &  &  &  &  &  &  &  &  &  &  \\
\hline
 &  &  &  &  &  &  &  &  &  &  &  &  &  &  &  &  &  &  &  &  &  &  &  \\
\hline
 &  &  &  &  &  &  &  &  &  &  &  &  &  &  &  &  &  &  &  &  &  &  &  \\
\hline
\end{tabular}
\end{center}

Zadanie 11.4. (0-1) \(\square \square \square\)\\
Funkcja kwadratowa \(g\) jest określona za pomocą funkcji \(f\) (zobacz rysunek na stronie 11) następująco: \(g(x)=f(x+1)\). Na jednym z rysunków A-D przedstawiono, w kartezjańskim układzie współrzędnych \((x, y)\), fragment wykresu funkcji \(y=g(x)\).

Dokończ zdanie. Wybierz właściwą odpowiedź spośród podanych.

Fragment wykresu funkcji \(y=g(x)\) przedstawiono na rysunku\\
A.\\
\includegraphics[max width=\textwidth, center]{2025_02_09_7ac42b6ab7ebca0d497ag-13(2)}\\
B.\\
\includegraphics[max width=\textwidth, center]{2025_02_09_7ac42b6ab7ebca0d497ag-13}\\
C.\\
\includegraphics[max width=\textwidth, center]{2025_02_09_7ac42b6ab7ebca0d497ag-13(3)}\\
D.\\
\includegraphics[max width=\textwidth, center]{2025_02_09_7ac42b6ab7ebca0d497ag-13(1)}

\begin{center}
\begin{tabular}{|c|c|c|c|c|c|c|c|c|c|c|c|c|c|c|c|c|c|c|c|c|c|c|c|c|c|}
\hline
\multicolumn{5}{|l|}{Brudnopis} &  &  &  &  &  &  &  &  &  &  &  &  &  &  &  &  &  &  &  &  &  \\
\hline
 &  &  &  &  &  &  &  &  &  &  &  &  &  &  &  &  &  &  &  &  &  &  &  &  &  \\
\hline
 &  &  &  &  &  &  &  &  &  &  &  &  &  &  &  &  &  &  &  &  &  &  &  &  &  \\
\hline
 &  &  &  &  &  &  &  &  &  &  &  &  &  &  &  &  &  &  &  &  &  &  &  &  &  \\
\hline
 &  &  &  &  &  &  &  &  &  &  &  &  &  &  &  &  &  &  &  &  &  &  &  &  &  \\
\hline
 &  &  &  &  &  &  &  &  &  &  &  &  &  &  &  &  &  &  &  &  &  &  &  &  &  \\
\hline
 &  &  &  &  &  &  &  &  &  &  &  &  &  &  &  &  &  &  &  &  &  &  &  &  &  \\
\hline
 &  &  &  &  &  &  &  &  &  &  &  &  &  &  &  &  &  &  &  &  &  &  &  &  &  \\
\hline
 &  &  &  &  &  &  &  &  &  &  &  &  &  &  &  &  &  &  &  &  &  &  &  &  &  \\
\hline
\end{tabular}
\end{center}

\section*{Zadanie 12. (0-1)}
Proces stygnięcia naparu z ziół w otoczeniu o stałej temperaturze \(22^{\circ} \mathrm{C}\) opisuje funkcja wykładnicza \(T(x)=78 \cdot 2^{-0,05 x}+22\), gdzie \(T(x)\) to temperatura naparu wyrażona w stopniach Celsjusza ( \({ }^{\circ} \mathrm{C}\) ) po \(x\) minutach liczonych od momentu \(x=0\), w którym zioła zalano wrzątkiem.

Dokończ zdanie. Wybierz właściwą odpowiedź spośród podanych.

Temperatura naparu po 20 minutach od momentu zalania ziół wrzątkiem jest równa\\
A. \(22^{\circ} \mathrm{C}\)\\
B. \(39^{\circ} \mathrm{C}\)\\
c. \(78{ }^{\circ} \mathrm{C}\)\\
D. \(61{ }^{\circ} \mathrm{C}\)

\begin{center}
\begin{tabular}{|c|c|c|c|c|c|c|c|c|c|c|c|c|c|c|c|c|c|c|c|c|c|}
\hline
\multicolumn{5}{|l|}{Brudnopis} &  &  &  &  &  &  &  &  &  &  & - &  &  &  &  &  &  \\
\hline
 &  &  &  &  &  &  &  &  &  &  &  &  &  &  &  &  &  &  &  &  &  \\
\hline
 &  &  &  &  &  &  &  &  &  &  &  &  &  &  &  &  &  &  &  &  &  \\
\hline
 &  &  &  &  &  &  &  &  &  &  &  &  &  &  &  &  &  &  &  &  &  \\
\hline
 &  &  &  &  &  &  &  &  &  &  &  &  &  &  &  &  &  &  &  &  &  \\
\hline
 &  &  &  &  &  &  &  &  &  &  &  &  &  &  &  &  &  &  &  &  &  \\
\hline
 &  &  &  &  &  &  &  &  &  &  &  &  &  &  &  &  &  &  &  &  &  \\
\hline
 &  &  &  &  &  &  &  &  &  &  &  &  &  &  &  &  &  &  &  &  &  \\
\hline
 &  &  &  &  &  &  &  &  &  &  &  &  &  &  &  &  &  &  &  &  &  \\
\hline
 &  &  &  &  &  &  &  &  &  &  &  &  &  &  &  &  &  &  &  &  &  \\
\hline
 &  &  &  &  &  &  &  &  &  &  &  &  &  &  &  &  &  &  &  &  &  \\
\hline
\end{tabular}
\end{center}

\section*{Zadanie 13. (0-1) 맘}
Ciąg arytmetyczny ( \(a_{n}\) ) jest określony dla każdej liczby naturalnej \(n \geq 1\). W tym ciągu \(a_{2}=4\) oraz \(a_{3}=9\).

Dokończ zdanie. Wybierz właściwą odpowiedź spośród podanych.\\
Szósty wyraz ciągu \(\left(a_{n}\right)\) jest równy\\
A. 24\\
B. 29\\
C. 36\\
D. 69\\
\includegraphics[max width=\textwidth, center]{2025_02_09_7ac42b6ab7ebca0d497ag-14}

\section*{Zadanie 14.(0-1)丁口冋}
Ciąg \(\left(a_{n}\right)\) jest określony dla każdej liczby naturalnej \(n \geq 1\) .Suma \(n\) począttkowych wyrazów tego ciągu jest określona wzorem \(S_{n}=4 \cdot\left(2^{n}-1\right)\) dla każdej liczby naturalnej \(n \geq 1\) .

Oceń prawdziwość poniższych stwierdzeń.Wybierz P,jeśli stwierdzenie jest prawdziwe,albo F-jeśli jest fałszywe.

\begin{center}
\begin{tabular}{|l|c|c|}
\hline
Pierwszy wyraz ciągu \(\left(a_{n}\right)\) jest równy 4. & \(\mathbf{P}\) & \(\mathbf{F}\) \\
\hline
Drugi wyraz ciągu \(\left(a_{n}\right)\) jest równy 12. & \(\mathbf{P}\) & \(\mathbf{F}\) \\
\hline
\end{tabular}
\end{center}

\begin{center}
\begin{tabular}{|c|c|c|c|c|c|c|c|c|c|c|c|c|c|c|c|c|c|c|c|c|c|c|}
\hline
\multicolumn{4}{|l|}{Brudnopis} &  &  &  &  &  &  &  &  &  &  &  &  &  &  &  &  &  &  &  \\
\hline
 &  &  &  &  &  &  &  &  &  &  &  &  &  &  &  &  &  &  &  &  &  &  \\
\hline
 &  &  &  &  &  &  &  &  &  &  &  &  &  &  &  &  &  &  &  &  &  &  \\
\hline
 &  &  &  &  &  &  &  &  &  &  &  &  &  &  &  &  &  &  &  &  &  &  \\
\hline
 &  &  &  &  &  &  &  &  &  &  &  &  &  &  &  &  &  &  &  &  &  &  \\
\hline
 &  &  &  &  &  &  &  &  &  &  &  &  &  &  &  &  &  &  &  &  &  &  \\
\hline
 &  &  &  &  &  &  &  &  &  &  &  &  &  &  &  &  &  &  &  &  &  &  \\
\hline
 &  &  &  &  &  &  &  &  &  &  &  &  &  &  &  &  &  &  &  &  &  &  \\
\hline
 &  &  &  &  &  &  &  &  &  &  &  &  &  &  &  &  &  &  &  &  &  &  \\
\hline
 &  &  &  &  &  &  &  &  &  &  &  &  &  &  &  &  &  &  &  &  &  &  \\
\hline
 &  &  &  &  &  &  &  &  &  &  &  &  &  &  &  &  &  &  &  &  &  &  \\
\hline
\end{tabular}
\end{center}

\section*{Zadanie 15.(0-1)임}
Trzywyrazowy ciąg( \(1-2 a, 12,48\) )jest geometryczny.

Dokończ zdanie.Wybierz właściwą odpowiedź spośród podanych.\\
Liczba a jest równa\\
A.\((-1)\)\\
B. 3\\
C. 4\\
D. 12,5\\
\includegraphics[max width=\textwidth, center]{2025_02_09_7ac42b6ab7ebca0d497ag-15}

\section*{Zadanie 16. (0-2)}
Dane są dwa kąty o miarach \(\alpha\) oraz \(\beta\), spełniające warunki:\\
\(\alpha \in\left(0^{\circ}, 180^{\circ}\right)\) i \(\operatorname{tg} \alpha=-\frac{2}{3}\) oraz \(\beta \in\left(0^{\circ}, 180^{\circ}\right)\) i \(\cos \beta=\frac{1}{\sqrt{10}}\).\\
Na rysunkach A-F w kartezjańskim układzie współrzędnych ( \(x, y\) ) zaznaczono różne kąty - w tym kąt o mierze \(\alpha\) oraz kąt o mierze \(\beta\). Jedno \(z\) ramion każdego \(z\) tych kątów pokrywa się z dodatnią półosią \(O x\), a drugie przechodzi przez jeden z punktów o współrzędnych całkowitych: \(A\) lub \(B\), lub \(C\), lub \(D\), lub \(E\), lub \(F\).

Uzupełnij tabelę. Wpisz w każdą pustą komórkę tabeli właściwą odpowiedź, wybraną spośród oznaczonych literami A-F.

\begin{center}
\begin{tabular}{|l|l|l|}
\hline
16.1. & Kąt \(\alpha\) jest zaznaczony na rysunku &  \\
\hline
16.2. & Kąt \(\beta\) jest zaznaczony na rysunku &  \\
\hline
\end{tabular}
\end{center}

A.\\
\includegraphics[max width=\textwidth, center]{2025_02_09_7ac42b6ab7ebca0d497ag-16}\\
B.\\
\includegraphics[max width=\textwidth, center]{2025_02_09_7ac42b6ab7ebca0d497ag-16(4)}\\
C.\\
\includegraphics[max width=\textwidth, center]{2025_02_09_7ac42b6ab7ebca0d497ag-16(2)}\\
D.\\
\includegraphics[max width=\textwidth, center]{2025_02_09_7ac42b6ab7ebca0d497ag-16(1)}\\
E.\\
\includegraphics[max width=\textwidth, center]{2025_02_09_7ac42b6ab7ebca0d497ag-16(3)}\\
F.\\
\includegraphics[max width=\textwidth, center]{2025_02_09_7ac42b6ab7ebca0d497ag-16(5)}

\begin{center}
\begin{tabular}{|c|c|c|c|c|c|c|c|c|c|c|c|c|c|c|c|c|c|c|c|c|c|c|c|c|c|}
\hline
 & Brudn & nopi &  &  &  &  &  &  &  &  &  &  &  &  &  &  &  &  &  &  &  &  &  &  &  \\
\hline
 &  &  &  &  &  &  &  &  &  &  &  &  &  &  &  &  &  &  &  &  &  &  &  &  &  \\
\hline
 &  &  &  &  &  &  &  &  &  &  &  &  &  &  &  &  &  &  &  &  &  &  &  &  &  \\
\hline
 &  &  &  &  &  &  &  &  &  &  &  &  &  &  &  &  &  &  &  &  &  &  &  &  &  \\
\hline
 &  &  &  &  &  &  &  &  &  &  &  &  &  &  &  &  &  &  &  &  &  &  &  &  &  \\
\hline
 &  &  &  &  &  &  &  &  &  &  &  &  &  &  &  &  &  &  &  &  &  &  &  &  &  \\
\hline
 &  &  &  &  &  &  &  &  &  &  &  &  &  &  &  &  &  &  &  &  &  &  &  &  &  \\
\hline
 &  &  &  &  &  &  &  &  &  &  &  &  &  &  &  &  &  &  &  &  &  &  &  &  &  \\
\hline
 &  &  &  &  &  &  &  &  &  &  &  &  &  &  &  &  &  &  &  &  &  &  &  &  &  \\
\hline
 &  &  &  &  &  &  &  &  &  &  &  &  &  &  &  &  &  &  &  &  &  &  &  &  &  \\
\hline
 &  &  &  &  &  &  &  &  &  &  &  &  &  &  &  &  &  &  &  &  &  &  &  &  &  \\
\hline
 &  &  &  &  &  &  &  &  &  &  &  &  &  &  &  &  &  &  &  &  &  &  &  &  &  \\
\hline
 &  &  &  &  &  &  &  &  &  &  &  &  &  &  &  &  &  &  &  &  &  &  &  &  &  \\
\hline
 &  &  &  &  &  &  &  &  &  &  &  &  &  &  &  &  &  &  &  &  &  &  &  &  &  \\
\hline
 &  &  &  &  &  &  &  &  &  &  &  &  &  &  &  &  &  &  &  &  &  &  &  &  &  \\
\hline
\end{tabular}
\end{center}

\section*{Zadanie 17. (0-1) 回}
Kąt \(\alpha\) jest ostry oraz \(\sin \alpha=\frac{\sqrt{5}}{3}\).

Dokończ zdanie. Wybierz właściwą odpowiedź spośród podanych.

Tangens kąta \(\alpha\) jest równy\\
A. \(\frac{\sqrt{5}}{2}\)\\
B. \(\frac{2}{3}\)\\
C. \(\frac{2 \sqrt{5}}{5}\)\\
D. \(\frac{3 \sqrt{5}}{5}\)

\begin{center}
\begin{tabular}{|c|c|c|c|c|c|c|c|c|c|c|c|c|c|c|c|c|c|c|c|c|c|c|c|c|c|c|}
\hline
 & \(d n\) & nopi &  &  &  &  &  &  &  &  &  &  &  &  &  &  &  &  &  &  &  &  &  &  &  &  \\
\hline
 &  &  &  &  &  &  &  &  &  &  &  &  &  &  &  &  &  &  &  &  &  &  &  &  &  &  \\
\hline
 &  &  &  &  &  &  &  &  &  &  &  &  &  &  &  &  &  &  &  &  &  &  &  &  &  &  \\
\hline
 &  &  &  &  &  &  &  &  &  &  &  &  &  &  &  &  &  &  &  &  &  &  &  &  &  &  \\
\hline
 &  &  &  &  &  &  &  &  &  &  &  &  &  &  &  &  &  &  &  &  &  &  &  &  &  &  \\
\hline
 &  &  &  &  &  &  &  &  &  &  &  &  &  &  &  &  &  &  &  &  &  &  &  &  &  &  \\
\hline
 &  &  &  &  &  &  &  &  &  &  &  &  &  &  &  &  &  &  &  &  &  &  &  &  &  &  \\
\hline
 &  &  &  &  &  &  &  &  &  &  &  &  &  &  &  &  &  &  &  &  &  &  &  &  &  &  \\
\hline
 &  &  &  &  &  &  &  &  &  &  &  &  &  &  &  &  &  &  &  &  &  &  &  &  &  &  \\
\hline
 &  &  &  &  &  &  &  &  &  &  &  &  &  &  &  &  &  &  &  &  &  &  &  &  &  &  \\
\hline
 &  &  &  &  &  &  &  &  &  &  &  &  &  &  &  &  &  &  &  &  &  &  &  &  &  &  \\
\hline
 &  &  &  &  &  &  &  &  &  &  &  &  &  &  &  &  &  &  &  &  &  &  &  &  &  &  \\
\hline
 &  &  &  &  &  &  &  &  &  &  &  &  &  &  &  &  &  &  &  &  &  &  &  &  &  &  \\
\hline
 &  &  &  &  &  &  &  &  &  &  &  &  &  &  &  &  &  &  &  &  &  &  &  &  &  &  \\
\hline
 &  &  &  &  &  &  &  &  &  &  &  &  &  &  &  &  &  &  &  &  &  &  &  &  &  &  \\
\hline
 &  &  &  &  &  &  &  &  &  &  &  &  &  &  &  &  &  &  &  &  &  &  &  &  &  &  \\
\hline
 &  &  &  &  &  &  &  &  &  &  &  &  &  &  &  &  &  &  &  &  &  &  &  &  &  &  \\
\hline
 &  &  &  &  &  &  &  &  &  &  &  &  &  &  &  &  &  &  &  &  &  &  &  &  &  &  \\
\hline
 &  &  &  &  &  &  &  &  &  &  &  &  &  &  &  &  &  &  &  &  &  &  &  &  &  &  \\
\hline
 &  &  &  &  &  &  &  &  &  &  &  &  &  &  &  &  &  &  &  &  &  &  &  &  &  &  \\
\hline
 &  &  &  &  &  &  &  &  &  &  &  &  &  &  &  &  &  &  &  &  &  &  &  &  &  &  \\
\hline
\end{tabular}
\end{center}

\section*{Zadanie 18. (0-1) 떰}
W kartezjańskim układzie współrzędnych \((x, y)\) dana jest prosta \(l\) o równaniu \(y=\frac{3}{2} x-\frac{15}{2}\). Prosta \(k\) jest prostopadła do prostej \(l\) i przechodzi przez punkt \(P=(6,0)\).

Dokończ zdanie. Wybierz właściwą odpowiedź spośród podanych.\\
Prosta \(k\) ma równanie\\
A. \(y=\frac{3}{2} x+6\)\\
B. \(y=-\frac{2}{3} x+6\)\\
C. \(y=\frac{3}{2} x-9\)\\
D. \(y=-\frac{2}{3} x+4\)

\begin{center}
\begin{tabular}{|c|c|c|c|c|c|c|c|c|c|c|c|c|c|c|c|c|c|c|c|c|c|c|}
\hline
\multicolumn{4}{|l|}{Brudnopis} &  &  &  & - &  &  & - &  &  &  &  &  &  &  &  & - &  &  &  \\
\hline
 &  &  &  &  &  &  &  &  &  &  &  &  &  &  &  &  &  &  &  &  &  &  \\
\hline
 &  &  &  &  &  &  &  &  &  &  &  &  &  &  &  &  &  &  &  &  &  &  \\
\hline
 &  &  &  &  &  &  &  &  &  &  &  &  &  &  &  &  &  &  &  &  &  &  \\
\hline
 &  &  &  &  &  &  &  &  &  &  &  &  &  &  &  &  &  &  &  &  &  &  \\
\hline
 &  &  &  &  &  &  &  &  &  &  &  &  &  &  &  &  &  &  &  &  &  &  \\
\hline
 &  &  &  &  &  &  &  &  &  &  &  &  &  &  &  &  &  &  &  &  &  &  \\
\hline
 &  &  &  &  &  &  &  &  &  &  &  &  &  &  &  &  &  &  &  &  &  &  \\
\hline
 &  &  &  &  &  &  &  &  &  &  &  &  &  &  &  &  &  &  &  &  &  &  \\
\hline
\end{tabular}
\end{center}

\section*{Zadanie 19. (0-1) 四问}
W kartezjańskim układzie współrzędnych \((x, y)\) dane są proste \(k\) oraz \(l\) o równaniach

\[
\begin{aligned}
& k: y=-\frac{1}{2} x-7 \\
& l: y=(2 m-1) x+13
\end{aligned}
\]

\section*{Dokończ zdanie. Wybierz właściwą odpowiedź spośród podanych.}
Proste \(k\) oraz \(l\) są równoległe, gdy\\
A. \(m=\left(-\frac{1}{2}\right)\)\\
B. \(m=\frac{1}{4}\)\\
C. \(m=\frac{3}{2}\)\\
D. \(m=2\)

\begin{center}
\begin{tabular}{|c|c|c|c|c|c|c|c|c|c|c|c|c|c|c|c|c|c|c|c|c|c|c|c|c|c|c|c|}
\hline
\multicolumn{5}{|l|}{Brudnopis} &  &  &  &  &  &  &  &  &  &  &  &  &  &  &  &  &  &  &  &  &  &  &  \\
\hline
 &  &  &  &  &  &  &  &  &  &  &  &  &  &  &  &  &  &  &  &  &  &  &  &  &  &  &  \\
\hline
 &  &  &  &  &  &  &  &  &  &  &  &  &  &  &  &  &  &  &  &  &  &  &  &  &  &  &  \\
\hline
 &  &  &  &  &  &  &  &  &  &  &  &  &  &  &  &  &  &  &  &  &  &  &  &  &  &  &  \\
\hline
 &  &  &  &  &  &  &  &  &  &  &  &  &  &  &  &  &  &  &  &  &  &  &  &  &  &  &  \\
\hline
 &  &  &  &  &  &  &  &  &  &  &  &  &  &  &  &  &  &  &  &  &  &  &  &  &  &  &  \\
\hline
 &  &  &  &  &  &  &  &  &  &  &  &  &  &  &  &  &  &  &  &  &  &  &  &  &  &  &  \\
\hline
 &  &  &  &  &  &  &  &  &  &  &  &  &  &  &  &  &  &  &  &  &  &  &  &  &  &  &  \\
\hline
 &  &  &  &  &  &  &  &  &  &  &  &  &  &  &  &  &  &  &  &  &  &  &  &  &  &  &  \\
\hline
\end{tabular}
\end{center}

\section*{Zadanie 20.(0-1)밈}
W kartezjańskim układzie współrzędnych \((x, y)\) dany jest okrąg \(\mathcal{O}\) o środku w punkcie \(S=(4,-2)\) .Okrąg \(\mathcal{O}\) jest styczny do osi \(O x\) układu współrzędnych.

Dokończ zdanie.Wybierz właściwą odpowiedź spośród podanych.\\
Okrąg \(\mathcal{O}\) jest określony równaniem\\
A.\((x-4)^{2}+(y+2)^{2}=4\)\\
B.\((x-4)^{2}+(y+2)^{2}=2\)\\
C.\((x+4)^{2}+(y-2)^{2}=4\)\\
D.\((x+4)^{2}+(y-2)^{2}=2\)

\begin{center}
\begin{tabular}{|c|c|c|c|c|c|c|c|c|c|c|c|c|c|c|c|c|c|c|c|c|c|}
\hline
\multicolumn{4}{|l|}{Brudnopis} &  &  &  &  &  &  & - &  &  &  &  &  &  &  &  &  &  &  \\
\hline
 &  &  &  &  &  &  &  &  &  &  &  &  &  &  &  &  &  &  &  &  &  \\
\hline
 &  &  &  &  &  &  &  &  &  &  &  &  &  &  &  &  &  &  &  &  &  \\
\hline
 &  &  &  &  &  &  &  &  &  &  &  &  &  &  &  &  &  &  &  &  &  \\
\hline
 &  &  &  &  &  &  &  &  &  &  &  &  &  &  &  &  &  &  &  &  &  \\
\hline
 &  &  &  &  &  &  &  &  &  &  &  &  &  &  &  &  &  &  &  &  &  \\
\hline
 &  &  &  &  &  &  &  &  &  &  &  &  &  &  &  &  &  &  &  &  &  \\
\hline
 &  &  &  &  &  &  &  &  &  &  &  &  &  &  &  &  &  &  &  &  &  \\
\hline
 &  &  &  &  &  &  &  &  &  &  &  &  &  &  &  &  &  &  &  &  &  \\
\hline
 &  &  &  &  &  &  &  &  &  &  &  &  &  &  &  &  &  &  &  &  &  \\
\hline
 &  &  &  &  &  &  &  &  &  &  &  &  &  &  &  &  &  &  &  &  &  \\
\hline
 &  &  &  &  &  &  &  &  &  &  &  &  &  &  &  &  &  &  &  &  &  \\
\hline
\end{tabular}
\end{center}

\section*{Zadanie 21.(0-1)}
W kartezjańskim układzie współrzędnych \((x, y)\) punkty \(K=(-7,-2)\) oraz \(L=(-1,4)\) są wierzchołkami trójkąta równobocznego \(K L M\) .

Dokończ zdanie.Wybierz właściwą odpowiedź spośród podanych.\\
Pole trójkąta \(K L M\) jest równe\\
A. \(17 \sqrt{2}\)\\
B. \(17 \sqrt{3}\)\\
C. \(18 \sqrt{2}\)\\
D. \(18 \sqrt{3}\)\\
\includegraphics[max width=\textwidth, center]{2025_02_09_7ac42b6ab7ebca0d497ag-19}

\section*{Zadanie 22. (0-1) 암}
Punkty \(A, B\) oraz \(C\) leżą na okręgu o środku w punkcie \(O\). Prosta \(k\) jest styczna do tego okręgu w punkcie \(A\) i tworzy z cięciwą \(A B\) kąt o mierze \(32^{\circ}\). Ponadto odcinek \(A C\) jest średnicą tego okręgu (zobacz rysunek).\\
\includegraphics[max width=\textwidth, center]{2025_02_09_7ac42b6ab7ebca0d497ag-20}

Dokończ zdanie. Wybierz właściwą odpowiedź spośród podanych.\\
Miara kąta rozwartego BOC jest równa\\
A. \(148^{\circ}\)\\
B. \(116^{\circ}\)\\
C. \(154^{\circ}\)\\
D. \(122^{\circ}\)\\
\includegraphics[max width=\textwidth, center]{2025_02_09_7ac42b6ab7ebca0d497ag-20(1)}

Zadanie 23. (0-1) \(\square \square\)\\
W rombie \(A B C D\) dłuższa przekątna \(A C\) ma długość 12 itworzy z bokiem \(A B\) kąt o mierze \(30^{\circ}\) (zobacz rysunek).\\
\includegraphics[max width=\textwidth, center]{2025_02_09_7ac42b6ab7ebca0d497ag-21}

Dokończ zdanie. Wybierz właściwą odpowiedź spośród podanych.

Pole rombu \(A B C D\) jest równe\\
A. 24\\
B. 36\\
C. \(24 \sqrt{3}\)\\
D. \(36 \sqrt{2}\)\\
\includegraphics[max width=\textwidth, center]{2025_02_09_7ac42b6ab7ebca0d497ag-21(1)}

\section*{Zadanie 24. (0-2)}
Dany jest okrąg \(\mathcal{O}\) o środku w punkcie \(S\). Średnica \(A B\) tego okręgu przecina cięciwę \(C D\) w punkcie \(P\) (zobacz rysunek). Ponadto: \(|P B|=4,|P C|=8\) oraz \(|P D|=5\).\\
\includegraphics[max width=\textwidth, center]{2025_02_09_7ac42b6ab7ebca0d497ag-22}

Oblicz promień okręgu \(\boldsymbol{O}\). Zapisz obliczenia.\\
\includegraphics[max width=\textwidth, center]{2025_02_09_7ac42b6ab7ebca0d497ag-22(1)}

Zadanie 25. (0-1) 밈\\
Dany jest sześcian \(A B C D E F G H\) o krawędzi długości 5 . Wewnątrz sześcianu znajduje się punkt \(P\) (zobacz rysunek).\\
\includegraphics[max width=\textwidth, center]{2025_02_09_7ac42b6ab7ebca0d497ag-23}

Dokończ zdanie. Wybierz właściwą odpowiedź spośród podanych.\\
Suma odległości punktu \(P\) od wszystkich ścian sześcianu \(\operatorname{ABCDEFGH}\) jest równa\\
A. 15\\
B. 20\\
C. 25\\
D. 30\\
\includegraphics[max width=\textwidth, center]{2025_02_09_7ac42b6ab7ebca0d497ag-23(1)}

Zadanie 26. (0-3)\\
Objętość ostrosłupa prawidłowego czworokątnego jest równa 384. Wysokość ściany bocznej tego ostrosłupa tworzy z płaszczyzną podstawy kąt o mierze \(\alpha\) taki, że \(\operatorname{tg} \alpha=\frac{4}{3}\) (zobacz rysunek).\\
\includegraphics[max width=\textwidth, center]{2025_02_09_7ac42b6ab7ebca0d497ag-24}

Oblicz wysokość ściany bocznej tego ostrosłupa. Zapisz obliczenia.\\
\includegraphics[max width=\textwidth, center]{2025_02_09_7ac42b6ab7ebca0d497ag-24(1)}\\
\includegraphics[max width=\textwidth, center]{2025_02_09_7ac42b6ab7ebca0d497ag-25}

\section*{Zadanie 27. (0-2)}
E-dowód ma zapisany na pierwszej stronie specjalny sześciocyfrowy numer CAN, który zabezpiecza go przed odczytaniem danych przez osoby nieuprawnione.

Oblicz, ile jest wszystkich sześciocyfrowych numerów CAN o różnych cyfrach, spełniających warunek: trzy pierwsze cyfry są kolejnymi wyrazami ciągu arytmetycznego o różnicy (-3). Zapisz obliczenia.

\begin{center}
\begin{tabular}{|c|c|c|c|c|c|c|c|c|c|c|c|c|c|c|c|c|c|c|c|c|c|c|c|c|}
\hline
 &  &  &  &  &  &  &  &  &  &  &  &  &  &  &  &  &  &  &  &  &  &  &  &  \\
\hline
 &  &  &  &  &  &  &  &  &  &  &  &  &  &  &  &  &  &  &  &  &  &  &  &  \\
\hline
 &  &  &  &  &  &  &  &  &  &  &  &  &  &  &  &  &  &  &  &  &  &  &  &  \\
\hline
 &  &  &  &  &  &  &  &  &  &  &  &  &  &  &  &  &  &  &  &  &  &  &  &  \\
\hline
 &  &  &  &  &  &  &  &  &  &  &  &  &  &  &  &  &  &  &  &  &  &  &  &  \\
\hline
 &  &  &  &  &  &  &  &  &  &  &  &  &  &  &  &  &  &  &  &  &  &  &  &  \\
\hline
 &  &  &  &  &  &  &  &  &  &  &  &  &  &  &  &  &  &  &  &  &  &  &  &  \\
\hline
 &  &  &  &  &  &  &  &  &  &  &  &  &  &  &  &  &  &  &  &  &  &  &  &  \\
\hline
 &  &  &  &  &  &  &  &  &  &  &  &  &  &  &  &  &  &  &  &  &  &  &  &  \\
\hline
 &  &  &  &  &  &  &  &  &  &  &  &  &  &  &  &  &  &  &  &  &  &  &  &  \\
\hline
 &  &  &  &  &  &  &  &  &  &  &  &  &  &  &  &  &  &  &  &  &  &  &  &  \\
\hline
 &  &  &  &  &  &  &  &  &  &  &  &  &  &  &  &  &  &  &  &  &  &  &  &  \\
\hline
 &  &  &  &  &  &  &  &  &  &  &  &  &  &  &  &  &  &  &  &  &  &  &  &  \\
\hline
 &  &  &  &  &  &  &  &  &  &  &  &  &  &  &  &  &  &  &  &  &  &  &  &  \\
\hline
 &  &  &  &  &  &  &  &  &  &  &  &  &  &  &  &  &  &  &  &  &  &  &  &  \\
\hline
 &  &  &  &  &  &  &  &  &  &  &  &  &  &  &  &  &  &  &  &  &  &  &  &  \\
\hline
 &  &  &  &  &  &  &  &  &  &  &  &  &  &  &  &  &  &  &  &  &  &  &  &  \\
\hline
 &  &  &  &  &  &  &  &  &  &  &  &  &  &  &  &  &  &  &  &  &  &  &  &  \\
\hline
 &  &  &  &  &  &  &  &  &  &  &  &  &  &  &  &  &  &  &  &  &  &  &  &  \\
\hline
 &  &  &  &  &  &  &  &  &  &  &  &  &  &  &  &  &  &  &  &  &  &  &  &  \\
\hline
 &  &  &  &  &  &  &  &  &  &  &  &  &  &  &  &  &  &  &  &  &  &  &  &  \\
\hline
 &  &  &  &  &  &  &  &  &  &  &  &  &  &  &  &  &  &  &  &  &  &  &  &  \\
\hline
 &  &  &  &  &  &  &  &  &  &  &  &  &  &  &  &  &  &  &  &  &  &  &  &  \\
\hline
 &  &  &  &  &  &  &  &  &  &  &  &  &  &  &  &  &  &  &  &  &  &  &  &  \\
\hline
 &  &  &  &  &  &  &  &  &  &  &  &  &  &  &  &  &  &  &  &  &  &  &  &  \\
\hline
 &  &  &  &  &  &  &  &  &  &  &  &  &  &  &  &  &  &  &  &  &  &  &  &  \\
\hline
 &  &  &  &  &  &  &  &  &  &  &  &  &  &  &  &  &  &  &  &  &  &  &  &  \\
\hline
 &  &  &  &  &  &  &  &  &  &  &  &  &  &  &  &  &  &  &  &  &  &  &  &  \\
\hline
 &  &  &  &  &  &  &  &  &  &  &  &  &  &  &  &  &  &  &  &  &  &  &  &  \\
\hline
 &  &  &  &  &  &  &  &  &  &  &  &  &  &  &  &  &  &  &  &  &  &  &  &  \\
\hline
 &  &  &  &  &  &  &  &  &  &  &  &  &  &  &  &  &  &  &  &  &  &  &  &  \\
\hline
 &  &  &  &  &  &  &  &  &  &  &  &  &  &  &  &  &  &  &  &  &  &  &  &  \\
\hline
 &  &  &  &  &  &  &  &  &  &  &  &  &  &  &  &  &  &  &  &  &  &  &  &  \\
\hline
 &  &  &  &  &  &  &  &  &  &  &  &  &  &  &  &  &  &  &  &  &  &  &  &  \\
\hline
 &  &  &  &  &  &  &  &  &  &  &  &  &  &  &  &  &  &  &  &  &  &  &  &  \\
\hline
 &  &  &  &  &  &  &  &  &  &  &  &  &  &  &  &  &  &  &  &  &  &  &  &  \\
\hline
 &  &  &  &  &  &  &  &  &  &  &  &  &  &  &  &  &  &  &  &  &  &  &  &  \\
\hline
 &  &  &  &  &  &  &  &  &  &  &  &  &  &  &  &  &  &  &  &  &  &  &  &  \\
\hline
 &  &  &  &  &  &  &  &  &  &  &  &  &  &  &  &  &  &  &  &  &  &  &  &  \\
\hline
 &  &  &  &  &  &  &  &  &  &  &  &  &  &  &  &  &  &  &  &  &  &  &  &  \\
\hline
 &  &  &  &  &  &  &  &  &  &  &  &  &  &  &  &  &  &  &  &  &  &  &  &  \\
\hline
\end{tabular}
\end{center}

Zadanie 28. (0-1) 回\\
Doświadczenie losowe polega na dwukrotnym rzucie symetryczną sześcienną kostką do gry, która na każdej ściance ma inną liczbę oczek - od jednego oczka do sześciu oczek.

Dokończ zdanie. Wybierz właściwą odpowiedź spośród podanych.

Prawdopodobieństwo zdarzenia polegającego na tym, że iloczyn liczb wyrzuconych oczek jest liczbą nieparzystą, jest równe\\
A. \(\frac{1}{2}\)\\
B. \(\frac{1}{5}\)\\
C. \(\frac{1}{4}\)\\
D. \(\frac{3}{4}\)\\
\includegraphics[max width=\textwidth, center]{2025_02_09_7ac42b6ab7ebca0d497ag-27}

\section*{Zadanie 29.}
W hurtowni owoców wyselekcjonowane jabłko spełnia normę jakości, gdy jego masa (po zaokrągleniu do pełnych dekagramów) mieści się w przedziale [19 dag, 21 dag]. Pobrano próbę kontrolną liczącą 50 jabłek i następnie zważono każde z nich. Na poniższym wykresie słupkowym przedstawiono rozkład masy jabłek w badanej próbie. Na osi poziomej podano - wyrażoną w dekagramach - masę jabłka (w zaokrągleniu do pełnych dekagramów), a na osi pionowej przedstawiono liczbę jabłek o określonej masie.\\
\includegraphics[max width=\textwidth, center]{2025_02_09_7ac42b6ab7ebca0d497ag-28(1)}

\section*{Zadanie 29.1. (0-1) 띰}
Spośród 50 zważonych jabłek z pobranej próby kontrolnej losujemy jedno jabłko.

\section*{Dokończ zdanie. Wybierz właściwą odpowiedź spośród podanych.}
Prawdopodobieństwo zdarzenia polegającego na tym, że wylosowane jabłko spełnia normę jakości, jest równe\\
A. \(\frac{3}{7}\)\\
B. \(\frac{5}{7}\)\\
C. \(\frac{18}{25}\)\\
D. \(\frac{9}{10}\)\\
\includegraphics[max width=\textwidth, center]{2025_02_09_7ac42b6ab7ebca0d497ag-28}

Zadanie 29.2. (0-1) 띰\\
Dokończ zdanie tak, aby było prawdziwe. Wybierz odpowiedź A albo B oraz jej uzasadnienie 1., 2. albo 3.

Dominanta masy 50 zważonych jabłek (w zaokrągleniu do pełnych dekagramów) z pobranej próby kontrolnej jest równa

\begin{center}
\begin{tabular}{|c|c|c|c|c|}
\hline
\multirow{2}{*}{A.} & \multirow{2}{*}{20 dag,} & \multirow{3}{*}{ponieważ} & 1. & ta masa jest największa w tej próbie. \\
\hline
 &  &  & 2. & iloczyn tej masy i liczby jabłek o takiej masie jest największy w tej próbie. \\
\hline
B. & 23 dag, &  & 3. & ta masa występuje najliczniej w tej próbie. \\
\hline
\end{tabular}
\end{center}

\begin{center}
\begin{tabular}{|c|c|c|c|c|c|c|c|c|c|c|c|c|c|c|c|c|c|c|c|c|c|c|}
\hline
 & Brudn & dnopi &  &  &  &  &  &  &  &  &  &  &  &  &  &  &  &  &  &  &  &  \\
\hline
 &  &  &  &  &  &  &  &  &  &  &  &  &  &  &  &  &  &  &  &  &  &  \\
\hline
 &  &  &  &  &  &  &  &  &  &  &  &  &  &  &  &  &  &  &  &  &  &  \\
\hline
 &  &  &  &  &  &  &  &  &  &  &  &  &  &  &  &  &  &  &  &  &  &  \\
\hline
 &  &  &  &  &  &  &  &  &  &  &  &  &  &  &  &  &  &  &  &  &  &  \\
\hline
 &  &  &  &  &  &  &  &  &  &  &  &  &  &  &  &  &  &  &  &  &  &  \\
\hline
 &  &  &  &  &  &  &  &  &  &  &  &  &  &  &  &  &  &  &  &  &  &  \\
\hline
 &  &  &  &  &  &  &  &  &  &  &  &  &  &  &  &  &  &  &  &  &  &  \\
\hline
 &  &  &  &  &  &  &  &  &  &  &  &  &  &  &  &  &  &  &  &  &  &  \\
\hline
 &  &  &  &  &  &  &  &  &  &  &  &  &  &  &  &  &  &  &  &  &  &  \\
\hline
 &  &  &  &  &  &  &  &  &  &  &  &  &  &  &  &  &  &  &  &  &  &  \\
\hline
 &  &  &  &  &  &  &  &  &  &  &  &  &  &  &  &  &  &  &  &  &  &  \\
\hline
 &  &  &  &  &  &  &  &  &  &  &  &  &  &  &  &  &  &  &  &  &  &  \\
\hline
 &  &  &  &  &  &  &  &  &  &  &  &  &  &  &  &  &  &  &  &  &  &  \\
\hline
 &  &  &  &  &  &  &  &  &  &  &  &  &  &  &  &  &  &  &  &  &  &  \\
\hline
 &  &  &  &  &  &  &  &  &  &  &  &  &  &  &  &  &  &  &  &  &  &  \\
\hline
 &  &  &  &  &  &  &  &  &  &  &  &  &  &  &  &  &  &  &  &  &  &  \\
\hline
 &  &  &  &  &  &  &  &  &  &  &  &  &  &  &  &  &  &  &  &  &  &  \\
\hline
 &  &  &  &  &  &  &  &  &  &  &  &  &  &  &  &  &  &  &  &  &  &  \\
\hline
 &  &  &  &  &  &  &  &  &  &  &  &  &  &  &  &  &  &  &  &  &  &  \\
\hline
 &  &  &  &  &  &  &  &  &  &  &  &  &  &  &  &  &  &  &  &  &  &  \\
\hline
 &  &  &  &  &  &  &  &  &  &  &  &  &  &  &  &  &  &  &  &  &  &  \\
\hline
 &  &  &  &  &  &  &  &  &  &  &  &  &  &  &  &  &  &  &  &  &  &  \\
\hline
 &  &  &  &  &  &  &  &  &  &  &  &  &  &  &  &  &  &  &  &  &  &  \\
\hline
 &  &  &  &  &  &  &  &  &  &  &  &  &  &  &  &  &  &  &  &  &  &  \\
\hline
 &  &  &  &  &  &  &  &  &  &  &  &  &  &  &  &  &  &  &  &  &  &  \\
\hline
 &  &  &  &  &  &  &  &  &  &  &  &  &  &  &  &  &  &  &  &  &  &  \\
\hline
 &  &  &  &  &  &  &  &  &  &  &  &  &  &  &  &  &  &  &  &  &  &  \\
\hline
 &  &  &  &  &  &  &  &  &  &  &  &  &  &  &  &  &  &  &  &  &  &  \\
\hline
 &  &  &  &  &  &  &  &  &  &  &  &  &  &  &  &  &  &  &  &  &  &  \\
\hline
 &  &  &  &  &  &  &  &  &  &  &  &  &  &  &  &  &  &  &  &  &  &  \\
\hline
 &  &  &  &  &  &  &  &  &  &  &  &  &  &  &  &  &  &  &  &  &  &  \\
\hline
 &  &  &  &  &  &  &  &  &  &  &  &  &  &  &  &  &  &  &  &  &  &  \\
\hline
\end{tabular}
\end{center}

Zadanie 30. (0-4)\\
Zgodnie z założeniem architekta okno na poddaszu ma mieć kształt trapezu równoramiennego, który nie jest równoległobokiem. Dłuższa podstawa trapezu ma mieć długość 12 dm , a suma długości krótszej podstawy i wysokości tego trapezu ma być równa 18 dm .

Oblicz, jaką długość powinna mieć krótsza podstawa tego trapezu, tak aby pole powierzchni okna było największe. Oblicz to pole. Zapisz obliczenia.

\begin{center}
\begin{tabular}{|c|c|c|c|c|c|c|c|c|c|c|c|c|c|c|c|c|c|c|c|c|c|c|}
\hline
 &  &  &  &  &  &  &  &  &  &  &  &  &  &  &  &  &  &  &  &  &  &  \\
\hline
 &  &  &  &  &  &  &  &  &  &  &  &  &  &  &  &  &  &  &  &  &  &  \\
\hline
 &  &  &  &  &  &  &  &  &  &  &  &  &  &  &  &  &  &  &  &  &  &  \\
\hline
 &  &  &  &  &  &  &  &  &  &  &  &  &  &  &  &  &  &  &  &  &  &  \\
\hline
 &  &  &  &  &  &  &  &  &  &  &  &  &  &  &  &  &  &  &  &  &  &  \\
\hline
 &  &  &  &  &  &  &  &  &  &  &  &  &  &  &  &  &  &  &  &  &  &  \\
\hline
 &  &  &  &  &  &  &  &  &  &  &  &  &  &  &  &  &  &  &  &  &  &  \\
\hline
 &  &  &  &  &  &  &  &  &  &  &  &  &  &  &  &  &  &  &  &  &  &  \\
\hline
 &  &  &  &  &  &  &  &  &  &  &  &  &  &  &  &  &  &  &  &  &  &  \\
\hline
 &  &  &  &  &  &  &  &  &  &  &  &  &  &  &  &  &  &  &  &  &  &  \\
\hline
 &  &  &  &  &  &  &  &  &  &  &  &  &  &  &  &  &  &  &  &  &  &  \\
\hline
 &  &  &  &  &  &  &  &  &  &  &  &  &  &  &  &  &  &  &  &  &  &  \\
\hline
 &  &  &  &  &  &  &  &  &  &  &  &  &  &  &  &  &  &  &  &  &  &  \\
\hline
 &  &  &  &  &  &  &  &  &  &  &  &  &  &  &  &  &  &  &  &  &  &  \\
\hline
 &  &  &  &  &  &  &  &  &  &  &  &  &  &  &  &  &  &  &  &  &  &  \\
\hline
 &  &  &  &  &  &  &  &  &  &  &  &  &  &  &  &  &  &  &  &  &  &  \\
\hline
 &  &  &  &  &  &  &  &  &  &  &  &  &  &  &  &  &  &  &  &  &  &  \\
\hline
 &  &  &  &  &  &  &  &  &  &  &  &  &  &  &  &  &  &  &  &  &  &  \\
\hline
 &  &  &  &  &  &  &  &  &  &  &  &  &  &  &  &  &  &  &  &  &  &  \\
\hline
 &  &  &  &  &  &  &  &  &  &  &  &  &  &  &  &  &  &  &  &  &  &  \\
\hline
 &  &  &  &  &  &  &  &  &  &  &  &  &  &  &  &  &  &  &  &  &  &  \\
\hline
 &  &  &  &  &  &  &  &  &  &  &  &  &  &  &  &  &  &  &  &  &  &  \\
\hline
 &  &  &  &  &  &  &  &  &  &  &  &  &  &  &  &  &  &  &  &  &  &  \\
\hline
 &  &  &  &  &  &  &  &  &  &  &  &  &  &  &  &  &  &  &  &  &  &  \\
\hline
 &  &  &  &  &  &  &  &  &  &  &  &  &  &  &  &  &  &  &  &  &  &  \\
\hline
 &  &  &  &  &  &  &  &  &  &  &  &  &  &  &  &  &  &  &  &  &  &  \\
\hline
 &  &  &  &  &  &  &  &  &  &  &  &  &  &  &  &  &  &  &  &  &  &  \\
\hline
 &  &  &  &  &  &  &  &  &  &  &  &  &  &  &  &  &  &  &  &  &  &  \\
\hline
 &  &  &  &  &  &  &  &  &  &  &  &  &  &  &  &  &  &  &  &  &  &  \\
\hline
 &  &  &  &  &  &  &  &  &  &  &  &  &  &  &  &  &  &  &  &  &  &  \\
\hline
 &  &  &  &  &  &  &  &  &  &  &  &  &  &  &  &  &  &  &  &  &  &  \\
\hline
 &  &  &  &  &  &  &  &  &  &  &  &  &  &  &  &  &  &  &  &  &  &  \\
\hline
 &  &  &  &  &  &  &  &  &  &  &  &  &  &  &  &  &  &  &  &  &  &  \\
\hline
 &  &  &  &  &  &  &  &  &  &  &  &  &  &  &  &  &  &  &  &  &  &  \\
\hline
 &  &  &  &  &  &  &  &  &  &  &  &  &  &  &  &  &  &  &  &  &  &  \\
\hline
 &  &  &  &  &  &  &  &  &  &  &  &  &  &  &  &  &  &  &  &  &  &  \\
\hline
 &  &  &  &  &  &  &  &  &  &  &  &  &  &  &  &  &  &  &  &  &  &  \\
\hline
 &  &  &  &  &  &  &  &  &  &  &  &  &  &  &  &  &  &  &  &  &  &  \\
\hline
 &  &  &  &  &  &  &  &  &  &  &  &  &  &  &  &  &  &  &  &  &  &  \\
\hline
\end{tabular}
\end{center}

\begin{center}
\includegraphics[max width=\textwidth]{2025_02_09_7ac42b6ab7ebca0d497ag-31}
\end{center}

BRUDNOPIS (nie podlega ocenie)

\begin{center}
\begin{tabular}{|c|c|c|c|c|c|c|c|c|c|c|c|c|c|c|c|c|c|c|c|c|c|c|c|}
\hline
 &  &  &  &  &  &  &  &  &  &  &  &  &  &  &  &  &  &  &  &  &  &  &  \\
\hline
 &  &  &  &  &  &  &  &  &  &  &  &  &  &  &  &  &  &  &  &  &  &  &  \\
\hline
 &  &  &  &  &  &  &  &  &  &  &  &  &  &  &  &  &  &  &  &  &  &  &  \\
\hline
 &  &  &  &  &  &  &  &  &  &  &  &  &  &  &  &  &  &  &  &  &  &  &  \\
\hline
 &  &  &  &  &  &  &  &  &  &  &  &  &  &  &  &  &  &  &  &  &  &  &  \\
\hline
 &  &  &  &  &  &  &  &  &  &  &  &  &  &  &  &  &  &  &  &  &  &  &  \\
\hline
 &  &  &  &  &  &  &  &  &  &  &  &  &  &  &  &  &  &  &  &  &  &  &  \\
\hline
 &  &  &  &  &  &  &  &  &  &  &  &  &  &  &  &  &  &  &  &  &  &  &  \\
\hline
 &  &  &  &  &  &  &  &  &  &  &  &  &  &  &  &  &  &  &  &  &  &  &  \\
\hline
 &  &  &  &  &  &  &  &  &  &  &  &  &  &  &  &  &  &  &  &  &  &  &  \\
\hline
 &  &  &  &  &  &  &  &  &  &  &  &  &  &  &  &  &  &  &  &  &  &  &  \\
\hline
 &  &  &  &  &  &  &  &  &  &  &  &  &  &  &  &  &  &  &  &  &  &  &  \\
\hline
 &  &  &  &  &  &  &  &  &  &  &  &  &  &  &  &  &  &  &  &  &  &  &  \\
\hline
 &  &  &  &  &  &  &  &  &  &  &  &  &  &  &  &  &  &  &  &  &  &  &  \\
\hline
 &  &  &  &  &  &  &  &  &  &  &  &  &  &  &  &  &  &  &  &  &  &  &  \\
\hline
 &  &  &  &  &  &  &  &  &  &  &  &  &  &  &  &  &  &  &  &  &  &  &  \\
\hline
 &  &  &  &  &  &  &  &  &  &  &  &  &  &  &  &  &  &  &  &  &  &  &  \\
\hline
 &  &  &  &  &  &  &  &  &  &  &  &  &  &  &  &  &  &  &  &  &  &  &  \\
\hline
 &  &  &  &  &  &  &  &  &  &  &  &  &  &  &  &  &  &  &  &  &  &  &  \\
\hline
 &  &  &  &  &  &  &  &  &  &  &  &  &  &  &  &  &  &  &  &  &  &  &  \\
\hline
 &  &  &  &  &  &  &  &  &  &  &  &  &  &  &  &  &  &  &  &  &  &  &  \\
\hline
 &  &  &  &  &  &  &  &  &  &  &  &  &  &  &  &  &  &  &  &  &  &  &  \\
\hline
 &  &  &  &  &  &  &  &  &  &  &  &  &  &  &  &  &  &  &  &  &  &  &  \\
\hline
 &  &  &  &  &  &  &  &  &  &  &  &  &  &  &  &  &  &  &  &  &  &  &  \\
\hline
 &  &  &  &  &  &  &  &  &  &  &  &  &  &  &  &  &  &  &  &  &  &  &  \\
\hline
 &  &  &  &  &  &  &  &  &  &  &  &  &  &  &  &  &  &  &  &  &  &  &  \\
\hline
 &  &  &  &  &  &  &  &  &  &  &  &  &  &  &  &  &  &  &  &  &  &  &  \\
\hline
 &  &  &  &  &  &  &  &  &  &  &  &  &  &  &  &  &  &  &  &  &  &  &  \\
\hline
 &  &  &  &  &  &  &  &  &  &  &  &  &  &  &  &  &  &  &  &  &  &  &  \\
\hline
 &  &  &  &  &  &  &  &  &  &  &  &  &  &  &  &  &  &  &  &  &  &  &  \\
\hline
 &  &  &  &  &  &  &  &  &  &  &  &  &  &  &  &  &  &  &  &  &  &  &  \\
\hline
 &  &  &  &  &  &  &  &  &  &  &  &  &  &  &  &  &  &  &  &  &  &  &  \\
\hline
 &  &  &  &  &  &  &  &  &  &  &  &  &  &  &  &  &  &  &  &  &  &  &  \\
\hline
 &  &  &  &  &  &  &  &  &  &  &  &  &  &  &  &  &  &  &  &  &  &  &  \\
\hline
 &  &  &  &  &  &  &  &  &  &  &  &  &  &  &  &  &  &  &  &  &  &  &  \\
\hline
 &  &  &  &  &  &  &  &  &  &  &  &  &  &  &  &  &  &  &  &  &  &  &  \\
\hline
 &  &  &  &  &  &  &  &  &  &  &  &  &  &  &  &  &  &  &  &  &  &  &  \\
\hline
 &  &  &  &  &  &  &  &  &  &  &  &  &  &  &  &  &  &  &  &  &  &  &  \\
\hline
 &  &  &  &  &  &  &  &  &  &  &  &  &  &  &  &  &  &  &  &  &  &  &  \\
\hline
 &  &  &  &  &  &  &  &  &  &  &  &  &  &  &  &  &  &  &  &  &  &  &  \\
\hline
 &  &  &  &  &  &  &  &  &  &  &  &  &  &  &  &  &  &  &  &  &  &  &  \\
\hline
 &  &  &  &  &  &  &  &  &  &  &  &  &  &  &  &  &  &  &  &  &  &  &  \\
\hline
 &  &  &  &  &  &  &  &  &  &  &  &  &  &  &  &  &  &  &  &  &  &  &  \\
\hline
 &  &  &  &  &  &  &  &  &  &  &  &  &  &  &  &  &  &  &  &  &  &  &  \\
\hline
 &  &  &  &  &  &  &  &  &  &  &  &  &  &  &  &  &  &  &  &  &  &  &  \\
\hline
 & - &  &  &  &  &  &  &  &  &  &  &  &  &  &  &  &  &  &  &  &  &  &  \\
\hline
 &  &  &  &  &  &  &  &  &  &  &  &  &  &  &  &  &  &  &  &  &  &  &  \\
\hline
 &  &  &  &  &  &  &  &  &  &  &  &  &  &  &  &  &  &  &  &  &  &  &  \\
\hline
\end{tabular}
\end{center}

\begin{center}
\includegraphics[max width=\textwidth]{2025_02_09_7ac42b6ab7ebca0d497ag-33}
\end{center}

\section*{MATEMATYKA}
\section*{Poziom podstawowy}
Formuła 2023

\section*{MATEMATYKA}
\section*{Poziom podstawowy}
Formuła 2023

\section*{MATEMATYKA}
\section*{Poziom podstawowy}
Formuła 2023


\end{document}