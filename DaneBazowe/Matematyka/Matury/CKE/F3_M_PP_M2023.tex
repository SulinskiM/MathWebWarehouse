\documentclass[a4paper,12pt]{article}
\usepackage{latexsym}
\usepackage{amsmath}
\usepackage{amssymb}
\usepackage{graphicx}
\usepackage{wrapfig}
\pagestyle{plain}
\usepackage{fancybox}
\usepackage{bm}

\begin{document}

CENTRALNA

KOMISJA

EGZAMINACYJNA

Arkusz zawiera informacje prawnie chronione

do momentu rozpoczecia egzaminu.

KOD

WYPELNIA ZOAJACY

PESEL

{\it Miejsce na naklejke}.

{\it Sprawdz}', {\it czy kod na naklejce to}

M-100.
\begin{center}
\includegraphics[width=21.900mm,height=10.164mm]{./F3_M_PP_M2023_page0_images/image001.eps}

\includegraphics[width=79.656mm,height=10.164mm]{./F3_M_PP_M2023_page0_images/image002.eps}
\end{center}
/{\it ezeli tak}- {\it przyklej naklejkq}.

/{\it ezeli nie}- {\it zgtoś to nauczycielowi}.

Egzamin maturalny

$\displaystyle \int$
\begin{center}
\includegraphics[width=193.344mm,height=75.792mm]{./F3_M_PP_M2023_page0_images/image003.eps}
\end{center}
Poziom  podstawowy

{\it Symbo arkusza}

MMAP-P0-100-2305

DATA: 8 maja 2023 r.

GODZINA R0ZP0CZECIA: 9:00

CZAS TRWANIA: $180 \displaystyle \min$ ut

WYPELNIA ZESPÓL NADZORUJACY

Uprawnienia $\mathrm{z}\mathrm{d}\mathrm{a}\mathrm{j}_{8}$cego do:

\fbox{} dostosowania zasad oceniania

\fbox{} dostosowania w zw. z dyskalkulia

\fbox{} nieprzenoszenia zaznaczeń na karte.

LICZBA PUNKTÓW DO UZYSKANIA 46

Przed rozpoczeciem pracy z arkuszem egzaminacyjnym

1.

Sprawd $\acute{\mathrm{z}}$, czy nauczyciel przekazal Ci wlaściwy arkusz egzaminacyjny,

tj. arkusz we wlaściwej formule, z w[aściwego przedmiotu na wlaściwym

poziomie.

2.

$\mathrm{J}\mathrm{e}\dot{\mathrm{z}}$ eli przekazano Ci niew[aściwy arkusz- natychmiast zgloś to nauczycielowi.

Nie rozrywaj banderol.

3. $\mathrm{J}\mathrm{e}\dot{\mathrm{z}}$ eli przekazano Ci w[aściwy arkusz- rozerwij banderole po otrzymaniu

takiego polecenia od nauczyciela. Zapoznaj $\mathrm{s}\mathrm{i}\mathrm{e}$ z instrukcjq na stronie 2.

$\mathrm{U}\mathrm{k}\}\mathrm{a}\mathrm{d}$ graficzny

\copyright CKE 2022 $\bullet$

$\Vert\Vert\Vert\Vert\Vert\Vert\Vert\Vert\Vert\Vert\Vert\Vert\Vert\Vert\Vert\Vert\Vert\Vert\Vert\Vert\Vert\Vert\Vert\Vert\Vert\Vert\Vert\Vert\Vert\Vert|$




lnstrukcja dla zdajqcego

l. Sprawdz', czy arkusz egzaminacyjny zawiera 31 stron (zadania $1-31$).

Ewentualny brak zgloś przewodniczqcemu zespolu nadzorujqcego egzamin.

2. Na pierwszej stronie arkusza oraz na karcie odpowiedzi wpisz swój numer PESEL

i przyklej naklejke z kodem.

3. Symbol ${}_{1\mathrm{g}}P$ zamieszczony w naglówku zadania oznacza, $\dot{\mathrm{z}}\mathrm{e}$ rozwiqzanie zadania

zamknietego musisz przenieśč na karte odpowiedzi.

4. Odpowiedzi do zadań zamknietych zaznacz na karcie odpowiedzi w cześci karty

przeznaczonej dla zdajqcego. Zamaluj $\blacksquare$ pola do tego przeznaczone. Bledne

zaznaczenie otocz kólkiem \copyright i zaznacz wlaściwe.

5. $\mathrm{P}\mathrm{a}\mathrm{m}\mathrm{i}_{9}\mathrm{t}\mathrm{a}\mathrm{j}, \dot{\mathrm{z}}\mathrm{e}$ pominiecie argumentacji lub istotnych obliczeń w rozwiqzaniu zadania

otwartego $\mathrm{m}\mathrm{o}\dot{\mathrm{z}}\mathrm{e}$ spowodować, $\dot{\mathrm{z}}\mathrm{e}$ za to rozwiazanie nie otrzymasz pelnej liczby punktów.

6. Rozwiqzania zadań i odpowiedzi wpisuj w miejscu na to przeznaczonym.

7. Pisz czytelnie i $\mathrm{u}\dot{\mathrm{z}}$ ywaj tylko dlugopisu lub pióra z czarnym tuszem lub atramentem.

8. Nie $\mathrm{u}\dot{\mathrm{z}}$ ywaj korektora, a bledne zapisy wyra $\acute{\mathrm{z}}$ nie przekreśl.

9. Nie wpisuj $\dot{\mathrm{z}}$ adnych znaków w tabelkach przeznaczonych dla egzaminatora.

Tabelki umieszczone sa na marginesie przy odpowiednich zadaniach.

10. Pamietaj, $\dot{\mathrm{z}}\mathrm{e}$ zapisy w brudnopisie nie $\mathrm{b}9\mathrm{d}\mathrm{a}$ oceniane.

11. $\mathrm{M}\mathrm{o}\dot{\mathrm{z}}$ esz korzystač z Wybranych wzorów matematycznych, cyrkla i linijki oraz kalkulatora

prostego. Upewnij si9, czy przekazano Ci broszure z ok1adka taka jak widoczna ponizej.

Strona 2 z31

$\mathrm{M}\mathrm{M}\mathrm{A}\mathrm{P}-\mathrm{P}0_{-}100$





Zadanie Y\S$*$(0-2)

Danyjest $\mathrm{p}\mathrm{r}\mathrm{o}\mathrm{s}\mathrm{t}\mathrm{o}\mathrm{k}_{\mathrm{c}1}\mathrm{t}$ o bokach dlugości a $\mathrm{i} b$, gdzie $a>b$. Obwód tego $\mathrm{p}\mathrm{r}\mathrm{o}\mathrm{s}\mathrm{t}\mathrm{o}\mathrm{k}_{\mathrm{c}}$]$\mathrm{t}\mathrm{a}$ jest

równy 30. Jeden z boków prostokqta jest o 5 krótszy od drugiego.

Uzupe[nij zdanie. Wybierz dwie w[aściwe odpowiedzi spośród oznaczonych literami

A-F i wpisz te litery w wykropkowanych miejscach.

Zalezności miedzy dlugościami boków tego prostokqta zapisano w ukladach równań

oznaczonych literami:

oraz

A. 

B. 

C. 

D. 

E. 

F. 

{\it Brud}$\underline{no}\underline{\sqrt{}is}_{-} -$

$\mathrm{M}\mathrm{M}\mathrm{A}\mathrm{P}-\mathrm{P}0_{-}100$

Strona ll z31





Zadanie 82.

$\mathrm{W}$ kartezjańskim ukladzie wspólrzednych $(x,y)$ narysowano wykres funkcji $y=f(x)$

(zobacz rysunek).

Zadanie 02.1. $\{0-8\} \beta$

Dokończ zdanie. Wybierz w[aściwq odpowied $\acute{\mathrm{z}}$ spośród podanych.

Dziedzinq funkcji $f$ jest zbiór

A. [-6, 5]

B. $(-6,5)$

C. $(-3,5]$

D. [-3, 5]

{\it Brudnopis}

Zadanie 82.2. (0-\S) ff

Dokończ zdanie. Wybierz w[aściwq odpowied $\acute{\mathrm{z}}$ spośród podanych.

Najwi9ksza wartośč funkcji f w przedzia1e [-4, 1] jest równa

A. 0

B. l

C. 2

D. 5

{\it Brudnopis}

Strona 12 z31

$\mathrm{M}\mathrm{M}\mathrm{A}\mathrm{P}-\mathrm{P}0_{-}100$





Zädanie 12,3. (0-\S)mp

Dokończ zdanie. Wybierz w[aściwq odpowied $\acute{\mathrm{z}}$ spośród podanych.

Funkcja f jest malejaca w zbiorze

A. $[-6,-3)$

B. [-3, 1]

{\it Brudnopis}

C. (1, 2]

D. [2, 5]

$\mathrm{Z}\mathrm{a}\mathrm{d}\mathrm{a}\mathrm{n}\dot{\mathrm{l}}\mathrm{e}13_{*}(0-9) \beta$

Funkcja liniowa $f$ jest określona wzorem

$f(x)=ax+b$, gdzie $a \mathrm{i} b$ sa pewnymi

liczbami rzeczywistymi. Na rysunku obok

przedstawiono fragment wykresu funkcji $f$

w kartezjańskim ukladzie wspólrzednych $(x,y).$
\begin{center}
\includegraphics[width=82.908mm,height=69.540mm]{./F3_M_PP_M2023_page12_images/image001.eps}
\end{center}
{\it y}

1

0 1  $\chi$

$y=f(x)$

Dokończ zdanie. Wybierz w[aściwq odpowied $\acute{\mathrm{z}}$ spośród podanych.

Liczba a oraz liczba b we wzorze funkcji f spelniaja warunki:

A. $a>0 \mathrm{i} b>0.$

B. $a>0 \mathrm{i} b<0.$

C. $a<0 \mathrm{i} b>0.$

D. $a<0 \mathrm{i} b<0.$

{\it Brudnopis}

$\mathrm{M}\mathrm{M}\mathrm{A}\mathrm{P}-\mathrm{P}0_{-}100$

Strona 13 z31





Zädanie 14. \{0-\S)

$\beta$

Jednym z miejsc zerowych funkcji kwadratowej $f$ jest liczba $(-5)$. Pierwsza wspólrz9dna

wierzcholka paraboli, $\mathrm{b}_{9}$dacej wykresem funkcji $f$, jest równa 3.

Dokończ zdanie. Wybierz w[aściwq odpowied $\acute{\mathrm{z}}$ spośród podanych.

Drugim miejscem zerowym funkcji f jest liczba

A. ll

B. l

C. $(-1)$

D. $(-13)$

{\it Brudnopis}

$| 1$

Strona 14 z31

$\mathrm{M}\mathrm{M}\mathrm{A}\mathrm{P}-\mathrm{P}0_{-}100$





Zädanie 95. (0-\S) $\beta$

Ciqg $(a_{n})$ jest określony wzorem $a_{n}=2^{n}$

$(n+1)$ dla $\mathrm{k}\mathrm{a}\dot{\mathrm{z}}$ dej liczby naturalnej $n\geq 1.$

Dokończ zdanie. Wybierz wlaściwq odpowied $\acute{\mathrm{z}}$ spośród podanych.

Wyraz $a_{4}$ jest równy

A. 64

B. 40

C. 48

D. 80

{\it Brudnopis}

$1 -$

Zadanie 86. (0-1) $\beta$

Trzywyrazowy ciag $($27, 9, $a-1)$ jest geometryczny.

Dokończ zdanie. Wybierz w[aściwq odpowied $\acute{\mathrm{z}}$ spośród podanych.

Liczba a jest równa

A. 3

B. 0

{\it Brudnopis}

4

D. 2

$\mathrm{M}\mathrm{M}\mathrm{A}\mathrm{P}-\mathrm{P}0_{-}100$

Strona 15 z31





Zadanie $\mathrm{t}7. (0-2)$

Pan Stanislaw splacil pozyczk9 w wysokości 8910 z1 w osiemnastu ratach. $\mathrm{K}\mathrm{a}\dot{\mathrm{z}}$ da kolejna

rata byla mniejsza od poprzedniej o 30 z1.

Oblicz kwote pierwszei raty. Zapisz obliczenia.

1

1

Strona 16 z31

$\mathrm{M}\mathrm{M}\mathrm{A}\mathrm{P}-\mathrm{P}0_{-}100$





Zädanie 1\S. \{0-\S) $\beta$

$\mathrm{W}$ kartezjańskim ukladzie wspólrzednych $(x,y)$ zaznaczono kqt $\alpha$ o wierzcholku

w punkcie $0=(0,0)$. Jedno z ramion tego kqta pokrywa si9 z dodatniq pó1osiq $0x,$

a drugie przechodzi przez punkt $P=(-3,1)$ (zobacz rysunek).
\begin{center}
\includegraphics[width=78.228mm,height=48.768mm]{./F3_M_PP_M2023_page16_images/image001.eps}
\end{center}
{\it y}

$P=(-3,1)$

$\alpha$

{\it 0}

$-3 -2 -1$

$-1$

1 2  3  $\chi$

Dokończ zdanie. Wybierz wlaściwq odpowied $\acute{\mathrm{z}}$ spośród podanych.

Tangens kqta $\alpha$ jest równy

A. -$\sqrt{}$110

B. $(-\displaystyle \frac{3}{\sqrt{10}})$

C. $(-\displaystyle \frac{3}{1})$

D. $(-\displaystyle \frac{1}{3})$

$Brudno\sqrt{}is$

-

Zadanie 19$*$(0-\S) $p$

Dokończ zdanie. Wybierz w[aściwq odpowied $\acute{\mathrm{z}}$ spośród podanych.

Dla $\mathrm{k}\mathrm{a}\dot{\mathrm{z}}$ dego kqta ostrego $\alpha$ wyrazenie $\sin^{4}\alpha +\sin^{2}\alpha\cdot\cos^{2}\alpha$ jest równe

A. $\sin^{2}\alpha$

B. $\sin^{6}\alpha\cdot\cos^{2}\alpha$

C. $\sin^{4}\alpha+1$

D. $\sin^{2}\alpha\cdot(\sin\alpha+\cos\alpha)\cdot(\sin\alpha-\cos\alpha)$

{\it Brudnopis}

$\mathrm{M}\mathrm{M}\mathrm{A}\mathrm{P}-\mathrm{P}0_{-}100$

Strona 17 z31





Zädanie 20. (0-\S) $\beta$

$\mathrm{W}$ rombie o boku dlugości $6\sqrt{2}$ kqt rozwarty ma miar9 $150^{\mathrm{o}}$

Dokończ zdanie. Wybierz wlaściwq odpowied $\acute{\mathrm{z}}$ spośród podanych.

lloczyn dlugości przekqtnych tego rombu jest równy

A. 24

B. 72

C. 36

D. $36\sqrt{2}$

{\it Brudnopis}

Zadan$\mathrm{e}2l. (0\infty 1)$

H $\beta$

Punkty $A, B, C \mathrm{l}\mathrm{e}\dot{\mathrm{z}}$ a na okr gu o środku w punkcie 0.

$\mathrm{K} \mathrm{t} AC0$ ma miar $70^{\mathrm{o}}$ (zobacz rysunek).
\begin{center}
\includegraphics[width=68.124mm,height=73.104mm]{./F3_M_PP_M2023_page17_images/image001.eps}
\end{center}
{\it B}

{\it 0}

{\it C}

{\it A}

odpowied $\acute{\mathrm{z}}$ spośród podanych.

Dokończ zdanie.

Wybierz w[aściw

Miara kata ostrego ABC jest równa

A. $10^{\mathrm{o}}$

B. $20^{\mathrm{o}}$

C. $35^{\mathrm{o}}$

D. $40^{\mathrm{o}}$

{\it Brudnopis}

Strona 18 z31

$\mathrm{M}\mathrm{M}\mathrm{A}\mathrm{P}-\mathrm{P}0_{-}100$





Zadanie 22. (0-2)

Trójkqty prostokatne $T_{1}$ i $T_{2}$ sq podobne. Przyprostokqtne trójkqta $T_{1}$ maja

dlugości 5 $\mathrm{i} 12$. Przeciwprostokatna trójkata $T_{2}$ ma dlugośč 26.

Oblicz pole tróikqta $T_{2}$. Zapisz obliczenia.

$\overline{1}$

$1-$

$\mathrm{M}\mathrm{M}\mathrm{A}\mathrm{P}-\mathrm{P}0_{-}100$

Strona 19 z31





Zädanie 23. \{0-\S) $\beta$

$\mathrm{W}$ kartezjańskim ukladzie wspólrzednych $(x,y)$ dane sa proste $k$ oraz $l$ o równaniach

{\it k}:

{\it y}$=$ -32 $\chi$

{\it l}:

$y=-\displaystyle \frac{3}{2}x+13$

Dokończ zdanie. Wybierz odpowied $\acute{\mathrm{z}}$ A albo $\mathrm{B}$ oraz $\mathrm{o}\mathrm{d}\mathrm{p}\mathrm{o}\mathrm{w}\mathrm{i}\mathrm{e}\mathrm{d}\acute{\mathrm{z}}1.$, 2. albo 3.

Proste $k$ oraz $l$

A. sq prostopadle

1. $(-6,-4)$

i przecinajq si9 w punkcie P o wspó1rzednych 2.

(6, 4)

B.

nie sq

prostopadle

3. $(-6,4)$

{\it Brud}$\underline{no}\underline{\sqrt{}is}_{-} -$

-

Strona 20 z31

$\mathrm{M}\mathrm{M}\mathrm{A}\mathrm{P}-\mathrm{P}0_{-}100$





Zadania egzaminacyjne sq wydrukowane

na nastepnych stronach.

$\mathrm{M}\mathrm{M}\mathrm{A}\mathrm{P}-\mathrm{P}0_{-}100$

Strona 3 z31





Zädanie 24. \{0-\S) $\beta$

$\mathrm{W}$ kartezjańskim ukladzie wspólrzednych $(x,y)$ dana jest prosta $k$ o równaniu

$y=-\displaystyle \frac{1}{3}x+2$

Dokończ zdanie. Wybierz w[aściwq odpowied $\acute{\mathrm{z}}$ spośród podanych.

Prosta o równaniu $y=ax+b$ jest równolegla do prostej $k$ i przechodzi przez

punkt $P=(3,5)$, gdy

A. $a=3 \mathrm{i} b=4.$

B. $a=-\displaystyle \frac{1}{3} \mathrm{i} b=4.$

C. $a=3 \mathrm{i} b=-4.$

D. $a=-\displaystyle \frac{1}{3} \mathrm{i} b=6.$

{\it Brud}$\underline{no}\underline{\sqrt{}is}_{-} -$

-

Zadanie 25. (0-{\$}) $\mathrm{R} \beta$

Dany jest graniastoslup prawidlowy czworokqtny, w którym krawedz' podstawy ma

dlugośč 15. Przekqtna graniastos1upa jest nachy1ona do p1aszczyzny podstawy pod

kqtem $\alpha$ takim, $\dot{\mathrm{z}}\mathrm{e} \displaystyle \cos\alpha=\frac{\sqrt{2}}{3}$

Dokończ zdanie. Wybierz w[aściwq odpowied $\acute{\mathrm{z}}$ spośród podanych.

Dlugośč przekqtnej tego graniastoslupa jest równa

A. $15\sqrt{2}$

B. 45

C. $5\sqrt{2}$

D. 10

{\it Brudnopis}

$\mathrm{M}\mathrm{M}\mathrm{A}\mathrm{P}-\mathrm{P}0_{-}100$

Strona 21 z31





Zadanie 26. (0-4)

Dany jest ostroslup prawidlowy czworokatny. Wysokośč ściany bocznej tego ostroslupa jest

nachylona do plaszczyzny podstawy pod $\mathrm{k}_{\mathrm{c}}$]$\mathrm{t}\mathrm{e}\mathrm{m} 30^{\mathrm{o}}$ i ma dlugośč równa 6 (zobacz rysunek).

Oblicz objetośč i pole powierzchni calkowitej tego ostros[upa. Zapisz obliczenia.

1

Strona 22 z31

$\mathrm{M}\mathrm{M}\mathrm{A}\mathrm{P}-\mathrm{P}0_{-}100$





1

$\overline{11}-$

-

$0_{-}100$

Strona 23 z31





Zädanie 27, \{0-\S)

$\beta$

$\mathrm{W}$ pewnym ostroslupie prawidlowym stosunek liczby $W$ wszystkich wierzcholków do

liczby $K$ wszystkich krawedzi jest równy $\displaystyle \frac{W}{K}=\frac{3}{5}$

Dokończ zdanie. Wybierz w[aściwq odpowied $\acute{\mathrm{z}}$ spośród podanych.

Podstawq tego ostroslupa jest

A. kwadrat.

B. $\mathrm{p}\mathrm{i}_{9}$ciokqt foremny.

C. sześciokqt foremny.

D. siedmiokqt foremny.

{\it Brudnopis}

Zadanie 2@. $(0\leftrightarrow 1) \mathrm{E} \beta$

Dokończ zdanie. Wybierz w[aściwq odpowied $\acute{\mathrm{z}}$ spośród podanych.

Wszystkich liczb naturalnych pieciocyfrowych, w których zapisie dziesietnym wystepujq tylko

cyfry 0, 5, 7 (np. 57075, 55555), jest

A. $5^{3}$

B. $2\cdot 4^{3}$

C. $2\cdot 3^{4}$

D. $3^{5}$

{\it Brudnopis}

Strona 24 z31

$\mathrm{M}\mathrm{M}\mathrm{A}\mathrm{P}-\mathrm{P}0_{-}100$





Zadanie 29. (0-2)

Na diagramie ponizej przedstawiono ceny pomidorów w szesnastu wybranych sklepach.

6

5

4

liczba

sklepów 3
\begin{center}
\includegraphics[width=154.176mm,height=80.364mm]{./F3_M_PP_M2023_page24_images/image001.eps}
\end{center}
2

1

0

5,05

5,60

5,70

6,00

6,30

cena za l kg pomidorów (w zl)

Uzupe[nij tabele. Wpisz w $\mathrm{k}\mathrm{a}\dot{\mathrm{z}}$ dq pustq komórke tabeli w[aściwq odpowied $\acute{\mathrm{z}}$, wybranq

spośród oznaczonych literami A-E.
\begin{center}
\begin{tabular}{|l|l|l|}
\hline
\multicolumn{1}{|l|}{$29.1.$}&	\multicolumn{1}{|l|}{$\begin{array}{l}\mbox{Mediana ceny kilograma pomidorów w tych wybranych sklepach jest}	\\	\mbox{równa}	\end{array}$}&	\multicolumn{1}{|l|}{}	\\
\hline
\multicolumn{1}{|l|}{ $29.2.$}&	\multicolumn{1}{|l|}{$\begin{array}{l}\mbox{ $\acute{\mathrm{S}}$ rednia cena kilograma pomidorów w tych wybranych sklepach jest}	\\	\mbox{równa}	\end{array}$}&	\multicolumn{1}{|l|}{}	\\
\hline
\end{tabular}

\end{center}
A. 5,80 z1

B. 5,73 z1

C. 5,85 z1

D. 6,00 z1

E. 5,70 z1

{\it Brudnopis}

$\mathrm{M}\mathrm{M}\mathrm{A}\mathrm{P}-\mathrm{P}0_{-}100$

Strona 25 z31





Zadanie $30_{\mathrm{L}}\{0-2$)

Ze zbioru ośmiu liczb \{2, 3, 4, 5, 6, 7, 8, 9\} 1osujemy ze zwracaniem ko1ejno dwa razy po

jednej liczbie.

Oblicz prawdopodobieństwo zdarzenia $A$ polegajqcego na tym, $\dot{\mathrm{z}}\mathrm{e}$ iloczyn

wylosowanych liczb jest podzielny przez 15. Zapisz ob1iczenia.

$-|1$

1

1

$1-$

1

Strona 26 z31

$\mathrm{M}\mathrm{M}\mathrm{A}\mathrm{P}-\mathrm{P}0_{-}100$





Zadanie 38.

Wlaściciel pewnej apteki przeanalizowal dane dotyczqce liczby obslugiwanych klientów

$\mathrm{z} 30$ kolejnych dni. Przyjmijmy, $\dot{\mathrm{z}}\mathrm{e}$ liczbe $L$ obslugiwanych klientów $n$-tego dnia opisuje

funkcja

$L(n)=-n^{2}+22n+279$

gdzie $n$ jest liczbq naturalnq$\mathrm{s}\mathrm{p}\mathrm{e}$niajqcq warunki $n\geq 1 \mathrm{i} n\leq 30.$

Zadanie $38_{\mathrm{r}}\S. (0-\not\in)\mathrm{E} p$

Oceń prawdziwośč ponizszych stwierdzeń. Wybierz $\mathrm{P}$, jeśli stwierdzenie jest

prawdziwe, albo $\mathrm{F}$ -jeśli jest fa[szywe.
\begin{center}
\begin{tabular}{|l|l|l|}
\hline
\multicolumn{1}{|l|}{ $\begin{array}{l}\mbox{Laczna liczba klientów obsluzonych w czasie wszystkich analizowanych dni}	\\	\mbox{jest równa $L(30).$}	\end{array}$}&	\multicolumn{1}{|l|}{P}&	\multicolumn{1}{|l|}{F}	\\
\hline
\multicolumn{1}{|l|}{$\mathrm{W}$ trzecim dniu analizowanego okresu obsluzono 336 klientów.}&	\multicolumn{1}{|l|}{P}&	\multicolumn{1}{|l|}{F}	\\
\hline
\end{tabular}

\end{center}
$B_{\Gamma}udno\sqrt{}is$

Zadanie 38.2. $(0-2J$

Którego dnia analizowanego okresu w aptece obslu $\dot{\mathrm{z}}$ ono najwiekszq liczbe klientów?

Oblicz liczbe klientów obslu $\dot{\mathrm{z}}$ onych tego dnia. Zapisz obliczenia.

$\mathrm{M}\mathrm{M}\mathrm{A}\mathrm{P}-\mathrm{P}0_{-}100$

Strona 27 z31





1

$\overline{11}-$

-

Strona 28 z31

$\mathrm{M}\mathrm{M}\mathrm{A}\mathrm{P}-\mathrm{P}0_{-}10$





BRUDNOPIS (nie podlega ocenie)

1

-PO-100

Strona 29 z31





$| 1$

Strona 30 z31

$\mathrm{M}\mathrm{M}\mathrm{A}\mathrm{P}-\mathrm{P}0_{-}10$





Zädanie 1. (0-t) $\beta$

Na osi liczbowej zaznaczono sum9 przedzia1ów.
\begin{center}
\includegraphics[width=143.964mm,height=11.424mm]{./F3_M_PP_M2023_page3_images/image001.eps}
\end{center}
$-2$  5  $\chi$

Dokończ zdanie. Wybierz w[aściwq odpowied $\acute{\mathrm{z}}$ spośród podanych.

Zbiór zaznaczony na osi jest zbiorem wszystkich rozwiqzań nierówności

A. $|x-3,5|\geq 1,5$

B. $|x-1,5|\geq 3,5$

C. $|x-3,5|\leq 1,5$

D. $|x-1,5|\leq 3,5$

$\underline{Brudno\sqrt{}is}$

$1 -$

Zadanie $2_{\mathrm{Y}}$ (0-\S\} $\bullet \beta$

Dokończ zdanie. Wybierz w[aściwq odpowied $\acute{\mathrm{z}}$ spośród podanych.

Liczba $\sqrt[3]{-\frac{27}{16}}\cdot\sqrt[3]{2}$ jest równa

A. $(-\displaystyle \frac{3}{2})$

B. -23

C. -32

D. $(-\displaystyle \frac{2}{3})$

{\it Brudnopis}

Strona 4 z31

$\mathrm{M}\mathrm{M}\mathrm{A}\mathrm{P}-\mathrm{P}0_{-}100$





$0_{-}100$

$| 1$

Strona 31 z31










Zadanie 3. $(0-2$\}

Wykaz, $\dot{\mathrm{z}}\mathrm{e}$ dla $\mathrm{k}\mathrm{a}\dot{\mathrm{z}}$ dej liczby naturalnej $n\geq 1$ liczba $(2n+1)^{2}-1$ jest podzielna

przez 8.

$\mathrm{M}\mathrm{M}\mathrm{A}\mathrm{P}-\mathrm{P}0_{-}100$

Strona 5 z31





Zädanie 4. (0-\S) $\beta$

Dokończ zdanie. Wybierz wlaściwq odpowied $\acute{\mathrm{z}}$ spośród podanych.

Liczba $\log_{9}27+\log_{9}3$ jest równa

A. 81

B. 9

{\it Brudnopis}

$1-$

4

D. 2

$\mathrm{Z}\mathrm{a}\mathrm{d}\mathrm{a}\mathrm{n}\dot{\mathrm{l}}\mathrm{e}5. (0-9) \beta$

Dokończ zdanie. Wybierz w[aściwq odpowied $\acute{\mathrm{z}}$ spośród podanych.

Dla $\mathrm{k}\mathrm{a}\dot{\mathrm{z}}$ dej liczby rzeczywistej $a$ wyrazenie $(2a-3)^{2}-(2a+3)^{2}$ jest równe

A. $-24a$

B. 0

[‡A][D]18

D. $16a^{2}-24a$

{\it Brudnopis}

Strona 6 z31

$\mathrm{M}\mathrm{M}\mathrm{A}\mathrm{P}-\mathrm{P}0_{-}100$





Zädanie 6. (0-t) $\beta$

Dokończ zdanie. Wybierz wlaściwq odpowied $\acute{\mathrm{z}}$ spośród podanych.

Zbiorem wszystkich rozwiqzań nierówności

$-2(x+3)\displaystyle \leq\frac{2-x}{3}$

jest przedzial

A. $(-\infty,-4]$

B. $(-\infty,4]$

C. $[-4,\infty)$

D. [4, $\infty)$

{\it Brudnopis}

$-||\mathrm{i}1$

1

$| 1$

-

Zadanie 7. (0-t) $\mathrm{w} p$

Dokończ zdanie. Wybierz w[aściwq odpowied $\acute{\mathrm{z}}$ spośród podanych.

Jednym z rozwiqzań równania $\sqrt{3}(x^{2}-2)(x+3)=0$ jest liczba

A. 3

B. 2

C. $\sqrt{3}$

D. $\sqrt{2}$

{\it Brudnopis}

$\mathrm{M}\mathrm{M}\mathrm{A}\mathrm{P}-\mathrm{P}0_{-}100$

Strona 7 z31





Zadanie 8. (0-t) $\beta$

Dokończ zdanie. Wybierz wlaściwq odpowied $\acute{\mathrm{z}}$ spośród podanych.

Równanie $\displaystyle \frac{(x+1)(x-1)^{2}}{(x-1)(x+1)^{2}}=0$ w zbiorze liczb rzeczywistych

A. nie ma rozwiqzania.

B. ma dokladnie jedno rozwiqzanie: $-1.$

C. ma dokladnie jedno rozwiqzanie: l.

D. ma dokladnie dwa rozwiqzania: $-1$ oraz l.

{\it Brudnopis} 

1

1

$1-$

Zadanie $\mathrm{g}. (0-3$\}

Rozwiqz równanie

$3x^{3}-2x^{2}-12x+8=0$

Zapisz obliczenia.

Strona 8 z31

$\mathrm{M}\mathrm{M}\mathrm{A}\mathrm{P}-\mathrm{P}0_{-}10$





1

$\overline{11}-$

-

$0_{-}100$

Strona 9 z31





Zadanie \S 0. (0-\S J $\mathrm{E}\mathrm{H}\mathrm{B}^{\beta}$

Na rysunku przedstawiono interpretacj9 $\displaystyle \mathrm{g}\mathrm{e}\mathrm{o}\mathrm{m}\mathrm{e}\mathrm{t}\mathrm{r}\mathrm{y}\mathrm{c}\mathrm{z}\bigcap_{\mathrm{c}1}$ w kartezjańskim ukladzie

wspólrz9dnych $(x,y)$ jednego z $\mathrm{n}\mathrm{i}\dot{\mathrm{z}}$ ej zapisanych ukladów równań A-D.

Dokończ zdanie. Wybierz w[aściwq odpowied $\acute{\mathrm{z}}$ spośród podanych.

Ukladem równań, którego interpretacje geometrycznq przedstawiono na rysunku, jest

A. 

B. 

C. 

D. 

{\it Brudnopis}

Strona 10 z31

$\mathrm{M}\mathrm{M}\mathrm{A}\mathrm{P}-\mathrm{P}0_{-}100$



\end{document}