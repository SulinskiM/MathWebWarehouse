\documentclass[a4paper,12pt]{article}
\usepackage{latexsym}
\usepackage{amsmath}
\usepackage{amssymb}
\usepackage{graphicx}
\usepackage{wrapfig}
\pagestyle{plain}
\usepackage{fancybox}
\usepackage{bm}

\begin{document}

Zädanie 12,3. (0-\S)mp

Dokończ zdanie. Wybierz w[aściwq odpowied $\acute{\mathrm{z}}$ spośród podanych.

Funkcja f jest malejaca w zbiorze

A. $[-6,-3)$

B. [-3, 1]

{\it Brudnopis}

C. (1, 2]

D. [2, 5]

$\mathrm{Z}\mathrm{a}\mathrm{d}\mathrm{a}\mathrm{n}\dot{\mathrm{l}}\mathrm{e}13_{*}(0-9) \beta$

Funkcja liniowa $f$ jest określona wzorem

$f(x)=ax+b$, gdzie $a \mathrm{i} b$ sa pewnymi

liczbami rzeczywistymi. Na rysunku obok

przedstawiono fragment wykresu funkcji $f$

w kartezjańskim ukladzie wspólrzednych $(x,y).$
\begin{center}
\includegraphics[width=82.908mm,height=69.540mm]{./F3_M_PP_M2023_page12_images/image001.eps}
\end{center}
{\it y}

1

0 1  $\chi$

$y=f(x)$

Dokończ zdanie. Wybierz w[aściwq odpowied $\acute{\mathrm{z}}$ spośród podanych.

Liczba a oraz liczba b we wzorze funkcji f spelniaja warunki:

A. $a>0 \mathrm{i} b>0.$

B. $a>0 \mathrm{i} b<0.$

C. $a<0 \mathrm{i} b>0.$

D. $a<0 \mathrm{i} b<0.$

{\it Brudnopis}

$\mathrm{M}\mathrm{M}\mathrm{A}\mathrm{P}-\mathrm{P}0_{-}100$

Strona 13 z31
\end{document}
