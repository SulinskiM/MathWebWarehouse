\documentclass[a4paper,12pt]{article}
\usepackage{latexsym}
\usepackage{amsmath}
\usepackage{amssymb}
\usepackage{graphicx}
\usepackage{wrapfig}
\pagestyle{plain}
\usepackage{fancybox}
\usepackage{bm}

\begin{document}

Zädanie 1\S. \{0-\S) $\beta$

$\mathrm{W}$ kartezjańskim ukladzie wspólrzednych $(x,y)$ zaznaczono kqt $\alpha$ o wierzcholku

w punkcie $0=(0,0)$. Jedno z ramion tego kqta pokrywa si9 z dodatniq pó1osiq $0x,$

a drugie przechodzi przez punkt $P=(-3,1)$ (zobacz rysunek).
\begin{center}
\includegraphics[width=78.228mm,height=48.768mm]{./F3_M_PP_M2023_page16_images/image001.eps}
\end{center}
{\it y}

$P=(-3,1)$

$\alpha$

{\it 0}

$-3 -2 -1$

$-1$

1 2  3  $\chi$

Dokończ zdanie. Wybierz wlaściwq odpowied $\acute{\mathrm{z}}$ spośród podanych.

Tangens kqta $\alpha$ jest równy

A. -$\sqrt{}$110

B. $(-\displaystyle \frac{3}{\sqrt{10}})$

C. $(-\displaystyle \frac{3}{1})$

D. $(-\displaystyle \frac{1}{3})$

$Brudno\sqrt{}is$

-

Zadanie 19$*$(0-\S) $p$

Dokończ zdanie. Wybierz w[aściwq odpowied $\acute{\mathrm{z}}$ spośród podanych.

Dla $\mathrm{k}\mathrm{a}\dot{\mathrm{z}}$ dego kqta ostrego $\alpha$ wyrazenie $\sin^{4}\alpha +\sin^{2}\alpha\cdot\cos^{2}\alpha$ jest równe

A. $\sin^{2}\alpha$

B. $\sin^{6}\alpha\cdot\cos^{2}\alpha$

C. $\sin^{4}\alpha+1$

D. $\sin^{2}\alpha\cdot(\sin\alpha+\cos\alpha)\cdot(\sin\alpha-\cos\alpha)$

{\it Brudnopis}

$\mathrm{M}\mathrm{M}\mathrm{A}\mathrm{P}-\mathrm{P}0_{-}100$

Strona 17 z31
\end{document}
