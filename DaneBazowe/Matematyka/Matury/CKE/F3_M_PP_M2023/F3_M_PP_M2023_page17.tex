\documentclass[a4paper,12pt]{article}
\usepackage{latexsym}
\usepackage{amsmath}
\usepackage{amssymb}
\usepackage{graphicx}
\usepackage{wrapfig}
\pagestyle{plain}
\usepackage{fancybox}
\usepackage{bm}

\begin{document}

Zädanie 20. (0-\S) $\beta$

$\mathrm{W}$ rombie o boku dlugości $6\sqrt{2}$ kqt rozwarty ma miar9 $150^{\mathrm{o}}$

Dokończ zdanie. Wybierz wlaściwq odpowied $\acute{\mathrm{z}}$ spośród podanych.

lloczyn dlugości przekqtnych tego rombu jest równy

A. 24

B. 72

C. 36

D. $36\sqrt{2}$

{\it Brudnopis}

Zadan$\mathrm{e}2l. (0\infty 1)$

H $\beta$

Punkty $A, B, C \mathrm{l}\mathrm{e}\dot{\mathrm{z}}$ a na okr gu o środku w punkcie 0.

$\mathrm{K} \mathrm{t} AC0$ ma miar $70^{\mathrm{o}}$ (zobacz rysunek).
\begin{center}
\includegraphics[width=68.124mm,height=73.104mm]{./F3_M_PP_M2023_page17_images/image001.eps}
\end{center}
{\it B}

{\it 0}

{\it C}

{\it A}

odpowied $\acute{\mathrm{z}}$ spośród podanych.

Dokończ zdanie.

Wybierz w[aściw

Miara kata ostrego ABC jest równa

A. $10^{\mathrm{o}}$

B. $20^{\mathrm{o}}$

C. $35^{\mathrm{o}}$

D. $40^{\mathrm{o}}$

{\it Brudnopis}

Strona 18 z31

$\mathrm{M}\mathrm{M}\mathrm{A}\mathrm{P}-\mathrm{P}0_{-}100$
\end{document}
