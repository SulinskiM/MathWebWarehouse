\documentclass[a4paper,12pt]{article}
\usepackage{latexsym}
\usepackage{amsmath}
\usepackage{amssymb}
\usepackage{graphicx}
\usepackage{wrapfig}
\pagestyle{plain}
\usepackage{fancybox}
\usepackage{bm}

\begin{document}

Zädanie 24. \{0-\S) $\beta$

$\mathrm{W}$ kartezjańskim ukladzie wspólrzednych $(x,y)$ dana jest prosta $k$ o równaniu

$y=-\displaystyle \frac{1}{3}x+2$

Dokończ zdanie. Wybierz w[aściwq odpowied $\acute{\mathrm{z}}$ spośród podanych.

Prosta o równaniu $y=ax+b$ jest równolegla do prostej $k$ i przechodzi przez

punkt $P=(3,5)$, gdy

A. $a=3 \mathrm{i} b=4.$

B. $a=-\displaystyle \frac{1}{3} \mathrm{i} b=4.$

C. $a=3 \mathrm{i} b=-4.$

D. $a=-\displaystyle \frac{1}{3} \mathrm{i} b=6.$

{\it Brud}$\underline{no}\underline{\sqrt{}is}_{-} -$

-

Zadanie 25. (0-{\$}) $\mathrm{R} \beta$

Dany jest graniastoslup prawidlowy czworokqtny, w którym krawedz' podstawy ma

dlugośč 15. Przekqtna graniastos1upa jest nachy1ona do p1aszczyzny podstawy pod

kqtem $\alpha$ takim, $\dot{\mathrm{z}}\mathrm{e} \displaystyle \cos\alpha=\frac{\sqrt{2}}{3}$

Dokończ zdanie. Wybierz w[aściwq odpowied $\acute{\mathrm{z}}$ spośród podanych.

Dlugośč przekqtnej tego graniastoslupa jest równa

A. $15\sqrt{2}$

B. 45

C. $5\sqrt{2}$

D. 10

{\it Brudnopis}

$\mathrm{M}\mathrm{M}\mathrm{A}\mathrm{P}-\mathrm{P}0_{-}100$

Strona 21 z31
\end{document}
