\documentclass[a4paper,12pt]{article}
\usepackage{latexsym}
\usepackage{amsmath}
\usepackage{amssymb}
\usepackage{graphicx}
\usepackage{wrapfig}
\pagestyle{plain}
\usepackage{fancybox}
\usepackage{bm}

\begin{document}

Zadanie 26. (0-4)

Dany jest ostroslup prawidlowy czworokatny. Wysokośč ściany bocznej tego ostroslupa jest

nachylona do plaszczyzny podstawy pod $\mathrm{k}_{\mathrm{c}}$]$\mathrm{t}\mathrm{e}\mathrm{m} 30^{\mathrm{o}}$ i ma dlugośč równa 6 (zobacz rysunek).

Oblicz objetośč i pole powierzchni calkowitej tego ostros[upa. Zapisz obliczenia.

1

Strona 22 z31

$\mathrm{M}\mathrm{M}\mathrm{A}\mathrm{P}-\mathrm{P}0_{-}100$
\end{document}
