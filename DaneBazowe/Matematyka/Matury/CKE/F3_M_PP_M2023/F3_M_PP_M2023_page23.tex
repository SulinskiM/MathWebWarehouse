\documentclass[a4paper,12pt]{article}
\usepackage{latexsym}
\usepackage{amsmath}
\usepackage{amssymb}
\usepackage{graphicx}
\usepackage{wrapfig}
\pagestyle{plain}
\usepackage{fancybox}
\usepackage{bm}

\begin{document}

Zädanie 27, \{0-\S)

$\beta$

$\mathrm{W}$ pewnym ostroslupie prawidlowym stosunek liczby $W$ wszystkich wierzcholków do

liczby $K$ wszystkich krawedzi jest równy $\displaystyle \frac{W}{K}=\frac{3}{5}$

Dokończ zdanie. Wybierz w[aściwq odpowied $\acute{\mathrm{z}}$ spośród podanych.

Podstawq tego ostroslupa jest

A. kwadrat.

B. $\mathrm{p}\mathrm{i}_{9}$ciokqt foremny.

C. sześciokqt foremny.

D. siedmiokqt foremny.

{\it Brudnopis}

Zadanie 2@. $(0\leftrightarrow 1) \mathrm{E} \beta$

Dokończ zdanie. Wybierz w[aściwq odpowied $\acute{\mathrm{z}}$ spośród podanych.

Wszystkich liczb naturalnych pieciocyfrowych, w których zapisie dziesietnym wystepujq tylko

cyfry 0, 5, 7 (np. 57075, 55555), jest

A. $5^{3}$

B. $2\cdot 4^{3}$

C. $2\cdot 3^{4}$

D. $3^{5}$

{\it Brudnopis}

Strona 24 z31

$\mathrm{M}\mathrm{M}\mathrm{A}\mathrm{P}-\mathrm{P}0_{-}100$
\end{document}
