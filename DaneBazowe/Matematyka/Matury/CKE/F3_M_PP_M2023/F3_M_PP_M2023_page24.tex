\documentclass[a4paper,12pt]{article}
\usepackage{latexsym}
\usepackage{amsmath}
\usepackage{amssymb}
\usepackage{graphicx}
\usepackage{wrapfig}
\pagestyle{plain}
\usepackage{fancybox}
\usepackage{bm}

\begin{document}

Zadanie 29. (0-2)

Na diagramie ponizej przedstawiono ceny pomidorów w szesnastu wybranych sklepach.

6

5

4

liczba

sklepów 3
\begin{center}
\includegraphics[width=154.176mm,height=80.364mm]{./F3_M_PP_M2023_page24_images/image001.eps}
\end{center}
2

1

0

5,05

5,60

5,70

6,00

6,30

cena za l kg pomidorów (w zl)

Uzupe[nij tabele. Wpisz w $\mathrm{k}\mathrm{a}\dot{\mathrm{z}}$ dq pustq komórke tabeli w[aściwq odpowied $\acute{\mathrm{z}}$, wybranq

spośród oznaczonych literami A-E.
\begin{center}
\begin{tabular}{|l|l|l|}
\hline
\multicolumn{1}{|l|}{$29.1.$}&	\multicolumn{1}{|l|}{$\begin{array}{l}\mbox{Mediana ceny kilograma pomidorów w tych wybranych sklepach jest}	\\	\mbox{równa}	\end{array}$}&	\multicolumn{1}{|l|}{}	\\
\hline
\multicolumn{1}{|l|}{ $29.2.$}&	\multicolumn{1}{|l|}{$\begin{array}{l}\mbox{ $\acute{\mathrm{S}}$ rednia cena kilograma pomidorów w tych wybranych sklepach jest}	\\	\mbox{równa}	\end{array}$}&	\multicolumn{1}{|l|}{}	\\
\hline
\end{tabular}

\end{center}
A. 5,80 z1

B. 5,73 z1

C. 5,85 z1

D. 6,00 z1

E. 5,70 z1

{\it Brudnopis}

$\mathrm{M}\mathrm{M}\mathrm{A}\mathrm{P}-\mathrm{P}0_{-}100$

Strona 25 z31
\end{document}
