\documentclass[a4paper,12pt]{article}
\usepackage{latexsym}
\usepackage{amsmath}
\usepackage{amssymb}
\usepackage{graphicx}
\usepackage{wrapfig}
\pagestyle{plain}
\usepackage{fancybox}
\usepackage{bm}

\begin{document}

Zadanie 38.

Wlaściciel pewnej apteki przeanalizowal dane dotyczqce liczby obslugiwanych klientów

$\mathrm{z} 30$ kolejnych dni. Przyjmijmy, $\dot{\mathrm{z}}\mathrm{e}$ liczbe $L$ obslugiwanych klientów $n$-tego dnia opisuje

funkcja

$L(n)=-n^{2}+22n+279$

gdzie $n$ jest liczbq naturalnq$\mathrm{s}\mathrm{p}\mathrm{e}$niajqcq warunki $n\geq 1 \mathrm{i} n\leq 30.$

Zadanie $38_{\mathrm{r}}\S. (0-\not\in)\mathrm{E} p$

Oceń prawdziwośč ponizszych stwierdzeń. Wybierz $\mathrm{P}$, jeśli stwierdzenie jest

prawdziwe, albo $\mathrm{F}$ -jeśli jest fa[szywe.
\begin{center}
\begin{tabular}{|l|l|l|}
\hline
\multicolumn{1}{|l|}{ $\begin{array}{l}\mbox{Laczna liczba klientów obsluzonych w czasie wszystkich analizowanych dni}	\\	\mbox{jest równa $L(30).$}	\end{array}$}&	\multicolumn{1}{|l|}{P}&	\multicolumn{1}{|l|}{F}	\\
\hline
\multicolumn{1}{|l|}{$\mathrm{W}$ trzecim dniu analizowanego okresu obsluzono 336 klientów.}&	\multicolumn{1}{|l|}{P}&	\multicolumn{1}{|l|}{F}	\\
\hline
\end{tabular}

\end{center}
$B_{\Gamma}udno\sqrt{}is$

Zadanie 38.2. $(0-2J$

Którego dnia analizowanego okresu w aptece obslu $\dot{\mathrm{z}}$ ono najwiekszq liczbe klientów?

Oblicz liczbe klientów obslu $\dot{\mathrm{z}}$ onych tego dnia. Zapisz obliczenia.

$\mathrm{M}\mathrm{M}\mathrm{A}\mathrm{P}-\mathrm{P}0_{-}100$

Strona 27 z31
\end{document}
