\documentclass[a4paper,12pt]{article}
\usepackage{latexsym}
\usepackage{amsmath}
\usepackage{amssymb}
\usepackage{graphicx}
\usepackage{wrapfig}
\pagestyle{plain}
\usepackage{fancybox}
\usepackage{bm}

\begin{document}

Zädanie 1. (0-t) $\beta$

Na osi liczbowej zaznaczono sum9 przedzia1ów.
\begin{center}
\includegraphics[width=143.964mm,height=11.424mm]{./F3_M_PP_M2023_page3_images/image001.eps}
\end{center}
$-2$  5  $\chi$

Dokończ zdanie. Wybierz w[aściwq odpowied $\acute{\mathrm{z}}$ spośród podanych.

Zbiór zaznaczony na osi jest zbiorem wszystkich rozwiqzań nierówności

A. $|x-3,5|\geq 1,5$

B. $|x-1,5|\geq 3,5$

C. $|x-3,5|\leq 1,5$

D. $|x-1,5|\leq 3,5$

$\underline{Brudno\sqrt{}is}$

$1 -$

Zadanie $2_{\mathrm{Y}}$ (0-\S\} $\bullet \beta$

Dokończ zdanie. Wybierz w[aściwq odpowied $\acute{\mathrm{z}}$ spośród podanych.

Liczba $\sqrt[3]{-\frac{27}{16}}\cdot\sqrt[3]{2}$ jest równa

A. $(-\displaystyle \frac{3}{2})$

B. -23

C. -32

D. $(-\displaystyle \frac{2}{3})$

{\it Brudnopis}

Strona 4 z31

$\mathrm{M}\mathrm{M}\mathrm{A}\mathrm{P}-\mathrm{P}0_{-}100$
\end{document}
