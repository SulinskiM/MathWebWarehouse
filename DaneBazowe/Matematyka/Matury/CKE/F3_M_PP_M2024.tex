\documentclass[a4paper,12pt]{article}
\usepackage{latexsym}
\usepackage{amsmath}
\usepackage{amssymb}
\usepackage{graphicx}
\usepackage{wrapfig}
\pagestyle{plain}
\usepackage{fancybox}
\usepackage{bm}

\begin{document}

CENTRALNA

KOMISJA

EGZAMINACYJNA

Arkusz zawiera informacje prawnie chronione

do momentu rozpoczecia egzaminu.

KOD

WYPELNIA ZOAJACY

PESEL

{\it Miejsce na naklejke}.

{\it Sprawdz}', {\it czy kod na naklejce to}

M-100.
\begin{center}
\includegraphics[width=21.900mm,height=10.164mm]{./F3_M_PP_M2024_page0_images/image001.eps}

\includegraphics[width=79.656mm,height=10.164mm]{./F3_M_PP_M2024_page0_images/image002.eps}
\end{center}
/{\it ezeli tak}- {\it przyklej naklejkq}.

/{\it ezeli nie}- {\it zgtoś to nauczycielowi}.

Egzamin maturalny

$\displaystyle \int$
\begin{center}
\includegraphics[width=193.344mm,height=75.792mm]{./F3_M_PP_M2024_page0_images/image003.eps}
\end{center}
$\mathrm{P}\mathrm{o}\mathrm{z}\mathrm{i}$\fcircle$\mathrm{m}$  podstawowy

{\it Symbo arkusza}

MMAP-P0-100-2405

DATA: 8 maja 2024 r.

GODZINA R0ZP0CZECIA: 9:00

CZAS TRWANIA: $180 \displaystyle \min$ ut

WYPetNIA ZESPÓL NAOZORUJACY

Uprawnienia $\mathrm{z}\mathrm{d}\mathrm{a}\mathrm{j}_{8}$cego do:

\fbox{} dostosowania zasad oceniania

\fbox{} dostosowania w zw. z dyskalkuliq

\fbox{} nieprzenoszenia odpowiedzi na karte.

LICZBA PUNKTÓW DO UZYSKANIA 46

Przed rozpoczeciem pracy z arkuszem egzaminacyjnym

1.

Sprawd $\acute{\mathrm{z}}$, czy nauczyciel przekazal Ci wlaściwy arkusz egzaminacyjny,

tj. arkusz we wlaściwej formule, z w[aściwego przedmiotu na wlaściwym

poziomie.

2.

$\mathrm{J}\mathrm{e}\dot{\mathrm{z}}$ eli przekazano Ci niew[aściwy arkusz- natychmiast zgloś to nauczycielowi.

Nie rozrywaj banderol.

3. $\mathrm{J}\mathrm{e}\dot{\mathrm{z}}$ eli przekazano Ci w[aściwy arkusz- rozerwij banderole po otrzymaniu

takiego polecenia od nauczyciela. Zapoznaj $\mathrm{s}\mathrm{i}\mathrm{e}$ z instrukcjq na stronie 2.

Uk\}ad graficzny

\copyright CKE 2022 O

$\Vert\Vert\Vert\Vert\Vert\Vert\Vert\Vert\Vert\Vert\Vert\Vert\Vert\Vert\Vert\Vert\Vert\Vert\Vert\Vert\Vert\Vert\Vert\Vert\Vert\Vert\Vert\Vert\Vert\Vert|$




lnstrukcja dla zdajqcego

l. Sprawdz', czy arkusz egzaminacyjny zawiera 30 stron (zadania $1-31$).

Ewentualny brak zgloś przewodniczacemu zespolu nadzorujacego egzamin.

2. Na pierwszej stronie arkusza oraz na karcie odpowiedzi wpisz swój numer PESEL

i przyklej naklejke z kodem.

3. Symbol $\overline{\mathrm{L}^{\mathrm{g}}\Leftrightarrow \mathrm{g}}\nearrow$zamieszczony w naglówku zadania oznacza, $\dot{\mathrm{z}}\mathrm{e}$ rozwiqzanie zadania

zamknietego musisz przenieśč na karte odpowiedzi. Ocenie podlegajq$\mathrm{w}\mathrm{y}$qcznie

odpowiedzi zaznaczone na karcie odpowiedzi.

4. Odpowiedzi do zadań $\mathrm{z}\mathrm{a}\mathrm{m}\mathrm{k}\mathrm{n}\mathrm{i}_{9}$tych zaznacz na karcie odpowiedzi w cześci karty

przeznaczonej dla zdajqcego. Zamaluj $\blacksquare$ pola do tego przeznaczone. $\mathrm{B}_{9}\mathrm{d}\mathrm{n}\mathrm{e}$

zaznaczenie otocz kólkiem \copyright i zaznacz wlaściwe.

5. Pamietaj, $\dot{\mathrm{z}}\mathrm{e}$ pominiecie argumentacji lub istotnych obliczeń w rozwiqzaniu zadania

otwartego $\mathrm{m}\mathrm{o}\dot{\mathrm{z}}\mathrm{e}$ spowodować, $\dot{\mathrm{z}}\mathrm{e}$ za to rozwiazanie nie otrzymasz pelnej liczby punktów.

6. Rozwiqzania zadań i odpowiedzi wpisuj w miejscu na to przeznaczonym.

7. Pisz czytelnie i $\mathrm{u}\dot{\mathrm{z}}$ ywaj tylko dlugopisu lub pióra z czarnym tuszem lub atramentem.

8. Nie $\mathrm{u}\dot{\mathrm{z}}$ ywaj korektora, a bledne zapisy wyra $\acute{\mathrm{z}}$ nie przekreśl.

9. Nie wpisuj $\dot{\mathrm{z}}$ adnych znaków w tabelkach przeznaczonych dla egzaminatora.

Tabelki umieszczone sa na marginesie przy odpowiednich zadaniach.

10. Pamietaj, $\dot{\mathrm{z}}\mathrm{e}$ zapisy w brudnopisie nie beda oceniane.

11. $\mathrm{M}\mathrm{o}\dot{\mathrm{z}}$ esz korzystač z Wybranych wzorów matematycznych, cyrkla i linijki oraz kalkulatora

prostego. Upewnij $\mathrm{s}\mathrm{i}\mathrm{e}$, czy przekazano Ci broszure z $\mathrm{o}\mathrm{k}$adkq takQ jak widoczna ponizej.

Strona 2 z30

$\mathrm{M}\mathrm{M}\mathrm{A}\mathrm{P}-\mathrm{P}0_{-}100$





Zadanie 9{\$}. (0-{\$}) $\overline{\mathrm{L}\mathfrak{B}\mathfrak{B}}$'

Na rysunku, w kartezjańskim ukladzie wspólrzednych $(x,y)$, przedstawiono dwie proste

równolegle, które sq interpretacjq geometrycznq jednego z ponizszych ukladów równań A-D.
\begin{center}
\includegraphics[width=97.176mm,height=100.884mm]{./F3_M_PP_M2024_page10_images/image001.eps}
\end{center}
{\it y}

1

0  1  $\chi$

Dokończ zdanie. Wybierz w[aściwq odpowied $\acute{\mathrm{z}}$ spośród podanych.

Ukladem równań, którego interpretacj9 geometrycznq przedstawiono na rysunku, jest

A. 

B. 

C. 

D. 

{\it Brudnopis}

$\mathrm{M}\mathrm{M}\mathrm{A}\mathrm{P}-\mathrm{P}0_{-}100$

Strona ll z30





$\mathrm{Z}\mathrm{a}\mathrm{d}\mathrm{a}\mathrm{n}\ddagger \mathrm{e}$ \S 2. $(0-1\} \overline{\mathrm{L}\mathfrak{B}\mathfrak{B}}$'

Funkcja liniowa $f$ jest określona wzorem $f(x)=(-2k+3)x+k-1$, gdzie $k\in \mathbb{R}.$

Dokończ zdanie. Wybierz wlaściwq odpowied $\acute{\mathrm{z}}$ spośród podanych.

Funkcja $f$ jest malejaca dla $\mathrm{k}\mathrm{a}\dot{\mathrm{z}}$ dej liczby $k$ nalezqcej do przedzialu

A. $(-\infty,1)$

B. $(-\displaystyle \infty,-\frac{3}{2})$

C. $(1,+\infty)$

D. $(\displaystyle \frac{3}{2},+\infty)$

{\it Brudnopis}

1

Zadanie \S 3$*$\{0$\infty$9) $\square \sqsupset\sqsupset\sqsupset$'

Funkcje liniowe $f$ oraz $g$, określone wzorami $f(x)=3x+6$ oraz $g(x)=ax+7$, maja

to samo miejsce zerowe.

Dokończ zdanie. Wybierz w[aściwq odpowied $\acute{\mathrm{z}}$ spośród podanych.

Wspólczynnik a we wzorze funkcji g jest równy

A. $(-\displaystyle \frac{7}{2})$

B. $(-\displaystyle \frac{2}{7})$

C. -72

D. -27

{\it Brudnopis}

Strona 12 z30

$\mathrm{M}\mathrm{M}\mathrm{A}\mathrm{P}-\mathrm{P}0_{-}100$





Zadanie t4.

W kartezjańskim ukladzie wspólrzednych (x, y) przedstawiono fragment paraboli, która jest

wykresem funkcji kwadratowej f (zobacz rysunek). Wierzcholek tej paraboli oraz punkty

przecipcia paraboli z osiami ukladu wspólrzednych maja obie wspólrzedne calkowite.

Zadanie \S 4.a. $(0-8$\}

Uzupe[nij ponizsze zdanie. Wpisz odpowiedni przedzia[w wykropkowanym miejscu

tak, aby zdanie bylo prawdziwe.

Zbiorem wszystkich rozwiazań nierówności $f(x)\geq 0$ jest przedzial

{\it Brudnopis}

Zadanie 94.2. (0-\S\} $\overline{1^{\mathrm{n}}\Delta \mathrm{g}\mathrm{g}}1$

Dokończ zdanie. Wybierz w[aściwq odpowied $\acute{\mathrm{z}}$ spośród podanych.

Funkcja kwadratowa f jest określona wzorem

A. $f(x)=-(x+1)^{2}-9$

B. $f(x)=-(x-1)^{2}+9$

C. $f(x)=-(x-1)^{2}-9$

D. $f(x)=-(x+1)^{2}+9$

{\it Brudnopis}

$\mathrm{M}\mathrm{M}\mathrm{A}\mathrm{P}-\mathrm{P}0_{-}100$

Strona 13 z30





Zadanie 14.3. (0-\S\} $\overline{\mathrm{L}\mathfrak{W}\mathfrak{B}}$;

Dokończ zdanie. Wybierz w[aściwq odpowied $\acute{\mathrm{z}}$ spośród podanych.

Dla funkcji f prawdziwa jest równośč

A. $f(-4)=f(6)$

B. $f(-4)=f(5)$

C. $f(-4)=f(4)$

D. $f(-4)=f(7)$

{\it Brudnopis}

$\mathrm{Z}\mathrm{a}\mathrm{d}\mathrm{a}\mathfrak{n}i\mathrm{e}*4.4_{*}\{0\infty 2)$

Funkcje kwadratowe $g$ oraz $h$ sa określone za pomocq funkcji $f$ (zobacz rysunek na

stronie 13) nastepujqco: $g(x)=f(x+3), h(x)=f(-x).$

Na rysunkach A-F przedstawiono, w kartezjańskim uk[adzie wspólrz9dnych $(x,y),$

fragmenty wykresów róznych funkcji-w tym fragment wykresu funkcji $g$ oraz fragment

wykresu funkcji $h.$

Uzupelnij tabele. $\mathrm{K}\mathrm{a}\dot{\mathrm{z}}$ dej z funkcji $g$ oraz $h$ przyporzqdkuj fragmentjej wykresu.

Wpisz w $\mathrm{k}\mathrm{a}\dot{\mathrm{z}}$ dq pustq komórke tabeli w[aściwq odpowied $\acute{\mathrm{z}}$, wybranq spośród

oznaczonych literami A-F.
\begin{center}
\begin{tabular}{|l|l|}
\hline
\multicolumn{1}{|l|}{Fragment wykresu funkcji $y=g(x)$ przedstawiono na rysunku}&	\multicolumn{1}{|l|}{}	\\
\hline
\multicolumn{1}{|l|}{Fragment wykresu funkcji $y=h(x)$ przedstawiono na rysunku}&	\multicolumn{1}{|l|}{}	\\
\hline
\end{tabular}

\end{center}
Strona 14 z30

$\mathrm{M}\mathrm{M}\mathrm{A}\mathrm{P}-\mathrm{P}0_{-}100$





{\it Brudnopis}

$\mathrm{M}\mathrm{M}\mathrm{A}\mathrm{P}-\mathrm{P}0_{-}100$

Strona 15 z30





Zadanie 15. $(0-1$\} $\overline{\mathrm{L}\mathfrak{B}\mathfrak{B}}$'

Ciqg $(a_{n})$ jest określony wzorem $a_{n}=(-1)^{n}\cdot(n-5)$ dla $\mathrm{k}\mathrm{a}\dot{\mathrm{z}}$ dej liczby

naturalnej $n\geq 1.$

Oceń prawdziwośč ponizszych stwierdzeń. Wybierz P, jeśli stwierdzenie jest

prawdziwe, albo F -jeśli jest fa[szywe.
\begin{center}
\begin{tabular}{|l|l|l|}
\hline
\multicolumn{1}{|l|}{$\begin{array}{l}\mbox{Pierwszy wyraz ciqgu $(a_{n})$ jest dwa razy wipkszy od trzeciego wyrazu tego}	\\	\mbox{ciqgu.}	\end{array}$}&	\multicolumn{1}{|l|}{P}&	\multicolumn{1}{|l|}{F}	\\
\hline
\multicolumn{1}{|l|}{Wszystkie wyrazy ciqgu $(a_{n})$ sq dodatnie.}&	\multicolumn{1}{|l|}{P}&	\multicolumn{1}{|l|}{F}	\\
\hline
\end{tabular}

\end{center}
{\it Brudnopis}

-

1

Zadanie k6. $\{0\infty \mathrm{B}$) $\overline{\mathrm{L}\emptyset\infty \mathrm{g}}$;

Trzywyrazowy ciag $(12,6,2m-1)$ jest geometryczny.

Dokończ zdanie. Wybierz odpowied $\acute{\mathrm{z}}$ A albo $\mathrm{B}$ oraz $\mathrm{o}\mathrm{d}\mathrm{p}\mathrm{o}\mathrm{w}\mathrm{i}\mathrm{e}\mathrm{d}\acute{\mathrm{z}}1., 2$. albo 3.

Ten ciqg jest

A.

rosnacy

1.

$m=\displaystyle \frac{1}{2}$

oraz

2.

$m=2$

B.

malejqcy

3.

$m=3$

{\it Brudnopis}

Strona 16 z30

$\mathrm{M}\mathrm{M}\mathrm{A}\mathrm{P}-\mathrm{P}0_{-}100$





Zadanie $\mathrm{t}7. (0-2)$

Ciqg arytmetyczny $(a_{n})$ jest określony dla $\mathrm{k}\mathrm{a}\dot{\mathrm{z}}$ dej liczby naturalnej $n\geq 1$. Trzeci wyraz

tego ciqgu jest równy $(-1)$, a suma piptnastu poczqtkowych kolejnych wyrazów tego ciqgu

jest równa $(-165).$

Oblicz róznice tego ciqgu. Zapisz obliczenia.

1

$1-$

1

$\mathrm{M}\mathrm{M}\mathrm{A}\mathrm{P}-\mathrm{P}0_{-}100$

Strona 17 z30





Zadanie $\mathrm{f}8_{*}(0-2)$

$\mathrm{W}$ kartezjańskim ukladzie wspólrzednych $(x,y)$ zaznaczono kqt o mierze $\alpha$ taki, $\dot{\mathrm{z}}\mathrm{e}$

tg $\alpha=-3$ oraz $90^{\mathrm{o}}<\alpha<180^{\mathrm{o}}$ (zobacz rysunek).

Uzupe[nij zdanie. Wybierz dwie w[aściwe odpowiedzi spośród oznaczonych literami

A-F i wpisz te litery w wykropkowanych miejscach.

Prawdziwe sq zalezności:

oraz

A. $\sin\alpha<0$

B. $\sin\alpha\cdot\cos\alpha<0$

C. $\sin\alpha\cdot\cos\alpha>0$

D. $\cos\alpha>0$

E. $\displaystyle \sin\alpha=-\frac{1}{3}\cos\alpha$

$\mathrm{F}.\ \sin\alpha=-3\cos\alpha$

$Brudno\sqrt{}is -$

Zadanie \S 9. $(0-\not\in) \overline{\llcorner \mathfrak{B}\mathrm{g}}$;

Dokończ zdanie. Wybierz w[aściwq odpowied $\acute{\mathrm{z}}$ spośród podanych.

Liczba $\sin^{3}20^{\mathrm{o}}+\cos^{2}20^{\mathrm{o}}\cdot\sin 20^{\mathrm{o}}$ jest równa

A. $\cos 20^{\mathrm{o}}$

B. $\sin 20^{\mathrm{o}}$

C. $\mathrm{t}\mathrm{g}20^{\mathrm{o}}$

$\mathrm{D}.\ \sin 20^{\mathrm{o}}\cdot\cos 20^{\mathrm{o}}$

{\it Brudnopis}

Strona 18 z30

$\mathrm{M}\mathrm{M}\mathrm{A}\mathrm{P}-\mathrm{P}0_{-}100$





Zadanie 20. $\{0-1$) $\overline{\mathrm{L}\mathrm{E}\mathrm{g}\mathrm{g}}$;

Danyjest trójkqt $KLM$, w którym $|KM|=a, |LM|=b$ oraz $a\neq b$. Dwusieczna kqta

$KML$ przecina bok $KL$ w punkcie $N$ takim, $\dot{\mathrm{z}}\mathrm{e} |KN|=c, |NL|=d$ oraz $|MN|=e$

(zobacz rysunek).
\begin{center}
\includegraphics[width=83.772mm,height=48.864mm]{./F3_M_PP_M2024_page18_images/image001.eps}
\end{center}
{\it M}

{\it a e  b}

{\it K N}

{\it c  d L}

Dokończ zdanie. Wybierz wlaściwq odpowied $\acute{\mathrm{z}}$ spośród podanych.

$\mathrm{W}$ trójkqcie $KLM$ prawdziwa jest równośč

A. $a\cdot b=c\cdot d$

B. $a\cdot d=b\cdot c$

C. $a\cdot c=b\cdot d$

D. $a\cdot b=e\cdot e$

$Brudno\sqrt{}is$

I I I $1 -$

ZadanIe 2{\$}. $(0-9$\}

$\square \sqsupset\sqsupset\sim\Sigma*$'

Danyjest równoleglobok o bokach dlugości 3 $\mathrm{i} 4$ oraz o kacie miedzy nimi o mierze $120^{\mathrm{o}}$

Dokończ zdanie. Wybierz w[aściwq odpowied $\acute{\mathrm{z}}$ spośród podanych.

Pole tego równolegloboku jest równe

A. 12

B. $12\sqrt{3}$

C. 6

D. $6\sqrt{3}$

{\it Brudnopis}

$\mathrm{M}\mathrm{M}\mathrm{A}\mathrm{P}-\mathrm{P}0_{-}100$

Strona 19 z30





Zadanie 22. (0-1)

$\overline{\mathrm{L}\mathrm{E}\mathrm{g}\mathrm{g}}$;

$\mathrm{W}$ trójkqcie $ABC$, wpisanym w $\mathrm{o}\mathrm{k}\mathrm{r}_{\mathrm{c}}$]$\mathrm{g}$ o środku w punkcie $S$, kqt $ACB$ ma miar9 $42^{\mathrm{o}}$

(zobacz rysunek).
\begin{center}
\includegraphics[width=67.968mm,height=65.076mm]{./F3_M_PP_M2024_page19_images/image001.eps}
\end{center}
{\it C}

$42^{\mathrm{o}}$  {\it S}

{\it A}

{\it B}

Dokończ zdanie. Wybierz w[aściwq odpowied $\acute{\mathrm{z}}$ spośród podanych.

Miara kqta ostrego BAS jest równa

A. $42^{\mathrm{o}}$

B. $45^{\mathrm{o}}$

C. $48^{\mathrm{o}}$

D. $69^{\mathrm{o}}$

$Brudno\sqrt{}is -$

Zadanie 23. (0-\S) $\overline{[similar]\alpha \mathrm{g}\mathrm{u}}$'

$\mathrm{W}$ kartezjańskim ukladzie wspólrzednych $(x,y)$ proste $k$ oraz $l$ sq określone równaniami

{\it k}:

$y=(m+1)x+7$

{\it l}:

$y=-2x+7$

Dokończ zdanie. Wybierz w[aściwq odpowied $\acute{\mathrm{z}}$ spośród podanych.

Proste k oraz l sa prostopadle, gdy liczba m jest równa

A. $(-\displaystyle \frac{1}{2})$

B. -21

C. $(-3)$

D. l

{\it Brudnopis}

Strona 20 z30

$\mathrm{M}\mathrm{M}\mathrm{A}\mathrm{P}-\mathrm{P}0_{-}100$





Zadania egzaminacyjne sq wydrukowane

na nastepnych stronach.

$\mathrm{M}\mathrm{M}\mathrm{A}\mathrm{P}-\mathrm{P}0_{-}100$

Strona 3 z30





Zadanie 24. (0-2)

$\mathrm{W}$ kartezjańskim ukladzie wspólrzednych $(x,y)$ danyjest równoleglobok ABCD, w którym

$A=(-2,6)$ oraz $B=(10,2)$. Przekqtne $AC$ oraz $BD$ tego równolegloboku przecinajq

sipw punkcie $P=(6,7).$

Oblicz dlugośč boku BC tego równoleg[oboku. Zapisz obliczenia.

1

1

$\mathrm{M}\mathrm{M}\mathrm{A}\mathrm{P}-\mathrm{P}0_{-}100$

Strona 21 z30





Zadanie $25_{\mathrm{r}}$

Wysokośč graniastoslupa prawidlowego sześciokatnego jest równa 6 (zobacz rysunek).

Pole podstawy tego graniastoslupa jest równe $15\sqrt{3}.$
\begin{center}
\includegraphics[width=62.328mm,height=71.016mm]{./F3_M_PP_M2024_page21_images/image001.eps}
\end{center}
I

I

I

I

I

I

I

I

I

I

I

I

I

$\underline{\mathrm{I}}$

I

I

I

I

I

I

I

I

I

I

I

I

6

Zadanie 25.1. $(\emptyset\infty \mathrm{e}$\} $\overline{\llcorner \mathfrak{B}\mathrm{g}}$;

Dokończ zdanie. Wybierz w[aściwq odpowied $\acute{\mathrm{z}}$ spośród podanych.

Pole $|\mathrm{e}\mathrm{d}\mathrm{n}\mathrm{e}1$ ściany bocznej tego graniastoslupa jest równe

A. $36\sqrt{10}$

B. 60

C. $6\sqrt{10}$

D. 360

$Brudno\sqrt{}is$

Strona 22 z30

$\mathrm{M}\mathrm{M}\mathrm{A}\mathrm{P}-\mathrm{P}0_{-}100$





Zadanie $25_{*}2_{\alpha}$ (0-\S) $\overline{\mathrm{L}\mathfrak{W}\mathfrak{B}}$;

Dokończ zdanie. Wybierz wlaściwq odpowied $\acute{\mathrm{z}}$ spośród podanych.

Kqt nachylenia najdlu $\dot{\mathrm{z}}$ szej przekqtnej graniastoslupa prawidlowego sześciokqtnego do

plaszczyzny podstawy jest zaznaczony na rysunku

A.
\begin{center}
\includegraphics[width=57.912mm,height=70.968mm]{./F3_M_PP_M2024_page22_images/image001.eps}
\end{center}
I

I

I

I

I

I

I

I

I

I

I

I

I

$\underline{\mathrm{I}}$

I

I

I

I

I

I

I

I

I

I

I

I

C.
\begin{center}
\includegraphics[width=57.816mm,height=70.920mm]{./F3_M_PP_M2024_page22_images/image002.eps}
\end{center}
I

I

I

I

I

I

I

I

I

I

I

I

I

$\prime\underline{\mathrm{I}}$

I

I

I

I

I

I

I

I

I

I

I

I

{\it Brudnopis}

B.
\begin{center}
\includegraphics[width=57.960mm,height=70.968mm]{./F3_M_PP_M2024_page22_images/image003.eps}
\end{center}
I

I

I

I

I

I

I

I

I

I

I

I

I

$\underline{\mathrm{I}}$

I

I

I

I

I

I

I

I

I

I

D.
\begin{center}
\includegraphics[width=57.912mm,height=70.920mm]{./F3_M_PP_M2024_page22_images/image004.eps}
\end{center}
I

I

I

I

I

I

I

I

I

I

I

I

I

$\underline{\mathrm{I}}-$

I

I

I

I

I

I

I

I

I

I

I

I

$-1$

$\mathrm{M}\mathrm{M}\mathrm{A}\mathrm{P}-\mathrm{P}0_{-}100$

Strona 23 z30





Zadanie 26. $(0-\not\in)$

Ostroslup $F_{1}$ jest podobny do ostroslupa $F_{2}.$

Obj9tośč ostros1upa $F_{1}$ jest równa 64.

Obj9tośč ostros1upa $F_{2}$ jest równa 512.

Uzupe[nij ponizsze zdanie. Wpisz odpowiedniq liczbe w wykropkowanym miejscu tak,

aby zdanie by[o prawdziwe.

Stosunek pola powierzchni calkowitej ostroslupa $F_{2}$ do pola powierzchni calkowitej

ostroslupa $F_{1}$ jest równy

{\it Brudnopis}

$-|1\mathrm{i}\mathrm{i}-$

$| 1$

$\mathrm{Z}\mathrm{a}\mathrm{d}\mathrm{a}\mathrm{n}\dot{\mathfrak{x}}\mathrm{e}Z7$. (0-{\$}) $\square \sqsupset\sqsupset 2$;

Rozwazamy wszystkie kody czterocyfrowe utworzone tylko z cyfr 1, 3, 6, 8, przy czym

w $\mathrm{k}\mathrm{a}\dot{\mathrm{z}}$ dym kodzie $\mathrm{k}\mathrm{a}\dot{\mathrm{z}}$ da z tych cyfr wystppuje dokladnie jeden raz.

Dokończ zdanie. Wybierz w[aściwq odpowied $\acute{\mathrm{z}}$ spośród podanych.

Liczba wszystkich takich kodówjest równa

A. 4

B. 10

C. 24

D. 16

{\it Brudnopis}

Strona 24 z30

$\mathrm{M}\mathrm{M}\mathrm{A}\mathrm{P}-\mathrm{P}0_{-}100$





Zadanie 28, (0-{\$}) $\overline{\mathrm{L}\mathfrak{B}\mathfrak{B}}$'

$\acute{\mathrm{S}}$ rednia arytmetyczna trzech liczb: $a, b, c$, jest równa 9.

Dokończ zdanie. Wybierz w[aściwq odpowied $\acute{\mathrm{z}}$ spośród podanych.

$\acute{\mathrm{S}}$ rednia arytmetyczna sześciu liczb: $a, a, b, b, c, c$, jest równa

A. 9

B. 6

C. 4,5

D. 18

{\it Brudnopis}

$\mathrm{Z}\mathrm{a}\mathrm{d}\mathrm{a}\mathrm{n}\mathrm{i}^{\vee}\mathrm{e}29. (0\infty 1\} \square \sqsupset\supset\supset 1$

Na diagramie przedstawiono wyniki sprawdzianu z matematyki w pewnej klasie maturalnej.

Na osi poziomej podano oceny, które uzyskali uczniowie tej klasy, a na osi pionowej podano

liczbe uczniów, którzy otrzymali danq ocen9.

8

7

6

liczba

uczniów

5

4
\begin{center}
\includegraphics[width=105.060mm,height=57.144mm]{./F3_M_PP_M2024_page24_images/image001.eps}
\end{center}
3

2

1

0

1 2

3 4

5 6

ocena

Dokończ zdanie. Wybierz w[aściwq odpowied $\acute{\mathrm{z}}$ spośród podanych.

Mediana ocen uzyskanych z tego sprawdzianu przez uczniów tej klasy jest równa

A. 4,5

B. 4

C. $3_{r}5$

D. 3

{\it Brudnopis}

$\mathrm{M}\mathrm{M}\mathrm{A}\mathrm{P}-\mathrm{P}0_{-}100$

Strona 25 z30





Zadanie $30_{\mathrm{L}}\{0-2$)

Dany jest piecioelementowy zbiór $K=\{5$, 6, 7, 8, 9$\}$. Wylosowanie $\mathrm{k}\mathrm{a}\dot{\mathrm{z}}$ dej liczby z tego

zbioru jestjednakowo prawdopodobne. Ze zbioru $K$ losujemy ze zwracaniem kolejno dwa

razy po jednej liczbie i zapisujemy je w kolejności losowania.

Oblicz prawdopodobieństwo zdarzenia $A$ polegajqcego na tym, $\dot{\mathrm{z}}\mathrm{e}$ suma

wylosowanych liczb jest liczbq parzystq. Zapisz obliczenia.

1

1

Strona 26 z30

$\mathrm{M}\mathrm{M}\mathrm{A}\mathrm{P}-\mathrm{P}0_{-}100$





Zadanie $38_{*}(0-4)$

$\mathrm{W}$ schronisku dla zwierzat, na $\mathrm{p}$\}askiej powierzchni, nale $\dot{\mathrm{z}}\mathrm{y}$ zbudowač ogrodzenie z siatki

wydzielajqce trzy identyczne wybiegi o $\underline{\mathrm{w}\mathrm{s}\mathrm{p}\text{ó} \mathrm{l}\mathrm{n}\mathrm{v}\mathrm{c}\mathrm{h}}$ ścianach wewnptrznych.

Podstawq$\mathrm{k}\mathrm{a}\dot{\mathrm{z}}$ dego z tych trzech wybiegów jest $\mathrm{p}\mathrm{r}\mathrm{o}\mathrm{s}\mathrm{t}\mathrm{o}\mathrm{k}_{\mathrm{c}1}\mathrm{t}$ (jak pokazano na rysunku).

Do wykonania tego ogrodzenia nale $\dot{\mathrm{z}}\mathrm{y}\mathrm{z}\mathrm{u}\dot{\mathrm{z}}$ yč 36 metrów biezqcych siatki.

Schematyczny rysunek trzech wybiegów (widok z góry).

Liniq przerywanq zaznaczono siatk9.
\begin{center}
\includegraphics[width=107.952mm,height=6.804mm]{./F3_M_PP_M2024_page26_images/image001.eps}
\end{center}
{\it y y  y}

$\chi$

$\iota \Gamma$ 1

I

I I I

I

$1 \mathrm{I}$ 1

I

I I I

I

I I I

I

I I I

I

I bi l bi 2 I bi 3 I

wybieg l. 1 wybieg 2. wybieg 3.

I I I

I

1 I I

1

I I I

I

I I I

I

1 I I

I

I I 1

I

$1 \llcorner$ 1

Oblicz wymiary $x$ oraz $y$ jednego wybiegu, przy których suma pól podstaw tych

trzech wybiegów bedzie najwieksza. $\mathrm{W}$ obliczeniach pomiń szerokośč wejścia na

$\mathrm{k}\mathrm{a}\dot{\mathrm{z}}\mathrm{d}\mathrm{y}$ z wybiegów. Zapisz obliczenia.

$\mathrm{M}\mathrm{M}\mathrm{A}\mathrm{P}-\mathrm{P}0_{-}100$

Strona 27 z30





1

$\overline{11}-$

-

Strona 28 z30

$\mathrm{M}\mathrm{M}\mathrm{A}\mathrm{P}-\mathrm{P}0_{-}10$





BRUDNOPIS (nie podlega ocenie)

1

-PO-100

Strona 29 z30





$| 1$

Strona 30 z30

$\mathrm{M}\mathrm{M}\mathrm{A}\mathrm{P}-\mathrm{P}0_{-}10$





Zadanie 8. (0-8) $\overline{\mathrm{L}\mathfrak{W}\mathfrak{B}}$;

Dana jest nierównośč

$|x-1|\geq 3$

Na którym rysunku poprawnie zaznaczono na osi liczbowej zbiór wszystkich liczb

rzeczywistych spe[niajqcych powy $\dot{\mathrm{z}}$ szq nierównośč? Wybierz w[aściwq $\mathrm{o}\mathrm{d}\mathrm{p}\mathrm{o}\mathrm{w}\mathrm{i}\mathrm{e}\mathrm{d}\acute{\mathrm{z}}$

spośród podanych.

A.

B.
\begin{center}
\includegraphics[width=83.820mm,height=10.764mm]{./F3_M_PP_M2024_page3_images/image001.eps}
\end{center}
$-2$  4  $\chi$
\begin{center}
\includegraphics[width=83.820mm,height=10.764mm]{./F3_M_PP_M2024_page3_images/image002.eps}
\end{center}
$-2$  4  $\chi$

C.

D.
\begin{center}
\includegraphics[width=83.820mm,height=10.764mm]{./F3_M_PP_M2024_page3_images/image003.eps}
\end{center}
$-2$  4  $\chi$
\begin{center}
\includegraphics[width=83.820mm,height=10.764mm]{./F3_M_PP_M2024_page3_images/image004.eps}
\end{center}
$-2$  4  $\chi$

$Brudno\sqrt{}is -$

1

Zadanie 2. (0-\S) $\square \sqsupset\infty\infty$;

Dokończ zdanie. Wybierz w[aściwq odpowied $\acute{\mathrm{z}}$ spośród podanych.

Liczba $(\displaystyle \frac{1}{16})^{8}\cdot 8^{16}$ jest równa

A. $2^{24}$

B. $2^{16}$

C. $2^{12}$

D. $2^{8}$

{\it Brudnopis}

Strona 4 z30

$\mathrm{M}\mathrm{M}\mathrm{A}\mathrm{P}-\mathrm{P}0_{-}100$















Zadanie 3. $(0-2$\}

Wykaz, $\dot{\mathrm{z}}\mathrm{e}$ dla $\mathrm{k}\mathrm{a}\dot{\mathrm{z}}$ dej liczby naturalnej $n\geq 1$ liczba $n^{2}+(n+1)^{2}+(n+2)^{2}$ przy

dzieleniu przez 3 daje reszte 2.

-

1

1

$\mathrm{M}\mathrm{M}\mathrm{A}\mathrm{P}-\mathrm{P}0_{-}100$

Strona 5 z30





Zadanie 4. \{0-\S) $\overline{\mathrm{L}\mathfrak{W}\mathfrak{B}}$;

Dokończ zdanie. Wybierz wlaściwq odpowied $\acute{\mathrm{z}}$ spośród podanych.

Liczba $\log_{\sqrt{3}}9$ jest równa

A. 2

B. 3

{\it Brudnopis}

4

D. 9

Zadanie 5. (0-P) $\overline{\omega \mathrm{D}\mathrm{B}\mathrm{B}}1$

Dokończ zdanie. Wybierz w[aściwq odpowied $\acute{\mathrm{z}}$ spośród podanych.

Dla $\mathrm{k}\mathrm{a}\dot{\mathrm{z}}$ dej liczby rzeczywistej $a$ i dla $\mathrm{k}\mathrm{a}\dot{\mathrm{z}}$ dej liczby rzeczywistej $b$ wartośč wyrazenia

$(2a+b)^{2}-(2a-b)^{2}$ jest równa wartości wyra $\dot{\mathrm{z}}$ enia

A. $8a^{2}$

B. 8ab

C. $-8ab$

D. $2b^{2}$

{\it Brudnopis}

Strona 6 z30

$\mathrm{M}\mathrm{M}\mathrm{A}\mathrm{P}-\mathrm{P}0_{-}100$





Zadanie 6. (0-8) $\overline{\mathrm{L}\mathfrak{W}\mathfrak{B}}$;

Dokończ zdanie. Wybierz wlaściwq odpowied $\acute{\mathrm{z}}$ spośród podanych.

Zbiorem wszystkich rozwiqzań nierówności

$1-\displaystyle \frac{3}{2}x<\frac{2}{3}-x$

jest przedzial

A. $(-\displaystyle \infty,-\frac{2}{3})$

B.(-$\infty$,-23)

C. $(-\displaystyle \frac{2}{3},+\infty)$

D. $(\displaystyle \frac{2}{3},+\infty)$

{\it Brudnopis}

-

Zadanie $7_{\mathrm{r}}\{\emptyset\infty 4$) $\square \sqsupset\infty 3*\nearrow$

Dokończ zdanie. Wybierz w[aściwq odpowied $\acute{\mathrm{z}}$ spośród podanych.

Równanie $\displaystyle \frac{x+1}{(x+2)(x-3)}=0$ w zbiorze liczb rzeczywistych

A. nie ma rozwiqzania.

B. ma dokladnie jedno rozwiqzanie: $(-1).$

C. ma dokladnie dwa rozwiqzania: $(-2)$ oraz 3.

D. ma dokladnie trzy rozwiqzania: $(-1), (-2)$ oraz 3.

{\it Brudnopis}

$\mathrm{M}\mathrm{M}\mathrm{A}\mathrm{P}-\mathrm{P}0_{-}100$

Strona 7 z30





Zadanie @$*$(0-\S) $\overline{\mathrm{L}\mathfrak{W}\mathfrak{B}}$;

Danyjest wielomian $W(x)=3x^{3}+6x^{2}+9x.$

Oceń prawdziwośč ponizszych stwierdzeń. Wybierz P, jeśli stwierdzenie jest

prawdziwe, albo F -jeśli jest fa[szywe.
\begin{center}
\begin{tabular}{|l|l|l|}
\hline
\multicolumn{1}{|l|}{Wielomian $W$ jest iloczynem wielomianów $F(x)=3x \mathrm{i} G(x)=\chi^{2}+2x+3.$}&	\multicolumn{1}{|l|}{P}&	\multicolumn{1}{|l|}{F}	\\
\hline
\multicolumn{1}{|l|}{Liczba $(-1)$ jest rozwiqzaniem równania $W(x)=0.$}&	\multicolumn{1}{|l|}{P}&	\multicolumn{1}{|l|}{F}	\\
\hline
\end{tabular}

\end{center}
{\it Brudnopis}

1

$| 1$

$1$

Zadanie $\mathrm{g}. (0-3\rangle$

Rozwiqz równanie

$x^{3}-2x^{2}-3x+6=0$

Zapisz obliczenia.

Strona 8 z30

$\mathrm{M}\mathrm{M}\mathrm{A}\mathrm{P}-\mathrm{P}0_{-}100$





1

$\overline{11}-$

-

$0_{-}100$

Strona 9 z30





Zadanie 10. (0-{\$}) $\overline{\mathrm{L}\mathfrak{B}\mathfrak{B}}$'

$\mathrm{W}$ paz'dzierniku 2022 roku za1ozono dwa sady, w których posadzono 1qcznie 1960 drzew.

Po roku stwierdzono, $\dot{\mathrm{z}}\mathrm{e}$ uschlo 5\% drzew w pierwszym sadzie i 10\% drzew w drugim

sadzie. Uschnipte drzewa usunieto, a nowych nie dosadzano.

Liczba drzew, które pozostaly w drugim sadzie, stanowila 60\% 1iczby drzew, które

pozostaly w pierwszym sadzie.

Niech $x$ oraz $\mathrm{y}$ oznaczaja liczby drzew posadzonych- odpowiednio-w pierwszym

i drugim sadzie.

Dokończ zdanie. Wybierz wlaściwq odpowied $\acute{\mathrm{z}}$ spośród podanych.

Ukladem równań, którego poprawne rozwiqzanie prowadzi do obliczenia liczby $x$ drzew

posadzonych w pierwszym sadzie oraz liczby $y$ drzew posadzonych w drugim sadzie, jest

A. 

B. 

C. 

D. 

$Brudno\sqrt{}is$

Strona 10 z30

$\mathrm{M}\mathrm{M}\mathrm{A}\mathrm{P}-\mathrm{P}0_{-}100$



\end{document}