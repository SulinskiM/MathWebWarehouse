\documentclass[a4paper,12pt]{article}
\usepackage{latexsym}
\usepackage{amsmath}
\usepackage{amssymb}
\usepackage{graphicx}
\usepackage{wrapfig}
\pagestyle{plain}
\usepackage{fancybox}
\usepackage{bm}

\begin{document}

Zadanie 9{\$}. (0-{\$}) $\overline{\mathrm{L}\mathfrak{B}\mathfrak{B}}$'

Na rysunku, w kartezjańskim ukladzie wspólrzednych $(x,y)$, przedstawiono dwie proste

równolegle, które sq interpretacjq geometrycznq jednego z ponizszych ukladów równań A-D.
\begin{center}
\includegraphics[width=97.176mm,height=100.884mm]{./F3_M_PP_M2024_page10_images/image001.eps}
\end{center}
{\it y}

1

0  1  $\chi$

Dokończ zdanie. Wybierz w[aściwq odpowied $\acute{\mathrm{z}}$ spośród podanych.

Ukladem równań, którego interpretacj9 geometrycznq przedstawiono na rysunku, jest

A. 

B. 

C. 

D. 

{\it Brudnopis}

$\mathrm{M}\mathrm{M}\mathrm{A}\mathrm{P}-\mathrm{P}0_{-}100$

Strona ll z30
\end{document}
