\documentclass[a4paper,12pt]{article}
\usepackage{latexsym}
\usepackage{amsmath}
\usepackage{amssymb}
\usepackage{graphicx}
\usepackage{wrapfig}
\pagestyle{plain}
\usepackage{fancybox}
\usepackage{bm}

\begin{document}

$\mathrm{Z}\mathrm{a}\mathrm{d}\mathrm{a}\mathrm{n}\ddagger \mathrm{e}$ \S 2. $(0-1\} \overline{\mathrm{L}\mathfrak{B}\mathfrak{B}}$'

Funkcja liniowa $f$ jest określona wzorem $f(x)=(-2k+3)x+k-1$, gdzie $k\in \mathbb{R}.$

Dokończ zdanie. Wybierz wlaściwq odpowied $\acute{\mathrm{z}}$ spośród podanych.

Funkcja $f$ jest malejaca dla $\mathrm{k}\mathrm{a}\dot{\mathrm{z}}$ dej liczby $k$ nalezqcej do przedzialu

A. $(-\infty,1)$

B. $(-\displaystyle \infty,-\frac{3}{2})$

C. $(1,+\infty)$

D. $(\displaystyle \frac{3}{2},+\infty)$

{\it Brudnopis}

1

Zadanie \S 3$*$\{0$\infty$9) $\square \sqsupset\sqsupset\sqsupset$'

Funkcje liniowe $f$ oraz $g$, określone wzorami $f(x)=3x+6$ oraz $g(x)=ax+7$, maja

to samo miejsce zerowe.

Dokończ zdanie. Wybierz w[aściwq odpowied $\acute{\mathrm{z}}$ spośród podanych.

Wspólczynnik a we wzorze funkcji g jest równy

A. $(-\displaystyle \frac{7}{2})$

B. $(-\displaystyle \frac{2}{7})$

C. -72

D. -27

{\it Brudnopis}

Strona 12 z30

$\mathrm{M}\mathrm{M}\mathrm{A}\mathrm{P}-\mathrm{P}0_{-}100$
\end{document}
