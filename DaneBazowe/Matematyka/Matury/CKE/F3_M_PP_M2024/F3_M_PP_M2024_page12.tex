\documentclass[a4paper,12pt]{article}
\usepackage{latexsym}
\usepackage{amsmath}
\usepackage{amssymb}
\usepackage{graphicx}
\usepackage{wrapfig}
\pagestyle{plain}
\usepackage{fancybox}
\usepackage{bm}

\begin{document}

Zadanie t4.

W kartezjańskim ukladzie wspólrzednych (x, y) przedstawiono fragment paraboli, która jest

wykresem funkcji kwadratowej f (zobacz rysunek). Wierzcholek tej paraboli oraz punkty

przecipcia paraboli z osiami ukladu wspólrzednych maja obie wspólrzedne calkowite.

Zadanie \S 4.a. $(0-8$\}

Uzupe[nij ponizsze zdanie. Wpisz odpowiedni przedzia[w wykropkowanym miejscu

tak, aby zdanie bylo prawdziwe.

Zbiorem wszystkich rozwiazań nierówności $f(x)\geq 0$ jest przedzial

{\it Brudnopis}

Zadanie 94.2. (0-\S\} $\overline{1^{\mathrm{n}}\Delta \mathrm{g}\mathrm{g}}1$

Dokończ zdanie. Wybierz w[aściwq odpowied $\acute{\mathrm{z}}$ spośród podanych.

Funkcja kwadratowa f jest określona wzorem

A. $f(x)=-(x+1)^{2}-9$

B. $f(x)=-(x-1)^{2}+9$

C. $f(x)=-(x-1)^{2}-9$

D. $f(x)=-(x+1)^{2}+9$

{\it Brudnopis}

$\mathrm{M}\mathrm{M}\mathrm{A}\mathrm{P}-\mathrm{P}0_{-}100$

Strona 13 z30
\end{document}
