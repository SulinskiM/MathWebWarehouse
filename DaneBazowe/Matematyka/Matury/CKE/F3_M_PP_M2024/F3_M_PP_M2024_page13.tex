\documentclass[a4paper,12pt]{article}
\usepackage{latexsym}
\usepackage{amsmath}
\usepackage{amssymb}
\usepackage{graphicx}
\usepackage{wrapfig}
\pagestyle{plain}
\usepackage{fancybox}
\usepackage{bm}

\begin{document}

Zadanie 14.3. (0-\S\} $\overline{\mathrm{L}\mathfrak{W}\mathfrak{B}}$;

Dokończ zdanie. Wybierz w[aściwq odpowied $\acute{\mathrm{z}}$ spośród podanych.

Dla funkcji f prawdziwa jest równośč

A. $f(-4)=f(6)$

B. $f(-4)=f(5)$

C. $f(-4)=f(4)$

D. $f(-4)=f(7)$

{\it Brudnopis}

$\mathrm{Z}\mathrm{a}\mathrm{d}\mathrm{a}\mathfrak{n}i\mathrm{e}*4.4_{*}\{0\infty 2)$

Funkcje kwadratowe $g$ oraz $h$ sa określone za pomocq funkcji $f$ (zobacz rysunek na

stronie 13) nastepujqco: $g(x)=f(x+3), h(x)=f(-x).$

Na rysunkach A-F przedstawiono, w kartezjańskim uk[adzie wspólrz9dnych $(x,y),$

fragmenty wykresów róznych funkcji-w tym fragment wykresu funkcji $g$ oraz fragment

wykresu funkcji $h.$

Uzupelnij tabele. $\mathrm{K}\mathrm{a}\dot{\mathrm{z}}$ dej z funkcji $g$ oraz $h$ przyporzqdkuj fragmentjej wykresu.

Wpisz w $\mathrm{k}\mathrm{a}\dot{\mathrm{z}}$ dq pustq komórke tabeli w[aściwq odpowied $\acute{\mathrm{z}}$, wybranq spośród

oznaczonych literami A-F.
\begin{center}
\begin{tabular}{|l|l|}
\hline
\multicolumn{1}{|l|}{Fragment wykresu funkcji $y=g(x)$ przedstawiono na rysunku}&	\multicolumn{1}{|l|}{}	\\
\hline
\multicolumn{1}{|l|}{Fragment wykresu funkcji $y=h(x)$ przedstawiono na rysunku}&	\multicolumn{1}{|l|}{}	\\
\hline
\end{tabular}

\end{center}
Strona 14 z30

$\mathrm{M}\mathrm{M}\mathrm{A}\mathrm{P}-\mathrm{P}0_{-}100$
\end{document}
