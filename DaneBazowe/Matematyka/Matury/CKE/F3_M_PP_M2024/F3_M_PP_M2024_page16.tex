\documentclass[a4paper,12pt]{article}
\usepackage{latexsym}
\usepackage{amsmath}
\usepackage{amssymb}
\usepackage{graphicx}
\usepackage{wrapfig}
\pagestyle{plain}
\usepackage{fancybox}
\usepackage{bm}

\begin{document}

Zadanie $\mathrm{t}7. (0-2)$

Ciqg arytmetyczny $(a_{n})$ jest określony dla $\mathrm{k}\mathrm{a}\dot{\mathrm{z}}$ dej liczby naturalnej $n\geq 1$. Trzeci wyraz

tego ciqgu jest równy $(-1)$, a suma piptnastu poczqtkowych kolejnych wyrazów tego ciqgu

jest równa $(-165).$

Oblicz róznice tego ciqgu. Zapisz obliczenia.

1

$1-$

1

$\mathrm{M}\mathrm{M}\mathrm{A}\mathrm{P}-\mathrm{P}0_{-}100$

Strona 17 z30
\end{document}
