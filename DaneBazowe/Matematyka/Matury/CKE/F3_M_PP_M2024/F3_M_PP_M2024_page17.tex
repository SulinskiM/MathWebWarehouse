\documentclass[a4paper,12pt]{article}
\usepackage{latexsym}
\usepackage{amsmath}
\usepackage{amssymb}
\usepackage{graphicx}
\usepackage{wrapfig}
\pagestyle{plain}
\usepackage{fancybox}
\usepackage{bm}

\begin{document}

Zadanie $\mathrm{f}8_{*}(0-2)$

$\mathrm{W}$ kartezjańskim ukladzie wspólrzednych $(x,y)$ zaznaczono kqt o mierze $\alpha$ taki, $\dot{\mathrm{z}}\mathrm{e}$

tg $\alpha=-3$ oraz $90^{\mathrm{o}}<\alpha<180^{\mathrm{o}}$ (zobacz rysunek).

Uzupe[nij zdanie. Wybierz dwie w[aściwe odpowiedzi spośród oznaczonych literami

A-F i wpisz te litery w wykropkowanych miejscach.

Prawdziwe sq zalezności:

oraz

A. $\sin\alpha<0$

B. $\sin\alpha\cdot\cos\alpha<0$

C. $\sin\alpha\cdot\cos\alpha>0$

D. $\cos\alpha>0$

E. $\displaystyle \sin\alpha=-\frac{1}{3}\cos\alpha$

$\mathrm{F}.\ \sin\alpha=-3\cos\alpha$

$Brudno\sqrt{}is -$

Zadanie \S 9. $(0-\not\in) \overline{\llcorner \mathfrak{B}\mathrm{g}}$;

Dokończ zdanie. Wybierz w[aściwq odpowied $\acute{\mathrm{z}}$ spośród podanych.

Liczba $\sin^{3}20^{\mathrm{o}}+\cos^{2}20^{\mathrm{o}}\cdot\sin 20^{\mathrm{o}}$ jest równa

A. $\cos 20^{\mathrm{o}}$

B. $\sin 20^{\mathrm{o}}$

C. $\mathrm{t}\mathrm{g}20^{\mathrm{o}}$

$\mathrm{D}.\ \sin 20^{\mathrm{o}}\cdot\cos 20^{\mathrm{o}}$

{\it Brudnopis}

Strona 18 z30

$\mathrm{M}\mathrm{M}\mathrm{A}\mathrm{P}-\mathrm{P}0_{-}100$
\end{document}
