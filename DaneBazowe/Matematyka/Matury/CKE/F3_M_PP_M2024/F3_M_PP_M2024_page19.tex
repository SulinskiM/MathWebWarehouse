\documentclass[a4paper,12pt]{article}
\usepackage{latexsym}
\usepackage{amsmath}
\usepackage{amssymb}
\usepackage{graphicx}
\usepackage{wrapfig}
\pagestyle{plain}
\usepackage{fancybox}
\usepackage{bm}

\begin{document}

Zadanie 22. (0-1)

$\overline{\mathrm{L}\mathrm{E}\mathrm{g}\mathrm{g}}$;

$\mathrm{W}$ trójkqcie $ABC$, wpisanym w $\mathrm{o}\mathrm{k}\mathrm{r}_{\mathrm{c}}$]$\mathrm{g}$ o środku w punkcie $S$, kqt $ACB$ ma miar9 $42^{\mathrm{o}}$

(zobacz rysunek).
\begin{center}
\includegraphics[width=67.968mm,height=65.076mm]{./F3_M_PP_M2024_page19_images/image001.eps}
\end{center}
{\it C}

$42^{\mathrm{o}}$  {\it S}

{\it A}

{\it B}

Dokończ zdanie. Wybierz w[aściwq odpowied $\acute{\mathrm{z}}$ spośród podanych.

Miara kqta ostrego BAS jest równa

A. $42^{\mathrm{o}}$

B. $45^{\mathrm{o}}$

C. $48^{\mathrm{o}}$

D. $69^{\mathrm{o}}$

$Brudno\sqrt{}is -$

Zadanie 23. (0-\S) $\overline{[similar]\alpha \mathrm{g}\mathrm{u}}$'

$\mathrm{W}$ kartezjańskim ukladzie wspólrzednych $(x,y)$ proste $k$ oraz $l$ sq określone równaniami

{\it k}:

$y=(m+1)x+7$

{\it l}:

$y=-2x+7$

Dokończ zdanie. Wybierz w[aściwq odpowied $\acute{\mathrm{z}}$ spośród podanych.

Proste k oraz l sa prostopadle, gdy liczba m jest równa

A. $(-\displaystyle \frac{1}{2})$

B. -21

C. $(-3)$

D. l

{\it Brudnopis}

Strona 20 z30

$\mathrm{M}\mathrm{M}\mathrm{A}\mathrm{P}-\mathrm{P}0_{-}100$
\end{document}
