\documentclass[a4paper,12pt]{article}
\usepackage{latexsym}
\usepackage{amsmath}
\usepackage{amssymb}
\usepackage{graphicx}
\usepackage{wrapfig}
\pagestyle{plain}
\usepackage{fancybox}
\usepackage{bm}

\begin{document}

Zadanie $25_{\mathrm{r}}$

Wysokośč graniastoslupa prawidlowego sześciokatnego jest równa 6 (zobacz rysunek).

Pole podstawy tego graniastoslupa jest równe $15\sqrt{3}.$
\begin{center}
\includegraphics[width=62.328mm,height=71.016mm]{./F3_M_PP_M2024_page21_images/image001.eps}
\end{center}
I

I

I

I

I

I

I

I

I

I

I

I

I

$\underline{\mathrm{I}}$

I

I

I

I

I

I

I

I

I

I

I

I

6

Zadanie 25.1. $(\emptyset\infty \mathrm{e}$\} $\overline{\llcorner \mathfrak{B}\mathrm{g}}$;

Dokończ zdanie. Wybierz w[aściwq odpowied $\acute{\mathrm{z}}$ spośród podanych.

Pole $|\mathrm{e}\mathrm{d}\mathrm{n}\mathrm{e}1$ ściany bocznej tego graniastoslupa jest równe

A. $36\sqrt{10}$

B. 60

C. $6\sqrt{10}$

D. 360

$Brudno\sqrt{}is$

Strona 22 z30

$\mathrm{M}\mathrm{M}\mathrm{A}\mathrm{P}-\mathrm{P}0_{-}100$
\end{document}
