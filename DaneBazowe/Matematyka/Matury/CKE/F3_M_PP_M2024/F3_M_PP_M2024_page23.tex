\documentclass[a4paper,12pt]{article}
\usepackage{latexsym}
\usepackage{amsmath}
\usepackage{amssymb}
\usepackage{graphicx}
\usepackage{wrapfig}
\pagestyle{plain}
\usepackage{fancybox}
\usepackage{bm}

\begin{document}

Zadanie 26. $(0-\not\in)$

Ostroslup $F_{1}$ jest podobny do ostroslupa $F_{2}.$

Obj9tośč ostros1upa $F_{1}$ jest równa 64.

Obj9tośč ostros1upa $F_{2}$ jest równa 512.

Uzupe[nij ponizsze zdanie. Wpisz odpowiedniq liczbe w wykropkowanym miejscu tak,

aby zdanie by[o prawdziwe.

Stosunek pola powierzchni calkowitej ostroslupa $F_{2}$ do pola powierzchni calkowitej

ostroslupa $F_{1}$ jest równy

{\it Brudnopis}

$-|1\mathrm{i}\mathrm{i}-$

$| 1$

$\mathrm{Z}\mathrm{a}\mathrm{d}\mathrm{a}\mathrm{n}\dot{\mathfrak{x}}\mathrm{e}Z7$. (0-{\$}) $\square \sqsupset\sqsupset 2$;

Rozwazamy wszystkie kody czterocyfrowe utworzone tylko z cyfr 1, 3, 6, 8, przy czym

w $\mathrm{k}\mathrm{a}\dot{\mathrm{z}}$ dym kodzie $\mathrm{k}\mathrm{a}\dot{\mathrm{z}}$ da z tych cyfr wystppuje dokladnie jeden raz.

Dokończ zdanie. Wybierz w[aściwq odpowied $\acute{\mathrm{z}}$ spośród podanych.

Liczba wszystkich takich kodówjest równa

A. 4

B. 10

C. 24

D. 16

{\it Brudnopis}

Strona 24 z30

$\mathrm{M}\mathrm{M}\mathrm{A}\mathrm{P}-\mathrm{P}0_{-}100$
\end{document}
