\documentclass[a4paper,12pt]{article}
\usepackage{latexsym}
\usepackage{amsmath}
\usepackage{amssymb}
\usepackage{graphicx}
\usepackage{wrapfig}
\pagestyle{plain}
\usepackage{fancybox}
\usepackage{bm}

\begin{document}

Zadanie 28, (0-{\$}) $\overline{\mathrm{L}\mathfrak{B}\mathfrak{B}}$'

$\acute{\mathrm{S}}$ rednia arytmetyczna trzech liczb: $a, b, c$, jest równa 9.

Dokończ zdanie. Wybierz w[aściwq odpowied $\acute{\mathrm{z}}$ spośród podanych.

$\acute{\mathrm{S}}$ rednia arytmetyczna sześciu liczb: $a, a, b, b, c, c$, jest równa

A. 9

B. 6

C. 4,5

D. 18

{\it Brudnopis}

$\mathrm{Z}\mathrm{a}\mathrm{d}\mathrm{a}\mathrm{n}\mathrm{i}^{\vee}\mathrm{e}29. (0\infty 1\} \square \sqsupset\supset\supset 1$

Na diagramie przedstawiono wyniki sprawdzianu z matematyki w pewnej klasie maturalnej.

Na osi poziomej podano oceny, które uzyskali uczniowie tej klasy, a na osi pionowej podano

liczbe uczniów, którzy otrzymali danq ocen9.

8

7

6

liczba

uczniów

5

4
\begin{center}
\includegraphics[width=105.060mm,height=57.144mm]{./F3_M_PP_M2024_page24_images/image001.eps}
\end{center}
3

2

1

0

1 2

3 4

5 6

ocena

Dokończ zdanie. Wybierz w[aściwq odpowied $\acute{\mathrm{z}}$ spośród podanych.

Mediana ocen uzyskanych z tego sprawdzianu przez uczniów tej klasy jest równa

A. 4,5

B. 4

C. $3_{r}5$

D. 3

{\it Brudnopis}

$\mathrm{M}\mathrm{M}\mathrm{A}\mathrm{P}-\mathrm{P}0_{-}100$

Strona 25 z30
\end{document}
