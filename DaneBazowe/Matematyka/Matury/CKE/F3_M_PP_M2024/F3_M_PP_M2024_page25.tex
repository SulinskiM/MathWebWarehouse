\documentclass[a4paper,12pt]{article}
\usepackage{latexsym}
\usepackage{amsmath}
\usepackage{amssymb}
\usepackage{graphicx}
\usepackage{wrapfig}
\pagestyle{plain}
\usepackage{fancybox}
\usepackage{bm}

\begin{document}

Zadanie $30_{\mathrm{L}}\{0-2$)

Dany jest piecioelementowy zbiór $K=\{5$, 6, 7, 8, 9$\}$. Wylosowanie $\mathrm{k}\mathrm{a}\dot{\mathrm{z}}$ dej liczby z tego

zbioru jestjednakowo prawdopodobne. Ze zbioru $K$ losujemy ze zwracaniem kolejno dwa

razy po jednej liczbie i zapisujemy je w kolejności losowania.

Oblicz prawdopodobieństwo zdarzenia $A$ polegajqcego na tym, $\dot{\mathrm{z}}\mathrm{e}$ suma

wylosowanych liczb jest liczbq parzystq. Zapisz obliczenia.

1

1

Strona 26 z30

$\mathrm{M}\mathrm{M}\mathrm{A}\mathrm{P}-\mathrm{P}0_{-}100$
\end{document}
