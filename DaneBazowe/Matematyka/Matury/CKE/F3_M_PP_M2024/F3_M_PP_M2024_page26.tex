\documentclass[a4paper,12pt]{article}
\usepackage{latexsym}
\usepackage{amsmath}
\usepackage{amssymb}
\usepackage{graphicx}
\usepackage{wrapfig}
\pagestyle{plain}
\usepackage{fancybox}
\usepackage{bm}

\begin{document}

Zadanie $38_{*}(0-4)$

$\mathrm{W}$ schronisku dla zwierzat, na $\mathrm{p}$\}askiej powierzchni, nale $\dot{\mathrm{z}}\mathrm{y}$ zbudowač ogrodzenie z siatki

wydzielajqce trzy identyczne wybiegi o $\underline{\mathrm{w}\mathrm{s}\mathrm{p}\text{ó} \mathrm{l}\mathrm{n}\mathrm{v}\mathrm{c}\mathrm{h}}$ ścianach wewnptrznych.

Podstawq$\mathrm{k}\mathrm{a}\dot{\mathrm{z}}$ dego z tych trzech wybiegów jest $\mathrm{p}\mathrm{r}\mathrm{o}\mathrm{s}\mathrm{t}\mathrm{o}\mathrm{k}_{\mathrm{c}1}\mathrm{t}$ (jak pokazano na rysunku).

Do wykonania tego ogrodzenia nale $\dot{\mathrm{z}}\mathrm{y}\mathrm{z}\mathrm{u}\dot{\mathrm{z}}$ yč 36 metrów biezqcych siatki.

Schematyczny rysunek trzech wybiegów (widok z góry).

Liniq przerywanq zaznaczono siatk9.
\begin{center}
\includegraphics[width=107.952mm,height=6.804mm]{./F3_M_PP_M2024_page26_images/image001.eps}
\end{center}
{\it y y  y}

$\chi$

$\iota \Gamma$ 1

I

I I I

I

$1 \mathrm{I}$ 1

I

I I I

I

I I I

I

I I I

I

I bi l bi 2 I bi 3 I

wybieg l. 1 wybieg 2. wybieg 3.

I I I

I

1 I I

1

I I I

I

I I I

I

1 I I

I

I I 1

I

$1 \llcorner$ 1

Oblicz wymiary $x$ oraz $y$ jednego wybiegu, przy których suma pól podstaw tych

trzech wybiegów bedzie najwieksza. $\mathrm{W}$ obliczeniach pomiń szerokośč wejścia na

$\mathrm{k}\mathrm{a}\dot{\mathrm{z}}\mathrm{d}\mathrm{y}$ z wybiegów. Zapisz obliczenia.

$\mathrm{M}\mathrm{M}\mathrm{A}\mathrm{P}-\mathrm{P}0_{-}100$

Strona 27 z30
\end{document}
