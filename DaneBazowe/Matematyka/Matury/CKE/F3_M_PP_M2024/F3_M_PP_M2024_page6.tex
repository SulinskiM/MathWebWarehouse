\documentclass[a4paper,12pt]{article}
\usepackage{latexsym}
\usepackage{amsmath}
\usepackage{amssymb}
\usepackage{graphicx}
\usepackage{wrapfig}
\pagestyle{plain}
\usepackage{fancybox}
\usepackage{bm}

\begin{document}

Zadanie 6. (0-8) $\overline{\mathrm{L}\mathfrak{W}\mathfrak{B}}$;

Dokończ zdanie. Wybierz wlaściwq odpowied $\acute{\mathrm{z}}$ spośród podanych.

Zbiorem wszystkich rozwiqzań nierówności

$1-\displaystyle \frac{3}{2}x<\frac{2}{3}-x$

jest przedzial

A. $(-\displaystyle \infty,-\frac{2}{3})$

B.(-$\infty$,-23)

C. $(-\displaystyle \frac{2}{3},+\infty)$

D. $(\displaystyle \frac{2}{3},+\infty)$

{\it Brudnopis}

-

Zadanie $7_{\mathrm{r}}\{\emptyset\infty 4$) $\square \sqsupset\infty 3*\nearrow$

Dokończ zdanie. Wybierz w[aściwq odpowied $\acute{\mathrm{z}}$ spośród podanych.

Równanie $\displaystyle \frac{x+1}{(x+2)(x-3)}=0$ w zbiorze liczb rzeczywistych

A. nie ma rozwiqzania.

B. ma dokladnie jedno rozwiqzanie: $(-1).$

C. ma dokladnie dwa rozwiqzania: $(-2)$ oraz 3.

D. ma dokladnie trzy rozwiqzania: $(-1), (-2)$ oraz 3.

{\it Brudnopis}

$\mathrm{M}\mathrm{M}\mathrm{A}\mathrm{P}-\mathrm{P}0_{-}100$

Strona 7 z30
\end{document}
