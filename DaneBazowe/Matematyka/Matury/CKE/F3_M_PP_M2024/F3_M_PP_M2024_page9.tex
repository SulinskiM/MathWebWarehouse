\documentclass[a4paper,12pt]{article}
\usepackage{latexsym}
\usepackage{amsmath}
\usepackage{amssymb}
\usepackage{graphicx}
\usepackage{wrapfig}
\pagestyle{plain}
\usepackage{fancybox}
\usepackage{bm}

\begin{document}

Zadanie 10. (0-{\$}) $\overline{\mathrm{L}\mathfrak{B}\mathfrak{B}}$'

$\mathrm{W}$ paz'dzierniku 2022 roku za1ozono dwa sady, w których posadzono 1qcznie 1960 drzew.

Po roku stwierdzono, $\dot{\mathrm{z}}\mathrm{e}$ uschlo 5\% drzew w pierwszym sadzie i 10\% drzew w drugim

sadzie. Uschnipte drzewa usunieto, a nowych nie dosadzano.

Liczba drzew, które pozostaly w drugim sadzie, stanowila 60\% 1iczby drzew, które

pozostaly w pierwszym sadzie.

Niech $x$ oraz $\mathrm{y}$ oznaczaja liczby drzew posadzonych- odpowiednio-w pierwszym

i drugim sadzie.

Dokończ zdanie. Wybierz wlaściwq odpowied $\acute{\mathrm{z}}$ spośród podanych.

Ukladem równań, którego poprawne rozwiqzanie prowadzi do obliczenia liczby $x$ drzew

posadzonych w pierwszym sadzie oraz liczby $y$ drzew posadzonych w drugim sadzie, jest

A. 

B. 

C. 

D. 

$Brudno\sqrt{}is$

Strona 10 z30

$\mathrm{M}\mathrm{M}\mathrm{A}\mathrm{P}-\mathrm{P}0_{-}100$
\end{document}
