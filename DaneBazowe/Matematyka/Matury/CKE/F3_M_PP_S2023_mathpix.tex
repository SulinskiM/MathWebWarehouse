% This LaTeX document needs to be compiled with XeLaTeX.
\documentclass[10pt]{article}
\usepackage[utf8]{inputenc}
\usepackage{ucharclasses}
\usepackage{amsmath}
\usepackage{amsfonts}
\usepackage{amssymb}
\usepackage[version=4]{mhchem}
\usepackage{stmaryrd}
\usepackage{graphicx}
\usepackage[export]{adjustbox}
\graphicspath{ {./images/} }
\usepackage[fallback]{xeCJK}
\usepackage{polyglossia}
\usepackage{fontspec}
\IfFontExistsTF{Noto Serif CJK KR}
{\setCJKmainfont{Noto Serif CJK KR}}
{\IfFontExistsTF{Apple SD Gothic Neo}
  {\setCJKmainfont{Apple SD Gothic Neo}}
  {\IfFontExistsTF{UnDotum}
    {\setCJKmainfont{UnDotum}}
    {\setCJKmainfont{Malgun Gothic}}
}}

\setmainlanguage{polish}
\setotherlanguages{thai}
\IfFontExistsTF{Noto Serif Thai}
{\newfontfamily\thaifont{Noto Serif Thai}}
{\IfFontExistsTF{Thonburi}
  {\newfontfamily\thaifont{Thonburi}}
  {\IfFontExistsTF{FreeSerif}
    {\newfontfamily\thaifont{FreeSerif}}
    {\IfFontExistsTF{Tahoma}
      {\newfontfamily\thaifont{Tahoma}}
      {\newfontfamily\thaifont{Arial Unicode MS}}
}}}
\IfFontExistsTF{CMU Serif}
{\newfontfamily\lgcfont{CMU Serif}}
{\IfFontExistsTF{DejaVu Sans}
  {\newfontfamily\lgcfont{DejaVu Sans}}
  {\newfontfamily\lgcfont{Georgia}}
}
\setDefaultTransitions{\lgcfont}{}
\setTransitionsFor{Thai}{\thaifont}{\lgcfont}

\begin{document}
CENTRALNA\\
KOMISJA\\
EGZAMINACYJNA

Arkusz zawiera informacje prawnie chronione do momentu rozpoczęcia egzaminu.

\section*{Miejsce na naklejkę.}
 Sprawdż, czy kod na naklejce to M-100.Jeżeli tak - przyklej naklejkę. Jeżeli nie - zgłoś to nauczycielowi.

\section*{Egzamin maturalny}
\section*{MATEMATYKA}
\section*{Poziom podstawowy}
Symbol arkusza мMAP-P0-100-2308

\section*{Data: 22 sierpnia 2023 r.}
 Godzina rozpoczecia: 9:00 CZAS trwania: \(\mathbf{1 8 0}\) minut\section*{WYPEŁNIA ZESPÓŁ NADZORUJĄCY}
Uprawnienia zdającego do:\\
dostosowania zasad oceniania dostosowania w zw. z dyskalkulią nieprzenoszenia zaznaczeń na kartę.

\section*{Liczba punktów do uzyskania: 46}
Przed rozpoczęciem pracy z arkuszem egzaminacyjnym

\begin{enumerate}
  \item Sprawdź, czy nauczyciel przekazał Ci właściwy arkusz egzaminacyjny, tj. arkusz we właściwej formule, z właściwego przedmiotu na właściwym poziomie.
  \item Jeżeli przekazano Ci niewłaściwy arkusz - natychmiast zgłoś to nauczycielowi. Nie rozrywaj banderol.
  \item Jeżeli przekazano Ci właściwy arkusz - rozerwij banderole po otrzymaniu takiego polecenia od nauczyciela. Zapoznaj się z instrukcją na stronie 2.\\
\includegraphics[max width=\textwidth, center]{2025_02_10_c6c2d6cd217fbcdc7465g-02}
\end{enumerate}

\section*{Instrukcja dla zdającego}
\begin{enumerate}
  \item Sprawdź, czy arkusz egzaminacyjny zawiera 31 stron (zadania 1-33). Ewentualny brak zgłoś przewodniczącemu zespołu nadzorującego egzamin.
  \item Na pierwszej stronie arkusza oraz na karcie odpowiedzi wpisz swój numer PESEL i przyklej naklejkę z kodem.
  \item Symbol zamieszczony w nagłówku zadania oznacza, że rozwiązanie zadania zamkniętego musisz przenieść na kartę odpowiedzi.
  \item Odpowiedzi do zadań zamkniętych zaznacz na karcie odpowiedzi w części karty przeznaczonej dla zdającego. Zamaluj \(\square\) pola do tego przeznaczone. Błędne zaznaczenie otocz kółkiem © i zaznacz właściwe.
  \item Pamiętaj, że pominięcie argumentacji lub istotnych obliczeń w rozwiązaniu zadania otwartego może spowodować, że za to rozwiązanie nie otrzymasz pełnej liczby punktów.
  \item Rozwiązania zadań i odpowiedzi wpisuj w miejscu na to przeznaczonym.
  \item Pisz czytelnie i używaj tylko długopisu lub pióra z czarnym tuszem lub atramentem.
  \item Nie używaj korektora, a błędne zapisy wyraźnie przekreśl.
  \item Pamiętaj, że zapisy w brudnopisie nie będą oceniane.
  \item Możesz korzystać z Wybranych wzorów matematycznych, cyrkla i linijki oraz kalkulatora prostego. Upewnij się, czy przekazano Ci broszurę z okładką taką jak widoczna poniżej.\\
\includegraphics[max width=\textwidth, center]{2025_02_10_c6c2d6cd217fbcdc7465g-02(1)}
\end{enumerate}

\section*{Zadania egzaminacyjne są wydrukowane na następnych stronach.}
Zadanie 1. (0-1) птт\\
Dana jest nierówność

\[
|x-5|<2
\]

Na którym rysunku poprawnie zaznaczono na osi liczbowej zbiór wszystkich liczb rzeczywistych spełniających powyższą nierówność? Wybierz właściwą odpowiedź spośród podanych.\\
A.\\
\includegraphics[max width=\textwidth, center]{2025_02_10_c6c2d6cd217fbcdc7465g-04(1)}\\
B.\\
\includegraphics[max width=\textwidth, center]{2025_02_10_c6c2d6cd217fbcdc7465g-04}\\
C.\\
\includegraphics[max width=\textwidth, center]{2025_02_10_c6c2d6cd217fbcdc7465g-04(2)}\\
D.\\
\includegraphics[max width=\textwidth, center]{2025_02_10_c6c2d6cd217fbcdc7465g-04(3)}

Zadanie 2. (0-1) 뚬\\
Dokończ zdanie. Wybierz właściwą odpowiedź spośród podanych.\\
Liczba \(3 \sqrt{45}-\sqrt{20}\) jest równa\\
A. \((7 \cdot 5)^{\frac{1}{2}}\)\\
B. \(5^{\frac{1}{2}}\)\\
C. 7\\
D. \(7 \cdot 5^{\frac{1}{2}}\)

\begin{center}
\begin{tabular}{|c|c|c|c|c|c|c|c|c|c|c|c|c|c|c|c|c|c|c|c|c|c|c|c|c|}
\hline
\multicolumn{5}{|l|}{Brudnopis} &  &  &  &  &  &  &  &  &  &  &  &  &  &  &  &  &  &  &  &  \\
\hline
 &  &  &  &  &  &  &  &  &  &  &  &  &  &  &  &  &  &  &  &  &  &  &  &  \\
\hline
 &  &  &  &  &  &  &  &  &  &  &  &  &  &  &  &  &  &  &  &  &  &  &  &  \\
\hline
 &  &  &  &  &  &  &  &  &  &  &  &  &  &  &  &  &  &  &  &  &  &  &  &  \\
\hline
 &  &  &  &  &  &  &  &  &  &  &  &  &  &  &  &  &  &  &  &  &  &  &  &  \\
\hline
 &  &  &  &  &  &  &  &  &  &  &  &  &  &  &  &  &  &  &  &  &  &  &  &  \\
\hline
 &  &  &  &  &  &  &  &  &  &  &  &  &  &  &  &  &  &  &  &  &  &  &  &  \\
\hline
 &  &  &  &  &  &  &  &  &  &  &  &  &  &  &  &  &  &  &  &  &  &  &  &  \\
\hline
 &  &  &  &  &  &  &  &  &  &  &  &  &  &  &  &  &  &  &  &  &  &  &  &  \\
\hline
 &  &  &  &  &  &  &  &  &  &  &  &  &  &  &  &  &  &  &  &  &  &  &  &  \\
\hline
 &  &  &  &  &  &  &  &  &  &  &  &  &  &  &  &  &  &  &  &  &  &  &  &  \\
\hline
 &  &  &  &  &  &  &  &  &  &  &  &  &  &  &  &  &  &  &  &  &  &  &  &  \\
\hline
\end{tabular}
\end{center}

\section*{Zadanie 3. (0-1) 뚜}
Dokończ zdanie. Wybierz właściwą odpowiedź spośród podanych.\\
Liczba \(\log _{25} 1-\frac{1}{2} \log _{25} 5\) jest równa\\
A. \(\left(-\frac{1}{4}\right)\)\\
B. \(\left(-\frac{1}{2}\right)\)\\
C. \(\frac{1}{4}\)\\
D. \(\frac{1}{2}\)\\
\includegraphics[max width=\textwidth, center]{2025_02_10_c6c2d6cd217fbcdc7465g-05}

Zadanie 4. (0-2)\\
Wykaż, że dla każdej liczby naturalnej \(n \geq 1\) liczba \(3 n^{3}+18 n^{2}+15 n\) jest podzielna przez 6.\\
\includegraphics[max width=\textwidth, center]{2025_02_10_c6c2d6cd217fbcdc7465g-06}

\section*{Zadanie 5. (0-1) 뚱}
Dokończ zdanie. Wybierz właściwą odpowiedź spośród podanych.\\
Wartość wyrażenia \(\frac{3^{-1}}{\left(-\frac{1}{9}\right)^{-2}} \cdot 81\) jest równa\\
A. \(\frac{1}{3}\)\\
B. \(\left(-\frac{1}{3}\right)\)\\
C. 3\\
D. \((-3)\)

\begin{center}
\begin{tabular}{|c|c|c|c|c|c|c|c|c|c|c|c|c|c|c|c|c|c|c|c|c|c|}
\hline
\multicolumn{4}{|l|}{Brudnopis} &  &  &  &  &  &  &  &  &  &  &  &  &  &  &  &  &  &  \\
\hline
 &  &  &  &  &  &  &  &  &  &  &  &  &  &  &  &  &  &  &  &  &  \\
\hline
 &  &  &  &  &  &  &  &  &  &  &  &  &  &  &  &  &  &  &  &  &  \\
\hline
 &  &  &  &  &  &  &  &  &  &  &  &  &  &  &  &  &  &  &  &  &  \\
\hline
 &  &  &  &  &  &  &  &  &  &  &  &  &  &  &  &  &  &  &  &  &  \\
\hline
 &  &  &  &  &  &  &  &  &  &  &  &  &  &  &  &  &  &  &  &  &  \\
\hline
 &  &  &  &  &  &  &  &  &  &  &  &  &  &  &  &  &  &  &  &  &  \\
\hline
 &  &  &  &  &  &  &  &  &  &  &  &  &  &  &  &  &  &  &  &  &  \\
\hline
 &  &  &  &  &  &  &  &  &  &  &  &  &  &  &  &  &  &  &  &  &  \\
\hline
 &  &  &  &  &  &  &  &  &  &  &  &  &  &  &  &  &  &  &  &  &  \\
\hline
 &  &  &  &  &  &  &  &  &  &  &  &  &  &  &  &  &  &  &  &  &  \\
\hline
 &  &  &  &  &  &  &  &  &  &  &  &  &  &  &  &  &  &  &  &  &  \\
\hline
 &  &  &  &  &  &  &  &  &  &  &  &  &  &  &  &  &  &  &  &  &  \\
\hline
 &  &  &  &  &  &  &  &  &  &  &  &  &  &  &  &  &  &  &  &  &  \\
\hline
\end{tabular}
\end{center}

Zadanie 6. (0-1)\\
Dokończ zdanie. Wybierz właściwą odpowiedź spośród podanych.\\
Wartość wyrażenia \((2-\sqrt{3})^{2}-(\sqrt{3}-2)^{2}\) jest równa\\
A. \((-2 \sqrt{3})\)\\
B. 0\\
C. 6\\
D. \(8 \sqrt{3}\)\\
\includegraphics[max width=\textwidth, center]{2025_02_10_c6c2d6cd217fbcdc7465g-07}

\section*{Zadanie 7. (0-1) 띰}
Dokończ zdanie. Wybierz właściwą odpowiedź spośród podanych.\\
Dla każdej liczby rzeczywistej \(x\) różnej od 0 wartość wyrażenia \(\frac{1}{2 x}-x\) jest równa wartości wyrażenia\\
A. \(\frac{1}{x}\)\\
B. \(\frac{1-x}{2 x}\)\\
C. \(\frac{1-2 x^{2}}{2 x}\)\\
D. \(-\frac{1}{2 x}\)

\begin{center}
\begin{tabular}{|c|c|c|c|c|c|c|c|c|c|c|c|c|c|c|c|c|c|c|c|c|c|c|c|}
\hline
\multicolumn{5}{|l|}{Brudnopis} &  &  &  &  &  &  &  &  &  &  &  &  &  &  &  &  &  &  &  \\
\hline
 &  &  &  &  &  &  &  &  &  &  &  &  &  &  &  &  &  &  &  &  &  &  &  \\
\hline
 &  &  &  &  &  &  &  &  &  &  &  &  &  &  &  &  &  &  &  &  &  &  &  \\
\hline
 &  &  &  &  &  &  &  &  &  &  &  &  &  &  &  &  &  &  &  &  &  &  &  \\
\hline
 &  &  &  &  &  &  &  &  &  &  &  &  &  &  &  &  &  &  &  &  &  &  &  \\
\hline
 &  &  &  &  &  &  &  &  &  &  &  &  &  &  &  &  &  &  &  &  &  &  &  \\
\hline
 &  &  &  &  &  &  &  &  &  &  &  &  &  &  &  &  &  &  &  &  &  &  &  \\
\hline
 &  &  &  &  &  &  &  &  &  &  &  &  &  &  &  &  &  &  &  &  &  &  &  \\
\hline
 &  &  &  &  &  &  &  &  &  &  &  &  &  &  &  &  &  &  &  &  &  &  &  \\
\hline
 &  &  &  &  &  &  &  &  &  &  &  &  &  &  &  &  &  &  &  &  &  &  &  \\
\hline
 &  &  &  &  &  &  &  &  &  &  &  &  &  &  &  &  &  &  &  &  &  &  &  \\
\hline
\(\square\) &  &  &  &  &  &  &  &  &  &  &  &  &  &  &  &  &  &  &  &  &  &  &  \\
\hline
\end{tabular}
\end{center}

\section*{Zadanie 8. (0-1)}
Dokończ zdanie. Wybierz właściwą odpowiedź spośród podanych.\\
Równanie \(\frac{\left(x^{2}-3 x\right)\left(x^{2}+1\right)}{x^{2}-25}=0 \mathrm{w}\) zbiorze liczb rzeczywistych ma dokładnie\\
A. jedno rozwiązanie.\\
B. dwa rozwiązania.\\
C. trzy rozwiązania.\\
D. cztery rozwiązania.

\begin{center}
\begin{tabular}{|c|c|c|c|c|c|c|c|c|c|c|c|c|c|c|c|c|c|c|c|c|c|c|c|c|c|c|}
\hline
\multicolumn{5}{|l|}{Brudnopis} &  &  &  &  &  &  &  &  &  &  &  &  &  &  &  &  &  &  &  &  &  &  \\
\hline
 &  &  &  &  &  &  &  &  &  &  &  &  &  &  &  &  &  &  &  &  &  &  &  &  &  &  \\
\hline
 &  &  &  &  &  &  &  &  &  &  &  &  &  &  &  &  &  &  &  &  &  &  &  &  &  &  \\
\hline
 &  &  &  &  &  &  &  &  &  &  &  &  &  &  &  &  &  &  &  &  &  &  &  &  &  &  \\
\hline
 &  &  &  &  &  &  &  &  &  &  &  &  &  &  &  &  &  &  &  &  &  &  &  &  &  &  \\
\hline
 &  &  &  &  &  &  &  &  &  &  &  &  &  &  &  &  &  &  &  &  &  &  &  &  &  &  \\
\hline
 &  &  &  &  &  &  &  &  &  &  &  &  &  &  &  &  &  &  &  &  &  &  &  &  &  &  \\
\hline
 &  &  &  &  &  &  &  &  &  &  &  &  &  &  &  &  &  &  &  &  &  &  &  &  &  &  \\
\hline
 &  &  &  &  &  &  &  &  &  &  &  &  &  &  &  &  &  &  &  &  &  &  &  &  &  &  \\
\hline
 &  &  &  &  &  &  &  &  &  &  &  &  &  &  &  &  &  &  &  &  &  &  &  &  &  &  \\
\hline
 &  &  &  &  &  &  &  &  &  &  &  &  &  &  &  &  &  &  &  &  &  &  &  &  &  &  \\
\hline
 &  &  &  &  &  &  &  &  &  &  &  &  &  &  &  &  &  &  &  &  &  &  &  &  &  &  \\
\hline
\end{tabular}
\end{center}

Zadanie 9. (0-3)\\
Rozwiąż równanie

\[
3 x^{3}-2 x^{2}-3 x+2=0
\]

\section*{Zapisz obliczenia.}
\begin{center}
\includegraphics[max width=\textwidth]{2025_02_10_c6c2d6cd217fbcdc7465g-09}
\end{center}

\section*{Zadanie 10. (0-1) 뚬}
Dokończ zdanie. Wybierz właściwą odpowiedź spośród podanych.\\
W kartezjańskim układzie współrzędnych \((x, y)\), punkt \((-8,6)\) jest punktem przecięcia prostych o równaniach\\
A. \(2 x+3 y=2 \quad\) i \(\quad-x+y=-14\).\\
B. \(3 x+2 y=-12 \quad\) i \(\quad 2 x+y=10\).\\
C. \(x+y=-2 \quad\) i \(\quad x-2 y=4\).\\
D. \(x-y=-14 \quad\) i \(\quad-2 x+y=22\).

\begin{center}
\begin{tabular}{|c|c|c|c|c|c|c|c|c|c|c|c|c|c|c|c|c|c|c|c|c|c|c|c|}
\hline
\multicolumn{5}{|l|}{Brudnopis} &  &  &  &  &  &  &  &  &  &  &  & - &  &  &  &  &  &  &  \\
\hline
 &  &  &  &  &  &  &  &  &  &  &  &  &  &  &  &  &  &  &  &  &  &  &  \\
\hline
 &  &  &  &  &  &  &  &  &  &  &  &  &  &  &  &  &  &  &  &  &  &  &  \\
\hline
 &  &  &  &  &  &  &  &  &  &  &  &  &  &  &  &  &  &  &  &  &  &  &  \\
\hline
 &  &  &  &  &  &  &  &  &  &  &  &  &  &  &  &  &  &  &  &  &  &  &  \\
\hline
 &  &  &  &  &  &  &  &  &  &  &  &  &  &  &  &  &  &  &  &  &  &  &  \\
\hline
 &  &  &  &  &  &  &  &  &  &  &  &  &  &  &  &  &  &  &  &  &  &  &  \\
\hline
 &  &  &  &  &  &  &  &  &  &  &  &  &  &  &  &  &  &  &  &  &  &  &  \\
\hline
 &  &  &  &  &  &  &  &  &  &  &  &  &  &  &  &  &  &  &  &  &  &  &  \\
\hline
 &  &  &  &  &  &  &  &  &  &  &  &  &  &  &  &  &  &  &  &  &  &  &  \\
\hline
 &  &  &  &  &  &  &  &  &  &  &  &  &  &  &  &  &  &  &  &  &  &  &  \\
\hline
 &  &  &  &  &  &  &  &  &  &  &  &  &  &  &  &  &  &  &  &  &  &  &  \\
\hline
\end{tabular}
\end{center}

\section*{Zadanie 11. (0-1)}
Miejscem zerowym funkcji liniowej \(f\) jest liczba 1 . Wykres tej funkcji przechodzi przez punkt \((-1,4)\).

Dokończ zdanie. Wybierz właściwą odpowiedź spośród podanych.\\
Wzór funkcji \(f\) ma postać\\
A. \(f(x)=-\frac{1}{2} x+1\)\\
B. \(f(x)=-\frac{1}{3} x+\frac{1}{3}\)\\
C. \(f(x)=-2 x+2\)\\
D. \(f(x)=-3 x+1\)\\
\includegraphics[max width=\textwidth, center]{2025_02_10_c6c2d6cd217fbcdc7465g-10}

\section*{Zadanie 12. (0-1) 뚬}
Funkcja \(f\) jest określona dla każdej liczby rzeczywistej \(x\) wzorem \(f(x)=\frac{x-k}{x^{2}+1}\), gdzie \(k\) jest pewną liczbą rzeczywistą. Ta funkcja spełnia warunek \(f(1)=2\).

Dokończ zdanie. Wybierz właściwą odpowiedź spośród podanych.\\
Wartość współczynnika \(k\) we wzorze tej funkcji jest równa\\
A. \((-3)\)\\
B. 3\\
C. \((-4)\)\\
D. 4

\begin{center}
\begin{tabular}{|c|c|c|c|c|c|c|c|c|c|c|c|c|c|c|c|c|c|c|c|c|}
\hline
\multicolumn{4}{|l|}{Brudnopis} &  & - & - &  &  & - &  &  &  &  & - &  &  & - &  & - & - \\
\hline
 &  &  &  &  &  &  &  &  &  &  &  &  &  &  &  &  &  &  &  &  \\
\hline
 &  &  &  &  &  &  &  &  &  &  &  &  &  &  &  &  &  &  &  &  \\
\hline
 &  &  &  &  &  &  &  &  &  &  &  &  &  &  &  &  &  &  &  &  \\
\hline
 &  &  &  &  &  &  &  &  &  &  &  &  &  &  &  &  &  &  &  &  \\
\hline
 &  &  &  &  &  &  &  &  &  &  &  &  &  &  &  &  &  &  &  &  \\
\hline
 &  &  &  &  &  &  &  &  &  &  &  &  &  &  &  &  &  &  &  &  \\
\hline
 &  &  &  &  &  &  &  &  &  &  &  &  &  &  &  &  &  &  &  &  \\
\hline
 &  &  &  &  &  &  &  &  &  &  &  &  &  &  &  &  &  &  &  &  \\
\hline
 &  &  &  &  &  &  &  &  &  &  &  &  &  &  &  &  &  &  &  &  \\
\hline
 &  &  &  &  &  &  &  &  &  &  &  &  &  &  &  &  &  &  &  &  \\
\hline
 &  &  &  &  &  &  &  &  &  &  &  &  &  &  &  &  &  &  &  &  \\
\hline
\end{tabular}
\end{center}

\section*{Zadanie 13. (0-1)}
Funkcja kwadratowa \(f\) jest określona wzorem \(f(x)=(x-13)^{2}-256\). Jednym z miejsc zerowych tej funkcji jest liczba ( -3 ).

Dokończ zdanie. Wybierz właściwą odpowiedź spośród podanych.\\
Drugim miejscem zerowym funkcji \(f\) jest liczba\\
A. \((-29)\)\\
B. \((-23)\)\\
C. 23\\
D. 29

\begin{center}
\begin{tabular}{|c|c|c|c|c|c|c|c|c|c|c|c|c|c|c|c|c|c|c|c|c|c|c|c|c|}
\hline
\multicolumn{5}{|l|}{Brudnopis} &  &  &  &  &  &  &  &  &  &  &  &  &  &  &  &  &  &  &  &  \\
\hline
 &  &  &  &  &  &  &  &  &  &  &  &  &  &  &  &  &  &  &  &  &  &  &  &  \\
\hline
 &  &  &  &  &  &  &  &  &  &  &  &  &  &  &  &  &  &  &  &  &  &  &  &  \\
\hline
 &  &  &  &  &  &  &  &  &  &  &  &  &  &  &  &  &  &  &  &  &  &  &  &  \\
\hline
 &  &  &  &  &  &  &  &  &  &  &  &  &  &  &  &  &  &  &  &  &  &  &  &  \\
\hline
 &  &  &  &  &  &  &  &  &  &  &  &  &  &  &  &  &  &  &  &  &  &  &  &  \\
\hline
 &  &  &  &  &  &  &  &  &  &  &  &  &  &  &  &  &  &  &  &  &  &  &  &  \\
\hline
 &  &  &  &  &  &  &  &  &  &  &  &  &  &  &  &  &  &  &  &  &  &  &  &  \\
\hline
 &  &  &  &  &  &  &  &  &  &  &  &  &  &  &  &  &  &  &  &  &  &  &  &  \\
\hline
 &  &  &  &  &  &  &  &  &  &  &  &  &  &  &  &  &  &  &  &  &  &  &  &  \\
\hline
 &  &  &  &  &  &  &  &  &  &  &  &  &  &  &  &  &  &  &  &  &  &  &  &  \\
\hline
 &  &  &  &  &  &  &  &  &  &  &  &  &  &  &  &  &  &  &  &  &  &  &  &  \\
\hline
\end{tabular}
\end{center}

\section*{Zadanie 14.}
W kartezjańskim układzie współrzędnych \((x, y)\) narysowano wykres funkcji \(y=f(x)\) (zobacz rysunek).\\
\includegraphics[max width=\textwidth, center]{2025_02_10_c6c2d6cd217fbcdc7465g-12(1)}

\section*{Zadanie 14.1. (0-1) 뚬}
Dokończ zdanie. Wybierz właściwą odpowiedź spośród podanych.\\
Funkcja \(f\) jest rosnąca w przedziale\\
A. \([-5,4]\)\\
B. \([5,7]\)\\
C. \([1,5]\)\\
D. \([-1,5]\)\\
\includegraphics[max width=\textwidth, center]{2025_02_10_c6c2d6cd217fbcdc7465g-12}

\section*{Zadanie 14.2. (0-1)}
Zapisz poniżej w postaci sumy przedziałów zbiór wszystkich argumentów, dla których funkcja \(f\) przyjmuje wartości większe od 1.\\
\(\qquad\)

\begin{center}
\begin{tabular}{|c|c|c|c|c|c|c|c|c|c|c|c|c|c|c|c|c|c|c|c|c|c|c|c|c|}
\hline
 & Brudno & opis &  &  &  &  &  &  &  &  &  &  &  &  &  &  &  &  &  &  &  &  &  &  \\
\hline
 &  &  &  &  &  &  &  &  &  &  &  &  &  &  &  &  &  &  &  &  &  &  &  &  \\
\hline
 &  &  &  &  &  &  &  &  &  &  &  &  &  &  &  &  &  &  &  &  &  &  &  &  \\
\hline
 &  &  &  &  &  &  &  &  &  &  &  &  &  &  &  &  &  &  &  &  &  &  &  &  \\
\hline
 &  &  &  &  &  &  &  &  &  &  &  &  &  &  &  &  &  &  &  &  &  &  &  &  \\
\hline
 &  &  &  &  &  &  &  &  &  &  &  &  &  &  &  &  &  &  &  &  &  &  &  &  \\
\hline
\end{tabular}
\end{center}

\section*{Zadanie 14.3. (0-1)}
Funkcja \(g\) jest określona za pomocą funkcji \(f\) następująco: \(g(x)=f(-x)\) dla każdego \(x \in[-7,-5] \cup[-4,4] \cup[5,7]\). Na jednym z rysunków A-D przedstawiono, w kartezjańskim układzie wspórrzędnych \((x, y)\), wykres funkcji \(y=g(x)\).

Dokończ zdanie. Wybierz właściwą odpowiedź spośród podanych.\\
Wykres funkcji \(y=g(x)\) przedstawiono na rysunku\\
A.\\
B.\\
\includegraphics[max width=\textwidth, center]{2025_02_10_c6c2d6cd217fbcdc7465g-13(3)}\\
\includegraphics[max width=\textwidth, center]{2025_02_10_c6c2d6cd217fbcdc7465g-13(1)}\\
C.\\
\includegraphics[max width=\textwidth, center]{2025_02_10_c6c2d6cd217fbcdc7465g-13}\\
D.\\
\includegraphics[max width=\textwidth, center]{2025_02_10_c6c2d6cd217fbcdc7465g-13(2)}

\section*{Zadanie 15. (0-2)}
Funkcje \(A, B, C, D, E\) oraz \(F\) są określone dla każdej liczby rzeczywistej \(x\). Wzory tych funkcji podano poniżej.

Uzupełnij zdanie. Wybierz dwie właściwe odpowiedzi spośród oznaczonych literami A-F i wpisz te litery w wykropkowanych miejscach.

Przedział ( \(-\infty\), 2] jest zbiorem wartości funkcji \(\qquad\) oraz \(\qquad\) .. .\\
A. \(A(x)=-(x-3)^{2}+2\)\\
B. \(B(x)=x^{2}+2\)\\
C. \(C(x)=-5(x-2)^{2}\)\\
D. \(D(x)=(x-2)^{2}\)\\
E. \(E(x)=2 x^{2}-8 x+10\)\\
F. \(F(x)=-2 x^{2}+4 x\)\\
\includegraphics[max width=\textwidth, center]{2025_02_10_c6c2d6cd217fbcdc7465g-14}

\section*{Zadanie 16. (0-1)}
Ciąg \(\left(a_{n}\right)\) jest określony wzorem \(a_{n}=(-1)^{n} \cdot \frac{n+1}{2}\) dla każdej liczby naturalnej \(n \geq 1\).\\
Dokończ zdanie. Wybierz właściwą odpowiedź spośród podanych.\\
Trzeci wyraz tego ciągu jest równy\\
A. 2\\
B. \((-2)\)\\
C. 3\\
D. \((-1)\)\\
\includegraphics[max width=\textwidth, center]{2025_02_10_c6c2d6cd217fbcdc7465g-15}

\section*{Zadanie 17. (0-1) 뚬}
Dany jest ciąg geometryczny \(\left(a_{n}\right)\), określony dla każdej liczby naturalnej \(n \geq 1\). Pierwszy wyraz tego ciągu jest równy 128 , natomiast iloraz ciągu jest równy \(\left(-\frac{1}{2}\right)\).

Oceń prawdziwość poniższych stwierdzeń. Wybierz \(P\), jeśli stwierdzenie jest prawdziwe, albo F - jeśli jest fałszywe.

\begin{center}
\begin{tabular}{|l|c|c|}
\hline
Wyraz \(a_{2023}\) jest liczbą ujemną. & \(\mathbf{P}\) & F \\
\hline
Różnica \(a_{3}-a_{2}\) jest równa 96. & \(\mathbf{P}\) & \(\mathbf{F}\) \\
\hline
\end{tabular}
\end{center}

\begin{center}
\includegraphics[max width=\textwidth]{2025_02_10_c6c2d6cd217fbcdc7465g-15(1)}
\end{center}

Zadanie 18. (0-2)\\
Ciąg \(\left(3 x^{2}+5 x, x^{2}, 20-x^{2}\right)\) jest arytmetyczny.\\
Oblicz \(x\). Zapisz obliczenia.\\
\includegraphics[max width=\textwidth, center]{2025_02_10_c6c2d6cd217fbcdc7465g-16}

\section*{Zadanie 19. (0-1)}
Kąt \(\alpha\) jest ostry i \(\cos \alpha=\frac{2 \sqrt{6}}{7}\).

Dokończ zdanie. Wybierz właściwą odpowiedź spośród podanych.\\
Sinus kąta \(\alpha\) jest równy\\
A. \(\frac{24}{49}\)\\
B. \(\frac{5}{7}\)\\
C. \(\frac{25}{49}\)\\
D. \(\frac{\sqrt{6}}{7}\)

\begin{center}
\begin{tabular}{|c|c|c|c|c|c|c|c|c|c|c|c|c|c|c|c|c|c|c|c|c|c|c|c|}
\hline
\multicolumn{5}{|l|}{Brudnopis} &  &  &  &  &  &  &  &  &  &  &  &  &  &  &  &  &  &  &  \\
\hline
 &  &  &  &  &  &  &  &  &  &  &  &  &  &  &  &  &  &  &  &  &  &  &  \\
\hline
 &  &  &  &  &  &  &  &  &  &  &  &  &  &  &  &  &  &  &  &  &  &  &  \\
\hline
 &  &  &  &  &  &  &  &  &  &  &  &  &  &  &  &  &  &  &  &  &  &  &  \\
\hline
 &  &  &  &  &  &  &  &  &  &  &  &  &  &  &  &  &  &  &  &  &  &  &  \\
\hline
 &  &  &  &  &  &  &  &  &  &  &  &  &  &  &  &  &  &  &  &  &  &  &  \\
\hline
 &  &  &  &  &  &  &  &  &  &  &  &  &  &  &  &  &  &  &  &  &  &  &  \\
\hline
 &  &  &  &  &  &  &  &  &  &  &  &  &  &  &  &  &  &  &  &  &  &  &  \\
\hline
 &  &  &  &  &  &  &  &  &  &  &  &  &  &  &  &  &  &  &  &  &  &  &  \\
\hline
 &  &  &  &  &  &  &  &  &  &  &  &  &  &  &  &  &  &  &  &  &  &  &  \\
\hline
 &  &  &  &  &  &  &  &  &  &  &  &  &  &  &  &  &  &  &  &  &  &  &  \\
\hline
 &  &  &  &  &  &  &  &  &  &  &  &  &  &  &  &  &  &  &  &  &  &  &  \\
\hline
 &  &  &  &  &  &  &  &  &  &  &  &  &  &  &  &  &  &  &  &  &  &  &  \\
\hline
\end{tabular}
\end{center}

\section*{Zadanie 20. (0-1) 뚜}
Trapez \(T_{1}\), o polu równym 52 i obwodzie 36 , jest podobny do trapezu \(T_{2}\). Pole trapezu \(T_{2}\) jest równe 13.

\section*{Dokończ zdanie. Wybierz właściwą odpowiedź spośród podanych.}
Obwód trapezu \(T_{2}\) jest równy\\
A. 18\\
B. 9\\
C. \(\frac{169}{9}\)\\
D. \(\frac{52}{3}\)

\begin{center}
\begin{tabular}{|c|c|c|c|c|c|c|c|c|c|c|c|c|c|c|c|c|c|c|c|c|c|c|c|c|}
\hline
\multicolumn{5}{|l|}{Brudnopis} &  &  &  &  &  &  & - &  & - &  &  &  & - & - &  & , & - &  &  &  \\
\hline
 &  &  &  &  &  &  &  &  &  &  &  &  &  &  &  &  &  &  &  & , &  &  &  &  \\
\hline
 &  &  &  &  &  &  &  &  &  &  &  &  &  &  &  &  &  &  &  &  &  &  &  &  \\
\hline
 &  &  &  &  &  &  &  &  &  &  &  &  &  &  &  &  &  &  &  &  &  &  &  &  \\
\hline
 &  &  &  &  &  &  &  &  &  &  &  &  &  &  &  &  &  &  &  &  &  &  &  &  \\
\hline
 &  &  &  &  &  &  &  &  &  &  &  &  &  &  &  &  &  &  &  &  &  &  &  &  \\
\hline
 &  &  &  &  &  &  &  &  &  &  &  &  &  & - &  &  &  &  &  &  &  &  &  &  \\
\hline
 &  &  &  &  &  &  &  &  &  &  &  &  &  &  &  &  &  &  &  &  &  &  &  &  \\
\hline
 &  &  &  &  &  &  &  &  &  &  &  &  &  &  &  &  &  &  &  &  &  &  &  &  \\
\hline
 &  &  &  &  &  &  &  &  &  &  &  &  &  &  &  &  &  &  &  &  &  &  &  &  \\
\hline
 &  &  &  &  &  &  &  &  &  &  &  &  &  &  &  &  &  &  &  &  &  &  &  &  \\
\hline
\end{tabular}
\end{center}

\section*{Zadanie 21. (0-1) 뚠}
Koło ma promień równy 3.\\
Dokończ zdanie. Wybierz właściwą odpowiedź spośród podanych.\\
Obwód wycinka tego koła o kącie środkowym \(30^{\circ}\) jest równy\\
A. \(\frac{3}{4} \pi\)\\
B. \(\frac{1}{2} \pi\)\\
C. \(\frac{3}{4} \pi+6\)\\
D. \(\frac{1}{2} \pi+6\)

\begin{center}
\begin{tabular}{|c|c|c|c|c|c|c|c|c|c|c|c|c|c|c|c|c|c|c|c|c|c|c|}
\hline
\multicolumn{4}{|l|}{Brudnopis} &  &  &  &  &  &  &  &  &  &  &  &  &  &  &  &  &  &  &  \\
\hline
 &  &  &  &  &  &  &  &  &  &  &  &  &  &  &  &  &  &  &  &  &  &  \\
\hline
 &  &  &  &  &  &  &  &  &  &  &  &  &  &  &  &  &  &  &  &  &  &  \\
\hline
 &  &  &  &  &  &  &  &  &  &  &  &  &  &  &  &  &  &  &  &  &  &  \\
\hline
 &  &  &  &  &  &  &  &  &  &  &  &  &  &  &  &  &  &  &  &  &  &  \\
\hline
 &  &  &  &  &  &  &  &  &  &  &  &  &  &  &  &  &  &  &  &  &  &  \\
\hline
 &  &  &  &  &  &  &  &  &  &  &  &  &  &  &  &  &  &  &  &  &  &  \\
\hline
 &  &  &  &  &  &  &  &  &  &  &  &  &  &  &  &  &  &  &  &  &  &  \\
\hline
 &  &  &  &  &  &  &  &  &  &  &  &  &  &  &  &  &  &  &  &  &  &  \\
\hline
 &  &  &  &  &  &  &  &  &  &  &  &  &  &  &  &  &  &  &  &  &  &  \\
\hline
 &  &  &  &  &  &  &  &  &  &  &  &  &  &  &  &  &  &  &  &  &  &  \\
\hline
 &  &  &  &  &  &  &  &  &  &  &  &  &  &  &  &  &  &  &  &  &  &  \\
\hline
\end{tabular}
\end{center}

\section*{Zadanie 22. (0-1)}
W okręgu \(\mathcal{O}\) kąt środkowy \(\beta\) oraz kąt wpisany \(\alpha\) są oparte na tym samym łuku. Kąt \(\beta\) ma miarę o \(40^{\circ}\) większą od kąta \(\alpha\).

Dokończ zdanie. Wybierz właściwą odpowiedź spośród podanych.\\
Miara kąta \(\beta\) jest równa\\
A. \(40^{\circ}\)\\
B. \(80^{\circ}\)\\
C. \(100^{\circ}\)\\
D. \(120^{\circ}\)

\begin{center}
\begin{tabular}{|c|c|c|c|c|c|c|c|c|c|c|c|c|c|c|c|c|c|c|c|c|c|c|c|}
\hline
\multicolumn{5}{|l|}{Brudnopis} &  &  &  &  &  &  &  &  &  &  &  & - & , &  &  & - &  &  &  \\
\hline
 &  &  &  &  &  &  &  &  &  &  &  &  &  &  &  &  &  &  &  &  &  &  &  \\
\hline
 &  &  &  &  &  &  &  &  &  &  &  &  &  &  &  &  &  &  &  &  &  &  &  \\
\hline
 &  &  &  &  &  &  &  &  &  &  &  &  &  &  &  &  &  &  &  &  &  &  &  \\
\hline
 &  &  &  &  &  &  &  &  &  &  &  &  &  &  &  &  &  &  &  &  &  &  &  \\
\hline
 &  &  &  &  &  &  &  &  &  &  &  &  &  &  &  &  &  &  &  &  &  &  &  \\
\hline
 &  &  &  &  &  &  &  &  &  &  &  &  &  &  &  &  &  &  &  &  &  &  &  \\
\hline
 &  &  &  &  &  &  &  &  &  &  &  &  &  &  &  &  &  &  &  &  &  &  &  \\
\hline
 &  &  &  &  &  &  &  &  &  &  &  &  &  &  &  &  &  &  &  &  &  &  &  \\
\hline
 &  &  &  &  &  &  &  &  &  &  &  &  &  &  &  &  &  &  &  &  &  &  &  \\
\hline
 &  &  &  &  &  &  &  &  &  &  &  &  &  &  &  &  &  &  &  &  &  &  &  \\
\hline
\end{tabular}
\end{center}

Zadanie 23. (0-1)\\
W trójkącie \(A B C\) długość boku \(A C\) jest równa 3 , a długość boku \(B C\) jest równa 4 .\\
Dwusieczna kąta \(A C B\) przecina bok \(A B\) w punkcie \(D\).\\
Dokończ zdanie. Wybierz właściwą odpowiedź spośród podanych.

Stosunek \(|A D|:|D B|\) jest równy\\
A. \(4: 3\)\\
B. \(4: 7\)\\
C. \(3: 4\)\\
D. \(3: 7\)

\begin{center}
\begin{tabular}{|c|c|c|c|c|c|c|c|c|c|c|c|c|c|c|c|c|c|c|c|c|c|c|c|c|c|c|c|c|c|c|}
\hline
 & ud & nop & ois &  &  &  &  &  &  &  &  &  &  &  &  &  &  &  &  &  &  &  &  &  &  &  &  &  &  &  \\
\hline
 &  &  &  &  &  &  &  &  &  &  &  &  &  &  &  &  &  &  &  &  &  &  &  &  &  &  &  &  &  &  \\
\hline
 &  &  &  &  &  &  &  &  &  &  &  &  &  &  &  &  &  &  &  &  &  &  &  &  &  &  &  &  &  &  \\
\hline
 &  &  &  &  &  &  &  &  &  &  &  &  &  &  &  &  &  &  &  &  &  &  &  &  &  &  &  &  &  &  \\
\hline
 &  &  &  &  &  &  &  &  &  &  &  &  &  &  &  &  &  &  &  &  &  &  &  &  &  &  &  &  &  &  \\
\hline
 &  &  &  &  &  &  &  &  &  &  &  &  &  &  &  &  &  &  &  &  &  &  &  &  &  &  &  &  &  &  \\
\hline
 &  &  &  &  &  &  &  &  &  &  &  &  &  &  &  &  &  &  &  &  &  &  &  &  &  &  &  &  &  &  \\
\hline
 &  &  &  &  &  &  &  &  &  &  &  &  &  &  &  &  &  &  &  &  &  &  &  &  &  &  &  &  &  &  \\
\hline
 &  &  &  &  &  &  &  &  &  &  &  &  &  &  &  &  &  &  &  &  &  &  &  &  &  &  &  &  &  &  \\
\hline
 &  &  &  &  &  &  &  &  &  &  &  &  &  &  &  &  &  &  &  &  &  &  &  &  &  &  &  &  &  &  \\
\hline
 &  &  &  &  &  &  &  &  &  &  &  &  &  &  &  &  &  &  &  &  &  &  &  &  &  &  &  &  &  &  \\
\hline
\end{tabular}
\end{center}

Zadanie 24. (0-2)\\
Dany jest trapez równoramienny \(A B C D\), w którym podstawa \(C D\) ma długość 6 , ramię \(A D\) ma długość 4, a kąty \(B A D\) oraz \(A B C\) mają miarę \(60^{\circ}\) (zobacz rysunek).\\
\includegraphics[max width=\textwidth, center]{2025_02_10_c6c2d6cd217fbcdc7465g-20(3)}

Oblicz pole tego trapezu. Zapisz obliczenia.

\begin{center}
\begin{tabular}{|c|c|c|c|c|c|c|c|c|c|c|c|c|c|c|c|c|c|c|c|c|c|c|c|c|c|c|c|c|c|c|}
\hline
 &  &  &  &  &  &  &  &  &  &  &  &  &  &  &  &  &  &  &  &  &  &  &  &  &  &  &  &  &  &  \\
\hline
 &  &  &  &  &  &  &  &  &  &  &  &  &  &  &  &  &  &  &  &  &  &  &  &  &  &  &  &  &  &  \\
\hline
 &  &  &  &  &  &  &  &  &  &  &  &  &  &  &  &  &  &  &  &  &  &  &  &  &  &  &  &  &  &  \\
\hline
 &  &  &  &  &  &  &  &  &  &  &  &  &  &  &  &  &  &  &  &  &  &  &  &  &  &  &  &  &  &  \\
\hline
 &  &  &  &  &  &  &  &  &  &  &  &  &  &  &  &  &  &  &  &  &  &  &  &  &  &  &  &  &  &  \\
\hline
 &  &  &  &  &  &  &  &  &  &  &  &  &  &  &  &  &  &  &  &  &  &  &  &  &  &  &  &  &  &  \\
\hline
 &  &  &  &  &  &  &  &  &  &  &  &  &  &  &  &  &  &  &  &  &  &  &  &  &  &  &  &  &  &  \\
\hline
 &  &  &  &  &  &  &  &  &  &  &  &  &  &  &  &  &  &  &  &  &  &  &  &  &  &  &  &  &  &  \\
\hline
 &  &  &  &  &  &  &  &  &  &  &  &  &  &  &  &  &  &  &  &  &  &  &  &  &  &  &  &  &  &  \\
\hline
 &  &  &  &  &  &  &  &  &  &  &  &  &  &  &  &  &  &  &  &  &  &  &  &  &  &  &  &  &  &  \\
\hline
 &  &  &  &  &  &  &  &  &  &  &  &  &  &  &  &  &  &  &  &  &  &  &  &  &  &  &  &  &  &  \\
\hline
\includegraphics[max width=\textwidth]{2025_02_10_c6c2d6cd217fbcdc7465g-20(1)}
 &  &  &  &  &  &  &  &  &  &  &  &  &  &  &  &  &  &  &  &  &  &  &  &  &  &  &  &  &  &  \\
\hline
 &  &  &  &  &  &  &  &  &  &  &  &  &  &  &  &  &  &  &  &  &  &  &  &  &  &  &  &  &  &  \\
\hline
 &  &  &  &  &  &  &  &  &  &  &  &  &  &  &  &  &  &  &  &  &  &  &  &  &  &  &  &  &  &  \\
\hline
 &  &  &  &  &  &  &  &  &  &  &  &  &  &  &  &  &  &  &  &  &  &  &  &  &  &  &  &  &  &  \\
\hline
 &  &  &  &  &  &  &  &  &  &  &  &  &  &  &  &  &  &  &  &  &  &  &  &  &  &  &  &  &  &  \\
\hline
 &  &  &  &  &  &  &  &  &  &  &  &  &  &  &  &  &  &  &  &  &  &  &  &  &  &  &  &  &  &  \\
\hline
 &  &  &  &  &  &  &  &  &  &  &  &  &  &  &  &  &  &  &  &  &  &  &  &  &  &  &  &  &  &  \\
\hline
 &  &  &  &  &  &  &  &  &  &  &  &  &  &  &  &  &  &  &  &  &  &  &  &  &  &  &  &  &  &  \\
\hline
 &  &  &  &  &  &  &  &  &  &  &  &  &  &  &  &  &  &  &  &  &  &  &  &  &  &  &  &  &  &  \\
\hline
\includegraphics[max width=\textwidth]{2025_02_10_c6c2d6cd217fbcdc7465g-20(2)}
 &  &  &  &  &  &  &  &  &  &  &  &  &  &  &  &  &  &  &  &  &  &  &  &  &  &  &  &  &  &  \\
\hline
 &  &  &  &  &  &  &  &  &  &  &  &  &  &  &  &  &  &  &  &  &  &  &  &  &  &  &  &  &  &  \\
\hline
 &  &  &  &  &  &  &  &  &  &  &  &  &  &  &  &  &  &  &  &  &  &  &  &  &  &  &  &  &  &  \\
\hline
 &  &  &  &  &  &  &  &  &  &  &  &  &  &  &  &  &  &  &  &  &  &  &  &  &  &  &  &  &  &  \\
\hline
 &  &  &  &  &  &  &  &  &  &  &  &  &  &  &  &  &  &  &  &  &  &  &  &  &  &  &  &  &  &  \\
\hline
 &  &  &  &  &  &  &  &  &  &  &  &  &  &  &  &  &  &  &  &  &  &  &  &  &  &  &  &  &  &  \\
\hline
 & - &  &  &  &  &  &  &  &  &  &  &  &  &  &  &  &  &  &  &  &  &  &  &  &  &  &  &  &  &  \\
\hline
 & \includegraphics[max width=\textwidth]{2025_02_10_c6c2d6cd217fbcdc7465g-20}
 &  &  &  &  &  &  &  &  &  &  &  &  &  &  &  &  &  &  &  &  &  &  &  &  &  &  &  &  &  \\
\hline
 &  &  &  &  &  &  &  &  &  &  &  &  &  &  &  &  &  &  &  &  &  &  &  &  &  &  &  &  &  &  \\
\hline
 &  &  &  &  &  &  &  &  &  &  &  &  &  &  &  &  &  &  &  &  &  &  &  &  &  &  &  &  &  &  \\
\hline
 &  &  &  &  &  &  &  &  &  &  &  &  &  &  &  &  &  &  &  &  &  &  &  &  &  &  &  &  &  &  \\
\hline
\end{tabular}
\end{center}

\section*{Zadanie 25. (0-1) 뚜ํ}
W kartezjańskim układzie współrzędnych \((x, y)\) dane są prosta \(k\) o równaniu \(y=\frac{3}{4} x-\frac{7}{4}\) oraz punkt \(P=(12,-1)\).

Dokończ zdanie. Wybierz właściwą odpowiedź spośród podanych.\\
Prosta przechodząca przez punkt \(P\) i równoległa do prostej \(k\) ma równanie\\
A. \(y=-\frac{3}{4} x+8\)\\
B. \(y=\frac{3}{4} x-10\)\\
C. \(y=\frac{4}{3} x-17\)\\
D. \(y=-\frac{4}{3} x+15\)\\
\includegraphics[max width=\textwidth, center]{2025_02_10_c6c2d6cd217fbcdc7465g-21(1)}

\section*{Zadanie 26. (0-1) 두무}
W kartezjańskim układzie współrzędnych \((x, y)\) dany jest okrąg \(\mathcal{O}\) o środku \(S=(-1,2)\) i promieniu 3.

Dokończ zdanie. Wybierz właściwą odpowiedź spośród podanych.\\
Okrąg \(\mathcal{O}\) jest określony równaniem\\
A. \((x-1)^{2}+(y+2)^{2}=9\)\\
B. \((x-1)^{2}+(y+2)^{2}=3\)\\
C. \((x+1)^{2}+(y-2)^{2}=9\)\\
D. \((x+1)^{2}+(y-2)^{2}=3\)\\
\includegraphics[max width=\textwidth, center]{2025_02_10_c6c2d6cd217fbcdc7465g-21}

Zadanie 27. (0-1) 뚱\\
W kartezjańskim układzie wspórrzędnych ( \(x, y\) ) proste o równaniach:

\begin{itemize}
  \item \(y=\sqrt{3} x+6\)
  \item \(y=-\sqrt{3} x+6\)
  \item \(y=-\frac{1}{\sqrt{3}} x-2\),\\
przecinają się w punktach, które są wierzchołkami trójkąta \(K L M\).\\
Dokończ zdanie tak, aby było prawdziwe. Wybierz odpowiedź A albo B oraz jej uzasadnienie 1., 2. albo 3.
\end{itemize}

Trójkąt \(K L M\) jest

\begin{center}
\begin{tabular}{|l|l|l|l|l|}
\hline
 &  &  & \begin{tabular}{l}
oś Ox przechodzi przez jeden z wierzchołków \\
A. \\
A.go trójkąta i środek jednego z boków tego \\
trójkąta. \\
\end{tabular} &  \\
\hline
 & równoramienny, &  & ponieważ & 2. \\
B. dwie z tych prostych są prostopadłe. &  &  &  &  \\
\hline
Brostokątny, &  & 3. & oś Oy zawiera dwusieczną tego trójkąta. &  \\
\hline
\end{tabular}
\end{center}

\begin{center}
\begin{tabular}{|c|c|c|c|c|c|c|c|c|c|c|c|c|c|c|c|c|c|c|c|c|c|c|}
\hline
\multicolumn{4}{|l|}{Brudnopis} &  &  &  &  &  &  &  &  &  &  &  &  &  &  &  &  &  &  &  \\
\hline
 &  &  &  &  &  &  &  &  &  &  &  &  &  &  &  &  &  &  &  &  &  &  \\
\hline
 &  &  &  &  &  &  &  &  &  &  & - &  &  & - & - & - &  & - &  &  &  &  \\
\hline
 &  &  &  &  &  &  &  &  &  &  &  &  &  &  &  &  &  &  &  &  &  &  \\
\hline
 &  &  &  &  &  &  &  &  &  &  &  &  &  &  &  &  &  &  &  &  &  &  \\
\hline
 &  &  &  &  &  &  &  &  &  &  &  &  &  &  &  &  &  &  &  &  &  &  \\
\hline
 &  &  &  &  &  &  &  &  &  &  &  &  &  &  &  &  &  &  &  &  &  &  \\
\hline
 &  &  &  &  &  &  &  &  &  &  &  &  &  &  &  &  &  &  &  &  &  &  \\
\hline
 &  &  &  &  &  &  &  &  &  &  &  &  &  &  &  &  &  &  &  &  &  &  \\
\hline
 &  &  &  &  &  &  &  &  &  &  &  &  &  &  &  &  &  &  &  &  &  &  \\
\hline
 &  &  &  &  &  &  &  &  &  &  &  &  &  &  &  &  &  &  &  &  &  &  \\
\hline
\end{tabular}
\end{center}

\section*{Zadanie 28. (0-1) 뚬}
W kartezjańskim układzie współrzędnych \((x, y)\) punkt \(A=(-1,-4)\) jest wierzchołkiem równoległoboku \(A B C D\). Punkt \(S=(2,2)\) jest środkiem symetrii tego równoległoboku.

Dokończ zdanie. Wybierz właściwą odpowiedź spośród podanych.\\
Długość przekątnej \(A C\) równoległoboku \(A B C D\) jest równa\\
A. \(\sqrt{5}\)\\
B. \(2 \sqrt{5}\)\\
C. \(3 \sqrt{5}\)\\
D. \(6 \sqrt{5}\)\\
\includegraphics[max width=\textwidth, center]{2025_02_10_c6c2d6cd217fbcdc7465g-23}

\section*{Zadanie 29.}
Każda krawędź graniastosłupa prawidłowego sześciokątnego ma długość równą 6.

\section*{Zadanie 29.1. (0-1)}
Dokończ zdanie. Wybierz właściwą odpowiedź spośród podanych.\\
Pole powierzchni całkowitej tego graniastosłupa jest równe\\
A. \(216+18 \sqrt{3}\)\\
B. \(216+54 \sqrt{3}\)\\
C. \(216+216 \sqrt{3}\)\\
D. \(216+108 \sqrt{3}\)

\begin{center}
\begin{tabular}{|c|c|c|c|c|c|c|c|c|c|c|c|c|c|c|c|c|c|c|c|c|c|c|c|c|c|c|}
\hline
\multicolumn{4}{|l|}{Brudnopis} &  &  &  &  &  &  &  &  &  &  &  &  &  &  &  &  &  &  &  &  &  &  &  \\
\hline
 &  &  &  &  &  &  &  &  &  &  &  &  &  &  &  &  &  &  &  &  &  &  &  &  &  &  \\
\hline
 &  &  &  &  &  &  &  &  &  &  &  &  &  &  &  &  &  &  &  &  &  &  &  &  &  &  \\
\hline
 &  &  &  &  &  &  &  &  &  &  &  &  &  &  &  &  &  &  &  &  &  &  &  &  &  &  \\
\hline
 &  &  &  &  &  &  &  &  &  &  &  &  &  &  &  &  &  &  &  &  &  &  &  &  &  &  \\
\hline
 &  &  &  &  &  &  &  &  &  &  &  &  &  &  &  &  &  &  &  &  &  &  &  &  &  &  \\
\hline
 &  &  &  &  &  &  &  &  &  &  &  &  &  &  &  &  &  &  &  &  &  &  &  &  &  &  \\
\hline
 &  &  &  &  &  &  &  &  &  &  &  &  &  &  &  &  &  &  &  &  &  &  &  &  &  &  \\
\hline
 &  &  &  &  &  &  &  &  &  &  &  &  &  &  &  &  &  &  &  &  &  &  &  &  &  &  \\
\hline
 &  &  &  &  &  &  &  &  &  &  &  &  &  &  &  &  &  &  &  &  &  &  &  &  &  &  \\
\hline
 &  &  &  &  &  &  &  &  &  &  &  &  &  &  &  &  &  &  &  &  &  &  &  &  &  &  \\
\hline
\end{tabular}
\end{center}

Zadanie 29.2. (0-1)\\
Oblicz cosinus kąta nachylenia dłuższej przekątnej tego graniastosłupa do płaszczyzny podstawy graniastosłupa. Zapisz obliczenia.

\begin{center}
\begin{tabular}{|c|c|c|c|c|c|c|c|c|c|c|c|c|c|c|c|c|c|c|c|c|c|c|c|c|c|c|c|c|c|c|}
\hline
 &  &  &  &  &  &  &  &  &  &  &  &  &  &  &  &  &  &  &  &  &  &  &  &  &  &  &  &  &  &  \\
\hline
 &  &  &  &  &  &  &  &  &  &  &  &  &  &  &  &  &  &  &  &  &  &  &  &  &  &  &  &  &  &  \\
\hline
 &  &  &  &  &  &  &  &  &  &  &  &  &  &  &  &  &  &  &  &  &  &  &  &  &  &  &  &  &  &  \\
\hline
 &  &  &  &  &  &  &  &  &  &  &  &  &  &  &  &  &  &  &  &  &  &  &  &  &  &  &  &  &  &  \\
\hline
 &  &  &  &  &  &  &  &  &  &  &  &  &  &  &  &  &  &  &  &  &  &  &  &  &  &  &  &  &  &  \\
\hline
 &  &  &  &  &  &  &  &  &  &  &  &  &  &  &  &  &  &  &  &  &  &  &  &  &  &  &  &  &  &  \\
\hline
\includegraphics{smile-f8588253a3d3b10a7aad8f8bf781aa2dffed198d} &  &  &  &  &  &  &  &  &  &  &  &  &  &  &  &  &  &  &  &  &  &  &  &  &  &  &  &  &  &  \\
\hline
\end{tabular}
\end{center}

\section*{Zadanie 30. (0-1) 뚱}
Dokończ zdanie. Wybierz właściwą odpowiedź spośród podanych.\\
Wszystkich liczb naturalnych czterocyfrowych, w których zapisie dziesiętnym cyfry się nie powtarzaja, jest\\
A. \(9 \cdot 10 \cdot 10 \cdot 10 \cdot 10\)\\
B. \(9.9 \cdot 9 \cdot 9\)\\
C. \(10 \cdot 9 \cdot 8 \cdot 7\)\\
D. \(9 \cdot 9 \cdot 8 \cdot 7\)\\
\includegraphics[max width=\textwidth, center]{2025_02_10_c6c2d6cd217fbcdc7465g-24(1)}

\section*{Zadanie 31. (0-2)}
Ze zbioru pięciu liczb \(\{1,2,3,4,5\}\) losujemy bez zwracania kolejno dwa razy po jednej liczbie.

Oblicz prawdopodobieństwo zdarzenia \(A\) polegającego na tym, że obie wylosowane liczby są nieparzyste. Zapisz obliczenia.\\
\includegraphics[max width=\textwidth, center]{2025_02_10_c6c2d6cd217fbcdc7465g-24}\\
\includegraphics[max width=\textwidth, center]{2025_02_10_c6c2d6cd217fbcdc7465g-25}

Zadanie 32. (0-1) 뚱\\
Na diagramie przedstawiono rozkład wynagrodzenia brutto wszystkich stu pracowników pewnej firmy za styczeń 2023 roku.\\
\includegraphics[max width=\textwidth, center]{2025_02_10_c6c2d6cd217fbcdc7465g-26}

Dokończ zdanie. Wybierz właściwą odpowiedź spośród podanych.\\
Średnia wynagrodzenia brutto wszystkich pracowników tej firmy za styczeń 2023 roku jest równa\\
A. 5690 zt\\
B. \(5280 \mathrm{zł}\)\\
C. \(6257 \mathrm{zł}\)\\
D. 5900 zt

\begin{center}
\begin{tabular}{|c|c|c|c|c|c|c|c|c|c|c|c|c|c|c|c|c|c|c|c|c|c|c|}
\hline
\multicolumn{4}{|l|}{Brudnopis} &  &  &  &  &  & - & - &  &  &  &  & - & , & - &  & - &  & , & - \\
\hline
 &  &  &  &  &  &  &  &  &  &  &  &  &  &  &  &  &  &  &  &  &  &  \\
\hline
 &  &  &  &  &  &  &  &  &  &  &  &  &  &  &  &  &  &  &  &  &  &  \\
\hline
 &  &  &  &  &  &  &  &  &  &  &  &  &  &  &  &  &  &  &  &  &  &  \\
\hline
 &  &  &  &  &  &  &  &  &  &  &  &  &  &  &  &  &  &  &  &  &  &  \\
\hline
 &  &  &  &  &  &  &  &  &  &  &  &  &  &  &  &  &  &  &  &  &  &  \\
\hline
 &  &  &  &  &  &  &  &  &  &  &  &  &  &  &  &  &  &  &  &  &  &  \\
\hline
 &  &  &  &  &  &  &  &  &  &  &  &  &  &  &  &  &  &  &  &  &  &  \\
\hline
 &  &  &  &  &  &  &  &  &  &  &  &  &  &  &  &  &  &  &  &  &  &  \\
\hline
 &  &  &  &  &  &  &  &  &  &  &  &  &  &  &  &  &  &  &  &  &  &  \\
\hline
 &  &  &  &  &  &  &  &  &  &  &  &  &  &  &  &  &  &  &  &  &  &  \\
\hline
\end{tabular}
\end{center}

\section*{Zadanie 33. (0-4)}
Zakład stolarski produkuje krzesła, które sprzedaje po 196 złotych za sztukę. Właściciel, na podstawie analizy rzeczywistych wpływów i wydatków, stwierdził, że:

\begin{itemize}
  \item przychód \(P\) (w złotych) ze sprzedaży \(x\) krzeseł można opisać funkcją \(P(x)=196 x\)
  \item koszt \(K\) (w złotych) produkcji \(x\) krzeseł dziennie można opisać funkcją
\end{itemize}

\[
K(x)=4 x^{2}+4 x+240
\]

Dziennie w zakładzie można wyprodukować co najwyżej 30 krzeseł.\\
Oblicz, ile krzeseł powinien dziennie sprzedawać zakład, aby zysk ze sprzedaży krzeseł wyprodukowanych przez ten zakład w ciągu jednego dnia był możliwie największy. Oblicz ten największy zysk.

Zapisz obliczenia.\\
Wskazówka: przyjmij, że zysk jest różnicą przychodu i kosztów.\\
\includegraphics[max width=\textwidth, center]{2025_02_10_c6c2d6cd217fbcdc7465g-27}\\
\includegraphics[max width=\textwidth, center]{2025_02_10_c6c2d6cd217fbcdc7465g-28}

BRUDNOPIS (nie podlega ocenie)

\begin{center}
\begin{tabular}{|c|c|c|c|c|c|c|c|c|c|c|c|c|c|c|c|c|c|c|c|c|c|}
\hline
 &  &  &  &  &  &  &  &  &  &  &  &  &  &  &  &  &  &  &  &  &  \\
\hline
 &  &  &  &  &  &  &  &  &  &  &  &  &  &  &  &  &  &  &  &  &  \\
\hline
 &  &  &  &  &  &  &  &  &  &  &  &  &  &  &  &  &  &  &  &  &  \\
\hline
 &  &  &  &  &  &  &  &  &  &  &  &  &  &  &  &  &  &  &  &  &  \\
\hline
 &  &  &  &  &  &  &  &  &  &  &  &  &  &  &  &  &  &  &  &  &  \\
\hline
 &  &  &  &  &  &  &  &  &  &  &  &  &  &  &  &  &  &  &  &  &  \\
\hline
 &  &  &  &  &  &  &  &  &  &  &  &  &  &  &  &  &  &  &  &  &  \\
\hline
 &  &  &  &  &  &  &  &  &  &  &  &  &  &  &  &  &  &  &  &  &  \\
\hline
 &  &  &  &  &  &  &  &  &  &  &  &  &  &  &  &  &  &  &  &  &  \\
\hline
 &  &  &  &  &  &  &  &  &  &  &  &  &  &  &  &  &  &  &  &  &  \\
\hline
 &  &  &  &  &  &  &  &  &  &  &  &  &  &  &  &  &  &  &  &  &  \\
\hline
 &  &  &  &  &  &  &  &  &  &  &  &  &  &  &  &  &  &  &  &  &  \\
\hline
 &  &  &  &  &  &  &  &  &  &  &  &  &  &  &  &  &  &  &  &  &  \\
\hline
 &  &  &  &  &  &  &  &  &  &  &  &  &  &  &  &  &  &  &  &  &  \\
\hline
 &  &  &  &  &  &  &  &  &  &  &  &  &  &  &  &  &  &  &  &  &  \\
\hline
 &  &  &  &  &  &  &  &  &  &  &  &  &  &  &  &  &  &  &  &  &  \\
\hline
 &  &  &  &  &  &  &  &  &  &  &  &  &  &  &  &  &  &  &  &  &  \\
\hline
 &  &  &  &  &  &  &  &  &  &  &  &  &  &  &  &  &  &  &  &  &  \\
\hline
 &  &  &  &  &  &  &  &  &  &  &  &  &  &  &  &  &  &  &  &  &  \\
\hline
 &  &  &  &  &  &  &  &  &  &  &  &  &  &  &  &  &  &  &  &  &  \\
\hline
 &  &  &  &  &  &  &  &  &  &  &  &  &  &  &  &  &  &  &  &  &  \\
\hline
 &  &  &  &  &  &  &  &  &  &  &  &  &  &  &  &  &  &  &  &  &  \\
\hline
 &  &  &  &  &  &  &  &  &  &  &  &  &  &  &  &  &  &  &  &  &  \\
\hline
 &  &  &  &  &  &  &  &  &  &  &  &  &  &  &  &  &  &  &  &  &  \\
\hline
 &  &  &  &  &  &  &  &  &  &  &  &  &  &  &  &  &  &  &  &  &  \\
\hline
 &  &  &  &  &  &  &  &  &  &  &  &  &  &  &  &  &  &  &  &  &  \\
\hline
 &  &  &  &  &  &  &  &  &  &  &  &  &  &  &  &  &  &  &  &  &  \\
\hline
 &  &  &  &  &  &  &  &  &  &  &  &  &  &  &  &  &  &  &  &  &  \\
\hline
 &  &  &  &  &  &  &  &  &  &  &  &  &  &  &  &  &  &  &  &  &  \\
\hline
 &  &  &  &  &  &  &  &  &  &  &  &  &  &  &  &  &  &  &  &  &  \\
\hline
 &  &  &  &  &  &  &  &  &  &  &  &  &  &  &  &  &  &  &  &  &  \\
\hline
 &  &  &  &  &  &  &  &  &  &  &  &  &  &  &  &  &  &  &  &  &  \\
\hline
 &  &  &  &  &  &  &  &  &  &  &  &  &  &  &  &  &  &  &  &  &  \\
\hline
 &  &  &  &  &  &  &  &  &  &  &  &  &  &  &  &  &  &  &  &  &  \\
\hline
 &  &  &  &  &  &  &  &  &  &  &  &  &  &  &  &  &  &  &  &  &  \\
\hline
 &  &  &  &  &  &  &  &  &  &  &  &  &  &  &  &  &  &  &  &  &  \\
\hline
 &  &  &  &  &  &  &  &  &  &  &  &  &  &  &  &  &  &  &  &  &  \\
\hline
 &  &  &  &  &  &  &  &  &  &  &  &  &  &  &  &  &  &  &  &  &  \\
\hline
 &  &  &  &  &  &  &  &  &  &  &  &  &  &  &  &  &  &  &  &  &  \\
\hline
 &  &  &  &  &  &  &  &  &  &  &  &  &  &  &  &  &  &  &  &  &  \\
\hline
 &  &  &  &  &  &  &  &  &  &  &  &  &  &  &  &  &  &  &  &  &  \\
\hline
 &  &  &  &  &  &  &  &  &  &  &  &  &  &  &  &  &  &  &  &  &  \\
\hline
 &  &  &  &  &  &  &  &  &  &  &  &  &  &  &  &  &  &  &  &  &  \\
\hline
 &  &  &  &  &  &  &  &  &  &  &  &  &  &  &  &  &  &  &  &  &  \\
\hline
 &  &  &  &  &  &  &  &  &  &  &  &  &  &  &  &  &  &  &  &  &  \\
\hline
 &  &  &  &  &  &  &  &  &  &  &  &  &  &  &  &  &  &  &  &  &  \\
\hline
 &  &  &  &  &  &  &  &  &  &  &  &  &  &  &  &  &  &  &  &  &  \\
\hline
\end{tabular}
\end{center}

\includegraphics[max width=\textwidth, center]{2025_02_10_c6c2d6cd217fbcdc7465g-30}\\
\includegraphics[max width=\textwidth, center]{2025_02_10_c6c2d6cd217fbcdc7465g-31}

\section*{MATEMATYKA}
\section*{Poziom podstawowy}
Formuła 2023

\section*{MATEMATYKA}
\section*{Poziom podstawowy}
Formuła 2023

\section*{MATEMATYKA}
\section*{Poziom podstawowy}
Formuła 2023


\end{document}