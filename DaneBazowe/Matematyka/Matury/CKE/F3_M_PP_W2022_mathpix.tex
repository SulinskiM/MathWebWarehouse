% This LaTeX document needs to be compiled with XeLaTeX.
\documentclass[10pt]{article}
\usepackage[utf8]{inputenc}
\usepackage{ucharclasses}
\usepackage{amsmath}
\usepackage{amsfonts}
\usepackage{amssymb}
\usepackage[version=4]{mhchem}
\usepackage{stmaryrd}
\usepackage{graphicx}
\usepackage[export]{adjustbox}
\graphicspath{ {./images/} }
\usepackage[fallback]{xeCJK}
\usepackage{polyglossia}
\usepackage{fontspec}
\IfFontExistsTF{Noto Serif CJK KR}
{\setCJKmainfont{Noto Serif CJK KR}}
{\IfFontExistsTF{Apple SD Gothic Neo}
  {\setCJKmainfont{Apple SD Gothic Neo}}
  {\IfFontExistsTF{UnDotum}
    {\setCJKmainfont{UnDotum}}
    {\setCJKmainfont{Malgun Gothic}}
}}
\IfFontExistsTF{Noto Serif CJK TC}
{\setCJKfallbackfamilyfont{\CJKrmdefault}{Noto Serif CJK TC}}
{\IfFontExistsTF{STSong}
  {\setCJKfallbackfamilyfont{\CJKrmdefault}{STSong}}
  {\IfFontExistsTF{Droid Sans Fallback}
    {\setCJKfallbackfamilyfont{\CJKrmdefault}{Droid Sans Fallback}}
    {\setCJKfallbackfamilyfont{\CJKrmdefault}{SimSun}}
}}

\setmainlanguage{polish}
\setotherlanguages{hindi, thai}
\IfFontExistsTF{Noto Serif Devanagari}
{\newfontfamily\hindifont{Noto Serif Devanagari}}
{\IfFontExistsTF{Kohinoor Devanagari}
  {\newfontfamily\hindifont{Kohinoor Devanagari}}
  {\IfFontExistsTF{Devanagari MT}
    {\newfontfamily\hindifont{Devanagari MT}}
    {\IfFontExistsTF{Lohit Devanagari}
      {\newfontfamily\hindifont{Lohit Devanagari}}
      {\IfFontExistsTF{FreeSerif}
        {\newfontfamily\hindifont{FreeSerif}}
        {\newfontfamily\hindifont{Arial Unicode MS}}
}}}}
\IfFontExistsTF{Noto Serif Thai}
{\newfontfamily\thaifont{Noto Serif Thai}}
{\IfFontExistsTF{Thonburi}
  {\newfontfamily\thaifont{Thonburi}}
  {\IfFontExistsTF{FreeSerif}
    {\newfontfamily\thaifont{FreeSerif}}
    {\IfFontExistsTF{Tahoma}
      {\newfontfamily\thaifont{Tahoma}}
      {\newfontfamily\thaifont{Arial Unicode MS}}
}}}
\IfFontExistsTF{CMU Serif}
{\newfontfamily\lgcfont{CMU Serif}}
{\IfFontExistsTF{DejaVu Sans}
  {\newfontfamily\lgcfont{DejaVu Sans}}
  {\newfontfamily\lgcfont{Georgia}}
}
\setDefaultTransitions{\lgcfont}{}
\setTransitionsForDevanagari{\hindifont}{\rmfamily}
\setTransitionsFor{Thai}{\thaifont}{\lgcfont}

\author{Liczba punktów do uzyskania: 46}
\date{}


\begin{document}
\maketitle
W JAWORZNIE\\
WYPEŁNIA ZDAJACY

KOD\\
PESEL

\begin{center}
\begin{tabular}{|l|l|}
\hline
 &  \\
\hline
\end{tabular}
\end{center} \(\square\)

Miejsce na naklejkę. Sprawdż, czy kod na naklejce to M-100.

Jeżeli tak - przyklej naklejkę. Jeżeli nie - zgłoś to nauczycielowi.

\section*{Egzamin maturalny}
\section*{MATEMATYKA}
\section*{Poziom podstawowy}
\section*{TEST DIAGNOSTYCZNY}
\section*{DAtA: 29 września 2022 r.}
Godzina rozpoczęcia: 9:00\\
Czas trwania: \(\mathbf{1 8 0}\) minut

\begin{verbatim}
WYPEKNIA ZESPÓŁ NADZORUJACY
Uprawnienia zdającego do:
    dostosowania zasad oceniania
        dostosowania w zw. z dyskalkulia
        nieprzenoszenia zaznaczeń na kartę.
\end{verbatim}

Przed rozpoczęciem pracy z arkuszem egzaminacyjnym

\begin{enumerate}
  \item Sprawdź, czy nauczyciel przekazał Ci właściwy arkusz egzaminacyjny, tj. arkusz z właściwego przedmiotu na właściwym poziomie.
  \item Jeżeli przekazano Ci niewłaściwy arkusz - natychmiast zgłoś to nauczycielowi. Nie rozrywaj banderol.
  \item Jeżeli przekazano Ci właściwy arkusz - rozerwij banderole po otrzymaniu takiego polecenia od nauczyciela. Zapoznaj się z instrukcją na stronie 2.\\
\includegraphics[max width=\textwidth, center]{2025_02_09_354e7a1d815e7eca7072g-02}
\end{enumerate}

\section*{Instrukcja dla zdającego}
\begin{enumerate}
  \item Sprawdź, czy arkusz egzaminacyjny zawiera 29 stron (zadania 1-26). Ewentualny brak zgłoś przewodniczącemu zespołu nadzorującego egzamin.
  \item Na stronie tytułowej arkusza oraz na karcie odpowiedzi wpisz swój numer PESEL i przyklej naklejkę z kodem. Nie wpisuj żadnych znaków w części przeznaczonej dla egzaminatora.
  \item Nie wpisuj żadnych znaków w tabelkach przeznaczonych dla egzaminatora. Tabelki umieszczone są na marginesie przy odpowiednich zadaniach.
  \item Rozwiązania zadań i odpowiedzi wpisuj w miejscu na to przeznaczonym.
  \item Symbol exal zamieszczony w nagłówku zadania oznacza, że rozwiązanie zadania zamkniętego musisz przenieść na kartę odpowiedzi.
  \item Odpowiedzi do zadań zamkniętych zaznacz na karcie odpowiedzi w części karty przeznaczonej dla zdającego. Zamaluj \(\square\) pola do tego przeznaczone. Błędne zaznaczenie otocz kółkiem i i zaznacz właściwe.
  \item Pamiętaj, że pominięcie argumentacji lub istotnych obliczeń w rozwiązaniu zadania otwartego może spowodować, że za to rozwiązanie nie otrzymasz pełnej liczby punktów.
  \item Pisz czytelnie i używaj tylko długopisu lub pióra z czarnym tuszem lub atramentem.
  \item Nie używaj korektora, a błędne zapisy wyraźnie przekreśl.
  \item Pamiętaj, że zapisy w brudnopisie nie będą oceniane.
  \item Możesz korzystać z Wybranych wzorów matematycznych, cyrkla i linijki oraz kalkulatora prostego. Upewnij się, czy przekazano Ci broszurę z taką okładką, jak poniżej.\\
\includegraphics[max width=\textwidth, center]{2025_02_09_354e7a1d815e7eca7072g-02(1)}
\end{enumerate}

Zadania egzaminacyine sa wydrukowane na kolejnych stronach.

\section*{Zadanie 1. (0-1) 뚜ํ}
Dokończ zdanie. Wybierz właściwą odpowiedź spośród podanych.\\
Wartość wyrażenia \(\left(1+3 \cdot 2^{-1}\right)^{-2}\) jest równa\\
A. \(\frac{25}{4}\)\\
B. \(\frac{4}{25}\)\\
C. \(\frac{36}{49}\)\\
D. \(\frac{40}{9}\)

\begin{center}
\begin{tabular}{|c|c|c|c|c|c|c|c|c|c|c|c|c|c|c|c|c|c|c|c|c|c|c|c|}
\hline
 & Brud & dnopi &  &  &  &  &  &  &  &  &  &  &  &  &  &  &  &  &  &  &  &  &  \\
\hline
 &  &  &  &  &  &  &  &  &  &  &  &  &  &  &  &  &  &  &  &  &  &  &  \\
\hline
 &  &  &  &  &  &  &  &  &  &  &  &  &  &  &  &  &  &  &  &  &  &  &  \\
\hline
 &  &  &  &  &  &  &  &  &  &  &  &  &  &  &  &  &  &  &  &  &  &  &  \\
\hline
 &  &  &  &  &  &  &  &  &  &  &  &  &  &  &  &  &  &  &  &  &  &  &  \\
\hline
 &  &  &  &  &  &  &  &  &  &  &  &  &  &  &  &  &  &  &  &  &  &  &  \\
\hline
 &  &  &  &  &  &  &  &  &  &  &  &  &  &  &  &  &  &  &  &  &  &  &  \\
\hline
 &  &  &  &  &  &  &  &  &  &  &  &  &  &  &  &  &  &  &  &  &  &  &  \\
\hline
 &  &  &  &  &  &  &  &  &  &  &  &  &  &  &  &  &  &  &  &  &  &  &  \\
\hline
 &  &  &  &  &  &  &  &  &  &  &  &  &  &  &  &  &  &  &  &  &  &  &  \\
\hline
 &  &  &  &  &  &  &  &  &  &  &  &  &  &  &  &  &  &  &  &  &  &  &  \\
\hline
 &  &  &  &  &  &  &  &  &  &  &  &  &  &  &  &  &  &  &  &  &  &  &  \\
\hline
 &  &  &  &  &  &  &  &  &  &  &  &  &  &  &  &  &  &  &  &  &  &  &  \\
\hline
 &  &  &  &  &  &  &  &  &  &  &  &  &  &  &  &  &  &  &  &  &  &  &  \\
\hline
 &  &  &  &  &  &  &  &  &  &  &  &  &  &  &  &  &  &  &  &  &  &  &  \\
\hline
 &  &  &  &  &  &  &  &  &  &  &  &  &  &  &  &  &  &  &  &  &  &  &  \\
\hline
\end{tabular}
\end{center}

\section*{Zadanie 2. (0-1)}
Dokończ zdanie. Wybierz właściwą odpowiedź spośród podanych.\\
Wartość wyrażenia \(2 \log _{5} 5+1-\frac{1}{2} \log _{5} 625\) jest równa\\
A. 1\\
B. 5\\
C. 10\\
D. 25

\begin{center}
\begin{tabular}{|c|c|c|c|c|c|c|c|c|c|c|c|c|c|c|c|c|c|c|c|c|c|c|c|c|c|c|c|c|c|c|}
\hline
\multicolumn{6}{|l|}{Brudnopis} &  &  &  &  &  &  &  &  &  &  &  &  &  &  &  &  &  &  &  &  &  &  &  &  &  \\
\hline
 &  &  &  &  &  &  &  &  &  &  &  &  &  &  &  &  &  &  &  &  &  &  &  &  &  &  &  &  &  &  \\
\hline
 &  &  &  &  &  &  &  &  &  &  &  &  &  &  &  &  &  &  &  &  &  &  &  &  &  &  &  &  &  &  \\
\hline
 &  &  &  &  &  &  &  &  &  &  &  &  &  &  &  &  &  &  &  &  &  &  &  &  &  &  &  &  &  &  \\
\hline
 &  &  &  &  &  &  &  &  &  &  &  &  &  &  &  &  &  &  &  &  &  &  &  &  &  &  &  &  &  &  \\
\hline
 &  &  &  &  &  &  &  &  &  &  &  &  &  &  &  &  &  &  &  &  &  &  &  &  &  &  &  &  &  &  \\
\hline
 &  &  &  &  &  &  &  &  &  &  &  &  &  &  &  &  &  &  &  &  &  &  &  &  &  &  &  &  &  &  \\
\hline
 &  &  &  &  &  &  &  &  &  &  &  &  &  &  &  &  &  &  &  &  &  &  &  &  &  &  &  &  &  &  \\
\hline
 &  &  &  &  &  &  &  &  &  &  &  &  &  &  &  &  &  &  &  &  &  &  &  &  &  &  &  &  &  &  \\
\hline
 &  &  &  &  &  &  &  &  &  &  &  &  &  &  &  &  &  &  &  &  &  &  &  &  &  &  &  &  &  &  \\
\hline
 &  &  &  &  &  &  &  &  &  &  &  &  &  &  &  &  &  &  &  &  &  &  &  &  &  &  &  &  &  &  \\
\hline
 &  &  &  &  &  &  &  &  &  &  &  &  &  &  &  &  &  &  &  &  &  &  &  &  &  &  &  &  &  &  \\
\hline
 &  &  &  &  &  &  &  &  &  &  &  &  &  &  &  &  &  &  &  &  &  &  &  &  &  &  &  &  &  &  \\
\hline
 &  &  &  &  &  &  &  &  &  &  &  &  &  &  &  &  &  &  &  &  &  &  &  &  &  &  &  &  &  &  \\
\hline
\end{tabular}
\end{center}

Zadanie 3. (0-1)\\
Dokończ zdanie. Wybierz właściwą odpowiedź spośród podanych.\\
Wszystkich różnych liczb naturalnych czterocyfrowych, które są nieparzyste i podzielne przez 25, jest\\
A. \(9 \cdot 9 \cdot 2\)\\
B. \(9 \cdot 10 \cdot 2\)\\
C. \(9 \cdot 9 \cdot 4\)\\
D. \(9 \cdot 10 \cdot 4\)

\begin{center}
\begin{tabular}{|c|c|c|c|c|c|c|c|c|c|c|c|c|c|c|c|c|c|c|c|c|c|}
\hline
\multicolumn{5}{|l|}{} &  &  &  &  &  &  &  &  &  &  &  &  &  &  &  &  &  \\
\hline
\multicolumn{4}{|l|}{Brudnopis} &  &  &  &  &  &  &  &  &  &  &  &  &  &  &  &  &  &  \\
\hline
 &  &  &  &  &  &  &  &  &  &  &  &  &  &  &  &  &  &  &  &  &  \\
\hline
 &  &  &  &  &  &  &  &  &  &  &  &  &  &  &  &  &  &  &  &  &  \\
\hline
 &  &  &  &  &  &  &  &  &  &  &  &  &  &  &  &  &  &  &  &  &  \\
\hline
 &  &  &  &  &  &  &  &  &  &  &  &  &  &  &  &  &  &  &  &  &  \\
\hline
 &  &  &  &  &  &  &  &  &  &  &  &  &  &  &  &  &  &  &  &  &  \\
\hline
 &  &  &  &  &  &  &  &  &  &  &  &  &  &  &  &  &  &  &  &  &  \\
\hline
 &  &  &  &  &  &  &  &  &  &  &  &  &  &  &  &  &  &  &  &  &  \\
\hline
 &  &  &  &  &  &  &  &  &  &  &  &  &  &  &  &  &  &  &  &  &  \\
\hline
 &  &  &  &  &  &  &  &  &  &  &  &  &  &  &  &  &  &  &  &  &  \\
\hline
 &  &  &  &  &  &  &  &  &  &  &  &  &  &  &  &  &  &  &  &  &  \\
\hline
 &  &  &  &  &  &  &  &  &  &  &  &  &  &  &  &  &  &  &  &  &  \\
\hline
\end{tabular}
\end{center}

\section*{Zadanie 4. (0-1)}
Dokończ zdanie. Wybierz właściwą odpowiedź spośród podanych.\\
Dla każdej liczby rzeczywistej \(x \neq 1\) wyrażenie \(\frac{2}{x-1}-5\) jest równe\\
A. \(\frac{-5 x+1}{x-1}\)\\
B. \(\frac{-5 x+7}{x-1}\)\\
C. \(\frac{-5 x+3}{x-1}\)\\
D. \(\frac{-5 x-3}{x-1}\)

\begin{center}
\begin{tabular}{|c|c|c|c|c|c|c|c|c|c|c|c|c|c|c|c|c|c|c|c|c|c|c|}
\hline
\multicolumn{5}{|l|}{Brudnopis} &  &  &  &  &  &  &  &  & - &  & - &  &  &  & - &  &  &  \\
\hline
 &  &  &  &  &  &  &  &  &  &  &  &  &  &  &  &  &  &  &  &  &  &  \\
\hline
 &  &  &  &  &  &  &  &  &  &  &  &  &  &  &  &  &  &  &  &  &  &  \\
\hline
 &  &  &  &  &  &  &  &  &  &  &  &  &  &  &  &  &  &  &  &  &  &  \\
\hline
 &  &  &  &  &  &  &  &  &  &  &  &  &  &  &  &  &  &  &  &  &  &  \\
\hline
 &  &  &  &  &  &  &  &  &  &  &  &  &  &  &  &  &  &  &  &  &  &  \\
\hline
 &  &  &  &  &  &  &  &  &  &  &  &  &  &  &  &  &  &  &  &  &  &  \\
\hline
 &  &  &  &  &  &  &  &  &  &  &  &  &  &  &  &  &  &  &  &  &  &  \\
\hline
 &  &  &  &  &  &  &  &  &  &  &  &  &  &  &  &  &  &  &  &  &  &  \\
\hline
 &  &  &  &  &  &  &  &  &  &  &  &  &  &  &  &  &  &  &  &  &  &  \\
\hline
 &  &  &  &  &  &  &  &  &  &  &  &  &  &  &  &  &  &  &  &  &  &  \\
\hline
 &  &  &  &  &  &  &  &  &  &  &  &  &  &  &  &  &  &  &  &  &  &  \\
\hline
 &  &  &  &  &  &  &  &  &  &  &  &  &  &  &  &  &  &  &  &  &  &  \\
\hline
 &  &  &  &  &  &  &  &  &  &  &  &  &  &  &  &  &  &  &  &  &  &  \\
\hline
\end{tabular}
\end{center}

Dokończ zdanie. Wybierz dwie właściwe odpowiedzi spośród podanych.\\
Dla każdej liczby rzeczywistej \(x\) i dla każdej liczby rzeczywistej \(y\) wyrażenie \(9-\left(x^{2}-2 x y+y^{2}\right)\) jest równe\\
A. \([3-(x-2 y)]^{2}\)\\
B. \([3+(x-2 y)]^{2}\)\\
C. \([3-(x+2 y)]^{2}\)\\
D. \([3-(x-y)] \cdot[3+(x-y)]\)\\
E. \([3-(x+2 y)] \cdot[3+(x+2 y)]\)\\
F. \(-[(x-y)-3] \cdot[(x-y)+3]\)

\begin{center}
\begin{tabular}{|c|c|c|c|c|c|c|c|c|c|c|c|c|c|c|c|c|c|c|c|c|c|}
\hline
 & Brudn & nopis &  &  &  &  &  &  &  &  &  &  &  &  &  &  &  &  &  &  &  \\
\hline
 &  &  &  &  &  &  &  &  &  &  &  &  &  &  &  &  &  &  &  &  &  \\
\hline
 &  &  &  &  &  &  &  &  &  &  &  &  &  &  &  &  &  &  &  &  &  \\
\hline
 &  &  &  &  &  &  &  &  &  &  &  &  &  &  &  &  &  &  &  &  &  \\
\hline
 &  &  &  &  &  &  &  &  &  &  &  &  &  &  &  &  &  &  &  &  &  \\
\hline
 &  &  &  &  &  &  &  &  &  &  &  &  &  &  &  &  &  &  &  &  &  \\
\hline
 &  &  &  &  &  &  &  &  &  &  &  &  &  &  &  &  &  &  &  &  &  \\
\hline
 &  &  &  &  &  &  &  &  &  &  &  &  &  &  &  &  &  &  &  &  &  \\
\hline
 &  &  &  &  &  &  &  &  &  &  &  &  &  &  &  &  &  &  &  &  &  \\
\hline
 &  &  &  &  &  &  &  &  &  &  &  &  &  &  &  &  &  &  &  &  &  \\
\hline
 &  &  &  &  &  &  &  &  &  &  &  &  &  &  &  &  &  &  &  &  &  \\
\hline
 &  &  &  &  &  &  &  &  &  &  &  &  &  &  &  &  &  &  &  &  &  \\
\hline
 &  &  &  &  &  &  &  &  &  &  &  &  &  &  &  &  &  &  &  &  &  \\
\hline
 &  &  &  &  &  &  &  &  &  &  &  &  &  &  &  &  &  &  &  &  &  \\
\hline
 &  &  &  &  &  &  &  &  &  &  &  &  &  &  &  &  &  &  &  &  &  \\
\hline
 &  &  &  &  &  &  &  &  &  &  &  &  &  &  &  &  &  &  &  &  &  \\
\hline
 &  &  &  &  &  &  &  &  &  &  &  &  &  &  &  &  &  &  &  &  &  \\
\hline
 &  &  &  &  &  &  &  &  &  &  &  &  &  &  &  &  &  &  &  &  &  \\
\hline
 &  &  &  &  &  &  &  &  &  &  &  &  &  &  &  &  &  &  &  &  &  \\
\hline
 &  &  &  &  &  &  &  &  &  &  &  &  &  &  &  &  &  &  &  &  &  \\
\hline
 &  &  &  &  &  &  &  &  &  &  &  &  &  &  &  &  &  &  &  &  &  \\
\hline
 &  &  &  &  &  &  &  &  &  &  &  &  &  &  &  &  &  &  &  &  &  \\
\hline
 &  &  &  &  &  &  &  &  &  &  &  &  &  &  &  &  &  &  &  &  &  \\
\hline
 &  &  &  &  &  &  &  &  &  &  &  &  &  &  &  &  &  &  &  &  &  \\
\hline
 &  &  &  &  &  &  &  &  &  &  &  &  &  &  &  &  &  &  &  &  &  \\
\hline
 &  &  &  &  &  &  &  &  &  &  &  &  &  &  &  &  &  &  &  &  &  \\
\hline
 &  &  &  &  &  &  &  &  &  &  &  &  &  &  &  &  &  &  &  &  &  \\
\hline
 &  &  &  &  &  &  &  &  &  &  &  &  &  &  &  &  &  &  &  &  &  \\
\hline
 &  &  &  &  &  &  &  &  &  &  &  &  &  &  &  &  &  &  &  &  &  \\
\hline
 &  &  &  &  &  &  &  &  &  &  &  &  &  &  &  &  &  &  &  &  &  \\
\hline
\end{tabular}
\end{center}

Zadanie 6. (0-3)\\
Rozwiąż równanie

\[
3 x^{3}-6 x^{2}-27 x+54=0
\]

Zapisz obliczenia.\\
\includegraphics[max width=\textwidth, center]{2025_02_09_354e7a1d815e7eca7072g-07}

\section*{Zadanie 7. (0-1)}
Dokończ zdanie. Wybierz właściwą odpowiedź spośród podanych.\\
Równanie

\[
\frac{\left(x^{2}+x\right)(x+3)(x-1)}{x^{2}-1}=0
\]

ma w zbiorze liczb rzeczywistych dokładnie\\
A. jedno rozwiązanie: \(x=-3\).\\
B. dwa rozwiązania: \(x=-3, x=0\).\\
C. trzy rozwiązania: \(x=-3, x=-1, x=0\).\\
D. cztery rozwiązania: \(x=-3, x=-1, x=0, x=1\).\\
\includegraphics[max width=\textwidth, center]{2025_02_09_354e7a1d815e7eca7072g-08(1)}

\section*{Zadanie 8. (0-1) 두}
Spośród nierówności A-D wybierz tę, której zbiór wszystkich rozwiązań zaznaczono na osi liczbowej.\\
\includegraphics[max width=\textwidth, center]{2025_02_09_354e7a1d815e7eca7072g-08}\\
A. \(|x+2| \leq 2\)\\
B. \(|x-2| \leq 2\)\\
C. \(|x+2| \geq 2\)\\
D. \(|x-2| \geq 2\)\\
\includegraphics[max width=\textwidth, center]{2025_02_09_354e7a1d815e7eca7072g-08(2)}

Zadanie 9. (0-1)\\
Klient banku wypłacił z bankomatu kwotę 1040 zł. Bankomat wydał kwotę w banknotach o nominałach \(20 \mathrm{zł}, 50 \mathrm{zł}\) oraz \(100 \mathrm{zł}\). Banknotów 100-złotowych było dwa razy więcej niż 50-złotowych, a banknotów 20-złotowych było o 2 mniej niż 50 -złotowych.

Niech \(x\) oznacza liczbę banknotów 50-złotowych, a y - liczbę banknotów 20-złotowych, które otrzymał ten klient.

Dokończ zdanie. Wybierz właściwą odpowiedź spośród podanych.\\
Poprawny układ równań prowadzący do obliczenia liczb \(x\) i \(y\) to\\
A. \(\left\{\begin{array}{l}20 y+50 x+100 \cdot 2 x=1040 \\ y=x-2\end{array}\right.\)\\
B. \(\left\{\begin{array}{l}20 y+50 x+50 x \cdot 2=1040 \\ y=x-2\end{array}\right.\)\\
c. \(\left\{\begin{array}{l}20 y+50 x+100 \cdot 2 x=1040 \\ x=y-2\end{array}\right.\)\\
D. \(\left\{\begin{array}{l}20 y+50 x+50 x \cdot 2=1040 \\ x=y-2\end{array}\right.\)\\
\includegraphics[max width=\textwidth, center]{2025_02_09_354e7a1d815e7eca7072g-09}

\section*{Zadanie 10.}
Na rysunku, w kartezjańskim układzie współrzędnych ( \(x, y\) ), przedstawiono wykres funkcji \(f\) określonej dla każdego \(x \in[-5,4)\). Na tym wykresie zaznaczono punkty o współrzędnych całkowitych.\\
\includegraphics[max width=\textwidth, center]{2025_02_09_354e7a1d815e7eca7072g-10}

Zadanie 10.1. (0-1)\\
Zapisz w wykropkowanym miejscu zbiór wartości funkcji \(f\).

\section*{Zadanie 10.2. (0-1)}
Oceń prawdziwość poniższych stwierdzeń. Wybierz P, jeśli stwierdzenie jest prawdziwe, albo F - jeśli jest fałszywe.

\begin{center}
\begin{tabular}{|l|c|c|}
\hline
\begin{tabular}{l}
Dla każdego argumentu z przedziału \((-4,-2)\) funkcja \(f\) przyjmuje wartości \\
ujemne. \\
\end{tabular} & \(\mathbf{P}\) & \(\mathbf{F}\) \\
\hline
Funkcja \(f\) ma trzy miejsca zerowe. & \(\mathbf{P}\) & \(\mathbf{F}\) \\
\hline
\end{tabular}
\end{center}

\begin{center}
\includegraphics[max width=\textwidth]{2025_02_09_354e7a1d815e7eca7072g-10(1)}
\end{center}

\section*{Zadanie 10.3.(0-1)ㄸ.⿷匚一}
Dokończ zdanie.Wybierz właściwą odpowiedź spośród podanych.\\
Najmniejsza wartość funkcji \(f\) w przedziale \([-4,0]\) jest równa\\
A.\((-4)\)\\
B.\((-3)\)\\
C.\((-2)\)\\
D. 0

\begin{center}
\begin{tabular}{|c|c|c|c|c|c|c|c|c|c|c|c|c|c|c|c|c|c|c|c|c|c|c|c|}
\hline
\multicolumn{4}{|l|}{Brudnopis} &  &  &  &  &  &  &  &  &  &  &  &  &  &  &  &  &  &  &  &  \\
\hline
 &  &  &  &  &  &  &  &  &  &  &  &  &  &  &  &  &  &  &  &  &  &  &  \\
\hline
 &  &  &  &  &  &  &  &  &  &  &  &  &  &  &  &  &  &  &  &  &  &  &  \\
\hline
 &  &  &  &  &  &  &  &  &  &  &  &  &  &  &  &  &  &  &  &  &  &  &  \\
\hline
 &  &  &  &  &  &  &  &  &  &  &  &  &  &  &  &  &  &  &  &  &  &  &  \\
\hline
 &  &  &  &  &  &  &  &  &  &  &  &  &  &  &  &  &  &  &  &  &  &  &  \\
\hline
 &  &  &  &  &  &  &  &  &  &  &  &  &  &  &  &  &  &  &  &  &  &  &  \\
\hline
 &  &  &  &  &  &  &  &  &  &  &  &  &  &  &  &  &  &  &  &  &  &  &  \\
\hline
 &  &  &  &  &  &  &  &  &  &  &  &  &  &  &  &  &  &  &  &  &  &  &  \\
\hline
 &  &  &  &  &  &  &  &  &  &  &  &  &  &  &  &  &  &  &  &  &  &  &  \\
\hline
 &  &  &  &  &  &  &  &  &  &  &  &  &  &  &  &  &  &  &  &  &  &  &  \\
\hline
 &  &  &  &  &  &  &  &  &  &  &  &  &  &  &  &  &  &  &  &  &  &  &  \\
\hline
 &  &  &  &  &  &  &  &  &  &  &  &  &  &  &  &  &  &  &  &  &  &  &  \\
\hline
 &  &  &  &  &  &  &  &  &  &  &  &  &  &  &  &  &  &  &  &  &  &  &  \\
\hline
\end{tabular}
\end{center}

\section*{Zadanie 11.(0-1)}
W kartezjańskim układzie wspórzzędnych \((x, y)\) dane są:punkt \(A=(8,11)\) oraz okrąg o równaniu \((x-3)^{2}+(y+1)^{2}=25\) .

Dokończ zdanie.Wybierz właściwą odpowiedź spośród podanych.\\
Odległość punktu \(A\) od środka tego okręgu jest równa\\
A. 25\\
B. 13\\
C.\(\sqrt{125}\)\\
D.\(\sqrt{265}\)\\
\includegraphics[max width=\textwidth, center]{2025_02_09_354e7a1d815e7eca7072g-11}

\section*{Zadanie 12.}
Basen ma długość 25 m . W najpłytszym miejscu jego głębokość jest równa \(1,2 \mathrm{~m}\). Przekrój podłużny tego basenu przedstawiono poglądowo na rysunku.\\
Głębokość \(y\) basenu zmienia się wraz z odległością \(x\) od brzegu w sposób opisany funkcją:

\[
y= \begin{cases}a x+b & \text { dla } 0 \leq x \leq 15 \mathrm{~m} \\ 0,18 x-0,9 & \text { dla } 15 \mathrm{~m} \leq x \leq 25 \mathrm{~m}\end{cases}
\]

Odległość \(x\) jest mierzona od płytszego brzegu w poziomie na powierzchni wody (zobacz rysunek). Wielkości \(x\) i \(y\) są wyrażone w metrach.\\
\includegraphics[max width=\textwidth, center]{2025_02_09_354e7a1d815e7eca7072g-12}

\section*{Zadanie 12.1. (0-1) 망}
Dokończ zdanie. Wybierz właściwą odpowiedź spośród podanych.\\
Największa głębokość basenu jest równa\\
A. \(5,4 \mathrm{~m}\)\\
B. \(3,6 \mathrm{~m}\)\\
C. \(2,2 \mathrm{~m}\)\\
D. \(1,8 \mathrm{~m}\)\\
\includegraphics[max width=\textwidth, center]{2025_02_09_354e7a1d815e7eca7072g-12(1)}

Zadanie 12.2. (0-2)\\
Oblicz wartość współczynnika a oraz wartość współczynnika b.

Zapisz obliczenia.\\
\includegraphics[max width=\textwidth, center]{2025_02_09_354e7a1d815e7eca7072g-13}

\section*{Zadanie 13.}
Funkcja kwadratowa \(f\) jest określona wzorem \(f(x)=-(x-1)^{2}+2\).

\section*{Zadanie 13.1. (0-1) 뚜}
Dokończ zdanie. Wybierz właściwą odpowiedź spośród podanych.\\
Wykresem funkcji \(f\) jest parabola, której wierzchołek ma współrzędne\\
A. \((1,2)\)\\
B. \((-1,2)\)\\
C. \((1,-2)\)\\
D. \((-1,-2)\)

\begin{center}
\begin{tabular}{|c|c|c|c|c|c|c|c|c|c|c|c|c|c|c|c|c|c|c|c|c|c|}
\hline
\multicolumn{4}{|l|}{Brudnopis} &  &  &  &  &  & - &  &  &  &  & - &  &  & - & , &  &  &  \\
\hline
 &  &  &  &  &  &  &  &  &  &  &  &  &  &  &  &  &  &  &  &  &  \\
\hline
 &  &  &  &  &  &  &  &  &  &  &  &  &  &  &  &  &  &  &  &  &  \\
\hline
 &  &  &  &  &  &  &  &  &  &  &  &  &  &  &  &  &  &  &  &  &  \\
\hline
 &  &  &  &  &  &  &  &  &  &  &  &  &  &  &  &  &  &  &  &  &  \\
\hline
 &  &  &  &  &  &  &  &  &  &  &  &  &  &  &  &  &  &  &  &  &  \\
\hline
 &  &  &  &  &  &  &  &  &  &  &  &  &  &  &  &  &  &  &  &  &  \\
\hline
 &  &  &  &  &  &  &  &  &  &  &  &  &  &  &  &  &  &  &  &  &  \\
\hline
 &  &  &  &  &  &  &  &  &  &  &  &  &  &  &  &  &  &  &  &  &  \\
\hline
 &  &  &  &  &  &  &  &  &  &  &  &  &  &  &  &  &  &  &  &  &  \\
\hline
 &  &  &  &  &  &  &  &  &  &  &  &  &  &  &  &  &  &  &  &  &  \\
\hline
 &  &  &  &  &  &  &  &  &  &  &  &  &  &  &  &  &  &  &  &  &  \\
\hline
 &  &  &  &  &  &  &  &  &  &  &  &  &  &  &  &  &  &  &  &  &  \\
\hline
\end{tabular}
\end{center}

\section*{Zadanie 13.2. (0-1) ㄸ.p}
Dokończ zdanie. Wybierz właściwą odpowiedź spośród podanych.\\
Zbiorem wartości funkcji \(f\) jest przedział\\
A. \((-\infty, 2]\)\\
B. \((-\infty, 2)\)\\
C. \((2,+\infty)\)\\
D. \([2,+\infty)\)

\begin{center}
\begin{tabular}{|c|c|c|c|c|c|c|c|c|c|c|c|c|c|c|c|c|c|c|c|c|c|c|c|c|c|c|c|c|c|c|c|}
\hline
\multicolumn{6}{|l|}{Brudnopis} &  &  &  &  &  &  &  &  &  &  &  &  &  &  &  &  &  &  &  &  &  &  &  &  &  &  \\
\hline
 &  &  &  &  &  &  &  &  &  &  &  &  &  &  &  &  &  &  &  &  &  &  &  &  &  &  &  &  &  &  &  \\
\hline
 &  &  &  &  &  &  &  &  &  &  &  &  &  &  &  &  &  &  &  &  &  &  &  &  &  &  &  &  &  &  &  \\
\hline
 &  &  &  &  &  &  &  &  &  &  &  &  &  &  &  &  &  &  &  &  &  &  &  &  &  &  &  &  &  &  &  \\
\hline
 &  &  &  &  &  &  &  &  &  &  &  &  &  &  &  &  &  &  &  &  &  &  &  &  &  &  &  &  &  &  &  \\
\hline
 &  &  &  &  &  &  &  &  &  &  &  &  &  &  &  &  &  &  &  &  &  &  &  &  &  &  &  &  &  &  &  \\
\hline
 &  &  &  &  &  &  &  &  &  &  &  &  &  &  &  &  &  &  &  &  &  &  &  &  &  &  &  &  &  &  &  \\
\hline
 &  &  &  &  &  &  &  &  &  &  &  &  &  &  &  &  &  &  &  &  &  &  &  &  &  &  &  &  &  &  &  \\
\hline
 &  &  &  &  &  &  &  &  &  &  &  &  &  &  &  &  &  &  &  &  &  &  &  &  &  &  &  &  &  &  &  \\
\hline
 &  &  &  &  &  &  &  &  &  &  &  &  &  &  &  &  &  &  &  &  &  &  &  &  &  &  &  &  &  &  &  \\
\hline
 &  &  &  &  &  &  &  &  &  &  &  &  &  &  &  &  &  &  &  &  &  &  &  &  &  &  &  &  &  &  &  \\
\hline
 &  &  &  &  &  &  &  &  &  &  &  &  &  &  &  &  &  &  &  &  &  &  &  &  &  &  &  &  &  &  &  \\
\hline
 &  &  &  &  &  &  &  &  &  &  &  &  &  &  &  &  &  &  &  &  &  &  &  &  &  &  &  &  &  &  &  \\
\hline
 &  &  &  &  &  &  &  &  &  &  &  &  &  &  &  &  &  &  &  &  &  &  &  &  &  &  &  &  &  &  &  \\
\hline
\end{tabular}
\end{center}

\section*{Zadanie 14.}
Dany jest ciąg \(\left(a_{n}\right)\) określony wzorem \(a_{n}=\frac{7^{n}}{21}\) dla każdej liczby naturalnej \(n \geq 1\).

\section*{Zadanie 14.1. (0-1) 뚠}
Dokończ zdanie. Wybierz właściwą odpowiedź spośród podanych.

Pięćdziesiątym wyrazem ciągu ( \(a_{n}\) ) jest\\
A. \(\frac{7^{49}}{3}\)\\
B. \(\frac{7^{50}}{3}\)\\
C. \(\frac{7^{51}}{3}\)\\
D. \(\frac{7^{52}}{3}\)\\
\includegraphics[max width=\textwidth, center]{2025_02_09_354e7a1d815e7eca7072g-15}

Zadanie 14.2. (0-1) 뚬\\
Oceń prawdziwość poniższych stwierdzeń. Wybierz P, jeśli stwierdzenie jest prawdziwe, albo F - jeśli jest fałszywe.

\begin{center}
\begin{tabular}{|l|c|c|}
\hline
Ciąg \(\left(a_{n}\right)\) jest geometryczny. & \(\mathbf{P}\) & \(\mathbf{F}\) \\
\hline
Suma trzech początkowych wyrazów ciągu \(\left(a_{n}\right)\) jest równa 20. & \(\mathbf{P}\) & \(\mathbf{F}\) \\
\hline
\end{tabular}
\end{center}

\begin{center}
\includegraphics[max width=\textwidth]{2025_02_09_354e7a1d815e7eca7072g-15(1)}
\end{center}

Zadanie 15. (0-1) \~{}ण्ण\\
Na płaszczyźnie, w kartezjańskim układzie współrzędnych \((x, y)\), dana jest prosta \(k\) o równaniu \(y=3 x+b\), przechodząca przez punkt \(A=(-1,3)\).

Dokończ zdanie. Wybierz właściwą odpowiedź spośród podanych.\\
Współczynnik \(b\) w równaniu tej prostej jest równy\\
A. 0\\
B. 6\\
C. \((-10)\)\\
D. 8\\
\includegraphics[max width=\textwidth, center]{2025_02_09_354e7a1d815e7eca7072g-16(1)}

\section*{Zadanie 16.}
Dany jest ciąg \(\left(a_{n}\right)\) określony wzorem \(a_{n}=3 n-1\) dla każdej liczby naturalnej \(n \geq 1\).\\
Zadanie 16.1. (0-1)\\
Dokończ zdanie tak, aby było prawdziwe. Wybierz odpowiedź A, B albo C oraz jej uzasadnienie 1., 2. albo 3.

Ciąg \(\left(a_{n}\right)\) jest

\begin{center}
\begin{tabular}{|l|l|l|l|l|}
\hline
A. & rosnący, &  & 1. & \(a_{n+1}-a_{n}=-1\) \\
\hline
B. & malejący, & ponieważ dla każdej liczby naturalnej \(n \geq 1\) & 2. & \(a_{n+1}-a_{n}=0\) \\
\cline { 1 - 1 }
 &  & 3. & \(a_{n+1}-a_{n}=3\) &  \\
\hline
C. & stały, &  &  &  \\
\hline
\end{tabular}
\end{center}

\begin{center}
\includegraphics[max width=\textwidth]{2025_02_09_354e7a1d815e7eca7072g-16}
\end{center}

\section*{Zadanie 16.2. (0-1)}
Dokończ zdanie. Wybierz właściwą odpowiedź spośród podanych.\\
Najmniejszą wartością \(n\), dla której wyraz \(a_{n}\) jest większy od 25 , jest\\
A. 8\\
B. 9\\
C. 7\\
D. 26

\begin{center}
\begin{tabular}{|c|c|c|c|c|c|c|c|c|c|c|c|c|c|c|c|c|c|c|c|c|c|c|}
\hline
\multicolumn{4}{|l|}{Brudnopis} &  &  &  &  &  &  &  &  &  &  &  &  &  &  &  &  &  &  &  \\
\hline
 &  &  &  &  &  &  &  &  &  &  &  &  &  &  &  &  &  &  &  &  &  &  \\
\hline
 &  &  &  &  &  &  &  &  &  &  &  &  &  &  &  &  &  &  &  &  &  &  \\
\hline
 &  &  &  &  &  &  &  &  &  &  &  &  &  &  &  &  &  &  &  &  &  &  \\
\hline
 &  &  &  &  &  &  &  &  &  &  &  &  &  &  &  &  &  &  &  &  &  &  \\
\hline
 &  &  &  &  &  &  &  &  &  &  &  &  &  &  &  &  &  &  &  &  &  &  \\
\hline
 &  &  &  &  &  &  &  &  &  &  &  &  &  &  &  &  &  &  &  &  &  &  \\
\hline
 &  &  &  &  &  &  &  &  &  &  &  &  &  &  &  &  &  &  &  &  &  &  \\
\hline
 &  &  &  &  &  &  &  &  &  &  &  &  &  &  &  &  &  &  &  &  &  &  \\
\hline
 &  &  &  &  &  &  &  &  &  &  &  &  &  &  &  &  &  &  &  &  &  &  \\
\hline
 &  &  &  &  &  &  &  &  &  &  &  &  &  &  &  &  &  &  &  &  &  &  \\
\hline
 &  &  &  &  &  &  &  &  &  &  &  &  &  &  &  &  &  &  &  &  &  &  \\
\hline
 &  &  &  &  &  &  &  &  &  &  &  &  &  &  &  &  &  &  &  &  &  &  \\
\hline
 &  &  &  &  &  &  &  &  &  &  &  &  &  &  &  &  &  &  &  &  &  &  \\
\hline
\end{tabular}
\end{center}

\section*{Zadanie 16.3. (0-1)}
Dokończ zdanie. Wybierz właściwą odpowiedź spośród podanych.\\
Suma \(n\) początkowych wyrazów ciągu \(\left(a_{n}\right)\) jest równa 57 dla \(n\) równego\\
A. 6\\
B. 23\\
C. 5\\
D. 11\\
\includegraphics[max width=\textwidth, center]{2025_02_09_354e7a1d815e7eca7072g-17}

Zadanie 17. (0-1)\\
Na płaszczyźnie, w kartezjańskim układzie współrzędnych \((x, y)\), dane są:

\begin{itemize}
  \item prosta \(k\) o równaniu \(y=\frac{1}{2} x+5\)
  \item prosta \(l\) o równaniu \(y-1=-2 x\).
\end{itemize}

Dokończ zdanie. Wybierz właściwą odpowiedź spośród podanych.

Proste \(k\) i \(l\)\\
A. się pokrywaja.\\
B. nie mają punktów wspólnych.\\
C. są prostopadłe.\\
D. przecinają się pod kątem \(30^{\circ}\).

\begin{center}
\begin{tabular}{|c|c|c|c|c|c|c|c|c|c|c|c|c|c|c|c|c|c|c|c|c|c|c|}
\hline
\multicolumn{4}{|l|}{Brudnopis} &  &  &  &  &  &  &  &  &  &  & - &  &  & - &  & - &  &  & - \\
\hline
 &  &  &  &  &  &  &  &  &  &  &  &  &  &  &  &  &  &  &  &  &  &  \\
\hline
 &  &  &  &  &  &  &  &  &  &  &  &  &  &  &  &  &  &  &  &  &  &  \\
\hline
 &  &  &  &  &  &  &  &  &  &  &  &  &  &  &  &  &  &  &  &  &  &  \\
\hline
 &  &  &  &  &  &  &  &  &  &  &  &  &  &  &  &  &  &  &  &  &  &  \\
\hline
 &  &  &  &  &  &  &  &  &  &  &  &  &  &  &  &  &  &  &  &  &  &  \\
\hline
 &  &  &  &  &  &  &  &  &  &  &  &  &  &  &  &  &  &  &  &  &  &  \\
\hline
 &  &  &  &  &  &  &  &  &  &  &  &  &  &  &  &  &  &  &  &  &  &  \\
\hline
\end{tabular}
\end{center}

\section*{Zadanie 18. (0-1)}
Dokończ zdanie. Wybierz właściwą odpowiedź spośród podanych.\\
Wartość wyrażenia \(\left(1-\cos 20^{\circ}\right) \cdot\left(1+\cos 20^{\circ}\right)-\sin ^{2} 20^{\circ}\) jest równa\\
A. \((-1)\)\\
B. 0\\
C. 1\\
D. 20\\
\includegraphics[max width=\textwidth, center]{2025_02_09_354e7a1d815e7eca7072g-18}

Zadanie 19. (0-1)\\
W pojemniku są wyłącznie kule białe i czerwone. Stosunek liczby kul białych do liczby kul czerwonych jest równy 4:5. Z pojemnika losujemy jedną kulę.

Dokończ zdanie. Wybierz właściwą odpowiedź spośród podanych.\\
Prawdopodobieństwo wylosowania kuli białej jest równe\\
A. \(\frac{4}{9}\)\\
B. \(\frac{4}{5}\)\\
C. \(\frac{1}{9}\)\\
D. \(\frac{1}{4}\)\\
\includegraphics[max width=\textwidth, center]{2025_02_09_354e7a1d815e7eca7072g-19(2)}

\section*{Zadanie 20. (0-1) 줌}
Punkty \(A, B\) oraz \(C\) leżą na okręgu o środku w punkcie \(O\). Kąt \(A B O\) ma miarę \(40^{\circ}\), a kąt OBC ma miarę \(10^{\circ}\) (zobacz rysunek).\\
\includegraphics[max width=\textwidth, center]{2025_02_09_354e7a1d815e7eca7072g-19}

Dokończ zdanie. Wybierz właściwą odpowiedź spośród podanych.\\
Miara kąta \(A C O\) jest równa\\
A. \(30^{\circ}\)\\
B. \(40^{\circ}\)\\
C. \(50^{\circ}\)\\
D. \(60^{\circ}\)\\
\includegraphics[max width=\textwidth, center]{2025_02_09_354e7a1d815e7eca7072g-19(1)}\\
21. Zadanie 21. (0-2)

Dany jest trójkąt \(A B C\) o bokach długości 6,7 oraz 8 .

Oblicz cosinus największego kąta tego trójkąta.

\section*{Zapisz obliczenia.}
\begin{center}
\includegraphics[max width=\textwidth]{2025_02_09_354e7a1d815e7eca7072g-20}
\end{center}

Zadanie 22. (0-1)\\
W trójkącie \(A B C\) bok \(A B\) ma długość 4 , a bok \(B C\) ma długość 4,6 . Dwusieczna kąta \(A B C\) przecina bok \(A C\) w punkcie \(D\) takim, że \(|A D|=3,2\) (zobacz rysunek).\\
\includegraphics[max width=\textwidth, center]{2025_02_09_354e7a1d815e7eca7072g-21}

Dokończ zdanie. Wybierz właściwą odpowiedź spośród podanych.\\
Odcinek \(C D\) ma długość\\
A. \(\frac{64}{23}\)\\
B. \(\frac{16}{5}\)\\
C. \(\frac{23}{4}\)\\
D. \(\frac{92}{25}\)\\
\includegraphics[max width=\textwidth, center]{2025_02_09_354e7a1d815e7eca7072g-21(1)}

Zadanie 23. (0-4)\\
Rodzinna firma stolarska produkuje małe wiatraki ogrodowe. Na podstawie analizy rzeczywistych wpływów i wydatków stwierdzono, że:

\begin{itemize}
  \item przychód \(P\) (w złotych) z tygodniowej sprzedaży \(x\) wiatraków można opisać funkcją \(P(x)=251 x\)
  \item koszt \(K\) (w złotych) produkcji \(x\) wiatraków w ciągu jednego tygodnia można określić funkcją \(K(x)=x^{2}+21 x+170\).
\end{itemize}

Tygodniowo w zakładzie można wyprodukować co najwyżej 150 wiatraków.\\
Oblicz, ile tygodniowo wiatraków należy sprzedać, aby zysk zakładu w ciągu jednego tygodnia był największy. Oblicz ten największy zysk.

\section*{Zapisz obliczenia.}
Wskazówka: przyjmij, że zysk jest różnicą przychodu i kosztów.\\
\includegraphics[max width=\textwidth, center]{2025_02_09_354e7a1d815e7eca7072g-22}\\
\includegraphics[max width=\textwidth, center]{2025_02_09_354e7a1d815e7eca7072g-23}

\section*{Zadanie 24.}
Firma \(\mathcal{F}\) zatrudnia 160 osób. Rozkład płac brutto pracowników tej firmy przedstawia poniższy diagram. Na osi poziomej podano - wyrażoną w złotych - miesięczną płacę brutto, a na osi pionowej podano liczbę pracowników firmy \(\mathcal{F}\), którzy otrzymują płacę miesięczną w danej wysokości.\\
\includegraphics[max width=\textwidth, center]{2025_02_09_354e7a1d815e7eca7072g-24}

\section*{Zadanie 24.1. (0-1)}
Dokończ zdanie. Wybierz właściwą odpowiedź spośród podanych.\\
Średnia miesięczna płaca brutto w firmie \(\mathcal{F}\) jest równa\\
A. \(4593,75 \mathrm{zł}\)\\
B. \(4800,00 \mathrm{z} \nmid\)\\
C. \(5360,00 \mathrm{zt}\)\\
D. \(2399,33 \mathrm{zł}\)\\
\includegraphics[max width=\textwidth, center]{2025_02_09_354e7a1d815e7eca7072g-24(1)}

\section*{Zadanie 24.2. (0-1) 뚱}
Dokończ zdanie. Wybierz właściwą odpowiedź spośród podanych.\\
Mediana miesięcznej płacy pracowników firmy \(\mathcal{F}\) jest równa\\
A. \(4000 \mathrm{zł}\)\\
B. 4800 zt\\
C. \(5000 \mathrm{zł}\)\\
D. \(5500 \mathrm{zł}\)

\begin{center}
\begin{tabular}{|c|c|c|c|c|c|c|c|c|c|c|c|c|c|c|c|c|c|c|c|c|c|c|c|c|c|c|c|c|c|}
\hline
Br & d & nop &  &  &  &  &  &  &  &  &  &  &  &  &  &  &  &  &  &  &  &  &  &  &  &  &  &  &  \\
\hline
 &  &  &  &  &  &  &  &  &  &  &  &  &  &  &  &  &  &  &  &  &  &  &  &  &  &  &  &  &  \\
\hline
 &  &  &  &  &  &  &  &  &  &  &  &  &  &  &  &  &  &  &  &  &  &  &  &  &  &  &  &  &  \\
\hline
 &  &  &  &  &  &  &  &  &  &  &  &  &  &  &  &  &  &  &  &  &  &  &  &  &  &  &  &  &  \\
\hline
 &  &  &  &  &  &  &  &  &  &  &  &  &  &  &  &  &  &  &  &  &  &  &  &  &  &  &  &  &  \\
\hline
 &  &  &  &  &  &  &  &  &  &  &  &  &  &  &  &  &  &  &  &  &  &  &  &  &  &  &  &  & \includegraphics[max width=\textwidth]{2025_02_09_354e7a1d815e7eca7072g-25}
 \\
\hline
\end{tabular}
\end{center}

\section*{Zadanie 24.3. (0-1)}
Dokończ zdanie. Wybierz właściwą odpowiedź spośród podanych.\\
Liczba pracowników firmy \(\mathcal{F}\), których miesięczna płaca brutto nie przewyższa\\
5000 zł, stanowi (w zaokrągleniu do 1\%)\\
A. \(91 \%\) liczby wszystkich pracowników tej firmy.\\
B. \(78 \%\) liczby wszystkich pracowników tej firmy.\\
C. \(53 \%\) liczby wszystkich pracowników tej firmy.\\
D. \(22 \%\) liczby wszystkich pracowników tej firmy.

\begin{center}
\begin{tabular}{|c|c|c|c|c|c|c|c|c|c|c|c|c|c|c|c|c|c|c|c|c|c|c|}
\hline
\multicolumn{4}{|l|}{Brudnopis} &  &  &  &  &  &  &  &  &  &  &  &  &  &  &  &  &  &  &  \\
\hline
 &  &  &  &  &  &  &  &  &  &  &  &  &  &  &  &  &  &  &  &  &  &  \\
\hline
 &  &  &  &  &  &  &  &  &  &  &  &  &  &  &  &  &  &  &  &  &  &  \\
\hline
 &  &  &  &  &  &  &  &  &  &  &  &  &  &  &  &  &  &  &  &  &  &  \\
\hline
 &  &  &  &  &  &  &  &  &  &  &  &  &  &  &  &  &  &  &  &  &  &  \\
\hline
 &  &  &  &  &  &  &  &  &  &  &  &  &  &  &  &  &  &  &  &  &  &  \\
\hline
 &  &  &  &  &  &  &  &  &  &  &  &  &  &  &  &  &  &  &  &  &  &  \\
\hline
 &  &  &  &  &  &  &  &  &  &  &  &  &  &  &  &  &  &  &  &  &  &  \\
\hline
 &  &  &  &  &  &  &  &  &  &  &  &  &  &  &  &  &  &  &  &  &  &  \\
\hline
 &  &  &  &  &  &  &  &  &  &  &  &  &  &  &  &  &  &  &  &  &  &  \\
\hline
 &  &  &  &  &  &  &  &  &  &  &  &  &  &  &  &  &  &  &  &  &  &  \\
\hline
 &  &  &  &  &  &  &  &  &  &  &  &  &  &  &  &  &  &  &  &  &  &  \\
\hline
 &  &  &  &  &  &  &  &  &  &  &  &  &  &  &  &  &  &  &  &  &  &  \\
\hline
 &  &  &  &  &  &  &  &  &  &  &  &  &  &  &  &  &  &  &  &  &  &  \\
\hline
 &  &  &  &  &  &  &  &  &  &  &  &  &  &  &  &  &  &  &  &  &  &  \\
\hline
 &  &  &  &  &  &  &  &  &  &  &  &  &  &  &  &  &  &  &  &  &  &  \\
\hline
 &  &  &  &  &  &  &  &  &  &  &  &  &  &  &  &  &  &  &  &  &  &  \\
\hline
\end{tabular}
\end{center}

Zadanie 25. (0-3)\\
Każda z krawędzi podstawy trójkątnej ostrosłupa ma długość \(10 \sqrt{3}\), a każda jego krawędź boczna ma długość 15.

Oblicz wysokość tego ostrosłupa.\\
Zapisz obliczenia.\\
\includegraphics[max width=\textwidth, center]{2025_02_09_354e7a1d815e7eca7072g-26}

Zadanie 26. (0-2)\\
Wykaż, że dla każdej liczby naturalnej \(n\) liczba \(10 n^{2}+30 n+8\) przy dzieleniu przez 5 daje resztę 3.

\begin{center}
\begin{tabular}{|c|c|c|c|c|c|c|c|c|c|c|c|c|c|c|c|c|c|c|c|c|c|}
\hline
 &  &  &  &  &  &  &  &  &  &  &  &  &  &  &  &  &  &  & - &  &  \\
\hline
 &  &  &  &  &  &  &  &  &  &  &  &  &  &  &  &  &  &  &  &  &  \\
\hline
 &  &  &  &  &  &  &  &  &  &  &  &  &  &  &  &  &  &  &  &  &  \\
\hline
 &  &  &  &  &  &  &  &  &  &  &  &  &  &  &  &  &  &  &  &  &  \\
\hline
 &  &  &  &  &  &  &  &  &  &  &  &  &  &  &  &  &  &  &  &  &  \\
\hline
 &  &  &  &  &  &  &  &  &  &  &  &  &  &  &  &  &  &  &  &  &  \\
\hline
 &  &  &  &  &  &  &  &  &  &  &  &  &  &  &  &  &  &  &  &  &  \\
\hline
 &  &  &  &  &  &  &  &  &  &  &  &  &  &  &  &  &  &  &  &  &  \\
\hline
 &  &  &  &  &  &  &  &  &  &  &  &  &  &  &  &  &  &  &  &  &  \\
\hline
 &  &  &  &  &  &  &  &  &  &  &  &  &  &  &  &  &  &  &  &  &  \\
\hline
 &  &  &  &  &  &  &  &  &  &  &  &  &  &  &  &  &  &  &  &  &  \\
\hline
 &  &  &  &  &  &  &  &  &  &  &  &  &  &  &  &  &  &  &  &  &  \\
\hline
 &  &  &  &  &  &  &  &  &  &  &  &  &  &  &  &  &  &  &  &  &  \\
\hline
 &  &  &  &  &  &  &  &  &  &  &  &  &  &  &  &  &  &  &  &  &  \\
\hline
 &  &  &  &  &  &  &  &  &  &  &  &  &  &  &  &  &  &  &  &  &  \\
\hline
 &  &  &  &  &  &  &  &  &  &  &  &  &  &  &  &  &  &  &  &  &  \\
\hline
 &  &  &  &  &  &  &  &  &  &  &  &  &  &  &  &  &  &  &  &  &  \\
\hline
 &  &  &  &  &  &  &  &  &  &  &  &  &  &  &  &  &  &  &  &  &  \\
\hline
 &  &  &  &  &  &  &  &  &  &  &  &  &  &  &  &  &  &  &  &  &  \\
\hline
 &  &  &  &  &  &  &  &  &  &  &  &  &  &  &  &  &  &  &  &  &  \\
\hline
 &  &  &  &  &  &  &  &  &  &  &  &  &  &  &  &  &  &  &  &  &  \\
\hline
 &  &  &  &  &  &  &  &  &  &  &  &  &  &  &  &  &  &  &  &  &  \\
\hline
 &  &  &  &  &  &  &  &  &  &  &  &  &  &  &  &  &  &  &  &  &  \\
\hline
 &  &  &  &  &  &  &  &  &  &  &  &  &  &  &  &  &  &  &  &  &  \\
\hline
 &  &  &  &  &  &  &  &  &  &  &  &  &  &  &  &  &  &  &  &  &  \\
\hline
 &  &  &  &  &  &  &  &  &  &  &  &  &  &  &  &  &  &  &  &  &  \\
\hline
 &  &  &  &  &  &  &  &  &  &  &  &  &  &  &  &  &  &  &  &  &  \\
\hline
 &  &  &  &  &  &  &  &  &  &  &  &  &  &  &  &  &  &  &  &  &  \\
\hline
 &  &  &  &  &  &  &  &  &  &  &  &  &  &  &  &  &  &  &  &  &  \\
\hline
 &  &  &  &  &  &  &  &  &  &  &  &  &  &  &  &  &  &  &  &  &  \\
\hline
 &  &  &  &  &  &  &  &  &  &  &  &  &  &  &  &  &  &  &  &  &  \\
\hline
 &  &  &  &  &  &  &  &  &  &  &  &  &  &  &  &  &  &  &  &  &  \\
\hline
 &  &  &  &  &  &  &  &  &  &  &  &  &  &  &  &  &  &  &  &  &  \\
\hline
 &  &  &  &  &  &  &  &  &  &  &  &  &  &  &  &  &  &  &  &  &  \\
\hline
 &  &  &  &  &  &  &  &  &  &  &  &  &  &  &  &  &  &  &  &  &  \\
\hline
 &  &  &  &  &  &  &  &  &  &  &  &  &  &  &  &  &  &  &  &  &  \\
\hline
 &  &  &  &  &  &  &  &  &  &  &  &  &  &  &  &  &  &  &  &  &  \\
\hline
 &  &  &  &  &  &  &  &  &  &  &  &  &  &  &  &  &  &  &  &  &  \\
\hline
 &  &  &  &  &  &  &  &  &  &  &  &  &  &  &  &  &  &  &  &  &  \\
\hline
 &  &  &  &  &  &  &  &  &  &  &  &  &  &  &  &  &  &  &  &  &  \\
\hline
 &  &  &  &  &  &  &  &  &  &  &  &  &  &  &  &  &  &  &  &  &  \\
\hline
 &  &  &  &  &  &  &  &  &  &  &  &  &  &  &  &  &  &  &  &  &  \\
\hline
 &  &  &  &  &  &  &  &  &  &  &  &  &  &  &  &  &  &  &  &  &  \\
\hline
 &  &  &  &  &  &  &  &  &  &  &  &  &  &  &  &  &  &  &  &  &  \\
\hline
\end{tabular}
\end{center}

BRUDNOPIS (nie podlega ocenie)

\begin{center}
\begin{tabular}{|c|c|c|c|c|c|c|c|c|c|c|c|c|c|c|c|c|c|c|c|c|c|c|c|c|}
\hline
 &  &  &  &  &  &  &  &  &  &  &  &  &  &  &  &  &  &  &  &  &  &  &  &  \\
\hline
 &  &  &  &  &  &  &  &  &  &  &  &  &  &  &  &  &  &  &  &  &  &  &  &  \\
\hline
 &  &  &  &  &  &  &  &  &  &  &  &  &  &  &  &  &  &  &  &  &  &  &  &  \\
\hline
 &  &  &  &  &  &  &  &  &  &  &  &  &  &  &  &  &  &  &  &  &  &  &  &  \\
\hline
 &  &  &  &  &  &  &  &  &  &  &  &  &  &  &  &  &  &  &  &  &  &  &  &  \\
\hline
 &  &  &  &  &  &  &  &  &  &  &  &  &  &  &  &  &  &  &  &  &  &  &  &  \\
\hline
 &  &  &  &  &  &  &  &  &  &  &  &  &  &  &  &  &  &  &  &  &  &  &  &  \\
\hline
 &  &  &  &  &  &  &  &  &  &  &  &  &  &  &  &  &  &  &  &  &  &  &  &  \\
\hline
 &  &  &  &  &  &  &  &  &  &  &  &  &  &  &  &  &  &  &  &  &  &  &  &  \\
\hline
 &  &  &  &  &  &  &  &  &  &  &  &  &  &  &  &  &  &  &  &  &  &  &  &  \\
\hline
 &  &  &  &  &  &  &  &  &  &  &  &  &  &  &  &  &  &  &  &  &  &  &  &  \\
\hline
 &  &  &  &  &  &  &  &  &  &  &  &  &  &  &  &  &  &  &  &  &  &  &  &  \\
\hline
 &  &  &  &  &  &  &  &  &  &  &  &  &  &  &  &  &  &  &  &  &  &  &  &  \\
\hline
 &  &  &  &  &  &  &  &  &  &  &  &  &  &  &  &  &  &  &  &  &  &  &  &  \\
\hline
 &  &  &  &  &  &  &  &  &  &  &  &  &  &  &  &  &  &  &  &  &  &  &  &  \\
\hline
 &  &  &  &  &  &  &  &  &  &  &  &  &  &  &  &  &  &  &  &  &  &  &  &  \\
\hline
 &  &  &  &  &  &  &  &  &  &  &  &  &  &  &  &  &  &  &  &  &  &  &  &  \\
\hline
 &  &  &  &  &  &  &  &  &  &  &  &  &  &  &  &  &  &  &  &  &  &  &  &  \\
\hline
 &  &  &  &  &  &  &  &  &  &  &  &  &  &  &  &  &  &  &  &  &  &  &  &  \\
\hline
 &  &  &  &  &  &  &  &  &  &  &  &  &  &  &  &  &  &  &  &  &  &  &  &  \\
\hline
 &  &  &  &  &  &  &  &  &  &  &  &  &  &  &  &  &  &  &  &  &  &  &  &  \\
\hline
 &  &  &  &  &  &  &  &  &  &  &  &  &  &  &  &  &  &  &  &  &  &  &  &  \\
\hline
 &  &  &  &  &  &  &  &  &  &  &  &  &  &  &  &  &  &  &  &  &  &  &  &  \\
\hline
 &  &  &  &  &  &  &  &  &  &  &  &  &  &  &  &  &  &  &  &  &  &  &  &  \\
\hline
 &  &  &  &  &  &  &  &  &  &  &  &  &  &  &  &  &  &  &  &  &  &  &  &  \\
\hline
 &  &  &  &  &  &  &  &  &  &  &  &  &  &  &  &  &  &  &  &  &  &  &  &  \\
\hline
 &  &  &  &  &  &  &  &  &  &  &  &  &  &  &  &  &  &  &  &  &  &  &  &  \\
\hline
 &  &  &  &  &  &  &  &  &  &  &  &  &  &  &  &  &  &  &  &  &  &  &  &  \\
\hline
 &  &  &  &  &  &  &  &  &  &  &  &  &  &  &  &  &  &  &  &  &  &  &  &  \\
\hline
 &  &  &  &  &  &  &  &  &  &  &  &  &  &  &  &  &  &  &  &  &  &  &  &  \\
\hline
 &  &  &  &  &  &  &  &  &  &  &  &  &  &  &  &  &  &  &  &  &  &  &  &  \\
\hline
 &  &  &  &  &  &  &  &  &  &  &  &  &  &  &  &  &  &  &  &  &  &  &  &  \\
\hline
 &  &  &  &  &  &  &  &  &  &  &  &  &  &  &  &  &  &  &  &  &  &  &  &  \\
\hline
 &  &  &  &  &  &  &  &  &  &  &  &  &  &  &  &  &  &  &  &  &  &  &  &  \\
\hline
 &  &  &  &  &  &  &  &  &  &  &  &  &  &  &  &  &  &  &  &  &  &  &  &  \\
\hline
 &  &  &  &  &  &  &  &  &  &  &  &  &  &  &  &  &  &  &  &  &  &  &  &  \\
\hline
 &  &  &  &  &  &  &  &  &  &  &  &  &  &  &  &  &  &  &  &  &  &  &  &  \\
\hline
 &  &  &  &  &  &  &  &  &  &  &  &  &  &  &  &  &  &  &  &  &  &  &  &  \\
\hline
 &  &  &  &  &  &  &  &  &  &  &  &  &  &  &  &  &  &  &  &  &  &  &  &  \\
\hline
 &  &  &  &  &  &  &  &  &  &  &  &  &  &  &  &  &  &  &  &  &  &  &  &  \\
\hline
 &  &  &  &  &  &  &  &  &  &  &  &  &  &  &  &  &  &  &  &  &  &  &  &  \\
\hline
 &  &  &  &  &  &  &  &  &  &  &  &  &  &  &  &  &  &  &  &  &  &  &  &  \\
\hline
 &  &  &  &  &  &  &  &  &  &  &  &  &  &  &  &  &  &  &  &  &  &  &  &  \\
\hline
 &  &  &  &  &  &  &  &  &  &  &  &  &  &  &  &  &  &  &  &  &  &  &  &  \\
\hline
 &  &  &  &  &  &  &  &  &  &  &  &  &  &  &  &  &  &  &  &  &  &  &  &  \\
\hline
 & - &  &  &  &  &  &  &  &  &  &  &  &  &  &  &  &  &  &  &  &  &  &  &  \\
\hline
 &  &  &  &  &  &  &  &  &  &  &  &  &  &  &  &  &  &  &  &  &  &  &  &  \\
\hline
 &  &  &  &  &  &  &  &  &  &  &  &  &  &  &  &  &  &  &  &  &  &  &  &  \\
\hline
\end{tabular}
\end{center}

\begin{center}
\includegraphics[max width=\textwidth]{2025_02_09_354e7a1d815e7eca7072g-29}
\end{center}

\section*{MATEMATYKA}
\section*{Poziom podstawowy}
Formuła 2023

\section*{MATEMATYKA}
\section*{Poziom podstawowy}
Formuła 2023

\section*{MATEMATYKA}
\section*{Poziom podstawowy}
Formuła 2023


\end{document}