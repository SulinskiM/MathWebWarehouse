\documentclass[10pt]{article}
\usepackage[polish]{babel}
\usepackage[utf8]{inputenc}
\usepackage[T1]{fontenc}
\usepackage{amsmath}
\usepackage{amsfonts}
\usepackage{amssymb}
\usepackage[version=4]{mhchem}
\usepackage{stmaryrd}
\usepackage{graphicx}
\usepackage[export]{adjustbox}
\graphicspath{ {./images/} }
\usepackage{bbold}

\newcommand\Varangle{\mathop{{<\!\!\!\!\!\text{\small)}}\:}\nolimits}

\begin{document}
CENTRALNA\\
KOMISJA\\
EGZAMINACYJNA

Arkusz zawiera informacje prawnie chronione do momentu rozpoczęcia egzaminu.

WYPEŁNIA ZDAJACY

KOD

\begin{center}
\begin{tabular}{|l|l|}
\hline
 &  \\
\hline
\end{tabular}
\end{center}

\begin{center}
\begin{tabular}{|l|l|l|l|l|l|l|l|l|l|l|}
\hline
 &  &  &  &  &  &  &  &  &  &  \\
\hline
\end{tabular}
\end{center}

\section*{Miejsce na naklejke.}
 Sprawdż, czy kod na naklejce to M-100.Jeżeli tak - przyklej naklejkę. Jeżeli nie - zgłoś to nauczycielowi.

\section*{Egzamin maturalny}
\section*{MATEMATYKA}
\section*{Poziom rozszerzony}
\section*{TEST DIAGNOSTYCZNY}
Symbol arkusza MMAP-R0-100-2212

\section*{DATA: 19 grudnia 2022 r. \\
 GODZINA ROZPOCZĘCIA: 9:00}
CzAS trwania: \(\mathbf{1 8 0}\) minut

\section*{WYPEŁNIA ZESPÓŁ NADZORUJĄCY}
Uprawnienia zdającego do:\\
\includegraphics[max width=\textwidth, center]{2025_02_09_3be06a6bac0961303cf7g-01}

LICZBA PUNKTÓW DO UZYSKANIA: 50

Przed rozpoczęciem pracy z arkuszem egzaminacyjnym

\begin{enumerate}
  \item Sprawdź, czy nauczyciel przekazał Ci właściwy arkusz egzaminacyjny, tj. arkusz we właściwej formule, z właściwego przedmiotu na właściwym poziomie.
  \item Jeżeli przekazano Ci niewłaściwy arkusz - natychmiast zgłoś to nauczycielowi. Nie rozrywaj banderol.
  \item Jeżeli przekazano Ci właściwy arkusz - rozerwij banderole po otrzymaniu takiego polecenia od nauczyciela. Zapoznaj się z instrukcją na stronie 2.\\
\includegraphics[max width=\textwidth, center]{2025_02_09_3be06a6bac0961303cf7g-01(1)}\\
\includegraphics[max width=\textwidth, center]{2025_02_09_3be06a6bac0961303cf7g-02}
\end{enumerate}

\section*{Instrukcja dla zdającego}
\begin{enumerate}
  \item Sprawdź, czy arkusz egzaminacyjny zawiera 25 stron (zadania 1-12). Ewentualny brak zgłoś przewodniczącemu zespołu nadzorującego egzamin.
  \item Na pierwszej stronie arkusza oraz na karcie odpowiedzi wpisz swój numer PESEL i przyklej naklejkę z kodem.
  \item Pamiętaj, że pominięcie argumentacji lub istotnych obliczeń w rozwiązaniu zadania otwartego może spowodować, że za to rozwiązanie nie otrzymasz pełnej liczby punktów.
  \item Rozwiązania zadań i odpowiedzi wpisuj w miejscu na to przeznaczonym.
  \item Pisz czytelnie i używaj tylko długopisu lub pióra z czarnym tuszem lub atramentem.
  \item Nie używaj korektora, a błędne zapisy wyraźnie przekreśl.
  \item Nie wpisuj żadnych znaków w tabelkach przeznaczonych dla egzaminatora. Tabelki umieszczone są na marginesie przy każdym zadaniu.
  \item Pamiętaj, że zapisy w brudnopisie nie będą oceniane.
  \item Możesz korzystać z Wybranych wzorów matematycznych, cyrkla i linijki oraz kalkulatora prostego. Upewnij się, czy przekazano Ci broszurę z okładką taką jak widoczna poniżej.\\
\includegraphics[max width=\textwidth, center]{2025_02_09_3be06a6bac0961303cf7g-02(1)}
\end{enumerate}

\section*{Zadania egzaminacyjne są wydrukowane na następnych stronach.}
\[
\frac{\log _{3} 5 \cdot \log _{25} 27}{\log _{7} \sqrt[6]{49}}
\]

\section*{Zapisz obliczenia}
\begin{center}
\begin{tabular}{|c|c|c|c|c|c|c|c|c|c|c|c|c|c|c|c|c|c|c|c|c|c|c|c|c|c|}
\hline
 &  &  &  &  &  &  &  &  &  &  &  &  &  &  &  &  &  &  &  &  &  &  &  &  &  \\
\hline
 &  &  &  &  &  &  &  &  &  &  &  &  &  &  &  &  &  &  &  &  &  &  &  &  &  \\
\hline
 &  &  &  &  &  &  &  &  &  &  &  &  &  &  &  &  &  &  &  &  &  &  &  &  &  \\
\hline
 &  &  &  &  &  &  &  &  &  &  &  &  &  &  &  &  &  &  &  &  &  &  &  &  &  \\
\hline
 &  &  &  &  &  &  &  &  &  &  &  &  &  &  &  &  &  &  &  &  &  &  &  &  &  \\
\hline
 &  &  &  &  &  &  &  &  &  &  &  &  &  &  &  &  &  &  &  &  &  &  &  &  &  \\
\hline
 &  &  &  &  &  &  &  &  &  &  &  &  &  &  &  &  &  &  &  &  &  &  &  &  &  \\
\hline
 &  &  &  &  &  &  &  &  &  &  &  &  &  &  &  &  &  &  &  &  &  &  &  &  &  \\
\hline
 &  &  &  &  &  &  &  &  &  &  &  &  &  &  &  &  &  &  &  &  &  &  &  &  &  \\
\hline
 &  &  &  &  &  &  &  &  &  &  &  &  &  &  &  &  &  &  &  &  &  &  &  &  &  \\
\hline
 &  &  &  &  &  &  &  &  &  &  &  &  &  &  &  &  &  &  &  &  &  &  &  &  &  \\
\hline
 &  &  &  &  &  &  &  &  &  &  &  &  &  &  &  &  &  &  &  &  &  &  &  &  &  \\
\hline
 &  &  &  &  &  &  &  &  &  &  &  &  &  &  &  &  &  &  &  &  &  &  &  &  &  \\
\hline
 &  &  &  &  &  &  &  &  &  &  &  &  &  &  &  &  &  &  &  &  &  &  &  &  &  \\
\hline
 &  &  &  &  &  &  &  &  &  &  &  &  &  &  &  &  &  &  &  &  &  &  &  &  &  \\
\hline
 &  &  &  &  &  &  &  &  &  &  &  &  &  &  &  &  &  &  &  &  &  &  &  &  &  \\
\hline
 &  &  &  &  &  &  &  &  &  &  &  &  &  &  &  &  &  &  &  &  &  &  &  &  &  \\
\hline
 &  &  &  &  &  &  &  &  &  &  &  &  &  &  &  &  &  &  &  &  &  &  &  &  &  \\
\hline
 &  &  &  &  &  &  &  &  &  &  &  &  &  &  &  &  &  &  &  &  &  &  &  &  &  \\
\hline
 &  &  &  &  &  &  &  &  &  &  &  &  &  &  &  &  &  &  &  &  &  &  &  &  &  \\
\hline
 &  &  &  &  &  &  &  &  &  &  &  &  &  &  &  &  &  &  &  &  &  &  &  &  &  \\
\hline
 &  &  &  &  &  &  &  &  &  &  &  &  &  &  &  &  &  &  &  &  &  &  &  &  &  \\
\hline
 &  &  &  &  &  &  &  &  &  &  &  &  &  &  &  &  &  &  &  &  &  &  &  &  &  \\
\hline
 &  &  &  &  &  &  &  &  &  &  &  &  &  &  &  &  &  &  &  &  &  &  &  &  &  \\
\hline
 &  &  &  &  &  &  &  &  &  &  &  &  &  &  &  &  &  &  &  &  &  &  &  &  &  \\
\hline
 &  &  &  &  &  &  &  &  &  &  &  &  &  &  &  &  &  &  &  &  &  &  &  &  &  \\
\hline
 &  &  &  &  &  &  &  &  &  &  &  &  &  &  &  &  &  &  &  &  &  &  &  &  &  \\
\hline
 &  &  &  &  &  &  &  &  &  &  &  &  &  &  &  &  &  &  &  &  &  &  &  &  &  \\
\hline
 &  &  &  &  &  &  &  &  &  &  &  &  &  &  &  &  &  &  &  &  &  &  &  &  &  \\
\hline
 &  &  &  &  &  &  &  &  &  &  &  &  &  &  &  &  &  &  &  &  &  &  &  &  &  \\
\hline
 &  &  &  &  &  &  &  &  &  &  &  &  &  &  &  &  &  &  &  &  &  &  &  &  &  \\
\hline
 &  &  &  &  &  &  &  &  &  &  &  &  &  &  &  &  &  &  &  &  &  &  &  &  &  \\
\hline
 &  &  &  &  &  &  &  &  &  &  &  &  &  &  &  &  &  &  &  &  &  &  &  &  &  \\
\hline
 &  &  &  &  &  &  &  &  &  &  &  &  &  &  &  &  &  &  &  &  &  &  &  &  &  \\
\hline
 &  &  &  &  &  &  &  &  &  &  &  &  &  &  &  &  &  &  &  &  &  &  &  &  &  \\
\hline
 &  &  &  &  &  &  &  &  &  &  &  &  &  &  &  &  &  &  &  &  &  &  &  &  &  \\
\hline
 &  &  &  &  &  &  &  &  &  &  &  &  &  &  &  &  &  &  &  &  &  &  &  &  &  \\
\hline
 &  &  &  &  &  &  &  &  &  &  &  &  &  &  &  &  &  &  &  &  &  &  &  &  &  \\
\hline
 &  &  &  &  &  &  &  &  &  &  &  &  &  &  &  &  &  &  &  &  &  &  &  &  &  \\
\hline
 &  &  &  &  &  &  &  &  &  &  &  &  &  &  &  &  &  &  &  &  &  &  &  &  &  \\
\hline
 & - &  &  &  &  &  &  &  &  &  & - &  &  &  &  &  &  &  &  &  &  &  &  &  &  \\
\hline
 & - &  &  &  &  &  &  &  &  &  &  &  &  &  &  &  &  &  &  &  &  &  &  &  &  \\
\hline
\end{tabular}
\end{center}

\section*{Zadanie 2.}
Funkcja \(g\) jest określona wzorem \(g(x)=\left|-\frac{1}{4} x^{2}+3 x-5\right|\) dla każdego \(x \in \mathbb{R}\).\\
Fragment wykresu funkcji \(g\) w kartezjańskim układzie współrzędnych ( \(x, y\) ) przedstawiono na rysunku (jednostki pominięto).\\
\includegraphics[max width=\textwidth, center]{2025_02_09_3be06a6bac0961303cf7g-05}

Zadanie 2.1. (0-2)\\
Wyznacz zbiór wszystkich wartości, jakie funkcja \(g\) przyjmuje w przedziale [9, 11]. Zapisz obliczenia.\\
\includegraphics[max width=\textwidth, center]{2025_02_09_3be06a6bac0961303cf7g-05(1)}

Zadanie 2.2. (0-2)\\
Wyznacz zbiór wszystkich wartości parametru \(\boldsymbol{m}\), dla których równanie \(\boldsymbol{g}(\boldsymbol{x})=|\boldsymbol{m}|\) ma dokładnie dwa rozwiązania dodatnie.\\
\includegraphics[max width=\textwidth, center]{2025_02_09_3be06a6bac0961303cf7g-06}

Zadanie 3. (0-3)\\
Wykaż, że dla każdej liczby rzeczywistej \(\boldsymbol{x}\) oraz dla każdej liczby rzeczywistej \(y\), spełniających warunek \(x+y \geq 1\), prawdziwa jest nierówność

\[
x^{3}+2 x y+y^{3} \geq x^{2}+x y(x+y)+y^{2}
\]

\begin{center}
\begin{tabular}{|c|c|c|c|c|c|c|c|c|c|c|c|c|c|c|c|c|c|c|c|c|c|}
\hline
 &  &  &  &  &  &  &  &  &  &  &  &  &  &  &  &  &  &  &  &  &  \\
\hline
 &  &  &  &  &  &  &  &  &  &  &  &  &  &  &  &  &  &  &  &  &  \\
\hline
 &  &  &  &  &  &  &  &  &  &  &  &  &  &  &  &  &  &  &  &  &  \\
\hline
 &  &  &  &  &  &  &  &  &  &  &  &  &  &  &  &  &  &  &  &  &  \\
\hline
 &  &  &  &  &  &  &  &  &  &  &  &  &  &  &  &  &  &  &  &  &  \\
\hline
 &  &  &  &  &  &  &  &  &  &  &  &  &  &  &  &  &  &  &  &  &  \\
\hline
 &  &  &  &  &  &  &  &  &  &  &  &  &  &  &  &  &  &  &  &  &  \\
\hline
 &  &  &  &  &  &  &  &  &  &  &  &  &  &  &  &  &  &  &  &  &  \\
\hline
 &  &  &  &  &  &  &  &  &  &  &  &  &  &  &  &  &  &  &  &  &  \\
\hline
 &  &  &  &  &  &  &  &  &  &  &  &  &  &  &  &  &  &  &  &  &  \\
\hline
 &  &  &  &  &  &  &  &  &  &  &  &  &  &  &  &  &  &  &  &  &  \\
\hline
 &  &  &  &  &  &  &  &  &  &  &  &  &  &  &  &  &  &  &  &  &  \\
\hline
 &  &  &  &  &  &  &  &  &  &  &  &  &  &  &  &  &  &  &  &  &  \\
\hline
 &  &  &  &  &  &  &  &  &  &  &  &  &  &  &  &  &  &  &  &  &  \\
\hline
 &  &  &  &  &  &  &  &  &  &  &  &  &  &  &  &  &  &  &  &  &  \\
\hline
 &  &  &  &  &  &  &  &  &  &  &  &  &  &  &  &  &  &  &  &  &  \\
\hline
 &  &  &  &  &  &  &  &  &  &  &  &  &  &  &  &  &  &  &  &  &  \\
\hline
 &  &  &  &  &  &  &  &  &  &  &  &  &  &  &  &  &  &  &  &  &  \\
\hline
 &  &  &  &  &  &  &  &  &  &  &  &  &  &  &  &  &  &  &  &  &  \\
\hline
 &  &  &  &  &  &  &  &  &  &  &  &  &  &  &  &  &  &  &  &  &  \\
\hline
 &  &  &  &  &  &  &  &  &  &  &  &  &  &  &  &  &  &  &  &  &  \\
\hline
 &  &  &  &  &  &  &  &  &  &  &  &  &  &  &  &  &  &  &  &  &  \\
\hline
 &  &  &  &  &  &  &  &  &  &  &  &  &  &  &  &  &  &  &  &  &  \\
\hline
 &  &  &  &  &  &  &  &  &  &  &  &  &  &  &  &  &  &  &  &  &  \\
\hline
 &  &  &  &  &  &  &  &  &  &  &  &  &  &  &  &  &  &  &  &  &  \\
\hline
 &  &  &  &  &  &  &  &  &  &  &  &  &  &  &  &  &  &  &  &  &  \\
\hline
 &  &  &  &  &  &  &  &  &  &  &  &  &  &  &  &  &  &  &  &  &  \\
\hline
 &  &  &  &  &  &  &  &  &  &  &  &  &  &  &  &  &  &  &  &  &  \\
\hline
 &  &  &  &  &  &  &  &  &  &  &  &  &  &  &  &  &  &  &  &  &  \\
\hline
 &  &  &  &  &  &  &  &  &  &  &  &  &  &  &  &  &  &  &  &  &  \\
\hline
 &  &  &  &  &  &  &  &  &  &  &  &  &  &  &  &  &  &  &  &  &  \\
\hline
 &  &  &  &  &  &  &  &  &  &  &  &  &  &  &  &  &  &  &  &  &  \\
\hline
 &  &  &  &  &  &  &  &  &  &  &  &  &  &  &  &  &  &  &  &  &  \\
\hline
 &  &  &  &  &  &  &  &  &  &  &  &  &  &  &  &  &  &  &  &  &  \\
\hline
 &  &  &  &  &  &  &  &  &  &  &  &  &  &  &  &  &  &  &  &  &  \\
\hline
 &  &  &  &  &  &  &  &  &  &  &  &  &  &  &  &  &  &  &  &  &  \\
\hline
 &  &  &  &  &  &  &  &  &  &  &  &  &  &  &  &  &  &  &  &  &  \\
\hline
 &  &  &  &  &  &  &  &  &  &  &  &  &  &  &  &  &  &  &  &  &  \\
\hline
 &  &  &  &  &  &  &  &  &  &  &  &  &  &  &  &  &  &  &  &  &  \\
\hline
 &  &  &  &  &  &  &  &  &  &  &  &  &  &  &  &  &  &  &  &  &  \\
\hline
 &  &  &  &  &  &  &  &  &  &  &  &  &  &  &  &  &  &  &  &  &  \\
\hline
 &  &  &  &  &  &  &  &  &  &  &  &  &  &  &  &  &  &  &  &  &  \\
\hline
 &  &  &  &  &  &  &  &  &  &  &  &  &  &  &  &  &  &  &  &  &  \\
\hline
\end{tabular}
\end{center}

\section*{Zadanie 4. (0-3)}
Maszyna napełnia torebki herbatą. Każda torebka ma zostać napełniona 200 g herbaty. Torebkę, która zawiera mniej niż 200 g herbaty, nazywamy torebką z niedowagą. Prawdopodobieństwo tego, że pojedyncza torebka napełniona przez tę maszynę jest z niedowagą, jest równe 0,1 . Kontroli poddano masę herbaty w torebkach napełnianych przez tę maszynę danego dnia. Do kontroli wybrano losowo 20 torebek.

Oblicz prawdopodobieństwo zdarzenia polegającego na tym, że wśród tych 20 losowo wybranych torebek znajdą się co najwyżej dwie torebki z niedowagą.

Zapisz obliczenia. Wynik zapisz w zaokrągleniu do drugiego miejsca po przecinku.

\begin{center}
\begin{tabular}{|c|c|c|c|c|c|c|c|c|c|c|c|c|c|c|c|c|c|c|c|c|c|c|c|c|c|}
\hline
 &  &  &  &  &  &  &  &  &  &  &  &  &  &  &  &  &  &  &  &  &  &  &  &  &  \\
\hline
 &  &  &  &  &  &  &  &  &  &  &  &  &  &  &  &  &  &  &  &  &  &  &  &  &  \\
\hline
 &  &  &  &  &  &  &  &  &  &  &  &  &  &  &  &  &  &  &  &  &  &  &  &  &  \\
\hline
 &  &  &  &  &  &  &  &  &  &  &  &  &  &  &  &  &  &  &  &  &  &  &  &  &  \\
\hline
 &  &  &  &  &  &  &  &  &  &  &  &  &  &  &  &  &  &  &  &  &  &  &  &  &  \\
\hline
 &  &  &  &  &  &  &  &  &  &  &  &  &  &  &  &  &  &  &  &  &  &  &  &  &  \\
\hline
 &  &  &  &  &  &  &  &  &  &  &  &  &  &  &  &  &  &  &  &  &  &  &  &  &  \\
\hline
 &  &  &  &  &  &  &  &  &  &  &  &  &  &  &  &  &  &  &  &  &  &  &  &  &  \\
\hline
 &  &  &  &  &  &  &  &  &  &  &  &  &  &  &  &  &  &  &  &  &  &  &  &  &  \\
\hline
 &  &  &  &  &  &  &  &  &  &  &  &  &  &  &  &  &  &  &  &  &  &  &  &  &  \\
\hline
 &  &  &  &  &  &  &  &  &  &  &  &  &  &  &  &  &  &  &  &  &  &  &  &  &  \\
\hline
 &  &  &  &  &  &  &  &  &  &  &  &  &  &  &  &  &  &  &  &  &  &  &  &  &  \\
\hline
 &  &  &  &  &  &  &  &  &  &  &  &  &  &  &  &  &  &  &  &  &  &  &  &  &  \\
\hline
 &  &  &  &  &  &  &  &  &  &  &  &  &  &  &  &  &  &  &  &  &  &  &  &  &  \\
\hline
 &  &  &  &  &  &  &  &  &  &  &  &  &  &  &  &  &  &  &  &  &  &  &  &  &  \\
\hline
 &  &  &  &  &  &  &  &  &  &  &  &  &  &  &  &  &  &  &  &  &  &  &  &  &  \\
\hline
 &  &  &  &  &  &  &  &  &  &  &  &  &  &  &  &  &  &  &  &  &  &  &  &  &  \\
\hline
 &  &  &  &  &  &  &  &  &  &  &  &  &  &  &  &  &  &  &  &  &  &  &  &  &  \\
\hline
 &  &  &  &  &  &  &  &  &  &  &  &  &  &  &  &  &  &  &  &  &  &  &  &  &  \\
\hline
 &  &  &  &  &  &  &  &  &  &  &  &  &  &  &  &  &  &  &  &  &  &  &  &  &  \\
\hline
 &  &  &  &  &  &  &  &  &  &  &  &  &  &  &  &  &  &  &  &  &  &  &  &  &  \\
\hline
 &  &  &  &  &  &  &  &  &  &  &  &  &  &  &  &  &  &  &  &  &  &  &  &  &  \\
\hline
 &  &  &  &  &  &  &  &  &  &  &  &  &  &  &  &  &  &  &  &  &  &  &  &  &  \\
\hline
 &  &  &  &  &  &  &  &  &  &  &  &  &  &  &  &  &  &  &  &  &  &  &  &  &  \\
\hline
 &  &  &  &  &  &  &  &  &  &  &  &  &  &  &  &  &  &  &  &  &  &  &  &  &  \\
\hline
 &  &  &  &  &  &  &  &  &  &  &  &  &  &  &  &  &  &  &  &  &  &  &  &  &  \\
\hline
 &  &  &  &  &  &  &  &  &  &  &  &  &  &  &  &  &  &  &  &  &  &  &  &  &  \\
\hline
 &  &  &  &  &  &  &  &  &  &  &  &  &  &  &  &  &  &  &  &  &  &  &  &  &  \\
\hline
 &  &  &  &  &  &  &  &  &  &  &  &  &  &  &  &  &  &  &  &  &  &  &  &  &  \\
\hline
 &  &  &  &  &  &  &  &  &  &  &  &  &  &  &  &  &  &  &  &  &  &  &  &  &  \\
\hline
 &  &  &  &  &  &  &  &  &  &  &  &  &  &  &  &  &  &  &  &  &  &  &  &  &  \\
\hline
 &  &  &  &  &  &  &  &  &  &  &  &  &  &  &  &  &  &  &  &  &  &  &  &  &  \\
\hline
 &  &  &  &  &  &  &  &  &  &  &  &  &  &  &  &  &  &  &  &  &  &  &  &  &  \\
\hline
 &  &  &  &  &  &  &  &  &  &  &  &  &  &  &  &  &  &  &  &  &  &  &  &  &  \\
\hline
 &  &  &  &  &  &  &  &  &  &  &  &  &  &  &  &  &  &  &  &  &  &  &  &  &  \\
\hline
 &  &  &  &  &  &  &  &  &  &  &  &  &  &  &  &  &  &  &  &  &  &  &  &  &  \\
\hline
 &  &  &  &  &  &  &  &  &  &  &  &  &  &  &  &  &  &  &  &  &  &  &  &  &  \\
\hline
\end{tabular}
\end{center}

Zadanie 5. (0-4)\\
Rozwiąż równanie

\[
6 \sin x+2 \sqrt{3} \cos x+3 \operatorname{tg} x+\sqrt{3}=0
\]

\section*{Zapisz obliczenia.}
\begin{center}
\includegraphics[max width=\textwidth]{2025_02_09_3be06a6bac0961303cf7g-09}
\end{center}

Zadanie 6. (0-4)\\
W trójkącie \(A B C\) poprowadzono dwusieczne kątów przecinające boki \(B C, A C\) i \(A B\) tego trójkąta w punktach - odpowiednio - \(K, L\) oraz \(M\). Punkt \(P\) jest punktem przecięcia tych dwusiecznych. Na czworokątach \(C L P K\) oraz \(B K P M\) można opisać okragg.

Udowodnij, że trójkąt \(A B C\) jest równoboczny.

\begin{center}
\begin{tabular}{|c|c|c|c|c|c|c|c|c|c|c|c|c|c|c|c|c|c|c|c|c|c|c|}
\hline
 &  &  &  &  &  &  &  &  &  &  &  &  &  &  &  &  &  &  &  &  &  &  \\
\hline
 &  &  &  &  &  &  &  &  &  &  &  &  &  &  &  &  &  &  &  &  &  &  \\
\hline
 &  &  &  &  &  &  &  &  &  &  &  &  &  &  &  &  &  &  &  &  &  &  \\
\hline
 &  &  &  &  &  &  &  &  &  &  &  &  &  &  &  &  &  &  &  &  &  &  \\
\hline
 &  &  &  &  &  &  &  &  &  &  &  &  &  &  &  &  &  &  &  &  &  &  \\
\hline
 &  &  &  &  &  &  &  &  &  &  &  &  &  &  &  &  &  &  &  &  &  &  \\
\hline
 &  &  &  &  &  &  &  &  &  &  &  &  &  &  &  &  &  &  &  &  &  &  \\
\hline
 &  &  &  &  &  &  &  &  &  &  &  &  &  &  &  &  &  &  &  &  &  &  \\
\hline
 &  &  &  &  &  &  &  &  &  &  &  &  &  &  &  &  &  &  &  &  &  &  \\
\hline
 &  &  &  &  &  &  &  &  &  &  &  &  &  &  &  &  &  &  &  &  &  &  \\
\hline
 &  &  &  &  &  &  &  &  &  &  &  &  &  &  &  &  &  &  &  &  &  &  \\
\hline
 &  &  &  &  &  &  &  &  &  &  &  &  &  &  &  &  &  &  &  &  &  &  \\
\hline
 &  &  &  &  &  &  &  &  &  &  &  &  &  &  &  &  &  &  &  &  &  &  \\
\hline
 &  &  &  &  &  &  &  &  &  &  &  &  &  &  &  &  &  &  &  &  &  &  \\
\hline
 &  &  &  &  &  &  &  &  &  &  &  &  &  &  &  &  &  &  &  &  &  &  \\
\hline
 &  &  &  &  &  &  &  &  &  &  &  &  &  &  &  &  &  &  &  &  &  &  \\
\hline
 &  &  &  &  &  &  &  &  &  &  &  &  &  &  &  &  &  &  &  &  &  &  \\
\hline
 &  &  &  &  &  &  &  &  &  &  &  &  &  &  &  &  &  &  &  &  &  &  \\
\hline
 &  &  &  &  &  &  &  &  &  &  &  &  &  &  &  &  &  &  &  &  &  &  \\
\hline
 &  &  &  &  &  &  &  &  &  &  &  &  &  &  &  &  &  &  &  &  &  &  \\
\hline
 &  &  &  &  &  &  &  &  &  &  &  &  &  &  &  &  &  &  &  &  &  &  \\
\hline
 &  &  &  &  &  &  &  &  &  &  &  &  &  &  &  &  &  &  &  &  &  &  \\
\hline
 &  &  &  &  &  &  &  &  &  &  &  &  &  &  &  &  &  &  &  &  &  &  \\
\hline
 &  &  &  &  &  &  &  &  &  &  &  &  &  &  &  &  &  &  &  &  &  &  \\
\hline
 &  &  &  &  &  &  &  &  &  &  &  &  &  &  &  &  &  &  &  &  &  &  \\
\hline
 &  &  &  &  &  &  &  &  &  &  &  &  &  &  &  &  &  &  &  &  &  &  \\
\hline
 &  &  &  &  &  &  &  &  &  &  &  &  &  &  &  &  &  &  &  &  &  &  \\
\hline
 &  &  &  &  &  &  &  &  &  &  &  &  &  &  &  &  &  &  &  &  &  &  \\
\hline
 &  &  &  &  &  &  &  &  &  &  &  &  &  &  &  &  &  &  &  &  &  &  \\
\hline
 &  &  &  &  &  &  &  &  &  &  &  &  &  &  &  &  &  &  &  &  &  &  \\
\hline
 &  &  &  &  &  &  &  &  &  &  &  &  &  &  &  &  &  &  &  &  &  &  \\
\hline
 &  &  &  &  &  &  &  &  &  &  &  &  &  &  &  &  &  &  &  &  &  &  \\
\hline
 &  &  &  &  &  &  &  &  &  &  &  &  &  &  &  &  &  &  &  &  &  &  \\
\hline
 &  &  &  &  &  &  &  &  &  &  &  &  &  &  &  &  &  &  &  &  &  &  \\
\hline
 &  &  &  &  &  &  &  &  &  &  &  &  &  &  &  &  &  &  &  &  &  &  \\
\hline
 &  &  &  &  &  &  &  &  &  &  &  &  &  &  &  &  &  &  &  &  &  &  \\
\hline
 &  &  &  &  &  &  &  &  &  &  &  &  &  &  &  &  &  &  &  &  &  &  \\
\hline
 &  &  &  &  &  &  &  &  &  &  &  &  &  &  &  &  &  &  &  &  &  &  \\
\hline
 &  &  &  &  &  &  &  &  &  &  &  &  &  &  &  &  &  &  &  &  &  &  \\
\hline
 &  &  &  &  &  &  &  &  &  &  &  &  &  &  &  &  &  &  &  &  &  &  \\
\hline
 &  &  &  &  &  &  &  &  &  &  &  &  &  &  &  &  &  &  &  &  &  &  \\
\hline
 &  &  &  &  &  &  &  &  &  &  &  &  &  &  &  &  &  &  &  &  &  &  \\
\hline
\end{tabular}
\end{center}

\begin{center}
\includegraphics[max width=\textwidth]{2025_02_09_3be06a6bac0961303cf7g-11}
\end{center}

\section*{Zadanie 7. (0-4)}
Olejarnia wytwarza olej ekologiczny. Aby produkcja była opłacalna, dzienna wielkość produkcji musi wynosić co najmniej 480 litrów i nie może przekroczyć 530 litrów (ze względu na ograniczone moce produkcyjne). Przy poziomie produkcji \((480+x)\) litrów dziennie przeciętny koszt \(K\) (w złotych) wytworzenia jednego litra oleju jest równy\\
\(K(x)=\frac{22 x^{2}-621,5 x+23430}{480+x}\), gdzie \(x \in[0,50]\).

Oblicz, ile litrów oleju dziennie powinna wytworzyć olejarnia, aby przeciętny koszt produkcji jednego litra oleju był najmniejszy (z zachowaniem opłacalności produkcji). Oblicz ten najmniejszy przeciętny koszt.

Zapisz obliczenia.\\
\includegraphics[max width=\textwidth, center]{2025_02_09_3be06a6bac0961303cf7g-12}\\
\includegraphics[max width=\textwidth, center]{2025_02_09_3be06a6bac0961303cf7g-13}\\
\(8 . \quad\) Zadanie 8. (0-5)

\[
\frac{x-1}{x^{2}-4}-\frac{1}{2-x} \geq \frac{3}{2+x}+2
\]

Zapisz obliczenia.\\
\includegraphics[max width=\textwidth, center]{2025_02_09_3be06a6bac0961303cf7g-14}\\
\includegraphics[max width=\textwidth, center]{2025_02_09_3be06a6bac0961303cf7g-15}

Zadanie 9. (0-5)\\
Wyznacz wszystkie wartości parametru m, dla których równanie

\[
x^{2}-(m-4) x+m^{2}-7 m+12=0
\]

ma dwa różne rozwiązania rzeczywiste \(x_{1}\) oraz \(x_{2}\), spełniające warunek

\[
x_{1}^{3}+x_{2}^{3}<5 x_{1}^{2} \cdot x_{2}+5 x_{1} \cdot x_{2}^{2}
\]

\section*{Zapisz obliczenia.}
\includegraphics[max width=\textwidth, center]{2025_02_09_3be06a6bac0961303cf7g-16}\\
\includegraphics[max width=\textwidth, center]{2025_02_09_3be06a6bac0961303cf7g-17}

\section*{Zadanie 10. (0-5)}
Dany jest ostrosłup prawidłowy czworokątny \(A B C D S\) o podstawie \(A B C D\). Krawędź podstawy tego ostrosłupa ma długość \(a\). Ściana boczna jest nachylona do płaszczyzny podstawy pod kątem o mierze \(\alpha\) takim, że \(\cos \alpha=\frac{\sqrt{10}}{10}\). Przez krawędź \(B C\) podstawy ostrosłupa poprowadzono płaszczyznę \(\pi\) prostopadłą do ściany bocznej \(S A D\).

Sporządź rysunek tego ostrosłupa, zaznacz na rysunku przekrój wyznaczony przez płaszczyznę \(\pi\) i nazwij figurę, która jest tym przekrojem. Oblicz pole otrzymanego przekroju.

\section*{Zapisz obliczenia.}
\includegraphics[max width=\textwidth, center]{2025_02_09_3be06a6bac0961303cf7g-18}\\
\includegraphics[max width=\textwidth, center]{2025_02_09_3be06a6bac0961303cf7g-19}

\section*{Zadanie 11. (0-5)}
Dany jest trapez \(A B C D\) o podstawach \(A B\) i \(C D\), w którym \(|A B|>|C D|\) oraz ramię \(B C\) ma długość 6 . Na tym trapezie opisano okrąg o promieniu \(R=5\). Miary kątów \(B A C\) i \(A B C\) tego trapezu spełniają warunek

\[
\frac{\sin |\Varangle B A C|}{\sin |\measuredangle A B C|}=\frac{5}{8}
\]

Oblicz pole i obwód trapezu \(A B C D\).

\section*{Zapisz obliczenia.}
\includegraphics[max width=\textwidth, center]{2025_02_09_3be06a6bac0961303cf7g-20}\\
\includegraphics[max width=\textwidth, center]{2025_02_09_3be06a6bac0961303cf7g-21}

\section*{Zadanie 12. (0-6)}
Prosta \(k\) o równaniu \(x+y-9=0\) przecina parabolę o równaniu \(y=\frac{1}{4} x^{2}-\frac{3}{2} x+\frac{1}{4}\) w punktach \(A\) oraz \(B\). Pierwsza współrzędna punktu \(A\) jest liczbą dodatnią; pierwsza współrzędna punktu \(B\) jest liczbą ujemną. Prosta \(l\) jest równoległa do prostej \(k\) i styczna do danej paraboli w punkcie \(C\).

Oblicz odległość punktu \(C\) od prostej \(k\) oraz pole trójkąta \(A B C\).

\section*{Zapisz obliczenia.}
\begin{center}
\begin{tabular}{|c|c|c|c|c|c|c|c|c|c|c|c|c|c|c|c|c|c|c|c|c|c|c|c|c|}
\hline
 &  &  &  &  &  &  &  &  &  &  &  &  &  &  &  &  &  &  &  &  &  &  &  &  \\
\hline
 &  &  &  &  &  &  &  &  &  &  &  &  &  &  &  &  &  &  &  &  &  &  &  &  \\
\hline
 &  &  &  &  &  &  &  &  &  &  &  &  &  &  &  &  &  &  &  &  &  &  &  &  \\
\hline
 &  &  &  &  &  &  &  &  &  &  &  &  &  &  &  &  &  &  &  &  &  &  &  &  \\
\hline
 &  &  &  &  &  &  &  &  &  &  &  &  &  &  &  &  &  &  &  &  &  &  &  &  \\
\hline
 &  &  &  &  &  &  &  &  &  &  &  &  &  &  &  &  &  &  &  &  &  &  &  &  \\
\hline
 &  &  &  &  &  &  &  &  &  &  &  &  &  &  &  &  &  &  &  &  &  &  &  &  \\
\hline
 &  &  &  &  &  &  &  &  &  &  &  &  &  &  &  &  &  &  &  &  &  &  &  &  \\
\hline
 &  &  &  &  &  &  &  &  &  &  &  &  &  &  &  &  &  &  &  &  &  &  &  &  \\
\hline
 &  &  &  &  &  &  &  &  &  &  &  &  &  &  &  &  &  &  &  &  &  &  &  &  \\
\hline
 &  &  &  &  &  &  &  &  &  &  &  &  &  &  &  &  &  &  &  &  &  &  &  &  \\
\hline
 &  &  &  &  &  &  &  &  &  &  &  &  &  &  &  &  &  &  &  &  &  &  &  &  \\
\hline
 &  &  &  &  &  &  &  &  &  &  &  &  &  &  &  &  &  &  &  &  &  &  &  &  \\
\hline
 &  &  &  &  &  &  &  &  &  &  &  &  &  &  &  &  &  &  &  &  &  &  &  &  \\
\hline
 &  &  &  &  &  &  &  &  &  &  &  &  &  &  &  &  &  &  &  &  &  &  &  &  \\
\hline
 &  &  &  &  &  &  &  &  &  &  &  &  &  &  &  &  &  &  &  &  &  &  &  &  \\
\hline
 &  &  &  &  &  &  &  &  &  &  &  &  &  &  &  &  &  &  &  &  &  &  &  &  \\
\hline
 &  &  &  &  &  &  &  &  &  &  &  &  &  &  &  &  &  &  &  &  &  &  &  &  \\
\hline
 &  &  &  &  &  &  &  &  &  &  &  &  &  &  &  &  &  &  &  &  &  &  &  &  \\
\hline
 &  &  &  &  &  &  &  &  &  &  &  &  &  &  &  &  &  &  &  &  &  &  &  &  \\
\hline
 &  &  &  &  &  &  &  &  &  &  &  &  &  &  &  &  &  &  &  &  &  &  &  &  \\
\hline
 &  &  &  &  &  &  &  &  &  &  &  &  &  &  &  &  &  &  &  &  &  &  &  &  \\
\hline
 &  &  &  &  &  &  &  &  &  &  &  &  &  &  &  &  &  &  &  &  &  &  &  &  \\
\hline
 &  &  &  &  &  &  &  &  &  &  &  &  &  &  &  &  &  &  &  &  &  &  &  &  \\
\hline
 &  &  &  &  &  &  &  &  &  &  &  &  &  &  &  &  &  &  &  &  &  &  &  &  \\
\hline
- &  &  &  &  &  &  &  &  &  &  &  &  &  &  &  &  &  &  &  &  &  &  &  &  \\
\hline
- &  &  &  &  &  &  &  &  &  &  &  &  &  &  &  &  &  &  &  &  &  &  &  &  \\
\hline
- &  &  &  &  &  &  &  &  &  &  &  &  &  &  &  &  &  &  &  &  &  &  &  &  \\
\hline
\(\square\) &  &  &  &  &  &  &  &  &  &  &  &  &  &  &  &  &  &  &  &  &  &  &  &  \\
\hline
 &  &  &  &  &  &  &  &  &  &  &  &  &  &  &  &  &  &  &  &  &  &  &  &  \\
\hline
 &  &  &  &  &  &  &  &  &  &  &  &  &  &  &  &  &  &  &  &  &  &  &  &  \\
\hline
 &  &  &  &  &  &  &  &  &  &  &  &  &  &  &  &  &  &  &  &  &  &  &  &  \\
\hline
 &  &  &  &  &  &  &  &  &  &  &  &  &  &  &  &  &  &  &  &  &  &  &  &  \\
\hline
 &  &  &  &  &  &  &  &  &  &  &  &  &  &  &  &  &  &  &  &  &  &  &  &  \\
\hline
 &  &  &  &  &  &  &  &  &  &  &  &  &  &  &  &  &  &  &  &  &  &  &  &  \\
\hline
 &  &  &  &  &  &  &  &  &  &  &  &  &  &  &  &  &  &  &  &  &  &  &  &  \\
\hline
 &  &  &  &  &  &  &  &  &  &  &  &  &  &  &  &  &  &  &  &  &  &  &  &  \\
\hline
 &  &  &  &  &  &  &  &  &  &  &  &  &  &  &  &  &  &  &  &  &  &  &  &  \\
\hline
\end{tabular}
\end{center}

\begin{center}
\includegraphics[max width=\textwidth]{2025_02_09_3be06a6bac0961303cf7g-23}
\end{center}

BRUDNOPIS (nie podlega ocenie)\\
\includegraphics[max width=\textwidth, center]{2025_02_09_3be06a6bac0961303cf7g-24}\\
\includegraphics[max width=\textwidth, center]{2025_02_09_3be06a6bac0961303cf7g-25}

\section*{MATEMATYKA}
\section*{Poziom rozszerzony}
Formuła 2023

\section*{MATEMATYKA}
\section*{Poziom rozszerzony}
Formuła 2023

\section*{MATEMATYKA}
\section*{Poziom rozszerzony}
Formuła 2023


\end{document}