\documentclass[10pt]{article}
\usepackage[polish]{babel}
\usepackage[utf8]{inputenc}
\usepackage[T1]{fontenc}
\usepackage{amsmath}
\usepackage{amsfonts}
\usepackage{amssymb}
\usepackage[version=4]{mhchem}
\usepackage{stmaryrd}
\usepackage{graphicx}
\usepackage[export]{adjustbox}
\graphicspath{ {./images/} }
\usepackage{multirow}

\title{Kolejne zadania egzaminacyjne są wydrukowane na następnych stronach. }

\author{}
\date{}


\begin{document}
\maketitle
CENTRALNA\\
KOMISJA\\
EGZAMINACYJNA

Arkusz zawiera informacje prawnie chronione do momentu rozpoczęcia egzaminu.

WYPEŁNIA ZDAJACY

KOD

\begin{center}
\begin{tabular}{|l|l|}
\hline
 &  \\
\hline
\end{tabular}
\end{center}

\begin{center}
\begin{tabular}{|l|l|l|l|l|l|l|l|l|l|l|}
\hline
 &  &  &  &  &  &  &  &  &  &  \\
\hline
\end{tabular}
\end{center}

\section*{Miejsce na naklejkę.}
 Sprawdź, czy kod na naklejce to M-100.Jeżeli tak - przyklej naklejkę. Jeżeli nie - zgłoś to nauczycielowi.

\section*{Egzamin maturalny}
\section*{MATEMATYKA}
\section*{Poziom rozszerzony}
\section*{TEST DIAGNOSTYCZNY}
Symbol arkusza\\
MMAP-R0-100-2412

\section*{DATA: \(\mathbf{1 2}\) grudnia 2024 r. \\
 Godzina rozpoczęcia: 9:00}
WYPEŁNIA ZESPÓŁ NADZORUJĄCY\\
Uprawnienia zdającego do:\\
dostosowania zasad oceniania.

Czas trwania: \(\mathbf{1 8 0}\) minut\\
LiczBa punktów do uzyskania: 50

Przed rozpoczęciem pracy z arkuszem egzaminacyjnym

\begin{enumerate}
  \item Sprawdź, czy nauczyciel przekazał Ci właściwy arkusz egzaminacyjny, tj. arkusz we właściwej formule, z właściwego przedmiotu na właściwym poziomie.
  \item Jeżeli przekazano Ci niewłaściwy arkusz - natychmiast zgłoś to nauczycielowi. Nie rozrywaj banderol.
  \item Jeżeli przekazano Ci właściwy arkusz - rozerwij banderole po otrzymaniu takiego polecenia od nauczyciela. Zapoznaj się z instrukcją na stronie 2.\\
\includegraphics[max width=\textwidth, center]{2025_02_09_f8fe43fb3ffece8ad22eg-01}\\
\includegraphics[max width=\textwidth, center]{2025_02_09_f8fe43fb3ffece8ad22eg-02(1)}
\end{enumerate}

\section*{Instrukcja dla zdającego}
\begin{enumerate}
  \item Sprawdź, czy arkusz egzaminacyjny zawiera 30 stron (zadania 1-13). Ewentualny brak zgłoś przewodniczącemu zespołu nadzorującego egzamin.
  \item Na pierwszej stronie arkusza oraz na karcie odpowiedzi wpisz swój numer PESEL i przyklej naklejkę z kodem.
  \item Pamiętaj, że pominięcie argumentacji lub istotnych obliczeń w rozwiązaniu zadania może spowodować, że za to rozwiązanie nie otrzymasz pełnej liczby punktów.
  \item Rozwiązania zadań i odpowiedzi wpisuj w miejscu na to przeznaczonym.
  \item Pisz czytelnie i używaj tylko długopisu lub pióra z czarnym tuszem lub atramentem.
  \item Nie używaj korektora, a błędne zapisy wyraźnie przekreśl.
  \item Nie wpisuj żadnych znaków w tabelkach przeznaczonych dla egzaminatora. Tabelki umieszczone są na marginesie przy każdym zadaniu.
  \item Pamiętaj, że zapisy w brudnopisie nie będą oceniane.
  \item Możesz korzystać z Wybranych wzorów matematycznych, cyrkla i linijki oraz z kalkulatora prostego. Upewnij się, czy przekazano Ci broszurę z okładką taką jak widoczna poniżej.\\
\includegraphics[max width=\textwidth, center]{2025_02_09_f8fe43fb3ffece8ad22eg-02}
\end{enumerate}

\section*{Zadania egzaminacyjne są wydrukowane na następnych stronach.}
\section*{Zadanie 1. (0-2)}
Ładunek elektryczny zgromadzony w kondensatorze można opisać zależnością

\[
Q(t)=Q_{0} \cdot \beta^{-t} \quad \text { dla } \quad t \geq 0
\]

gdzie:\\
\(Q_{0}\) - ładunek elektryczny zgromadzony w kondensatorze w chwili początkowej \((t=0)\) wyrażony w milikulombach\\
\(Q\) - ładunek elektryczny zgromadzony w kondensatorze w chwili \(t\) (licząc od chwili początkowej) wyrażony w milikulombach\\
\(\beta\) - stała dodatnia\\
\(t\) - czas wyrażony w sekundach.\\
Wiadomo, że w chwili \(t=4 \mathrm{~s} \mathrm{w}\) kondensatorze był zgromadzony ładunek 2 milikulombów, a w chwili \(t=6 \mathrm{~s}\)-ładunek 18 milikulombów.

\begin{enumerate}
  \item Oblicz, ile milikulombów ładunku było zgromadzone w tym kondensatorze w chwili \(t=5 \mathrm{~s}\). Zapisz obliczenia.
\end{enumerate}

\begin{center}
\begin{tabular}{|c|c|c|c|c|c|c|c|c|c|c|c|c|c|c|c|c|c|c|c|c|c|c|c|c|c|c|c|c|c|c|}
\hline
 &  &  &  &  &  &  &  &  &  &  &  &  &  &  &  &  &  &  &  &  &  &  &  &  &  &  &  &  &  &  \\
\hline
 &  &  &  &  &  &  &  &  &  &  &  &  &  &  &  &  &  &  &  &  &  &  &  &  &  &  &  &  &  &  \\
\hline
 &  &  &  &  &  &  &  &  &  &  &  &  &  &  &  &  &  &  &  &  &  &  &  &  &  &  &  &  &  &  \\
\hline
 &  &  &  &  &  &  &  &  &  &  &  &  &  &  &  &  &  &  &  &  &  &  &  &  &  &  &  &  &  &  \\
\hline
 &  &  &  &  &  &  &  &  &  &  &  &  &  &  &  &  &  &  &  &  &  &  &  &  &  &  &  &  &  &  \\
\hline
 &  &  &  &  &  &  &  &  &  &  &  &  &  &  &  &  &  &  &  &  &  &  &  &  &  &  &  &  &  &  \\
\hline
 &  &  &  &  &  &  &  &  &  &  &  &  &  &  &  &  &  &  &  &  &  &  &  &  &  &  &  &  &  &  \\
\hline
 &  &  &  &  &  &  &  &  &  &  &  &  &  &  &  &  &  &  &  &  &  &  &  &  &  &  &  &  &  &  \\
\hline
 &  &  &  &  &  &  &  &  &  &  &  &  &  &  &  &  &  &  &  &  &  &  &  &  &  &  &  &  &  &  \\
\hline
 &  &  &  &  &  &  &  &  &  &  &  &  &  &  &  &  &  &  &  &  &  &  &  &  &  &  &  &  &  &  \\
\hline
 &  &  &  &  &  &  &  &  &  &  &  &  &  &  &  &  &  &  &  &  &  &  &  &  &  &  &  &  &  &  \\
\hline
 &  &  &  &  &  &  &  &  &  &  &  &  &  &  &  &  &  &  &  &  &  &  &  &  &  &  &  &  &  &  \\
\hline
 &  &  &  &  &  &  &  &  &  &  &  &  &  &  &  &  &  &  &  &  &  &  &  &  &  &  &  &  &  &  \\
\hline
 &  &  &  &  &  &  &  &  &  &  &  &  &  &  &  &  &  &  &  &  &  &  &  &  &  &  &  &  &  &  \\
\hline
 &  &  &  &  &  &  &  &  &  &  &  &  &  &  &  &  &  &  &  &  &  &  &  &  &  &  &  &  &  &  \\
\hline
 &  &  &  &  &  &  &  &  &  &  &  &  &  &  &  &  &  &  &  &  &  &  &  &  &  &  &  &  &  &  \\
\hline
 &  &  &  &  &  &  &  &  &  &  &  &  &  &  &  &  &  &  &  &  &  &  &  &  &  &  &  &  &  &  \\
\hline
 &  &  &  &  &  &  &  &  &  &  &  &  &  &  &  &  &  &  &  &  &  &  &  &  &  &  &  &  &  &  \\
\hline
 &  &  &  &  &  &  &  &  &  &  &  &  &  &  &  &  &  &  &  &  &  &  &  &  &  &  &  &  &  &  \\
\hline
 &  &  &  &  &  &  &  &  &  &  &  &  &  &  &  &  &  &  &  &  &  &  &  &  &  &  &  &  &  &  \\
\hline
 &  &  &  &  &  &  &  &  &  &  &  &  &  &  &  &  &  &  &  &  &  &  &  &  &  &  &  &  &  &  \\
\hline
 &  &  &  &  &  &  &  &  &  &  &  &  &  &  &  &  &  &  &  &  &  &  &  &  &  &  &  &  &  &  \\
\hline
 &  &  &  &  &  &  &  &  &  &  &  &  &  &  &  &  &  &  &  &  &  &  &  &  &  &  &  &  &  &  \\
\hline
 &  &  &  &  &  &  &  &  &  &  &  &  &  &  &  &  &  &  &  &  &  &  &  &  &  &  &  &  &  &  \\
\hline
 &  &  &  &  &  &  &  &  &  &  &  &  &  &  &  &  &  &  &  &  &  &  &  &  &  &  &  &  &  &  \\
\hline
 &  &  &  &  &  &  &  &  &  &  &  &  &  &  &  &  &  &  &  &  &  &  &  &  &  &  &  &  &  &  \\
\hline
 &  &  &  &  &  &  &  &  &  &  &  &  &  &  &  &  &  &  &  &  &  &  &  &  &  &  &  &  &  &  \\
\hline
\end{tabular}
\end{center}

Zadanie 2. (0-2)\\
Okrąg \(\mathcal{O}\) jest styczny do boków \(A C\) i \(B C\) trójkąta \(A B C\) oraz przecina bok \(A B\) tego trójkąta w punktach \(M\) oraz \(N\), przy czym \(0<|A M|<|A N|<|A B|\).

Wykaż, że jeśli \(|A M|=|B N|\), to trójkąt \(A B C\) jest równoramienny.\\
\includegraphics[max width=\textwidth, center]{2025_02_09_f8fe43fb3ffece8ad22eg-05}

\section*{Zadanie 3. (0-3)}
lloczyn długości średnicy podstawy walca i wysokości walca jest równy \(12 \sqrt{3}\).\\
Pole powierzchni całkowitej tego walca jest równe \(12 \pi(\sqrt{3}+1)\).\\
Oblicz objętość tego walca. Zapisz obliczenia.\\
\includegraphics[max width=\textwidth, center]{2025_02_09_f8fe43fb3ffece8ad22eg-06}

Zadanie 4. (0-3)\\
Wykaż, że

\[
\frac{1}{\log _{2} 35+1}+\frac{1}{\log _{7} 140-\log _{7} 2}+\frac{1}{\log _{5} 7+\log _{5} 2+1}=1
\]

\begin{center}
\includegraphics[max width=\textwidth]{2025_02_09_f8fe43fb3ffece8ad22eg-07}
\end{center}

\section*{Zadanie 5. (0-3)}
W pewnej lokalnej społeczności \(35 \%\) osób ma wyższe wykształcenie. W tej społeczności językiem niemieckim dobrze włada \(70 \%\) osób mających wyższe wykształcenie i \(40 \%\) osób bez wyższego wykształcenia.\\
Spośród członków tej społeczności wybieramy losowo jedną osobę.\\
Oblicz prawdopodobieństwo zdarzenia polegającego na tym, że wybierzemy osobę z wyższym wykształceniem, jeżeli wiadomo, że ta osoba dobrze włada językiem niemieckim. Wynik zapisz w postaci ułamka dziesiętnego w zaokrągleniu do części setnych. Zapisz obliczenia.

\begin{center}
\begin{tabular}{|c|c|c|c|c|c|c|c|c|c|c|c|c|c|c|c|c|c|c|c|c|c|c|c|c|}
\hline
 &  &  &  &  &  &  &  &  &  &  &  &  &  &  &  &  &  &  &  &  &  &  &  &  \\
\hline
 &  &  &  &  &  &  &  &  &  &  &  &  &  &  &  &  &  &  &  &  &  &  &  &  \\
\hline
 &  &  &  &  &  &  &  &  &  &  &  &  &  &  &  &  &  &  &  &  &  &  &  &  \\
\hline
 &  &  &  &  &  &  &  &  &  &  &  &  &  &  &  &  &  &  &  &  &  &  &  &  \\
\hline
 &  &  &  &  &  &  &  &  &  &  &  &  &  &  &  &  &  &  &  &  &  &  &  &  \\
\hline
 &  &  &  &  &  &  &  &  &  &  &  &  &  &  &  &  &  &  &  &  &  &  &  &  \\
\hline
 &  &  &  &  &  &  &  &  &  &  &  &  &  &  &  &  &  &  &  &  &  &  &  &  \\
\hline
 &  &  &  &  &  &  &  &  &  &  &  &  &  &  &  &  &  &  &  &  &  &  &  &  \\
\hline
 &  &  &  &  &  &  &  &  &  &  &  &  &  &  &  &  &  &  &  &  &  &  &  &  \\
\hline
 &  &  &  &  &  &  &  &  &  &  &  &  &  &  &  &  &  &  &  &  &  &  &  &  \\
\hline
 &  &  &  &  &  &  &  &  &  &  &  &  &  &  &  &  &  &  &  &  &  &  &  &  \\
\hline
- &  &  &  &  &  &  &  &  &  &  &  &  &  &  &  &  &  &  &  &  &  &  &  &  \\
\hline
 &  &  &  &  &  &  &  &  &  &  &  &  &  &  &  &  &  &  &  &  &  &  &  &  \\
\hline
 &  &  &  &  &  &  &  &  &  &  &  &  &  &  &  &  &  &  &  &  &  &  &  &  \\
\hline
 &  &  &  &  &  &  &  &  &  &  &  &  &  &  &  &  &  &  &  &  &  &  &  &  \\
\hline
- &  &  &  &  &  &  &  &  &  &  &  &  &  &  &  &  &  &  &  &  &  &  &  &  \\
\hline
 &  &  &  &  &  &  &  &  &  &  &  &  &  &  &  &  &  &  &  &  &  &  &  &  \\
\hline
 &  &  &  &  &  &  &  &  &  &  &  &  &  &  &  &  &  &  &  &  &  &  &  &  \\
\hline
- &  &  &  &  &  &  &  &  &  &  &  &  &  &  &  &  &  &  &  &  &  &  &  &  \\
\hline
- &  &  &  &  &  &  &  &  &  &  &  &  &  &  &  &  &  &  &  &  &  &  &  &  \\
\hline
- &  &  &  &  &  &  &  &  &  &  &  &  &  &  &  &  &  &  &  &  &  &  &  &  \\
\hline
- &  &  &  &  &  &  &  &  &  &  &  &  &  &  &  &  &  &  &  &  &  &  &  &  \\
\hline
 &  &  &  &  &  &  &  &  &  &  &  &  &  &  &  &  &  &  &  &  &  &  &  &  \\
\hline
- &  &  &  &  &  &  &  &  &  &  &  &  &  &  &  &  &  &  &  &  &  &  &  &  \\
\hline
- &  &  &  &  &  &  &  &  &  &  &  &  &  &  &  &  &  &  &  &  &  &  &  &  \\
\hline
- &  &  &  &  &  &  &  &  &  &  &  &  &  &  &  &  &  &  &  &  &  &  &  &  \\
\hline
- &  &  &  &  &  &  &  &  &  &  &  &  &  &  &  &  &  &  &  &  &  &  &  &  \\
\hline
- &  &  &  &  &  &  &  &  &  &  &  &  &  &  &  &  &  &  &  &  &  &  &  &  \\
\hline
 &  &  &  &  &  &  &  &  &  &  &  &  &  &  &  &  &  &  &  &  &  &  &  &  \\
\hline
 &  &  &  &  &  &  &  &  &  &  &  &  &  &  &  &  &  &  &  &  &  &  &  &  \\
\hline
 & - &  &  &  &  &  &  &  &  &  &  &  &  &  &  &  &  &  &  &  &  &  &  &  \\
\hline
 &  &  &  &  &  &  &  &  &  &  &  &  &  &  &  &  &  &  &  &  &  &  &  &  \\
\hline
 &  &  &  &  &  &  &  &  &  &  &  &  &  &  &  &  &  &  &  &  &  &  &  &  \\
\hline
 &  &  &  &  &  &  &  &  &  &  &  &  &  &  &  &  &  &  &  &  &  &  &  &  \\
\hline
 &  &  &  &  &  &  &  &  &  &  &  &  &  &  &  &  &  &  &  &  &  &  &  &  \\
\hline
\end{tabular}
\end{center}

Zadanie 6. (0-4)\\
Rozwiąż równanie

\[
|4 x-8|+|x-2|=|2-x|+|x+2|+4
\]

\section*{Zapisz obliczenia.}
\includegraphics[max width=\textwidth, center]{2025_02_09_f8fe43fb3ffece8ad22eg-10}\\
\includegraphics[max width=\textwidth, center]{2025_02_09_f8fe43fb3ffece8ad22eg-11}

\section*{Zadanie 7. (0-4)}
W kartezjańskim układzie współrzędnych \((x, y)\) dane są:\\
okrąg o równaniu \((x+1)^{2}+(y-3)^{2}=50\) i punkty \(A=(6,4)\) oraz \(B=(-6,8)\).\\
Punkt \(C\) leży na tym okręgui \(|A C|=|B C|\).\\
Oblicz współrzędne punktu C. Rozważ wszystkie przypadki. Zapisz obliczenia.\\
\includegraphics[max width=\textwidth, center]{2025_02_09_f8fe43fb3ffece8ad22eg-12}\\
\includegraphics[max width=\textwidth, center]{2025_02_09_f8fe43fb3ffece8ad22eg-13}

Zadanie 8. (0-4)\\
Oblicz granicę

\[
\lim _{n \rightarrow+\infty} \frac{1+3+5+7+\ldots+(2 n+1)}{\binom{n}{2}}
\]

gdzie \(1+3+5+7+\ldots+(2 n+1)\) jest sumą kolejnych liczb naturalnych nieparzystych. Zapisz obliczenia.

\begin{center}
\begin{tabular}{|c|c|c|c|c|c|c|c|c|c|c|c|c|c|c|c|c|c|c|c|c|c|c|c|c|}
\hline
 &  &  &  &  &  &  &  &  &  &  &  &  &  &  &  &  &  &  &  &  &  &  &  &  \\
\hline
\multicolumn{25}{|l|}{\multirow[t]{2}{*}{}} \\
\hline
 &  &  &  &  &  &  &  &  &  &  &  &  &  &  &  &  &  &  &  &  &  &  &  &  \\
\hline
 &  &  &  &  &  &  &  &  &  &  &  &  &  &  &  &  &  &  &  &  &  &  &  &  \\
\hline
\multicolumn{25}{|l|}{} \\
\hline
\multicolumn{25}{|l|}{} \\
\hline
\multicolumn{25}{|l|}{\(\square\)} \\
\hline
\multicolumn{25}{|l|}{\multirow[t]{2}{*}{}} \\
\hline
 &  &  &  &  &  &  &  &  &  &  &  &  &  &  &  &  &  &  &  &  &  &  &  &  \\
\hline
\multicolumn{25}{|l|}{} \\
\hline
\multicolumn{25}{|l|}{} \\
\hline
 &  &  &  &  &  &  &  &  &  &  &  &  &  &  &  &  &  &  &  &  &  &  &  &  \\
\hline
\multicolumn{25}{|l|}{} \\
\hline
\multicolumn{25}{|l|}{} \\
\hline
\multicolumn{25}{|l|}{} \\
\hline
\multicolumn{25}{|l|}{} \\
\hline
\multicolumn{25}{|l|}{} \\
\hline
\multicolumn{25}{|l|}{} \\
\hline
\multicolumn{25}{|l|}{} \\
\hline
\multicolumn{25}{|l|}{} \\
\hline
\multicolumn{25}{|l|}{} \\
\hline
\multicolumn{25}{|l|}{} \\
\hline
\multicolumn{25}{|l|}{} \\
\hline
\multicolumn{25}{|l|}{} \\
\hline
\multicolumn{25}{|l|}{} \\
\hline
\multicolumn{25}{|l|}{} \\
\hline
\multicolumn{25}{|l|}{} \\
\hline
\multicolumn{25}{|l|}{} \\
\hline
\multicolumn{25}{|l|}{} \\
\hline
\multicolumn{25}{|l|}{} \\
\hline
\multicolumn{25}{|l|}{} \\
\hline
\multicolumn{25}{|l|}{} \\
\hline
\multicolumn{25}{|l|}{} \\
\hline
\multicolumn{25}{|l|}{} \\
\hline
\multicolumn{25}{|l|}{} \\
\hline
\multicolumn{25}{|l|}{} \\
\hline
\multicolumn{25}{|l|}{\multirow[t]{3}{*}{\includegraphics[max width=\textwidth]{2025_02_09_f8fe43fb3ffece8ad22eg-14}
}} \\
\hline
 &  &  &  &  &  &  &  &  &  &  &  &  &  &  &  &  &  &  &  &  &  &  &  &  \\
\hline
 &  &  &  &  &  &  &  &  &  &  &  &  &  &  &  &  &  &  &  &  &  &  &  &  \\
\hline
\end{tabular}
\end{center}

\begin{center}
\includegraphics[max width=\textwidth]{2025_02_09_f8fe43fb3ffece8ad22eg-15}
\end{center}

\[
\sin ^{4} x=\sin x \cdot \cos x-\cos ^{4} x
\]

w zbiorze \([-\pi, 2 \pi]\). Zapisz obliczenia.\\
\includegraphics[max width=\textwidth, center]{2025_02_09_f8fe43fb3ffece8ad22eg-16}\\
\includegraphics[max width=\textwidth, center]{2025_02_09_f8fe43fb3ffece8ad22eg-17}

Zadanie 10. (0-5)\\
Trzeci i piąty wyraz malejącego ciągu arytmetycznego \(\left(a_{n}\right)\), określonego dla każdej liczby naturalnej \(n \geq 1\), spełniają warunek \(a_{3}+a_{5}=10\).\\
Trzywyrazowy ciąg ( \(2 a_{1}+4, a_{4}-1,-\frac{1}{8} a_{7}\) ) jest geometryczny.\\
Oblicz wyrazy tego ciągu geometrycznego. Zapisz obliczenia.

\begin{center}
\begin{tabular}{|c|c|c|c|c|c|c|c|c|c|c|c|c|c|c|c|c|c|c|c|c|c|c|c|c|c|c|c|c|}
\hline
 &  &  &  &  &  &  &  &  &  &  &  &  &  &  &  &  &  &  &  &  &  &  &  &  &  &  &  &  \\
\hline
 &  &  &  &  &  &  &  &  &  &  &  &  &  &  &  &  &  &  &  &  &  &  &  &  &  &  &  &  \\
\hline
 &  &  &  &  &  &  &  &  &  &  &  &  &  &  &  &  &  &  &  &  &  &  &  &  &  &  &  &  \\
\hline
 &  &  &  &  &  &  &  &  &  &  &  &  &  &  &  &  &  &  &  &  &  &  &  &  &  &  &  &  \\
\hline
 &  &  &  &  &  &  &  &  &  &  &  &  &  &  &  &  &  &  &  &  &  &  &  &  &  &  &  &  \\
\hline
 &  &  &  &  &  &  &  &  &  &  &  &  &  &  &  &  &  &  &  &  &  &  &  &  &  &  &  &  \\
\hline
 &  &  &  &  &  &  &  &  &  &  &  &  &  &  &  &  &  &  &  &  &  &  &  &  &  &  &  &  \\
\hline
 &  &  &  &  &  &  &  &  &  &  &  &  &  &  &  &  &  &  &  &  &  &  &  &  &  &  &  &  \\
\hline
 &  &  &  &  &  &  &  &  &  &  &  &  &  &  &  &  &  &  &  &  &  &  &  &  &  &  &  &  \\
\hline
 &  &  &  &  &  &  &  &  &  &  &  &  &  &  &  &  &  &  &  &  &  &  &  &  &  &  &  &  \\
\hline
 &  &  &  &  &  &  &  &  &  &  &  &  &  &  &  &  &  &  &  &  &  &  &  &  &  &  &  &  \\
\hline
 &  &  &  &  &  &  &  &  &  &  &  &  &  &  &  &  &  &  &  &  &  &  &  &  &  &  &  &  \\
\hline
 &  &  &  &  &  &  &  &  &  &  &  &  &  &  &  &  &  &  &  &  &  &  &  &  &  &  &  &  \\
\hline
 &  &  &  &  &  &  &  &  &  &  &  &  &  &  &  &  &  &  &  &  &  &  &  &  &  &  &  &  \\
\hline
 &  &  &  &  &  &  &  &  &  &  &  &  &  &  &  &  &  &  &  &  &  &  &  &  &  &  &  &  \\
\hline
 &  &  &  &  &  &  &  &  &  &  &  &  &  &  &  &  &  &  &  &  &  &  &  &  &  &  &  &  \\
\hline
 &  &  &  &  &  &  &  &  &  &  &  &  &  &  &  &  &  &  &  &  &  &  &  &  &  &  &  &  \\
\hline
 &  &  &  &  &  &  &  &  &  &  &  &  &  &  &  &  &  &  &  &  &  &  &  &  &  &  &  &  \\
\hline
 &  &  &  &  &  &  &  &  &  &  &  &  &  &  &  &  &  &  &  &  &  &  &  &  &  &  &  &  \\
\hline
 &  &  &  &  &  &  &  &  &  &  &  &  &  &  &  &  &  &  &  &  &  &  &  &  &  &  &  &  \\
\hline
 &  &  &  &  &  &  &  &  &  &  &  &  &  &  &  &  &  &  &  &  &  &  &  &  &  &  &  &  \\
\hline
 &  &  &  &  &  &  &  &  &  &  &  &  &  &  &  &  &  &  &  &  &  &  &  &  &  &  &  &  \\
\hline
 &  &  &  &  &  &  &  &  &  &  &  &  &  &  &  &  &  &  &  &  &  &  &  &  &  &  &  &  \\
\hline
 &  &  &  &  &  &  &  &  &  &  &  &  &  &  &  &  &  &  &  &  &  &  &  &  &  &  &  &  \\
\hline
 &  &  &  &  &  &  &  &  &  &  &  &  &  &  &  &  &  &  &  &  &  &  &  &  &  &  &  &  \\
\hline
 &  &  &  &  &  &  &  &  &  &  &  &  &  &  &  &  &  &  &  &  &  &  &  &  &  &  &  &  \\
\hline
 &  &  &  &  &  &  &  &  &  &  &  &  &  &  &  &  &  &  &  &  &  &  &  &  &  &  &  &  \\
\hline
 &  &  &  &  &  &  &  &  &  &  &  &  &  &  &  &  &  &  &  &  &  &  &  &  &  &  &  &  \\
\hline
 &  &  &  &  &  &  &  &  &  &  &  &  &  &  &  &  &  &  &  &  &  &  &  &  &  &  &  &  \\
\hline
 &  &  &  &  &  &  &  &  &  &  &  &  &  &  &  &  &  &  &  &  &  &  &  &  &  &  &  &  \\
\hline
 &  &  &  &  &  &  &  &  &  &  &  &  &  &  &  &  &  &  &  &  &  &  &  &  &  &  &  &  \\
\hline
 &  &  &  &  &  &  &  &  &  &  &  &  &  &  &  &  &  &  &  &  &  &  &  &  &  &  &  &  \\
\hline
 &  &  &  &  &  &  &  &  &  &  &  &  &  &  &  &  &  &  &  &  &  &  &  &  &  &  &  &  \\
\hline
 &  &  &  &  &  &  &  &  &  &  &  &  &  &  &  &  &  &  &  &  &  &  &  &  &  &  &  &  \\
\hline
 &  &  &  &  &  &  &  &  &  &  &  &  &  &  &  &  &  &  &  &  &  &  &  &  &  &  &  &  \\
\hline
 &  &  &  &  &  &  &  &  &  &  &  &  &  &  &  &  &  &  &  &  &  &  &  &  & \includegraphics[max width=\textwidth]{2025_02_09_f8fe43fb3ffece8ad22eg-18}
 &  &  &  \\
\hline
 &  &  &  &  &  &  &  &  &  &  &  &  &  &  &  &  &  &  &  &  &  &  &  &  &  &  &  &  \\
\hline
 &  &  &  &  &  &  &  &  &  &  &  &  &  &  &  &  &  &  &  &  &  &  &  &  &  &  &  &  \\
\hline
\end{tabular}
\end{center}

\begin{center}
\includegraphics[max width=\textwidth]{2025_02_09_f8fe43fb3ffece8ad22eg-19}
\end{center}

\section*{Zadanie 11. (0-5)}
Funkcja kwadratowa \(f\) zmiennej rzeczywistej \(x\) jest określona wzorem

\[
f(x)=x^{2}-3 x-m^{2}+m+3
\]

Wyznacz wszystkie wartości parametru \(m\), dla których funkcja \(f\) ma dwa różne miejsca zerowe \(x_{1}, x_{2}\) spełniające warunek \(\left|x_{1}^{2}-x_{2}^{2}\right| \leq 12\). Zapisz obliczenia.\\
\includegraphics[max width=\textwidth, center]{2025_02_09_f8fe43fb3ffece8ad22eg-20}\\
\includegraphics[max width=\textwidth, center]{2025_02_09_f8fe43fb3ffece8ad22eg-21}

Zadanie 12. (0-5)\\
W trójkącie ostrokątnym \(A B C\) miara kąta \(B A C\) jest dwa razy większa od miary kąta \(A B C\). Punkt \(D\) jest środkiem boku \(A B\). Niech \(\alpha\) oznacza miarę kąta \(A B C\), natomiast \(\beta\)-miarę kąta \(A D C\) (zobacz rysunek).\\
\includegraphics[max width=\textwidth, center]{2025_02_09_f8fe43fb3ffece8ad22eg-22}

Oblicz \(\frac{\operatorname{tg} \beta}{\sin (2 \alpha)}\). Zapisz obliczenia.\\
\includegraphics[max width=\textwidth, center]{2025_02_09_f8fe43fb3ffece8ad22eg-22(1)}\\
\includegraphics[max width=\textwidth, center]{2025_02_09_f8fe43fb3ffece8ad22eg-23}\\
\includegraphics[max width=\textwidth, center]{2025_02_09_f8fe43fb3ffece8ad22eg-24}

\section*{Zadanie 13.}
Funkcja \(f\) jest określona wzorem \(f(x)=\frac{12 x-84}{x-8}\) dla każdego \(x \in(-\infty, 8)\).\\
W kartezjańskim układzie współrzędnych \((x, y)\) rozważamy wszystkie czworokąty \(O B C D\), w których:

\begin{itemize}
  \item wierzchołek \(O\) ma wspórzzędne \((0,0)\)
  \item wierzchołki \(B\) oraz \(D\) są punktami przecięcia wykresu funkcji \(f\) z osią odpowiednio - Ox i Oy
  \item wierzchołek \(C\) ma obie współrzędne dodatnie i leży na wykresie funkcji \(f\) (zobacz rysunek).\\
\includegraphics[max width=\textwidth, center]{2025_02_09_f8fe43fb3ffece8ad22eg-25(1)}
\end{itemize}

\section*{Zadanie 13.1. (0-2)}
Wykaż, że pole \(P\) czworokąta \(O B C D\) w zależności od pierwszej współrzędnej \(\boldsymbol{x}\) punktu \(C\) jest określone wzorem

\[
P(x)=\frac{21}{4} \cdot \frac{x^{2}-56}{x-8}
\]

\includegraphics[max width=\textwidth, center]{2025_02_09_f8fe43fb3ffece8ad22eg-25}\\
\includegraphics[max width=\textwidth, center]{2025_02_09_f8fe43fb3ffece8ad22eg-26}

Zadanie 13.2. (0-4)\\
Pole \(P\) czworokąta \(O B C D\) w zależności od pierwszej współrzędnej \(x\) punktu \(C\) jest określone wzorem

\[
P(x)=\frac{21}{4} \cdot \frac{x^{2}-56}{x-8}
\]

dla \(x \in(0,7)\).\\
Oblicz współrzędne wierzchołka \(C\), dla których pole czworokąta \(O B C D\) jest największe. Zapisz obliczenia.\\
\includegraphics[max width=\textwidth, center]{2025_02_09_f8fe43fb3ffece8ad22eg-27}\\
\includegraphics[max width=\textwidth, center]{2025_02_09_f8fe43fb3ffece8ad22eg-28}

BRUDNOPIS (nie podlega ocenie)

\begin{center}
\begin{tabular}{|c|c|c|c|c|c|c|c|c|c|c|c|c|c|c|c|c|c|c|c|c|c|c|c|}
\hline
 &  &  &  &  &  &  &  &  &  &  &  &  &  &  &  &  &  &  &  &  &  &  &  \\
\hline
 &  &  &  &  &  &  &  &  &  &  &  &  &  &  &  &  &  &  &  &  &  &  &  \\
\hline
 &  &  &  &  &  &  &  &  &  &  &  &  &  &  &  &  &  &  &  &  &  &  &  \\
\hline
 &  &  &  &  &  &  &  &  &  &  &  &  &  &  &  &  &  &  &  &  &  &  &  \\
\hline
 &  &  &  &  &  &  &  &  &  &  &  &  &  &  &  &  &  &  &  &  &  &  &  \\
\hline
 &  &  &  &  &  &  &  &  &  &  &  &  &  &  &  &  &  &  &  &  &  &  &  \\
\hline
 &  &  &  &  &  &  &  &  &  &  &  &  &  &  &  &  &  &  &  &  &  &  &  \\
\hline
 &  &  &  &  &  &  &  &  &  &  &  &  &  &  &  &  &  &  &  &  &  &  &  \\
\hline
 &  &  &  &  &  &  &  &  &  &  &  &  &  &  &  &  &  &  &  &  &  &  &  \\
\hline
 &  &  &  &  &  &  &  &  &  &  &  &  &  &  &  &  &  &  &  &  &  &  &  \\
\hline
 &  &  &  &  &  &  &  &  &  &  &  &  &  &  &  &  &  &  &  &  &  &  &  \\
\hline
 &  &  &  &  &  &  &  &  &  &  &  &  &  &  &  &  &  &  &  &  &  &  &  \\
\hline
 &  &  &  &  &  &  &  &  &  &  &  &  &  &  &  &  &  &  &  &  &  &  &  \\
\hline
 &  &  &  &  &  &  &  &  &  &  &  &  &  &  &  &  &  &  &  &  &  &  &  \\
\hline
 &  &  &  &  &  &  &  &  &  &  &  &  &  &  &  &  &  &  &  &  &  &  &  \\
\hline
 &  &  &  &  &  &  &  &  &  &  &  &  &  &  &  &  &  &  &  &  &  &  &  \\
\hline
 &  &  &  &  &  &  &  &  &  &  &  &  &  &  &  &  &  &  &  &  &  &  &  \\
\hline
 &  &  &  &  &  &  &  &  &  &  &  &  &  &  &  &  &  &  &  &  &  &  &  \\
\hline
 &  &  &  &  &  &  &  &  &  &  &  &  &  &  &  &  &  &  &  &  &  &  &  \\
\hline
 &  &  &  &  &  &  &  &  &  &  &  &  &  &  &  &  &  &  &  &  &  &  &  \\
\hline
 &  &  &  &  &  &  &  &  &  &  &  &  &  &  &  &  &  &  &  &  &  &  &  \\
\hline
 &  &  &  &  &  &  &  &  &  &  &  &  &  &  &  &  &  &  &  &  &  &  &  \\
\hline
 &  &  &  &  &  &  &  &  &  &  &  &  &  &  &  &  &  &  &  &  &  &  &  \\
\hline
 &  &  &  &  &  &  &  &  &  &  &  &  &  &  &  &  &  &  &  &  &  &  &  \\
\hline
 &  &  &  &  &  &  &  &  &  &  &  &  &  &  &  &  &  &  &  &  &  &  &  \\
\hline
 &  &  &  &  &  &  &  &  &  &  &  &  &  &  &  &  &  &  &  &  &  &  &  \\
\hline
 &  &  &  &  &  &  &  &  &  &  &  &  &  &  &  &  &  &  &  &  &  &  &  \\
\hline
 &  &  &  &  &  &  &  &  &  &  &  &  &  &  &  &  &  &  &  &  &  &  &  \\
\hline
 &  &  &  &  &  &  &  &  &  &  &  &  &  &  &  &  &  &  &  &  &  &  &  \\
\hline
 &  &  &  &  &  &  &  &  &  &  &  &  &  &  &  &  &  &  &  &  &  &  &  \\
\hline
 &  &  &  &  &  &  &  &  &  &  &  &  &  &  &  &  &  &  &  &  &  &  &  \\
\hline
 &  &  &  &  &  &  &  &  &  &  &  &  &  &  &  &  &  &  &  &  &  &  &  \\
\hline
 &  &  &  &  &  &  &  &  &  &  &  &  &  &  &  &  &  &  &  &  &  &  &  \\
\hline
 &  &  &  &  &  &  &  &  &  &  &  &  &  &  &  &  &  &  &  &  &  &  &  \\
\hline
 &  &  &  &  &  &  &  &  &  &  &  &  &  &  &  &  &  &  &  &  &  &  &  \\
\hline
 &  &  &  &  &  &  &  &  &  &  &  &  &  &  &  &  &  &  &  &  &  &  &  \\
\hline
 &  &  &  &  &  &  &  &  &  &  &  &  &  &  &  &  &  &  &  &  &  &  &  \\
\hline
 &  &  &  &  &  &  &  &  &  &  &  &  &  &  &  &  &  &  &  &  &  &  &  \\
\hline
 &  &  &  &  &  &  &  &  &  &  &  &  &  &  &  &  &  &  &  &  &  &  &  \\
\hline
 &  &  &  &  &  &  &  &  &  &  &  &  &  &  &  &  &  &  &  &  &  &  &  \\
\hline
 &  &  &  &  &  &  &  &  &  &  &  &  &  &  &  &  &  &  &  &  &  &  &  \\
\hline
 &  &  &  &  &  &  &  &  &  &  &  &  &  &  &  &  &  &  &  &  &  &  &  \\
\hline
 &  &  &  &  &  &  &  &  &  &  &  &  &  &  &  &  &  &  &  &  &  &  &  \\
\hline
 &  &  &  &  &  &  &  &  &  &  &  &  &  &  &  &  &  &  &  &  &  &  &  \\
\hline
 &  &  &  &  &  &  &  &  &  &  &  &  &  &  &  &  &  &  &  &  &  &  &  \\
\hline
 &  &  &  &  &  &  &  &  &  &  &  &  &  &  &  &  &  &  &  &  &  &  &  \\
\hline
 &  &  &  &  &  &  &  &  &  &  &  &  &  &  &  &  &  &  &  &  &  &  &  \\
\hline
\end{tabular}
\end{center}

\begin{center}
\includegraphics[max width=\textwidth]{2025_02_09_f8fe43fb3ffece8ad22eg-30}
\end{center}

\section*{MATEMATYKA}
\section*{Poziom rozszerzony}
Formuła 2023

\section*{MATEMATYKA}
\section*{Poziom rozszerzony}
Formuła 2023

\section*{MATEMATYKA}
\section*{Poziom rozszerzony}
Formuła 2023


\end{document}