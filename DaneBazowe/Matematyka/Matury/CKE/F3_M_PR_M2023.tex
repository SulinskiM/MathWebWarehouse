\documentclass[a4paper,12pt]{article}
\usepackage{latexsym}
\usepackage{amsmath}
\usepackage{amssymb}
\usepackage{graphicx}
\usepackage{wrapfig}
\pagestyle{plain}
\usepackage{fancybox}
\usepackage{bm}

\begin{document}

CENTRALNA

KOMISJA

EGZAMINACYJNA

Arkusz zawiera informacje prawnie chronione

do momentu rozpoczecia egzaminu.

KOD

WYPELNIA ZDAJACY

PESEL

{\it Miejsce na naklejke}.

{\it Sprawdz}', {\it czy kod na naklejce to}

M-100.
\begin{center}
\includegraphics[width=21.900mm,height=10.164mm]{./F3_M_PR_M2023_page0_images/image001.eps}

\includegraphics[width=79.656mm,height=10.164mm]{./F3_M_PR_M2023_page0_images/image002.eps}
\end{center}
/{\it ezeli tak}- {\it przyklej naklejkq}.

/{\it ezeli nie}- {\it zgtoś to nauczycielowi}.

Egzamin maturalny

$\displaystyle \int$
\begin{center}
\includegraphics[width=193.344mm,height=78.180mm]{./F3_M_PR_M2023_page0_images/image003.eps}
\end{center}
Poziom  rozszerzony

{\it Symbol arkusza}

MMAP-R0-100-2305

DATA: 12 maja 2023 r.
\begin{center}
\begin{tabular}{|l|}
\hline
\multicolumn{1}{|l|}{P LNIA $\mathrm{E}\mathrm{S}\mathrm{P}6\mathrm{L}$ NADZORUJACY}	\\
\hline
\multicolumn{1}{|l|}{$\begin{array}{l}\mbox{Uprawnienia zdaj cego do:}	\\	\mbox{dostosowania zasad oceniania}	\\	\mbox{dostosowania w zw. z dyskalkuli}	\end{array}$}	\\
\hline
\end{tabular}

\end{center}
GODZINA R0ZP0CZECIA: 9:00

CZAS TRWANIA: $180 \displaystyle \min$ ut

LICZBA PUNKTÓW DO UZYSKANIA 50

Przed rozpoczeciem pracy z arkuszem egzaminacyjnym

1.

Sprawd $\acute{\mathrm{z}}$, czy nauczyciel przekazal Ci wlaściwy arkusz egzaminacyjny,

tj. arkusz we wlaściwej formule, z w[aściwego przedmiotu na wlaściwym

poziomie.

2.

$\mathrm{J}\mathrm{e}\dot{\mathrm{z}}$ eli przekazano Ci niew[aściwy arkusz- natychmiast zgloś to nauczycielowi.

Nie rozrywaj banderol.

3.

$\mathrm{J}\mathrm{e}\dot{\mathrm{z}}$ eli przekazano Ci w[aściwy arkusz- rozerwij banderole po otrzymaniu

takiego polecenia od nauczyciela. Zapoznaj $\mathrm{s}\mathrm{i}\mathrm{e}$ z instrukcjq na stronie 2.

$\mathrm{U}\mathrm{k}\}\mathrm{a}\mathrm{d}$ graficzny

\copyright CKE 2022 $\blacksquare$

$\Vert\Vert\Vert\Vert\Vert\Vert\Vert\Vert\Vert\Vert\Vert\Vert\Vert\Vert\Vert\Vert\Vert\Vert\Vert\Vert\Vert\Vert\Vert\Vert\Vert\Vert\Vert\Vert\Vert\Vert|$




lnstrukcja dla zdajqcego

1.

2.

3.

4.

5.

6.

7.

8.

9.

Sprawd $\acute{\mathrm{z}}$, czy arkusz egzaminacyjny zawiera 27 stron (zadania $1-13$).

Ewentualny brak zgloś przewodniczqcemu zespolu nadzorujqcego egzamin.

Na pierwszej stronie arkusza oraz na karcie odpowiedzi wpisz swój numer PESEL

i przyklej naklejke z kodem.

$\mathrm{P}\mathrm{a}\mathrm{m}\mathrm{i}_{9}\mathrm{t}\mathrm{a}\mathrm{j}, \dot{\mathrm{z}}\mathrm{e}$ pominiecie argumentacji lub istotnych obliczeń w rozwiqzaniu zadania

otwartego $\mathrm{m}\mathrm{o}\dot{\mathrm{z}}\mathrm{e}$ spowodować, $\dot{\mathrm{z}}\mathrm{e}$ za to rozwiazanie nie otrzymasz pelnej liczby punktów.

Rozwiqzania zadań i odpowiedzi wpisuj w miejscu na to przeznaczonym.

Pisz czytelnie i $\mathrm{u}\dot{\mathrm{z}}$ ywaj tylko dlugopisu lub pióra z czarnym tuszem lub atramentem.

Nie $\mathrm{u}\dot{\mathrm{z}}$ ywaj korektora, a bledne zapisy wyra $\acute{\mathrm{z}}$ nie przekreśl.

Nie wpisuj $\dot{\mathrm{z}}$ adnych znaków w tabelkach przeznaczonych dla egzaminatora. Tabelki

umieszczone sq na marginesie przy $\mathrm{k}\mathrm{a}\dot{\mathrm{z}}$ dym zadaniu.

$\mathrm{P}\mathrm{a}\mathrm{m}\mathrm{i}_{9}\mathrm{t}\mathrm{a}\mathrm{j}, \dot{\mathrm{z}}\mathrm{e}$ zapisy w brudnopisie nie beda oceniane.

$\mathrm{M}\mathrm{o}\dot{\mathrm{z}}$ esz korzystač z Wybranych wzorów matematycznych, cyrkla i linijki oraz kalkulatora

prostego. Upewnij $\mathrm{s}\mathrm{i}\mathrm{e}$, czy przekazano Ci broszur9 z ok1adka taka jak widoczna ponizej.

Strona 2 z27

$\mathrm{M}\mathrm{M}\mathrm{A}\mathrm{P}-\mathrm{R}0_{-}100$





Zadanie 7. $(0-4$\}

Danyjest sześcian ABCDEFGH o krawpdzi

dlugości 6. Punkt $S$ jest punktem przeciecia

przekqtnych $AH \mathrm{i}$ DE ściany bocznej ADHE

(zobacz rysunek).

Oblicz wysokośč tróikqta SBH poprowadzonq z punktu S na bok BH tego trójkqta.

Zapisz obliczenia.

$\mathrm{M}\mathrm{M}\mathrm{A}\mathrm{P}-\mathrm{R}0_{-}100$

Strona ll z27





Zadanie 8. $(0-4$\}

Czworokqt ABCD, w którym $|BC|=4 \mathrm{i} |CD|=5$, jest opisany na okregu. Przekatna $AC$

tego czworokata tworzy z bokiem $BC$ kqt o mierze $60^{\mathrm{o}}$, natomiast z bokiem $AB-$ kqt ostry,

którego sinus jest równy $\displaystyle \frac{1}{4}.$

Oblicz obwód czworokqta ABCD. Zapisz obliczenia.

Strona 12 z27

$\mathrm{M}\mathrm{M}\mathrm{A}\mathrm{P}-\mathrm{R}0_{-}100$





RO-100

Strona 13 z27





Zadanie $\mathrm{g}. (0-4$\}

Rozwiqz nierównośč

$\displaystyle \sqrt{x^{2}+4x+4}<\frac{25}{3}-\sqrt{x^{2}-6x+9}$

Zapisz obliczenia.

{\it Wskazówka}: {\it skorzystaj z tego, ze} $\sqrt{a^{2}}=|a|$ {\it dla kazdej liczby} $ rz\mathrm{e}\mathrm{c}z\gamma${\it wi}s{\it t}e{\it j} $a.$

Strona 14 z27

$\mathrm{M}\mathrm{M}\mathrm{A}\mathrm{P}-\mathrm{R}0_{-}10$





RO-100

Strona 15 z27





Zadanie $\mathrm{f}0_{\mathrm{L}}\{0-4$)

Określamy kwadraty $K_{1}, K_{2}, K_{3}$, następujqco:

$\bullet K_{1}$ jest kwadratem o boku dlugości $a$

$\bullet K_{2}$ jest kwadratem, którego $\mathrm{k}\mathrm{a}\dot{\mathrm{z}}\mathrm{d}\mathrm{y}$ wierzcholek $\mathrm{l}\mathrm{e}\dot{\mathrm{z}}\mathrm{y}$ na innym boku kwadratu $K_{1}$

ten bok w stosunku 1 : 3

i dzieli

$\bullet K_{3}$ jest kwadratem, którego $\mathrm{k}\mathrm{a}\dot{\mathrm{z}}\mathrm{d}\mathrm{y}$ wierzcholek $\mathrm{l}\mathrm{e}\dot{\mathrm{z}}\mathrm{y}$ na innym boku kwadratu $K_{2}$ i dzieli

ten bok w stosunku 1 : 3

i ogólnie, dla $\mathrm{k}\mathrm{a}\dot{\mathrm{z}}$ dej liczby naturalnej $n\geq 2,$

$\bullet K_{n}$ jest kwadratem, którego $\mathrm{k}\mathrm{a}\dot{\mathrm{z}}\mathrm{d}\mathrm{y}$ wierzcholek $\mathrm{l}\mathrm{e}\dot{\mathrm{z}}\mathrm{y}$ na innym boku kwadratu $K_{n-1}$

i dzieli ten bok w stosunku 1 : 3.

Obwody wszystkich kwadratów określonych powyzej tworzq nieskończony ciqg

geometryczny.

Na rysunku przedstawiono kwadraty utworzone w sposób opisany powyzej.

{\it a}
\begin{center}
\includegraphics[width=58.824mm,height=58.872mm]{./F3_M_PR_M2023_page15_images/image001.eps}
\end{center}
{\it a}

Oblicz sume wszystkich wyrazów tego nieskończonego ciqgu. Zapisz obliczenia.

$\dagger$

Strona 16 z27

$\mathrm{M}\mathrm{M}\mathrm{A}\mathrm{P}-\mathrm{R}0_{-}100$





RO-100

Strona 17 z27





Zadanie Y\S$*$(0-5)

Wyznacz wszystkie wartości parametru $m\neq 2$, dla których równanie

$x^{2}+4x-\displaystyle \frac{m-3}{m-2}=0$

ma dwa rózne rozwiqzania rzeczywiste $x_{1}, x_{2}$ spelniajqce warunek $x_{1}^{3}+x_{2}^{3}>-28.$

Zapisz obliczenia.

Strona 18 z27

$\mathrm{M}\mathrm{M}\mathrm{A}\mathrm{P}-\mathrm{R}0_{-}100$





RO-100

Strona 19 z27





Zadanie 82.

Funkcja $f$ jest określona wzorem $f(x)=81^{\log_{3}x}+\displaystyle \frac{2\cdot\log_{2}\sqrt{27}\cdot\log_{3}2}{3}\cdot x^{2}-6x$ dla

$\mathrm{k}\mathrm{a}\dot{\mathrm{z}}$ dej liczby dodatniel $x.$

Zadanie \S 2.a. $\{0-2\}$

Wykaz, $\dot{\mathrm{z}}\mathrm{e}$ dla $\mathrm{k}\mathrm{a}\dot{\mathrm{z}}\mathrm{d}\mathrm{e}\mathrm{i}$ liczby dodatniej $x$ wyra $\dot{\mathrm{z}}$ enie

$81^{\log_{3}x}+\displaystyle \frac{2\cdot\log_{2}\sqrt{27}\cdot\log_{3}2}{3}\cdot x^{2}-6x$

$\mathrm{m}\mathrm{o}\dot{\mathrm{z}}$ na równowaznie przekszta[cič do postaci $x^{4}+x^{2}-6x.$

Strona 20 z27

$\mathrm{M}\mathrm{M}\mathrm{A}\mathrm{P}-\mathrm{R}0_{-}100$





Zadania egzaminacyine sq wydrukowane

na nastepnych stronach.

$\mathrm{M}\mathrm{M}\mathrm{A}\mathrm{P}-\mathrm{R}0_{-}100$

Strona 3 z27





Zadanie n2.2. (0-4)

Oblicz najmniejszq wartośč funkcji $f$ określonej dla $\mathrm{k}\mathrm{a}\dot{\mathrm{z}}$ dej liczby dodatniei $x.$

Zapisz obliczenia.

{\it Wskazówka}: {\it przyjmij, ze wzór funkcii} $f$ {\it mozna przedstawič w postaci} $f(x)=x^{4}+x^{2}-6x.$

$\mathrm{M}\mathrm{M}\mathrm{A}\mathrm{P}-\mathrm{R}0_{-}100$

Strona 21 z27





Zadanie 83. (0-6)

$\mathrm{W}$ kartezjańskim ukladzie wspólrzednych $(x,y)$ prosta $l$ o równaniu $x-y-2=0$

przecina parabo19 o równaniu $y=4x^{2}-7x+1$ w punktach $A$ oraz $B$. Odcinek $AB$ jest

średnicq okrpgu $O$. Punkt $C \mathrm{l}\mathrm{e}\dot{\mathrm{z}}\mathrm{y}$ na okrpgu $O$ nad prostq $l$, a kqt $BAC$ jest ostry i ma

miar9 $\alpha$ takq, $\dot{\mathrm{z}}\mathrm{e} \displaystyle \mathrm{t}\mathrm{g}\alpha=\frac{1}{3}$ (zobacz rysunek).
\begin{center}
\includegraphics[width=122.580mm,height=132.024mm]{./F3_M_PR_M2023_page21_images/image001.eps}
\end{center}
{\it y}

1  $y=4x^{2}-7x+1$

{\it l}

$x-y-2=0$

1 {\it C}  $\chi$

{\it B}

$\alpha$

{\it A}

Oblicz wspó[rzedne punktu C. Zapisz obliczenia.

$\rfloor$

$\rceil_{1}$

$\rfloor$

$\rfloor$

$\rfloor$

$\rfloor$

$i$

$\mathrm{t}^{:}$

Strona 22 z27

$\mathrm{M}\mathrm{M}\mathrm{A}\mathrm{P}-\mathrm{R}0_{-}100$





RO-100

Strona 23 z27





Strona 24 z27

$\mathrm{M}\mathrm{M}\mathrm{A}\mathrm{P}-\mathrm{R}0_{-}10$





: RUDNOPIS (nie podlega ocenie)

$\Psi-\mathrm{R}0_{-}100$

Strona 25 z27





1

-$|\mathfrak{l} \mathfrak{l} \mathfrak{l}|$ -

Strona 26 z27

$\mathrm{M}\mathrm{M}\mathrm{A}\mathrm{P}-\mathrm{R}0_{-}10$





1

-$|\mathfrak{l} \mathfrak{l} \mathfrak{l}|$ -

RO-100

Strona 27 z27










Zadanie 8. $(0-2$\}

$\mathrm{W}$ chwili poczqtkowej $(t=0)$ masa substancji jest równa 4 gramom. Wskutek rozpadu

czqsteczek tej substancji jej masa si9 zmniejsza. Po $\mathrm{k}\mathrm{a}\dot{\mathrm{z}}$ dej kolejnej dobie ubywa

19\% masy, jaka byla na koniec doby poprzedniej. Dla $\mathrm{k}\mathrm{a}\dot{\mathrm{z}}$ dej liczby calkowitej $t\geq 0$

funkcja $m(t)$ określa mase substancji w gramach po $t$ pelnych dobach (czas liczymy od

chwili poczatkowej).

Wyznacz wzór funkcji m(t). Oblicz, po ilu pe[nych dobach masa tej substancji bedzie

po raz pierwszy mniejsza od 1, 5 grama.

Zapisz obliczenia.

$1 1$

Strona 4 z27

$\mathrm{M}\mathrm{M}\mathrm{A}\mathrm{P}-\mathrm{R}0_{-}100$





Zadanie 2. $(0-3$\}

Tomek i Romek postanowili rozegrač między sobq pieč partii szachów. Prawdopodobieństwo

wygrania pojedynczej partii przez Tomka jest równe $\displaystyle \frac{1}{4}.$

Oblicz prawdopodobieństwo wygrania przez Tomka co najmniej czterech z pieciu

partii. Wynik podaj w postaci ulamka zwyk[ego nieskracalnego. Zapisz obliczenia.

$\mathrm{M}\mathrm{M}\mathrm{A}\mathrm{P}-\mathrm{R}0_{-}100$

Strona 5 z27





Zadanie 3. $(0-3$\}

Funkcja $f$ jest określona wzorem $f(x)=\displaystyle \frac{3x^{2}-2x}{x^{2}+2x+8}$ dla $\mathrm{k}\mathrm{a}\dot{\mathrm{z}}$ dej liczby rzeczywistej $x.$

Punkt $P=(x_{0}$, 3$)$ nalez $\mathrm{y}$ do wykresu funkcji $f.$

Oblicz $x_{0}$ oraz wyznacz równanie stycznej do wykresu funkcji $f$ w punkcie $P.$

Zapisz obliczenia.

$-\mathrm{i}1$

$1 -$

Strona 6 z27

$\mathrm{M}\mathrm{M}\mathrm{A}\mathrm{P}-\mathrm{R}0_{-}100$





Zadanie 4. $(0-3$\}

Liczby rzeczywiste $x$ oraz $y$ spelniajqjednocześnie równanie $x+y=4$ i nierównośč

$x^{3}-x^{2}\mathrm{y}\leq x\mathrm{y}^{2}-y^{3}$

Wykaz, $\dot{\mathrm{z}}\mathrm{e} x=2$ oraz $y=2.$

$\mathrm{M}\mathrm{M}\mathrm{A}\mathrm{P}-\mathrm{R}0_{-}100$

Strona 7 z27





Zadanie 5. $(0-3$\}

Danyjest trójkqt prostokqtny $ABC$, w którym $|4ABC|=90^{\mathrm{o}}$ oraz $|4\mathrm{C}AB|=60^{\mathrm{o}}$ Punkty

$K \mathrm{i} L \mathrm{l}\mathrm{e}\dot{\mathrm{z}}$ a na bokach- odpowiednio -$AB \mathrm{i} BC$ tak, $\dot{\mathrm{z}}\mathrm{e} |BK|=|BL|=1$ (zobacz

rysunek). Odcinek $KL$ przecina wysokośč $BD$ tego trójkqta w punkcie $N$, a ponadto

$|AD|=2.$
\begin{center}
\includegraphics[width=133.404mm,height=81.480mm]{./F3_M_PR_M2023_page7_images/image001.eps}
\end{center}
{\it A}

$60^{\mathrm{o}}$

2

{\it D}

{\it K}

{\it N}

1

{\it B 1 L  C}

Wykaz, $\dot{\mathrm{z}}\mathrm{e} |ND|=\sqrt{3}+1.$

Strona 8 z27

$\mathrm{M}\mathrm{M}\mathrm{A}\mathrm{P}-\mathrm{R}0_{-}100$





RO-100

Strona 9 z27





Zadanie 6. $(0-3$\}

Rozwiqz równanie

$4\sin(4x)\cos(6x)=2\sin(10x)+1$

Zapisz obliczenia.

Strona 10 z27

$\mathrm{M}\mathrm{M}\mathrm{A}\mathrm{P}-\mathrm{R}0_{-}10$



\end{document}