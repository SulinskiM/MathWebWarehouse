\documentclass[a4paper,12pt]{article}
\usepackage{latexsym}
\usepackage{amsmath}
\usepackage{amssymb}
\usepackage{graphicx}
\usepackage{wrapfig}
\pagestyle{plain}
\usepackage{fancybox}
\usepackage{bm}

\begin{document}

Zadanie $\mathrm{f}0_{\mathrm{L}}\{0-4$)

Określamy kwadraty $K_{1}, K_{2}, K_{3}$, następujqco:

$\bullet K_{1}$ jest kwadratem o boku dlugości $a$

$\bullet K_{2}$ jest kwadratem, którego $\mathrm{k}\mathrm{a}\dot{\mathrm{z}}\mathrm{d}\mathrm{y}$ wierzcholek $\mathrm{l}\mathrm{e}\dot{\mathrm{z}}\mathrm{y}$ na innym boku kwadratu $K_{1}$

ten bok w stosunku 1 : 3

i dzieli

$\bullet K_{3}$ jest kwadratem, którego $\mathrm{k}\mathrm{a}\dot{\mathrm{z}}\mathrm{d}\mathrm{y}$ wierzcholek $\mathrm{l}\mathrm{e}\dot{\mathrm{z}}\mathrm{y}$ na innym boku kwadratu $K_{2}$ i dzieli

ten bok w stosunku 1 : 3

i ogólnie, dla $\mathrm{k}\mathrm{a}\dot{\mathrm{z}}$ dej liczby naturalnej $n\geq 2,$

$\bullet K_{n}$ jest kwadratem, którego $\mathrm{k}\mathrm{a}\dot{\mathrm{z}}\mathrm{d}\mathrm{y}$ wierzcholek $\mathrm{l}\mathrm{e}\dot{\mathrm{z}}\mathrm{y}$ na innym boku kwadratu $K_{n-1}$

i dzieli ten bok w stosunku 1 : 3.

Obwody wszystkich kwadratów określonych powyzej tworzq nieskończony ciqg

geometryczny.

Na rysunku przedstawiono kwadraty utworzone w sposób opisany powyzej.

{\it a}
\begin{center}
\includegraphics[width=58.824mm,height=58.872mm]{./F3_M_PR_M2023_page15_images/image001.eps}
\end{center}
{\it a}

Oblicz sume wszystkich wyrazów tego nieskończonego ciqgu. Zapisz obliczenia.

$\dagger$

Strona 16 z27

$\mathrm{M}\mathrm{M}\mathrm{A}\mathrm{P}-\mathrm{R}0_{-}100$
\end{document}
