\documentclass[a4paper,12pt]{article}
\usepackage{latexsym}
\usepackage{amsmath}
\usepackage{amssymb}
\usepackage{graphicx}
\usepackage{wrapfig}
\pagestyle{plain}
\usepackage{fancybox}
\usepackage{bm}

\begin{document}

Zadanie 82.

Funkcja $f$ jest określona wzorem $f(x)=81^{\log_{3}x}+\displaystyle \frac{2\cdot\log_{2}\sqrt{27}\cdot\log_{3}2}{3}\cdot x^{2}-6x$ dla

$\mathrm{k}\mathrm{a}\dot{\mathrm{z}}$ dej liczby dodatniel $x.$

Zadanie \S 2.a. $\{0-2\}$

Wykaz, $\dot{\mathrm{z}}\mathrm{e}$ dla $\mathrm{k}\mathrm{a}\dot{\mathrm{z}}\mathrm{d}\mathrm{e}\mathrm{i}$ liczby dodatniej $x$ wyra $\dot{\mathrm{z}}$ enie

$81^{\log_{3}x}+\displaystyle \frac{2\cdot\log_{2}\sqrt{27}\cdot\log_{3}2}{3}\cdot x^{2}-6x$

$\mathrm{m}\mathrm{o}\dot{\mathrm{z}}$ na równowaznie przekszta[cič do postaci $x^{4}+x^{2}-6x.$

Strona 20 z27

$\mathrm{M}\mathrm{M}\mathrm{A}\mathrm{P}-\mathrm{R}0_{-}100$
\end{document}
