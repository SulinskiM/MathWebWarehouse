\documentclass[a4paper,12pt]{article}
\usepackage{latexsym}
\usepackage{amsmath}
\usepackage{amssymb}
\usepackage{graphicx}
\usepackage{wrapfig}
\pagestyle{plain}
\usepackage{fancybox}
\usepackage{bm}

\begin{document}

Zadanie 83. (0-6)

$\mathrm{W}$ kartezjańskim ukladzie wspólrzednych $(x,y)$ prosta $l$ o równaniu $x-y-2=0$

przecina parabo19 o równaniu $y=4x^{2}-7x+1$ w punktach $A$ oraz $B$. Odcinek $AB$ jest

średnicq okrpgu $O$. Punkt $C \mathrm{l}\mathrm{e}\dot{\mathrm{z}}\mathrm{y}$ na okrpgu $O$ nad prostq $l$, a kqt $BAC$ jest ostry i ma

miar9 $\alpha$ takq, $\dot{\mathrm{z}}\mathrm{e} \displaystyle \mathrm{t}\mathrm{g}\alpha=\frac{1}{3}$ (zobacz rysunek).
\begin{center}
\includegraphics[width=122.580mm,height=132.024mm]{./F3_M_PR_M2023_page21_images/image001.eps}
\end{center}
{\it y}

1  $y=4x^{2}-7x+1$

{\it l}

$x-y-2=0$

1 {\it C}  $\chi$

{\it B}

$\alpha$

{\it A}

Oblicz wspó[rzedne punktu C. Zapisz obliczenia.

$\rfloor$

$\rceil_{1}$

$\rfloor$

$\rfloor$

$\rfloor$

$\rfloor$

$i$

$\mathrm{t}^{:}$

Strona 22 z27

$\mathrm{M}\mathrm{M}\mathrm{A}\mathrm{P}-\mathrm{R}0_{-}100$
\end{document}
