\documentclass[a4paper,12pt]{article}
\usepackage{latexsym}
\usepackage{amsmath}
\usepackage{amssymb}
\usepackage{graphicx}
\usepackage{wrapfig}
\pagestyle{plain}
\usepackage{fancybox}
\usepackage{bm}

\begin{document}

Zadanie 8. $(0-2$\}

$\mathrm{W}$ chwili poczqtkowej $(t=0)$ masa substancji jest równa 4 gramom. Wskutek rozpadu

czqsteczek tej substancji jej masa si9 zmniejsza. Po $\mathrm{k}\mathrm{a}\dot{\mathrm{z}}$ dej kolejnej dobie ubywa

19\% masy, jaka byla na koniec doby poprzedniej. Dla $\mathrm{k}\mathrm{a}\dot{\mathrm{z}}$ dej liczby calkowitej $t\geq 0$

funkcja $m(t)$ określa mase substancji w gramach po $t$ pelnych dobach (czas liczymy od

chwili poczatkowej).

Wyznacz wzór funkcji m(t). Oblicz, po ilu pe[nych dobach masa tej substancji bedzie

po raz pierwszy mniejsza od 1, 5 grama.

Zapisz obliczenia.

$1 1$

Strona 4 z27

$\mathrm{M}\mathrm{M}\mathrm{A}\mathrm{P}-\mathrm{R}0_{-}100$
\end{document}
