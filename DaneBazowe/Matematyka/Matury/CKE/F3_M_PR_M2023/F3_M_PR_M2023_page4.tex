\documentclass[a4paper,12pt]{article}
\usepackage{latexsym}
\usepackage{amsmath}
\usepackage{amssymb}
\usepackage{graphicx}
\usepackage{wrapfig}
\pagestyle{plain}
\usepackage{fancybox}
\usepackage{bm}

\begin{document}

Zadanie 2. $(0-3$\}

Tomek i Romek postanowili rozegrač między sobq pieč partii szachów. Prawdopodobieństwo

wygrania pojedynczej partii przez Tomka jest równe $\displaystyle \frac{1}{4}.$

Oblicz prawdopodobieństwo wygrania przez Tomka co najmniej czterech z pieciu

partii. Wynik podaj w postaci ulamka zwyk[ego nieskracalnego. Zapisz obliczenia.

$\mathrm{M}\mathrm{M}\mathrm{A}\mathrm{P}-\mathrm{R}0_{-}100$

Strona 5 z27
\end{document}
