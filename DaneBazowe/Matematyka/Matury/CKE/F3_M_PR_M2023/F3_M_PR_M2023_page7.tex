\documentclass[a4paper,12pt]{article}
\usepackage{latexsym}
\usepackage{amsmath}
\usepackage{amssymb}
\usepackage{graphicx}
\usepackage{wrapfig}
\pagestyle{plain}
\usepackage{fancybox}
\usepackage{bm}

\begin{document}

Zadanie 5. $(0-3$\}

Danyjest trójkqt prostokqtny $ABC$, w którym $|4ABC|=90^{\mathrm{o}}$ oraz $|4\mathrm{C}AB|=60^{\mathrm{o}}$ Punkty

$K \mathrm{i} L \mathrm{l}\mathrm{e}\dot{\mathrm{z}}$ a na bokach- odpowiednio -$AB \mathrm{i} BC$ tak, $\dot{\mathrm{z}}\mathrm{e} |BK|=|BL|=1$ (zobacz

rysunek). Odcinek $KL$ przecina wysokośč $BD$ tego trójkqta w punkcie $N$, a ponadto

$|AD|=2.$
\begin{center}
\includegraphics[width=133.404mm,height=81.480mm]{./F3_M_PR_M2023_page7_images/image001.eps}
\end{center}
{\it A}

$60^{\mathrm{o}}$

2

{\it D}

{\it K}

{\it N}

1

{\it B 1 L  C}

Wykaz, $\dot{\mathrm{z}}\mathrm{e} |ND|=\sqrt{3}+1.$

Strona 8 z27

$\mathrm{M}\mathrm{M}\mathrm{A}\mathrm{P}-\mathrm{R}0_{-}100$
\end{document}
