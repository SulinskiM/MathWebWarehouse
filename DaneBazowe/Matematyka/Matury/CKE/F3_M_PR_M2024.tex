\documentclass[a4paper,12pt]{article}
\usepackage{latexsym}
\usepackage{amsmath}
\usepackage{amssymb}
\usepackage{graphicx}
\usepackage{wrapfig}
\pagestyle{plain}
\usepackage{fancybox}
\usepackage{bm}

\begin{document}

CENTRALNA

KOMISJA

EGZAMINACYJNA

Arkusz zawiera informacje prawnie chronione

do momentu rozpoczecia egzaminu.

KOD

WYPELNIA ZDAJACY

PESEL

{\it Miejsce na naklejke}.

{\it Sprawdz}', {\it czy kod na naklejce to}

M-100.
\begin{center}
\includegraphics[width=21.900mm,height=10.164mm]{./F3_M_PR_M2024_page0_images/image001.eps}

\includegraphics[width=79.656mm,height=10.164mm]{./F3_M_PR_M2024_page0_images/image002.eps}
\end{center}
/{\it ezeli tak}- {\it przyklej naklejkq}.

/{\it ezeli nie}- {\it zgtoś to nauczycielowi}.

Egzamin maturalny

$\displaystyle \int$
\begin{center}
\includegraphics[width=193.344mm,height=78.180mm]{./F3_M_PR_M2024_page0_images/image003.eps}
\end{center}
Poziom  rozszerzony

{\it Symbol arkusza}

MMAP-R0-100-2405

DATA: 15 maja 2024 r.
\begin{center}
\begin{tabular}{|l|}
\hline
\multicolumn{1}{|l|}{WYP N1A $\mathrm{S}\mathrm{P}6$ NADZORUJACY}	\\
\hline
\multicolumn{1}{|l|}{$\begin{array}{l}\mbox{Uprawnienia zdaj cego do:}	\\	\mbox{dostosowania zasad oceniania.}	\end{array}$}	\\
\hline
\end{tabular}

\end{center}
GODZINA R0ZP0CZECIA: 9:00

CZAS TRWANIA: $180 \displaystyle \min$ ut

LICZBA PUNKTÓW DO UZYSKANIA 50

Przed rozpoczeciem pracy z arkuszem egzaminacyjnym

1.

Sprawd $\acute{\mathrm{z}}$, czy nauczyciel przekazal Ci wlaściwy arkusz egzaminacyjny,

tj. arkusz we wlaściwej formule, z w[aściwego przedmiotu na wlaściwym

poziomie.

2.

$\mathrm{J}\mathrm{e}\dot{\mathrm{z}}$ eli przekazano Ci niew[aściwy arkusz- natychmiast zgloś to nauczycielowi.

Nie rozrywaj banderol.

3.

$\mathrm{J}\mathrm{e}\dot{\mathrm{z}}$ eli przekazano Ci w[aściwy arkusz- rozerwij banderole po otrzymaniu

takiego polecenia od nauczyciela. Zapoznaj $\mathrm{s}\mathrm{i}\mathrm{e}$ z instrukcjq na stronie 2.

$\mathrm{U}\mathrm{k}\}\mathrm{a}\mathrm{d}$ graficzny

\copyright CKE 2022 $\bullet 1$

$\Vert\Vert\Vert\Vert\Vert\Vert\Vert\Vert\Vert\Vert\Vert\Vert\Vert\Vert\Vert\Vert\Vert\Vert\Vert\Vert\Vert\Vert\Vert\Vert\Vert\Vert\Vert\Vert\Vert\Vert|$




lnstrukcja dla zdajqcego

1.

2.

3.

4.

5.

6.

7.

8.

9.

Sprawd $\acute{\mathrm{z}}$, czy arkusz egzaminacyjny zawiera 27 stron (zadania $1-13$).

Ewentualny brak zgloś przewodniczqcemu zespolu nadzorujqcego egzamin.

Na pierwszej stronie arkusza oraz na karcie odpowiedzi wpisz swój numer PESEL

i przyklej naklejke z kodem.

$\mathrm{P}\mathrm{a}\mathrm{m}\mathrm{i}_{9}\mathrm{t}\mathrm{a}\mathrm{j}, \dot{\mathrm{z}}\mathrm{e}$ pominiecie argumentacji lub istotnych obliczeń w rozwiqzaniu zadania

otwartego $\mathrm{m}\mathrm{o}\dot{\mathrm{z}}\mathrm{e}$ spowodować, $\dot{\mathrm{z}}\mathrm{e}$ za to rozwiazanie nie otrzymasz pelnej liczby punktów.

Rozwiqzania zadań i odpowiedzi wpisuj w miejscu na to przeznaczonym.

Pisz czytelnie i $\mathrm{u}\dot{\mathrm{z}}$ ywaj tylko dlugopisu lub pióra z czarnym tuszem lub atramentem.

Nie $\mathrm{u}\dot{\mathrm{z}}$ ywaj korektora, a bledne zapisy wyra $\acute{\mathrm{z}}$ nie przekreśl.

Nie wpisuj $\dot{\mathrm{z}}$ adnych znaków w tabelkach przeznaczonych dla egzaminatora. Tabelki

umieszczone $\mathrm{s}_{\mathrm{c}}1$ na marginesie przy $\mathrm{k}\mathrm{a}\dot{\mathrm{z}}$ dym zadaniu.

$\mathrm{P}\mathrm{a}\mathrm{m}\mathrm{i}_{9}\mathrm{t}\mathrm{a}\mathrm{j}, \dot{\mathrm{z}}\mathrm{e}$ zapisy w brudnopisie nie bedq oceniane.

$\mathrm{M}\mathrm{o}\dot{\mathrm{z}}$ esz korzystač z Wybranych wzorów matematycznych, cyrkla i linijki oraz kalkulatora

prostego. Upewnij $\mathrm{s}\mathrm{i}\mathrm{e}$, czy przekazano Ci broszur9 z ok1adka taka jak widoczna ponizej.

Strona 2 z27

$\mathrm{M}\mathrm{M}\mathrm{A}\mathrm{P}-\mathrm{R}0_{-}100$





RO-100

Strona ll z27





Zadanie 8. $(0-4$\}

Danyjest trójkqt $ABC$, który nie jest równoramienny. $\mathrm{W}$ tym trójkqcie miara kqta $ABC$ jest

dwa razy wieksza od miary kqta $BAC.$

Wykaz, $\dot{\mathrm{z}}\mathrm{e}$ dlugości boków tego trójkqta spe[niajq warunek

$|AC|^{2}=|BC|^{2}+|AB|$

$|BC|$

Strona 12 z27

$\mathrm{M}\mathrm{M}\mathrm{A}\mathrm{P}-\mathrm{R}0_{-}100$





RO-100

Strona 13 z27





Zadanie $\mathrm{g}. (0-4$\}

Danyjest kwadrat ABCD o boku dlugości $a$. Punkt $E$ jest środkiem boku $CD$. Przekatna

$BD$ dzieli trójkat ACE na dwie figury: $AGF$ oraz CEFG (zobacz rysunek).
\begin{center}
\includegraphics[width=66.096mm,height=70.668mm]{./F3_M_PR_M2024_page13_images/image001.eps}
\end{center}
{\it D E  C}

{\it F}

{\it G}

{\it A  a  B}

Oblicz pola figur AGF oraz CEFG. Zapisz obliczenia.

Strona 14 z27

$\mathrm{M}\mathrm{M}\mathrm{A}\mathrm{P}-\mathrm{R}0_{-}100$





RO-100

Strona 15 z27





Zadanie $\mathrm{f}0. (0-5)$

Rozwiqz równanie

$\sin(4x)-\sin(2x)=4\cos^{2}x-3$

w zbiorze $[0,2\pi]$. Zapisz obliczenia.

Strona 16 z27

$\mathrm{M}\mathrm{M}\mathrm{A}\mathrm{P}-\mathrm{R}0_{-}10$





RO-100

Strona 17 z27





Zadanie Y\S$*$(0-5)

$\mathrm{W}$ kartezjańskim ukladzie wspólrzednych $(x,y)$ środek $S$ okregu o promieniu $\sqrt{5} \mathrm{l}\mathrm{e}\dot{\mathrm{z}}\mathrm{y}$ na

prostej o równaniu $y=x+1$. Przez punkt $A=(1,2)$, którego odleglośč od punktu $S$ jest

wipksza od $\sqrt{5}$, poprowadzono dwie proste styczne do tego okregu w punktach-

odpowiednio- $B \mathrm{i} C$. Pole czworokqta ABSC jest równe 15.

Oblicz wspólrzqdne punktu $S.$ Rozwa $\dot{\mathrm{z}}$ wszystkie przypadki. Zapisz obliczenia.

Strona 18 z27

$\mathrm{M}\mathrm{M}\mathrm{A}\mathrm{P}-\mathrm{R}0_{-}100$





RO-100

Strona 19 z27





Zadanie 82. (0-6)

Wyznacz wszystkie wartości parametru $m$, dla których równanie

$x^{2}-(3m+1)\cdot x+2m^{2}+m+1=0$

ma dwa rózne rozwiqzania rzeczywiste $x_{1}, x_{2}$ spelniajqce warunek

$x_{1}^{3}+x_{2}^{3}+3\cdot x_{1}\cdot x_{2}$

$(x_{1}+x_{2}-3)\leq 3m-7$

Zapisz obliczenia.

Strona 20 z27

$\mathrm{M}\mathrm{M}\mathrm{A}\mathrm{P}-\mathrm{R}0_{-}10$





Zadania egzaminacyine sq wydrukowane

na nastepnych stronach.

$\mathrm{M}\mathrm{M}\mathrm{A}\mathrm{P}-\mathrm{R}0_{-}100$

Strona 3 z27





RO-100

Strona 21 z27





Strona 22 z27

$\mathrm{M}\mathrm{M}\mathrm{A}\mathrm{P}-\mathrm{R}0_{-}10$





Zadanie 83

Rozwazamy wszystkie graniastoslupy prawidlowe trójkqtne o objetości 3456, których

$\mathrm{k}\mathrm{r}\mathrm{a}\mathrm{w}9^{\mathrm{d}\acute{\mathrm{Z}}}$ podstawy ma dlugośč nie większq $\mathrm{n}\mathrm{i}\dot{\mathrm{z}} 8\sqrt{3}.$

Zadanie $83.9_{1}(0-2)$

Wykaz, $\dot{\mathrm{z}}\mathrm{e}$ pole $P$ powierzchni ca[kowitej graniastoslupa w zale $\dot{\mathrm{z}}$ ności od d[ugości $a$

krawedzi podstawy graniastos[upa jest określone wzorem

$P(a)=\displaystyle \frac{a^{2}\cdot\sqrt{3}}{2}+\frac{13824\sqrt{3}}{a}$

$\mathrm{M}\mathrm{M}\mathrm{A}\mathrm{P}-\mathrm{R}0_{-}100$

Strona 23 z27





Zadanie 83[‡C]2. (0-4)

Pole $P$ powierzchni calkowitej graniastoslupa w zalezności od d$\dagger$ugości $a$ krawedzi

podstawy graniastoslupa jest określone wzorem

$P(a)=\displaystyle \frac{a^{2}\cdot\sqrt{3}}{2}+\frac{13824\sqrt{3}}{a}$

dla $a\in(0,8\sqrt{3}].$

Wyznacz d[ugośč krawedzi podstawy tego z rozwa $\dot{\mathrm{z}}$ anych graniastos[upów, którego

pole powierzchni calkowitej jest najmniejsze. Oblicz to najmniejsze pole. Zapisz

obliczenia.

Strona 24 z27

$\mathrm{M}\mathrm{M}\mathrm{A}\mathrm{P}-\mathrm{R}0_{-}100$





RO-100

Strona 25 z27





: RUDNOPIS (nie podlega ocenie)

Strona 26 z27

$\mathrm{M}\mathrm{M}\mathrm{A}\mathrm{P}-\mathrm{R}0_{-}10$





1

-$|\mathfrak{l} \mathfrak{l} \mathfrak{l}|$ -

RO-100

Strona 27 z27










Zadanie 8. $(0-2$\}

$\mathrm{W}$ chwili poczqtkowej$(t=0)$ filizanka z goracq kawq znajduje si9 w pokoju, a temperatura

tej kawy jest równa $80^{\mathrm{o}}\mathrm{C}$. Temperatura w pokoju (temperatura otoczenia)jest stala

i równa $20^{\mathrm{o}}\mathrm{C}$. Temperatura $T$ tej kawy zmienia si9 w czasie zgodnie z za1eznościq

$T(t)=(T_{p}-T_{Z})\cdot k^{-r}+T_{Z}$ dla

$r\geq 0$

gdzie:

T - temperatura kawy wyrazona w stopniach Celsjusza,

$t -$ czas wyrazony w minutach, liczony od chwili poczqtkowej,

$T_{\mathrm{P}}-$ temperatura poczqtkowa kawy wyrazona w stopniach Celsjusza,

$T_{Z}-$ temperatura otoczenia wyrazona w stopniach Celsjusza,

$k -$ stala charakterystyczna dla danej cieczy.

Po 10 minutach, 1iczqc od chwi1i poczatkowej, kawa ostyg1a do temperatury 65 $\mathrm{o}\mathrm{C}.$

Oblicz temperature tej kawy po nastepnych pieciu minutach. Wynik podaj w stopniach

Celsjusza, w zaokrqgleniu do jedności. Zapisz obliczenia.

Strona 4 z27

$\mathrm{M}\mathrm{M}\mathrm{A}\mathrm{P}-\mathrm{R}0_{-}100$





Zadanie 2. $(0-2$\}

Oblicz granice

Zapisz obliczenia.

$\mathrm{M}\mathrm{M}\mathrm{A}\mathrm{P}-\mathrm{R}0_{-}100$

$\chi$li$\rightarrow$m2--($\chi\chi$3--28)2

Strona 5 z27





Zadanie 3. $(0-3$\}

$\mathrm{W}$ pewnym zakladzie mleczarskim śmietana produkowana jest w 200-gramowych

opakowaniach. Prawdopodobieństwo zdarzenia, $\dot{\mathrm{z}}\mathrm{e}$ w losowo wybranym opakowaniu

śmietana zawiera mniej $\mathrm{n}\mathrm{i}\dot{\mathrm{z}}$ 36\% tluszczu, jest równe 0,0l. Kontroli poddajemy l0 losowo

wybranych opakowań ze śmietanq.

Oblicz prawdopodobieństwo zdarzenia polegajqcego na tym, $\dot{\mathrm{z}}\mathrm{e}$ wśród opakowań

poddanych $\mathrm{t}\mathrm{e}\mathrm{i}$ kontroli bedzie co najwy $\dot{\mathrm{z}}$ ej jedno opakowanie ze śmietanq, która

zawiera mniej $\mathrm{n}\mathrm{i}\dot{\mathrm{z}}$ 36\% tluszczu. Wynik zapisz w postaci ulamka dziesietnego

w zaokrqgleniu do cześci tysiecznych. Zapisz obliczenia.

Strona 6 z27

$\mathrm{M}\mathrm{M}\mathrm{A}\mathrm{P}-\mathrm{R}0_{-}100$





Zadaníe 4. $(0-3$\}

Funkcja $f$ jest określona wzorem

$f(x)=\displaystyle \frac{x^{3}-3x+2}{\chi}$

dla $\mathrm{k}\mathrm{a}\dot{\mathrm{z}}$ dej liczby rzeczywistej $x$ róznej od zera. $\mathrm{W}$ kartezjańskim ukladzie wspólrz9dnych

$(x,\mathrm{y})$ punkt $P$, o pierwszej wspólrzednej równej 2, na1ez $\mathrm{y}$ do wykresu funkcji $f.$

Prosta o równaniu $y=ax+b$ jest styczna do wykresu funkcji $f$ w punkcie $P.$

Oblicz wspólczynniki a oraz b w równaniu tei stycznej. Zapisz obliczenia.

$\mathrm{M}\mathrm{M}\mathrm{A}\mathrm{P}-\mathrm{R}0_{-}100$

Strona 7 z27





Zadanie $5*(0-3$\}

Wyka $\dot{\mathrm{z}}, \dot{\mathrm{z}}\mathrm{e}\mathrm{j}\mathrm{e}\dot{\mathrm{z}}$ eli $\log_{5}4=a$ oraz log43 $=b$, to log1280$=\displaystyle \frac{2a+1}{a\cdot(1+b)}$

Strona 8 z27

$\mathrm{M}\mathrm{M}\mathrm{A}\mathrm{P}-\mathrm{R}0_{-}10$





Zadanie 6, $(0-3$\}

Rozwazamy wszystkie liczby naturalne, w których zapisie $\mathrm{d}\mathrm{z}\mathrm{i}\mathrm{e}\mathrm{s}\mathrm{i}9$tnym nie powtarza si9

jakakolwiek cyfra oraz dokladnie trzy cyfry sq nieparzyste i dokladnie dwie cyfry sq parzyste.

Oblicz, ile jest wszystkich takich liczb. Zapisz obliczenia.

$\mathrm{M}\mathrm{M}\mathrm{A}\mathrm{P}-\mathrm{R}0_{-}100$

Strona 9 z27





Zadanie 7. $(0-4$\}

Trzywyrazowy ciag $(x,y,z)$ jest geometryczny i rosnqcy. Suma wyrazów tego ciqgu jest

równa 105. Liczby $x, y$ oraz $z$ sq- odpowiednio-pierwszym, drugim oraz szóstym

wyrazem ciqgu arytmetycznego $(a_{n})$, określonego dla $\mathrm{k}\mathrm{a}\dot{\mathrm{z}}$ dej liczby naturalnej $n\geq 1.$

Oblicz x, y oraz z. Zapisz obliczenia.

Strona 10 z27

$\mathrm{M}\mathrm{M}\mathrm{A}\mathrm{P}-\mathrm{R}0_{-}100$



\end{document}