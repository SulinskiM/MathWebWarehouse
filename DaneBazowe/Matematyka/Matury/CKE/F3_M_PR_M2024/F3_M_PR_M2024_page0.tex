\documentclass[a4paper,12pt]{article}
\usepackage{latexsym}
\usepackage{amsmath}
\usepackage{amssymb}
\usepackage{graphicx}
\usepackage{wrapfig}
\pagestyle{plain}
\usepackage{fancybox}
\usepackage{bm}

\begin{document}

CENTRALNA

KOMISJA

EGZAMINACYJNA

Arkusz zawiera informacje prawnie chronione

do momentu rozpoczecia egzaminu.

KOD

WYPELNIA ZDAJACY

PESEL

{\it Miejsce na naklejke}.

{\it Sprawdz}', {\it czy kod na naklejce to}

M-100.
\begin{center}
\includegraphics[width=21.900mm,height=10.164mm]{./F3_M_PR_M2024_page0_images/image001.eps}

\includegraphics[width=79.656mm,height=10.164mm]{./F3_M_PR_M2024_page0_images/image002.eps}
\end{center}
/{\it ezeli tak}- {\it przyklej naklejkq}.

/{\it ezeli nie}- {\it zgtoś to nauczycielowi}.

Egzamin maturalny

$\displaystyle \int$
\begin{center}
\includegraphics[width=193.344mm,height=78.180mm]{./F3_M_PR_M2024_page0_images/image003.eps}
\end{center}
Poziom  rozszerzony

{\it Symbol arkusza}

MMAP-R0-100-2405

DATA: 15 maja 2024 r.
\begin{center}
\begin{tabular}{|l|}
\hline
\multicolumn{1}{|l|}{WYP N1A $\mathrm{S}\mathrm{P}6$ NADZORUJACY}	\\
\hline
\multicolumn{1}{|l|}{$\begin{array}{l}\mbox{Uprawnienia zdaj cego do:}	\\	\mbox{dostosowania zasad oceniania.}	\end{array}$}	\\
\hline
\end{tabular}

\end{center}
GODZINA R0ZP0CZECIA: 9:00

CZAS TRWANIA: $180 \displaystyle \min$ ut

LICZBA PUNKTÓW DO UZYSKANIA 50

Przed rozpoczeciem pracy z arkuszem egzaminacyjnym

1.

Sprawd $\acute{\mathrm{z}}$, czy nauczyciel przekazal Ci wlaściwy arkusz egzaminacyjny,

tj. arkusz we wlaściwej formule, z w[aściwego przedmiotu na wlaściwym

poziomie.

2.

$\mathrm{J}\mathrm{e}\dot{\mathrm{z}}$ eli przekazano Ci niew[aściwy arkusz- natychmiast zgloś to nauczycielowi.

Nie rozrywaj banderol.

3.

$\mathrm{J}\mathrm{e}\dot{\mathrm{z}}$ eli przekazano Ci w[aściwy arkusz- rozerwij banderole po otrzymaniu

takiego polecenia od nauczyciela. Zapoznaj $\mathrm{s}\mathrm{i}\mathrm{e}$ z instrukcjq na stronie 2.

$\mathrm{U}\mathrm{k}\}\mathrm{a}\mathrm{d}$ graficzny

\copyright CKE 2022 $\bullet 1$

$\Vert\Vert\Vert\Vert\Vert\Vert\Vert\Vert\Vert\Vert\Vert\Vert\Vert\Vert\Vert\Vert\Vert\Vert\Vert\Vert\Vert\Vert\Vert\Vert\Vert\Vert\Vert\Vert\Vert\Vert|$
\end{document}
