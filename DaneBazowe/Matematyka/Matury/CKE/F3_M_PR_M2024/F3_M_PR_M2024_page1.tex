\documentclass[a4paper,12pt]{article}
\usepackage{latexsym}
\usepackage{amsmath}
\usepackage{amssymb}
\usepackage{graphicx}
\usepackage{wrapfig}
\pagestyle{plain}
\usepackage{fancybox}
\usepackage{bm}

\begin{document}

lnstrukcja dla zdajqcego

1.

2.

3.

4.

5.

6.

7.

8.

9.

Sprawd $\acute{\mathrm{z}}$, czy arkusz egzaminacyjny zawiera 27 stron (zadania $1-13$).

Ewentualny brak zgloś przewodniczqcemu zespolu nadzorujqcego egzamin.

Na pierwszej stronie arkusza oraz na karcie odpowiedzi wpisz swój numer PESEL

i przyklej naklejke z kodem.

$\mathrm{P}\mathrm{a}\mathrm{m}\mathrm{i}_{9}\mathrm{t}\mathrm{a}\mathrm{j}, \dot{\mathrm{z}}\mathrm{e}$ pominiecie argumentacji lub istotnych obliczeń w rozwiqzaniu zadania

otwartego $\mathrm{m}\mathrm{o}\dot{\mathrm{z}}\mathrm{e}$ spowodować, $\dot{\mathrm{z}}\mathrm{e}$ za to rozwiazanie nie otrzymasz pelnej liczby punktów.

Rozwiqzania zadań i odpowiedzi wpisuj w miejscu na to przeznaczonym.

Pisz czytelnie i $\mathrm{u}\dot{\mathrm{z}}$ ywaj tylko dlugopisu lub pióra z czarnym tuszem lub atramentem.

Nie $\mathrm{u}\dot{\mathrm{z}}$ ywaj korektora, a bledne zapisy wyra $\acute{\mathrm{z}}$ nie przekreśl.

Nie wpisuj $\dot{\mathrm{z}}$ adnych znaków w tabelkach przeznaczonych dla egzaminatora. Tabelki

umieszczone $\mathrm{s}_{\mathrm{c}}1$ na marginesie przy $\mathrm{k}\mathrm{a}\dot{\mathrm{z}}$ dym zadaniu.

$\mathrm{P}\mathrm{a}\mathrm{m}\mathrm{i}_{9}\mathrm{t}\mathrm{a}\mathrm{j}, \dot{\mathrm{z}}\mathrm{e}$ zapisy w brudnopisie nie bedq oceniane.

$\mathrm{M}\mathrm{o}\dot{\mathrm{z}}$ esz korzystač z Wybranych wzorów matematycznych, cyrkla i linijki oraz kalkulatora

prostego. Upewnij $\mathrm{s}\mathrm{i}\mathrm{e}$, czy przekazano Ci broszur9 z ok1adka taka jak widoczna ponizej.

Strona 2 z27

$\mathrm{M}\mathrm{M}\mathrm{A}\mathrm{P}-\mathrm{R}0_{-}100$
\end{document}
