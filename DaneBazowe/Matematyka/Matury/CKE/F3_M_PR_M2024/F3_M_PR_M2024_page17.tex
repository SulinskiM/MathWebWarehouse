\documentclass[a4paper,12pt]{article}
\usepackage{latexsym}
\usepackage{amsmath}
\usepackage{amssymb}
\usepackage{graphicx}
\usepackage{wrapfig}
\pagestyle{plain}
\usepackage{fancybox}
\usepackage{bm}

\begin{document}

Zadanie Y\S$*$(0-5)

$\mathrm{W}$ kartezjańskim ukladzie wspólrzednych $(x,y)$ środek $S$ okregu o promieniu $\sqrt{5} \mathrm{l}\mathrm{e}\dot{\mathrm{z}}\mathrm{y}$ na

prostej o równaniu $y=x+1$. Przez punkt $A=(1,2)$, którego odleglośč od punktu $S$ jest

wipksza od $\sqrt{5}$, poprowadzono dwie proste styczne do tego okregu w punktach-

odpowiednio- $B \mathrm{i} C$. Pole czworokqta ABSC jest równe 15.

Oblicz wspólrzqdne punktu $S.$ Rozwa $\dot{\mathrm{z}}$ wszystkie przypadki. Zapisz obliczenia.

Strona 18 z27

$\mathrm{M}\mathrm{M}\mathrm{A}\mathrm{P}-\mathrm{R}0_{-}100$
\end{document}
