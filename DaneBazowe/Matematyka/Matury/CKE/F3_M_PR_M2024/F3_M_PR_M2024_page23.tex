\documentclass[a4paper,12pt]{article}
\usepackage{latexsym}
\usepackage{amsmath}
\usepackage{amssymb}
\usepackage{graphicx}
\usepackage{wrapfig}
\pagestyle{plain}
\usepackage{fancybox}
\usepackage{bm}

\begin{document}

Zadanie 83[‡C]2. (0-4)

Pole $P$ powierzchni calkowitej graniastoslupa w zalezności od d$\dagger$ugości $a$ krawedzi

podstawy graniastoslupa jest określone wzorem

$P(a)=\displaystyle \frac{a^{2}\cdot\sqrt{3}}{2}+\frac{13824\sqrt{3}}{a}$

dla $a\in(0,8\sqrt{3}].$

Wyznacz d[ugośč krawedzi podstawy tego z rozwa $\dot{\mathrm{z}}$ anych graniastos[upów, którego

pole powierzchni calkowitej jest najmniejsze. Oblicz to najmniejsze pole. Zapisz

obliczenia.

Strona 24 z27

$\mathrm{M}\mathrm{M}\mathrm{A}\mathrm{P}-\mathrm{R}0_{-}100$
\end{document}
