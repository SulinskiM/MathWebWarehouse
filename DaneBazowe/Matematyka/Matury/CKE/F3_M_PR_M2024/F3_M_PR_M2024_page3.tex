\documentclass[a4paper,12pt]{article}
\usepackage{latexsym}
\usepackage{amsmath}
\usepackage{amssymb}
\usepackage{graphicx}
\usepackage{wrapfig}
\pagestyle{plain}
\usepackage{fancybox}
\usepackage{bm}

\begin{document}

Zadanie 8. $(0-2$\}

$\mathrm{W}$ chwili poczqtkowej$(t=0)$ filizanka z goracq kawq znajduje si9 w pokoju, a temperatura

tej kawy jest równa $80^{\mathrm{o}}\mathrm{C}$. Temperatura w pokoju (temperatura otoczenia)jest stala

i równa $20^{\mathrm{o}}\mathrm{C}$. Temperatura $T$ tej kawy zmienia si9 w czasie zgodnie z za1eznościq

$T(t)=(T_{p}-T_{Z})\cdot k^{-r}+T_{Z}$ dla

$r\geq 0$

gdzie:

T - temperatura kawy wyrazona w stopniach Celsjusza,

$t -$ czas wyrazony w minutach, liczony od chwili poczqtkowej,

$T_{\mathrm{P}}-$ temperatura poczqtkowa kawy wyrazona w stopniach Celsjusza,

$T_{Z}-$ temperatura otoczenia wyrazona w stopniach Celsjusza,

$k -$ stala charakterystyczna dla danej cieczy.

Po 10 minutach, 1iczqc od chwi1i poczatkowej, kawa ostyg1a do temperatury 65 $\mathrm{o}\mathrm{C}.$

Oblicz temperature tej kawy po nastepnych pieciu minutach. Wynik podaj w stopniach

Celsjusza, w zaokrqgleniu do jedności. Zapisz obliczenia.

Strona 4 z27

$\mathrm{M}\mathrm{M}\mathrm{A}\mathrm{P}-\mathrm{R}0_{-}100$
\end{document}
