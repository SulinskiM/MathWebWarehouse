\documentclass[a4paper,12pt]{article}
\usepackage{latexsym}
\usepackage{amsmath}
\usepackage{amssymb}
\usepackage{graphicx}
\usepackage{wrapfig}
\pagestyle{plain}
\usepackage{fancybox}
\usepackage{bm}

\begin{document}

Zadanie 3. $(0-3$\}

$\mathrm{W}$ pewnym zakladzie mleczarskim śmietana produkowana jest w 200-gramowych

opakowaniach. Prawdopodobieństwo zdarzenia, $\dot{\mathrm{z}}\mathrm{e}$ w losowo wybranym opakowaniu

śmietana zawiera mniej $\mathrm{n}\mathrm{i}\dot{\mathrm{z}}$ 36\% tluszczu, jest równe 0,0l. Kontroli poddajemy l0 losowo

wybranych opakowań ze śmietanq.

Oblicz prawdopodobieństwo zdarzenia polegajqcego na tym, $\dot{\mathrm{z}}\mathrm{e}$ wśród opakowań

poddanych $\mathrm{t}\mathrm{e}\mathrm{i}$ kontroli bedzie co najwy $\dot{\mathrm{z}}$ ej jedno opakowanie ze śmietanq, która

zawiera mniej $\mathrm{n}\mathrm{i}\dot{\mathrm{z}}$ 36\% tluszczu. Wynik zapisz w postaci ulamka dziesietnego

w zaokrqgleniu do cześci tysiecznych. Zapisz obliczenia.

Strona 6 z27

$\mathrm{M}\mathrm{M}\mathrm{A}\mathrm{P}-\mathrm{R}0_{-}100$
\end{document}
