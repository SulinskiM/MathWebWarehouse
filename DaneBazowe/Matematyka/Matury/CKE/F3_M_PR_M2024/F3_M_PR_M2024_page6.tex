\documentclass[a4paper,12pt]{article}
\usepackage{latexsym}
\usepackage{amsmath}
\usepackage{amssymb}
\usepackage{graphicx}
\usepackage{wrapfig}
\pagestyle{plain}
\usepackage{fancybox}
\usepackage{bm}

\begin{document}

Zadaníe 4. $(0-3$\}

Funkcja $f$ jest określona wzorem

$f(x)=\displaystyle \frac{x^{3}-3x+2}{\chi}$

dla $\mathrm{k}\mathrm{a}\dot{\mathrm{z}}$ dej liczby rzeczywistej $x$ róznej od zera. $\mathrm{W}$ kartezjańskim ukladzie wspólrz9dnych

$(x,\mathrm{y})$ punkt $P$, o pierwszej wspólrzednej równej 2, na1ez $\mathrm{y}$ do wykresu funkcji $f.$

Prosta o równaniu $y=ax+b$ jest styczna do wykresu funkcji $f$ w punkcie $P.$

Oblicz wspólczynniki a oraz b w równaniu tei stycznej. Zapisz obliczenia.

$\mathrm{M}\mathrm{M}\mathrm{A}\mathrm{P}-\mathrm{R}0_{-}100$

Strona 7 z27
\end{document}
