\documentclass[a4paper,12pt]{article}
\usepackage{latexsym}
\usepackage{amsmath}
\usepackage{amssymb}
\usepackage{graphicx}
\usepackage{wrapfig}
\pagestyle{plain}
\usepackage{fancybox}
\usepackage{bm}

\begin{document}

Zadanie 6, $(0-3$\}

Rozwazamy wszystkie liczby naturalne, w których zapisie $\mathrm{d}\mathrm{z}\mathrm{i}\mathrm{e}\mathrm{s}\mathrm{i}9$tnym nie powtarza si9

jakakolwiek cyfra oraz dokladnie trzy cyfry sq nieparzyste i dokladnie dwie cyfry sq parzyste.

Oblicz, ile jest wszystkich takich liczb. Zapisz obliczenia.

$\mathrm{M}\mathrm{M}\mathrm{A}\mathrm{P}-\mathrm{R}0_{-}100$

Strona 9 z27
\end{document}
