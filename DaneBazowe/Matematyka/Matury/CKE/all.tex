\documentclass[a4paper,12pt]{article}
\usepackage{latexsym}
\usepackage{amsmath}
\usepackage{amssymb}
\usepackage{graphicx}
\usepackage{wrapfig}
\pagestyle{plain}
\usepackage{fancybox}
\usepackage{bm}

\begin{document}

Centralna Komisja Egzaminacyjna

Arkusz zawiera informacje prawnie chronione do momentu rozpoczęcia egzaminu.

WPISUJE ZDAJACY

KOD PESEL

{\it Miejsce}

{\it na naklejkę}

{\it z kodem}
\begin{center}
\includegraphics[width=21.432mm,height=9.804mm]{./F1_M_PP_C2012_page0_images/image001.eps}

\includegraphics[width=82.092mm,height=9.804mm]{./F1_M_PP_C2012_page0_images/image002.eps}
\end{center}
\fbox{} dysleksja
\begin{center}
\includegraphics[width=204.060mm,height=216.048mm]{./F1_M_PP_C2012_page0_images/image003.eps}
\end{center}
EGZAMIN MATU LNY

Z MATEMATYKI

CZERWIEC 2012

POZIOM PODSTAWOWY

1. Sprawd $\acute{\mathrm{z}}$, czy arkusz egzaminacyjny zawiera 18 stron

(zadania $1-34$). Ewentualny brak zgłoś przewodniczącemu

zespo nadzorującego egzamin.

2. Rozwiązania zadań i odpowiedzi wpisuj w miejscu na to

przeznaczonym.

3. Odpowiedzi do zadań za niętych (l-24) przenieś

na ka ę odpowiedzi, zaznaczając je w części ka $\mathrm{y}$

przeznaczonej dla zdającego. Zamaluj $\blacksquare$ pola do tego

przeznaczone. Błędne zaznaczenie otocz kółkiem \fcircle$\bullet$

i zaznacz właściwe.

4. Pamiętaj, $\dot{\mathrm{z}}\mathrm{e}$ pominięcie argumentacji lub istotnych

obliczeń w rozwiązaniu zadania otwa ego (25-34) $\mathrm{m}\mathrm{o}\dot{\mathrm{z}}\mathrm{e}$

spowodować, $\dot{\mathrm{z}}\mathrm{e}$ za to rozwiązanie nie będziesz mógł

dostać pełnej liczby punktów.

5. Pisz czytelnie i uzywaj tvlko długopisu lub -Dióra

z czarnym tuszem lub atramentem.

6. Nie uzywaj korektora, a błędne zapisy wyrazínie prze eśl.

7. Pamiętaj, $\dot{\mathrm{z}}\mathrm{e}$ zapisy w brudnopisie nie będą oceniane.

8. $\mathrm{M}\mathrm{o}\dot{\mathrm{z}}$ esz korzystać z zestawu wzorów matematycznych,

cyrkla i linijki oraz kalkulatora.

9. Na tej stronie oraz na karcie odpowiedzi wpisz swój

numer PESEL i przyklej naklejkę z kodem.

10. Nie wpisuj $\dot{\mathrm{z}}$ adnych znaków w części przeznaczonej

dla egzaminatora.

Czas pracy:

170 minut

Liczba punktów

do uzyskania: 50

$\Vert\Vert\Vert\Vert\Vert\Vert\Vert\Vert\Vert\Vert\Vert\Vert\Vert\Vert\Vert\Vert\Vert\Vert\Vert\Vert\Vert\Vert\Vert\Vert|  \mathrm{M}\mathrm{M}\mathrm{A}-\mathrm{P}1_{-}1\mathrm{P}-123$




{\it 2}

{\it Egzamin maturalny z matematyki}

{\it Poziom podstawowy}

ZADANIA ZAMKNIĘTE

{\it Wzadaniach} $\theta d1.$ {\it do 24. wybierz i zaznacz na karcie odpowiedzipoprawnq odpowiedzí}.

Zadanie l. $(1pkt)$

Ułamek $\displaystyle \frac{\sqrt{5}+2}{\sqrt{5}-2}$ jest równy

A. 1 B. $-1$

C. $7+4\sqrt{5}$

D. $9+4\sqrt{5}$

Zadanie 2. $(1pkt)$

Liczbami spełniającymi równanie $|2x+3|=5$ są

A. $1\mathrm{i}-4$

B. l i 2

C. $-1\mathrm{i}4$

D. $-2\mathrm{i}2$

Zadanie 3. $(1pkt)$

Równanie $(x+5)(x-3)(x^{2}+1)=0$ ma

A.

B.

C.

D.

dwa rozwiązania: $x=-5, x=3.$

dwa rozwiązania: $x=-3, x=5.$

cztery rozwiązania: $x=-5, x=-1, x=1, x=3.$

cztery rozwiązania: $x=-3, x=-1, x=1, x=5.$

Zadanie 4. (1pkt)

Marza równa 1,5\% kwoty pozyczonego kapitału była równa 3000 zł.

pozyczono

Wynika stąd, $\dot{\mathrm{z}}\mathrm{e}$

A. 45 zł

B. 2000 zł

C. 200000 zł

D. 450000 zł

Zadanie 5. $(1pkt)$

Najednym z ponizszych rysunków przedstawiono fragment wykresu funkcji $y=x^{2}+2x-3.$

Wskaz ten rysunek.
\begin{center}
\includegraphics[width=4.932mm,height=22.812mm]{./F1_M_PP_C2012_page1_images/image001.eps}

\begin{tabular}{|l|l|}
\hline
\multicolumn{1}{|l|}{ $\begin{array}{l}\mbox{$4$}	\\	\mbox{ $3$}	\\	\mbox{ $2$}	\\	\mbox{ $1$}	\end{array}$}&	\multicolumn{1}{|l|}{ $\mathrm{y}$}	\\
\hline
\multicolumn{1}{|l|}{ $\begin{array}{l}\mbox{ $-4-2-1$}	\\	\mbox{ $-1$}	\\	\mbox{ $-2$}	\\	\mbox{ $-3$}	\\	\mbox{ $-4$}	\end{array}$}&	\multicolumn{1}{|l|}{ $234$}	\\
\hline
\end{tabular}


\begin{tabular}{|l|l|}
\hline
\multicolumn{1}{|l|}{ $\begin{array}{l}\mbox{$4$}	\\	\mbox{ $3$}	\\	\mbox{ $1$}	\end{array}$}&	\multicolumn{1}{|l|}{ $\mathrm{y}$}	\\
\hline
\multicolumn{1}{|l|}{ $-4-3-21^{1}4321$}&	\multicolumn{1}{|l|}{ $124$}	\\
\hline
\end{tabular}


\begin{tabular}{|l|l|}
\hline
\multicolumn{1}{|l|}{ $\begin{array}{l}\mbox{$4$}	\\	\mbox{ $3$}	\\	\mbox{ $2$}	\\	\mbox{ $1$}	\end{array}$}&	\multicolumn{1}{|l|}{ $\mathrm{y}$}	\\
\hline
\multicolumn{1}{|l|}{ $\begin{array}{l}\mbox{ $-43-2-1$}	\\	\mbox{ $-1$}	\\	\mbox{ $-2$}	\\	\mbox{ $-3$}	\\	\mbox{ $-4$}	\end{array}$}&	\multicolumn{1}{|l|}{ $234$}	\\
\hline
\end{tabular}


\includegraphics[width=4.932mm,height=22.812mm]{./F1_M_PP_C2012_page1_images/image002.eps}

\begin{tabular}{|l|l|}
\hline
\multicolumn{1}{|l|}{ $\begin{array}{l}\mbox{$4$}	\\	\mbox{ $3$}	\\	\mbox{ $2$}	\\	\mbox{ $1$}	\end{array}$}&	\multicolumn{1}{|l|}{ $\mathrm{y}$}	\\
\hline
\multicolumn{1}{|l|}{ $\begin{array}{l}\mbox{ $-4-3-2$}	\\	\mbox{ $-1$}	\\	\mbox{ $-3$}	\\	\mbox{ $-4$}	\end{array}$}&	\multicolumn{1}{|l|}{ $124$}	\\
\hline
\end{tabular}


\includegraphics[width=5.232mm,height=22.860mm]{./F1_M_PP_C2012_page1_images/image003.eps}
\end{center}
A.

B.

C.

D.





{\it Egzamin maturalny z matematyki}

{\it Poziom podstawowy}

{\it 11}

Zadanie 27. (2pkt)

Podstawy trapezu prostokątnego mają długości 6 i 10 oraz tangens jego kąta ostrego jest

równy 3. Ob1icz po1e tego trapezu.

Odpowiedzí :

Zadanie 28. $(2pkt)$

Uzasadnij, $\dot{\mathrm{z}}$ ejezeli $\alpha$ jest kątem ostrym, to $\sin^{4}\alpha+\cos^{2}\alpha=\sin^{2}\alpha+\cos^{4}\alpha.$





{\it 12}

{\it Egzamin maturalny z matematyki}

{\it Poziom podstawowy}

Zadanie 29. $(2pkt)$

Uzasadnij, $\dot{\mathrm{z}}\mathrm{e}$ suma kwadratów trzech kolejnych liczb całkowitych przy dzieleniu przez 3 daje

resztę 2.

Zadanie 30. $(2pkt)$

Suma $S_{n}=a_{1}+a_{2}+\ldots+a_{n}$ początkowych $n$ wyrazów pewnego ciągu arytmetycznego $(a_{n})$

jest określona wzorem $S_{n}=n^{2}-2n$ dla $n\geq 1$. Wyznacz wzór na n-ty wyraz tego ciągu.

Odpowied $\acute{\mathrm{z}}$:





{\it Egzamin maturalny z matematyki}

{\it Poziom podstawowy}

{\it 13}

Zadanie 31. $(2pkt)$

Dany jest romb, którego kąt ostry ma miarę $45^{\mathrm{o}}$, a jego pole jest równe $50\sqrt{2}$. Oblicz

wysokość tego rombu.

Odpowied $\acute{\mathrm{z}}$:





{\it 14}

{\it Egzamin maturalny z matematyki}

{\it Poziom podstawowy}

Zadanie 32. $(4pkt)$

Punkty $A=(2,11), B=(8,23), C=(6,14)$ są wierzchołkami trójkąta. Wysokość trójkąta

poprowadzona z wierzchołka $C$ przecina prostą AB w punkcie $D$. Oblicz współrzędne punktu $D.$

Odpowied $\acute{\mathrm{z}}$:





{\it Egzamin maturalny z matematyki}

{\it Poziom podstawowy}

{\it 15}

Zadanie 33. (4pkt)

Oblicz, ile jest liczb naturalnych pięciocyfrowych, w zapisie których nie występuje zero, jest

dokładniejedna cyfra 7 i dokładniejedna cyfra parzysta.

Odpowiedzí :





{\it 16}

{\it Egzamin maturalny z matematyki}

{\it Poziom podstawowy}

Zadanie 34. (4pkt)

Dany jest graniastosłup prawidłowy trójkątny ABCDEF o podstawach ABC i DEF

i krawędziach bocznych AD, BE iCF (zobacz rysunek). Długość krawędzi podstawy AB jest

równa 8, a po1e trójkąta ABFjest równe 52. Ob1icz objętość tego graniastosłupa.





{\it Egzamin maturalny z matematyki}

{\it Poziom podstawowy}

{\it 1}7

Odpowied $\acute{\mathrm{z}}$:





{\it 18}

{\it Egzamin maturalny z matematyki}

{\it Poziom podstawowy}

BRUDNOPIS





{\it Egzamin maturalny z matematyki}

{\it Poziom podstawowy}

{\it 3}

BRUDNOPIS





{\it 4}

{\it Egzamin maturalny z matematyki}

{\it Poziom podstawowy}

Zadanie 6. $(1pkt)$

Wierzchołkiem paraboli będącej wykresem ffinkcji określonej wzorem $f(x)=x^{2}-4x+4$

jest punkt o współrzędnych

A. (0,2)

B. $(0,-2)$

C. $(-2,0)$

D. (2, 0)

Zadanie 7. $(1pkt)$

Jeden kąt trójkąta ma miarę $54^{\mathrm{o}} \mathrm{Z}$ pozostałych dwóch kątów tego trójkątajedenjest 6 razy

większy od drugiego. Miary pozostałych kątów są równe

A. $21^{\mathrm{o}}$ i $105^{\mathrm{o}}$

B. $11^{\mathrm{o}}$ i $66^{\mathrm{o}}$

C. $18^{\mathrm{o}}$ i $108^{\mathrm{o}}$

D. $16^{\mathrm{o}}\mathrm{i}96^{\mathrm{o}}$

Zadanie 8. $(1pkt)$

Krótszy bok prostokąta ma długość 6. Kąt między przekątną prostokąta i dłuzszym bokiem

ma miarę $30^{\mathrm{o}}$. Dłuzszy bok prostokąta ma długość

A. $2\sqrt{3}$

B. $4\sqrt{3}$

C. $6\sqrt{3}$

D. 12

Zadanie 9. (1pkt)

Cięciwa okręgu ma długość 8 cm ijest odda1ona odjego środka o 3 cm. Promień tego okręgu

ma długość

A. 3 cm

B. 4 cm

C. 5 cm

D. 8 cm

Zadanie 10. (1pkt)

Punkt O jest środkiem okręgu. Kąt wpisany BAD ma miarę

A. $150^{\mathrm{o}}$
\begin{center}
\includegraphics[width=44.040mm,height=46.380mm]{./F1_M_PP_C2012_page3_images/image001.eps}
\end{center}
{\it D  C}

$130^{\circ}$

{\it O}

$60^{\circ}$

{\it B}

{\it A}

$115^{\mathrm{o}}$

$120^{\mathrm{o}}$

C.

B.

D. $85^{\mathrm{o}}$

Zadanie ll. (lpkt)

Pięciokąt ABCDE jest foremny. Wskaz trójkąt przystający do trójkąta ECD

A.

$\Delta ABF$

B.

$\Delta CAB$
\begin{center}
\includegraphics[width=55.884mm,height=50.088mm]{./F1_M_PP_C2012_page3_images/image002.eps}
\end{center}
{\it D}

{\it E  I H  C}

{\it J  G}

{\it F}

{\it A B}

$\Delta ABD$

D.

$\Delta IHD$

C.





{\it Egzamin maturalny z matematyki}

{\it Poziom podstawowy}

{\it 5}

BRUDNOPIS





{\it 6}

{\it Egzamin maturalny z matematyki}

{\it Poziom podstawowy}

Zadanie 12. (1pkt)

Punkt O jest środkiem okręgu przedstawionego na rysunku. Równanie tego okręgu ma postać:

A.

B.
\begin{center}
\includegraphics[width=65.436mm,height=64.104mm]{./F1_M_PP_C2012_page5_images/image001.eps}
\end{center}
y

4

2

{\it o}

$-1$  1 2  3 4

x

5

$-2$

D.

C.

Zadanie 13. $(1pkt)$

Wyra $\dot{\mathrm{z}}$ enie $\displaystyle \frac{3x+1}{x-2}-\frac{2x-1}{x+3}$ jest równe

A.

-({\it xx}2-$+$21)5({\it xx} $++$31)

B.

$\displaystyle \frac{x+2}{(x-2)(x+3)}$

$(x-2)^{2}+(y-1)^{2}=9$

$(x-2)^{2}+(y-1)^{2}=3$

$(x+2)^{2}+(y+1)^{2}=9$

$(x+2)^{2}+(y+1)^{2}=3$

C. $\displaystyle \frac{x}{(x-2)(x+3)}$

D.

$\displaystyle \frac{x+2}{-5}$

Zadanie 14. $(1pkt)$

Ciąg $(a_{n})$ jest określony wzorem $a_{n}=\sqrt{2n+4}$ dla $n\geq 1$. Wówczas

A. $a_{8}=2\sqrt{5}$

B. $a_{8}=8$

C. $a_{8}=5\sqrt{2}$

D. $a_{8}=\sqrt{12}$

Zadanie 15. $(1pkt)$

Ciąg $(2\sqrt{2},4,a)$ jest geometryczny. Wówczas

A. $a=8\sqrt{2}$

B. $a=4\sqrt{2}$

C. $a=8-2\sqrt{2}$

D. $a=8+2\sqrt{2}$

Zadanie 16. $(1pkt)$

Kąt $\alpha$ jest ostry i $\mathrm{t}\mathrm{g}\alpha=1$. Wówczas

A. $\alpha<30^{\mathrm{o}}$

B. $\alpha=30^{\mathrm{o}}$

C. $\alpha=45^{\mathrm{o}}$

D. $\alpha>45^{\mathrm{o}}$

Zadanie 17. (1pkt)

Wiadomo, $\dot{\mathrm{z}}\mathrm{e}$ dziedziną funkcji

$(-\infty,2)\cup(2,+\infty)$. Wówczas

$f$ określonej wzorem $f(x)=\displaystyle \frac{x-7}{2x+a}$ jest zbiór

A. $a=2$

B. $a=-2$

C. $a=4$

D. $a=-4$





{\it Egzamin maturalny z matematyki}

{\it Poziom podstawowy}

7

BRUDNOPIS





{\it 8}

{\it Egzamin maturalny z matematyki}

{\it Poziom podstawowy}

Zadanie 18. $(1pkt)$

Jeden z rysunków przedstawia wykres ffinkcji liniowej $f(x)=ax+b$, gdzie $a>0\mathrm{i}b<0$. Wskaz

ten wykres.
\begin{center}
\includegraphics[width=45.108mm,height=45.108mm]{./F1_M_PP_C2012_page7_images/image001.eps}
\end{center}
$\gamma$

{\it x}

0
\begin{center}
\includegraphics[width=45.012mm,height=51.864mm]{./F1_M_PP_C2012_page7_images/image002.eps}
\end{center}
$\gamma$

{\it x}

0

D.

A.
\begin{center}
\includegraphics[width=45.156mm,height=51.864mm]{./F1_M_PP_C2012_page7_images/image003.eps}
\end{center}
$\gamma$

{\it x}

0

B.
\begin{center}
\includegraphics[width=44.904mm,height=51.912mm]{./F1_M_PP_C2012_page7_images/image004.eps}
\end{center}
$\gamma$

{\it x}

0

C.

Zadanie 19. $(1pkt)$

Punkt $S=(2,7)$ jest środkiem odcinka $AB$, w którym $A=(-1,3)$. Punkt $B$ ma współrzędne:

A. $B=(5,11)$ B. $B=(\displaystyle \frac{1}{2},2)$ C. $B=(-\displaystyle \frac{3}{2},-5)$ D. $B=(3,11)$

Zadanie 20. $(1pkt)$

$\mathrm{W}$ kolejnych sześciu rzutach kostką otrzymano następujące wyniki: 6, 3, 1, 2, 5, 5. Mediana

tych wynikówjest równa:

A. 3

B. 3,5

C. 4

D. 5

Zadanie 21. $(1pkt)$

RównoŚć $(a+2\sqrt{2})^{2}=a^{2}+28\sqrt{2}+8$ zachodzi dla

A. $a=14$ B. $a=7\sqrt{2}$ C.

$a=7$

D. $a=2\sqrt{2}$

Zadanie 22. (1pkt)

Trójkąt prostokątny o przyprostokątnych 4 i 6 obracamy wokół dłuzszej przyprostokątnej.

Objętość powstałego stozkajest równa

A. $ 96\pi$

B. $ 48\pi$

C. $ 32\pi$

D. $ 8\pi$

Zadanie 23. $(1pkt)$

$\mathrm{J}\mathrm{e}\dot{\mathrm{z}}$ eli $A \mathrm{i} B$ są zdarzeniami losowymi, $B'$ jest zdarzeniem przeciwnym do $B, P(A)=0,3,$

$P(B')=0,4$ oraz $ A\cap B=\emptyset$, to $P(A\cup B)$ jest równe

A. 0,12

B. 0,18

C. 0,6

D. 0,9

Zadanie 24. $(1pkt)$

Przekrój osiowy walca jest kwadratem o boku $a. \mathrm{J}\mathrm{e}\dot{\mathrm{z}}$ eli $r$ oznacza promień podstawy walca,

$h$ oznacza wysokość walca, to

A. $r+h=a$

B.

$h-r=\displaystyle \frac{a}{2}$

C.

{\it r-h}$=$ -{\it a}2

D. $r^{2}+h^{2}=a^{2}$





{\it Egzamin maturalny z matematyki}

{\it Poziom podstawowy}

{\it 9}

BRUDNOPIS





$ 1\theta$

{\it Egzamin maturalny z matematyki}

{\it Poziom podstawowy}

ZADANIA OTWARTE

{\it Rozwiqzania zadań o numerach od 25. do 34. nalezy zapisać w} $wyznacz\theta nych$ {\it miejscach}

{\it pod treściq zadania}.

Zadanie 25. $(2pkt)$

Rozwiąz nierówność $x^{2}-3x-10<0.$

Odpowiedz:

Zadanie 26. (2pkt)

Średnia wieku w pewnej giupie studentówjest równa 231ata. Średnia wieku tych studentów

i ich opiekunajest równa 241ata. Opiekun ma 391at. Ob1icz, i1u studentówjest w tej giupie.

Odpowiedzí:





\begin{center}
\begin{tabular}{l|l}
\multicolumn{1}{l|}{{\it dysleksja}}&	\multicolumn{1}{|l}{}	\\
\hline
\multicolumn{1}{l|}{ $\begin{array}{l}\mbox{MATERIAL DIAGNOSTYCZNY}	\\	\mbox{Z MATEMATYKI}	\\	\mbox{Arkusz I}	\\	\mbox{POZIOM PODSTAWOWY}	\\	\mbox{Czas pracy 120 minut}	\\	\mbox{Instrukcja dla ucznia}	\\	\mbox{1. $\mathrm{S}\mathrm{p}\mathrm{r}\mathrm{a}\mathrm{w}\mathrm{d}\acute{\mathrm{z}}$, czy arkusz zawiera 12 ponumerowanych stron.}	\\	\mbox{Ewentualny brak zgłoś przewodniczącemu zespo}	\\	\mbox{nadzorującego badanie.}	\\	\mbox{2. Rozwiązania i odpowiedzi zapisz w miejscu na to}	\\	\mbox{przeznaczonym.}	\\	\mbox{3. $\mathrm{W}$ rozwiązaniach zadań przedstaw tok rozumowania}	\\	\mbox{prowadzący do ostatecznego wyniku.}	\\	\mbox{4. Pisz czytelnie. Uzywaj długopisu pióra tylko z czamym}	\\	\mbox{tusze atramentem.}	\\	\mbox{5. Nie uzywaj korektora, a błędne zapisy $\mathrm{w}\mathrm{y}\mathrm{r}\mathrm{a}\acute{\mathrm{z}}\mathrm{n}\mathrm{i}\mathrm{e}$ prze eśl.}	\\	\mbox{6. Pamiętaj, $\dot{\mathrm{z}}\mathrm{e}$ zapisy w brudnopisie nie podlegają ocenie.}	\\	\mbox{7. $\mathrm{M}\mathrm{o}\dot{\mathrm{z}}$ esz korzystać z zestawu wzorów matematycznych, cyrkla}	\\	\mbox{i linijki oraz kalkulatora.}	\\	\mbox{8. Wypełnij tę część ka $\mathrm{y}$ odpowiedzi, którą koduje uczeń. Nie}	\\	\mbox{wpisuj $\dot{\mathrm{z}}$ adnych znaków w części przeznaczonej dla}	\\	\mbox{oceniającego.}	\\	\mbox{9. Na karcie odpowiedzi wpisz swoją datę urodzenia i PESEL.}	\\	\mbox{Zamaluj $\blacksquare$ pola odpowiadające cyfrom numeru PESEL. Błędne}	\\	\mbox{zaznaczenie otocz kółkiem \fcircle i zaznacz właściwe.}	\\	\mbox{{\it Zyczymy} $p\theta wodzenia'$}	\end{array}$}&	\multicolumn{1}{|l}{$\begin{array}{l}\mbox{ARKUSZ I}	\\	\mbox{GRUDZIEN}	\\	\mbox{ROK 2005}	\\	\mbox{Za rozwiązanie}	\\	\mbox{wszystkich zadań}	\\	\mbox{mozna otrzymać}	\\	\mbox{łącznie}	\\	\mbox{50 punktów}	\end{array}$}	\\
\hline
\multicolumn{1}{l|}{$\begin{array}{l}\mbox{W ełnia uczeń rzed roz oczęciem rac}	\\	\mbox{PESEL UCZNIA}	\end{array}$}&	\multicolumn{1}{|l}{$\begin{array}{l}\mbox{Wypełnia uczeń}	\\	\mbox{przed rozpoczęciem}	\\	\mbox{pracy}	\\	\mbox{KOD UCZNIA}	\end{array}$}
\end{tabular}


\includegraphics[width=80.724mm,height=12.756mm]{./F1_M_PP_G2005_page0_images/image001.eps}

\includegraphics[width=23.616mm,height=9.852mm]{./F1_M_PP_G2005_page0_images/image002.eps}
\end{center}



{\it 2}

{\it Materialpomocniczy do doskonalenia nauczycieli w zakresie diagnozowania, oceniania i egzaminowania}

{\it Matematyka}- {\it grudzień 2005 r}.

Zadanie l. $(4pkt)$

Wielomian $P(x)=x^{3}-21x+20$ rozłóz na czynniki liniowe, to znaczy zapisz go w postaci

iloczynu trzech wielomianów stopnia pierwszego.





{\it Materialpomocniczy do doskonalenia nauczycieli w zakresie diagnozowania, oceniania i egzaminowania ll}

{\it Matematyka}- {\it grudzień 2005 r}.

Zadanie 10. $(7pkt)$

Pole powierzchni całkowitej prawidłowego ostrosłupa trójkątnego równa

a polejego powierzchni bocznej $96\sqrt{3}$. Oblicz objętość tego ostrosłupa.

się $144\sqrt{3},$





{\it 12 Materiatpomocniczy do doskonalenia nauczycieli w zakresie diagnozowania, oceniania i egzaminowania}

{\it Matematyka}- {\it grudzień 2005 r}.

BRUDNOPIS





{\it Materialpomocniczy do doskonalenia nauczycieli w zakresie diagnozowania, oceniania i egzaminowania}

{\it Matematyka}- {\it grudzień 2005 r}.

{\it 3}

Zadanie 2. (4pkt)

W roku 2005 na uroczystości urodzin zapytano jubi1ata, i1e ma 1at.

Jubilat odpowiedział:,,Jeśli swój wiek sprzed 101at pomnozę przez swój wiek za 111at,

to otrzymam rok mojego urodzenia'' Ułóz odpowiednie równanie, rozwiąz je i zapisz,

w którym roku urodził się tenjubilat.





{\it 4}

{\it Materialpomocniczy do doskonalenia nauczycieli w zakresie diagnozowania, oceniania i egzaminowania}

{\it Matematyka}- {\it grudzień 2005 r}.

Zadanie 3. $(5pkt)$

Funkcja $f(x)$ jest określona wzorem: $f(x)=$

a) Sprawd $\acute{\mathrm{z}}$, czy liczba $a=(0,25)^{-0,5}$ nalezy do dziedziny funkcji $f(x).$

b) Oblicz $f(2)$ oraz $f(3).$

c) Sporządz$\acute{}$ wykres funkcji $f(x).$

d) Podaj rozwiązanie równania $f(x)=0.$

e) Zapisz zbiór wartości funkcji $f(x).$





{\it Materialpomocniczy do doskonalenia nauczycieli w zakresie diagnozowania, oceniania i egzaminowania}

{\it Matematyka}- {\it grudzień 2005 r}.

{\it 5}

Zadanie 4. $(6pkt)$

$\mathrm{W}$ układzie współrzędnych są dane dwa punkty: $A=(-2,2)\mathrm{i}B=(4,4).$

a) Wyznacz równanie prostej $AB.$

b) Prosta $AB$ oraz prosta o równaniu $9x-6y-26=0$ przecinają się w punkcie

Oblicz współrzędne punktu $C.$

c) Wyznacz równanie symetralnej odcinka $AB.$

{\it C}.





{\it 6}

{\it Materialpomocniczy do doskonalenia nauczycieli w zakresie diagnozowania, oceniania i egzaminowania}

{\it Matematyka}- {\it grudzień 2005 r}.

Zadanie 5. $(5pkt)$

Nieskończony ciąg liczbowy $(a_{n})$ jest określony wzorem $a_{n}=4n-31, n=1,2,3,\ldots.$

Wyrazy $a_{k}, a_{k+1}, a_{k+2}$ danego ciągu $(a_{n})$, wzięte w takim porządku, powiększono: wyraz

$a_{k} 01$, wyraz $a_{k+1} 03$ oraz wyraz $a_{k+2}023. \mathrm{W}$ ten sposób otrzymano trzy pierwsze wyrazy

pewnego ciągu geometrycznego. Wyznacz $k$ oraz czwarty wyraz tego ciągu geometrycznego.





{\it Materialpomocniczy do doskonalenia nauczycieli w zakresie diagnozowania, oceniania i egzaminowania}

{\it Matematyka}- {\it grudzień 2005 r}.

7

Zadanie 6. $(4pkt)$

Do szkolnych zawodów szachowych zgłosiło się 16 uczniów, wśród których było dwóch

faworytów. Organizatorzy zawodów zamierzają losowo podzielić szachistów na dwie

jednakowo liczne grupy eliminacyjne, Niebieską i Zółtą. Oblicz prawdopodobieństwo

zdarzenia polegającego na tym, $\dot{\mathrm{z}}\mathrm{e}$ faworyci tych zawodów nie znajdą się w tej samej grupie

eliminacyjnej. Końcowy wynik obliczeń zapisz w postaci ułamka nieskracalnego.





{\it 8}

{\it Materialpomocniczy do doskonalenia nauczycieli w zakresie diagnozowania, oceniania i egzaminowania}

{\it Matematyka}- {\it grudzień 2005 r}.

Zadanie 7. $(3pkt)$

Aby wyznaczyć wszystkie liczby całkowite $c$, dla których liczba postaci $\displaystyle \frac{c-3}{c-5}$ jest takz $\mathrm{e}$

liczbą całkowitą mozna postąpić w następujący sposób:

a) Wyrazenie w liczniku ułamka zapisujemy w postaci sumy, której jednym

ze składnikówjest wyrazenie z mianownika:

$\displaystyle \frac{c-3}{c-5}=\frac{(c-5)+2}{c-5}$

b) Zapisujemy powyzszy ułamek w postaci sumy liczby l oraz pewnego ułamka:

$\displaystyle \frac{c-5+2}{c-5}=\frac{c-5}{c-5}+\frac{2}{c-5}=1+\frac{2}{c-5}$

c) Zauwazamy, $\dot{\mathrm{z}}\mathrm{e}$ ułamek $\displaystyle \frac{2}{c-5}$ jest liczbą całkowitą wtedy i tylko wtedy, gdy liczba

$(c-5)$ jest całkowitym dzielnikiem liczby 2, czy1i $\dot{\mathrm{z}}\mathrm{e}(c-5)\in\{-1,1,-2,2\}.$

d) Rozwiązujemy kolejno równania $c-5=-1, c-5=1, c-5=-2, c-5=2,$

i otrzymujemy odpowiedzí: liczba postaci $\displaystyle \frac{c-3}{c-5}$ jest całkowita dla:

$c=4,c=6,c=3,c=7.$

Rozumując analogicznie, wyznacz wszystkie liczby całkowite $x$, dla których liczba postaci

$\displaystyle \frac{x}{x-3}$ jest liczbą całkowitą.





{\it Materialpomocniczy do doskonalenia nauczycieli w zakresie diagnozowania, oceniania i egzaminowania}

{\it Matematyka}- {\it grudzień 2005 r}.

{\it 9}

Zadanie 8. $(5pkt)$

$\mathrm{W}$ kwadrat ABCD wpisano kwadrat EFGH, jak pokazano na ponizszym rysunku. Wiedząc,

$\dot{\mathrm{z}}\mathrm{e}|AB|=1$ oraz tangens kąta $AEH$ równa się $\displaystyle \frac{2}{5}$, oblicz pole kwadratu EFGH.

{\it A}
\begin{center}
\includegraphics[width=95.304mm,height=93.168mm]{./F1_M_PP_G2005_page8_images/image001.eps}
\end{center}
{\it D  G  C}

{\it F}

{\it H}

{\it E B}





$ 1\theta$ {\it Materiatpomocniczy do doskonalenia nauczycieli w zakresie diagnozowania, oceniania i egzaminowania}

{\it Matematyka}- {\it grudzień 2005} $r.$

Zadanie 9. $(7pkt)$

Liczbę naturalną $t_{n}$ nazywamy $n$ -tą liczbą trójkątn\% $\mathrm{j}\mathrm{e}\dot{\mathrm{z}}$ eli jest ona sumą $n$

kolejnych,

początkowych liczb naturalnych. Liczbami trójkątnymi są zatem: $t_{1}=1, t_{2}=1+2=3,$

$t_{3}=1+2+3=6, t_{4}=1+2+3+4=10, t_{5}=1+2+3+4+5=15$. Stosując tę definicję:

a) wyznacz liczbę $t_{17}.$

b) ułóz odpowiednie równanie i zbadaj, czy liczba $7626$jest liczbą trójkątną.

c) wyznacz największą czterocyfrową liczbę trójkątną.






\begin{center}
\begin{tabular}{l|l}
\multicolumn{1}{l|}{$\begin{array}{l}\mbox{{\it dysleksja}}	\\	\mbox{Miejsce}	\\	\mbox{na na ejkę}	\\	\mbox{z kodem szkoly}	\end{array}$}&	\multicolumn{1}{|l}{}	\\
\hline
\multicolumn{1}{l|}{ $\begin{array}{l}\mbox{PRÓBNY EGZAMIN}	\\	\mbox{MATURALNY}	\\	\mbox{Z MATEMATYKI}	\\	\mbox{POZIOM PODSTAWOWY}	\\	\mbox{Czas pracy 120 minut}	\\	\mbox{Instrukcja dla zdającego}	\\	\mbox{1. $\mathrm{S}\mathrm{p}\mathrm{r}\mathrm{a}\mathrm{w}\mathrm{d}\acute{\mathrm{z}}$, czy arkusz egzaminacyjny zawiera 15 stron}	\\	\mbox{(zadania $1-11$). Ewentualny brak zgłoś przewodniczącemu}	\\	\mbox{zespo nadzorującego egzamin.}	\\	\mbox{2. Rozwiązania zadań i odpowiedzi zamieść w miejscu na to}	\\	\mbox{przeznaczonym.}	\\	\mbox{3. $\mathrm{W}$ rozwiązaniach zadań przedstaw tok rozumowania}	\\	\mbox{prowadzący do ostatecznego wyniku.}	\\	\mbox{4. Pisz czytelnie. Uzywaj długopisu pióra tylko z czamym}	\\	\mbox{tusze atramentem.}	\\	\mbox{5. Nie uzywaj korektora, a błędne zapisy prze eśl.}	\\	\mbox{6. Pamiętaj, $\dot{\mathrm{z}}\mathrm{e}$ zapisy w brudnopisie nie podlegają ocenie.}	\\	\mbox{7. $\mathrm{M}\mathrm{o}\dot{\mathrm{z}}$ esz korzystać z zestawu wzorów matematycznych, cyrkla}	\\	\mbox{i linijki oraz kalkulatora.}	\\	\mbox{8. Wypełnij tę część ka $\mathrm{y}$ odpowiedzi, którą koduje zdający.}	\\	\mbox{Nie wpisuj $\dot{\mathrm{z}}$ adnych znaków w części przeznaczonej dla}	\\	\mbox{egzaminatora.}	\\	\mbox{9. Na karcie odpowiedzi wpisz swoją datę urodzenia i PESEL.}	\\	\mbox{Zamaluj $\blacksquare$ pola odpowiadające cyfrom numeru PESEL. Błędne}	\\	\mbox{zaznaczenie otocz kółkiem $\mathrm{O}$ i zaznacz właściwe.}	\\	\mbox{{\it Zyczymy} $p\theta wodzenia'$}	\end{array}$}&	\multicolumn{1}{|l}{$\begin{array}{l}\mbox{LISTOPAD}	\\	\mbox{ROK 2006}	\\	\mbox{Za rozwiązanie}	\\	\mbox{wszystkich zadań}	\\	\mbox{mozna otrzymać}	\\	\mbox{łącznie}	\\	\mbox{50 punktów}	\end{array}$}	\\
\hline
\multicolumn{1}{l|}{$\begin{array}{l}\mbox{Wypelnia zdający przed}	\\	\mbox{roz oczęciem racy}	\\	\mbox{PESEL ZDAJACEGO}	\end{array}$}&	\multicolumn{1}{|l}{$\begin{array}{l}\mbox{KOD}	\\	\mbox{ZDAJACEGO}	\end{array}$}
\end{tabular}


\includegraphics[width=21.840mm,height=9.852mm]{./F1_M_PP_L2006_page0_images/image001.eps}

\includegraphics[width=78.792mm,height=13.356mm]{./F1_M_PP_L2006_page0_images/image002.eps}
\end{center}



{\it 2}

{\it Próbny egzamin maturalny z matematyki}

{\it Poziom podstawowy}

Zadanie 1. (3pkt)

Wzrost kursu euro w stosunku do złotego spowodował podwyzkę ceny wycieczki

zagranicznej o 5\%. Poniewaz nowa cena nie była zachęcająca, postanowiono obnizyć ją

0 8\%, ustalając cenę promocyjną równą l449 zł. Oblicz pierwotną cenę wycieczki dla

jednego uczestnika.





{\it Próbny egzamin maturalny z matematyki}

{\it Poziom podstawowy}

{\it 11}

Zadanie 9. $(4pkt)$

Nauczyciele informatyki, chcąc wyłonić reprezentację szkoły na wojewódzki konkurs

informatyczny, przeprowadzili w klasach I A i I $\mathrm{B}$ test z zakresu poznanych wiadomości.

$\mathrm{K}\mathrm{a}\dot{\mathrm{z}}\mathrm{d}\mathrm{y}$ z nich przygotował zestawienie wyników swoich uczniów w innej formie.

Na podstawie analizy przedstawionych ponizej wyników obu klas:

a) oblicz średni wynik z testu $\mathrm{k}\mathrm{a}\dot{\mathrm{z}}$ dej klasy,

b) oblicz, ile procent uczniów klasy I $\mathrm{B}$ uzyskało wynik $\mathrm{w}\mathrm{y}\dot{\mathrm{z}}$ szy $\mathrm{n}\mathrm{i}\dot{\mathrm{z}}$ średni w swojej klasie,

c) podaj medianę wyników uzyskanych w klasie I A.

$\mathrm{w}\mathfrak{n}\mathrm{i}\mathrm{k}\mathrm{l}$ testu $\displaystyle \inf \mathrm{u}-\mathrm{i}\acute{\mathrm{r}}$ kl. lA
\begin{center}
\includegraphics[width=98.244mm,height=74.832mm]{./F1_M_PP_L2006_page10_images/image001.eps}
\end{center}
5

4

1

0

0 1

2 3 4 5 6 7 8

1J
\begin{center}
\begin{tabular}{|l|l|}
\hline
\multicolumn{1}{|l|}{Liczba punktów}&	\multicolumn{1}{|l|}{Liczba uczniów}	\\
\hline
\multicolumn{1}{|l|}{$0$}&	\multicolumn{1}{|l|}{ $1$}	\\
\hline
\multicolumn{1}{|l|}{ $1$}&	\multicolumn{1}{|l|}{ $2$}	\\
\hline
\multicolumn{1}{|l|}{ $2$}&	\multicolumn{1}{|l|}{ $1$}	\\
\hline
\multicolumn{1}{|l|}{ $3$}&	\multicolumn{1}{|l|}{ $2$}	\\
\hline
\multicolumn{1}{|l|}{ $4$}&	\multicolumn{1}{|l|}{ $1$}	\\
\hline
\multicolumn{1}{|l|}{ $5$}&	\multicolumn{1}{|l|}{ $2$}	\\
\hline
\multicolumn{1}{|l|}{ $6$}&	\multicolumn{1}{|l|}{ $4$}	\\
\hline
\multicolumn{1}{|l|}{ $7$}&	\multicolumn{1}{|l|}{ $4$}	\\
\hline
\multicolumn{1}{|l|}{ $8$}&	\multicolumn{1}{|l|}{ $1$}	\\
\hline
\multicolumn{1}{|l|}{ $9$}&	\multicolumn{1}{|l|}{ $2$}	\\
\hline
\multicolumn{1}{|l|}{ $10$}&	\multicolumn{1}{|l|}{ $5$}	\\
\hline
\end{tabular}

\end{center}
Wyniki testu informatycznego

uczniów kl. l B.
\begin{center}
\includegraphics[width=195.168mm,height=145.488mm]{./F1_M_PP_L2006_page10_images/image002.eps}
\end{center}




{\it 12}

{\it Próbny egzamin maturalny z matematyki}

{\it Poziom podstawowy}

Zadanie 10. $(6pkt)$

Dane są zbiory:

$A=\{x\in R:|5-x|\geq 3\}, B=\{x\in R:x^{2}-9\geq 0\} \mathrm{i} C=\displaystyle \{x\in R:\frac{x+1}{x-1}\leq 1\}.$

a) Zaznacz na osi liczbowej zbiory $A, B \mathrm{i}C.$

b) Wyznacz i zapisz za pomocą przedziału liczbowego zbiór $C\backslash (A\cap B).$
\begin{center}
\includegraphics[width=192.072mm,height=72.240mm]{./F1_M_PP_L2006_page11_images/image001.eps}
\end{center}
zbiór A
\begin{center}
\includegraphics[width=192.072mm,height=72.288mm]{./F1_M_PP_L2006_page11_images/image002.eps}
\end{center}
zbiór B
\begin{center}
\includegraphics[width=192.072mm,height=72.240mm]{./F1_M_PP_L2006_page11_images/image003.eps}
\end{center}
zbiór C





{\it Próbny egzamin maturalny z matematyki}

{\it Poziom podstawowy}

{\it 13}
\begin{center}
\includegraphics[width=195.168mm,height=290.784mm]{./F1_M_PP_L2006_page12_images/image001.eps}
\end{center}




{\it 14}

{\it Próbny egzamin maturalny z matematyki}

{\it Poziom podstawowy}

Zadanie ll. $(4pkt)$

Funkcja $f$ przyporządkowuje $\mathrm{k}\mathrm{a}\dot{\mathrm{z}}$ dej liczbie rzeczywistej $x$ z przedziału $\langle-4,-2\rangle$ połowę

kwadratu tej liczby pomniejszoną o 8.

a) Podaj wzór tej funkcji.

b) Wyznacz najmniejszą wartość funkcji $f$ w podanym przedziale.
\begin{center}
\includegraphics[width=195.168mm,height=254.412mm]{./F1_M_PP_L2006_page13_images/image001.eps}
\end{center}




{\it Próbny egzamin maturalny z matematyki}

{\it Poziom podstawowy}

{\it 15}

BRUDNOPIS





{\it Próbny egzamin maturalny z matematyki}

{\it Poziom podstawowy}

{\it 3}

Zadanie 2. $(4pkt)$

Dany jest kwadrat o boku długości $a. \mathrm{W}$ prostokącie ABCD bok $AB$ jest dwa razy dłuzszy $\mathrm{n}\mathrm{i}\dot{\mathrm{z}}$

bok kwadratu, a bok $AD$ jest o 2 cm krótszy od boku kwadratu. Po1e tego prostokąta jest

$012\mathrm{c}\mathrm{m}^{2}$ większe od pola kwadratu. Oblicz długość boku kwadratu.
\begin{center}
\includegraphics[width=195.168mm,height=266.544mm]{./F1_M_PP_L2006_page2_images/image001.eps}
\end{center}




{\it 4}

{\it Próbny egzamin maturalny z matematyki}

{\it Poziom podstawowy}

Zadanie 3. (5pkt)

Z prostokąta o szerokości 60 cm wycina się deta1e w kształcie półko1a o promieniu 60 cm.

Sposób wycinania detali ilustruje ponizszy rysunek.

Oblicz najmniejszą długość prostokąta potrzebnego do wycięcia dwóch takich detali. Wynik

zaokrąglij do pełnego centymetra.
\begin{center}
\includegraphics[width=195.168mm,height=218.136mm]{./F1_M_PP_L2006_page3_images/image001.eps}
\end{center}




{\it Próbny egzamin maturalny z matematyki}

{\it Poziom podstawowy}

{\it 5}

Zadanie 4. $(3pkt)$

Wielomian $W(x)=-2x^{4}+5x^{3}+9x^{2}-15x-9$

Wyznacz pierwiastki tego wielomianu.

jest podzielny przez

dwumian $(2x+1).$
\begin{center}
\includegraphics[width=195.168mm,height=266.544mm]{./F1_M_PP_L2006_page4_images/image001.eps}
\end{center}




{\it 6}

{\it Próbny egzamin maturalny z matematyki}

{\it Poziom podstawowy}

Zadanie 5. $(5pkt)$

Dane sąproste o równaniach $2x-y-3=0\mathrm{i}2x-3y-7=0.$

a) Zaznacz w prostokątnym układzie współrzędnych na płaszczyzínie kąt

układem nierówności 

b) Oblicz odległość punktu przecięcia się tych prostych od punktu $S=(3,-8).$

opisany
\begin{center}
\includegraphics[width=165.204mm,height=151.536mm]{./F1_M_PP_L2006_page5_images/image001.eps}
\end{center}
7 J

5

4

3

2

1

{\it x}

$-7  -5$ -$4  -3$ -$2  -1 0 1$ 2  1 2 3 4 5  7

$-1$

$-2$

$-3$

$-4$

$-5$

$-7$
\begin{center}
\includegraphics[width=195.168mm,height=97.080mm]{./F1_M_PP_L2006_page5_images/image002.eps}
\end{center}




{\it Próbny egzamin maturalny z matematyki}

{\it Poziom podstawowy}

7
\begin{center}
\includegraphics[width=195.168mm,height=290.784mm]{./F1_M_PP_L2006_page6_images/image001.eps}
\end{center}




{\it 8}

{\it Próbny egzamin maturalny z matematyki}

{\it Poziom podstawowy}

Zadanie 6. $(5pkt)$

$\mathrm{W}$ utnie znajdują się kule z kolejnymi liczbami 10, 11, 12, 13, 50, przy czym ku1

z liczbą 10 jest 10, ku1 z 1iczbą 11 jest 11 itd., a ku1 z 1iczbą $50$jest 5$0. \mathrm{Z}$ umy tej losujemy

jedną kulę. Oblicz prawdopodobieństwo, $\dot{\mathrm{z}}\mathrm{e}$ wylosujemy kulę z liczbą parzystą.





{\it Próbny egzamin maturalny z matematyki}

{\it Poziom podstawowy}

{\it 9}

Zadanie 7. $(6pkt)$

$\mathrm{W}$ graniastosłupie prawidłowym czworokątnym przekątna podstawy ma długość 8 cm

i tworzy z przekątną ściany bocznej, z którą ma wspólny wierzchołek kąt, którego cosinus

jest równy $\displaystyle \frac{2}{3}$. Oblicz objętość i pole powierzchni całkowitej tego graniastosłupa.
\begin{center}
\includegraphics[width=195.168mm,height=254.460mm]{./F1_M_PP_L2006_page8_images/image001.eps}
\end{center}




$ 1\theta$

{\it Próbny egzamin maturalny z matematyki}

{\it Poziom podstawowy}

Zadanie 8. $(5pkt)$

Dany jest wykres funkcji $y=f(x)$ określonej dla $x\in\langle-6, 6\rangle.$
\begin{center}
\begin{tabular}{|l|l|}
\hline
\multicolumn{1}{|l|}{ $\begin{array}{l}\mbox{$7$}	\\	\mbox{ $6$}	\\	\mbox{ $5$}	\\	\mbox{ $4$}	\\	\mbox{ $3$}	\\	\mbox{ $2$}	\end{array}$}&	\multicolumn{1}{|l|}{ $\mathrm{y}$}	\\
\hline
\multicolumn{1}{|l|}{ $\begin{array}{l}\mbox{-f $-8 -7 -6 -4 -3 -2$ 1}	\\	\mbox{$-1$}	\\	\mbox{ $-2$}	\\	\mbox{ $-3$}	\\	\mbox{ $-4$}	\\	\mbox{ $-5$}	\\	\mbox{ $-6$}	\\	\mbox{ $-7$}	\end{array}$}&	\multicolumn{1}{|l|}{ $2346789$}	\\
\hline
\end{tabular}


\includegraphics[width=35.508mm,height=42.576mm]{./F1_M_PP_L2006_page9_images/image001.eps}

\includegraphics[width=35.760mm,height=42.576mm]{./F1_M_PP_L2006_page9_images/image002.eps}
\end{center}
Korzystając z wykresu ffinkcji zapisz:

a) maksymalne przedziały, w których funkcjajest rosnąca,

b) zbiór argumentów, dla których ffinkcja przyjmuje wartości dodatnie,

c) największąwartość ffinkcji $f$ w przedziale $\langle-5, 5\rangle,$

d) miejsca zerowe ffinkcji $g(x)=f(x-1),$

e) najmniejszą wartość funkcji $h(x)=f(x)+2.$






\begin{center}
\includegraphics[width=7.212mm,height=14.220mm]{./F1_M_PP_L2009_page0_images/image001.eps}

\includegraphics[width=25.140mm,height=9.900mm]{./F1_M_PP_L2009_page0_images/image002.eps}
\end{center}
Centralna

Komisja

Egzaminacyjna

APiTAtL z l

NA O OWAST A Eclk l

Materiał współfmansowany ze środków Unii Europejskiej

w ramach Europejskiego Funduszu Społecznego

$\displaystyle \mathrm{F}\cup \mathrm{N}\mathrm{O}\cup \mathrm{s}\mathrm{z}\mathrm{o}r\mathrm{s}_{\mathrm{n}}\bigcup_{\mathrm{S}\mathrm{P}}^{\mathrm{N}1\mathrm{A}\mathrm{E}\cup}\mathrm{R}\mathrm{O}_{\varsigma \mathrm{z}\mathrm{N}?}$
\begin{center}
\includegraphics[width=20.772mm,height=13.920mm]{./F1_M_PP_L2009_page0_images/image003.eps}

\begin{tabular}{|l|l|l}
\cline{1-1}
\multicolumn{1}{|l|}{$\begin{array}{l}\mbox{Miejsce}	\\	\mbox{na na ejkę}	\end{array}$}&	\multicolumn{1}{|l|}{$\begin{array}{l}\mbox{{\it ARKUSZ ZA WIE}}	\\	\mbox{{\it INFO ACJE}}	\\	\mbox{{\it P WNIE CHRONIONE}}	\\	\mbox{{\it DO MOMENTU}}	\\	\mbox{{\it ROZPOCZĘCIA}}	\\	\mbox{{\it EGZAMINU}.'}	\end{array}$}&	\multicolumn{1}{|l}{ $\mathrm{M}\mathrm{M}\mathrm{A}-\mathrm{P}1_{-}1\mathrm{P}-095$}	\\
\hline
&	\multicolumn{1}{|l}{$\begin{array}{l}\mbox{LISTOPAD}	\\	\mbox{ROK 2009}	\\	\mbox{Za rozwiązanie}	\\	\mbox{wszystkich zadań}	\\	\mbox{mozna otrzymać}	\\	\mbox{łącznie}	\\	\mbox{50 punktów}	\end{array}$}	\\
\cline{3-3}
&	\multicolumn{1}{|l}{$\begin{array}{l}\mbox{KOD}	\\	\mbox{ZDAJACEGO}	\end{array}$}
\end{tabular}


\includegraphics[width=21.840mm,height=9.852mm]{./F1_M_PP_L2009_page0_images/image004.eps}

\includegraphics[width=78.744mm,height=13.308mm]{./F1_M_PP_L2009_page0_images/image005.eps}
\end{center}



{\it 2}

{\it Próbny egzamin maturalny z matematyki}

{\it Poziom podstawowy}

ZADANIA ZAMKNIĘTE

$W$ {\it zadaniach} $\theta d1. d_{\theta}25$. {\it wybierz i zaznacz na karcie} $\theta dp\theta${\it wiedzijednq}

{\it poprawnq odpowied} $\acute{z}.$

Zadanie l. $(1pkt)$

Wskaz nierówność, która opisuje sumę przedziałów zaznaczonych na osi liczbowej.
\begin{center}
\includegraphics[width=174.852mm,height=13.416mm]{./F1_M_PP_L2009_page1_images/image001.eps}
\end{center}
$-2$  6  {\it x}

A. $|x-2|>4$

B. $|x-2|<4$

C. $|x-4|<2$

D. $|x-4|>2$

Zadanie 2. (1pkt)

Na seans filmowy sprzedano 280 bi1etów, w tym 126 u1gowych. Jaki procent sprzedanych

biletów stanowiły bilety ulgowe?

A. 22\%

B. 33\%

Zadanie 3. (1pkt)

6\% 1iczby x jest równe 9. Wtedy

A. $x=240$

B. $x=150$

Zadanie 4. $(1pkt)$

Iloraz $32^{-3}$ : $(\displaystyle \frac{1}{8})^{4}$ jest równy

A. $2^{-27}$ B. $2^{-3}$

Zadanie 5. $(1pkt)$

$\mathrm{O}$ liczbie $x$ wiadomo, $\dot{\mathrm{z}}\mathrm{e}\log_{3}x=9$. Zatem

A. {\it x}$=$2 B. {\it x}$=- 21$

Zadanie 6. $(1pkt)$

Wyrazenie $27x^{3}+y^{3}$ jest równe iloczynowi

A.

B.

C.

D.

$(3x+y)(9x^{2}-3xy+y^{2})$

$(3x+y)(9x^{2}+3xy+y^{2})$

$(3x-y)(9x^{2}+3xy+y^{2})$

$(3x-y)(9x^{2}-3xy+y^{2})$

C. 45\%

D. 63\%

C. $x=24$

D. $x=15$

C. $2^{3}$

D. $2^{27}$

C. $x=3^{9}$

D. $x=9^{3}$

Zadanie 7. $(1pkt)$

Dane sąwielomiany: $W(x)=x^{3}-3x+1$ oraz $V(x)=2x^{3}$. Wielomian $W(x)\cdot\nabla(x)$ jest równy

A. $2x^{5}-6x^{4}+2x^{3}$

B. $2x^{6}-6x^{4}+2x^{3}$

C. $2x^{5}+3x+1$

D. $2x^{5}+6x^{4}+2x^{3}$





{\it Próbny egzamin maturalny z matematyki}

{\it Poziom podstawowy}

{\it 11}

Zadanie 28. $(2pkt)$

$\mathrm{W}$ układzie współrzędnych na płaszczyzínie punkty $A=(2,5)$ i $\mathrm{C}=(6,7)$ są przeciwległymi

wierzchołkami kwadratu ABCD. Wyznacz równanie prostej $BD.$

Odpowied $\acute{\mathrm{z}}$:

Zadanie 29. $(2pkt)$

Kąt $a$ jest ostry i $\displaystyle \mathrm{t}\mathrm{g}\alpha=\frac{4}{3}$. Oblicz $\sin\alpha+\cos\alpha.$

Odpowiedzí :





{\it 12}

{\it Próbny egzamin maturalny z matematyki}

{\it Poziom podstawowy}

Zadanie 30. $(2pkt)$

Wykaz, $\dot{\mathrm{z}}\mathrm{e}$ dla $\mathrm{k}\mathrm{a}\dot{\mathrm{z}}$ dego $m$ ciąg $(\displaystyle \frac{m+1}{4},\frac{m+3}{6},\frac{m+9}{12})$ jest arytmetyczny.





{\it Próbny egzamin maturalny z matematyki}

{\it Poziom podstawowy}

{\it 13}

Zadanie 31. $(2pkt)$

Trójkąty $ABC\mathrm{i}CDE$ są równoboczne. Punkty $A, C\mathrm{i}E$ lez$\cdot$ą najednej prostej. Punkty $K, L\mathrm{i}M$

są środkami odcinków $AC$, {\it CE} $\mathrm{i} BD$ (zobacz rysunek). Wykaz, $\dot{\mathrm{z}}\mathrm{e}$ punkty $K, L \mathrm{i} M$

są wierzchołkami trójkąta równobocznego.
\begin{center}
\includegraphics[width=116.640mm,height=65.124mm]{./F1_M_PP_L2009_page12_images/image001.eps}
\end{center}
{\it D}

{\it M}

{\it B}

{\it A  E}

{\it K C  L}





{\it 14}

{\it Próbny egzamin maturalny z matematyki}

{\it Poziom podstawowy}

Zadanie 32. $(5pkt)$

Uczeń przeczytał ksiązkę liczącą480 stron, przy czym $\mathrm{k}\mathrm{a}\dot{\mathrm{z}}$ dego dnia czytał jednakową liczbę

stron. Gdyby czytał $\mathrm{k}\mathrm{a}\dot{\mathrm{z}}$ dego dnia o 8 stron więcej, to przeczytałby tę ksiązkę o 3 dni

wcześniej. Oblicz, ile dni uczeń czytał tę ksiązkę.

Odpowiedzí:





{\it Próbny egzamin maturalny z matematyki}

{\it Poziom podstawowy}

{\it 15}

Zadanie 33. $(4pkt)$

Punkty $A=(2,0) \mathrm{i} B=(12,0)$ są wierzchołkami trójkąta prostokątnego $ABC$

o przeciwprostokątnej $AB$. Wierzchołek $C$ lezy na prostej o równaniu $y=x$. Oblicz

współrzędne punktu $C.$

Odpowiedzí :





{\it 16}

{\it Próbny egzamin maturalny z matematyki}

{\it Poziom podstawowy}

Zadanie 34. $(4pkt)$

Pole trójkąta prostokątnego jest równe 60 $\mathrm{c}\mathrm{m}^{2}$ Jedna przyprostokątna jest o 7 cm diuzsza

od drugiej. Oblicz długość przeciwprostokątnej tego trójkąta.

Odpowiedzí:





{\it Próbny egzamin maturalny z matematyki}

{\it Poziom podstawowy}

{\it 1}7

BRUDNOPIS





{\it Próbny egzamin maturalny z matematyki}

{\it Poziom podstawowy}

{\it 3}

BRUDNOPIS





{\it 4}

{\it Próbny egzamin maturalny z matematyki}

{\it Poziom podstawowy}

Zadanie 8. $(1pkt)$

Wierzchołek paraboli o równaniu $y=-3(x+1)^{2}$ ma współrzędne

A. $(-1,0)$ B. $(0,-1)$ C. $($1, $0)$

D. (0,1)

Zadanie 9. $(1pkt)$

Do wykresu funkcji $f(x)=x^{2}+x-2$ nalezy punkt

A. $(-1,-4)$

B. $(-1,1)$

C. $(-1,-1)$

D. $(-1,-2)$

Zadanie 10. $(1pkt)$

Rozwiązaniem równania $\displaystyle \frac{x-5}{x+3}=\frac{2}{3}$ jest liczba

A. 21 B. 7

C.

$\displaystyle \frac{17}{3}$

D. 0

Zadanie ll. $(1pkt)$

Zbiór rozwiązań nierównoŚci $(x+1)(x-3)>0$ przedstawionyjest na rysunku
\begin{center}
\includegraphics[width=170.988mm,height=15.804mm]{./F1_M_PP_L2009_page3_images/image001.eps}
\end{center}
$-1$  3  {\it x}

A.
\begin{center}
\includegraphics[width=171.756mm,height=13.716mm]{./F1_M_PP_L2009_page3_images/image002.eps}
\end{center}
{\it x}

1

$-3$

B.
\begin{center}
\includegraphics[width=171.048mm,height=15.852mm]{./F1_M_PP_L2009_page3_images/image003.eps}
\end{center}
$-1$  3  {\it x}

C.
\begin{center}
\includegraphics[width=171.756mm,height=13.716mm]{./F1_M_PP_L2009_page3_images/image004.eps}
\end{center}
{\it x}

1

$-3$

D.

Zadanie 12. $(1pkt)$

Dla $ n=1,2,3,\ldots$ ciąg $(a_{n})$ jest określony wzorem: $a_{n}=(-1)^{n}\cdot(3-n)$. Wtedy

A. $a_{3}<0$

B. $a_{3}=0$

C. $a_{3}=1$

D. $a_{3}>1$

Zadanie 13. (1pkt)

W ciągu arytmetycznym trzeci wyraz jest równy 14, ajedenasty jest równy 34. Róznica tego

ciągu jest równa

A. 9 B. -25 C. 2 D. -25

Zadanie 14. $(1pkt)$

$\mathrm{W}$ ciągu geometrycznym $(a_{n})$ dane są: $a_{1}=32 \mathrm{i}a_{4}=-4$. Iloraz tego ciągujest równy

A. 12 B. $\displaystyle \frac{1}{2}$ C. $-\displaystyle \frac{1}{2}$ D. $-12$





{\it Próbny egzamin maturalny z matematyki}

{\it Poziom podstawowy}

{\it 5}

BRUDNOPIS





{\it 6}

{\it Próbny egzamin maturalny z matematyki}

{\it Poziom podstawowy}

Zadanie 15. $(1pkt)$

Kąt $\alpha$ jest ostry i $\displaystyle \sin\alpha=\frac{8}{9}$. Wtedy $\cos\alpha$ jest równy

A. -91 B. -98 C. --$\sqrt{}$917

D.

$\displaystyle \frac{\sqrt{65}}{9}$

Zadanie 16. $(1pkt)$

Danyjest trójkąt prostokątny (patrz rysunek). Wtedy tg $\alpha$ jest równy
\begin{center}
\includegraphics[width=57.660mm,height=36.420mm]{./F1_M_PP_L2009_page5_images/image001.eps}
\end{center}
$\sqrt{3}$

1

$\alpha$

A. $\sqrt{2}$

B.

$\sqrt{2}$

$\sqrt{3}$

$\sqrt{2}$

C.

$\sqrt{3}$

$\sqrt{2}$

D.

-$\sqrt{}$12

Zadanie 17. (1pkt)

W trójkącie równoramiennym ABC dane są

opuszczona z wierzchołka C jest równa

$|AC|=|BC|=7$

oraz

$|AB|=12.$

Wysokość

A. $\sqrt{13}$

B. $\sqrt{5}$

C. l

D. 5

Zadanie 18. $(1pkt)$

Oblicz $\mathrm{d}$ gość odcinka $AE$ wiedząc, $\dot{\mathrm{z}}\mathrm{e}AB||CD \mathrm{i} AB=6, AC=4, CD=8.$
\begin{center}
\includegraphics[width=102.108mm,height=50.040mm]{./F1_M_PP_L2009_page5_images/image002.eps}
\end{center}
{\it D}

{\it B}

8

6

{\it E  A}  4  {\it C}

A.

$|AE|=2$

B.

$|AE|=4$

C.

$|AE|=6$

D.

$|AE|=12$

Zadanie 19. $(1pkt)$

Dane sąpunkty $A=(-2,3)$ oraz $B=(4,6)$. Długość odcinka $AB$ jest równa

A. $\sqrt{208}$

B. $\sqrt{52}$

C. $\sqrt{45}$

D. $\sqrt{40}$

Zadanie 20. $(1pkt)$

Promień okręgu o równaniu $(x-1)^{2}+y^{2}=16$ jest równy

A. l

B. 2

C. 3

D. 4





{\it Próbny egzamin maturalny z matematyki}

{\it Poziom podstawowy}

7

BRUDNOPIS





{\it 8}

{\it Próbny egzamin maturalny z matematyki}

{\it Poziom podstawowy}

Zadanie 21. $(1pkt)$

Wykres ffinkcji liniowej określonej wzorem $f(x)=3x+2$ jest prostą prostopadłą do prostej

o równaniu:

A. $y=-\displaystyle \frac{1}{3}x-1$ B. $y=\displaystyle \frac{1}{3}x+1$ C. $y=3x+1$ D. $y=3x-1$

Zadanie 22. $(1pkt)$

Prosta o równaniu $y=-4x+(2m-7)$ przechodzi przez punkt $A=(2,-1)$. Wtedy

A. $m=7$

B.

{\it m}$=$2 -21

C.

{\it m}$=$ - -21

D. $m=-17$

Zadanie 23. $(1pkt)$

Pole powierzchni całkowitej sześcianu jest równe 150 $\mathrm{c}\mathrm{m}^{2}$ Długość krawędzi tego sześcianu

jest równa

A. 3,5 cm

B. 4 cm

C. 4,5 cm

D. 5 cm

Zadanie 24. (1pkt)

Średnia arytmetyczna pięciu liczb: 5, x, 1, 3, 1 jest równa 3. Wtedy

A. $x=2$

B. $x=3$

C. $x=4$

D. $x=5$

Zadanie 25. $(1pkt)$

Wybieramy liczbę $a$ ze zbioru $A=\{2,3,4,5\}$ oraz liczbę $b$ ze zbioru $B=\{1,4\}$. Ilejest takich par

$(a,b), \dot{\mathrm{z}}\mathrm{e}$ iloczyn $a\cdot b$ jest liczbą nieparzystą?

A. 2

B. 3

C. 5

D. 20





{\it Próbny egzamin maturalny z matematyki}

{\it Poziom podstawowy}

{\it 9}

BRUDNOPIS





$ 1\theta$

{\it Próbny egzamin maturalny z matematyki}

{\it Poziom podstawowy}

ZADANIA OTWARTE

{\it Rozwiqzania zadań o numerach od 26. do 34. nalezy zapisać w} $wyznacz\theta nych$ {\it miejscach}

{\it pod treściq zadania}.

Zadanie 26. $(2pkt)$

Rozwiąz nierówność $x^{2}-3x+2\leq 0.$

Odpowiedzí:

Zadanie 27. $(2pkt)$

Rozwiąz równanie $x^{3}-7x^{2}+2x-14=0.$

Odpowied $\acute{\mathrm{z}}$:







$\mathrm{g}$ NARODOWASTRATECIASPóJNOS$\subseteq$lKAPITALL$\cup$DZKl Centralna Komisja Egzaminacyjna $\mathrm{F}\cup \mathrm{N}\mathrm{D}\cup \mathrm{s}\mathrm{z}\mathrm{S}\mathrm{P}\mathrm{O}\mathrm{L}\mathrm{E}\mathrm{C}\mathrm{Z}\mathrm{N}\mathrm{Y}\cup \mathrm{N}\mathrm{l}\mathrm{A}\mathrm{E}\cup \mathrm{R}\mathrm{O}\mathrm{p}\mathrm{E}\mathrm{J}\mathrm{S}\mathrm{K}\mathrm{A}\mathrm{E}\cup \mathrm{R}\mathrm{O}\mathrm{P}\mathrm{E}\rfloor 5\mathrm{K}\mathrm{l}$\fbox{}

Materiał współfinansowany ze środków Unii Europejskiej w ramach Europejskiego Funduszu Społecznego.

Arkusz zawiera informacje prawnie chronione do momentu rozpoczęcia egzaminu.

WPISUJE ZDAJACY

KOD PESEL

{\it Miejsce}

{\it na naklejkę}

{\it z kodem}
\begin{center}
\includegraphics[width=21.432mm,height=9.804mm]{./F1_M_PP_L2010_page0_images/image001.eps}

\includegraphics[width=82.092mm,height=9.804mm]{./F1_M_PP_L2010_page0_images/image002.eps}

\includegraphics[width=204.012mm,height=216.048mm]{./F1_M_PP_L2010_page0_images/image003.eps}
\end{center}
PRÓBNY EGZAMIN MATU

Z MATEMATY

LNY

POZIOM PODSTAWOWY  LISTOPAD 2010

1.

2.

3.

Sprawdzí, czy arkusz egzaminacyjny zawiera 19 stron

(zadania $1-34$). Ewentualny brak zgłoś przewodniczącemu

zespo nadzorującego egzamin.

Rozwiązania zadań i odpowiedzi wpisuj w miejscu na to

przeznaczonym.

Odpowiedzi do zadań zamkniętych (1-25) przenieś

na ka ę odpowiedzi, zaznaczając je w części ka $\mathrm{y}$

przeznaczonej dla zdającego. Zamaluj $\blacksquare$ pola do tego

przeznaczone. Błędne zaznaczenie otocz kółkiem

i zaznacz właściwe.

4. Pamiętaj, $\dot{\mathrm{z}}\mathrm{e}$ pominięcie argumentacji lub istotnych

obliczeń w rozwiązaniu zadania otwa ego (26-34) $\mathrm{m}\mathrm{o}\dot{\mathrm{z}}\mathrm{e}$

spowodować, $\dot{\mathrm{z}}\mathrm{e}$ za to rozwiązanie nie będziesz mógł

dostać pełnej liczby punktów.

5. Pisz cz elnie i $\mathrm{u}\dot{\mathrm{z}}$ aj tvlko długopisu lub -Dióra

z czarnym tuszem lub atramentem.

6. Nie $\mathrm{u}\dot{\mathrm{z}}$ aj korektora, a błędne zapisy wyrazínie prze eśl.

7. Pamiętaj, $\dot{\mathrm{z}}\mathrm{e}$ zapisy w brudnopisie nie będą oceniane.

8. $\mathrm{M}\mathrm{o}\dot{\mathrm{z}}$ esz korzystać z zestawu wzorów matematycznych,

cyrkla i linijki oraz kalkulatora.

9. Na karcie odpowiedzi wpisz i zakoduj swój numer

PESEL.

10. Nie wpisuj $\dot{\mathrm{z}}$ adnych znaków w części przeznaczonej dla

egzaminatora.

Czas pracy:

170 minut

Liczba punktów

do uzyskania: 50

$\Vert\Vert\Vert\Vert\Vert\Vert\Vert\Vert\Vert\Vert\Vert\Vert\Vert\Vert\Vert\Vert\Vert\Vert\Vert\Vert\Vert\Vert\Vert\Vert|  \mathrm{M}\mathrm{M}\mathrm{A}-\mathrm{P}1_{-}1\mathrm{P}-105$




{\it 2}

{\it Próbny egzamin maturalny z matematyki}

{\it Poziom podstawowy}

ZADANIA ZAMKNIĘTE

{\it Wzadaniach od l. do 25. wybierz i zaznacz na karcie odpowiedzijednq}

{\it poprawnq odpowied} $\acute{z}.$

Zadanie l. $(1pkt)$

Liczba $|5-7|-|-3+4|$ jest równa

A. $-3$ B. $-5$

C. l

D. 3

Zadanie 2. $(1pkt)$

Wskaz rysunek, na którym jest przedstawiony zbiór rozwiązań nierówności $|x-2|\geq 3.$
\begin{center}
\includegraphics[width=173.280mm,height=14.532mm]{./F1_M_PP_L2010_page1_images/image001.eps}
\end{center}
$-1$  5  {\it x}

A.
\begin{center}
\includegraphics[width=172.716mm,height=15.648mm]{./F1_M_PP_L2010_page1_images/image002.eps}
\end{center}
$-1$  5  {\it x}

B.
\begin{center}
\includegraphics[width=171.756mm,height=13.104mm]{./F1_M_PP_L2010_page1_images/image003.eps}
\end{center}
3  {\it x}

C.
\begin{center}
\includegraphics[width=172.716mm,height=15.648mm]{./F1_M_PP_L2010_page1_images/image004.eps}
\end{center}
5  {\it x}

D.

Zadanie 3. (1pkt)

Samochód kosztował 30000 zł. Jego cenę obnizono o 10\%, a następnie cenę po tej obnizce

ponownie obnizono o 10\%. Po tych obnizkach samochód kosztował

A. 24400 zł

B. 24700 zł

C. 24000 zł

D. 24300 zł

Zadanie 4. $(1pkt)$

Danajest liczba $x=63^{2}\displaystyle \cdot(\frac{1}{3})^{4}$. Wtedy

A. $x=7^{2}$ B. $x=7^{-2}$

C. $x=3^{8}\cdot 7^{2}$

D. $x=3\cdot 7$

Zadanie 5. $(1pkt)$

Kwadrat liczby $x=5+2\sqrt{3}$ jest równy

A. 37 B. $25+4\sqrt{3}$

C. $37+20\sqrt{3}$

D. 147

Zadanie 6. $(1pkt)$

Liczba $\log_{5}5-\log_{5}125$ jest równa

A. $-2$ B. $-1$

C.

$\displaystyle \frac{1}{25}$

D. 4





{\it Próbny egzamin maturalny z matematyki}

{\it Poziom podstawowy}

{\it 11}

BRUDNOPIS





{\it 12}

{\it Próbny egzamin maturalny z matematyki}

{\it Poziom podstawowy}

ZADANIA OTWARTE

{\it Rozwiqzania zadań o numerach od 26. do 34. nalezy zapisać w wyznaczonych miejscach}

{\it pod treściq zadania}.

Zadanie 26. $(2pkt)$

Rozwiąz nierówność $x^{2}+11x+30\leq 0.$

Odpowiedzí:

Zadanie 27. $(2pkt)$

Rozwiąz równanie $x^{3}+2x^{2}-5x-10=0.$

Odpowiedzí:





{\it Próbny egzamin maturalny z matematyki}

{\it Poziom podstawowy}

{\it 13}

Zadanie 28. (2pkt)

Przeciwprostokątna trójkąta prostokątnego jest dłuzsza od jednej przyprostokątnej o l cm

i od drugiej przyprostokątnej o 32 cm. Ob1icz długości boków tego trójkąta.

Odpowiedzí:





{\it 14}

{\it Próbny egzamin maturalny z matematyki}

{\it Poziom podstawowy}

Zadanie 29. (2pkt)

Dany jest prostokąt ABCD. Okręgi o średnicach AB $\mathrm{i}$ AD przecinają się w punktach $A\mathrm{i}P$

(zobacz rysunek). Wykaz, $\dot{\mathrm{z}}\mathrm{e}$ punkty $B, P\mathrm{i}D$ lez$\cdot$ą najednej prostej.
\begin{center}
\includegraphics[width=81.228mm,height=71.628mm]{./F1_M_PP_L2010_page13_images/image001.eps}
\end{center}
{\it D  C}

{\it P}

{\it A  B}





{\it Próbny egzamin maturalny z matematyki}

{\it Poziom podstawowy}

{\it 15}

Zadanie 30. $(2pkt)$

Uzasadnij, $\dot{\mathrm{z}}$ ejeśli $(a^{2}+b^{2})(c^{2}+d^{2})=(ac+bd)^{2}$, to {\it ad}$=bc.$

Zadanie 31. (2pkt)

Oblicz, ile jest liczb naturalnych czterocyfrowych, w których zapisie pierwsza cyfra jest

parzysta, a pozostałe nieparzyste.

Odpowiedzí:





{\it 16}

{\it Próbny egzamin maturalny z matematyki}

{\it Poziom podstawowy}

Zadanie 32. $(4pkt)$

Ciąg $(1,x,y-1)$ jest arytmetyczny, natomiast

Oblicz $x$ oraz $y$ i podaj ten ciąg geometryczny.

ciąg (x, y, 12)

jest geometryczny.

Odpowiedzí:





{\it Próbny egzamin maturalny z matematyki}

{\it Poziom podstawowy}

{\it 1}7

Zadanie 33. $(4pkt)$

Punkty $A=(1,5), B=(14,31), C=(4,31)$ są wierzchołkami trójkąta. Prosta zawierająca

wysokość tego trójkąta poprowadzona z wierzchołka $C$ przecina prostą AB w punkcie $D.$

Oblicz długość odcinka $BD.$

Odpowiedzí:





{\it 18}

{\it Próbny egzamin maturalny z matematyki}

{\it Poziom podstawowy}

Zadanie 34. $(5pkt)$

Droga z miasta A do miasta $\mathrm{B}$ ma długość 474 km. Samochódjadący z miasta A do miasta $\mathrm{B}$

wyrusza godzinę pózíniej $\mathrm{n}\mathrm{i}\dot{\mathrm{z}}$ samochód z miasta $\mathrm{B}$ do miasta A. Samochody te spotykają się

w odległości 300 km od miasta B. Średnia prędkość samochodu, który wyjechał z miasta $\mathrm{A},$

liczona od chwili wyjazdu z A do momentu spotkania, była o 17 $\mathrm{k}\mathrm{m}/\mathrm{h}$ mniejsza od średniej

prędkości drugiego samochodu liczonej od chwili wyjazdu z $\mathrm{B}$ do chwili spotkania. Oblicz

średniąprędkość $\mathrm{k}\mathrm{a}\dot{\mathrm{z}}$ dego samochodu do chwili spotkania.

Odpowiedzí:





{\it Próbny egzamin maturalny z matematyki}

{\it Poziom podstawowy}

{\it 19}

BRUDNOPIS





{\it Próbny egzamin maturalny z matematyki}

{\it Poziom podstawowy}

{\it 3}

BRUDNOPIS





{\it 4}

{\it Próbny egzamin maturalny z matematyki}

{\it Poziom podstawowy}

{\it W zadaniach 7, 8 i9 wykorzystaj przedstawiony ponizej wykres funkcji f}
\begin{center}
\begin{tabular}{|l|l|l|l|l|l|l|l|l|l|l|l|l|l|l|l|l|l|}
\hline
\multicolumn{1}{|l|}{}&	\multicolumn{1}{|l|}{}&	\multicolumn{1}{|l|}{}&	\multicolumn{1}{|l|}{}&	\multicolumn{1}{|l|}{}&	\multicolumn{1}{|l|}{}&	\multicolumn{1}{|l|}{}&	\multicolumn{1}{|l|}{$y$}&	\multicolumn{1}{|l|}{}&	\multicolumn{1}{|l|}{}&	\multicolumn{1}{|l|}{}&	\multicolumn{1}{|l|}{}&	\multicolumn{1}{|l|}{}&	\multicolumn{1}{|l|}{}&	\multicolumn{1}{|l|}{}&	\multicolumn{1}{|l|}{}&	\multicolumn{1}{|l|}{}&	\multicolumn{1}{|l|}{}	\\
\hline
\multicolumn{1}{|l|}{}&	\multicolumn{1}{|l|}{}&	\multicolumn{1}{|l|}{}&	\multicolumn{1}{|l|}{}&	\multicolumn{1}{|l|}{}&	\multicolumn{1}{|l|}{}&	\multicolumn{1}{|l|}{}&	\multicolumn{1}{|l|}{}&	\multicolumn{1}{|l|}{}&	\multicolumn{1}{|l|}{}&	\multicolumn{1}{|l|}{}&	\multicolumn{1}{|l|}{}&	\multicolumn{1}{|l|}{}&	\multicolumn{1}{|l|}{}&	\multicolumn{1}{|l|}{}&	\multicolumn{1}{|l|}{}&	\multicolumn{1}{|l|}{}&	\multicolumn{1}{|l|}{}	\\
\hline
\multicolumn{1}{|l|}{}&	\multicolumn{1}{|l|}{}&	\multicolumn{1}{|l|}{}&	\multicolumn{1}{|l|}{}&	\multicolumn{1}{|l|}{}&	\multicolumn{1}{|l|}{}&	\multicolumn{1}{|l|}{}&	\multicolumn{1}{|l|}{}&	\multicolumn{1}{|l|}{}&	\multicolumn{1}{|l|}{}&	\multicolumn{1}{|l|}{}&	\multicolumn{1}{|l|}{}&	\multicolumn{1}{|l|}{}&	\multicolumn{1}{|l|}{}&	\multicolumn{1}{|l|}{}&	\multicolumn{1}{|l|}{}&	\multicolumn{1}{|l|}{}&	\multicolumn{1}{|l|}{}	\\
\hline
\multicolumn{1}{|l|}{}&	\multicolumn{1}{|l|}{}&	\multicolumn{1}{|l|}{}&	\multicolumn{1}{|l|}{}&	\multicolumn{1}{|l|}{}&	\multicolumn{1}{|l|}{}&	\multicolumn{1}{|l|}{}&	\multicolumn{1}{|l|}{}&	\multicolumn{1}{|l|}{}&	\multicolumn{1}{|l|}{}&	\multicolumn{1}{|l|}{}&	\multicolumn{1}{|l|}{}&	\multicolumn{1}{|l|}{}&	\multicolumn{1}{|l|}{}&	\multicolumn{1}{|l|}{}&	\multicolumn{1}{|l|}{}&	\multicolumn{1}{|l|}{}&	\multicolumn{1}{|l|}{}	\\
\hline
\multicolumn{1}{|l|}{}&	\multicolumn{1}{|l|}{}&	\multicolumn{1}{|l|}{}&	\multicolumn{1}{|l|}{}&	\multicolumn{1}{|l|}{}&	\multicolumn{1}{|l|}{}&	\multicolumn{1}{|l|}{}&	\multicolumn{1}{|l|}{}&	\multicolumn{1}{|l|}{}&	\multicolumn{1}{|l|}{}&	\multicolumn{1}{|l|}{}&	\multicolumn{1}{|l|}{}&	\multicolumn{1}{|l|}{}&	\multicolumn{1}{|l|}{}&	\multicolumn{1}{|l|}{}&	\multicolumn{1}{|l|}{}&	\multicolumn{1}{|l|}{}&	\multicolumn{1}{|l|}{}	\\
\hline
\multicolumn{1}{|l|}{}&	\multicolumn{1}{|l|}{}&	\multicolumn{1}{|l|}{}&	\multicolumn{1}{|l|}{}&	\multicolumn{1}{|l|}{}&	\multicolumn{1}{|l|}{}&	\multicolumn{1}{|l|}{}&	\multicolumn{1}{|l|}{}&	\multicolumn{1}{|l|}{}&	\multicolumn{1}{|l|}{}&	\multicolumn{1}{|l|}{}&	\multicolumn{1}{|l|}{}&	\multicolumn{1}{|l|}{}&	\multicolumn{1}{|l|}{}&	\multicolumn{1}{|l|}{}&	\multicolumn{1}{|l|}{}&	\multicolumn{1}{|l|}{}&	\multicolumn{1}{|l|}{}	\\
\hline
\multicolumn{1}{|l|}{}&	\multicolumn{1}{|l|}{}&	\multicolumn{1}{|l|}{}&	\multicolumn{1}{|l|}{}&	\multicolumn{1}{|l|}{}&	\multicolumn{1}{|l|}{}&	\multicolumn{1}{|l|}{}&	\multicolumn{1}{|l|}{}&	\multicolumn{1}{|l|}{}&	\multicolumn{1}{|l|}{}&	\multicolumn{1}{|l|}{}&	\multicolumn{1}{|l|}{}&	\multicolumn{1}{|l|}{}&	\multicolumn{1}{|l|}{}&	\multicolumn{1}{|l|}{}&	\multicolumn{1}{|l|}{}&	\multicolumn{1}{|l|}{}&	\multicolumn{1}{|l|}{ $x$}	\\
\hline
\multicolumn{1}{|l|}{}&	\multicolumn{1}{|l|}{}&	\multicolumn{1}{|l|}{}&	\multicolumn{1}{|l|}{}&	\multicolumn{1}{|l|}{}&	\multicolumn{1}{|l|}{}&	\multicolumn{1}{|l|}{}&	\multicolumn{1}{|l|}{}&	\multicolumn{1}{|l|}{}&	\multicolumn{1}{|l|}{}&	\multicolumn{1}{|l|}{}&	\multicolumn{1}{|l|}{}&	\multicolumn{1}{|l|}{}&	\multicolumn{1}{|l|}{}&	\multicolumn{1}{|l|}{}&	\multicolumn{1}{|l|}{}&	\multicolumn{1}{|l|}{ $1$}&	\multicolumn{1}{|l|}{}	\\
\hline
\multicolumn{1}{|l|}{}&	\multicolumn{1}{|l|}{}&	\multicolumn{1}{|l|}{}&	\multicolumn{1}{|l|}{}&	\multicolumn{1}{|l|}{}&	\multicolumn{1}{|l|}{}&	\multicolumn{1}{|l|}{}&	\multicolumn{1}{|l|}{}&	\multicolumn{1}{|l|}{}&	\multicolumn{1}{|l|}{}&	\multicolumn{1}{|l|}{}&	\multicolumn{1}{|l|}{}&	\multicolumn{1}{|l|}{}&	\multicolumn{1}{|l|}{}&	\multicolumn{1}{|l|}{}&	\multicolumn{1}{|l|}{}&	\multicolumn{1}{|l|}{}&	\multicolumn{1}{|l|}{}	\\
\hline
\multicolumn{1}{|l|}{}&	\multicolumn{1}{|l|}{}&	\multicolumn{1}{|l|}{}&	\multicolumn{1}{|l|}{}&	\multicolumn{1}{|l|}{}&	\multicolumn{1}{|l|}{}&	\multicolumn{1}{|l|}{}&	\multicolumn{1}{|l|}{}&	\multicolumn{1}{|l|}{}&	\multicolumn{1}{|l|}{}&	\multicolumn{1}{|l|}{}&	\multicolumn{1}{|l|}{}&	\multicolumn{1}{|l|}{}&	\multicolumn{1}{|l|}{}&	\multicolumn{1}{|l|}{}&	\multicolumn{1}{|l|}{}&	\multicolumn{1}{|l|}{}&	\multicolumn{1}{|l|}{}	\\
\hline
\multicolumn{1}{|l|}{}&	\multicolumn{1}{|l|}{}&	\multicolumn{1}{|l|}{}&	\multicolumn{1}{|l|}{}&	\multicolumn{1}{|l|}{}&	\multicolumn{1}{|l|}{}&	\multicolumn{1}{|l|}{}&	\multicolumn{1}{|l|}{}&	\multicolumn{1}{|l|}{}&	\multicolumn{1}{|l|}{}&	\multicolumn{1}{|l|}{}&	\multicolumn{1}{|l|}{}&	\multicolumn{1}{|l|}{}&	\multicolumn{1}{|l|}{}&	\multicolumn{1}{|l|}{}&	\multicolumn{1}{|l|}{}&	\multicolumn{1}{|l|}{}&	\multicolumn{1}{|l|}{}	\\
\hline
\end{tabular}

\end{center}
Zadanie 7. (1pkt)

Zbiorem wartości ffinkcjifjest

A. $\langle-2,5\rangle$

B. $\langle-4,8\rangle$

C. $\langle-1,4\rangle$

D. $\langle$5, $ 8\rangle$

Zadanie 8. (1pkt)

Korzystając z wykresu ffinkcjif, wskaz nierówność prawdziwą.

A. $f(-1)<f(1)$

B. $f(1)<f(3)$

C. $f(-1)<f(3)$

D. $f(3)<f(0)$

Zadanie 9. $(1pkt)$

Wykres ffinkcji $g$ określonej wzorem $g(x)=f(x)+2$ jest przedstawiony na rysunku

A. B.





{\it Próbny egzamin maturalny z matematyki}

{\it Poziom podstawowy}

{\it 5}

BRUDNOPIS





{\it 6}

{\it Próbny egzamin maturalny z matematyki}

{\it Poziom podstawowy}

Zadanie 10. $(1pkt)$

Liczby $x_{1}$ i $x_{2}$ sąpierwiastkami równania $x^{2}+10x-24=0\mathrm{i}x_{1}<x_{2}$. Oblicz $2x_{1}+x_{2}.$

A. $-22$

B. $-17$

C. 8

D. 13

Zadanie ll. (lpkt)

Liczba 2 jest pierwiastkiem wie1omianu

równy

$W(x)=x^{3}+ax^{2}+6x-4$. Współczynnik $a$ jest

A. 2

B. $-2$

C. 4

D. $-4$

Zadanie 12. $(1pkt)$

Wskaz $m$, dla którego ffinkcja liniowa określona wzorem $f(x)=(m-1)x+3$ jest stała.

A. $m=1$

B. $m=2$

C. $m=3$

D. $m=-1$

Zadanie 13. $(1pkt)$

Zbiorem rozwiązań nierówności $(x-2)(x+3)\geq 0$ jest

A.

B.

C.

D.

$\langle-2,3\rangle$

$\langle-3,2\rangle$

$(-\infty,-3\rangle\cup\langle 2,+\infty)$

$(-\infty,-2\rangle\cup\langle 3,+\infty)$

Zadanie 14. $(1pkt)$

$\mathrm{W}$ ciągu geometrycznym $(a_{n})$ dane są: $a_{1}=2\mathrm{i}a_{2}=12$. Wtedy

A. $a_{4}=26$

B. $a_{4}=432$

C. $a_{4}=32$

D. $a_{4}=2592$

Zadanie 15. $(1pkt)$

$\mathrm{W}$ ciągu arytmetycznym $a_{1}=3$ oraz $a_{20}=7$. Wtedy suma $S_{20}=a_{1}+a_{2}+\ldots+a_{19}+a_{20}$ jest

równa

A. 95

B. 200

C. 230

D. 100

Zadanie 16. $(1pkt)$

Na rysunku zaznaczono długości boków i kąt $\alpha$ trójkąta prostokątnego (zobacz rysunek). Wtedy
\begin{center}
\includegraphics[width=87.984mm,height=32.868mm]{./F1_M_PP_L2010_page5_images/image001.eps}
\end{center}
13

5

12

A.

$\displaystyle \cos\alpha=\frac{5}{13}$

B.

$\displaystyle \mathrm{t}\mathrm{g}\alpha=\frac{13}{12}$

C.

$\displaystyle \cos\alpha=\frac{12}{13}$

D.

$\displaystyle \mathrm{t}\mathrm{g}\alpha=\frac{12}{5}$





{\it Próbny egzamin maturalny z matematyki}

{\it Poziom podstawowy}

7

BRUDNOPIS





{\it 8}

{\it Próbny egzamin maturalny z matematyki}

{\it Poziom podstawowy}

Zadanie 17. (1pkt)

Ogród ma kształt prostokąta o bokach długości 20 m i 40 m. Na dwóch końcach przekątnej

tego prostokąta wbito słupki. Odległość między tymi słupkamijest

A.

B.

C.

D.

równa 40 $\mathrm{m}$

większa $\mathrm{n}\mathrm{i}\dot{\mathrm{z}}50\mathrm{m}$

większa $\mathrm{n}\mathrm{i}\dot{\mathrm{z}}40\mathrm{m}$ i mniejsza $\mathrm{n}\mathrm{i}\dot{\mathrm{z}}45\mathrm{m}$

większa $\mathrm{n}\mathrm{i}\dot{\mathrm{z}}45\mathrm{m}$ i mniejsza $\mathrm{n}\mathrm{i}\dot{\mathrm{z}}50\mathrm{m}$

Zadanie 18. (1pkt)

Pionowy słupek o wysokości 90 cm rzuca cień o długości 60 cm. W tej samej chwi1i stojąca

obok wieza rzuca cień długości 12 m. Jakajest wysokość wiezy?

A. 18 m

B. 8m

C. 9m

D. 16 m

Zadanie 19. $(1pkt)$

Punkty $A, B \mathrm{i} C$ lez$\cdot$ą na okręgu o środku $S$ (zobacz rysunek). Miara zaznaczonego kąta

wpisanego $ACB$ jest równa
\begin{center}
\includegraphics[width=53.796mm,height=52.728mm]{./F1_M_PP_L2010_page7_images/image001.eps}
\end{center}
{\it C}

{\it A  B}

{\it S}

$230^{\mathrm{o}}$

A. $65^{\mathrm{o}}$

B. $100^{\mathrm{o}}$

C. $115^{\mathrm{o}}$

D. $130^{\mathrm{o}}$

Zadanie 20. $(1pkt)$

Dane sąpunkty $S=(2,1), M=(6,4)$. Równanie okręgu o środku $S$ i przechodzącego przez

punkt $M$ ma postać

A.

B.

C.

D.

$(x-2)^{2}+(y-1)^{2}=5$

$(x-2)^{2}+(y-1)^{2}=25$

$(x-6)^{2}+(y-4)^{2}=5$

$(x-6)^{2}+(y-4)^{2}=25$





{\it Próbny egzamin maturalny z matematyki}

{\it Poziom podstawowy}

{\it 9}

BRUDNOPIS





$ 1\theta$

{\it Próbny egzamin maturalny z matematyki}

{\it Poziom podstawowy}

Zadanie 21. $(1pkt)$

Proste o równaniach $y=2x+3$ oraz $y=-\displaystyle \frac{1}{3}x+2$

A. są równoległe i rózne

B. sąprostopadłe

C. przecinają się pod kątem innym $\mathrm{n}\mathrm{i}\dot{\mathrm{z}}$ prosty

D. pokrywają się

Zadanie 22. $(1pkt)$

Wskaz równanie prostej, którajest osią symetrii paraboli o równaniu $y=x^{2}-4x+2010.$

A. $x=4$

B. $x=-4$

C. $x=2$

D. $x=-2$

Zadanie 23. $(1pkt)$

Kąt $\alpha$ jest ostry i $\displaystyle \cos\alpha=\frac{3}{7}$. Wtedy

A.

$\displaystyle \sin\alpha=\frac{2\sqrt{10}}{7}$

B.

$\displaystyle \sin\alpha=\frac{\sqrt{10}}{7}$

C.

$\displaystyle \sin\alpha=\frac{4}{7}$

D.

$\displaystyle \sin\alpha=\frac{3}{4}$

Zadanie 24. (1pkt)

W karcie dań jest 5 zup i 4 drugie dania. Na i1e sposobów mozna zamówić obiad s$\mathbb{H}$adający się

zjednej zupy ijednego drugiego dania?

A. 25

B. 20

C. 16

D. 9

Zadanie 25. (1pkt)

W czterech rzutach sześcienną kostką do gry otrzymano następujące liczby oczek: 6, 3, 1, 4.

Mediana tych danychjest równa

A. 2

B. 2,5

C. 5

D. 3,5






\begin{center}
\begin{tabular}{l|l}
\multicolumn{1}{l|}{$\begin{array}{l}\mbox{{\it dysleksja}}	\\	\mbox{Miejsce}	\\	\mbox{na na ejkę}	\\	\mbox{z kodem szkoly}	\end{array}$}&	\multicolumn{1}{|l}{MMA-PIAIP-052}	\\
\hline
\multicolumn{1}{l|}{$\begin{array}{l}\mbox{EGZAMIN MATURALNY}	\\	\mbox{Z MATEMATYKI}	\\	\mbox{Arkusz I}	\\	\mbox{POZIOM PODSTAWOWY}	\\	\mbox{Czas pracy 120 minut}	\\	\mbox{Instrukcja dla zdającego}	\\	\mbox{1. $\mathrm{S}\mathrm{p}\mathrm{r}\mathrm{a}\mathrm{w}\mathrm{d}\acute{\mathrm{z}}$, czy arkusz egzaminacyjny zawiera 13 stron.}	\\	\mbox{Ewentualny brak zgłoś przewodniczącemu zespo}	\\	\mbox{nadzorującego egzamin.}	\\	\mbox{2. Rozwiązania zadań i odpowiedzi zamieść w miejscu na to}	\\	\mbox{przeznaczonym.}	\\	\mbox{3. $\mathrm{W}$ rozwiązaniach zadań przedstaw tok rozumowania}	\\	\mbox{prowadzący do ostatecznego wyniku.}	\\	\mbox{4. Pisz czytelnie. Uzywaj długopisu pióra tylko z czatnym}	\\	\mbox{tusze atramentem.}	\\	\mbox{5. Nie uzywaj korektora. Błędne zapisy prze eśl.}	\\	\mbox{6. Pamiętaj, $\dot{\mathrm{z}}\mathrm{e}$ zapisy w $\mathrm{b}$ dnopisie nie podlegają ocenie.}	\\	\mbox{7. Obok $\mathrm{k}\mathrm{a}\dot{\mathrm{z}}$ dego zadania podanajest maksymalna liczba punktów,}	\\	\mbox{którą mozesz uzyskać zajego poprawne rozwiązanie.}	\\	\mbox{8. $\mathrm{M}\mathrm{o}\dot{\mathrm{z}}$ esz korzystać z zestawu wzorów matematycznych, cyrkla}	\\	\mbox{i linijki oraz kalkulatora.}	\\	\mbox{9. Wypełnij tę część ka $\mathrm{y}$ odpowiedzi, którą koduje zdający.}	\\	\mbox{Nie wpisuj $\dot{\mathrm{z}}$ adnych znaków w części przeznaczonej}	\\	\mbox{dla egzaminatora.}	\\	\mbox{10. Na karcie odpowiedzi wpisz swoją datę urodzenia i PESEL.}	\\	\mbox{Zamaluj $\blacksquare$ pola odpowiadające cyfrom numeru PESEL. Błędne}	\\	\mbox{zaznaczenie otocz kółkiem i zaznacz właściwe.}	\\	\mbox{{\it Zyczymy powodzenia}.'}	\end{array}$}&	\multicolumn{1}{|l}{$\begin{array}{l}\mbox{ARKUSZ I}	\\	\mbox{MAJ}	\\	\mbox{ROK 2005}	\\	\mbox{Za rozwiązanie}	\\	\mbox{wszystkich zadań}	\\	\mbox{mozna otrzymać}	\\	\mbox{łącznie}	\\	\mbox{50 punktów}	\end{array}$}	\\
\hline
\multicolumn{1}{l|}{$\begin{array}{l}\mbox{Wypelnia zdający przed}	\\	\mbox{roz oczęciem racy}	\\	\mbox{PESEL ZDAJACEGO}	\end{array}$}&	\multicolumn{1}{|l}{$\begin{array}{l}\mbox{tylko}	\\	\mbox{O Kraków,}	\\	\mbox{OKE Wroclaw}	\\	\mbox{KOD}	\\	\mbox{ZDAJACEGO}	\end{array}$}
\end{tabular}


\includegraphics[width=78.792mm,height=13.356mm]{./F1_M_PP_M2005_page0_images/image001.eps}

\includegraphics[width=21.840mm,height=9.804mm]{./F1_M_PP_M2005_page0_images/image002.eps}
\end{center}



{\it 2}

{\it Egzamin maturalny z matematyki}

{\it Arkusz I}

Zadanie 1. (3pkt)

W pudełku są trzy kule białe i pięć kul czarnych. Do pudełka mozna albo dołozyć jedną kulę

białą albo usunąč z niegojedną kulę czarn4 a następnie wy1osować z tego pudełkajedną ku1ę.

W którym z tych przypadków wylosowanie kuli białej jest bardziej prawdopodobne?

Wykonaj odpowiednie obliczenia.
\begin{center}
\includegraphics[width=192.588mm,height=252.684mm]{./F1_M_PP_M2005_page1_images/image001.eps}
\end{center}




{\it Egzamin maturalny z matematyki}

{\it Arkusz I}

{\it 11}

Zadanie 10. $(7pkt)$

$\mathrm{W}$ ostrosłupie czworokątnym prawidłowym wysokości przeciwległych ścian bocznych

poprowadzone z wierzchołka ostrosłupa mają długości $h$ i tworzą kąt o mierze $ 2\alpha$. Oblicz

objętość tego ostrosłupa.
\begin{center}
\includegraphics[width=192.588mm,height=258.720mm]{./F1_M_PP_M2005_page10_images/image001.eps}
\end{center}




{\it 12}

{\it Egzamin maturalny z matematyki}

{\it Arkusz I}

BRUDNOPIS





{\it Egzamin maturalny z matematyki}

{\it Arkusz I}

{\it 13}





{\it Egzamin maturalny z matematyki}

{\it Arkusz I}

{\it 3}
\begin{center}
\includegraphics[width=193.548mm,height=290.220mm]{./F1_M_PP_M2005_page2_images/image001.eps}
\end{center}
Zadanie 2. $(4pkt)$

Dany jest ciąg $(a_{n})$, gdzie $a_{n}=\displaystyle \frac{n+2}{3n+1}$ dla $ n=1,2,3\ldots$ Wyznacz wszystkie wyrazy tego ciągu

większe od $\displaystyle \frac{1}{2}$





{\it 4}

{\it Egzamin maturalny z matematyki}

{\it Arkusz I}

Zadanie 3. (4pkt)

Dany jest wielomian $W(x)=x^{3}+kx^{2}-4.$

a) Wyznacz współczynnik $k$ tego wielomianu wiedząc, $\dot{\mathrm{z}}\mathrm{e}$ wielomian ten jest podzielny

przez dwumian $x+2.$

b) Dla wyznaczonej wartości $k$ rozłóz wielomian na czynniki i podaj wszystkie jego

pierwiastki.
\begin{center}
\includegraphics[width=192.588mm,height=240.696mm]{./F1_M_PP_M2005_page3_images/image001.eps}
\end{center}




{\it Egzamin maturalny z matematyki}

{\it Arkusz I}

{\it 5}

Zadanie 4. $(5pkt)$

Na trzech półkach ustawiono 76 płyt kompaktowych. Okazało się, $\dot{\mathrm{z}}\mathrm{e}$ liczby płyt na półkach

gótnej, środkowej i dolnej tworzą rosnący ciąg geometryczny. Na środkowej półce stoją

24 płyty. Oblicz, ile płyt stoi na półce gótnej, a ile płyt stoi na półce dolnej.
\begin{center}
\includegraphics[width=192.588mm,height=258.720mm]{./F1_M_PP_M2005_page4_images/image001.eps}
\end{center}




{\it 6}

{\it Egzamin maturalny z matematyki}

{\it Arkusz I}

Zadanie 5. $(4pkt)$

Sklep sprowadza z hurtowni kurtki płacąc po 100 zł za sztukę i sprzedaje średnio 40 sztuk

miesięcznie po 160 zł. Zaobserwowano, $\dot{\mathrm{z}}\mathrm{e} \mathrm{k}\mathrm{a}\dot{\mathrm{z}}$ da kolejna obnizka ceny sprzedaz$\mathrm{y}$ kurtki

$01$ zł zwiększa sprzedaz miesięczną o l sztukę. Jaką cenę kurtki powinien ustalić

sprzedawca, abyjego miesięczny zysk był największy?
\begin{center}
\includegraphics[width=192.588mm,height=252.684mm]{./F1_M_PP_M2005_page5_images/image001.eps}
\end{center}




{\it Egzamin maturalny z matematyki}

{\it Arkusz I}

7

Zadanie 6. (6pkt)

Dane są zbiory liczb rzeczywistych:

$A=\{x:|x+2|\langle 3\}$

$B=\{x:(2x-1)^{3}\leq 8x^{3}-13x^{2}+6x+3\}$

Zapisz w postaci przedziałów liczbowych zbiory $A, B, A\cap B$ oraz $B-A.$
\begin{center}
\includegraphics[width=192.588mm,height=240.696mm]{./F1_M_PP_M2005_page6_images/image001.eps}
\end{center}




{\it 8}

{\it Egzamin maturalny z matematyki}

{\it Arkusz I}

Zadanie 7. (5pkt)

W ponizszej tabeli przedstawiono wyniki sondazu przeprowadzonego w grupie uczniów,

dotyczącego czasu przeznaczanego dziennie na przygotowanie zadań domowych.
\begin{center}
\begin{tabular}{|l|l|l|l|l|}
\hline
\multicolumn{1}{|l|}{$\begin{array}{l}\mbox{Czas}	\\	\mbox{(w godzinach)}	\end{array}$}&	\multicolumn{1}{|l|}{ $1$}&	\multicolumn{1}{|l|}{ $2$}&	\multicolumn{1}{|l|}{ $3$}&	\multicolumn{1}{|l|}{ $4$}	\\
\hline
\multicolumn{1}{|l|}{$\begin{array}{l}\mbox{Liczba}	\\	\mbox{uczniów}	\end{array}$}&	\multicolumn{1}{|l|}{ $5$}&	\multicolumn{1}{|l|}{ $10$}&	\multicolumn{1}{|l|}{ $15$}&	\multicolumn{1}{|l|}{ $10$}	\\
\hline
\end{tabular}

\end{center}
a) Naszkicuj diagram s

wyniki tego sondazu.

pkowy ilustrujący

b) Oblicz średnią liczbę godzin, jaką

uczniowie przeznaczają dziennie na

przygotowanie zadań domowych.
\begin{center}
\includegraphics[width=96.972mm,height=96.924mm]{./F1_M_PP_M2005_page7_images/image001.eps}
\end{center}
c)

Oblicz wariancję i odchylenie

standardowe czasu przeznaczonego

dziennie na przygotowanie zadań

domowych. Wynik podaj z dokładnością

do 0,01.
\begin{center}
\includegraphics[width=192.588mm,height=126.492mm]{./F1_M_PP_M2005_page7_images/image002.eps}
\end{center}




{\it Egzamin maturalny z matematyki}

{\it Arkusz I}

{\it 9}

Zadanie 8. (6pkt)

Z kawałka materiału o kształcie i wymiarach

czworokąta ABCD (patrz na rysunek obok)

wycięto okrągłą serwetkę o promieniu 3 dm.

Oblicz, ile procent całego materiału stanowi

jego niewykorzystana część. Wynik podaj

z dokładnością do 0,01 procenta.
\begin{center}
\includegraphics[width=71.376mm,height=82.092mm]{./F1_M_PP_M2005_page8_images/image001.eps}
\end{center}
{\it c}

{\it D}

10

{\it o}

3
\begin{center}
\includegraphics[width=192.588mm,height=204.624mm]{./F1_M_PP_M2005_page8_images/image002.eps}
\end{center}




$ 1\theta$

{\it Egzamin maturalny z matematyki}

{\it Arkusz I}

Zadanie 9. (6pkt)
\begin{center}
\includegraphics[width=193.644mm,height=280.620mm]{./F1_M_PP_M2005_page9_images/image001.eps}
\end{center}
Rodzeństwo w wieku 8 $\mathrm{i} 10$ lat otrzymało razem w spadku 84100 zł. Kwotę tę złozono

w banku, który stosuje kapitalizację roczną przy rocznej stopie procentowej 5\%. $\mathrm{K}\mathrm{a}\dot{\mathrm{z}}$ de

z dzieci otrzyma swoją część spadku z chwilą osiągnięcia wieku 211at. $\dot{\mathrm{Z}}$ yczeniem

spadkodawcy było takie podzielenie kwoty spadku, aby w przyszłości obie wypłacone części

spadku zaokrąglone do l zł były równe. Jak nalez$\mathrm{y}$ podzielić kwotę 84100 zł między

rodzeńs $0$? Za isz wszystkie wykon ane obliczenia.






\begin{center}
\begin{tabular}{l|l}
\multicolumn{1}{l|}{$\begin{array}{l}\mbox{{\it dysleksja}}	\\	\mbox{Miejsce}	\\	\mbox{na na ejkę}	\\	\mbox{z kodem szkoly}	\end{array}$}&	\multicolumn{1}{|l}{MMA-PIAIP-062}	\\
\hline
\multicolumn{1}{l|}{ $\begin{array}{l}\mbox{EGZAMIN MATURALNY}	\\	\mbox{Z MATEMATYKI}	\\	\mbox{Arkusz I}	\\	\mbox{POZIOM PODSTAWOWY}	\\	\mbox{Czas pracy 120 minut}	\\	\mbox{Instrukcja dla zdającego}	\\	\mbox{1. Sprawdzí, czy arkusz egzaminacyjny zawiera 14 stron (zadania}	\\	\mbox{$1-11)$. Ewentualny brak zgłoś przewodniczącemu zespo}	\\	\mbox{nadzorującego egzamin.}	\\	\mbox{2. Rozwiązania zadań i odpowiedzi zamieść w miejscu na to}	\\	\mbox{przeznaczonym.}	\\	\mbox{3. $\mathrm{W}$ rozwiązaniach zadań przedstaw tok rozumowania}	\\	\mbox{prowadzący do ostatecznego wyniku.}	\\	\mbox{4. Pisz czytelnie. $\mathrm{U}\dot{\mathrm{z}}$ aj długopisu pióra tylko z czarnym}	\\	\mbox{tusze atramentem.}	\\	\mbox{5. Nie uzywaj korektora, a błędne zapisy prze eśl.}	\\	\mbox{6. Pamiętaj, $\dot{\mathrm{z}}\mathrm{e}$ zapisy w $\mathrm{b}$ dnopisie nie podlegają ocenie.}	\\	\mbox{7. Obok $\mathrm{k}\mathrm{a}\dot{\mathrm{z}}$ dego zadania podanajest maksymalna liczba punktów,}	\\	\mbox{którą mozesz uzyskać zajego poprawne rozwiązanie.}	\\	\mbox{8. $\mathrm{M}\mathrm{o}\dot{\mathrm{z}}$ esz korzystać z zestawu wzorów matematycznych, cyrkla}	\\	\mbox{i linijki oraz kalkulatora.}	\\	\mbox{9. Wypełnij tę część ka $\mathrm{y}$ odpowiedzi, którą koduje zdający.}	\\	\mbox{Nie wpisuj $\dot{\mathrm{z}}$ adnych znaków w części przeznaczonej dla}	\\	\mbox{egzaminatora.}	\\	\mbox{10. Na karcie odpowiedzi wpisz swoją datę urodzenia i PESEL.}	\\	\mbox{Zamaluj $\blacksquare$ pola odpowiadające cyfrom numeru PESEL. Błędne}	\\	\mbox{zaznaczenie otocz kółkiem $\mathrm{O}$ i zaznacz właściwe.}	\\	\mbox{{\it Zyczymy} $p\theta wodzenia'$}	\end{array}$}&	\multicolumn{1}{|l}{$\begin{array}{l}\mbox{ARKUSZ I}	\\	\mbox{MAJ}	\\	\mbox{ROK 2006}	\\	\mbox{Za rozwiązanie}	\\	\mbox{wszystkich zadań}	\\	\mbox{mozna otrzymać}	\\	\mbox{łącznie}	\\	\mbox{50 punktów}	\end{array}$}	\\
\hline
\multicolumn{1}{l|}{$\begin{array}{l}\mbox{Wypelnia zdający przed}	\\	\mbox{roz oczęciem racy}	\\	\mbox{PESEL ZDAJACEGO}	\end{array}$}&	\multicolumn{1}{|l}{$\begin{array}{l}\mbox{KOD}	\\	\mbox{ZDAJACEGO}	\end{array}$}
\end{tabular}


\includegraphics[width=21.840mm,height=9.852mm]{./F1_M_PP_M2006_page0_images/image001.eps}

\includegraphics[width=78.792mm,height=13.356mm]{./F1_M_PP_M2006_page0_images/image002.eps}
\end{center}



{\it 2}

{\it Egzamin maturalny z matematyki}

{\it Arkusz I}

Zadanie l. $(3pkt)$

Dane są zbiory: $A=\{x\in R:|x-4|\geq 7\}, B=\{x\in R$:

a) zbiór $A,$

b) zbiór $B,$

c) zbiór $C=B\backslash A.$

$x^{2}>0$. Zaznacz na osi liczbowej:

a)

b)

c)
\begin{center}
\includegraphics[width=192.072mm,height=72.240mm]{./F1_M_PP_M2006_page1_images/image001.eps}

\includegraphics[width=192.072mm,height=72.240mm]{./F1_M_PP_M2006_page1_images/image002.eps}

\includegraphics[width=192.072mm,height=72.240mm]{./F1_M_PP_M2006_page1_images/image003.eps}

\includegraphics[width=109.932mm,height=17.580mm]{./F1_M_PP_M2006_page1_images/image004.eps}
\end{center}
WypelnÍa

egzaminator!

Nr czynności

Maks. liczba kt

1.1.

1

1.2.

1.3.

1

Uzyskana liczba pkt





{\it Egzamin maturalny z matematyki}

{\it Arkusz I}

{\it 11}

Zadanie 10. $(6pkt)$

Liczby 3 $\mathrm{i}-1$ sąpierwiastkami wielomianu $W(x)=2x^{3}+ax^{2}+bx+30.$

a) Wyznacz wartości współczynników a $\mathrm{i}b.$

b) Oblicz trzeci pierwiastek tego wielomianu.
\begin{center}
\includegraphics[width=192.276mm,height=260.508mm]{./F1_M_PP_M2006_page10_images/image001.eps}

\includegraphics[width=151.788mm,height=17.580mm]{./F1_M_PP_M2006_page10_images/image002.eps}
\end{center}
Wypelnia

egzamÍnator!

Nr czynności

Maks. liczba kt

10.1.

1

10.2.

1

10.3.

1

10.4.

1

10.5.

1

1

Uzyskana liczba pkt





{\it 12}

{\it Egzamin maturalny z matematyki}

{\it Arkusz I}

Zadanie ll. $(3pkt)$

Sumę $S=\displaystyle \frac{3}{1\cdot 4}+\frac{3}{4\cdot 7}+\frac{3}{7\cdot 10}+\ldots+\frac{3}{301\cdot 304}+\frac{3}{304\cdot 307}$ mozna obliczyć w następujący sposób:

a) sumę $S$ zapisujemy w postaci

$S=\displaystyle \frac{4-1}{4\cdot 1}+\frac{7-4}{7\cdot 4}+\frac{10-7}{10\cdot 7}+\ldots+\frac{304-301}{304\cdot 301}+\frac{307-304}{307\cdot 304}$

b) $\mathrm{k}\mathrm{a}\dot{\mathrm{z}}\mathrm{d}\mathrm{y}$ składnik tej sumy przedstawiamy jako róznicę ułamków

$S=(\displaystyle \frac{4}{4\cdot 1}-\frac{1}{4\cdot 1})+(\frac{7}{7\cdot 4}-\frac{4}{7\cdot 4})+(\frac{10}{10\cdot 7}-\frac{7}{10\cdot 7})+\ldots+(\frac{304}{304\cdot 301}-\frac{301}{304\cdot 301})+(\frac{307}{307\cdot 304}-\frac{304}{307\cdot 304})$

stąd $S=(1-\displaystyle \frac{1}{4})+(\frac{1}{4}-\frac{1}{7})+(\frac{1}{7}-\frac{1}{10})+\ldots+(\frac{1}{301}-\frac{1}{304})+(\frac{1}{304}-\frac{1}{307})$

więc $S=1-\displaystyle \frac{1}{4}+\frac{1}{4}-\frac{1}{7}+\frac{1}{7}-\frac{1}{10}+\ldots+\frac{1}{301}-\frac{1}{304}+\frac{1}{304}-\frac{1}{307}$

c) obliczamy sumę, redukując parami wyrazy sąsiednie, poza pierwszym i ostatnim

$S=1-\displaystyle \frac{1}{307}=\frac{306}{307}.$

Postępując w analogiczny sposób, oblicz sumę $S_{1}=\displaystyle \frac{4}{1\cdot 5}+\frac{4}{5\cdot 9}+\frac{4}{9\cdot 13}+\ldots+\frac{4}{281\cdot 285}$
\begin{center}
\includegraphics[width=192.228mm,height=193.956mm]{./F1_M_PP_M2006_page11_images/image001.eps}
\end{center}




{\it Egzamin maturalny z matematyki}

{\it Arkusz I}

{\it 13}
\begin{center}
\includegraphics[width=192.276mm,height=290.784mm]{./F1_M_PP_M2006_page12_images/image001.eps}

\includegraphics[width=109.980mm,height=17.580mm]{./F1_M_PP_M2006_page12_images/image002.eps}
\end{center}
Nr czynno\S ci

WypelnÍa Maks. liczba kt

egzaminator! Uzyskana liczba pkt

11.1.

1

11.2.

11.3.

1





{\it 14}

{\it Egzamin maturalny z matematyki}

{\it Arkusz I}

BRUDNOPIS





{\it Egzamin maturalny z matematyki}

{\it Arkusz I}

{\it 3}
\begin{center}
\includegraphics[width=192.276mm,height=289.200mm]{./F1_M_PP_M2006_page2_images/image001.eps}
\end{center}
Zadanie 2. $(3pkt)$

$\mathrm{W}$ wycieczce szkolnej bierze udział 16 uczniów, wśród których ty1ko czworo zna oko1icę.

Wychowawca chce wybrać w sposób losowy 3 osoby, które mają pójść do sk1epu. Ob1icz

prawdopodobieństwo tego, $\dot{\mathrm{z}}\mathrm{e}$ wśród wybranych trzech osób będą dokładnie dwie znające

okolicę.
\begin{center}
\includegraphics[width=109.980mm,height=17.580mm]{./F1_M_PP_M2006_page2_images/image002.eps}
\end{center}
Nr czynno\S ci

WypelnÍa Maks. liczba $\llcorner\prime \mathrm{t}$

egzaminator! Uzyskana liczba pkt

2.1.

1

2.2.

2.3.

1





{\it 4}

{\it Egzamin maturalny z matematyki}

{\it Arkusz I}

Zadanie 3. (5pkt)

Kostka masła produkowanego przez pewien zakład mleczarski ma nominalną masę

20 dag. W czasie kontroli zakładu zwazono l50 losowo wybranych kostek masła. Wyniki

badań przedstawiono w tabeli.
\begin{center}
\begin{tabular}{|l|l|l|l|l|l|l|}
\hline
\multicolumn{1}{|l|}{Masa kostki masła (w dag)}&	\multicolumn{1}{|l|}{$16$}&	\multicolumn{1}{|l|}{ $18$}&	\multicolumn{1}{|l|}{ $19$}&	\multicolumn{1}{|l|}{ $20$}&	\multicolumn{1}{|l|}{ $21$}&	\multicolumn{1}{|l|}{ $22$}	\\
\hline
\multicolumn{1}{|l|}{Liczba kostek masła}&	\multicolumn{1}{|l|}{$1$}&	\multicolumn{1}{|l|}{ $15$}&	\multicolumn{1}{|l|}{ $24$}&	\multicolumn{1}{|l|}{ $68$}&	\multicolumn{1}{|l|}{ $26$}&	\multicolumn{1}{|l|}{ $16$}	\\
\hline
\end{tabular}

\end{center}
a) Na podstawie danych przedstawionych w tabeli oblicz średnią arytmetyczną oraz

odchylenie standardowe masy kostki masła.

b) Kontrola wypada pozytywnie, jeśli średnia masa kostki masła jest równa masie

nominalnej i odchylenie standardowe nie przekracza l dag. Czy kontrola zakładu

wypadła pozytywnie? Odpowiedzí uzasadnij.
\begin{center}
\includegraphics[width=192.228mm,height=212.088mm]{./F1_M_PP_M2006_page3_images/image001.eps}

\includegraphics[width=109.932mm,height=17.580mm]{./F1_M_PP_M2006_page3_images/image002.eps}
\end{center}
Wypelnia

egzaminator!

Nr czynnoŚci

Maks. liczba kt

3.1.

2

3.2.

2

3.3.

Uzyskana liczba pkt





{\it Egzamin maturalny z matematyki}

{\it Arkusz I}

{\it 5}
\begin{center}
\includegraphics[width=192.276mm,height=286.668mm]{./F1_M_PP_M2006_page4_images/image001.eps}
\end{center}
Zadanie 4. $(4pkt)$

Dany jest rosnący ciąg geometryczny, w którym $a_{1}=12, a_{3}=27.$

a) Wyznacz iloraz tego ciągu.

b) Zapisz wzór, na podstawie którego mozna obliczyć wyraz $a_{n}$, dla $\mathrm{k}\mathrm{a}\dot{\mathrm{z}}$ dej liczby naturalnej

$n\geq 1.$

c) Oblicz wyraz $a_{6}.$
\begin{center}
\includegraphics[width=109.980mm,height=17.580mm]{./F1_M_PP_M2006_page4_images/image002.eps}
\end{center}
Nr czynności

Wypelnia Maks. liczba $\llcorner\prime \mathrm{t}$

egzaminator! Uzyskana liczba pkt

4.1.

2

4.2.

1

4.3.





{\it 6}

{\it Egzamin maturalny z matematyki}

{\it Arkusz I}

Zadanie 5. $(3pkt)$

Wiedząc, $\dot{\mathrm{z}}\mathrm{e}0^{\mathrm{o}}\leq\alpha\leq 360^{\mathrm{o}}, \sin\alpha<0$ oraz 4 tg $\alpha=3\sin^{2}\alpha+3\cos^{2}\alpha$

a) oblicz $\mathrm{t}\mathrm{g}\alpha,$

b) zaznacz w układzie współrzędnych kąt $\alpha$ i podaj współrzędne dowolnego punktu,

róznego od początku układu współrzędnych, który lezy na końcowym ramieniu tego
\begin{center}
\includegraphics[width=84.024mm,height=108.408mm]{./F1_M_PP_M2006_page5_images/image001.eps}
\end{center}
kąta.
\begin{center}
\includegraphics[width=90.732mm,height=145.488mm]{./F1_M_PP_M2006_page5_images/image002.eps}

\includegraphics[width=109.932mm,height=17.628mm]{./F1_M_PP_M2006_page5_images/image003.eps}
\end{center}
Wypelnia

egzaminator!

Nr czynnoŚci

Maks. liczba kl

5.1.

1

5.2.

1

5.3.

Uzyskana liczba pkt





{\it Egzamin maturalny z matematyki}

{\it Arkusz I}

7

Zadanie 6. $(7pkt)$

Państwo Nowakowie przeznaczyli 26000 zł na zakup działki. Do jednej z ofert dołączono

rysunek dwóch przylegających do siebie działek w skali 1:1000. Jeden metr kwadratowy

gruntu w tej ofercie kosztuje 35 zł. Ob1icz, czy przeznaczona przez państwa Nowaków kwota

wystarczy na zakup działki $\mathrm{P}_{2}.$

E
\begin{center}
\includegraphics[width=126.240mm,height=55.020mm]{./F1_M_PP_M2006_page6_images/image001.eps}
\end{center}
D

$\mathrm{P}_{1}$

$\mathrm{P}_{2}$

A  B  C

AE $=5$ cm,

EC $=13$ cm,

BC $=6,5$ cm.
\begin{center}
\includegraphics[width=193.044mm,height=193.908mm]{./F1_M_PP_M2006_page6_images/image002.eps}

\includegraphics[width=165.756mm,height=17.580mm]{./F1_M_PP_M2006_page6_images/image003.eps}
\end{center}
Wypelnia

egzaminator!

Nr czynnoŚci

Maks. liczba kt

1

1

1

1

1

1

Uzyskana liczba pkt





{\it 8}

{\it Egzamin maturalny z matematyki}

{\it Arkusz I}

Zadanie 7. $(5pkt)$

Szkic przedstawia kanał ciepłowniczy, którego przekrój poprzeczny jest prostokątem.

Wewnątrz kanału znajduje się rurociąg składający się z trzech rur, $\mathrm{k}\mathrm{a}\dot{\mathrm{z}}$ da o średnicy

zewnętrznej l $\mathrm{m}$. Oblicz wysokość i szerokość kanału ciepłowniczego. Wysokość zaokrąglij

do 0,01 $\mathrm{m}.$
\begin{center}
\includegraphics[width=192.636mm,height=97.380mm]{./F1_M_PP_M2006_page7_images/image001.eps}

\includegraphics[width=192.228mm,height=157.632mm]{./F1_M_PP_M2006_page7_images/image002.eps}

\includegraphics[width=123.900mm,height=17.580mm]{./F1_M_PP_M2006_page7_images/image003.eps}
\end{center}
Wypelnia

egzamÍnator!

Nr czynności

Maks. liczba kt

7.1.

1

7.2.

1

7.3.

2

7.4.

1

Uzyskana liczba pkt





{\it Egzamin maturalny z matematyki}

{\it Arkusz I}

{\it 9}

Zadanie 8. $(5pkt)$

Dana jest ffinkcja $f(x) =-x^{2} +6x-5.$

a) Naszkicuj wykres funkcji $f$ i podaj jej zbiór wartości.

b) Podaj rozwiązanie nierówności $f(x)\geq 0.$
\begin{center}
\includegraphics[width=176.940mm,height=103.476mm]{./F1_M_PP_M2006_page8_images/image001.eps}

\includegraphics[width=192.276mm,height=151.584mm]{./F1_M_PP_M2006_page8_images/image002.eps}

\includegraphics[width=137.820mm,height=17.580mm]{./F1_M_PP_M2006_page8_images/image003.eps}
\end{center}
WypelnÍa

egzaminator!

Nr czynności

Maks. liczba kt

8.1.

1

8.2.

1

8.3.

1

8.4.

1

8.5.

1

Uzyskana liczba pkt





$ 1\theta$

{\it Egzamin maturalny z matematyki}

{\it Arkusz I}

Zadanie 9. $(6pkt)$

Dach wiez$\mathrm{y}$ ma kształt powierzchni bocznej ostrosłupa prawidłowego czworokątnego,

którego krawędzí podstawy ma długość 4 $\mathrm{m}$. Ściana boczna tego ostrosłupajest nachylona do

płaszczyzny podstawy pod kątem $60^{\mathrm{o}}$

a) Sporządz$\acute{}$ pomocniczy rysunek i zaznacz na nim podane w zadaniu wielkości.

b) Oblicz, ile sztuk dachówek nalez$\mathrm{y}$ kupić, aby pokryć ten dach, wiedząc, $\dot{\mathrm{z}}\mathrm{e}$ do pokrycia

$1\mathrm{m}^{2}$ potrzebne są24 dachówki. Przy zakupie na1ez$\mathrm{y}$ doliczyć 8\% dachówek na zapas.
\begin{center}
\includegraphics[width=192.228mm,height=242.364mm]{./F1_M_PP_M2006_page9_images/image001.eps}

\includegraphics[width=137.868mm,height=17.628mm]{./F1_M_PP_M2006_page9_images/image002.eps}
\end{center}
Nr czynności

Wypelnia Maks. liczba kt

egzaminator! Uzyskana liczba pkt

1

1

1

2

1






\begin{center}
\begin{tabular}{l|l}
\multicolumn{1}{l|}{$\begin{array}{l}\mbox{{\it dysleksja}}	\\	\mbox{Miejsce}	\\	\mbox{na naklejkę}	\\	\mbox{z kodem szkoly}	\end{array}$}&	\multicolumn{1}{|l}{ $\mathrm{M}\mathrm{M}\mathrm{A}-\mathrm{P}1_{-}1\mathrm{P}-072$}	\\
\hline
\multicolumn{1}{l|}{$\begin{array}{l}\mbox{EGZAMIN MATURALNY}	\\	\mbox{Z MATEMATYKI}	\\	\mbox{POZIOM PODSTAWOWY}	\\	\mbox{Czas pracy 120 minut}	\\	\mbox{Instrukcja dla zdającego}	\\	\mbox{1. Sprawdzí, czy arkusz egzaminacyjny zawiera 15 stron (zadania}	\\	\mbox{$1-11)$. Ewentualny brak zgłoś przewodniczącemu zespołu}	\\	\mbox{nadzorującego egzamin.}	\\	\mbox{2. Rozwiązania zadań i odpowiedzi zamieść w miejscu na to}	\\	\mbox{przeznaczonym.}	\\	\mbox{3. $\mathrm{W}$ rozwiązaniach zadań przedstaw tok rozumowania}	\\	\mbox{prowadzący do ostatecznego wyniku.}	\\	\mbox{4. Pisz czytelnie. Uzywaj $\mathrm{d}$ gopisu pióra tylko z czatnym}	\\	\mbox{tusze atramentem.}	\\	\mbox{5. Nie uzywaj korektora, a błędne zapisy prze eśl.}	\\	\mbox{6. Pamiętaj, $\dot{\mathrm{z}}\mathrm{e}$ zapisy w brudnopisie nie podlegają ocenie.}	\\	\mbox{7. Obok $\mathrm{k}\mathrm{a}\dot{\mathrm{z}}$ dego zadania podanajest maksymalna liczba punktów,}	\\	\mbox{którą mozesz uzyskać zajego poprawne rozwiązanie.}	\\	\mbox{8. $\mathrm{M}\mathrm{o}\dot{\mathrm{z}}$ esz korzystać z zestawu wzorów matematycznych, cyrkla}	\\	\mbox{i linijki oraz kalkulatora.}	\\	\mbox{9. Wypełnij tę część ka $\mathrm{y}$ odpowiedzi, którą koduje zdający.}	\\	\mbox{Nie wpisuj $\dot{\mathrm{z}}$ adnych znaków w części przeznaczonej dla}	\\	\mbox{egzaminatora.}	\\	\mbox{10. Na karcie odpowiedzi wpisz swoją datę urodzenia i PESEL.}	\\	\mbox{Zamaluj $\blacksquare$ pola odpowiadające cyfrom numeru PESEL. Błędne}	\\	\mbox{zaznaczenie otocz kółkiem $\mathrm{O}$ i zaznacz właściwe.}	\\	\mbox{{\it Zyczymy powodzenia}.'}	\end{array}$}&	\multicolumn{1}{|l}{$\begin{array}{l}\mbox{MAJ}	\\	\mbox{ROK 2007}	\\	\mbox{Za rozwiązanie}	\\	\mbox{wszystkich zadań}	\\	\mbox{mozna otrzymać}	\\	\mbox{łącznie}	\\	\mbox{50 punktów}	\end{array}$}	\\
\hline
\multicolumn{1}{l|}{$\begin{array}{l}\mbox{Wypelnia zdający}	\\	\mbox{rzed roz oczęciem racy}	\\	\mbox{PESEL ZDAJACEGO}	\end{array}$}&	\multicolumn{1}{|l}{$\begin{array}{l}\mbox{KOD}	\\	\mbox{ZDAJACEGO}	\end{array}$}
\end{tabular}


\includegraphics[width=21.840mm,height=9.852mm]{./F1_M_PP_M2007_page0_images/image001.eps}

\includegraphics[width=78.792mm,height=13.356mm]{./F1_M_PP_M2007_page0_images/image002.eps}
\end{center}



{\it 2}

{\it Egzamin maturalny z matematyki}

{\it Poziom podstawowy}

Zadanie 1. (5pkt)

Znajdzí wzór funkcji kwadratowej $y=f(x)$, której wykresem jest parabola o wierzchołku

$(1,-9)$ przechodząca przez punkt o współrzędnych $(2,-8)$. Otrzymaną funkcję przedstaw

w postaci kanonicznej. Obliczjej miejsca zerowe i naszkicuj wykres.
\begin{center}
\includegraphics[width=137.868mm,height=17.628mm]{./F1_M_PP_M2007_page1_images/image001.eps}
\end{center}
Nr czynnoŚci

Wypelnia Maks. liczba kt

egzaminator! Uzyskana liczba pkt

1.1.

1

1.2.

1

1.3.

1

1.4.

1

1.5.

1





{\it Egzamin maturalny z matematyki}

{\it Poziom podstawowy}

{\it 11}
\begin{center}
\includegraphics[width=151.788mm,height=17.580mm]{./F1_M_PP_M2007_page10_images/image001.eps}
\end{center}
WypelnÍa

egzaminator!

Nr czynności

Maks. lÍczba kt

1

1

1

1

1

Uzyskana liczba pkt





{\it 12}

{\it Egzamin maturalny z matematyki}

{\it Poziom podstawowy}

Zadanie 10. (5pkt)

Dany jest graniastosłup czworokątny prosty ABCDEFGH o podstawach ABCD $\mathrm{i}$ {\it EFGH oraz}

krawędziach bocznych $AE, BF, CG, DH$. Podstawa ABCD graniastosłupajest rombem o boku

długości 8 cm i kątach ostrych $A \mathrm{i} C$ o mierze $60^{\circ}$ Przekątna graniastosłupa $CE$ jest

nachylona do płaszczyzny podstawy pod kątem $60^{\circ}$ Sporządz$\acute{}$ rysunek pomocniczy i zaznacz

na nim wymienione w zadaniu kąty. Oblicz objętość tego graniastosłupa.





{\it Egzamin maturalny z matematyki}

{\it Poziom podstawowy}

{\it 13}
\begin{center}
\includegraphics[width=137.820mm,height=17.580mm]{./F1_M_PP_M2007_page12_images/image001.eps}
\end{center}
Wypelnia

egzaminator!

Nr czynno\S ci

Maks. liczba kt

10.1.

1

10.2.

1

10.3.

1

10.4.

1

10.5.

1

Uzyskana liczba pkt





{\it 14}

{\it Egzamin maturalny z matematyki}

{\it Poziom podstawowy}

Zadanie 11. (4pkt)

Dany jest rosnący ciąg geometryczny $(a_{n})$ dla

Oblicz $x$ oraz $y, \mathrm{j}\mathrm{e}\dot{\mathrm{z}}$ eli wiadomo, $\dot{\mathrm{z}}\mathrm{e}x+y=35$

$n\geq 1$, w którym $a_{1}=x, a_{2}=14, a_{3}=y.$
\begin{center}
\includegraphics[width=123.900mm,height=17.628mm]{./F1_M_PP_M2007_page13_images/image001.eps}
\end{center}
Wypelnia

egzaminator!

Nr czynności

Maks. liczba kt

11.1.

1

1

11.4.

1

Uzyskana liczba pkt





{\it Egzamin maturalny z matematyki}

{\it Poziom podstawowy}

{\it 15}

BRUDNOPIS





{\it Egzamin maturalny z matematyki}

{\it Poziom podstawowy}

{\it 3}

Zadanie 2. (3pkt)

Wysokość prowizji, którą klient płaci w pewnym biurze maklerskim przy $\mathrm{k}\mathrm{a}\dot{\mathrm{z}}$ dej zawieranej

transakcji kupna lub sprzedaz$\mathrm{y}$ akcji jest uzalezniona od wartości transakcji. Zalezność ta

została przedstawiona w tabeli:
\begin{center}
\begin{tabular}{|l|l|}
\hline
\multicolumn{1}{|l|}{Wartość transakcji}&	\multicolumn{1}{|l|}{Wysokość rowizji}	\\
\hline
\multicolumn{1}{|l|}{do 500 zł}&	\multicolumn{1}{|l|}{15 zł}	\\
\hline
\multicolumn{1}{|l|}{od 500,01 zł do 3000 zł}&	\multicolumn{1}{|l|}{2\% wartości $\mathrm{t}\mathrm{r}\mathrm{a}\mathrm{n}\mathrm{s}\mathrm{a}\mathrm{k}\mathrm{c}\mathrm{j}\mathrm{i}+5$ zł}	\\
\hline
\multicolumn{1}{|l|}{od 3000,01 zł do 8000 zł}&	\multicolumn{1}{|l|}{1,5\% wartości $\mathrm{t}\mathrm{r}\mathrm{a}\mathrm{n}\mathrm{s}\mathrm{a}\mathrm{k}\mathrm{c}\mathrm{j}\mathrm{i}+20$ zł}	\\
\hline
\multicolumn{1}{|l|}{od 8000,01 zł do 15000 zł}&	\multicolumn{1}{|l|}{1\% wartości $\mathrm{t}\mathrm{r}\mathrm{a}\mathrm{n}\mathrm{s}\mathrm{a}\mathrm{k}\mathrm{c}\mathrm{j}\mathrm{i}+60$ zł}	\\
\hline
\multicolumn{1}{|l|}{powyzej 15000 zł}&	\multicolumn{1}{|l|}{0,7\% wartości $\mathrm{t}\mathrm{r}\mathrm{a}\mathrm{n}\mathrm{s}\mathrm{a}\mathrm{k}\mathrm{c}\mathrm{j}\mathrm{i}+105$ zł}	\\
\hline
\end{tabular}

\end{center}
Klient zakupił za pośrednictwem tego biura maklerskiego 530 akcji w cenie 25 zł za jedną

akcję. Po roku sprzedał wszystkie kupione akcje po 45 zł zajedną sztukę. Ob1icz, i1e zarobił

na tych transakcjach po uwzględnieniu prowizji, które zapłacił.
\begin{center}
\includegraphics[width=109.980mm,height=17.580mm]{./F1_M_PP_M2007_page2_images/image001.eps}
\end{center}
Nr czynności

Wypelnia Maks. liczba kt

egzaminator! Uzyskana liczba pkt

2.1.

1

2.2.

1

2.3.

1





{\it 4}

{\it Egzamin maturalny z matematyki}

{\it Poziom podstawowy}

Zadanie 3. (4pkt)

Korzystając z danych przedstawionych na rysunku, oblicz wartość wyrazenia:

$\mathrm{t}\mathrm{g}^{2}\beta-5\sin\beta$. ctg $\alpha+\sqrt{1-\cos^{2}\alpha}.$

{\it C}
\begin{center}
\includegraphics[width=81.684mm,height=35.256mm]{./F1_M_PP_M2007_page3_images/image001.eps}
\end{center}
8  6

$\beta$

{\it A}  $\alpha$  {\it B}
\begin{center}
\includegraphics[width=123.900mm,height=17.628mm]{./F1_M_PP_M2007_page3_images/image002.eps}
\end{center}
Wypelnia

egzaminator!

Nr czynności

Maks. liczba kt

3.1.

1

3.2.

1

3.3.

1

3.4.

1

Uzyskana liczba pkt





{\it Egzamin maturalny z matematyki}

{\it Poziom podstawowy}

{\it 5}

Zadanie 4. (5pkt)

Samochód przebył w pewnym czasie 210 km. Gdybyjechał ze średnią prędkością o 10 km/h

większib to czas przejazdu skróciłby się o pół godziny. Oblicz, z jaką średnią prędkością

jechał ten samochód.
\begin{center}
\includegraphics[width=137.820mm,height=17.580mm]{./F1_M_PP_M2007_page4_images/image001.eps}
\end{center}
Wypelnia

egzaminator!

Nr czynno\S ci

Maks. liczba kt

4.1.

1

4.2.

1

4.3.

1

4.4.

1

4.5.

1

Uzyskana liczba pkt





{\it 6}

{\it Egzamin maturalny z matematyki}

{\it Poziom podstawowy}

Zadanie 5. (5pkt)

Dany jest ciąg arytmetyczny $(a_{n})$, gdzie $n\geq 1$. Wiadomo, $\dot{\mathrm{z}}\mathrm{e}$ dla $\mathrm{k}\mathrm{a}\dot{\mathrm{z}}$ dego $n\geq 1$

$n$ początkowych wyrazów $S_{n}=a_{1}+a_{2}+\ldots+a_{n}$ wyraza się wzorem: $S_{n}=-n^{2}+13n.$

a) Wyznacz wzór na $n-\mathrm{t}\mathrm{y}$ wyraz ciągu $(a_{n}).$

b) Oblicz a200$7^{\cdot}$

c) Wyznacz liczbę $n$, dla której $a_{n}=0.$

suma
\begin{center}
\includegraphics[width=137.868mm,height=17.580mm]{./F1_M_PP_M2007_page5_images/image001.eps}
\end{center}
Nr czynności

Wypelnia Maks. liczba kt

egzaminator! Uzyskana liczba pkt

5.1.

5.2.

1

5.3.

1

5.4.

1

5.5.

1





{\it Egzamin maturalny z matematyki}

{\it Poziom podstawowy}

7

Zadanie 6. (4pkt)

Dany jest wielomian $W(x)=2x^{3}+ax^{2}-14x+b.$

a) Dla $a=0 \mathrm{i} b=0$ otrzymamy wielomian $W(x)=2x^{3}-14x$. Rozwiąz równanie

$2x^{3}-14x=0.$

b) Dobierz wartości $a\mathrm{i}b$ tak, aby wielomian $W(x)$ był podzielny jednocześnie przez $x-2$

oraz przez $x+3.$
\begin{center}
\includegraphics[width=123.900mm,height=17.580mm]{./F1_M_PP_M2007_page6_images/image001.eps}
\end{center}
Nr czynności

Wypelnia Maks. liczba kt

egzamÍnator! Uzyskana liczba pkt

1

1

1





{\it 8}

{\it Egzamin maturalny z matematyki}

{\it Poziom podstawowy}

Zadanie 7. (5pkt)

Dany jest punkt $C=(2,3)$ i prosta o równaniu $y=2x-8$ będąca symetralną odcinka $BC.$

Wyznacz współrzędne punktu $B$. Wykonaj obliczenia uzasadniające odpowiedz.
\begin{center}
\includegraphics[width=137.868mm,height=17.628mm]{./F1_M_PP_M2007_page7_images/image001.eps}
\end{center}
Nr czynnoŚci

Wypelnia Maks. liczba kt

egzaminator! Uzyskana liczba pkt

7.1.

7.2.

1

7.3.

1

7.4.

1

7.5.

1





{\it Egzamin maturalny z matematyki}

{\it Poziom podstawowy}

{\it 9}

Zadanie 8. (4pkt)

Na stole $\mathrm{l}\mathrm{e}\dot{\mathrm{z}}$ ało 14 banknotów: 2 banknoty o nomina1e 100 zł, 2 banknoty o nomina1e 50 zł

$\mathrm{i} 10$ banknotów o nominale 20 zł. Wiatr zdmuchnął na podłogę 5 banknotów. Ob1icz

prawdopodobieństwo tego, $\dot{\mathrm{z}}\mathrm{e}$ na podłodze lezy dokładnie 130 zł. Odpowied $\acute{\mathrm{z}}$ podaj w postaci

ułamka nieskracalnego.
\begin{center}
\includegraphics[width=123.900mm,height=17.628mm]{./F1_M_PP_M2007_page8_images/image001.eps}
\end{center}
Nr czynnoŚci

Wypelnia Maks. liczba kt

egzaminator! Uzyskana liczba pkt

8.1.

1

8.2.

8.3.

1

8.4.

1





$ 1\theta$

{\it Egzamin maturalny z matematyki}

{\it Poziom podstawowy}

Zadanie 9. (6pkt)

Oblicz pole czworokąta wypukłego ABCD, w którym kąty wewnętrzne mają odpowiednio

miary: $4A=90^{\circ}, \triangleleft B=75^{\circ}, \triangleleft C=60^{\circ}, \triangleleft D=135^{\circ}$, a boki AB $\mathrm{i}$ AD mają długość 3 cm.

Sporządzí rysunek pomocniczy.







{\it ARKUSZ ZA WIERA INFORMACJE} $PRA$ {\it WNIE CHRONIONE}

{\it DO MOMENTU ROZPOCZĘCIA EGZAMINU}.$\displaystyle \int$
\begin{center}
\begin{tabular}{|l|l|l}
\cline{1-1}
\multicolumn{1}{|l|}{$\begin{array}{l}\mbox{Miejsce}	\\	\mbox{na na ejkę}	\end{array}$}&	\multicolumn{1}{|l|}{}&	\multicolumn{1}{|l}{ $\mathrm{M}\mathrm{M}\mathrm{A}-\mathrm{P}1_{-}1\mathrm{P}-082$}	\\
\hline
&	\multicolumn{1}{|l}{$\begin{array}{l}\mbox{MAJ}	\\	\mbox{ROK 2008}	\\	\mbox{Za rozwiązanie}	\\	\mbox{wszystkich zadań}	\\	\mbox{mozna otrzymać}	\\	\mbox{łącznie}	\\	\mbox{50 punktów}	\end{array}$}	\\
\cline{3-3}
&	\multicolumn{1}{|l}{$\begin{array}{l}\mbox{KOD}	\\	\mbox{ZDAJACEGO}	\end{array}$}
\end{tabular}


\includegraphics[width=21.840mm,height=9.852mm]{./F1_M_PP_M2008_page0_images/image001.eps}

\includegraphics[width=78.792mm,height=13.356mm]{./F1_M_PP_M2008_page0_images/image002.eps}
\end{center}



{\it 2 Egzamin maturalny z matematyki}

{\it Poziom podstawowy}

Zadanie l. $(4pkt)$

Na ponizszym rysunku przedstawiono łamaną ABCD, którajest wykresem ffinkcji $y=f(x).$
\begin{center}
\includegraphics[width=108.108mm,height=107.952mm]{./F1_M_PP_M2008_page1_images/image001.eps}
\end{center}
{\it y}

{\it C  D}

3

1

$-3 -2  -1  0 1_{1} 2_{1}$ 3  $1_{1} 2_{1}$  4  {\it x}

1

$-2$

{\it A  B}  $-4$

Korzystając z tego wykresu:

a) zapisz w postaci przedziału zbiór wartości funkcji $f,$

b) podaj wartość funkcji $f$ dla argumentu $x=1-\sqrt{10},$

c) wyznacz równanie prostej $BC,$

d) oblicz długość odcinka $BC.$





{\it Egzamin maturalny z matematyki ll}

{\it Poziom podstawowy}
\begin{center}
\includegraphics[width=123.900mm,height=17.832mm]{./F1_M_PP_M2008_page10_images/image001.eps}
\end{center}
Nr zadania

Wypelnia Maks. liczba kt

egzaminator! Uzyskana liczba pkt

7.1

1

7.2

1

7.3

1

7.4

1





{\it 12 Egzamin maturalny z matematyki}

{\it Poziom podstawowy}

Zadanie 8. $(4pkt)$

Dany jest wielomian $W(x)=x^{3}-5x^{2}-9x+45.$

a) Sprawdzí, czy punkt $A=(1$, 30$)$ nalezy do wykresu tego wielomianu.

b) Zapisz wielomian $W$ w postaci iloczynu trzech wielomianów stopnia pierwszego.
\begin{center}
\includegraphics[width=123.948mm,height=17.784mm]{./F1_M_PP_M2008_page11_images/image001.eps}
\end{center}
Nr zadania

Wypelnia Maks. liczba kt

egzaminator! Uzyskana lÍczba pkt

8.1

1

8.2

1

8.3

1

8.4

1





{\it Egzamin maturalny z matematyki 13}

{\it Poziom podstawowy}

Zadanie 9. (5pkt)

Oblicz najmniejszą i

w przedziale $\langle-2, 2\rangle.$

największą wartość

ffinkcji kwadratowej

$f(x)=(2x+1)(x-2)$
\begin{center}
\includegraphics[width=137.928mm,height=17.832mm]{./F1_M_PP_M2008_page12_images/image001.eps}
\end{center}
Wypelnia

egzaminator!

Nr zadania

Maks. liczba kt

1

1

Uzyskana liczba pkt





{\it 14 Egzamin maturalny z matematyki}

{\it Poziom podstawowy}

Zadanie 10. $(3pkt)$

Rysunek przedstawia fragment wykresu funkcji $h$, określonej wzorem $h(x)=\displaystyle \frac{a}{x}$ dla $x\neq 0.$

Wiadomo, $\dot{\mathrm{z}}\mathrm{e}$ do wykresu ffinkcji $h$ nalezy punkt $P=(2,5).$

a) Oblicz wartość współczynnika $a.$

b) Ustal, czy liczba $h(\pi)-h(-\pi)$ jest dodatnia czy ujemna.

c) Rozwiąz nierówność $h(x)>5.$
\begin{center}
\includegraphics[width=140.664mm,height=112.824mm]{./F1_M_PP_M2008_page13_images/image001.eps}
\end{center}
{\it y}

1

1  {\it x}





{\it Egzamin maturalny z matematyki 15}

{\it Poziom podstawowy}
\begin{center}
\includegraphics[width=109.980mm,height=17.832mm]{./F1_M_PP_M2008_page14_images/image001.eps}
\end{center}
Nr zadania

Wypelnia Maks. liczba kt

egzaminator! Uzyskana liczba pkt

10.1

10.2

10.3

1





{\it 16 Egzamin maturalny z matematyki}

{\it Poziom podstawowy}

Zadanie ll. $(5pkt)$

Pole powierzchni bocznej ostrosłupa prawidłowego trójkątnego równa się $\displaystyle \frac{a^{2}\sqrt{15}}{4}$, gdzie

$a$ oznacza długość krawędzi podstawy tego ostrosłupa. Zaznacz na ponizszym rysunku kąt

nachylenia ściany bocznej ostrosłupa do płaszczyzny jego podstawy. Miarę tego kąta oznacz

symbolem $\beta$. Oblicz $\cos\beta$ i korzystając z tablic funkcji trygonometrycznych odczytaj

przyblizoną wartość $\beta$ z dokładnością do $1^{\mathrm{o}}$





{\it Egzamin maturalny z matematyki 17}

{\it Poziom podstawowy}
\begin{center}
\includegraphics[width=137.928mm,height=17.832mm]{./F1_M_PP_M2008_page16_images/image001.eps}
\end{center}
Wypelnia

egzaminator!

Nr zadania

Maks. liczba kt

1

11.2

1

11.3

1

11.4

1

11.5

Uzyskana liczba pkt





{\it 18 Egzamin maturalny z matematyki}

{\it Poziom podstawowy}

Zadanie 12. $(4pkt)$

Rzucamy dwa razy symetryczną sześcienną kostką do gry. Oblicz prawdopodobieństwo

$\mathrm{k}\mathrm{a}\dot{\mathrm{z}}$ dego z następujących zdarzeń:

a) $A-\mathrm{w}\mathrm{k}\mathrm{a}\dot{\mathrm{z}}$ dym rzucie wypadnie nieparzysta liczba oczek.

b) $B-$ suma oczek otrzymanych w obu rzutachjest liczbą większą od 9.

c) $C-$ suma oczek otrzymanych w obu rzutachjest liczbą nieparzystą i większą od 9.
\begin{center}
\includegraphics[width=123.948mm,height=17.832mm]{./F1_M_PP_M2008_page17_images/image001.eps}
\end{center}
Wypelnia

egzaminator!

Nr zadania

Maks. liczba kt

12.1

1

12.2

1

12.3

1

12.4

1

Uzyskana liczba pkt





{\it Egzamin maturalny z matematyki 19}

{\it Poziom podstawowy}

BRUDNOPIS





{\it Egzamin maturalny z matematyki 3}

{\it Poziom podstawowy}
\begin{center}
\includegraphics[width=123.900mm,height=17.832mm]{./F1_M_PP_M2008_page2_images/image001.eps}
\end{center}
Nr zadania

Wypelnia Maks. liczba kt

egzaminator! Uzyskana liczba pkt

1.1

1

1.2

1

1.3

1

1.4

1





{\it 4 Egzamin maturalny z matematyki}

{\it Poziom podstawowy}

Zadanie 2. (4pkt)

Liczba przekątnych wielokąta wypukłego, w którymjest $n$ boków i $n\geq 3$ wyraza się wzorem

$P(n)=\displaystyle \frac{n(n-3)}{2}.$

Wykorzystując ten wzór:

a) oblicz liczbę przekątnych w dwudziestokącie wypukłym.

b) oblicz, ile boków ma wielokąt wypukły, w którym liczba przekątnych jest pięć razy

większa od liczby boków.

c) sprawd $\acute{\mathrm{z}}$, czy jest prawdziwe następujące stwierdzenie:

{\it Kazdy wielokqt wypukly o parzystej liczbie boków ma parzystq liczbę przekqtnych}.

Odpowied $\acute{\mathrm{z}}$ uzasadnij.
\begin{center}
\includegraphics[width=123.948mm,height=17.784mm]{./F1_M_PP_M2008_page3_images/image001.eps}
\end{center}
Nr zadania

Wypelnia Maks. liczba kt

egzamÍnator! Uzyskana lÍczba pkt

2.1

1

2.2

1

2.3

1

2.4

1





{\it Egzamin maturalny z matematyki 5}

{\it Poziom podstawowy}

Zadanie 3. $(4pkt)$

Rozwiąz. równanie $4^{23}x-32^{9}x=16^{4}\cdot(4^{4})^{4}$

Zapisz rozwiązanie tego równania w postaci $2^{k}$, gdzie kjest liczbą całkowitą.
\begin{center}
\includegraphics[width=123.900mm,height=17.784mm]{./F1_M_PP_M2008_page4_images/image001.eps}
\end{center}
Nr zadania

Wypelnia Maks. liczba kt

egzaminator! Uzyskana liczba pkt

3.1

1

3.2

1

3.3

1

3.4

1





{\it 6 Egzamin maturalny z matematyki}

{\it Poziom podstawowy}

Zadanie 4. (3pkt)

Koncetn paliwowy podnosił dwukrotnie w jednym tygodniu cenę benzyny, pierwszy raz

010\%, a drugi raz o 5\%. Po obu tych podwyzkachjeden litr benzyny, wyprodukowanej przez

ten koncern, kosztuje 4,62 zł. Ob1icz cenę jednego 1itra benzyny przed omawianymi

podwyzkami.
\begin{center}
\includegraphics[width=109.932mm,height=17.832mm]{./F1_M_PP_M2008_page5_images/image001.eps}
\end{center}
Wypelnia

egzaminator!

Nr zadania

Maks. liczba kt

4.2

4.3

1

Uzyskana liczba pkt





{\it Egzamin maturalny z matematyki 7}

{\it Poziom podstawowy}

Zadanie 5. $(5pkt)$

Nieskończony ciąg liczbowy $(a_{n})$ jest określony wzorem $a_{n}=2-\displaystyle \frac{1}{n}, n=1$, 2, 3,$\ldots.$

a) Oblicz, ile wyrazów ciągu $(a_{n})$ jest mniejszych od 1,975.

b) Dla pewnej liczby $x$ trzywyrazowy ciąg $(a_{2},a_{7},x)$ jest arytmetyczny. Oblicz $x.$
\begin{center}
\includegraphics[width=137.928mm,height=17.784mm]{./F1_M_PP_M2008_page6_images/image001.eps}
\end{center}
Nr zadania

Wypelnia Maks. liczba $\mathrm{k}\iota$

egzaminator! Uzyskana lÍczba pkt

5.1

1

5.2

1

5.3

1

5.4

5.5

1





{\it 8 Egzamin maturalny z matematyki}

{\it Poziom podstawowy}

Zadanie 6. $(5pkt)$

Prosta o równaniu $5x+4y-10=0$ przecina oś $Ox$ układu współrzędnych w punkcie $A$ oraz

oś $Oy$ w punkcie $B$. Oblicz współrzędne wszystkich punktów $C$ lez$\cdot$ących na osi $Ox$ i takich,

$\dot{\mathrm{z}}\mathrm{e}$ trójkąt $ABC$ ma pole równe 35.





{\it Egzamin maturalny z matematyki 9}

{\it Poziom podstawowy}
\begin{center}
\includegraphics[width=137.928mm,height=17.832mm]{./F1_M_PP_M2008_page8_images/image001.eps}
\end{center}
Wypelnia

egzaminator!

Nr zadania

Maks. liczba kt

1

1

1

1

Uzyskana liczba pkt





$ 1\theta$ {\it Egzamin maturalny z matematyki}

{\it Poziom podstawowy}

Zadanie 7. $(4pkt)$

Dany jest trapez, w którym podstawy mają długość 4 cm i 10 cm oraz ramiona tworzą

z dłuzsząpodstawą kąty o miarach $30^{\mathrm{o}}$ i $45^{\mathrm{o}}$. Oblicz wysokość tego trapezu.







{\it ARKUSZ ZA WIERA INFORMACJE} $PRA$ {\it WNIE CHRONIONE}

{\it DO MOMENTU ROZPOCZĘCIA EGZAMINU}.$\displaystyle \int$
\begin{center}
\includegraphics[width=192.024mm,height=288.084mm]{./F1_M_PP_M2009_page0_images/image001.eps}
\end{center}
Miejsce

na na ejkę

MMA-PI IP-092

EGZAMIN MATURALNY

MAJ

Z MATEMATYKI

POZIOM PODSTAWOWY

Czas pracy 120 minut

Instrukcja dla zdającego

1.

2.

3.

4.

5.

6.

7.

8.

9.

Sprawd $\acute{\mathrm{z}}$, czy arkusz egzaminacyjny zawiera 16 stron (zadania

$1-11)$. Ewentualny brak zgłoś przewodniczącemu zespołu

nadzorującego egzamin.

Rozwiązania zadań i odpowiedzi zamieść w miejscu na to

przeznaczonym.

W rozwiązaniach zadań przedstaw tok rozumowania

prowadzący do ostatecznego wyniku.

Pisz czytelnie. Uzywaj $\mathrm{d}$ gopisu pióra tylko z czatnym

tusze atramentem.

Nie uzywaj korektora, a błędne zapisy prze eśl.

Pamiętaj, $\dot{\mathrm{z}}\mathrm{e}$ zapisy w brudnopisie nie podlegają ocenie.

Obok $\mathrm{k}\mathrm{a}\dot{\mathrm{z}}$ dego zadania podanajest maksymalna liczba punktów,

którą $\mathrm{m}\mathrm{o}\dot{\mathrm{z}}$ esz uzyskać zajego poprawne rozwiązanie.

$\mathrm{M}\mathrm{o}\dot{\mathrm{z}}$ esz korzystać z zesta wzorów matematycznych, cyrkla

i linijki oraz kalkulatora.

Na karcie odpowiedzi wpisz swoją datę urodzenia i PESEL.

Nie wpisuj $\dot{\mathrm{z}}$ adnych znaków w części przeznaczonej

dla egzaminatora.

Za rozwiązanie

wszystkich zadań

mozna otrzymać

łącznie

50 punktów

{\it Zyczymy} $pow\theta dzenia'$

Wypelnia zdający

rzed roz oczęciem racy

PESEL ZDAJACEGO

KOD

ZDAJACEGO




{\it 2}

{\it Egzamin maturalny z matematyki}

{\it Poziom podstawowy}

Zadanie l. $(5pkt)$

Funkcja $f$ określona jest wzorem $f(x)=$

a) Uzupełnij tabelę:

dla $x<2$

dla $2\leq x\leq 4$
\begin{center}
\begin{tabular}{|l|l|l|l|}
\hline
\multicolumn{1}{|l|}{$x$}&	\multicolumn{1}{|l|}{ $-3$}&	\multicolumn{1}{|l|}{ $3$}&	\multicolumn{1}{|l|}{}	\\
\hline
\multicolumn{1}{|l|}{ $f(x)$}&	\multicolumn{1}{|l|}{}&	\multicolumn{1}{|l|}{}&	\multicolumn{1}{|l|}{ $0$}	\\
\hline
\end{tabular}

\end{center}
b) Narysuj wykres funkcji $f.$

c) Podaj wszystkie liczby całkowite $x$, spełniające nierówność $f(x)\geq-6.$
\begin{center}
\includegraphics[width=137.868mm,height=17.832mm]{./F1_M_PP_M2009_page1_images/image001.eps}
\end{center}
Nr zadania

Wypelnia Maks. liczba kt

egzaminator! Uzyskana liczba pkt

1.1

1

1

1.3

1

1.4

1

1.5





{\it Egzamin maturalny z matematyki}

{\it Poziom podstawowy}

{\it 11}
\begin{center}
\includegraphics[width=192.276mm,height=290.784mm]{./F1_M_PP_M2009_page10_images/image001.eps}

\includegraphics[width=123.900mm,height=17.784mm]{./F1_M_PP_M2009_page10_images/image002.eps}
\end{center}
Nr zadania

Wypelnia Maks. liczba kt

egzamÍnator! Uzyskana liczba pkt

8.1

1

8.2

1

8.3

1

8.4

1





{\it 12}

{\it Egzamin maturalny z matematyki}

{\it Poziom podstawowy}

Zadanie 9. $(4pkt)$

Punkty $B=(0,10) \mathrm{i} O=(0,0)$ są wierzchołkami trójkąta prostokątnego $OAB$, w którym

$|\neq OAB|=90^{\mathrm{o}}$ Przyprostokątna $OA$ zawiera się w prostej o równaniu

współrzędne punktu $A$ i długość przyprostokątnej $OA.$

$y=\displaystyle \frac{1}{2}x$. Oblicz
\begin{center}
\includegraphics[width=192.228mm,height=254.460mm]{./F1_M_PP_M2009_page11_images/image001.eps}

\includegraphics[width=123.948mm,height=17.784mm]{./F1_M_PP_M2009_page11_images/image002.eps}
\end{center}
Nr zadania

Wypelnia Maks. liczba kt

egzamÍnator! Uzyskana lÍczba pkt

1

1

1

1





{\it Egzamin maturalny z matematyki}

{\it Poziom podstawowy}

{\it 13}

Zadanie 10. $(5pkt)$

Tabela przedstawia wyniki części teoretycznej egzaminu na prawo jazdy. Zdający uzyskał

wynik pozytywny, $\mathrm{j}\mathrm{e}\dot{\mathrm{z}}$ eli popełnił co najwyzej dwa błędy.
\begin{center}
\begin{tabular}{|l|l|l|l|l|l|l|l|l|l|}
\hline
\multicolumn{1}{|l|}{liczba błędów}&	\multicolumn{1}{|l|}{$0$}&	\multicolumn{1}{|l|}{ $1$}&	\multicolumn{1}{|l|}{ $2$}&	\multicolumn{1}{|l|}{ $3$}&	\multicolumn{1}{|l|}{ $4$}&	\multicolumn{1}{|l|}{ $5$}&	\multicolumn{1}{|l|}{ $6$}&	\multicolumn{1}{|l|}{ $7$}&	\multicolumn{1}{|l|}{ $8$}	\\
\hline
\multicolumn{1}{|l|}{liczba zdających}&	\multicolumn{1}{|l|}{$8$}&	\multicolumn{1}{|l|}{ $5$}&	\multicolumn{1}{|l|}{ $8$}&	\multicolumn{1}{|l|}{ $5$}&	\multicolumn{1}{|l|}{ $2$}&	\multicolumn{1}{|l|}{ $1$}&	\multicolumn{1}{|l|}{ $0$}&	\multicolumn{1}{|l|}{ $0$}&	\multicolumn{1}{|l|}{ $1$}	\\
\hline
\end{tabular}

\end{center}
a) Oblicz średnią arytmetyczną liczby błędów popełnionych przez zdających ten egzamin.

Wynik podaj w zaokrągleniu do całości.

b) Oblicz prawdopodobieństwo, $\dot{\mathrm{z}}\mathrm{e}$ wśród dwóch losowo wybranych zdających tylko jeden

uzyskał wynik pozytywny. Wynik zapisz w postaci ułamka zwykłego nieskracalnego.
\begin{center}
\includegraphics[width=192.276mm,height=218.136mm]{./F1_M_PP_M2009_page12_images/image001.eps}

\includegraphics[width=137.928mm,height=17.832mm]{./F1_M_PP_M2009_page12_images/image002.eps}
\end{center}
Nr zadania

Wypelnia Maks. liczba kt

egzaminator! Uzyskana liczba pkt

10.1

1

10.2

1

10.3

1

10.4

1

10.5





{\it 14}

{\it Egzamin maturalny z matematyki}

{\it Poziom podstawowy}

Zadanie ll. $(5pkt)$

Powierzchnia boczna walca po rozwinięciu na płaszczyznę jest prostokątem. Przekątna tego

prostokąta ma długość 12 i tworzy z bokiem, którego długość jest równa wysokości wa1ca,

kąt o mierze $30^{\circ}$

a) Oblicz pole powierzchni bocznej tego walca.

b) Sprawdzí, czy objętość tego walcajest większa od $18\sqrt{3}$. Odpowiedzí uzasadnij.
\begin{center}
\includegraphics[width=192.228mm,height=272.640mm]{./F1_M_PP_M2009_page13_images/image001.eps}
\end{center}




{\it Egzamin maturalny z matematyki}

{\it Poziom podstawowy}

{\it 15}
\begin{center}
\includegraphics[width=192.276mm,height=290.784mm]{./F1_M_PP_M2009_page14_images/image001.eps}

\includegraphics[width=137.928mm,height=17.784mm]{./F1_M_PP_M2009_page14_images/image002.eps}
\end{center}
Nr zadania

Wypelnia Maks. liczba kt

egzaminator! Uzyskana lÍczba pkt

11.1

1

11.2

1

11.3

11.4

11.5

1





{\it 16}

{\it Egzamin maturalny z matematyki}

{\it Poziom podstawowy}

BRUDNOPIS





{\it Egzamin maturalny z matematyki}

{\it Poziom podstawowy}

{\it 3}

Zadanie 2. $(3pkt)$

Dwaj rzemieślnicy przyjęli zlecenie wykonania wspólnie 980 deta1i. Zap1anowa1i, $\dot{\mathrm{z}}\mathrm{e}$

$\mathrm{k}\mathrm{a}\dot{\mathrm{z}}$ dego dnia pierwszy z nich wykona $m$, a drugi $n$ detali. Obliczyli, $\dot{\mathrm{z}}\mathrm{e}$ razem wykonają

zlecenie w ciągu 7 dni. Po pierwszym dniu pracy pierwszy z rzemieś1ników rozchorował się

i wtedy drugi, aby wykonać całe zlecenie, musiał pracować o 8 dni dłuzej $\mathrm{n}\mathrm{i}\dot{\mathrm{z}}$ planował, (nie

zmieniając liczby wykonywanych codziennie detali). Oblicz $m \mathrm{i} n.$
\begin{center}
\includegraphics[width=192.276mm,height=248.364mm]{./F1_M_PP_M2009_page2_images/image001.eps}

\includegraphics[width=109.980mm,height=17.784mm]{./F1_M_PP_M2009_page2_images/image002.eps}
\end{center}
Nr zadania

Wypelnia Maks. liczba kt

egzaminator! Uzyskana lÍczba pkt

2.1

2.2

1

2.3

1





{\it 4}

{\it Egzamin maturalny z matematyki}

{\it Poziom podstawowy}

Zadanie 3. $(5pkt)$

Wykres funkcji $f$ danej wzorem $f(x)=-2x^{2}$ przesunięto wzdłuz osi $Ox 0 3$ jednostki

w prawo oraz wzdłuz osi $Oy\mathrm{o}$ 8jednostek w górę, otrzymując wykres funkcji $g.$

a) Rozwiąz nierówność $f(x)+5<3x.$

b) Podaj zbiór wartości funkcji $g.$

c) Funkcja $g$ określonajest wzorem $g(x)=-2x^{2}+bx+c$. Oblicz $b\mathrm{i}c.$
\begin{center}
\includegraphics[width=192.228mm,height=260.508mm]{./F1_M_PP_M2009_page3_images/image001.eps}
\end{center}




{\it Egzamin maturalny z matematyki}

{\it Poziom podstawowy}

{\it 5}
\begin{center}
\includegraphics[width=192.276mm,height=290.784mm]{./F1_M_PP_M2009_page4_images/image001.eps}

\includegraphics[width=137.928mm,height=17.784mm]{./F1_M_PP_M2009_page4_images/image002.eps}
\end{center}
Nr zadanÍa

Wypelnia Maks. liczba kt

egzaminator! Uzyskana lÍczba pkt

1

3.2

1

3.3

3.4

3.5

1





{\it 6}

{\it Egzamin maturalny z matematyki}

{\it Poziom podstawowy}

Zadanie 4. $(3pkt)$

Wykaz, $\dot{\mathrm{z}}\mathrm{e}$ liczba $3^{54}$jest rozwiązaniem równania $243^{11}-81^{14}+7x=9^{27}$
\begin{center}
\includegraphics[width=109.932mm,height=17.832mm]{./F1_M_PP_M2009_page5_images/image001.eps}
\end{center}
Wypelnia

egzaminator!

Nr zadania

Maks. liczba kt

4.2

4.3

1

Uzyskana liczba pkt





{\it Egzamin maturalny z matematyki}

{\it Poziom podstawowy}

7

Zadanie 5. $(5pkt)$

Wielomian $W$ dany jest wzorem $W(x)=x^{3}+ax^{2}-4x+b.$

a) Wyznacz $a, b$ oraz $c$ tak, aby wielomian $W$ był równy wielomianowi $P$, gdy

$P(x)=x^{3}+(2a+3)x^{2}+(a+b+c)x-1.$

b) Dla $a=3 \mathrm{i} b=0$ zapisz wielomian $W$ w postaci iloczynu trzech wielomianów stopnia

pierwszego.
\begin{center}
\includegraphics[width=137.928mm,height=17.832mm]{./F1_M_PP_M2009_page6_images/image001.eps}
\end{center}
Nr zadania

Wypelnia Maks. liczba kt

egzaminator! Uzyskana liczba pkt

5.1

1

5.2

1

5.3

1

5.4

1

5.5





{\it 8}

{\it Egzamin maturalny z matematyki}

{\it Poziom podstawowy}

Zadanie 6. $(5pkt)$

Miarajednego z kątów ostrych w trójkącie prostokątnymjest równa $\alpha.$

a) Uzasadnij, ze spełnionajest nierówność $\sin\alpha-\mathrm{t}\mathrm{g}\alpha<0.$

b) Dla $\displaystyle \sin\alpha=\frac{\mathrm{z}\sqrt{2}}{3}$ oblicz wartość wyrazenia $\cos^{3}\alpha+\cos\alpha\cdot\sin^{2}\alpha.$
\begin{center}
\includegraphics[width=192.228mm,height=254.460mm]{./F1_M_PP_M2009_page7_images/image001.eps}

\includegraphics[width=137.868mm,height=17.832mm]{./F1_M_PP_M2009_page7_images/image002.eps}
\end{center}
Nr zadania

Wypelnia Maks. liczba kt

egzaminator! Uzyskana liczba pkt

1

1





{\it Egzamin maturalny z matematyki}

{\it Poziom podstawowy}

{\it 9}

Zadanie 7. $(6pkt)$

Dany jest ciąg arytmetyczny $(a_{n})$ dla $n\geq 1$, w którym $a_{7}=1, a_{11}=9.$

a) Oblicz pierwszy wyraz $a_{1}$ i róznicę $r$ ciągu $(a_{n}).$

b) Sprawdzí, czy ciąg $(a_{7},a_{8},a_{11})$ jest geometryczny.

c) Wyznacz takie $n$, aby suma $n$ początkowych wyrazów ciągu

najmniejszą.

$(a_{n})$

miała wartość
\begin{center}
\includegraphics[width=192.276mm,height=242.316mm]{./F1_M_PP_M2009_page8_images/image001.eps}

\includegraphics[width=151.836mm,height=17.784mm]{./F1_M_PP_M2009_page8_images/image002.eps}
\end{center}
Wypelnia

egzaminator!

Nr zadania

Maks. liczba kt

7.1

1

7.2

1

7.3

1

7.4

1

7.5

1

1

Uzyskana liczba pkt





$ 1\theta$

{\it Egzamin maturalny z matematyki}

{\it Poziom podstawowy}

Zadanie 8. (4pkt)

W trapezie ABCD długość podstawy CD jest równa 18, a długości ramion trapezu AD iBC

są odpowiednio równe 25 i 15. Kąty ADBi DCB, zaznaczone na rysunku, mają równe miary.

Oblicz obwód tego trapezu.
\begin{center}
\includegraphics[width=136.704mm,height=55.680mm]{./F1_M_PP_M2009_page9_images/image001.eps}
\end{center}
{\it D  C}

{\it A  B}







Centralna Komisja Egzaminacyjna

Arkusz zawiera informacje prawnie chronione do momentu rozpoczęcia egzaminu.

WPISUJE ZDAJACY

KOD PESEL

{\it Miejsce}

{\it na naklejkę}

{\it z kodem}
\begin{center}
\includegraphics[width=21.432mm,height=9.852mm]{./F1_M_PP_M2010_page0_images/image001.eps}

\includegraphics[width=82.092mm,height=9.852mm]{./F1_M_PP_M2010_page0_images/image002.eps}

\includegraphics[width=204.060mm,height=216.048mm]{./F1_M_PP_M2010_page0_images/image003.eps}
\end{center}
EGZAMIN MATU

Z MATEMATY

LNY

POZIOM PODSTAWOWY  MAJ 2010

l. Sprawdzí, czy arkusz egzaminacyjny zawiera 20 stron

(zadania $1-34$). Ewentualny brak zgłoś przewodniczącemu

zespo nadzo jącego egzamin.

2. Rozwiązania zadań i odpowiedzi wpisuj w miejscu na to

przeznaczonym.

3. Odpowiedzi do zadań za iętych (l-25) przenieś

na ka ę odpowiedzi, zaznaczając je w części ka $\mathrm{y}$

przeznaczonej dla zdającego. Zamaluj $\blacksquare$ pola do tego

przeznaczone. Błędne zaznaczenie otocz kółkiem \fcircle$\bullet$

i zaznacz właściwe.

4. Pamiętaj, $\dot{\mathrm{z}}\mathrm{e}$ pominięcie argumentacji lub istotnych

obliczeń w rozwiązaniu zadania otwa ego (26-34) $\mathrm{m}\mathrm{o}\dot{\mathrm{z}}\mathrm{e}$

spowodować, $\dot{\mathrm{z}}\mathrm{e}$ za to rozwiązanie nie będziesz mógł

dostać pełnej liczby punktów.

5. Pisz czytelnie i $\mathrm{u}\dot{\mathrm{z}}$ aj tvlko $\mathrm{d}$ gopisu lub -Dióra

z czamym tuszem lub atramentem.

6. Nie uzywaj korektora, a błędne zapisy wyra $\acute{\mathrm{z}}\mathrm{n}\mathrm{i}\mathrm{e}$ prze eśl.

7. Pamiętaj, $\dot{\mathrm{z}}\mathrm{e}$ zapisy w brudnopisie nie będą oceniane.

8. $\mathrm{M}\mathrm{o}\dot{\mathrm{z}}$ esz korzystać z zesta wzorów matematycznych,

cyrkla i linijki oraz kalkulatora.

9. Na karcie odpowiedzi wpisz swój numer PESEL i przyklej

naklejkę z kodem.

10. Nie wpisuj $\dot{\mathrm{z}}$ adnych znaków w części przeznaczonej dla

egzaminatora.

Czas pracy:

170 minut

Liczba punktów

do uzyskania: 50

$\Vert\Vert\Vert\Vert\Vert\Vert\Vert\Vert\Vert\Vert\Vert\Vert\Vert\Vert\Vert\Vert\Vert\Vert\Vert\Vert\Vert\Vert\Vert\Vert|  \mathrm{M}\mathrm{M}\mathrm{A}-\mathrm{P}1_{-}1\mathrm{P}-102$




{\it 2}

{\it Egzamin maturalny z matematyki}

{\it Poziom podstawowy}

ZADANIA ZAMKNIĘTE

{\it Wzadaniach} $\theta d1.$ {\it do 25. wybierz i zaznacz na karcie odpowiedzipoprawnq odpowied} $\acute{z}.$

Zadanie l. $(1pkt)$

Wskaz rysunek, na którymjest przedstawiony zbiór rozwiązań nierówności $|x+7|>5.$
\begin{center}
\includegraphics[width=173.280mm,height=13.212mm]{./F1_M_PP_M2010_page1_images/image001.eps}
\end{center}
$-12$  2  {\it x}

A.
\begin{center}
\includegraphics[width=175.008mm,height=13.812mm]{./F1_M_PP_M2010_page1_images/image002.eps}
\end{center}
2  12  {\it x}

B.
\begin{center}
\includegraphics[width=173.280mm,height=13.260mm]{./F1_M_PP_M2010_page1_images/image003.eps}
\end{center}
$-12  -2$  {\it x}

C.
\begin{center}
\includegraphics[width=171.756mm,height=13.104mm]{./F1_M_PP_M2010_page1_images/image004.eps}
\end{center}
$-2$  12  {\it x}

D.

Zadanie 2. (1pkt)

Spodnie po obnizce ceny o 30\% kosztują 126 zł. I1e kosztowały spodnie przed obnizką?

A. 163,80 zł

B. 180 zł

C. 294 zł

D. 420 zł

Zadanie 3. $(1pkt)$

Liczba $(\displaystyle \frac{2^{-2}\cdot 3^{-1}}{2^{-1}3^{-2}})^{0}$ jest równa

A. I B. 4

C. 9

D. 36

Zadanie 4. (1pkt)

Liczba $\log_{4}8+\log_{4}2$ jest równa

A. l

B. 2

C. $\log_{4}6$

D. log410

Zadanie 5. $(1pkt)$

Dane są wielomiany $W(x)=-2x^{3}+5x^{2}-3$ oraz $P(x)=2x^{3}+12x$. Wielomian $W(x)+P(x)$

jest równy

A. $5x^{2}+12x-3$

B. $4x^{3}+5x^{2}+12x-3$

C. $4x^{6}+5x^{2}+12x-3$

D. $4x^{3}+12x^{2}-3$





{\it Egzamin maturalny z matematyki}

{\it Poziom podstawowy}

{\it 11}

Zadanie 28. (2pkt)

Trójkąty prostokątne równoramienne $ABC\mathrm{i}CDE$ są połozone tak, jak na ponizszym rysunku

(w obu trójkątach kąt przy wierzchołku $C$ jest prosty). Wykaz, $\dot{\mathrm{z}}\mathrm{e}|AD|=|BE|.$

{\it C}
\begin{center}
\includegraphics[width=76.404mm,height=36.072mm]{./F1_M_PP_M2010_page10_images/image001.eps}
\end{center}
{\it E}

{\it D}

{\it A  B}
\begin{center}
\includegraphics[width=109.980mm,height=17.832mm]{./F1_M_PP_M2010_page10_images/image002.eps}
\end{center}
Nr zadani,`

Wypelnia Maks. liczba kt

egzaminator

Uzyskana liczba pkt

2

27.

2

28.

2





{\it 12}

{\it Egzamin maturalny z matematyki}

{\it Poziom podstawowy}

Zadanie 29. $(2pkt)$

Kąt $\alpha$ jest ostry i $\displaystyle \mathrm{t}\mathrm{g}\alpha=\frac{5}{12}$. Oblicz $\cos\alpha.$

Odpowied $\acute{\mathrm{z}}$:

Zadanie 30. $(2pkt)$

Wyka $\dot{\mathrm{z}}$, ze jeśli $a>0$, to $\displaystyle \frac{a^{2}+1}{a+1}\geq\frac{a+1}{2}.$





{\it Egzamin maturalny z matematyki}

{\it Poziom podstawowy}

{\it 13}

Zadanie 31. (2pkt)

W trapezie prostokątnym krótsza przekątna dzieli go na trójkąt prostokątny

równoboczny. Dłuzsza podstawa trapezujest równa 6. Ob1icz obwód tego trapezu.

i trójkąt

Odpowiedzí :
\begin{center}
\includegraphics[width=109.980mm,height=17.784mm]{./F1_M_PP_M2010_page12_images/image001.eps}
\end{center}
Nr zadania

Wypelnia Maks. liczba kt

egzaminator

Uzyskana lÍczba pkt

2

30.

2

31.

2





{\it 14}

{\it Egzamin maturalny z matematyki}

{\it Poziom podstawowy}

Zadanie 32. $(4pkt)$

Podstawą ostrosłupa ABCD jest trójkąt $ABC$. Krawędzí AD jest wysokością ostrosłupa (zobacz

rysunek). Oblicz objętość ostrostupa ABCD, jeśli wiadomo, $\dot{\mathrm{z}}\mathrm{e} |AD|=12, |BC|=6,$

$|BD|=|CD|=13.$

{\it D}





{\it Egzamin maturalny z matematyki}

{\it Poziom podstawowy}

{\it 15}

Odpowiedzí :
\begin{center}
\includegraphics[width=82.044mm,height=17.832mm]{./F1_M_PP_M2010_page14_images/image001.eps}
\end{center}
Wypelnia

egzaminator

Nr zadania

Maks. liczba kt

32.

4

Uzyskana liczba pkt





{\it 16}

{\it Egzamin maturalny z matematyki}

{\it Poziom podstawowy}

Zadanie 33. $(4pkt)$

Doświadczenie losowe polega na dwukrotnym rzucie symetryczną sześcienną kostką do gry.

Oblicz prawdopodobieństwo zdarzenia $A$ polegającego na tym, $\dot{\mathrm{z}}\mathrm{e}$ w pierwszym rzucie

otrzymamy parzystą liczbę oczek i iloczyn liczb oczek w obu rzutach będzie podzielny przez 12.

Wynik przedstaw w postaci ułamka zwykłego nieskracalnego.





{\it Egzamin maturalny z matematyki}

{\it Poziom podstawowy}

{\it 1}7

Odpowiedzí :
\begin{center}
\includegraphics[width=82.044mm,height=17.784mm]{./F1_M_PP_M2010_page16_images/image001.eps}
\end{center}
Wypelnia

egzaminator

Nr zadania

Maks. lÍczba kt

33.

4

Uzyskana lÍczba pkt





{\it 18}

{\it Egzamin maturalny z matematyki}

{\it Poziom podstawowy}

Zadanie 34. $(5pkt)$

$\mathrm{W}$ dwóch hotelach wybudowano prostokątne baseny. Basen w pierwszym hotelu

ma powierzchnię 240 $\mathrm{m}^{2}$ Basen w drugim hotelu ma powierzchnię 350 $\mathrm{m}^{2}$ oraz jest o 5 $\mathrm{m}$

dłuzszy i 2 $\mathrm{m}$ szerszy $\mathrm{n}\mathrm{i}\dot{\mathrm{z}}$ w pierwszym hotelu. Oblicz, jakie wymiary mogą mieć baseny

w obu hotelach. Podaj wszystkie $\mathrm{m}\mathrm{o}\dot{\mathrm{z}}$ liwe odpowiedzi.





{\it Egzamin maturalny z matematyki}

{\it Poziom podstawowy}

{\it 19}

Odpowiedzí :
\begin{center}
\includegraphics[width=82.044mm,height=17.832mm]{./F1_M_PP_M2010_page18_images/image001.eps}
\end{center}
Wypelnia

egzaminator

Nr zadania

Maks. liczba kt

34.

5

Uzyskana liczba pkt





$ 2\theta$

{\it Egzamin maturalny z matematyki}

{\it Poziom podstawowy}

BRUDNOPIS





{\it Egzamin maturalny z matematyki}

{\it Poziom podstawowy}

{\it 3}

BRUDNOPIS





{\it 4}

{\it Egzamin maturalny z matematyki}

{\it Poziom podstawowy}

Zadanie 6. $(1pkt)$

Rozwiązaniem równania $\displaystyle \frac{3x-1}{7x+1}=\frac{2}{5}$ jest

A. 1 B. -73

C.

-47

D. 7

Zadanie 7. $(1pkt)$

Do zbioru rozwiązań nierównoŚci $(x-2)(x+3)<0$ nalezy liczba

A. 9 B. 7 C. 4

D. l

Zadanie 8. $(1pkt)$

Wykresem funkcji kwadratowej $f(x)=-3x^{2}+3$ jest parabola o wierzchołku w punkcie

A. $($3, $0)$ B. $(0,3)$ C. $(-3,0)$ D. $(0,-3)$

Zadanie 9. $(1pkt)$

Prosta o równaniu $y=-2x+(3m+3)$ przecina w układzie współrzędnych oś $Oy$ w punkcie

(0,2). Wtedy

A. {\it m}$=$--23

B.

{\it m}$=$- -31

C.

{\it m}$=$ -31

D.

{\it m}$=$ -35

Zadanie 10. $(1pkt)$

Na rysunku jest przedstawiony wykres funkcji $y=f(x).$
\begin{center}
\begin{tabular}{|l|l|l|l|l|l|l|l|l|l|l|l|l|l|l|}
\hline
\multicolumn{1}{|l|}{}&	\multicolumn{1}{|l|}{}&	\multicolumn{1}{|l|}{}&	\multicolumn{1}{|l|}{$y$}&	\multicolumn{1}{|l|}{}&	\multicolumn{1}{|l|}{}&	\multicolumn{1}{|l|}{}&	\multicolumn{1}{|l|}{}&	\multicolumn{1}{|l|}{}&	\multicolumn{1}{|l|}{}&	\multicolumn{1}{|l|}{}&	\multicolumn{1}{|l|}{}&	\multicolumn{1}{|l|}{}&	\multicolumn{1}{|l|}{}&	\multicolumn{1}{|l|}{}	\\
\hline
\multicolumn{1}{|l|}{}&	\multicolumn{1}{|l|}{}&	\multicolumn{1}{|l|}{}&	\multicolumn{1}{|l|}{}&	\multicolumn{1}{|l|}{}&	\multicolumn{1}{|l|}{}&	\multicolumn{1}{|l|}{}&	\multicolumn{1}{|l|}{}&	\multicolumn{1}{|l|}{}&	\multicolumn{1}{|l|}{}&	\multicolumn{1}{|l|}{}&	\multicolumn{1}{|l|}{}&	\multicolumn{1}{|l|}{}&	\multicolumn{1}{|l|}{}&	\multicolumn{1}{|l|}{}	\\
\hline
\multicolumn{1}{|l|}{}&	\multicolumn{1}{|l|}{}&	\multicolumn{1}{|l|}{}&	\multicolumn{1}{|l|}{}&	\multicolumn{1}{|l|}{}&	\multicolumn{1}{|l|}{}&	\multicolumn{1}{|l|}{}&	\multicolumn{1}{|l|}{}&	\multicolumn{1}{|l|}{}&	\multicolumn{1}{|l|}{}&	\multicolumn{1}{|l|}{}&	\multicolumn{1}{|l|}{}&	\multicolumn{1}{|l|}{}&	\multicolumn{1}{|l|}{}&	\multicolumn{1}{|l|}{}	\\
\hline
\multicolumn{1}{|l|}{}&	\multicolumn{1}{|l|}{}&	\multicolumn{1}{|l|}{}&	\multicolumn{1}{|l|}{}&	\multicolumn{1}{|l|}{}&	\multicolumn{1}{|l|}{}&	\multicolumn{1}{|l|}{}&	\multicolumn{1}{|l|}{}&	\multicolumn{1}{|l|}{}&	\multicolumn{1}{|l|}{}&	\multicolumn{1}{|l|}{}&	\multicolumn{1}{|l|}{}&	\multicolumn{1}{|l|}{}&	\multicolumn{1}{|l|}{}&	\multicolumn{1}{|l|}{}	\\
\hline
\multicolumn{1}{|l|}{}&	\multicolumn{1}{|l|}{}&	\multicolumn{1}{|l|}{}&	\multicolumn{1}{|l|}{}&	\multicolumn{1}{|l|}{}&	\multicolumn{1}{|l|}{}&	\multicolumn{1}{|l|}{}&	\multicolumn{1}{|l|}{}&	\multicolumn{1}{|l|}{}&	\multicolumn{1}{|l|}{}&	\multicolumn{1}{|l|}{}&	\multicolumn{1}{|l|}{}&	\multicolumn{1}{|l|}{}&	\multicolumn{1}{|l|}{}&	\multicolumn{1}{|l|}{}	\\
\hline
\multicolumn{1}{|l|}{}&	\multicolumn{1}{|l|}{}&	\multicolumn{1}{|l|}{}&	\multicolumn{1}{|l|}{}&	\multicolumn{1}{|l|}{}&	\multicolumn{1}{|l|}{}&	\multicolumn{1}{|l|}{}&	\multicolumn{1}{|l|}{}&	\multicolumn{1}{|l|}{}&	\multicolumn{1}{|l|}{}&	\multicolumn{1}{|l|}{}&	\multicolumn{1}{|l|}{}&	\multicolumn{1}{|l|}{}&	\multicolumn{1}{|l|}{}&	\multicolumn{1}{|l|}{}	\\
\hline
\multicolumn{1}{|l|}{}&	\multicolumn{1}{|l|}{}&	\multicolumn{1}{|l|}{}&	\multicolumn{1}{|l|}{}&	\multicolumn{1}{|l|}{}&	\multicolumn{1}{|l|}{}&	\multicolumn{1}{|l|}{}&	\multicolumn{1}{|l|}{}&	\multicolumn{1}{|l|}{}&	\multicolumn{1}{|l|}{}&	\multicolumn{1}{|l|}{}&	\multicolumn{1}{|l|}{}&	\multicolumn{1}{|l|}{}&	\multicolumn{1}{|l|}{}&	\multicolumn{1}{|l|}{}	\\
\hline
\multicolumn{1}{|l|}{}&	\multicolumn{1}{|l|}{}&	\multicolumn{1}{|l|}{}&	\multicolumn{1}{|l|}{}&	\multicolumn{1}{|l|}{}&	\multicolumn{1}{|l|}{}&	\multicolumn{1}{|l|}{}&	\multicolumn{1}{|l|}{}&	\multicolumn{1}{|l|}{}&	\multicolumn{1}{|l|}{}&	\multicolumn{1}{|l|}{}&	\multicolumn{1}{|l|}{}&	\multicolumn{1}{|l|}{}&	\multicolumn{1}{|l|}{}&	\multicolumn{1}{|l|}{}	\\
\hline
\multicolumn{1}{|l|}{}&	\multicolumn{1}{|l|}{}&	\multicolumn{1}{|l|}{}&	\multicolumn{1}{|l|}{}&	\multicolumn{1}{|l|}{}&	\multicolumn{1}{|l|}{}&	\multicolumn{1}{|l|}{}&	\multicolumn{1}{|l|}{}&	\multicolumn{1}{|l|}{}&	\multicolumn{1}{|l|}{}&	\multicolumn{1}{|l|}{}&	\multicolumn{1}{|l|}{}&	\multicolumn{1}{|l|}{}&	\multicolumn{1}{|l|}{}&	\multicolumn{1}{|l|}{ $x$}	\\
\hline
\multicolumn{1}{|l|}{}&	\multicolumn{1}{|l|}{}&	\multicolumn{1}{|l|}{ $0$}&	\multicolumn{1}{|l|}{}&	\multicolumn{1}{|l|}{}&	\multicolumn{1}{|l|}{}&	\multicolumn{1}{|l|}{}&	\multicolumn{1}{|l|}{}&	\multicolumn{1}{|l|}{}&	\multicolumn{1}{|l|}{}&	\multicolumn{1}{|l|}{}&	\multicolumn{1}{|l|}{}&	\multicolumn{1}{|l|}{ $1$}&	\multicolumn{1}{|l|}{O 1}&	\multicolumn{1}{|l|}{$1$}	\\
\hline
\multicolumn{1}{|l|}{}&	\multicolumn{1}{|l|}{}&	\multicolumn{1}{|l|}{}&	\multicolumn{1}{|l|}{}&	\multicolumn{1}{|l|}{}&	\multicolumn{1}{|l|}{}&	\multicolumn{1}{|l|}{}&	\multicolumn{1}{|l|}{}&	\multicolumn{1}{|l|}{}&	\multicolumn{1}{|l|}{}&	\multicolumn{1}{|l|}{}&	\multicolumn{1}{|l|}{}&	\multicolumn{1}{|l|}{}&	\multicolumn{1}{|l|}{}&	\multicolumn{1}{|l|}{}	\\
\hline
\end{tabular}

\end{center}
Które równanie ma dokładnie trzy rozwiązania?

A. $f(x)=0$

B. $f(x)=1$

C. $f(x)=2$

D. $f(x)=3$

Zadanie ll. $(1pkt)$

$\mathrm{W}$ ciągu arytmetycznym $(a_{n})$ dane są: $a_{3}=13\mathrm{i}a_{5}=39$. Wtedy wyraz $a_{1}$ jest równy

A. 13

B. 0

C. $-13$

D. $-26$

Zadanie 12. $(1pkt)$

$\mathrm{W}$ ciągu geometrycznym $(a_{n})$ dane są: $a_{1}=3\mathrm{i}a_{4}=24$. Iloraz tego ciągujest równy

A. 8 B. 2 C. -81 D. --21





{\it Egzamin maturalny z matematyki}

{\it Poziom podstawowy}

{\it 5}

BRUDNOPIS





{\it 6}

{\it Egzamin maturalny z matematyki}

{\it Poziom podstawowy}

Zadanie 13. (1pkt)

Liczba przekątnych siedmiokąta foremnegojest równa

A. 7

B. 14

C. 21

D. 28

Zadanie 14. $(1pkt)$

Kąt $\alpha$ jest ostry i $\displaystyle \sin\alpha=\frac{3}{4}$. Wartość wyrazenia $ 2-\cos^{2}\alpha$ jest równa

A. --2165 B. -23 C. --1176 D.

$\displaystyle \frac{31}{16}$

Zadanie 15. (1pkt)

Okrąg opisany na kwadracie ma promień 4. Długość boku tego kwadratujest równa

A. $4\sqrt{2}$

B. $2\sqrt{2}$

C. 8

D. 4

Zadanie 16. (1pkt)

Podstawa trójkąta równoramiennego ma długość 6, a ramię ma długość 5.

opuszczona na podstawę ma długość

Wysokość

A. 3

B. 4

C. $\sqrt{34}$

D. $\sqrt{61}$

Zadanie 17. (1pkt)

Odcinki AB i DE są równoległe. Długości odcinków CD, DE i AB są odpowiednio równe

1, 3 i 9. Długość odcinka AD jest równa
\begin{center}
\includegraphics[width=90.624mm,height=47.652mm]{./F1_M_PP_M2010_page5_images/image001.eps}
\end{center}
{\it C}

1

{\it D E}

3

{\it A}  9  {\it B}

A. 2

B. 3

C. 5

D. 6

Zadanie 18. $(1pkt)$

Punkty $A, B, C$ lez$\cdot$ące na okręgu o środku $S$ są wierzchołkami trójkąta równobocznego. Miara

zaznaczonego na rysunku kąta środkowego $ASB$ jest równa
\begin{center}
\includegraphics[width=65.436mm,height=70.968mm]{./F1_M_PP_M2010_page5_images/image002.eps}
\end{center}
{\it C}

{\it S}

{\it A  B}

B. $90^{\mathrm{o}}$  C. $60^{\mathrm{o}}$

A. $120^{\mathrm{o}}$

D. $30^{\mathrm{o}}$





{\it Egzamin maturalny z matematyki}

{\it Poziom podstawowy}

7

BRUDNOPIS





{\it 8}

{\it Egzamin maturalny z matematyki}

{\it Poziom podstawowy}

Zadanie 19. (1pkt)

Latawiec ma wymiary podane na

zacieniowanego trójkątajest równa

rysunku. Powierzchnia

A. 3200 $\mathrm{c}\mathrm{m}^{2}$

B. 6400 $\mathrm{c}\mathrm{m}^{2}$
\begin{center}
\includegraphics[width=33.480mm,height=80.676mm]{./F1_M_PP_M2010_page7_images/image001.eps}
\end{center}
30

1600 $\mathrm{c}\mathrm{m}^{2}$

800 $\mathrm{c}\mathrm{m}^{2}$

C.

D.

Zadanie 20. $(1pkt)$

Współczynnik kierunkowy prostej równoległej do prostej o równaniu $y=-3x+5$ jest równy:

A.

- -31

B. $-3$

C.

-31

D. 3

Zadanie 21. (1pkt)

Wskaz równanie okręgu o promieniu 6.

A. $x^{2}+y^{2}=3$

B. $x^{2}+y^{2}=6$

C. $x^{2}+y^{2}=12$

D. $x^{2}+y^{2}=36$

Zadanie 22. $(1pkt)$

Punkty $A=(-5,2) \mathrm{i} B=(3,-2)$ są wierzchołkami trójkąta równobocznego $ABC$. Obwód

tego trójkątajest równy

A. 30

B. $4\sqrt{5}$

C. $12\sqrt{5}$

D. 36

Zadanie 23. $(1pkt)$

Pole powierzchni całkowitej prostopadłoŚcianu o wymiarach $5\times 3\times 4$ jest równe

A. 94

B. 60

C. 47

D. 20

Zadanie 24. (1pkt)

Ostrosłup ma 18 wierzchołków. Liczba wszystkich krawędzi tego ostrosłupajest równa

A. ll

B. 18

C. 27

D. 34

Zadanie 25. (1pkt)

Średnia arytmetyczna dziesięciu liczb x, 3, 1, 4, 1, 5, 1, 4, 1, 5jest równa 3. Wtedy

A. $x=2$

B. $x=3$

C. $x=4$

D. $x=5$





{\it Egzamin maturalny z matematyki}

{\it Poziom podstawowy}

{\it 9}

BRUDNOPIS





$ 1\theta$

{\it Egzamin maturalny z matematyki}

{\it Poziom podstawowy}

ZADANIA OTWARTE

{\it Rozwiqzania zadań o numerach od 26. do 34. nalezy zapisać w} $wyznacz\theta nych$ {\it miejscach}

{\it pod treściq zadania}.

Zadanie 26. $(2pkt)$

Rozwiąz nierówność $x^{2}-x-2\leq 0.$

Odpowied $\acute{\mathrm{z}}$:

Zadanie 27. $(2pkt)$

Rozwiąz równanie $x^{3}-7x^{2}-4x+28=0.$

Odpowied $\acute{\mathrm{z}}$:







Centralna Komisja Egzaminacyjna

Arkusz zawiera informacje prawnie chronione do momentu rozpoczęcia egzaminu.

WPISUJE ZDAJACY

KOD PESEL

{\it Miejsce}

{\it na naklejkę}

{\it z kodem}
\begin{center}
\includegraphics[width=21.432mm,height=9.804mm]{./F1_M_PP_M2011_page0_images/image001.eps}

\includegraphics[width=82.092mm,height=9.804mm]{./F1_M_PP_M2011_page0_images/image002.eps}

\includegraphics[width=204.060mm,height=216.048mm]{./F1_M_PP_M2011_page0_images/image003.eps}
\end{center}
EGZAMIN MATU

Z MATEMATY

LNY

POZIOM PODSTAWOWY  MAJ 2011

1. Sprawd $\acute{\mathrm{z}}$, czy arkusz egzaminacyjny zawiera 20 stron

(zadania $1-33$). Ewentualny brak zgłoś przewodniczącemu

zespo nadzorującego egzamin.

2. Rozwiązania zadań i odpowiedzi wpisuj w miejscu na to

przeznaczonym.

3. Odpowiedzi do zadań za iętych (l-23) przenieś

na ka ę odpowiedzi, zaznaczając je w części ka $\mathrm{y}$

przeznaczonej dla zdającego. Zamaluj $\blacksquare$ pola do tego

przeznaczone. Błędne zaznaczenie otocz kółkiem \fcircle$\bullet$

i zaznacz właściwe.

4. Pamiętaj, $\dot{\mathrm{z}}\mathrm{e}$ pominięcie argumentacji lub istotnych

obliczeń w rozwiązaniu zadania otwa ego (24-33) $\mathrm{m}\mathrm{o}\dot{\mathrm{z}}\mathrm{e}$

spowodować, $\dot{\mathrm{z}}\mathrm{e}$ za to rozwiązanie nie będziesz mógł

dostać pełnej liczby punktów.

5. Pisz czytelnie i $\mathrm{u}\dot{\mathrm{z}}$ aj tvlko $\mathrm{d}$ gopisu lub -Dióra

z czatnym tuszem lub atramentem.

6. Nie uzywaj korektora, a błędne zapisy wyrazínie prze eśl.

7. Pamiętaj, $\dot{\mathrm{z}}\mathrm{e}$ zapisy w brudnopisie nie będą oceniane.

8. $\mathrm{M}\mathrm{o}\dot{\mathrm{z}}$ esz korzystać z zestawu wzorów matematycznych,

cyrkla i linijki oraz kalkulatora.

9. Na karcie odpowiedzi wpisz swój numer PESEL i przyklej

naklejkę z kodem.

10. Nie wpisuj $\dot{\mathrm{z}}$ adnych znaków w części przeznaczonej dla

egzaminatora.

Czas pracy:

170 minut

Liczba punktów

do uzyskania: 50

$\Vert\Vert\Vert\Vert\Vert\Vert\Vert\Vert\Vert\Vert\Vert\Vert\Vert\Vert\Vert\Vert\Vert\Vert\Vert\Vert\Vert\Vert\Vert\Vert|  \mathrm{M}\mathrm{M}\mathrm{A}-\mathrm{P}1_{-}1\mathrm{P}-112$




{\it 2}

{\it Egzamin maturalny z matematyki}

{\it Poziom podstawowy}

ZADANIA ZAMKNIĘTE

{\it Wzadaniach} $\theta d1.$ {\it do 23. wybierz i zaznacz na karcie odpowiedzipoprawnq odpowied} $\acute{z}.$

Zadanie l. $(1pkt)$

Wska $\dot{\mathrm{z}}$ nierówność, którą spełnia liczba $\pi.$

A. $|x+1|>5$ B. $|x-1|<2$

C.

$|x+\displaystyle \frac{2}{3}|\leq 4$

D.

$|x-\displaystyle \frac{1}{3}|\geq 3$

Zadanie 2. (1pkt)

Pierwsza rata, która stanowi 9\% ceny roweru, jest równa 189zł. Rower kosztuje

A. 1701 zł.

B. 2100 zł.

C. 1890 zł.

D. 2091 zł.

Zadanie 3. $(1pkt)$

Wyrazenie $5a^{2}-10ab+15a$ jest równe iloczynowi

A. $5a^{2}(1-10b+3)$

B. $5a(a-2b+3)$

C. $5a(a-10b+15)$

D. $5(a-2b+3)$

Zadanie 4. (1pkt)

Układ równań 

A. $a=-1$

B. $a=0$

C. $a=2$

D. $a=3$

Zadanie 5. $(1pkt)$

Rozwiązanie równania $x(x+3)-49=x(x-4)$ nalezy do przedziału

A.

$(-\infty,3)$

B. $(10,+\infty)$

C. $(-5,-1)$

D. $(2,+\infty)$

Zadanie 6. $(1pkt)$

Najmniejszą liczbą całkowitą nalezącą do zbioru rozwiązań nierówności $\displaystyle \frac{3}{8}+\frac{x}{6}<\frac{5x}{12}$ jest

A. l

B. 2

C. $-1$

D. $-2$

Zadanie 7. $(1pkt)$

Wskaz, który zbiór przedstawiony na osi liczbowej jest zbiorem liczb spełniających

jednocześnie następujące nierówności: 3 $(x-1)(x-5)\leq 0 \mathrm{i} x>1.$
\begin{center}
\includegraphics[width=45.264mm,height=7.320mm]{./F1_M_PP_M2011_page1_images/image001.eps}
\end{center}
A.

B.

1

$\check{}$6

$\underline{x}$
\begin{center}
\includegraphics[width=36.168mm,height=7.320mm]{./F1_M_PP_M2011_page1_images/image002.eps}

\includegraphics[width=84.228mm,height=14.172mm]{./F1_M_PP_M2011_page1_images/image003.eps}
\end{center}
{\it x}

1 5

D.

$\underline{x}$

-$\check{}$5

C.

1





{\it Egzamin maturalny z matematyki}

{\it Poziom podstawowy}

{\it 11}

Zadanie 26. (2pkt)

Na rysunku przedstawiono wykres funkcjif.
\begin{center}
\includegraphics[width=154.836mm,height=87.372mm]{./F1_M_PP_M2011_page10_images/image001.eps}
\end{center}
Odczytaj z wykresu i zapisz:

a) zbiór wartości funkcjif,

b) przedział maksymalnej długości, w którym funkcja f jest malejąca.

Odpowied $\acute{\mathrm{z}}$:
\begin{center}
\includegraphics[width=109.980mm,height=17.784mm]{./F1_M_PP_M2011_page10_images/image002.eps}
\end{center}
Nr zadania

Wypelnia Maks. liczba kt

egzaminator

Uzyskana lÍczba pkt

24.

2

25.

2

2





{\it 12}

{\it Egzamin maturalny z matematyki}

{\it Poziom podstawowy}

Zadanie 27. $(2pkt)$

Liczby $x, y$, 19 w podanej kolejności tworzą ciąg arytmetyczny, przy czym $x+y=8.$

Oblicz $x\mathrm{i}y.$

Odpowied $\acute{\mathrm{z}}$:

Zadanie 28. $(2pkt)$

Kąt $\alpha$ jest ostry i $\displaystyle \frac{\sin\alpha}{\cos\alpha}+\frac{\cos\alpha}{\sin\alpha}=2$. Oblicz wartość wyrazenia $\sin\alpha\cdot\cos\alpha.$

Odpowiedzí:





{\it Egzamin maturalny z matematyki}

{\it Poziom podstawowy}

{\it 13}

Zadanie 29. $(2pkt)$

Dany jest czworokąt ABCD, w którym AB $\Vert$ CD. Na boku $BC$ wybrano taki punkt $E,$

$\dot{\mathrm{z}}\mathrm{e}|EC|=|CD|\mathrm{i}|EB|=|BA|$. Wykaz$\cdot, \dot{\mathrm{z}}\mathrm{e}$ kąt $AED$ jest prosty.

Odpowiedzí :
\begin{center}
\includegraphics[width=109.980mm,height=17.832mm]{./F1_M_PP_M2011_page12_images/image001.eps}
\end{center}
Nr zadania

Wypelnia Maks. liczba kt

egzaminator

Uzyskana liczba pkt

27.

2

28.

2

2





{\it 14}

{\it Egzamin maturalny z matematyki}

{\it Poziom podstawowy}

Zadanie 30. (2pkt)

Ze zbioru liczb \{1, 2, 3 7\} 1osujemy ko1ejno dwa razy po jednej 1iczbie ze zwracaniem.

Oblicz prawdopodobieństwo wylosowania liczb, których sumajest podzielna przez 3.

Odpowied $\acute{\mathrm{z}}$:





{\it Egzamin maturalny z matematyki}

{\it Poziom podstawowy}

{\it 15}

Zadanie 31. $(4pkt)$

Okrąg o środku w punkcie $S=(3,7)$ jest styczny do prostej o równaniu $y=2x-3$. Oblicz

współrzędne punktu styczności.

Odpowied $\acute{\mathrm{z}}$:
\begin{center}
\includegraphics[width=96.012mm,height=17.784mm]{./F1_M_PP_M2011_page14_images/image001.eps}
\end{center}
WypelnÍa

egzaminator

Nr zadania

Maks. liczba kt

30.

2

31.

4

Uzyskana liczba pkt





{\it 16}

{\it Egzamin maturalny z matematyki}

{\it Poziom podstawowy}

Zadanie 32. $(5pkt)$

Pewien turysta pokonał trasę 112 km, przechodząc $\mathrm{k}\mathrm{a}\dot{\mathrm{z}}$ dego dnia tę samą liczbę kilometrów.

Gdyby mógł przeznaczyć na tę wędrówkę o 3 dni więcej, to w ciągu $\mathrm{k}\mathrm{a}\dot{\mathrm{z}}$ dego dnia mógłby

przechodzić o 12 km mniej. Ob1icz, i1e ki1ometrów dziennie przechodził ten turysta.





{\it Egzamin maturalny z matematyki}

{\it Poziom podstawowy}

17

Odpowiedzí :
\begin{center}
\includegraphics[width=82.044mm,height=17.832mm]{./F1_M_PP_M2011_page16_images/image001.eps}
\end{center}
Wypelnia

egzaminator

Nr zadania

Maks. liczba kt

32.

5

Uzyskana liczba pkt





{\it 18}

{\it Egzamin maturalny z matematyki}

{\it Poziom podstawowy}

Zadanie 33. (4pkt)

Punkty K, L iM są środkami krawędzi BC, GHi AE szeŚcianu ABCDEFGH o krawędzi

długości l (zobacz rysunek). Oblicz pole trójkąta KLM.





{\it Egzamin maturalny z matematyki}

{\it Poziom podstawowy}

{\it 19}

Odpowiedzí :
\begin{center}
\includegraphics[width=82.044mm,height=17.832mm]{./F1_M_PP_M2011_page18_images/image001.eps}
\end{center}
Wypelnia

egzaminator

Nr zadania

Maks. liczba kt

33.

4

Uzyskana liczba pkt





$ 2\theta$

{\it Egzamin maturalny z matematyki}

{\it Poziom podstawowy}

BRUDNOPIS





{\it Egzamin maturalny z matematyki}

{\it Poziom podstawowy}

{\it 3}

BRUDNOPIS





$\blacksquare$

$\blacksquare$

$\Vert\Vert\Vert\Vert\Vert\Vert\Vert\Vert\Vert\Vert\Vert\Vert\Vert\Vert\Vert\Vert\Vert\Vert\Vert\Vert\Vert\Vert\Vert\Vert|$
\begin{center}
\includegraphics[width=79.452mm,height=15.804mm]{./F1_M_PP_M2011_page20_images/image001.eps}
\end{center}
PESEL

$\mathrm{M}\mathrm{M}\mathrm{A}-\mathrm{P}1_{-}1$ P-112

WYPELNIA ZDAJACY
\begin{center}
\begin{tabular}{|l|l|l|l|l|}
\cline{1-1}
\multicolumn{1}{|l|}{$\begin{array}{l}\mbox{Nr}	\\	\mbox{zad.}	\end{array}$}	\\
\cline{1-1}
\multicolumn{1}{|l|}{ $1$}&	\multicolumn{1}{|l|}{ $\fbox{$\mathrm{A}$}$}&	\multicolumn{1}{|l|}{ $\fbox{$\mathrm{B}$}$}&	\multicolumn{1}{|l|}{ $\underline{\mathrm{H}\mathrm{c}}-$}&	\multicolumn{1}{|l|}{ $\Gamma \mathrm{D}\lrcorner$}	\\
\hline
\multicolumn{1}{|l|}{ $2$}&	\multicolumn{1}{|l|}{ $\fbox{$\mathrm{A}$}$}&	\multicolumn{1}{|l|}{[‡L]}&	\multicolumn{1}{|l|}{$\fbox{$\mathrm{c}$}$}&	\multicolumn{1}{|l|}{ $\fbox{$\mathrm{D}$}$}	\\
\hline
\multicolumn{1}{|l|}{ $3$}&	\multicolumn{1}{|l|}{ $\fbox{$\mathrm{A}$}$}&	\multicolumn{1}{|l|}{ $\fbox{$\mathrm{B}$}$}&	\multicolumn{1}{|l|}{ $\underline{\mathrm{H}\mathrm{c}}-$}&	\multicolumn{1}{|l|}{ $\Gamma \mathrm{D}\lrcorner$}	\\
\hline
\multicolumn{1}{|l|}{ $4$}&	\multicolumn{1}{|l|}{ $\fbox{$\mathrm{A}$}$}&	\multicolumn{1}{|l|}{ $\fbox{$\mathrm{B}$}$}&	\multicolumn{1}{|l|}{ $\fbox{$\mathrm{c}$}$}&	\multicolumn{1}{|l|}{ $\fbox{$\mathrm{D}$}$}	\\
\hline
\multicolumn{1}{|l|}{ $5$}&	\multicolumn{1}{|l|}{ $\displaystyle \prod$}&	\multicolumn{1}{|l|}{ $\fbox{$\mathrm{B}$}$}&	\multicolumn{1}{|l|}{ $\fbox{$\mathrm{c}$}$}&	\multicolumn{1}{|l|}{ $\mathrm{g}$}	\\
\hline
\multicolumn{1}{|l|}{ $6$}&	\multicolumn{1}{|l|}{ $\fbox{$\mathrm{A}$}$}&	\multicolumn{1}{|l|}{ $\fbox{$\mathrm{B}$}$}&	\multicolumn{1}{|l|}{ $\fbox{$\mathrm{c}$}$}&	\multicolumn{1}{|l|}{ $\fbox{$\mathrm{D}$}$}	\\
\hline
\multicolumn{1}{|l|}{ $7$}&	\multicolumn{1}{|l|}{ $\fbox{$\mathrm{A}$}$}&	\multicolumn{1}{|l|}{ $\fbox{$\mathrm{B}$}$}&	\multicolumn{1}{|l|}{ $\fbox{$\mathrm{c}$}$}&	\multicolumn{1}{|l|}{ $\fbox{$\mathrm{D}$}$}	\\
\hline
\multicolumn{1}{|l|}{ $8$}&	\multicolumn{1}{|l|}{ $\cap$}&	\multicolumn{1}{|l|}{ $\fbox{$\mathrm{B}$}$}&	\multicolumn{1}{|l|}{ $\fbox{$\mathrm{c}$}$}&	\multicolumn{1}{|l|}{ $\fbox{$\mathrm{D}$}$}	\\
\hline
\multicolumn{1}{|l|}{ $9$}&	\multicolumn{1}{|l|}{ $\fbox{$\mathrm{A}$}$}&	\multicolumn{1}{|l|}{ $\fbox{$\mathrm{B}$}$}&	\multicolumn{1}{|l|}{ $\fbox{$\mathrm{c}$}$}&	\multicolumn{1}{|l|}{ $\fbox{$\mathrm{D}$}$}	\\
\hline
\multicolumn{1}{|l|}{ $10$}&	\multicolumn{1}{|l|}{ $\fbox{$\mathrm{A}$}$}&	\multicolumn{1}{|l|}{ $\fbox{$\mathrm{B}$}$}&	\multicolumn{1}{|l|}{ $\fbox{$\mathrm{c}$}$}&	\multicolumn{1}{|l|}{ $\fbox{$\mathrm{D}$}$}	\\
\hline
\multicolumn{1}{|l|}{ $11$}&	\multicolumn{1}{|l|}{ $\fbox{$\mathrm{A}$}$}&	\multicolumn{1}{|l|}{ $\fbox{$\mathrm{B}$}$}&	\multicolumn{1}{|l|}{ $\underline{\mathrm{H}\mathrm{c}}-$}&	\multicolumn{1}{|l|}{ $\fbox{$\mathrm{D}$}$}	\\
\hline
\multicolumn{1}{|l|}{ $12$}&	\multicolumn{1}{|l|}{ $\fbox{$\mathrm{A}$}$}&	\multicolumn{1}{|l|}{ $\fbox{$\mathrm{B}$}$}&	\multicolumn{1}{|l|}{ $\fbox{$\mathrm{c},$}$}&	\multicolumn{1}{|l|}{ $\fbox{$\mathrm{D}$}$}	\\
\hline
\multicolumn{1}{|l|}{ $13$}&	\multicolumn{1}{|l|}{ $\displaystyle \prod$}&	\multicolumn{1}{|l|}{ $\fbox{$\mathrm{B}$}$}&	\multicolumn{1}{|l|}{ $\fbox{$\mathrm{c}$}$}&	\multicolumn{1}{|l|}{ $\fbox{$\mathrm{D}$}$}	\\
\hline
\multicolumn{1}{|l|}{ $14$}&	\multicolumn{1}{|l|}{ $\fbox{$\mathrm{A}$}$}&	\multicolumn{1}{|l|}{[‡N]}&	\multicolumn{1}{|l|}{$\fbox{$\zeta \mathrm{i},$}$}&	\multicolumn{1}{|l|}{ $\fbox{$\mathrm{D}$}$}	\\
\hline
\multicolumn{1}{|l|}{ $15$}&	\multicolumn{1}{|l|}{ $\fbox{$\mathrm{A}$}$}&	\multicolumn{1}{|l|}{ $\fbox{$\mathrm{B}$}$}&	\multicolumn{1}{|l|}{ $\mathrm{H}$-c-}&	\multicolumn{1}{|l|}{$\fbox{$\mathrm{D}$}$}	\\
\hline
\multicolumn{1}{|l|}{ $16$}&	\multicolumn{1}{|l|}{ $\fbox{$\mathrm{A}$}$}&	\multicolumn{1}{|l|}{[‡N]}&	\multicolumn{1}{|l|}{$\fbox{$\mathrm{c}$}$}&	\multicolumn{1}{|l|}{ $\fbox{$\mathrm{D}$}$}	\\
\hline
\multicolumn{1}{|l|}{ $17$}&	\multicolumn{1}{|l|}{ $\fbox{$\mathrm{A}$}$}&	\multicolumn{1}{|l|}{ $\fbox{$\mathrm{B}$}$}&	\multicolumn{1}{|l|}{ $\underline{\mathrm{H}\mathrm{c}}-$}&	\multicolumn{1}{|l|}{ $\fbox{$\ulcorner)$}$}	\\
\hline
\multicolumn{1}{|l|}{ $18$}&	\multicolumn{1}{|l|}{ $\fbox{$\mathrm{A}$}$}&	\multicolumn{1}{|l|}{[‡N]}&	\multicolumn{1}{|l|}{$\fbox{$\mathrm{c}$}$}&	\multicolumn{1}{|l|}{ $\fbox{$\mathrm{D}$}$}	\\
\hline
\multicolumn{1}{|l|}{ $19$}&	\multicolumn{1}{|l|}{ $\fbox{$\mathrm{A}$}$}&	\multicolumn{1}{|l|}{ $\fbox{$\mathrm{B}$}$}&	\multicolumn{1}{|l|}{ $\fbox{$\mathrm{c}$}$}&	\multicolumn{1}{|l|}{ $\fbox{$\ulcorner)$}$}	\\
\hline
\multicolumn{1}{|l|}{ $20$}&	\multicolumn{1}{|l|}{ $\fbox{$\mathrm{A}$}$}&	\multicolumn{1}{|l|}{ $\fbox{$\mathrm{B}$}$}&	\multicolumn{1}{|l|}{ $\fbox{$\mathrm{c}$}$}&	\multicolumn{1}{|l|}{ $\fbox{$\mathrm{D}$}$}	\\
\hline
\multicolumn{1}{|l|}{ $21$}&	\multicolumn{1}{|l|}{ $\displaystyle \prod$}&	\multicolumn{1}{|l|}{ $\fbox{$\mathrm{B}$}$}&	\multicolumn{1}{|l|}{ $\fbox{$\mathrm{c}$}$}&	\multicolumn{1}{|l|}{ $\ulcorner \mathrm{D}\rfloor$}	\\
\hline
\multicolumn{1}{|l|}{ $22$}&	\multicolumn{1}{|l|}{ $\fbox{$\mathrm{A}$}$}&	\multicolumn{1}{|l|}{ $\fbox{$\mathrm{B}$}$}&	\multicolumn{1}{|l|}{ $\fbox{$\mathrm{c}$}$}&	\multicolumn{1}{|l|}{ $\fbox{$\mathrm{D}$}$}	\\
\hline
\multicolumn{1}{|l|}{ $23$}&	\multicolumn{1}{|l|}{ $\fbox{$\mathrm{A}$}$}&	\multicolumn{1}{|l|}{ $\fbox{$\mathrm{B}$}$}&	\multicolumn{1}{|l|}{ $\fbox{$\mathrm{c}$}$}&	\multicolumn{1}{|l|}{ $\fbox{$ 1\supset$}$}	\\
\hline
\end{tabular}

\end{center}
Miejsce na naKlej$\kappa$e

z rr PESE-

WYPELNIA EGZAMINATOR
\begin{center}
\begin{tabular}{|l|l|l|l|l|l|l|}
	\\
&	\multicolumn{1}{|l|}{$0$}&	\multicolumn{1}{|l|}{ $1$}&	\multicolumn{1}{|l|}{ $2$}&	\multicolumn{1}{|l|}{ $3$}&	\multicolumn{1}{|l|}{ $4$}&	\multicolumn{1}{|l|}{ $5$}	\\
\cline{2-7}
\multicolumn{1}{|l|}{ $24$}&	\multicolumn{1}{|l|}{ $\square $}&	\multicolumn{1}{|l|}{ $\square $}&	\multicolumn{1}{|l|}{ $\square $}&	\multicolumn{1}{|l|}{}&	\multicolumn{1}{|l|}{}&	\multicolumn{1}{|l|}{}	\\
\hline
\multicolumn{1}{|l|}{ $25$}&	\multicolumn{1}{|l|}{ $\square $}&	\multicolumn{1}{|l|}{ $\square $}&	\multicolumn{1}{|l|}{ $\square $}&	\multicolumn{1}{|l|}{}&	\multicolumn{1}{|l|}{}&	\multicolumn{1}{|l|}{}	\\
\hline
\multicolumn{1}{|l|}{ $26$}&	\multicolumn{1}{|l|}{ $\square $}&	\multicolumn{1}{|l|}{ $\square $}&	\multicolumn{1}{|l|}{ $\square $}&	\multicolumn{1}{|l|}{}&	\multicolumn{1}{|l|}{}&	\multicolumn{1}{|l|}{}	\\
\hline
\multicolumn{1}{|l|}{ $27$}&	\multicolumn{1}{|l|}{ $\square $}&	\multicolumn{1}{|l|}{ $\square $}&	\multicolumn{1}{|l|}{ $\square $}&	\multicolumn{1}{|l|}{}&	\multicolumn{1}{|l|}{}&	\multicolumn{1}{|l|}{}	\\
\hline
\multicolumn{1}{|l|}{ $28$}&	\multicolumn{1}{|l|}{ $\square $}&	\multicolumn{1}{|l|}{ $\square $}&	\multicolumn{1}{|l|}{ $\square $}&	\multicolumn{1}{|l|}{}&	\multicolumn{1}{|l|}{}&	\multicolumn{1}{|l|}{}	\\
\hline
\multicolumn{1}{|l|}{ $29$}&	\multicolumn{1}{|l|}{ $\square $}&	\multicolumn{1}{|l|}{ $\square $}&	\multicolumn{1}{|l|}{ $\square $}&	\multicolumn{1}{|l|}{}&	\multicolumn{1}{|l|}{}&	\multicolumn{1}{|l|}{}	\\
\hline
\multicolumn{1}{|l|}{ $30$}&	\multicolumn{1}{|l|}{ $\square $}&	\multicolumn{1}{|l|}{ $\square $}&	\multicolumn{1}{|l|}{ $\square $}&	\multicolumn{1}{|l|}{}&	\multicolumn{1}{|l|}{}&	\multicolumn{1}{|l|}{}	\\
\hline
\multicolumn{1}{|l|}{ $31$}&	\multicolumn{1}{|l|}{ $\square $}&	\multicolumn{1}{|l|}{ $\square $}&	\multicolumn{1}{|l|}{ $\square $}&	\multicolumn{1}{|l|}{ $\square $}&	\multicolumn{1}{|l|}{ $\square $}&	\multicolumn{1}{|l|}{}	\\
\hline
\multicolumn{1}{|l|}{ $32$}&	\multicolumn{1}{|l|}{ $\square $}&	\multicolumn{1}{|l|}{ $\square $}&	\multicolumn{1}{|l|}{ $\square $}&	\multicolumn{1}{|l|}{ $\square $}&	\multicolumn{1}{|l|}{ $\square $}&	\multicolumn{1}{|l|}{ $\square $}	\\
\hline
\multicolumn{1}{|l|}{ $33$}&	\multicolumn{1}{|l|}{ $\square $}&	\multicolumn{1}{|l|}{ $\square $}&	\multicolumn{1}{|l|}{ $\square $}&	\multicolumn{1}{|l|}{ $\square $}&	\multicolumn{1}{|l|}{ $\square $}&	\multicolumn{1}{|l|}{}	\\
\hline
\end{tabular}


\includegraphics[width=14.580mm,height=9.852mm]{./F1_M_PP_M2011_page20_images/image002.eps}
\end{center}
$\blacksquare$

$\blacksquare$

SUMA

PUNKTÓW

D \square  \square  \square  \square  \square  \square  \square  \square  \square  \square 

J

0 1 2 3 4 5 6 7 8 9

0 1 2 3 4 5 6 7 8 9

$\blacksquare$




\begin{center}
\includegraphics[width=73.152mm,height=11.028mm]{./F1_M_PP_M2011_page21_images/image001.eps}
\end{center}
KOD EGZAMINATORA

Czytelny podpis egzaminatora
\begin{center}
\includegraphics[width=21.840mm,height=9.852mm]{./F1_M_PP_M2011_page21_images/image002.eps}
\end{center}
KOD ZDAJACEGO





{\it 4}

{\it Egzamin maturalny z matematyki}

{\it Poziom podstawowy}

Zadanie 8. $(1pkt)$

Wyrazenie $\log_{4}(2x-1)$ jest określone dla wszystkich liczb $x$ spełniających warunek

A.

$x\displaystyle \leq\frac{1}{2}$

B.

$x>\displaystyle \frac{1}{2}$

C. $x\leq 0$

D. $x>0$

Zadanie 9. $(1pkt)$

Dane są funkcje liniowe $f(x)=x-2$ oraz $g(x)=x+4$ określone dla wszystkich liczb

rzeczywistych $x$. Wskaz, który z ponizszych wykresów jest wykresem funkcji

$h(x)=f(x)\cdot g(x).$
\begin{center}
\includegraphics[width=29.868mm,height=49.380mm]{./F1_M_PP_M2011_page3_images/image001.eps}
\end{center}
{\it y}

{\it x}

$-4$  2
\begin{center}
\includegraphics[width=30.024mm,height=49.380mm]{./F1_M_PP_M2011_page3_images/image002.eps}
\end{center}
{\it y}

$-2$

{\it x}

4
\begin{center}
\includegraphics[width=29.868mm,height=49.380mm]{./F1_M_PP_M2011_page3_images/image003.eps}
\end{center}
{\it y}

{\it x}

$-4$  2
\begin{center}
\includegraphics[width=29.868mm,height=49.380mm]{./F1_M_PP_M2011_page3_images/image004.eps}
\end{center}
{\it y}

$-2$

{\it X}

4

A.

B.

C.

D.

Zadanie 10 $(1pkt)$

Funkcja liniowa określona jest wzorem $f(x)=-\sqrt{2}x+4$. Miejscem zerowym tej funkcjijest

liczba

A. $-2\sqrt{2}$

B.

-$\sqrt{}$22

C.

- -$\sqrt{}$22

D. $2\sqrt{2}$

Zadanie ll. $(1pkt)$

Danyjest nieskończony ciąg geometryczny $(a_{n})$, w którym $a_{3}=1 \displaystyle \mathrm{i}a_{4}=\frac{2}{3}$. Wtedy

A. {\it a}1$=- 23$ B. {\it a}1$=- 49$ C. {\it a}1$=$-23 D. {\it a}1$=$-49

Zadanie 12. $(1pkt)$

Danyjest nieskończony rosnący ciąg arytmetyczny $(a_{n})$ o wyrazach dodatnich. Wtedy

A. $a_{4}+a_{7}=a_{10}$

B. $a_{4}+a_{6}=a_{3}+a_{8}$

C. $a_{2}+a_{9}=a_{3}+a_{8}$

D. $a_{5}+a_{7}=2a_{8}$

Zadanie 13. $(1pkt)$

Kąt $\alpha$ jest ostry i $\displaystyle \cos\alpha=\frac{5}{13}$. Wtedy

A. $\displaystyle \sin\alpha=\frac{12}{13}$ oraz $\displaystyle \mathrm{t}\mathrm{g}\alpha=\frac{12}{5}$

C. $\displaystyle \sin\alpha=\frac{12}{5}$ oraz $\displaystyle \mathrm{t}\mathrm{g}\alpha=\frac{12}{13}$

B. $\displaystyle \sin\alpha=\frac{12}{13}$ oraz $\displaystyle \mathrm{t}\mathrm{g}\alpha=\frac{5}{12}$

D. $\displaystyle \sin\alpha=\frac{5}{12}$ oraz $\displaystyle \mathrm{t}\mathrm{g}\alpha=\frac{12}{13}$





{\it Egzamin maturalny z matematyki}

{\it Poziom podstawowy}

{\it 5}

BRUDNOPIS





{\it 6}

{\it Egzamin maturalny z matematyki}

{\it Poziom podstawowy}

Zadanie 14. $(1pkt)$

Wartość wyrazenia $\displaystyle \frac{\sin^{2}38^{\mathrm{o}}+\cos^{2}38^{\mathrm{o}}-1}{\sin^{2}52^{\mathrm{o}}+\cos^{2}52^{\mathrm{o}}+1}$ jest równa

A.

-21

B. 0

C.

- -21

D. l

Zadanie 15. $(1pkt)$

$\mathrm{W}$ prostopadłoŚcianie ABCDEFGH mamy: $|AB|=5, |AD|=4, |AE|=3$. Który z odcinków

{\it AB}, $BG, GE, EB$ jest najdłuzszy?

A.

{\it AB}

B.

{\it BG}

C.

{\it GE}

{\it D. EB}

Zadanie 16. $(1pkt)$

Punkt $O$ jest środkiem okręgu. Kąt wpisany $\alpha$ ma miarę
\begin{center}
\includegraphics[width=66.348mm,height=60.912mm]{./F1_M_PP_M2011_page5_images/image001.eps}
\end{center}
{\it B}

$\alpha$

{\it A}

$160^{\mathrm{o}}$  {\it C}

{\it O}

A. $80^{\mathrm{o}}$

B. $100^{\mathrm{o}}$

C. $110^{\mathrm{o}}$

D. $120^{\mathrm{o}}$

Zadanie 17. $(1pkt)$

Wysokość rombu o boku długości 6 i kącie ostrym $60^{\mathrm{o}}$ jest równa

A. $3\sqrt{3}$

B. 3

C. $6\sqrt{3}$

D. 6

Zadanie 18. $(1pkt)$

Prosta $k$ ma równanie $y=2x-3$. Wskaz równanie prostej $l$ równoległej do prostej $k$

i przechodzącej przez punkt $D$ o współrzędnych $(-2,1).$

A. $y=-2x+3$

B. $y=2x+1$

C. $y=2x+5$

D. $y=-x+1$





{\it Egzamin maturalny z matematyki}

{\it Poziom podstawowy}

7

BRUDNOPIS





{\it 8}

{\it Egzamin maturalny z matematyki}

{\it Poziom podstawowy}

Zadanie 19. $(1pkt)$

Styczną do okręgu $(x-1)^{2}+y^{2}-4=0$ jest prosta o równaniu

A. $x=1$

B. $x=3$

C. $y=0$

D. $y=4$

Zadanie 20. (1pkt)

Pole powierzchni całkowitej sześcianu jest równe 54. Długość przekątnej tego sześcianu jest

równa

A. $\sqrt{6}$

B. 3

C. 9

D. $3\sqrt{3}$

Zadanie 21. (1pkt)

Objętość stozka o wysokości 8 i średnicy podstawy 12jest równa

A. $ 124\pi$

B. $ 96\pi$

C. $ 64\pi$

D. $ 32\pi$

Zadanie 22. (1pkt)

Rzucamy dwa razy symetryczną sześcienną kostką do gry. Prawdopodobieństwo otrzymania

sumy oczek równej trzy wynosi

A.

-61

B.

-91

C.

$\displaystyle \frac{1}{12}$

D.

$\displaystyle \frac{1}{18}$

Zadanie 23. (1pkt)

Uczniowie pewnej klasy zostali poproszeni o odpowiedzí na pytanie:,,Ile osób liczy twoja

rodzina?'' Wyniki przedstawiono w tabeli:
\begin{center}
\begin{tabular}{|l|l|}
\hline
\multicolumn{1}{|l|}{$\begin{array}{l}\mbox{Liczba osób}	\\	\mbox{w rodzinie}	\end{array}$}&	\multicolumn{1}{|l|}{$\begin{array}{l}\mbox{liczba}	\\	\mbox{uczniów}	\end{array}$}	\\
\hline
\multicolumn{1}{|l|}{ $3$}&	\multicolumn{1}{|l|}{ $6$}	\\
\hline
\multicolumn{1}{|l|}{ $4$}&	\multicolumn{1}{|l|}{ $12$}	\\
\hline
\multicolumn{1}{|l|}{ $x$}&	\multicolumn{1}{|l|}{ $2$}	\\
\hline
\end{tabular}

\end{center}
Średnia liczba osób w rodzinie dla uczniów tej klasyjest równa 4. Wtedy 1iczba x jest równa

A. 3

B. 4

C. 5

D. 7





{\it Egzamin maturalny z matematyki}

{\it Poziom podstawowy}

{\it 9}

BRUDNOPIS





$ 1\theta$

{\it Egzamin maturalny z matematyki}

{\it Poziom podstawowy}

ZADANIA OTWARTE

{\it Rozwiqzania zadań o numerach od 24. do 33. nalezy zapisać w} $wyznacz\theta nych$ {\it miejscach}

{\it pod treściq zadania}.

Zadanie 24. $(2pkt)$

Rozwiąz nierówność $3x^{2}-10x+3\leq 0.$

Odpowiedzí:

Zadanie 25. $(2pkt)$

Uzasadnij, $\dot{\mathrm{z}}\mathrm{e}\mathrm{j}\mathrm{e}\dot{\mathrm{z}}$ eli $a+b=1$

$\mathrm{i} a^{2}+b^{2}=7$, to $a^{4}+b^{4}=31.$







$1-$

$-1\cup 1$

$-\mapsto 1$

$\mathrm{r}--$

Centralna Komisja Egzaminacyjna

Arkusz zawiera informacje prawnie chronione do momentu rozpoczęcia egzaminu.

WPISUJE ZDAJACY

KOD PESEL

{\it Miejsce}

{\it na naklejkę}

{\it z kodem}
\begin{center}
\includegraphics[width=21.432mm,height=9.804mm]{./F1_M_PP_M2012_page0_images/image001.eps}

\includegraphics[width=82.092mm,height=9.804mm]{./F1_M_PP_M2012_page0_images/image002.eps}
\end{center}
\fbox{} dysleksja
\begin{center}
\includegraphics[width=204.060mm,height=216.048mm]{./F1_M_PP_M2012_page0_images/image003.eps}
\end{center}
EGZAMIN MATU LNY

Z MATEMATYKI

MAJ 2012

POZIOM PODSTAWOWY

1. Sprawd $\acute{\mathrm{z}}$, czy arkusz egzaminacyjny zawiera 18 stron

(zadania $1-34$). Ewentualny brak zgłoś przewodniczącemu

zespo nadzorującego egzamin.

2. Rozwiązania zadań i odpowiedzi wpisuj w miejscu na to

przeznaczonym.

3. Odpowiedzi do zadań za niętych (l-25) przenieś

na ka ę odpowiedzi, zaznaczając je w części ka $\mathrm{y}$

przeznaczonej dla zdającego. Zamaluj $\blacksquare$ pola do tego

przeznaczone. Błędne zaznaczenie otocz kółkiem \fcircle$\bullet$

i zaznacz właściwe.

4. Pamiętaj, $\dot{\mathrm{z}}\mathrm{e}$ pominięcie argumentacji lub istotnych

obliczeń w rozwiązaniu zadania otwa ego (26-34) $\mathrm{m}\mathrm{o}\dot{\mathrm{z}}\mathrm{e}$

spowodować, $\dot{\mathrm{z}}\mathrm{e}$ za to rozwiązanie nie będziesz mógł

dostać pełnej liczby punktów.

5. Pisz czytelnie i uzywaj tvlko długopisu lub -Dióra

z czarnym tuszem lub atramentem.

6. Nie uzywaj korektora, a błędne zapisy wyrazínie prze eśl.

7. Pamiętaj, $\dot{\mathrm{z}}\mathrm{e}$ zapisy w brudnopisie nie będą oceniane.

8. $\mathrm{M}\mathrm{o}\dot{\mathrm{z}}$ esz korzystać z zestawu wzorów matematycznych,

cyrkla i linijki oraz kalkulatora.

9. Na tej stronie oraz na karcie odpowiedzi wpisz swój

numer PESEL i przyklej naklejkę z kodem.

10. Nie wpisuj $\dot{\mathrm{z}}$ adnych znaków w części przeznaczonej

dla egzaminatora.

Czas pracy:

170 minut

Liczba punktów

do uzyskania: 50

$\Vert\Vert\Vert\Vert\Vert\Vert\Vert\Vert\Vert\Vert\Vert\Vert\Vert\Vert\Vert\Vert\Vert\Vert\Vert\Vert\Vert\Vert\Vert\Vert|  \mathrm{M}\mathrm{M}\mathrm{A}-\mathrm{P}1_{-}1\mathrm{P}-122$




{\it 2}

{\it Egzamin maturalny z matematyki}

{\it Poziom podstawowy}

ZADANIA ZAMKNIĘTE

{\it Wzadaniach} $\theta d1.$ {\it do 25. wybierz i zaznacz na karcie odpowiedzipoprawnq odpowied} $\acute{z}.$

Zadanie l. (lpkt)

Cenę nart obnizono o 20\%, a po miesiącu nową cenę obnizono o da1sze 30\%. W wyniku obu

obnizek cena nart zmniejszyła się o

A. 44\%

B. 50\%

C. 56\%

D. 60\%

Zadanie 2. $(1pkt)$

3

Liczba $\sqrt[3]{(-8)^{-1}}\cdot 16^{\overline{4}}$ jest równa

A. $-8$

B. $-4$

C. 2

D. 4

Zadanie 3. $(1pkt)$

Liczba $(3-\sqrt{2})^{2}+4(2-\sqrt{2})$ jest równa

A. $19-10\sqrt{2}$

B. $17-4\sqrt{2}$

C. $15+14\sqrt{2}$

D. $19+6\sqrt{2}$

Zadanie 4. $(1pkt)$

Iloczyn 2$\cdot\log_{1}9$ jest równy

-3

A. $-6$ B. $-4$

C. $-1$

D. l

Zadanie 5. $(1pkt)$

Wska $\dot{\mathrm{z}}$ liczbę, która spełnia równanie $|3x+1|=4x.$

A. $x=-1$

B. $x=1$

C. $x=2$

D. $x=-2$

Zadanie 6. $(1pkt)$

Liczby $x_{1}, x_{2}$ sąróz$\cdot$nymi rozwiązaniami równania $2x^{2}+3x-7=0$. Suma $x_{1}+x_{2}$ jest równa

A.

- -27

B.

- -47

C.

- -23

D.

- -43

Zadanie 7. $(1pkt)$

Miejscami zerowymi ffinkcji kwadratowej $y=-3(x-7\mathrm{X}x+2)$ są

A. $x=7, x=-2$

B. $x=-7, x=-2$

C. $x=7, x=2$

D. $x=-7, x=2$

Zadanie 8. $(1pkt)$

Funkcja liniowafjest określona wzorem $f(x)=ax+6$, gdzie $a>0$. Wówczas spełniony jest

warunek

A. $f(1)>1$

B. $f(2)=2$

C. $f(3)<3$

D. $f(4)=4$





{\it Egzamin maturalny z matematyki}

{\it Poziom podstawowy}

{\it 11}

Zadanie 28. $(2pkt)$

Liczby $x_{1}=-4 \mathrm{i} x_{2}=3$ są pierwiastkami

trzeci pierwiastek tego wielomianu.

Odpowiedzí :

wielomianu $W(x)=x^{3}+4x^{2}-9x-36$. Oblicz

Zadanie 29. $(2pkt)$

Wyznacz równanie symetralnej odcinka o końcach $A=(-2,2)\mathrm{i}B=(2,10).$

Odpowiedzí :
\begin{center}
\includegraphics[width=123.900mm,height=17.832mm]{./F1_M_PP_M2012_page10_images/image001.eps}
\end{center}
Nr zadania

Wypelnia Maks. liczba kt

egzaminator

Uzyskana liczba pkt

2

27.

2

28.

2

2





{\it 12}

{\it Egzamin maturalny z matematyki}

{\it Poziom podstawowy}

Zadanie 30. $(2pkt)$

$\mathrm{W}$ trójkącie $ABC$ poprowadzono dwusieczne kątów A $\mathrm{i}B$. Dwusieczne te przecinają się

w punkcie $P$. Uzasadnij, $\dot{\mathrm{z}}\mathrm{e}$ kąt $APB$ jest rozwarty.





{\it Egzamin maturalny z matematyki}

{\it Poziom podstawowy}

{\it 13}

Zadanie 31. (2pkt)

Ze zbioru liczb \{1,2,3,4,5,6,7\} 1osujemy dwa razy po jednej 1iczbie ze zwracaniem. Ob1icz

prawdopodobieństwo zdarzenia A, polegającego na wylosowaniu liczb, których iloczyn jest

podzielny przez 6.

Odpowied $\acute{\mathrm{z}}$:
\begin{center}
\includegraphics[width=95.964mm,height=17.784mm]{./F1_M_PP_M2012_page12_images/image001.eps}
\end{center}
Wypelnia

egzaminator

Nr zadania

Maks. liczba kt

30.

2

31.

2

Uzyskana liczba pkt





{\it 14}

{\it Egzamin maturalny z matematyki}

{\it Poziom podstawowy}

Zadanie 32. (4pkt)

Ciąg (9, x,19) jest arytmetyczny, a ciąg (x,42,y,z) jest geometryczny. Ob1icz x, y oraz z.

Odpowiedzí:





{\it Egzamin maturalny z matematyki}

{\it Poziom podstawowy}

{\it 15}

Zadanie 33. $(4pkt)$

$\mathrm{W}$ graniastosłupie prawidłowym czworokątnym ABCDEFGH przekątna $AC$ podstawy

ma długość 4. Kąt ACE jest równy $60^{\mathrm{o}}$. Oblicz objętość ostrosłupa ABCDE przedstawionego

na ponizszym rysunku.

Odpowiedzí :
\begin{center}
\includegraphics[width=95.964mm,height=17.784mm]{./F1_M_PP_M2012_page14_images/image001.eps}
\end{center}
Wypelnia

egzaminator

Nr zadania

Maks. liczba kt

32.

4

33.

4

Uzyskana liczba pkt





{\it 16}

{\it Egzamin maturalny z matematyki}

{\it Poziom podstawowy}

Zadanie 34. $(5pkt)$

Miasto $A$ i miasto $B$ łączy linia kolejowa długości 210 km. Średnia prędkość pociągu

pospiesznego na tej trasie jest o 24 $\mathrm{k}\mathrm{m}/\mathrm{h}$ większa od średniej prędkości pociągu osobowego.

Pociąg pospieszny pokonuje tę trasę o l godzinę krócej $\mathrm{n}\mathrm{i}\dot{\mathrm{z}}$ pociąg osobowy. Oblicz czas

pokonania tej drogi przez pociąg pospieszny.





{\it Egzamin maturalny z matematyki}

{\it Poziom podstawowy}

{\it 1}7

Odpowied $\acute{\mathrm{z}}$:
\begin{center}
\includegraphics[width=82.044mm,height=17.832mm]{./F1_M_PP_M2012_page16_images/image001.eps}
\end{center}
Wypelnia

egzaminator

Nr zadania

Maks. liczba kt

34.

5

Uzyskana liczba pkt





{\it 18}

{\it Egzamin maturalny z matematyki}

{\it Poziom podstawowy}

BRUDNOPIS





{\it Egzamin maturalny z matematyki}

{\it Poziom podstawowy}

{\it 3}

BRUDNOPIS





{\it 4}

{\it Egzamin maturalny z matematyki}

{\it Poziom podstawowy}

Zadanie 9. $(1pkt)$

Wskaz wykres funkcji, która w przedziale $\langle-4,4\rangle$ ma dokładniejedno miejsce zerowe.

A.
\begin{center}
\includegraphics[width=57.408mm,height=58.368mm]{./F1_M_PP_M2012_page3_images/image001.eps}
\end{center}
4

3

y

2

1

$-4$ -$3  -2$

$-1$

$-1$

1 2

$\mathrm{x}$

$3\backslash ^{4}$

$-2$

$-3$

$-4$

C.
\begin{center}
\includegraphics[width=58.668mm,height=58.980mm]{./F1_M_PP_M2012_page3_images/image002.eps}
\end{center}
y

3

2

1

x

$-4$ -$3  -2  -1$  1 2 3 4

$-2$

$-3$

Zadanie 10. $(1pkt)$

Liczba tg $30^{\mathrm{o}}-\sin 30^{\mathrm{o}}$ jest równa

A. $\sqrt{3}-1$

B.

- -$\sqrt{}$63

B.
\begin{center}
\includegraphics[width=57.144mm,height=58.164mm]{./F1_M_PP_M2012_page3_images/image003.eps}
\end{center}
4  y

2

1

$-4$ -$3  -2$

$-1$

$-1$

1 2  3 4

$-2$

$-3$

$-4$

D.
\begin{center}
\includegraphics[width=56.088mm,height=57.300mm]{./F1_M_PP_M2012_page3_images/image004.eps}
\end{center}
4  y

2

1

$-4  -2$

$-1$

$-1$

1 3  4

$-2$

$-3$

$-4$

C.

$\displaystyle \frac{\sqrt{3}-1}{6}$

D.

$\displaystyle \frac{2\sqrt{3}-3}{6}$

Zadanie ll. $(1pkt)$

$\mathrm{W}$ trójkącie prostokątnym $ABC$ odcinek $AB$ jest przeciwprostokątną

$|BC|=12$. Wówczas sinus kąta ABCjest równy

i

$|AB|=13$

oraz

A.

-1123

B.

$\displaystyle \frac{5}{13}$

C.

$\displaystyle \frac{5}{12}$

D.

$\displaystyle \frac{13}{12}$

Zadanie 12. (1pkt)

W trójkącie równoramiennym ABC dane

Podstawa AB tego trójkąta ma długość

są

$|AC|=|BC|=5$

oraz wysokość

$|CD|=2.$

A. 6

B. $\mathrm{z}\sqrt{21}$

C. $\mathrm{z}\sqrt{29}$

D. 14





{\it Egzamin maturalny z matematyki}

{\it Poziom podstawowy}

{\it 5}

BRUDNOPIS





{\it 6}

{\it Egzamin maturalny z matematyki}

{\it Poziom podstawowy}

Zadanie 13. $(1pkt)$

$\mathrm{W}$ trójkącie prostokątnym dwa dłuzsze boki mają długości 5 $\mathrm{i}7$. Obwód tego trójkąta jest

równy

A. $16\sqrt{6}$ B. $14\sqrt{6}$ C. $12+4\sqrt{6}$ D. $12+2\sqrt{6}$

Zadanie 14. $(1pkt)$

Odcinki AB $\mathrm{i}$ CD są równoległe i $|AB|=5, |AC|=2, |CD|=7$ (zobacz rysunek). Długość

odcinka $AE$ jest równa

A.

$\displaystyle \frac{10}{7}$

B.

$\displaystyle \frac{14}{5}$
\begin{center}
\includegraphics[width=68.628mm,height=61.116mm]{./F1_M_PP_M2012_page5_images/image001.eps}
\end{center}
{\it D}

{\it B}

7

5

{\it E  A} 2  {\it C}

5

C. 3

D. 5

Zadanie 15. (1pkt)

Pole kwadratu wpisanego w okrąg o promieniu 5jest równe

A. 25

B. 50

C. 75

D. 100

Zadanie 16. $(1pkt)$

Punkty $A, B, C, D$ dzielą okrąg na 4 równe łuki. Miara zaznaczonego na rysunku kąta

wpisanego $ACD$ jest równa

A. $90^{\mathrm{o}}$

B. $60^{\mathrm{o}}$
\begin{center}
\includegraphics[width=50.388mm,height=50.388mm]{./F1_M_PP_M2012_page5_images/image002.eps}
\end{center}
{\it C}

$D$

{\it B}

{\it A}

D. $30^{\mathrm{o}}$

C. $45^{\mathrm{o}}$

Zadanie 17. (1pkt)

Miary kątów czworokąta tworzą ciąg arytmetyczny o róznicy

czworokąta ma miarę

$20^{\mathrm{o}}$ Najmniejszy kąt tego

A. $40^{\mathrm{o}}$

B. $50^{\mathrm{o}}$

C. $60^{\mathrm{o}}$

D. $70^{\mathrm{o}}$

Zadanie 18. $(1pkt)$

Dany jest ciąg $(a_{n})$ określony wzorem $a_{n}=(-1)^{n}\displaystyle \cdot\frac{2-n}{n^{2}}$ dla $n\geq 1$. Wówczas wyraz $a_{5}$ tego

ciągujest równy

A. - $\displaystyle \frac{3}{25}$ B. $\displaystyle \frac{3}{25}$ C. - $\displaystyle \frac{7}{25}$ D. $\displaystyle \frac{7}{25}$





{\it Egzamin maturalny z matematyki}

{\it Poziom podstawowy}

7

BRUDNOPIS





{\it 8}

{\it Egzamin maturalny z matematyki}

{\it Poziom podstawowy}

Zadanie 19. (1pkt)

Pole powierzchni jednej ściany sześcianujest równe 4. Objętość tego sześcianujest równa

A. 6

B. 8

C. 24

D. 64

Zadanie 20. $(1pkt)$

Tworząca stozka ma długość 4 i jest nachy1ona do płaszczyzny podstawy pod kątem $45^{\mathrm{o}}$

Wysokość tego stozkajest równa

A. $2\sqrt{2}$

B. $ 16\pi$

C. $4\sqrt{2}$

D. $ 8\pi$

Zadanie 21. $(1pkt)$

Wskaz równanie prostej równoległej do prostej o równaniu $3x-6y+7=0.$

A. {\it y}$=$-21{\it x} B. {\it y}$=$--21{\it x} C. {\it y}$=$2{\it x} D. {\it y}$=- 2x$

Zadanie 22. (1pkt)

Punkt A ma współrzędne (5,2012). Punkt B jest symetryczny do punktu A wzg1ędem osi Ox,

a punkt Cjest symetryczny do punktu B względem osi Oy. Punkt C ma współrzędne

A. $(-5,-2012)$

B. $(-2012,-5)$

C. $(-5$, 2012$)$

D. $(-2012,5)$

Zadanie 23. $(1pkt)$

Na okręgu o równaniu $(x-2)^{2}+(y+7)^{2}=4\mathrm{l}\mathrm{e}\dot{\mathrm{z}}\mathrm{y}$ punkt

A. $A=(-2,5)$

B. $B=(2,-5)$

C. $C=(2,-7)$

D. $D=(7,-2)$

Zadanie 24. (1pkt)

Flagę, takąjak pokazano na rysunku, nalezy zszyć

z trzech jednakowej szerokości pasów kolorowej

tkaniny. Oba pasy zewnętrzne mają być tego

samego koloru, a pas znajdujący się między nimi

ma być innego koloru.

Liczba róznych takich flag, które mozna uszyć,

mając do dyspozycji tkaniny w 10 ko1orach, jest

równa

A. 100

B. 99

C. 90

D. 19

Zadanie 25. (1pkt)

Średnia arytmetyczna cen sześciu akcji na giełdzie jest równa 500 zł. Za pięć z tych akcji

zapłacono 2300 zł. Cena szóstej akcjijest równa

A. 400 zł

B. 500 zł

C. 600 zł

D. 700 zł





{\it Egzamin maturalny z matematyki}

{\it Poziom podstawowy}

{\it 9}

BRUDNOPIS





$ 1\theta$

{\it Egzamin maturalny z matematyki}

{\it Poziom podstawowy}

ZADANIA OTWARTE

{\it Rozwiqzania zadań o numerach od 26. do 34. nalezy zapisać w} $wyznacz\theta nych$ {\it miejscach}

{\it pod treściq zadania}.

Zadanie 26. $(2pkt)$

Rozwiąz nierówność $x^{2}+8x+15>0.$

Odpowiedzí:

Zadanie 27. $(2pkt)$

Uzasadnij, $\dot{\mathrm{z}}\mathrm{e}$ jeśli liczby rzeczywiste $a,$

$\displaystyle \frac{a+b+c}{3}>\frac{a+b}{2}.$

$b, c$ spełniają nierówności $0<a<b<c$, to







Centralna Komisja Egzaminacyjna

Arkusz zawiera informacje prawnie chronione do momentu rozpoczęcia egzaminu.

WPISUJE ZDAJACY

KOD PESEL

{\it Miejsce}

{\it na naklejkę}

{\it z kodem}
\begin{center}
\includegraphics[width=21.432mm,height=9.804mm]{./F1_M_PP_M2013_page0_images/image001.eps}

\includegraphics[width=82.092mm,height=9.804mm]{./F1_M_PP_M2013_page0_images/image002.eps}
\end{center}
\fbox{} dysleksja
\begin{center}
\includegraphics[width=204.060mm,height=216.048mm]{./F1_M_PP_M2013_page0_images/image003.eps}
\end{center}
EGZAMIN MATU LNY

Z MATEMATYKI

MAJ 2013

POZIOM PODSTAWOWY

1.

2.

3.

Sprawd $\acute{\mathrm{z}}$, czy ar sz egzaminacyjny zawiera 22 strony

(zadania $1-34$). Ewentualny brak zgłoś przewodniczącemu

zespo nadzorującego egzamin.

Rozwiązania zadań i odpowiedzi wpisuj w miejscu na to

przeznaczonym.

Odpowiedzi do zadań za iętych (1-25) przenieś

na ka ę odpowiedzi, zaznaczając je w części ka $\mathrm{y}$

przeznaczonej dla zdającego. Zamaluj $\blacksquare$ pola do tego

przeznaczone. Błędne zaznaczenie otocz kółkiem

i zaznacz właściwe.

4. Pamiętaj, $\dot{\mathrm{z}}\mathrm{e}$ pominięcie argumentacji lub istotnych

obliczeń w rozwiązaniu zadania otwa ego (26-34) $\mathrm{m}\mathrm{o}\dot{\mathrm{z}}\mathrm{e}$

spowodować, $\dot{\mathrm{z}}\mathrm{e}$ za to rozwiązanie nie będziesz mógł

dostać pełnej liczby punktów.

5. Pisz czytelnie i uzywaj tvlko długopisu lub -Dióra

z czamym tuszem lub atramentem.

6. Nie uzywaj korektora, a błędne zapisy wyra $\acute{\mathrm{z}}\mathrm{n}\mathrm{i}\mathrm{e}$ przekreśl.

7. Pamiętaj, $\dot{\mathrm{z}}\mathrm{e}$ zapisy w brudnopisie nie będą oceniane.

8. $\mathrm{M}\mathrm{o}\dot{\mathrm{z}}$ esz korzystać z zestawu wzorów matematycznych,

cyrkla i linijki oraz kalkulatora.

9. Na tej stronie oraz na karcie odpowiedzi wpisz swój

numer PESEL i przyklej naklejkę z kodem.

10. Nie wpisuj $\dot{\mathrm{z}}$ adnych znaków w części przeznaczonej

dla egzaminatora.

Czas pracy:

170 minut

Liczba punktów

do uzyskania: 50

$\Vert\Vert\Vert\Vert\Vert\Vert\Vert\Vert\Vert\Vert\Vert\Vert\Vert\Vert\Vert\Vert\Vert\Vert\Vert\Vert\Vert\Vert\Vert\Vert|  \mathrm{M}\mathrm{M}\mathrm{A}-\mathrm{P}1_{-}1\mathrm{P}-132$




{\it 2}

{\it Egzamin maturalny z matematyki}

{\it Poziom podstawowy}

ZADANIA ZAMKNIĘTE

{\it Wzadaniach l-25 wybierz i zaznacz na karcie odpowiedzipoprawnq odpowiedzí}.

Zadanie l. $(1pkt)$

Wskaz rysunek, na którym zaznaczony

spełniających nierówność $|x+4|<5.$

jest zbiór

wszystkich liczb rzeczywistych
\begin{center}
\includegraphics[width=165.552mm,height=12.240mm]{./F1_M_PP_M2013_page1_images/image001.eps}
\end{center}
A.
\begin{center}
\includegraphics[width=165.612mm,height=17.832mm]{./F1_M_PP_M2013_page1_images/image002.eps}
\end{center}
$-9  -4$  1  {\it X}

B.
\begin{center}
\includegraphics[width=165.552mm,height=18.036mm]{./F1_M_PP_M2013_page1_images/image003.eps}
\end{center}
$-1$  4 9  {\it X}

C.
\begin{center}
\includegraphics[width=165.552mm,height=17.784mm]{./F1_M_PP_M2013_page1_images/image004.eps}
\end{center}
$-9  -5  -1$  {\it X}

1 5  9  {\it X}

D.

Zadanie 2. $(1pkt)$

Liczby $a\mathrm{i}b$ są dodatnie oraz 12\% 1iczby $a$ jest równe 15\% 1iczby $b$. Stąd wynika, $\dot{\mathrm{z}}\mathrm{e}a$ jest

równe

A. 103\% 1iczby $b$ B. 125\% 1iczby $b$ C. 150\% 1iczby $b$ D. 153\% 1iczby $b$

Zadanie 3. $(1pkt)$

Liczba $\log 100-\log_{2}8$ jest równa

A. $-2$

B. $-1$

C. 0

D. l

Zadanie 4. $(1pkt)$

Rozwiązaniem układu równań 

A. $x=-3 \mathrm{i}y=4$

B. $x=-3 \mathrm{i}y=6$

C. $x=3 \mathrm{i}y=-4$

D. $x=9 \mathrm{i}y=4$

Zadanie 5. $(1pkt)$

Punkt $A=(0,1)$ lezy na wykresie ffinkcji liniowej $f(x)=(m-2)x+m-3$. Stąd wynika, $\dot{\mathrm{z}}\mathrm{e}$

A. $m=1$

B. $m=2$

C. $m=3$

D. $m=4$

Zadanie 6. $(1pkt)$

Wierzchołkiem paraboli o równaniu $y=-3(x-2)^{2}+4$ jest punkt o współrzędnych

A. $(-2,-4)$

B. $(-2,4)$

C. $(2,-4)$

D. (2, 4)

Zadanie 7. $(1pkt)$

Dla $\mathrm{k}\mathrm{a}\dot{\mathrm{z}}$ dej liczby rzeczywistej $x$, wyrazenie $4x^{2}-12x+9$ jest równe

A. $(4x+3)(x+3)$

B. $(2x-3)(2x+3)$

C. $(2x-3)(2x-3)$

D. $(x-3)(4x-3)$





{\it Egzamin maturalny z matematyki}

{\it Poziom podstawowy}

{\it 11}

Zadanie 27. $(2pkt)$

Kąt $\alpha$ jest ostry i $\displaystyle \sin\alpha=\frac{\sqrt{3}}{2}$. Oblicz wartość wyrazenia $\sin^{2}\alpha-3\cos^{2}\alpha.$

Odpowied $\acute{\mathrm{z}}$:
\begin{center}
\includegraphics[width=95.964mm,height=17.832mm]{./F1_M_PP_M2013_page10_images/image001.eps}
\end{center}
Wypelnia

egzaminator

2

27.

2

Uzyskana liczba pkt





{\it 12}

{\it Egzamin maturalny z matematyki}

{\it Poziom podstawowy}

Zadanie 28. $(2pkt)$

Udowodnij, $\dot{\mathrm{z}}\mathrm{e}$ dla dowolnych liczb rzeczywistych $x, y, z$ takich, $\dot{\mathrm{z}}\mathrm{e}x+y+z=0$, prawdziwa

jest nierówność $xy+yz+zx\leq 0.$

$\mathrm{M}\mathrm{o}\dot{\mathrm{z}}$ esz skorzystać z $\mathrm{t}\mathrm{o}\dot{\mathrm{z}}$ samości $(x+y+z)^{2}=x^{2}+y^{2}+z^{2}+2xy+2xz+2yz.$





{\it Egzamin maturalny z matematyki}

{\it Poziom podstawowy}

{\it 13}

Zadanie 29. $(2pkt)$

Na rysunku przedstawiony jest wykres funkcji $f(x)$ określonej dla $x\in\langle-7,8\rangle.$
\begin{center}
\includegraphics[width=162.564mm,height=98.292mm]{./F1_M_PP_M2013_page12_images/image001.eps}
\end{center}
Odczytaj z wykresu i zapisz:

a) największą wartość funkcji f,

b) zbiór rozwiązań nierówności $f(x)<0.$
\begin{center}
\includegraphics[width=96.012mm,height=17.832mm]{./F1_M_PP_M2013_page12_images/image002.eps}
\end{center}
Wypelnia

egzaminator

Nr zadania

Maks. liczba kt

28.

2

2

Uzyskana liczba pkt





{\it 14}

{\it Egzamin maturalny z matematyki}

{\it Poziom podstawowy}

Zadanie 30. $(2pkt)$

Rozwiąz nierówność $2x^{2}-7x+5\geq 0.$

Odpowiedzí:





{\it Egzamin maturalny z matematyki}

{\it Poziom podstawowy}

{\it 15}

Zadanie 31. $(2pkt)$

Wykaz, $\dot{\mathrm{z}}\mathrm{e}$ liczba $6^{100}-2\cdot 6^{99}+10\cdot 6^{98}$ jest podzielna przez 17.
\begin{center}
\includegraphics[width=95.964mm,height=17.784mm]{./F1_M_PP_M2013_page14_images/image001.eps}
\end{center}
Wypelnia

egzaminator

Nr zadania

Maks. liczba kt

30.

2

31.

2

Uzyskana liczba pkt





{\it 16}

{\it Egzamin maturalny z matematyki}

{\it Poziom podstawowy}

Zadanie 32. (4pkt)

Punkt S jest środkiem okręgu opisanego na trójkącie ostrokątnym ABC. Kąt ACS jest trzy razy

większy od kąta BAS, a kąt CBSjest dwa razy większy od kąta BAS. Oblicz kąty trójkąta ABC.
\begin{center}
\includegraphics[width=93.828mm,height=90.168mm]{./F1_M_PP_M2013_page15_images/image001.eps}
\end{center}
{\it C}

{\it S}

{\it A  B}





{\it Egzamin maturalny z matematyki}

{\it Poziom podstawowy}

17

Odpowied $\acute{\mathrm{z}}$:
\begin{center}
\includegraphics[width=82.044mm,height=17.832mm]{./F1_M_PP_M2013_page16_images/image001.eps}
\end{center}
Wypelnia

egzaminator

Nr zadania

Maks. liczba kt

32.

4

Uzyskana liczba pkt





{\it 18}

{\it Egzamin maturalny z matematyki}

{\it Poziom podstawowy}

Zadanie 33. $(4pkt)$

Pole podstawy ostrosłupa prawidłowego czworokątnego jest równe 100

pole powierzchni bocznej jest równe 260 $\mathrm{c}\mathrm{m}^{2}$. Oblicz objętość tego ostrosłupa.

$\mathrm{c}\mathrm{m}^{2}$, a jego





{\it Egzamin maturalny z matematyki}

{\it Poziom podstawowy}

{\it 19}

Odpowiedzí :
\begin{center}
\includegraphics[width=82.044mm,height=17.832mm]{./F1_M_PP_M2013_page18_images/image001.eps}
\end{center}
Wypelnia

egzaminator

Nr zadania

Maks. liczba kt

33.

4

Uzyskana liczba pkt





$ 2\theta$

{\it Egzamin maturalny z matematyki}

{\it Poziom podstawowy}

Zadanie 34. $(5pkt)$

Dwa miasta łączy linia kolejowa o długości 336 ki1ometrów. Pierwszy pociąg przebył tę trasę

w czasie o 40 minut krótszym $\mathrm{n}\mathrm{i}\dot{\mathrm{z}}$ drugi pociąg. Średnia prędkość pierwszego pociągu na tej

trasie była o 9 $\mathrm{k}\mathrm{n}\vee \mathrm{h}$ większa od średniej prędkości drugiego pociągu. Oblicz średnią

prędkość $\mathrm{k}\mathrm{a}\dot{\mathrm{z}}$ dego z tych pociągów na tej trasie.





{\it Egzamin maturalny z matematyki}

{\it Poziom podstawowy}

{\it 3}

BRUDNOPIS





{\it Egzamin maturalny z matematyki}

{\it Poziom podstawowy}

{\it 21}

Odpowiedzí :
\begin{center}
\includegraphics[width=82.044mm,height=17.832mm]{./F1_M_PP_M2013_page20_images/image001.eps}
\end{center}
Wypelnia

egzaminator

Nr zadania

Maks. liczba kt

34.

5

Uzyskana liczba pkt





{\it 22}

{\it Egzamin maturalny z matematyki}

{\it Poziom podstawowy}

BRUDNOPIS





{\it 4}

{\it Egzamin maturalny z matematyki}

{\it Poziom podstawowy}

Zadanie 8. $(1pkt)$

Prosta o równaniu $y=\displaystyle \frac{2}{m}x+1$ jest prostopadła do prostej o równaniu $y=-\displaystyle \frac{3}{2}x-1$. Stąd

wynika, $\dot{\mathrm{z}}\mathrm{e}$

A. $m=-3$

B.

{\it m}$=$ -23

C.

{\it m}$=$ -23

D. $m=3$

Zadanie 9. $(1pkt)$

Na rysunku przedstawiony jest fragment wykresu pewnej funkcji liniowej $y=ax+b.$
\begin{center}
\includegraphics[width=66.036mm,height=50.748mm]{./F1_M_PP_M2013_page3_images/image001.eps}
\end{center}
$y$

0  {\it x}

Jakie znaki mają współczynniki a ib?

A. $a<0 \mathrm{i}b<0$

B. $a<0 \mathrm{i}b>0$

C. $a>0 \mathrm{i}b<0$

D. $a>0\mathrm{i}b>0$

Zadanie 10. (1pkt)

Najmniejszą liczbą całkowitą spełniającą nierówność $\displaystyle \frac{x}{2}\leq\frac{2x}{3}+\frac{1}{4}$ jest

A. $-2$

B. $-1$

C. 0

D. l

Zadanie ll. $(1pkt)$

Na rysunku l przedstawiony jest wykres funkcji $y=f(x)$ określonej dla $x\in\langle-7,4\rangle.$
\begin{center}
\includegraphics[width=184.500mm,height=59.280mm]{./F1_M_PP_M2013_page3_images/image002.eps}
\end{center}
Rysunek 2 przedstawia wykres ffinkcji

A. $y=f(x+2)$ B. $y=f(x)-2$

C. $y=f(x-2)$

D. $y=f(x)+2$

Zadanie 12. $(1pkt)$

Ciąg $($27, 18, $x+5)$ jest geometryczny. Wtedy

A. $x=4$

B. $x=5$

C. $x=7$

D. $x=9$





{\it Egzamin maturalny z matematyki}

{\it Poziom podstawowy}

{\it 5}

BRUDNOPIS





{\it 6}

{\it Egzamin maturalny z matematyki}

{\it Poziom podstawowy}

Zadanie 13. $(1pkt)$

Ciąg $(a_{n})$ określony dla $n\geq 1$ jest arytmetyczny oraz $a_{3}=10 \mathrm{i}a_{4}=14$. Pierwszy wyraz tego

ciągu jest równy

A. $a_{1}=-2$ B. $a_{1}=2$ C. $a_{1}=6$ D. $a_{1}=12$

Zadanie 14. $(1pkt)$

Kąt $\alpha$ jest ostry i $\displaystyle \sin\alpha=\frac{\sqrt{3}}{2}$. Wartość wyrazenia $\cos^{2}\alpha-2$ jest równa

A.

- -47

B.

- -41

C.

-21

D.

-$\sqrt{}$23

Zadanie 15. $(1pkt)$

Średnice AB $\mathrm{i}$ CD okręgu o środku $S$ przecinają się pod kątem $50^{\mathrm{o}}$ (takjak na rysunku).
\begin{center}
\includegraphics[width=65.124mm,height=65.628mm]{./F1_M_PP_M2013_page5_images/image001.eps}
\end{center}
{\it B}

{\it D}

$\alpha$

{\it S  M}

$50^{\mathrm{o}}$

{\it C}

{\it A}

Miara kąta $\alpha$ jest równa

A. $25^{\mathrm{o}}$

B. $30^{\mathrm{o}}$

C. $40^{\mathrm{o}}$

D. $50^{\mathrm{o}}$

Zadanie 16. $(1pkt)$

Liczba rzeczywistych rozwiązań równania $(x+1)(x+2)(x^{2}+3)=0$ jest równa

A. 0

B. l

C. 2

D. 4

Zadanie 17. $(1pkt)$

Punkty $A=(-1,2) \mathrm{i}B=(5,-2)$ są dwoma sąsiednimi wierzchołkami rombu ABCD. Obwód

tego rombujest równy

A. $\sqrt{13}$

B. 13

C. 676

D. $8\sqrt{13}$

Zadanie 18. $(1pkt)$

Punkt $S=(-4,7)$ jest środkiem odcinka

współrzędne

$PQ$, gdzie $Q=(17,12)$. Zatem punkt $P$ ma

A. $P=(2,-25)$

B. $P=(38,17)$

C. $P=(-25,2)$

D. $P=(-12,4)$





{\it Egzamin maturalny z matematyki}

{\it Poziom podstawowy}

7

BRUDNOPIS





{\it 8}

{\it Egzamin maturalny z matematyki}

{\it Poziom podstawowy}

Zadanie 19. $(1pkt)$

Odległość między środkami okręgów o równaniach $(x+1)^{2}+(y-2)^{2}=9$ oraz $x^{2}+y^{2}=10$

jest równa

A. $\sqrt{5}$

B. $\sqrt{10}-3$

C. 3

D. 5

Zadanie 20. $(1pkt)$

Liczba wszystkich krawędzi graniastosłupajest o 10 większa od 1iczby wszystkichjego ścian

bocznych. Stąd wynika, $\dot{\mathrm{z}}\mathrm{e}$ podstawą tego graniastosłupajest

A. czworokąt

B. pięciokąt

C. sześciokąt

D. dziesięciokąt

Zadanie 21. (1pkt)

Pole powierzchni bocznej stozka o wysokości 4 i promieniu podstawy 3 jest równe

A. $ 9\pi$

B. $ 12\pi$

C. $ 15\pi$

D. $ 16\pi$

Zadanie 22. $(1pkt)$

Rzucamy dwa razy symetryczną sześcienną kostką do gry. Niech $p$ oznacza

prawdopodobieństwo zdarzenia, $\dot{\mathrm{z}}\mathrm{e}$ iloczyn liczb wyrzuconych oczekjest równy 5. Wtedy

A.

$p=\displaystyle \frac{1}{36}$

B.

$p=\displaystyle \frac{1}{18}$

C.

$p=\displaystyle \frac{1}{12}$

D.

{\it p}$=$ -91

Zadanie 23. $(1pkt)$

Liczba $\displaystyle \frac{\sqrt{50}-\sqrt{18}}{\sqrt{2}}$ jest równa

A. $2\sqrt{2}$ B. 2

C. 4

D. $\sqrt{10}-\sqrt{6}$

Zadanie 24. (1pkt)

Mediana uporządkowanego niemalejąco zestawu sześciu liczb:

Wtedy

1, 2, 3, x, 5, 8 jest równa 4.

A. $x=2$

B. $x=3$

C. $x=4$

D. $x=5$

Zadanie 25. $(1pkt)$

Objętość graniastosłupa prawidłowego trójkątnego o wysokości $7$jest równa $28\sqrt{3}$. Długość

krawędzi podstawy tego graniastosłupajest równa

A. 2

B. 4

C. 8

D. 16





{\it Egzamin maturalny z matematyki}

{\it Poziom podstawowy}

{\it 9}

BRUDNOPIS





$ 1\theta$

{\it Egzamin maturalny z matematyki}

{\it Poziom podstawowy}

ZADANIA OTWARTE

{\it Rozwiqzania zadań} $26-34$ {\it nalezy zapisać w wyznaczonych miejscach} $p\theta d$ {\it treściq zadania}.

Zadanie 26. $(2pkt)$

Rozwiąz równanie $x^{3}+2x^{2}-8x-16=0$

Odpowiedzí:







$--\cdot\backslash \tau^{\mathrm{l}}\cdots-\cdot i1\dot{\text{‡C}}$

$:_{}^{\prime:=_{1\text{‡@}}^{1}}\overline{\iota}_{:\dot{!^{\mathrm{f}}:}!_{\vee}}^{-1}..$

-r$\equiv$:$\grave{}=$-J-$=$.-$\acute{}$z--,-.-[‡@]n-w$\Omega$-.:-.R-.J--n-$\llcorner\ulcorner-\simeq-\breve{}$.---$\lrcorner$.--[‡@]R-[‡@]-.--{\it l}-$\iota$-

Arkusz zawiera informacje prawnie chronione do momentu rozpoczęcia egzaminu.

WPISUJE ZDAJACY

KOD PESEL

{\it Miejsce}

{\it na naklejkę}

{\it z kodem}
\begin{center}
\includegraphics[width=21.432mm,height=9.852mm]{./F1_M_PP_M2014_page0_images/image001.eps}

\includegraphics[width=82.044mm,height=9.852mm]{./F1_M_PP_M2014_page0_images/image002.eps}
\end{center}
\fbox{} dysleksja
\begin{center}
\includegraphics[width=204.060mm,height=219.000mm]{./F1_M_PP_M2014_page0_images/image003.eps}
\end{center}
EGZAMIN MATU LNY

Z MATEMATYKI

MAJ 2014

POZIOM PODSTAWOWY

1. Sprawd $\acute{\mathrm{z}}$, czy arkusz egzaminacyjny zawiera 19 stron

(zadania $1-34$). Ewentualny brak zgłoś przewodniczącemu

zespo nadzo jącego egzamin.

2. Rozwiązania zadań i odpowiedzi wpisuj w miejscu na to

przeznaczonym.

3. Odpowiedzi do zadań za niętych (l-25) przenieś

na ka ę odpowiedzi, zaznaczając je w części ka $\mathrm{y}$

przeznaczonej dla zdającego. Zamaluj $\blacksquare$ pola do tego

przeznaczone. Błędne zaznaczenie otocz kółkiem \fcircle$\bullet$

i zaznacz właściwe.

4. Pamiętaj, $\dot{\mathrm{z}}\mathrm{e}$ pominięcie argumentacji lub istotnych

obliczeń w rozwiązaniu zadania otwa ego (26-34) $\mathrm{m}\mathrm{o}\dot{\mathrm{z}}\mathrm{e}$

spowodować, $\dot{\mathrm{z}}\mathrm{e}$ za to rozwiązanie nie otrzymasz pełnej

liczby punktów.

5. Pisz czytelnie i uzywaj tvlko długopisu lub -Dióra

z czatnym tuszem lub atramentem.

6. Nie uzywaj korektora, a błędne zapisy wyrazínie prze eśl.

7. Pamiętaj, $\dot{\mathrm{z}}\mathrm{e}$ zapisy w brudnopisie nie będą oceniane.

8. $\mathrm{M}\mathrm{o}\dot{\mathrm{z}}$ esz korzystać z zestawu wzorów matematycznych,

cyrkla i linijki oraz kalkulatora.

9. Na tej stronie oraz na karcie odpowiedzi wpisz swój

numer PESEL i przyklej naklejkę z kodem.

10. Nie wpisuj $\dot{\mathrm{z}}$ adnych znaków w części przeznaczonej

dla egzaminatora.

Czas pracy:

170 minut

Liczba punktów

do uzyskania: 50

$\Vert\Vert\Vert\Vert\Vert\Vert\Vert\Vert\Vert\Vert\Vert\Vert\Vert\Vert\Vert\Vert\Vert\Vert\Vert\Vert\Vert\Vert\Vert\Vert|  \mathrm{M}\mathrm{M}\mathrm{A}-\mathrm{P}1_{-}1\mathrm{P}-142$




{\it 2}

{\it Egzamin maturalny z matematyki}

{\it Poziom podstawowy}

ZADANIA ZAMKNIĘTE

{\it Wzadaniach od l. do 25. wybierz i zaznacz na karcie odpowiedzipoprawnq} $odp\theta wied\acute{z}.$

Zadanie l. $(1pkt)$

Na rysunku przedstawiono geometryczną interpretację jednego z $\mathrm{n}\mathrm{i}\dot{\mathrm{z}}$ ej zapisanych układów

równań.
\begin{center}
\includegraphics[width=62.844mm,height=50.340mm]{./F1_M_PP_M2014_page1_images/image001.eps}
\end{center}
4  {\it y}

$-3$

2

$-2$ -$1$

0

$-1$

1 2 3  {\it x}

Wskaz ten układ.

A.

$\left\{\begin{array}{l}
y=x+1\\
y=-2x+4
\end{array}\right.$

B.

$\left\{\begin{array}{l}
y=x-1\\
y=2x+4
\end{array}\right.$

C.

$\left\{\begin{array}{l}
y=x-1\\
y=-2x+4
\end{array}\right.$

D.

$\left\{\begin{array}{l}
y=x+1\\
y=2x+4
\end{array}\right.$

Zadanie 2. $(1pkt)$

$\mathrm{J}\mathrm{e}\dot{\mathrm{z}}$ eli liczba $78$jest o 50\% większa od 1iczby $c$, to

A. $c=60$

B. $c=52$

C. $c=48$

D. $c=39$

Zadanie 3. $(1pkt)$

Wartość wyrazenia $\displaystyle \frac{2}{\sqrt{3}-1}-\frac{2}{\sqrt{3}+1}$ jest równa

A. $-2$ B. $-2\sqrt{3}$

C. 2

D. $2\sqrt{3}$

Zadanie $4.(1pkt)$

Suma $\log_{8}16+1$ jest równa

A. 3

B.

-23

C. log817

D.

-73

Zadanie 5. $(1pkt)$

Wspólnym pierwiastkiem równań $(x^{2}-1)(x-10)(x-5)=0$ oraz $\displaystyle \frac{2x-10}{x-1}=0$ jest liczba

A. $-1$

B. l

C. 5

D. 10





{\it Egzamin maturalny z matematyki}

{\it Poziom podstawowy}

{\it 11}

Zadanie 27. $(2pkt)$

Rozwiąz równanie $9x^{3}+18x^{2}-4x-8=0.$

Odpowiedzí :
\begin{center}
\includegraphics[width=90.372mm,height=17.580mm]{./F1_M_PP_M2014_page10_images/image001.eps}
\end{center}
Wypelnia

egzamÍnator

Nr zadania

Maks. liczba kt

2

27.

2

Uzyskana liczba pkt





{\it 12}

{\it Egzamin maturalny z matematyki}

{\it Poziom podstawowy}

Zadanie 28. $(2pkt)$

Udowodnij, $\dot{\mathrm{z}}\mathrm{e}\mathrm{k}\mathrm{a}\dot{\mathrm{z}}$ da liczba całkowita $k$, która przy dzieleniu przez 7 daje resztę 2, ma tę

własność, $\dot{\mathrm{z}}\mathrm{e}$ reszta z dzielenia liczby $3k^{2}$ przez $7$jest równa 5.





{\it Egzamin maturalny z matematyki}

{\it Poziom podstawowy}

{\it 13}

Zadanie 29. $(2pkt)$

Na rysunku przedstawiono fragment wykresu ffinkcji $f$, który powstał w wyniku przesunięcia

wykresu funkcji określonej wzorem $y=\displaystyle \frac{1}{x}$ dla $\mathrm{k}\mathrm{a}\dot{\mathrm{z}}$ dej liczby rzeczywistej $x\neq 0.$
\begin{center}
\includegraphics[width=98.760mm,height=87.372mm]{./F1_M_PP_M2014_page12_images/image001.eps}
\end{center}
a) Odczytaj z wykresu i zapisz zbiór tych wszystkich argumentów, dla których wartości

funkcji $f$ są większe od 0.

b) Podaj miejsce zerowe funkcji $g$ określonej wzorem $g(x)=f(x-3).$

Odpowied $\acute{\mathrm{z}}:\mathrm{a})$

b)
\begin{center}
\includegraphics[width=90.372mm,height=17.580mm]{./F1_M_PP_M2014_page12_images/image002.eps}
\end{center}
Wypelnia

egzamÍnator

Nr zadania

Maks. liczba kt

28.

2

2

Uzyskana liczba pkt





{\it 14}

{\it Egzamin maturalny z matematyki}

{\it Poziom podstawowy}

Zadanie 30. (2pkt)

Ze zbioiu liczb \{1, 2, 3, 4, 5, 6, 7, 8\} 1osujemy dwa razy po jednej 1iczbie ze zwracaniem.

Oblicz prawdopodobieństwo zdarzenia A, polegającego na wylosowaniu liczb, z których

pierwszajest większa od drugiej o 41ub 6.

Odpowied $\acute{\mathrm{z}}$:





{\it Egzamin maturalny z matematyki}

{\it Poziom podstawowy}

{\it 15}

Zadanie 31. (2pkt)

Środek S okręgu opisanego na trójkącie równoramiennym ABC, o ramionach ACiBC, lezy

wewnątrz tego trójkąta (zobacz rysunek).
\begin{center}
\includegraphics[width=60.708mm,height=65.076mm]{./F1_M_PP_M2014_page14_images/image001.eps}
\end{center}
{\it C}

{\it S}

{\it A  B}

{\it ASB}

kąta wypukłego

Wykaz, $\dot{\mathrm{z}}\mathrm{e}$ miara

wypukłego $SBC.$

est cztery

razy większa

od miary kąta
\begin{center}
\includegraphics[width=90.372mm,height=17.580mm]{./F1_M_PP_M2014_page14_images/image002.eps}
\end{center}
Wypelnia

egzamÍnator

Nr zadania

Maks. liczba kt

30.

2

31.

2

Uzyskana liczba pkt





{\it 16}

{\it Egzamin maturalny z matematyki}

{\it Poziom podstawowy}

Zadanie 32. (4pkt)

Pole powierzchni całkowitej prostopadłościanu jest równe 198. Stosunki długości krawędzi

prostopadłościanu wychodzących z tego samego wierzchołka prostopadłościanu to 1: 2: 3.

Oblicz długość przekątnej tego prostopadłościanu.

Odpowied $\acute{\mathrm{z}}$:





{\it Egzamin maturalny z matematyki}

{\it Poziom podstawowy}

{\it 1}7

Zadanie 33. $(5pkt)$

Turysta zwiedzał zamek stojący na wzgórzu. Droga łącząca parking z zamkiem ma długość

2,1 km. Lączny czas wędrówki turysty z parkingu do zamku i z powrotem, nie licząc czasu

poświęconego na zwiedzanie, był równy l godzinę i 4 minuty. Ob1icz, z jaką średnią

prędkością turysta wchodził na wzgórze, $\mathrm{j}\mathrm{e}\dot{\mathrm{z}}$ eli prędkość ta była o $1 \displaystyle \frac{\mathrm{k}\mathrm{m}}{\mathrm{h}}$ mniejsza od średniej

prędkości, zjaką schodził ze wzgórza.

Odpowied $\acute{\mathrm{z}}$:
\begin{center}
\includegraphics[width=90.372mm,height=17.628mm]{./F1_M_PP_M2014_page16_images/image001.eps}
\end{center}
Wypelnia

egzaminator

Nr zadania

Maks. liczba kt

32.

4

33.

5

Uzyskana liczba pkt





{\it 18}

{\it Egzamin maturalny z matematyki}

{\it Poziom podstawowy}

Zadanie 34. $(4pkt)$

Kąt CAB trójkąta prostokątnego $ACB$ ma miarę $30^{\mathrm{o}}$. Pole kwadratu DEFG, wpisanego w ten

trójkąt (zobacz rysunek), jest równe 4. Ob1icz po1e trójkąta $ACB.$
\begin{center}
\includegraphics[width=68.880mm,height=43.692mm]{./F1_M_PP_M2014_page17_images/image001.eps}
\end{center}
{\it B}

{\it F}

{\it E}

{\it G}

$30^{\mathrm{o}}$

{\it C D  A}

Odpowiedzí :
\begin{center}
\includegraphics[width=78.840mm,height=17.580mm]{./F1_M_PP_M2014_page17_images/image002.eps}
\end{center}
Wypelnia

egzaminator

Nr zadania

Maks. liczba kt

34.

4

Uzyskana liczba pkt





{\it Egzamin maturalny z matematyki}

{\it Poziom podstawowy}

{\it 19}

BRUDNOPIS





{\it Egzamin maturalny z matematyki}

{\it Poziom podstawowy}

{\it 3}

BRUDNOPIS





{\it 4}

{\it Egzamin maturalny z matematyki}

{\it Poziom podstawowy}

Zadanie 6. $(1pkt)$

Funkcja liniowa $f(x)=(m^{2}-4)x+2$ jest malejąca, gdy

A. $m\in\{-2,2\}$

B. $m\in(-2,2)$

C.

$m\in(-\infty,-2)$

D. $m\in(2,+\infty)$

Zadanie 7. (1pkt)

Na rysunku przedstawiono fragment wykresu funkcji kwadratowej f
\begin{center}
\includegraphics[width=56.436mm,height=49.788mm]{./F1_M_PP_M2014_page3_images/image001.eps}
\end{center}
{\it y}

{\it x}

0

Funkcjafjest określona wzorem

A.

C.

$f(x)=\displaystyle \frac{1}{2}(x+3)(x-1)$

$f(x)=-\displaystyle \frac{1}{2}(x+3)(x-1)$

B.

D.

$f(x)=\displaystyle \frac{1}{2}(x-3)(x+1)$

$f(x)=-\displaystyle \frac{1}{2}(x-3)(x+1)$

Zadanie 8. $(1pkt)$

Punkt $C=(0,2)$ jest wierzchołkiem trapezu ABCD, którego podstawa $AB$ jest zawarta

w prostej o równaniu $y=2x-4$. Wskaz równanie prostej zawierającej podstawę CD.

A. $y=\displaystyle \frac{1}{2}x+2$ B. $y=-2x+2$ C. $y=-\displaystyle \frac{1}{2}x+2$ D. $y=2x+2$

Zadanie 9. $(1pkt)$

Dla $\mathrm{k}\mathrm{a}\dot{\mathrm{z}}$ dej liczby $x$, spełniającej warunek-3$<x<0$, wyrazenie $\displaystyle \frac{|x+3|-x+3}{x}$ jest równe

A. 2 B. 3 C. --{\it x}6 D. -{\it x}6

Zadanie 10. $(1pkt)$

Pierwiastki $x_{1}, x_{2}$ równania $2(x+2)(x-2)=0$ spełniają warunek

A.

$\underline{1}\underline{1}+=-1$

$x_{1} x_{2}$

B.

$\underline{1}+\underline{1}=0$

$x_{1} x_{2}$

C.

-{\it x}1  1 $+$ -{\it x}12 $=$ -41

D.

-{\it x}1  1 $+$ -{\it x}12 $=$ -21

Zadanie ll. $(1pkt)$

Liczby $2, -1, -4$ są trzema początkowymi wyrazami ciągu arytmetycznego

określonego dla liczb naturalnych $n\geq 1$. Wzór ogólny tego ciągu ma postać

A. $a_{n}=-3n+5$ B. $a_{n}=n-3$ C. $a_{n}=-n+3$ D. $a_{n}=3n-5$

$(a_{n}),$





{\it Egzamin maturalny z matematyki}

{\it Poziom podstawowy}

{\it 5}

BRUDNOPIS





{\it 6}

{\it Egzamin maturalny z matematyki}

{\it Poziom podstawowy}

Zadanie 12. $(1pkt)$

$\mathrm{J}\mathrm{e}\dot{\mathrm{z}}$ eli trójkąty $ABC \mathrm{i} A'B'C'$ są podobne, a ich pola $\mathrm{S}i\mathrm{L}$ odpowiednio, równe 25 $\mathrm{c}\mathrm{m}^{2}$

$\mathrm{i}50\mathrm{c}\mathrm{m}^{2}$, to skala podobieństwa $\displaystyle \frac{A'B'}{AB}$ jest równa

A. 2 B. -21 C. $\sqrt{}$2 D. --$\sqrt{}$22

Zadanie 13. $(1pkt)$

Liczby: $x-2$, 6, 12, w podanej kolejności,

geometrycznego. Liczba $x$ jest równa

są trzema

kolejnymi wyrazami

ciągu

A. 0

B. 2

C. 3

D. 5

Zadanie 14. $(1pkt)$

$\mathrm{J}\mathrm{e}\dot{\mathrm{z}}$ eli $\alpha$ jest kątem ostrym oraz $\displaystyle \mathrm{t}\mathrm{g}\alpha=\frac{2}{5}$, to wartość wyrazenia $\displaystyle \frac{3\cos\alpha-2\sin\alpha}{\sin\alpha-5\cos\alpha}$ jest równa

A.

$-\displaystyle \frac{11}{23}$

B.

$\displaystyle \frac{24}{5}$

C.

- -2131

D.

$\displaystyle \frac{5}{24}$

Zadanie 15. (1pkt)

Liczba punktów wspólnych okręgu

współrzędnychjest równa

A. 0 B. 1

o równaniu $(x+2)^{2}+(y-3)^{2}=4$

C. 2 D.

z osiami układu

4

Zadanie 16. $(1pkt)$

Wysokość trapezu równoramiennego o kącie ostrym $60^{\mathrm{o}}$ i ramieniu długości $2\sqrt{3}$ jest równa

A. $\sqrt{3}$ B. 3 C. $2\sqrt{3}$ D. 2

Zadanie 17. $(1pkt)$

Kąt środkowy oparty na iuku, którego diugośćjest równa $\displaystyle \frac{4}{9}$ diugości okręgu, ma miarę

A. $160^{\mathrm{o}}$

B. $80^{\mathrm{o}}$

C. $40^{\mathrm{o}}$

D. $20^{\mathrm{o}}$

Zadanie 18. $(1pkt)$

$\mathrm{O}$ funkcji liniowej $f$ wiadomo, $\dot{\mathrm{z}}\mathrm{e}f(1)=2$. Do wykresu tej funkcji nalez$\mathrm{y}$ punkt $P=(-2,3).$

Wzór funkcji $f$ to

A. $f(x)=-\displaystyle \frac{1}{3}x+\frac{7}{3}$ B. $f(x)=-\displaystyle \frac{1}{2}x+2$ C. $f(x)=-3x+7$ D. $f(x)=-2x+4$

Zadanie 19. $(1pkt)$

$\mathrm{J}\mathrm{e}\dot{\mathrm{z}}$ eli ostrosłup ma 10 krawędzi, to 1iczba ścian bocznychjest równa

A. 5

B. 7

C. 8

D. 10





{\it Egzamin maturalny z matematyki}

{\it Poziom podstawowy}

7

BRUDNOPIS





{\it 8}

{\it Egzamin maturalny z matematyki}

{\it Poziom podstawowy}

Zadanie 20. (1pkt)

Stozek i walec mają takie same podstawy i równe pola powierzchni bocznych. Wtedy

tworząca stozka jest

A. sześć razy dłuzsza od wysokości walca.

B. trzy razy dłuzsza od wysokości walca.

C. dwa razy dłuzsza od wysokości walca.

D. równa wysokości walca.

Zadanie 21. $(1pkt)$

Liczba $(\displaystyle \frac{1}{(\sqrt[3]{729}+\sqrt[4]{256}+2)^{0}})^{-2}$ jest równa

A. $\displaystyle \frac{1}{225}$ B. $\displaystyle \frac{1}{15}$

C. l

D. 15

Zadanie 22. $(1pkt)$

Do wykresu ffinkcji, określonej dla wszystkich liczb rzeczywistych wzorem $y=-2^{x-2}$, nalez$\mathrm{y}$

punkt

A. $A=(1,-2)$ B. $B=(2,-1)$ C. $C=(1,\displaystyle \frac{1}{2})$ D. $D=(4,4)$

Zadanie 23. $(1pkt)$

$\mathrm{J}\mathrm{e}\dot{\mathrm{z}}$ eli $A$ jest zdarzeniem losowym, $\mathrm{a}$

zachodzi równość $P(A)=2\cdot P(A^{\uparrow})$, to

A. $P(A)=\displaystyle \frac{2}{3}$ B. $P(A)=\displaystyle \frac{1}{2}$

A ` -zdarzeniem przeciwnym do zdarzenia A oraz

C. $P(A)=\displaystyle \frac{1}{3}$ D. $P(A)=\displaystyle \frac{1}{6}$

Zadanie 24. (1pkt)

Na ile sposobów mozna wybrać dwóch graczy spośród 10 zawodników?

A. 100 B. 90 C. 45 D.

20

Zadanie 25. $(1pkt)$

Mediana zestawu danych 2, 12, $a$, 10, 5, 3 jest równa 7. Wówczas

A. $a=4$ B. $a=6$ C. $a=7$

D. $a=9$





{\it Egzamin maturalny z matematyki}

{\it Poziom podstawowy}

{\it 9}

BRUDNOPIS





$ 1\theta$

{\it Egzamin maturalny z matematyki}

{\it Poziom podstawowy}

ZADANIA OTWARTE

{\it Rozwiqzania zadań o numerach od 26. do 34. nalezy zapisać}

{\it w wyznaczonych miejscach} $p\theta d$ {\it treściq zadania}.

Zadanie 26. $(2pkt)$

Wykresem funkcji kwadratowej $f(x)=2x^{2}+bx+c$ jest parabola, której wierzchołkiemjest

punkt $W=(4,0)$. Oblicz wartości współczynników $b\mathrm{i}c.$

Odpowied $\acute{\mathrm{z}}$:







Arkusz zawiera informacje prawnie chronione do momentu rozpoczęcia egzaminu.

UZUPELNIA ZDAJACY

KOD PESEL

{\it Miejsce}

{\it na naklejkę}

{\it z kodem}
\begin{center}
\includegraphics[width=21.432mm,height=9.852mm]{./F1_M_PP_M2015_page0_images/image001.eps}

\includegraphics[width=82.092mm,height=9.852mm]{./F1_M_PP_M2015_page0_images/image002.eps}
\end{center}
\fbox{} dysleksja
\begin{center}
\includegraphics[width=204.060mm,height=216.048mm]{./F1_M_PP_M2015_page0_images/image003.eps}
\end{center}
EGZAMIN MATU LNY

Z MATEMATYKI

POZIOM PODSTAWOWY  5 MAJA 20I5

Instrukcja dla zdającego

l. Sprawdzí, czy arkusz egzaminacyjny zawiera 24 strony

(zadania $1-34$). Ewentualny brak zgłoś przewodniczącemu

zespo nadzorującego egzamin.

2. Rozwiązania zadań i odpowiedzi wpisuj w miejscu na to

przeznaczonym.

3. Odpowiedzi do zadań za niętych (l-25) przenieś

na ka ę odpowiedzi, zaznaczając je w części ka $\mathrm{y}$

przeznaczonej dla zdającego. Zamaluj $\blacksquare$ pola do tego

przeznaczone. Błędne zaznaczenie otocz kółkiem \fcircle$\bullet$

i zaznacz właściwe.

4. Pamiętaj, $\dot{\mathrm{z}}\mathrm{e}$ pominięcie argumentacji lub istotnych

obliczeń w rozwiązaniu zadania otwartego (26-34) $\mathrm{m}\mathrm{o}\dot{\mathrm{z}}\mathrm{e}$

spowodować, $\dot{\mathrm{z}}\mathrm{e}$ za to rozwiązanie nie będziesz mógł

dostać pełnej liczby punktów.

5. Pisz czytelnie i $\mathrm{u}\dot{\mathrm{z}}$ aj tvlko długopisu lub -Dióra

z czamym tuszem lub atramentem.

6. Nie uzywaj korektora, a błędne zapisy wyra $\acute{\mathrm{z}}\mathrm{n}\mathrm{i}\mathrm{e}$ prze eśl.

7. Pamiętaj, $\dot{\mathrm{z}}\mathrm{e}$ zapisy w brudnopisie nie będą oceniane.

8. $\mathrm{M}\mathrm{o}\dot{\mathrm{z}}$ esz korzystać z zestawu wzorów matematycznych,

cyrkla i linijki oraz kalkulatora prostego.

9. Na tej stronie oraz na karcie odpowiedzi wpisz swój

numer PESEL i przyklej naklejkę z kodem.

10. Nie wpisuj $\dot{\mathrm{z}}$ adnych znaków w części przeznaczonej dla

egzaminatora.

Godzina rozpoczęcia:

Czas pracy:

170 minut

Liczba punktów

do uzyskania: 50

$\Vert\Vert\Vert\Vert\Vert\Vert\Vert\Vert\Vert\Vert\Vert\Vert\Vert\Vert\Vert\Vert\Vert\Vert\Vert\Vert\Vert\Vert\Vert\Vert|  \mathrm{M}\mathrm{M}\mathrm{A}-\mathrm{P}1_{-}1\mathrm{P}-152$




{\it Egzamin maturalny z matematyki}

{\it Poziom podstawowy}

{\it Wzadaniach od l. do 25. wybierz i zaznacz na karcie odpowiedzi poprawnq odpowiedzí}.

Zadanie l. (lpkt)

Cena pewnego towaru wraz z 7-procentowym podatkiem VAT jest równa 34347 zł. Cena

tego samego towaru wraz z 23-procentowym podatkiem VAT będzie równa

A. 37236 zł

B. 39842, 52 zł

C. 39483 zł

D. 42246, 81 zł

Zadanie 2. $(1pkt)$

Najmniejszą liczbą całkowitą dodatnią spełniającą nierówność $|x+4,5|\geq 6$ jest

A. $x=1$

B. $x=2$

C. $x=3$

D. $x=6$

Zadanie 3. $(1pkt)$

Liczba $2^{\frac{4}{3}}. \sqrt[3]{2^{5}}$ jest równa

A.

$2^{\frac{20}{3}}$

B. 2

C.

2-45

D. $2^{3}$

Zadanie 4. $(1pkt)$

Liczba 2 $\log_{5}10-\log_{5}4$ jest równa

A. 2 B. 1og596

C. $2\log_{5}6$

D. 5

$\mathrm{Z}\mathrm{a}\mathrm{d}\mathrm{a}\mathrm{n}\mathrm{i}\varepsilon 5. (1pkt)$

Zbiór wszystkich liczb rzeczywistych spełniających nierówność $\displaystyle \frac{3}{5}-\frac{2x}{3}\geq\frac{x}{6}$ jest przedziałem

A.

$\displaystyle \langle\frac{9}{15},+\infty)$

B.

$(-\displaystyle \infty,\frac{18}{25}\}$

C.

$\displaystyle \{\frac{1}{30},+\infty)$

D.

(-$\infty$ , -95$\rangle$

Zadanie 6. $(1pkt)$

Dziedziną funkcji $f$ określonej wzorem $f(x)=\displaystyle \frac{x+4}{x^{2}-4x}\mathrm{m}\mathrm{o}\dot{\mathrm{z}}\mathrm{e}$ być zbiór

A. wszystkich liczb rzeczywistych róznych od 0 i od 4.

B. wszystkich liczb rzeczywistych róznych od $-4$ i od 4.

C. wszystkich liczb rzeczywistych róznych od -A i od 0.

D. wszystkich liczb rzeczywistych.

$\mathrm{Z}\mathrm{a}\mathrm{d}\mathrm{a}\mathrm{n}\mathrm{i}\varepsilon 7. (1pkt)$

Rozwiązaniem równania $\displaystyle \frac{2x-4}{3-x}=\frac{4}{3}$ jest liczba

A. $x=0$

B.

$x=\displaystyle \frac{12}{5}$

C. $x=2$

Strona 2 z24

D.

{\it x}$=$ -2151

MMA-IP





{\it Egzamin maturalny z matematyki}

{\it Poziom podstawowy}

{\it BRUDNOPIS} ({\it nie podlega ocenie})

MMA-IP

Strona ll z24





{\it Egzamin maturalny z matematyki}

{\it Poziom podstawowy}

Zadanie $2\not\in. (2pki)$

Wykaz, $\dot{\mathrm{z}}\mathrm{e}$ dla $\mathrm{k}\mathrm{a}\dot{\mathrm{z}}$ dej liczby rzeczywistej $x$ i dla $\mathrm{k}\mathrm{a}\dot{\mathrm{z}}$ dej liczby rzeczywistej $y$ prawdziwajest

nierówność $4x^{2}-8xy+5y^{2}\geq 0.$

Strona 12 z24

MMA-IP





{\it Egzamin maturalny z matematyki}

{\it Poziom podstawowy}

Zadanie 27. $(2pkt)$

Rozwiąz nierówność $2x^{2}-4x\geq x-2.$

Odpowied $\acute{\mathrm{z}}$:
\begin{center}
\includegraphics[width=96.012mm,height=17.784mm]{./F1_M_PP_M2015_page12_images/image001.eps}
\end{center}
Wypelnia

egzaminator

Nr zadania

Maks. liczba kt

2

27.

2

Uzyskana liczba pkt

MMA-IP

Strona 13 z24





{\it Egzamin maturalny z matematyki}

{\it Poziom podstawowy}

Zadanie $2\mathrm{S}. (2pkt)$

Rozwiąz równanie $4x^{3}+4x^{2}-x-1=0.$

Odpowiedzí:

Strona 14 z24

MMA-IP





{\it Egzamin maturalny z matematyki}

{\it Poziom podstawowy}

Zadanie $29_{n}(2pkt)$

Na rysunku przedstawiono wykres funkcji $f.$
\begin{center}
\includegraphics[width=143.052mm,height=110.136mm]{./F1_M_PP_M2015_page14_images/image001.eps}
\end{center}
$y$

5

4

3

2

1

{\it x}

$-4$ -$3  -2$ -$1 0$  1 2  3 4  5 6

$-1$

$-2$

Funkcja $h$ określona jest dla $x\in\langle-3,  5\rangle$ wzorem $h(x)=f(x)+q$, gdzie $q$ jest pewną liczbą

rzeczywistą. Wiemy, $\dot{\mathrm{z}}$ ejednym z miejsc zerowych funkcji $h$ jest liczba $x_{0}=-1.$

a) Wyznacz q.

b) Podaj wszystkie pozostałe miejsca zerowe funkcji h.

Odpowiedzí :
\begin{center}
\includegraphics[width=96.012mm,height=23.676mm]{./F1_M_PP_M2015_page14_images/image002.eps}
\end{center}
Wypelnia

egzaminator

Nr zadania

Maks. liczba kt

28.

2

2

Uzyskana liczba pkt

Strona 15 z24

MMA-IP





{\it Egzamin maturalny z matematyki}

{\it Poziom podstawowy}

Zadanie 30. $(2pki)$

Dany jest skończony ciąg, w którym pierwszy wyraz jest równy 444, a ostatni jest

równy 653. $\mathrm{K}\mathrm{a}\dot{\mathrm{z}}\mathrm{d}\mathrm{y}$ wyraz tego ciągu, począwszy od drugiego, jest o ll większy od wyrazu

bezpośrednio go poprzedzającego. Oblicz sumę wszystkich wyrazów tego ciągu.

Odpowiedzí:

Strona 16 z24

MMA-IP





{\it Egzamin maturalny z matematyki}

{\it Poziom podstawowy}

Zadanie 31. $(2pkt)$

Dany jest okrąg o środku w punkcie $O$. Prosta $KL$ jest styczna do tego okręgu w punkcie $L,$

a środek $O$ tego okręgu lezy na odcinku KM (zob. rysunek). Udowodnij, $\dot{\mathrm{z}}\mathrm{e}$ kąt $KML$ ma

miarę $31^{\mathrm{o}}$
\begin{center}
\includegraphics[width=112.320mm,height=75.180mm]{./F1_M_PP_M2015_page16_images/image001.eps}
\end{center}
{\it L}

{\it M} ?

{\it O}

$28^{\mathrm{o}}$

{\it K}
\begin{center}
\includegraphics[width=96.012mm,height=17.784mm]{./F1_M_PP_M2015_page16_images/image002.eps}
\end{center}
Wypelnia

egzaminator

Nr zadania

Maks. liczba kt

30.

2

31.

2

Uzyskana liczba pkt

MMA-IP

Strona 17 z24





{\it Egzamin maturalny z matematyki}

{\it Poziom podstawowy}

Zadanie 32. $(4pki)$

Wysokość graniastosłupa prawidłowego czworokątnego jest równa 16. Przekątna graniastosłupa

jest nachylona do płaszczyzny jego podstawy pod kątem, którego cosinus jest równy $\displaystyle \frac{3}{5}$. Oblicz

pole powierzchni całkowitej tego graniastosłupa.

Strona 18 z24

MMA-IP





{\it Egzamin maturalny z matematyki}

{\it Poziom podstawowy}

Odpowied $\acute{\mathrm{z}}$:
\begin{center}
\includegraphics[width=82.044mm,height=17.832mm]{./F1_M_PP_M2015_page18_images/image001.eps}
\end{center}
Nr zadania

Wypelnia Maks. liczba kt

egzaminator

Uzyskana liczba pkt

32.

4

MMA-IP

Strona 19 z24





{\it Egzamin maturalny z matematyki}

{\it Poziom podstawowy}

Zadanie 33. (4pkt)

Wśród 115 osób przeprowadzono badania ankietowe, związane z zakupami w pewnym

kiosku. W ponizszej tabeli przedstawiono informacje o tym, ile osób kupiło bilety

tramwajowe ulgowe oraz ile osób kupiło bilety tramwajowe normalne.
\begin{center}
\begin{tabular}{|l|l|}
\hline
\multicolumn{1}{|l|}{$\begin{array}{l}\mbox{Rodzaj kupionych}	\\	\mbox{biletów}	\end{array}$}&	\multicolumn{1}{|l|}{Liczba osób}	\\
\hline
\multicolumn{1}{|l|}{ulgowe}&	\multicolumn{1}{|l|}{$76$}	\\
\hline
\multicolumn{1}{|l|}{normalne}&	\multicolumn{1}{|l|}{$41$}	\\
\hline
\end{tabular}

\end{center}
Uwaga! 27 osób spośród ankietowanych kupiło oba rodzaje bi1etów.

Oblicz prawdopodobieństwo zdarzenia polegającego na tym, $\dot{\mathrm{z}}\mathrm{e}$ osoba losowo wybrana

spośród ankietowanych nie kupiła $\dot{\mathrm{z}}$ adnego biletu. Wynik przedstaw w formie nieskracalnego

ułamka.

Strona 20 z24

MMA-IP





{\it Egzamin maturalny z matematyki}

{\it Poziom podstawowy}

{\it BRUDNOPIS} ({\it nie podlega ocenie})

MMA-IP

Strona 3 z24





{\it Egzamin maturalny z matematyki}

{\it Poziom podstawowy}

Odpowied $\acute{\mathrm{z}}$:
\begin{center}
\includegraphics[width=82.044mm,height=17.832mm]{./F1_M_PP_M2015_page20_images/image001.eps}
\end{center}
Wypelnia

egzaminator

Nr zadania

Maks. liczba kt

33.

4

Uzyskana liczba pkt

MMA-IP

Strona 21 z24





{\it Egzamin maturalny z matematyki}

{\it Poziom podstawowy}

Zadanie 34. $\beta 5pkt$)

Biegacz narciarski Borys wyruszył na trasę biegu o 10 minut pózíniej $\mathrm{n}\mathrm{i}\dot{\mathrm{z}}$ inny zawodnik,

Adam. Metę zawodów, po przebyciu 15-ki1ometrowej trasy biegu, obaj zawodnicy pokona1i

równocześnie. Okazało się, $\dot{\mathrm{z}}\mathrm{e}$ wartość średniej prędkości na całej trasie w przypadku Borysa

była o $4,5 \displaystyle \frac{\mathrm{k}\mathrm{m}}{\mathrm{h}}$ większa $\mathrm{n}\mathrm{i}\dot{\mathrm{z}}$ w przypadku Adama. Oblicz, wjakim czasie Adam pokonał całą

trasę biegu.

Strona 22 z24

MMA-IP





{\it Egzamin maturalny z matematyki}

{\it Poziom podstawowy}

Odpowied $\acute{\mathrm{z}}$:
\begin{center}
\includegraphics[width=82.044mm,height=17.832mm]{./F1_M_PP_M2015_page22_images/image001.eps}
\end{center}
Nr zadania

Wypelnia Maks. liczba kt

egzaminator

Uzyskana liczba pkt

34.

5

MMA-IP

Strona 23 z24





{\it Egzamin maturalny z matematyki}

{\it Poziom podstawowy}

{\it BRUDNOPIS} ({\it nie podlega ocenie})

Strona 24 z24

MMA-IP





{\it Egzamin maturalny z matematyki}

{\it Poziom podstawowy}

Zadanie 8. $(1pkt)$

Miejscem zerowym funkcji liniowej określonej wzorem $f(x)=-\displaystyle \frac{2}{3}x+4$ jest

A. 0

B. 6

C. 4

D. $-6$

ZadanÍe 9. $(1pkt)$

Punkt $M=(\displaystyle \frac{1}{2},3)$

nalezy do

wykresu funkcji

liniowej określonej

wzorem

$f(x)=(3-2a)x+2$. Wtedy

A.

{\it a}$=$- -21

B. $a=2$

C.

{\it a}$=$ -21

D. $a=-2$

Zadanie $l0. (1pkt)$

Na rysunku przedstawiono fragment prostej o równaniu $y=ax+b.$
\begin{center}
\includegraphics[width=125.328mm,height=84.840mm]{./F1_M_PP_M2015_page3_images/image001.eps}
\end{center}
{\it y}

7

6

5

$P=(2,5)$

4  $Q=(5,3)$

3

2

1

{\it x}

$-1$

0

$-1$

1 2 3 4  5 6 7 8  9 1

Współczynnik kierunkowy tej prostej jest równy

A.

{\it a}$=$- -23

B.

{\it a}$=$- -23

C.

{\it a}$=$- -25

D.

{\it a}$=$- -53

Zadanie ll. (lpkt)

$\mathrm{W}$ ciągu arytmetycznym $(a_{n})$ określonym dla

wyrazem tego ciągujest liczba 156?

$n\geq 1$ dane są $a_{1}=-4$

i

$r=2$. Którym

A. 81.

B. 80.

C. 76.

D. 77.

Zadanie 12. (1pkt)

W rosnącym ciągu geometrycznym

$(a_{n})$, określonym dla $n\geq 1$, spełniony jest warunek

$a_{4}=3a_{1}$. Iloraz $q$ tego ciągu jest równy

A.

{\it q}$=$ -31

B.

{\it q}$=$ -$\sqrt{}$313

C. $q=\sqrt[3]{3}$

Strona 4 $\mathrm{z}24$

D. $q=3$

MMA-IP





{\it Egzamin maturalny z matematyki}

{\it Poziom podstawowy}

{\it BRUDNOPIS} ({\it nie podlega ocenie})

MMA-IP

Strona 5 z24





{\it Egzamin maturalny z matematyki}

{\it Poziom podstawowy}

Zadanie 13. (1pkt)

Drabinę o długości 4 metrów oparto o pionowy mur,

w odległości 1,30 m od tego muru (zobacz rysunek).

a jej podstawę umieszczono
\begin{center}
\includegraphics[width=27.072mm,height=53.388mm]{./F1_M_PP_M2015_page5_images/image001.eps}
\end{center}
4m

$\alpha$

1,30 $\mathrm{m}$

Kąt $\alpha$, podjakim ustawiono drabinę, spełnia warunek

A. $0^{\mathrm{o}}<\alpha<30^{\mathrm{o}}$

B. $30^{\mathrm{o}}<\alpha<45^{\mathrm{o}}$

C. $45^{\mathrm{o}}<\alpha<60^{\mathrm{o}}$

D. $60^{\mathrm{o}}<\alpha<90^{\mathrm{o}}$

Zadanie 14. $(1pkt)$

Kąt $\alpha$jest ostry i $\displaystyle \sin\alpha=\frac{2}{5}$. Wówczas $\cos\alpha$ jest równy

A. -25 B. --$\sqrt{}$421 C. -53

D.

$\displaystyle \frac{\sqrt{21}}{5}$

Zadanie $15_{\mathfrak{v}}(1pkt)$

$\mathrm{W}$ trójkącie równoramiennym $ABC$ spełnione są warunki: $|AC|=|BC|, |\neq CAB|=50^{\mathrm{o}}$

Odcinek $BD$ jest dwusieczną kąta $ABC$, a odcinek $BE$ jest wysokoŚcią opuszczoną

z wierzchołka $B$ na bok $AC$. Miara kąta $EBD$ jest równa
\begin{center}
\includegraphics[width=136.200mm,height=91.392mm]{./F1_M_PP_M2015_page5_images/image002.eps}
\end{center}
{\it C}

{\it E}

{\it D}

?

$50^{\mathrm{o}}$

{\it A  B}

B. 12, $5^{\mathrm{o}}$

A. $10^{\mathrm{o}}$

C. 13, $5^{\mathrm{o}}$

D. $15^{\mathrm{o}}$

Strona 6 z24

MMA-IP





{\it Egzamin maturalny z matematyki}

{\it Poziom podstawowy}

{\it BRUDNOPIS} ({\it nie podlega ocenie})

MMA-IP

Strona 7 z24





{\it Egzamin maturalny z matematyki}

{\it Poziom podstawowy}

Zadanie $l\not\in. (1pki)$

Przedstawione na rysunku trójkąty są podobne.
\begin{center}
\includegraphics[width=60.912mm,height=31.092mm]{./F1_M_PP_M2015_page7_images/image001.eps}
\end{center}
{\it a}

4

$\alpha  \beta$
\begin{center}
\includegraphics[width=121.416mm,height=61.980mm]{./F1_M_PP_M2015_page7_images/image002.eps}
\end{center}
{\it b}

$\alpha  \beta$

6

15

12

Wówczas

A. $a=13, b=17$

B. $a=10, b=18$

C. $a=9, b=19$

D. $a=11, b=13$

Zadauie 17. $(1pkt)$

Proste o równaniach: $y=2mx-m^{2}-1$ oraz $y=4m^{2}x+m^{2}+1$ są prostopadłe dla

A. {\it m}$=$--21 B. {\it m}$=$-21 C. {\it m}$=$1 D. {\it m}$=$2

Zadanie 18. (1pkt)

Dane są punkty $M=(3,-5)$ oraz $N=(-1,7)$. Prosta przechodząca przez te punkty ma

równanie

A. $y=-3x+4$

B. $y=3x-4$

C.

$y=-\displaystyle \frac{1}{3}x+4$

D. $y=3x+4$

ZadaBie 19. $(1pkt)$

Dane są punkty: $P=(-2,-2), Q=(3$, 3$)$. Odległość punktu $P$ od punktu $Q$ jest równa

A. I B. 5 C. $5\sqrt{2}$ D. $2\sqrt{5}$

Zadanie 20. $(1pkt)$

Punkt $K=(-4,4)$ jest końcem odcinka $KL$, punkt $L$ lezy na osi $Ox$, a środek $S$ tego odcinka

lezy na osi $Oy$. Wynika stąd, $\dot{\mathrm{z}}\mathrm{e}$

A. $S=(0,2)$

B. $S=(-2,0)$

C. $S=(4,0)$

D. $S=(0,4)$

Strona 8 z24

MMA-IP





{\it Egzamin maturalny z matematyki}

{\it Poziom podstawowy}

{\it BRUDNOPIS} ({\it nie podlega ocenie})

MMA-IP

Strona 9 z24





{\it Egzamin maturalny z matematyki}

{\it Poziom podstawowy}

Zadanie 21. $(1pki)$

Okrąg przedstawiony na rysunku ma środek w punkcie $O=(3,1)$ i przechodzi przez punkty

$S=(0,4)\mathrm{i}T=(0,-2)$. Okrąg tenjest opisany przez równanie
\begin{center}
\includegraphics[width=99.420mm,height=89.460mm]{./F1_M_PP_M2015_page9_images/image001.eps}
\end{center}
{\it y}

6

5

4 {\it S}

3

2

1

{\it O}

{\it x}

0

1

1 2  3 4 5 6  8

$-2$  {\it T}

A. $(x+3)^{2}+(y+1)^{2}=18$

B. $(x-3)^{2}+(y+1)^{2}=18$

C. $(x-3)^{2}+(y-1)^{2}=18$

D. $(x+3)^{2}+(y-1)^{2}=18$

Zadanie 22. (1pkt)

Przekątna ściany sześcianu ma długość 2. Po1e powierzchni całkowitej tego sześcianu jest

równe

A. 24

B. $12\sqrt{2}$

C. 12

D. $16\sqrt{2}$

Zadanie 23. $(1pkt)$

Kula o promieniu 5 cm i stozek o promieniu podstawy

Wysokość stozkajest równa

A. $\displaystyle \frac{25}{\pi}$ cm B. $10\mathrm{c}\mathrm{m}$ C. $\displaystyle \frac{10}{\pi}$ cm

10 cm mają równe objętości.

D. 5 cm

Zadanie 24. (1pki)

Średnia arytmetyczna zestawu danych:

2, 4, 7, 8, 9

jest taka sama jak średnia arytmetyczna zestawu danych:

2, 4, 7, 8, 9, $x.$

Wynika stąd, $\dot{\mathrm{z}}\mathrm{e}$

A. $x=0$

B. $x=3$

C. $x=5$

D. $x=6$

Zadanie $25_{\mathfrak{v}}(1pkt)$

$\mathrm{W}$ pewnej klasie stosunek liczby dziewcząt do liczby chłopców jest równy 4: 5. Losujemy

jedną osobę z tej klasy. Prawdopodobieństwo tego, $\dot{\mathrm{z}}\mathrm{e}$ będzie to dziewczyna, jest równe

A. -45 B. -49 C. -41 D. -91 MMA-1P

Strona 10 $\mathrm{z}24$







Arkusz zawiera informacje prawnie chronione do momentu rozpoczęcia egzaminu.

UZUPELNIA ZDAJACY

KOD PESEL

{\it miejsce}

{\it na naklejkę}
\begin{center}
\includegraphics[width=21.432mm,height=9.852mm]{./F1_M_PP_M2016_page0_images/image001.eps}

\includegraphics[width=82.092mm,height=9.852mm]{./F1_M_PP_M2016_page0_images/image002.eps}
\end{center}
\square  dyskalkulia

\fbox{} dysleksja
\begin{center}
\includegraphics[width=204.060mm,height=216.048mm]{./F1_M_PP_M2016_page0_images/image003.eps}
\end{center}
EGZAMIN MATU

Z MATEMATY

LNY

POZIOM PODSTAWOWY  5 MAJA 20I

Instrukcja dla zdającego

l. Sprawdzí, czy arkusz egzaminacyjny zawiera 24 strony

(zadania $1-34$). Ewentualny brak zgłoś przewodniczącemu

zespo nadzorującego egzamin.

2. Rozwiązania zadań i odpowiedzi wpisuj w miejscu na to

przeznaczonym.

3. Odpowiedzi do zadań zamkniętych $(1-25)$ zaznacz

na karcie odpowiedzi, w części ka $\mathrm{y}$ przeznaczonej dla

zdającego. Zamaluj $\blacksquare$ pola do tego przeznaczone. Błędne

zaznaczenie otocz kółkiem $\mathrm{O}$ i zaznacz właściwe.

4. Pamiętaj, $\dot{\mathrm{z}}\mathrm{e}$ pominięcie argumentacji lub istotnych

obliczeń w rozwiązaniu zadania otwa ego (26-34) $\mathrm{m}\mathrm{o}\dot{\mathrm{z}}\mathrm{e}$

spowodować, $\dot{\mathrm{z}}\mathrm{e}$ za to rozwiązanie nie będziesz mógł

dostać pełnej liczby punktów.

5. Pisz czytelnie i $\mathrm{u}\dot{\mathrm{z}}$ aj tylko $\mathrm{d}$ gopisu lub pióra

z czarnym tuszem lub atramentem.

6. Nie $\mathrm{u}\dot{\mathrm{z}}$ aj korektora, a błędne zapisy wyra $\acute{\mathrm{z}}\mathrm{n}\mathrm{i}\mathrm{e}$ prze eśl.

7. Pamiętaj, $\dot{\mathrm{z}}\mathrm{e}$ zapisy w brudnopisie nie będą oceniane.

8. $\mathrm{M}\mathrm{o}\dot{\mathrm{z}}$ esz korzystać z zestawu wzorów matematycznych,

cyrkla i linijki oraz kalkulatora prostego.

9. Na tej stronie oraz na karcie odpowiedzi wpisz swój

numer PESEL i przyklej naklejkę z kodem.

10. Nie wpisuj $\dot{\mathrm{z}}$ adnych znaków w części przeznaczonej dla

egzaminatora.

Godzina rozpoczęcia:

Czas pracy:

170 minut

Liczba punktów

do uzyskania: 50

$\Vert\Vert\Vert\Vert\Vert\Vert\Vert\Vert\Vert\Vert\Vert\Vert\Vert\Vert\Vert\Vert\Vert\Vert\Vert\Vert\Vert\Vert\Vert\Vert|  \mathrm{M}\mathrm{M}\mathrm{A}-\mathrm{P}1_{-}1\mathrm{P}-162$




{\it Egzamin maturalny z matematyki}

{\it Poziom podstawowy}

ZADANIA ZAMKNIĘTE

{\it Wzadaniach od l. do 25. wybierz i zaznacz na karcie odpowiedzipoprawnq} $odp\theta wied\acute{z}.$

Zadanie l, (l pkţ)

Dla $\mathrm{k}\mathrm{a}\dot{\mathrm{z}}$ dej dodatniej liczby $a$ iloraz $\displaystyle \frac{a^{-2,6}}{a^{1,3}}$ jest równy

A.

$a^{-3,9}$

B.

$a^{-2}$

C.

$a^{-1,3}$

D.

$a^{1,3}$

Zadanie 2. $(1pkt)$

Liczba $\log_{\sqrt{2}}(2\sqrt{2})$ jest równa

A.

-23

B. 2

C.

-25

D. 3

Zadanie 3. $(1pkt)$

Liczby $a\mathrm{i}c$ są dodatnie. Liczba $b$ stanowi 48\% 1iczby $a$ oraz 32\% 1iczby $c$. Wynika stąd, $\dot{\mathrm{z}}\mathrm{e}$

A. $c=1,5a$

B. $c=1,6a$

C. $c=0,8a$

D. $c=0,16a$

ZadanÍe 4. $(1pkt)$

RównoŚć $(2\sqrt{2}-a)^{2}=17-12\sqrt{2}$ jest prawdziwa dla

A. $a=3$

B. $a=1$

C. $a=-2$

D. $a=-3$

Zadanie 5. $(1pktJ$

Jedną z liczb, które spełniają nierówność $-x^{5}+x^{3}-x<-2$, jest

A. l

B. $-1$

C. 2

D. $-2$

Zadanie $\epsilon. (1pkt)$

Proste o równaniach $2x-3y=4\mathrm{i}5x-6y=7$ przecinają się w punkcie $P$. Stąd wynika, $\dot{\mathrm{z}}\mathrm{e}$

A. $P=(1,2)$

B. $P=(-1,2)$

C. $P=(-1,-2)$

D. $P=(1,-2)$

ZadanÍe 7. (1pkt)

Punkty ABCD $\mathrm{l}\mathrm{e}\dot{\mathrm{z}}$ ą na o ęgu o środku $S$ (zobacz

Miara kąta $BDC$ jest równa

A. $91^{\mathrm{o}}$

sunek).

B. $72,5^{\mathrm{o}}$
\begin{center}
\includegraphics[width=78.132mm,height=79.452mm]{./F1_M_PP_M2016_page1_images/image001.eps}
\end{center}
{\it D}

{\it C}

$27^{\mathrm{o}}$

{\it S}

18

{\it B}

{\it A}

Strona 2 z24

D. $32^{\mathrm{o}}$

C. $18^{\mathrm{o}}$

MMA-IP





{\it Egzamin maturalny z matematyki}

{\it Poziom podstawowy}

{\it BRUDNOPIS} ({\it nie podlega ocenie})

MMA-IP

Strona ll z24





{\it Egzamin maturalny z matematyki}

{\it Poziom podstawowy}

ZADANIA OTWARTE

{\it Rozwiqzania zadań o numerach od 26. do 34. nalezy zapisać w wyznaczonych miejscach pod treściq}

{\it zadania}.

Zadanie 26. (2pkt)

Rozwiąz nierównoŚć $2x^{2}+5x-3>0.$

Odpowied $\acute{\mathrm{z}}$:

Strona 12 $\mathrm{z}24$

MMA-IP





{\it Egzamin maturalny z matematyki}

{\it Poziom podstawowy}

Zadanie 27. (2pkt)

Rozwiąz równanie $x^{3}+3x^{2}+2x+6=0.$

Odpowied $\acute{\mathrm{z}}$:
\begin{center}
\includegraphics[width=96.012mm,height=17.832mm]{./F1_M_PP_M2016_page12_images/image001.eps}
\end{center}
Wypelnia

egzaminator

Nr zadania

Maks. liczba kt

2

27.

2

Uzyskana liczba pkt

MMA-IP

Strona 13 z24





{\it Egzamin maturalny z matematyki}

{\it Poziom podstawowy}

Zadanie 28. (2pktJ

Kąt $\alpha$ jest ostry $\displaystyle \mathrm{i}(\sin\alpha+\cos\alpha)^{2}=\frac{3}{2}$. Oblicz wartość wyrazenia $\sin\alpha\cdot\cos\alpha.$

Odpowiedzí :

Strona 14 z24

MMA-IP





{\it Egzamin maturalny z matematyki}

{\it Poziom podstawowy}

Zadanie 29. (2pkt)

Dany jest trójkąt prostokątny $ABC$. Na przyprostokątnych $AC\mathrm{i}$ AB tego trójkąta obrano

odpowiednio punkty $D\mathrm{i}G$. Na przeciwprostokątnej $BC$ wyznaczono punkty $E\mathrm{i}F$ takie, $\dot{\mathrm{z}}\mathrm{e}$

$|\wedge DEC|=|\triangleleft BGF|=90^{\mathrm{o}}$ (zobacz rysunek). Wykaz, $\dot{\mathrm{z}}\mathrm{e}$ trójkąt $CDE$ jest podobny do

trójkąta $FBG.$
\begin{center}
\includegraphics[width=87.528mm,height=55.476mm]{./F1_M_PP_M2016_page14_images/image001.eps}
\end{center}
{\it C}

{\it E}

{\it F}

{\it D}

{\it A  G B}
\begin{center}
\includegraphics[width=96.012mm,height=17.784mm]{./F1_M_PP_M2016_page14_images/image002.eps}
\end{center}
Wypelnia

egzaminator

Nr zadania

Maks. liczba kt

28.

2

2

Uzyskana liczba pkt

MMA-IP

Strona 15 z24





{\it Egzamin maturalny z matematyki}

{\it Poziom podstawowy}

Zadanie 30. (2pktJ

Ciąg $(a_{n})$ jest określony wzorem $a_{n}=2n^{2}+2n$ dla $n\geq 1$. Wykaz, $\dot{\mathrm{z}}\mathrm{e}$ suma $\mathrm{k}\mathrm{a}\dot{\mathrm{z}}$ dych dwóch

kolejnych wyrazów tego ciągu jest kwadratem liczby naturalnej.

Strona 16 z24

MMA-IP





{\it Egzamin maturalny z matematyki}

{\it Poziom podstawowy}

{\it Zadanie 3l}. ({\it 2pktJ}

$\mathrm{W}$ skończonym ciągu arytmetycznym $(a_{n})$ pierwszy wyraz $a_{1}$ jest równy 7 oraz ostatni

wyraz $a_{n}$ jest równy 89. Suma wszystkich wyrazów tego ciągujest równa 2016.

Oblicz, ile wyrazów ma ten ciąg.

Odpowied $\acute{\mathrm{z}}$:
\begin{center}
\includegraphics[width=96.012mm,height=17.832mm]{./F1_M_PP_M2016_page16_images/image001.eps}
\end{center}
Wypelnia

egzaminator

Nr zadania

Maks. liczba kt

30.

2

31.

2

Uzyskana liczba pkt

MMA-IP

Strona 17 z24





{\it Egzamin maturalny z matematyki}

{\it Poziom podstawowy}

Zadanie 32. (4pktJ

Jeden z kątów trójkąta jest trzy razy większy od mniejszego z dwóch pozostałych kątów,

które róznią się o $50^{\mathrm{o}}$. Oblicz kąty tego trójkąta.

Strona 18 z24

MMA-IP





{\it Egzamin maturalny z matematyki}

{\it Poziom podstawowy}

Odpowiedzí :
\begin{center}
\includegraphics[width=82.044mm,height=17.784mm]{./F1_M_PP_M2016_page18_images/image001.eps}
\end{center}
Nr zadanÍa

WypelnÍa Maks. liczba kt

egzaminator

Uzyskana liczba pkt

32.

4

MMA-IP

Strona 19 z24





{\it Egzamin maturalny z matematyki}

{\it Poziom podstawowy}

Zadanie 33. (5pktJ

Grupa znajomych wyjez $\mathrm{d}\dot{\mathrm{z}}$ ających na biwak wynajęła bus. Koszt wynajęcia busa jest równy

960 złotych i tę kwotę rozłozono po równo pomiędzy uczestników wyjazdu. Do grupy

wyjez $\mathrm{d}\dot{\mathrm{z}}$ ających dołączyło w ostatniej chwili dwóch znajomych. Wtedy koszt wyjazdu

przypadający na jednego uczestnika zmniejszył się o 16 złotych. Ob1icz, i1e osób wyjechało

na biwak.

Strona 20 z24

MMA-IP





{\it Egzamin maturalny z matematyki}

{\it Poziom podstawowy}

{\it BRUDNOPIS} ({\it nie podlega ocenie})

MMA-IP





{\it Egzamin maturalny z matematyki}

{\it Poziom podstawowy}
\begin{center}
\includegraphics[width=82.044mm,height=17.832mm]{./F1_M_PP_M2016_page20_images/image001.eps}
\end{center}
WypelnÍa

egzaminator

Nr zadanÍa

Maks. liczba kt

33.

5

Uzyskana liczba pkt

MMA-IP

Strona 21 z24





{\it Egzamin maturalny z matematyki}

{\it Poziom podstawowy}

Zadanie 34. (4pktJ

Ze zbioru wszystkich liczb naturalnych dwucyfrowych losujemy kolejno dwa razy po jednej

liczbie bez zwracania. Oblicz prawdopodobieństwo zdarzenia polegającego na tym, $\dot{\mathrm{z}}\mathrm{e}$ suma

wylosowanych liczb będzie równa 30. Wynik zapisz w postaci ułamka zwykłego

nieskracalnego.

Strona 22 z24

MMA-IP





{\it Egzamin maturalny z matematyki}

{\it Poziom podstawowy}

Odpowiedzí :
\begin{center}
\includegraphics[width=82.044mm,height=17.832mm]{./F1_M_PP_M2016_page22_images/image001.eps}
\end{center}
Wypelnia

egzaminator

Nr zadania

Maks. liczba kt

34.

4

Uzyskana liczba pkt

MMA-IP

Strona 23 z24





{\it Egzamin maturalny z matematyki}

{\it Poziom podstawowy}

{\it BRUDNOPIS} ({\it nie podlega ocenie})

Strona 24 z24

MMA-IP





{\it Egzamin maturalny z matematyki}

{\it Poziom podstawowy}

Zadam$\mathrm{e}8. (1pkt)$

Danajest ffinkcja liniowa $f(x)=\displaystyle \frac{3}{4}x+6$. Miejscem zerowym tej funkcjijest liczba

A. 8

B. 6

C. $-6$

D. $-8$

Zadanie $g. (1pktJ$

Równanie wymierne $\displaystyle \frac{3x-1}{x+5}=3$, gdzie $x\neq-5,$

A.

B.

C.

D.

nie ma rozwiązań rzeczywistych.

ma dokładniejedno rozwiązanie rzeczywiste.

ma dokładnie dwa rozwiązania rzeczywiste.

ma dokładnie trzy rozwiązania rzeczywiste.

Informacja do zadań 10. $\mathrm{i}l1.$

Na rysunku przedstawiony jest fragment paraboli będącej wykresem funkcji kwadratowej $f.$

Wierzchołkiem tej parabolijest punkt $W=(1,9)$. Liczby $-2\mathrm{i}4$ to miejsca zerowe funkcji $f.$
\begin{center}
\includegraphics[width=192.228mm,height=118.104mm]{./F1_M_PP_M2016_page3_images/image001.eps}
\end{center}
Zadanie 10. (1pkt)

Zbiorem wartości funkcji f jest przedział

A.

$(-\infty'-2\rangle$

B. $\langle-2,  4\rangle$

C.

$\langle 4,+\infty)$

D. $(-\infty$' $ 9\rangle$

Zadanie $ll. (1pkt)$

Najmniejsza wartość funkcji $f$ w przedziale $\langle-1,2\rangle$ jest równa

A. 2

B. 5

C. 8

D. 9

Strona 4 z24

MMA-IP





{\it Egzamin maturalny z matematyki}

{\it Poziom podstawowy}

{\it BRUDNOPIS} ({\it nie podlega ocenie})

MMA-IP

Strona 5 z24





{\it Egzamin maturalny z matematyki}

{\it Poziom podstawowy}

Zadanie $l2. (1pkt)$

Funkcja $f$ określona jest wzorem $f(x)=\displaystyle \frac{2x^{3}}{x^{6}+1}$ dla kazdej liczby rzeczywistej $x$. Wtedy

$f(-\sqrt[3]{3})$ jest równa

A.

$-\displaystyle \frac{\sqrt[3]{9}}{2}$

B.

- -53

C.

-53

D.

$\displaystyle \frac{\sqrt[3]{3}}{2}$

Zadanie 13. $(1pktJ$

$\mathrm{W}$ okręgu o środku w punkcie $S$ poprowadzono cięciwę AB, która utworzyła z promieniem

$AS$ kąt o mierze $31^{\mathrm{o}}$ (zobacz rysunek). Promień tego okręgu ma długość 10. Od1egłość punktu

$S$ od cięciwy $AB$ jest liczbą z przedziału

A. $\displaystyle \{\frac{9}{2},\frac{11}{2}\}$

B. $\displaystyle \frac{11}{2}, \displaystyle \frac{13}{2}$

C. $\displaystyle \frac{13}{2}, \displaystyle \frac{19}{2}$
\begin{center}
\includegraphics[width=72.588mm,height=76.200mm]{./F1_M_PP_M2016_page5_images/image001.eps}
\end{center}
$B$

{\it K}

{\it S}

31

{\it A}

$\displaystyle \frac{19}{2}, \displaystyle \frac{37}{2}$

D.

Zadanie 14. $(1pkt)$

Czternasty wyraz ciągu arytmetycznegojest równy 8, a róznica tego ciągujest równa $(-\displaystyle \frac{3}{2}).$

Siódmy wyraz tego ciągu jest równy

A.

$\displaystyle \frac{37}{2}$

B.

$-\displaystyle \frac{37}{2}$

C.

- -25

D.

-25

Zadanie 15. $(1pki)$

Ciąg $(x,2x+3,4x+3)$ jest geometryczny. Pierwszy wyraz tego ciągujest równy

A. $-4$

B. l

C. 0

D. $-1$

Zadanie 16. (1pkt)

Przedstawione na rysunku trójkąty ABCi PQR są podobne. Bok AB trójkąta ABC ma długość

A. 8

B. 8,5

C. 9,5
\begin{center}
\includegraphics[width=105.660mm,height=60.504mm]{./F1_M_PP_M2016_page5_images/image002.eps}
\end{center}
18

{\it Q}  $62^{\mathrm{o}}$  {\it R}

{\it C}

17

9

$70^{\mathrm{o}}$

$70^{\mathrm{o}}  48^{\mathrm{o}}$

{\it A B}

{\it x  P}

D. 10

Strona 6 z24

MMA-IP





{\it Egzamin maturalny z matematyki}

{\it Poziom podstawowy}

{\it BRUDNOPIS} ({\it nie podlega ocenie})

MMA-IP

Strona 7 z24





{\it Egzamin maturalny z matematyki}

{\it Poziom podstawowy}

Zadanie 17. $(1pkt)$

Kąt $\alpha$ jest ostry i $\displaystyle \mathrm{t}\mathrm{g}\alpha=\frac{2}{3}$. Wtedy

A.

$\displaystyle \sin\alpha=\frac{3\sqrt{13}}{26}$

B.

$\displaystyle \sin\alpha=\frac{\sqrt{13}}{13}$

C.

$\mathrm{s}$i$\displaystyle \mathrm{n}\alpha=\frac{2\sqrt{13}}{13}$

D.

$\displaystyle \sin\alpha=\frac{3\sqrt{13}}{13}$

Zadanie $l\mathrm{S}. (1pkt)$

$\mathrm{Z}$ odcinków o długościach: 5, $2a+1, a-1$ mozna zbudować trójkąt równoramienny. Wynika

stąd, $\dot{\mathrm{z}}\mathrm{e}$

A. $a=6$

B. $a=4$

C. $a=3$

D. $a=2$

Zadanie 19. (1pRt)

Okręgi o promieniach 3 i 4 są styczne zewnętrznie. Prosta styczna do okręgu

o promieniu 4 w punkcie P przechodzi przez środek okręgu o promieniu 3 (zobacz rysunek).
\begin{center}
\includegraphics[width=171.504mm,height=116.184mm]{./F1_M_PP_M2016_page7_images/image001.eps}
\end{center}
{\it P}

$O_{1}$  3 4  $O_{2}$

Pole trójkąta, którego wierzchołkami są środki okręgów i punkt styczności P, jest równe

A. 14

B. $2\sqrt{33}$

C. $4\sqrt{33}$

D. 12

Zadanie 20. $(1pkt)$

Proste opisane równaniami $y=\displaystyle \frac{2}{m-1}x+m-2$ oraz $y=mx+\displaystyle \frac{1}{m+1}$ są prostopadłe, gdy

A. $m=2$

B.

{\it m}$=$ -21

C.

{\it m}$=$ -31

D. $m=-2$

Strona 8 z24

MMA-IP





{\it Egzamin maturalny z matematyki}

{\it Poziom podstawowy}

{\it BRUDNOPIS} ({\it nie podlega ocenie})

MMA-IP

Strona 9 z24





{\it Egzamin maturalny z matematyki}

{\it Poziom podstawowy}

Zadanie $2l. (1pkt)$

$\mathrm{W}$ układzie współrzędnych dane są punkty $A=(a,6)$ oraz $B=(7,b)$. Środkiem odcinka $AB$

jest punkt $M=(3,4)$. Wynika stąd, $\dot{\mathrm{z}}\mathrm{e}$

A. $a=5 \mathrm{i}b=5$

B. $a=-1 \mathrm{i}b=2$

C. $a=4\mathrm{i}b=10$

D. $a=-4 \mathrm{i}b=-2$

Zadanie 22. (Ipkt)

Rzucamy trzy razy symetryczną monetą. Niech p oznacza prawdopodobieństwo otrzymania

dokładnie dwóch orłów w tych trzech rzutach. Wtedy

A. $0\leq p<0,2$

B. $0,2\leq p\leq 0,35$

C. $0,35<p\leq 0,5$

D. $0,5<p\leq 1$

Zadanie 23. $(1pki)$

Kąt rozwarcia stozka ma miarę $120^{\mathrm{o}}$, a tworząca tego stozka ma długość 4. Objętość tego

stozkajest równa

A. $ 36\pi$

B. $ 18\pi$

C. $ 24\pi$

D. $ 8\pi$

Zadanie 24. (1pki)

Przekątna podstawy graniastosłupa prawidłowego czworokątnego jest dwa razy dłuzsza od

wysokości graniastosłupa. Graniastosłup przecięto płaszczyzną przechodzącą przez przekątną

podstawy ijeden wierzchołek drugiej podstawy (patrz rysunek).

Płaszczyzna przekroju tworzy z podstawą graniastosłupa kąt $\alpha$ o mierze

A. $30^{\mathrm{o}}$

B. $45^{\mathrm{o}}$

C. $60^{\mathrm{o}}$

D. $75^{\mathrm{o}}$

Zadanie 25. $(1pki)$

Średnia arytmetyczna sześciu liczb naturalnych: 31, 16, 25, 29, 27, $x$, jest równa $\displaystyle \frac{x}{2}$. Mediana

tych liczb jest równa

A. 26

B. 27

C. 28

D. 29

Strona 10 z24

MMA-IP







Arkusz zawiera informacje prawnie chronione do momentu rozpoczęcia egzaminu.

UZUPELNIA ZDAJACY

KOD PESEL

{\it miejsce}

{\it na naklejkę}
\begin{center}
\includegraphics[width=21.432mm,height=9.852mm]{./F1_M_PP_M2017_page0_images/image001.eps}

\includegraphics[width=82.092mm,height=9.852mm]{./F1_M_PP_M2017_page0_images/image002.eps}

\includegraphics[width=204.060mm,height=216.048mm]{./F1_M_PP_M2017_page0_images/image003.eps}
\end{center}
EGZAMIN MATU

Z MATEMATY

LNY

POZIOM PODSTAWOWY

Instrukcja dla zdającego

l. Sprawdzí, czy arkusz egzaminacyjny zawiera 26 stron

(zadania $1-34$). Ewentualny brak zgłoś przewodniczącemu

zespo nadzorującego egzamin.

2. Rozwiązania zadań i odpowiedzi wpisuj w miejscu na to

przeznaczonym.

3. Odpowiedzi do zadań zamkniętych $(1-25)$ zaznacz

na karcie odpowiedzi, w części ka $\mathrm{y}$ przeznaczonej dla

zdającego. Zamaluj $\blacksquare$ pola do tego przeznaczone. Błędne

zaznaczenie otocz kółkiem $\mathrm{O}$ i zaznacz właściwe.

4. Pamiętaj, $\dot{\mathrm{z}}\mathrm{e}$ pominięcie argumentacji lub istotnych

obliczeń w rozwiązaniu zadania otwa ego (26-34) $\mathrm{m}\mathrm{o}\dot{\mathrm{z}}\mathrm{e}$

spowodować, $\dot{\mathrm{z}}\mathrm{e}$ za to rozwiązanie nie otrzymasz pełnej

liczby punktów.

5. Pisz czytelnie i $\mathrm{u}\dot{\mathrm{z}}$ aj tylko $\mathrm{d}$ gopisu lub pióra

z czarnym tuszem lub atramentem.

6. Nie $\mathrm{u}\dot{\mathrm{z}}$ aj korektora, a błędne zapisy wyra $\acute{\mathrm{z}}\mathrm{n}\mathrm{i}\mathrm{e}$ prze eśl.

7. Pamiętaj, $\dot{\mathrm{z}}\mathrm{e}$ zapisy w brudnopisie nie będą oceniane.

8. $\mathrm{M}\mathrm{o}\dot{\mathrm{z}}$ esz korzystać z zestawu wzorów matematycznych,

cyrkla i linijki oraz kalkulatora prostego.

9. Na tej stronie oraz na karcie odpowiedzi wpisz swój

numer PESEL i przyklej naklejkę z kodem.

10. Nie wpisuj $\dot{\mathrm{z}}$ adnych znaków w części przeznaczonej dla

egzaminatora.

5 MAJA 20I7

Godzina rozpoczęcia:

9:00

Czas pracy:

170 minut

Liczba punktów

do uzyskania: 50

$\Vert\Vert\Vert\Vert\Vert\Vert\Vert\Vert\Vert\Vert\Vert\Vert\Vert\Vert\Vert\Vert\Vert\Vert\Vert\Vert\Vert\Vert\Vert\Vert|  \mathrm{M}\mathrm{M}\mathrm{A}-\mathrm{P}1_{-}1\mathrm{P}-172$




{\it Egzamin maturalny z matematyki}

{\it Poziom podstawowy}

ZADANIA ZAMKNIĘTE

{\it Wzadaniach od l. do 25. wybierz i zaznacz na karcie odpowiedzipoprawnq} $odp\theta wied\acute{z}.$

Zadanie l. $(1pktJ$

Liczba $5^{8}.16^{-2}$ jest równa

A. $(\displaystyle \frac{5}{2})^{8}$ B.

-25

Zadanie 2. $(1pktJ$

Liczba $\sqrt[3]{54}-\sqrt[3]{2}$ jest równa

A. $\sqrt[3]{52}$

B. 3

Zadanie 3. $(1pktJ$

Liczba 2 $\log_{2}3-2\log_{2}5$ jest równa

A.

$\displaystyle \log_{2}\frac{9}{25}$

B.

$\log_{2} \displaystyle \frac{3}{5}$

C. $10^{8}$

D. 10

C. $\mathrm{z}\sqrt[3]{2}$

D. 2

C.

$\log_{2} \displaystyle \frac{9}{5}$

D.

$\displaystyle \log_{2}\frac{6}{25}$

Zadanie 4. (1pktJ

Liczba osobników pewnego zagrozonego wyginięciem gatunku zwierząt wzrosła w stosunku

do liczby tych zwierząt z 31 grudnia 2011 r. 0120\% i obecnie jest równa 8910. I1e zwierząt

liczyła populacja tego gatunku w ostatnim dniu 2011 roku?

A. 4050

B. 1782

C. 7425

D. 7128

Zadame 5. $(1pkt)$

Równość $(x\sqrt{2}-2)^{2}=(2+\sqrt{2})^{2}$ jest

A. prawdziwa dla $x=-\sqrt{2}.$

B. prawdziwa dla $x=\sqrt{2}.$

C. prawdziwa dla $x=-1.$

D. fałszywa dla $\mathrm{k}\mathrm{a}\dot{\mathrm{z}}$ dej liczby $x.$

Strona 2 z 26

MMA-IP





{\it Egzamin maturalny z matematyki}

{\it Poziom podstawowy}

{\it BRUDNOPIS} ({\it nie podlega ocenie})

MMA-IP

Strona ll z 26





{\it Egzamin maturalny z matematyki}

{\it Poziom podstawowy}

Zadanie 18. $(1pkt)$

Na rysunku przedstawiona jest prosta $k$ o równaniu $y=ax$, przechodząca przez punkt

$A=(2,-3)$ i przez początek układu współrzędnych, oraz zaznaczony jest kąt $\alpha$ nachylenia

tej prostej do osi $Ox.$
\begin{center}
\includegraphics[width=70.716mm,height=67.512mm]{./F1_M_PP_M2017_page11_images/image001.eps}
\end{center}
{\it k}

{\it y}

5

4

3

2

1

$\alpha$

{\it x}

$-5$ -$4  -3$ -$2$

$-1 0$ 1

$-1$

2 3  4 5

$-2$

$-3  -A$

$-4$

Zatem

A.

{\it a}$=$- -23

B.

{\it a}$=$- -23

C.

{\it a}$=$ -23

D.

{\it a}$=$ -23

Zadanie $l9*(1pkt)$

Na płaszczyz$\acute{}$nie z układem współrzędnych proste $k\mathrm{i} l$ przecinają się pod kątem prostym

w punkcie $A=(-2,4)$. Prosta $k$ jest określona równaniem $y=-\displaystyle \frac{1}{4}x+\frac{7}{2}$ Zatem prostą $l$

opisuje równanie

A.

{\it y}$=$ -41 {\it x}$+$ -27

B.

{\it y}$=$- -41 {\it x}- -27

C. $y=4x-12$

D. $y=4x+12$

Zadanie 20. $(1pkt)$

Dany jest okrąg o środku $S=(2,3)$ i promieniu $r=5$. Który z podanych punktów lezy na

tym okręgu?

A. $A=(-1,7)$

B. $B=(2,-3)$

C. $C=(3,2)$

D. $D=(5,3)$

Zadanie 21. (1pkt)

Pole powierzchni całkowitej graniastosiupa prawidłowego czworokątnego, w którym

wysokość jest 3 razy dłuzsza od krawędzi podstawy, jest równe 140. Zatem krawędz$\acute{}$

podstawy tego graniastosłupajest równa

A. $\sqrt{10}$

B.

$3\sqrt{10}$

C. $\sqrt{42}$

D. $3\sqrt{42}$

Strona 12 z 26

MMA-IP





{\it Egzamin maturalny z matematyki}

{\it Poziom podstawowy}

{\it BRUDNOPIS} ({\it nie podlega ocenie})

MMA-IP

Strona 13 z 26





{\it Egzamin maturalny z matematyki}

{\it Poziom podstawowy}

Zadanie 22. (1pkt)

Promień AS podstawy walca jest równy wysokości OS tego walca. Sinus kąta OAS (zobacz

rysunek) jest równy
\begin{center}
\includegraphics[width=49.272mm,height=39.984mm]{./F1_M_PP_M2017_page13_images/image001.eps}
\end{center}
{\it O}

{\it S}

{\it A}

A.

-21

B.

-$\sqrt{}$22

C.

-$\sqrt{}$23

D. l

Zadanie 23. (1pkt)

Dany jest stozek o wysokości 4 i średnicy podstawy 12. Objętość tego stozkajest równa

A. $ 576\pi$

B. $ 192\pi$

C. $ 144\pi$

D. $ 48\pi$

Zadanie 24. (1pkt)

Średnia arytmetyczna ośmiu liczb: 3, 5, 7, 9, x, 15, 17, 19jest równa 11. Wtedy

A. $x=1$

B. $x=2$

C. $x=11$

D. $x=13$

Zadanie 25. $(1pkt)$

Ze zbioru dwudziesm czterech kolejnych liczb naturalnych od l do 241osujemy jedną 1iczbę.

Niech $A$ oznacza zdarzenie, $\dot{\mathrm{z}}\mathrm{e}$ wylosowana liczba będzie dzielnikiem liczby 24. Wtedy

prawdopodobieństwo zdarzenia $A$ jest równe

A.

-41

B.

-31

C.

-81

D.

-61

Strona 14 z26

MMA-IP





{\it Egzamin maturalny z matematyki}

{\it Poziom podstawowy}

{\it BRUDNOPIS} ({\it nie podlega ocenie})

MMA-IP

Strona 15 z 26





{\it Egzamin maturalny z matematyki}

{\it Poziom podstawowy}

Zadanie 26. $(2pktJ$

Rozwiąz nierówność $8x^{2}-72x\leq 0.$

Odpowied $\acute{\mathrm{z}}$:

Strona 16 $\mathrm{z}26$

MMA-IP





{\it Egzamin maturalny z matematyki}

{\it Poziom podstawowy}

Zadanie 27, $(2pktJ$

Wykaz, $\dot{\mathrm{z}}\mathrm{e}$ liczba $4^{2017}+4^{2018}+4^{2019}+4^{2020}$ jest podzielna przez 17.
\begin{center}
\includegraphics[width=96.012mm,height=17.784mm]{./F1_M_PP_M2017_page16_images/image001.eps}
\end{center}
Wypelnia

egzamÍnator

Nr zadania

Maks. liczba kt

2

27.

2

Uzyskana liczba pkt

MMA-IP

Strona 17 z26





{\it Egzamin maturalny z matematyki}

{\it Poziom podstawowy}

Zadanie 2{\$}. $(2pktJ$

Dane są dwa okręgi o środkach w punktach $P \mathrm{i} R$, styczne zewnętrznie w punkcie $C.$

Prosta $AB$ jest styczna do obu okręgów odpowiednio w punktach $A \mathrm{i}B$ oraz $|\triangleleft APC|=\alpha$

$\mathrm{i}|<ABC|=\beta$ (zobacz rysunek). Wykaz, $\dot{\mathrm{z}}\mathrm{e}\alpha=180^{\mathrm{o}}-2\beta.$
\begin{center}
\includegraphics[width=190.908mm,height=44.904mm]{./F1_M_PP_M2017_page17_images/image001.eps}
\end{center}
{\it P}

$\alpha$  {\it C  R}

$(\beta$

{\it A}  -{\it B}

Strona 18 z26

MMA-IP





{\it Egzamin maturalny z matematyki}

{\it Poziom podstawowy}

Zadanie 29. $(4pkt)$

Funkcja kwadratowa $f$ jest określona dla wszystkich liczb rzeczywistych $x$ wzorem

$f(x)=ax^{2}+bx+c$. Największa wartość funkcji $f$ jest równa 6 oraz $f(-6)=f(0)=\displaystyle \frac{3}{2}.$

Oblicz wartość współczynnika $a.$

Odpowiedzí :
\begin{center}
\includegraphics[width=96.012mm,height=17.784mm]{./F1_M_PP_M2017_page18_images/image001.eps}
\end{center}
Wypelnia

egzamÍnator

Nr zadani,

Maks. liczba kt

28.

2

4

Uzyskana liczba pkt

MMA-IP

Strona 19 z26





{\it Egzamin maturalny z matematyki}

{\it Poziom podstawowy}

Zadanie 30. (2pkt)

Przeciwprostokątna trójkąta prostokątnego ma długość 26 cm, a jedna z przyprostokątnych

jest o 14 cm dłuzsza od drugiej. Ob1icz obwód tego trójkąta.

Odpowied $\acute{\mathrm{z}}$:

Strona 20 $\mathrm{z}26$

MMA-IP





{\it Egzamin maturalny z matematyki}

{\it Poziom podstawowy}

{\it BRUDNOPIS} ({\it nie podlega ocenie})

MMA-IP

Strona 3 z 26





{\it Egzamin maturalny z matematyki}

{\it Poziom podstawowy}

Zadanie 31. (2pkt)

$\mathrm{W}$ ciągu arytmetycznym $(a_{n})$, określonym dla $n\geq 1$, dane są: wyraz $a_{1}=8$ i suma trzech

początkowych wyrazów tego ciągu $S_{3}=33$. Oblicz róznicę $a_{16}-a_{13}.$

Odpowiedzí :
\begin{center}
\includegraphics[width=96.012mm,height=17.784mm]{./F1_M_PP_M2017_page20_images/image001.eps}
\end{center}
Wypelnia

egzamÍnator

Nr zadani,

Maks. liczba kt

30.

2

31.

2

Uzyskana liczba pkt

MMA-IP

Strona 21 z26





{\it Egzamin maturalny z matematyki}

{\it Poziom podstawowy}

Zadanie 32. $(SpktJ$

Dane są punkty $A=(-4,0) \mathrm{i}M=(2,9)$ oraz prosta $k$ o równaniu $y=-2x+10$. Wierzchołek

$B$ trójkąta $ABC$ to punkt przecięcia prostej $k$ z osią $Ox$ układu współrzędnych, a wierzchołek

$C$ jest punktem przecięcia prostej $k$ z prostą AM. Oblicz pole trójkąta $ABC.$

Odpowied $\acute{\mathrm{z}}$:

Strona 22 $\mathrm{z}26$

MMA-IP





{\it Egzamin maturalny z matematyki}

{\it Poziom podstawowy}

Zadanie 33. $(2pkt)$

Ze zbioru wszystkich liczb naturalnych dwucyfrowych losujemy jedną liczbę. Oblicz

prawdopodobieństwo zdarzenia, $\dot{\mathrm{z}}\mathrm{e}$ wylosujemy liczbę, która jest równocześnie mniejsza od

40 i podzielna przez 3. Wynik zapisz w postaci ułamka zwykłego nieskracalnego.

Odpowiedzí:
\begin{center}
\includegraphics[width=96.012mm,height=17.784mm]{./F1_M_PP_M2017_page22_images/image001.eps}
\end{center}
Wypelnia

egzamÍnator

Nr zadania

Maks. liczba kt

32.

5

33.

2

Uzyskana liczba pkt

MMA-IP

Strona 23 z 26





{\it Egzamin maturalny z matematyki}

{\it Poziom podstawowy}

Zadanie 34. $(4pktJ$

$\mathrm{W}$ ostrosłupie prawidłowym trójkątnym wysokość ściany bocznej prostopadła do krawędzi

podstawy ostrosłupa jest równa $\displaystyle \frac{5\sqrt{3}}{4}$, a pole powierzchni bocznej tego ostrosłupa jest

równe $\displaystyle \frac{15\sqrt{3}}{4}$. Oblicz objętość tego ostrosłupa.

Strona 24 z26

MMA-IP





{\it Egzamin maturalny z matematyki}

{\it Poziom podstawowy}

Odpowied $\acute{\mathrm{z}}$:
\begin{center}
\includegraphics[width=82.044mm,height=17.832mm]{./F1_M_PP_M2017_page24_images/image001.eps}
\end{center}
Wypelnia

egzaminator

Nr zadanÍa

Maks. lÍczba kt

34.

4

Uzyskana liczba pkt

MMA-IP

Strona 25 z26





{\it Egzamin maturalny z matematyki}

{\it Poziom podstawowy}

{\it BRUDNOPIS} ({\it nie podlega ocenie})

Strona 26 z26

MMA-IP





{\it Egzamin maturalny z matematyki}

{\it Poziom podstawowy}

Zadanie 6. $(1pkt)$

Do zbioru rozwiązań nierówności $(x^{4}+1)(2-x)>0$ nie nalez$\mathrm{v}$ liczba

A. $-3$

B. $-1$

C. l

D. 3

Zadam$\mathrm{e}7. (1pkt)$

Wskaz rysunek, na którym jest przedstawiony zbiór wszystkich rozwiązań nierówności

$2-3x\geq 4.$

A.
\begin{center}
\includegraphics[width=167.940mm,height=17.676mm]{./F1_M_PP_M2017_page3_images/image001.eps}
\end{center}
-23  {\it x}

B.
\begin{center}
\includegraphics[width=167.940mm,height=17.784mm]{./F1_M_PP_M2017_page3_images/image002.eps}
\end{center}
-23  {\it x}

C.
\begin{center}
\includegraphics[width=168.000mm,height=17.832mm]{./F1_M_PP_M2017_page3_images/image003.eps}
\end{center}
- -23  {\it x}

D.
\begin{center}
\includegraphics[width=168.048mm,height=17.832mm]{./F1_M_PP_M2017_page3_images/image004.eps}
\end{center}
- -23  {\it x}

Zadanie $S, (1pktJ$

Równanie $x(x^{2}-4)(x^{2}+4)=0$ z niewiadomą $x$

A. nie ma rozwiązań w zbiorze liczb rzeczywistych.

B. ma dokładnie dwa rozwiązania w zbiorze liczb rzeczywistych.

C. ma dokładnie trzy rozwiązania w zbiorze liczb rzeczywistych.

D. ma dokładnie pięć rozwiązań w zbiorze liczb rzeczywistych.

{\it Zadanie g}. ({\it lpkt})

Miejscem zerowym funkcji liniowej

$f(x)=\sqrt{3}(x+1)-12$ jest liczba

A. $\sqrt{3}-4$

B. $-2\sqrt{3}+1$

C. $4\sqrt{3}-1$

D. $-\sqrt{3}+12$

Strona 4 z 26

MMA-IP





{\it Egzamin maturalny z matematyki}

{\it Poziom podstawowy}

{\it BRUDNOPIS} ({\it nie podlega ocenie})

MMA-IP

Strona 5 z 26





{\it Egzamin maturalny z matematyki}

{\it Poziom podstawowy}

Zadanie 10. $(1pktJ$

Na rysunku przedstawiono fragment wykresu funkcji

o miejscach zerowych: $-3 \mathrm{i}1.$

kwadratowej $f(x)=ax^{2}+bx+c,$
\begin{center}
\includegraphics[width=86.004mm,height=100.380mm]{./F1_M_PP_M2017_page5_images/image001.eps}
\end{center}
{\it 5y}

)4

3

2

1

{\it X}

$\rightarrow 2$

$-4$

$-5$

Współczynnik c we wzorze funkcji f jest równy

A. l

B. 2

C. 3

D. 4

Zadanie ll. $(Ipkt)$

Na rysunku przedstawiono fragment wykresu funkcji wykładniczej $f$ określonej wzorem

$f(x)=a^{x}$. Punkt $A=(1,2)$ nalezy do tego wykresu funkcji.
\begin{center}
\includegraphics[width=143.052mm,height=75.588mm]{./F1_M_PP_M2017_page5_images/image002.eps}
\end{center}
Podstawa a potęgijest równa

A.

- -21

B.

-21

C. $-2$

D. 2

Strona 6 z 26

MMA-IP





{\it Egzamin maturalny z matematyki}

{\it Poziom podstawowy}

{\it BRUDNOPIS} ({\it nie podlega ocenie})

MMA-IP

Strona 7 z 26





{\it Egzamin maturalny z matematyki}

{\it Poziom podstawowy}

Zadanie 12. $(1pkt)$

$\mathrm{W}$ ciągu arytmetycznym $(a_{n})$, określonym dla $n\geq 1$, dane są: $a_{1}=5, a_{2}=11$. Wtedy

A. $a_{14}=71$

B. $a_{12}=71$

C. $a_{11}=71$

D. $a_{10}=71$

Zadanie 13. $(1pkt)$

Dany jest trzywyrazowy ciąg geometryczny $($24, 6, $a-1)$. Stąd wynika, $\dot{\mathrm{z}}\mathrm{e}$

A.

{\it a}$=$ -25

B.

{\it a}$=$ -25

C.

{\it a}$=$ -23

D.

{\it a}$=$ -23

Zadanie 14. $(1pkt)$

Jeśli $m=\sin 50^{\mathrm{o}}$, to

A.

$m=\sin 40^{\mathrm{o}}$

B. $m=\cos 40^{\mathrm{o}}$

C. $m=\cos 50^{\mathrm{o}}$

D. $m=\mathrm{t}\mathrm{g}50^{\mathrm{o}}$

Zadanie 15. (I pkt)

Na okręgu o środku w punkcie O lezy punkt C (zobacz rysunek). Odcinek AB jest średnicą

tego okręgu. Zaznaczony na rysunku kąt środkowy a ma miarę
\begin{center}
\includegraphics[width=70.260mm,height=66.552mm]{./F1_M_PP_M2017_page7_images/image001.eps}
\end{center}
{\it C}

$56^{\mathrm{o}}$

{\it A}

$\alpha$

{\it O}

{\it B}

A. $116^{\mathrm{o}}$

B. $114^{\mathrm{o}}$

C. $112^{\mathrm{o}}$

D. $110^{\mathrm{o}}$

Strona 8 z 26

MMA-IP





{\it Egzamin maturalny z matematyki}

{\it Poziom podstawowy}

{\it BRUDNOPIS} ({\it nie podlega ocenie})

MMA-IP

Strona 9 z 26





{\it Egzamin maturalny z matematyki}

{\it Poziom podstawowy}

Zadanie 16. $(1pktJ$

$\mathrm{W}$ trójkącie $ABC$ punkt $D$ lezy na boku $BC$, a punkt $E$ lezy na boku $AB$. Odcinek $DE$ jest

równoległy do boku $AC$, a ponadto $|BD|=10, |BC|=12 \mathrm{i}|AC|=24$ (zobacz rysunek).
\begin{center}
\includegraphics[width=117.756mm,height=49.020mm]{./F1_M_PP_M2017_page9_images/image001.eps}
\end{center}
{\it B}

10

{\it D}

2

{\it C}

{\it E}

{\it A}

24

Długość odcinka DE jest równa

A. 22 B. 20

C. 12

D. ll

{\it Zadanie l7}. ({\it lpktJ}

Obwód trójkąta ABC, przedstawionego na rysunku, jest równy

A. $(3+\displaystyle \frac{\sqrt{3}}{2})a$
\begin{center}
\includegraphics[width=78.132mm,height=48.816mm]{./F1_M_PP_M2017_page9_images/image002.eps}
\end{center}
{\it C}

{\it a}

$30^{\mathrm{o}}$

{\it A  B}

C. $(3+\sqrt{3})a$

B. $(2+\displaystyle \frac{\sqrt{2}}{2})a$

D. $(2+\sqrt{2})a$

Strona 10 z 26

MMA-IP







CENTRALNA

KOMISJA

EGZAMINACYJNA

Arkusz zawiera informacje prawnie chronione do momentu rozpoczęcia egzaminu.

UZUPELNIA ZDAJACY

KOD PESEL

{\it miejsce}

{\it na naklejkę}
\begin{center}
\includegraphics[width=21.432mm,height=9.852mm]{./F1_M_PP_M2018_page0_images/image001.eps}

\includegraphics[width=82.140mm,height=9.852mm]{./F1_M_PP_M2018_page0_images/image002.eps}

\includegraphics[width=204.060mm,height=216.048mm]{./F1_M_PP_M2018_page0_images/image003.eps}
\end{center}
EGZAMIN MATU LNY

Z MATEMATYKI

POZIOM PODSTAWOWY

Instrukcja dla zdającego

1. Sprawd $\acute{\mathrm{z}}$, czy arkusz egzaminacyjny zawiera 26 stron

(zadania $1-34$). Ewentualny brak zgłoś przewodniczącemu

zespo nadzorującego egzamin.

2. Rozwiązania zadań i odpowiedzi wpisuj w miejscu na to

przeznaczonym.

3. Odpowiedzi do zadań zam iętych $(1-25)$ zaznacz

na karcie odpowiedzi, w części ka $\mathrm{y}$ przeznaczonej dla

zdającego. Zamaluj $\blacksquare$ pola do tego przeznaczone. Błędne

zaznaczenie otocz kółkiem $\mathrm{O}$ i zaznacz właściwe.

4. Pamiętaj, $\dot{\mathrm{z}}\mathrm{e}$ pominięcie argumentacji lub istotnych

obliczeń w rozwiązaniu zadania otwa ego (26-34) $\mathrm{m}\mathrm{o}\dot{\mathrm{z}}\mathrm{e}$

spowodować, $\dot{\mathrm{z}}\mathrm{e}$ za to rozwiązanie nie otrzymasz pełnej

liczby punktów.

5. Pisz czytelnie i uzywaj tylko długopisu lub pióra

z czarnym tuszem lub atramentem.

6. Nie uzywaj korektora, a błędne zapisy wyrazínie prze eśl.

7. Pamiętaj, $\dot{\mathrm{z}}\mathrm{e}$ zapisy w brudnopisie nie będą oceniane.

8. $\mathrm{M}\mathrm{o}\dot{\mathrm{z}}$ esz korzystać z zestawu wzorów matematycznych,

cyrkla i linijki oraz kalkulatora prostego.

9. Na tej stronie oraz na karcie odpowiedzi wpisz swój

numer PESEL i przyklej naklejkę z kodem.

10. Nie wpisuj $\dot{\mathrm{z}}$ adnych znaków w części przeznaczonej dla

egzaminatora.

7 MAJA 20I8

Godzina rozpoczęcia:

Czas pracy:

170 minut

Liczba punktów

do uzyskania: 50

$\Vert\Vert\Vert\Vert\Vert\Vert\Vert\Vert\Vert\Vert\Vert\Vert\Vert\Vert\Vert\Vert\Vert\Vert\Vert\Vert\Vert\Vert\Vert\Vert|  \mathrm{M}\mathrm{M}\mathrm{A}-\mathrm{P}1_{-}1\mathrm{P}-182$




{\it Egzamin maturalny z matematyki}

{\it Poziom podstawowy}

ZADANIA ZAMKNIĘTE

$W$ {\it kazdym z zadań od l. do 25. wybierz i zaznacz na karcie odpowiedzipoprawnq odpowied} $\acute{z}.$

Zadanie l. $(1pktJ$

Liczba 2 $\log_{3}6-\log_{3}4$ jest równa

A. 4

B. 2

Zadanie 2. $(1pkt)$

Liczba $\sqrt[3]{\frac{7}{3}}\cdot\sqrt[3]{\frac{81}{56}}$ jest równa

A.

-$\sqrt{}$23

B.

$\displaystyle \frac{3}{2\sqrt[3]{21}}$

C. $2\log_{3}2$

D. $\log_{3}8$

C.

-23

D.

-49

Zadanie 3. $(1pkt)$

Dane są liczby $a=3,6\cdot 10^{-12}$ oraz $b=2,4\cdot 10^{-20}$. Wtedy iloraz $\displaystyle \frac{a}{b}$ jest równy

A. $8,64\cdot 10^{-32}$

B. $1,5\cdot 10^{-8}$

C. $1,5\cdot 10^{8}$

D. $8,64\cdot 10^{32}$

Zadame4. (1pkt)

Cena roweru po obnizce o 15\% była równa 850 zł. Przed tą obnizką rower ten kosztował

A. 865,00 zł

B. 850,15 zł

C. 1000,00 zł

D. 977,50 zł

Zadanie 5. $(1pkt)$

Zbiorem wszystkich rozwiązań nierówności $\displaystyle \frac{1-2x}{2}>\frac{1}{3}$ jest przedział

A.

(-$\infty$' -61)

B.

(-$\infty$' -23)

C.

$(\displaystyle \frac{1}{6},+\infty)$

D.

$(\displaystyle \frac{2}{3},+\infty)$

{\it Zadanie 6}. ({\it lpkt})

Funkcja kwadratowa określona jest wzorem

róznymi miejscami zerowymi ffinkcjif. Zatem

$f(x)=-2(x+3)(x-5)$. Liczby

$x_{1}, x_{2}$

są

A. $x_{1}+x_{2}=-8$

B. $x_{1}+x_{2}=-2$

C. $x_{1}+x_{2}=2$

D. $x_{1}+x_{2}=8$

Strona 2 z 26

MMA-IP





{\it Egzamin maturalny z matematyki}

{\it Poziom podstawowy}

{\it BRUDNOPIS} ({\it nie podlega ocenie})

MMA-IP

Strona ll z 26





{\it Egzamin maturalny z matematyki}

{\it Poziom podstawowy}

Zadanie 23. $(1pktJ$

$\mathrm{W}$ zestawie $\displaystyle \frac{2,2,2,\ldots,2}{m1\mathrm{i}\mathrm{c}\mathrm{z}\mathrm{b}}\frac{4,4,4,\ldots,4}{m1\mathrm{i}\mathrm{c}\mathrm{z}\mathrm{b}}$ jest $2m$ liczb $(m\geq 1)$ ` w tym $m$ liczb 2 $\mathrm{i} m$ liczb 4.

Odchylenie standardowe tego zestawu liczb jest równe

A. 2

B. l

C.

-$\sqrt{}$12

D. $\sqrt{2}$

Zadanie 24. $(1pktJ$

Ile jest wszystkich liczb naturalnych czterocyfrowych mniejszych $\mathrm{n}\mathrm{i}\dot{\mathrm{z}}$ 2018 i podzielnych

przez 5?

A. 402

B. 403

C. 203

D. 204

Zadanie 25, $(1pktJ$

$\mathrm{W}$ pudełku jest 50 kuponów, wśród których jest 15 kuponów przegrywających, a pozostałe

kupony są wygrywające. $\mathrm{Z}$ tego pudełka w sposób losowy wyciągamy jeden kupon.

Prawdopodobieństwo zdarzenia polegającego na tym, $\dot{\mathrm{z}}\mathrm{e}$ wyciągniemy kupon wygrywający, jest

równe

A.

$\displaystyle \frac{15}{35}$

B.

$\displaystyle \frac{1}{50}$

C.

$\displaystyle \frac{15}{50}$

D.

$\displaystyle \frac{35}{50}$

Strona 12 z 26

MMA-IP





{\it Egzamin maturalny z matematyki}

{\it Poziom podstawowy}

{\it BRUDNOPIS} ({\it nie podlega ocenie})

MMA-IP

Strona 13 z 26





{\it Egzamin maturalny z matematyki}

{\it Poziom podstawowy}

Zadanie 26. $(2pktJ$

Rozwiąz nierówność $2x^{2}-3x>5.$

Odpowiedzí :

Strona 14 z 26

MMA-IP





{\it Egzamin maturalny z matematyki}

{\it Poziom podstawowy}

Zadanie 27, $(2pktJ$

Rozwiąz równanie $x^{3}-7x^{2}-4x+28=0.$

Odpowiedzí :
\begin{center}
\includegraphics[width=96.012mm,height=17.832mm]{./F1_M_PP_M2018_page14_images/image001.eps}
\end{center}
Wypelnia

egzaminator

Nr zadania

Maks. liczba kt

2

27.

2

Uzyskana liczba pkt

MMA-IP

Strona 15 z 26





{\it Egzamin maturalny z matematyki}

{\it Poziom podstawowy}

Zadanie 2{\$}. $(2pktJ$

Udowodnij, $\dot{\mathrm{z}}\mathrm{e}$ dla dowolnych liczb dodatnich $a, b$ prawdziwajest nierówność

$\displaystyle \frac{1}{2a}+\frac{1}{2f_{i}}\geq\frac{2}{a+b}.$

Strona 16 z 26

MMA-IP





{\it Egzamin maturalny z matematyki}

{\it Poziom podstawowy}

Zadanie 29. $(2pktJ$

Okręgi o środkach odpowiednio $A\mathrm{i}B$ są styczne zewnętrznie i $\mathrm{k}\mathrm{a}\dot{\mathrm{z}}\mathrm{d}\mathrm{y}$ z nichjest styczny do obu

ramion danego kąta prostego (zobacz rysunek). Promień okręgu o środku $A$ jest równy 2.

{\it A}.

{\it B}.

Uzasadnij, $\dot{\mathrm{z}}\mathrm{e}$ promień okręgu o środku $B$ jest mniejszy od $\sqrt{2}-1.$
\begin{center}
\includegraphics[width=96.012mm,height=17.784mm]{./F1_M_PP_M2018_page16_images/image001.eps}
\end{center}
Wypelnia

egzaminator

Nr zadania

Maks. liczba kt

28.

2

2

Uzyskana liczba pkt

MMA-IP

Strona 17 z 26





{\it Egzamin maturalny z matematyki}

{\it Poziom podstawowy}

Zadanie 30. $(2pkt)$

Do wykresu funkcji wykładniczej, określonej

$f(x)=a^{x}$ (gdzie $a>0 \mathrm{i} a\neq 1$), nalezy punkt

ffinkcji $g$, określonej wzorem $g(x)=f(x)-2$

dla $\mathrm{k}\mathrm{a}\dot{\mathrm{z}}$ dej liczby rzeczywistej $x$ wzorem

$P=(2,9)$. Oblicz $a$ i zapisz zbiór wartości

Odpowiedzí :

Strona 18 z 26

MMA-IP





{\it Egzamin maturalny z matematyki}

{\it Poziom podstawowy}

Zadanie 31. $(2pktJ$

Dwunasty wyraz ciągu arytmetycznego $(a_{n})$, określonego dla $n\geq 1$, jest równy 30, a sumajego

dwunastu początkowych wyrazówjest równa 162. Ob1icz pierwszy wyraz tego ciągu.

Odpowiedzí :
\begin{center}
\includegraphics[width=96.012mm,height=17.832mm]{./F1_M_PP_M2018_page18_images/image001.eps}
\end{center}
Wypelnia

egzaminator

Nr zadania

Maks. liczba kt

30.

2

31.

2

Uzyskana liczba pkt

MMA-IP

Strona 19 z 26





{\it Egzamin maturalny z matematyki}

{\it Poziom podstawowy}

Zadanie 32. $(SpktJ$

$\mathrm{W}$ układzie współrzędnych punkty $A=(4,3) \mathrm{i} B=(10,5)$ są wierzchołkami trójkąta $ABC.$

Wierzchołek $C$ lezy na prostej o równaniu $y=2x+3$. Oblicz współrzędne punktu $C$, dla którego

kąt $ABC$ jest prosty.

Strona 20 z 26

MMA-IP





{\it Egzamin maturalny z matematyki}

{\it Poziom podstawowy}

{\it BRUDNOPIS} ({\it nie podlega ocenie})

MMA-IP

Strona 3 z 26





{\it Egzamin maturalny z matematyki}

{\it Poziom podstawowy}

Odpowiedzí :
\begin{center}
\includegraphics[width=82.044mm,height=17.832mm]{./F1_M_PP_M2018_page20_images/image001.eps}
\end{center}
Wypelnia

egzaminator

Nr zadania

Maks. liczba kt

32.

5

Uzyskana liczba pkt

MMA-IP

Strona 21 z 26





{\it Egzamin maturalny z matematyki}

{\it Poziom podstawowy}

Zadanie 33. $(4pktJ$

Dane są dwa zbiory: $A=\{100$, 200, 300, 400, 500, 600, 700$\} \mathrm{i} B=\{10$, 11, 12, 13, 14, 15, 16$\}.$

$\mathrm{Z}\mathrm{k}\mathrm{a}\dot{\mathrm{z}}$ dego z nich losujemyjedną liczbę. Oblicz prawdopodobieństwo zdarzenia polegającego na

tym, $\dot{\mathrm{z}}\mathrm{e}$ suma wylosowanych liczb będzie podzielna przez 3. Ob1iczone prawdopodobieństwo

zapisz w postaci nieskracalnego ułamka zwykłego.

Strona 22 z 26

MMA-IP





{\it Egzamin maturalny z matematyki}

{\it Poziom podstawowy}

Odpowiedzí :
\begin{center}
\includegraphics[width=82.044mm,height=17.832mm]{./F1_M_PP_M2018_page22_images/image001.eps}
\end{center}
Wypelnia

egzaminator

Nr zadania

Maks. liczba kt

33.

4

Uzyskana liczba pkt

MMA-IP

Strona 23 z 26





{\it Egzamin maturalny z matematyki}

{\it Poziom podstawowy}

Zadanie 34. $(4pktJ$

Dany jest graniastosłup prawidłowy trójkątny (zobacz rysunek). Pole powierzchni całkowitej

tego graniastosłupa jest równe $45\sqrt{3}$. Pole podstawy graniastosłupa jest równe polu jednej

ściany bocznej. Oblicz objętość tego graniastosłupa.
\begin{center}
\includegraphics[width=61.980mm,height=42.828mm]{./F1_M_PP_M2018_page23_images/image001.eps}
\end{center}
{\it F}

{\it E}

{\it C  D}

{\it B}

{\it A}

Strona 24 z 26

MMA-IP





{\it Egzamin maturalny z matematyki}

{\it Poziom podstawowy}

Odpowiedzí :
\begin{center}
\includegraphics[width=82.044mm,height=17.832mm]{./F1_M_PP_M2018_page24_images/image001.eps}
\end{center}
Wypelnia

egzaminator

Nr zadania

Maks. liczba kt

34.

4

Uzyskana liczba pkt

MMA-IP

Strona 25 z 26





{\it Egzamin maturalny z matematyki}

{\it Poziom podstawowy}

{\it BRUDNOPIS} ({\it nie podlega ocenie})

Strona 26 z 26

MMA-IP





{\it Egzamin maturalny z matematyki}

{\it Poziom podstawowy}

Zadanie 7. $(1pkt)$

Równanie $\displaystyle \frac{x^{2}+2x}{x^{2}-4}=0$

A. ma trzy rozwiązania: $x=-2, x=0, x=2$

B. ma dwa rozwiązania: $x=0, x=2$

C. ma dwa rozwiązania: $x=-2, x=2$

D. majedno rozwiązanie: $x=0$

Zadanie S, (lpkt)

Funkcja liniowa f określona jest wzorem

rzeczywistych x. Wskaz zdanie prawdziwe.

$f(x)=\displaystyle \frac{1}{3}x-1,$

dla wszystkich

liczb

A. Funkcja $f$ jest malejącaijej wykres przecina oś $oy$ w punkcie $P=(0,\displaystyle \frac{1}{3}).$

B. Funkcja $f$ jest malejącaijej wykres przecina oś $Oy$ w punkcie $P=(0,-1).$

C. Funkcja $f$ jest rosnąca ijej wykres przecina oś $oy$ w punkcie $P=(0,\displaystyle \frac{1}{3}).$

D. Funkcja $f$ jest rosnącaijej wykres przecina oś $Oy$ w punkcie $P=(0,-1).$

Zadam$\mathrm{e}9. (1pkt)$

Wykresem funkcji kwadratowej $f(x)=x^{2}-6x-3$ jest parabola, której wierzchołkiem jest

punkt o współrzędnych

A. $(-6,-3)$

B. $(-6,69)$

C. $(3,-12)$

D. $(6,-3)$

Zadanie $l0. (1pkt)$

Liczba l jest miejscem zerowym funkcji liniowej $f(x)=ax+b$, a punkt $M=(3,-2)$ nalezy

do wykresu tej funkcji. Współczynnik $a$ we wzorze tej funkcjijest równy

A. l

B.

-23

C.

- -23

D. $-1$

Zadanie ll. $(1pktJ$

Dany jest ciąg $(a_{n})$ jest określony wzorem $a_{n}=\displaystyle \frac{5-2n}{6}$ dla $n\geq 1$. Ciąg tenjest

A.

B.

C.

D.

arytmetyczny ijego róznicajest równa $r=-\displaystyle \frac{1}{3}$

arytmetyczny ijego róznicajest równa $r=-2.$

geometryczny ijego iloraz jest równy $q=-\displaystyle \frac{1}{3}.$

geometryczny ijego iloraz jest równy $q=\displaystyle \frac{5}{6}$

Strona 4 $\mathrm{z}26$

MMA-IP





{\it Egzamin maturalny z matematyki}

{\it Poziom podstawowy}

{\it BRUDNOPIS} ({\it nie podlega ocenie})

MMA-IP

Strona 5 z 26





{\it Egzamin maturalny z matematyki}

{\it Poziom podstawowy}

Zadanie 12. $(1pktJ$

Dla ciągu arytmetycznego $(a_{n})$, określonego dla $n\geq 1$, jest spetniony wamnek $a_{4}+a_{5}+a_{6}=12.$

Wtedy

A. $a_{5}=4$

B. $a_{5}=3$

C. $a_{5}=6$

D. $a_{5}=5$

Zadanie $l3. (1pktJ$

Dany jest ciąg geometryczny $(a_{n})$, określony dla $n\geq 1$, w którym $a_{1}=\sqrt{2},$

$a_{3}=4\sqrt{2}$. Wzór na n-ty wyraz tego ciągu ma postać

$a_{2}=2\sqrt{2},$

A. $a_{n}=(\sqrt{2})^{n}$

B.

{\it an}$=$ -$\sqrt{}$22{\it n}

C.

{\it an}$=$(-$\sqrt{}$22){\it n}

D.

$a_{n}=\displaystyle \frac{(\sqrt{2})}{2}n$

Zadanie 14. (1pkt)

Przyprostokątna LM trójkąta prostokątnego KLM ma długość 3, a przeciwprostokątna KL ma

długość 8 (zobacz rysunek).

3
\begin{center}
\includegraphics[width=82.656mm,height=35.808mm]{./F1_M_PP_M2018_page5_images/image001.eps}
\end{center}
{\it L}

8

$\alpha$

{\it M  K}

Wówczas miara $\alpha$ kąta ostrego $LMK$ tego trójkąta spełnia waiunek

A. $27^{\mathrm{o}}<\alpha\leq 30^{\mathrm{o}}$

B. $24^{\mathrm{o}}<\alpha\leq 27^{\mathrm{o}}$

C. $21^{\mathrm{o}}<\alpha\leq 24^{\mathrm{o}}$

D. $18^{\mathrm{o}}<\alpha\leq 21^{\mathrm{o}}$

Zadanie 15. $(1pkt)$

Dany jest trójkąt o bokach długości: $2\sqrt{5}, 3\sqrt{5}, 4\sqrt{5}$. Trójkątem podobnym do tego trójkąta

jest trójkąt, którego boki mają długości

A. 10, 15, 20

B. 20, 45, 80

C. $\sqrt{2}, \sqrt{3}, \sqrt{4}$

D. $\sqrt{5}, 2\sqrt{5}, 3\sqrt{5}$

Strona 6 z 26

MMA-IP





{\it Egzamin maturalny z matematyki}

{\it Poziom podstawowy}

{\it BRUDNOPIS} ({\it nie podlega ocenie})

MMA-IP

Strona 7 z 26





{\it Egzamin maturalny z matematyki}

{\it Poziom podstawowy}

Zadanie 16. $(1pkt)$

Dany jest okrąg o środku $S$. Punkty $K, L\mathrm{i}M$ lez$\cdot$ą na tym okręgu. Na łuku $KL$ tego okręgu są

oparte kąty $KSL \mathrm{i} KML$ (zobacz rysunek), których miary a $\mathrm{i} \beta$, spełniają warunek

$\alpha+\beta=111^{\mathrm{o}}$. Wynika stąd, $\dot{\mathrm{z}}\mathrm{e}$
\begin{center}
\includegraphics[width=64.368mm,height=61.620mm]{./F1_M_PP_M2018_page7_images/image001.eps}
\end{center}
{\it M}

{\it K  L}

A. $\alpha=74^{\mathrm{o}}$

B. $\alpha=76^{\mathrm{o}}$

C. $\alpha=70^{\mathrm{o}}$

D. $\alpha=72^{\mathrm{o}}$

Zadanie 17. $(1pkt)$

Dany jest trapez prostokątny KLMN, którego podstawy mają długoŚci $|KL|=a, |MN|=b,$

$a>b$. Kąt $KLM$ ma miarę $60^{\mathrm{o}}$. DługoŚć ramienia $LM$ tego trapezujest równa

{\it N b}

{\it M}
\begin{center}
\includegraphics[width=89.712mm,height=41.556mm]{./F1_M_PP_M2018_page7_images/image002.eps}
\end{center}
{\it K  L}

{\it a}

A. $a-b$

B. $2(a-b)$

C.

$a+\displaystyle \frac{1}{2}b$

D.

-{\it a} $+$2 {\it b}

Zadanie $l8.(1pkt)$

Średnicą okręgu jest odcinek $Kl$, gdzie $K=(6,8), L=(-6,-8)$. Równanie tego okręgu ma

postać

A. $x^{2}+y^{2}=200$

B. $x^{2}+y^{2}=100$

C. $x^{2}+y^{2}=400$

D. $x^{2}+y^{2}=300$

Zadanie $1\vartheta. (1pkt)$

Proste o równaniach $y=(m+2)x+3$ oraz $y=(2m-1)x-3$ są równoległe, gdy

A. $m=2$

B. $m=3$

C. $m=0$

D. $m=1$

Strona 8 z 26

MMA-IP





{\it Egzamin maturalny z matematyki}

{\it Poziom podstawowy}

{\it BRUDNOPIS} ({\it nie podlega ocenie})

MMA-IP

Strona 9 z 26





{\it Egzamin maturalny z matematyki}

{\it Poziom podstawowy}

Zadanie 20. $(1pktJ$

Podstawą ostrosłupa jest kwadrat KLMN o boku długości 4. Wysokością tego ostrosłupajest

krawędzí $NS$, ajej długość $\mathrm{t}\mathrm{e}\dot{\mathrm{z}}$ jest równa 4 (zobacz rysunek).

Kąt $\alpha$, jaki tworzą krawędzie $KS\mathrm{i}MS$, spełnia warunek

A. $\alpha=45^{\mathrm{o}}$

B. $45^{\mathrm{o}}<\alpha<60^{\mathrm{o}}$

C. $a>60^{\mathrm{o}}$

D. $\alpha=60^{\mathrm{o}}$

Zadanie 21. $(1pkt)$

Podstawą graniastosłupa prostegojest prostokąt o bokach długości 3 $\mathrm{i}4$. Kąt $\alpha$, jaki przekątna

tego graniastosłupa tworzy zjego podstawą, jest równy $45^{\mathrm{o}}$ (zobacz rysunek).

Wysokość graniastosłupa jest równa

A. 5

B. $3\sqrt{2}$

C. $5\sqrt{2}$

D.

$\displaystyle \frac{5\sqrt{3}}{3}$

Zadanie 22. (1pkt)

Na rysunku przedstawiono bryłę zbudowaną z walca i półkuli. Wysokość walcajest równa r

ijest taka samajak promień półkuli oraz taka sama jak promień podstawy walca.

Objętość tej bryłyjest równa

A.

$\displaystyle \frac{5}{3}\pi r^{3}$

B.

$\displaystyle \frac{4}{3}\pi r^{3}$

C.

$\displaystyle \frac{2}{3}\pi r^{3}$

D.

$\displaystyle \frac{1}{3}\pi r^{3}$

Strona 10 z 26

MMA-IP







CENTRALNA

KOMISJA

EGZAMiNACYJNA

Arkusz zawiera informacje prawnie chronione do momentu rozpoczęcia egzaminu.

UZUPELNIA ZDAJACY

KOD PESEL

{\it miejsce}

{\it na naklejkę}
\begin{center}
\includegraphics[width=21.432mm,height=9.852mm]{./F1_M_PP_M2019_page0_images/image001.eps}

\includegraphics[width=82.140mm,height=9.852mm]{./F1_M_PP_M2019_page0_images/image002.eps}

\includegraphics[width=204.060mm,height=216.108mm]{./F1_M_PP_M2019_page0_images/image003.eps}
\end{center}
EGZAMIN MATU LNY

Z MATEMATYKI

POZIOM PODSTAWOWY

Instrukcja dla zdającego

1. Sprawd $\acute{\mathrm{z}}$, czy arkusz egzaminacyjny zawiera 26 stron

(zadania $1-34$). Ewentualny brak zgłoś przewodniczącemu

zespo nadzorującego egzamin.

2. Rozwiązania zadań i odpowiedzi wpisuj w miejscu na to

przeznaczonym.

3. Odpowiedzi do zadań zam iętych $(1-25)$ zaznacz

na karcie odpowiedzi, w części ka $\mathrm{y}$ przeznaczonej dla

zdającego. Zamaluj $\blacksquare$ pola do tego przeznaczone. Błędne

zaznaczenie otocz kółkiem \copyright i zaznacz właściwe.

4. Pamiętaj, $\dot{\mathrm{z}}\mathrm{e}$ pominięcie argumentacji lub istotnych

obliczeń w rozwiązaniu zadania otwa ego (26-34) $\mathrm{m}\mathrm{o}\dot{\mathrm{z}}\mathrm{e}$

spowodować, $\dot{\mathrm{z}}\mathrm{e}$ za to rozwiązanie nie otrzymasz pełnej

liczby punktów.

5. Pisz czytelnie i uzywaj tylko długopisu lub pióra

z czamym tuszem lub atramentem.

6. Nie uzywaj korektora, a błędne zapisy wyra $\acute{\mathrm{z}}\mathrm{n}\mathrm{i}\mathrm{e}$ prze eśl.

7. Pamiętaj, $\dot{\mathrm{z}}\mathrm{e}$ zapisy w brudnopisie nie będą oceniane.

8. $\mathrm{M}\mathrm{o}\dot{\mathrm{z}}$ esz korzystać z zestawu wzorów matematycznych,

cyrkla i linijki oraz kalkulatora prostego.

9. Na tej stronie oraz na karcie odpowiedzi wpisz swój

numer PESEL i przyklej naklejkę z kodem.

10. Nie wpisuj $\dot{\mathrm{z}}$ adnych znaków w części przeznaczonej dla

egzaminatora.

Godzina rozpoczęcia:

9:00

Czas pracy:

170 minut

Liczba punktów

do uzyskania: 50

$\Vert\Vert\Vert\Vert\Vert\Vert\Vert\Vert\Vert\Vert\Vert\Vert\Vert\Vert\Vert\Vert\Vert\Vert\Vert\Vert\Vert\Vert\Vert\Vert|  \mathrm{M}\mathrm{M}\mathrm{A}-\mathrm{P}1_{-}1\mathrm{P}-192$




{\it Egzamin maturalny z matematyki}

{\it Poziom podstawowy}

ZADANIA ZAMKNIĘTE

$W$ {\it kazdym z zadań} $\theta d1.$ {\it do 25. wybierz i zaznacz na karcie} $\theta owiedipprawnq$ {\it odpowiedz}$\acute{}$.{\it í}

Zadanie l. $(1pkt)$

Liczba $\log_{\sqrt{2}}2$ jest równa

A. 2

B. 4

C.

$\sqrt{2}$

D.

-21

Zadanie 2. $(1pkt)$

Liczba naturalna $n=2^{14}\cdot 5^{15}$ w zapisie dziesiętnym ma

A. 14 cyfr

B. 15 cyfr

C. 16 cyfr

D. 30 cyfr

$\mathrm{Z}_{\vartheta}\mathrm{d}\mathrm{a}\mathrm{n}\mathrm{i}\S 3. (1pkt)$

$\mathrm{W}$ pewnym banku prowizja od udzielanych kredytów hipotecznych przez cały styczeń była

równa 4\%. Na początku 1utego ten bank obnizył wysokość prowizji od wszystkich kredytów

$0 1$ punkt procentowy. Oznacza to, $\dot{\mathrm{z}}\mathrm{e}$ prowizja od kredytów hipotecznych w tym banku

zmniejszyła się o

A. l\%

B. 25\%

C. 33\%

D. 75\%

Zadanie 4, $(1pkt)$

Równość $\displaystyle \frac{1}{4}+\frac{1}{5}+\frac{1}{a}=1$ jest prawdziwa dla

A.

$a=\displaystyle \frac{11}{20}$

B.

{\it a}$=$ -98

C.

{\it a}$=$ -98

D.

{\it a}$=$ -2101

Zadanie 5. $(1pkt)$

Para liczb $x=2 \mathrm{i}y=2$ jest rozwiązaniem układu równań 

A. $a=-1$

B. $a=1$

C. $a=-2$

D. $a=2$

Zadanie $\epsilon. (1pkt)$

Równanie $\displaystyle \frac{(x-1)(x+2)}{x-3}=0$

A. ma trzy rózne rozwiązania: $x=1, x=3, x=-2.$

B. ma trzy rózne rozwiązania: $x=-1, x=-3, x=2.$

C. ma dwa rózne rozwiązania: $x=1, x=-2.$

D. ma dwa rózne rozwiązania: $x=-1, x=2.$

Strona 2 z26

MMA-IP





{\it Egzamin maturalny z matematyki}

{\it Poziom podstawowy}

{\it BRUDNOPIS} ({\it nie podlega ocenie})

Strona ll z 26





{\it Egzamin maturalny z matematyki}

{\it Poziom podstawowy}

Zadanie 22. (1pktJ

Podstawą ostrosłupa prawidłowego czworokątnego ABCDS jest kwadrat ABCD. Wszystkie

ściany boczne tego ostrosłupa są trójkątami równobocznymi.

Miara kąta SAC jest równa

A. $90^{\mathrm{o}}$

B. $75^{\mathrm{o}}$

C. $60^{\mathrm{o}}$

D. $45^{\mathrm{o}}$

Zadanie 23. (1pkt)

Mediana zestawu sześciu danych liczb: 4, 8, 21, a, 16, 25, jest równa 14. Zatem

A. $a=7$

B. $a=12$

C. $a=14$

D. $a=20$

Zadanie 24. (1pkt)

Wszystkich liczb pięciocyfrowych, w których występują wyłącznie cyfry 0, 2, 5, jest

A.

12

B. 36

C. 162

D. 243

Zadanie 25. $(1pkt)$

$\mathrm{W}$ pudełku jest 40 ku1. Wśród nich jest 35 ku1 białych, a pozostałe to ku1e czerwone.

Prawdopodobieństwo wylosowania $\mathrm{k}\mathrm{a}\dot{\mathrm{z}}$ dej kulijest takie samo. $\mathrm{Z}$ pudełka losujemyjedną kulę.

Prawdopodobieństwo zdarzenia polegającego na tym, $\dot{\mathrm{z}}\mathrm{e}$ otrzymamy kulę czerwoną, jest równe

A.

-81

B.

-51

C.

$\displaystyle \frac{1}{40}$

D.

$\displaystyle \frac{1}{35}$

Strona 12 z 26

MMA-IP





{\it Egzamin maturalny z matematyki}

{\it Poziom podstawowy}

{\it BRUDNOPIS} ({\it nie podlega ocenie})

Strona 13 z 26





{\it Egzamin maturalny z matematyki}

{\it Poziom podstawowy}

Zadanie 26. $(2pktJ$

Rozwiąz równanie $x^{3}-5x^{2}-9x+45=0.$

Odpowied $\acute{\mathrm{z}}$:

Strona 14 $\mathrm{z}26$





{\it Egzamin maturalny z matematyki}

{\it Poziom podstawowy}

Zadanie 27, $(2pktJ$

Rozwiąz nierównoŚć $3x^{2}-16x+16>0.$

Odpowiedzí :
\begin{center}
\includegraphics[width=96.012mm,height=17.784mm]{./F1_M_PP_M2019_page14_images/image001.eps}
\end{center}
Wypelnia

egzaminator

Nr zadania

Maks. liczba kt

2

27.

2

Uzyskana liczba pkt

MMA-IP

Strona 15 z 26





{\it Egzamin maturalny z matematyki}

{\it Poziom podstawowy}

Zadanie 2{\$}. $(2pktJ$

Wykaz, $\dot{\mathrm{z}}\mathrm{e}$ dla dowolnych liczb rzeczywistych $a\mathrm{i}b$ prawdziwajest nierówność

$3a^{2}-2ab+3b^{2}\geq 0.$

Strona 16 z 26





{\it Egzamin maturalny z matematyki}

{\it Poziom podstawowy}

Zadanie 29. $(2pktJ$

Danyjest okrąg o środku w punkcie $S$ i promieniu $r$. Na przedłuzeniu cięciwy $AB$ poza punkt $B$

odłozono odcinek $BC$ równy promieniowi danego okręgu. Przez punkty $C\mathrm{i}S$ poprowadzono

prostą. Prosta $CS$ przecina dany okrąg w punktach $D\mathrm{i}E$ (zobacz rysunek). Wykaz, $\dot{\mathrm{z}}$ ejezeli

miara kąta ACSjest równa $\alpha$, to miara kąta $ASD$ jest równa $3\alpha.$
\begin{center}
\includegraphics[width=96.924mm,height=62.580mm]{./F1_M_PP_M2019_page16_images/image001.eps}
\end{center}
{\it D  r}

{\it S}

{\it r}

{\it E}

{\it r  r}

{\it C}

{\it A  B}
\begin{center}
\includegraphics[width=96.012mm,height=17.832mm]{./F1_M_PP_M2019_page16_images/image002.eps}
\end{center}
Wypelnia

egzaminator

Nr zadania

Maks. liczba kt

28.

2

2

Uzyskana liczba pkt

MMA-IP

Strona 17 z 26





{\it Egzamin maturalny z matematyki}

{\it Poziom podstawowy}

Zadanie 30. (2pktJ

Ze zbioru liczb \{1, 2, 3, 4, 5\} 1osujemy dwa razy po jednej 1iczbie ze zwracaniem. Ob1icz

prawdopodobieństwo zdarzenia A polegającego na wylosowaniu liczb, których iloczyn jest

liczbą nieparzystą.

Odpowied $\acute{\mathrm{z}}$:

Strona 18 $\mathrm{z}26$

MMA-IP





{\it Egzamin maturalny z matematyki}

{\it Poziom podstawowy}

Zadanie 31. $(2pktJ$

$\mathrm{W}$ trapezie prostokątnym ABCD dłuzsza podstawa $AB$ ma długość 8. Przekątna $AC$ tego trapezu

ma długość 4 i tworzy z krótszą podstawą trapezu kąt o mierze $30^{\mathrm{o}}$ (zobacz rysunek). Oblicz

długość przekątnej $BD$ tego trapezu.
\begin{center}
\includegraphics[width=106.728mm,height=35.352mm]{./F1_M_PP_M2019_page18_images/image001.eps}
\end{center}
{\it D  C}

4

{\it A}  8  {\it B}

Odpowied $\acute{\mathrm{z}}$:
\begin{center}
\includegraphics[width=96.012mm,height=17.832mm]{./F1_M_PP_M2019_page18_images/image002.eps}
\end{center}
Wypelnia

egzaminator

Nr zadania

Maks. liczba kt

30.

2

31.

2

Uzyskana liczba pkt

MMA-IP

Strona 19 z 26





{\it Egzamin maturalny z matematyki}

{\it Poziom podstawowy}

Zadanie 32. $(4pktJ$

Ciąg arytmetyczny $(a_{n})$ jest określony dla $\mathrm{k}\mathrm{a}\dot{\mathrm{z}}$ dej liczby naturalnej $n\geq 1$. Róznicą tego

ciągujest liczba $r=-4$, a średnia arytmetyczna początkowych sześciu wyrazów tego ciągu:

$a_{1}, a_{2}, a_{3}, a_{4}, a_{5}, a_{6}$, jest równa 16.

a) Oblicz pierwszy wyraz tego ciągu.

b) Oblicz liczbę $k$, dla której $a_{k}=-78.$

Strona 20 z 26

MMA-IP





{\it Egzamin maturalny z matematyki}

{\it Poziom podstawowy}

{\it BRUDNOPIS} ({\it nie podlega ocenie})

Strona 3 z26





Odpowiedzí :

{\it Egzamin maturalny z matematyki}

{\it Poziom podstawowy}
\begin{center}
\includegraphics[width=82.044mm,height=17.832mm]{./F1_M_PP_M2019_page20_images/image001.eps}
\end{center}
Wypelnia

egzaminator

Nr zadania

Maks. liczba kt

32.

4

Uzyskana liczba pkt

MMA-IP

Strona 21 z 26





{\it Egzamin maturalny z matematyki}

{\it Poziom podstawowy}

Zadanie 33. $(4pktJ$

Danyjest punkt $A=(-18,10)$. Prosta o równaniu $y=3x$ jest symetralną odcinka $AB$. Wyznacz

współrzędne punktu $B.$

Strona 22 z 26

MMA-IP





{\it Egzamin maturalny z matematyki}

{\it Poziom podstawowy}

Odpowied $\acute{\mathrm{z}}$:
\begin{center}
\includegraphics[width=82.044mm,height=17.832mm]{./F1_M_PP_M2019_page22_images/image001.eps}
\end{center}
Nr zadania

Wypelnia Maks. liczba kt

egzaminator

Uzyskana liczba pkt

33.

4

MMA-IP

Strona 23 z 26





{\it Egzamin maturalny z matematyki}

{\it Poziom podstawowy}

Zadanie 34. $(SpktJ$

Długość krawędzi podstawy ostrosłupa prawidłowego czworokątnego jest równa 6. Po1e

powierzchni całkowitej tego ostrosłupajest cztery razy większe od polajego podstawy. Kąt $\alpha$

jest kątem nachylenia krawędzi bocznej tego ostrosłupa do płaszczyzny podstawy (zobacz

rysunek). Oblicz cosinus kąta $\alpha.$

Strona 24 z 26

MMA-IP





{\it Egzamin maturalny z matematyki}

{\it Poziom podstawowy}

Odpowied $\acute{\mathrm{z}}$:
\begin{center}
\includegraphics[width=82.044mm,height=17.832mm]{./F1_M_PP_M2019_page24_images/image001.eps}
\end{center}
Nr zadania

Wypelnia Maks. liczba kt

egzaminator

Uzyskana liczba pkt

34.

5

MMA-IP

Strona 25 z 26





{\it Egzamin maturalny z matematyki}

{\it Poziom podstawowy}

{\it BRUDNOPIS} ({\it nie podlega ocenie})

Strona 26 z 26





{\it Egzamin maturalny z matematyki}

{\it Poziom podstawowy}

Zadanie 7. $(1pkt)$

Miejscem zerowym funkcji liniowej $f$ określonej wzorem $f(x)=3(x+1)-6\sqrt{3}$ jest liczba

A. $3-6\sqrt{3}$

B.

$1-6\sqrt{3}$

C. $2\sqrt{3}-1$

D.

$2\displaystyle \sqrt{3}-\frac{1}{3}$

Informacja do zadań S.-10.

Na rysunku przedstawiony jest fragment paraboli będącej wykresem funkcji kwadratowej $f.$

Wierzchołkiem tej parabolijest punkt $W=(2,-4)$. Liczby 0 $\mathrm{i}4$ to miejsca zerowe funkcji $f.$
\begin{center}
\includegraphics[width=127.248mm,height=105.060mm]{./F1_M_PP_M2019_page3_images/image001.eps}
\end{center}
{\it y}

4

3

1

{\it x}

$-3 -2$

$-1 0$

$-1$

1 2 3 4  5 6

$-2$

$-3$

{\it W}

$\langle 0,  4\rangle$

B.

$(-\infty,  0\rangle$

A.

Zadanie 8. (1pkt)

Zbiorem wartości funkcji f jest przedział

C.

$\langle-4, +\infty)$

D. $\langle 4, +\infty)$

Zadam$\mathrm{e}9\cdot(1pkt)$

Największa wartość funkcji $f$ w przedziale $\langle$1, $ 4\rangle$ jest równa

A. $-3$

B. $-4$

C. 4

D. 0

Zadanie 10. (1pkt)

Osią symetrii wykresu funkcji f jest prosta o równaniu

A. $y=-4$

B. $x=-4$

C. $y=2$

D. $x=2$

Strona 4 z26

MMA-IP





{\it Egzamin maturalny z matematyki}

{\it Poziom podstawowy}

{\it BRUDNOPIS} ({\it nie podlega ocenie})

Strona 5 z 26





{\it Egzamin maturalny z matematyki}

{\it Poziom podstawowy}

Zadanie ll. $(1pktJ$

$\mathrm{W}$ ciągu arytmetycznym $(a_{n})$, określonym dla $n\geq 1$, dane są dwa wyrazy: $a_{1}=7\mathrm{i}a_{8}=-49.$

Suma ośmiu początkowych wyrazów tego ciągujest równa

A. $-168$

B. $-189$

C. $-21$

D. $-42$

Zadanie $l2. (1pkt)$

Dany jest ciąg geometryczny $(a_{n})$, określony dla $n\geq 1$. Wszystkie wyrazy tego ciągu są

dodatnie i spełnionyjest watunek $\displaystyle \frac{a_{5}}{a_{3}}=\frac{1}{9}$. Iloraz tego ciągujest równy

A.

-31

B.

-$\sqrt{}$13

C. 3

D. $\sqrt{3}$

Zadanie 13. $(1pktJ$

Sinus kąta ostrego $\alpha$ jest równy $\displaystyle \frac{4}{5}$. Wtedy

A.

$\displaystyle \cos\alpha=\frac{5}{4}$

B.

$\displaystyle \cos\alpha=\frac{1}{5}$

C.

$\displaystyle \cos\alpha=\frac{9}{25}$

D.

$\displaystyle \cos\alpha=\frac{3}{5}$

Zadanie 14. $(1pktJ$

Punkty $D\mathrm{i}E$ lez$\cdot$ą na okręgu opisanym na trójkącie równobocznym $ABC$ (zobacz rysunek).

Odcinek $CD$ jest średnicą tego okręgu. Kąt wpisany $DEB$ ma miarę $\alpha.$
\begin{center}
\includegraphics[width=46.788mm,height=52.680mm]{./F1_M_PP_M2019_page5_images/image001.eps}
\end{center}
{\it C}

{\it E}

$\alpha$

{\it A  B}

{\it D}

Zatem

A. $\alpha=30^{\mathrm{o}}$

B. $\alpha<30^{\mathrm{o}}$

C. $\alpha>45^{\mathrm{o}}$

D. $\alpha=45^{\mathrm{o}}$

Strona 6 z26

MMA-IP





{\it Egzamin maturalny z matematyki}

{\it Poziom podstawowy}

{\it BRUDNOPIS} ({\it nie podlega ocenie})

Strona 7 z 26





{\it Egzamin maturalny z matematyki}

{\it Poziom podstawowy}

Zadanie 15. (1pktJ

Dane są dwa okręgi: okrąg o środku w punkcie O i promieniu 5 oraz okrąg o środku

w punkcie P i promieniu 3. Odcinek OP ma długość 16. Prosta AB jest styczna do tych okręgów

w punktach A iB. Ponadto prosta AB przecina odcinek OP w punkcie K(zobacz rysunek).
\begin{center}
\includegraphics[width=155.292mm,height=65.436mm]{./F1_M_PP_M2019_page7_images/image001.eps}
\end{center}
{\it B}

{\it O  K}

{\it P}

{\it A}

Wtedy

A.

$|OK|=6$

B.

$|OK|=8$

C.

$|OK|=10$

D.

$|OK|=12$

Zadanie $1\mathrm{f}\cdot(1pkt)$

Dany jest romb o boku długości 4 i kącie rozwartym $150^{\mathrm{o}}$. Pole tego rombujest równe

A. 8

B. 12

C. $8\sqrt{3}$

D. 16

Zadanie $l7. (1pktJ$

Proste o równaniach $y=(2m+2)x-2019$ oraz $y=(3m-3)x+2019$ są równoległe, gdy

A. $m=-1$

B. $m=0$

C. $m=1$

D. $m=5$

Zadanie 18. (1pktJ

Prosta o równaniu $y=ax+b$ jest prostopadła do prostej o równaniu $y=-4x+1$ i przechodzi

przez punkt $P=(\displaystyle \frac{1}{2},0)$, gdy

A. $a=-4\mathrm{i}b=-2$

B. {\it a}$=$-41i{\it b}$=$--81

C. $a=-4\mathrm{i}b=2$

D. {\it a}$=$-41i{\it b}$=$-21

Strona 8 z 26

MMA-IP





{\it Egzamin maturalny z matematyki}

{\it Poziom podstawowy}

{\it BRUDNOPIS} ({\it nie podlega ocenie})

Strona 9 z 26





{\it Egzamin maturalny z matematyki}

{\it Poziom podstawowy}

Zadanie 19. $(1pktJ$

Na rysunku przedstawiony jest fragment wykresu funkcji liniowej $f$ Na wykresie tej ffinkcji

$\mathrm{l}\mathrm{e}\dot{\mathrm{z}}$ ą punkty $A=(0,4)\mathrm{i}B=(2,2).$
\begin{center}
\includegraphics[width=65.376mm,height=67.920mm]{./F1_M_PP_M2019_page9_images/image001.eps}
\end{center}
$y$

$5$

-$4^{A}$

3

2

$B1$

1

{\it x}

$-4  -3$ -$2$ -$1$ -$10$  1 2 3 4  $-5$

$-2$

$-3$

$-4$

Obrazem prostej AB w symetrii względem początku układu współrzędnych jest wykres

funkcji g określonej wzorem

A. $g(x)=x+4$

B. $g(x)=x-4$

C. $g(x)=-x-4$

D. $g(x)=-x+4$

Zadanie 20. $(1pktJ$

Dane są punkty o współrzędnych $A=(-2,5)$ oraz $B=(4,-1)$. Średnica okręgu wpisanego

w kwadrat o boku $AB$ jest równa

A. 12

B. 6

C.

$6\sqrt{2}$

D. $2\sqrt{6}$

Zadanie 21. (1pkt)

Promień AS podstawy walca jest równy połowie wysokości OS tego walca. Sinus kąta OAS

(zobacz rysunek) jest równy
\begin{center}
\includegraphics[width=47.904mm,height=76.404mm]{./F1_M_PP_M2019_page9_images/image002.eps}
\end{center}
{\it O}

{\it S}

{\it A}

A.

-$\sqrt{}$25

B.

$\displaystyle \frac{2\sqrt{5}}{5}$

C.

-21

D. l

Strona 10 z 26

MMA-IP







CENTRALNA

KOMISJA

EGZAMINACYJNA

Arkusz zawiera informacje prawnie chronione do momentu rozpoczęcia egzaminu.

UZUPELNIA ZDAJACY

KOD PESEL

{\it miejsce}

{\it na naklejkę}
\begin{center}
\includegraphics[width=21.432mm,height=9.852mm]{./F1_M_PP_M2020_page0_images/image001.eps}

\includegraphics[width=82.140mm,height=9.852mm]{./F1_M_PP_M2020_page0_images/image002.eps}

\includegraphics[width=204.060mm,height=216.048mm]{./F1_M_PP_M2020_page0_images/image003.eps}
\end{center}
EGZAMIN MATU LNY

Z MATEMATYKI

POZIOM PODSTAWOWY

Instrukcja dla zdającego

1. Sprawd $\acute{\mathrm{z}}$, czy arkusz egzaminacyjny zawiera 26 stron

(zadania $1-34$). Ewentualny brak zgłoś przewodniczącemu

zespo nadzo jącego egzamin.

2. Rozwiązania zadań i odpowiedzi wpisuj w miejscu na to

przeznaczonym.

3. Odpowiedzi do zadań zamkniętych $(1-25)$ zaznacz

na karcie odpowiedzi, w części ka $\mathrm{y}$ przeznaczonej dla

zdającego. Zamaluj $\blacksquare$ pola do tego przeznaczone. Błędne

zaznaczenie otocz kółkiem $\mathrm{O}$ i zaznacz właściwe.

4. Pamiętaj, $\dot{\mathrm{z}}\mathrm{e}$ pominięcie argumentacji lub istotnych

obliczeń w rozwiązaniu zadania otwartego (26-34) $\mathrm{m}\mathrm{o}\dot{\mathrm{z}}\mathrm{e}$

spowodować, $\dot{\mathrm{z}}\mathrm{e}$ za to rozwiązanie nie otrzymasz pełnej

liczby punktów.

5. Pisz czytelnie i uzywaj tylko długopisu lub pióra

z czatnym tuszem lub atramentem.

6. Nie uzywaj korektora, a błędne zapisy wyra $\acute{\mathrm{z}}\mathrm{n}\mathrm{i}\mathrm{e}$ prze eśl.

7. Pamiętaj, $\dot{\mathrm{z}}\mathrm{e}$ zapisy w brudnopisie nie będą oceniane.

8. $\mathrm{M}\mathrm{o}\dot{\mathrm{z}}$ esz korzystać z zestawu wzorów matematycznych,

cyrkla i linijki oraz kalkulatora prostego.

9. Na tej stronie oraz na karcie odpowiedzi wpisz swój

numer PESEL i przyklej naklejkę z kodem.

10. Nie wpisuj $\dot{\mathrm{z}}$ adnych znaków w części przeznaczonej dla

egzaminatora.

5 MAJA 2020

Godzina rozpoczęcia:

Czas pracy:

170 minut

Liczba punktów

do uzyskania: 50

$\Vert\Vert\Vert\Vert\Vert\Vert\Vert\Vert\Vert\Vert\Vert\Vert\Vert\Vert\Vert\Vert\Vert\Vert\Vert\Vert\Vert\Vert\Vert\Vert|  \mathrm{M}\mathrm{M}\mathrm{A}-\mathrm{P}1_{-}1\mathrm{P}-202$




{\it Egzamin maturalny z matematyki}

{\it Poziom podstawowy}

ZADANIA ZAMKNIETE

$W$ {\it kazdym z zadań od l. do 25. wybierz i zaznacz na karcie odpowiedzipoprawnq odpowied} $\acute{z}.$

Zadanie l. $(1pkt)$

Wartość wyrazenia $x^{2}-6x+9$ dla $x=\sqrt{3}+3$ jest równa

A. l

B. 3

Zadanie 2. $(1pkt)$

Liczba $\displaystyle \frac{2^{50}\cdot 3^{40}}{36^{10}}$ jest równa

A. $6^{70}$

B. $6^{45}$

Zadanie 3. $(1pkt)$

Liczba $\log_{5}\sqrt{125}$ jest równa

A.

-23

B. 2

C. $1+2\sqrt{3}$

D. $1-2\sqrt{3}$

C. $2^{30}\cdot 3^{20}$

D. $2^{10}\cdot 3^{20}$

C. 3

D.

-23

Zadanie 4. (1pkt)

Cenę x pewnego towaru obnizono o 20\% i otrzymano cenę y. Aby przywrócić cenę x, nową

cenę y nalezy podnieść o

A. 25\%

B. 20\%

C. 15\%

D. 12\%

Zadanie 5. $(1pkt)$

Zbiorem wszystkich rozwiązań nierówności 3 $(1-x)>2(3x-1)-12x$ jest przedział

A.

$(-\displaystyle \frac{5}{3},+\infty)$

B.

(-$\infty$, -35)

C.

$(\displaystyle \frac{5}{3},+\infty)$

D.

(-$\infty$'- -35)

Zadanie $\epsilon. (1pkt)$

Suma wszystkich rozwiązań równania $x(x-3)(x+2)=0$ jest równa

A. 0

B. l

C. 2

D. 3

Strona 2 z26

MMA-IP





{\it Egzamin maturalny z matematyki}

{\it Poziom podstawowy}

{\it BRUDNOPIS} ({\it nie podlega ocenie})

MMA-IP

Strona ll z26





{\it Egzamin maturalny z matematyki}

{\it Poziom podstawowy}

Zadanie 23. (1pktJ

Cztery liczby: 2, 3, a, 8, tworzące zestaw danych, są uporządkowane rosnąco. Mediana tego

zestawu czterech danychjest równa medianie zestawu pięciu danych: 5, 3, 6, 8, 2. Zatem

A. $a=7$

B. $a=6$

C. $a=5$

D. $a=4$

Zadanie 24. $(1pkt)$

Dany jest sześcian ABCDEFGH. Sinus kąta $\alpha$ nachylenia przekątnej $HB$ tego sześcianu do

płaszczyzny podstawy ABCD (zobacz rysunek) jest równy

A.

$\displaystyle \frac{\sqrt{3}}{3}$

B.

$\displaystyle \frac{\sqrt{6}}{3}$

C.

-$\sqrt{}$22

D.

-$\sqrt{}$26
\begin{center}
\includegraphics[width=61.572mm,height=58.116mm]{./F1_M_PP_M2020_page11_images/image001.eps}
\end{center}
{\it H  G}

{\it E  F}

{\it D  C}

$\alpha$

{\it A  B}

Zadanie 25. $(1pkt)$

Danyjest stozek o objętości $ 18\pi$, którego przekrojem osiowymjest trójkąt ABC(zobacz rysunek).

Kąt $CBA$ jest kątem nachylenia tworzącej $l$ tego stozka do płaszczyzny jego podstawy.

Tangens kąta $CBA$ jest równy 2.
\begin{center}
\includegraphics[width=64.464mm,height=48.408mm]{./F1_M_PP_M2020_page11_images/image002.eps}
\end{center}
{\it C}

{\it l}

{\it h}

{\it A  B}

Wynika stąd, $\dot{\mathrm{z}}\mathrm{e}$ wysokość $h$ tego stozkajest równa

A. 12

B. 6

C. 4

D. 2

Strona 12 z26

MMA-IP





{\it Egzamin maturalny z matematyki}

{\it Poziom podstawowy}

{\it BRUDNOPIS} ({\it nie podlega ocenie})

MMA-IP

Strona 13 z26





{\it Egzamin maturalny z matematyki}

{\it Poziom podstawowy}

Zadanie 26. $(2pktJ$

Rozwiąz nierówność 2 $(x-1)(x+3)>x-1.$

Odpowied $\acute{\mathrm{z}}$:

Strona 14 $\mathrm{z}26$

MMA-IP





{\it Egzamin maturalny z matematyki}

{\it Poziom podstawowy}

Zadanie 27. $(2pktJ$

Rozwiąz równanie $x^{3}-9x^{2}-4x+36=0.$

Odpowiedzí:
\begin{center}
\includegraphics[width=96.012mm,height=17.784mm]{./F1_M_PP_M2020_page14_images/image001.eps}
\end{center}
WypelnÍa

egzaminator

Nr zadania

Maks. lÍczba kt

2

27.

2

Uzyskana liczba pkt

MMA-IP

Strona 15 z26





{\it Egzamin maturalny z matematyki}

{\it Poziom podstawowy}

Zadanie 28. $(2pktJ$

Wykaz, ze dla kazdych dwóch róznych liczb rzeczywistych $a\mathrm{i}b$ prawdziwajest nierówność

$a(a-2b)+2b^{2}>0.$

Strona 16 z26

MMA-IP





{\it Egzamin maturalny z matematyki}

{\it Poziom podstawowy}

Zadanie 29. $(2pktJ$

Trójkąt ABCjest równoboczny. Punkt $E$ lezy na wysokości $CD$ tego trójkąta oraz $|CE|=\displaystyle \frac{3}{4}|CD|.$

Punkt $F$ lezy na boku $BC$ i odcinek $EF$ jest prostopadły do $BC$ (zobacz rysunek).
\begin{center}
\includegraphics[width=82.140mm,height=70.812mm]{./F1_M_PP_M2020_page16_images/image001.eps}
\end{center}
{\it C}

{\it F}

{\it E}

{\it A  D  B}

Wykaz, $\displaystyle \dot{\mathrm{z}}\mathrm{e}|CF|=\frac{9}{16}|CB|.$
\begin{center}
\includegraphics[width=96.012mm,height=17.784mm]{./F1_M_PP_M2020_page16_images/image002.eps}
\end{center}
Wypelnia

egzaminator

Nr zadania

Maks. liczba kt

28.

2

2

Uzyskana liczba pkt

MMA-IP

Strona 17 z26





{\it Egzamin maturalny z matematyki}

{\it Poziom podstawowy}

Zadanie 30. $(2pktJ$

Rzucamy dwa razy symetryczną sześcienną kostką do gry, która na $\mathrm{k}\mathrm{a}\dot{\mathrm{z}}$ dej ściance ma inną

liczbę oczek-odjednego oczka do sześciu oczek. Oblicz prawdopodobieństwo zdarzenia $A$

polegającego na tym, ze co najmniej jeden raz wypadnie ścianka z pięcioma oczkami.

Odpowiedzí:

Strona 18 z26

MMA-IP





{\it Egzamin maturalny z matematyki}

{\it Poziom podstawowy}

Zadanie 31. $(2pktJ$

Kąt $\alpha$ jest ostry i spełnia warunek $\displaystyle \frac{2\sin\alpha+3\cos\alpha}{\cos\alpha}=4$. Oblicz tangens kąta $\alpha.$

Odpowied $\acute{\mathrm{z}}$:
\begin{center}
\includegraphics[width=96.012mm,height=17.784mm]{./F1_M_PP_M2020_page18_images/image001.eps}
\end{center}
WypelnÍa

egzaminator

Nr zadanÍa

Maks. lÍczba kt

30.

2

31.

2

Uzyskana liczba pkt

MMA-IP

Strona 19 z26





{\it Egzamin maturalny z matematyki}

{\it Poziom podstawowy}

Zadanie 32. $(4pktJ$

Dany jest kwadrat ABCD, w którym $A=(5,-\displaystyle \frac{5}{3})$. Przekątna $BD$ tego kwadratu jest zawarta

w prostej o równaniu $y=\displaystyle \frac{4}{3}x$. Oblicz współrzędne punktu przecięcia przekątnych $AC\mathrm{i}BD$ oraz

pole kwadratu ABCD.

Strona 20 z26

MMA-IP





{\it Egzamin maturalny z matematyki}

{\it Poziom podstawowy}

{\it BRUDNOPIS} ({\it nie podlega ocenie})

MMA-IP

Strona 3 z26





{\it Egzamin maturalny z matematyki}

{\it Poziom podstawowy}

Odpowiedzí:
\begin{center}
\includegraphics[width=82.044mm,height=17.832mm]{./F1_M_PP_M2020_page20_images/image001.eps}
\end{center}
Wypelnia

egzaminator

Nr zadania

Maks. liczba kt

32.

4

Uzyskana liczba pkt

MMA-IP

Strona 21 z 26





{\it Egzamin maturalny z matematyki}

{\it Poziom podstawowy}

Zadanie 33. $(4pktJ$

Wszystkie wyrazy ciągu geometrycznego $(a_{n})$, określonego dla $n\geq 1$, są dodatnie. Wyrazy tego

ciągu spełniają warunek $6a_{1}-5a_{2}+a_{3}=0$. Oblicz iloraz

$\langle 2\sqrt{2}, 3\sqrt{2}\rangle.$

q tego ciągu nalezący do przedziału

Strona 22 z 26

MMA-IP





{\it Egzamin maturalny z matematyki}

{\it Poziom podstawowy}

Odpowiedzí:
\begin{center}
\includegraphics[width=82.044mm,height=17.832mm]{./F1_M_PP_M2020_page22_images/image001.eps}
\end{center}
Wypelnia

egzaminator

Nr zadania

Maks. liczba kt

33.

4

Uzyskana liczba pkt

MMA-IP

Strona 23 z 26





{\it Egzamin maturalny z matematyki}

{\it Poziom podstawowy}

Zadanie 34. $(SpktJ$

Dany jest ostrosłup prawidłowy czworokątny ABCDS, którego krawędzí boczna ma długość 6

(zobacz rysunek). Ściana boczna tego ostrosłupajest nachylona do płaszczyzny podstawy pod

kątem, którego tangensjest równy $\sqrt{7}$. Oblicz objętość tego ostrosłupa.

Strona 24 z 26

MMA-IP





{\it Egzamin maturalny z matematyki}

{\it Poziom podstawowy}

Odpowiedzí:
\begin{center}
\includegraphics[width=82.044mm,height=17.832mm]{./F1_M_PP_M2020_page24_images/image001.eps}
\end{center}
Wypelnia

egzaminator

Nr zadania

Maks. liczba kt

34.

5

Uzyskana liczba pkt

MMA-IP

Strona 25 z 26





{\it Egzamin maturalny z matematyki}

{\it Poziom podstawowy}

{\it BRUDNOPIS} ({\it nie podlega ocenie})

Strona 26 z 26

MMA-IP





{\it Egzamin maturalny z matematyki}

{\it Poziom podstawowy}

Informacja do zadań 7.$-9.$

Funkcja kwadratowa

f jest określona

wzorem

$f(x)=a(x-1)(x-3)$. Na rysunku

przedstawiono fragment paraboli będącej wykresem tej ffinkcji. Wierzchołkiem tej parabolijest

punkt $W=(2,1).$
\begin{center}
\includegraphics[width=117.960mm,height=97.128mm]{./F1_M_PP_M2020_page3_images/image001.eps}
\end{center}
4  {\it y}

3

2

{\it W}

1

$-3  -2 -1 0$  2 3  4 5 x

$-1$

$-2$

$-3$

Zadanie 7. (1pkt)

Współczynnik a we wzorze funkcji f jest równy

A. l

B. 2

C. $-2$

D. $-1$

Zadanie 8. $(1pkt)$

Największa wartość funkcji $f$ w przedziale $\langle$1, $ 4\rangle$ jest równa

A. $-3$

B. 0

C. l

D. 2

Zadanie 9. (1pkt)

Osią symetrii paraboli będącej wykresem funkcji f jest prosta o równaniu

A. $x=1$

B. $x=2$

C.

$y=1$

D. $y=2$

Strona 4 z26

MMA-IP





{\it Egzamin maturalny z matematyki}

{\it Poziom podstawowy}

{\it BRUDNOPIS} ({\it nie podlega ocenie})

MMA-IP

Strona 5 z26





{\it Egzamin maturalny z matematyki}

{\it Poziom podstawowy}

Zadanie 10. $(1pktJ$

Równanie $x(x-2)=(x-2)^{2}$ w zbiorze liczb rzeczywistych

A. nie ma rozwiązań.

B. ma dokładniejedno rozwiązanie: $x=2.$

C. ma dokładniejedno rozwiązanie: $x=0.$

D. ma dwa rózne rozwiązania: $x=1 \mathrm{i}x=2.$

Zadanie $l1. (1pktJ$

Na iysunku przedstawiono fiiagment wykresu funkcji liniowej $f$ określonej wzorem $f(x)=ax+b.$
\begin{center}
\includegraphics[width=118.056mm,height=97.584mm]{./F1_M_PP_M2020_page5_images/image001.eps}
\end{center}
4  {\it y}

3

1

$-3 -2$

$-1 0$

$-1$

1 2 3 4  5  {\it x}

$-2$

$-3$

Współczynniki a oraz b we wzorze funkcji f spełniają zalezność

A. $a+b>0$

B. $a+b=0$

C. $a\cdot b>0$

D. $a\cdot b<0$

Zadanie 12. $(1pktJ$

Funkcja $f$ jest określona wzorem $f(x)=4^{-x}+1$ dla $\mathrm{k}\mathrm{a}\dot{\mathrm{z}}$ dej liczby rzeczywistej $x$. Liczba $f(\displaystyle \frac{1}{2})$

jest równa

A.

-21

B.

-23

C. 3

D. 17

Zadanie 13. $(1pktJ$

Proste o równaniach $y=(m-2)x$ oraz $y=\displaystyle \frac{3}{4}x+7$ są równoległe. Wtedy

A.

{\it m}$=$- -45

B.

{\it m}$=$ -23

C.

$m=\displaystyle \frac{11}{4}$

D.

$m=\displaystyle \frac{10}{3}$

Strona 6 z26

MMA-IP





{\it Egzamin maturalny z matematyki}

{\it Poziom podstawowy}

{\it BRUDNOPIS} ({\it nie podlega ocenie})

MMA-IP

Strona 7 z 26





{\it Egzamin maturalny z matematyki}

{\it Poziom podstawowy}

Zadanie $1_{[}4. (1pktJ$

Ciąg $(a_{n})$ jest określony wzorem $a_{n}=2n^{2}$ dla $n\geq 1$. Róz$\cdot$nica $a_{5}-a_{4}$ jest równa

A. 4

B. 20

C. 36

D. 18

Zadanie 15. $(1pkt)$

$\mathrm{W}$ ciągu arytmetycznym $(a_{n})$, określonym dla $n\geq 1$, czwarty wyraz jest równy 3, a róznica

tego ciągujest równa 5. Suma $a_{1}+a_{2}+a_{3}+a_{4}$ jest równa

A. $-42$

B. $-36$

C. $-18$

D. 6

Zadanie $l6. (1pkt)$

Punkt $A=(\displaystyle \frac{1}{3},-1)$ nalezy do wykresu ffinkcji liniowej $f$ określonej wzorem $f(x)=3x+b.$

Wynika stąd, $\dot{\mathrm{z}}\mathrm{e}$

A. $b=2$

B. $b=1$

C. $b=-1$

D. $b=-2$

Zadanie $17_{c}(1pkt)$

Punkty $A, B, C, D$ lez$\cdot$ą na okręgu o środku w punkcie $O$. Kąt środkowy DOC ma miarę $118^{\mathrm{o}}$

(zobacz sunek).
\begin{center}
\includegraphics[width=57.612mm,height=61.572mm]{./F1_M_PP_M2020_page7_images/image001.eps}
\end{center}
{\it B} $D$

{\it O}  $118^{\mathrm{o}}$

{\it A C}

Miara kąta ABC jest równa

A. $59^{\mathrm{o}}$

B. $48^{\mathrm{o}}$

C. $62^{\mathrm{o}}$

D. $31^{\mathrm{o}}$

Zadanie 18. $(1pkt)$

Prosta przechodząca przez punkty $A=(3,-2)\mathrm{i}B=(-1,6)$ jest określona równaniem

A.

$y=-2x+4$

B. $y=-2x-8$

C.

$y=2x+8$

D. $y=2x-4$

Strona 8 z 26

MMA-IP





{\it Egzamin maturalny z matematyki}

{\it Poziom podstawowy}

{\it BRUDNOPIS} ({\it nie podlega ocenie})

MMA-IP

Strona 9 z 26





{\it Egzamin maturalny z matematyki}

{\it Poziom podstawowy}

Zadanie 19. $(1pktJ$

Danyjest trójkąt prostokątny o kątach ostrych $\alpha \mathrm{i}\beta$ (zobacz rysunek).

Wyrazenie $ 2\cos\alpha-\sin\beta$ jest równe

A. $ 2\sin\beta$

B.

$\cos\alpha$

C. 0

D. 2

Zadanie 20. $(1pktJ$

Punkt $B$ jest obrazem punktu $A=(-3,5) \mathrm{w}$

współrzędnych. Długość odcinka $AB$ jest równa

symetrii względem

początku układu

A. $2\sqrt{34}$

B. 8

C. $\sqrt{34}$

D. 12

Zadanie 21. (1pktJ

Ilejest wszystkich dwucyfrowych liczb naturalnych utworzonych z cyfr: 1, 3, 5, 7, 9, w których

cyfry się nie powtarzają?

A. 10

B. 15

C. 20

D. 25

Zadanie 22. $(1pkt)$

Pole prostokąta ABCDjest równe 90. Na bokach AB $\mathrm{i}$ CD wybrano- odpowiednio- punkty $P\mathrm{i}R,$

takie, $\displaystyle \dot{\mathrm{z}}\mathrm{e}\frac{|AP|}{|PB|}=\frac{|CR|}{|RD|}=\frac{3}{2}$ (zobacz rysunek).
\begin{center}
\includegraphics[width=78.180mm,height=48.672mm]{./F1_M_PP_M2020_page9_images/image001.eps}
\end{center}
{\it D R  C}

{\it A  P B}

Pole czworokąta APCR jest równe

A. 36

B. 40

C. 54

D. 60

Strona 10 z 26

MMA-IP







$1-$

$-1\cup 1$

$-\mapsto 1$

$\mathrm{r}--$

Centralna Komisja Egzaminacyjna

Arkusz zawiera informacje prawnie chronione do momentu rozpoczęcia egzaminu.

WPISUJE ZDAJACY

KOD PESEL

{\it Miejsce}

{\it na naklejkę}

{\it z kodem}
\begin{center}
\includegraphics[width=21.432mm,height=9.804mm]{./F1_M_PP_S2012_page0_images/image001.eps}

\includegraphics[width=82.092mm,height=9.804mm]{./F1_M_PP_S2012_page0_images/image002.eps}
\end{center}
\fbox{} dysleksja
\begin{center}
\includegraphics[width=204.060mm,height=216.048mm]{./F1_M_PP_S2012_page0_images/image003.eps}
\end{center}
EGZAMIN MATU LNY

Z MATEMATYKI

SIERPIE $\acute{\mathrm{N}}$ 2012

POZIOM PODSTAWOWY

1. Sprawd $\acute{\mathrm{z}}$, czy arkusz egzaminacyjny zawiera 20 stron

(zadania $1-34$). Ewentualny brak zgłoś przewodniczącemu

zespo nadzorującego egzamin.

2. Rozwiązania zadań i odpowiedzi wpisuj w miejscu na to

przeznaczonym.

3. Odpowiedzi do zadań za niętych (l-25) przenieś

na ka ę odpowiedzi, zaznaczając je w części ka $\mathrm{y}$

przeznaczonej dla zdającego. Zamaluj $\blacksquare$ pola do tego

przeznaczone. Błędne zaznaczenie otocz kółkiem \fcircle$\bullet$

i zaznacz właściwe.

4. Pamiętaj, $\dot{\mathrm{z}}\mathrm{e}$ pominięcie argumentacji lub istotnych

obliczeń w rozwiązaniu zadania otwa ego (26-34) $\mathrm{m}\mathrm{o}\dot{\mathrm{z}}\mathrm{e}$

spowodować, $\dot{\mathrm{z}}\mathrm{e}$ za to rozwiązanie nie będziesz mógł

dostać pełnej liczby punktów.

5. Pisz czytelnie i uzywaj tvlko długopisu lub -Dióra

z czarnym tuszem lub atramentem.

6. Nie uzywaj korektora, a błędne zapisy wyrazínie prze eśl.

7. Pamiętaj, $\dot{\mathrm{z}}\mathrm{e}$ zapisy w brudnopisie nie będą oceniane.

8. $\mathrm{M}\mathrm{o}\dot{\mathrm{z}}$ esz korzystać z zestawu wzorów matematycznych,

cyrkla i linijki oraz kalkulatora.

9. Na tej stronie oraz na karcie odpowiedzi wpisz swój

numer PESEL i przyklej naklejkę z kodem.

10. Nie wpisuj $\dot{\mathrm{z}}$ adnych znaków w części przeznaczonej

dla egzaminatora.

Czas pracy:

170 minut

Liczba punktów

do uzyskania: 50

$\Vert\Vert\Vert\Vert\Vert\Vert\Vert\Vert\Vert\Vert\Vert\Vert\Vert\Vert\Vert\Vert\Vert\Vert\Vert\Vert\Vert\Vert\Vert\Vert|  \mathrm{M}\mathrm{M}\mathrm{A}-\mathrm{P}1_{-}1\mathrm{P}-124$




{\it 2}

{\it Egzamin maturalny z matematyki}

{\it Poziom podstawowy}

ZADANIA ZAMKNIĘTE

{\it Wzadaniach} $\theta d1.$ {\it do 25. wybierz i zaznacz na karcie odpowiedzipoprawnq odpowied} $\acute{z}.$

Zadanie l. $(1pkt)$

Długość boku kwadratu $k_{2}$ jest o 10\% większa od długości boku kwadratu $k_{1}$. Wówczas pole

kwadratu $k_{2}$ jest większe od pola kwadratu $k_{1}$

A. 010\%

B. 0110\%

C. 021\%

D. 0121\%

Zadanie 2. $(1pkt)$

Iloczyn $9^{-5}\cdot 3^{8}$ jest równy

A. $3^{-4}$

B. $3^{-9}$

C. $9^{-1}$

D. $9^{-9}$

Zadanie 3. $(1pkt)$

Liczba $\log_{3}27-\log_{3}1$ jest równa

A. 0

B. l

C. 2

D. 3

Zadanie 4. $(1pkt)$

Liczba $(2-3\sqrt{2})^{2}$ jest równa

A. $-14$ B. 22

$\mathrm{C}.\ -14-12\sqrt{2}$

D. $22-12\sqrt{2}$

Zadanie 5. $(1pkt)$

Liczba $(-2)$ jest miejscem zerowym ffinkcji liniowej $f(x)=mx+2$. Wtedy

A. $m=3$

B. $m=1$

C. $m=-2$

D. $m=-4$

Zadanie 6. $(1pkt)$

Wskaz rysunek, na którym jest przedstawiony zbiór rozwiązań nierówności $|x+4|\leq 7.$
\begin{center}
\includegraphics[width=172.464mm,height=13.260mm]{./F1_M_PP_S2012_page1_images/image001.eps}
\end{center}
$-11$  3  {\it x}

A.
\begin{center}
\includegraphics[width=174.804mm,height=13.416mm]{./F1_M_PP_S2012_page1_images/image002.eps}
\end{center}
$-3$  11  {\it x}

B.
\begin{center}
\includegraphics[width=175.560mm,height=13.212mm]{./F1_M_PP_S2012_page1_images/image003.eps}
\end{center}
$-11$  3  {\it x}

C.
\begin{center}
\includegraphics[width=174.804mm,height=13.356mm]{./F1_M_PP_S2012_page1_images/image004.eps}
\end{center}
$-3$  11  {\it x}

D.





{\it Egzamin maturalny z matematyki}

{\it Poziom podstawowy}

{\it 11}

Zadanie 28. (2pkt)

Pierwszy wyraz ciągu arytmetycznegojest równy 3, czwarty wyraz tego ciągu jest równy 15.

Oblicz sumę szeŚciu początkowych wyrazów tego ciągu.

Odpowiedzí :

Zadanie 29. $(2pkt)$

$\mathrm{W}$ trójkącie równoramiennym $ABC$ dane są $|AC|=|BC|=6 \mathrm{i}|\wedge ACB|=30^{\mathrm{o}}$ (zobacz rysunek).

Oblicz wysokoŚć AD trójkąta opuszczoną z wierzchołka $A$ na bok $BC.$
\begin{center}
\includegraphics[width=39.168mm,height=62.736mm]{./F1_M_PP_S2012_page10_images/image001.eps}
\end{center}
{\it C}

$30^{\mathrm{o}}$

{\it D}

{\it A B}

Odpowiedzí :





{\it 12}

{\it Egzamin maturalny z matematyki}

{\it Poziom podstawowy}

Zadanie 30. (2pkt)

Dany jest równoległobok ABCD. Na przedłuzeniu przekątnej $AC$ wybrano punkt $E$ tak, $\dot{\mathrm{z}}\mathrm{e}$

$|CE|=\displaystyle \frac{1}{2}|AC|$ (zobacz rysunek). Uzasadnij, $\dot{\mathrm{z}}\mathrm{e}$ pole równoległoboku ABCD jest cztery razy

większe od pola trójkąta $DCE.$
\begin{center}
\includegraphics[width=123.948mm,height=46.332mm]{./F1_M_PP_S2012_page11_images/image001.eps}
\end{center}
{\it E}

{\it D}

{\it C}

{\it A  B}





{\it Egzamin maturalny z matematyki}

{\it Poziom podstawowy}

{\it 13}

Zadanie 31. $(2pkt)$

Wykaz, $\dot{\mathrm{z}}\mathrm{e}\mathrm{j}\mathrm{e}\dot{\mathrm{z}}$ eli $c<0$, to trójmian kwadratowy

$y=x^{2}+bx+c$ ma dwa rózne miejsca

zerowe.





{\it 14}

{\it Egzamin maturalny z matematyki}

{\it Poziom podstawowy}

Zadanie 32. (4pkt)

Dany jest trójkąt równoramienny $ABC$, w którym $|AC|=|BC|$ oraz $A=(2,1) \mathrm{i} C=(1,9).$

Podstawa $AB$ tego trójkątajest zawarta w prostej $y=\displaystyle \frac{1}{2}x$. Oblicz współrzędne wierzchołka $B.$





{\it Egzamin maturalny z matematyki}

{\it Poziom podstawowy}

{\it 15}

Odpowied $\acute{\mathrm{z}}$:





{\it 16}

{\it Egzamin maturalny z matematyki}

{\it Poziom podstawowy}

Zadanie 33. (4pkt)

W ostrosłupie prawidłowym czworokątnym ABCDS o podstawie ABCD i wierzchołku S

trójkąt ACS jest równoboczny i ma bok długości 8. Ob1icz sinus kąta nachy1enia ściany

bocznej do płaszczyzny podstawy tego ostrosłupa (zobacz rysunek).





{\it Egzamin maturalny z matematyki}

{\it Poziom podstawowy}

{\it 1}7

Odpowied $\acute{\mathrm{z}}$:





{\it 18}

{\it Egzamin maturalny z matematyki}

{\it Poziom podstawowy}

Zadanie 34. (5pkt)

Kolarz pokonał trasę 114 km. Gdyby jechał ze średnią prędkością mniejszą o 9,5 km/h,

to pokonałby tę trasę w czasie o 2 godziny dłuzszym. Ob1icz, zjaką średnią prędkościąjechał

ten kolarz.





{\it Egzamin maturalny z matematyki}

{\it Poziom podstawowy}

{\it 19}

Odpowied $\acute{\mathrm{z}}$:





$ 2\theta$

{\it Egzamin maturalny z matematyki}

{\it Poziom podstawowy}

BRUDNOPIS





{\it Egzamin maturalny z matematyki}

{\it Poziom podstawowy}

{\it 3}

BRUDNOPIS





{\it 4}

{\it Egzamin maturalny z matematyki}

{\it Poziom podstawowy}

Zadanie 7. (1pkt)

Dana jest parabola o równaniu

paraboli jest równa

$y=x^{2}+8x-14$. Pierwsza współrzędna wierzchołka tej

A. $x=-8$

B. $x=-4$

C. $x=4$

D. $x=8$

Zadanie 8. $(1pkt)$

Wskaz fragment wykresu funkcji kwadratowej, której zbiorem wartościjest $\langle-2,+\infty$).
\begin{center}
\includegraphics[width=142.236mm,height=52.524mm]{./F1_M_PP_S2012_page3_images/image001.eps}
\end{center}
A.  B.  C.

3
\begin{center}
\includegraphics[width=43.788mm,height=52.476mm]{./F1_M_PP_S2012_page3_images/image002.eps}
\end{center}
D.

Zadanie 9. $(1pkt)$

Zbiorem rozwiązań nierówności $x(x+6)<0$ jest

A.

B.

C.

D.

$(-6,0)$

$(0,6)$

$(-\infty,-6)\cup(0,+\infty)$

$(-\infty,0)\cup(6,+\infty)$

Zadanie 10. (1pkt)

Wielomian $W(x)=x^{6}+x^{3}-2$ jest równy iloczynowi

A. $(x^{3}+1)(x^{2}-2)$

B. $(x^{3}-1)(x^{3}+2)$

C. $(x^{2}+2)(x^{4}-1)$

D. $(x^{4}-2)(x+1)$

Zadanie ll. (lpkt)

Równanie $\displaystyle \frac{(x+3)(x-2)}{(x-3)(x+2)}=0$ ma

A.

B.

C.

D.

dokładnie jedno rozwiązanie

dokładnie dwa rozwiązania

dokładnie trzy rozwiązania

dokładnie cztery rozwiązania

Zadanie 12. $(1pkt)$

Danyjest ciąg $(a_{n})$ określony wzorem $a_{n}=\displaystyle \frac{n}{(-2)^{n}}$ dla $n\geq 1$. Wówczas

A.

{\it a}3$=$ -21

B.

{\it a}3$=$ - -21

C.

{\it a}3$=$ -83

D.

{\it a}3$=$- -83





{\it Egzamin maturalny z matematyki}

{\it Poziom podstawowy}

{\it 5}

BRUDNOPIS





{\it 6}

{\it Egzamin maturalny z matematyki}

{\it Poziom podstawowy}

Zadanie 13. $(1pkt)$

$\mathrm{W}$ ciągu geometrycznym $(a_{n})$ dane są: $a_{1}=36, a_{2}=18$. Wtedy

A. $a_{4}=-18$

B. $a_{4}=0$

C. $a_{4}=4,5$

D. $a_{4}=144$

Zadanie 14. $(1pkt)$

Kąt $\alpha$ jest ostry i $\displaystyle \sin\alpha=\frac{7}{13}$. Wtedy $\mathrm{t}\mathrm{g}\alpha$ jest równy

A.

-76

B.

$\displaystyle \frac{7\cdot 13}{120}$

C.

$\displaystyle \frac{7}{\sqrt{120}}$

D.

$\displaystyle \frac{7}{13\sqrt{120}}$

Zadanie 15. (1pkt)

W trójkącie prostokątnym dane są długości boków (zobacz rysunek). Wtedy
\begin{center}
\includegraphics[width=37.344mm,height=72.384mm]{./F1_M_PP_S2012_page5_images/image001.eps}
\end{center}
$\alpha$

11

9

$2\sqrt{10}$

C.

$\displaystyle \sin\alpha=\frac{9}{11}$

B.

$\displaystyle \cos\alpha=\frac{9}{11}$

A.

$\displaystyle \sin\alpha=\frac{11}{2\sqrt{10}}$

D.

$\displaystyle \cos\alpha=\frac{2\sqrt{10}}{11}$

Zadanie 16. (1pkt)

Przekątna AC prostokąta ABCD ma długość

Diugość boku BC jest równa

14. Bok AB tego prostokąta ma długość 6.

A. 8

B. $4\sqrt{10}$

C. $\mathrm{z}\sqrt{58}$

D. 10

Zadanie 17. $(1pkt)$

Punkty $A, B \mathrm{i} C$ lez$\cdot$ą na okręgu o środku $S$ (zobacz rysunek). Miara zaznaczonego kąta

wpisanego $ACB$ jest równa
\begin{center}
\includegraphics[width=54.864mm,height=53.952mm]{./F1_M_PP_S2012_page5_images/image002.eps}
\end{center}
{\it C}

{\it A  B}

{\it S}

$230^{\mathrm{o}}$

A. $65^{\mathrm{o}}$

B. $100^{\mathrm{o}}$

C. $115^{\mathrm{o}}$

D. $130^{\mathrm{o}}$





{\it Egzamin maturalny z matematyki}

{\it Poziom podstawowy}

7

BRUDNOPIS





{\it 8}

{\it Egzamin maturalny z matematyki}

{\it Poziom podstawowy}

Zadanie 18. $(1pkt)$

Długość boku trójkąta równobocznego jest równa $24\sqrt{3}$. Promień okręgu wpisanego w ten

trójkątjest równy

A. 36

B. 18

C. 12

D. 6

Zadanie 19. $(1pkt)$

Wskaz równanie prostej przechodzącej przez początek układu współrzędnych i prostopadłej

do prostej o równaniu $y=-\displaystyle \frac{1}{3}x+2.$

A. $y=3x$ B. $y=-3x$ C. $y=3x+2$ D. $y=\displaystyle \frac{1}{3}x+2$

Zadanie 20. $(1pkt)$

Punkty $B=(-2,4) \mathrm{i} C=(5,1)$ są dwoma sąsiednimi wierzchołkami kwadratu ABCD. Pole

tego kwadratu jest równe

A. 74

B. 58

C. 40

D. 29

Zadanie 21. $(1pkt)$

Danyjest okrąg o równaniu $(x+4)^{2}+(y-6)^{2}=100$. Środek tego okręgu ma współrzędne

A. $(-4,-6)$

B. (4, 6)

C. $(4,-6)$

D. $(-4,6)$

Zadanie 22. (1pkt)

Objętość sześcianujest równa 64. Po1e powierzchni całkowitej tego sześcianu jest równe

A. 512

B. 384

C. 96

D. 16

Zadanie 23. (1pkt)

Przekrój osiowy stozka jest trójkątem równobocznym o boku $a$. Objętość tego stozka wyraz $\mathrm{a}$

się wzorem

A. $\displaystyle \frac{\sqrt{3}}{6}\pi a^{3}$ B. $\displaystyle \frac{\sqrt{3}}{8}\pi a^{3}$ C. $\displaystyle \frac{\sqrt{3}}{12}\pi a^{3}$ D. $\displaystyle \frac{\sqrt{3}}{24}\pi a^{3}$

Zadanie 24. $(1pkt)$

Pewna firma zatrudnia 6 osób. Dyrektor zarabia 8000 zł, a pensje pozostałych pracowników

są równe: 2000 zł, 2800 zł, 3400 zł, 3600 zł, 4200 zł. Mediana zarobków tych 6 osób jest

równa

A. 3400 zł

B. 3500 zł

C. 6000 zł

D. 7000 zł

Zadanie 25. (1pkt)

Ze zbioru \{1, 2, 3, 4, 5, 6, 7, 8, 9, 10, 11, 12, 13, 14, 15\} wybieramy 1osowo jedną 1iczbę. Niech

p oznacza prawdopodobieństwo otrzymania liczby podzielnej przez 4. Wówczas

A.

{\it p}$<$ -51

B.

{\it p}$=$ -51

C.

{\it p}$=$ -41

D.

{\it p}$>$ -41





{\it Egzamin maturalny z matematyki}

{\it Poziom podstawowy}

{\it 9}

BRUDNOPIS





$ 1\theta$

{\it Egzamin maturalny z matematyki}

{\it Poziom podstawowy}

ZADANIA OTWARTE

{\it Rozwiqzania zadań o numerach od 26. do 34. nalezy zapisać w wyznaczonych miejscach}

{\it pod treściq zadania}.

Zadanie 26. $(2pkt)$

Rozwiąz nierówność $x^{2}-8x+7\geq 0.$

Odpowiedzí:

Zadanie 27. $(2pkt)$

Rozwiąz równanie $x^{3}-6x^{2}-9x+54=0.$

Odpowiedzí:







Centralna Komisja Egzaminacyjna

Arkusz zawiera informacje prawnie chronione do momentu rozpoczęcia egzaminu.

WPISUJE ZDAJACY

KOD PESEL

{\it Miejsce}

{\it na naklejkę}

{\it z kodem}
\begin{center}
\includegraphics[width=21.432mm,height=9.804mm]{./F1_M_PR_C2012_page0_images/image001.eps}

\includegraphics[width=82.092mm,height=9.804mm]{./F1_M_PR_C2012_page0_images/image002.eps}
\end{center}
\fbox{} dysleksja
\begin{center}
\includegraphics[width=204.060mm,height=216.048mm]{./F1_M_PR_C2012_page0_images/image003.eps}
\end{center}
EGZAMIN MATU LNY

Z MATEMATYKI

CZERWIEC 2012

POZIOM ROZSZERZONY

1.

3.

Sprawd $\acute{\mathrm{z}}$, czy arkusz egzaminacyjny zawiera 20 stron

(zadania $1-12$). Ewentualny brak zgłoś

przewodniczącemu zespołu nadzorującego egzamin.

Rozwiązania zadań i odpowiedzi wpisuj w miejscu na to

przeznaczonym.

Pamiętaj, $\dot{\mathrm{z}}\mathrm{e}$ pominięcie argumentacji lub istotnych

obliczeń w rozwiązaniu zadania otwa ego $\mathrm{m}\mathrm{o}\dot{\mathrm{z}}\mathrm{e}$

spowodować, $\dot{\mathrm{z}}\mathrm{e}$ za to rozwiązanie nie będziesz mógł

dostać pełnej liczby punktów.

Pisz czytelnie i uzywaj tvlko długopisu lub -Dióra

z czarnym tuszem lub atramentem.

Nie uzywaj korektora, a błędne zapisy wyra $\acute{\mathrm{z}}\mathrm{n}\mathrm{i}\mathrm{e}$ prze eśl.

Pamiętaj, $\dot{\mathrm{z}}\mathrm{e}$ zapisy w brudnopisie nie będą oceniane.

$\mathrm{M}\mathrm{o}\dot{\mathrm{z}}$ esz korzystać z zestawu wzorów matematycznych,

cyrkla i linijki oraz kalkulatora.

Na tej stronie oraz na karcie odpowiedzi wpisz swój

numer PESEL i przyklej naklejkę z kodem.

Nie wpisuj $\dot{\mathrm{z}}$ adnych znaków w części przeznaczonej

dla egzaminatora.

Czas pracy:

180 minut

2.

4.

5.

6.

7.

8.

9.

Liczba punktów

do uzyskania: 50

$\Vert\Vert\Vert\Vert\Vert\Vert\Vert\Vert\Vert\Vert\Vert\Vert\Vert\Vert\Vert\Vert\Vert\Vert\Vert\Vert\Vert\Vert\Vert\Vert|  \mathrm{M}\mathrm{M}\mathrm{A}-\mathrm{R}1_{-}1\mathrm{P}-123$




{\it 2}

{\it Egzamin maturalny z matematyki}

{\it Poziom rozszerzony}

Zadanie l. $(4pkt)$

Rozwiąz nierówność $|x-2|+|x+1|\geq 3x-3.$





{\it Egzamin maturalny z matematyki}

{\it Poziom rozszerzony}

{\it 11}

Odpowiedzí :





{\it 12}

{\it Egzamin maturalny z matematyki}

{\it Poziom rozszerzony}

Zadanie 8. $(5pkt)$

$\mathrm{W}$ czworokącie ABCD dane są długości boków: $|AB|=24, |CD|=15, |AD|=7$. Ponadto kąty

$DAB$ oraz $BCD$ sąproste. Oblicz pole tego czworokąta oraz długościjego przekątnych.





{\it Egzamin maturalny z matematyki}

{\it Poziom rozszerzony}

{\it 13}

Odpowiedzí :





{\it 14}

{\it Egzamin maturalny z matematyki}

{\it Poziom rozszerzony}

Zadanie 9. (3pkt)

Oblicz, ile jest liczb naturalnych trzycyfrowych podzielnych przez 6 1ub

przez 15.

podzielnych

Odpowiedzí:





{\it Egzamin maturalny z matematyki}

{\it Poziom rozszerzony}

{\it 15}

Zadanie 10. $(4pkt)$

Na płaszczyzínie dane są punkty $A=(3,-2) \mathrm{i}B=(11,4)$. Na prostej o równaniu $y=8x+10$

znajdz$\ovalbox{\tt\small REJECT}$ punkt $P$, dla którego suma $|AP|^{2}+|BP|^{2}$ jest najmniejsza.

Odpowiedzí :





{\it 16}

{\it Egzamin maturalny z matematyki}

{\it Poziom rozszerzony}

Zadanie ll. $(5pkt)$

Podstawą ostrosłupa ABCS jest trójkąt równoramienny $ABC$, w którym $|AB|=30,$

$|BC|=|AC|=39$ i spodek wysokości ostrosłupa nalez$\mathrm{y}$ do jego podstawy. $\mathrm{K}\mathrm{a}\dot{\mathrm{z}}$ da wysokość

ściany bocznej poprowadzona z wierzchołka $S$ ma długość 26. Ob1icz objętość tego

ostrosłupa.





{\it Egzamin maturalny z matematyki}

{\it Poziom rozszerzony}

17

Odpowied $\acute{\mathrm{z}}$:





{\it 18}

{\it Egzamin maturalny z matematyki}

{\it Poziom rozszerzony}

Zadanie 12. $(3pkt)$

Zdarzenia losowe $A, B$ są zawarte w $\Omega$ oraz $P(A\cap B')=0,1 \mathrm{i}P(A'\cap B)=0,2$. Wykaz, $\dot{\mathrm{z}}\mathrm{e}$

$P(A\cap B)\leq 0,7$ (A'oznacza zdarzenie przeciwne do zdarzenia

przeciwne do zdarzenia $B$).

A, B'oznacza zdarzenie





{\it Egzamin maturalny z matematyki}

{\it Poziom rozszerzony}

{\it 19}

Odpowied $\acute{\mathrm{z}}$:





$ 2\theta$

{\it Egzamin maturalny z matematyki}

{\it Poziom rozszerzony}

BRUDNOPIS





{\it Egzamin maturalny z matematyki}

{\it Poziom rozszerzony}

{\it 3}

Odpowied $\acute{\mathrm{z}}$:





{\it 4}

{\it Egzamin maturalny z matematyki}

{\it Poziom rozszerzony}

Zadanie 2. $(4pkt)$

Wielomian $W(x)=x^{4}+ax^{3}+bx^{2}-24x+9$ jest kwadratem wielomianu $P(x)=x^{2}+cx+d.$

Oblicz $a$ oraz $b.$

Odpowiedzí:





{\it Egzamin maturalny z matematyki}

{\it Poziom rozszerzony}

{\it 5}

Zadanie 3. $(5pkt)$

Kąt $\alpha$ jest taki, $\displaystyle \dot{\mathrm{z}}\mathrm{e}\cos\alpha+\sin\alpha=\frac{4}{3}$. Oblicz wartość wyrazenia $|\cos\alpha-\sin\alpha|.$

Odpowied $\acute{\mathrm{z}}$:





{\it 6}

{\it Egzamin maturalny z matematyki}

{\it Poziom rozszerzony}

Zadanie 4. $(5pkt)$

Wyznacz wszystkie wartości parametru $m$, dla których równanie $2x^{2}+(3-2m)x-m+1=0$

ma dwa rózne pierwiastki $x_{1}, x_{2}$ takie, $\dot{\mathrm{z}}\mathrm{e}|x_{1}-x_{2}|=3.$





{\it Egzamin maturalny z matematyki}

{\it Poziom rozszerzony}

7

Odpowied $\acute{\mathrm{z}}$:





{\it 8}

{\it Egzamin maturalny z matematyki}

{\it Poziom rozszerzony}

Zadanie 5. (5pkt)

W ciągu arytmetycznym

$(a_{n})$, dla $n\geq 1$, dane są

$a_{1}=-2$ oraz róz$\mathrm{m}\mathrm{c}\mathrm{a} r=3$. Oblicz

największe $n$ takie, $\dot{\mathrm{z}}\mathrm{e}a_{1}+a_{2}+\ldots+a_{n}<2012.$

Odpowiedzí:





{\it Egzamin maturalny z matematyki}

{\it Poziom rozszerzony}

{\it 9}

Zadanie 6. $(3pkt)$

Udowodnij, $\dot{\mathrm{z}}\mathrm{e}$ dla dowolnych liczb dodatnich $a, b, c \mathrm{i} d$ prawdziwa jest nierówność

$ac+bd\leq\sqrt{a^{2}+b^{2}}\cdot\sqrt{c^{2}+d^{2}}.$





$ 1\theta$

{\it Egzamin maturalny z matematyki}

{\it Poziom rozszerzony}

Zadanie 7. $(4pkt)$

Okrąg jest styczny do osi układu współrzędnych w punktach $A=(0,2) \mathrm{i}B=(2,0)$ oraz jest

styczny do prostej $l$ w punkcie $C=(1,a)$, gdzie $a>1$. Wyznacz równanie prostej $l.$






\begin{center}
\begin{tabular}{l|l}
\multicolumn{1}{l|}{{\it dysleksja}}&	\multicolumn{1}{|l}{}	\\
\hline
\multicolumn{1}{l|}{ $\begin{array}{l}\mbox{MATERIAL DIAGNOSTYCZNY}	\\	\mbox{Z MATEMATYKI}	\\	\mbox{Arkusz II}	\\	\mbox{POZIOM ROZSZERZONY}	\\	\mbox{Czas pracy 150 minut}	\\	\mbox{Instrukcja dla ucznia}	\\	\mbox{1. $\mathrm{S}\mathrm{p}\mathrm{r}\mathrm{a}\mathrm{w}\mathrm{d}\acute{\mathrm{z}}$, czy arkusz zawiera 12 ponumerowanych stron.}	\\	\mbox{Ewentualny brak zgłoś przewodniczącemu zespo}	\\	\mbox{nadzorującego badanie.}	\\	\mbox{2. Rozwiązania i odpowiedzi zapisz w miejscu na to}	\\	\mbox{przeznaczonym.}	\\	\mbox{3. $\mathrm{W}$ rozwiązaniach zadań przedstaw tok rozumowania}	\\	\mbox{prowadzący do ostatecznego wyniku.}	\\	\mbox{4. Pisz czytelnie. Uzywaj długopisu pióra tylko z czamym}	\\	\mbox{tusze atramentem.}	\\	\mbox{5. Nie uzywaj korektora, a błędne zapisy $\mathrm{w}\mathrm{y}\mathrm{r}\mathrm{a}\acute{\mathrm{z}}\mathrm{n}\mathrm{i}\mathrm{e}$ prze eśl.}	\\	\mbox{6. Pamiętaj, $\dot{\mathrm{z}}\mathrm{e}$ zapisy w brudnopisie nie podlegają ocenie.}	\\	\mbox{7. $\mathrm{M}\mathrm{o}\dot{\mathrm{z}}$ esz korzystać z zestawu wzorów matematycznych, cyrkla}	\\	\mbox{i linijki oraz kalkulatora.}	\\	\mbox{8. Wypełnij tę część ka $\mathrm{y}$ odpowiedzi, którą koduje uczeń. Nie}	\\	\mbox{wpisuj $\dot{\mathrm{z}}$ adnych znaków w części przeznaczonej dla}	\\	\mbox{oceniającego.}	\\	\mbox{9. Na karcie odpowiedzi wpisz swoją datę urodzenia i PESEL.}	\\	\mbox{Zamaluj $\blacksquare$ pola odpowiadające cyfrom numeru PESEL. Błędne}	\\	\mbox{zaznaczenie otocz kółkiem \fcircle i zaznacz właściwe.}	\\	\mbox{{\it Zyczymy} $p\theta wodzenia'$}	\end{array}$}&	\multicolumn{1}{|l}{$\begin{array}{l}\mbox{ARKUSZ II}	\\	\mbox{GRUDZIEN}	\\	\mbox{ROK 2005}	\\	\mbox{Za rozwiązanie}	\\	\mbox{wszystkich zadań}	\\	\mbox{mozna otrzymać}	\\	\mbox{łącznie}	\\	\mbox{50 punktów}	\end{array}$}	\\
\hline
\multicolumn{1}{l|}{$\begin{array}{l}\mbox{W ełnia uczeń rzed roz oczęciem rac}	\\	\mbox{PESEL UCZNIA}	\end{array}$}&	\multicolumn{1}{|l}{$\begin{array}{l}\mbox{Wypełnia uczeń}	\\	\mbox{przed rozpoczęciem}	\\	\mbox{pracy}	\\	\mbox{KOD UCZNIA}	\end{array}$}
\end{tabular}


\includegraphics[width=80.724mm,height=12.756mm]{./F1_M_PR_G2005_page0_images/image001.eps}

\includegraphics[width=23.616mm,height=9.852mm]{./F1_M_PR_G2005_page0_images/image002.eps}
\end{center}



{\it 2}

{\it Materialpomocniczy do doskonalenia nauczycieli w zakresie diagnozowania, oceniania i egzaminowania}

{\it Matematyka}- {\it grudzień 2005 r}.

Zadanie 11. (6pkt)

Wyznacz wszystkie liczby całkowite $k$, dla których funkcja

przyjmuje wartości dodatnie dla $\mathrm{k}\mathrm{a}\dot{\mathrm{z}}$ dego $x\in R.$

$f(x)=x^{2}-2^{k}\displaystyle \cdot x+2^{k}+\frac{5}{4}$





{\it Materialpomocniczy do doskonalenia nauczycieli w zakresie diagnozowania, oceniania i egzaminowania ll}

{\it Matematyka}- {\it grudzień 2005 r}.

Zadanie 19. (6pkt)

Korzystając z zasady indukcji matematycznej, udowodnij, $\dot{\mathrm{z}}\mathrm{e}\mathrm{k}\mathrm{a}\dot{\mathrm{z}}$ da liczba naturalna $n\geq 5$

spełnia nierówność $2^{n}>n^{2}+n-1.$





{\it 12 Materiatpomocniczy do doskonalenia nauczycieli w zakresie diagnozowania, oceniania i egzaminowania}

{\it Matematyka}- {\it grudzień 2005 r}.

BRUDNOPIS





{\it Materialpomocniczy do doskonalenia nauczycieli w zakresie diagnozowania, oceniania i egzaminowania}

{\it Matematyka}- {\it grudzień 2005 r}.

{\it 3}

Zadanie 12. (5pkt)
\begin{center}
\includegraphics[width=128.976mm,height=120.444mm]{./F1_M_PR_G2005_page2_images/image001.eps}
\end{center}
y

$-2$  1  x

Powyzszy rysunek przedstawia fragment wykresu pewnej funkcji wielomianowej $W(x)$

stopnia trzeciego. Jedynymi miejscami zerowymi tego wielomianu są liczby $(-2)$ oraz l,

a pochodna $W'(-2)=18.$

a) Wyznacz wzór wielomianu $W(x).$

b) Wyznacz równanie prostej stycznej do wykresu tego wielomianu w punkcie o odciętej

$x=3.$





{\it 4}

{\it Materialpomocniczy do doskonalenia nauczycieli w zakresie diagnozowania, oceniania i egzaminowania}

{\it Matematyka}- {\it grudzień 2005 r}.





{\it Materialpomocniczy do doskonalenia nauczycieli w zakresie diagnozowania, oceniania i egzaminowania}

{\it Matematyka}- {\it grudzień 2005 r}.

{\it 5}

Zadanie 13. (5pkt)

Sporządzí wykres funkcji $f(x)=|\displaystyle \frac{x-4}{x-2}|$, a następnie korzystając z tego wykresu, wyznacz

wszystkie wartości parametru $k$, dla których równanie $|\displaystyle \frac{x-4}{x-2}|=k$, ma dwa rozwiązania,

których iloczyn jest liczbą ujemną.





{\it 6}

{\it Materialpomocniczy do doskonalenia nauczycieli w zakresie diagnozowania, oceniania i egzaminowania}

{\it Matematyka}- {\it grudzień 2005 r}.

Zadanie 14. (4pkt)

Niech $A, B \subset \Omega$ będą zdarzeniami losowymi, takimi $\displaystyle \dot{\mathrm{z}}\mathrm{e}P(A)=\frac{5}{12}$ oraz $P(B)=\displaystyle \frac{7}{11}$

Zbadaj, czy zdarzenia $A\mathrm{i}B$ są rozłączne.





{\it Materialpomocniczy do doskonalenia nauczycieli w zakresie diagnozowania, oceniania i egzaminowania}

{\it Matematyka}- {\it grudzień 2005 r}.

7

Zadanie 15. $(5pkt)$

Dany jest nieskończony ciąg geometryczny postaci: 2, $\displaystyle \frac{2}{p-1}, \displaystyle \frac{2}{(p-1)^{2}}, \displaystyle \frac{2}{(p-1)^{3}}$, .

Wyznacz wszystkie wartości $p$, dla których granicą tego ciągu jest liczba:

a) 0.

b) 2.





{\it 8}

{\it Materialpomocniczy do doskonalenia nauczycieli w zakresie diagnozowania, oceniania i egzaminowania}

{\it Matematyka}- {\it grudzień 2005 r}.

Zadanie 16. (7pkt)

Danejest równanie postaci $(\cos x-1)\cdot(\cos x+p+1)=0$, gdzie $p\in R$ jest parametrem.

a) Dla $p=-1$ wypisz wszystkie rozwiązania tego równania nalezące do przedziału $\langle 0;5\rangle.$

b) Wyznacz wszystkie wartości parametru $p$, dla których dane równanie

ma w przedziale $\langle-\pi;\pi\rangle$ trzy rózne rozwiązania.





{\it Materialpomocniczy do doskonalenia nauczycieli w zakresie diagnozowania, oceniania i egzaminowania}

{\it Matematyka}- {\it grudzień 2005 r}.

{\it 9}

Zadanie 17. $(4pkt)$

$\mathrm{W}$ trójkącie prostokątnym $ABC(\triangleleft BCA=90^{\circ})$ dane są długości przyprostokątnych: $|BC|=a$

$\mathrm{i} |CA|=b$. Dwusieczna kąta prostego tego trójkąta przecina przeciwprostokątną

$AB$ w punkcie $D$. Wykaz, $\dot{\mathrm{z}}\mathrm{e}$ długość odcinka $CD$ jest równa $\displaystyle \frac{a\cdot b}{a+b}.\sqrt{2}$. Sporządzí

pomocniczy rysunek uwzględniając podane oznaczenia.





$ 1\theta$ {\it Materiatpomocniczy do doskonalenia nauczycieli w zakresie diagnozowania, oceniania i egzaminowania}

{\it Matematyka}- {\it grudzień 2005} $r.$

Zadanie 18. (8pkt)

Oblicz miary kątów dowolnego czworokąta wpisanego w okrąg o promieniu $R=5\sqrt{2},$

wiedząc ponadto, $\dot{\mathrm{z}}$ ejedna z przekątnych tego czworokąta ma długość 10, zaś i1oczyn sinusów

wszystkichjego kątów wewnętrznych równa się $\displaystyle \frac{3}{8}$






\begin{center}
\begin{tabular}{l|l}
\multicolumn{1}{l|}{$\begin{array}{l}\mbox{{\it dysleksja}}	\\	\mbox{Miejsce}	\\	\mbox{na na ejkę}	\\	\mbox{z kodem szkoly}	\end{array}$}&	\multicolumn{1}{|l}{}	\\
\hline
\multicolumn{1}{l|}{ $\begin{array}{l}\mbox{PRÓBNY EGZAMIN}	\\	\mbox{MATURALNY}	\\	\mbox{Z MATEMATYKI}	\\	\mbox{POZIOM ROZSZERZONY}	\\	\mbox{Czas pracy 180 minut}	\\	\mbox{Instrukcja dla zdającego}	\\	\mbox{1. $\mathrm{S}\mathrm{p}\mathrm{r}\mathrm{a}\mathrm{w}\mathrm{d}\acute{\mathrm{z}}$, czy arkusz egzaminacyjny zawiera 16 stron}	\\	\mbox{(zadania $1-12$). Ewentualny brak zgłoś przewodniczącemu}	\\	\mbox{zespo nadzorującego egzamin.}	\\	\mbox{2. Rozwiązania zadań i odpowiedzi zamieść w miejscu na to}	\\	\mbox{przeznaczonym.}	\\	\mbox{3. $\mathrm{W}$ rozwiązaniach zadań przedstaw tok rozumowania}	\\	\mbox{prowadzący do ostatecznego wyniku.}	\\	\mbox{4. Pisz czytelnie. Uzywaj długopisu pióra tylko z czamym}	\\	\mbox{tusze atramentem.}	\\	\mbox{5. Nie uzywaj korektora, a błędne zapisy prze eśl.}	\\	\mbox{6. Pamiętaj, $\dot{\mathrm{z}}\mathrm{e}$ zapisy w brudnopisie nie podlegają ocenie.}	\\	\mbox{7. $\mathrm{M}\mathrm{o}\dot{\mathrm{z}}$ esz korzystać z zestawu wzorów matematycznych, cyrkla}	\\	\mbox{i linijki oraz kalkulatora.}	\\	\mbox{8. Wypełnij tę część ka $\mathrm{y}$ odpowiedzi, którą koduje zdający.}	\\	\mbox{Nie wpisuj $\dot{\mathrm{z}}$ adnych znaków w części przeznaczonej dla}	\\	\mbox{egzaminatora.}	\\	\mbox{9. Na karcie odpowiedzi wpisz swoją datę urodzenia i PESEL.}	\\	\mbox{Zamaluj $\blacksquare$ pola odpowiadające cyfrom numeru PESEL. Błędne}	\\	\mbox{zaznaczenie otocz kółkiem $\mathrm{O}$ i zaznacz właściwe.}	\\	\mbox{{\it Zyczymy} $p\theta wodzenia'$}	\end{array}$}&	\multicolumn{1}{|l}{$\begin{array}{l}\mbox{LISTOPAD}	\\	\mbox{ROK 2006}	\\	\mbox{Za rozwiązanie}	\\	\mbox{wszystkich zadań}	\\	\mbox{mozna otrzymać}	\\	\mbox{łącznie}	\\	\mbox{50 punktów}	\end{array}$}	\\
\hline
\multicolumn{1}{l|}{$\begin{array}{l}\mbox{Wypelnia zdający przed}	\\	\mbox{roz oczęciem racy}	\\	\mbox{PESEL ZDAJACEGO}	\end{array}$}&	\multicolumn{1}{|l}{$\begin{array}{l}\mbox{KOD}	\\	\mbox{ZDAJACEGO}	\end{array}$}
\end{tabular}


\includegraphics[width=21.840mm,height=9.852mm]{./F1_M_PR_L2006_page0_images/image001.eps}

\includegraphics[width=78.792mm,height=13.356mm]{./F1_M_PR_L2006_page0_images/image002.eps}
\end{center}



{\it 2 Próbny egzamin maturalny z matematyki}

{\it Poziom rozszerzony}

Zadanie l. $(5pkt)$

Funkcja homograficzna

parametrem i $|p|\neq\sqrt{3}.$

f jest

określona

wzorem

$f(x)=\underline{px-3},$

$x-p$

gdzie

$p\in R$

jest

a) Dla $p=1$ zapisz wzór ffinkcji w postaci $f(x)=k+\displaystyle \frac{m}{x-1}$, gdzie $k$ oraz $m$

są liczbami rzeczywistymi.

b) Wyznacz wszystkie wartości parametru $p$, dla których w przedziale $(p,+\infty)$ funkcja $f$

jest malejąca.





{\it Próbny egzamin maturalny z matematyki ll}

{\it Poziom rozszerzony}

Zadanie 9. (3pkt)

Niech $ A\subset\Omega \mathrm{i}  B\subset\Omega$ będą zdarzeniami losowymi. Mając dane prawdopodobieństwa

zdarzeń: $P(A)=0,5, P(B)=0,4 \mathrm{i} P(A\backslash B)=0,3$, zbadaj, czy $A \mathrm{i} B$ są zdarzeniami

niezaleznymi.
\begin{center}
\includegraphics[width=195.168mm,height=254.460mm]{./F1_M_PR_L2006_page10_images/image001.eps}
\end{center}




{\it 12 Próbny egzamin maturalny z matematyki}

{\it Poziom rozszerzony}

Zadanie 10. (5pkt)

Ciąg liczbowy

$(a_{n})$

jest określony

dla $\mathrm{k}\mathrm{a}\dot{\mathrm{z}}$ dej

liczby naturalnej

$n\geq 1$ wzorem

$a_{n}=(n-3)(2-p^{2})$, gdzie $p\in R.$

a) Wykaz, $\dot{\mathrm{z}}\mathrm{e}$ dla $\mathrm{k}\mathrm{a}\dot{\mathrm{z}}$ dej wartości $p$ ciąg $(a_{n})$ jest arytmetyczny.

b) Dla $p=2$ oblicz sumę $a_{20}+a_{21}+a_{22}\cdots+a_{40}.$

c) Wyznacz wszystkie wartości $p$, dla których ciąg $(b_{n})$ określony wzorem $b_{n}=a_{n}-pn$

jest stały.
\begin{center}
\includegraphics[width=195.168mm,height=224.184mm]{./F1_M_PR_L2006_page11_images/image001.eps}
\end{center}




{\it Próbny egzamin maturalny z matematyki 13}

{\it Poziom rozszerzony}

Zadanie 11. (3pkt)

Funkcja $f$ przyporządkowuje $\mathrm{k}\mathrm{a}\dot{\mathrm{z}}$ dej liczbie naturalnej $n>1$ największą liczbę całkowitą

spełniającą nierówność $x^{2}-3nx+2n^{2}<0$ o niewiadomej $x$. Wyznacz wzór funkcji $f$





{\it 14 Próbny egzamin maturalny z matematyki}

{\it Poziom rozszerzony}

Zadanie 12. (4pkt)

Dwa okręgi, $\mathrm{k}\mathrm{a}\dot{\mathrm{z}}\mathrm{d}\mathrm{y}$ o promieniu 8, są styczne zewnętrznie. Ze środka jednego z nich

poprowadzono styczne do drugiego okręgu. Oblicz pole zacieniowanej figury (patrz rysunek).
\begin{center}
\includegraphics[width=96.516mm,height=84.432mm]{./F1_M_PR_L2006_page13_images/image001.eps}
\end{center}
{\it A  B}
\begin{center}
\includegraphics[width=195.168mm,height=163.632mm]{./F1_M_PR_L2006_page13_images/image002.eps}
\end{center}




{\it Próbny egzamin maturalny z matematyki 15}

{\it Poziom rozszerzony}
\begin{center}
\includegraphics[width=195.168mm,height=290.724mm]{./F1_M_PR_L2006_page14_images/image001.eps}
\end{center}




{\it 16 Próbny egzamin maturalny z matematyki}

{\it Poziom rozszerzony}

BRUDNOPIS





{\it Próbny egzamin maturalny z matematyki 3}

{\it Poziom rozszerzony}

Zadanie 2. (5pkt)

Wyznacz wszystkie wartości

$W(x)=(x^{2}-8x+12)\cdot(x-k)$ są

geometrycznego.

$k\in R,$

trzema

dla których pierwiastki wielomianu

kolejnymi wyrazami rosnącego ciągu
\begin{center}
\includegraphics[width=195.168mm,height=260.508mm]{./F1_M_PR_L2006_page2_images/image001.eps}
\end{center}




{\it 4 Próbny egzamin maturalny z matematyki}

{\it Poziom rozszerzony}

Zadanie 3. (4pkt)

Na rysunku ponizej przedstawiono wykres funkcji logarytmicznej f.

Rozwiąz równanie $(f(x))^{2}-16=0.$
\begin{center}
\includegraphics[width=195.168mm,height=157.584mm]{./F1_M_PR_L2006_page3_images/image001.eps}
\end{center}




{\it Próbny egzamin maturalny z matematyki 5}

{\it Poziom rozszerzony}
\begin{center}
\includegraphics[width=195.168mm,height=290.724mm]{./F1_M_PR_L2006_page4_images/image001.eps}
\end{center}




{\it 6 Próbny egzamin maturalny z matematyki}

{\it Poziom rozszerzony}

Zadanie 4. $(7pkt)$

Trójkąt prostokątny $ABC$, w którym $|\infty BCA|=90^{\circ} \mathrm{i} |\infty CAB|=30^{\circ}$, jest opisany na okręgu

o promieniu $\sqrt{3}$. Oblicz odległość wierzchołka $C$ trójkąta od punktu styczności tego okręgu

z przeciwprostokątną. Wykonaj odpowiedni rysunek.





{\it Próbny egzamin maturalny z matematyki 7}

{\it Poziom rozszerzony}

Zadanie 5. $(3pkt)$

Sporządzí wykres funkcji $f$ danej wzorem $f(x)=2|x|-x^{2}$, a następnie, korzystając z niego,

podaj wszystkie wartości $x$, dla których funkcja $f$ przyjmuje maksima lokalne i wszystkie

wartości $x$, dla których przyjmuje minima lokalne.
\begin{center}
\includegraphics[width=195.168mm,height=260.508mm]{./F1_M_PR_L2006_page6_images/image001.eps}
\end{center}




{\it 8 Próbny egzamin maturalny z matematyki}

{\it Poziom rozszerzony}

Zadanie 6. $(4pkt)$

Podstawa $AB$ trapezu ABCD jest zawarta w osi $Ox$, wierzchołek $D$ jest punktem przecięcia

paraboli o równaniu $y=-\displaystyle \frac{1}{3}x^{2}+x+6$ z osią $oy$. Pozostałe wierzchołki trapezu równiez $\mathrm{l}\mathrm{e}\dot{\mathrm{z}}$ ą

na tej paraboli (patrz rysunek). Oblicz pole tego trapezu.
\begin{center}
\includegraphics[width=83.724mm,height=69.444mm]{./F1_M_PR_L2006_page7_images/image001.eps}
\end{center}
{\it y}

{\it D C}

{\it A  B}

0  {\it x}
\begin{center}
\includegraphics[width=195.168mm,height=169.728mm]{./F1_M_PR_L2006_page7_images/image002.eps}
\end{center}




{\it Próbny egzamin maturalny z matematyki 9}

{\it Poziom rozszerzony}

Zadanie 7. (3pkt)

Wyznacz wszystkie rozwiązania równania $2\cos^{2}x=\cos x$ nalezące do przedziału $\langle 0,2\pi\rangle.$
\begin{center}
\includegraphics[width=195.168mm,height=266.544mm]{./F1_M_PR_L2006_page8_images/image001.eps}
\end{center}




$ 1\theta$ {\it Próbny egzamin maturalny z matematyki}

{\it Poziom rozszerzony}

Zadanie 8. (4pkt)

Uczeń analizował własności funkcji $f$, której dziedziną jest zbiór wszystkich liczb

rzeczywistych i która ma pochodną $f'(x)$ dla $\mathrm{k}\mathrm{a}\dot{\mathrm{z}}$ dego $x\in R$. Wyniki tej analizy zapisał

w tabeli.
\begin{center}
\begin{tabular}{|l|l|l|l|l|l|l|l|}
\hline
\multicolumn{1}{|l|}{$x$}&	\multicolumn{1}{|l|}{ $(-\infty,-1)$}&	\multicolumn{1}{|l|}{ $-1$}&	\multicolumn{1}{|l|}{ $(-1,2)$}&	\multicolumn{1}{|l|}{ $2$}&	\multicolumn{1}{|l|}{ $(2,3)$}&	\multicolumn{1}{|l|}{ $3$}&	\multicolumn{1}{|l|}{ $(3,+\infty)$}	\\
\hline
\multicolumn{1}{|l|}{ $f'(x)$}&	\multicolumn{1}{|l|}{ $(+)$}&	\multicolumn{1}{|l|}{ $0$}&	\multicolumn{1}{|l|}{ $(-)$}&	\multicolumn{1}{|l|}{ $0$}&	\multicolumn{1}{|l|}{ $(-)$}&	\multicolumn{1}{|l|}{ $0$}&	\multicolumn{1}{|l|}{ $(-)$}	\\
\hline
\multicolumn{1}{|l|}{ $f(x)$}&	\multicolumn{1}{|l|}{}&	\multicolumn{1}{|l|}{ $2$}&	\multicolumn{1}{|l|}{}&	\multicolumn{1}{|l|}{ $-1$}&	\multicolumn{1}{|l|}{}&	\multicolumn{1}{|l|}{ $1$}&	\multicolumn{1}{|l|}{}	\\
\hline
\end{tabular}

\end{center}
Niestety, wpisując znaki pochodnej, popełniłjeden błąd.

a) Przekreśl błędnie wpisany znak pochodnej i wstaw obok prawidłowy.

b) Napisz, czy po poprawieniu błędu w tabeli, zawarte w niej dane pozwolą określić

dokładną liczbę miejsc zerowych ffinkcji $f$. Uzasadniając swoją odpowiedzí $\mathrm{m}\mathrm{o}\dot{\mathrm{z}}$ esz

naszkicować przykładowe wykresy funkcji.






\begin{center}
\begin{tabular}{l|l}
\multicolumn{1}{l|}{$\begin{array}{l}\mbox{{\it dysleksja}}	\\	\mbox{Miejsce}	\\	\mbox{na na ejkę}	\\	\mbox{z kodem szkoly}	\end{array}$}&	\multicolumn{1}{|l}{MMA-RIAIP-052}	\\
\hline
\multicolumn{1}{l|}{$\begin{array}{l}\mbox{EGZAMIN MATURALNY}	\\	\mbox{Z MATEMATYKI}	\\	\mbox{Arkusz II}	\\	\mbox{POZIOM ROZSZERZONY}	\\	\mbox{Czas pracy 150 minut}	\\	\mbox{Instrukcja dla zdającego}	\\	\mbox{1. $\mathrm{S}\mathrm{p}\mathrm{r}\mathrm{a}\mathrm{w}\mathrm{d}\acute{\mathrm{z}}$, czy arkusz egzaminacyjny zawiera 15 stron.}	\\	\mbox{Ewentualny brak zgłoś przewodniczącemu zespo}	\\	\mbox{nadzorującego egzamin.}	\\	\mbox{2. Rozwiązania zadań i odpowiedzi zamieść w miejscu na to}	\\	\mbox{przeznaczonym.}	\\	\mbox{3. $\mathrm{W}$ rozwiązaniach zadań przedstaw tok rozumowania}	\\	\mbox{prowadzący do ostatecznego wyniku.}	\\	\mbox{4. Pisz czytelnie. Uzywaj długopisu pióra tylko z czatnym}	\\	\mbox{tusze atramentem.}	\\	\mbox{5. Nie uzywaj korektora. Błędne zapisy prze eśl.}	\\	\mbox{6. Pamiętaj, $\dot{\mathrm{z}}\mathrm{e}$ zapisy w $\mathrm{b}$ dnopisie nie podlegają ocenie.}	\\	\mbox{7. Obok $\mathrm{k}\mathrm{a}\dot{\mathrm{z}}$ dego zadania podanajest maksymalna liczba punktów,}	\\	\mbox{którą mozesz uzyskać zajego poprawne rozwiązanie.}	\\	\mbox{8. $\mathrm{M}\mathrm{o}\dot{\mathrm{z}}$ esz korzystać z zestawu wzorów matematycznych, cyrkla}	\\	\mbox{i linijki oraz kalkulatora.}	\\	\mbox{9. Wypełnij tę część ka $\mathrm{y}$ odpowiedzi, którą koduje zdający.}	\\	\mbox{Nie wpisuj $\dot{\mathrm{z}}$ adnych znaków w części przeznaczonej}	\\	\mbox{dla egzaminatora.}	\\	\mbox{10. Na karcie odpowiedzi wpisz swoją datę urodzenia i PESEL.}	\\	\mbox{Zamaluj $\blacksquare$ pola odpowiadające cyfrom numeru PESEL. Błędne}	\\	\mbox{zaznaczenie otocz kółkiem i zaznacz właściwe.}	\\	\mbox{{\it Zyczymy powodzenia}.'}	\end{array}$}&	\multicolumn{1}{|l}{$\begin{array}{l}\mbox{ARKUSZ II}	\\	\mbox{MAJ}	\\	\mbox{ROK 2005}	\\	\mbox{Za rozwiązanie}	\\	\mbox{wszystkich zadań}	\\	\mbox{mozna otrzymać}	\\	\mbox{łącznie}	\\	\mbox{50 punktów}	\end{array}$}	\\
\hline
\multicolumn{1}{l|}{$\begin{array}{l}\mbox{Wypelnia zdający przed}	\\	\mbox{roz oczęciem racy}	\\	\mbox{PESEL ZDAJACEGO}	\end{array}$}&	\multicolumn{1}{|l}{$\begin{array}{l}\mbox{tylko}	\\	\mbox{O Kraków,}	\\	\mbox{OKE Wroclaw}	\\	\mbox{KOD}	\\	\mbox{ZDAJACEGO}	\end{array}$}
\end{tabular}


\includegraphics[width=78.792mm,height=13.356mm]{./F1_M_PR_M2005_page0_images/image001.eps}

\includegraphics[width=21.840mm,height=9.804mm]{./F1_M_PR_M2005_page0_images/image002.eps}
\end{center}



{\it 2}

{\it Egzamin maturalny z matematyki}

{\it Arkusz II}

Zadanie 11. (3pkt)

Wyznacz dziedzinę funkcji

przedziałów liczbowych.

$f(x)=\log_{x^{2}-3}(x^{3}+4x^{2}-x-4)$ i zapisz ją w postaci sumy
\begin{center}
\includegraphics[width=192.588mm,height=258.720mm]{./F1_M_PR_M2005_page1_images/image001.eps}
\end{center}




{\it Egzamin maturalny z matematyki}

{\it Arkusz II}

{\it 11}
\begin{center}
\includegraphics[width=192.588mm,height=294.792mm]{./F1_M_PR_M2005_page10_images/image001.eps}
\end{center}




{\it 12}

{\it Egzamin maturalny z matematyki}

{\it Arkusz II}

Zadanie 19. $(1\theta pkt)$

Dane jest równanie: $x^{2}+(m-5)x+m^{2}+m+\displaystyle \frac{1}{4}=0.$

Zbadaj, dla jakich wartości parametru $m$ stosunek sumy pierwiastków rzeczywistych

równania do ich iloczynu przyjmuje wartość najmniejszą. Wyznacz tę wartość.
\begin{center}
\includegraphics[width=192.588mm,height=258.720mm]{./F1_M_PR_M2005_page11_images/image001.eps}
\end{center}




{\it Egzamin maturalny z matematyki}

{\it Arkusz II}

{\it 13}
\begin{center}
\includegraphics[width=192.588mm,height=294.792mm]{./F1_M_PR_M2005_page12_images/image001.eps}
\end{center}




{\it 14}

{\it Egzamin maturalny z matematyki}

{\it Arkusz II}

BRUDNOPIS





{\it Egzamin maturalny z matematyki}

{\it Arkusz II}

{\it 15}





{\it Egzamin maturalny z matematyki}

{\it Arkusz II}

{\it 3}

Zadanie 12. (4pkt)

Dana jest funkcja: $f(x)=\cos x-\sqrt{3}\sin x,$

a) Naszkicuj wykres funkcji $f.$

b) Rozwiąz równanie: $f(x)=1.$

$x\in R.$
\begin{center}
\includegraphics[width=202.740mm,height=101.652mm]{./F1_M_PR_M2005_page2_images/image001.eps}
\end{center}
i

2

-2

$\rightarrow 2$
\begin{center}
\includegraphics[width=192.588mm,height=156.516mm]{./F1_M_PR_M2005_page2_images/image002.eps}
\end{center}




{\it 4}

{\it Egzamin maturalny z matematyki}

{\it Arkusz II}

Zadanie 13. (4pkt)

Rzucamy $n$ razy dwiema symetrycznymi sześciennymi kostkami do gry. Oblicz, dlajakich $n$

prawdopodobieństwo otrzymania co najmniej raz tej samej liczby oczek na obu kostkachjest

mniejsze od $\displaystyle \frac{671}{1296}.$
\begin{center}
\includegraphics[width=192.588mm,height=252.732mm]{./F1_M_PR_M2005_page3_images/image001.eps}
\end{center}




{\it Egzamin maturalny z matematyki}

{\it Arkusz II}

{\it 5}

Zadanie 14. (5pkt)

Ob1icz: {\it n}1i$\rightarrow$m$\infty$ -51 $++$47 $++$79 $++$...... $++$((32{\it nn} -$+$23)).
\begin{center}
\includegraphics[width=192.588mm,height=258.720mm]{./F1_M_PR_M2005_page4_images/image001.eps}
\end{center}




{\it 6}

{\it Egzamin maturalny z matematyki}

{\it Arkusz II}

Zadanie 15. (4pkt)

W dowolnym trójkącie ABC punkty MiN są odpowiednio środkami boków ACiBC (Rys. l).

{\it c}
\begin{center}
\includegraphics[width=86.712mm,height=43.992mm]{./F1_M_PR_M2005_page5_images/image001.eps}
\end{center}
Rys. l

{\it A  B}

Zapoznaj się uwaznie z następującym rozumowaniem:

Korzystając z własności wektorów i działań na wektorach, zapisujemy równoŚci:

oraz

$\vec{MN}=\vec{MA}+\vec{AB}+\vec{BN}$ (1)

$\vec{MN}=\vec{MC}+\vec{CN}$ (2)

Po dodaniu równości (l) $\mathrm{i}$ (2) stronami otrzymujemy:

2. $\vec{MN}=\vec{MA}+\vec{MC}+\vec{AB}+\vec{BN}+\vec{CN}$

Poniewaz $\vec{MC}=-\vec{MA}$ oraz $\vec{CN}=-\vec{BN}$, więc:

2. $\vec{MN}=\vec{MA}-\vec{MA}+\vec{AB}+\vec{BN}-\vec{BN}$

2. $\vec{MN}=\vec{\text{Õ}}+\vec{AB}+\vec{0}$

$\displaystyle \vec{MN}=\frac{1}{2}\cdot\vec{AB}.$

Wykorzystując własności iloczynu wektora przez liczbę, ostatnią równość

zinterpretować następująco:

mozna

odcinek lączący środki dwóch boków dowolnego trójkąta jest równolegly do trzeciego

boku tego trójkąta, zaś jego dlugośćjest równa polowie dlugości tego boku.

Przeprowadzając analogiczne rozumowanie, ustal związek pomiędzy wektorem $\vec{MN}$ oraz

wektorami $\vec{AB} \mathrm{i} \vec{DC}$, wiedząc, $\dot{\mathrm{z}}\mathrm{e}$ czworokąt ABCD jest dowolnym trapezem, zaś punkty

$M\mathrm{i}N$ są odpowiednio środkami ramion AD $\mathrm{i}BC$ tego trapezu (Rys. 2).

Rys. 2
\begin{center}
\includegraphics[width=91.536mm,height=46.020mm]{./F1_M_PR_M2005_page5_images/image002.eps}
\end{center}
{\it A}

Podaj interpretację otrzymanego wyniku.





{\it Egzamin maturalny z matematyki}

{\it Arkusz II}

7
\begin{center}
\includegraphics[width=192.588mm,height=294.792mm]{./F1_M_PR_M2005_page6_images/image001.eps}
\end{center}




{\it 8}

{\it Egzamin maturalny z matematyki}

{\it Arkusz II}

Zadanie 16. (5pkt)

Sześcian o krawędzi długości $a$ przecięto płaszczyzną przechodzącą przez przekątną

podstawy i nachyloną do płaszczyzny podstawy pod kątem $\displaystyle \frac{\pi}{3}$. Sporządz$\acute{}$ odpowiedni rysunek.

Oblicz pole otrzymanego przekroju.
\begin{center}
\includegraphics[width=192.588mm,height=252.732mm]{./F1_M_PR_M2005_page7_images/image001.eps}
\end{center}




{\it Egzamin maturalny z matematyki}

{\it Arkusz II}

{\it 9}

Zadanie 17. (7pkt)

Wykaz, bez uzycia kalkulatora i tablic, $\dot{\mathrm{z}}\mathrm{e}\sqrt[3]{5\sqrt{2}+7}-\sqrt[3]{5\sqrt{2}-7}$jest liczbą całkowitą.
\begin{center}
\includegraphics[width=192.588mm,height=264.720mm]{./F1_M_PR_M2005_page8_images/image001.eps}
\end{center}




$ 1\theta$

{\it Egzamin maturalny z matematyki}

{\it Arkusz II}

Zadanie 18. (8pkt)

Pary liczb $(x,y)$ spełniające układ równań:

$\left\{\begin{array}{l}
-4x^{2}+y^{2}+2y+1=0\\
-x^{2}+y+4=0
\end{array}\right.$

są współrzędnymi wierzchołków czworokąta wypukłego ABCD.

a) Wyznacz współrzędne punktów: $A, B, C, D.$

b) Wykaz, $\dot{\mathrm{z}}\mathrm{e}$ czworokąt ABCD jest trapezem równoramiennym.

c) Wyznacz równanie okręgu opisanego na czworokącie ABCD.
\begin{center}
\includegraphics[width=192.588mm,height=228.648mm]{./F1_M_PR_M2005_page9_images/image001.eps}
\end{center}





\begin{center}
\begin{tabular}{l|l}
\multicolumn{1}{l|}{$\begin{array}{l}\mbox{{\it dysleksja}}	\\	\mbox{Miejsce}	\\	\mbox{na na ejkę}	\\	\mbox{z kodem szkoly}	\end{array}$}&	\multicolumn{1}{|l}{MMA-RIAIP-02}	\\
\hline
\multicolumn{1}{l|}{ $\begin{array}{l}\mbox{EGZAMIN MATURALNY}	\\	\mbox{Z MATEMATYKI}	\\	\mbox{Arkusz II}	\\	\mbox{POZIOM ROZSZERZONY}	\\	\mbox{Czas pracy 150 minut}	\\	\mbox{Instrukcja dla zdającego}	\\	\mbox{1. Sprawdzí, czy arkusz egzaminacyjny zawiera 14 stron}	\\	\mbox{(zadania $12-21$). Ewentualny brak zgłoś przewodniczącemu}	\\	\mbox{zespo nadzorującego egzamin.}	\\	\mbox{2. Rozwiązania zadań i odpowiedzi zamieść w miejscu na to}	\\	\mbox{przeznaczonym.}	\\	\mbox{3. $\mathrm{W}$ rozwiązaniach zadań przedstaw tok rozumowania}	\\	\mbox{prowadzący do ostatecznego wyniku.}	\\	\mbox{4. Pisz czytelnie. $\mathrm{U}\dot{\mathrm{z}}$ aj długopisu pióra tylko z czarnym}	\\	\mbox{tusze atramentem.}	\\	\mbox{5. Nie uzywaj korektora, a błędne zapisy prze eśl.}	\\	\mbox{6. Pamiętaj, $\dot{\mathrm{z}}\mathrm{e}$ zapisy w $\mathrm{b}$ dnopisie nie podlegają ocenie.}	\\	\mbox{7. Obok $\mathrm{k}\mathrm{a}\dot{\mathrm{z}}$ dego zadania podanajest maksymalna liczba punktów,}	\\	\mbox{którą mozesz uzyskać zajego poprawne rozwiązanie.}	\\	\mbox{8. $\mathrm{M}\mathrm{o}\dot{\mathrm{z}}$ esz korzystać z zestawu wzorów matematycznych, cyrkla}	\\	\mbox{i linijki oraz kalkulatora.}	\\	\mbox{9. Wypełnij tę część ka $\mathrm{y}$ odpowiedzi, którą koduje zdający.}	\\	\mbox{Nie wpisuj $\dot{\mathrm{z}}$ adnych znaków w części przeznaczonej dla}	\\	\mbox{egzaminatora.}	\\	\mbox{10. Na karcie odpowiedzi wpisz swoją datę urodzenia i PESEL.}	\\	\mbox{Zamaluj $\blacksquare$ pola odpowiadające cyfrom numeru PESEL. Błędne}	\\	\mbox{zaznaczenie otocz kółkiem $\mathrm{O}$ i zaznacz właściwe.}	\\	\mbox{{\it Zyczymy} $p\theta wodzenia'$}	\end{array}$}&	\multicolumn{1}{|l}{$\begin{array}{l}\mbox{ARKUSZ II}	\\	\mbox{MAJ}	\\	\mbox{ROK 2006}	\\	\mbox{Za rozwiązanie}	\\	\mbox{wszystkich zadań}	\\	\mbox{mozna otrzymać}	\\	\mbox{łącznie}	\\	\mbox{50 punktów}	\end{array}$}	\\
\hline
\multicolumn{1}{l|}{$\begin{array}{l}\mbox{Wypelnia zdający przed}	\\	\mbox{roz oczęciem racy}	\\	\mbox{PESEL ZDAJACEGO}	\end{array}$}&	\multicolumn{1}{|l}{$\begin{array}{l}\mbox{KOD}	\\	\mbox{ZDAJACEGO}	\end{array}$}
\end{tabular}


\includegraphics[width=21.840mm,height=9.852mm]{./F1_M_PR_M2006_page0_images/image001.eps}

\includegraphics[width=78.792mm,height=13.356mm]{./F1_M_PR_M2006_page0_images/image002.eps}
\end{center}



{\it 2}

{\it Egzamin maturalny z matematyki}

{\it Arkusz II}
\begin{center}
\includegraphics[width=192.228mm,height=288.036mm]{./F1_M_PR_M2006_page1_images/image001.eps}
\end{center}
Zadanie 12. $(5pkt)$

Korzystając z zasady indukcji matematycznej wykaz, $\dot{\mathrm{z}}\mathrm{e}$ dla $\mathrm{k}\mathrm{a}\dot{\mathrm{z}}$ dej liczby naturalnej $n\geq 1$

prawdziwy jest wzór: l$\cdot$ 3$\cdot(1!)^{2}+2\cdot 4\cdot(2!)^{2}+\cdots+n(n+2)(n!)^{2}=[(n+1)!]^{2}-1.$
\begin{center}
\includegraphics[width=137.868mm,height=17.580mm]{./F1_M_PR_M2006_page1_images/image002.eps}
\end{center}
Nr czynnoścÍ

WypelnÍa Maks. liczba kt

egzaminator! Uzyskana liczba pkt

12.1.

1

12.2.

1

12.3.

12.4.

1

12.5.

1





{\it Egzamin maturalny z matematyki}

{\it Arkusz II}

{\it 11}

Zadanie 20. $(4pkt)$

Dane są funkcje $f(x)=3^{x^{2}-5x} \mathrm{i} g(x)=(\displaystyle \frac{1}{9})^{-2x^{2}-3x+2}$

Oblicz, dla których argumentów $x$ wartości funkcji $f$ sąwiększe od wartości funkcji $g.$
\begin{center}
\includegraphics[width=192.276mm,height=260.508mm]{./F1_M_PR_M2006_page10_images/image001.eps}

\includegraphics[width=123.900mm,height=17.628mm]{./F1_M_PR_M2006_page10_images/image002.eps}
\end{center}
Nr czynnoŚci

Wypelnia Maks. liczba kt

egzaminator! Uzyskana liczba pkt

20.1.

1

20.2.

1

20.3.

1

20.4.

1





{\it 12}

{\it Egzamin maturalny z matematyki}

{\it Arkusz II}

Zadanie 21. $(5pkt)$

$\mathrm{W}$ trakcie badania przebiegu zmienności funkcji ustalono, $\dot{\mathrm{z}}\mathrm{e}$ ffinkcja

własności:

- jej dziedzinąjest zbiór wszystkich liczb rzeczywistych,

- $f$ jest funkcją nieparzyst\%

- $f$ jest funkcją ciągłą

oraz:

$f'(x)<0$ dla $x\in(-8,-3),$

$f'(x)>0$ dla $x\in(-3,-1),$

f ma następujące

$f'(x)<0$ dla $x\in(-1,0),$

$f'(-3)=f'(-1)=0,$

$f(-8)=0,$

$f(-3)=-2,$

$f(-2)=0,$

$f(-1)=1.$

$\mathrm{W}$ prostokątnym układzie współrzędnych na płaszczyz$\acute{}$nie naszkicuj wykres funkcji $f$

w przedziale $\langle-8,8\rangle$, wykorzystując podane powyzej informacje ojej własnościach.
\begin{center}
\includegraphics[width=192.228mm,height=48.672mm]{./F1_M_PR_M2006_page11_images/image001.eps}
\end{center}




{\it Egzamin maturalny z matematyki}

{\it Arkusz II}

{\it 13}
\begin{center}
\includegraphics[width=192.276mm,height=290.724mm]{./F1_M_PR_M2006_page12_images/image001.eps}

\includegraphics[width=109.980mm,height=17.580mm]{./F1_M_PR_M2006_page12_images/image002.eps}
\end{center}
Nr czynno\S ci

WypelnÍa Maks. liczba kt

egzaminator! Uzyskana liczba pkt

21.1.

1

21.2.

2

21.3.

2





{\it 14}

{\it Egzamin maturalny z matematyki}

{\it Arkusz II}

BRUDNOPIS





{\it Egzamin maturalny z matematyki}

{\it Arkusz II}

{\it 3}

Zadanie 13. $(5pkt)$

Dany jest ciąg $(a_{n})$, gdzie $a_{n}=\displaystyle \frac{5n+6}{10(n+1)}$ dla $\mathrm{k}\mathrm{a}\dot{\mathrm{z}}$ dej liczby naturalnej $n\geq 1.$

a) Zbadaj monotoniczność ciągu $(a_{n}).$

b) Oblicz $\displaystyle \lim_{n\rightarrow\infty}a_{n}.$

c) Podaj największą liczbę $a$ i najnmiejszą liczbę $b$ takie, $\dot{\mathrm{z}}\mathrm{e}$ dla $\mathrm{k}\mathrm{a}\dot{\mathrm{z}}$ dego $n$ spełniony jest

warunek $a\leq a_{n}\leq b.$
\begin{center}
\includegraphics[width=192.276mm,height=236.268mm]{./F1_M_PR_M2006_page2_images/image001.eps}

\includegraphics[width=137.928mm,height=17.580mm]{./F1_M_PR_M2006_page2_images/image002.eps}
\end{center}
Nr czynności

Wypelnia Maks. liczba kt

egzaminator! Uzyskana liczba pkt

13.1.

1

13.2.

13.3.

13.4.

1

13.5.

1





{\it 4}

{\it Egzamin maturalny z matematyki}

{\it Arkusz II}

Zadanie 14. $(4pkt)$

a) Naszkicuj wykres funkcji $y=\sin 2x$ w przedziale $<-2\pi,2\pi>.$
\begin{center}
\includegraphics[width=192.228mm,height=121.308mm]{./F1_M_PR_M2006_page3_images/image001.eps}
\end{center}
b)

Naszkicuj wykres funkcji $y=\displaystyle \frac{|\sin 2x|}{\sin 2x}$ w przedziale $<-2\pi,2\pi>$

i zapisz, dla których liczb z tego przedziału spełnionajest nierówność $\displaystyle \frac{|\sin 2x|}{\sin 2x}<0.$
\begin{center}
\includegraphics[width=192.228mm,height=121.308mm]{./F1_M_PR_M2006_page3_images/image002.eps}
\end{center}




{\it Egzamin maturalny z matematyki}

{\it Arkusz II}

{\it 5}
\begin{center}
\includegraphics[width=192.276mm,height=290.724mm]{./F1_M_PR_M2006_page4_images/image001.eps}

\includegraphics[width=123.900mm,height=17.580mm]{./F1_M_PR_M2006_page4_images/image002.eps}
\end{center}
Nr czynnoścÍ

Wypelnia Maks. liczba kt

egzaminator! Uzyskana liczba pkt

14.1.

1

14.2.

1

14.3.

1

14.4.

1





{\it 6}

{\it Egzamin maturalny z matematyki}

{\it Arkusz II}

Zadanie 15. $(4pkt)$

Uczniowie dojez $\mathrm{d}\dot{\mathrm{z}}$ ający do szkoły zaobserwowali, $\dot{\mathrm{z}}\mathrm{e}$ spózínienie autobusu zalez$\mathrm{y}$ od tego,

który z trzech kierowców prowadzi autobus. Przeprowadzili badania statystyczne i obliczyli,

$\dot{\mathrm{z}}\mathrm{e}$ w przypadku, gdy autobus prowadzi kierowca $\mathrm{A}$, spózínienie zdarza się w 5\% jego kursów,

gdy prowadzi kierowca $\mathrm{B}$ w 20\% jego kursów, a gdy prowadzi kierowca $\mathrm{C}$ w 50\% jego

kursów. $\mathrm{W}$ ciągu 5-dniowego tygodnia nauki dwa razy prowadzi autobus kierowca $\mathrm{A}$, dwa

razy kierowca $\mathrm{B}$ i jeden raz kierowca C. Oblicz prawdopodobieństwo spózínienia się

szkolnego autobusu w losowo wybrany dzień nauki.
\begin{center}
\includegraphics[width=192.228mm,height=242.364mm]{./F1_M_PR_M2006_page5_images/image001.eps}

\includegraphics[width=123.948mm,height=17.580mm]{./F1_M_PR_M2006_page5_images/image002.eps}
\end{center}
Nr czynno\S ci

Wypelnia Maks. liczba kt

egzamÍnator! Uzyskana liczba pkt

15.1.

1

15.2.

1

15.3.

1

15.4.

1





{\it Egzamin maturalny z matematyki}

{\it Arkusz II}

7

Zadanie 16. $(3pkt)$

Obiekty $A\mathrm{i}B$ lez$\cdot$ą po dwóch stronach jeziora. $\mathrm{W}$ terenie dokonano pomiarów odpowiednich

kątów i ich wyniki przedstawiono na rysunku. Odległość między obiektami $B\mathrm{i}C$ jest równa

400 $\mathrm{m}$. Oblicz odległość w linii prostej między obiektami $A\mathrm{i}B$ i podaj wynik, zaokrąglając

go do jednego metra.
\begin{center}
\includegraphics[width=67.968mm,height=35.808mm]{./F1_M_PR_M2006_page6_images/image001.eps}
\end{center}
{\it A}
\begin{center}
\includegraphics[width=192.276mm,height=212.088mm]{./F1_M_PR_M2006_page6_images/image002.eps}

\includegraphics[width=109.980mm,height=17.628mm]{./F1_M_PR_M2006_page6_images/image003.eps}
\end{center}
Nr czynności

egzaminator! Uzyskana liczba pkt

1

1

16.3.





{\it 8}

{\it Egzamin maturalny z matematyki}

{\it Arkusz II}

Zadanie 17. $(6pkt)$

Na okręgu o promieniu $r$ opisano trapez równoramienny ABCD o dłuzszej podstawie $AB$

i krótszej $CD$. Punkt styczności $S$ dzieli ramię $BC$ tak, $\displaystyle \dot{\mathrm{z}}\mathrm{e}\frac{|CS|}{|SB|}=\frac{2}{5}.$

a) Wyznacz długość ramienia tego trapezu.

b) Oblicz cosinus $|\wedge CBD|.$
\begin{center}
\includegraphics[width=192.228mm,height=248.412mm]{./F1_M_PR_M2006_page7_images/image001.eps}

\includegraphics[width=151.896mm,height=17.580mm]{./F1_M_PR_M2006_page7_images/image002.eps}
\end{center}
Nr czynno\S ci

Wypelnia Maks. liczba kt

egzamÍnator! Uzyskana liczba pkt

17.1.

1

17.2.

1

17.3.

1

17.4.

1

17.5.

1

1





{\it Egzamin maturalny z matematyki}

{\it Arkusz II}

{\it 9}

Zadanie 18. $(7pkt)$

Wśród wszystkich graniastosłupów prawidłowych trójkątnych o objętości równej 2 $\mathrm{m}^{3}$

istnieje taki, którego pole powierzchni całkowitej jest najmniejsze. Wyznacz długości

krawędzi tego graniastosłupa.
\begin{center}
\includegraphics[width=192.276mm,height=260.508mm]{./F1_M_PR_M2006_page8_images/image001.eps}

\includegraphics[width=165.864mm,height=17.628mm]{./F1_M_PR_M2006_page8_images/image002.eps}
\end{center}
Wypelnia

egzaminator!

Nr czynności

Maks. liczba kt

18.1.

1

18.2.

1

18.3.

18.4.

18.5.

18.6.

18.7.

1

Uzyskana liczba pkt





$ 1\theta$

{\it Egzamin maturalny z matematyki}

{\it Arkusz II}

Zadanie 19. (7pkt)

Nieskończony ciąg

geometryczny

$(a_{n})$

jest

zdefiniowany

wzorem

rekurencyjnym: $a_{1}=2, a_{n+1}=a_{n}\cdot\log_{2}(k-2)$, dla $\mathrm{k}\mathrm{a}\dot{\mathrm{z}}$ dej liczby naturalnej $n\geq 1$. Wszystkie

wyrazy tego ciągu są rózne od zera. Wyznacz wszystkie wartości parametru $k$, dla których

istnieje suma wszystkich wyrazów nieskończonego ciągu $(a_{n}).$
\begin{center}
\includegraphics[width=192.228mm,height=254.460mm]{./F1_M_PR_M2006_page9_images/image001.eps}

\includegraphics[width=151.896mm,height=17.580mm]{./F1_M_PR_M2006_page9_images/image002.eps}
\end{center}
Nr czynności

Wypelnia Maks. liczba kt

egzamÍnator! Uzyskana liczba pkt

1

1

1

1

2

1






\begin{center}
\begin{tabular}{l|l}
\multicolumn{1}{l|}{$\begin{array}{l}\mbox{{\it dysleksja}}	\\	\mbox{Miejsce}	\\	\mbox{na naklejkę}	\\	\mbox{z kodem szkoly}	\end{array}$}&	\multicolumn{1}{|l}{ $\mathrm{M}\mathrm{M}\mathrm{A}-\mathrm{R}1_{-}1\mathrm{P}-072$}	\\
\hline
\multicolumn{1}{l|}{$\begin{array}{l}\mbox{EGZAMIN MATURALNY}	\\	\mbox{Z MATEMATYKI}	\\	\mbox{POZIOM ROZSZERZONY}	\\	\mbox{Czas pracy 180 minut}	\\	\mbox{Instrukcja dla zdającego}	\\	\mbox{1. Sprawdzí, czy arkusz egzaminacyjny zawiera 15 stron}	\\	\mbox{(zadania $1-11$). Ewentualny brak zgłoś przewodniczącemu}	\\	\mbox{zespo nadzo jącego egzamin.}	\\	\mbox{2. Rozwiązania zadań i odpowiedzi zamieść w miejscu na to}	\\	\mbox{przeznaczonym.}	\\	\mbox{3. $\mathrm{W}$ rozwiązaniach zadań przedstaw tok rozumowania}	\\	\mbox{prowadzący do ostatecznego wyniku.}	\\	\mbox{4. Pisz czytelnie. Uzywaj $\mathrm{d}$ gopisu pióra tylko z czatnym}	\\	\mbox{tusze atramentem.}	\\	\mbox{5. Nie uzywaj korektora, a błędne zapisy prze eśl.}	\\	\mbox{6. Pamiętaj, $\dot{\mathrm{z}}\mathrm{e}$ zapisy w brudnopisie nie podlegają ocenie.}	\\	\mbox{7. Obok $\mathrm{k}\mathrm{a}\dot{\mathrm{z}}$ dego zadania podanajest maksymalna liczba punktów,}	\\	\mbox{którą mozesz uzyskać zajego poprawne rozwiązanie.}	\\	\mbox{8. $\mathrm{M}\mathrm{o}\dot{\mathrm{z}}$ esz korzystać z zestawu wzorów matematycznych, cyrkla}	\\	\mbox{i linijki oraz kalkulatora.}	\\	\mbox{9. Wypełnij tę część ka $\mathrm{y}$ odpowiedzi, którą koduje zdający.}	\\	\mbox{Nie wpisuj $\dot{\mathrm{z}}$ adnych znaków w części przeznaczonej dla}	\\	\mbox{egzaminatora.}	\\	\mbox{10. Na karcie odpowiedzi wpisz swoją datę urodzenia i PESEL.}	\\	\mbox{Zamaluj $\blacksquare$ pola odpowiadające cyfrom numeru PESEL. Błędne}	\\	\mbox{zaznaczenie otocz kółkiem $\mathrm{O}$ i zaznacz właściwe.}	\\	\mbox{{\it Zyczymy powodzenia}.'}	\end{array}$}&	\multicolumn{1}{|l}{$\begin{array}{l}\mbox{MAJ}	\\	\mbox{ROK 2007}	\\	\mbox{Za rozwiązanie}	\\	\mbox{wszystkich zadań}	\\	\mbox{mozna otrzymać}	\\	\mbox{łącznie}	\\	\mbox{50 punktów}	\end{array}$}	\\
\hline
\multicolumn{1}{l|}{$\begin{array}{l}\mbox{Wypelnia zdający}	\\	\mbox{rzed roz oczęciem racy}	\\	\mbox{PESEL ZDAJACEGO}	\end{array}$}&	\multicolumn{1}{|l}{$\begin{array}{l}\mbox{KOD}	\\	\mbox{ZDAJACEGO}	\end{array}$}
\end{tabular}


\includegraphics[width=21.840mm,height=9.852mm]{./F1_M_PR_M2007_page0_images/image001.eps}

\includegraphics[width=78.792mm,height=13.356mm]{./F1_M_PR_M2007_page0_images/image002.eps}
\end{center}



{\it 2}

{\it Egzamin maturalny z matematyki}

{\it Poziom rozszerzony}

Zadanie 1. (5pkt)

Danajest funkcja $f(x)=|x-1|-|x+2|$ dla $x\in R.$

a) Wyznacz zbiór wartości funkcji $f$ dla $x\in(-\infty,-2).$

b) Naszkicuj wykres tej funkcji.

c) Podaj jej miejsca zerowe.

d) Wyznacz wszystkie wartości parametru $m$, dla których równanie $f(x)=m$ nie ma

rozwiązania.
\begin{center}
\includegraphics[width=137.868mm,height=17.580mm]{./F1_M_PR_M2007_page1_images/image001.eps}
\end{center}
Nr czynno\S ci

Wypelnia Maks. liczba kt

egzaminator! Uzyskana liczba pkt

1.1.

1

1.2.

1.3.

1

1.4.

1

1.5.

1





{\it Egzamin maturalny z matematyki}

{\it Poziom rozszerzony}

{\it 11}
\begin{center}
\includegraphics[width=109.980mm,height=17.580mm]{./F1_M_PR_M2007_page10_images/image001.eps}
\end{center}
Nr czynności

Wypelnia Maks. liczba kt

egzaminator! Uzyskana liczba pkt

8.1.

1

8.2.

1

8.3.

1





{\it 12}

{\it Egzamin maturalny z matematyki}

{\it Poziom rozszerzony}

Zadanie 9. (3pkt)

Przedstaw wielomian $W(x)=x^{4}-2x^{3}-3x^{2}+4x-1$ w postaci iloczynu dwóch wielomianów

stopnia drugiego o współczynnikach całkowitych i takich, $\dot{\mathrm{z}}\mathrm{e}$ współczynniki przy drugich

potęgach są równe jeden.
\begin{center}
\includegraphics[width=109.932mm,height=17.628mm]{./F1_M_PR_M2007_page11_images/image001.eps}
\end{center}
Wypelnia

egzaminator!

Nr czynności

Maks. liczba kt

1

1

1

Uzyskana liczba pkt





{\it Egzamin maturalny z matematyki}

{\it Poziom rozszerzony}

{\it 13}

Zadanie 10. $(4pkt)$

Na kole opisany jest romb. Stosunek pola koła do pola rombu wynosi $\displaystyle \frac{\pi\sqrt{3}}{8}$. Wyznacz miarę

kąta ostrego rombu.
\begin{center}
\includegraphics[width=123.900mm,height=17.628mm]{./F1_M_PR_M2007_page12_images/image001.eps}
\end{center}
Nr czynności

Wypelnia Maks. liczba kt

egzaminator! Uzyskana liczba pkt

10.2.

1

10.3.

1

10.4.

1





{\it 14}

{\it Egzamin maturalny z matematyki}

{\it Poziom rozszerzony}

Zadanie ll. $(4pkt)$

Suma $n$ początkowych wyrazów ciągu arytmetycznego $(a_{n})$

$S_{n}=2n^{2}+n$ dla $n\geq 1.$

a) Oblicz sumę 50 początkowych wyrazów tego ciągu o

$a_{2}+a_{4}+a_{6}+\ldots+a_{100}.$

b) Oblicz $\displaystyle \lim_{n\rightarrow\infty}\frac{S_{n}}{3n^{2}-2}.$

wyraza się wzorem

numerach parzystych:
\begin{center}
\includegraphics[width=123.900mm,height=17.628mm]{./F1_M_PR_M2007_page13_images/image001.eps}
\end{center}
Wypelnia

egzaminator!

Nr czynności

Maks. liczba kt

11.1.

1

11.2.

1

1

11.4.

1

Uzyskana liczba pkt





{\it Egzamin maturalny z matematyki}

{\it Poziom rozszerzony}

{\it 15}

BRUDNOPIS





{\it Egzamin maturalny z matematyki}

{\it Poziom rozszerzony}

{\it 3}

Zadanie 2. $(5pkt)$

Rozwiąz nierówność: $\log_{\frac{1}{3}}(x^{2}-1)+\log_{\frac{1}{3}}(5-x)>\log_{\frac{1}{3}}(3(x+1)).$
\begin{center}
\includegraphics[width=137.928mm,height=17.580mm]{./F1_M_PR_M2007_page2_images/image001.eps}
\end{center}
Wypelnia

egzaminator!

Nr czynności

Maks. liczba kt

2.1.

1

2.2.

2.3.

1

2.4.

1

2.5.

1

Uzyskana liczba pkt





{\it 4}

{\it Egzamin maturalny z matematyki}

{\it Poziom rozszerzony}

Zadanie 3. $(5pkt)$

Kapsuła lądownika ma kształt stozka zakończonego w podstawie półkulą o tym samym

promieniu co promień podstawy stozka. Wysokość stozka jest o l $\mathrm{m}$ większa $\mathrm{n}\mathrm{i}\dot{\mathrm{z}}$ promień

półkuli. Objętość stozka stanowi $\displaystyle \frac{2}{3}$ objętości całej kapsuły. Oblicz objętość kapsuły

lądownika.
\begin{center}
\includegraphics[width=137.868mm,height=17.580mm]{./F1_M_PR_M2007_page3_images/image001.eps}
\end{center}
Nr czynno\S ci

Wypelnia Maks. liczba kt

egzaminator! Uzyskana liczba pkt

3.1.

1

3.2.

3.3.

1

3.4.

1

3.5.

1





{\it Egzamin maturalny z matematyki}

{\it Poziom rozszerzony}

{\it 5}

Zadanie 4. $(3pkt)$

Dany jest trójkąt o bokach długości l, $\displaystyle \frac{3}{2}$, 2. Oblicz cosinus i sinus kąta lez$\cdot$ącego naprzeciw

najkrótszego boku tego trójkąta.
\begin{center}
\includegraphics[width=109.980mm,height=17.580mm]{./F1_M_PR_M2007_page4_images/image001.eps}
\end{center}
Nr czynności

Wypelnia Maks. liczba kt

egzaminator! Uzyskana liczba pkt

4.1.

1

4.2.

1

4.3.

1





{\it 6}

{\it Egzamin maturalny z matematyki}

{\it Poziom rozszerzony}

Zadanie 5. $(7pkt)$

Wierzchołki trójkąta równobocznego $ABC$ są punktami paraboli $y=-x^{2}+6x$. Punkt $C$ jest

jej wierzchołkiem, a bok $AB$ jest równoległy do osi $\mathrm{O}x$. Sporządzí rysunek w układzie

współrzędnych i wyznacz współrzędne wierzchołków tego trójkąta.
\begin{center}
\includegraphics[width=165.816mm,height=17.628mm]{./F1_M_PR_M2007_page5_images/image001.eps}
\end{center}
Nr czynności

Wypelnia Maks. liczba kt

egzaminator! Uzyskana liczba pkt

5.1.

1

5.2.

1

5.3.

1

5.4.

1

5.5.

1

5.7.

1





{\it Egzamin maturalny z matematyki}

{\it Poziom rozszerzony}

7

Zadanie 6. (4pkt)

Niech $A, B$ będą zdarzeniami o prawdopodobieństwach $P(A) \mathrm{i} P(B)$. Wykaz, $\dot{\mathrm{z}}\mathrm{e}\mathrm{j}\mathrm{e}\dot{\mathrm{z}}$ eli

$P(A)=0,85 \mathrm{i} P(B)=0,75$, to prawdopodobieństwo warunkowe spełnia nierówność

$P(A|B)\geq 0,8.$
\begin{center}
\includegraphics[width=123.900mm,height=17.580mm]{./F1_M_PR_M2007_page6_images/image001.eps}
\end{center}
Nr czynności

Wypelnia Maks. liczba kt

egzamÍnator! Uzyskana liczba pkt

1

1

1





{\it 8}

{\it Egzamin maturalny z matematyki}

{\it Poziom rozszerzony}

Zadanie 7. $(7pkt)$

Dany jest układ równań: 

Dla $\mathrm{k}\mathrm{a}\dot{\mathrm{z}}$ dej wartości parametru $m$ wyznacz parę liczb $(x,y)$, która jest rozwiązaniem tego

układu równań. Wyznacz najmniejszą wartość sumy $x+y$ dla $m\in\langle 2,4\rangle.$





{\it Egzamin maturalny z matematyki}

{\it Poziom rozszerzony}

{\it 9}
\begin{center}
\includegraphics[width=165.864mm,height=17.580mm]{./F1_M_PR_M2007_page8_images/image001.eps}
\end{center}
Nr czynności

Wypelnia Maks. lÍczba kt

egzaminator! Uzyskana liczba pkt

7.1.

1

7.2.

1

7.3.

1

7.4.

7.5.

1

7.7.

1





$ 1\theta$

{\it Egzamin maturalny z matematyki}

{\it Poziom rozszerzony}

Zadanie 8. $(3pkt)$

Danajest funkcja $f$ określona wzorem $f(x)=\displaystyle \frac{\sin^{2}x-|\sin x|}{\sin x}$ dla $x\in(0,\pi)\cup(\pi,2\pi).$

a) Naszkicuj wykres funkcji $f.$

b) Wyznacz miejsca zerowe funkcji $f$







{\it ARKUSZ ZA WIERA INFORMACJE PRA WNIE CHRONIONE}

{\it DO MOMENTU ROZPOCZĘCIA EGZAMINU}.'
\begin{center}
\begin{tabular}{|l|l|l}
\cline{1-1}
\multicolumn{1}{|l|}{$\begin{array}{l}\mbox{Miejsce}	\\	\mbox{na na ejkę}	\end{array}$}&	\multicolumn{1}{|l|}{}&	\multicolumn{1}{|l}{ $\mathrm{M}\mathrm{M}\mathrm{A}-\mathrm{R}1_{-}1\mathrm{P}-082$}	\\
\hline
&	\multicolumn{1}{|l}{$\begin{array}{l}\mbox{MAJ}	\\	\mbox{ROK 2008}	\\	\mbox{Za rozwiązanie}	\\	\mbox{wszystkich zadań}	\\	\mbox{mozna otrzymać}	\\	\mbox{łącznie}	\\	\mbox{50 punktów}	\end{array}$}	\\
\cline{3-3}
&	\multicolumn{1}{|l}{$\begin{array}{l}\mbox{KOD}	\\	\mbox{ZDAJACEGO}	\end{array}$}
\end{tabular}


\includegraphics[width=21.840mm,height=9.852mm]{./F1_M_PR_M2008_page0_images/image001.eps}

\includegraphics[width=78.792mm,height=13.356mm]{./F1_M_PR_M2008_page0_images/image002.eps}
\end{center}



{\it 2 Egzamin maturalny z matematyki}

{\it Poziom rozszerzony}

Zadanie l. $(4pkt)$

Wielomian $f$, którego fragment wykresu przedstawiono na ponizszym rysunku spełnia

warunek $f(0)=90$. Wielomian $g$ dany jest wzorem $g(x)=x^{3}-14x^{2}+63x-90$. Wykaz$\cdot,$

$\dot{\mathrm{z}}\mathrm{e}g(x)=-f(-x)$ dla $x\in R.$
\begin{center}
\includegraphics[width=160.428mm,height=128.724mm]{./F1_M_PR_M2008_page1_images/image001.eps}
\end{center}
{\it y}

{\it f}

1

$-5  -3$  0 1  {\it x}





{\it Egzamin maturalny z matematyki ll}

{\it Poziom rozszerzony}
\begin{center}
\includegraphics[width=195.168mm,height=284.688mm]{./F1_M_PR_M2008_page10_images/image001.eps}

\includegraphics[width=123.900mm,height=17.832mm]{./F1_M_PR_M2008_page10_images/image002.eps}
\end{center}
Nr zadania

Wypelnia Maks. liczba kt

egzaminator! Uzyskana liczba pkt

8.1

1

8.2

1

8.3

1

8.4

1





{\it 12 Egzamin maturalny z matematyki}

{\it Poziom rozszerzony}

Zadanie 9. $(4pkt)$

Wyznacz dziedzinę i najmniejszą wartość funkcji $f(x)=\log_{\frac{\sqrt{2}}{2}}(8x-x^{2}).$
\begin{center}
\includegraphics[width=195.228mm,height=260.508mm]{./F1_M_PR_M2008_page11_images/image001.eps}

\includegraphics[width=123.948mm,height=17.832mm]{./F1_M_PR_M2008_page11_images/image002.eps}
\end{center}
Wypelnia

egzamÍnator!

Nr zadania

Maks. liczba kt

1

1

1

1

Uzyskana liczba pkt





{\it Egzamin maturalny z matematyki 13}

{\it Poziom rozszerzony}

Zadanie 10. $(4pkt)$

$\mathrm{Z}$ pewnej grupy osób, w której jest dwa razy więcej męzczyzn $\mathrm{n}\mathrm{i}\dot{\mathrm{z}}$ kobiet, wybrano losowo

dwuosobową delegację. Prawdopodobieństwo tego, $\dot{\mathrm{z}}\mathrm{e}$ w delegacji znajdą się tylko kobiety

jest równe 0,1. Ob1icz, i1e kobiet i i1u męzczyzn jest w tej grupie.
\begin{center}
\includegraphics[width=195.168mm,height=254.412mm]{./F1_M_PR_M2008_page12_images/image001.eps}

\includegraphics[width=123.900mm,height=17.784mm]{./F1_M_PR_M2008_page12_images/image002.eps}
\end{center}
Nr zadania

Wypelnia Maks. liczba kt

egzaminator! Uzyskana liczba pkt

10.1

1

10.2

1

10.3

1

10.4

1





{\it 14 Egzamin maturalny z matematyki}

{\it Poziom rozszerzony}

Zadanie ll. $(5pkt)$

$\mathrm{W}$ ostrosłupie prawidłowym czworokątnym dane są: $H$ -wysokość ostrosłupa oraz

$\alpha-$ miara kąta utworzonego przez krawędz$\acute{}$ boczną i krawędz$\acute{}$ podstawy $(45^{\circ}<\alpha<90^{\circ}).$

a) Wykaz, $\dot{\mathrm{z}}\mathrm{e}$ objętość $V$ tego ostrosłupajest równa $\displaystyle \frac{4}{3}.\frac{H^{3}}{\mathrm{t}\mathrm{g}^{2}\alpha-1}.$

b) Oblicz miarę kąta $\alpha$, dla której objętość $V$ danego ostrosłupajest równa $\displaystyle \frac{2}{9}H^{3}$ Wynik

podaj w zaokrągleniu do całkowitej liczby stopni.
\begin{center}
\includegraphics[width=195.228mm,height=103.128mm]{./F1_M_PR_M2008_page13_images/image001.eps}
\end{center}




{\it Egzamin maturalny z matematyki 15}

{\it Poziom rozszerzony}
\begin{center}
\includegraphics[width=195.168mm,height=284.688mm]{./F1_M_PR_M2008_page14_images/image001.eps}

\includegraphics[width=137.928mm,height=17.832mm]{./F1_M_PR_M2008_page14_images/image002.eps}
\end{center}
Wypelnia

egzaminator!

Nr zadania

Maks. liczba kt

1

11.2

1

11.3

1

11.4

1

11.5

Uzyskana liczba pkt





{\it 16 Egzamin maturalny z matematyki}

{\it Poziom rozszerzony}

Zadanie 12. $(4pkt)$

$\mathrm{W}$ trójkącie prostokątnym $ABC$ przyprostokątne mają długości: $|BC|=9, |CA|=12$. Na boku

$AB$ wybrano punkt $D$ tak, $\dot{\mathrm{z}}\mathrm{e}$ odcinki $BC \mathrm{i}$ CD mają równe długości. Oblicz długość

odcinka $AD.$
\begin{center}
\includegraphics[width=195.228mm,height=266.544mm]{./F1_M_PR_M2008_page15_images/image001.eps}
\end{center}




{\it Egzamin maturalny z matematyki 17}

{\it Poziom rozszerzony}
\begin{center}
\includegraphics[width=195.168mm,height=284.688mm]{./F1_M_PR_M2008_page16_images/image001.eps}

\includegraphics[width=123.900mm,height=17.832mm]{./F1_M_PR_M2008_page16_images/image002.eps}
\end{center}
Nr zadania

Wypelnia Maks. liczba kt

egzaminator! Uzyskana liczba pkt

12.1

1

12.2

1

12.3

1

1





{\it 18 Egzamin maturalny z matematyki}

{\it Poziom rozszerzony}

BRUDNOPIS





{\it Egzamin maturalny z matematyki 3}

{\it Poziom rozszerzony}
\begin{center}
\includegraphics[width=123.900mm,height=17.832mm]{./F1_M_PR_M2008_page2_images/image001.eps}
\end{center}
Nr zadania

Wypelnia Maks. liczba kt

egzaminator! Uzyskana liczba pkt

1.1

1

1.2

1

1.3

1

1.4

1





{\it 4 Egzamin maturalny z matematyki}

{\it Poziom rozszerzony}

Zadanie 2. $(4pkt)$

Rozwiąz nierówność $|x-2|+|3x-6|<|x|.$
\begin{center}
\includegraphics[width=195.228mm,height=266.544mm]{./F1_M_PR_M2008_page3_images/image001.eps}

\includegraphics[width=123.948mm,height=17.784mm]{./F1_M_PR_M2008_page3_images/image002.eps}
\end{center}
Nr zadania

Wypelnia Maks. liczba kt

egzaminator! Uzyskana lÍczba pkt

2.1

1

2.2

1

2.3

1

2.4

1





{\it Egzamin maturalny z matematyki 5}

{\it Poziom rozszerzony}

Zadanie 3. $(5pkt)$

Liczby $x_{1}=5+\sqrt{23}\mathrm{i}x_{2}=5-\sqrt{23}$ sąrozwiązaniami równania $x^{2}-(p^{2}+q^{2})x+(p+q)=0$

z niewiadomą $x$. Oblicz wartości $p \mathrm{i}q.$
\begin{center}
\includegraphics[width=195.168mm,height=260.508mm]{./F1_M_PR_M2008_page4_images/image001.eps}

\includegraphics[width=137.928mm,height=17.832mm]{./F1_M_PR_M2008_page4_images/image002.eps}
\end{center}
Wypelnia

egzaminator!

Nr zadania

Maks. liczba kt

3.1

1

3.2

1

3.3

1

3.4

1

3.5

Uzyskana liczba pkt





{\it 6 Egzamin maturalny z matematyki}

{\it Poziom rozszerzony}

Zadanie 4. $(4pkt)$

Rozwiąz równanie 4 $\cos^{2}x=4\sin x+1$ w przedziale $\langle 0,2\pi\rangle.$
\begin{center}
\includegraphics[width=195.228mm,height=266.544mm]{./F1_M_PR_M2008_page5_images/image001.eps}

\includegraphics[width=123.948mm,height=17.784mm]{./F1_M_PR_M2008_page5_images/image002.eps}
\end{center}
Nr zadania

Wypelnia Maks. liczba kt

egzaminator! Uzyskana lÍczba pkt

4.1

1

4.2

1

4.3

1

4.4

1





{\it Egzamin maturalny z matematyki 7}

{\it Poziom rozszerzony}

Zadanie 5. $(5pkt)$

Dane jest równanie $|\displaystyle \frac{2}{x}+3|=p$ z niewiadomą $x.$

w zalezności od parametru $p.$

Wyznacz liczbę rozwiązań tego równania
\begin{center}
\includegraphics[width=195.168mm,height=254.460mm]{./F1_M_PR_M2008_page6_images/image001.eps}

\includegraphics[width=137.928mm,height=17.784mm]{./F1_M_PR_M2008_page6_images/image002.eps}
\end{center}
Wypelnia

egzaminator!

Nr zadanÍa

Maks. liczba kt

1

5.2

1

5.3

5.4

5.5

1

Uzyskana lÍczba pkt





{\it 8 Egzamin maturalny z matematyki}

{\it Poziom rozszerzony}

Zadanie 6. $(3pkt)$

Udowodnij, $\dot{\mathrm{z}}\mathrm{e} \mathrm{j}\mathrm{e}\dot{\mathrm{z}}$ eli

to $a=b=c.$

Ciąg

(a, b, c) jest jednocześnie

arytmetyczny

i

geometryczny,
\begin{center}
\includegraphics[width=195.228mm,height=260.508mm]{./F1_M_PR_M2008_page7_images/image001.eps}

\includegraphics[width=109.932mm,height=17.832mm]{./F1_M_PR_M2008_page7_images/image002.eps}
\end{center}
Wypelnia

egzaminator!

Nr zadania

Maks. liczba kt

1

1

1

Uzyskana liczba pkt





{\it Egzamin maturalny z matematyki 9}

{\it Poziom rozszerzony}

Zadanie 7. $(4pkt)$

Uzasadnij, $\dot{\mathrm{z}}\mathrm{e}\mathrm{k}\mathrm{a}\dot{\mathrm{z}}\mathrm{d}\mathrm{y}$ punkt paraboli o równaniu $y=\displaystyle \frac{1}{4}x^{2}+1$ jest równoodległy od osi

punktu $F=(0,2).$

Ox i od
\begin{center}
\includegraphics[width=195.168mm,height=254.460mm]{./F1_M_PR_M2008_page8_images/image001.eps}

\includegraphics[width=123.900mm,height=17.784mm]{./F1_M_PR_M2008_page8_images/image002.eps}
\end{center}
Nr zadania

Wypelnia Maks. liczba kt

egzamÍnator! Uzyskana liczba pkt

7.1

1

7.2

1

7.3

1

7.4

1





$ 1\theta$ {\it Egzamin maturalny z matematyki}

{\it Poziom rozszerzony}

Zadanie 8. $(4pkt)$

Wyznacz współrzędne środka jednokładności, w której obrazem okręgu o równaniu

$(x-16)^{2}+y^{2}=4$ jest okrąg o równaniu $(x-6)^{2}+(y-4)^{2}=16$, a skala tej jednokładności

jest liczbą ujemną.
\begin{center}
\includegraphics[width=195.228mm,height=260.556mm]{./F1_M_PR_M2008_page9_images/image001.eps}
\end{center}






{\it ARKUSZ ZA WIERA INFORMACJE} $PRA$ {\it WNIE CHRONIONE}

{\it DO MOMENTU ROZPOCZĘCIA EGZAMINU}.$\displaystyle \int$
\begin{center}
\includegraphics[width=192.024mm,height=288.084mm]{./F1_M_PR_M2009_page0_images/image001.eps}
\end{center}
Miejsce

na na ejkę

EGZAMIN MATURALNY

Z MATEMATYKI

MAJ

POZIOM ROZSZERZONY

Czas pracy 180 minut

Instrukcja dla zdającego

1.

2.

3.

4.

5.

6.

7.

8.

9.

Sprawd $\acute{\mathrm{z}}$, czy arkusz egzaminacyjny zawiera 16 stron

(zadania $1-11$). Ewentualny brak zgłoś przewodniczącemu

zespo nadzorującego egzamin.

Rozwiązania zadań i odpowiedzi zamieść w miejscu na to

przeznaczonym.

W rozwiązaniach zadań przedstaw tok rozumowania

prowadzący do ostatecznego wyniku.

Pisz czytelnie. Uzywaj $\mathrm{d}$ gopisu pióra tylko z czatnym

tusze atramentem.

Nie uzywaj korektora, a błędne zapisy prze eśl.

Pamiętaj, $\dot{\mathrm{z}}\mathrm{e}$ zapisy w brudnopisie nie podlegają ocenie.

Obok $\mathrm{k}\mathrm{a}\dot{\mathrm{z}}$ dego zadania podanajest maksymalna liczba punktów,

którą $\mathrm{m}\mathrm{o}\dot{\mathrm{z}}$ esz uzyskać zajego poprawne rozwiązanie.

$\mathrm{M}\mathrm{o}\dot{\mathrm{z}}$ esz korzystać z zestawu wzorów matematycznych, cyrkla

i linijki oraz kalkulatora.

Na karcie odpowiedzi wpisz swoją datę urodzenia i PESEL.

Nie wpisuj $\dot{\mathrm{z}}$ adnych znaków w części przeznaczonej dla

egzaminatora.

Za rozwiązanie

wszystkich zadań

mozna otrzymać

łącznie

50 punktów

{\it Zyczymy powodzenia}.'

Wypelnia zdający

rzed roz oczęciem racy

PESEL ZDAJACEGO

KOD

ZDAJACEGO




{\it 2}

{\it Egzamin maturalny z matematyki}

{\it Poziom rozszerzony}

Zadanie l. $(4pkt)$

Funkcja liniowa $f$ określona jest wzorem $f(x)=ax+b$ dla $x\in R.$

a) Dla $a=2008\mathrm{i}b=2009$ zbadaj, czy do wykresu tej ffinkcji nalezypunkt $P=(2009,2009^{2}).$

b) Narysuj w układzie współrzędnych zbiór

$A=\displaystyle \{(x,y):x\in\langle-1,3\rangle\mathrm{i}y=-\frac{1}{2}x+b\mathrm{i}b\in\langle-2,1\rangle\}.$
\begin{center}
\includegraphics[width=123.900mm,height=17.628mm]{./F1_M_PR_M2009_page1_images/image001.eps}
\end{center}
Wypelnia

egzaminator!

Nr czynności

Maks. liczba kt

1

1.2.

1.3.

1.4.

1

Uzyskana liczba pkt





{\it Egzamin maturalny z matematyki}

{\it Poziom rozszerzony}

{\it 11}
\begin{center}
\includegraphics[width=123.900mm,height=17.628mm]{./F1_M_PR_M2009_page10_images/image001.eps}
\end{center}
Nr czynnoŚci

Wypelnia Maks. liczba kt

egzaminator! Uzyskana lÍczba pkt

8.1.

1

8.2.

1

8.3.

1

8.4.

1





{\it 12}

{\it Egzamin maturalny z matematyki}

{\it Poziom rozszerzony}

Zadanie 9. $(5pkt)$

$\mathrm{W}$ układzie współrzędnych narysuj okrąg o równaniu $(x+2)^{2}+(y-3)^{2}=4$ oraz zaznacz

punkt $A=(0,-1)$. Prosta o równaniu $x=0$ jest jedną ze stycznych do tego okręgu

przechodzących przez punkt $A$. Wyznacz równanie drugiej stycznej do tego okręgu,

przechodzącej przez punkt $A.$
\begin{center}
\includegraphics[width=137.868mm,height=17.580mm]{./F1_M_PR_M2009_page11_images/image001.eps}
\end{center}
Nr czynnoŚci

Wypelnia Maks. liczba kt

egzaminator! Uzyskana lÍczba pkt

1

1

1





{\it Egzamin maturalny z matematyki}

{\it Poziom rozszerzony}

{\it 13}

Zadanie 10. $(4pkt)$

$\mathrm{W}$ urnie znajdują się jedynie kule białe i czarne. Kul białych jest trzy razy więcej

$\mathrm{n}\mathrm{i}\dot{\mathrm{z}}$ czarnych. Oblicz, ile jest kul w umie, jeśli przy jednoczesnym losowaniu dwóch kul

prawdopodobieństwo otrzymania kul o róznych kolorachjest większe od $\displaystyle \frac{9}{22}$
\begin{center}
\includegraphics[width=123.900mm,height=17.628mm]{./F1_M_PR_M2009_page12_images/image001.eps}
\end{center}
Nr czynnoŚci

Wypelnia Maks. liczba kt

egzaminator! Uzyskana lÍczba pkt

10.1.

1

10.2.

10.3.

10.4.

1





{\it 14}

{\it Egzamin maturalny z matematyki}

{\it Poziom rozszerzony}

Zadanie 11. (6pkt)

Dany jest ostrosłup prawidłowy trójkątny, w którym krawędzí podstawy ma długość a

i krawędzí bocznajest od niej dwa razy dłuzsza. Oblicz cosinus kąta między krawędzią boczną

i krawędzią podstawy ostrosłupa. Narysuj przekrój ostrosłupa płaszczyzną przechodzącą

przez krawędzí podstawy i środek przeciwległej krawędzi bocznej i oblicz pole tego przekroju.





{\it Egzamin maturalny z matematyki}

{\it Poziom rozszerzony}

{\it 15}
\begin{center}
\includegraphics[width=151.788mm,height=17.580mm]{./F1_M_PR_M2009_page14_images/image001.eps}
\end{center}
Wypelnia

egzamÍnator!

Nr czynnoścÍ

Maks. liczba kt

11.1.

1

11.2.

1

11.3.

1

11.4.

11.5.

1

1

Uzyskana liczba pkt





{\it 16}

{\it Egzamin maturalny z matematyki}

{\it Poziom rozszerzony}

BRUDNOPIS





{\it Egzamin maturalny z matematyki}

{\it Poziom rozszerzony}

{\it 3}

Zadanie 2. $(4pkt)$

Przy dzieleniu wielomianu $W(x)$ przez dwumian $(x-1)$ otrzymujemy

$Q(x)=8x^{2}+4x-14$ oraz resztę $R(x)=-5$. Oblicz pierwiastki wielomianu $W(x).$

iloraz
\begin{center}
\includegraphics[width=123.900mm,height=17.580mm]{./F1_M_PR_M2009_page2_images/image001.eps}
\end{center}
Nr czynnoŚci

Wypelnia Maks. liczba kt

egzamÍnator! Uzyskana lÍczba pkt

2.1.

1

2.2.

2.3.

2.4.

1





{\it 4}

{\it Egzamin maturalny z matematyki}

{\it Poziom rozszerzony}

Zadanie 3. $(4pkt)$

Na rysunku przedstawiony jest wykres funkcji wykładniczej $f(x)=a^{x}$ dla $x\in R.$
\begin{center}
\includegraphics[width=112.272mm,height=113.388mm]{./F1_M_PR_M2009_page3_images/image001.eps}
\end{center}
$\gamma$

$5$

4

3

2

$-4 -3  -2 -1$  0 1

$111$

$1$

$1$

$1$

$1$

$1$

$1$

$1$

$\rangle$

$1$

$1$

$111$

2

3 4 x

$-1$

$-2$

$-3$

a) Oblicz $a.$

b) Narysuj wykres funkcji $g(x)=|f(x)-2|$ i podaj wszystkie wartości parametru $m\in R,$

dla których równanie $g(x)=m$ ma dokładniejedno rozwiązanie.





{\it Egzamin maturalny z matematyki}

{\it Poziom rozszerzony}

{\it 5}
\begin{center}
\includegraphics[width=123.900mm,height=17.580mm]{./F1_M_PR_M2009_page4_images/image001.eps}
\end{center}
Nr czynności

Wypelnia Maks. liczba kt

egzamÍnator! Uzyskana liczba pkt

3.1.

1

3.2.

3.3.

3.4.

1





{\it 6}

{\it Egzamin maturalny z matematyki}

{\it Poziom rozszerzony}

Zadanie 4. $(5pkt)$

$\mathrm{W}$ skarbcu królewskim było $k$ monet. Pierwszego dnia rano skarbnik dorzucił 25 monet,

a $\mathrm{k}\mathrm{a}\dot{\mathrm{z}}$ dego następnego ranka dorzucał o 2 monety więcej $\mathrm{n}\mathrm{i}\dot{\mathrm{z}}$ dnia poprzedniego. Jednocześnie

ze skarbca król zabierał w południe $\mathrm{k}\mathrm{a}\dot{\mathrm{z}}$ dego dnia 50 monet. Ob1icz najmniejszą 1iczbę $k$, dla

której w kazdym dniu w skarbcu była co najmniej jedna moneta, a następnie dla tej wartości $k$

oblicz, w którym dniu w skarbcu była najmniejsza liczba monet.
\begin{center}
\includegraphics[width=137.868mm,height=17.628mm]{./F1_M_PR_M2009_page5_images/image001.eps}
\end{center}
Nr czynnoŚci

Wypelnia Maks. liczba kt

egzaminator! Uzyskana lÍczba pkt

4.1.

1

4.2.

1

4.3.

1

4.4.

1

4.5.





{\it Egzamin maturalny z matematyki}

{\it Poziom rozszerzony}

7

Zadanie 5. $(3pkt)$

Wykaz, $\dot{\mathrm{z}}\mathrm{e}\mathrm{j}\mathrm{e}\dot{\mathrm{z}}$ eli $A=3^{4\sqrt{2}+2}$

$\mathrm{i} B=3^{2\sqrt{2}+3}$, to $B=9\sqrt{A}.$
\begin{center}
\includegraphics[width=109.980mm,height=17.580mm]{./F1_M_PR_M2009_page6_images/image001.eps}
\end{center}
Nr czynności

Wypelnia Maks. liczba kt

egzaminator! Uzyskana liczba pkt

5.1.

1

5.2.

1

5.3.

1





{\it 8}

{\it Egzamin maturalny z matematyki}

{\it Poziom rozszerzony}

Zadanie 6. $(5pkt)$

Wyznacz dziedzinę funkcji $f(x)=\log_{2\cos x}(9-x^{2})$ i zapisz ją w postaci sumy przedziałów

liczbowych.
\begin{center}
\includegraphics[width=137.868mm,height=17.580mm]{./F1_M_PR_M2009_page7_images/image001.eps}
\end{center}
Nr czynności

Wypelnia Maks. liczba kt

egzaminator! Uzyskana liczba pkt

1

1

1

1





{\it Egzamin maturalny z matematyki}

{\it Poziom rozszerzony}

{\it 9}

Zadanie 7. $(6pkt)$

Ciąg $(x-3,x+3,6x+2,\ldots)$

jest nieskończonym ciągiem geometrycznym o wyrazach

dodatnich. Oblicz iloraz tego ciągu i uzasadnij,

n początkowych wyrazów tego ciągu.

$\dot{\mathrm{z}}\mathrm{e} \displaystyle \frac{S_{19}}{S_{20}}<\frac{1}{4}$, gdzie $S_{n}$ oznacza sumę
\begin{center}
\includegraphics[width=151.788mm,height=17.628mm]{./F1_M_PR_M2009_page8_images/image001.eps}
\end{center}
Wypelnia

egzaminator!

Nr czynnoŚci

Maks. liczba kt

7.1.

1

7.2.

1

7.3.

1

7.4.

7.5.

1

1

Uzyskana lÍczba pkt





$ 1\theta$

{\it Egzamin maturalny z matematyki}

{\it Poziom rozszerzony}

Zadanie 8. $(4pkt)$

Dwa okręgi o środkach $A\mathrm{i}B$ są styczne zewnętrznie i $\mathrm{k}\mathrm{a}\dot{\mathrm{z}}\mathrm{d}\mathrm{y}$ z nichjestjednocześnie styczny

do ramion tego samego kąta prostego (patrz rysunek). Udowodnij, $\dot{\mathrm{z}}\mathrm{e}$ stosunek promienia

większego z tych okręgów do promienia mniejszegojest równy $3+2\sqrt{2}.$

{\it B}.

{\it A}.







Centralna Komisja Egzaminacyjna

Arkusz zawiera informacje prawnie chronione do momentu rozpoczęcia egzaminu.

WPISUJE ZDAJACY

KOD PESEL

{\it Miejsce}

{\it na naklejkę}

{\it z kodem}
\begin{center}
\includegraphics[width=21.432mm,height=9.852mm]{./F1_M_PR_M2010_page0_images/image001.eps}

\includegraphics[width=82.092mm,height=9.852mm]{./F1_M_PR_M2010_page0_images/image002.eps}

\includegraphics[width=204.060mm,height=216.048mm]{./F1_M_PR_M2010_page0_images/image003.eps}
\end{center}
EGZAMIN MATU  LNY

Z MATEMATY  MAJ 2010

POZIOM ROZSZERZONY

1.

Czas pracy:

180 minut

3.

Sprawdzí, czy arkusz egzaminacyjny zawiera 24 strony

(zadania $1-11$). Ewentualny brak zgłoś

przewodniczącemu zespo nadzo jącego egzamin.

Rozwiązania zadań i odpowiedzi wpisuj w miejscu na to

przeznaczonym.

Pamiętaj, $\dot{\mathrm{z}}\mathrm{e}$ pominięcie argumentacji lub istotnych

obliczeń w rozwiązaniu zadania otwa ego $\mathrm{m}\mathrm{o}\dot{\mathrm{z}}\mathrm{e}$

spowodować, $\dot{\mathrm{z}}\mathrm{e}$ za to rozwiązanie nie będziesz mógł

dostać pełnej liczby punktów.

Pisz czytelnie i $\mathrm{u}\dot{\mathrm{z}}$ aj tvlko $\mathrm{d}$ gopisu lub -Dióra

z czatnym tuszem lub atramentem.

Nie uzywaj korektora, a błędne zapisy wyrazínie prze eśl.

Pamiętaj, $\dot{\mathrm{z}}\mathrm{e}$ zapisy w brudnopisie nie będą oceniane.

$\mathrm{M}\mathrm{o}\dot{\mathrm{z}}$ esz korzystać z zestawu wzorów matematycznych,

cyrkla i linijki oraz kalkulatora.

Na karcie odpowiedzi wpisz swój numer PESEL i przyklej

naklejkę z kodem.

Nie wpisuj $\dot{\mathrm{z}}$ adnych znaków w części przeznaczonej dla

egzaminatora.

2.

4.

5.

6.

7.

8.

9.

Liczba punktów

do uzyskania: 50

$\Vert\Vert\Vert\Vert\Vert\Vert\Vert\Vert\Vert\Vert\Vert\Vert\Vert\Vert\Vert\Vert\Vert\Vert\Vert\Vert\Vert\Vert\Vert\Vert|  \mathrm{M}\mathrm{M}\mathrm{A}-\mathrm{R}1_{-}1\mathrm{P}-102$




{\it 2}

{\it Egzamin maturalny z matematyki}

{\it Poziom rozszerzony}

Zadanie l. $(4pkt)$

Rozwiąz nierówność $|2x+4|+|x-1|\leq 6.$





{\it Egzamin maturalny z matematyki}

{\it Poziom rozszerzony}

{\it 11}
\begin{center}
\includegraphics[width=82.044mm,height=17.832mm]{./F1_M_PR_M2010_page10_images/image001.eps}
\end{center}
Wypelnia

egzaminator

Nr zadania

Maks. liczba kt

5.

5

Uzyskana liczba pkt





{\it 12}

{\it Egzamin maturalny z matematyki}

{\it Poziom rozszerzony}

Zadanie 6. $(5pkt)$

Wyznacz wszystkie wartości parametru $m$, dla których równanie $x^{2}+mx+2=0$ ma dwa

rózne pierwiastki rzeczywiste takie, $\dot{\mathrm{z}}\mathrm{e}$ suma ich kwadratówjest większa od $2m^{2}-13.$





{\it Egzamin maturalny z matematyki}

{\it Poziom rozszerzony}

{\it 13}
\begin{center}
\includegraphics[width=82.044mm,height=17.832mm]{./F1_M_PR_M2010_page12_images/image001.eps}
\end{center}
Wypelnia

egzaminator

Nr zadania

Maks. liczba kt

5

Uzyskana liczba pkt





{\it 14}

{\it Egzamin maturalny z matematyki}

{\it Poziom rozszerzony}

Zadanie 7. $(6pkt)$

Punkt $A=(-2,5)$ jest jednym z wierzchołków trójkąta równoramiennego $ABC$, w którym

$|AC|=|BC|$. Pole tego trójkąta jest równe 15. Bok $BC$ jest zawarty w prostej o równaniu

$y=x+1$. Oblicz współrzędne wierzchołka $C.$





{\it Egzamin maturalny z matematyki}

{\it Poziom rozszerzony}

{\it 15}
\begin{center}
\includegraphics[width=82.044mm,height=17.832mm]{./F1_M_PR_M2010_page14_images/image001.eps}
\end{center}
Wypelnia

egzaminator

Nr zadania

Maks. liczba kt

7.

Uzyskana liczba pkt





{\it 16}

{\it Egzamin maturalny z matematyki}

{\it Poziom rozszerzony}

Zadanie 8. (5pkt)

Rysunek przedstawia fragment

wykresu funkcji

$f(x)=\displaystyle \frac{1}{x^{2}}.$

Przeprowadzono prostą

równoległą do osi $Ox$, która przecięła wykres tej funkcji w punktach $A$

$C=(3,-1)$. Wykaz, $\dot{\mathrm{z}}\mathrm{e}$ pole trójkąta ABCjest większe lub równe 2.

i B. Niech
\begin{center}
\includegraphics[width=161.748mm,height=105.252mm]{./F1_M_PR_M2010_page15_images/image001.eps}
\end{center}




{\it Egzamin maturalny z matematyki}

{\it Poziom rozszerzony}

17
\begin{center}
\includegraphics[width=82.044mm,height=17.832mm]{./F1_M_PR_M2010_page16_images/image001.eps}
\end{center}
Wypelnia

egzamÍnator

Nr zadania

Maks. liczba kt

8.

5

Uzyskana liczba pkt





{\it 18}

{\it Egzamin maturalny z matematyki}

{\it Poziom rozszerzony}

Zadanie 9. $(4pkt)$

Na bokach $BC\mathrm{i}$ CD równoległoboku ABCD zbudowano kwadraty CDEF $\mathrm{i}$ {\it BCGH} ({\it zobacz}

rysunek). Udowodnij, $\dot{\mathrm{z}}\mathrm{e}|AC|=|FG|.$
\begin{center}
\includegraphics[width=96.216mm,height=93.324mm]{./F1_M_PR_M2010_page17_images/image001.eps}
\end{center}
{\it E} $F$

{\it D  C}

{\it G}

{\it A  B}

{\it H}





{\it Egzamin maturalny z matematyki}

{\it Poziom rozszerzony}

{\it 19}
\begin{center}
\includegraphics[width=82.044mm,height=17.832mm]{./F1_M_PR_M2010_page18_images/image001.eps}
\end{center}
Wypelnia

egzaminator

Nr zadania

Maks. liczba kt

4

Uzyskana liczba pkt





$ 2\theta$

{\it Egzamin maturalny z matematyki}

{\it Poziom rozszerzony}

Zadanie 10. (4pkt)

Oblicz prawdopodobieństwo tego, ze w trzech rzutach symetryczną sześcienną kostką do gry suma

kwadratów liczb uzyskanych oczek bę\& ie podzielna przez 3.





{\it Egzamin maturalny z matematyki}

{\it Poziom rozszerzony}

{\it 3}
\begin{center}
\includegraphics[width=82.044mm,height=17.832mm]{./F1_M_PR_M2010_page2_images/image001.eps}
\end{center}
Wypelnia

egzamÍnator

Nr zadania

Maks. liczba kt

1.

4

Uzyskana liczba pkt





{\it Egzamin maturalny z matematyki}

{\it Poziom rozszerzony}

{\it 21}
\begin{center}
\includegraphics[width=82.044mm,height=17.832mm]{./F1_M_PR_M2010_page20_images/image001.eps}
\end{center}
Wypelnia

egzaminator

Nr zadania

Maks. liczba kt

10.

4

Uzyskana liczba pkt





{\it 22}

{\it Egzamin maturalny z matematyki}

{\it Poziom rozszerzony}

Zadanie ll. $(5pkt)$

$\mathrm{W}$ ostrosłupie prawidłowym trójkątnym krawędzí podstawy ma długość $a$. Ściany boczne są

trójkątami ostrokątnymi. Miara kąta między sąsiednimi ścianami bocznymi jest równa $2\alpha.$

Wyznacz objętość tego ostrosłupa.





{\it Egzamin maturalny z matematyki}

{\it Poziom rozszerzony}

{\it 23}
\begin{center}
\includegraphics[width=82.044mm,height=17.832mm]{./F1_M_PR_M2010_page22_images/image001.eps}
\end{center}
Wypelnia

egzaminator

Nr zadania

Maks. liczba kt

11.

5

Uzyskana liczba pkt





{\it 24}

{\it Egzamin maturalny z matematyki}

{\it Poziom rozszerzony}

BRUDNOPIS





{\it 4}

{\it Egzamin maturalny z matematyki}

{\it Poziom rozszerzony}

Zadanie 2. $(4pkt)$

Wyznacz wszystkie rozwiązania równania 2 $\cos^{2}x-5\sin x-4=0$

$\langle 0, 2\pi\rangle.$

nalezące do przedziału





{\it Egzamin maturalny z matematyki}

{\it Poziom rozszerzony}

{\it 5}
\begin{center}
\includegraphics[width=82.044mm,height=17.832mm]{./F1_M_PR_M2010_page4_images/image001.eps}
\end{center}
Wypelnia

egzaminator

Nr zadania

Maks. liczba kt

2.

4

Uzyskana liczba pkt





{\it 6}

{\it Egzamin maturalny z matematyki}

{\it Poziom rozszerzony}

Zadanie 3. $(4pkt)$

Bok kwadratu ABCD ma długość l. Na bokach $BC\mathrm{i}$ CD wybrano odpowiednio punkty $E\mathrm{i}F$

umieszczone tak, by $|CE|=2|DF|$. Oblicz wartość $x=|DF|$, dla której pole trójkąta $AEF$

jest najmniejsze.





{\it Egzamin maturalny z matematyki}

{\it Poziom rozszerzony}

7
\begin{center}
\includegraphics[width=82.044mm,height=17.832mm]{./F1_M_PR_M2010_page6_images/image001.eps}
\end{center}
Wypelnia

egzaminator

Nr zadania

Maks. liczba kt

3.

4

Uzyskana liczba pkt





{\it 8}

{\it Egzamin maturalny z matematyki}

{\it Poziom rozszerzony}

Zadanie 4. $(4pkt)$

Wyznacz wartości $a\mathrm{i}b$ współczynników wielomianu $W(x)=x^{3}+ax^{2}+bx+1$

wiedząc, $\dot{\mathrm{z}}\mathrm{e}$

$W(2)=7$ oraz, $\dot{\mathrm{z}}\mathrm{e}$ reszta z dzielenia $W(x)$ przez $(x-3)$ jest równa 10.





{\it Egzamin maturalny z matematyki}

{\it Poziom rozszerzony}

{\it 9}
\begin{center}
\includegraphics[width=82.044mm,height=17.832mm]{./F1_M_PR_M2010_page8_images/image001.eps}
\end{center}
Wypelnia

egzaminator

Nr zadania

Maks. liczba kt

4.

4

Uzyskana liczba pkt





$ 1\theta$

{\it Egzamin maturalny z matematyki}

{\it Poziom rozszerzony}

Zadanie 5. $(5pkt)$

$\mathrm{O}$ liczbach $a, b, c$ wiemy, $\dot{\mathrm{z}}\mathrm{e}$ ciąg $(a,b,c)$ jest arytmetyczny

$(a+1,b+4,c+19)$ jest geometryczny. Wyznacz te liczby.

i

$a+c=10$, zaś ciąg







Centralna Komisja Egzaminacyjna

Arkusz zawiera informacje prawnie chronione do momentu rozpoczęcia egzaminu.

WPISUJE ZDAJACY

KOD PESEL

{\it Miejsce}

{\it na naklejkę}

{\it z kodem}
\begin{center}
\includegraphics[width=21.432mm,height=9.804mm]{./F1_M_PR_M2011_page0_images/image001.eps}

\includegraphics[width=82.092mm,height=9.804mm]{./F1_M_PR_M2011_page0_images/image002.eps}

\includegraphics[width=204.060mm,height=216.048mm]{./F1_M_PR_M2011_page0_images/image003.eps}
\end{center}
EGZAMIN MATU

Z MATEMATY

LNY

MAJ 2011

POZIOM ROZSZE ONY

1.

Czas pracy:

180 minut

3.

Sprawd $\acute{\mathrm{z}}$, czy arkusz egzaminacyjny zawiera 19 stron

(zadania $1-12$). Ewentualny brak zgłoś

przewodniczącemu zespo nadzorującego egzamin.

Rozwiązania zadań i odpowiedzi wpisuj w miejscu na to

przeznaczonym.

Pamiętaj, $\dot{\mathrm{z}}\mathrm{e}$ pominięcie argumentacji lub istotnych

obliczeń w rozwiązaniu zadania otwa ego $\mathrm{m}\mathrm{o}\dot{\mathrm{z}}\mathrm{e}$

spowodować, $\dot{\mathrm{z}}\mathrm{e}$ za to rozwiązanie nie będziesz mógł

dostać pełnej liczby punktów.

Pisz czytelnie i $\mathrm{u}\dot{\mathrm{z}}$ aj tvlko $\mathrm{d}$ gopisu lub -Dióra

z czamym tuszem lub atramentem.

Nie $\mathrm{u}\dot{\mathrm{z}}$ aj korektora, a błędne zapisy wyra $\acute{\mathrm{z}}\mathrm{n}\mathrm{i}\mathrm{e}$ prze eśl.

Pamiętaj, $\dot{\mathrm{z}}\mathrm{e}$ zapisy w brudnopisie nie będą oceniane.

$\mathrm{M}\mathrm{o}\dot{\mathrm{z}}$ esz korzystać z zestawu wzorów matematycznych,

cyrkla i linijki oraz kalkulatora.

Na karcie odpowiedzi wpisz swój numer PESEL i przyklej

naklejkę z kodem.

Nie wpisuj $\dot{\mathrm{z}}$ adnych znaków w części przeznaczonej dla

egzaminatora.

2.

4.

5.

6.

7.

8.

9.

Liczba punktów

do uzyskania: 50

$\Vert\Vert\Vert\Vert\Vert\Vert\Vert\Vert\Vert\Vert\Vert\Vert\Vert\Vert\Vert\Vert\Vert\Vert\Vert\Vert\Vert\Vert\Vert\Vert|  \mathrm{M}\mathrm{M}\mathrm{A}-\mathrm{R}1_{-}1\mathrm{P}-112$




{\it 2}

{\it Egzamin maturalny z matematyki}

{\it Poziom rozszerzony}

Zadanie l. $(4pkt)$

Uzasadnij, $\dot{\mathrm{z}}\mathrm{e}$ dla $\mathrm{k}\mathrm{a}\dot{\mathrm{z}}$ dej liczby całkowitej $k$ liczba $k^{6}-2k^{4}+k^{2}$ jest podzielna przez 36.





{\it Egzamin maturalny z matematyki}

{\it Poziom rozszerzony}

{\it 11}

Odpowiedzí :
\begin{center}
\includegraphics[width=82.044mm,height=17.784mm]{./F1_M_PR_M2011_page10_images/image001.eps}
\end{center}
Wypelnia

egzaminator

Nr zadania

Maks. lÍczba kt

7.

4

Uzyskana lÍczba pkt





{\it 12}

{\it Egzamin maturalny z matematyki}

{\it Poziom rozszerzony}

Zadanie 8. (4pkt)

Wśród wszystkich graniastosłupów prawidłowych sześciokątnych, w których suma długości

wszystkich krawędzi jest równa 24, jest taki, który ma największe po1e powierzchni bocznej.

Oblicz długość krawędzi podstawy tego graniastosłupa.





{\it Egzamin maturalny z matematyki}

{\it Poziom rozszerzony}

{\it 13}

Odpowiedzí :
\begin{center}
\includegraphics[width=82.044mm,height=17.784mm]{./F1_M_PR_M2011_page12_images/image001.eps}
\end{center}
Wypelnia

egzaminator

Nr zadania

Maks. lÍczba kt

8.

4

Uzyskana lÍczba pkt





{\it 14}

{\it Egzamin maturalny z matematyki}

{\it Poziom rozszerzony}

Zadanie 9. (4pkt)

Oblicz, ile jest liczb ośmiocyfrowych, w zapisie których nie występuje zero, natomiast

występują dwie dwójki i występują trzy trójki.

Odpowiedzí:





{\it Egzamin maturalny z matematyki}

{\it Poziom rozszerzony}

{\it 15}

Zadanie 10. $(3pkt)$

Dany jest czworokąt wypukły ABCD niebędący równoległobokiem. Punkty $M, N$ są

odpowiednio środkami boków AB $\mathrm{i}$ CD. Punkty $P, Q$ są odpowiednio środkami przekątnych

$AC\mathrm{i}BD$. Uzasadnij, $\dot{\mathrm{z}}\mathrm{e}MQ||PN.$
\begin{center}
\includegraphics[width=95.964mm,height=17.832mm]{./F1_M_PR_M2011_page14_images/image001.eps}
\end{center}
Wypelnia

egzaminator

Nr zadania

Maks. liczba kt

4

10.

3

Uzyskana liczba pkt





{\it 16}

{\it Egzamin maturalny z matematyki}

{\it Poziom rozszerzony}

Zadanie ll. $(6pkt)$

Dany jest ostrosłup prawidłowy czworokątny ABCDS o podstawie ABCD. $\mathrm{W}$ trójkącie

równoramiennym $ASC$ stosunek długości podstawy do długości ramienia jest równy

$|AC|:|AS|=6:5$. Oblicz sinus kąta nachylenia ściany bocznej do płaszczyzny podstawy.





{\it Egzamin maturalny z matematyki}

{\it Poziom rozszerzony}

{\it 1}7

Odpowiedzí :
\begin{center}
\includegraphics[width=82.044mm,height=17.784mm]{./F1_M_PR_M2011_page16_images/image001.eps}
\end{center}
Wypelnia

egzaminator

Nr zadania

Maks. lÍczba kt

11.

Uzyskana lÍczba pkt





{\it 18}

{\it Egzamin maturalny z matematyki}

{\it Poziom rozszerzony}

Zadanie 12. $(3pkt)$

$A, B$ są zdarzeniami losowymi zawartymi w $\Omega$. Wykaz, $\dot{\mathrm{z}}$ ejezeli $P(A)=0,9 \mathrm{i}P(B)=0,7,$

to $P(A\cap B')\leq 0,3$ ($B'$ oznacza zdarzenie przeciwne do zdarzenia $B$).

Odpowiedzí:
\begin{center}
\includegraphics[width=81.996mm,height=17.784mm]{./F1_M_PR_M2011_page17_images/image001.eps}
\end{center}
Nr zadania

Wypelnia Maks. liczba kt

egzaminator

Uzyskana liczba pkt

12.

3





{\it Egzamin maturalny z matematyki}

{\it Poziom rozszerzony}

{\it 19}

BRUDNOPIS










{\it Egzamin maturalny z matematyki}

{\it Poziom rozszerzony}

{\it 3}

Zadanie 2. $(4pkt)$

Uzasadnij, $\dot{\mathrm{z}}$ ejezeli $a\neq b, a\neq c, b\neq c\mathrm{i}a+b=2c$, to $\displaystyle \frac{a}{a-c}+\frac{b}{b-c}=2.$
\begin{center}
\includegraphics[width=95.964mm,height=17.832mm]{./F1_M_PR_M2011_page2_images/image001.eps}
\end{center}
Wypelnia

egzaminator

VIaks. liczba kt

1.

4

2.

4

Uzyskana liczba pkt





$\blacksquare$

$\blacksquare$

$\blacksquare$

$\Vert\Vert\Vert\Vert\Vert\Vert\Vert\Vert\Vert\Vert\Vert\Vert\Vert\Vert\Vert\Vert\Vert\Vert\Vert\Vert\Vert\Vert\Vert\Vert|$
\begin{center}
\includegraphics[width=79.452mm,height=15.804mm]{./F1_M_PR_M2011_page20_images/image001.eps}
\end{center}
PESEL

$\mathrm{M}\mathrm{M}\mathrm{A}-\mathrm{R}1_{-}1$ P-112

WYPELNIA ZDAJACY

Miejsce na naKlej$\kappa$e

z rr PESE-

WYPELNIA EGZAMINATOR
\begin{center}
\begin{tabular}{|l|l|l|l|l|l|l|l|}
	\\
&	\multicolumn{1}{|l|}{$0$}&	\multicolumn{1}{|l|}{ $1$}&	\multicolumn{1}{|l|}{ $2$}&	\multicolumn{1}{|l|}{ $3$}&	\multicolumn{1}{|l|}{ $4$}&	\multicolumn{1}{|l|}{ $5$}&	\multicolumn{1}{|l|}{ $6$}	\\
\cline{2-8}
\multicolumn{1}{|l|}{ $1$}&	\multicolumn{1}{|l|}{ $\square $}&	\multicolumn{1}{|l|}{ $\square $}&	\multicolumn{1}{|l|}{ $\square $}&	\multicolumn{1}{|l|}{ $\square $}&	\multicolumn{1}{|l|}{ $\square $}&	\multicolumn{1}{|l|}{}&	\multicolumn{1}{|l|}{}	\\
\hline
\multicolumn{1}{|l|}{ $2$}&	\multicolumn{1}{|l|}{ $\square $}&	\multicolumn{1}{|l|}{ $\square $}&	\multicolumn{1}{|l|}{ $\square $}&	\multicolumn{1}{|l|}{ $\square $}&	\multicolumn{1}{|l|}{ $\square $}&	\multicolumn{1}{|l|}{}&	\multicolumn{1}{|l|}{}	\\
\hline
\multicolumn{1}{|l|}{ $3$}&	\multicolumn{1}{|l|}{ $\square $}&	\multicolumn{1}{|l|}{ $\square $}&	\multicolumn{1}{|l|}{ $\square $}&	\multicolumn{1}{|l|}{ $\square $}&	\multicolumn{1}{|l|}{ $\square $}&	\multicolumn{1}{|l|}{ $\square $}&	\multicolumn{1}{|l|}{ $\square $}	\\
\hline
\multicolumn{1}{|l|}{ $4$}&	\multicolumn{1}{|l|}{ $\square $}&	\multicolumn{1}{|l|}{ $\square $}&	\multicolumn{1}{|l|}{ $\square $}&	\multicolumn{1}{|l|}{ $\square $}&	\multicolumn{1}{|l|}{ $\square $}&	\multicolumn{1}{|l|}{}&	\multicolumn{1}{|l|}{}	\\
\hline
\multicolumn{1}{|l|}{ $5$}&	\multicolumn{1}{|l|}{ $\square $}&	\multicolumn{1}{|l|}{ $\square $}&	\multicolumn{1}{|l|}{ $\square $}&	\multicolumn{1}{|l|}{ $\square $}&	\multicolumn{1}{|l|}{ $\square $}&	\multicolumn{1}{|l|}{}&	\multicolumn{1}{|l|}{}	\\
\hline
\multicolumn{1}{|l|}{ $6$}&	\multicolumn{1}{|l|}{ $\square $}&	\multicolumn{1}{|l|}{ $\square $}&	\multicolumn{1}{|l|}{ $\square $}&	\multicolumn{1}{|l|}{ $\square $}&	\multicolumn{1}{|l|}{ $\square $}&	\multicolumn{1}{|l|}{}&	\multicolumn{1}{|l|}{}	\\
\hline
\multicolumn{1}{|l|}{ $7$}&	\multicolumn{1}{|l|}{ $\square $}&	\multicolumn{1}{|l|}{ $\square $}&	\multicolumn{1}{|l|}{ $\square $}&	\multicolumn{1}{|l|}{ $\square $}&	\multicolumn{1}{|l|}{ $\square $}&	\multicolumn{1}{|l|}{}&	\multicolumn{1}{|l|}{}	\\
\hline
\multicolumn{1}{|l|}{ $8$}&	\multicolumn{1}{|l|}{ $\square $}&	\multicolumn{1}{|l|}{ $\square $}&	\multicolumn{1}{|l|}{ $\square $}&	\multicolumn{1}{|l|}{ $\square $}&	\multicolumn{1}{|l|}{ $\square $}&	\multicolumn{1}{|l|}{}&	\multicolumn{1}{|l|}{}	\\
\hline
\multicolumn{1}{|l|}{ $9$}&	\multicolumn{1}{|l|}{ $\square $}&	\multicolumn{1}{|l|}{ $\square $}&	\multicolumn{1}{|l|}{ $\square $}&	\multicolumn{1}{|l|}{ $\square $}&	\multicolumn{1}{|l|}{ $\square $}&	\multicolumn{1}{|l|}{}&	\multicolumn{1}{|l|}{}	\\
\hline
\multicolumn{1}{|l|}{ $10$}&	\multicolumn{1}{|l|}{ $\square $}&	\multicolumn{1}{|l|}{ $\square $}&	\multicolumn{1}{|l|}{ $\square $}&	\multicolumn{1}{|l|}{ $\square $}&	\multicolumn{1}{|l|}{}&	\multicolumn{1}{|l|}{}&	\multicolumn{1}{|l|}{}	\\
\hline
\multicolumn{1}{|l|}{ $11$}&	\multicolumn{1}{|l|}{ $\square $}&	\multicolumn{1}{|l|}{ $\square $}&	\multicolumn{1}{|l|}{ $\square $}&	\multicolumn{1}{|l|}{ $\square $}&	\multicolumn{1}{|l|}{ $\square $}&	\multicolumn{1}{|l|}{ $\square $}&	\multicolumn{1}{|l|}{ $\square $}	\\
\hline
\multicolumn{1}{|l|}{ $12$}&	\multicolumn{1}{|l|}{ $\square $}&	\multicolumn{1}{|l|}{ $\square $}&	\multicolumn{1}{|l|}{ $\square $}&	\multicolumn{1}{|l|}{ $\square $}&	\multicolumn{1}{|l|}{}&	\multicolumn{1}{|l|}{}&	\multicolumn{1}{|l|}{}	\\
\hline
\end{tabular}

\end{center}
SUMA

PUNKTÓW
\begin{center}
\includegraphics[width=14.532mm,height=9.756mm]{./F1_M_PR_M2011_page20_images/image002.eps}
\end{center}
D $\square  \square  \square  \square  \square  \square  \square  \square  \square  \square $

0 1 2 3 4 5 6 7 8 9

J $\square  \square  \square  \square  \square  \square  \square  \square  \square  \square $

0 1 2 3 4 5 6 7 8 9

$\blacksquare$

$\blacksquare$




\begin{center}
\includegraphics[width=73.152mm,height=11.028mm]{./F1_M_PR_M2011_page21_images/image001.eps}
\end{center}
KOD EGZAMINATORA

Czytelny podpis egzaminatora
\begin{center}
\includegraphics[width=21.840mm,height=9.852mm]{./F1_M_PR_M2011_page21_images/image002.eps}
\end{center}
KOD ZDAJACEGO





{\it 4}

{\it Egzamin maturalny z matematyki}

{\it Poziom rozszerzony}

Zadanie 3. $(6pkt)$

Wyznacz wszystkie wartości parametru $m$, dla których

$x^{2}-4mx-m^{3}+6m^{2}+m-2=0$ ma dwa rózne pierwiastki rzeczywiste $x_{1}, x_{2}$

$(x_{1}-x_{2})^{2}<8(m+1).$

równanie

takie, $\dot{\mathrm{z}}\mathrm{e}$





{\it Egzamin maturalny z matematyki}

{\it Poziom rozszerzony}

{\it 5}

Odpowiedzí :
\begin{center}
\includegraphics[width=82.044mm,height=17.784mm]{./F1_M_PR_M2011_page4_images/image001.eps}
\end{center}
Wypelnia

egzaminator

Nr zadania

Maks. lÍczba kt

3.

Uzyskana lÍczba pkt





{\it 6}

{\it Egzamin maturalny z matematyki}

{\it Poziom rozszerzony}

Zadanie 4. $(4pkt)$

Rozwiąz równanie 2 $\sin^{2}x-2\sin^{2}x\cos x=1-\cos x$ w przedziale $\langle 0,2\pi\rangle.$

Odpowiedzí:





{\it Egzamin maturalny z matematyki}

{\it Poziom rozszerzony}

7

Zadanie 5. $(4pkt)$

$\mathrm{O}$ ciągu $(x_{n})$ dla $n\geq 1$ wiadomo, $\dot{\mathrm{z}}\mathrm{e}$:

a) ciąg $(a_{n})$ określony wzorem $a_{n}=3^{x_{n}}$ dla $n\geq 1$ jest geometryczny o ilorazie $q=27.$

b) $x_{1}+x_{2}+\ldots+x_{10}=145.$

Oblicz $x_{1}.$
\begin{center}
\includegraphics[width=95.964mm,height=17.832mm]{./F1_M_PR_M2011_page6_images/image001.eps}
\end{center}
Wypelnia

egzaminator

Nr zadania

Maks. liczba kt

4.

4

5.

4

Uzyskana liczba pkt





{\it 8}

{\it Egzamin maturalny z matematyki}

{\it Poziom rozszerzony}

Zadanie 6. $(4pkt)$

Podstawa $AB$ trójkąta równoramiennego $ABC$ ma długość 8 oraz $|\neq BAC|=30^{\mathrm{o}}$

długość środkowej $AD$ tego trójkąta.

Oblicz





{\it Egzamin maturalny z matematyki}

{\it Poziom rozszerzony}

{\it 9}

Odpowiedzí :
\begin{center}
\includegraphics[width=82.044mm,height=17.784mm]{./F1_M_PR_M2011_page8_images/image001.eps}
\end{center}
Wypelnia

egzaminator

Nr zadania

Maks. lÍczba kt

4

Uzyskana lÍczba pkt





$ 1\theta$

{\it Egzamin maturalny z matematyki}

{\it Poziom rozszerzony}

Zadanie 7. $(4pkt)$

Oblicz miarę kąta między stycznymi do okręgu $x^{2}+y^{2}+2x-2y-3=0$ poprowadzonymi

przez punkt $A=(2,$ 0$).$







$1-$

$-1\cup 1$

$-\mapsto 1$

$\mathrm{r}--$

Centralna Komisja Egzaminacyjna

Arkusz zawiera informacje prawnie chronione do momentu rozpoczęcia egzaminu.

WPISUJE ZDAJACY

KOD PESEL

{\it Miejsce}

{\it na naklejkę}

{\it z kodem}
\begin{center}
\includegraphics[width=21.432mm,height=9.804mm]{./F1_M_PR_M2012_page0_images/image001.eps}

\includegraphics[width=82.092mm,height=9.804mm]{./F1_M_PR_M2012_page0_images/image002.eps}
\end{center}
\fbox{} dysleksja
\begin{center}
\includegraphics[width=204.060mm,height=216.048mm]{./F1_M_PR_M2012_page0_images/image003.eps}
\end{center}
EGZAMIN MATU LNY

Z MATEMATYKI

MAJ 2012

POZIOM ROZSZERZONY

1.

3.

Sprawd $\acute{\mathrm{z}}$, czy arkusz egzaminacyjny zawiera 19 stron

(zadania $1-11$). Ewentualny brak zgłoś

przewodniczącemu zespołu nadzorującego egzamin.

Rozwiązania zadań i odpowiedzi wpisuj w miejscu na to

przeznaczonym.

Pamiętaj, $\dot{\mathrm{z}}\mathrm{e}$ pominięcie argumentacji lub istotnych

obliczeń w rozwiązaniu zadania otwa ego $\mathrm{m}\mathrm{o}\dot{\mathrm{z}}\mathrm{e}$

spowodować, $\dot{\mathrm{z}}\mathrm{e}$ za to rozwiązanie nie będziesz mógł

dostać pełnej liczby punktów.

Pisz czytelnie i uzywaj tvlko długopisu lub -Dióra

z czarnym tuszem lub atramentem.

Nie uzywaj korektora, a błędne zapisy wyra $\acute{\mathrm{z}}\mathrm{n}\mathrm{i}\mathrm{e}$ prze eśl.

Pamiętaj, $\dot{\mathrm{z}}\mathrm{e}$ zapisy w brudnopisie nie będą oceniane.

$\mathrm{M}\mathrm{o}\dot{\mathrm{z}}$ esz korzystać z zestawu wzorów matematycznych,

cyrkla i linijki oraz kalkulatora.

Na tej stronie oraz na karcie odpowiedzi wpisz swój

numer PESEL i przyklej naklejkę z kodem.

Nie wpisuj $\dot{\mathrm{z}}$ adnych znaków w części przeznaczonej

dla egzaminatora.

Czas pracy:

180 minut

2.

4.

5.

6.

7.

8.

9.

Liczba punktów

do uzyskania: 50

$\Vert\Vert\Vert\Vert\Vert\Vert\Vert\Vert\Vert\Vert\Vert\Vert\Vert\Vert\Vert\Vert\Vert\Vert\Vert\Vert\Vert\Vert\Vert\Vert|  \mathrm{M}\mathrm{M}\mathrm{A}-\mathrm{R}1_{-}1\mathrm{P}-122$




{\it 2}

{\it Egzamin maturalny z matematyki}

{\it Poziom rozszerzony}

Zadanie l. $(4pkt)$

Wyznacz cztery kolejne liczby całkowite takie, $\dot{\mathrm{z}}\mathrm{e}$ największa z nich jest równa sumie

kwadratów trzech pozostałych liczb.





{\it Egzamin maturalny z matematyki}

{\it Poziom rozszerzony}

{\it 11}

Odpowied $\acute{\mathrm{z}}$:
\begin{center}
\includegraphics[width=82.044mm,height=17.832mm]{./F1_M_PR_M2012_page10_images/image001.eps}
\end{center}
Wypelnia

egzaminator

Nr zadania

Maks. liczba kt

Uzyskana liczba pkt





{\it 12}

{\it Egzamin maturalny z matematyki}

{\it Poziom rozszerzony}

Zadanie 7. $(3pkt)$

Udowodnij, $\dot{\mathrm{z}}$ ejezeli $a+b\geq 0$, to prawdziwajest nierówność $a^{3}+b^{3}\geq a^{2}b+ab^{2}$





{\it Egzamin maturalny z matematyki}

{\it Poziom rozszerzony}

{\it 13}

Zadanie 8. $(4pkt)$

Oblicz, ile jest liczb naturalnych ośmiocyfrowych takich, $\dot{\mathrm{z}}\mathrm{e}$ iloczyn cyfr w ich zapisie

dziesiętnymjest równy 12.

Odpowied $\acute{\mathrm{z}}$:
\begin{center}
\includegraphics[width=95.964mm,height=17.832mm]{./F1_M_PR_M2012_page12_images/image001.eps}
\end{center}
Wypelnia

egzaminator

Nr zadania

Maks. liczba kt

7.

3

8.

4

Uzyskana liczba pkt





{\it 14}

{\it Egzamin maturalny z matematyki}

{\it Poziom rozszerzony}

Zadanie 9. $(5pkt)$

Dany jest prostokąt ABCD, w którym $|AB|=a, |BC|=b \mathrm{i}a>b$. Odcinek $AE$ jest wysokością

trójkąta $DAB$ opuszczoną najego bok $BD.$ Wyra $\acute{\mathrm{z}}$ pole trójkąta $AED$ za pomocą a $\mathrm{i}b.$





{\it Egzamin maturalny z matematyki}

{\it Poziom rozszerzony}

{\it 15}

Odpowiedzí :
\begin{center}
\includegraphics[width=82.044mm,height=17.784mm]{./F1_M_PR_M2012_page14_images/image001.eps}
\end{center}
Wypelnia

egzamÍnator

Nr zadania

Maks. liczba kt

5

Uzyskana liczba pkt





{\it 16}

{\it Egzamin maturalny z matematyki}

{\it Poziom rozszerzony}

Zadanie 10. $(5pkt)$

Podstawą ostrosłupa ABCS jest trójkąt równoramienny $ABC$. Krawędzí AS jest wysokością

ostrosiupa oraz $|AS|=8\sqrt{210}, |BS|=118, |CS|=131$. Oblicz objętość tego ostrosłupa.





{\it Egzamin maturalny z matematyki}

{\it Poziom rozszerzony}

17

Odpowied $\acute{\mathrm{z}}$:
\begin{center}
\includegraphics[width=82.044mm,height=17.832mm]{./F1_M_PR_M2012_page16_images/image001.eps}
\end{center}
Wypelnia

egzaminator

Nr zadania

Maks. liczba kt

10.

5

Uzyskana liczba pkt





{\it 18}

{\it Egzamin maturalny z matematyki}

{\it Poziom rozszerzony}

Zadanie ll. $(3pkt)$

Zdarzenia losowe $A, B$ są zawarte w $\Omega$ oraz $P(A\cap B')=0,7 (A'$ oznacza zdarzenie

przeciwne do zdarzenia $A, B'$ oznacza zdarzenie przeciwne do zdarzenia $B$).

Wykaz, $\dot{\mathrm{z}}\mathrm{e}P(A'\cap B)\leq 0,3.$
\begin{center}
\includegraphics[width=81.996mm,height=17.784mm]{./F1_M_PR_M2012_page17_images/image001.eps}
\end{center}
Nr zadania

Wypelnia Maks. liczba kt

egzaminator

Uzyskana liczba pkt

11.

3





{\it Egzamin maturalny z matematyki}

{\it Poziom rozszerzony}

{\it 19}

BRUDNOPIS





{\it Egzamin maturalny z matematyki}

{\it Poziom rozszerzony}

{\it 3}

Odpowiedzí:
\begin{center}
\includegraphics[width=82.044mm,height=17.832mm]{./F1_M_PR_M2012_page2_images/image001.eps}
\end{center}
Wypelnia

egzaminator

Nr zadania

Maks. liczba kt

1.

4

Uzyskana liczba pkt





{\it 4}

{\it Egzamin maturalny z matematyki}

{\it Poziom rozszerzony}

Zadanie 2. $(4pkt)$

Rozwiąz nierówność $x^{4}+x^{2}\geq 2x.$

Odpowiedzí:





{\it Egzamin maturalny z matematyki}

{\it Poziom rozszerzony}

{\it 5}

Zadanie 3. $(4pkt)$

Rozwiąz równanie $\cos 2x+2=3\cos x.$

Odpowiedzí :
\begin{center}
\includegraphics[width=95.964mm,height=17.784mm]{./F1_M_PR_M2012_page4_images/image001.eps}
\end{center}
Wypelnia

egzamÍnator

Nr zadania

Maks. liczba kt

2.

4

3.

4

Uzyskana liczba pkt





{\it 6}

{\it Egzamin maturalny z matematyki}

{\it Poziom rozszerzony}

Zadanie 4. $(6pkt)$

Oblicz wszystkie wartości parametru $m$, dla których równanie $x^{2}-(m+2)x+m+4=0$

ma dwa rózne pierwiastki rzeczywiste $x_{1}, x_{2}$ takie, $\dot{\mathrm{z}}\mathrm{e}x_{1}^{4}+x_{2}^{4}=4m^{3}+6m^{2}-32m+12.$





{\it Egzamin maturalny z matematyki}

{\it Poziom rozszerzony}

7

Odpowied $\acute{\mathrm{z}}$:
\begin{center}
\includegraphics[width=82.044mm,height=17.832mm]{./F1_M_PR_M2012_page6_images/image001.eps}
\end{center}
Wypelnia

egzaminator

Nr zadania

Maks. liczba kt

4.

Uzyskana liczba pkt





{\it 8}

{\it Egzamin maturalny z matematyki}

{\it Poziom rozszerzony}

Zadanie 5. $(6pkt)$

Trzy liczby tworzą ciąg geometryczny. $\mathrm{J}\mathrm{e}\dot{\mathrm{z}}$ eli do drugiej liczby dodamy 8, to ciąg ten zmieni

się w arytmetyczny. $\mathrm{J}\mathrm{e}\dot{\mathrm{z}}$ eli zaś do ostatniej liczby nowego ciągu arytmetycznego dodamy 64,

to tak otrzymany ciąg będzie znów geometryczny. Znajdzí te liczby. Uwzględnij wszystkie

$\mathrm{m}\mathrm{o}\dot{\mathrm{z}}$ liwości.





{\it Egzamin maturalny z matematyki}

{\it Poziom rozszerzony}

{\it 9}

Odpowied $\acute{\mathrm{z}}$:
\begin{center}
\includegraphics[width=82.044mm,height=17.832mm]{./F1_M_PR_M2012_page8_images/image001.eps}
\end{center}
Wypelnia

egzaminator

Nr zadania

Maks. liczba kt

5.

Uzyskana liczba pkt





$ 1\theta$

{\it Egzamin maturalny z matematyki}

{\it Poziom rozszerzony}

Zadanie 6. $(6pkt)$

$\mathrm{W}$ układzie współrzędnych rozwazmy wszystkie punkty $P$ postaci: $P=(\displaystyle \frac{1}{2}m+\frac{5}{2},m),$

gdzie $ m\in\langle-1,7\rangle$. Oblicz najmniejszą i największąwartość $|PQ|^{2}$, gdzie $Q=(\displaystyle \frac{55}{2},0).$







Centralna Komisja Egzaminacyjna

Arkusz zawiera informacje prawnie chronione do momentu rozpoczęcia egzaminu.

WPISUJE ZDAJACY

KOD PESEL

{\it Miejsce}

{\it na naklejkę}

{\it z kodem}
\begin{center}
\includegraphics[width=21.432mm,height=9.804mm]{./F1_M_PR_M2013_page0_images/image001.eps}

\includegraphics[width=82.092mm,height=9.804mm]{./F1_M_PR_M2013_page0_images/image002.eps}
\end{center}
\fbox{} dysleksja
\begin{center}
\includegraphics[width=204.060mm,height=216.048mm]{./F1_M_PR_M2013_page0_images/image003.eps}
\end{center}
EGZAMIN MATU LNY

Z MATEMATYKI

MAJ 2013

POZIOM ROZSZERZONY

1.

3.

Sprawdzí, czy arkusz egzaminacyjny zawiera 20 stron

(zadania $1-12$). Ewentualny brak zgłoś

przewodniczącemu zespołu nadzorującego egzamin.

Rozwiązania zadań i odpowiedzi wpisuj w miejscu na to

przeznaczonym.

Pamiętaj, $\dot{\mathrm{z}}\mathrm{e}$ pominięcie argumentacji lub istotnych

obliczeń w rozwiązaniu zadania otwa ego $\mathrm{m}\mathrm{o}\dot{\mathrm{z}}\mathrm{e}$

spowodować, $\dot{\mathrm{z}}\mathrm{e}$ za to rozwiązanie nie będziesz mógł

dostać pełnej liczby punktów.

Pisz czytelnie i uzywaj tvlko długopisu lub -Dióra

z czatnym tuszem lub atramentem.

Nie $\mathrm{u}\dot{\mathrm{z}}$ aj korektora, a błędne zapisy wyrazínie przekreśl.

Pamiętaj, $\dot{\mathrm{z}}\mathrm{e}$ zapisy w brudnopisie nie będą oceniane.

$\mathrm{M}\mathrm{o}\dot{\mathrm{z}}$ esz korzystać z zestawu wzorów matematycznych,

cyrkla i linijki oraz kalkulatora.

Na tej stronie oraz na karcie odpowiedzi wpisz swój

numer PESEL i przyklej naklejkę z kodem.

Nie wpisuj $\dot{\mathrm{z}}$ adnych znaków w części przeznaczonej

dla egzaminatora.

Czas pracy:

180 minut

2.

4.

5.

6.

7.

8.

9.

Liczba punktów

do uzyskania: 50

$\Vert\Vert\Vert\Vert\Vert\Vert\Vert\Vert\Vert\Vert\Vert\Vert\Vert\Vert\Vert\Vert\Vert\Vert\Vert\Vert\Vert\Vert\Vert\Vert|  \mathrm{M}\mathrm{M}\mathrm{A}-\mathrm{R}1_{-}1\mathrm{P}-132$




{\it 2}

{\it Egzamin maturalny z matematyki}

{\it Poziom rozszerzony}

Zadanie l. $(4pkt)$

Rozwiąz nierównoŚć $|2x-5|-|x+4|\leq 2-2x.$

Odpowiedzí:





{\it Egzamin maturalny z matematyki}

{\it Poziom rozszerzony}

{\it 11}

Odpowied $\acute{\mathrm{z}}$:
\begin{center}
\includegraphics[width=82.044mm,height=17.784mm]{./F1_M_PR_M2013_page10_images/image001.eps}
\end{center}
WypelnÍa

egzaminator

Nr zadania

Maks. liczba kt

7.

4

Uzyskana liczba pkt





{\it 12}

{\it Egzamin maturalny z matematyki}

{\it Poziom rozszerzony}

Zadanie 8. $(4pkt)$

Reszta z dzielenia wielomianu $W(x)=4x^{3}-5x^{2}-23x+m$ przez dwumian $x+1$ jest równa 20.

Oblicz wartość współczynnika $m$ oraz pierwiastki tego wielomianu.

Odpowiedzí:





{\it Egzamin maturalny z matematyki}

{\it Poziom rozszerzony}

{\it 13}

Zadanie 9. $(5pkt)$

Dany jest trójkąt $ABC$, w którym $|AC|=17 \mathrm{i} |BC|=10$. Na boku AB lezy punkt $D$ taki, $\dot{\mathrm{z}}\mathrm{e}$

$|AD|:|DB|=3:4$ oraz $|DC|=10$. Oblicz pole trójkąta $ABC.$

Odpowied $\acute{\mathrm{z}}$:
\begin{center}
\includegraphics[width=95.964mm,height=17.832mm]{./F1_M_PR_M2013_page12_images/image001.eps}
\end{center}
Wypelnia

egzaminator

Nr zadania

Maks. liczba kt

8.

4

5

Uzyskana liczba pkt





{\it 14}

{\it Egzamin maturalny z matematyki}

{\it Poziom rozszerzony}

Zadanie 10. (4pkt)

W ostrosłupie ABCS podstawa ABC jest trójkątem równobocznym o boku długości a.

Krawędzí AS jest prostopadła do płaszczyzny podstawy. Odległość wierzchołka A od ściany

BCSjest równa d. Wyznacz objętość tego ostrosłupa.





{\it Egzamin maturalny z matematyki}

{\it Poziom rozszerzony}

{\it 15}

Odpowied $\acute{\mathrm{z}}$:
\begin{center}
\includegraphics[width=82.044mm,height=17.784mm]{./F1_M_PR_M2013_page14_images/image001.eps}
\end{center}
WypelnÍa

egzaminator

Nr zadania

Maks. liczba kt

10.

4

Uzyskana liczba pkt





{\it 16}

{\it Egzamin maturalny z matematyki}

{\it Poziom rozszerzony}

Zadanie ll. $(4pkt)$

Rzucamy cztery razy symetryczną sześcienną kostką do gry. Oblicz prawdopodobieństwo

zdarzenia polegającego na tym, $\dot{\mathrm{z}}\mathrm{e}$ iloczyn liczb oczek otrzymanych we wszystkich czterech

rzutach będzie równy 60.





{\it Egzamin maturalny z matematyki}

{\it Poziom rozszerzony}

17

Odpowied $\acute{\mathrm{z}}$:
\begin{center}
\includegraphics[width=82.044mm,height=17.784mm]{./F1_M_PR_M2013_page16_images/image001.eps}
\end{center}
WypelnÍa

egzaminator

Nr zadania

Maks. liczba kt

11.

4

Uzyskana liczba pkt





{\it 18}

{\it Egzamin maturalny z matematyki}

{\it Poziom rozszerzony}

Zadanie 12. $(3pkt)$

Na rysunku przedstawiony jest fragment wykresu funkcji logarytmicznej $f$ określonej wzorem

$f(x)=\log_{2}(x-p).$
\begin{center}
\includegraphics[width=117.852mm,height=97.536mm]{./F1_M_PR_M2013_page17_images/image001.eps}
\end{center}
a) Podaj wartoŚć p.

b) Narysuj wykres funkcji określonej wzorem $y=|f(x)|.$

c) Podaj wszystkie wartości parametru $m$, dla których równanie

rozwiązania o przeciwnych znakach.

$|f(x)|=m$ ma dwa





{\it Egzamin maturalny z matematyki}

{\it Poziom rozszerzony}

{\it 19}

Odpowied $\acute{\mathrm{z}}$:
\begin{center}
\includegraphics[width=82.044mm,height=17.784mm]{./F1_M_PR_M2013_page18_images/image001.eps}
\end{center}
WypelnÍa

egzaminator

Nr zadania

Maks. liczba kt

12.

3

Uzyskana liczba pkt





$ 2\theta$

{\it Egzamin maturalny z matematyki}

{\it Poziom rozszerzony}

BRUDNOPIS





{\it Egzamin maturalny z matematyki}

{\it Poziom rozszerzony}

{\it 3}

Zadanie 2. $(4pkt)$

Trapez równoramienny ABCD o podstawach AB $\mathrm{i}$ CD jest opisany na okręgu o promieniu $r.$

Wykaz, $\dot{\mathrm{z}}\mathrm{e}4r^{2}=|AB|\cdot|CD|.$
\begin{center}
\includegraphics[width=95.964mm,height=17.784mm]{./F1_M_PR_M2013_page2_images/image001.eps}
\end{center}
Wypelnia

egzaminator

Nr zadania

Maks. liczba kt

1.

4

2.

4

Uzyskana liczba pkt





{\it 4}

{\it Egzamin maturalny z matematyki}

{\it Poziom rozszerzony}

Zadanie 3. (3pkt)

Oblicz, ile jest liczb naturalnych sześciocyfrowych, w zapisie których występuje dokładnie

trzy razy cyfra 0 i dokładnie raz występuje cyfra 5.





{\it Egzamin maturalny z matematyki}

{\it Poziom rozszerzony}

{\it 5}

Odpowiedzí:
\begin{center}
\includegraphics[width=82.044mm,height=17.832mm]{./F1_M_PR_M2013_page4_images/image001.eps}
\end{center}
Wypelnia

egzaminator

Nr zadania

Maks. liczba kt

3.

3

Uzyskana liczba pkt





{\it 6}

{\it Egzamin maturalny z matematyki}

{\it Poziom rozszerzony}

Zadanie 4. $(4pkt)$

Rozwiąz równanie $\cos 2x+\cos x+1=0$ dla $x\in\langle 0,2\pi\rangle.$

Odpowied $\acute{\mathrm{z}}$:





{\it Egzamin maturalny z matematyki}

{\it Poziom rozszerzony}

7

Zadanie 5. $(5pkt)$

Ciąg liczbowy $(a,b,c)$ jest arytmetyczny i $a+b+c=33$, natomiast ciąg $(a-1,b+5,c+19)$

jest geometryczny. Oblicz $a, b, c.$

Odpowiedzí:
\begin{center}
\includegraphics[width=95.964mm,height=17.832mm]{./F1_M_PR_M2013_page6_images/image001.eps}
\end{center}
Wypelnia

egzaminator

Nr zadania

Maks. liczba kt

4.

4

5.

5

Uzyskana liczba pkt





{\it 8}

{\it Egzamin maturalny z matematyki}

{\it Poziom rozszerzony}

Zadanie 6. $(6pkt)$

Wyznacz wszystkie wartości parametru $m$, dla których równanie $x^{2}+2(1-m)x+m^{2}-m=0$

ma dwa rózne rozwiązania rzeczywiste $x_{1}, x_{2}$ spełniające warunek $x_{1}\cdot x_{2}\leq 6m\leq x_{1}^{2}+x_{2}^{2}.$





{\it Egzamin maturalny z matematyki}

{\it Poziom rozszerzony}

{\it 9}

Odpowied $\acute{\mathrm{z}}$:
\begin{center}
\includegraphics[width=82.044mm,height=17.784mm]{./F1_M_PR_M2013_page8_images/image001.eps}
\end{center}
WypelnÍa

egzaminator

Nr zadania

Maks. liczba kt

Uzyskana liczba pkt





$ 1\theta$

{\it Egzamin maturalny z matematyki}

{\it Poziom rozszerzony}

Zadanie 7. $(4pkt)$

Prosta o równaniu $3x-4y-36=0$ przecina okrąg o Środku $S=(3,12)$ w punktach A $\mathrm{i}B.$

Długość odcinka $AB$ jest równa 40. Wyznacz równanie tego okręgu.







Arkusz zawiera informacje prawnie chronione do momentu rozpoczęcia egzaminu.

WPISUJE ZDAJACY

KOD PESEL

{\it Miejsce}

{\it na naklejkę}

{\it z kodem}
\begin{center}
\includegraphics[width=21.432mm,height=9.852mm]{./F1_M_PR_M2014_page0_images/image001.eps}

\includegraphics[width=82.092mm,height=9.852mm]{./F1_M_PR_M2014_page0_images/image002.eps}
\end{center}
\fbox{} dysleksja
\begin{center}
\includegraphics[width=204.060mm,height=217.272mm]{./F1_M_PR_M2014_page0_images/image003.eps}
\end{center}
EGZAMIN MATU

Z MATEMATY

LNY  MAJ 2014

POZIOM ROZSZERZONY

1.

2.

3.

4.

Sprawdzí, czy arkusz egzaminacyjny zawiera 19 stron

(zadania $1-11$). Ewentualny brak zgłoś

przewodniczącemu zespo nadzo jącego egzamin.

Rozwiązania zadań i odpowiedzi wpisuj w miejscu na to

przeznaczonym.

Pamiętaj, $\dot{\mathrm{z}}\mathrm{e}$ pominięcie argumentacji lub istotnych

obliczeń w rozwiązaniu zadania otwa ego $\mathrm{m}\mathrm{o}\dot{\mathrm{z}}\mathrm{e}$

spowodować, $\dot{\mathrm{z}}\mathrm{e}$ za to rozwiązanie nie otrzymasz pełnej

liczby punktów.

Pisz czytelnie i $\mathrm{u}\dot{\mathrm{z}}$ aj tvlko $\mathrm{d}$ gopisu lub -Dióra

z czatnym tuszem lub atramentem.

Nie uzywaj korektora, a błędne zapisy wyrazínie prze eśl.

Pamiętaj, $\dot{\mathrm{z}}\mathrm{e}$ zapisy w brudnopisie nie będą oceniane.

$\mathrm{M}\mathrm{o}\dot{\mathrm{z}}$ esz korzystać z zestawu wzorów matematycznych,

cyrkla i linijki oraz kalkulatora.

Na karcie odpowiedzi wpisz swój numer PESEL i przyklej

naklejkę z kodem.

Nie wpisuj $\dot{\mathrm{z}}$ adnych znaków w części przeznaczonej dla

egzaminatora.

Czas pracy:

180 minut

5.

6.

7.

8.

9.

Liczba punktów

do uzyskania: 50

$\Vert\Vert\Vert\Vert\Vert\Vert\Vert\Vert\Vert\Vert\Vert\Vert\Vert\Vert\Vert\Vert\Vert\Vert\Vert\Vert\Vert\Vert\Vert\Vert|  \mathrm{M}\mathrm{M}\mathrm{A}-\mathrm{R}1_{-}1\mathrm{P}-142$




{\it 2}

{\it Egzamin maturalny z matematyki}

{\it Poziom rozszerzony}

Zadanie l. $(4pkt)$

Dana jest ffinkcja $f$ określona wzorem $f(x)=\displaystyle \frac{|x+3|+|x-3|}{x}$ dla $\mathrm{k}\mathrm{a}\dot{\mathrm{z}}$ dej liczby rzeczywistej

$x\neq 0$. Wyznacz zbiór wartości tej funkcji.





{\it Egzamin maturalny z matematyki}

{\it Poziom rozszerzony}

{\it 11}

Odpowied $\acute{\mathrm{z}}$:
\begin{center}
\includegraphics[width=78.840mm,height=17.628mm]{./F1_M_PR_M2014_page10_images/image001.eps}
\end{center}
Wypelnia

egzaminator

Nr zadania

Maks. liczba kt

3

Uzyskana liczba pkt





{\it 12}

{\it Egzamin maturalny z matematyki}

{\it Poziom rozszerzony}

Zadanie 7. $(6pkt)$

Ciąg geometryczny $(a_{n})$ ma 100 wyrazów i są one 1iczbami dodatnimi. Suma wszystkich

wyrazów o numerach nieparzystych jest sto razy większa od sumy wszystkich wyrazów

o numerach parzystych oraz $\log a_{1}+\log a_{2}+\log a_{3}+\ldots+\log a_{100}=100$. Oblicz $a_{1}.$

Odpowiedzí :





{\it Egzamin maturalny z matematyki}

{\it Poziom rozszerzony}

{\it 13}

Zadanie 8. $(4pkt)$

Punkty $A, B, C, D, E, F$ są kolejnymi wierzchołkami sześciokąta foremnego, przy czym

$A=(0,2\sqrt{3}), B=(2,0)$, a $C$ lezy na osi $ox$. Wyznacz równanie stycznej do okręgu

opisanego na tym sześciokącie przechodzącej przez wierzchołek $E.$

Odpowiedzí :
\begin{center}
\includegraphics[width=90.276mm,height=17.580mm]{./F1_M_PR_M2014_page12_images/image001.eps}
\end{center}
Wypelnia

egzaminator

Nr zadania

Maks. liczba kt

7.

8.

4

Uzyskana liczba pkt





{\it 14}

{\it Egzamin maturalny z matematyki}

{\it Poziom rozszerzony}

Zadanie 9. (6pkt)

Oblicz objętość ostrosłupa trójkątnego ABCS, którego siatkę przedstawiono na rysunku.





{\it Egzamin maturalny z matematyki}

{\it Poziom rozszerzony}

{\it 15}

Odpowied $\acute{\mathrm{z}}$:
\begin{center}
\includegraphics[width=78.840mm,height=17.628mm]{./F1_M_PR_M2014_page14_images/image001.eps}
\end{center}
Wypelnia

egzaminator

Nr zadania

Maks. liczba kt

Uzyskana liczba pkt





{\it 16}

{\it Egzamin maturalny z matematyki}

{\it Poziom rozszerzony}

Zadanie 10. $(5pkt)$

Wyznacz wszystkie całkowite wartości parametru $m$, dla których równanie

$(x^{3}+2x^{2}+2x+1)[x^{2}-(2m+1)x+m^{2}+m]=0$ ma trzy, parami rózne, pierwiastki

rzeczywiste, takie $\dot{\mathrm{z}}$ ejeden z nichjest średnią arytmetyczną dwóch pozostałych.





{\it Egzamin maturalny z matematyki}

{\it Poziom rozszerzony}

17

Odpowied $\acute{\mathrm{z}}$:
\begin{center}
\includegraphics[width=78.840mm,height=17.628mm]{./F1_M_PR_M2014_page16_images/image001.eps}
\end{center}
Wypelnia

egzaminator

Nr zadania

Maks. liczba kt

10.

5

Uzyskana liczba pkt





{\it 18}

{\it Egzamin maturalny z matematyki}

{\it Poziom rozszerzony}

Zadanie ll. $(4pkt)$

$\mathrm{Z}$ urny zawierającej 10 ku1 ponumerowanych ko1ejnymi 1iczbami od 1 do 101osujemy

jednocześnie trzy kule. Oblicz prawdopodobieństwo zdarzenia $A$ polegającego na tym, $\dot{\mathrm{z}}\mathrm{e}$

numerjednej z wylosowanych kuljest równy sumie numerów dwóch pozostałych kul.

Odpowiedzí :
\begin{center}
\includegraphics[width=78.840mm,height=17.580mm]{./F1_M_PR_M2014_page17_images/image001.eps}
\end{center}
Wypelnia

egzaminator

Nr zadania

Maks. liczba kt

11.

4

Uzyskana liczba pkt





{\it Egzamin maturalny z matematyki}

{\it Poziom rozszerzony}

{\it 19}

BRUDNOPIS





{\it Egzamin maturalny z matematyki}

{\it Poziom rozszerzony}

{\it 3}

Odpowied $\acute{\mathrm{z}}$:
\begin{center}
\includegraphics[width=78.840mm,height=17.628mm]{./F1_M_PR_M2014_page2_images/image001.eps}
\end{center}
Wypelnia

egzaminator

Nr zadania

Maks. liczba kt

1.

4

Uzyskana liczba pkt





{\it 4}

{\it Egzamin maturalny z matematyki}

{\it Poziom rozszerzony}

Zadanie 2. $(6pkt)$

Wyznacz wszystkie wartości parametru $m$, dla których funkcja kwadratowa

$f(x)=x^{2}-(2m+2)x+2m+5$ ma dwa rózne pierwiastki $x_{1}, x_{2}$ takie, $\dot{\mathrm{z}}\mathrm{e}$ suma kwadratów

odległości punktów $A=(x_{1}$, 0$) \mathrm{i}B=(x_{2}$, 0$)$ od prostej o równaniu $x+y+1=0$ jest równa 6.





{\it Egzamin maturalny z matematyki}

{\it Poziom rozszerzony}

{\it 5}

Odpowied $\acute{\mathrm{z}}$:
\begin{center}
\includegraphics[width=78.840mm,height=17.628mm]{./F1_M_PR_M2014_page4_images/image001.eps}
\end{center}
Wypelnia

egzaminator

Nr zadania

Maks. liczba kt

2.

Uzyskana liczba pkt





{\it 6}

{\it Egzamin maturalny z matematyki}

{\it Poziom rozszerzony}

Zadanie 3. $(4pkt)$

Rozwiąz równanie $\sqrt{3}\cdot\cos x=1+\sin x$ w przedziale $\langle 0, 2\pi\rangle.$

Odpowied $\acute{\mathrm{z}}$:





{\it Egzamin maturalny z matematyki}

{\it Poziom rozszerzony}

7

Zadanie 4. $(3pkt)$

Udowodnij, $\dot{\mathrm{z}}\mathrm{e}$ dla $\mathrm{k}\mathrm{a}\dot{\mathrm{z}}$ dych dwóch liczb rzeczywistych dodatnich $x, y$ prawdziwa jest

nierówność $(x+1)\displaystyle \frac{x}{y}+(y+1)\frac{y}{x}>2.$
\begin{center}
\includegraphics[width=90.276mm,height=17.580mm]{./F1_M_PR_M2014_page6_images/image001.eps}
\end{center}
Wypelnia

egzaminator

Nr zadania

Maks. liczba kt

3.

4

4.

3

Uzyskana liczba pkt





{\it 8}

{\it Egzamin maturalny z matematyki}

{\it Poziom rozszerzony}

Zadanie 5. $(5pkt)$

Dane są trzy okręgi o środkach $A, B, C$ i promieniach równych odpowiednio $r, 2r, 3r. \mathrm{K}\mathrm{a}\dot{\mathrm{z}}$ de

dwa z tych okręgów są zewnętrznie styczne: pierwszy z drugim w punkcie $K$, drugi z trzecim

w punkcie $L$ i trzeci z pierwszym w punkcie $M$. Oblicz stosunek pola trójkąta $KLM$ do pola

trójkąta $ABC.$





{\it Egzamin maturalny z matematyki}

{\it Poziom rozszerzony}

{\it 9}

Odpowied $\acute{\mathrm{z}}$:
\begin{center}
\includegraphics[width=78.840mm,height=17.628mm]{./F1_M_PR_M2014_page8_images/image001.eps}
\end{center}
Wypelnia

egzaminator

Nr zadania

Maks. liczba kt

5.

5

Uzyskana liczba pkt





$ 1\theta$

{\it Egzamin maturalny z matematyki}

{\it Poziom rozszerzony}

Zadanie 6. $(3pkt)$

Trójkąt $ABC$ jest wpisany w okrąg o środku $S$. Kąty wewnętrzne CAB, $ABC\mathrm{i}BCA$ tego

trójkąta są równe, odpowiednio, $\alpha,  2\alpha \mathrm{i}  4\alpha$. Wykaz, $\dot{\mathrm{z}}\mathrm{e}$ trójkąt $ABC$ jest rozwartokątny,

i udowodnij, $\dot{\mathrm{z}}\mathrm{e}$ miary wypukłych kątów środkowych $ASB, ASC\mathrm{i}BSC$ tworzą w podanej

kolejności ciąg arytmetyczny.







Arkusz zawiera informacje prawnie chronione do momentu rozpoczęcia egzaminu.

UZUPELNIA ZDAJACY

KOD PESEL

{\it Miejsce}

{\it na naklejkę}

{\it z kodem}
\begin{center}
\includegraphics[width=21.432mm,height=9.852mm]{./F1_M_PR_M2015_page0_images/image001.eps}

\includegraphics[width=82.092mm,height=9.852mm]{./F1_M_PR_M2015_page0_images/image002.eps}
\end{center}
\fbox{} dysleksja
\begin{center}
\includegraphics[width=204.060mm,height=197.820mm]{./F1_M_PR_M2015_page0_images/image003.eps}
\end{center}
EGZAMIN MATU LNY

Z MATEMATYKI

POZIOM ROZSZERZONY  8 MAJA 20I5

Instrukcja dla zdającego

l. Sprawdzí, czy arkusz egzaminacyjny zawiera 17 stron

(zadania $1-11$). Ewentualny brak zgłoś przewodniczącemu

zespo nadzorującego egzamin.

2. Rozwiązania zadań i odpowiedzi wpisuj w miejscu na to

przeznaczonym.

3. Pamiętaj, $\dot{\mathrm{z}}\mathrm{e}$ pominięcie argumentacji lub istotnych

obliczeń w rozwiązaniu zadania otwa ego $\mathrm{m}\mathrm{o}\dot{\mathrm{z}}\mathrm{e}$

spowodować, $\dot{\mathrm{z}}\mathrm{e}$ za to rozwiązanie nie będziesz mógł

dostać pełnej liczby punktów.

4. Pisz czytelnie i uzywaj tvlko długopisu lub -Dióra

z czarnym tuszem lub atramentem.

5. Nie uzywaj korektora, a błędne zapisy wyrazínie prze eśl.

6. Pamiętaj, $\dot{\mathrm{z}}\mathrm{e}$ zapisy w brudnopisie nie będą oceniane.

7. $\mathrm{M}\mathrm{o}\dot{\mathrm{z}}$ esz korzystać z zestawu wzorów matematycznych,

cyrkla i linijki oraz kalkulatora prostego.

8. Na tej stronie oraz na karcie odpowiedzi wpisz swój

numer PESEL i przyklej naklejkę z kodem.

9. Nie wpisuj $\dot{\mathrm{z}}$ adnych znaków w części przeznaczonej dla

egzaminatora.

Godzina rozpoczęcia:

Czas pracy:

180 minut

Liczba punktów

do uzyskania: 50

$\Vert\Vert\Vert\Vert\Vert\Vert\Vert\Vert\Vert\Vert\Vert\Vert\Vert\Vert\Vert\Vert\Vert\Vert\Vert\Vert\Vert\Vert\Vert\Vert|  \mathrm{M}\mathrm{M}\mathrm{A}-\mathrm{R}1_{-}1\mathrm{P}-152$




{\it Egzamin maturalny z matematyki}

{\it Poziom rozszerzony}

Zadanie l.$(3pkt)$

Wykaz, $\dot{\mathrm{z}}\mathrm{e}$ dla $\mathrm{k}\mathrm{a}\dot{\mathrm{z}}$ dej dodatniej liczby rzeczywistej $x$ róz$\cdot$nej od l oraz dla $\mathrm{k}\mathrm{a}\dot{\mathrm{z}}$ dej dodatniej

liczby rzeczywistej $y$ róz$\cdot$nej od l prawdziwajest równość

$\displaystyle \log_{x}(xy)\cdot\log_{y}(\frac{y}{x})=\log_{y}(xy)\cdot\log_{x}(\frac{y}{x}).$

Strona 2 z 17

MMA-IR





Odpowied $\acute{\mathrm{z}}$:

{\it Egzamin maturalny z matematyki}

{\it Poziom rozszerzony}
\begin{center}
\includegraphics[width=82.044mm,height=17.832mm]{./F1_M_PR_M2015_page10_images/image001.eps}
\end{center}
Wypelnia

egzaminator

Nr zadania

Maks. liczba kt

8.

4

Uzyskana liczba pkt

MMA-IR

Strona ll z 17





{\it Egzamin maturalny z matematyki}

{\it Poziom rozszerzony}

Zadanie 9, $(5pkt)$

Wyznacz równania prostych stycznych do okręgu o równaniu

i zarazem prostopadłych do prostej $x+2y-6=0.$

$x^{2}+y^{2}+4x-6y-3=0$

Strona 12 z 17

MMA-IR





Odpowied $\acute{\mathrm{z}}$:

{\it Egzamin maturalny z matematyki}

{\it Poziom rozszerzony}
\begin{center}
\includegraphics[width=82.044mm,height=17.784mm]{./F1_M_PR_M2015_page12_images/image001.eps}
\end{center}
Wypelnia

egzamÍnator

Nr zadania

Maks. liczba kt

5

Uzyskana liczba pkt

MMA-IR

Strona 13 z 17





{\it Egzamin maturalny z matematyki}

{\it Poziom rozszerzony}

Zadanie $l0. (6pki)$

Krawędzí podstawy ostrosłupa prawidłowego czworokątnego ABCDS ma długość $a$. Ściana

boczna jest nachylona do płaszczyzny podstawy ostrosłupa pod kątem $ 2\alpha$. Ostrosłup ten

przecięto płaszczyzną, która przechodzi przez krawędzí podstawy i dzieli na połowy kąt

pomiędzy ścianą boczną i podstawą. Oblicz pole powstałego przekroju tego ostrosłupa.

Strona 14 z 17

MMA-IR





Odpowied $\acute{\mathrm{z}}$:

{\it Egzamin maturalny z matematyki}

{\it Poziom rozszerzony}
\begin{center}
\includegraphics[width=82.044mm,height=17.784mm]{./F1_M_PR_M2015_page14_images/image001.eps}
\end{center}
Wypelnia

egzamÍnator

Nr zadania

Maks. liczba kt

10.

Uzyskana liczba pkt

MMA-IR

Strona 15 z 17





{\it Egzamin maturalny z matematyki}

{\it Poziom rozszerzony}

Zadanie $l1_{1}. (3pkt)$

Rozwazmy rzut sześcioma kostkami do gry, z których $\mathrm{k}\mathrm{a}\dot{\mathrm{z}}$ da ma inny kolor. Oblicz

prawdopodobieństwo zdarzenia polegającego na tym, $\dot{\mathrm{z}}\mathrm{e}$ uzyskany wynik rzutu spełnia

równoczeŚnie trzy warunki:

dokładnie na dwóch kostkach otrzymano pojednym oczku;

dokładnie na trzech kostkach otrzymano po sześć oczek;

suma wszystkich otrzymanych liczb oczekjest parzysta.

Odpowiedzí:
\begin{center}
\includegraphics[width=82.044mm,height=17.784mm]{./F1_M_PR_M2015_page15_images/image001.eps}
\end{center}
Nr zadania

Wypelnia Maks. liczba kt

egzaminator

Uzyskana liczba pkt

11.

3

Strona 16 z 17

MMA-IR





{\it Egzamin maturalny z matematyki}

{\it Poziom rozszerzony}

{\it BRUDNOPIS} ({\it nie podlega ocenie})

MMA-IR

Strona 17 z 17





{\it Egzamin maturalny z matematyki}

{\it Poziom rozszerzony}

Zadanie 2. $(5pkt)$

Dany jest wielomian $W(x)=x^{3}-3mx^{2}+(3m^{2}-1)x-9m^{2}+20m+4$. Wykres tego

wielomianu, po przesunięciu o wektor $u=[-3,0]$, przechodzi przez początek układu

współrzędnych. Wyznacz wszystkie pierwiastki wielomianu $W.$

Odpowied $\acute{\mathrm{z}}$:
\begin{center}
\includegraphics[width=96.012mm,height=17.784mm]{./F1_M_PR_M2015_page2_images/image001.eps}
\end{center}
Wypelnia

egzaminator

Nr zadania

Maks. liczba kt

1.

3

2.

5

Uzyskana liczba pkt

MMA-IR

Strona 3 z 17





{\it Egzamin maturalny z matematyki}

{\it Poziom rozszerzony}

Zadanie 3. $(6pkt)$

Wyznacz wszystkie wartości parametru $m$, dla których równanie $(m^{2}-m)x^{2}-x+1=0$ ma

dwa rózne rozwiązania rzeczywiste $x_{1}, x_{2}$ takie, $\displaystyle \dot{\mathrm{z}}\mathrm{e}\frac{1}{x_{1}+x_{2}}\leq\frac{m}{3}\leq\frac{1}{x_{1}}+\frac{1}{x_{2}}$

Strona 4 z17

MMA-IR





Odpowied $\acute{\mathrm{z}}$:

{\it Egzamin maturalny z matematyki}

{\it Poziom rozszerzony}
\begin{center}
\includegraphics[width=82.044mm,height=17.784mm]{./F1_M_PR_M2015_page4_images/image001.eps}
\end{center}
Wypelnia

egzamÍnator

Nr zadania

Maks. liczba kt

3.

Uzyskana liczba pkt

MMA-IR

Strona 5 z 17





{\it Egzamin maturalny z matematyki}

{\it Poziom rozszerzony}

Zadanie 4. (6pkt)

Trzy liczby tworzą ciąg arytmetyczny. Jeśli do pierwszej z nich dodamy 5, do diugiej 3, a do

trzeciej 4, to otrzymamy rosnący ciąg geometryczny, w którym trzeci wyraz jest cztery razy

większy od pierwszego. Znajdzí te liczby.

Odpowiedzí:

Strona 6 z17

MMA-IR





{\it Egzamin maturalny z matematyki}

{\it Poziom rozszerzony}

Zadanie 5. $(4pkt)$

Rozwiąz równanie $\sin^{2}2x-4\sin^{2}x+1=0$ w przedziale $\langle 0,2\pi\rangle.$

Odpowied $\acute{\mathrm{z}}$:
\begin{center}
\includegraphics[width=96.012mm,height=17.784mm]{./F1_M_PR_M2015_page6_images/image001.eps}
\end{center}
Wypelnia

egzaminator

Nr zadania

Maks. liczba kt

4.

5.

4

Uzyskana liczba pkt

MMA-IR

Strona 7 z 17





{\it Egzamin maturalny z matematyki}

{\it Poziom rozszerzony}

Zadanie 6. $(4pkt)$

Rozwiąz nierówność $|2x-6|+|x+7|\geq 17.$

Odpowiedzí:

Strona 8 z17

MMA-IR





{\it Egzamin maturalny z matematyki}

{\it Poziom rozszerzony}

Zadanie 7. $(4pkt)$

$\mathrm{O}$ trapezie ABCD wiadomo, $\dot{\mathrm{z}}\mathrm{e}$ mozna w niego wpisać okrąg, a ponadto długościjego boków

{\it AB}, $BC$, {\it CD}, $AD-\mathrm{w}$ podanej kolejności- tworzą ciąg geometryczny. Uzasadnij, $\dot{\mathrm{z}}\mathrm{e}$ trapez

ABCD jest rombem.
\begin{center}
\includegraphics[width=96.012mm,height=17.832mm]{./F1_M_PR_M2015_page8_images/image001.eps}
\end{center}
Wypelnia

egzaminator

Nr zadania

Maks. liczba kt

4

7.

4

Uzyskana liczba pkt

MMA-IR

Strona 9 z 17





{\it Egzamin maturalny z matematyki}

{\it Poziom rozszerzony}

$\mathrm{Z}\mathrm{a}\mathrm{d}\mathrm{a}\mathrm{n}\mathrm{i}\varepsilon 8. (4pkt)$

Na boku $AB$ trójkąta równobocznego $ABC$ wybrano punkt $D$ taki, $\dot{\mathrm{z}}\mathrm{e}|AD|$ : $|DB|=2:3.$

Oblicz tangens kąta $ACD.$

Strona 10 z 17

MMA-IR







Arkusz zawiera informacje prawnie chronione do momentu rozpoczęcia egzaminu.

UZUPELNIA ZDAJACY

KOD PESEL

{\it miejsce}

{\it na naklejkę}
\begin{center}
\includegraphics[width=21.432mm,height=9.852mm]{./F1_M_PR_M2016_page0_images/image001.eps}

\includegraphics[width=82.092mm,height=9.852mm]{./F1_M_PR_M2016_page0_images/image002.eps}
\end{center}
\fbox{} dysleksja
\begin{center}
\includegraphics[width=204.060mm,height=197.868mm]{./F1_M_PR_M2016_page0_images/image003.eps}
\end{center}
EGZAMIN MATU

Z MATEMATY

LNY

Instrukc.ia dla zda.iacego

1. Sprawd $\acute{\mathrm{z}}$, czy arkusz egzaminacyjny zawiera 24 strony

(zadania $1-11$). Ewentualny brak zgłoś przewodniczącemu

zespo nadzorującego egzamin.

2. Rozwiązania zadań i odpowiedzi wpisuj w miejscu na to

przeznaczonym.

3. Pamiętaj, $\dot{\mathrm{z}}\mathrm{e}$ pominięcie argumentacji lub istotnych

obliczeń w rozwiązaniu zadania otwartego $\mathrm{m}\mathrm{o}\dot{\mathrm{z}}\mathrm{e}$

spowodować, $\dot{\mathrm{z}}\mathrm{e}$ za to rozwiązanie nie będziesz mógł

dostać pełnej liczby punktów.

4. Pisz czytelnie i $\mathrm{u}\dot{\mathrm{z}}$ aj tylko $\mathrm{d}$ gopisu lub pióra

z czarnym tuszem lub atramentem.

5. Nie $\mathrm{u}\dot{\mathrm{z}}$ aj korektora, a błędne zapisy wyra $\acute{\mathrm{z}}\mathrm{n}\mathrm{i}\mathrm{e}$ prze eśl.

6. Pamiętaj, $\dot{\mathrm{z}}\mathrm{e}$ zapisy w brudnopisie nie będą oceniane.

7. $\mathrm{M}\mathrm{o}\dot{\mathrm{z}}$ esz korzystać z zestawu wzorów matematycznych,

cyrkla i linijki oraz kalkulatora prostego.

8. Na tej stronie oraz na karcie odpowiedzi wpisz swój

numer PESEL i przyklej naklejkę z kodem.

9. Nie wpisuj $\dot{\mathrm{z}}$ adnych znaków w części przeznaczonej dla

egzaminatora.

Godzina rozpoczęcia:

9:00

Czas pracy:

180 minut

Liczba punktów

do uzyskania: 50

$\Vert\Vert\Vert\Vert\Vert\Vert\Vert\Vert\Vert\Vert\Vert\Vert\Vert\Vert\Vert\Vert\Vert\Vert\Vert\Vert\Vert\Vert\Vert\Vert|  \mathrm{M}\mathrm{M}\mathrm{A}-\mathrm{R}1_{-}1\mathrm{P}-162$




{\it Egzamin maturalny z matematyki}

{\it Poziom rozszerzony}

Zadanie 1. (3pkt)

Niech $\log_{7}4=a$. Wyznacz $\log_{\sqrt{2}}49$ w zalezności od $a.$

Strona 2 z24

MMA-IR





Odpowiedzí:

{\it Egzamin maturalny z matematyki}

{\it Poziom rozszerzony}
\begin{center}
\includegraphics[width=82.044mm,height=17.784mm]{./F1_M_PR_M2016_page10_images/image001.eps}
\end{center}
Wypelnia

egzaminator

Nr zadania

Maks. liczba kt

5.

Uzyskana liczba pkt

MMA-IR

Strona ll z24





{\it Egzamin maturalny z matematyki}

{\it Poziom rozszerzony}

Zadaníe 6. (6pkt)

Punkty $A=(1,1) \mathrm{i} B=(6,2)$ są wierzchołkami trójkąta $ABC$. Wysokości trójkąta $ABC$

przecinają się w punkcie $M=(3,3)$. Oblicz pole tego trójkąta.

Strona 12 z24

MMA-IR





Odpowiedzí:

{\it Egzamin maturalny z matematyki}

{\it Poziom rozszerzony}
\begin{center}
\includegraphics[width=82.044mm,height=17.784mm]{./F1_M_PR_M2016_page12_images/image001.eps}
\end{center}
Wypelnia

egzaminator

Nr zadania

Maks. liczba kt

Uzyskana liczba pkt

MMA-IR

Strona 13 z24





{\it Egzamin maturalny z matematyki}

{\it Poziom rozszerzony}

Zadaníe 7. (3pkt)

Reszta z dzielenia liczby naturalnej $a$ przez 6 jest równa 1. Reszta z dzie1enia 1iczby

naturalnej $b$ przez $6$jest równa 5. Uzasadnij, $\dot{\mathrm{z}}\mathrm{e}$ liczba $a^{2}-b^{2}$ jest podzielna przez 24.

Strona 14 z24

MMA-IR





{\it Egzamin maturalny z matematyki}

{\it Poziom rozszerzony}
\begin{center}
\includegraphics[width=82.044mm,height=17.784mm]{./F1_M_PR_M2016_page14_images/image001.eps}
\end{center}
Wypelnia

egzaminator

Nr zadania

Maks. liczba kt

7.

3

Uzyskana liczba pkt

MMA-IR

Strona 15 z24





{\it Egzamin maturalny z matematyki}

{\it Poziom rozszerzony}

Zadaníe 8. (6pkt)

$\mathrm{W}$ ostrosłupie prawidłowym czworokątnym ABCDS o podstawie ABCD wysokość jest równa 5,

a kąt między sąsiednimi ścianami bocznymi ostrosłupa ma miarę $120^{\mathrm{o}}$ Oblicz objętość tego

ostrosłupa.

Strona 16 z24

MMA-IR





Odpowiedzí:

{\it Egzamin maturalny z matematyki}

{\it Poziom rozszerzony}
\begin{center}
\includegraphics[width=82.044mm,height=17.784mm]{./F1_M_PR_M2016_page16_images/image001.eps}
\end{center}
Wypelnia

egzaminator

Nr zadania

Maks. liczba kt

8.

Uzyskana liczba pkt

MMA-IR

Strona 17 z24





{\it Egzamin maturalny z matematyki}

{\it Poziom rozszerzony}

Zadaníe 9. (3pkt)

Dany jest okrąg o średnicy $AB$ i środku $S$ oraz dwa okręgi o średnicach AS $\mathrm{i}BS$. Okrąg

o środku $M$ i promieniu $r$ ma z $\mathrm{k}\mathrm{a}\dot{\mathrm{z}}$ dym z danych okręgów dokładnie jeden punkt wspólny

(zobacz rysunek). Wykaz, $\displaystyle \dot{\mathrm{z}}\mathrm{e}r=\frac{1}{6}|AB|.$
\begin{center}
\includegraphics[width=91.392mm,height=82.752mm]{./F1_M_PR_M2016_page17_images/image001.eps}
\end{center}
{\it M}

{\it A  K S  L  B}

Strona 18 z24

MMA-IR





{\it Egzamin maturalny z matematyki}

{\it Poziom rozszerzony}
\begin{center}
\includegraphics[width=82.044mm,height=17.784mm]{./F1_M_PR_M2016_page18_images/image001.eps}
\end{center}
Wypelnia

egzaminator

Nr zadania

Maks. liczba kt

3

Uzyskana liczba pkt

MMA-IR

Strona 19 z24





{\it Egzamin maturalny z matematyki}

{\it Poziom rozszerzony}

Zadanie 10. (5pkt)

$\mathrm{W}$ urnie znajduje się 20 ku1: 9 białych, 9 czerwonych i 2 zie1one. $\mathrm{Z}$ tej umy losujemy bez

zwracania 3 ku1e. Ob1icz prawdopodobieństwo zdarzenia po1egającego na tym, $\dot{\mathrm{z}}\mathrm{e}$ co najmniej

dwie z wylosowanych kul są tego samego koloru.

Strona 20 z24

MMA-IR





Odpowiedzí:

{\it Egzamin maturalny z matematyki}

{\it Poziom rozszerzony}
\begin{center}
\includegraphics[width=82.044mm,height=17.784mm]{./F1_M_PR_M2016_page2_images/image001.eps}
\end{center}
Wypelnia

egzaminator

Nr zadania

Maks. liczba kt

1.

3

Uzyskana liczba pkt

MMA-IR

Strona 3 z24





Odpowiedzí:

{\it Egzamin maturalny z matematyki}

{\it Poziom rozszerzony}
\begin{center}
\includegraphics[width=82.044mm,height=17.832mm]{./F1_M_PR_M2016_page20_images/image001.eps}
\end{center}
Wypelnia

egzaminator

Nr zadania

Maks. liczba kt

5

Uzyskana liczba pkt

MMA-IR

Strona 21 z24





{\it Egzamin maturalny z matematyki}

{\it Poziom rozszerzony}

Zadanie 11. (3pkt)

Rozpatrujemy wszystkie liczby naturalne dziesięciocyfrowe, w zapisie których mogą

występować wyłącznie cyfry 1, 2, 3, przy czym cyfra 1 występuje dokładnie trzy razy.

Uzasadnij, $\dot{\mathrm{z}}\mathrm{e}$ takich liczb jest 15360.

Strona 22 z24

MMA-IR





Odpowiedzí:

{\it Egzamin maturalny z matematyki}

{\it Poziom rozszerzony}
\begin{center}
\includegraphics[width=82.044mm,height=17.832mm]{./F1_M_PR_M2016_page22_images/image001.eps}
\end{center}
Wypelnia

egzaminator

Nr zadania

Maks. liczba kt

3

Uzyskana liczba pkt

MMA-IR

Strona 23 z24





{\it Egzamin maturalny z matematyki}

{\it Poziom rozszerzony}

{\it BRUDNOPIS} ({\it nie podlega ocenie})

Strona 24 z24

MMA-IR





{\it Egzamin maturalny z matematyki}

{\it Poziom rozszerzony}

Zadanie 2. (5pkt)

Wielomian $W(x)=2x^{3}+mx^{2}-22x+n$ jest podzielny przez $\mathrm{k}\mathrm{a}\dot{\mathrm{z}}\mathrm{d}\mathrm{y}$ z dwumianów $x+3$

$\mathrm{i}x-4$. Oblicz wartości współczynników $n\mathrm{i}m$ oraz rozwiąz nierówność $W(x)\geq 0.$

Strona 4 z24

MMA-IR





Odpowiedzí:

{\it Egzamin maturalny z matematyki}

{\it Poziom rozszerzony}
\begin{center}
\includegraphics[width=82.044mm,height=17.784mm]{./F1_M_PR_M2016_page4_images/image001.eps}
\end{center}
Wypelnia

egzaminator

Nr zadania

Maks. liczba kt

2.

5

Uzyskana liczba pkt

MMA-IR

Strona 5 z24





{\it Egzamin maturalny z matematyki}

{\it Poziom rozszerzony}

Zadaníe 3. (4pkt)

Rozwiąz równanie $-2\cos^{2}x+3\sin x+3=0$ w przedziale$<0,2\pi>.$

Strona 6 z24

MMA-IR





Odpowiedzí:

{\it Egzamin maturalny z matematyki}

{\it Poziom rozszerzony}
\begin{center}
\includegraphics[width=82.044mm,height=17.784mm]{./F1_M_PR_M2016_page6_images/image001.eps}
\end{center}
Wypelnia

egzaminator

Nr zadania

Maks. liczba kt

3.

4

Uzyskana liczba pkt

MMA-IR

Strona 7 z24





{\it Egzamin maturalny z matematyki}

{\it Poziom rozszerzony}

Zadaníe 4. (6pkt)

Ciąg $(a,4,b,c)$ jest arytmetyczny, a ciąg $(a,a+b,4c)$jest geometryczny. Oblicz $a, b\mathrm{i}c.$

Strona 8 z24

MMA-IR





Odpowiedzí:

{\it Egzamin maturalny z matematyki}

{\it Poziom rozszerzony}
\begin{center}
\includegraphics[width=82.044mm,height=17.784mm]{./F1_M_PR_M2016_page8_images/image001.eps}
\end{center}
Wypelnia

egzaminator

Nr zadania

Maks. liczba kt

4.

Uzyskana liczba pkt

MMA-IR

Strona 9 z24





{\it Egzamin maturalny z matematyki}

{\it Poziom rozszerzony}

Zadaníe 5. (6pkt)

$\mathrm{W}$ trapezie równoramiennym ABCD, w którym AB $\Vert$ CD, dane są

$|BC|=|AD|=40$. Oblicz promień okręgu wpisanego w trójkąt $ABP,$

przecięcia przekątnych tego trapezu.

$|AB|=84, |CD|=36,$

gdzie P jest punktem

Strona 10 z24

MMA-IR







Arkusz zawiera informacje prawnie chronione do momentu rozpoczęcia egzaminu.

UZUPELNIA ZDAJACY

KOD PESEL

{\it miejsce}

{\it na naklejkę}
\begin{center}
\includegraphics[width=21.432mm,height=9.852mm]{./F1_M_PR_M2017_page0_images/image001.eps}

\includegraphics[width=82.092mm,height=9.852mm]{./F1_M_PR_M2017_page0_images/image002.eps}

\includegraphics[width=204.060mm,height=197.820mm]{./F1_M_PR_M2017_page0_images/image003.eps}
\end{center}
EGZAMIN MATU LNY

Z MATEMATYKI

POZIOM ROZSZERZONY

1.

2.

Sprawdzí, czy arkusz egzaminacyjny zawiera 20 stron

(zadania $1-11$). Ewentualny brak zgłoś przewodniczącemu

zespo nadzorującego egzamin.

Rozwiązania zadań i odpowiedzi wpisuj w miejscu na to

przeznaczonym.

Pamiętaj, $\dot{\mathrm{z}}\mathrm{e}$ pominięcie argumentacji lub istotnych

obliczeń w rozwiązaniu zadania otwa ego $\mathrm{m}\mathrm{o}\dot{\mathrm{z}}\mathrm{e}$

spowodować, $\dot{\mathrm{z}}\mathrm{e}$ za to rozwiązanie nie otrzymasz pełnej

liczby punktów.

Pisz czytelnie i uzywaj tvlko długopisu lub -Dióra

z czarnym tuszem lub atramentem.

Nie uzywaj korektora, a błędne zapisy wyrazínie prze eśl.

Pamiętaj, $\dot{\mathrm{z}}\mathrm{e}$ zapisy w brudnopisie nie będą oceniane.

$\mathrm{M}\mathrm{o}\dot{\mathrm{z}}$ esz korzystać z zestawu wzorów matematycznych,

cyrkla i linijki oraz kalkulatora prostego.

Na tej stronie oraz na karcie odpowiedzi wpisz swój

numer PESEL i przyklej naklejkę z kodem.

Nie wpisuj $\dot{\mathrm{z}}$ adnych znaków w części przeznaczonej dla

egzaminatora.

9 MAJA 20I7

3.

Godzina rozpoczęcia:

4.

5.

6.

7.

8.

9.

Czas pracy:

180 minut

Liczba punktów

do uzyskania: 50

$\Vert\Vert\Vert\Vert\Vert\Vert\Vert\Vert\Vert\Vert\Vert\Vert\Vert\Vert\Vert\Vert\Vert\Vert\Vert\Vert\Vert\Vert\Vert\Vert|  \mathrm{M}\mathrm{M}\mathrm{A}-\mathrm{R}1_{-}1\mathrm{P}-172$




{\it Egzamin maturalny z matematyki}

{\it Poziom rozszerzony}

Zadanie l. $(4pkt)$

Rozwiąz nierówność $|x-1|+|x-5|\leq 10-2x.$

Strona 2 z20

MMA-IR





{\it Egzamin maturalny z matematyki}

{\it Poziom rozszerzony}

Zadanie 6. $(3pkt)$

$\mathrm{W}$ trójkącie ostrokątnym $ABC$ bok $AB$ ma długość $c$, długość boku $BC$ jest równa $a$ oraz

$|\neq ABC|=\beta$. Dwusieczna kąta $ABC$ przecina bok $AC$ trójkąta w punkcie $E.$

Wykaz, $\dot{\mathrm{z}}\mathrm{e}$ długość odcinka $BE$ jest równa $\displaystyle \frac{2ac\cdot\cos\frac{\beta}{2}}{a+c}$
\begin{center}
\includegraphics[width=96.012mm,height=17.832mm]{./F1_M_PR_M2017_page10_images/image001.eps}
\end{center}
Wypelnia

egzaminator

5.

3

3

Jzyskana liczba pkt

MMA-IR

Strona ll z20





{\it Egzamin maturalny z matematyki}

{\it Poziom rozszerzony}

Zadanie 7. (4pkt)

Oblicz, ile jest liczb sześciocyfrowych, w których zapisie nie występuje zero, natomiast

występują dwie dziewiątki, jedna szóstka i suma wszystkich cyfrjest równa 30.

Odpowiedzí:

$ 0\neg$trona 1$2\mathrm{z}20$

MMA-IR





{\it Egzamin maturalny z matematyki}

{\it Poziom rozszerzony}

Zadanie 8. (3pkt)

W dwóch pudełkach umieszczono po pięć kul, przy czym w pierwszym pudełku: 2 ku1e białe

i3 ku1e czerwone, a w drugim pudełku: 1 ku1ę białą i 4 ku1e czerwone. Z pierwszego pudełka

losujemy jedną kulę i bez oglądania wkładamy ją do drugiego pudełka. Następnie

losujemyjedną kulę z drugiego pudełka. Oblicz prawdopodobieństwo wylosowania kuli

białej z drugiego pudełka.

Odpowiedzí:
\begin{center}
\includegraphics[width=96.012mm,height=17.784mm]{./F1_M_PR_M2017_page12_images/image001.eps}
\end{center}
Wypelnia

egzaminator

Nr zadania

Maks. liczba kt

7.

4

8.

3

Uzyskana liczba pkt

MMA-IR

Strona 13 z20





{\it Egzamin maturalny z matematyki}

{\it Poziom rozszerzony}

ZadanÍe 9. (6pkt)

W trójkącie równoramiennym wysokość opuszczona na podstawę jest równa 36, a promień

okręgu wpisanego w ten trójkąt jest równy 10. Ob1icz długości boków tego trójkąta i promień

okręgu opisanego na tym trójkącie.

Strona 14 z20

MMA-IR





Odpowied $\acute{\mathrm{z}}$:

{\it Egzamin maturalny z matematyki}

{\it Poziom rozszerzony}
\begin{center}
\includegraphics[width=82.044mm,height=17.784mm]{./F1_M_PR_M2017_page14_images/image001.eps}
\end{center}
Wypelnia

egzamÍnator

Nr zadania

Maks. liczba kt

Uzyskana liczba pkt

MMA-IR

Strona 15 z20





{\it Egzamin maturalny z matematyki}

{\it Poziom rozszerzony}

Zadanie $l0. (6pki)$

Przekątne sąsiednich ścian bocznych prostopadłościanu wychodzące z jednego wierzchołka

tworzą zjego podstawą kąty o miarach $\displaystyle \frac{\pi}{3} \mathrm{i} \alpha$. Cosinus kąta między tymi przekątnymi jest

równy $\displaystyle \frac{\sqrt{6}}{4}$. Wyznacz miarę kąta $a.$

Strona 16 z20

MMA-IR





Odpowied $\acute{\mathrm{z}}$:

{\it Egzamin maturalny z matematyki}

{\it Poziom rozszerzony}
\begin{center}
\includegraphics[width=82.044mm,height=17.784mm]{./F1_M_PR_M2017_page16_images/image001.eps}
\end{center}
Wypelnia

egzamÍnator

Nr zadania

Maks. liczba kt

10.

Uzyskana liczba pkt

MMA-IR

Strona 17 z20





{\it Egzamin maturalny z matematyki}

{\it Poziom rozszerzony}

Zadanie $l1_{1}. (5pkt)$

Wyznacz równanie okręgu przechodzącego przez punkty $A=(-5,3) \mathrm{i} B=(0,6)$, którego

środek lezy na prostej o równaniu $x-3y+1=0.$

Strona 18 z20

MMA-IR





{\it Egzamin maturalny z matematyki}

{\it Poziom rozszerzony}

Odpowied $\acute{\mathrm{z}}$:
\begin{center}
\includegraphics[width=82.044mm,height=17.784mm]{./F1_M_PR_M2017_page18_images/image001.eps}
\end{center}
Wypelnia

egzamÍnator

Nr zadania

Maks. liczba kt

11.

5

Uzyskana liczba pkt

MMA-IR

Strona 19 z20





{\it Egzamin maturalny z matematyki}

{\it Poziom rozszerzony}

{\it BRUDNOPIS} ({\it nie podlega ocenie})

Strona 20 z20

MMA-IR





Odpowied $\acute{\mathrm{z}}$:

{\it Egzamin maturalny z matematyki}

{\it Poziom rozszerzony}
\begin{center}
\includegraphics[width=82.044mm,height=17.784mm]{./F1_M_PR_M2017_page2_images/image001.eps}
\end{center}
Wypelnia

egzamÍnator

Nr zadania

Maks. liczba kt

1.

4

Uzyskana liczba pkt

MMA-IR

Strona 3 z20





{\it Egzamin maturalny z matematyki}

{\it Poziom rozszerzony}

Zadanie 2. $(Spkt)$

Dany jest wielomian $W(x)=2x^{3}+ax^{2}-13x+b$. Liczba 3 jest jednym z pierwiastków tego

wielomianu. Reszta z dzielenia wielomianu $W(x)$ przez $(x+2)$ jest równa 20. Ob1icz

współczynniki $a\mathrm{i}b$ oraz pozostałe pierwiastki wielomianu $W(x).$

Strona 4 z20

MMA-IR





Odpowied $\acute{\mathrm{z}}$:

{\it Egzamin maturalny z matematyki}

{\it Poziom rozszerzony}
\begin{center}
\includegraphics[width=82.044mm,height=17.784mm]{./F1_M_PR_M2017_page4_images/image001.eps}
\end{center}
Wypelnia

egzamÍnator

Nr zadania

Maks. liczba kt

2.

5

Uzyskana liczba pkt

MMA-IR

Strona 5 z20





{\it Egzamin maturalny z matematyki}

{\it Poziom rozszerzony}

Zadanie 3. $(Spkt)$

Wyznacz wszystkie wartości parametru $m$, dla których równanie

$4x^{2}-6mx+(2m+3)(m-3)=0$

ma dwa rózne rozwiązania rzeczywiste $x_{1}$ i $x_{2}$, przy czym $x_{1}<x_{2}$, spełniające warunek

$(4x_{1}-4x_{2}-1)(4x_{1}-4x_{2}+1)<0.$

Strona 6 z20

MMA-IR





Odpowied $\acute{\mathrm{z}}$:

{\it Egzamin maturalny z matematyki}

{\it Poziom rozszerzony}
\begin{center}
\includegraphics[width=82.044mm,height=17.784mm]{./F1_M_PR_M2017_page6_images/image001.eps}
\end{center}
Wypelnia

egzamÍnator

Nr zadania

Maks. liczba kt

3.

5

Uzyskana liczba pkt

MMA-IR

Strona 7 z20





{\it Egzamin maturalny z matematyki}

{\it Poziom rozszerzony}

Zadanie 4. $(6pkt)$

Liczby $a, b, c$ są - odpowiednio - pierwszym, drugim i trzecim wyrazem ciągu

arytmetycznego. Suma tych liczb jest równa 27. Ciąg $(a-2,b,2c+1)$ jest geometryczny.

Wyznacz liczby $a, b, c.$

Strona 8 z20

MMA-IR





Odpowied $\acute{\mathrm{z}}$:

{\it Egzamin maturalny z matematyki}

{\it Poziom rozszerzony}
\begin{center}
\includegraphics[width=82.044mm,height=17.784mm]{./F1_M_PR_M2017_page8_images/image001.eps}
\end{center}
Wypelnia

egzamÍnator

Nr zadania

Maks. liczba kt

4.

Uzyskana liczba pkt

MMA-IR

Strona 9 z20





{\it Egzamin maturalny z matematyki}

{\it Poziom rozszerzony}

Zadanie 5. $(3pkt)$

Udowodnij, $\dot{\mathrm{z}}\mathrm{e}$ dla dowolnych róznych liczb rzeczywistych $x, y$ prawdziwajest nierówność

$x^{2}y^{2}+2x^{2}+2y^{2}-8xy+4>0.$

Strona 10 z20

MMA-IR







CENTRALNA

KOMISJA

EGZAMINACYJNA

Arkusz zawiera informacje prawnie chronione do momentu rozpoczęcia egzaminu.

UZUPELNIA ZDAJACY

KOD PESEL

{\it miejsce}

{\it na naklejkę}
\begin{center}
\includegraphics[width=21.432mm,height=9.852mm]{./F1_M_PR_M2018_page0_images/image001.eps}

\includegraphics[width=82.140mm,height=9.852mm]{./F1_M_PR_M2018_page0_images/image002.eps}

\includegraphics[width=204.060mm,height=197.820mm]{./F1_M_PR_M2018_page0_images/image003.eps}
\end{center}
EGZAMIN MATU LNY

Z MATEMATYKI

POZIOM ROZSZERZONY

1.

2.

Sprawdzí, czy arkusz egzaminacyjny zawiera 20 stron

(zadania $1-11$). Ewenmalny brak zgłoś przewodniczącemu

zespo nadzorującego egzamin.

Rozwiązania zadań i odpowiedzi wpisuj w miejscu na to

przeznaczonym.

Pamiętaj, $\dot{\mathrm{z}}\mathrm{e}$ pominięcie argumentacji lub istotnych

obliczeń w rozwiązaniu zadania otwa ego $\mathrm{m}\mathrm{o}\dot{\mathrm{z}}\mathrm{e}$

spowodować, $\dot{\mathrm{z}}\mathrm{e}$ za to rozwiązanie nie otrzymasz pełnej

liczby punktów.

Pisz czytelnie i uzywaj tvlko długopisu lub -Dióra

z czarnym tuszem lub atramentem.

Nie uzywaj korektora, a błędne zapisy wyrazínie prze eśl.

Pamiętaj, $\dot{\mathrm{z}}\mathrm{e}$ zapisy w brudnopisie nie będą oceniane.

$\mathrm{M}\mathrm{o}\dot{\mathrm{z}}$ esz korzystać z zestawu wzorów matematycznych,

cyrkla i linijki oraz kalkulatora prostego.

Na tej stronie oraz na karcie odpowiedzi wpisz swój

numer PESEL i przyklej naklejkę z kodem.

Nie wpisuj $\dot{\mathrm{z}}$ adnych znaków w części przeznaczonej dla

egzaminatora.

9 MAJA 20I8

3.

Godzina rozpoczęcia:

4.

5.

6.

7.

8.

9.

Czas pracy:

180 minut

Liczba punktów

do uzyskania: 50

$\Vert\Vert\Vert\Vert\Vert\Vert\Vert\Vert\Vert\Vert\Vert\Vert\Vert\Vert\Vert\Vert\Vert\Vert\Vert\Vert\Vert\Vert\Vert\Vert|  \mathrm{M}\mathrm{M}\mathrm{A}-\mathrm{R}1_{-}1\mathrm{P}-182$




{\it Egzamin maturalny z matematyki}

{\it Poziom rozszerzony}

Zadanie l.$(4pkt)$

Rozwiąz równanie $3|x+2|=|x-3|+11.$

Strona 2 z20

$\mathrm{M}\mathrm{M}_{p}$





{\it Egzamin maturalny z matematyki}

{\it Poziom rozszerzony}

Zadanie 6. $(3pkt)$

Udowodnij, $\dot{\mathrm{z}}\mathrm{e}$ dla $\mathrm{k}\mathrm{a}\dot{\mathrm{z}}$ dej liczby całkowitej $k$ i dla $\mathrm{k}\mathrm{a}\dot{\mathrm{z}}$ dej liczby całkowitej $m$ liczba $k^{3}m-km^{3}$

jest podzielna przez 6.
\begin{center}
\includegraphics[width=96.012mm,height=17.784mm]{./F1_M_PR_M2018_page10_images/image001.eps}
\end{center}
Wypelnia

egzaminator

Nr zadania

Maks. liczba kt

5.

3

3

Uzyskana liczba pkt

MMA-IR

Strona ll z20





{\it Egzamin maturalny z matematyki}

{\it Poziom rozszerzony}

Zadanie 7. $(4pkt)$

Rozwiąz równanie $2\cos^{2}x+3\sin x=0$ w przedziale $\displaystyle \{-\frac{\pi}{2},\frac{3\pi}{2}\}.$

Odpowied $\acute{\mathrm{z}}$:

Strona 12 z20

MMA-IR





{\it Egzamin maturalny z matematyki}

{\it Poziom rozszerzony}

Zadanie 8. $(5pkt)$

Liczba $\displaystyle \frac{2}{5}$ jest pierwiastkiem wielomianu $W(x)=5x^{3}-7x^{2}-3x+p$. Wyznacz pozostałe

pierwiastki tego wielomianu i rozwiąz nierówność $W(x)>0.$

Odpowied $\acute{\mathrm{z}}$:
\begin{center}
\includegraphics[width=96.012mm,height=17.784mm]{./F1_M_PR_M2018_page12_images/image001.eps}
\end{center}
Wypelnia

egzaminator

Nr zadania

Maks. liczba kt

7.

4

8.

5

Uzyskana liczba pkt

MMA-IR

Strona 13 z20





{\it Egzamin maturalny z matematyki}

{\it Poziom rozszerzony}

ZadanÍe 9. $(6pkt)$

Wyznacz wszystkie wartości parametru $m$, dla których równanie $x^{2}+(m+1)x-m^{2}+1=0$ ma

dwa rozwiązania rzeczywiste $x_{1} \mathrm{i}x_{2}(x_{1}\neq x_{2})$, spełniające waiunek $x_{1}^{3}+x_{2}^{3}>-7x_{1}x_{2}.$

Strona 14 z20

MMA-IR





{\it Egzamin maturalny z matematyki}

{\it Poziom rozszerzony}

Odpowiedzí :
\begin{center}
\includegraphics[width=82.044mm,height=17.832mm]{./F1_M_PR_M2018_page14_images/image001.eps}
\end{center}
Wypelnia

egzaminator

Nr zadania

Maks. liczba kt

Uzyskana liczba pkt

MMA-IR

Strona 15 z20





{\it Egzamin maturalny z matematyki}

{\it Poziom rozszerzony}

Zadanie $l0. (6pki)$

Punkt $A=(7,-1)$ jest wierzchołkiem trójkąta równoramiennego $ABC$, w którym $|AC|=|BC|.$

Obie współrzędne wierzchołka $C$ są liczbami ujemnymi. Okrąg wpisany w trójkąt $ABC$ ma

równanie $x^{2}+y^{2}=10$. Oblicz współrzędne wierzchołków $B\mathrm{i}C$ tego trójkąta.

Strona 16 z20

MMA-IR





{\it Egzamin maturalny z matematyki}

{\it Poziom rozszerzony}

Odpowiedzí :
\begin{center}
\includegraphics[width=82.044mm,height=17.832mm]{./F1_M_PR_M2018_page16_images/image001.eps}
\end{center}
Wypelnia

egzaminator

Nr zadania

Maks. liczba kt

10.

Uzyskana liczba pkt

MMA-IR

Strona 17 z20





{\it Egzamin maturalny z matematyki}

{\it Poziom rozszerzony}

Zadanie $l1. (5pktJ$

Przekrój ostrosłupa prawidłowego trójkątnego ABCS płaszczyzną przechodzącą przez

wierzchołek $S$ i wysokości dwóch ścian bocznych jest trójkątem równobocznym. Krawędzí

boczna tego ostrosłupa ma długość $\displaystyle \frac{4\sqrt{3}}{3}$. Oblicz objętość tego ostrosłupa.

Strona 18 z20

MMA-IR





{\it Egzamin maturalny z matematyki}

{\it Poziom rozszerzony}

Odpowiedzí :
\begin{center}
\includegraphics[width=82.044mm,height=17.832mm]{./F1_M_PR_M2018_page18_images/image001.eps}
\end{center}
Wypelnia

egzaminator

Nr zadania

Maks. liczba kt

11.

5

Uzyskana liczba pkt

MMA-IR

Strona 19 z20





{\it Egzamin maturalny z matematyki}

{\it Poziom rozszerzony}

{\it BRUDNOPIS} ({\it nie podlega ocenie})

Strona 20 z20

MM





{\it Egzamin maturalny z matematyki}

{\it Poziom rozszerzony}

Odpowiedzí:
\begin{center}
\includegraphics[width=82.044mm,height=17.832mm]{./F1_M_PR_M2018_page2_images/image001.eps}
\end{center}
Wypelnia

egzaminator

Nr zadania

Maks. liczba kt

1.

4

Uzyskana liczba pkt

MMA-IR

$\urcorner$trona 3$\mathrm{z}20$





{\it Egzamin maturalny z matematyki}

{\it Poziom rozszerzony}

Zadanie 2. $(Spkt)$

Liczby $a, b, c$, spełniające warunek $3a+b+3c=77$, są odpowiednio pierwszym, drugim

i trzecim wyrazem ciągu arytmetycznego. Ciąg $(a,b+1,2c)$ jest geometryczny. Wyznacz

liczby $a, b, c$ oraz podaj wyrazy ciągu geometrycznego.

Strona 4 z20

MMA-IR





{\it Egzamin maturalny z matematyki}

{\it Poziom rozszerzony}

Odpowiedzí:
\begin{center}
\includegraphics[width=82.044mm,height=17.832mm]{./F1_M_PR_M2018_page4_images/image001.eps}
\end{center}
Wypelnia

egzaminator

Nr zadania

Maks. liczba kt

2.

5

Uzyskana liczba pkt

MMA-IR

$\urcorner$trona 5$\mathrm{z}20$





{\it Egzamin maturalny z matematyki}

{\it Poziom rozszerzony}

Zadanie 3. (Spkt)

Dany jest czworokąt wypukły ABCD, w którym

$|AD|=|AB|=|BC|=a, |\triangleleft BAD|=60^{\mathrm{o}}$

$\mathrm{i}|4ADC|=135^{\mathrm{o}}$. Oblicz pole czworokąta ABCD.

Strona 6 z20

MMA-IR





{\it Egzamin maturalny z matematyki}

{\it Poziom rozszerzony}

Odpowiedzí:
\begin{center}
\includegraphics[width=82.044mm,height=17.832mm]{./F1_M_PR_M2018_page6_images/image001.eps}
\end{center}
Wypelnia

egzaminator

Nr zadania

Maks. liczba kt

3.

5

Uzyskana liczba pkt

MMA-IR

$\urcorner$trona 7$\mathrm{z}20$





{\it Egzamin maturalny z matematyki}

{\it Poziom rozszerzony}

Zadanie 4. $(4pkt)$

$\mathrm{Z}$ liczb ośmioelementowego zbioru $Z=\{1$, 2, 3, 4, 5, 6, 7, 9$\}$ tworzymy ośmiowyrazowy ciąg,

którego wyrazy nie powtarzają się. Oblicz prawdopodobieństwo zdarzenia polegającego na

tym, $\dot{\mathrm{z}}\mathrm{e}\dot{\mathrm{z}}$ adne dwie liczby parzyste nie są sąsiednimi wyrazami utworzonego ciągu. Wynik

przedstaw w postaci ułamka zwykłego nieskracalnego.

Strona 8 z20

MMA-IR





{\it Egzamin maturalny z matematyki}

{\it Poziom rozszerzony}

Odpowiedzí:
\begin{center}
\includegraphics[width=82.044mm,height=17.832mm]{./F1_M_PR_M2018_page8_images/image001.eps}
\end{center}
Wypelnia

egzaminator

Nr zadania

Maks. liczba kt

4.

4

Uzyskana liczba pkt

MMA-IR

$\urcorner$trona 9$\mathrm{z}20$





{\it Egzamin maturalny z matematyki}

{\it Poziom rozszerzony}

Zadanie 5. $(3pkt)$

Trójkąt $ABC$ jest ostrokątny oraz $|AC|>|BC|$. Dwusieczna $d_{c}$ kąta $ACB$ przecina bok $AB$

w punkcie $K$. Punkt $L$ jest obrazem punktu $K$ w symetrii osiowej względem dwusiecznej $d_{A}$

kąta $BAC$, punkt Mjest obrazem punktu $L$ w symetrii osiowej względem dwusiecznej $d_{c}$ kąta

$ACB$, a punkt $N$ jest obrazem punktu $M$ w symetrii osiowej względem dwusiecznej $d_{B}$ kąta

$ABC$ (zobacz rysunek).
\begin{center}
\includegraphics[width=88.344mm,height=83.976mm]{./F1_M_PR_M2018_page9_images/image001.eps}
\end{center}
{\it C}

{\it L}

{\it M}

{\it A  K N  B}

Udowodnij, $\dot{\mathrm{z}}\mathrm{e}$ na czworokącie KNML mozna opisać okrąg.

Strona 10 z20

MMA-IR







CENTRALNA

KOMfSJA

EGZAMiNACYJNA

Arkusz zawiera informacje prawnie chronione do momentu rozpoczęcia egzaminu.

UZUPELNIA ZDAJACY

KOD PESEL

{\it miejsce}

{\it na naklejkę}
\begin{center}
\includegraphics[width=21.432mm,height=9.852mm]{./F1_M_PR_M2019_page0_images/image001.eps}

\includegraphics[width=82.140mm,height=9.852mm]{./F1_M_PR_M2019_page0_images/image002.eps}

\includegraphics[width=204.060mm,height=197.868mm]{./F1_M_PR_M2019_page0_images/image003.eps}
\end{center}
EGZAMIN MATU LNY

Z MATEMATYKI

POZIOM ROZSZERZONY

Instrukcja dla zdającego

1.

3.

Sprawd $\acute{\mathrm{z}}$, czy arkusz egzaminacyjny zawiera 24 strony

(zadania $1-11$). Ewentualny brak zgłoś przewodniczącemu

zespo nadzorującego egzamin.

Rozwiązania zadań i odpowiedzi wpisuj w miejscu na to

przeznaczonym.

Pamiętaj, $\dot{\mathrm{z}}\mathrm{e}$ pominięcie argumentacji lub istotnych

obliczeń w rozwiązaniu zadania otwa ego $\mathrm{m}\mathrm{o}\dot{\mathrm{z}}\mathrm{e}$

spowodować, $\dot{\mathrm{z}}\mathrm{e}$ za to rozwiązanie nie otrzymasz pełnej

liczby punktów.

Pisz czytelnie i uzywaj tvlko długopisu lub -Dióra

z czarnym tuszem lub atramentem.

Nie uzywaj korektora, a błędne zapisy wyra $\acute{\mathrm{z}}\mathrm{n}\mathrm{i}\mathrm{e}$ prze eśl.

Pamiętaj, $\dot{\mathrm{z}}\mathrm{e}$ zapisy w brudnopisie nie będą oceniane.

$\mathrm{M}\mathrm{o}\dot{\mathrm{z}}$ esz korzystać z zestawu wzorów matematycznych,

cyrkla i linijki oraz kalkulatora prostego.

Na tej stronie oraz na karcie odpowiedzi wpisz swój

numer PESEL i przyklej naklejkę z kodem.

Nie wpisuj $\dot{\mathrm{z}}$ adnych znaków w części przeznaczonej dla

egzaminatora.

Godzina rozpoczęcia:

9:00

4.

5.

6.

7.

8.

9.

Czas pracy:

180 minut

Liczba punktów

do uzyskania: 50

$\Vert\Vert\Vert\Vert\Vert\Vert\Vert\Vert\Vert\Vert\Vert\Vert\Vert\Vert\Vert\Vert\Vert\Vert\Vert\Vert\Vert\Vert\Vert\Vert|  \mathrm{M}\mathrm{M}\mathrm{A}-\mathrm{R}1_{-}1\mathrm{P}-192$




{\it Egzamin maturalny z matematyki}

{\it Poziom rozszerzony}

Zadanie l. $(5pkt)$

Funkcja $f$ jest określona wzorem $f(x)=\displaystyle \frac{|x+2|}{x+2}-x+3|x-1|$, dla $\mathrm{k}\mathrm{a}\dot{\mathrm{z}}$ dej liczby rzeczywistej

$x\neq-2$. Wyznacz zbiór wartości tej ffinkcji.

Strona 2 z24

MMA-IR





Odpowied $\acute{\mathrm{z}}$:

{\it Egzamin maturalny z matematyki}

{\it Poziom rozszerzony}
\begin{center}
\includegraphics[width=82.044mm,height=17.784mm]{./F1_M_PR_M2019_page10_images/image001.eps}
\end{center}
Wypelnia

egzamÍnator

Nr zadania

Maks. liczba kt

5.

Uzyskana liczba pkt

MMA-IR

Strona ll z24





{\it Egzamin maturalny z matematyki}

{\it Poziom rozszerzony}

Zadanie 6. $(5pkt)$

Wielomian określony wzorem $W(x)=2x^{3}+(m^{3}+2)x^{2}-11x-2(2m+1)$ jest podzielny przez

dwumian $(x-2)$ oraz przy dzieleniu przez dwumian $(x+1)$ daje resztę 6. Ob1icz $m$ oraz

pierwiastki wielomianu $W$ dla wyznaczonej wartości $m.$

Strona 12 z24

MMA-IR





Odpowied $\acute{\mathrm{z}}$:

{\it Egzamin maturalny z matematyki}

{\it Poziom rozszerzony}
\begin{center}
\includegraphics[width=82.044mm,height=17.784mm]{./F1_M_PR_M2019_page12_images/image001.eps}
\end{center}
Wypelnia

egzamÍnator

Nr zadania

Maks. liczba kt

5

Uzyskana liczba pkt

MMA-IR

Strona 13 z24





{\it Egzamin maturalny z matematyki}

{\it Poziom rozszerzony}

Zadanie 7. $(4pkt)$

Rozwiąz równanie $\cos 2x=\sin x+1$ w przedziale $\langle 0,2\pi\rangle.$

Strona 14 z24





Odpowiedzí:

{\it Egzamin maturalny z matematyki}

{\it Poziom rozszerzony}
\begin{center}
\includegraphics[width=82.044mm,height=17.832mm]{./F1_M_PR_M2019_page14_images/image001.eps}
\end{center}
Wypelnia

egzaminator

Nr zadania

Maks. liczba kt

7.

4

Uzyskana liczba pkt

MMA-IR

Strona 15 z24





{\it Egzamin maturalny z matematyki}

{\it Poziom rozszerzony}

Zadanie 8. $(4pkt)$

Punkt $D$ lezy na boku $AB$ trójkąta $ABC$ oraz $|AC|=16, |AD|=6, |CD|=14 \mathrm{i} |BC|=|BD|.$

Oblicz obwód trójkąta $ABC.$

Strona 16 z24

MMA-IR





Odpowied $\acute{\mathrm{z}}$:

{\it Egzamin maturalny z matematyki}

{\it Poziom rozszerzony}
\begin{center}
\includegraphics[width=82.044mm,height=17.784mm]{./F1_M_PR_M2019_page16_images/image001.eps}
\end{center}
Wypelnia

egzamÍnator

Nr zadania

Maks. liczba kt

8.

4

Uzyskana liczba pkt

MMA-IR

Strona 17 z24





{\it Egzamin maturalny z matematyki}

{\it Poziom rozszerzony}

Zadanie 9. $(6pkt)$

Wyznacz wszystkie wartości parametru $m$, dla których funkcja kwadratowa

wzorem

$f(x)=(2m+1)x^{2}+(m+2)x+m-3$

f określona

ma dwa rózne pierwiastki rzeczywiste $x_{1}, x_{2}$ spełniające warunek $(x_{1}-x_{2})^{2}+5x_{1}x_{2}\geq 1.$

Strona 18 z24

MMA-IR





Odpowiedzí:

{\it Egzamin maturalny z matematyki}

{\it Poziom rozszerzony}
\begin{center}
\includegraphics[width=82.044mm,height=17.832mm]{./F1_M_PR_M2019_page18_images/image001.eps}
\end{center}
Wypelnia

egzaminator

Nr zadania

Maks. liczba kt

Uzyskana liczba pkt

MMA-IR

Strona 19 z24





{\it Egzamin maturalny z matematyki}

{\it Poziom rozszerzony}

Zadanie 10. $(3pkt)$

Ze zbioru \{1, 2, 3, 4, 5, 6, 7, 8, 9\} 1osujemy ko1ejno ze zwracaniem trzy 1iczby. Ob1icz

prawdopodobieństwo zdarzenia polegającego na tym, $\dot{\mathrm{z}}\mathrm{e}$ dokładnie dwie spośród trzech

wylosowanych liczb będą równe. Wynik zapisz w postaci ułamka nieskracalnego.

Strona 20 z24

MMA-IR





Odpowiedzí:

{\it Egzamin maturalny z matematyki}

{\it Poziom rozszerzony}
\begin{center}
\includegraphics[width=82.044mm,height=17.784mm]{./F1_M_PR_M2019_page2_images/image001.eps}
\end{center}
Wypelnia

egzamÍnator

Nr zadania

Maks. liczba kt

1.

5

Uzyskana liczba pkt

MMA-IR

Strona 3 z24





Odpowiedzí:

{\it Egzamin maturalny z matematyki}

{\it Poziom rozszerzony}
\begin{center}
\includegraphics[width=82.044mm,height=17.832mm]{./F1_M_PR_M2019_page20_images/image001.eps}
\end{center}
Wypelnia

egzaminator

Nr zadania

Maks. liczba kt

10.

3

Uzyskana liczba pkt

MMA-IR

Strona 21 z24





{\it Egzamin maturalny z matematyki}

{\it Poziom rozszerzony}

Zadanie ll. $(6pkt)$

Podstawą ostrosłupa ABCDS jest prostokąt ABCD, którego boki mają długości $|AB|=32$

$\mathrm{i}|BC|=18$. Ściany boczne $ABS\mathrm{i}CDS$ są trójkątami przystającymi i $\mathrm{k}\mathrm{a}\dot{\mathrm{z}}$ da z nichjest nachylona

do płaszczyzny podstawy ostrosłupa pod kątem $\alpha$. Ściany boczne $BCS\mathrm{i}ADS$ są trójkątami

przystającymi i $\mathrm{k}\mathrm{a}\dot{\mathrm{z}}$ da z nich jest nachylona do płaszczyzny podstawy pod kątem $\beta$. Miary

kątów $\alpha \mathrm{i} \beta$ spełniają warunek: $\alpha+\beta=90^{\mathrm{o}}$ Oblicz pole powierzchni całkowitej tego

ostrosłupa.

Strona 22 z24

MMA-IR





Odpowiedzí:

{\it Egzamin maturalny z matematyki}

{\it Poziom rozszerzony}
\begin{center}
\includegraphics[width=82.044mm,height=17.832mm]{./F1_M_PR_M2019_page22_images/image001.eps}
\end{center}
Wypelnia

egzaminator

Nr zadania

Maks. liczba kt

11.

Uzyskana liczba pkt

MMA-IR

Strona 23 z24





{\it Egzamin maturalny z matematyki}

{\it Poziom rozszerzony}

{\it BRUDNOPIS} ({\it nie podlega ocenie})

Strona 24 z24





{\it Egzamin maturalny z matematyki}

{\it Poziom rozszerzony}

Zadanie 2. $(3pkt)$

Udowodnij, $\dot{\mathrm{z}}\mathrm{e}$ dla dowolnych dodatnich liczb rzeczywistych $x\mathrm{i}y$, takich $\dot{\mathrm{z}}\mathrm{e}x<y$, i dowolnej

dodatniej liczby rzeczywistej $a$ prawdziwajest nierówność $\displaystyle \frac{x+a}{y+a}+\frac{y}{x}>2.$

Strona 4 z24

MMA-IR





{\it Egzamin maturalny z matematyki}

{\it Poziom rozszerzony}
\begin{center}
\includegraphics[width=82.044mm,height=17.784mm]{./F1_M_PR_M2019_page4_images/image001.eps}
\end{center}
Wypelnia

egzamÍnator

Nr zadania

Maks. liczba kt

2.

3

Uzyskana liczba pkt

Strona 5 z24





{\it Egzamin maturalny z matematyki}

{\it Poziom rozszerzony}

Zadanie 3. $(3pkt)$

Dany jest trójkąt równoramienny $ABC$, w którym $|AC|=|BC|$. Na ramieniu $AC$ tego trójkąta

wybrano punkt $M(M\neq A\mathrm{i}M\neq C)$, a na ramieniu $BC$ wybrano punkt $N$, w taki sposób, $\dot{\mathrm{z}}\mathrm{e}$

$|AM|=|CN|$. Przez punkty $M\mathrm{i}N$ poprowadzono proste prostopadłe do podstawy $AB$ tego

trójkąta, które wyznaczają na niej punkty $S\mathrm{i}T$. Udowodnij, $\displaystyle \dot{\mathrm{z}}\mathrm{e}|ST|=\frac{1}{2}|AB|.$

Strona 6 z24

MMA-IR





{\it Egzamin maturalny z matematyki}

{\it Poziom rozszerzony}
\begin{center}
\includegraphics[width=82.044mm,height=17.784mm]{./F1_M_PR_M2019_page6_images/image001.eps}
\end{center}
Nr zadania

Wypelnia Maks. liczba kt

egzaminator

Uzyskana liczba pkt

3.

3

Strona 7 z24





{\it Egzamin maturalny z matematyki}

{\it Poziom rozszerzony}

Zadanie 4. $(5pkt)$

Ciąg $(a,b,c)$ jest geometryczny, ciąg $(a+1,b+5,c)$ jest malejącym ciągiem arytmetycznym

oraz $a+b+c=39$. Oblicz $a, b, c.$

Strona 8 z24

MMA-IR





Odpowied $\acute{\mathrm{z}}$:

{\it Egzamin maturalny z matematyki}

{\it Poziom rozszerzony}
\begin{center}
\includegraphics[width=82.044mm,height=17.784mm]{./F1_M_PR_M2019_page8_images/image001.eps}
\end{center}
Nr zadania

Wypelnia Maks. liczba kt

egzamÍnator

Uzyskana liczba pkt

4.

5

MMA-IR

Strona 9 z24





{\it Egzamin maturalny z matematyki}

{\it Poziom rozszerzony}

Zadanie 5. $(6pkt)$

Dane są okręgi o równaniach $x^{2}+y^{2}-12x-8y+43=0 \mathrm{i} x^{2}+y^{2}-2ax+4y+a^{2}-77=0.$

Wyznacz wszystkie wartości parametru $a$, dla których te okręgi mają dokładnie jeden punkt

wspólny. Rozwaz wszystkie przypadki.

Strona 10 z24

MMA-IR







{\it f}

CENTRALNA

f KOMfSJA

EGZAMINACYJNA

Arkusz zawiera informacje prawnie chronione do momentu rozpoczęcia egzaminu.

UZUPELNIA ZDAJACY

KOD PESEL

{\it miejsce}

{\it na naklejkę}
\begin{center}
\includegraphics[width=21.432mm,height=9.852mm]{./F1_M_PR_M2020_page0_images/image001.eps}

\includegraphics[width=82.140mm,height=9.852mm]{./F1_M_PR_M2020_page0_images/image002.eps}

\includegraphics[width=204.060mm,height=197.868mm]{./F1_M_PR_M2020_page0_images/image003.eps}
\end{center}
EGZAMIN MATU LNY

Z MATEMATYKI

POZIOM ROZSZERZONY

Instrukcja dla zdającego

1.

2.  7 MAJA 2020

3.

Sprawd $\acute{\mathrm{z}}$, czy arkusz egzaminacyjny zawiera 22 strony

(zadania $1-12$). Ewentualny brak zgłoś przewodniczącemu

zespo nadzo jącego egzamin.

Rozwiązania zadań i odpowiedzi wpisuj w miejscu na to

przeznaczonym.

Pamiętaj, $\dot{\mathrm{z}}\mathrm{e}$ pominięcie argumentacji lub istotnych

obliczeń w rozwiązaniu zadania otwartego $\mathrm{m}\mathrm{o}\dot{\mathrm{z}}\mathrm{e}$

spowodować, $\dot{\mathrm{z}}\mathrm{e}$ za to rozwiązanie nie otrzymasz pełnej

liczby punktów.

Pisz czytelnie i uzywaj tvlko długopisu lub -Dióra

z czamym tuszem lub atramentem.

Nie uzywaj korektora, a błędne zapisy wyra $\acute{\mathrm{z}}\mathrm{n}\mathrm{i}\mathrm{e}$ prze eśl.

Pamiętaj, $\dot{\mathrm{z}}\mathrm{e}$ zapisy w brudnopisie nie będą oceniane.

$\mathrm{M}\mathrm{o}\dot{\mathrm{z}}$ esz korzystać z zestawu wzorów matematycznych,

cyrkla i linijki oraz kalkulatora prostego.

Na tej stronie oraz na karcie odpowiedzi wpisz swój

numer PESEL i przyklej naklejkę z kodem.

Nie wpisuj $\dot{\mathrm{z}}$ adnych znaków w części przeznaczonej dla

egzaminatora.

Godzina rozpoczęcia:

9:00

4.

5.

6.

7.

8.

9.

Czas pracy:

180 minut

Liczba punktów

do uzyskania: 50

$\Vert\Vert\Vert\Vert\Vert\Vert\Vert\Vert\Vert\Vert\Vert\Vert\Vert\Vert\Vert\Vert\Vert\Vert\Vert\Vert\Vert\Vert\Vert\Vert|  \mathrm{M}\mathrm{M}\mathrm{A}-\mathrm{R}1_{-}1\mathrm{P}-202$




{\it Egzamin maturalny z matematyki}

{\it Poziom rozszerzony}

Zadanie l. $(4pkt)$

Rozwiąz nierówność $(\displaystyle \frac{1}{x}-1)^{-1}\leq 1.$

Odpowied $\acute{\mathrm{z}}$:

Strona 2 z22





Odpowiedzí:

{\it Egzamin maturalny z matematyki}

{\it Poziom rozszerzony}
\begin{center}
\includegraphics[width=82.044mm,height=17.832mm]{./F1_M_PR_M2020_page10_images/image001.eps}
\end{center}
Wypelnia

egzaminator

Nr zadania

Maks. liczba kt

7.

4

Uzyskana liczba pkt

MMA-IR

Strona ll z22





{\it Egzamin maturalny z matematyki}

{\it Poziom rozszerzony}

Zadanie 8. $(4pkt)$

$\mathrm{W}$ trójkącie równoramiennym $ABC$: $|AC|=|BC|=10$, a miara kąta $ABC$ jest równa $30^{\mathrm{o}}$

Na boku $BC$ wybrano punkt $P$, taki, $\dot{\mathrm{z}}\mathrm{e} \displaystyle \frac{|BP|}{|PC|}=\frac{2}{3}$. Oblicz sinus kąta $\alpha$ (zobacz rysunek).
\begin{center}
\includegraphics[width=141.120mm,height=46.020mm]{./F1_M_PR_M2020_page11_images/image001.eps}
\end{center}
{\it C}

{\it P}

{\it A}  $\alpha$  {\it B}

Strona 12 z22

MMA-IR





Odpowiedzí:

{\it Egzamin maturalny z matematyki}

{\it Poziom rozszerzony}
\begin{center}
\includegraphics[width=82.044mm,height=17.832mm]{./F1_M_PR_M2020_page12_images/image001.eps}
\end{center}
Wypelnia

egzaminator

Nr zadania

Maks. liczba kt

8.

4

Uzyskana liczba pkt

MMA-IR

Strona 13 z22





{\it Egzamin maturalny z matematyki}

{\it Poziom rozszerzony}

Zadanie 9. $(5pkt)$

Prosta o równaniu $x+y-10=0$

przecina okrąg

o równaniu $x^{2}+y^{2}-8x-6y+8=0$

wpunktach $K\mathrm{i}L$. Punkt $S$ jest środkiem cięciwy $KL$. Wyznacz równanie obrazu tego okręgu

wjednokładności o środku $S$ i skali $k=-3.$

Strona 14 z22

MMA-IR





Odpowiedzí:

{\it Egzamin maturalny z matematyki}

{\it Poziom rozszerzony}
\begin{center}
\includegraphics[width=82.044mm,height=17.832mm]{./F1_M_PR_M2020_page14_images/image001.eps}
\end{center}
Wypelnia

egzaminator

Nr zadania

Maks. liczba kt

5

Uzyskana liczba pkt

MMA-IR

Strona 15 z22





{\it Egzamin maturalny z matematyki}

{\it Poziom rozszerzony}

Zadanie 10. $(5pkt)$

Dany jest kwadrat ABCD o boku długości 2. Na bokach $BC\mathrm{i}$ CD tego kwadratu wybrano

- odpowiednio- punkty $P\mathrm{i}Q$, takie, $\dot{\mathrm{z}}\mathrm{e}$ długość odcinka $|PC|=|QD|=x$ (zobacz rysunek).

Wyznacz tę wartość $x$, dla której pole trójkąta $APQ$ osiąga wartość najmniejszą. Oblicz to

najmniejsze pole.

{\it Q}

2
\begin{center}
\includegraphics[width=69.396mm,height=62.784mm]{./F1_M_PR_M2020_page15_images/image001.eps}
\end{center}
{\it D  x}  $2-x$  {\it C}

{\it x}

{\it P}

{\it A} 2  {\it B}

Strona 16 z22

MMA-IR





Odpowiedzí:

{\it Egzamin maturalny z matematyki}

{\it Poziom rozszerzony}
\begin{center}
\includegraphics[width=82.044mm,height=17.832mm]{./F1_M_PR_M2020_page16_images/image001.eps}
\end{center}
Wypelnia

egzaminator

Nr zadania

Maks. liczba kt

10.

5

Uzyskana liczba pkt

MMA-IR

Strona 17 z22





{\it Egzamin maturalny z matematyki}

{\it Poziom rozszerzony}

Zadanie 11. (4pkt)

Oblicz, ilejest wszystkich siedmiocyfrowych liczb naturalnych, w których zapisie dziesiętnym

występują dokładnie trzy cyfry l i dokładnie dwie cyfry 2.

Strona 18 z22

MMA-IR





Odpowiedzí:

{\it Egzamin maturalny z matematyki}

{\it Poziom rozszerzony}
\begin{center}
\includegraphics[width=82.044mm,height=17.832mm]{./F1_M_PR_M2020_page18_images/image001.eps}
\end{center}
Wypelnia

egzaminator

Nr zadania

Maks. liczba kt

11.

4

Uzyskana liczba pkt

MMA-IR

Strona 19 z22





{\it Egzamin maturalny z matematyki}

{\it Poziom rozszerzony}

Zadanie 12. $(6pkt)$

Podstawą ostrosłupa czworokątnego ABCDS jest trapez ABCD (AB $||$ CD). Ramiona tego

trapezu mają długości $|AD|=10 \mathrm{i}|BC|=16$, a miara kąta $ABC$ jest równa $30^{\mathrm{o}}. \mathrm{K}\mathrm{a}\dot{\mathrm{z}}$ da ściana

boczna tego ostrosłupa tworzy z płaszczyzną podstawy kąt $\alpha$, taki, ze $\displaystyle \mathrm{t}\mathrm{g}\alpha=\frac{9}{2}$. Oblicz objętość

tego ostrosłupa.

Strona 20 z22

MMA-IR





{\it Egzamin maturalny z matematyki}

{\it Poziom rozszerzony}

Zadanie 2. $(3pkt)$

Wyznacz wszystkie wartości parametru $a$, dla których równanie $|x-5|=(a-1)^{2}-4$ ma dwa

rózne rozwiązania dodatnie.

Odpowiedzí:
\begin{center}
\includegraphics[width=96.012mm,height=17.784mm]{./F1_M_PR_M2020_page2_images/image001.eps}
\end{center}
Wypelnia

egzaminator

Nr zadania

Maks. lÍczba kt

1.

4

2.

3

Uzyskana liczba pkt

MMA-IR

Strona 3 z22





Odpowiedzí:

{\it Egzamin maturalny z matematyki}

{\it Poziom rozszerzony}
\begin{center}
\includegraphics[width=82.044mm,height=17.832mm]{./F1_M_PR_M2020_page20_images/image001.eps}
\end{center}
Wypelnia

egzaminator

Nr zadania

Maks. liczba kt

12.

Uzyskana liczba pkt

MMA-IR

Strona 21 z22





{\it Egzamin maturalny z matematyki}

{\it Poziom rozszerzony}

{\it BRUDNOPIS} ({\it nie podlega ocenie})

Strona 22 z22















{\it Egzamin maturalny z matematyki}

{\it Poziom rozszerzony}

Zadanie 3. $(3pkt)$

Liczby dodatnie $a\mathrm{i}b$ spełniają równość $a^{2}+2a=4b^{2}+4b$. Wykaz, $\dot{\mathrm{z}}\mathrm{e}a=2b.$

Strona 4 z22





{\it Egzamin maturalny z matematyki}

{\it Poziom rozszerzony}

Zadanie 4. $(3pkt)$

Dany jest trójkąt równoramienny $ABC$, w którym $|AC|=|BC|=6$, a punkt $D$ jest środkiem

podstawy $AB$. Okrąg o środku $D$ jest styczny do prostej $AC$ w punkcie $M$. Punkt $K$ lezy na boku

$AC$, punkt $L$ lezy na boku $BC$, odcinek $KL$ jest styczny do rozwazanego okręgu oraz $|KC|=|LC|=2$

(zobacz rysunek).
\begin{center}
\includegraphics[width=101.496mm,height=64.260mm]{./F1_M_PR_M2020_page4_images/image001.eps}
\end{center}
{\it C}

{\it K  L}

{\it M}

{\it A  D  B}

Wykaz, $\displaystyle \dot{\mathrm{z}}\mathrm{e}\frac{|AM|}{|MC|}=\frac{4}{5}$
\begin{center}
\includegraphics[width=96.012mm,height=17.832mm]{./F1_M_PR_M2020_page4_images/image002.eps}
\end{center}
Wypelnia

egzaminator

Nr zadania

Maks. lÍczba kt

3.

3

4.

3

Uzyskana liczba pkt

MMA-IR

Strona 5 z22





{\it Egzamin maturalny z matematyki}

{\it Poziom rozszerzony}

Zadanie 5. $(5pkt)$

$\mathrm{W}$ trzywyrazowym ciągu geometrycznym $(a_{1},a_{2},a_{3})$ spełnionajest równość $a_{1}+a_{2}+a_{3}=\displaystyle \frac{21}{4}$

Wyrazy $a_{1}, a_{2}, a_{3}$ są- odpowiednio-czwartym, drugim i pierwszym wyrazem rosnącego

ciągu arytmetycznego. Oblicz $a_{1}.$

Strona 6 z22

MMA-IR





Odpowied $\acute{\mathrm{z}}$:

{\it Egzamin maturalny z matematyki}

{\it Poziom rozszerzony}
\begin{center}
\includegraphics[width=82.044mm,height=17.784mm]{./F1_M_PR_M2020_page6_images/image001.eps}
\end{center}
Nr zadania

Wypelnia Maks. liczba kt

egzamÍnator

Uzyskana liczba pkt

5.

5

MMA-IR

Strona 7 z22





{\it Egzamin maturalny z matematyki}

{\it Poziom rozszerzony}

Zadanie 6. $(4pkt)$

Rozwiąz równanie $3\cos 2x+10\cos^{2}x=24\sin x-3$ dla $x\in\langle 0, 2\pi\rangle.$

Strona 8 z22





Odpowied $\acute{\mathrm{z}}$:

{\it Egzamin maturalny z matematyki}

{\it Poziom rozszerzony}
\begin{center}
\includegraphics[width=82.044mm,height=17.784mm]{./F1_M_PR_M2020_page8_images/image001.eps}
\end{center}
Wypelnia

egzamÍnator

Nr zadania

Maks. liczba kt

4

Uzyskana liczba pkt

MMA-IR

Strona 9 z22





{\it Egzamin maturalny z matematyki}

{\it Poziom rozszerzony}

Zadanie 7. $(4pkt)$

Dane jest równanie kwadratowe $x^{2}-(3m+2)x+2m^{2}+7m-15=0$ z niewiadomą $x$. Wyznacz

wszystkie wartoŚci parametru $m$, dla których rózne rozwiązania $x_{1}$ i $x_{2}$ tego równania

i spełniają warunek

$2x_{1}^{2}+5x_{1}x_{2}+2x_{2}^{2}=2.$

istnieją

Strona 10 z22

MMA-IR






\begin{center}
\includegraphics[width=181.440mm,height=312.000mm]{./F2_M_PP_M2015_page0_images/image001.eps}
\end{center}
Arkusz zawiera info acje

prawnie chronione do momentu

rozpoczęcia egzaminu.

1

UZUPEL  A ZDAJACY

KOD  PESEL

{\it miejsce}

{\it na naklejkę}

dysleksja

EGZAMIN MATU  LNY Z MATEMATYKI

POZIOM PODSTAWOWY

DATA: 5 maja 2015 r.

CZAS P CY: 170 minut

LICZBA P  KTÓW DO UZYS NIA: 50

Instrukcja dla zdającego

1.

2.

3.

Sprawd $\acute{\mathrm{z}}$, czy arkusz egzaminacyjny zawiera 24 strony (zadania $1-34$).

Ewentualny brak zgłoś przewodniczącemu zespo nadzorującego

egzamin.

Rozwiązania zadań i odpowiedzi wpisuj w miejscu na to przeznaczonym.

Odpowiedzi do zadań zamkniętych $(1-25)$ przenieś na ka ę odpowiedzi,

zaznaczając je w części ka $\mathrm{y}$ przeznaczonej dla zdającego. Zamaluj $\blacksquare$

pola do tego przeznaczone. Błędne zaznaczenie otocz kółkiem \fcircle$\bullet$

i zaznacz właściwe.

4.

5.

Pamiętaj, $\dot{\mathrm{z}}\mathrm{e}$ pominięcie argumentacji lub istotnych obliczeń

w rozwiązaniu zadania otwa ego (26-34) $\mathrm{m}\mathrm{o}\dot{\mathrm{z}}\mathrm{e}$ spowodować, $\dot{\mathrm{z}}\mathrm{e}$ za to

rozwiązanie nie otrzymasz pełnej liczby punktów.

Pisz czytelnie i $\mathrm{u}\dot{\mathrm{z}}$ aj tvlko $\mathrm{d}$ gopisu lub -Dióra z czamym tuszem lub

atramentem.

6. Nie uzywaj korektora, a błędne zapisy wyrazínie prze eśl.

7. Pamiętaj, $\dot{\mathrm{z}}\mathrm{e}$ zapisy w brudnopisie nie będą oceniane.

8. $\mathrm{M}\mathrm{o}\dot{\mathrm{z}}$ esz korzystać z zesta wzorów matematycznych, cyrkla i linijki oraz

kalkulatora prostego.

9. Na tej stronie oraz na karcie odpowiedzi wpisz swój numer PESEL

i przyklej naklejkę z kodem.

10. Nie wpisuj $\dot{\mathrm{z}}$ adnych znaków w części przeznaczonej dla egzaminatora.

$\Vert\Vert\Vert\Vert\Vert\Vert\Vert\Vert\Vert\Vert\Vert\Vert\Vert\Vert\Vert\Vert\Vert\Vert\Vert\Vert\Vert\Vert\Vert\Vert|$

$\mathrm{M}\mathrm{M}\mathrm{A}-\mathrm{P}1_{-}1\mathrm{P}-152$

Układ graficzny

\copyright CKE 2015

1




{\it Wzadaniach od l. do 25. wybierz i zaznacz na karcie odpowiedzi poprawnq odpowiedzí}.

Zadanie $*.(0-l\rangle$

Wskaz rysunek, na którym przedstawiono przedział, będący zbiorem wszystkich rozwiązań

nierówności $-4\leq x-1\leq 4.$
\begin{center}
\includegraphics[width=173.376mm,height=13.512mm]{./F2_M_PP_M2015_page1_images/image001.eps}
\end{center}
$-5$  3  {\it x}

A.
\begin{center}
\includegraphics[width=173.328mm,height=15.192mm]{./F2_M_PP_M2015_page1_images/image002.eps}
\end{center}
$-3$  5  {\it x}

B.
\begin{center}
\includegraphics[width=173.376mm,height=14.220mm]{./F2_M_PP_M2015_page1_images/image003.eps}
\end{center}
$-3$  {\it x}

5

C.
\begin{center}
\includegraphics[width=173.328mm,height=14.280mm]{./F2_M_PP_M2015_page1_images/image004.eps}
\end{center}
$-5$  {\it x}

3

D.

Zadanie 2. (0-1)

Dane są liczby $a=-\displaystyle \frac{1}{27}, b=\log_{\frac{1}{4}}64, c=\log_{\frac{1}{3}}27$. Iloczyn $abc$ jest równy

A. -9 B. --31 C. -31 D. 3

Zadanie 3. (0-1)

Kwotę 1000 zł u1okowano w banku na roczną 1okatę oprocentowaną w wysokości 4\%

w stosunku rocznym. Po zakończeniu lokaty od naliczonych odsetek odprowadzany jest

podatek w wysokości 19\%. Maksyma1na kwota, jaką po upływie roku będzie mozna wypłacić

z banku, jest równa

A.

1000 $(1-\displaystyle \frac{81}{100}\cdot\frac{4}{100})$

B.

1000 $(1+\displaystyle \frac{19}{100}\cdot\frac{4}{100})$

C.

1000 $(1+\displaystyle \frac{81}{100}\cdot\frac{4}{100})$

D.

1000 $(1-\displaystyle \frac{19}{100}\cdot\frac{4}{100})$

Zadam$\mathrm{e}4.(0-1)$

Równość $\displaystyle \frac{m}{5-\sqrt{5}}=\frac{5+\sqrt{5}}{5}$ zachodzi dla

A. $m=5$

B. $m=4$

C. $m=1$

D. $m=-5$

Strona 2 z24

MMA-IP





{\it BRUDNOPIS} ({\it nie podlega ocenie})

Strona ll z24





Zadanie 26. $(0-2\rangle$

Rozwiąz nierówność $2x^{2}-4x>(x+3)(x-2).$

Odpowiedzí:

Strona 12 z24

MD





Zadanie 27. (0-2)

Wykaz, $\dot{\mathrm{z}}\mathrm{e}$ dla $\mathrm{k}\mathrm{a}\dot{\mathrm{z}}$ dej liczby rzeczywistej $x$ i dla $\mathrm{k}\mathrm{a}\dot{\mathrm{z}}$ dej liczby rzeczywistej $y$ prawdziwa jest

nierówność $4x^{2}-8xy+5y^{2}\geq 0.$
\begin{center}
\includegraphics[width=96.012mm,height=17.832mm]{./F2_M_PP_M2015_page12_images/image001.eps}
\end{center}
Wypelnia

egzaminator

Nr zadania

Maks. liczba kt

2

27.

2

Uzyskana liczba pkt

IMA-IP

Strona 13 z24





Zadanie 28. (0-2)

Dany jest kwadrat ABCD. Przekątne $AC\mathrm{i}BD$ przecinają się w punkcie $E$. Punkty $K\mathrm{i}M$ są

środkami odcinków- odpowiednio -$AE\mathrm{i}EC$. Punkty $L\mathrm{i}N$ lez$\cdot$ą na przekątnej $BD$ tak, $\dot{\mathrm{z}}\mathrm{e}$

$|BL|=\displaystyle \frac{1}{3}|BE| \mathrm{i} |DN|=\displaystyle \frac{1}{3}|DE|$ (zobacz rysunek). Wykaz, $\dot{\mathrm{z}}\mathrm{e}$ stosunek pola czworokąta KLMN

do pola kwadratu ABCD jest równy 1: 3.
\begin{center}
\includegraphics[width=60.864mm,height=60.252mm]{./F2_M_PP_M2015_page13_images/image001.eps}
\end{center}
{\it D  C}

{\it N}

{\it M}

{\it K  L}

{\it A  B}

Strona 14 z24

MMA-IP





Zadanie 29. $(0-2\rangle$

Oblicz najmniejszą

w przedziale $\langle 0, 4\rangle.$

i

największą wartość

funkcji kwadratowej

$f(x)=x^{2}-6x+3$

Odpowied $\acute{\mathrm{z}}$:
\begin{center}
\includegraphics[width=96.012mm,height=17.784mm]{./F2_M_PP_M2015_page14_images/image001.eps}
\end{center}
Wypelnia

egzaminator

Nr zadania

Maks. liczba kt

28.

2

2

Uzyskana liczba pkt

IMA-IP

Strona 15 z24





Zadanie 30. (0-2)

$\mathrm{W}$ układzie współrzędnych są dane punkty $A=(-43,-12), B=(50,19)$. Prosta $AB$ przecina

oś $Ox$ w punkcie $P$. Oblicz pierwszą współrzędną punktu $P.$

Odpowiedzí:

Strona 16 z24

MMA-IP





Zadanie $3l. (0-2)$

$\mathrm{J}\mathrm{e}\dot{\mathrm{z}}$ eli do licznika i do mianownika nieskracalnego dodatniego ułamka dodamy połowę jego

licznika, to otrzymamy $\displaystyle \frac{4}{7}$, ajezeli do licznika i do mianownika dodamy l, to otrzymamy $\displaystyle \frac{1}{2}.$

Wyznacz ten ułamek.

Odpowied $\acute{\mathrm{z}}$:
\begin{center}
\includegraphics[width=96.012mm,height=17.784mm]{./F2_M_PP_M2015_page16_images/image001.eps}
\end{center}
Wypelnia

egzaminator

Nr zadania

Maks. liczba kt

30.

2

31.

2

Uzyskana liczba pkt

IMA-IP

Strona 17 z24





Zadanie 32. (0-4)

Wysokość graniastosłupa prawidłowego czworokątnego jest równa 16. Przekątna graniastosłupa

jest nachylona do płaszczyzny jego podstawy pod kątem, którego cosinus jest równy $\displaystyle \frac{3}{5}$. Oblicz

pole powierzchni całkowitej tego graniastosłupa.

Strona 18 z24

MMA-IP





Odpowiedzí :
\begin{center}
\includegraphics[width=82.044mm,height=17.784mm]{./F2_M_PP_M2015_page18_images/image001.eps}
\end{center}
Wypelnia

egzamÍnator

Nr zadania

Maks. liczba kt

32.

4

Uzyskana liczba pkt

IMA-IP

Strona 19 z24





Zadanie 33. (0-4)

Wśród 115 osób przeprowadzono badania ankietowe, związane z zakupami w pewnym

kiosku. W ponizszej tabeli przedstawiono informacje o tym, ile osób kupiło bilety

tramwajowe ulgowe oraz ile osób kupiło bilety tramwajowe normalne.
\begin{center}
\begin{tabular}{|l|l|}
\hline
\multicolumn{1}{|l|}{$\begin{array}{l}\mbox{Rodzaj kupionych}	\\	\mbox{biletów}	\end{array}$}&	\multicolumn{1}{|l|}{Liczba osób}	\\
\hline
\multicolumn{1}{|l|}{ulgowe}&	\multicolumn{1}{|l|}{$76$}	\\
\hline
\multicolumn{1}{|l|}{normalne}&	\multicolumn{1}{|l|}{$41$}	\\
\hline
\end{tabular}

\end{center}
Uwaga! 27 osób spośród ankietowanych kupiło oba rodzaje bi1etów.

Oblicz prawdopodobieństwo zdarzenia polegającego na tym, $\dot{\mathrm{z}}\mathrm{e}$ osoba losowo wybrana

spośród ankietowanych nie kupiła $\dot{\mathrm{z}}$ adnego biletu. Wynik przedstaw w formie nieskracalnego

ułamka.

Strona 20 z24

MMA-IP





{\it BRUDNOPIS} ({\it nie podlega ocenie})

Strona 3 z24





Odpowiedzí :
\begin{center}
\includegraphics[width=82.044mm,height=17.784mm]{./F2_M_PP_M2015_page20_images/image001.eps}
\end{center}
Wypelnia

egzamÍnator

Nr zadania

Maks. liczba kt

33.

4

Uzyskana liczba pkt

IMA-IP

Strona 21 z24





Zadanie 34. $\zeta 0-5\rangle$

$\mathrm{W}$ nieskończonym ciągu arytmetycznym $(a_{n})$, określonym dla $n\geq 1$, suma jedenastu

początkowych wyrazów tego ciągu jest równa 187. Średnia arytmetyczna pierwszego,

trzeciego i dziewiątego wyrazu tego ciągu, jest równa 12. Wyrazy $a_{1}, a_{3}, a_{k}$ ciągu $(a_{n}),$

w podanej kolejności, tworzą nowy ciąg- trzywyrazowy ciąg geometryczny $(b_{n})$. Oblicz $k.$

Strona 22 z24

MMA-IP





Odpowiedzí :
\begin{center}
\includegraphics[width=82.044mm,height=17.784mm]{./F2_M_PP_M2015_page22_images/image001.eps}
\end{center}
Wypelnia

egzamÍnator

Nr zadania

Maks. liczba kt

34.

5

Uzyskana liczba pkt

IMA-IP

Strona 23 z24





{\it BRUDNOPIS} ({\it nie podlega ocenie})

Strona 24 z24

MD





Zadanie 5. (0-1)

Układ równań 

A. zbiór pusty.

B. dokładnie jeden punkt.

C. dokładnie dwa rózne punkty.

D. zbiór nieskończony.

Zadanie 6. (0-1)

Suma wszystkich pierwiastków równania $(x+3)(x+7)(x-11)=0$ jest równa

A. $-1$

B. 21

C. l

D. $-21$

Zadanie 7. $(0-1\rangle$

Równanie $\displaystyle \frac{x-1}{x+1}=x-1$

A. ma dokładniejedno rozwiązanie: $x=1.$

B. ma dokładniejedno rozwiązanie: $x=0.$

C. ma dokładniejedno rozwiązanie: $x=-1.$

D. ma dokładnie dwa rozwiązania: $x=0, x=1.$

Zadanie 8. (0-1)

Na rysunku przedstawiono wykres funkcjif
\begin{center}
\includegraphics[width=120.492mm,height=75.996mm]{./F2_M_PP_M2015_page3_images/image001.eps}
\end{center}
{\it y}

3

2

1

$-4 -3  -2 -1$  {\it x}

0  1 2 3 4  5

$-1$

$-2$

$-3$

Zbiorem wartości ffinkcji $f$ jest

A. $(-2,2)$ B. $\langle-2$, 2$)$

C. $\langle-2,  2\rangle$

D. $(-2,2\rangle$

Zadanie $g. (0-1)$

Na wykresie funkcji liniowej określonej wzorem $f(x)=(m-1)x+3$ lezy punkt $S=(5,-2).$

Zatem

A. $m=-1$

B. $m=0$

C. $m=1$

D. $m=2$

Strona 4 z24

MMA-IP





{\it BRUDNOPIS} ({\it nie podlega ocenie})

Strona 5 z24





Zadanie $l0. (0-1\rangle$

Funkcja liniowa $f$ określona wzorem $f(x)=2x+b$ ma takie samo miejsce zerowe, jakie ma

funkcja liniowa $g(x)=-3x+4$. Stąd wynika, $\dot{\mathrm{z}}\mathrm{e}$

A. $b=4$

B.

{\it b}$=$- -23

C.

{\it b}$=$- -38

D.

{\it b}$=$ -43

Zadanie ll. $(0-l\rangle$

Funkcja kwadratowa określonajest wzorem $f(x)=x^{2}+x+c. \mathrm{J}\mathrm{e}\dot{\mathrm{z}}$ eli $f(3)=4$, to

A. $f(1)=-6$

B. $f(1)=0$

C. $f(1)=6$

D. $f(1)=18$

$\mathrm{Z}\mathrm{a}\mathrm{d}\mathrm{a}\mathrm{n}\ddagger \mathrm{e}12. (0-1\rangle$

Ile liczb całkowitych $x$ spełnia nierówność $\displaystyle \frac{2}{7}<\frac{x}{14}<\frac{4}{3}$ ?

A. 14

B. 15

C. 16

D. 17

$\mathrm{Z}\mathrm{a}\mathrm{d}\mathrm{a}\mathrm{n}\ddagger \mathrm{e}13. (0-1)$

$\mathrm{W}$ rosnącym ciągu geometrycznym $(a_{n})$, określonym dla $n\geq 1$, spełniony jest warunek

$a_{4}=3a_{1}$. Iloraz $q$ tego ciągu jest równy

A.

{\it q}$=$ -31

B.

{\it q}$=$ -$\sqrt{}$313

C. $q=\sqrt[3]{3}$

D. $q=3$

Zadanie 14. (0-1)

Tangens kąta a zaznaczonego na

su ujest równy

A. -

$\sqrt{3}$

3

B. --45
\begin{center}
\includegraphics[width=83.364mm,height=58.776mm]{./F2_M_PP_M2015_page5_images/image001.eps}
\end{center}
{\it y}

6

{\it P}

5

4

3

2

{\it x}

$-5$

1

$a$

$-3-2-1 0$ 1

$-1$

2 3 4 5

D. - -45

C. $-1$

$P=(-4,5)$

Zadanie 15. $(0-1\rangle$

$\mathrm{J}\mathrm{e}\dot{\mathrm{z}}$ eli $0^{\mathrm{o}}<\alpha<90^{\mathrm{o}}$ oraz $\mathrm{t}\mathrm{g}\alpha=2\sin\alpha$, to

A.

$\displaystyle \cos\alpha=\frac{1}{2}$

B.

$\displaystyle \cos\alpha=\frac{\sqrt{2}}{2}$

C.

$\displaystyle \cos\alpha=\frac{\sqrt{3}}{2}$

D. $\cos\alpha=1$

Strona 6 z24

MMA-IP





{\it BRUDNOPIS} ({\it nie podlega ocenie})

Strona 7 z24





Zadanie $1\epsilon. (0-1\rangle$

Miara kąta wpisanego w okrąg jest o $20^{\mathrm{o}}$ mniejsza od miary kąta środkowego opartego na

tym samym łuku. Wynika stąd, $\dot{\mathrm{z}}\mathrm{e}$ miara kąta wpisanegojest równa

A. $5^{\mathrm{o}}$

B. $10^{\mathrm{o}}$

C. $20^{\mathrm{o}}$

D. $30^{\mathrm{o}}$

$\mathrm{Z}\mathrm{a}\mathrm{d}\mathrm{a}\mathrm{n}\ddagger \mathrm{e}17. (0-1\rangle$

Pole rombu o obwodzie $8$jest równe l. Kąt ostry tego rombu ma miarę $\alpha$. Wtedy

A. $14^{\mathrm{o}}<\alpha<15^{\mathrm{o}}$

B. $29^{\mathrm{o}}<\alpha<30^{\mathrm{o}}$

C. $60^{\mathrm{o}}<\alpha<61^{\mathrm{o}}$

D. $75^{\mathrm{o}}<\alpha<76^{\mathrm{o}}$

Zadanie 18. (0-1)

Prosta $l$ o równaniu $y=m^{2}x+3$ jest równoległa do prostej $k$ o równaniu $y=(4m-4)x-3.$

Zatem

A. $m=2$ B. $m=-2$ C. $m=-2-2\sqrt{2}$ D. $m=2+2\sqrt{2}$

$\mathrm{Z}\mathrm{a}\mathrm{d}\mathrm{a}\mathrm{n}\ddagger \mathrm{e}l9. (0-1\rangle$

Proste o równaniach: $y=2mx-m^{2}-1$ oraz $y=4m^{2}x+m^{2}+1$ są prostopadłe dla

A. {\it m}$=$--21 B. {\it m}$=$-21 C. {\it m}$=$1 D. {\it m}$=$2

Zadanie 20. (0-1)

Dane są punkty $M=(-2,1) \mathrm{i} N=(-1,3)$. Punkt $K$ jest środkiem odcinka $MN$. Obrazem

punktu $K$ w symetrii względem początku układu współrzędnychjest punkt

A.

{\it K}$\prime =$(2, - -23)

B.

{\it K}$\prime =$(2, -23)

C.

{\it K}$\prime =$(-23 , 2)

D. {\it K}$\prime =$(-23 , -2)

Zadanie 21. (0-1)

W graniastosłupie prawidłowym czworokątnym EFGHIJKL wierzchołki E, G, L połączono

odcinkami (takjak na rysunku).
\begin{center}
\includegraphics[width=54.912mm,height=78.432mm]{./F2_M_PP_M2015_page7_images/image001.eps}
\end{center}
{\it L K}

{\it I J}

{\it H G}

{\it O}

{\it E F}

Wskaz kąt między wysokością OL trójkąta EGL i płaszczyzną podstawy tego graniastosłupa.

A.

$\neq HOL$

B.

$\neq OGL$

C. $\neq HLO$

D.

$\neq OHL$

Strona 8 z24

MMA-IP





{\it BRUDNOPIS} ({\it nie podlega ocenie})

Strona 9 z24





Zadanie 22. $(0-1\rangle$

Przekrojem osiowym stozka jest trójkąt równoboczny o boku długości

stozkajest równa

6. Objętość tego

A. $27\pi\sqrt{3}$

B. $9\pi\sqrt{3}$

C. $ 18\pi$

D. $ 6\pi$

Zadanie 23. (0-1)

$\mathrm{K}\mathrm{a}\dot{\mathrm{z}}$ da krawędz graniastosłupa prawidłowego trójkątnego ma długość

powierzchni całkowitej tego graniastosłupajest równe

równą 8. Po1e

A.

$\displaystyle \frac{8^{2}}{3}(\frac{\sqrt{3}}{2}+3)$

B. $8^{2}\cdot\sqrt{3}$

C.

$\displaystyle \frac{8^{2}\sqrt{6}}{3}$

D.

$8^{2}(\displaystyle \frac{\sqrt{3}}{2}+3)$

Zadanie 24. $(0-1\rangle$

Średnia arytmetyczna zestawu danych:

2, 4, 7, 8, 9

jest taka samajak średnia arytmetyczna zestawu danych:

2, 4, 7, 8, 9, $x.$

Wynika stąd, $\dot{\mathrm{z}}\mathrm{e}$

A. $x=0$

B. $x=3$

C. $x=5$

D. $x=6$

Zadanie 25. (0-1)

$\mathrm{W} \mathrm{k}\mathrm{a}\dot{\mathrm{z}}$ dym z trzech pojemników znajduje się para kul, z których jedna jest czerwona,

a druga - niebieska. $\mathrm{Z} \mathrm{k}\mathrm{a}\dot{\mathrm{z}}$ dego pojemnika losujemy jedną kulę. Niech $p$ oznacza

prawdopodobieństwo zdarzenia polegającego na tym, $\dot{\mathrm{z}}\mathrm{e}$ dokładnie dwie z trzech

wylosowanych kul będą czerwone. Wtedy

A.

{\it p}$=$ -41

B.

{\it p}$=$ -83

C.

{\it p}$=$ -21

D.

{\it p}$=$ -23

Strona 10 z24

MMA-IP







gCK$\epsilon$6OENZffl$\tau$AfRS{\it m}JA$\xi$ANLNAACY{\it S}NA

Arkusz zawiera info acje

prawnie chronione do momentu

rozpoczęcia egzaminu.
\begin{center}
\includegraphics[width=22.908mm,height=19.248mm]{./F2_M_PP_M2016_page0_images/image001.eps}
\end{center}
1  $\iota$

UZUPELNIA ZDAJACY

{\it miejsce}

{\it na naklejkę}
\begin{center}
\includegraphics[width=21.900mm,height=14.736mm]{./F2_M_PP_M2016_page0_images/image002.eps}
\end{center}
KOD
\begin{center}
\includegraphics[width=79.656mm,height=14.736mm]{./F2_M_PP_M2016_page0_images/image003.eps}
\end{center}
PESEL
\begin{center}
\includegraphics[width=194.616mm,height=246.432mm]{./F2_M_PP_M2016_page0_images/image004.eps}
\end{center}
dyskalkulia  dysleksja

EGZAMIN MATU  LNY Z MATEMATY

POZIOM PODSTAWOWY

LICZBA P  KTÓW DO UZYS NIA: 50

Instrukcja dla zdającego

1.

2.

3.

4.

5.

Sprawdzí, czy arkusz egzaminacyjny zawiera 24 strony (zadania $1-34$).

Ewentualny brak zgłoś przewodniczącemu zespo nadzorującego

egzamin.

Rozwiązania zadań i odpowiedzi wpisuj w miejscu na to przeznaczonym.

Odpowiedzi do zadań za ię ch $(1-25)$ zaznacz na karcie odpowiedzi,

w części ka $\mathrm{y}$ przeznaczonej dla zdającego. Zamaluj $\blacksquare$ pola do tego

przeznaczone. Błędne zaznaczenie otocz kólkiem \copyright i zaznacz wlaściwe.

Pamiętaj, $\dot{\mathrm{z}}\mathrm{e}$ pominięcie argumentacji lub istotnych obliczeń

w rozwiązaniu zadania otwa ego (26-34) $\mathrm{m}\mathrm{o}\dot{\mathrm{z}}\mathrm{e}$ spowodować, $\dot{\mathrm{z}}\mathrm{e}$ za to

rozwiązanie nie otrzymasz pelnej liczby pu tów.

Pisz cz elnie i $\mathrm{u}\dot{\mathrm{z}}$ aj lko $\mathrm{d}$ gopisu lub pióra z czarnym tuszem lub

atramentem.

6. Nie $\mathrm{u}\dot{\mathrm{z}}$ aj korektora, a błędne zapisy $\mathrm{r}\mathrm{a}\acute{\mathrm{z}}\mathrm{n}\mathrm{i}\mathrm{e}$ prze eśl.

7. Pamiętaj, $\dot{\mathrm{z}}\mathrm{e}$ zapisy w brudnopisie nie będą oceniane.

8. $\mathrm{M}\mathrm{o}\dot{\mathrm{z}}$ esz korzystać z zesta wzorów matema cznych, cyrkla i linijki,

a ta $\mathrm{e}$ z kalkulatora prostego.

9. Na tej stronie oraz na karcie odpowiedzi wpisz swój numer PESEL

i przyklej naklejkę z kodem.

10. Nie wpisuj $\dot{\mathrm{z}}$ adnych znaków w części przeznaczonej dla egzaminatora.

$\Vert\Vert\Vert\Vert\Vert\Vert\Vert\Vert\Vert\Vert\Vert\Vert\Vert\Vert\Vert\Vert\Vert\Vert\Vert\Vert\Vert\Vert\Vert\Vert|$

$\mathrm{M}\mathrm{M}\mathrm{A}-\mathrm{P}1_{-}1\mathrm{P}-162$

Układ graficzny

\copyright CKE 2015




{\it Wzadaniach od l. do 25. wybierz i zaznacz na karcie odpowiedzi poprawnq odpowiedzí}.

Zadanie $*.(0-l\rangle$

Dla $\mathrm{k}\mathrm{a}\dot{\mathrm{z}}$ dej dodatniej liczby $a$ iloraz $\displaystyle \frac{a^{-2,6}}{a^{1,3}}$ jest równy

A.

$a^{-3,9}$

B.

$a^{-2}$

C.

$a^{-1,3}$

D.

$a^{1,3}$

Zadam$\mathrm{e}2. (0-1)$

Liczba $\log_{\sqrt{2}}(2\sqrt{2})$ jest równa

A.

-23

B. 2

C.

-25

D. 3

Zadanie 3. (0-1)

Liczby $a\mathrm{i}c$ są dodatnie. Liczba $b$ stanowi 48\% 1iczby $a$ oraz 32\% 1iczby $c$. Wynika stąd, $\dot{\mathrm{z}}\mathrm{e}$

A. $c=1,5a$

B. $c=1,6a$

C. $c=0,8a$

D. $c=0,16a$

Zadam$\mathrm{e}4.(0-1)$

Równość $(2\sqrt{2}-a)^{2}=17-12\sqrt{2}$ jest prawdziwa dla

A. $a=3$

B. $a=1$

C. $a=-2$

D. $a=-3$

Zadanie 5. $(0-1\rangle$

Jedną z liczb, które spełniają nierówność $-x^{5}+x^{3}-x<-2$, jest

A. l

B. $-1$

C. 2

D. $-2$

Zadam$\mathrm{e}6.(0-1)$

Proste o równaniach $2x-3y=4\mathrm{i}5x-6y=7$ przecinają się w punkcie $P$. Stąd wynika, $\dot{\mathrm{z}}\mathrm{e}$

A. $P=(1,2)$

B. $P=(-1,2)$

C. $P=(-1,-2)$

D. $P=(1,-2)$

Zadanie 7. (0-1)

Punkty ABCD $\mathrm{l}\mathrm{e}\dot{\mathrm{z}}$ ą na o ęgu o środku $\mathrm{S}$ (zobacz

Miara kąta $BDC$ jest równa

A. $91^{\mathrm{o}}$

B. $72,5^{\mathrm{o}}$

C. $18^{\mathrm{o}}$
\begin{center}
\includegraphics[width=90.828mm,height=95.352mm]{./F2_M_PP_M2016_page1_images/image001.eps}
\end{center}
sunek).

{\it D}

{\it C}

$27^{\mathrm{o}}$

{\it S}

$118^{\mathrm{o}}$

{\it B}

{\it A}

Strona 2 z24

D. $32^{\mathrm{o}}$

MMA-IP





{\it BRUDNOPIS} ({\it nie podlega ocenie})

Strona ll z24





Zadanie 26. $(0-2\rangle$

$\mathrm{W}$ tabeli przedstawiono roczne przyrosty wysokości pewnej sosny w ciągu sześciu kolejnych

lat.
\begin{center}
\begin{tabular}{|l|l|l|l|l|l|l|}
\hline
\multicolumn{1}{|l|}{kolejne lata}&	\multicolumn{1}{|l|}{$1$}&	\multicolumn{1}{|l|}{ $2$}&	\multicolumn{1}{|l|}{ $3$}&	\multicolumn{1}{|l|}{ $4$}&	\multicolumn{1}{|l|}{ $5$}&	\multicolumn{1}{|l|}{ $6$}	\\
\hline
\multicolumn{1}{|l|}{przyrost (w cm)}&	\multicolumn{1}{|l|}{$10$}&	\multicolumn{1}{|l|}{ $10$}&	\multicolumn{1}{|l|}{ $7$}&	\multicolumn{1}{|l|}{ $8$}&	\multicolumn{1}{|l|}{ $8$}&	\multicolumn{1}{|l|}{ $7$}	\\
\hline
\end{tabular}

\end{center}
Oblicz średni roczny przyrost wysokości tej sosny w badanym okresie sześciu lat. Otrzymany

wynik zaokrąglij do l cm. Oblicz błąd względny otrzymanego przyblizenia. Podaj ten błąd

w procentach.

Odpowiedzí:

Strona 12 z24

MMA-IP





Zadanie 27. (0-2)

Rozwiąz nierówność $2x^{2}-4x>3x^{2}-6x.$

Odpowied $\acute{\mathrm{z}}$:
\begin{center}
\includegraphics[width=96.012mm,height=17.784mm]{./F2_M_PP_M2016_page12_images/image001.eps}
\end{center}
Wypelnia

egzaminator

Nr zadania

Maks. liczba kt

2

27.

2

Uzyskana liczba pkt

IMA-IP

Strona 13 z24





Zadanie 28. (0-2)

Rozwiąz równanie $(4-x)(x^{2}+2x-15)=0.$

Odpowiedzí:

Strona 14 z24

MD





Zadanie 29. $(0-2\rangle$

Dany jest trójkąt prostokątny $ABC$. Na przyprostokątnych $AC\mathrm{i}$ AB tego trójkąta obrano

odpowiednio punkty $D\mathrm{i}G$. Na przeciwprostokątnej $BC$ wyznaczono punkty $E\mathrm{i}F$ takie, $\dot{\mathrm{z}}\mathrm{e}$

$|\triangleleft DEC|=|\wedge BGF|=90^{\mathrm{o}}$ (zobacz rysunek). Wykaz, $\dot{\mathrm{z}}\mathrm{e}$ trójkąt $CDE$ jest podobny do

trójkąta FBG.
\begin{center}
\includegraphics[width=87.780mm,height=55.728mm]{./F2_M_PP_M2016_page14_images/image001.eps}
\end{center}
{\it C}

{\it E}

{\it F}

{\it D}

{\it A  G B}
\begin{center}
\includegraphics[width=96.012mm,height=17.784mm]{./F2_M_PP_M2016_page14_images/image002.eps}
\end{center}
Wypelnia

egzaminator

Nr zadania

Maks. liczba kt

28.

2

2

Uzyskana liczba pkt

IMA-IP

Strona 15 z24





Zadanie 30. (0-2)

Ciąg $(a_{n})$ jest określony wzorem $a_{n}=2n^{2}+2n$ dla $n\geq 1$. Wykaz, $\dot{\mathrm{z}}\mathrm{e}$ suma $\mathrm{k}\mathrm{a}\dot{\mathrm{z}}$ dych dwóch

kolejnych wyrazów tego ciągu jest kwadratem liczby naturalnej.

Strona 16 z24

MMA-IP





Zadanie $3l. (0-2)$

Skala Richtera słuz$\mathrm{y}$ do określania siły trzęsień ziemi. Siła ta opisana jest wzorem

$R=\displaystyle \log\frac{A}{4_{\mathfrak{c}}}$, gdzie $A$ oznacza amplitudę trzęsienia wyrazoną w centymetrach, $A_{0}=10^{\rightarrow\iota}$ cm

jest stałą, nazywaną amplitudą wzorcową. 5 maja 2014 roku w Taj1andii miało miejsce

trzęsienie ziemi o sile 6,2 w ska1i Richtera. Ob1icz amp1itudę trzęsienia ziemi w Taj1andii

i rozstrzygnij, czyjest ona większa, czy- mniejsza od 100 cm.

Odpowied $\acute{\mathrm{z}}$:
\begin{center}
\includegraphics[width=96.012mm,height=17.784mm]{./F2_M_PP_M2016_page16_images/image001.eps}
\end{center}
Wypelnia

egzaminator

Nr zadania

Maks. liczba kt

30.

2

31.

2

Uzyskana liczba pkt

IMA-IP

Strona 17 z24





Zadanie 32. $(0-4$

Jeden z kątów trójkąta jest trzy razy większy od mniejszego z dwóch pozostałych kątów,

które róznią się o $50^{\mathrm{o}}$. Oblicz kąty tego trójkąta.

Strona 18 z24

MMA-IP





Odpowiedzí :
\begin{center}
\includegraphics[width=82.044mm,height=17.784mm]{./F2_M_PP_M2016_page18_images/image001.eps}
\end{center}
Wypelnia

egzamÍnator

Nr zadania

Maks. liczba kt

32.

4

Uzyskana liczba pkt

IMA-IP

Strona 19 z24





Zadanie 33. $(0-5\rangle$

Podstawą ostrosłupa prawidłowego trójkątnego ABCS jest trójkąt równoboczny $ABC.$

Wysokość SO tego ostrosłupajest równa wysokościjego podstawy. Objętość tego ostrosiupa

jest równa 27. Ob1icz po1e powierzchni bocznej ostrosłupa ABCS oraz cosinus kąta, jaki

tworzą wysokość ściany bocznej i płaszczyzna podstawy ostrosłupa.

Strona 20 z24

MMA-IP





{\it BRUDNOPIS} ({\it nie podlega ocenie})

Strona 3 z24





Odpowiedzí :
\begin{center}
\includegraphics[width=82.044mm,height=17.784mm]{./F2_M_PP_M2016_page20_images/image001.eps}
\end{center}
Wypelnia

egzamÍnator

Nr zadania

Maks. liczba kt

33.

5

Uzyskana liczba pkt

IMA-IP

Strona 21 z24





Zadanie 34. $\zeta 0-4$)

Ze zbioru wszystkich liczb naturalnych dwucyfrowych losujemy kolejno dwa razy po jednej

liczbie bez zwracania. Oblicz prawdopodobieństwo zdarzenia polegającego na tym, $\dot{\mathrm{z}}\mathrm{e}$ suma

wylosowanych liczb będzie równa 30. Wynik zapisz w postaci ułamka zwykłego

nieskracalnego.

Strona 22 z24

MMA-IP





Odpowiedzí :
\begin{center}
\includegraphics[width=82.044mm,height=17.784mm]{./F2_M_PP_M2016_page22_images/image001.eps}
\end{center}
Wypelnia

egzamÍnator

Nr zadania

Maks. liczba kt

34.

4

Uzyskana liczba pkt

IMA-IP

Strona 23 z24





{\it BRUDNOPIS} ({\it nie podlega ocenie})

Strona 24 z24

MD





Zadanie 8. (0-1)

Danajest funkcja liniowa $f(x)=\displaystyle \frac{3}{4}x+6$. Miejscem zerowym tej funkcjijest liczba

A. 8

B. 6

C. $-6$

D. $-8$

Zadam$\mathrm{e}9.(0-1)$

Równanie wymietne $\displaystyle \frac{3x-1}{x+5}=3$, gdzie $x\neq-5,$

A.

B.

C.

D.

nie ma rozwiązań rzeczywistych.

ma dokładniejedno rozwiązanie rzeczywiste.

ma dokładnie dwa rozwiązania rzeczywiste.

ma dokładnie trzy rozwiązania rzeczywiste.

InformaCja do zadat 10. $i11.$

Na rysunku przedstawiony jest fragment paraboli będącej wykresem funkcji kwadratowej $f$

Wierzchołkiem tej parabolijest punkt $W=(1,9)$. Liczby $-2\mathrm{i}4$ to miejsca zerowe funkcji $f.$
\begin{center}
\includegraphics[width=192.228mm,height=118.164mm]{./F2_M_PP_M2016_page3_images/image001.eps}
\end{center}
Zadanie NO. (0-1)

Zbiorem wartości funkcji f jest przedział

A.

$(-\infty'-2\rangle$

B. $\langle-2,  4\rangle$

C.

$\langle 4,+\infty)$

D. $(-\infty$' $ 9\rangle$

Zadanie ll. $(0-1\rangle$

Najmniejsza wartość funkcji $f$ w przedziale $\langle-1,2\rangle$ jest równa

A. 2

B. 5

C. 8

D. 9

Strona 4 z24

MMA-IP





{\it BRUDNOPIS} ({\it nie podlega ocenie})

Strona 5 z24





Zadanie 12. $(0-1\rangle$

Funkcja $f$ określona jest wzorem $f(x)=\displaystyle \frac{2x^{3}}{x^{6}+1}$ dla $\mathrm{k}\mathrm{a}\dot{\mathrm{z}}$ dej liczby rzeczywistej $x$. Wtedy

$f(-\sqrt[3]{3})$ jest równa

A.

$-\displaystyle \frac{\sqrt[3]{9}}{2}$

B.

- -53

C.

-53

D.

$\displaystyle \frac{\sqrt[3]{3}}{2}$

Zadanie 13. $(0-\mathrm{f}\rangle$

$\mathrm{W}$ okręgu o środku w punkcie $S$ poprowadzono cięciwę AB, która utworzyła z promieniem

$AS$ kąt o mierze $31^{\mathrm{o}}$ (zobacz rysunek). Promień tego okręgu ma długość 10. Od1egłość punktu

$S$ od cięciwy $AB$ jest liczbą z przedziału

A. $\displaystyle \{\frac{9}{2},\frac{11}{2}\}$

B. $\displaystyle \frac{11}{2}, \displaystyle \frac{13}{2}$

C. $\displaystyle \frac{13}{2}, \displaystyle \frac{19}{2}$
\begin{center}
\includegraphics[width=72.588mm,height=76.200mm]{./F2_M_PP_M2016_page5_images/image001.eps}
\end{center}
$B$

{\it K}

{\it S}

31

{\it A}

$\displaystyle \frac{19}{2}, \displaystyle \frac{37}{2}\}$

D.

Zadanie 14. $(0-1\rangle$

Cztetnasty wyraz ciągu arytmetycznegojest równy 8, a róznica tego ciągujest równa $(-\displaystyle \frac{3}{2}).$

Siódmy wyraz tego ciągujest równy

A.

$\displaystyle \frac{37}{2}$

B.

$-\displaystyle \frac{37}{2}$

C.

- -25

D.

-25

Zadanie 15. (0-1)

Ciąg $(x,2x+3,4x+3)$ jest geometryczny. Pierwszy wyraz tego ciągu jest równy

A. $-4$

B. l

C. 0

D. $-1$

Zadanie $l6. (0-1\rangle$

Przedstawione na rysunku trójkąty $ABC\mathrm{i}PQR$ są podobne. Bok $AB$ trójkąta $ABC$ ma długość

A. 8

B. 8,5

C. 9,5
\begin{center}
\includegraphics[width=105.660mm,height=60.456mm]{./F2_M_PP_M2016_page5_images/image002.eps}
\end{center}
18

{\it Q} $62^{\mathrm{o}} R$

{\it C}

17

9

$70^{\mathrm{o}}$

$70^{\mathrm{o}}  48^{\mathrm{o}}$

{\it A B}

{\it x  P}

D. 10

Strona 6 z24

MMA-IP





{\it BRUDNOPIS} ({\it nie podlega ocenie})

Strona 7 z24





Zadanie 17. $(0-1\rangle$

Kąt $\alpha$ jest ostry i $\displaystyle \mathrm{t}\mathrm{g}\alpha=\frac{2}{3}$. Wtedy

A.

$\mathrm{s}$i$\displaystyle \mathrm{n}\alpha=\frac{3\sqrt{13}}{26}$

B.

$\mathrm{s}$i$\displaystyle \mathrm{n}\alpha=\frac{\sqrt{13}}{13}$

C.

$\displaystyle \sin\alpha=\frac{2\sqrt{13}}{13}$

D.

$\mathrm{s}$i$\displaystyle \mathrm{n}\alpha=\frac{3\sqrt{13}}{13}$

Zadanie 18. (0-1)

$\mathrm{Z}$ odcinków o długościach: 5, $2a+1, a-1$ mozna zbudować trójkąt równoramienny. Wynika

stąd, $\dot{\mathrm{z}}\mathrm{e}$

A. $a=6$

B. $a=4$

C. $a=3$

D. $a=2$

ZadanÎe $l9. (0-1)$

Okręgi o promieniach 3 $\mathrm{i} 4$ są styczne zewnętrznie. Prosta styczna do okręgu

o promieniu 4 w punkcie $P$ przechodzi przez środek okręgu o promieniu 3 (zobacz rysunek).
\begin{center}
\includegraphics[width=171.504mm,height=116.184mm]{./F2_M_PP_M2016_page7_images/image001.eps}
\end{center}
{\it P}

$O_{1}$  3 4  $O_{2}$

Pole trójkąta, którego wierzchołkami są środki okręgów i punkt styczności P, jest równe

A. 14

B. $2\sqrt{33}$

C. $4\sqrt{33}$

D. 12

Zadanie 20. $(0-1\rangle$

Proste opisane równaniami $y=\displaystyle \frac{2}{m-1}x+m-2$ oraz $y=mx+\displaystyle \frac{1}{m+1}$ są prostopadłe, gdy

A. $m=2$

B.

{\it m}$=$ -21

C.

{\it m}$=$ -31

D. $m=-2$

Strona 8 z 24

MMA-IP





{\it BRUDNOPIS} ({\it nie podlega ocenie})

Strona 9 z24





Zadanie 21. $(0-\mathrm{f}\rangle$

$\mathrm{W}$ układzie współrzędnych dane są punkty $A=(a,6)$ oraz $B=(7,b)$. Środkiem odcinka $AB$

jest punkt $M=(3,4)$. Wynika stąd, $\dot{\mathrm{z}}\mathrm{e}$

A. $a=5 \mathrm{i}b=5$

B. $a=-1 \mathrm{i}b=2$

C. $a=4\mathrm{i}b=10$

D. $a=-4 \mathrm{i}b=-2$

Zadanie 32. (0-1)

Rzucamy trzy razy symetryczną monetą. Niech p oznacza prawdopodobieństwo otrzymania

dokładnie dwóch orłów w tych trzech rzutach. Wtedy

A. $0\leq p<0,2$

B. $0,2\leq p\leq 0,35$

C. $0,35<p\leq 0,5$

D. $0,5<p\leq 1$

Zadanie 23. $(0-1\rangle$

Kąt rozwarcia stozka ma miarę $120^{\mathrm{o}}$, a tworząca tego stozka ma długość 4. Objętość tego

stozkajest równa

A. $ 36\pi$

B. $ 18\pi$

C. $ 24\pi$

D. $ 8\pi$

Zadanie 24. (0-1)

Przekątna podstawy graniastosłupa prawidłowego czworokątnego jest dwa razy dłuzsza od

wysokości graniastosłupa. Graniastosłup przecięto płaszczyzną przechodzącą przez przekątną

podstawy ijeden wierzchołek drugiej podstawy (patrz rysunek).

Płaszczyzna przekroju tworzy z podstawą graniastosłupa kąt $\alpha$ o mierze

A. $30^{\mathrm{o}}$

B. $45^{\mathrm{o}}$

C. $60^{\mathrm{o}}$

D. $75^{\mathrm{o}}$

Zadanie 25. $(0-1\rangle$

Średnia arytmetyczna szeŚciu liczb naturalnych: 31, 16, 25, 29, 27, $x$, jest równa $\displaystyle \frac{x}{2}$. Mediana

tych liczb jest równa

A. 26

B. 27

C. 28

D. 29

Strona 10 z24

MMA-IP







$\mathrm{g}_{\mathrm{E}\mathrm{G}\mathrm{Z}\mathrm{A}\mathrm{M}\mathrm{I}\mathrm{N}\mathrm{A}\subset \mathrm{Y}\mathrm{J}\mathrm{N}\mathrm{A}}^{\mathrm{C}\mathrm{E}\mathrm{N}\mathrm{T}\mathrm{R}\mathrm{A}\mathrm{L}\mathrm{N}\mathrm{A}}\mathrm{K}\mathrm{O}\mathrm{M}1\mathrm{S}\mathrm{J}\mathrm{A}$

Arkusz zawiera informacje

prawnie chronione do momentu

rozpoczęcia egzaminu.

UZUPELNIA ZDAJACY

{\it miejsce}

{\it na naklejkę}
\begin{center}
\includegraphics[width=21.900mm,height=16.056mm]{./F2_M_PP_M2017_page0_images/image001.eps}
\end{center}
KOD
\begin{center}
\includegraphics[width=79.656mm,height=16.104mm]{./F2_M_PP_M2017_page0_images/image002.eps}
\end{center}
PESEL
\begin{center}
\includegraphics[width=195.684mm,height=235.608mm]{./F2_M_PP_M2017_page0_images/image003.eps}
\end{center}
EGZAMIN MATU  LNY

Z MATEMATY

POZIOM PODSTAWOWY

DATA: 5 maja 2017 r.

LICZBA P KTÓW DO UZYS NIA: 50

Instrukcja dla zdającego

1. Sprawd $\acute{\mathrm{z}}$, czy arkusz egzaminacyjny zawiera 26 stron (zadania $1-34$).

Ewentualny brak zgłoś przewodniczącemu zespo nadzorującego

egzamin.

2. Rozwiązania zadań i odpowiedzi wpisuj w miejscu na to przeznaczonym.

3. Odpowiedzi do zadań za ię ch $(1-25)$ zaznacz na karcie odpowiedzi,

w części ka $\mathrm{y}$ przeznaczonej dla zdającego. Zamaluj $\blacksquare$ pola do tego

przeznaczone. Błędne zaznaczenie otocz kólkiem \copyright i zaznacz wlaściwe.

4. Pamiętaj, $\dot{\mathrm{z}}\mathrm{e}$ pominięcie argumentacji lub istotnych obliczeń

w rozwiązaniu zadania otwa ego (26-34) $\mathrm{m}\mathrm{o}\dot{\mathrm{z}}\mathrm{e}$ spowodować, $\dot{\mathrm{z}}\mathrm{e}$ za to

rozwiązanie nie otrzymasz pelnej liczby pu tów.

5. Pisz cz elnie i $\mathrm{u}\dot{\mathrm{z}}$ aj lko $\mathrm{d}$ gopisu lub pióra z czarnym tuszem lub

atramentem.

6. Nie $\mathrm{u}\dot{\mathrm{z}}$ aj korektora, a błędne zapisy $\mathrm{r}\mathrm{a}\acute{\mathrm{z}}\mathrm{n}\mathrm{i}\mathrm{e}$ prze eśl.

7. Pamiętaj, $\dot{\mathrm{z}}\mathrm{e}$ zapisy w brudnopisie nie będą oceniane.

8. $\mathrm{M}\mathrm{o}\dot{\mathrm{z}}$ esz korzystać z zesta wzorów matema cznych, cyrkla i linijki,

a ta $\mathrm{e}$ z kalkulatora prostego.

9. Na tej stronie oraz na karcie odpowiedzi wpisz swój numer PESEL

i przyklej naklejkę z kodem.

10. Nie wpisuj $\dot{\mathrm{z}}$ adnych znaków w części przeznaczonej dla egzaminatora.

$\Vert\Vert\Vert\Vert\Vert\Vert\Vert\Vert\Vert\Vert\Vert\Vert\Vert\Vert\Vert\Vert\Vert\Vert\Vert\Vert\Vert\Vert\Vert\Vert|$

$\mathrm{M}\mathrm{M}\mathrm{A}-\mathrm{P}1_{-}1\mathrm{P}-172$

Układ graficzny

\copyright CKE 2015




{\it Wzadaniach od l. do 25. wybierz i zaznacz na karcie odpowiedzi poprawnq odpowiedzí}.

Zadanie l. $(0-l)$

Liczba $5^{8}\cdot 16^{-2}$ jest równa

A. $(\displaystyle \frac{5}{2})^{8}$ B.

-25

Zadanie 2. (0-1)

Liczba $\sqrt[3]{54}-\sqrt[3]{2}$ jest równa

A. $\sqrt[3]{52}$ B. 3

Zadanie 3. $(0-l\rangle$

Liczba 2 $\log_{2}3-2\log_{2}5$ jest równa

A.

$\displaystyle \log_{2}\frac{9}{25}$

B.

$\log_{2} \displaystyle \frac{3}{5}$

C. $10^{8}$

D. 10

C. $2\sqrt[3]{2}$

D. 2

C.

$\log_{2} \displaystyle \frac{9}{5}$

D.

$\displaystyle \log_{2}\frac{6}{25}$

Zadanie 4. (0-1)

Liczba osobników pewnego zagrozonego wyginięciem gatunku zwierząt wzrosła w stosunku

do liczby tych zwierząt z 31 grudnia 2011 r. 0120\% i obecnie jest równa 8910. I1e zwierząt

liczyła populacja tego gatunku w ostatnim dniu 2011 roku?

A. 4050

B. 1782

C. 7425

D. 7128

Zadanie 5. (0-1)

Równość $(x\sqrt{2}-2)^{2}=(2+\sqrt{2})^{2}$ jest

A. prawdziwa dla $x=-\sqrt{2}.$

B. prawdziwa dla $x=\sqrt{2}.$

C. prawdziwa dla $x=-1.$

D. fałszywa dla $\mathrm{k}\mathrm{a}\dot{\mathrm{z}}$ dej liczby $x.$

Strona 2 z 26

MMA-II





{\it BRUDNOPIS} ({\it nie podlega ocenie})

Strona ll z 26





Zadanie $l8. (0-1\rangle$

Na rysunku przedstawiona jest prosta $k$, przechodząca przez punkt $A=(2,-3)$ i przez

początek układu współrzędnych, oraz zaznaczonyjest kąt $\alpha$ nachylenia tej prostej do osi $Ox.$
\begin{center}
\includegraphics[width=70.716mm,height=67.464mm]{./F2_M_PP_M2017_page11_images/image001.eps}
\end{center}
{\it k}

{\it y}

5

4

3

2

1

$\alpha$

{\it x}

$-5$ -$4  -3$ -$2$

$-1 0$ 1

$-1$

2 3  4 5

$-2$

$-3  -A$

$-4$

Zatem

A.

$\displaystyle \mathrm{t}\mathrm{g}\alpha=-\frac{2}{3}$

B.

$\displaystyle \mathrm{t}\mathrm{g}\alpha=-\frac{3}{2}$

C.

$\displaystyle \mathrm{t}\mathrm{g}\alpha=\frac{2}{3}$

D.

$\displaystyle \mathrm{t}\mathrm{g}\alpha=\frac{3}{2}$

Zadanie 19. (0-1)

Na płaszczyzínie z układem współrzędnych proste $k\mathrm{i} l$ przecinają się pod kątem prostym

w punkcie $A=(-2,4)$. Prosta $k$ jest określona równaniem $y=-\displaystyle \frac{1}{4}x+\frac{7}{2}$ Zatem prostą $l$

opisuje równanie

A.

{\it y}$=$ -41 {\it x}$+$ -27

B.

{\it y}$=$- -41 {\it x}- -27

C. $y=4x-12$

D. $y=4x+12$

Zadanie 20. $(0-1\rangle$

Dany jest okrąg o środku $S=(2,3)$ i promieniu $r=5$. Który z podanych punktów lezy na

tym okręgu?

A. $A=(-1,7)$

B. $B=(2,-3)$

C. $C=(3,2)$

D. $D=(5,3)$

Zadanie 21. $\zeta 0-1\rangle$

Pole powierzchni całkowitej graniastosłupa prawidłowego czworokątnego, w którym

wysokość jest 3 razy dłuzsza od krawędzi podstawy, jest równe 140. Zatem krawędzí

podstawy tego graniastosłupajest równa

A. $\sqrt{10}$

B. $3\sqrt{10}$

C. $\sqrt{42}$

D. $3\sqrt{42}$

Strona 12 z 26

MMA-II





{\it BRUDNOPIS} ({\it nie podlega ocenie})

Strona 13 z 26





Zadanie 22. $(0-1\rangle$

Promień AS podstawy walca jest równy wysokości $OS$ tego walca. Sinus kąta $OAS$ (zobacz

rysunek) jest równy
\begin{center}
\includegraphics[width=49.332mm,height=39.924mm]{./F2_M_PP_M2017_page13_images/image001.eps}
\end{center}
{\it O}

$\nearrow$

{\it S}

{\it A}

A.

-$\sqrt{}$23

B.

-$\sqrt{}$22

C.

-21

D. l

Zadanie 23. $(0-l\rangle$

Dany jest stozek o wysokości 4 i średnicy podstawy 12. Objętość tego stozkajest równa

A. $ 576\pi$

B. $ 192\pi$

C. $ 144\pi$

D. $ 48\pi$

Zadanie 24. $(0-1\rangle$

Średnia arytmetyczna oŚmiu liczb: 3, 5, 7, 9, $x$, 15, 17, $19$jest równa ll. Wtedy

A. $x=1$

B. $x=2$

C. $x=11$

D. $x=13$

Zadanie 25. (0-1)

Ze zbioru dwudziestu czterech kolejnych liczb naturalnych od l do 241osujemy jedną 1iczbę.

Niech $A$ oznacza zdarzenie, $\dot{\mathrm{z}}\mathrm{e}$ wylosowana liczba będzie dzielnikiem liczby 24. Wtedy

prawdopodobieństwo zdarzenia $A$ jest równe

A.

-41

B.

-31

C.

-81

D.

-61

Strona 14 z26

MMA-II





{\it BRUDNOPIS} ({\it nie podlega ocenie})

Strona 15 z 26





Zadanie 26. (0-2)

Rozwiąz nierówność $8x^{2}-72x\leq 0.$

Odpowiedzí

Strona 16 z 26

n





Zadanie 27. (0-2)

Wykaz, $\dot{\mathrm{z}}\mathrm{e}$ liczba $4^{2017}+4^{2018}+4^{2019}+4^{2020}$ jest podzielna przez 17.
\begin{center}
\includegraphics[width=96.012mm,height=17.784mm]{./F2_M_PP_M2017_page16_images/image001.eps}
\end{center}
Wypelnia

egzaminator

Nr zadania

Maks. liczba kt

2

27.

2

Uzyskana liczba pkt

MMA-IP

Strona 17 z 26





Zadanie 28. $(0-2\rangle$

Dane są dwa okręgi o środkach w punktach $P \mathrm{i} R$, styczne zewnętrznie w punkcie $C.$

Prosta $AB$ jest styczna do obu okręgów odpowiednio w punktach $A \mathrm{i}B$ oraz $|<APC|=\alpha$

$\mathrm{i}|\triangleleft ABC|=\beta$ (zobacz rysunek). Wykaz, $\dot{\mathrm{z}}\mathrm{e}\alpha=180^{\mathrm{o}}-2\beta.$

{\it P}
\begin{center}
\includegraphics[width=190.908mm,height=41.148mm]{./F2_M_PP_M2017_page17_images/image001.eps}
\end{center}
$\alpha$  {\it C  R}

$(\beta$

{\it A  B}

Strona 18 z 26

MMA-I]





Zadanie 29. (0-4)

Funkcja kwadratowa $f$ jest określona dla wszystkich liczb rzeczywistych $x$ wzorem

$f(x)=ax^{2}+bx+c$. Największa wartość funkcji $f$ jest równa 6 oraz $f(-6)=f(0)=\displaystyle \frac{3}{2}.$

Oblicz wartość współczynnika $a.$

Odpowied $\acute{\mathrm{z}}.$
\begin{center}
\includegraphics[width=96.012mm,height=17.832mm]{./F2_M_PP_M2017_page18_images/image001.eps}
\end{center}
Wypelnia

egzaminator

Nr zadania

Maks. liczba kt

28.

2

4

Uzyskana liczba pkt

MMA-IP

Strona 19 z26





$\mathrm{Z}\mathrm{a}\mathrm{d}\mathrm{a}\mathrm{n}\ddagger \mathrm{e}30. (0-2)$

Przeciwprostokątna trójkąta prostokątnego ma długość 26 cm, a jedna z przyprostokątnych

jest o 14 cm diuzsza od diugiej. Ob1icz obwód tego trójkąta.

Odpowiedzí

Strona 20 z 26

MMA-II





{\it BRUDNOPIS} ({\it nie podlega ocenie})

Strona 3 z26





Zadanie $3l. (0-2)$

$\mathrm{W}$ ciągu arytmetycznym $(a_{n})$, określonym dla $n\geq 1$, dane są: wyraz $a_{1}=8$ i suma trzech

początkowych wyrazów tego ciągu $S_{3}=33$. Oblicz róznicę $a_{16}-a_{13}.$

Odpowied $\acute{\mathrm{z}}$
\begin{center}
\includegraphics[width=96.012mm,height=17.832mm]{./F2_M_PP_M2017_page20_images/image001.eps}
\end{center}
Wypelnia

egzaminator

Jaks. liczba kt

30.

2

31.

2

Jzyskana liczba pkt

MMA-IP

Strona 21 z 26





Zadanie 32. (0-5)

Dane są punkty $A=(-4,0) \mathrm{i}M=(2,9)$ oraz prosta $k$ o równaniu $y=-2x+10$. Wierzchołek

$B$ trójkąta $ABC$ to punkt przecięcia prostej $k$ z osią $Ox$ układu współrzędnych, a wierzchołek

$C$ jest punktem przecięcia prostej $k$ z prostą AM. Oblicz pole trójkąta $ABC.$

Odpowiedzí

Strona 22 z 26

MMA-I]





Zadanie 33. (0-2)

Ze zbioru wszystkich liczb naturalnych dwucyfrowych losujemy jedną liczbę. Oblicz

prawdopodobieństwo zdarzenia, $\dot{\mathrm{z}}\mathrm{e}$ wylosujemy liczbę, która jest równocześnie mniejsza od

40 i podzielna przez 3. Wynik zapisz w postaci ułamka zwykłego nieskracalnego.

Odpowiedzí
\begin{center}
\includegraphics[width=96.012mm,height=17.784mm]{./F2_M_PP_M2017_page22_images/image001.eps}
\end{center}
WypelnÍa

egzaminator

Nr zadania

Maks. liczba kt

32.

5

33.

2

Uzyskana liczba pkt

MMA-IP

Strona 23 z 26





Zadanie 34. (0-4)

$\mathrm{W}$ ostrosłupie prawidłowym trójkątnym wysokość ściany bocznej prostopadła do krawędzi

podstawy ostrosłupa jest równa $\displaystyle \frac{5\sqrt{3}}{4}$, a pole powierzchni bocznej tego ostrosłupa jest

równe $\displaystyle \frac{15\sqrt{3}}{4}$. Oblicz objętość tego ostrosłupa.

Strona 24 z 26

MMA-II





Odpowied $\acute{\mathrm{z}}$
\begin{center}
\includegraphics[width=82.044mm,height=17.832mm]{./F2_M_PP_M2017_page24_images/image001.eps}
\end{center}
Wypelnia

egzaminator

Nr zadania

Maks. liczba kt

34.

4

Uzyskana liczba pkt

MMA-IP

Strona 25 z 26





{\it BRUDNOPIS} ({\it nie podlega ocenie})

Strona 26 z 26

n





Zadanie 6. (0-1)

Do zbioru rozwiązań nierówności $(x^{4}+1)(2-x)>0$ nie nalez$\mathrm{v}$ liczba

A. $-3$

B. $-1$

C. l

D. 3

Zadailie 7. (0-1)

Wskaz rysunek, na którym jest przedstawiony zbiór wszystkich rozwiązań nierówności

$2-3x\geq 4.$

A.
\begin{center}
\includegraphics[width=168.000mm,height=17.784mm]{./F2_M_PP_M2017_page3_images/image001.eps}
\end{center}
-23  {\it x}

B.
\begin{center}
\includegraphics[width=168.000mm,height=17.724mm]{./F2_M_PP_M2017_page3_images/image002.eps}
\end{center}
-23  {\it x}

C.
\begin{center}
\includegraphics[width=168.048mm,height=17.880mm]{./F2_M_PP_M2017_page3_images/image003.eps}
\end{center}
- -23  {\it x}

D.
\begin{center}
\includegraphics[width=168.048mm,height=17.832mm]{./F2_M_PP_M2017_page3_images/image004.eps}
\end{center}
- -23  {\it x}

Zadanie 8. $(0-l)$

Równanie $x(x^{2}-4)(x^{2}+4)=0$ z niewiadomą $x$

A. nie ma rozwiązań w zbiorze liczb rzeczywistych.

B. ma dokładnie dwa rozwiązania w zbiorze liczb rzeczywistych.

C. ma dokładnie trzy rozwiązania w zbiorze liczb rzeczywistych.

D. ma dokładnie pięć rozwiązań w zbiorze liczb rzeczywistych.

Zadanie $g. (0-1)$

Miejscem zerowym funkcji liniowej $f(x)=\sqrt{3}(x+1)-12$ jest liczba

A. $\sqrt{3}-4$

B. $-2\sqrt{3}+1$

C. $4\sqrt{3}-1$

D. $-\sqrt{3}+12$

Strona 4 z 26

MMA-I]





{\it BRUDNOPIS} ({\it nie podlega ocenie})

Strona 5 z26





Zadanie 10. $(0-1\rangle$

Na rysunku przedstawiono fragment wykresu

której miejsca zerowe to: $-3 \mathrm{i}1.$

funkcji kwadratowej $f(x)=ax^{2}+bx+c,$
\begin{center}
\includegraphics[width=86.004mm,height=100.380mm]{./F2_M_PP_M2017_page5_images/image001.eps}
\end{center}
{\it 5y}

)4

3

2

1

{\it x}

$-5$ -$4  -3 -2  0 \xi$

$-2$

$\rightarrow 3$

$-4$

Współczynnik c we wzorze funkcji f jest równy

A. l

B. 2

C. 3

D. 4

Zadanie 11. (0-1)

Na rysunku przedstawiono fragment wykresu funkcji wykładniczej $f$ określonej wzorem

$f(x)=a^{x}$. Punkt $A=(1,2)$ nalezy do tego wykresu ffinkcji.
\begin{center}
\includegraphics[width=143.460mm,height=75.588mm]{./F2_M_PP_M2017_page5_images/image002.eps}
\end{center}
Podstawa a potęgijest równa

A.

- -21

B.

-21

C. $-2$

D. 2

Strona 6 z26

MMA-II





{\it BRUDNOPIS} ({\it nie podlega ocenie})

Strona 7 z26





Zadanie 12. $(0-1\rangle$

$\mathrm{W}$ ciągu arytmetycznym $(a_{n})$, określonym dla $n\geq 1$, dane są: $a_{1}=5, a_{2}=11$. Wtedy

A. $a_{14}=71$

B. $a_{12}=71$

C. $a_{11}=71$

D. $a_{10}=71$

Zadanie 13. (0-1)

Danyjest trzywyrazowy ciąg geometryczny $($24, 6, $a-1)$. Stąd wynika, $\dot{\mathrm{z}}\mathrm{e}$

A.

{\it a}$=$ -25

B.

{\it a}$=$ -25

C.

{\it a}$=$ -23

D.

{\it a}$=$ -23

Zadanie $l4\cdot(0-1)$

Jeśli $m=\sin 50^{\mathrm{o}}$, to

A. $m=\sin 40^{\mathrm{o}}$

B. $m=\cos 40^{\mathrm{o}}$

C. $m=\cos 50^{\mathrm{o}}$

D. $m=\mathrm{t}\mathrm{g}50^{\mathrm{o}}$

Zadanie 15. (0-1)

Na okręgu o środku w punkcie O lezy punkt C (zobacz rysunek). Odcinek AB jest średnicą

tego okręgu. Zaznaczony na rysunku kąt środkowy a ma miarę
\begin{center}
\includegraphics[width=70.260mm,height=66.600mm]{./F2_M_PP_M2017_page7_images/image001.eps}
\end{center}
{\it C}

$56^{\mathrm{o}}$

{\it A}

$\alpha$

{\it O}

{\it B}

A. $116^{\mathrm{o}}$

B. $114^{\mathrm{o}}$

C. $112^{\mathrm{o}}$

D. $110^{\mathrm{o}}$

Strona 8 z 26

MMA-II





{\it BRUDNOPIS} ({\it nie podlega ocenie})

Strona 9 z26





Zadanie $l6. (0-1\rangle$

$\mathrm{W}$ trójkącie $ABC$ punkt $D$ lezy na boku $BC$, a punkt $E$ lezy na boku $AB$. Odcinek $DE$ jest

równoległy do boku $AC$, a ponadto $|BD|=10, |BC|=12 \mathrm{i}|AC|=24$ (zobacz rysunek).
\begin{center}
\includegraphics[width=117.804mm,height=49.020mm]{./F2_M_PP_M2017_page9_images/image001.eps}
\end{center}
{\it B}

10

{\it D}

2

{\it C}

{\it E}

{\it A}

24

B. 20

A. 22

DługoŚć odcinka DE jest równa

C. 12

D. ll

Zadanie 17. $(0-l\rangle$

Obwód trójkąta $ABC$, przedstawionego na rysunku, jest równy

A. $(3+\displaystyle \frac{\sqrt{3}}{2})a$
\begin{center}
\includegraphics[width=78.132mm,height=48.816mm]{./F2_M_PP_M2017_page9_images/image002.eps}
\end{center}
{\it C}

{\it a}

$30^{\mathrm{o}}$

{\it A  B}

C. $(3+\sqrt{3})a$

B. $(2+\displaystyle \frac{\sqrt{2}}{2})a$

D. $(2+\sqrt{2})a$

Strona 10 z 26

MMA-I]







$\mathrm{g}_{\mathrm{E}\mathrm{G}\mathrm{Z}\mathrm{A}\mathrm{M}\mathrm{I}\mathrm{N}\mathrm{A}\mathrm{C}\mathrm{Y}\mathrm{J}\mathrm{N}\mathrm{A}}^{\mathrm{C}\mathrm{E}\mathrm{N}\mathrm{T}\mathrm{R}\mathrm{A}\mathrm{L}\mathrm{N}\mathrm{A}}$KOMISJA

Arkusz zawiera informacje

prawnie chronione do momentu

rozpoczęcia egzaminu.

UZUPELNIA ZDAJACY

{\it miejsce}

{\it na naklejkę}
\begin{center}
\includegraphics[width=21.900mm,height=16.104mm]{./F2_M_PP_M2018_page0_images/image001.eps}
\end{center}
KOD
\begin{center}
\includegraphics[width=79.608mm,height=16.104mm]{./F2_M_PP_M2018_page0_images/image002.eps}
\end{center}
PESEL
\begin{center}
\includegraphics[width=193.548mm,height=250.644mm]{./F2_M_PP_M2018_page0_images/image003.eps}
\end{center}
EGZAMIN MATU  LNY

Z MATEMATY

POZIOM PODSTAWOWY

DATA: 7 maja 2018 $\mathrm{r}.$

LICZBA P KTÓW DO UZYS NIA: 50

Instrukcja dla zdającego

1.

2.

3.

4.

5.

Sprawdzí, czy arkusz egzaminacyjny zawiera 26 stron (zadania $1-34$).

Ewentualny brak zgłoś przewodniczącemu zespo nadzo jącego

egzamin.

Rozwiązania zadań i odpowiedzi wpisuj w miejscu na to przeznaczonym.

Odpowiedzi do zadań za iętych $(1-25)$ zaznacz na karcie odpowiedzi,

w części ka przeznaczonej dla zdającego. Zamaluj $\blacksquare$ pola do tego

przeznaczone. Błędne zaznaczenie otocz kółkiem \copyright i zaznacz właściwe.

Pamiętaj, $\dot{\mathrm{z}}\mathrm{e}$ pominięcie argumentacji lub istotnych obliczeń

w rozwiązaniu zadania o a ego (26-34) $\mathrm{m}\mathrm{o}\dot{\mathrm{z}}\mathrm{e}$ spowodować, $\dot{\mathrm{z}}\mathrm{e}$ za to

rozwiązanie nie otrzymasz pełnej liczby pu tów.

Pisz czytelnie i $\mathrm{u}\dot{\mathrm{z}}$ aj tylko $\mathrm{d}$ gopisu lub pióra z czatnym tuszem lub

atramentem.

6. Nie $\mathrm{u}\dot{\mathrm{z}}$ aj korektora, a błędne zapisy $\mathrm{r}\mathrm{a}\acute{\mathrm{z}}\mathrm{n}\mathrm{i}\mathrm{e}$ prze eśl.

7. Pamiętaj, $\dot{\mathrm{z}}\mathrm{e}$ zapisy w brudnopisie nie będą oceniane.

8. $\mathrm{M}\mathrm{o}\dot{\mathrm{z}}$ esz korzystać z zesta wzorów matema cznych, cyrkla i linijki,

a ta $\mathrm{e}$ z kalkulatora prostego.

9. Na tej stronie oraz na karcie odpowiedzi wpisz swój numer PESEL

i przyklej naklejkę z kodem.

10. Nie wpisuj $\dot{\mathrm{z}}$ adnych znaków w części przeznaczonej dla egzaminatora.

$\Vert\Vert\Vert\Vert\Vert\Vert\Vert\Vert\Vert\Vert\Vert\Vert\Vert\Vert\Vert\Vert\Vert\Vert\Vert\Vert\Vert\Vert\Vert\Vert|$

$\mathrm{M}\mathrm{M}\mathrm{A}-\mathrm{P}1_{-}1\mathrm{P}-1\mathrm{S}2$
\begin{center}
\includegraphics[width=22.908mm,height=19.200mm]{./F2_M_PP_M2018_page0_images/image004.eps}
\end{center}
$1$ :

Układ graficzny

\copyright CKE 2015




{\it W kazdym z zadań od l. do 25. wybierz i zaznacz na karcie odpowiedzi poprawnq odpowiedzí}.

Zadanie 1. (0-1)

Liczba 2 $\log_{3}6-\log_{3}4$ jest równa

A. 4

B. 2

Zadanie 2. $(0-1\rangle$

Liczba $\sqrt[3]{\frac{7}{3}}\cdot\sqrt[3]{\frac{81}{56}}$ jest równa

A.

-$\sqrt{}$23

B.

$\displaystyle \frac{3}{2\sqrt[3]{21}}$

C. $2\log_{3}2$

D. $\log_{3}8$

C.

-23

D.

-49

Zadanie 3. (0-1)

Dane są liczby $a=3,6\cdot 10^{-12}$ oraz $b=2,4\cdot 10^{-20}$. Wtedy iloraz $\displaystyle \frac{a}{b}$ jest równy

A. $8,64\cdot 10^{-32}$

B. $1,5\cdot 10^{-8}$

C. $1,5\cdot 10^{8}$

D. $8,64\cdot 10^{32}$

Zadanie 4. (0-1)

Cena roweru po obnizce o 15\% była równa 850 zł. Przed obnizką ten rower kosztował

A. 865,00 zł

B. 850,15 zł

C. 1000,00 zł

D. 977,50 zł

Zadanie 5. (0-1)

Zbiorem wszystkich rozwiązań nierówności $\displaystyle \frac{1-2x}{2}>\frac{1}{3}$ jest przedział

A.

(-$\infty$' -61)

B.

(-$\infty$' -23)

C.

$(\displaystyle \frac{1}{6},+\infty)$

D.

$(\displaystyle \frac{2}{3},+\infty)$

Zadanie 6. $(0-l)$

Funkcja kwadratowa jest określona wzorem

róznymi miejscami zerowymi ffinkcji $f$ Zatem

$f(x)=-2(x+3)(x-5)$. Liczby

$x_{1}, x_{2}$ są

A. $x_{1}+x_{2}=-8$

B. $x_{1}+x_{2}=-2$

C. $x_{1}+x_{2}=2$

D. $x_{1}+x_{2}=8$

Strona 2 z26

MMA-IP





{\it BRUDNOPIS} ({\it nie podlega ocenie})

$\mathrm{A}_{-}1\mathrm{P}$

Strona ll z26





Zadanie 23. $(0-l)$

$\mathrm{W}$ zestawie $\displaystyle \frac{2,2,2,\ldots,2}{m1\mathrm{i}\mathrm{c}\mathrm{z}\mathrm{b}}\frac{4,4,4,\ldots,4}{m1\mathrm{i}\mathrm{c}\mathrm{z}\mathrm{b}}$ jest $2m$ liczb $(m\geq 1)$ ` w tym $m$ liczb 2 $\mathrm{i} m$ liczb 4.

Odchylenie standardowe tego zestawu liczb jest równe

A. 2

B. l

C.

-$\sqrt{}$12

D. $\sqrt{2}$

Zadanie 24. $(0-l)$

Ile jest wszystkich liczb naturalnych czterocyfrowych mniejszych od 2018 i podzie1nych

przez 5?

A. 402

B. 403

C. 203

D. 204

Zadanie 25. $(0-l)$

$\mathrm{W}$ pudełku jest 50 kuponów, wśród których jest 15 kuponów przegrywających, a pozostałe

kupony są wygrywające. $\mathrm{Z}$ tego pudełka w sposób losowy wyciągamy jeden kupon.

Prawdopodobieństwo zdarzenia polegającego na tym, $\dot{\mathrm{z}}\mathrm{e}$ wyciągniemy kupon wygrywający, jest

równe

A.

$\displaystyle \frac{15}{35}$

B.

$\displaystyle \frac{1}{50}$

C.

$\displaystyle \frac{15}{50}$

D.

$\displaystyle \frac{35}{50}$

Strona 12 z26

MMA-IP





{\it BRUDNOPIS} ({\it nie podlega ocenie})

$\mathrm{A}_{-}1\mathrm{P}$

Strona 13 z26





Zadanie 26. $(0\rightarrow 2\rangle$

Rozwiąz nierówność $2x^{2}-3x>5.$

Odpowiedzí :

Strona 14 z26

MMA-I]





Zadanie 27. (0-2)

Rozwiąz równanie $(x^{3}+125)(x^{2}-64)=0.$

Odpowied $\acute{\mathrm{z}}$:
\begin{center}
\includegraphics[width=96.012mm,height=17.784mm]{./F2_M_PP_M2018_page14_images/image001.eps}
\end{center}
Wypelnia

egzaminator

Nr zadania

Maks. liczba kt

2

27.

2

Uzyskana liczba pkt

MMA-IP

Strona 15 z26





Zadanie 2@. (0-2)

Udowodnij, $\dot{\mathrm{z}}\mathrm{e}$ dla dowolnych liczb dodatnich $a, b$ prawdziwajest nierówność

$\displaystyle \frac{1}{2a}+\frac{1}{2b}\geq\frac{2}{a+b}.$

Strona 16 z26

MMA-I]





Zadanie 29. (0-2)

Okręgi o środkach odpowiednio $A\mathrm{i}B$ są styczne zewnętrznie i $\mathrm{k}\mathrm{a}\dot{\mathrm{z}}\mathrm{d}\mathrm{y}$ z nich jest styczny do

obu ramion danego kąta prostego (zobacz rysunek). Promień okręgu o środku $A$ jest równy 2.

{\it A}.

{\it B}.

Uzasadnij, $\dot{\mathrm{z}}\mathrm{e}$ promień okręgu o środku $B$ jest mniejszy od $\sqrt{2}-1.$
\begin{center}
\includegraphics[width=96.012mm,height=17.832mm]{./F2_M_PP_M2018_page16_images/image001.eps}
\end{center}
Wypelnia

egzaminator

Nr zadania

Maks. liczba kt

28.

2

2

Uzyskana liczba pkt

MMA-IP

Strona 17 z26





Zadanie 30. (0-2)

Do wykresu funkcji wykładniczej, określonej dla $\mathrm{k}\mathrm{a}\dot{\mathrm{z}}$ dej liczby rzeczywistej $x$ wzorem

$f(x)=a^{x}$ (gdzie $a>0 \mathrm{i} a\neq 1$), nalezy punkt $P=(2,9)$. Oblicz $a$ i zapisz zbiór wartości

funkcji $g$, określonej wzorem $g(x)=f(x)-2.$

Odpowiedzí :

Strona 18 z26

MMA-IP





Zadanie 31. (0-2)

Dwunasty wyraz ciągu arytmetycznego $(a_{n})$, określonego dla $n\geq 1$, jest równy 30, a sumajego

dwunastu początkowych wyrazówjest równa 162. Ob1icz pierwszy wyraz tego ciągu.

Odpowiedzí :
\begin{center}
\includegraphics[width=96.012mm,height=17.832mm]{./F2_M_PP_M2018_page18_images/image001.eps}
\end{center}
Wypelnia

egzaminator

Nr zadania

Maks. liczba kt

30.

2

31.

2

Uzyskana liczba pkt

MMA-IP

Strona 19 z26





Zadanie 32. (0-5)

$\mathrm{W}$ układzie współrzędnych punkty $A=(4,3) \mathrm{i} B=(10,5)$ są wierzchołkami trójkąta $ABC.$

Wierzchołek $C$ lezy na prostej o równaniu $y=2x+3$. Oblicz współrzędne punktu $C$, dla którego

kąt $ABC$ jest prosty.

Strona 20 z26

MMA-IP





{\it BRUDNOPIS} ({\it nie podlega ocenie})

$\mathrm{A}_{-}1\mathrm{P}$

Strona 3 z 26





Odpowiedzí :
\begin{center}
\includegraphics[width=82.044mm,height=17.832mm]{./F2_M_PP_M2018_page20_images/image001.eps}
\end{center}
Wypelnia

egzaminator

Nr zadania

Maks. liczba kt

32.

5

Uzyskana liczba pkt

MMA-IP

Strona 21 z26





Zadanie 33. (0-4)

Dane są dwa zbiory: $A=\{100$, 200, 300, 400, 500, 600, 700$\} \mathrm{i} B=\{10$, 11, 12, 13, 14, 15, 16$\}.$

$\mathrm{Z}\mathrm{k}\mathrm{a}\dot{\mathrm{z}}$ dego z nich losujemy jedną liczbę. Oblicz prawdopodobieństwo zdarzenia polegającego

na tym, $\dot{\mathrm{z}}\mathrm{e}$ suma wylosowanych liczb będzie podzielna przez 3. Ob1iczone

prawdopodobieństwo zapisz w postaci nieskracalnego ułamka zwykłego.

Strona 22 z26

MMA-IP





Odpowiedzí :
\begin{center}
\includegraphics[width=82.044mm,height=17.832mm]{./F2_M_PP_M2018_page22_images/image001.eps}
\end{center}
Wypelnia

egzaminator

Nr zadania

Maks. liczba kt

33.

4

Uzyskana liczba pkt

MMA-IP

Strona 23 z26





Zadanie 34. (0-4)

Dany jest graniastosłup prawidłowy trójkątny (zobacz rysunek). Pole powierzchni całkowitej

tego graniastosłupa jest równe $45\sqrt{3}$. Pole podstawy graniastosłupa jest równe polu jednej

ściany bocznej. Oblicz objętość tego graniastosłupa.
\begin{center}
\includegraphics[width=60.864mm,height=42.768mm]{./F2_M_PP_M2018_page23_images/image001.eps}
\end{center}
{\it E}

{\it D}

{\it B}

Strona 24 z26

MMA-IP





Odpowiedzí :
\begin{center}
\includegraphics[width=82.044mm,height=17.832mm]{./F2_M_PP_M2018_page24_images/image001.eps}
\end{center}
Wypelnia

egzaminator

Nr zadania

Maks. liczba kt

34.

4

Uzyskana liczba pkt

MMA-IP

Strona 25 z26





{\it BRUDNOPIS} ({\it nie podlega ocenie})

Strona 26 z26

MMA-I]





Zadanie 7. $(0\rightarrow 1\rangle$

Równanie $\displaystyle \frac{x^{2}+2x}{x^{2}-4}=0$

A. ma trzy rozwiązania: $x=-2, x=0, x=2$

B. ma dwa rozwiązania: $x=0, x=-2$

C. ma dwa rozwiązania: $x=-2, x=2$

D. majedno rozwiązanie: $x=0$

Zadanie @. $(0-1\rangle$

Funkcja liniowa $f$ określona jest wzorem

rzeczywistych $x$. Wskaz zdanie prawdziwe.

$f(x)=\displaystyle \frac{1}{3}x-1,$

dla wszystkich

liczb

A. Funkcja $f$ jest malejąca ijej wykres przecina oś $oy$ w punkcie $P=(0,\displaystyle \frac{1}{3}).$

B. Funkcja $f$ jest malejąca ijej wykres przecina oś $Oy$ w punkcie $P=(0,-1).$

C. Funkcja $f$ jest rosnąca ijej wykres przecina oś $oy$ w punkcie $P=(0,\displaystyle \frac{1}{3}).$

D. Funkcja $f$ jest rosnąca ijej wykres przecina oś $Oy$ w punkcie $P=(0,-1).$

Zadanie $g.(0-1)$

Wykresem funkcji kwadratowej $f(x)=x^{2}-6x-3$ jest parabola, której wierzchołkiem jest

punkt o współrzędnych

A. $(-6,-3)$

B. $(-6,69)$

C. $(3,-12)$

D. $(6,-3)$

Zadanie 10, (0-1)

Liczba l jest miejscem zerowym funkcji liniowej $f(x)=ax+b$, a punkt $M=(3,-2)$ nalezy

do wykresu tej funkcji. Współczynnik $a$ we wzorze tej funkcjijest równy

A. l

B.

-23

C.

- -23

D. $-1$

Zadanie $ll. (0\rightarrow l)$

Dany jest ciąg $(a_{n})$ określony wzorem $a_{n}=\displaystyle \frac{5-2n}{6}$ dla $n\geq 1$. Ciąg tenjest

A.

B.

C.

D.

arytmetyczny ijego róznicajest równa $r=-\displaystyle \frac{1}{3}$

arytmetyczny ijego róznicajest równa $r=-2.$

geometryczny ijego iloraz jest równy $q=-\displaystyle \frac{1}{3}.$

geometryczny ijego iloraz jest równy $q=\displaystyle \frac{5}{6}$

Strona 4 z 26

MMA-IP





{\it BRUDNOPIS} ({\it nie podlega ocenie})

$\mathrm{A}_{-}1\mathrm{P}$

Strona 5 z 26





Zadanie 12. $(0-l)$

Dla ciągu arytmetycznego

$(a_{n}),$

określonego dla

$n\geq 1$, jest

spełniony warunek

$a_{4}+a_{5}+a_{6}=12$. Wtedy

A. $a_{5}=4$

B. $a_{5}=3$

C. $a_{5}=6$

D. $a_{5}=5$

Zadanie 13, (0-1)

Dany jest ciąg geometryczny $(a_{n})$, określony dla

$a_{3}=4\sqrt{2}$. Wzór na n-ty wyraz tego ciągu ma postać

$n\geq 1$, w którym $a_{1}=\sqrt{2}, a_{2}=2\sqrt{2},$

A.

$a_{n}=(\sqrt{2})^{n}$

B.

{\it an}$=$ -$\sqrt{}$22{\it n}

C.

{\it an}$=$(-$\sqrt{}$22){\it n}

D.

$a_{n}=\displaystyle \frac{(\sqrt{2})}{2}n$

Zadanie 14. $(0-1\rangle$

Przyprostokątna $LM$ trójkąta prostokątnego $KLM$ ma długość 3, a przeciwprostokątna $KL$ ma

długość 8 (zobacz rysunek).

3
\begin{center}
\includegraphics[width=82.596mm,height=37.080mm]{./F2_M_PP_M2018_page5_images/image001.eps}
\end{center}
{\it L}

8

$\alpha$

{\it M  K}

Wtedy miara $\alpha$ kąta ostrego $LKM$ tego trójkąta spełnia warunek

A. $27^{\mathrm{o}}<\alpha\leq 30^{\mathrm{o}}$

B. $24^{\mathrm{o}}<\alpha\leq 27^{\mathrm{o}}$

C. $21^{\mathrm{o}}<\alpha\leq 24^{\mathrm{o}}$

D. $18^{\mathrm{o}}<\alpha\leq 21^{\mathrm{o}}$

Zadanie $l5. (0-l)$

Dany jest trójkąt o bokach długości: $2\sqrt{5}, 3\sqrt{5}, 4\sqrt{5}$. Trójkątem podobnym do tego trójkąta

jest trójkąt, którego boki mają długości

A. 10, 15, 20

B. 20, 45, 80

C. $\sqrt{2}, \sqrt{3}, \sqrt{4}$

D. $\sqrt{5}, 2\sqrt{5}, 3\sqrt{5}$

Strona 6 z 26

MMA-IP





{\it BRUDNOPIS} ({\it nie podlega ocenie})

$\mathrm{A}_{-}1\mathrm{P}$

Strona 7 z26





Zadanie 16. $(0-l)$

Dany jest okrąg o środku $S$. Punkty $K, L\mathrm{i}M$ lez$\cdot$ą na tym okręgu. Na łuku $KL$ tego okręgu są

oparte kąty $KSL \mathrm{i} KML$ (zobacz rysunek), których miary $a \mathrm{i} \beta$ spełniają warunek

$\alpha+\beta=111^{\mathrm{o}}$. Wynika stąd, $\dot{\mathrm{z}}\mathrm{e}$
\begin{center}
\includegraphics[width=64.368mm,height=61.620mm]{./F2_M_PP_M2018_page7_images/image001.eps}
\end{center}
{\it M}

$\alpha$

{\it K  L}

A. $\alpha=74^{\mathrm{o}}$

B. $\alpha=76^{\mathrm{o}}$

C. $\alpha=70^{\mathrm{o}}$

D. $\alpha=72^{\mathrm{o}}$

Zadanie $17_{\alpha}(0-1)$

Dany jest trapez prostokątny KLMN, którego podstawy mają długości $|KL|=a, |MN|=b,$

$a>b$. Kąt $KLM$ ma miarę $60^{\mathrm{o}}$. Długość ramienia $LM$ tego trapezujest równa

{\it b}
\begin{center}
\includegraphics[width=89.868mm,height=45.516mm]{./F2_M_PP_M2018_page7_images/image002.eps}
\end{center}
{\it N  M}

{\it K  a  L}

A. $a-b$

B. $2(a-b)$

C.

$a+\displaystyle \frac{1}{2}b$

D.

-{\it a} $+$2 {\it b}

Zadanie 18. $(0-l)$

Punkt $K=(2,2)$ jest wierzchołkiem trójkąta równoramiennego $KLM$, w którym $|KM|=|LM|$

Odcinek MNjest wysokością trójkąta i $N=(4,3)$. Zatem

A. $L=(5,3)$

B. $L=(6,4)$

C. $L=(3,5)$

D. $L=(4,6)$

Zadanie $l9. (0\rightarrow l)$

Proste o równaniach $y=(m+2)x+3$ oraz $y=(2m-1)x-3$ są równoległe, gdy

A. $m=2$

B. $m=3$

C. $m=0$

D. $m=1$

Strona 8 z26

MMA-IP





{\it BRUDNOPIS} ({\it nie podlega ocenie})

$\mathrm{A}_{-}1\mathrm{P}$

Strona 9 z26





Zadanie 20. $(0-l)$

Podstawą ostrosłupa jest kwadrat KLMN o boku długości 4. Wysokością tego ostrosłupajest

krawędzí $NS$, ajej długość $\mathrm{t}\mathrm{e}\dot{\mathrm{z}}$ jest równa 4 (zobacz rysunek).

Kąt $\alpha$, jaki tworzą krawędzie $KS\mathrm{i}MS$, spełnia warunek

A. $\alpha=45^{\mathrm{o}}$

B. $45^{\mathrm{o}}<\alpha<60^{\mathrm{o}}$

C. $\alpha>60^{\mathrm{o}}$

D. $\alpha=60^{\mathrm{o}}$

Zadanie $2l. (0-l)$

Podstawą graniastosłupa prostegojest prostokąt o bokach długości 3 $\mathrm{i}4$. Kąt $cx$, jaki przekątna

tego graniastosłupa tworzy zjego podstawą, jest równy $45^{\mathrm{o}}$ (zobacz rysunek).

Wysokość graniastosłupajest równa

A. 5

B. $3\sqrt{2}$

C. $5\sqrt{2}$

D.

$\displaystyle \frac{5\sqrt{3}}{3}$

Zadanie 22. (0-1)

Na rysunku przedstawiono bryłę zbudowaną z walca i półkuli. Wysokość walcajest równa r

ijest taka samajak promień półkuli oraz taka samajak promień podstawy walca.

Objętość tej bryłyjest równa

A.

$\displaystyle \frac{5}{3}\pi r^{3}$

B.

$\displaystyle \frac{4}{3}\pi r^{3}$

C.

$\displaystyle \frac{2}{3}\pi r^{3}$

D.

$\displaystyle \frac{1}{3}\pi r^{3}$

Strona 10 z26

MMA-IP







$\mathrm{g}_{\mathrm{E}\mathrm{G}\mathrm{Z}\mathrm{A}\mathrm{M}\mathrm{I}\mathrm{N}\mathrm{A}\mathrm{C}\mathrm{Y}\mathrm{J}\mathrm{N}\mathrm{A}}^{\mathrm{C}\mathrm{E}\mathrm{N}\mathrm{T}\mathrm{R}\mathrm{A}\mathrm{L}\mathrm{N}\mathrm{A}}$KOMISJA

Arkusz zawiera informacje

prawnie chronione do momentu

rozpoczęcia egzaminu.

UZUPELNIA ZDAJACY

{\it miejsce}

{\it na naklejkę}
\begin{center}
\includegraphics[width=21.900mm,height=16.104mm]{./F2_M_PP_M2019_page0_images/image001.eps}
\end{center}
KOD
\begin{center}
\includegraphics[width=79.608mm,height=16.104mm]{./F2_M_PP_M2019_page0_images/image002.eps}
\end{center}
PESEL
\begin{center}
\includegraphics[width=195.636mm,height=245.976mm]{./F2_M_PP_M2019_page0_images/image003.eps}
\end{center}
EGZAMIN MATU  LNY

Z MATEMATY

POZIOM PODSTAWOWY

DATA: 7 maja 2019 $\mathrm{r}.$

LICZBA P KTÓW DO UZYS NIA: 50

Instrukcja dla zdającego

1.

2.

3.

4.

5.

Sprawdzí, czy arkusz egzaminacyjny zawiera 26 stron (zadania $1-34$).

Ewentualny brak zgłoś przewodniczącemu zespo nadzo jącego

egzamin.

Rozwiązania zadań i odpowiedzi wpisuj w miejscu na to przeznaczonym.

Odpowiedzi do zadań za iętych $(1-25)$ zaznacz na karcie odpowiedzi,

w części ka przeznaczonej dla zdającego. Zamaluj $\blacksquare$ pola do tego

przeznaczone. Błędne zaznaczenie otocz kółkiem \copyright i zaznacz właściwe.

Pamiętaj, $\dot{\mathrm{z}}\mathrm{e}$ pominięcie argumentacji lub istotnych obliczeń

w rozwiązaniu zadania o a ego (26-34) $\mathrm{m}\mathrm{o}\dot{\mathrm{z}}\mathrm{e}$ spowodować, $\dot{\mathrm{z}}\mathrm{e}$ za to

rozwiązanie nie otrzymasz pełnej liczby pu tów.

Pisz cz elnie i $\mathrm{u}\dot{\mathrm{z}}$ aj tylko $\mathrm{d}$ gopisu lub pióra z czamym tuszem lub

atramentem.

6. Nie $\mathrm{u}\dot{\mathrm{z}}$ aj korektora, a blędne zapisy razínie prze eśl.

7. Pamiętaj, $\dot{\mathrm{z}}\mathrm{e}$ zapisy w brudnopisie nie będą oceniane.

8. $\mathrm{M}\mathrm{o}\dot{\mathrm{z}}$ esz korzystać z zesta wzorów matema cznych, cyrkla i linijki,

a ta $\mathrm{e}$ z kalkulatora prostego.

9. Na tej stronie oraz na karcie odpowiedzi wpisz swój numer PESEL

i przyklej naklejkę z kodem.

10. Nie wpisuj $\dot{\mathrm{z}}$ adnych znaków w części przeznaczonej dla egzaminatora.

$\Vert\Vert\Vert\Vert\Vert\Vert\Vert\Vert\Vert\Vert\Vert\Vert\Vert\Vert\Vert\Vert\Vert\Vert\Vert\Vert\Vert\Vert\Vert\Vert|$

$\mathrm{M}\mathrm{M}\mathrm{A}-\mathrm{P}1_{-}1\mathrm{P}-192$

Układ graficzny

\copyright CKE 2015




{\it W kazdym z zadań od l. do 25. wybierz i zaznacz na karcie odpowiedzi poprawnq odpowiedzí}.

Zadanie l. $(0-1\rangle$

Liczba $\log_{\sqrt{2}}2$ jest równa

A. 2

B. 4

C.

$\sqrt{2}$

D.

-21

Zadanie 2. (0-1)

Liczba naturalna $n=2^{14}\cdot 5^{15}$ w zapisie dziesiętnym ma

A. 14 cyfr

B. 15 cyfr

C. 16 cyfr

D. 30 cyfr

Zadanie 3. (0-1)

$\mathrm{W}$ pewnym banku prowizja od udzielanych kredytów hipotecznych przez cały styczeń była

równa 4\%. Na początku 1utego ten bank obnizył wysokość prowizji od wszystkich kredytów

$0 1$ punkt procentowy. Oznacza to, $\dot{\mathrm{z}}\mathrm{e}$ prowizja od kredytów hipotecznych w tym banku

zmniejszyła się o

A. l\%

B. 25\%

C. 33\%

D. 75\%

Zadanie 4. $(0-l)$

Równość $\displaystyle \frac{1}{4}+\frac{1}{5}+\frac{1}{a}=1$ jest prawdziwa dla

A.

$a=\displaystyle \frac{11}{20}$

B.

{\it a}$=$ -98

C.

{\it a}$=$ -98

D.

{\it a}$=$ -2101

Zadanie 5. $(0-l)$

Para liczb $x=2 \mathrm{i}y=2$ jest rozwiązaniem układu równań 

A. $a=-1$

B. $a=1$

C. $a=-2$

D. $a=2$

Zadanie 6. $(0-l)$

Równanie $\displaystyle \frac{(x-1)(x+2)}{x-3}=0$

A. ma trzy rózne rozwiązania: $x=1, x=3, x=-2.$

B. ma trzy rózne rozwiązania: $x=-1, x=-3, x=2.$

C. ma dwa rózne rozwiązania: $x=1, x=-2.$

D. ma dwa rózne rozwiązania: $x=-1, x=2.$

Strona 2 z26

MMA-IP





{\it BRUDNOPIS} ({\it nie podlega ocenie})

$\mathrm{A}_{-}1\mathrm{P}$

Strona ll z 26





Zadanie 22. $(0\rightarrow 1)$

Promień kuli i promień podstawy stozka są równe 4. Po1e powierzchni ku1i jest równe po1u

powierzchni całkowitej stozka. Długość tworzącej stozka jest równa

A. 8

B. 4

C. 16

D. 12

Zadanie 23. $(0-l)$

Mediana zestawu sześciu danych liczb: 4, 8, 21, $a$, 16, 25, jest równa l4. Zatem

A. $a=7$

B. $a=12$

C. $a=14$

D. $a=20$

Zadanie 24, (0-1)

Wszystkich liczb pięciocyfrowych, w których występują wyłącznie cyfry 0, 2, 5, jest

A.

12

B. 36

C. 162

D. 243

Zadanie 25. (0-1)

$\mathrm{W}$ pudełku jest 40 ku1. Wśród nich jest 35 ku1 białych, a pozostałe to ku1e czerwone.

Prawdopodobieństwo wylosowania kazdej kuli jest takie samo. $\mathrm{Z}$ pudełka losujemy jedną

kulę. Prawdopodobieństwo zdarzenia polegającego na tym, $\dot{\mathrm{z}}\mathrm{e}$ otrzymamy kulę czerwoną, jest

równe

A.

-81

B.

-51

C.

$\displaystyle \frac{1}{40}$

D.

$\displaystyle \frac{1}{35}$

Strona 12 z 26

MMA-IP





{\it BRUDNOPIS} ({\it nie podlega ocenie})

$\mathrm{A}_{-}1\mathrm{P}$

Strona 13 z 26





Zadanie $2\epsilon. (0-2)$

Rozwiąz równanie $(x^{3}-8)(x^{2}-4x-5)=0.$

Odpowiedzí :

Strona 14 z 26

MMA-I]





Zadanie 27. $(0\rightarrow 2\rangle$

Rozwiąz nierówność $3x^{2}-16x+16>0.$

Odpowied $\acute{\mathrm{z}}$:
\begin{center}
\includegraphics[width=96.012mm,height=17.832mm]{./F2_M_PP_M2019_page14_images/image001.eps}
\end{center}
Wypelnia

egzaminator

Nr zadania

Maks. liczba kt

2

27.

2

Uzyskana liczba pkt

MMA-IP

Strona 15 z 26





Zadanie 2@. (0-2)

Wykaz, $\dot{\mathrm{z}}\mathrm{e}$ dla dowolnych liczb rzeczywistych $a\mathrm{i}b$ prawdziwajest nierówność

$3a^{2}-2ab+3b^{2}\geq 0.$

Strona 16 z 26

MMA-IF





Zadanie 29. $(0\rightarrow 2\rangle$

Dany jest okrąg o Środku w punkcie $S$ i promieniu $r$. Na przedłuzeniu cięciwy $AB$ poza

punkt $B$ odłozono odcinek $BC$ równy promieniowi danego okręgu. Przez punkty $C \mathrm{i} S$

poprowadzono prostą. Prosta $CS$ przecina dany okrąg w punktach $D\mathrm{i}E$ (zobacz rysunek).

Wykaz, $\dot{\mathrm{z}}$ ejezeli miara kąta ACSjest równa $a$, to miara kąta $ASD$ jest równa $3a.$
\begin{center}
\includegraphics[width=96.924mm,height=62.580mm]{./F2_M_PP_M2019_page16_images/image001.eps}
\end{center}
{\it D  r}

{\it S}

{\it r}

{\it E}

{\it r  r}

{\it C}

{\it A  B}
\begin{center}
\includegraphics[width=96.012mm,height=17.832mm]{./F2_M_PP_M2019_page16_images/image002.eps}
\end{center}
Wypelnia

egzaminator

Nr zadania

Maks. liczba kt

28.

2

2

Uzyskana liczba pkt

MMA-IP

Strona 17 z 26





Zadanie30. $(0\rightarrow 2\rangle$

Ze zbioru liczb \{1, 2, 3, 4, 5\} 1osujemy dwa razy po jednej 1iczbie ze zwracaniem. Ob1icz

prawdopodobieństwo zdarzenia $A$ polegającego na wylosowaniu liczb, których iloczyn jest

liczbą nieparzystą.

Odpowiedzí :

Strona 18 z 26

MMA-IP





Zadanie 31. (0-2)

$\mathrm{W}$ trapezie prostokątnym ABCD dłuzsza podstawa $AB$ ma długoŚć 8. Przekątna $AC$ tego trapezu

ma długość 4 i tworzy z krótszą podstawą trapezu kąt o mierze $30^{\mathrm{o}}$ (zobacz rysunek). Oblicz

długoŚć przekątnej $BD$ tego trapezu.
\begin{center}
\includegraphics[width=106.776mm,height=35.256mm]{./F2_M_PP_M2019_page18_images/image001.eps}
\end{center}
{\it D  C}

4

8  {\it B}

MMA-IP
\begin{center}
\includegraphics[width=96.012mm,height=17.832mm]{./F2_M_PP_M2019_page18_images/image002.eps}
\end{center}
Wypelnia

egzaminator

Nr zadania

Maks. liczba kt

30.

2

31.

2

Uzyskana liczba pkt

Odpowiedzí :

Strona 19 z 26





Zadanie 32. $(0\rightarrow 4\rangle$

Ciąg arytmetyczny $(a_{n})$ jest określony dla $\mathrm{k}\mathrm{a}\dot{\mathrm{z}}$ dej liczby naturalnej $n\geq 1$. Róz$\cdot$nicą tego ciągu

jest liczba $r=-4$, a średnia arytmetyczna początkowych sześciu wyrazów tego ciągu:

$a_{1}, a_{2}, a_{3}, a_{4}, a_{5}, a_{6}$, jestrówna 16.

a) Oblicz pierwszy wyraz tego ciągu.

b) Oblicz liczbę $k$, dla której $a_{k}=-78.$

Strona 20 z 26

MMA-IP





{\it BRUDNOPIS} ({\it nie podlega ocenie})

$\mathrm{A}_{-}1\mathrm{P}$

Strona 3 z26





Odpowied $\acute{\mathrm{z}}$:
\begin{center}
\includegraphics[width=82.044mm,height=17.784mm]{./F2_M_PP_M2019_page20_images/image001.eps}
\end{center}
Wypelnia

egzaminator

Nr zadania

Maks. liczba kt

32.

4

Uzyskana liczba pkt

MMA-IP

Strona 21 z 26





Zadanie 33. $(0\rightarrow 4\rangle$

Danyjest punkt $A=(-18,10)$. Prosta o równaniu $y=3x$ jest symetralną odcinka $AB$. Wyznacz

współrzędne punktu $B.$

Strona 22 z 26

MMA-IP





Odpowiedzí :
\begin{center}
\includegraphics[width=82.044mm,height=17.832mm]{./F2_M_PP_M2019_page22_images/image001.eps}
\end{center}
Wypelnia

egzaminator

Nr zadania

Maks. liczba kt

33.

4

Uzyskana liczba pkt

MMA-IP

Strona 23 z 26





Zadanie 34. $(0-5\rangle$

Długość krawędzi podstawy ostrosłupa prawidłowego czworokątnego jest równa 6. Po1e

powierzchni całkowitej tego ostrosłupa jest cztery razy większe od pola jego podstawy.

Kąt $\alpha$ jest kątem nachylenia krawędzi bocznej tego ostrosłupa do płaszczyzny podstawy

(zobacz rysunek). Oblicz cosinus kąta $\alpha.$

Strona 24 z 26

MMA-IP





Odpowiedzí :
\begin{center}
\includegraphics[width=82.044mm,height=17.832mm]{./F2_M_PP_M2019_page24_images/image001.eps}
\end{center}
Wypelnia

egzaminator

Nr zadania

Maks. liczba kt

34.

5

Uzyskana liczba pkt

MMA-IP

Strona 25 z 26





{\it BRUDNOPIS} ({\it nie podlega ocenie})

Strona 26 z 26

MMA-I]





Zadanie 7. $(0-l)$

Miejscem zerowym funkcji liniowej $f$ określonej wzorem $f(x)=3(x+1)-6\sqrt{3}$ jest liczba

A. $3-6\sqrt{3}$

B. $1-6\sqrt{3}$

C. $2\sqrt{3}-1$

D.

$2\displaystyle \sqrt{3}-\frac{1}{3}$

Informacja do zadań $8\rightarrow l0.$

Na rysunku przedstawiony jest fragment paraboli będącej wykresem funkcji kwadratowej $f.$

Wierzchołkiem tej parabolijest punkt $W=(2,-4)$. Liczby 0 $\mathrm{i}4$ to miejsca zerowe funkcji $f.$
\begin{center}
\includegraphics[width=127.248mm,height=105.108mm]{./F2_M_PP_M2019_page3_images/image001.eps}
\end{center}
{\it y}

4

3

1

{\it x}

$-4 -3  -2$

$-1 0$

$-1$

1 2 3 4  5 6

$-2$

$-3$

{\it W}

B. $\langle 0,  4\rangle$

$(-\infty,  0\rangle$

A.

Zadanie 8, $(0-l)$

Zbiorem wartości funkcji $f$ jest przedział

C. $\langle-4, +\infty)$

D. $\langle 4, +\infty)$

Zadanie 9. $(0-l)$

Największa wartość funkcji $f$ w przedziale $\langle$1, $ 4\rangle$ jest równa

A. $-3$

B. $-4$

C. 4

D. 0

Zadanie 10. (0-1)

Osią symetrii wykresu ffinkcji f jest prosta o równaniu

A. $y=-4$

B. $x=-4$

C. $y=2$

D. $x=2$

Strona 4 z26

MMA-IP





{\it BRUDNOPIS} ({\it nie podlega ocenie})

$\mathrm{A}_{-}1\mathrm{P}$

Strona 5 z 26





Zadanie ll. $(0-l)$

$\mathrm{W}$ ciągu arytmetycznym $(a_{n})$, określonym dla $n\geq 1$, dane są dwa wyrazy: $a_{1}=7\mathrm{i}a_{8}=-49.$

Suma ośmiu początkowych wyrazów tego ciągujest równa

A. $-168$

B. $-189$

C. $-21$

D. $-42$

Zadanie 12. (0-1)

Dany jest ciąg geometryczny $(a_{n})$, określony dla $n\geq 1$. Wszystkie wyrazy tego ciągu są

dodatnie i spełnionyjest watunek $\displaystyle \frac{a_{5}}{a_{3}}=\frac{1}{9}$. Iloraz tego ciągu jest równy

A.

-31

B.

-$\sqrt{}$13

C. 3

D. $\sqrt{3}$

Zadanie 13. (0-1)

Sinus kąta ostrego $a$ jest równy $\displaystyle \frac{4}{5}$. Wtedy

A.

$\displaystyle \cos\alpha=\frac{5}{4}$

B.

$\displaystyle \cos\alpha=\frac{1}{5}$

C.

$\displaystyle \cos\alpha=\frac{9}{25}$

D.

$\displaystyle \cos\alpha=\frac{3}{5}$

Zadanie 14, $(0-l)$

Punkty $D\mathrm{i}E$ lez$\cdot$ą na okręgu opisanym na trójkącie równobocznym $ABC$ (zobacz rysunek).

Odcinek $CD$ jest średnicą tego okręgu. Kąt wpisany $DEB$ ma miarę $\alpha.$
\begin{center}
\includegraphics[width=46.788mm,height=52.680mm]{./F2_M_PP_M2019_page5_images/image001.eps}
\end{center}
{\it C}

{\it E}

$\alpha$

{\it A  B}

{\it D}

Zatem

A. $\alpha=30^{\mathrm{o}}$

B. $\alpha<30^{\mathrm{o}}$

C. $\alpha>45^{\mathrm{o}}$

D. $\alpha=45^{\mathrm{o}}$

Strona 6 z26

MMA-IP





{\it BRUDNOPIS} ({\it nie podlega ocenie})

$\mathrm{A}_{-}1\mathrm{P}$

Strona 7 z 26





Zadanie 15. $(0-l)$

Dane są dwa okręgi: okrąg o środku w punkcie $O$ i promieniu 5 oraz okrąg o środku

w punkcie $P$ i promieniu 3. Odcinek $OP$ ma długość 16. Prosta $AB$ jest styczna do tych

okręgów w punktach A $\mathrm{i}B$. Ponadto prosta $AB$ przecina odcinek $OP$ w punkcie $K$ (zobacz

rysunek).
\begin{center}
\includegraphics[width=155.244mm,height=65.484mm]{./F2_M_PP_M2019_page7_images/image001.eps}
\end{center}
{\it B}

{\it O  K}

{\it P}

{\it A}

Wtedy

A.

$|OK|=6$

B.

$|OK|=8$

C.

$|OK|=10$

D.

$|OK|=12$

Zadanie 16. (0-1)

Dany jest romb o boku długości 4 i kącie rozwartym $150^{\mathrm{o}}$. Pole tego rombujest równe

A. 8

B. 12

C. $8\sqrt{3}$

D. 16

Zadanie $l7. (0-l)$

Proste o równaniach $y=(2m+2)x-2019$ oraz $y=(3m-3)x+2019$ są równoległe, gdy

A. $m=-1$

B. $m=0$

C. $m=1$

D. $m=5$

Zadanie 18. (0-1)

Prosta o równaniu $y=ax+b$ jest prostopadła do prostej o równaniu $y=-4x+1$ i przechodzi

przez punkt $P=(\displaystyle \frac{1}{2},0)$, gdy

A. $a=-4\mathrm{i}b=-2$

B. {\it a}$=$-41i{\it b}$=$--81

C. $a=-4\mathrm{i}b=2$

D. {\it a}$=$-41i{\it b}$=$-21

Strona 8 z 26

MMA-IP





{\it BRUDNOPIS} ({\it nie podlega ocenie})

$\mathrm{A}_{-}1\mathrm{P}$

Strona 9 z 26





Zadanie 19. $(0-l)$

Na rysunku przedstawiony jest fragment wykresu funkcji liniowej $f$. Na wykresie tej ffinkcji

$\mathrm{l}\mathrm{e}\dot{\mathrm{z}}$ ą punkty $A=(0,4)\mathrm{i}B=(2,2).$
\begin{center}
\includegraphics[width=65.376mm,height=67.920mm]{./F2_M_PP_M2019_page9_images/image001.eps}
\end{center}
$y$

$5$

-$4^{A}$

3

2

$B1$

1

{\it x}

$-4  -3$ -$2$ -$1$ -$10$  1 2 3 4  $-5$

$-2$

$-3$

$-4$

Obrazem prostej AB w symetrii względem początku układu współrzędnych jest wykres

funkcji g określonej wzorem

A. $g(x)=x+4$

B. $g(x)=x-4$

C. $g(x)=-x-4$

D. $g(x)=-x+4$

Zadanie 20. (0-1)

Dane są punkty o współrzędnych $A=(-2,5)$ oraz $B=(4,-1)$. Średnica okręgu wpisanego

w kwadrat o boku $AB$ jest równa

A. 12

B. 6

C.

$6\sqrt{2}$

D. $\mathrm{z}\sqrt{6}$

Zadanie 21. (0-1)

Pudełko w kształcie prostopadłościanu ma wymiary 5 dm$\rangle\langle 3$ dm$\rangle\langle 2$ dm (zobacz rysunek).
\begin{center}
\includegraphics[width=106.776mm,height=57.252mm]{./F2_M_PP_M2019_page9_images/image002.eps}
\end{center}
{\it L}

2 dm

3 dm

{\it K}

{\it K}

5 dm

Przekątna KL tego prostopadłościanujest-z dokładnością do 0,01 dm- równa

A. 5,83 dm

B. 6,16 dm

C. 3,61 dm

D. 5,39 dm

Strona 10 z 26

MMA-IP







$\mathrm{g}_{\mathrm{E}\mathrm{G}\mathrm{Z}\mathrm{A}\mathrm{M}\mathrm{I}\mathrm{N}\mathrm{A}\mathrm{C}\mathrm{Y}\mathrm{J}\mathrm{N}\mathrm{A}}^{\mathrm{C}\mathrm{E}\mathrm{N}\mathrm{T}\mathrm{R}\mathrm{A}\mathrm{L}\mathrm{N}\mathrm{A}}$KOMISJA

Arkusz zawiera informacje

prawnie chronione do momentu

rozpoczęcia egzaminu.

WYPELNIA ZDAJACY

{\it miejsce}

{\it na naklejkę}
\begin{center}
\includegraphics[width=21.900mm,height=16.104mm]{./F2_M_PP_M2020_page0_images/image001.eps}
\end{center}
KOD
\begin{center}
\includegraphics[width=79.608mm,height=16.104mm]{./F2_M_PP_M2020_page0_images/image002.eps}
\end{center}
PESEL
\begin{center}
\includegraphics[width=193.644mm,height=264.720mm]{./F2_M_PP_M2020_page0_images/image003.eps}
\end{center}
EGZAMIN MATU  LNY

Z MATEMATY

POZIOM PODSTAWOWY

DATA: 5 maja 2020 $\mathrm{r}.$

LICZBA P KTÓW DO UZYS NIA: 50

Instrukcja dla zdającego

1.

2.

3.

4.

5.

Sprawdzí, czy ar sz egzaminacyjny zawiera 26 stron (zadania $1-34$).

Ewentualny brak zgłoś przewodniczącemu zespo nadzorującego

egzamin.

Rozwiązania zadań i odpowiedzi wpisuj w miejscu na to przeznaczonym.

Odpowiedzi do zadań za ię ch $(1-25)$ zaznacz na karcie odpowiedzi,

w części ka przeznaczonej dla zdającego. Zamaluj $\blacksquare$ pola do tego

przeznaczone. Blędne zaznaczenie otocz kólkiem \copyright i zaznacz wlaściwe.

Pamiętaj, $\dot{\mathrm{z}}\mathrm{e}$ pominięcie argumentacji lub istotnych obliczeń

w rozwiązaniu zadania otwa ego (26-34) $\mathrm{m}\mathrm{o}\dot{\mathrm{z}}\mathrm{e}$ spowodować, $\dot{\mathrm{z}}\mathrm{e}$ za to

rozwiązanie nie otr masz pełnej liczby pu tów.

Pisz cz elnie i $\mathrm{u}\dot{\mathrm{z}}$ aj lko $\mathrm{d}$ gopisu lub pióra z czamym tuszem lub

atramentem.

6. Nie $\mathrm{u}\dot{\mathrm{z}}$ aj korektora, a błędne zapisy razínie prze eśl.

7. Pamiętaj, $\dot{\mathrm{z}}\mathrm{e}$ zapisy w brudnopisie nie będą oceniane.

8. $\mathrm{M}\mathrm{o}\dot{\mathrm{z}}$ esz korzystać z zesta wzorów matema cznych, cyrkla i linijki,

a ta $\mathrm{e}$ z kalkulatora prostego.

9. Na tej stronie oraz na karcie odpowiedzi wpisz swój numer PESEL

i przyklej naklejkę z kodem.

10. Nie wpisuj $\dot{\mathrm{z}}$ adnych znaków w części przeznaczonej dla egzaminatora.

$\Vert\Vert\Vert\Vert\Vert\Vert\Vert\Vert\Vert\Vert\Vert\Vert\Vert\Vert\Vert\Vert\Vert\Vert\Vert\Vert\Vert\Vert\Vert\Vert|$

$\mathrm{M}\mathrm{M}\mathrm{A}-\mathrm{P}1_{-}1\mathrm{P}-202$

Układ graficzny

\copyright CKE 2015

$| 1$




{\it W kazdym z zadań od l. do 25. wybierz i zaznacz na karcie odpowiedzi poprawnq odpowiedzí}.

Zadanie 1. (0-1)

Wartość wyrazenia $x^{2}-6x+9$ dla $x=\sqrt{3}+3$

A. l

B. 3

Zadanie2. (0-1)

Liczba $\displaystyle \frac{2^{50}\cdot 3^{40}}{36^{10}}$ jest równa

A.

$6^{70}$

B. $6^{45}$

Zadanie 3. $(0-1\rangle$

Liczba $\log_{5}\sqrt{125}$ jest równa

A.

-23

B. 2

est równa

C. $1+2\sqrt{3}$

D. $1-2\sqrt{3}$

C. $2^{30}\cdot 3^{20}$

D. $2^{10}\cdot 3^{20}$

C. 3

D.

-23

Zadanie 4. $(0-1\rangle$

Cenę $x$ pewnego towaru obnizono o 20\% i otrzymano cenę $y$. Aby przywrócić cenę $x$, nową

cenę $y$ nalezy podnieść o

A. 25\%

B. 20\%

C. 15\%

D. 12\%

Zadanie 5, $(0-1\rangle$

Zbiorem wszystkich rozwiązań nierówności 3 $(1-x)>2(3x-1)-12x$ jest przedział

A.

$(-\displaystyle \frac{5}{3},+\infty)$

B.

(-$\infty$, -35)

C.

$(\displaystyle \frac{5}{3},+\infty)$

D.

(-$\infty$'- -35)

Zadanie 6. (0-1)

Suma wszystkich rozwiązań równania $x(x-3)(x+2)=0$ jest równa

A. 0

B. l

C. 2

D. 3

Strona 2 z26

MMA-IP





{\it BRUDNOPIS} ({\it nie podlega ocenie})

$\mathrm{A}_{-}1\mathrm{P}$

Strona ll z26





Zadanie 23. (0-1)

Cztery liczby: 2, 3, a, 8, tworzące zestaw danych, są uporządkowane rosnąco. Mediana tego

zestawu czterech danychjest równa medianie zestawu pięciu danych: 5, 3, 6, 8, 2. Zatem

A. $a=7$

B. $a=6$

C. $a=5$

D. $a=4$

ZadanIe 24. $(0\rightarrow 1$\}

Przekątna sześcianu ma długość $4\sqrt{3}$. Pole powierzchni tego sześcianujest równe

A. 96

B. $24\sqrt{3}$

C. 192

D. $16\sqrt{3}$

Zadanie 25. $(0\rightarrow 1)$

Dwa stozki o takich samych podstawach połączono podstawami w taki sposób jak na rysunku.

Stosunek wysokości tych stozkówjest równy 3: 2. Objętość stozka o krótszej wysokościjest

równa 12 $\mathrm{c}\mathrm{m}^{3}$

Objętość bryły utworzonej z połączonych stozkówjest równa

A.

20 $\mathrm{c}\mathrm{m}^{3}$

B.

$30\mathrm{c}\mathrm{m}^{3}$

C.

$39\mathrm{c}\mathrm{m}^{3}$

D. 52, $5\mathrm{c}\mathrm{m}^{3}$

Strona 12 z26

MMA-IP





{\it BRUDNOPIS} ({\it nie podlega ocenie})

$\mathrm{A}_{-}1\mathrm{P}$

Strona 13 z26





Zadanie $2\epsilon. (0-2)$

Rozwiąz nierówność 2 $(x-1)(x+3)>x-1.$

Odpowiedzí:

Strona 14 z26

MMA-I]





Zadanie 27. (0-2)

Rozwiąz równanie $(x^{2}-1)(x^{2}-2x)=0.$

Odpowiedzí:
\begin{center}
\includegraphics[width=96.012mm,height=17.784mm]{./F2_M_PP_M2020_page14_images/image001.eps}
\end{center}
WypelnÍa

egzaminator

Nr zadanÍa

Maks. lÍczba kt

2

27.

2

Uzyskana liczba pkt

MMA-IP

Strona 15 z26





Zadanie 2@. (0-2)

Wykaz, $\dot{\mathrm{z}}\mathrm{e}$ dlakazdych dwóch róznych liczb rzeczywistych $a\mathrm{i}b$ prawdziwajest nierówność

$a(a-2b)+2b^{2}>0.$

Strona 16 z26

MMA-IP





Zadanie 29. (0-2)

Trójkąt ABCjest równoboczny. Punkt $E$ lezy na wysokości $CD$ tego trójkąta oraz $|CE|=\displaystyle \frac{3}{4}|CD|.$

Punkt $F$ lezy na boku $BC$ i odcinek $EF$ jest prostopadły do $BC$ (zobacz rysunek).
\begin{center}
\includegraphics[width=82.140mm,height=73.968mm]{./F2_M_PP_M2020_page16_images/image001.eps}
\end{center}
{\it C}

{\it F}

{\it A  D  B}

Wykaz, $\displaystyle \dot{\mathrm{z}}\mathrm{e}|CF|=\frac{9}{16}|CB|.$
\begin{center}
\includegraphics[width=96.012mm,height=17.784mm]{./F2_M_PP_M2020_page16_images/image002.eps}
\end{center}
WypelnÍa

egzaminator

Nr zadanÍa

Maks. lÍczba kt

28.

2

2

Uzyskana liczba pkt

MMA-IP

Strona 17 z26





Zadani\S 30. (0-2)

Rzucamy dwa razy symetryczną sześcienną kostką do gry, która na $\mathrm{k}\mathrm{a}\dot{\mathrm{z}}$ dej ściance ma inną

liczbę oczek-odjednego oczka do sześciu oczek. Oblicz prawdopodobieństwo zdarzenia $A$

polegającego na tym, ze co najmniej jeden raz wypadnie ścianka z pięcioma oczkami.

Odpowiedzí:

Strona 18 z26

MMA-IP





Zadani\S $3l. (0-2)$

Kąt $\alpha$ jest ostry i spełnia warunek $\displaystyle \frac{2\sin\alpha+3\cos\alpha}{\cos\alpha}=4$. Oblicz tangens kąta $\alpha.$

Odpowiedzí:
\begin{center}
\includegraphics[width=96.012mm,height=17.832mm]{./F2_M_PP_M2020_page18_images/image001.eps}
\end{center}
Wypelnia

egzaminator

Nr zadania

Maks. liczba kt

30.

2

31.

2

Uzyskana liczba pkt

MMA-IP

Strona 19 z26





Zadani\S 32. $(0-4\rangle$

Dany jest kwadrat ABCD, w którym $A=(5,-\displaystyle \frac{5}{3})$. Przekątna $BD$ tego kwadratu jest zawarta

w prostej o równaniu $y=\displaystyle \frac{4}{3}x$. Oblicz współrzędne punktu przecięcia przekątnych $AC\mathrm{i}BD$ oraz

pole kwadratu ABCD.

Strona 20 z26

MMA-IP





{\it BRUDNOPIS} ({\it nie podlega ocenie})

$\mathrm{A}_{-}1\mathrm{P}$

Strona 3 z 26





Odpowied $\acute{\mathrm{z}}$:
\begin{center}
\includegraphics[width=82.044mm,height=17.784mm]{./F2_M_PP_M2020_page20_images/image001.eps}
\end{center}
Nr zadania

Wypelnia Maks. liczba kt

egzamÍnator

Uzyskana liczba pkt

32.

4

MMA-IP

Strona 21 z26





Zadani\S 33. $(0-4\rangle$

Wszystkie wyrazy ciągu geometrycznego $(a_{n})$, określonego dla $n\geq 1$, są dodatnie. Wyrazy tego

ciągu spełniają warunek $6a_{1}-5a_{2}+a_{3}=0$. Oblicz iloraz

$\langle 2\sqrt{2}, 3\sqrt{2}\rangle.$

q tego ciągu nalezący do przedziatu

Strona 22 z26

MMA-IP





Odpowiedzí:
\begin{center}
\includegraphics[width=82.044mm,height=17.832mm]{./F2_M_PP_M2020_page22_images/image001.eps}
\end{center}
Wypelnia

egzaminator

Nr zadania

Maks. liczba kt

33.

4

Uzyskana liczba pkt

MMA-IP

Strona 23 z26





Zadani\S 34. $(0-5\rangle$

Dany jest ostrosłup prawidłowy czworokątny ABCDS, którego krawędzí boczna ma długość 6

(zobacz rysunek). Ściana boczna tego ostrosłupajest nachylona do płaszczyzny podstawy pod

kątem, którego tangensjest równy $\sqrt{7}$. Oblicz objętość tego ostrosłupa.

Strona 24 z26

MMA-IP





Odpowiedzí:
\begin{center}
\includegraphics[width=82.044mm,height=17.832mm]{./F2_M_PP_M2020_page24_images/image001.eps}
\end{center}
Wypelnia

egzaminator

Nr zadania

Maks. liczba kt

34.

5

Uzyskana liczba pkt

MMA-IP

Strona 25 z26





{\it BRUDNOPIS} ({\it nie podlega ocenie})

Strona 26 z26

MMA-I]





Imformacja do zadań 7.$-9.$

Funkcja

kwadratowa f jest

określona

wzorem

$f(x)=a(x-1)(x-3)$. Na rysunku

przedstawiono fragment paraboli będącej wykresem tej ffinkcji. Wierzchołkiem tej parabolijest

punkt $W=(2,1).$
\begin{center}
\includegraphics[width=117.912mm,height=97.128mm]{./F2_M_PP_M2020_page3_images/image001.eps}
\end{center}
4  {\it y}

3

2

{\it W}

1

$| 1  | 1$  1

$-4 -3  -2 -1$  0  1 2 3 4  {\it 5 x}

$-1$

$-2$

$-3$

$-4$

Zadanie 7. (0-1)

Współczynnik a we wzorze funkcji f jest równy

A. l

B. 2

C. $-2$

D. $-1$

Zadaqie @. (0-1)

Największa wartość funkcji $f$ w przedziale $\langle$1, $ 4\rangle$ jest równa

A. $-3$

B. 0

C. l

D. 2

Zadanie 9. $(0-1\rangle$

Osią symetrii paraboli będącej wykresem ffinkcji $f$ jest prosta o równaniu

A. $x=1$

B. $x=2$

C. $y=1$

D. $y=2$

Strona 4 z 26

MMA-IP





{\it BRUDNOPIS} ({\it nie podlega ocenie})

$\mathrm{A}_{-}1\mathrm{P}$

Strona 5 z 26





Zadanie $l0. (0\rightarrow 1)$

Równanie $x(x-2)=(x-2)^{2}$ w zbiorze liczb rzeczywistych

A. nie ma rozwiązań.

B. ma dokładniejedno rozwiązanie: $x=2.$

C. ma dokładniejedno rozwiązanie: $x=0.$

D. ma dwa rózne rozwiązania: $x=1 \mathrm{i}x=2.$

Zadanie ll, $(0-1\rangle$

Na iysunku przedstawiono fiiagment wykresu funkcji liniowej $f$ określonej wzorem $f(x)=ax+b.$
\begin{center}
\includegraphics[width=118.008mm,height=97.788mm]{./F2_M_PP_M2020_page5_images/image001.eps}
\end{center}
4  {\it y}

3

1

$-4 -3  -2$

$-1 0$

$-1$

1 2 3 4  5  {\it x}

$-2$

$-3$

$-4$

Współczynniki a oraz b we wzorze funkcji f spełniają zalezność

A. $a+b>0$

B. $a+b=0$

C. $a\cdot b>0$

D. $a\cdot b<0$

ZadanIe 12. $(0\rightarrow 1$\}

Funkcja $f$ jest określona wzorem $f(x)=4^{-x}+1$ dla kazdej liczby rzeczywistej $x$. Liczba $f(\displaystyle \frac{1}{2})$

jest równa

A.

-21

B.

-23

C. 3

D. 17

Zadanie 13. $(0-1\rangle$

Proste o równaniach $y=(m-2)x$ oraz $y=\displaystyle \frac{3}{4}x+7$ są równoległe. Wtedy

A.

{\it m}$=$- -45

B.

{\it m}$=$ -23

C.

$m=\displaystyle \frac{11}{4}$

D.

$m=\displaystyle \frac{10}{3}$

Strona 6 z 26

MMA-IP





{\it BRUDNOPIS} ({\it nie podlega ocenie})

$\mathrm{A}_{-}1\mathrm{P}$

Strona 7 z26





Zadanie 14. $(0\rightarrow 1)$

Ciąg $(a_{n})$ jest określony wzorem $a_{n}=2n^{2}$ dla $n\geq 1$. Róz$\cdot$nica $a_{5}-a_{4}$ jest równa

A. 4

B. 20

C. 36

D. 18

Zadanie 15. $(0\rightarrow 1)$

$\mathrm{W}$ ciągu arytmetycznym $(a_{n})$, określonym dla $n\geq 1$, czwarty wyraz jest równy 3, a róznica

tego ciągujest równa 5. Suma $a_{1}+a_{2}+a_{3}+a_{4}$ jest równa

A. $-42$

B. $-36$

C. $-18$

D. 6

Zadanie $l6. (0-1\rangle$

Punkt $A=(\displaystyle \frac{1}{3},-1)$ nalezy do wykresu funkcji liniowej $f$ określonej wzorem $f(x)=3x+b.$

Wynika stąd, $\dot{\mathrm{z}}\mathrm{e}$

A. $b=2$

B. $b=1$

C. $b=-1$

D. $b=-2$

Zadanie $l7. (0-1\rangle$

Punkty $A, B, C, D$ lez$\cdot$ą na okręgu o środku w punkcie $O$. Kąt środkowy DOC ma miarę $118^{\mathrm{o}}$

(zobacz rysunek).
\begin{center}
\includegraphics[width=57.612mm,height=60.804mm]{./F2_M_PP_M2020_page7_images/image001.eps}
\end{center}
{\it B D}

{\it O}  $118^{\mathrm{o}}$

{\it A C}

Miara kąta ABC jest równa

A. $59^{\mathrm{o}}$

B. $48^{\mathrm{o}}$

C. $62^{\mathrm{o}}$

D. $31^{\mathrm{o}}$

Zadanie $l8. (0-1)$

Prosta przechodząca przez punkty $A=(3,-2)\mathrm{i}B=(-1,6)$ jest określona równaniem

A. $y=-2x+4$

B. $y=-2x-8$

C. $y=2x+8$

D. $y=2x-4$

Strona 8 z26

MMA-IP





{\it BRUDNOPIS} ({\it nie podlega ocenie})

$\mathrm{A}_{-}1\mathrm{P}$

Strona 9 z26





Zadanie $l9. (0\rightarrow 1)$

Danyjest trójkąt prostokątny o kątach ostrych $\alpha \mathrm{i}\beta$ (zobacz rysunek).

Wyrazenie $ 2\cos\alpha-\sin\beta$ jest równe

A. $ 2\sin\beta$

B. $\cos\alpha$

C. 0

D. 2

Zadanie 20. $(0-1\rangle$

Punkt $B$ jest obrazem punktu $A=(-3,5) \mathrm{w}$

współrzędnych. DługoŚć odcinka $AB$ jest równa

symetrii względem

początku układu

A. $2\sqrt{34}$

B. 8

C. $\sqrt{34}$

D. 12

Zadanie 21. $(0-1\rangle$

Ilejest wszystkich dwucyfrowych liczb naturalnych utworzonych z cyfr: 1, 3, 5, 7, 9, w których

cyfry się nie powtarzają?

A. 10

B. 15

C. 20

D. 25

Zadanie 22. $(0-1\rangle$

Pole prostokąta ABCD jest równe 90. Na bokachAB $\mathrm{i}$ {\it CD} wybrano -odpowiednio -punkty {\it P}$\mathrm{i}R,$

takie, $\displaystyle \dot{\mathrm{z}}\mathrm{e}\frac{|AP|}{|PB|}=\frac{|CR|}{|RD|}=\frac{3}{2}$ (zobacz rysunek).
\begin{center}
\includegraphics[width=78.180mm,height=48.672mm]{./F2_M_PP_M2020_page9_images/image001.eps}
\end{center}
{\it D R  C}

{\it A  P B}

Pole czworokąta APCR jest równe

A. 36

B. 40

C. 54

D. 60

Strona 10 z26

MMA-IP







CENTRALNA

KOMISJA

EGZAMINACYJNA

Arkusz zawiera informacje prawnie chronione

do momentu rozpoczecia egzaminu.

KOD

WYPELNIA ZDAJACY

PESEL

{\it Miejsce na naklejke}.

{\it Sprawdz}', {\it czy kod na naklejce to}

E-100.
\begin{center}
\includegraphics[width=21.900mm,height=10.212mm]{./F2_M_PP_M2021_page0_images/image001.eps}

\includegraphics[width=79.608mm,height=10.212mm]{./F2_M_PP_M2021_page0_images/image002.eps}
\end{center}
$J\mathrm{e}\dot{\mathrm{z}}$ {\it eli tak}- {\it przyklej naklejkq}.

{\it lezeli nie}- {\it zgtoś to nauczycielowi}.

EGZAMIN MATURALNY Z MATEMATYKI

POZIOM PODSTAWOWY

WYPELN[A ZESPÓL NADZORUJACY

DAT$\mathrm{A}^{\cdot}$ 5 maja 202l $\mathrm{r}.$

GODZINA $\mathrm{R}\mathrm{O}\mathrm{Z}\mathrm{P}\mathrm{O}\mathrm{C}\mathrm{Z}\xi \mathrm{C}\mathrm{l}\mathrm{A}:9:00$

CZAS PRACY: $\{70 \displaystyle \min \mathrm{u}\mathrm{t}$

LICZBA PUNKTÓW DO UZYSKANIA 45

Uprawnienia zdajqcego do:

\fbox{} dostosowania zasad oceniania

\fbox{} dostosowania w zw. z dyskalkulia

\fbox{} nieprzenoszenia zaznaczeń na karte.

$\Vert\Vert\Vert\Vert\Vert\Vert\Vert\Vert\Vert\Vert\Vert\Vert\Vert\Vert\Vert\Vert\Vert\Vert\Vert\Vert\Vert\Vert\Vert\Vert\Vert\Vert\Vert\Vert\Vert\Vert|$

EMAP-P0-100-2105

lnstrukcja dla zdajqcego

l. Sprawdz', czy arkusz egzaminacyjny zawiera 25 stron (zadania $1-35$).

Ewentualny brak zgloś przewodniczqcemu zespolu nadzorujqcego egzamin.

2. Na tej stronie oraz na karcie odpowiedzi wpisz swój numer PESEL i przyklej naklejk9

z kodem.

3. Nie wpisuj $\dot{\mathrm{z}}$ adnych znaków w cześci przeznaczonej dla egzaminatora.

4. Rozwiqzania zadań i odpowiedzi wpisuj w miejscu na to przeznaczonym.

5. Odpowiedzi do zadań $\mathrm{z}\mathrm{a}\mathrm{m}\mathrm{k}\mathrm{n}\mathrm{i}_{9}$tych ($1-28)$ zaznacz na karcie odpowiedzi w cz9ści

karty przeznaczonej dla zdajqcego. Zamaluj $\blacksquare$ pola do tego przeznaczone. Bledne

zaznaczenie otocz kólkiem @ i zaznacz wlaściwe.

6. Pamietaj, $\dot{\mathrm{z}}\mathrm{e}$ pominiecie argumentacji lub istotnych obliczeń w rozwiqzaniu zadania

otwartego (29-35) $\mathrm{m}\mathrm{o}\dot{\mathrm{z}}\mathrm{e}$ spowodowač, $\dot{\mathrm{z}}\mathrm{e}$ za to rozwiqzanie nie otrzymasz pelnej

liczby punktów.

7. Pisz czytelnie i $\mathrm{u}\dot{\mathrm{z}}$ ywaj tylko dlugopisu lub pióra z czarnym tuszem lub atramentem.

8. Nie $\mathrm{u}\dot{\mathrm{z}}$ ywaj korektora, a bledne zapisy wyraz'nie przekreśl.

9. Pamiptaj, $\dot{\mathrm{z}}\mathrm{e}$ zapisy w brudnopisie nie $\mathrm{b}9\mathrm{d}\mathrm{q}$ oceniane.

10. $\mathrm{M}\mathrm{o}\dot{\mathrm{z}}$ esz korzystač z zestawu wzorów matematycznych, cyrkla i linijki oraz kalkulatora

prostego.

Uklad graficzny

\copyright CKE 2021




{\it Wkazdym z zadań od f. do 28. wybierz izaznacz na karcie odpowiedzi poprawnq odpowiedz}'.

Zadanie 1. (0-1)

Liczba $100^{5}\cdot(0,1)^{-6}$ jest równa

A. $10^{13}$

B. $10^{16}$

C. $10^{-1}$

D. $10^{-30}$

Zadanie 2. $\{0-l\mathrm{I}$

Liczba 78 stanowi 150\% 1iczby $c$. Wtedy liczba $c$ jest równa

A. 60

B. 52

C. 48

D. 39

Zadanie 3. $\langle 0-ll$

Rozwazamy przedzialy liczbowe $(-\infty,5) \mathrm{i} \langle-1, +\infty$). lle jest wszystkich liczb calkowitych,

które nalezq jednocześnie do obu rozwazanych przedzialów?

A. 6

B. 5

C. 4

D. 7

Zadanie 4. $\{0-l\mathrm{I}$

Suma 2 $\log\sqrt{10}+\log 10^{\mathrm{s}}$ jest równa

A. 2

B. 3

C. 4

D. 5

Zadänie 5. $\langle 0-ll$

Róznica $0,(3)-\displaystyle \frac{23}{33}$ jest równa

A. $-0,(39)$

B. $-\displaystyle \frac{39}{100}$

C. $-0,36$

D. $-\displaystyle \frac{4}{11}$

Zadänie 6. ćO-1)

Zbiorem wszystkich rozwiqzań nierówności $\displaystyle \frac{2-x}{2}-2x\geq 1$ jest przedzial

A. $\langle 0, +\infty)$

B. $(-\infty,  0\rangle$

C. $(-\infty,  5\rangle$

D.(-$\infty$,-31\}

Strona 2 z25

$\mathrm{E}\mathrm{M}\mathrm{A}\mathrm{P}-\mathrm{P}0_{-}100$





-{\it RUDNOPIS} \{{\it nie podlega ocenie}\}

$-\mathrm{P}0_{-}100$

Strona ll z 25





Zadänie 22. (0-1)

$\mathrm{W}$ równolegloboku ABCD, przedstawionym na rysunku, kqt $\alpha$ ma miar9 $70^{\mathrm{o}}$
\begin{center}
\includegraphics[width=94.392mm,height=45.720mm]{./F2_M_PP_M2021_page11_images/image001.eps}
\end{center}
{\it D  C}

$\alpha  \beta$

{\it A  B}

Wtedy kqt $\beta$ ma miar9

A. $80^{\mathrm{o}}$

B. $70^{\mathrm{o}}$

C. $60^{\mathrm{o}}$

D. $50^{\mathrm{o}}$

Zadanie 23. $\langle 0-1$)

$\mathrm{W}\mathrm{k}\mathrm{a}\dot{\mathrm{z}}$ dym $n$-kqcie wypuklym $(n\geq 3)$ liczba przekqtnych jest równa $\displaystyle \frac{n(n-3)}{2}$ Wielokqtem

wypuklym, w którym liczba przekqtnych jest o 25 wieksza od 1iczby boków, jest

A. siedmiokqt.

B. dziesieciokqt.

C. dwunastokqt.

D. pietnastokqt.

Zadänie 24. (0-1)

Pole figury $F_{1}$ zlozonej z dwóch stycznych zewnetrznie kól o promieniach l $\mathrm{i} 3$ jest równe

polu figury $F_{2}$ zlozonej z dwóch stycznych zewnptrznie kól o promieniach dlugości $r$ (zobacz

rysunek).

Figura $F_{1}$

Figura $F_{2}$
\begin{center}
\includegraphics[width=60.456mm,height=45.768mm]{./F2_M_PP_M2021_page11_images/image002.eps}

\includegraphics[width=75.588mm,height=38.148mm]{./F2_M_PP_M2021_page11_images/image003.eps}
\end{center}
{\it r r}

D\}ugośč r promieniajest równa

A. $\sqrt{3}$

B. 2

C. $\sqrt{5}$

D. 3

Strona 12 z25

$\mathrm{E}\mathrm{M}\mathrm{A}\mathrm{P}-\mathrm{P}0_{-}100$





-{\it RUDNOPIS} \{{\it nie podlega ocenie}\}

$-\mathrm{P}0_{-}100$

Strona 13 z25





Zadänie 25. (0-1)

Punkt $A=(3,-5)$ jest wierzcholkiem kwadratu ABCD, a punkt $M=(1,3)$ jest punktem

$\mathrm{p}\mathrm{r}\mathrm{z}\mathrm{e}\mathrm{c}\mathrm{i}_{9}\mathrm{c}\mathrm{i}\mathrm{a}$ si9 przekatnych tego kwadratu. Wynika stqd, $\dot{\mathrm{z}}\mathrm{e}$ pole kwadratu ABCD jest równe

A. 68

B. 136

C. $2\sqrt{34}$

D. $8\sqrt{34}$

Zadanie 26. (0-1)

$\mathrm{Z}$ wierzcholków sześcianu ABCDEFGH losujemy jednocześnie dwa rózne wierzcholki.

Prawdopodobieństwo tego, $\dot{\mathrm{z}}\mathrm{e}$ wierzcholki te bedq końcami przekqtnej sześcianu

{\it ABCDEFGH, jest równe}

A. -71

B. -47

C. $\displaystyle \frac{1}{14}$

D. -73

Zadanie 27. $\{0-1$)

Wszystkich liczb naturalnych trzycyfrowych, wiekszych od 700, w których $\mathrm{k}\mathrm{a}\dot{\mathrm{z}}$ da cyfra nalez $\mathrm{y}$

do zbioru \{1, 2, 3, 7, 8, 9\} i $\dot{\mathrm{z}}$ adna cyfra $\mathrm{s}\mathrm{i}\mathrm{e}$ nie powtarza, jest

A. 108

B. 60

C. 40

D. 299

Zadanie 28. $\{0-1\}$

Sześciowyrazowy ciqg liczbowy $(1,2,2x,x+2,5,6)$

tego ciqgu jest równa 4. Wynika stqd, $\dot{\mathrm{z}}\mathrm{e}$

jest niemalejqcy. Mediana wyrazów

A. $x=1$

B. $\chi=$ -23

$-. x=2$

D. $\chi=$ -83

Strona 14 z25

$\mathrm{E}\mathrm{M}\mathrm{A}\mathrm{P}-\mathrm{P}0_{-}100$





-{\it RUDNOPIS} \{{\it nie podlega ocenie}\}

$-\mathrm{P}0_{-}100$

Strona 15z 25





Zadänie 29. $(0-2)$

Rozwiqz nierównośč:

$x^{2}-5x\leq 14$

Odpowiedz':

Strona 16 z25

$\mathrm{E}\mathrm{M}\mathrm{A}\mathrm{P}-\mathrm{P}0_{-}100$





Zadänie 30. $(0-2)$

Wykaz, $\dot{\mathrm{z}}\mathrm{e}$ dla $\mathrm{k}\mathrm{a}\dot{\mathrm{z}}$ dych trzech dodatnich liczb $a, b$

nierównośč

-{\it ab}$<$--{\it ba}$++${\it cc}

i

$c$ takich, $\dot{\mathrm{z}}\mathrm{e} a<b$, spelniona jest
\begin{center}
\begin{tabular}{|l|l|l|l|}
\cline{2-4}
&	\multicolumn{1}{|l|}{Nr zadania}&	\multicolumn{1}{|l|}{$29.$}&	\multicolumn{1}{|l|}{ $30.$}	\\
\cline{2-4}
&	\multicolumn{1}{|l|}{Maks. liczba pkt}&	\multicolumn{1}{|l|}{$2$}&	\multicolumn{1}{|l|}{ $2$}	\\
\cline{2-4}
\multicolumn{1}{|l|}{egzaminator}&	\multicolumn{1}{|l|}{Uzyskana liczba pkt}&	\multicolumn{1}{|l|}{}&	\multicolumn{1}{|l|}{}	\\
\hline
\end{tabular}

\end{center}
$\mathrm{E}\mathrm{M}\mathrm{A}\mathrm{P}-\mathrm{P}0_{-}100$

Strona 17 z25





Zadänie 31. $(0-2l$

Funkcja liniowa $f$ przyjmuje wartośč

Wyznacz wzór funkcji $f.$

2 dla argumentu

0, a ponadto $f(4)-f(2)=6.$

Odpowiedz':

Strona 18 z25

$\mathrm{E}\mathrm{M}\mathrm{A}\mathrm{P}-\mathrm{P}0_{-}100$





Zadänie 32. $(0-2)$

Rozwiqz równanie:

$\displaystyle \frac{3x+2}{3x-2}=4-x$

Odpowiedz':
\begin{center}
\begin{tabular}{|l|l|l|l|}
\cline{2-4}
&	\multicolumn{1}{|l|}{Nr zadania}&	\multicolumn{1}{|l|}{$31.$}&	\multicolumn{1}{|l|}{ $32.$}	\\
\cline{2-4}
&	\multicolumn{1}{|l|}{Maks. liczba pkt}&	\multicolumn{1}{|l|}{$2$}&	\multicolumn{1}{|l|}{ $2$}	\\
\cline{2-4}
\multicolumn{1}{|l|}{egzaminator}&	\multicolumn{1}{|l|}{Uzyskana liczba pkt}&	\multicolumn{1}{|l|}{}&	\multicolumn{1}{|l|}{}	\\
\hline
\end{tabular}

\end{center}
$\mathrm{E}\mathrm{M}\mathrm{A}\mathrm{P}-\mathrm{P}0_{-}100$

Strona 19 z25





Zadänie 33. $(0-2)$

Trójkqt równoboczny $ABC$ ma pole równe $9\sqrt{3}$. Prosta równolegla do boku $BC$ przecina

boki AB $\mathrm{i} AC -$ odpowiednio-w punktach $K \mathrm{i} L$. Trójkqty $ABC \mathrm{i} AKL$ sq podobne,

a stosunek dlugości boków tych trójkqtówjest równy $\displaystyle \frac{3}{2}$. Oblicz dlugośč boku trójkqta $AKL.$

Odpowiedz':

Strona 20 z25

$\mathrm{E}\mathrm{M}\mathrm{A}\mathrm{P}-\mathrm{P}0_{-}100$





-{\it RUDNOPIS} \{{\it nie podlega ocenie}\}

$-\mathrm{P}0_{-}100$

Strona 3 z25





Zadänie 34. $(0-2)$

Gracz rzuca dwukrotnie symetrycznq sześciennq kostkq do gry i oblicza sum9 1iczb

wyrzuconych oczek. Oblicz prawdopodobieństwo zdarzenia $A$ polegajqcego na tym, $\dot{\mathrm{z}}\mathrm{e}$ suma

liczb wyrzuconych oczekjest równa 4 1ub 5, 1ub 6.

Odpowiedz':
\begin{center}
\begin{tabular}{|l|l|l|l|}
\cline{2-4}
&	\multicolumn{1}{|l|}{Nr zadania}&	\multicolumn{1}{|l|}{$33.$}&	\multicolumn{1}{|l|}{ $34.$}	\\
\cline{2-4}
&	\multicolumn{1}{|l|}{Maks. liczba pkt}&	\multicolumn{1}{|l|}{$2$}&	\multicolumn{1}{|l|}{ $2$}	\\
\cline{2-4}
\multicolumn{1}{|l|}{egzaminator}&	\multicolumn{1}{|l|}{Uzyskana liczba pkt}&	\multicolumn{1}{|l|}{}&	\multicolumn{1}{|l|}{}	\\
\hline
\end{tabular}

\end{center}
$\mathrm{E}\mathrm{M}\mathrm{A}\mathrm{P}-\mathrm{P}0_{-}100$

Strona 21 z25





Zadänie 35. $(0-5$\}

Punkty $A=(-20,12) \mathrm{i} B=(7,3)$ sq wierzcholkami trójkqta równoramiennego $ABC,$

w którym $|AC|=|BC|$. Wierzcholek $C \mathrm{l}\mathrm{e}\dot{\mathrm{z}}\mathrm{y}$ na osi $0\mathrm{y}$ ukladu wspólrzednych. Oblicz

wspólrz9dne wierzcho1ka $C$ oraz obwód tego trójkqta.

Strona 22 z25

$\mathrm{E}\mathrm{M}\mathrm{A}\mathrm{P}-\mathrm{P}0_{-}100$





Odpowiedz':

Wypelnia

egzaminator

Nr zadania

Maks. liczba pkt

Uzyskana liczba pkt

35.

5

$\mathrm{E}\mathrm{M}\mathrm{A}\mathrm{P}-\mathrm{P}0_{-}100$

Strona 23 z25





-{\it RUDNOPIS} \{{\it nie podlega ocenie}\}

Strona 24z 25

$\mathrm{E}\mathrm{M}\mathrm{A}\mathrm{P}-\mathrm{P}0_{-}1$





$|-100$

Strona 25 z25




















Zadänie $7_{1}. (0-1)$

Na ponizszym rysunku przedstawiono wykres funkcji $f$ określonej w zbiorze $\langle-6, 5\rangle.$

Funkcja g

prawdziwe.

jest określona wzorem

$g(x)=f(x)-2$ dla

$\chi\in\langle-6,5\rangle$. Wskaz zdanie

A. Liczba $f(2)+g(2)$ jest równa $(-2).$

B. Zbiory wartości funkcji $f \mathrm{i} g$ sq równe.

C. Funkcje $f \mathrm{i} g$ majq te same miejsca zerowe.

D. Punkt $P=(0,-2)$ nalez $\mathrm{y}$ do wykresów funkcji $f \mathrm{i} g.$

Zadanie S. $\langle 0-1$)

Na rysunku obok przedstawiono geometrycznq

interpretacje jednego z $\mathrm{n}\mathrm{i}\dot{\mathrm{z}}$ ej zapisanych ukladów

równań. Wska $\dot{\mathrm{z}}$ ten uklad, którego geometrycznq

interpretacje przedstawiono na rysunku.

A. 

B. 

C. 

D. 

Strona 4 z25

$\mathrm{E}\mathrm{M}\mathrm{A}\mathrm{P}-\mathrm{P}0_{-}100$





-{\it RUDNOPIS} \{{\it nie podlega ocenie}\}

$-\mathrm{P}0_{-}100$

Strona 5 z25





Zadänie 9. (0-1)

Proste o równaniach $y=3x-5$ oraz $y=\displaystyle \frac{m-3}{2}x+\frac{9}{2}$ sq równolegle, gdy

A. $m=1$

B. $m=3$

C. $m=6$

D. $m=9$

Zadanie 10. (0-1)

Funkcja $f$ jest określona wzorem $f(x)=\displaystyle \frac{\chi^{2}}{2x-2}$ dla $\mathrm{k}\mathrm{a}\dot{\mathrm{z}}$ dej liczby rzeczywistej $\chi\neq 1$. Wtedy

dla argumentu $x=\sqrt{3}-1$ wartośč funkcji $f$ jest równa

A. $\displaystyle \frac{1}{\sqrt{3}-1}$

B. $-1$

C. l

D. $\displaystyle \frac{1}{\sqrt{3}-2}$

Zadanie $l1_{\varepsilon}(0-1)$

Do wykresu funkcji $f$ określonej dla $\mathrm{k}\mathrm{a}\dot{\mathrm{z}}$ dej liczby rzeczywistej $x$

nalez $\mathrm{y}$ punkt o wspólrzednych

wzorem $f(x)=3^{\chi}-2$

A. $(-1,-5)$

B. $(0,-2)$

C. $(0,-1)$

D. (2, 4)

Zadanie $12_{s}(0-1)$

Funkcja kwadratowa

w przedziale

f określona wzorem

$f(x)=-2(x+1)(x-3)$

jest malejqca

A. $\langle 1, +\infty)$

B. $(-\infty,  1\rangle$

C. $(-\infty, -8\rangle$

D. $\langle-8, +\infty)$

Zadanie 13. $(0-1$\}

Trzywyrazowy ciag $(15,3x,\displaystyle \frac{5}{3})$ jest geometryczny i wszystkiejego wyrazy sq dodatnie. Stqd

wynika, $\dot{\mathrm{z}}\mathrm{e}$

A. $\chi=$ -53

B. $\chi=$ -45

C. $x=1$

D. $\chi=$ -35

Zadanie 14. (0-1)

Ciqg $(b_{n})$ jest określonywzorem $b_{n}=3n^{2}-25n$ dla $\mathrm{k}\mathrm{a}\dot{\mathrm{z}}$ dej liczby naturalnej $n\geq 1$. Liczba

niedodatnich wyrazów ciqgu $(b_{n})$ jest równa

A. 14

B. 13

C. 9

D. 8

Strona 6 z25

$\mathrm{E}\mathrm{M}\mathrm{A}\mathrm{P}-\mathrm{P}0_{-}100$





-{\it RUDNOPIS} \{{\it nie podlega ocenie}\}

$-\mathrm{P}0_{-}100$

Strona 7 z25





Zadänie 15. (0-1)

Ciqg arytmetyczny $(a_{n})$ jest określony dla $\mathrm{k}\mathrm{a}\dot{\mathrm{z}}$ dej liczby naturalnej $n\geq 1$. Trzeci i piqty wyraz

ciqgu spelniajq warunek $a_{3}+a_{5}=58$. Wtedy czwarty wyraz tego ciqgu jest równy

A. 28

B. 29

C. 33

D. 40

Zadanie 16. (0-1)

Dla $\mathrm{k}\mathrm{a}\dot{\mathrm{z}}$ dego kqta ostrego $\alpha$ iloczyn $\displaystyle \frac{\cos\alpha}{1-\sin^{2}\alpha} \displaystyle \frac{1-\cos^{2}\alpha}{\sin\alpha}$ jest równy

A. $\sin\alpha$

B. $\mathrm{t}\mathrm{g}\alpha$

C. $\cos\alpha$

D. $\sin^{2}\alpha$

Zadanie 17. (0-1)

Prosta $k$ jest styczna w punkcie $A$ do okregu o środku 0. Punkt $B \mathrm{l}\mathrm{e}\dot{\mathrm{z}}\mathrm{y}$ na tym okregu

i miara kqta $A0B$ jest równa $80^{\mathrm{o}}$. Przez punkty 0 $\mathrm{i} B$ poprowadzono prostq, która przecina

prostq $k$ w punkcie $C$ (zobacz rysunek).
\begin{center}
\includegraphics[width=143.460mm,height=35.208mm]{./F2_M_PP_M2021_page7_images/image001.eps}
\end{center}
{\it 0}

{\it B}

$80^{\mathrm{o}}$

{\it C  k}

{\it A}

B. $\displaystyle \frac{37}{3}$

A. 12

Pole tego trójkqta jest równe

Zadänie $l8. (0-1)$

Przyprostokqtna $AC$ trójkata prostokqtnego

rysunek).

B. $30^{\mathrm{o}}$

A. $10^{\mathrm{o}}$

Miara kqta BAC jest równa

C. $40^{\mathrm{o}}$

D. $50^{\mathrm{o}}$

$ABC$ ma dlugośč 8 oraz $\displaystyle \mathrm{t}\mathrm{g}\alpha=\frac{2}{5}$

(zobacz
\begin{center}
\includegraphics[width=66.852mm,height=33.684mm]{./F2_M_PP_M2021_page7_images/image002.eps}
\end{center}
{\it B}

$\alpha$

{\it C} 8  {\it A}

C. $\displaystyle \frac{62}{5}$

D. $\displaystyle \frac{64}{5}$

Strona 8 z25

$\mathrm{E}\mathrm{M}\mathrm{A}\mathrm{P}-\mathrm{P}0_{-}100$





-{\it RUDNOPIS} \{{\it nie podlega ocenie}\}

$-\mathrm{P}0_{-}100$

Strona 9 z25





Zadänie $l9. (0-1)$

Pole pewnego trójkqta równobocznego jest równe $\displaystyle \frac{4\sqrt{3}}{9}$. Obwód tego trójkqta jest równy

A. 4

B. 2

C. -43

D. -23

Zadanie 20. $(0-1$\}

$\mathrm{W}$ trójkqcie $ABC$ bok $BC$ ma dlugość 13, a wysokośč

$CD$ tego trójkqta dzieli bok $AB$ na odcinki o dtugościach

$|AD|=3 \mathrm{i} |BD|=12$ (zobacz rysunek obok). Dlugośč

boku $AC$ jest równa
\begin{center}
\includegraphics[width=64.572mm,height=28.452mm]{./F2_M_PP_M2021_page9_images/image001.eps}
\end{center}
{\it C}

13

{\it A} 3  {\it D}  12  {\it B}

A. $\sqrt{34}$

B. $\displaystyle \frac{13}{4}$

C. $2\sqrt{14}$

D. $3\sqrt{45}$

Zadänie 21. (0-1)

Punkty $A, B, C \mathrm{i} D \mathrm{l}\mathrm{e}\dot{\mathrm{z}}\mathrm{q}$ na okregu o środku $S$. Miary kqtów $SBC, BCD, CDA \mathrm{s}_{\mathrm{c}}1$ równe

odpowiednio: $|4SBC|=60^{\mathrm{o}}, |4BCD|=110^{\mathrm{o}}, |4CDA|=90^{\mathrm{o}}$ (zobacz rysunek).
\begin{center}
\includegraphics[width=60.192mm,height=64.464mm]{./F2_M_PP_M2021_page9_images/image002.eps}
\end{center}
{\it C}

$110^{\mathrm{o}}$

$60^{\mathrm{o}}$  {\it B}

{\it D  S}

$\alpha$

{\it A}

Wynika stqd, $\dot{\mathrm{z}}\mathrm{e}$ miara $\alpha$ kqta DAS jest równa

A. $25^{\mathrm{o}}$

B. $30^{\mathrm{o}}$

C. $35^{\mathrm{o}}$

D. $40^{\mathrm{o}}$

Strona 10 z25

$\mathrm{E}\mathrm{M}\mathrm{A}\mathrm{P}-\mathrm{P}0_{-}100$







CENTRALNA

KOMISJA

EGZAMINACYJNA

Arkusz zawiera informacje prawnie chronione

do momentu rozpoczecia egzaminu.

WYPELNIA ZDAJACY

{\it Miejsce na naklejke}.

{\it Sprawdz}', {\it czy kod na naklejce to}

e-100.
\begin{center}
\includegraphics[width=21.900mm,height=16.260mm]{./F2_M_PP_M2022_page0_images/image001.eps}
\end{center}
KOD
\begin{center}
\includegraphics[width=79.656mm,height=16.260mm]{./F2_M_PP_M2022_page0_images/image002.eps}
\end{center}
PESEL

{\it Jezeli tak}- {\it przyklej naklejkq}.

{\it Jezeli nie}- {\it zgtoś to nauczycielowi}.

EGZAMIN MATURALNY Z MATEMATYKI

POZIOM PODSTAWOWY

WYPELNIA ZESPÓt NADZORUJACY

DATA: 5 maja 2022 $\mathrm{r}.$

GODZINA ROZPOCZeClA: 9: 00

CZAS PRACY: $\{70 \displaystyle \min \mathrm{u}\mathrm{t}$

LICZBA PUNKTÓW DO UZYSKANIA: 45

Uprawnienia zdaj\S cego do:

\fbox{} nieprzenoszenia zaznaczeń na karte

\fbox{} dostosowania zasad oceniania

\fbox{} dostosowania w zw. z dyskalkuliq.

$\Vert\Vert\Vert\Vert\Vert\Vert\Vert\Vert\Vert\Vert\Vert\Vert\Vert\Vert\Vert\Vert\Vert\Vert\Vert\Vert\Vert\Vert\Vert\Vert\Vert\Vert\Vert\Vert\Vert\Vert|$

EMAP-P0-100-2205

lnstrukcja dla zdajqcego

l. Sprawdz', czy arkusz egzaminacyjny zawiera 25 stron (zadania $1-35$).

Ewentualny brak zgloś przewodniczacemu zespolu nadzorujacego egzamin.

2. Na tej stronie oraz na karcie odpowiedzi wpisz swój numer PESEL i przyklej naklejke

z kodem.

3. Nie wpisuj $\dot{\mathrm{z}}$ adnych znaków w cz9ści przeznaczonej d1a egzaminatora.

4. Rozwiqzania zadań i odpowiedzi wpisuj w miejscu na to przeznaczonym.

5. Odpowiedzi do zadań $\mathrm{z}\mathrm{a}\mathrm{m}\mathrm{k}\mathrm{n}\mathrm{i}_{9}$tych ($1-28)$ zaznacz na karcie odpowiedzi w cześci

karty przeznaczonej dla zdajacego. Zamaluj $\blacksquare$ pola do tego przeznaczone. $\mathrm{B}$pdne

zaznaczenie otocz kólkiem @ i zaznacz wlaściwe.

6. Pamietaj, $\dot{\mathrm{z}}\mathrm{e}$ pominiecie argumentacji lub istotnych obliczeń w rozwiqzaniu zadania

otwartego (29-35) $\mathrm{m}\mathrm{o}\dot{\mathrm{z}}\mathrm{e}$ spowodowač, $\dot{\mathrm{z}}\mathrm{e}$ za to rozwiqzanie nie otrzymasz pelnej

liczby punktów.

7. Pisz czytelnie i $\mathrm{u}\dot{\mathrm{z}}$ ywaj tylko dlugopisu lub pióra z czarnym tuszem lub atramentem.

8. Nie $\mathrm{u}\dot{\mathrm{z}}$ ywaj korektora, a bledne zapisy wyraz'nie przekreśl.

9. Pamietaj, $\dot{\mathrm{z}}\mathrm{e}$ zapisy w brudnopisie nie bedq oceniane.

10. $\mathrm{M}\mathrm{o}\dot{\mathrm{z}}$ esz korzystač z zestawu wzorów matematycznych, cyrkla i linijki oraz kalkulatora

prostego.

Uk\}ad graficzny

\copyright CKE 2021




{\it Wkazdym z zadań od} $f.$ {\it do 28. wybierz izaznacz na karcie odpowiedzi poprawna} $od\sqrt{}owi\mathrm{e}d\acute{z}.$

Zadanie $\mathrm{f}. (0-1$\}

Liczba $(2\sqrt{8}-3\sqrt{2})^{2}$ jest równa

A. 2

B. l

C. 26

D. 14

ZadanIe 2. $(0-1$\}

Dodatnie liczby $x \mathrm{i} y \mathrm{s}\mathrm{p}\mathrm{e}$niajq warunek $2x=3y$. Wynika stqd, $\dot{\mathrm{z}}\mathrm{e}$ wartośč wyrazenia

$\displaystyle \frac{x^{2}+y^{2}}{x\cdot y}$ jest równa

A. -23

B. $\displaystyle \frac{13}{6}$

C. $\displaystyle \frac{6}{13}$

D. -23

Zadanie 3. (0-1)

Liczba $4\log_{4}2+2\log_{4}8$ jest równa

A. 61og410

B. 16

C. 5

D. 61og416

Zädanie 4. (0-1)

Cena dzialki po kolejnych dwóch obnizkach, za $\mathrm{k}\mathrm{a}\dot{\mathrm{z}}$ dym razem o 10\% w odniesieniu do ceny

obowiqzujqcej w danym momencie, jest równa 78732 z1. Cena tej dzia1ki przed obiema

obnizkami byla, w zaokragleniu do l zl, równa

A. 98732 z1

B. 97200 z1

C. 95266 z1

D. 94478 z1

Zadanie 5. $(0-1\rangle$

Liczba 3 $2+\displaystyle \frac{1}{4}$ jest równa

A. $3^{2} \sqrt[4]{3}$

B. $\sqrt[4]{3^{3}}$

C. $3^{2}+\sqrt[4]{3}$

D. $3^{2}+ \sqrt{3^{4}}$

Strona 2 z25

$\mathrm{E}\mathrm{M}\mathrm{A}\mathrm{P}-\mathrm{P}0_{-}100$





: {\it RU DNOPIS} \{{\it nie podlega ocenie}\}

$\mathrm{h}\mathrm{P}-\mathrm{P}0_{-}100$

Strona ll z25





$\mathrm{Z}\mathrm{a}\mathrm{d}\mathrm{a}*\mathrm{i}\mathrm{e}19. \langle 0-1\}$

Wysokośč trójkqta równobocznego jest równa $6\sqrt{3}$. Pole tego trójkqta jest równe

A. $3\sqrt{3}$

B. $4\sqrt{3}$

C. $27\sqrt{3}$

D. $36\sqrt{3}$

Zadanie 20. (0-1)

Boki równolegloboku maja dlugości 6 $\mathrm{i} 10$, a kqt rozwarty mipdzy tymi bokami ma

miar9 $120^{\mathrm{o}}$ Pole tego równolegloboku jest równe

A. $30\sqrt{3}$

B. 30

C. $60\sqrt{3}$

D. 60

Zadanie 21. $\langle 0\rightarrow 1$)

Punkty $A=(-2,6)$ oraz $B=(3,b) \mathrm{l}\mathrm{e}\dot{\mathrm{z}}$ a na prostej, która przechodzi przez poczatek

ukladu wspólrzednych. Wtedy $b$ jest równe

A. 9

B. $(-9)$

C. $(-4)$

D. 4

Zadanie $22_{r}(0-1)$

Dane sq cztery proste $k, l, m, n$ o równaniach:

$k$: $\mathrm{y}=-x+1$

$l$: $y=\displaystyle \frac{2}{3}x+1$

$m$: $\displaystyle \mathrm{y}=-\frac{3}{2}x+4$

$n$: $y=-\displaystyle \frac{2}{3}x-1$

Wśród tych prostych prostopadle sa

A. proste k oraz l.

B. proste k oraz n.

C. proste l oraz m.

D. proste m oraz n.

Zadanie 23. (0-1)

Punkty $K=(4,-10) \mathrm{i} L=(b,2)$ sa końcami odcinka $KL$. Pierwsza wspólrz9dna środka

odcinka $KL$ jest równa $(-12)$. Wynika stqd, $\dot{\mathrm{z}}\mathrm{e}$

A. $b=-28$

B. $b=-14$

C. $b=-24$

D. $b=-10$

Strona 12 z25

$\mathrm{E}\mathrm{M}\mathrm{A}\mathrm{P}-\mathrm{P}0_{-}100$





: {\it RU DNOPIS} \{{\it nie podlega ocenie}\}

$\mathrm{h}\mathrm{P}-\mathrm{P}0_{-}100$

Strona 13 z25





Zadarie 24. $(0-1$\}

Punkty $A=(-4,4) \mathrm{i} B=(4,0)$ sq sqsiednimi wierzcholkami kwadratu ABCD. Przekqtna

tego kwadratu ma dlugośč

A. $4\sqrt{10}$

B. $4\sqrt{2}$

C. $4\sqrt{5}$

D. $4\sqrt{7}$

Zadanie 25. (0-1)

Podstawq graniastoslupa prostego jest romb o przekqtnych dlugości 7 cm i 10 cm.

Wysokośč tego graniastoslupa jest krótsza od dluzszej przekqtnej rombu o 2 cm. Wtedy

obj9tośč graniastos1upa jest równa

A. 560 $\mathrm{c}\mathrm{m}^{3}$

B. 280 $\mathrm{c}\mathrm{m}^{3}$

C. $\displaystyle \frac{280}{3}\mathrm{c}\mathrm{m}^{3}$

D. $\displaystyle \frac{560}{3}\mathrm{c}\mathrm{m}^{3}$

Zadanie $26_{*}(0-1)$

Danyjest sześcian ABCDEFGH o krawedzi dlugości $a.$

Punkty $E, F, G, B$ sa wierzcholkami ostroslupa EFGB

(zobacz rysunek).
\begin{center}
\includegraphics[width=57.864mm,height=56.484mm]{./F2_M_PP_M2022_page13_images/image001.eps}
\end{center}
{\it H}

II

{\it G}

{\it E}  IIII

{\it F}

III

I

I

I

I

I

I

I

-- --- $C$

{\it A  B}

{\it a}

Pole powierzchni calkowitej ostroslupa EFGB jest równe

A. $a^{2}$

B. $\displaystyle \frac{3\sqrt{3}}{2}\cdot a^{2}$

C. -23 {\it a}2

D. $\displaystyle \frac{3+\sqrt{3}}{2}\cdot a^{2}$

Zadanie 27. (0-1)

Wszystkich róznych liczb naturalnych czterocyfrowych nieparzystych podzielnych przez 5

jest

A. $9\cdot 8\cdot 7\cdot 2$

B. $9\cdot 10\cdot 10\cdot 1$

C. $9\cdot 10\cdot 10\cdot 2$

D. $9\cdot 9\cdot 8\cdot 1$

Zadanie 28. (0-1)

$\acute{\mathrm{S}}$ rednia arytmetyczna zestawu sześciu liczb: $2x$, 4, 6, 8, 11, 13, jest równa 5. Wynika stqd, $\dot{\mathrm{z}}\mathrm{e}$

A. $x=-1$

B. $x=7$

C. $x=-6$

D. $x=6$

Strona 14 z25

$\mathrm{E}\mathrm{M}\mathrm{A}\mathrm{P}-\mathrm{P}0_{-}100$





: {\it RU DNOPIS} \{{\it nie podlega ocenie}\}

$\mathrm{h}\mathrm{P}-\mathrm{P}0_{-}100$

Strona 15 z25





Zadarie 29. (0-2)

Rozwiqz nierównośč:

$3x^{2}-2x-9\geq 7$

Strona 16 z25

$\mathrm{E}\mathrm{M}\mathrm{A}\mathrm{P}-\mathrm{P}0_{-}10$





Zadarie 30. (0-2)

$\mathrm{W}$ ciqgu arytmetycznym $(a_{n})$, określonym dla $\mathrm{k}\mathrm{a}\dot{\mathrm{z}}$ dej liczby naturalnej $n\geq 1,$

$a_{1}=-1 \mathrm{i} a_{4}=8$. Oblicz sum9 stu poczqtkowych ko1ejnych wyrazów tego ciagu.
\begin{center}
\begin{tabular}{|l|l|l|l|}
\cline{2-4}
&	\multicolumn{1}{|l|}{Nr zadania}&	\multicolumn{1}{|l|}{$29.$}&	\multicolumn{1}{|l|}{ $30.$}	\\
\cline{2-4}
&	\multicolumn{1}{|l|}{Maks. liczba pkt}&	\multicolumn{1}{|l|}{$2$}&	\multicolumn{1}{|l|}{ $2$}	\\
\cline{2-4}
\multicolumn{1}{|l|}{egzaminator}&	\multicolumn{1}{|l|}{Uzyskana liczba pkt}&	\multicolumn{1}{|l|}{}&	\multicolumn{1}{|l|}{}	\\
\hline
\end{tabular}

\end{center}
$\mathrm{E}\mathrm{M}\mathrm{A}\mathrm{P}-\mathrm{P}0_{-}100$

Strona 17 z25





Zadanie 31. (0-2)

Wykaz, $\dot{\mathrm{z}}\mathrm{e}$ dla $\mathrm{k}\mathrm{a}\dot{\mathrm{z}}$ dej liczby rzeczywistej $a$ i $\mathrm{k}\mathrm{a}\dot{\mathrm{z}}$ dej liczby rzeczywistej $b$ takich, $\dot{\mathrm{z}}\mathrm{e} b\neq a,$

spelniona jest nierównośč

-{\it a}2 $+$2 {\it b}2 $>$ (-{\it a} $+$2 {\it b})2

Strona 18 z25

$\mathrm{E}\mathrm{M}\mathrm{A}\mathrm{P}-\mathrm{P}0_{-}100$





Zadanie 32. (0-2)

Kqt $\alpha$ jest ostry i tg $\alpha=2$. Oblicz wartośč wyrazenia $\sin^{2}\alpha.$
\begin{center}
\begin{tabular}{|l|l|l|l|}
\cline{2-4}
&	\multicolumn{1}{|l|}{Nr zadania}&	\multicolumn{1}{|l|}{$31.$}&	\multicolumn{1}{|l|}{ $32.$}	\\
\cline{2-4}
&	\multicolumn{1}{|l|}{Maks. liczba pkt}&	\multicolumn{1}{|l|}{$2$}&	\multicolumn{1}{|l|}{ $2$}	\\
\cline{2-4}
\multicolumn{1}{|l|}{egzaminator}&	\multicolumn{1}{|l|}{Uzyskana liczba pkt}&	\multicolumn{1}{|l|}{}&	\multicolumn{1}{|l|}{}	\\
\hline
\end{tabular}

\end{center}
$\mathrm{E}\mathrm{M}\mathrm{A}\mathrm{P}-\mathrm{P}0_{-}100$

Strona 19 z25





Zadanie 33. $\{0-2\}$

Danyjest trójkqt równoramienny $ABC$, w którym $|AC|=|BC|$. Dwusieczna kqta $BAC$

przecina bok $BC$ w takim punkcie $D, \dot{\mathrm{z}}\mathrm{e}$ trójkqty $ABC \mathrm{i} BDA$ sa podobne (zobacz

rysunek). Oblicz miare kqta $BAC.$

{\it C}
\begin{center}
\includegraphics[width=35.412mm,height=51.456mm]{./F2_M_PP_M2022_page19_images/image001.eps}
\end{center}
{\it D}

{\it A B}

Strona 20 z25

$\mathrm{E}\mathrm{M}\mathrm{A}\mathrm{P}-\mathrm{P}0_{-}100$





: {\it RU DNOPIS} \{{\it nie podlega ocenie}\}

$\mathrm{h}\mathrm{P}-\mathrm{P}0_{-}100$

Strona 3 z25





Zadarie 34. (0-2)

Ze zbioru dziewiecioelementowego $M=\{1$, 2, 3, 4, 5, 6, 7, 8, 9$\}$ losujemy kolejno ze

zwracaniem dwa razy po jednej liczbie. Zdarzenie $A$ polega na wylosowaniu dwóch liczb ze

zbioru $M$, których iloczyn jest równy 24. Ob1icz prawdopodobieństwo zdarzenia $A.$
\begin{center}
\begin{tabular}{|l|l|l|l|}
\cline{2-4}
&	\multicolumn{1}{|l|}{Nr zadania}&	\multicolumn{1}{|l|}{$33.$}&	\multicolumn{1}{|l|}{ $34.$}	\\
\cline{2-4}
&	\multicolumn{1}{|l|}{Maks. liczba pkt}&	\multicolumn{1}{|l|}{$2$}&	\multicolumn{1}{|l|}{ $2$}	\\
\cline{2-4}
\multicolumn{1}{|l|}{egzaminator}&	\multicolumn{1}{|l|}{Uzyskana liczba pkt}&	\multicolumn{1}{|l|}{}&	\multicolumn{1}{|l|}{}	\\
\hline
\end{tabular}

\end{center}
$\mathrm{E}\mathrm{M}\mathrm{A}\mathrm{P}-\mathrm{P}0_{-}100$

Strona 21 z25





Zadanie 35. (0-5)

Wykres funkcji kwadratowej $f$ określonej wzorem $f(x)=ax^{2}+bx+c$ ma z prostq

o równaniu $\mathrm{y}=6$ dokladniejeden punkt wspólny. Punkty $A=(-5,0) \mathrm{i} B=(3,0)$

nalez $\mathrm{c}$] do wykresu funkcji $f$. Oblicz wartości wspólczynników $a, b$ oraz $c.$

Strona 22 z25

$\mathrm{E}\mathrm{M}\mathrm{A}\mathrm{P}-\mathrm{P}0_{-}100$





Wypelnia

egzaminator

Nr zadania

Maks. liczba pkt

Uzyskana liczba pkt

35.

5

-PO-100

Strona 23 z25





: {\it RU DNOPIS} \{{\it nie podlega ocenie}\}

Strona 24z 25

$\mathrm{E}\mathrm{M}\mathrm{A}\mathrm{P}-\mathrm{P}0_{-}10$





$0_{-}100$

Strona 25 z25




















Zadarie 6. $(0-1$\}

Rozwiqzaniem ukladu równań 

A. $\chi_{0}>0 \mathrm{i}$

$\mathrm{y}_{0}>0$

B. $\chi_{0}>0 \mathrm{i}$

$y_{0}<0$

C. $\chi_{0}<0 \mathrm{i}$

$\mathrm{y}_{0}>0$

D. $\chi_{0}<0 \mathrm{i}$

$y_{0}<0$

Zadanie 7. $(0-1$\}

Zbiorem wszystkich rozwiqzań nierówności $\displaystyle \frac{2}{5}-\frac{\chi}{3}>\frac{\chi}{5}$ jest przedzial

A. $(-\infty,0)$

B. $(0,+\infty)$

C.(-$\infty$,-43)

D. $(\displaystyle \frac{3}{4},+\infty)$

Zadanie 8. $\langle 0-1$)

lloczyn wszystkich rozwiazań równania $2x(x^{2}-9)(x+1)=0$ jest równy

A. $(-3)$

B. 3

C. 0

D. 9

Zadanie 9. (0-1)

Na rysunku przedstawiono wykres funkcji f.
\begin{center}
\includegraphics[width=161.640mm,height=89.148mm]{./F2_M_PP_M2022_page3_images/image001.eps}
\end{center}
{\it y}

1 0  1  10 $\chi$

B. $(-8)$

A. $(-12)$

lloczyn $f(-3)\cdot f(0)\cdot f(4)$ jest równy

C. 0

D. 16

Strona 4 z25

$\mathrm{E}\mathrm{M}\mathrm{A}\mathrm{P}-\mathrm{P}0_{-}100$





: {\it RU DNOPIS} \{{\it nie podlega ocenie}\}

$\mathrm{h}\mathrm{P}-\mathrm{P}0_{-}100$

Strona 5 z25





Zadanie 10. $\{0-1\}$

Na rysunku l. przedstawiono wykres funkcji $f$ określonej na zbiorze $\langle-4, 5\rangle.$

Rysunek l.

Funkcje g określono za pomocq funkcji f. Wykres funkcji g przedstawiono na rysunku 2.

Rysunek 2.

Wynika stqd, $\dot{\mathrm{z}}\mathrm{e}$

A. $g(x)=f(x)-2$

C. $g(x)=f(x)+2$

B. $g(x)=f(x-2)$

D. $g(x)=f(x+2)$

Strona 6 z25

$\mathrm{E}\mathrm{M}\mathrm{A}\mathrm{P}-\mathrm{P}0_{-}100$





: {\it RU DNOPIS} \{{\it nie podlega ocenie}\}

$\mathrm{h}\mathrm{P}-\mathrm{P}0_{-}100$

Strona 7 z25





Zadanie ll. $\langle 0-1$\}

Miejscem zerowym funkcji liniowej $f$ określonej wzorem $f(x)=-\displaystyle \frac{1}{3}(x+3)+5$ jest liczba

A. $(-3)$

B. -92

C. 5

D. 12

Zadan$\mathrm{e}12. \langle 0-1$)

Wykresem funkcji kwadratowej $f(x)=3x^{2}+bx+c$ jest parabola o wierzcholku w punkcie

$W=(-3,2)$. Wzór tej funkcji w postaci kanonicznej to

A. $f(x)=3(x-3)^{2}+2$

B. $f(x)=3(x+3)^{2}+2$

C. $f(x)=(x-3)^{2}+2$

D. $f(x)=(x+3)^{2}+2$

Zadanie 13. (0-1)

Ciqg $(a_{n})$ jest określony wzorem $a_{n}=\displaystyle \frac{2n^{2}-30n}{n}$ dla $\mathrm{k}\mathrm{a}\dot{\mathrm{z}}$ dej liczby naturalnej $n\geq 1.$

Wtedy $a_{7}$ jest równy

A. $(-196)$

B. $(-32)$

C. $(-26)$

D. $(-16)$

Zadanie 14. $\langle 0-1$)

$\mathrm{W}$ ciqgu arytmetycznym $(a_{n})$, określonym dla $\mathrm{k}\mathrm{a}\dot{\mathrm{z}}$ dej liczby naturalnej $n\geq 1,$

$a_{5}=-31$ oraz $a_{10}=-66$. Róznica tego ciagu jest równa

A. $(-7)$

B. $(-19,4)$

C. 7

D. 19,4

Zadanie \{5. (0-1)

Wszystkie wyrazy nieskończonego ciqgu geometrycznego $(a_{n})$, określonego dla $\mathrm{k}\mathrm{a}\dot{\mathrm{z}}$ dej

liczby naturalnej $n\geq 1$, sa dodatnie i $9a_{5}=4a_{3}$. Wtedy iloraz tego ciqgu jest równy

A. -23

B. -23

C. -92

D. -92

Zadanie 16. $(0-1$\}

Liczba $\cos 12^{\mathrm{o}}\cdot\sin 78^{\mathrm{o}}+\sin 12^{\mathrm{o}}\cdot\cos 78^{\mathrm{o}}$ jest równa

A. -21

B. $\displaystyle \frac{\sqrt{2}}{2}$

C. $\displaystyle \frac{\sqrt{3}}{2}$

D. l

Strona 8 z25

$\mathrm{E}\mathrm{M}\mathrm{A}\mathrm{P}-\mathrm{P}0_{-}100$





: {\it RU DNOPIS} \{{\it nie podlega ocenie}\}

$\mathrm{h}\mathrm{P}-\mathrm{P}0_{-}100$

Strona 9 z25





$\mathrm{Z}\mathrm{a}\mathrm{d}\mathrm{a}*\mathrm{i}\mathrm{e}17, \langle 0-1\}$

Punkty $A, B, C \mathrm{l}\mathrm{e}\dot{\mathrm{z}}\mathrm{q}$ na okregu o środku $S$. Punkt $D$ jest punktem przeciecia $\mathrm{c}\mathrm{i}_{9}$ciwy $AC$

i średnicy okregu poprowadzonej z punktu $B$. Miara kqta $BSC$ jest równa $\alpha$, a miara kqta

$ADB$ jest równa $\gamma$ (zobacz rysunek).
\begin{center}
\includegraphics[width=64.668mm,height=60.456mm]{./F2_M_PP_M2022_page9_images/image001.eps}
\end{center}
{\it A}

{\it S}

{\it D}

$\gamma$

{\it C}

$\alpha$

{\it B}

B. $ 180^{\mathrm{o}}-\displaystyle \frac{\alpha}{2}-\gamma$

A. $\displaystyle \frac{\alpha}{2}+\gamma-180^{\mathrm{o}}$

Wtedy kqt ABD ma miare

C. $ 180^{\mathrm{o}}-\alpha-\gamma$

D. $\alpha+\gamma-180^{\mathrm{o}}$

Zadanie 18. (0-1)

Punkty $A, B, P \mathrm{l}\mathrm{e}\dot{\mathrm{z}}\mathrm{q}$ na okregu o środku $S$ i promieniu 6. Czworokat ASBP jest rombem,

w którym $\mathrm{k}\mathrm{a}\mathrm{t}$ ostry PAS ma miare $60^{\mathrm{o}}$ (zobacz rysunek).
\begin{center}
\includegraphics[width=76.296mm,height=79.296mm]{./F2_M_PP_M2022_page9_images/image002.eps}
\end{center}
{\it P}

{\it A}

{\it B}

{\it S}

Pole zakreskowanej na rysunku figury jest równe

A. $ 6\pi$

B. $ 9\pi$

C. $ 10\pi$

D. $ 12\pi$

Strona 10 z25

$\mathrm{E}\mathrm{M}\mathrm{A}\mathrm{P}-\mathrm{P}0_{-}100$







CENTRALNA

KOMISJA

EGZAMINACYJNA

Arkusz zawiera informacje prawnie chronione

do momentu rozpoczecia egzaminu.

KOD

WYPELNIA ZDAJACY

PESEL

{\it Miejsce na naklejke}.

{\it Sprawdz}', {\it czy kod na naklejce to}

e-100.
\begin{center}
\includegraphics[width=21.900mm,height=10.164mm]{./F2_M_PP_M2023_page0_images/image001.eps}

\includegraphics[width=79.656mm,height=10.164mm]{./F2_M_PP_M2023_page0_images/image002.eps}
\end{center}
/{\it ezeli tak}- {\it przyklej naklejkq}.

/{\it ezeli nie}- {\it zgtoś to nauczycielowi}.

Egzamin maturalny

DATA: 8 maja 2023 r.

GODZINA R0ZP0CZECIA: 9:00

CZAS TRWANIA: $170 \displaystyle \min$ ut

MAP-P0-100-2305

WyPEtNlA ZESPÓL NADZORUJACY

Uprawnienia $\mathrm{z}\mathrm{d}\mathrm{a}\mathrm{j}_{8}$cego do:

\fbox{} dostosowania zasad oceniania

\fbox{} dostosowania w zw. z dyskalkuliq

\fbox{} nieprzenoszenia zaznaczeń na karte.

LICZBA PUNKTÓW DO UZYSKANIA 46

Przed rozpoczeciem pracy z arkuszem egzaminacyjnym

1.

Sprawd $\acute{\mathrm{z}}$, czy nauczyciel przekazal Ci wlaściwy arkusz egzaminacyjny,

tj. arkusz we wlaściwej formule, z w[aściwego przedmiotu na wlaściwym

poziomie.

2.

$\mathrm{J}\mathrm{e}\dot{\mathrm{z}}$ eli przekazano Ci niew[aściwy arkusz- natychmiast zgloś to nauczycielowi.

Nie rozrywaj banderol.

3.

$\mathrm{J}\mathrm{e}\dot{\mathrm{z}}$ eli przekazano Ci w[aściwy arkusz- rozerwij banderole po otrzymaniu

takiego polecenia od nauczyciela. Zapoznaj $\mathrm{s}\mathrm{i}\mathrm{e}$ z instrukcjq na stronie 2.

Uk\}ad graficzny

\copyright CKE 2022

$\Vert\Vert\Vert\Vert\Vert\Vert\Vert\Vert\Vert\Vert\Vert\Vert\Vert\Vert\Vert\Vert\Vert\Vert\Vert\Vert\Vert\Vert\Vert\Vert\Vert\Vert\Vert\Vert\Vert\Vert|$




lnstrukcja dla zdajqcego

l. Sprawdz', czy arkusz egzaminacyjny zawiera 30 stron (zadania $1-36$).

Ewentualny brak zgloś przewodniczqcemu zespolu nadzorujacego egzamin.

2. Na pierwszej stronie arkusza oraz na karcie odpowiedzi wpisz swój numer PESEL

i przyklej naklejke z kodem.

3. Odpowiedzi do zadań zamknietych $(1-29)$ zaznacz na karcie odpowiedzi w cz9ści karty

przeznaczonej dla zdajacego. Zamaluj $\blacksquare$ pola do tego przeznaczone. $\mathrm{B}_{9}\mathrm{d}\mathrm{n}\mathrm{e}$

zaznaczenie otocz kólkiem \copyright i zaznacz wlaściwe.

4. Pamiptaj, $\dot{\mathrm{z}}\mathrm{e}$ pominiecie argumentacji lub istotnych obliczeń w rozwiqzaniu zadania

otwartego (30-36) $\mathrm{m}\mathrm{o}\dot{\mathrm{z}}\mathrm{e}$ spowodowač, $\dot{\mathrm{z}}\mathrm{e}$ za to rozwiqzanie nie otrzymasz pelnej liczby

punktów.

5. Rozwiqzania zadań i odpowiedzi wpisuj w miejscu na to przeznaczonym.

6. Pisz czytelnie i $\mathrm{u}\dot{\mathrm{z}}$ ywaj tylko dlugopisu lub pióra z czarnym tuszem lub atramentem.

7. Nie $\mathrm{u}\dot{\mathrm{z}}$ ywaj korektora, a bledne zapisy wyra $\acute{\mathrm{z}}\mathrm{n}\mathrm{i}\mathrm{e}$ przekreśl.

8. Nie wpisuj $\dot{\mathrm{z}}$ adnych znaków w cześci przeznaczonej dla egzaminatora.

9. $\mathrm{P}\mathrm{a}\mathrm{m}\mathrm{i}_{9}\mathrm{t}\mathrm{a}\mathrm{j}, \dot{\mathrm{z}}\mathrm{e}$ zapisy w brudnopisie nie bedq oceniane.

10. $\mathrm{M}\mathrm{o}\dot{\mathrm{z}}$ esz korzystač z Wybranych wzorów matematycznych, cyrkla i linijki oraz kalkulatora

prostego. Upewnij $\mathrm{s}\mathrm{i}\mathrm{e}$, czy przekazano Ci broszur9 z ok1adka taka jak widoczna ponizej.

Wybrane wzo y

matematyczne

$a\wedge=.\dot{f}^{-\prime}(x_{(},)$

$q\cdot\underline{\wedge n\tau\wedge}\iota\omega \mathrm{o}\mathrm{n}$

Strona 2 z30

$\mathrm{E}\mathrm{M}\mathrm{A}\mathrm{P}-\mathrm{P}0_{-}100$





: {\it RU DNOPIS} \{{\it nie podlega ocenie}\}

$-\mathrm{P}0_{-}100$

Strona llz30





Zadanie 15. $\langle 0-1$\}

Ciqg $(a_{n})$ jest określony wzorem $a_{n}=2^{n}$

Wyraz $a_{4}$ jest równy

$(n+1)$ dla $\mathrm{k}\mathrm{a}\dot{\mathrm{z}}$ dej liczby naturalnej $n\geq 1.$

A. 64

B. 40

C. 48

D. 80

Zadanie 16. $(0-1\mathrm{J}$

Trzywyrazowy ciag $($27, 9, $a-1)$ jest geometryczny.

Liczba $a$ jest równa

A. 3

B. 0

Zadqnie 17. $(0-1$\}

$\mathrm{W}$ ukladzie wspólrzednych zaznaczono $\mathrm{k}\mathrm{a}\mathrm{t} 0$

o wierzcholku w punkcie $0=(0,0)$. Jedno

z ramion tego kqta pokrywa $\mathrm{s}\mathrm{i}\mathrm{e}$ z dodatnia

póosiq $0x$, a drugie przechodzi przez punkt

$P=(-3,1)$ (zobacz rysunek).

Tangens kqta $\alpha$ jest równy

A. -$\sqrt{}$110

B. $(-\displaystyle \frac{3}{\sqrt{10}})$

C. 4

D. 2
\begin{center}
\includegraphics[width=78.288mm,height=48.768mm]{./F2_M_PP_M2023_page11_images/image001.eps}
\end{center}
{\it y}

$P=(-3,1)$

$\alpha$

{\it 0}

$-1$

1 2 3  $\chi$

C. $(-\displaystyle \frac{3}{1})$

D. $(-\displaystyle \frac{1}{3})$

Zädanie $l8. (0-1$\}

Dla $\mathrm{k}\mathrm{a}\dot{\mathrm{z}}$ dego kqta ostrego $\alpha$ wyrazenie $\sin^{4}\alpha+\sin^{2}\alpha\cdot\cos^{2}\alpha$ jest równe

A. $\sin^{2}\alpha$

B. $\sin^{6}\alpha\cdot\cos^{2}\alpha$

C. $\sin^{4}\alpha+1$

D. $\sin^{2}\alpha\cdot(\sin\alpha+\cos\alpha)\cdot(\sin\alpha-\cos\alpha)$

Strona 12 z30

$\mathrm{E}\mathrm{M}\mathrm{A}\mathrm{P}-\mathrm{P}0_{-}100$





: {\it RU DNOPIS} \{{\it nie podlega ocenie}\}

$-\mathrm{P}0_{-}100$

Strona 13z30





Zadanie 19. $\langle 0-1$\}

Punkty $A, B, C \mathrm{l}\mathrm{e}\dot{\mathrm{z}}\mathrm{q}$ na okregu o środku w punkcie 0.

Kqt $AC0$ ma miar9 $70^{\mathrm{o}}$ (zobacz rysunek).
\begin{center}
\includegraphics[width=68.172mm,height=72.384mm]{./F2_M_PP_M2023_page13_images/image001.eps}
\end{center}
{\it B}

{\it 0}

$70^{\mathrm{o}}$

{\it C}

{\it A}

Miara kata ostrego ABC jest równa

A. $10^{\mathrm{o}}$

B. $20^{\mathrm{o}}$

C. $35^{\mathrm{o}}$

D. $40^{\mathrm{o}}$

Zadanie 20. $\langle 0-1$\}

$\mathrm{W}$ rombie o boku dlugości $6\sqrt{2} \mathrm{k}\mathrm{a}\mathrm{t}$ rozwarty ma miare $150^{\mathrm{o}}$

lloczyn dlugości przekqtnych tego rombu jest równy

A. 24

B. 72

Zadanie 21. (0-1)

Przez punkty A $\mathrm{i} B, \mathrm{l}\mathrm{e}\dot{\mathrm{z}}$ ace na okregu

o środku 0, poprowadzono proste styczne

do tego okrpgu, przecinajace $\mathrm{s}\mathrm{i}\mathrm{e}$

w punkcie $C$ (zobacz rysunek).

Miara kata ACB jest równa

A. $20^{\mathrm{o}}$

B. $35^{\mathrm{o}}$

C. 36

D. $36\sqrt{2}$
\begin{center}
\includegraphics[width=95.604mm,height=62.736mm]{./F2_M_PP_M2023_page13_images/image002.eps}
\end{center}
{\it B}

$0140^{\mathrm{o}}$

{\it A  C}

C. $40^{\mathrm{o}}$

D. $70^{\mathrm{o}}$

Strona 14 z30

$\mathrm{E}\mathrm{M}\mathrm{A}\mathrm{P}-\mathrm{P}0_{-}100$





: {\it RU DNOPIS} \{{\it nie podlega ocenie}\}

$-\mathrm{P}0_{-}100$

Strona 15z30





Zadanie 22. $\{0-1\}$

Danyjest trójkqt $ABC$, w którym

$|BC|=6$. Miara kqta $ACB$ jest

równa $150^{\mathrm{o}}$ (zobacz rysunek).
\begin{center}
\includegraphics[width=97.536mm,height=39.372mm]{./F2_M_PP_M2023_page15_images/image001.eps}
\end{center}
{\it B}

6

$150^{\mathrm{o}}$

{\it C  A}

Wysokośč trójkata ABC opuszczona z wierzcholka B jest równa

A. 3

B. 4

C. $3\sqrt{3}$

D. $4\sqrt{3}$

Zadanie 23. $[0-1$\}

Dana jest prosta $k$ o równaniu $y=-\displaystyle \frac{1}{3}x+2.$

Prosta o równaniu $y=ax+b$ jest równolegla do prostej $k$ i przechodzi przez

punkt $P=(3,5)$, gdy

A. $a=3 \mathrm{i} b=4.$

B. $a=-\displaystyle \frac{1}{3} \mathrm{i} b=4.$

C. $a=3 \mathrm{i} b=-4.$

D. $a=-\displaystyle \frac{1}{3} \mathrm{i} b=6.$

Zadanie 24. $(0-1$\}

Dane sa punkty $K=(-3,-7)$ oraz $S=(5,3)$. Punkt $S$ jest środkiem odcinka $KL$. Wtedy

punkt $L$ ma wspólrz9dne

A. (13, 10)

B. (13, 13)

C. $(1,-2)$

D. $(7,-1)$

Zadanie 25. (0-1)

Dana jest prosta o równaniu $y=2x-3$. Obrazem tej prostej w symetrii środkowej

wzgledem poczqtku ukladu wspólrzędnych jest prosta o równaniu

A. $y=2x+3$

B. $y=-2x-3$

C. $y=-2x+3$

D. $y=2x-3$

Strona 16 z30

$\mathrm{E}\mathrm{M}\mathrm{A}\mathrm{P}-\mathrm{P}0_{-}100$





: {\it RU DNOPIS} \{{\it nie podlega ocenie}\}

$-\mathrm{P}0_{-}100$

Strona 17 z30





Zadarie $26_{d}(0-1$\}

Dany jest graniastoslup prawidlowy czworokqtny, w którym krawpd $\acute{\mathrm{z}}$ podstawy ma

dlugośč 15. Przekatna graniastos1upa jest nachy1ona do p1aszczyzny podstawy pod

kqtem $\alpha$ takim, $\dot{\mathrm{z}}\mathrm{e} \displaystyle \cos\alpha=\frac{\sqrt{2}}{3}$

Dlugośč przekqtnej tego graniastoslupa jest równa

A. $15\sqrt{2}$

B. 45

C. $5\sqrt{2}$

D. 10

ZadanIe 27. $\zeta 0-1$\}

$\acute{\mathrm{S}}$ rednia arytmetyczna liczb $x, y, z$ jest równa 4.

$\acute{\mathrm{S}}$ rednia arytmetyczna czterech liczb: $1+x, 2+y, 3+z$, 14, jest równa

A. 6

B. 9

C. 8

D. 13

Zadanie28. $\langle 0-1$\}

Wszystkich liczb naturalnych pieciocyfrowych, w których zapisie dziesietnym wystepujq tylko

cyfry 0, 5, 7 (np. 57075, 55555), jest

A. $5^{3}$

B. $2\cdot 4^{3}$

C. $2\cdot 3^{4}$

D. $3^{5}$

Zadanie 29. $(0-1$\}

$\mathrm{W}$ pewnym ostroslupie prawidlowym stosunek liczby $W$ wszystkich wierzcholków do

liczby $K$ wszystkich krawedzi jest równy $\displaystyle \frac{W}{K}=\frac{3}{5}.$

Podstawa tego ostroslupa jest

A. kwadrat.

B. piciokt foremny.

C. sześciokat foremny.

D. siedmiokat foremny.

Strona 18 z30

$\mathrm{E}\mathrm{M}\mathrm{A}\mathrm{P}-\mathrm{P}0_{-}100$





: {\it RU DNOPIS} \{{\it nie podlega ocenie}\}

$-\mathrm{P}0_{-}100$

Strona 19z30





Zadarie 30. (0-2)

Rozwiqz nierównośč

$x(x-2)>2x^{2}-3$

Strona 20 z30

$\mathrm{E}\mathrm{M}\mathrm{A}\mathrm{P}-\mathrm{P}0_{-}10$





Zadania egzaminacyine sq wydrukowane

na nastepnych stronach.

$\mathrm{E}\mathrm{M}\mathrm{A}\mathrm{P}-\mathrm{P}0_{-}100$

Strona 3 z30





Zadarie 31. (0-2)

Pan Stanislaw splacil $\mathrm{p}\mathrm{o}\dot{\mathrm{z}}$ yczkę w wysokości

rata byla mniejsza od poprzedniej o 30 z1.

Oblicz kwot9 pierwszej raty.

8910 zl w osiemnastu ratach. $\mathrm{K}\mathrm{a}\dot{\mathrm{z}}$ da kolejna
\begin{center}
\begin{tabular}{|l|l|l|l|}
\cline{2-4}
&	\multicolumn{1}{|l|}{Nr zadania}&	\multicolumn{1}{|l|}{$30.$}&	\multicolumn{1}{|l|}{ $31.$}	\\
\cline{2-4}
&	\multicolumn{1}{|l|}{Maks. liczba pkt}&	\multicolumn{1}{|l|}{$2$}&	\multicolumn{1}{|l|}{ $2$}	\\
\cline{2-4}
\multicolumn{1}{|l|}{egzaminator}&	\multicolumn{1}{|l|}{Uzyskana liczba pkt}&	\multicolumn{1}{|l|}{}&	\multicolumn{1}{|l|}{}	\\
\hline
\end{tabular}

\end{center}
$\mathrm{E}\mathrm{M}\mathrm{A}\mathrm{P}-\mathrm{P}0_{-}100$

Strona 21 z30





Zadanie 32. (0-2)

Wykaz, $\dot{\mathrm{z}}\mathrm{e}$ dla $\mathrm{k}\mathrm{a}\dot{\mathrm{z}}$ dej liczby rzeczywistej $x\neq 1$ i dla $\mathrm{k}\mathrm{a}\dot{\mathrm{z}}$ dej liczby rzeczywistej

prawdziwa jest nierównośč

$x^{2}+y^{2}+5>2x+4y$

{\it y}

Strona 22 z30

$\mathrm{E}\mathrm{M}\mathrm{A}\mathrm{P}-\mathrm{P}0_{-}100$





Zadanie 33. (0-2)

Trójkqty prostokatne $T_{1}$ i $T_{2}$ sq podobne. Przyprostokqtne trójkata

5 $\mathrm{i} 12$. Przeciwprostokqtna trójkqta $T_{2}$ ma dlugośč 26.

Oblicz pole trójkqta $T_{2}.$

$T_{1}$ maja dlugości
\begin{center}
\begin{tabular}{|l|l|l|l|}
\cline{2-4}
&	\multicolumn{1}{|l|}{Nr zadania}&	\multicolumn{1}{|l|}{$32.$}&	\multicolumn{1}{|l|}{ $33.$}	\\
\cline{2-4}
&	\multicolumn{1}{|l|}{Maks. liczba pkt}&	\multicolumn{1}{|l|}{$2$}&	\multicolumn{1}{|l|}{ $2$}	\\
\cline{2-4}
\multicolumn{1}{|l|}{egzaminator}&	\multicolumn{1}{|l|}{Uzyskana liczba pkt}&	\multicolumn{1}{|l|}{}&	\multicolumn{1}{|l|}{}	\\
\hline
\end{tabular}

\end{center}
$\mathrm{E}\mathrm{M}\mathrm{A}\mathrm{P}-\mathrm{P}0_{-}100$

Strona 23 z30





Zadarie 34. (0-2)

$\mathrm{W}$ kwadracie ABCD punkty $A=(-8,-2)$ oraz $C=(0,4)$ sa końcami przekqtnej.

Wyznacz równanie prostej zawierajacej przekqtnq $BD$ tego kwadratu.

Strona 24 z30

$\mathrm{E}\mathrm{M}\mathrm{A}\mathrm{P}-\mathrm{P}0_{-}100$





Zadarie 35. (0-2)

Ze zbioru ośmiu liczb \{2, 3, 4, 5, 6, 7, 8, 9\} 1osujemy ze zwracaniem ko1ejno dwa razy po

jednej liczbie.

Oblicz prawdopodobieństwo zdarzenia $A$ polegajqcego na tym, $\dot{\mathrm{z}}\mathrm{e}$ iloczyn wylosowanych

liczb jest podzielny przez 15.
\begin{center}
\begin{tabular}{|l|l|l|l|}
\cline{2-4}
&	\multicolumn{1}{|l|}{Nr zadania}&	\multicolumn{1}{|l|}{$34.$}&	\multicolumn{1}{|l|}{ $35.$}	\\
\cline{2-4}
&	\multicolumn{1}{|l|}{Maks. liczba pkt}&	\multicolumn{1}{|l|}{$2$}&	\multicolumn{1}{|l|}{ $2$}	\\
\cline{2-4}
\multicolumn{1}{|l|}{egzaminator}&	\multicolumn{1}{|l|}{Uzyskana liczba pkt}&	\multicolumn{1}{|l|}{}&	\multicolumn{1}{|l|}{}	\\
\hline
\end{tabular}

\end{center}
$\mathrm{E}\mathrm{M}\mathrm{A}\mathrm{P}-\mathrm{P}0_{-}100$

Strona 25 z30





Zadarie 36. $\langle 0-5$)

Podstawq graniastoslupa prostego ABCDEF jest trójkqt

równoramienny $ABC$, w którym $|AC|=|BC|, |AB|=8.$

Wysokośč trójkata $ABC$, poprowadzona z wierzcholka $C,$

ma dlugośč 3. Przekqtna CE ściany bocznej tworzy

z krawpdziq $CB$ podstawy $ABC \triangleright_{\iota}\mathrm{q}\mathrm{t} 60^{\mathrm{o}}$ (zobacz

rysunek).

Oblicz pole powierzchni calkowitej oraz objetośč tego graniastoslupa.

Strona 26 z30

$\mathrm{E}\mathrm{M}\mathrm{A}\mathrm{P}-\mathrm{P}0_{-}100$





Wypelnia

egzaminator

Nr zadania

Maks. liczba pkt

Uzyskana liczba pkt

36.

5

-PO-100

Strona 27 z30





: {\it RU DNOPIS} \{{\it nie podlega ocenie}\}

Strona 28z 30

$\mathrm{E}\mathrm{M}\mathrm{A}\mathrm{P}-\mathrm{P}0_{-}10$





$0_{-}100$

Strona 29 z30





Strona 30 z30

$\mathrm{E}\mathrm{M}\mathrm{A}\mathrm{P}-\mathrm{P}0_{-}10$





{\it Wkazdym z zadań od} $f.$ {\it do 29. wybierz izaznacz na karcie odpowiedzi poprawna} $od\sqrt{}owi\mathrm{e}d\acute{z}.$

Zadanie $\mathrm{f}. (0-1$\}

Liczba $\log_{9}27+\log_{9}3$ jest równa

A. 81

B. 9

C. 4

D. 2

Zadan$\mathrm{e}2. (0-1$\}

Liczba $\sqrt[3]{-\frac{27}{16}}\cdot\sqrt[3]{2}$ jest równa

A. $(-\displaystyle \frac{3}{2})$

B. -23

C. -32

D. $(-\displaystyle \frac{2}{3})$

Zadanie $3_{r}(0-4)$

Cene aparatu fotograficznego obnizono o 15\%, a nastepnie-o 20\% w odniesieniu do

ceny obowiqzujacej w danym momencie. Po tych dwóch obnizkach aparat kosztuje 340 z1.

Przed obiema obnizkami cena tego aparatu byla równa

A. 500 z1

B. 425 z1

C. 400 z1

D. 375 z1

Zadanie 4. $(0-1\rangle$

Dla $\mathrm{k}\mathrm{a}\dot{\mathrm{z}}$ dej liczby rzeczywistej $a$ wyrazenie $(2a-3)^{2}-(2a+3)^{2}$ jest równe

A. $-24a$

B. 0

C. 18

D. $16a^{2}-24a$

Strona 4 z30

$\mathrm{E}\mathrm{M}\mathrm{A}\mathrm{P}-\mathrm{P}0_{-}100$















: {\it RU DNOPIS} \{{\it nie podlega ocenie}\}

$-\mathrm{P}0_{-}100$

Strona 5z30





Zadanie 5. $(0-1$\}

Na rysunku przedstawiono interpretacj9 $\displaystyle \mathrm{g}\mathrm{e}\mathrm{o}\mathrm{m}\mathrm{e}\mathrm{t}\mathrm{r}\mathrm{y}\mathrm{c}\mathrm{z}\bigcap_{\mathrm{c}1}$ jednego z $\mathrm{n}\mathrm{i}\dot{\mathrm{z}}$ ej zapisanych ukladów

równań.

Wskaz ten uklad równań, którego interpretacje geometryczna przedstawiono na rysunku.

A. 

B. 

C. 

D. 

Zädanie 6. $\{0-1\}$

Zbiorem wszystkich rozwiqzań nierówności

$-2(x+3)\displaystyle \leq\frac{2-x}{3}$

jest przedzial

A. $(-\infty, -4\rangle$

B. $(-\infty,  4\rangle$

C. $\langle-4, \infty)$

D. $\langle$4, $\infty)$

Zadanie 7. (0-1)

Jednym z rozwiqzań równania $\sqrt{3}(x^{2}-2)(x+3)=0$ jest liczba

A. 3

B. 2

C. $\sqrt{3}$

D. $\sqrt{2}$

Strona 6 z30

$\mathrm{E}\mathrm{M}\mathrm{A}\mathrm{P}-\mathrm{P}0_{-}100$





: {\it RU DNOPIS} \{{\it nie podlega ocenie}\}

$-\mathrm{P}0_{-}100$

Strona 7 z30





Zadanie @. $(0-\mathrm{t}\rangle$

Równanie $\displaystyle \frac{(x+1)(x-1)^{2}}{(x-1)(x+1)^{2}}=0$ w zbiorze liczb rzeczywistych

A. nie ma rozwiqzania.

B. ma dokladnie jedno rozwiqzanie: $-1.$

C. ma dokladnie jedno rozwiqzanie: l.

D. ma dokladnie dwa rozwiqzania: $-1$ oraz l.

Zadanie 9. (0-4)

Miejscem zerowym funkcji liniowej $f(x)=(2p-1)x+p$ jest liczba $(-4)$. Wtedy

A. {\it p}$=$ -49

B. {\it p} $=$ -47

C. $p=-4$

D. {\it p}$=$ - -47

Zadanie 10. $\langle 0-1$\}

Funkcja liniowa $f$ jest określona wzorem

$f(x)=ax+b$, gdzie $a \mathrm{i} b$ sa pewnymi

liczbami rzeczywistymi. Na rysunku obok

przedstawiono fragment wykresu funkcji $f$

w ukladzie wspólrzednych $(x,y).$
\begin{center}
\includegraphics[width=82.956mm,height=69.540mm]{./F2_M_PP_M2023_page7_images/image001.eps}
\end{center}
{\it y}

1

0 1  $\chi$

$y=f(x)$

Liczba $a$ oraz liczba $b$ we wzorze funkcji $f \mathrm{s}\mathrm{p}\mathrm{e}$niajq warunki:

A. $a>0 \mathrm{i} b>0.$

B. $a>0 \mathrm{i} b<0.$

C. $a<0 \mathrm{i} b>0.$

D. $a<0 \mathrm{i} b<0.$

Strona 8 z30

$\mathrm{E}\mathrm{M}\mathrm{A}\mathrm{P}-\mathrm{P}0_{-}100$





: {\it RU DNOPIS} \{{\it nie podlega ocenie}\}

$-\mathrm{P}0_{-}100$

Strona 9z30





lnformacja do zadań ll.$-13.$

$\mathrm{W}$ ukladzie wspólrzednych $(x,y)$

narysowano wykres funkcji $y=f(x)$

(zobacz sunek).

Zadanie ll. $\langle 0-1$\}

Dziedzinq funkcji $f$ jest zbiór

A. $\langle-6,  5\rangle$

B. $(-6,5)$

Zadanie 12. $\langle 0-1$)

Funkcja $f$ jest malejqca w zbiorze

A. $\langle-6, -3)$

B. $\langle-3,1\rangle$
\begin{center}
\includegraphics[width=100.680mm,height=83.820mm]{./F2_M_PP_M2023_page9_images/image001.eps}
\end{center}
{\it y}

0  $\chi$

C. $(-3,5\rangle$

D. $\langle-3,  5\rangle$

C. (l, $ 2\rangle$

D. $\langle$2, $ 5\rangle$

Zadanie 43. (0-1)

$\mathrm{N}\mathrm{a}\mathrm{j}\mathrm{w}\mathrm{i}_{9}$ksza wartośč funkcji $f$ w przedziale $\langle-4,  1\rangle$ jest równa

A. 0

B. l

C. 2

D. 5

Zädanie $l4. (0-1)$

Jednym z miejsc zerowych funkcji kwadratowej $f$ jest liczba $(-5)$. Pierwsza wspólrzedna

wierzcholka paraboli, $\mathrm{b}_{9}$dqcej wykresem funkcji $f$, jest równa 3.

Drugim miejscem zerowym funkcji $f$ jest liczba

A. ll

B. l

C. $(-1)$

D. $(-13)$

Strona 10 z30

$\mathrm{E}\mathrm{M}\mathrm{A}\mathrm{P}-\mathrm{P}0_{-}100$







CENTRALNA

KOMISJA

EGZAMINACYJNA

Arkusz zawiera informacje prawnie chronione

do momentu rozpoczecia egzaminu.

KOD

WYPELNIA ZDAJACY

PESEL

{\it Miejsce na naklejke}.

{\it Sprawdz}', {\it czy kod na naklejce to}

e-100.
\begin{center}
\includegraphics[width=21.900mm,height=10.164mm]{./F2_M_PP_M2024_page0_images/image001.eps}

\includegraphics[width=79.656mm,height=10.164mm]{./F2_M_PP_M2024_page0_images/image002.eps}
\end{center}
/{\it ezeli tak}- {\it przyklej naklejkq}.

/{\it ezeli nie}- {\it zgtoś to nauczycielowi}.

Egzamin maturalny

DATA: 8 maja 2024 r.

GODZINA R0ZP0CZECIA: 9:00

CZAS TRWANIA: $170 \displaystyle \min$ ut

MAP-P0-100-2405

WYPELNIA ZESPÓt NADZORUJACY

Uprawnienia zdajqcego do:

\fbox{} dostosowania zasad oceniania

\fbox{} dostosowania w zw. z dyskalkuliq

\fbox{} nieprzenoszenia odpowiedzi na karte.

LICZBA PUNKTÓW DO UZYSKANIA 46

Przed rozpoczeciem pracy z arkuszem egzaminacyjnym

1.

Sprawd $\acute{\mathrm{z}}$, czy nauczyciel przekazal Ci wlaściwy arkusz egzaminacyjny,

tj. arkusz we wlaściwej formule, z w[aściwego przedmiotu na wlaściwym

poziomie.

2.

$\mathrm{J}\mathrm{e}\dot{\mathrm{z}}$ eli przekazano Ci niew[aściwy arkusz- natychmiast zgloś to nauczycielowi.

Nie rozrywaj banderol.

3.

$\mathrm{J}\mathrm{e}\dot{\mathrm{z}}$ eli przekazano Ci w[aściwy arkusz- rozerwij banderole po otrzymaniu

takiego polecenia od nauczyciela. Zapoznaj $\mathrm{s}\mathrm{i}\mathrm{e}$ z instrukcjq na stronie 2.

Uk\}ad graficzny

\copyright CKE 2022

$\Vert\Vert\Vert\Vert\Vert\Vert\Vert\Vert\Vert\Vert\Vert\Vert\Vert\Vert\Vert\Vert\Vert\Vert\Vert\Vert\Vert\Vert\Vert\Vert\Vert\Vert\Vert\Vert\Vert\Vert|$




lnstrukcja dla zdajqcego

l. Sprawdz', czy arkusz egzaminacyjny zawiera 31 stron (zadania $1-36$).

Ewentualny brak zg\}oś przewodniczqcemu zespolu nadzorujqcego egzamin.

2. Na pierwszej stronie arkusza oraz na karcie odpowiedzi wpisz swój numer PESEL

i przyklej naklejke z kodem.

3. Odpowiedzi do zadań zamknietych $(1-29)$ zaznacz na karcie odpowiedzi w cz9ści karty

przeznaczonej dla zdajacego. Zamaluj $\blacksquare$ pola do tego przeznaczone. $\mathrm{B}_{9}\mathrm{d}\mathrm{n}\mathrm{e}$

zaznaczenie otocz kólkiem \copyright i zaznacz wlaściwe.

4. Pamiptaj, $\dot{\mathrm{z}}\mathrm{e}$ pominiecie argumentacji lub istotnych obliczeń w rozwiqzaniu zadania

otwartego (30-36) $\mathrm{m}\mathrm{o}\dot{\mathrm{z}}\mathrm{e}$ spowodowač, $\dot{\mathrm{z}}\mathrm{e}$ za to rozwiazanie nie otrzymasz pelnej liczby

punktów.

5. Rozwiqzania zadań i odpowiedzi wpisuj w miejscu na to przeznaczonym.

6. Pisz czytelnie i $\mathrm{u}\dot{\mathrm{z}}$ ywaj tylko dlugopisu lub pióra z czarnym tuszem lub atramentem.

7. Nie $\mathrm{u}\dot{\mathrm{z}}$ ywaj korektora, a bledne zapisy wyra $\acute{\mathrm{z}}\mathrm{n}\mathrm{i}\mathrm{e}$ przekreśl.

8. Nie wpisuj $\dot{\mathrm{z}}$ adnych znaków w cześci przeznaczonej dla egzaminatora.

9. $\mathrm{P}\mathrm{a}\mathrm{m}\mathrm{i}_{9}\mathrm{t}\mathrm{a}\mathrm{j}, \dot{\mathrm{z}}\mathrm{e}$ zapisy w brudnopisie nie bedq oceniane.

10. $\mathrm{M}\mathrm{o}\dot{\mathrm{z}}$ esz korzystač z Wybranych wzorów matematycznych, cyrkla i linijki oraz kalkulatora

prostego. Upewnij $\mathrm{s}\mathrm{i}\mathrm{e}$, czy przekazano Ci broszure z okladka taka jak widoczna ponizej.

Wybrane wzory

matematyczne

Strona 2 z31

$\mathrm{E}\mathrm{M}\mathrm{A}\mathrm{P}-\mathrm{P}0_{-}100$





: {\it RU DNOPIS} \{{\it nie podlega ocenie}\}

$\mathrm{h}\mathrm{P}-\mathrm{P}0_{-}100$

Strona llz31





lnformacja do zadań 14.$-15.$

Na rysunku przedstawiono fragment paraboli, która jest wykresem funkcji kwadratowej $f$

(zobacz rysunek). Wierzcholek tej paraboli oraz punkty przeciecia paraboli z osiami ukladu

wspólrz9dnych maja obie wspó1rz9dne ca1kowite.

Zadanie 14. $\langle 0-1$\}

Funkcja kwadratowa $f$ jest określona wzorem

A. $f(x)=-(x+1)^{2}-9$

B. $f(x)=-(x-1)^{2}+9$

C. $f(x)=-(x-1)^{2}-9$

D. $f(x)=-(x+1)^{2}+9$

Zädanie i5. (0-1)

Dla funkcji f prawdziwa jest równośč

A. $f(-4)=f(6)$

B. $f(-4)=f(4)$

C. $f(-4)=f(5)$

D. $f(-4)=f(7)$

Strona 12 z31

$\mathrm{E}\mathrm{M}\mathrm{A}\mathrm{P}-\mathrm{P}0_{-}100$





: {\it RU DNOPIS} \{{\it nie podlega ocenie}\}

$\mathrm{h}\mathrm{P}-\mathrm{P}0_{-}100$

Strona 13z31





Zadanie 16. $\langle 0-1$)

$\mathrm{W}$ ciqgu arytmetycznym $(a_{n})$, określonym dla $\mathrm{k}\mathrm{a}\dot{\mathrm{z}}$ dej liczby naturalnej $n\geq 1$, dane sa

wyrazy $a_{4}=-2$ oraz $a_{6}=16.$

Piqty wyraz tego ciqgu jest równy

A. -27

B. -92

C. 7

D. 9

Zadanie 17. $(0-1$\}

Ciqg geometryczny $(a_{n})$ jest określony wzorem $a_{n}=2^{n-1}$, dla $\mathrm{k}\mathrm{a}\dot{\mathrm{z}}$ dej liczby naturalnej $n\geq 1.$

lloraz tego ciqgu jest równy

A. -21

B. $(-2)$

C. 2

D. l

Zadanie 18. $(0-1$\}

Ciqg $(b_{n})$ jest określony wzorem $b_{n}=(n+2)(7-n)$, dla $\mathrm{k}\mathrm{a}\dot{\mathrm{z}}$ dej liczby naturalnej $n\geq 1.$

Liczba dodatnich wyrazów ciqgu $(b_{n})$ jest równa

A. 6

B. 7

C. 8

D. 9

Zädanie $l9. (0-1)$

Liczba $\sin^{3}20^{\mathrm{o}}+\cos^{2}20^{\mathrm{o}}\cdot\sin 20^{\mathrm{o}}$ jest równa

A. $\cos 20^{\mathrm{o}}$

B. $\sin 20^{\mathrm{o}}$

C. $\mathrm{t}\mathrm{g}20^{\mathrm{o}}$

D. $\sin 20^{\mathrm{o}}\cdot\cos 20^{\mathrm{o}}$

Zadanie 20. (0-1)

$\mathrm{K}\mathrm{a}\mathrm{t} \alpha$ jest ostry oraz $\cos\alpha= \displaystyle \frac{5}{13}$. Wtedy

A. $\displaystyle \mathrm{t}\mathrm{g}\alpha=\frac{12}{13}$

B. $\displaystyle \mathrm{t}\mathrm{g}\alpha=\frac{12}{5}$

C. $\displaystyle \mathrm{t}\mathrm{g}\alpha=\frac{5}{12}$

D. $\displaystyle \mathrm{t}\mathrm{g}\alpha=\frac{13}{12}$

Strona 14 z31

$\mathrm{E}\mathrm{M}\mathrm{A}\mathrm{P}-\mathrm{P}0_{-}100$





: {\it RU DNOPIS} \{{\it nie podlega ocenie}\}

$\mathrm{h}\mathrm{P}-\mathrm{P}0_{-}100$

Strona 15z31





Zadarie 21. $(0-1$\}

Danyjest równoleglobok o bokach dlugości 3 $\mathrm{i} 4$ oraz o kqcie mipdzy nimi o mierze $120^{\mathrm{o}}$

Pole tego równolegloboku jest równe

A. 6

B. $6\sqrt{3}$

C. 12

D. $12\sqrt{3}$

Zadanie 22. $\langle 0-1$\}

$\mathrm{W}$ trójkacie $MKC$ bok $MK$ ma d\}ugośč 24. Prosta równo1eg$\dagger$a do boku $MK$ przecina boki

$MC \mathrm{i} KC -$ odpowiednio-w punktach $A$ oraz $B$ takich, $\dot{\mathrm{z}}\mathrm{e} |AB|=6 \mathrm{i} |AC|=3$

(zobacz rysunek).

{\it C}
\begin{center}
\includegraphics[width=123.084mm,height=56.232mm]{./F2_M_PP_M2024_page15_images/image001.eps}
\end{center}
3

{\it A B}

6

{\it M}  24  {\it K}

Dlugośč odcinka MA jest równa

A. 18

B. 15

C. 9

D. 12

Zadanie 23. $\{0-1\}$

$\mathrm{W}$ trójkqcie $ABC$, wpisanym w $\mathrm{o}\mathrm{k}\mathrm{r}_{\mathrm{c}}\mathrm{l}\mathrm{g}$ o środku w punkcie $S, \mathrm{k}\mathrm{a}\mathrm{t} ACB$ ma miare $42^{\mathrm{o}}$

(zobacz rysunek).
\begin{center}
\includegraphics[width=68.016mm,height=65.172mm]{./F2_M_PP_M2024_page15_images/image002.eps}
\end{center}
{\it C}

$42^{\mathrm{o}}$  {\it S}

{\it A}

{\it B}

Miara kqta ostrego BAS jest równa

A. $42^{\mathrm{o}}$

B. $45^{\mathrm{o}}$

C. $48^{\mathrm{o}}$

D. $69^{\mathrm{o}}$

Strona 16 z31

$\mathrm{E}\mathrm{M}\mathrm{A}\mathrm{P}-\mathrm{P}0_{-}100$





: {\it RU DNOPIS} \{{\it nie podlega ocenie}\}

$\mathrm{h}\mathrm{P}-\mathrm{P}0_{-}100$

Strona 17 z31





Zadanie 24. (0-1)

Proste k oraz l sq określone równaniami

{\it k}:

$y=(m+1)x+7$

{\it l}:

$y=-2x+7$

Proste k oraz l sq prostopadle, gdy liczba m jest równa

A. $(-\displaystyle \frac{1}{2})$

B. -21

C. $(-3)$

D. l

Zadanie 25. $\langle 0-1$\}

Na prostej $l$ o wspólczynniku kierunkowym $\displaystyle \frac{1}{2}\mathrm{l}\mathrm{e}\dot{\mathrm{z}}$ a punkty $A=(2,-4)$ oraz $B=(0,b).$

Wtedy liczba $b$ jest równa

A. $(-5)$

B. 10

C. $(-2)$

D. 0

Zadqnie 26. (0-1)

Wysokośč graniastoslupa prawidlowego sześciokatnego jest równa 6 (zobacz rysunek).

Pole podstawy tego graniastoslupa jest równe $15\sqrt{3}.$
\begin{center}
\includegraphics[width=62.328mm,height=70.968mm]{./F2_M_PP_M2024_page17_images/image001.eps}
\end{center}
I

I

I

I

I

I

I

I

I

I

I

I

I

$\underline{\mathrm{I}}$

I

I

I

I

I

I

I

I

I

I

I

I

6

Pole lednel ściany bocznej tego graniastoslupa jest równe

A. $36\sqrt{10}$

B. 60

C. $6\sqrt{10}$

D. 360

Strona 18 z31

$\mathrm{E}\mathrm{M}\mathrm{A}\mathrm{P}-\mathrm{P}0_{-}100$





: {\it RU DNOPIS} \{{\it nie podlega ocenie}\}

$\mathrm{h}\mathrm{P}-\mathrm{P}0_{-}100$

Strona 19z31





Zadanie 27. $\langle 0-1$)

Kqt nachylenia najdluzszej przekqtnej graniastoslupa prawidlowego sześciokqtnego do

plaszczyzny podstawy jest zaznaczony na rysunku

A.
\begin{center}
\includegraphics[width=52.176mm,height=63.852mm]{./F2_M_PP_M2024_page19_images/image001.eps}
\end{center}
I

I

I

I

I

I

I

I

I

I

I

I

I

I

I

I

I

I

I

I

I

I

I

C.
\begin{center}
\includegraphics[width=52.068mm,height=63.804mm]{./F2_M_PP_M2024_page19_images/image002.eps}
\end{center}
I

I

I

I

I

I

I

I

I

I

I

I

I

I

I

I

I

I

I

I

I

I

I

B.
\begin{center}
\includegraphics[width=52.176mm,height=63.852mm]{./F2_M_PP_M2024_page19_images/image003.eps}
\end{center}
I

I

I

I

I

I

I

I

I

I

I

I

I

I

I

I

I

I

I

I

I

I

D.
\begin{center}
\includegraphics[width=52.176mm,height=63.852mm]{./F2_M_PP_M2024_page19_images/image004.eps}
\end{center}
I

I

I

I

I

I

I

I

I

I

I

I

I

I

I

I

I

I

I

I

I

I

I

Zadanie 28. $(0-1$\}

Obj9tośč ostros1upa prawid1owego czworokatnego jest równa 64. Wysokośč tego ostros1upa

jest równa 12.

Dlugośč krawedzi podstawy tego ostroslupa jest równa

A. 2

B. 4

C. 6

D. 8

Zadanie 29. (0-1)

Rozwazamy wszystkie kody czterocyfrowe utworzone tylko z cyfr 1, 3, 6, 8, przy czym

w $\mathrm{k}\mathrm{a}\dot{\mathrm{z}}$ dym kodzie $\mathrm{k}\mathrm{a}\dot{\mathrm{z}}$ da z tych cyfr wystepuje dokladnie jeden raz.

Liczba wszystkich takich kodów jest równa

A. 4

B. 10

C. 24

D. 16

Strona 20 z31

$\mathrm{E}\mathrm{M}\mathrm{A}\mathrm{P}-\mathrm{P}0_{-}100$





Zadania egzaminacyine sq wydrukowane

na nastepnych stronach.

$\mathrm{E}\mathrm{M}\mathrm{A}\mathrm{P}-\mathrm{P}0_{-}100$

Strona 3 z31





: {\it RU DNOPIS} \{{\it nie podlega ocenie}\}

$\mathrm{h}\mathrm{P}-\mathrm{P}0_{-}100$

Strona 21 z 31





Zadarie 30. (0-2)

Rozwiqz nierównośč

$x^{2}-4\leq 3x$

Strona 22 z31

$\mathrm{E}\mathrm{M}\mathrm{A}\mathrm{P}-\mathrm{P}0_{-}10$





Zadarie 31. (0-2)

Wykaz, $\dot{\mathrm{z}}\mathrm{e}$ dla $\mathrm{k}\mathrm{a}\dot{\mathrm{z}}$ dej liczby rzeczywistej $x$ i dla $\mathrm{k}\mathrm{a}\dot{\mathrm{z}}$ dej liczby rzeczywistej $y$ takich,

$\dot{\mathrm{z}}\mathrm{e} x\neq \mathrm{y}$, prawdziwa jest nierównośč

$(3x+y)(x+3y)>16xy$
\begin{center}
\begin{tabular}{|l|l|l|l|}
\cline{2-4}
&	\multicolumn{1}{|l|}{Nr zadania}&	\multicolumn{1}{|l|}{$30.$}&	\multicolumn{1}{|l|}{ $31.$}	\\
\cline{2-4}
&	\multicolumn{1}{|l|}{Maks. liczba pkt}&	\multicolumn{1}{|l|}{$2$}&	\multicolumn{1}{|l|}{ $2$}	\\
\cline{2-4}
\multicolumn{1}{|l|}{egzaminator}&	\multicolumn{1}{|l|}{Uzyskana liczba pkt}&	\multicolumn{1}{|l|}{}&	\multicolumn{1}{|l|}{}	\\
\hline
\end{tabular}

\end{center}
$\mathrm{E}\mathrm{M}\mathrm{A}\mathrm{P}-\mathrm{P}0_{-}100$

Strona 23 z31





Zadanie 32. (0-2)

Osiq symetrii wykresu funkcji kwadratowej $f(x)=x^{2}+bx+c$ jest prosta o równaniu

$\chi=-2$. Jednym z miejsc zerowych funkcji $f$ jest liczba l.

Oblicz wspólczynniki $b$ oraz $c.$

Strona 24 z31

$\mathrm{E}\mathrm{M}\mathrm{A}\mathrm{P}-\mathrm{P}0_{-}100$





Zadanie 33. (0-2)

Ciqg arytmetyczny $(a_{n})$ jest określony dla $\mathrm{k}\mathrm{a}\dot{\mathrm{z}}$ dej liczby naturalnej $n\geq 1$. Trzeci wyraz

tego ciqgu jest równy $(-1)$, a suma piptnastu poczqtkowych kolejnych wyrazów tego ciqgu

jest równa $(-165).$

Oblicz róznic9 tego ciagu.
\begin{center}
\begin{tabular}{|l|l|l|l|}
\cline{2-4}
&	\multicolumn{1}{|l|}{Nr zadania}&	\multicolumn{1}{|l|}{$32.$}&	\multicolumn{1}{|l|}{ $33.$}	\\
\cline{2-4}
&	\multicolumn{1}{|l|}{Maks. liczba pkt}&	\multicolumn{1}{|l|}{$2$}&	\multicolumn{1}{|l|}{ $2$}	\\
\cline{2-4}
\multicolumn{1}{|l|}{egzaminator}&	\multicolumn{1}{|l|}{Uzyskana liczba pkt}&	\multicolumn{1}{|l|}{}&	\multicolumn{1}{|l|}{}	\\
\hline
\end{tabular}

\end{center}
$\mathrm{E}\mathrm{M}\mathrm{A}\mathrm{P}-\mathrm{P}0_{-}100$

Strona 25 z31





Zadarie 34. (0-2)

Danyjest równoleglobok ABCD, w którym $A=(-2,6)$ oraz $B=(10,2)$. Przekqtne $AC$

oraz $BD$ tego równolegloboku przecinajq $\mathrm{s}\mathrm{i}\mathrm{e}$ w punkcie $P=(6,7).$

Oblicz dlugośč boku $BC$ tego równolegloboku.

Strona 26 z31

$\mathrm{E}\mathrm{M}\mathrm{A}\mathrm{P}-\mathrm{P}0_{-}100$





Zadarie 35. (0-2)

Dany jest piecioelementowy zbiór $K=\{5$, 6, 7, 8, 9$\}$. Wylosowanie $\mathrm{k}\mathrm{a}\dot{\mathrm{z}}$ dej liczby z tego

zbioru jestjednakowo prawdopodobne. Ze zbioru $K$ losujemy ze zwracaniem kolejno dwa

razy po jednej liczbie i zapisujemy je w kolejności losowania.

Oblicz prawdopodobieństwo zdarzenia $A$ polegajqcego na tym, $\dot{\mathrm{z}}\mathrm{e}$ suma wylosowanych

liczb jest liczbq parzystq.
\begin{center}
\begin{tabular}{|l|l|l|l|}
\cline{2-4}
&	\multicolumn{1}{|l|}{Nr zadania}&	\multicolumn{1}{|l|}{$34.$}&	\multicolumn{1}{|l|}{ $35.$}	\\
\cline{2-4}
&	\multicolumn{1}{|l|}{Maks. liczba pkt}&	\multicolumn{1}{|l|}{$2$}&	\multicolumn{1}{|l|}{ $2$}	\\
\cline{2-4}
\multicolumn{1}{|l|}{egzaminator}&	\multicolumn{1}{|l|}{Uzyskana liczba pkt}&	\multicolumn{1}{|l|}{}&	\multicolumn{1}{|l|}{}	\\
\hline
\end{tabular}

\end{center}
$\mathrm{E}\mathrm{M}\mathrm{A}\mathrm{P}-\mathrm{P}0_{-}100$

Strona 27 z31





Zadarie 36. $\langle 0-5$)

$\mathrm{W}$ graniastoslupie prawidlowym czworokqtnym o objętości równej 108 stosunek d1ugości

krawedzi podstawy do wysokości graniastoslupa jest równy $\displaystyle \frac{1}{4}.$

Przekqtna tego graniastoslupa jest nachylona do plaszczyzny jego podstawy pod kqtem $\alpha$

(zobacz rysunek).

Oblicz cosinus kqta $\alpha$ oraz pole powierzchni calkowitej tego graniastoslupa.

Strona 28 z31

$\mathrm{E}\mathrm{M}\mathrm{A}\mathrm{P}-\mathrm{P}0_{-}100$





Wypelnia

egzaminator

Nr zadania

Maks. liczba pkt

Uzyskana liczba pkt

36.

5

-PO-100

Strona 29 z31





: {\it RU DNOPIS} \{{\it nie podlega ocenie}\}

Strona 30z31

$\mathrm{E}\mathrm{M}\mathrm{A}\mathrm{P}-\mathrm{P}0_{-}10$





{\it Wkazdym z zadań od} $f.$ {\it do 29. wybierz izaznacz na karcie odpowiedzi poprawna} $od\sqrt{}owi\mathrm{e}d\acute{z}.$

Zadanie $\mathrm{f}. (0-1$\}

Na poczqtku sezonu letniego cen9 $x$ pary sandalów podwyzszono o 20\%. Po miesiqcu

nowq cenę obnizono o 10\%. Po obu tych zmianach ta para sanda1ów kosztowa1a 81 z1.

Poczqtkowa cena $x$ pary sandalów byta równa

A. 45 z1

B. 73,63 z1

Zadanie 2. (0-1)

Liczba $(\displaystyle \frac{1}{16})^{8}\cdot 8^{16}$ jest równa

A. $2^{24}$

B. $2^{16}$

Zadanie 3, (0-1)

Liczba $\log_{\sqrt{3}}9$ jest równa

A. 2

B. 3

C. 75 z1

D. 87,48 z1

C. $2^{12}$

D. $2^{8}$

C. 4

D. 9

Zädanie 4. (0-1)

Dla $\mathrm{k}\mathrm{a}\dot{\mathrm{z}}$ dej liczby rzeczywistej $a$ i dla $\mathrm{k}\mathrm{a}\dot{\mathrm{z}}$ dej liczby rzeczywistej $b$ wartość wyrazenia

$(2a+b)^{2}-(2a-b)^{2}$ jest równa wartości wyra $\dot{\mathrm{z}}$ enia

A. $8a^{2}$

B. 8ab

C. $-8ab$

D. $2b^{2}$

Zadanie 5. (0-1)

Zbiorem wszystkich rozwiqzań nierówności

1- -23 $\chi<$ -32-$\chi$

jest przedzial

A. $(-\displaystyle \infty,-\frac{2}{3})$

B.(-$\infty$,-23)

C. $(-\displaystyle \frac{2}{3}r+\infty)$

D. $(\displaystyle \frac{2}{3},+\infty)$

Strona 4 z31

$\mathrm{E}\mathrm{M}\mathrm{A}\mathrm{P}-\mathrm{P}0_{-}100$





$0_{-}100$

Strona 31 z31










: {\it RU DNOPIS} \{{\it nie podlega ocenie}\}

$-\mathrm{P}0_{-}100$

Strona 5z31





Zadarie 6. $(0-1$\}

$\mathrm{N}\mathrm{a}\mathrm{j}\mathrm{w}\mathrm{i}_{9}$ksza liczbq bedqcq rozwiazaniem rzeczywistym równania $x(x+2)(x^{2}+9)=0$ jest

A. $(-2)$

B. 0

C. 2

D. 3

Zadanie 7. (0-1)

Równanie $\displaystyle \frac{x+1}{(x+2)(x-3)}=0$ w zbiorze liczb rzeczywistych

A. nie ma rozwiqzania.

B. ma dokladnie jedno rozwiqzanie: $(-1).$

C. ma dokladnie dwa rozwiqzania: $(-2)$ oraz 3.

D. ma dokladnie trzy rozwiazania: $(-1), (-2)$ oraz 3.

Zadqnie 8. $\langle 0-1$)

$\mathrm{W}$ paz'dzierniku 2022 roku za1ozono dwa sady, w których posadzono 1acznie 1960 drzew.

Po roku stwierdzono, $\dot{\mathrm{z}}\mathrm{e}$ uschlo 5\% drzew w pierwszym sadzie i 10\% drzew w drugim

sadzie. Uschniete drzewa usunieto, a nowych nie dosadzano.

Liczba drzew, które pozostaly w drugim sadzie, stanowila 60\% 1iczby drzew, które

pozostaly w pierwszym sadzie.

Niech $x$ oraz $y$ oznaczajq liczby drzew posadzonych- odpowiednio-w pierwszym

i drugim sadzie.

Ukladem równań, którego poprawne rozwiqzanie prowadzi do obliczenia liczby $x$ drzew

posadzonych w pierwszym sadzie oraz liczby $y$ drzew posadzonych w drugim sadzie, jest

A. 

B. 

C. 

D. 

Strona 6 z31

$\mathrm{E}\mathrm{M}\mathrm{A}\mathrm{P}-\mathrm{P}0_{-}100$





: {\it RU DNOPIS} \{{\it nie podlega ocenie}\}

$\mathrm{h}\mathrm{P}-\mathrm{P}0_{-}100$

Strona 7 z31





Zadanie 9. $(0-1$\}

$\acute{\mathrm{S}}$ rednia arytmetyczna trzech liczb: $a, b, c$, jest równa 9.

$\acute{\mathrm{S}}$ rednia arytmetyczna sześciu liczb: $a, a, b, b, c, c$, jest równa

A. 9

B. 6

C. 4,5

D. 18

Zadanie 10. (0-1)

Na rysunku przedstawiono dwie proste równolegle, które sq interpretacjq geometrycznq

jednego z ponizszych ukladów równań A-D.
\begin{center}
\includegraphics[width=97.176mm,height=83.520mm]{./F2_M_PP_M2024_page7_images/image001.eps}
\end{center}
{\it y}

1

0  1  $\chi$

Ukladem równań, którego interpretacje geometryczna przedstawiono na rysunku, jest

A. 

B. 

C. 

D. 

Strona 8 z31

$\mathrm{E}\mathrm{M}\mathrm{A}\mathrm{P}-\mathrm{P}0_{-}100$





: {\it RU DNOPIS} \{{\it nie podlega ocenie}\}

$\mathrm{h}\mathrm{P}-\mathrm{P}0_{-}100$

Strona 9z31





Zadanie ll. $\langle 0-1$)

Na rysunku przedstawiono wykres funkcji $f.$

Zbiorem wartości tej funkcji jest

A. $(-6,6\rangle$

B. $\langle$1, 4$)$

C. $\langle$1, $ 4\rangle$

D. $\langle-6,  6\rangle$

Zadanie 12. $\langle 0-1$\}

Funkcja liniowa $f$ jest określona wzorem $f(x)=(-2k+3)x+k-1$, gdzie $k\in \mathbb{R}.$

Funkcja $f$ jest malejqca dla $\mathrm{k}\mathrm{a}\dot{\mathrm{z}}$ dej liczby $k$ nalezacej do przedzialu

A. $(-\infty,1)$

B. $(-\displaystyle \infty,-\frac{3}{2})$

C. $(1,+\infty)$

D. $(\displaystyle \frac{3}{2},+\infty)$

Zädanie $l3. (0-1$\}

Funkcje liniowe $f$ oraz $g$, określone wzorami $f(x)=3x+6$ oraz $g(x)=ax+7$, maja

to samo miejsce zerowe.

Wspólczynnik $a$ we wzorze funkcji $g$ jest równy

A. $(-\displaystyle \frac{7}{2})$

B. $(-\displaystyle \frac{2}{7})$

C. -72

D. -27

Strona 10 z31

$\mathrm{E}\mathrm{M}\mathrm{A}\mathrm{P}-\mathrm{P}0_{-}100$






\begin{center}
\includegraphics[width=181.368mm,height=312.000mm]{./F2_M_PR_M2015_page0_images/image001.eps}
\end{center}
Arkusz zawiera info acje

prawnie chronione do momentu

rozpoczęcia egzaminu.

1

UZUPELNIA ZDAJACY

KOD  PESEL

{\it miejsce}

{\it na naklejkę}

dysleksja

EGZAMIN MATU  LNY Z MATEMATYKI

POZIOM ROZSZERZONY

DATA: 8 maja 2015 r.

LICZBA P  KTÓW DO UZYS NIA: 50

Instrukcja dla zdającego

1.

2.

3.

Sprawdzí, czy arkusz egzaminacyjny zawiera 22 strony (zadania $1-16$).

Ewentualny brak zgłoś przewodniczącemu zespo nadzorującego

egzamin.

Rozwiązania zadań i odpowiedzi wpisuj w miejscu na to przeznaczonym.

Odpowiedzi do zadań zamkniętych $(1-5)$ przenieś na kartę odpowiedzi,

zaznaczając je w części ka $\mathrm{y}$ przeznaczonej dla zdającego. Zamaluj $\blacksquare$

pola do tego przeznaczone. Błędne zaznaczenie otocz kółkiem $\mathrm{O}\bullet$

i zaznacz właściwe.

4.

5.

Pamiętaj, $\dot{\mathrm{z}}\mathrm{e}$ pominięcie argumentacji lub istotnych obliczeń

w rozwiązaniu zadania otwa ego (7-16) $\mathrm{m}\mathrm{o}\dot{\mathrm{z}}\mathrm{e}$ spowodować, $\dot{\mathrm{z}}\mathrm{e}$ za to

rozwiązanie nie otrzymasz pełnej liczby punktów.

Pisz cz elnie i $\mathrm{u}\dot{\mathrm{z}}$ aj tvlko $\mathrm{d}$ gopisu lub -Dióra z czatnym tuszem lub

atramentem.

6. Nie uzywaj korektora, a błędne zapisy wyra $\acute{\mathrm{z}}\mathrm{n}\mathrm{i}\mathrm{e}$ prze eśl.

7. Pamiętaj, $\dot{\mathrm{z}}\mathrm{e}$ zapisy w brudnopisie nie będą oceniane.

8. $\mathrm{M}\mathrm{o}\dot{\mathrm{z}}$ esz korzystać z zestawu wzorów matematycznych, cyrkla i linijki oraz

kalkulatora prostego.

9. Na tej stronie oraz na karcie odpowiedzi wpisz swój numer PESEL

i przyklej naklejkę z kodem.

10. Nie wpisuj $\dot{\mathrm{z}}$ adnych znaków w części przeznaczonej dla egzaminatora.

$\Vert\Vert\Vert\Vert\Vert\Vert\Vert\Vert\Vert\Vert\Vert\Vert\Vert\Vert\Vert\Vert\Vert\Vert\Vert\Vert\Vert\Vert\Vert\Vert|$

$\mathrm{M}\mathrm{M}\mathrm{A}-\mathrm{R}1_{-}1\mathrm{P}-152$

Układ graficzny

\copyright CKE 2015

1




{\it Wzadaniach od l. do 5. wybierz i zaznacz na karcie odpowiedzi poprawnq odpowiedzí}.

ZadaOie $l.(0-1)$

Na rysunku przedstawiony

nierównoŚć $|2x-8|\leq 10.$

jest zbiór

wszystkich liczb

rzeczywistych

spełniających
\begin{center}
\includegraphics[width=165.504mm,height=11.988mm]{./F2_M_PR_M2015_page1_images/image001.eps}
\end{center}
$-1$  {\it k  x}

Stąd wynika, $\dot{\mathrm{z}}\mathrm{e}$

A. $k=2$

B. $k=4$

C. $k=5$

D. $k=9$

Zadanie 2. $(0-l\rangle$

Dana jest funkcja $f$ określona wzorem $f(x)=$

Równanie $f(x)=1$ ma dokładnie

A. jedno rozwiązanie.

B. dwa rozwiązania.

C. cztery rozwiązania.

D. pięć rozwiązań.

Zadanie 3. (0-1)

Liczba $(3-2\sqrt{3})^{3}$ jest równa

A. $27-24\sqrt{3}$ B. $27-30\sqrt{3}$

C. $135-78\sqrt{3}$

D. $135-30\sqrt{3}$

Zadanie 4. $(0-l\rangle$

Równanie 2 $\sin x+3\cos x=6$ w przedziale $(0,2\pi)$

A. nie ma rozwiązań rzeczywistych.

B. ma dokładniejedno rozwiązanie rzeczywiste.

C. ma dokładnie dwa rozwiązania rzeczywiste.

D. ma więcej $\mathrm{n}\mathrm{i}\dot{\mathrm{z}}$ dwa rozwiązania rzeczywiste.

Ządanie 5. $(0-1\rangle$

Odległość początku układu współrzędnych od prostej o równaniu $y=2x+4$ jest równa

A.

$\displaystyle \frac{\sqrt{5}}{5}$

B.

$\displaystyle \frac{4\sqrt{5}}{5}$

C.

-45

D. 4

Strona 2 z22

MMA-IR





Zadanie 11. (0-4)

$\mathrm{W}$ pierwszej utnie umieszczono 3 ku1e białe i 5 ku1 czamych, a w drugiej urnie 7 ku1 białych

$\mathrm{i}2$ kule czarne. Losujemy jedną kulę z pierwszej umy, przekładamy ją do urny drugiej

i dodatkowo dokładamy do umy drugiej jeszcze dwie kule tego samego koloru, co

wylosowana kula. Następnie losujemy dwie kule z umy drugiej. Oblicz prawdopodobieństwo

zdarzenia polegającego na tym, $\dot{\mathrm{z}}\mathrm{e}$ obie kule wylosowane z drugiej urny będą białe.

Odpowied $\acute{\mathrm{z}}$:
\begin{center}
\includegraphics[width=96.012mm,height=17.832mm]{./F2_M_PR_M2015_page10_images/image001.eps}
\end{center}
Wypelnia

egzaminator

Nr zadania

Maks. liczba kt

10.

4

11.

4

Uzyskana liczba pkt

IMA-IR

Strona ll z22





Zadanie $l2. (0-4)$

Funkcja $f$ określona jest wzorem $f(x)=x^{3}-2x^{2}+1$ dla $\mathrm{k}\mathrm{a}\dot{\mathrm{z}}$ dej liczby rzeczywistej $x.$

Wyznacz równania tych stycznych do wykresu funkcji $f$, które są równoległe do prostej

o równaniu $y=4x.$

Strona 12 z22

MMA-IR





Odpowiedzí :
\begin{center}
\includegraphics[width=82.044mm,height=17.784mm]{./F2_M_PR_M2015_page12_images/image001.eps}
\end{center}
Wypelnia

egzamÍnator

Nr zadania

Maks. liczba kt

12.

4

Uzyskana liczba pkt

IMA-IR

Strona 13 z22





Zadanie 13. $(0-5\rangle$

Dany jest trójmian kwadratowy $f(x)=(m+1)x^{2}+2(m-2)x-m+4$. Wyznacz wszystkie

wartości parametru $m$, dla których trójmian $f$ ma dwa rózne pierwiastki rzeczywiste $x_{1}, x_{2},$

spełniające warunek $x_{1}^{2}-x_{2}^{2}=x_{1}^{4}-x_{2}^{4}$

Strona 14 z22

MMA-IR





Odpowiedzí :
\begin{center}
\includegraphics[width=82.044mm,height=17.832mm]{./F2_M_PR_M2015_page14_images/image001.eps}
\end{center}
Wypelnia

egzaminator

Nr zadania

Maks. liczba kt

13.

5

Uzyskana liczba pkt

IMA-IR

Strona 15 z22





Zadanie $l4. \zeta 0-5\rangle$

Podstawą ostrosłupa ABCDS jest kwadrat ABCD. Krawędzí boczna $SD$ jest wysokością

ostrosłupa, ajej długość jest dwa razy większa od długości krawędzi podstawy. Oblicz sinus

kąta między ścianami bocznymi $ABS\mathrm{i}CBS$ tego ostrosłupa.

Strona 16 z22

MMA-IR





Odpowiedzí :
\begin{center}
\includegraphics[width=82.044mm,height=17.784mm]{./F2_M_PR_M2015_page16_images/image001.eps}
\end{center}
Wypelnia

egzamÍnator

Nr zadania

Maks. liczba kt

14.

5

Uzyskana liczba pkt

IMA-IR

Strona 17 z22





Zadanie 15. $(0-6)$

Suma wszystkich

czterech współczynników wielomianu

$W(x)=x^{3}+ax^{2}+bx+c$ jest

równa 0. Trzy pierwiastki tego wie1omianu tworzą ciąg arytmetyczny o róznicy równej 3.

Oblicz współczynniki $a, b \mathrm{i}c$. Rozwaz wszystkie $\mathrm{m}\mathrm{o}\dot{\mathrm{z}}$ liwe przypadki.

Strona 18 z22

MMA-IR





Odpowiedzí :
\begin{center}
\includegraphics[width=82.044mm,height=17.784mm]{./F2_M_PR_M2015_page18_images/image001.eps}
\end{center}
Wypelnia

egzamÍnator

Nr zadania

Maks. liczba kt

15.

Uzyskana liczba pkt

IMA-IR

Strona 19 z22





Zadanie $1\epsilon. (0-7)$

Rozpatrujemy wszystkie stozki, których przekrojem osiowym jest trójkąt o obwodzie 20.

Oblicz wysokość i promień podstawy tego stozka, którego objętość jest największa. Oblicz

objętość tego stozka.

Strona 20 z22

MMA-IR





{\it BRUDNOPIS} ({\it nie podlega ocenie})

Strona 3 z22





Odpowiedzí :
\begin{center}
\includegraphics[width=82.044mm,height=17.784mm]{./F2_M_PR_M2015_page20_images/image001.eps}
\end{center}
Wypelnia

egzamÍnator

Nr zadania

Maks. liczba kt

7

Uzyskana liczba pkt

IMA-IR

Strona 21 z22





{\it BRUDNOPIS} ({\it nie podlega ocenie})

Strona 22 z22

MD





Zadanie $\epsilon.(0-2)$

Oblicz granicę $\displaystyle \lim_{n\rightarrow\infty}(\frac{11n^{3}+6n+5}{6n^{3}+1}-\frac{2n^{2}+2n+1}{5n^{2}-4}). \mathrm{W}$ ponizsze kratki wpisz kolejno cyfrę

jedności i pierwsze dwie cyfry po przecinku rozwinięcia dziesiętnego otrzymanego wyniku.
\begin{center}
\includegraphics[width=22.500mm,height=10.920mm]{./F2_M_PR_M2015_page3_images/image001.eps}
\end{center}
Strona 4 z22

MMA-IR





Zadanie 7. (0-2)

Liczby $(-1) \mathrm{i}3$ są miejscami zerowymi funkcji kwadratowej $f$. Oblicz $\displaystyle \frac{f(6)}{f(12)}.$

Odpowied $\acute{\mathrm{z}}$:
\begin{center}
\includegraphics[width=96.012mm,height=17.832mm]{./F2_M_PR_M2015_page4_images/image001.eps}
\end{center}
Wypelnia

egzaminator

Nr zadania

Maks. liczba kt

2

7.

2

Uzyskana liczba pkt

IMA-IR

Strona 5 z22





Zadanie S. (0-3)

Udowodnij, $\dot{\mathrm{z}}\mathrm{e}$ dla $\mathrm{k}\mathrm{a}\dot{\mathrm{z}}$ dej liczby rzeczywistej $x$ prawdziwajest nierówność

$x^{4}-x^{2}-2x+3>0.$

Strona 6 z22

MD




\begin{center}
\includegraphics[width=82.044mm,height=17.784mm]{./F2_M_PR_M2015_page6_images/image001.eps}
\end{center}
Wypelnia

egzamÍnator

Nr zadania

Maks. liczba kt

8.

3

Uzyskana liczba pkt

Strona 7 z22





Zadanie 9. (0-3)

Dwusieczne czworokąta ABCD wpisanego w okrąg przecinają się w czterech róznych

punktach: P, Q, R, S (zobacz rysunek)
\begin{center}
\includegraphics[width=119.628mm,height=112.212mm]{./F2_M_PR_M2015_page7_images/image001.eps}
\end{center}
{\it D}

{\it A  S}

{\it P  R}

{\it Q}

{\it B}

{\it C}

Wykaz, $\dot{\mathrm{z}}\mathrm{e}$ na czworokącie PQRS mozna opisać okrąg.

Strona 8 z22

MMA-IR




\begin{center}
\includegraphics[width=82.044mm,height=17.784mm]{./F2_M_PR_M2015_page8_images/image001.eps}
\end{center}
Wypelnia

egzamÍnator

Nr zadania

Maks. liczba kt

3

Uzyskana liczba pkt

Strona 9 z22





Zadanie 10. (0-4)

Długości boków czworokąta ABCD są równe: $|AB|=2, |BC|=3, |CD|=4, |DA|=5.$

Na czworokącie ABCD opisano okrąg. Oblicz długość przekątnej $AC$ tego czworokąta.

Odpowiedzí:

Strona 10 z22

MMA-IR







$\mathrm{c}\varepsilon \mathrm{N}\mathrm{T}\mathrm{R}\mathrm{A}\mathrm{l}\aleph$AKOMfSJAECZAMINACYJNA

Arkusz zawiera info acje

prawnie chronione do momentu

rozpoczęcia egzaminu.
\begin{center}
\includegraphics[width=22.908mm,height=19.248mm]{./F2_M_PR_M2016_page0_images/image001.eps}
\end{center}
1  $\iota$

UZUPELNIA ZDAJACY

{\it miejsce}

{\it na naklejkę}
\begin{center}
\includegraphics[width=21.900mm,height=14.736mm]{./F2_M_PR_M2016_page0_images/image002.eps}
\end{center}
KOD
\begin{center}
\includegraphics[width=79.656mm,height=14.736mm]{./F2_M_PR_M2016_page0_images/image003.eps}
\end{center}
PESEL
\begin{center}
\includegraphics[width=194.568mm,height=251.616mm]{./F2_M_PR_M2016_page0_images/image004.eps}
\end{center}
dysleksja

EGZAMIN MATU  LNY Z MATEMATY

POZIOM ROZSZE  ONY

LICZBA P  KTÓW DO UZYS NIA: 50

Instrukcja dla zdającego

1.

2.

3.

4.

5.

6.

Sprawdzí, czy arkusz egzaminacyjny zawiera 22 strony (zadania $1-16$).

Ewentualny brak zgloś przewodniczącemu zespo nadzo jącego

egzamin.

Rozwiązania zadań i odpowiedzi wpisuj w miejscu na to przeznaczonym.

Odpowiedzi do zadań za ię ch $(1-5)$ zaznacz na karcie odpowiedzi

w części karty przeznaczonej dla zdającego. Zamaluj $\blacksquare$ pola do tego

przeznaczone. Błędne zaznaczenie otocz kółkiem \copyright i zaznacz właściwe.

$\mathrm{W}$ zadaniu 6. wpisz odpowiednie cyf w atki pod treścią zadania.

Pamiętaj, $\dot{\mathrm{z}}\mathrm{e}$ pominięcie argumentacji lub istotnych obliczeń

w rozwiązaniu zadania otwa ego (7-16) $\mathrm{m}\mathrm{o}\dot{\mathrm{z}}\mathrm{e}$ spowodować, $\dot{\mathrm{z}}\mathrm{e}$ za to

rozwiązanie nie otr masz pełnej liczby pu tów.

Pisz cz elnie i $\mathrm{u}\dot{\mathrm{z}}$ aj lko $\mathrm{d}$ gopisu lub pióra z czarnym tuszem lub

atramentem.

7. Nie $\mathrm{u}\dot{\mathrm{z}}$ aj korektora, a błędne zapisy $\mathrm{r}\mathrm{a}\acute{\mathrm{z}}\mathrm{n}\mathrm{i}\mathrm{e}$ prze eśl.

8. Pamiętaj, $\dot{\mathrm{z}}\mathrm{e}$ zapisy w brudnopisie nie będą oceniane.

9. $\mathrm{M}\mathrm{o}\dot{\mathrm{z}}$ esz korzystać z zesta wzorów matematycznych, cyrkla i linijki oraz

kalkulatora prostego.

10. Na tej stronie oraz na karcie odpowiedzi wpisz swój numer PESEL

i przyklej naklejkę z kodem.

ll. Nie wpisuj $\dot{\mathrm{z}}$ adnych znaków w części przeznaczonej dla egzaminatora.

$\Vert\Vert\Vert\Vert\Vert\Vert\Vert\Vert\Vert\Vert\Vert\Vert\Vert\Vert\Vert\Vert\Vert\Vert\Vert\Vert\Vert\Vert\Vert\Vert|$

$\mathrm{M}\mathrm{M}\mathrm{A}-\mathrm{R}1_{-}1\mathrm{P}-162$
\begin{center}
\includegraphics[width=22.908mm,height=19.248mm]{./F2_M_PR_M2016_page0_images/image005.eps}
\end{center}
1  $\iota$

Układ graficzny

\copyright CKE 2015




{\it Wzadaniach od l. do 5. wybierz i zaznacz na karcie odpowiedzi poprawnq odpowiedzí}.

Zadaoie $l.(0-1)$

$\mathrm{W}$ rozwinięciu wyrazenia $(2\sqrt{3}x+4y)^{3}$ współczynnik przy iloczynie $xy^{2}$ jest równy

A. $32\sqrt{3}$

B. 48

C. $96\sqrt{3}$

D. 144

Zadanie 2. (0-1)

Wielomian $W(x)=6x^{3}+3x^{2}-5x+p$ jest podzielny przez dwumian $x-1$ dla $p$ równego

A. 4

B. $-2$

C. 2

D. $-4$

Zadanie 3. (0-1)

Na rysunku przedstawiono fragment wykresu

dziedziną jest zbiór $D=(-\infty,3)\cup(3,+\infty).$

funkcji homograficznej $y=f(x)$, której

Równanie $|f(x)|=p$ z niewiadomą $x$ ma dokładniejedno rozwiązanie

A.

C.

w dwóch przypadkach: $p=0$ lub $p=3.$

tylko wtedy, gdy $p=3.$

B.

D.

w dwóch przypadkach: $p=0$ lub $p=2.$

tylko wtedy, gdy $p=2.$

Zadanie 4. (0-1)

Funkcja $f(x)=\displaystyle \frac{3x-1}{x^{2}+4}$ jest określona dla $\mathrm{k}\mathrm{a}\dot{\mathrm{z}}$ dej liczby rzeczywistej $x$. Pochodna tej funkcji

jest określona wzorem

A.

$f'(x)=\displaystyle \frac{-3x^{2}+2x+12}{(x^{2}+4)^{2}}$

B.

$f'(x)=\displaystyle \frac{-9x^{2}+2x-12}{(x^{2}+4)^{2}}$

C.

$f'(x)=\displaystyle \frac{3x^{2}-2x-12}{(x^{2}+4)^{2}}$

D.

$f'(x)=\displaystyle \frac{9x^{2}-2x+12}{(x^{2}+4)^{2}}$

Strona 2 z22

MMA-IR





Zadanie 11. (0-4)

Rozwiąz nierówność $\displaystyle \frac{2\cos x-\sqrt{3}}{\cos^{2}x}<0$ w przedziale $\langle 0, 2\pi\rangle.$

Odpowiedzí :
\begin{center}
\includegraphics[width=96.012mm,height=17.832mm]{./F2_M_PR_M2016_page10_images/image001.eps}
\end{center}
Wypelnia

egzaminator

Nr zadania

Maks. liczba kt

10.

4

11.

4

Uzyskana liczba pkt

IMA-IR

Strona ll z22





Zadanie $l2\cdot(0-6)$

Dany jest trójmian

kwadratowy $f(x)=x^{2}+2(m+1)x+6m+1.$

Wyznacz wszystkie

rzeczywiste wartości parametru $m$, dla których ten trójmian ma dwa rózne pierwiastki $x_{1}, x_{2}$

tego samego znaku, spełniające warunek $|x_{1}-x_{2}|<3.$

Strona 12 z22

MMA-IR





Odpowiedzí :
\begin{center}
\includegraphics[width=82.044mm,height=17.832mm]{./F2_M_PR_M2016_page12_images/image001.eps}
\end{center}
Wypelnia

egzaminator

Nr zadania

Maks. liczba kt

12.

Uzyskana liczba pkt

IMA-IR

Strona 13 z22





Zadanie 13. $(0-5\rangle$

Punkty $A=(30,32) \mathrm{i} B=(0,8)$ są sąsiednimi wierzchołkami czworokąta ABCD wpisanego

w okrąg. Prosta o równaniu $x-y+2=0$ jest jedyną osią symetrii tego czworokąta i zawiera

przekątną $AC$. Oblicz współrzędne wierzchołków $C\mathrm{i}D$ tego czworokąta.

Strona 14 z22

MMA-IR





Odpowiedzí :
\begin{center}
\includegraphics[width=82.044mm,height=17.832mm]{./F2_M_PR_M2016_page14_images/image001.eps}
\end{center}
Wypelnia

egzaminator

Nr zadania

Maks. liczba kt

13.

5

Uzyskana liczba pkt

IMA-IR

Strona 15 z22





Zadanie $l4. \zeta 0-3$)

Rozpatrujemy wszystkie liczby naturalne dziesięciocyfrowe, w zapisie których mogą

występować wyłącznie cyfry 1, 2, 3, przy czym cyfra 1 występuje dokładnie trzy razy.

Uzasadnij, $\dot{\mathrm{z}}\mathrm{e}$ takich liczb jest 15360.

Strona 16 z22

MMA-IR





Odpowiedzí :
\begin{center}
\includegraphics[width=82.044mm,height=17.784mm]{./F2_M_PR_M2016_page16_images/image001.eps}
\end{center}
Wypelnia

egzamÍnator

Nr zadania

Maks. liczba kt

14.

3

Uzyskana liczba pkt

IMA-IR

Strona 17 z22





Zadanie 15. $(0-6)$

$\mathrm{W}$ ostrosłupie prawidłowym czworokątnym ABCDS o podstawie ABCD wysokość jest równa 5,

a kąt między sąsiednimi ścianami bocznymi ostrosłupa ma miarę $120^{\mathrm{o}}$ Oblicz objętość tego

ostrosłupa.

Strona 18 z22

MMA-IR





Odpowiedzí :
\begin{center}
\includegraphics[width=82.044mm,height=17.832mm]{./F2_M_PR_M2016_page18_images/image001.eps}
\end{center}
Wypelnia

egzaminator

Nr zadania

Maks. liczba kt

15.

Uzyskana liczba pkt

IMA-IR

Strona 19 z22





Zadanie $1\epsilon. (0-7)$

Parabola o równaniu $y=2-\displaystyle \frac{1}{2}x^{2}$ przecina oś $Ox$ układu współrzędnych w punktach

$A=(-2,0) \mathrm{i} B=(2,0)$. Rozpatrujemy wszystkie trapezy równoramienne ABCD, których

dłuzszą podstawą jest odcinek $AB$, a końce $C\mathrm{i}D$ krótszej podstawy lez$\cdot$ą na paraboli (zobacz

rysunek).
\begin{center}
\includegraphics[width=87.120mm,height=50.592mm]{./F2_M_PR_M2016_page19_images/image001.eps}
\end{center}
{\it D C}

1

{\it A}

2

{\it B}

$-1$  0 1

Wyznacz pole trapezu ABCD w zalezności od pierwszej współrzędnej wierzchołka C. Oblicz

współrzędne wierzchołka C tego z rozpatrywanych trapezów, którego polejest największe.

Strona 20 z22

MMA-IR





{\it BRUDNOPIS} ({\it nie podlega ocenie})

Strona 3 z22





Odpowiedzí :
\begin{center}
\includegraphics[width=82.044mm,height=17.832mm]{./F2_M_PR_M2016_page20_images/image001.eps}
\end{center}
Wypelnia

egzaminator

Nr zadania

Maks. liczba kt

7

Uzyskana liczba pkt

IMA-IR

Strona 21 z22





{\it BRUDNOPIS} ({\it nie podlega ocenie})

Strona 22 z22

MD





Zadanie 5. $(0-l)$

Granica $\displaystyle \lim_{n\rightarrow\infty}\frac{(pn^{2}+4n)^{3}}{5n^{6}-4}=-\frac{8}{5}$. Wynika stąd, $\dot{\mathrm{z}}\mathrm{e}$

A.

$p=-8$

B.

$p=4$

C.

$p=2$

D.

$p=-2$

Zadanie $\epsilon$, (0-2)

Wśród 10 tysięcy mieszkańców pewnego miasta przeprowadzono sondaz dotyczący budowy

przedszkola publicznego. Wyniki sondaz$\mathrm{u}$ przedstawiono w tabeli.
\begin{center}
\begin{tabular}{|l|l|l|}
\hline
\multicolumn{1}{|l|}{Badane grupy}&	\multicolumn{1}{|l|}{$\begin{array}{l}\mbox{Liczba osób popierających}	\\	\mbox{budowę przedszkola}	\end{array}$}&	\multicolumn{1}{|l|}{$\begin{array}{l}\mbox{Liczba osób niepopierających}	\\	\mbox{budowy przedszkola}	\end{array}$}	\\
\hline
\multicolumn{1}{|l|}{Kobiety}&	\multicolumn{1}{|l|}{$5140$}&	\multicolumn{1}{|l|}{ $1860$}	\\
\hline
\multicolumn{1}{|l|}{Męzczyzíni}&	\multicolumn{1}{|l|}{$2260$}&	\multicolumn{1}{|l|}{ $740$}	\\
\hline
\end{tabular}

\end{center}
Oblicz prawdopodobieństwo zdarzenia polegającego na tym, $\dot{\mathrm{z}}\mathrm{e}$ losowo wybrana osoba,

spośród ankietowanych, popiera budowę przedszkola, jeśli wiadomo, $\dot{\mathrm{z}}\mathrm{e}$ jest męzczyzną.

Zakoduj trzy pierwsze cyfry po przecinku nieskończonego rozwinięcia dziesiętnego

otrzymanego wyniku.
\begin{center}
\includegraphics[width=22.500mm,height=10.920mm]{./F2_M_PR_M2016_page3_images/image001.eps}
\end{center}
{\it BRUDNOPIS} ({\it nie podlega ocenie})

Strona 4 z22

MMA-IR





Zadanie 7. (0-2)

Dany jest ciąg geometryczny $(a_{n})$ określony wzorem $a_{n}=(\displaystyle \frac{1}{2x-371})^{n}$ dla $n\geq 1$. Wszystkie

wyrazy tego ciągu są dodatnie. Wyznacz najmniejszą liczbę całkowitą $x$, dla której

nieskończony szereg $a_{1}+a_{2}+a_{3}+$ jest zbiezny.

Odpowied $\acute{\mathrm{z}}$:
\begin{center}
\includegraphics[width=96.012mm,height=17.832mm]{./F2_M_PR_M2016_page4_images/image001.eps}
\end{center}
Wypelnia

egzaminator

Nr zadania

Maks. liczba kt

2

7.

2

Uzyskana liczba pkt

IMA-IR

Strona 5 z22





Zadanie S. (0-3)

Wykaz, $\dot{\mathrm{z}}\mathrm{e}$ dla dowolnych dodatnich liczb rzeczywistych $x \mathrm{i} y$ takich, $\dot{\mathrm{z}}\mathrm{e}$

prawdziwajest nierówność $x+y\leq 2.$

$x^{2}+y^{2}=2,$

Strona 6 z22

MMA-IR




\begin{center}
\includegraphics[width=82.044mm,height=17.784mm]{./F2_M_PR_M2016_page6_images/image001.eps}
\end{center}
Wypelnia

egzamÍnator

Nr zadania

Maks. liczba kt

8.

3

Uzyskana liczba pkt

Strona 7 z22





Zadanie 9. (0-3)

Dany jest prostokąt ABCD. Okrąg wpisany w trójkąt BCD jest styczny do przekątnej BD

w punkcie N. Okrąg wpisany w trójkąt ABD jest styczny do boku AD w punkcie M, a środek S

tego okręgu lezy na odcinku MN, jak na rysunku.
\begin{center}
\includegraphics[width=77.016mm,height=51.204mm]{./F2_M_PR_M2016_page7_images/image001.eps}
\end{center}
{\it D C}

{\it M  N  S}

{\it A  B}

Wykaz, $\dot{\mathrm{z}}\mathrm{e}|MN|=|AD|.$

Strona 8 z22

MMA-IR




\begin{center}
\includegraphics[width=82.044mm,height=17.832mm]{./F2_M_PR_M2016_page8_images/image001.eps}
\end{center}
Wypelnia

egzaminator

Nr zadania

Maks. liczba kt

3

Uzyskana liczba pkt

Strona 9 z22





Zadanie 10. (0-4)

Wyznacz wszystkie wartości parametru $a$, dla których wykresy funkcji $f\mathrm{i}g$, określonych

wzorami $f(x)=x-2$ oraz $g(x)=5-ax$, przecinają się w punkcie o obu współrzędnych

dodatnich.

Odpowiedzí:

Strona 10 z22

MMA-IR







$\mathrm{g}_{\mathrm{E}\mathrm{G}\mathrm{Z}\mathrm{A}\mathrm{M}\mathrm{I}\mathrm{N}\mathrm{A}\subset \mathrm{Y}\mathrm{J}\mathrm{N}\mathrm{A}}^{\mathrm{C}\mathrm{E}\mathrm{N}\mathrm{T}\mathrm{R}\mathrm{A}\mathrm{L}\mathrm{N}\mathrm{A}}\mathrm{K}\mathrm{O}\mathrm{M}1\mathrm{S}\mathrm{J}\mathrm{A}$

Arkusz zawiera informacje

prawnie chronione do momentu

rozpoczęcia egzaminu.

UZUPELNIA ZDAJACY

{\it miejsce}

{\it na naklejkę}
\begin{center}
\includegraphics[width=21.900mm,height=16.056mm]{./F2_M_PR_M2017_page0_images/image001.eps}
\end{center}
KOD
\begin{center}
\includegraphics[width=79.656mm,height=16.104mm]{./F2_M_PR_M2017_page0_images/image002.eps}
\end{center}
PESEL
\begin{center}
\includegraphics[width=195.984mm,height=236.676mm]{./F2_M_PR_M2017_page0_images/image003.eps}
\end{center}
EGZAMIN MATU  LNY

Z MATEMATY

POZIOM ROZSZE  ONY

DATA: 9 maja 2017 $\mathrm{r}.$

CZAS P CY: $ 18\Uparrow$ minut

LICZBA P KTÓW DO UZYS NIA: 50

Instrukcja dla zdającego

l. Sprawdzí, czy arkusz egzaminacyjny zawiera 18 stron (zadania $1-15$).

Ewentualny brak zgłoś przewodniczącemu zespo nadzorującego

egzamin.

2. Rozwiązania zadań i odpowiedzi wpisuj w miejscu na to przeznaczonym.

3. Odpowiedzi do zadań za ię ch (l ) zaznacz na karcie odpowiedzi

w części ka $\mathrm{y}$ przeznaczonej dla zdającego. Zamaluj $\blacksquare$ pola do tego

przeznaczone. Błędne zaznaczenie otocz kólkiem \copyright i zaznacz wlaściwe.

4. $\mathrm{W}$ zadaniu 5. wpisz odpowiednie cyf w atki pod treścią zadania.

5. Pamiętaj, $\dot{\mathrm{z}}\mathrm{e}$ pominięcie argumentacji lub istotnych obliczeń

w rozwiązaniu zadania otwa ego (6-15) $\mathrm{m}\mathrm{o}\dot{\mathrm{z}}\mathrm{e}$ spowodować, $\dot{\mathrm{z}}\mathrm{e}$ za to

rozwiązanie nie otrzymasz pelnej liczby pu tów.

6. Pisz cz elnie i $\mathrm{u}\dot{\mathrm{z}}$ aj lko $\mathrm{d}$ gopisu lub pióra z czamym tuszem lub

atramentem.

7. Nie $\mathrm{u}\dot{\mathrm{z}}$ aj korektora, a błędne zapisy razínie prze eśl.

8. Pamiętaj, $\dot{\mathrm{z}}\mathrm{e}$ zapisy w brudnopisie nie będą oceniane.

9. $\mathrm{M}\mathrm{o}\dot{\mathrm{z}}$ esz korzystać z zesta wzorów matematycznych, cyrkla i linijki oraz

kal latora prostego.

10. Na tej stronie oraz na karcie odpowiedzi wpisz swój numer PESEL

i przyklej naklejkę z kodem.

ll. Nie wpisuj $\dot{\mathrm{z}}$ adnych znaków w części przeznaczonej dla egzaminatora.

$\Vert\Vert\Vert\Vert\Vert\Vert\Vert\Vert\Vert\Vert\Vert\Vert\Vert\Vert\Vert\Vert\Vert\Vert\Vert\Vert\Vert\Vert\Vert\Vert|$

$\mathrm{M}\mathrm{M}\mathrm{A}-\mathrm{R}1_{-}1\mathrm{P}-172$

Układ graficzny

\copyright CKE 2015




{\it Wzadaniach od l. do 4. wybierz i zaznacz na karcie odpowiedzi poprawnq odpowiedzí}.

Zadaoie $l.(0-1)$

Liczba $(\sqrt{2-\sqrt{3}}-\sqrt{2+\sqrt{3}})^{2}$ jest równa

A. 2

B. 4

C. $\sqrt{3}$

D.

$2\sqrt{3}$

Zadanie 2. $(0-l)$

Nieskończony ciąg

Wtedy

liczbowy jest określony wzorem

{\it an}$=$ -({\it n}22-{\it n}130$+${\it nn}) (22$+$-33{\it n})

dla $n\geq 1.$

A.

$\displaystyle \lim_{n\rightarrow\infty}a_{n}=\frac{1}{2}$

B.

$\displaystyle \lim_{n\rightarrow\infty}a_{n}=0$

C.

$\displaystyle \lim_{n\rightarrow\infty}a_{n}=-\infty$

D.

$\displaystyle \lim_{n\rightarrow\infty}a_{n}=-\frac{3}{2}$

Zadanie 3. $(0-l)$

Odcinek $CD$ jest wysokością trójkąta $ABC$, w którym $|AD|=|CD|=\displaystyle \frac{1}{2}|BC|$ (zobacz rysunek).

Okrąg o środku $C$ i promieniu $CD$ jest styczny do prostej $AB$. Okrąg ten przecina boki

$AC\mathrm{i}BC$ trójkąta odpowiednio w punktach $K\mathrm{i}L.$
\begin{center}
\includegraphics[width=63.804mm,height=47.652mm]{./F2_M_PR_M2017_page1_images/image001.eps}
\end{center}
{\it M}

$\alpha$

{\it C}

{\it L}

{\it K}

{\it A  D  B}

Zaznaczony na rysunku kąt $\alpha$ wpisany w okrągjest równy

A. $37,5^{\mathrm{o}}$

B. $45^{\mathrm{o}}$

C. 52, $5^{\mathrm{o}}$

D. $60^{\mathrm{o}}$

Zadanie 4. (0-1)

Dane są punkt $B=(-4,7)$ i wektor $\vec{u}=[-3,5]$. Punkt $A$, taki, $\dot{\mathrm{z}}\mathrm{e}\vec{AB}=-3\vec{u}$, ma współrzędne

A. $A=(5,-8)$

B. $A=(-13,22)$

C. $A=(9,-15)$

D. $A=(12,24)$

Strona 2 z18

MMA-IR





Odpowied $\acute{\mathrm{z}}$:
\begin{center}
\includegraphics[width=82.044mm,height=17.784mm]{./F2_M_PR_M2017_page10_images/image001.eps}
\end{center}
Wypelnia

egzamÍnator

Nr zadania

Maks. liczba kt

12.

5

Uzyskana liczba pkt

IMA-IR

Strona ll z18





Zadanie 13. $(0-5\rangle$

Wyznacz równanie okręgu przechodzącego przez punkty $A=(-5,3) \mathrm{i} B=(0,6)$, którego

środek lezy na prostej o równaniu $x-3y+1=0.$

Strona 12 z18

MMA-IR





Odpowied $\acute{\mathrm{z}}$:
\begin{center}
\includegraphics[width=82.044mm,height=17.784mm]{./F2_M_PR_M2017_page12_images/image001.eps}
\end{center}
Wypelnia

egzamÍnator

Nr zadania

Maks. liczba kt

13.

5

Uzyskana liczba pkt

IMA-IR

Strona 13 z18





Zadanie $l4. (0-6)$

Liczby $a, b, c$ są - odpowiednio - pierwszym, drugim i trzecim wyrazem ciągu

arytmetycznego. Suma tych liczb jest równa 27. Ciąg $(a-2,b,2c+1)$ jest geometryczny.

Wyznacz liczby $a, b, c.$

Strona 14 z18

MMA-IR





Odpowied $\acute{\mathrm{z}}$:
\begin{center}
\includegraphics[width=82.044mm,height=17.784mm]{./F2_M_PR_M2017_page14_images/image001.eps}
\end{center}
Wypelnia

egzamÍnator

Nr zadania

Maks. liczba kt

14.

Uzyskana liczba pkt

IMA-IR

Strona 15 z18





Zadanie 15. (0-7)

Rozpatrujemy wszystkie walce o danym polu powierzchni całkowitej P. Oblicz wysokość

i promień podstawy tego walca, którego objętość jest największa. Oblicz tę największą

objętość.

Strona 16 z18

MMA-IR





Odpowied $\acute{\mathrm{z}}$:
\begin{center}
\includegraphics[width=82.044mm,height=17.784mm]{./F2_M_PR_M2017_page16_images/image001.eps}
\end{center}
Wypelnia

egzamÍnator

Nr zadania

Maks. liczba kt

15.

7

Uzyskana liczba pkt

IMA-IR

Strona 17 z18





{\it BRUDNOPIS} ({\it nie podlega ocenie})

Strona 18 z18

Nl





{\it BRUDNOPIS} ({\it nie podlega ocenie})

Strona 3 z18





Zadanie 5. (0-2)

Reszta z dzielenia wielomianu $W(x)=x^{3}-2x^{2}+ax+\displaystyle \frac{3}{4}$ przez dwumian $x-2$ jest równa l.

Oblicz wartość współczynnika $a.$

$\mathrm{W}$ ponizsze kratki wpisz kolejno trzy pierwsze cyfry po przecinku rozwinięcia dziesiętnego

otrzymanego wyniku.
\begin{center}
\includegraphics[width=22.500mm,height=10.920mm]{./F2_M_PR_M2017_page3_images/image001.eps}
\end{center}
{\it BRUDNOPIS} ({\it nie podlega ocenie})

Zadanie 6. (0-3)

Funkcja $f$ jest określona wzorem $f(x)=\displaystyle \frac{x-1}{x^{2}+1}$ dla $\mathrm{k}\mathrm{a}\dot{\mathrm{z}}$ dej liczby rzeczywistej $x$. Wyznacz

równanie stycznej do wykresu tej funkcji w punkcie $P=(1,0).$

Odpowiedzí:

Strona 4 z18

MMA-IR





Zadanie 7. (0-3)

Udowodnij, $\dot{\mathrm{z}}\mathrm{e}$ dla dowolnych róznych liczb rzeczywistych $x,y$ prawdziwajest nierówność

$x^{2}y^{2}+2x^{2}+2y^{2}-8xy+4>0.$
\begin{center}
\includegraphics[width=109.980mm,height=17.832mm]{./F2_M_PR_M2017_page4_images/image001.eps}
\end{center}
Nr zadania

Wypelnia Maks. liczba kt

egzaminator

Uzyskana liczba pkt

5.

2

3

7.

3

IMA-IR

Strona 5 z18





Zadanie 8. $(0\rightarrow 3)$

$\mathrm{W}$ trójkącie ostrokątnym $ABC$ bok $AB$ ma długość $c$, długość boku $BC$ jest równa $a$ oraz

$|\leq ABC|=\beta$. Dwusieczna kąta $ABC$ przecina bok $AC$ trójkąta w punkcie $E.$

Wykaz, $\dot{\mathrm{z}}\mathrm{e}$ długość odcinka $BE$ jest równa $\displaystyle \frac{2ac\cdot\cos\frac{\beta}{2}}{a+c}$

Strona 6 z18

MMA-IR





Zadanie 9. (0-4)

$\mathrm{W}$ czworościanie, którego wszystkie krawędzie mają taką samą długość 6, umieszczono ku1ę

tak, $\dot{\mathrm{z}}\mathrm{e}$ ma ona dokładniejeden punkt wspólny z $\mathrm{k}\mathrm{a}\dot{\mathrm{z}}$ dą ścianą czworościanu. Płaszczyzna $\pi,$

równoległa do podstawy tego czworościanu, dzieli go na dwie bryły: ostrosłup o objętości

równej $\displaystyle \frac{8}{27}$ objętości dzielonego czworościanu i ostrosłup ścięty. Oblicz odległość środka $S$

kuli od płaszczyzny $\pi$, tj. długość najkrótszego spośród odcinków $SP$, gdzie Pjest punktem

płaszczyzny $\pi.$

Odpowiedzí :
\begin{center}
\includegraphics[width=96.012mm,height=17.832mm]{./F2_M_PR_M2017_page6_images/image001.eps}
\end{center}
Wypelnia

egzaminator

Nr zadania

Maks. liczba kt

8.

3

4

Uzyskana liczba pkt

IMA-IR

Strona 7 z18





Zadanie 10. (0-4)

Rozwiąz równanie $\cos 2x+3\cos x=-2$ w przedziale $\langle 0, 2\pi\rangle.$

Odpowiedzí:

Strona 8 z18

M





Zadanie 11. (0-4)

$\mathrm{W}$ pudełku znajduje się 8 piłeczek oznaczonych ko1ejnymi 1iczbami natura1nymi od 1 do 8.

Losujemy jedną piłeczkę, zapisujemy liczbę na niej występującą, a następnie zwracamy

piłeczkę do umy. Tę procedurę wykonujemy jeszcze dwa razy i tym samym otrzymujemy

zapisane trzy liczby. Oblicz prawdopodobieństwo wylosowania takich piłeczek, $\dot{\mathrm{z}}\mathrm{e}$ iloczyn

trzech zapisanych liczb jest podzielny przez 4. 1Vynik podaj w postaci ułamka zwykłego.

Odpowiedzí :
\begin{center}
\includegraphics[width=96.012mm,height=17.832mm]{./F2_M_PR_M2017_page8_images/image001.eps}
\end{center}
Wypelnia

egzaminator

Nr zadania

Maks. liczba kt

10.

4

11.

4

Uzyskana liczba pkt

IMA-IR

Strona 9 z18





Zadanie $ l2.(0-5\rangle$

Wyznacz wszystkie wartości parametru $m$, dla których równanie

$4x^{2}-6mx+(2m+3)(m-3)=0$

ma dwa rózne rozwiązania rzeczywiste $x_{1}$ i $x_{2}$, przy czym $x_{1}<x_{2}$, spełniające warunek

$(4x_{1}-4x_{2}-1)(4x_{1}-4x_{2}+1)<0.$

Strona 10 z18

$\mathrm{M}\mathrm{M}\mathrm{A}_{-}$







$\mathrm{g}_{\mathrm{E}\mathrm{G}\mathrm{Z}\mathrm{A}\mathrm{M}\mathrm{I}\mathrm{N}\mathrm{A}\mathrm{C}\mathrm{Y}\mathrm{J}\mathrm{N}\mathrm{A}}^{\mathrm{C}\mathrm{E}\mathrm{N}\mathrm{T}\mathrm{R}\mathrm{A}\mathrm{L}\mathrm{N}\mathrm{A}}$KOMISJA

Arkusz zawiera informacje

prawnie chronione do momentu

rozpoczęcia egzaminu.

UZUPELNIA ZDAJACY

{\it miejsce}

{\it na naklejkę}
\begin{center}
\includegraphics[width=21.900mm,height=16.104mm]{./F2_M_PR_M2018_page0_images/image001.eps}
\end{center}
KOD
\begin{center}
\includegraphics[width=79.608mm,height=16.104mm]{./F2_M_PR_M2018_page0_images/image002.eps}
\end{center}
PESEL
\begin{center}
\includegraphics[width=193.752mm,height=268.884mm]{./F2_M_PR_M2018_page0_images/image003.eps}
\end{center}
EGZAMIN MATU  LNY

Z MATEMATY

POZIOM ROZSZE ONY

DATA: 9 maja 2018 r.

LICZBA P KTÓW DO UZYS NIA: 50

Instrukcja dla zdającego

1.

2.

3.

4.

5.

6.

Sprawdzí, czy arkusz egzaminacyjny zawiera 18 stron (zadania $1-15$).

Ewentualny brak zgłoś przewodniczącemu zespo nadzorującego

egzamin.

Rozwiązania zadań i odpowiedzi wpisuj w miejscu na to przeznaczonym.

Odpowiedzi do zadań za iętych (l ) zaznacz na karcie odpowiedzi

w części ka przeznaczonej dla zdającego. Zamaluj $\blacksquare$ pola do tego

przeznaczone. Błędne zaznaczenie otocz kółkiem \copyright i zaznacz właściwe.

$\mathrm{W}$ zadaniu 5. wpisz odpowiednie cyf w atki pod treścią zadania.

Pamiętaj, $\dot{\mathrm{z}}\mathrm{e}$ pominięcie argumentacji lub istotnych obliczeń

w rozwiązaniu zadania otwa ego (6-15) $\mathrm{m}\mathrm{o}\dot{\mathrm{z}}\mathrm{e}$ spowodować, $\dot{\mathrm{z}}\mathrm{e}$ za to

rozwiązanie nie otrzymasz pełnej liczby pu tów.

Pisz czytelnie i $\mathrm{u}\dot{\mathrm{z}}$ aj tylko $\mathrm{d}$ gopisu lub pióra z czatnym tuszem lub

atramentem.

7. Nie $\mathrm{u}\dot{\mathrm{z}}$ aj korektora, a błędne zapisy $\mathrm{r}\mathrm{a}\acute{\mathrm{z}}\mathrm{n}\mathrm{i}\mathrm{e}$ prze eśl.

8. Pamiętaj, $\dot{\mathrm{z}}\mathrm{e}$ zapisy w $\mathrm{b}$ dnopisie nie będą oceniane.

9. $\mathrm{M}\mathrm{o}\dot{\mathrm{z}}$ esz korzystać z zesta wzorów matema cznych, cyrkla i linijki oraz

kalkulatora prostego.

10. Na tej stronie oraz na karcie odpowiedzi wpisz swój numer PESEL

i przyklej naklejkę z kodem.

ll. Nie wpisuj $\dot{\mathrm{z}}$ adnych znaków w części przeznaczonej dla egzaminatora.

$\Vert\Vert\Vert\Vert\Vert\Vert\Vert\Vert\Vert\Vert\Vert\Vert\Vert\Vert\Vert\Vert\Vert\Vert\Vert\Vert\Vert\Vert\Vert\Vert|$

$\mathrm{M}\mathrm{M}\mathrm{A}-\mathrm{R}1_{-}1\mathrm{P}-1\mathrm{S}2$

Układ graficzny

\copyright CKE 2015

$1$ :




{\it Wzadaniach od l. do 4. wybierz i zaznacz na karcie odpowiedzi poprawnq odpowiedzí}.

Zadanie $l. (0-1)$

Dane są liczby: $a=\displaystyle \frac{4\sqrt{8}}{2}, b=\displaystyle \frac{1}{2^{4}\sqrt{8}}, c=^{4}\sqrt{8}, d=\displaystyle \frac{2}{4\sqrt{8}}$ oraz $k=2^{-\frac{1}{4}}$. Prawdziwajest równość

A. $k=a$

B. $k=b$

C. $k=c$

D. $k=d$

Zadanie 2. (0-1)

Równanie $||x|-2|=|x|+2$

A. nie ma rozwiązań.

B. ma dokładniejedno rozwiązanie.

C. ma dokładnie dwa rozwiązania.

D. ma dokładnie cztery rozwiązania.

Zadanie 3. $(0-1\rangle$

Wartość wyrazenia 21og510- $\displaystyle \frac{1}{\log_{20}5}$ jest równa

A. $-1$

B. 0

Zadanie 4. (0-1)

Granica $\displaystyle \lim_{x\rightarrow 3^{-}}\frac{-x+2}{x^{2}-5x+6}$ jest równa

A.

$-\infty$

B. $-1$

C. l

D. 2

C. 0

D. $+\infty$

Strona 2 z18

MMA-IR





Odpowied $\acute{\mathrm{z}}$:
\begin{center}
\includegraphics[width=82.044mm,height=17.784mm]{./F2_M_PR_M2018_page10_images/image001.eps}
\end{center}
Wypelnia

egzamÍnator

Nr zadania

Maks. liczba kt

12.

Uzyskana liczba pkt

MMA-IR

Strona ll z18





Zadanie 13. (0-4)

Wyrazy ciągu geometrycznego $(a_{n})$, określonego dla $n\geq 1$, spełniają układ równań

$\left\{\begin{array}{l}
a_{3}+a_{6}=-84\\
a_{4}+a_{7}=168
\end{array}\right.$

Wyznacz liczbę $n$ początkowych wyrazów tego ciągu, których suma $S_{n}$ jest równa 32769.

Strona 12 z18

MMA-IR





Odpowied $\acute{\mathrm{z}}$:
\begin{center}
\includegraphics[width=82.044mm,height=17.784mm]{./F2_M_PR_M2018_page12_images/image001.eps}
\end{center}
Wypelnia

egzamÍnator

Nr zadania

Maks. liczba kt

13.

4

Uzyskana liczba pkt

MMA-IR

Strona 13 z18





Zadanie 14. $(0-6)$

Punkt $A=(7,-1)$ jest wierzchołkiem trójkąta równoramiennego $ABC$, w którym $|AC|=|BC|.$

Obie współrzędne wierzchołka $C$ są liczbami ujemnymi. Okrąg wpisany w trójkąt $ABC$ ma

równanie $x^{2}+y^{2}=10$. Oblicz współrzędne wierzchołków $B\mathrm{i}C$ tego trójkąta.

Strona 14 z18

MMA-IR





Odpowied $\acute{\mathrm{z}}$:
\begin{center}
\includegraphics[width=82.044mm,height=17.832mm]{./F2_M_PR_M2018_page14_images/image001.eps}
\end{center}
Wypelnia

egzaminator

Nr zadania

Maks. liczba kt

14.

Uzyskana liczba pkt

MMA-IR

Strona 15 z18





Zadanie $l5.(0\rightarrow 7)$

Rozpatrujemy wszystkie trapezy równoramienne, w które mozna wpisać okrąg, spełniające

warunek: suma długości dłuzszej podstawy $a$ i wysokości trapezujest równa 2.

a) Wyznacz wszystkie wartości $a$, dla których istnieje trapez o podanych własnościach.

b) Wykaz, $\dot{\mathrm{z}}\mathrm{e}$ obwód $L$ takiego trapezu, jako funkcja długości $a$ dłuzszej podstawy trapezu,

wyraza się wzorem $L(a)=\displaystyle \frac{4a^{2}-8a+8}{a}$

c)

Oblicz tangens kąta ostrego tego spośród rozpatrywanych trapezów, którego obwódjest

najmniejszy.

Strona 16 z18

MMA-IR





Odpowied $\acute{\mathrm{z}}$:
\begin{center}
\includegraphics[width=82.044mm,height=17.784mm]{./F2_M_PR_M2018_page16_images/image001.eps}
\end{center}
Wypelnia

egzamÍnator

Nr zadania

Maks. liczba kt

15.

7

Uzyskana liczba pkt

MMA-IR

Strona 17 z18





{\it BRUDNOPIS} ({\it nie podlega ocenie})

Strona 18 z18

MMA-II





{\it BRUDNOPIS} ({\it nie podlega ocenie})

$\mathrm{A}_{-}1\mathrm{R}$

Strona 3 z18





Zadanie 5. $(0-2\rangle$

Punkt $A=(-5,3)$ jest środkiem symetrii wykresu ffinkcji homograficznej określonej wzorem

$f(x)=\displaystyle \frac{ax+7}{x+d}$, gdy $x\neq-d$. Oblicz iloraz $\displaystyle \frac{d}{a}$

$\mathrm{W}$ ponizsze kratki wpisz kolejno cyfrę jedności i pierwsze dwie cyfry po przecinku

nieskończonego rozwinięcia dziesiętnego otrzymanego wyniku.
\begin{center}
\includegraphics[width=22.500mm,height=10.872mm]{./F2_M_PR_M2018_page3_images/image001.eps}
\end{center}
{\it BRUDNOPIS} ({\it nie podlega ocenie})

Zadanie 6. $(0-3\rangle$

Styczna do paraboli o równaniu $y=\sqrt{3}x^{2}-1$ w punkcie $P=(x_{0},y_{0})$ jest nachylona do osi $ox$

pod kątem $30^{\mathrm{o}}$. Oblicz współrzędne punktu $P.$

Odpowied $\acute{\mathrm{z}}$:

Strona 4 z18

MMA-IR





Zadanie 7. (0-3)

Trójkąt $ABC$ jest ostrokątny oraz $|AC|>|BC|$. Dwusieczna $d_{c}$ kąta $ACB$ przecina bok $AB$

w punkcie $K$. Punkt $L$ jest obrazem punktu $K$ w symetrii osiowej względem dwusiecznej $d_{A}$

kąta $BAC$, punkt Mjest obrazem punktu $L$ w symetrii osiowej względem dwusiecznej $d_{c}$ kąta

$ACB$, a punkt $N$ jest obrazem punktu $M$ w symetrii osiowej względem dwusiecznej $d_{B}$ kąta

$ABC$ (zobacz rysunek).
\begin{center}
\includegraphics[width=88.392mm,height=84.072mm]{./F2_M_PR_M2018_page4_images/image001.eps}
\end{center}
{\it C}

{\it L}

{\it M}

{\it A  K N  B}

Udowodnij, $\dot{\mathrm{z}}\mathrm{e}$ na czworokącie KNML mozna opisać okrąg.
\begin{center}
\includegraphics[width=109.980mm,height=17.832mm]{./F2_M_PR_M2018_page4_images/image002.eps}
\end{center}
Wypelnia

egzaminator

Nr zadania

Maks. lÍczba kt

5.

2

3

7.

3

Uzyskana liczba pkt

MMA-IR

Strona 5 z18





Zadanie S. (0-3)

Udowodnij, $\dot{\mathrm{z}}\mathrm{e}$ dla $\mathrm{k}\mathrm{a}\dot{\mathrm{z}}$ dej liczby całkowitej $k$ i dla $\mathrm{k}\mathrm{a}\dot{\mathrm{z}}$ dej liczby całkowitej $m$ liczba $k^{3}m-km^{3}$

jest podzielna przez 6.

Strona 6 z18

MMA-IR





Zadanie $g. (0-4)$

$\mathrm{Z}$ liczb ośmioelementowego zbioru $Z=\{1$, 2, 3, 4, 5, 6, 7, 9$\}$ tworzymy ośmiowyrazowy ciąg,

którego wyrazy się nie powtarzają. Oblicz prawdopodobieństwo zdarzenia polegającego na

tym, $\dot{\mathrm{z}}\mathrm{e}\dot{\mathrm{z}}$ adne dwie liczby parzyste nie są sąsiednimi wyrazami utworzonego ciągu. Wynik

przedstaw w postaci ułamka zwykłego nieskracalnego.

Odpowied $\acute{\mathrm{z}}$:
\begin{center}
\includegraphics[width=96.012mm,height=17.784mm]{./F2_M_PR_M2018_page6_images/image001.eps}
\end{center}
Wypelnia

egzaminator

Nr zadania

Maks. liczba kt

8.

3

4

Uzyskana liczba pkt

MMA-IR

Strona 7 z18





Zadanie 10.(0-4)

Objętość stozka ściętego (przedstawionego na rysunku) mozna obliczyć ze wzoru

$V=\displaystyle \frac{1}{3}\pi H(r^{2}+rR+R^{2})$, gdzie $r\mathrm{i}R$ są promieniami podstaw $(r<R)$, a $H$ jest wysokością

bryły. Danyjest stozek ścięty, którego wysokośćjest równa 10, objętość $ 840\pi$, a $r=6$. Oblicz

cosinus kąta nachylenia przekątnej przekroju osiowego tej bryły do jednej zjej podstaw.

Odpowiedzí :

Strona 8 z18

MMA-IR





Zadanie $1l. (0-4)$

Rozwiąz równanie $\sin 6x+\cos 3x=2\sin 3x+1$ w przedziale $\langle 0, \pi\rangle.$

Odpowiedzí :
\begin{center}
\includegraphics[width=96.012mm,height=17.832mm]{./F2_M_PR_M2018_page8_images/image001.eps}
\end{center}
Wypelnia

egzaminator

Nr zadania

Maks. liczba kt

10.

4

11.

4

Uzyskana liczba pkt

MMA-IR

Strona 9 z18





Zadanie 12. $(0-6)$

Wyznacz wszystkie wartości parametru $m$, dla których równanie $x^{2}+(m+1)x-m^{2}+1=0$ ma

dwa rozwiązania rzeczywiste $x_{1}$ i $x_{2}(x_{1}\neq x_{2})$, spełniające warunek $x_{1}^{3}+x_{2}^{3}>-7x_{1}x_{2}.$

Strona 10 z18

MMA-IR







$\mathrm{g}_{\mathrm{E}\mathrm{G}\mathrm{Z}\mathrm{A}\mathrm{M}\mathrm{I}\mathrm{N}\mathrm{A}\mathrm{C}\mathrm{Y}\mathrm{J}\mathrm{N}\mathrm{A}}^{\mathrm{C}\mathrm{E}\mathrm{N}\mathrm{T}\mathrm{R}\mathrm{A}\mathrm{L}\mathrm{N}\mathrm{A}}$KOMISJA

Arkusz zawiera informacje

prawnie chronione do momentu

rozpoczęcia egzaminu.

UZUPELNIA ZDAJACY

{\it miejsce}

{\it na naklejkę}
\begin{center}
\includegraphics[width=21.900mm,height=16.104mm]{./F2_M_PR_M2019_page0_images/image001.eps}
\end{center}
KOD
\begin{center}
\includegraphics[width=79.608mm,height=16.104mm]{./F2_M_PR_M2019_page0_images/image002.eps}
\end{center}
PESEL
\begin{center}
\includegraphics[width=196.440mm,height=246.984mm]{./F2_M_PR_M2019_page0_images/image003.eps}
\end{center}
EGZAMIN MATU  LNY

Z MATEMATY

POZIOM ROZSZE ONY

DATA: 9 maja 2019 r.

LICZBA P KTÓW DO UZYS NIA: 50

Instrukcja dla zdającego

1.

2.

3.

4.

5.

6.

Sprawdzí, czy arkusz egzaminacyjny zawiera 22 strony (zadania $1-15$).

Ewentualny brak zgłoś przewodniczącemu zespo nadzorującego

egzamin.

Rozwiązania zadań i odpowiedzi wpisuj w miejscu na to przeznaczonym.

Odpowiedzi do zadań za iętych (l ) zaznacz na karcie odpowiedzi

w części ka przeznaczonej dla zdającego. Zamaluj $\blacksquare$ pola do tego

przeznaczone. Błędne zaznaczenie otocz kółkiem \copyright i zaznacz właściwe.

$\mathrm{W}$ zadaniu 5. wpisz odpowiednie cyf w atki pod treścią zadania.

Pamiętaj, $\dot{\mathrm{z}}\mathrm{e}$ pominięcie argumentacji lub istotnych obliczeń

w rozwiązaniu zadania otwa ego (6-15) $\mathrm{m}\mathrm{o}\dot{\mathrm{z}}\mathrm{e}$ spowodować, $\dot{\mathrm{z}}\mathrm{e}$ za to

rozwiązanie nie otrzymasz pełnej liczby pu tów.

Pisz czytelnie i $\mathrm{u}\dot{\mathrm{z}}$ aj tylko $\mathrm{d}$ gopisu lub pióra z czatnym tuszem lub

atramentem.

7. Nie $\mathrm{u}\dot{\mathrm{z}}$ aj korektora, a błędne zapisy $\mathrm{r}\mathrm{a}\acute{\mathrm{z}}\mathrm{n}\mathrm{i}\mathrm{e}$ prze eśl.

8. Pamiętaj, $\dot{\mathrm{z}}\mathrm{e}$ zapisy w brudnopisie nie będą oceniane.

9. $\mathrm{M}\mathrm{o}\dot{\mathrm{z}}$ esz korzystać z zesta wzorów matema cznych, cyrkla i linijki oraz

kalkulatora prostego.

10. Na tej stronie oraz na karcie odpowiedzi wpisz swój numer PESEL

i przyklej naklejkę z kodem.

ll. Nie wpisuj $\dot{\mathrm{z}}$ adnych znaków w części przeznaczonej dla egzaminatora.

$\Vert\Vert\Vert\Vert\Vert\Vert\Vert\Vert\Vert\Vert\Vert\Vert\Vert\Vert\Vert\Vert\Vert\Vert\Vert\Vert\Vert\Vert\Vert\Vert|$

$\mathrm{M}\mathrm{M}\mathrm{A}-\mathrm{R}1_{-}1\mathrm{P}-192$

Układ graficzny

\copyright CKE 2015




{\it Wkazdym z zadań od l. do 4. wybierz i zaznacz na karcie odpowiedzi poprawnq odpowiedzí}.

Zadanie 1. (0-1)

Dla dowolnych liczb $x>0, x\neq 1, y>0, y\neq 1$ wartość wyrazenia $(\log_{\frac{1}{x}}y)\cdot(\log_{y}\perp x)$ jest

równa

$\underline{1}$

A. $x\cdot y$ B. C. $-1$ D. l

$x\cdot y$

Zadanie 2. (0-1)

Liczba $\cos^{2}105^{\mathrm{o}}-\sin^{2}105^{\mathrm{o}}$ jest równa

A.

- -$\sqrt{}$23

B.

- -21

C.

-21

D.

-$\sqrt{}$23

Zadanie 3. (0-1)

Na rysunku przedstawiono fragment wykresu ffinkcji $y=f(x)$, który jest złozony z dwóch

półprostych AD $\mathrm{i}$ CE oraz dwóch odcinków AB $\mathrm{i} BC$, gdzie $A=(-1,0), B=(1,2),$

$C=(3,0), D=(-4,3), E=(6,3).$
\begin{center}
\includegraphics[width=91.284mm,height=62.940mm]{./F2_M_PR_M2019_page1_images/image001.eps}
\end{center}
{\it y}

5

{\it D}

4

3

2

$E_{1}$

{\it B}

{\it x}

$-5$ -$4  -3  -2A$ -$1$  0  1 2  $3C^{4}$  5 6  7

$-1$

Wzór funkcji $f$ to

A. $f(x)=|x+1|+|x-1|$

B. $f(x)=\Vert x-1|-2|$

C. $f(x)=\Vert x-1|+2|$

D. $f(x)=|x-1|+2$

Zadanie 4. (0-1)

Zdarzenia losowe $A \mathrm{i} B$ zawarte w $\Omega$

zdarzenia $B'$, przeciwnego do zdarzenia $B,$

warunkowe $P(A|B)=\displaystyle \frac{1}{5}$. Wynika stąd, $\dot{\mathrm{z}}\mathrm{e}$

A. $P(A\displaystyle \cap B)=\frac{1}{20}$ B. $P(A\displaystyle \cap B)=\frac{4}{15}$

są takie, ze prawdopodobieństwo $P(B')$

est równe $\displaystyle \frac{1}{4}$ Ponadto prawdopodobieństwo

C. $P(A\displaystyle \cap B)=\frac{3}{20}$ D. $P(A\displaystyle \cap B)=\frac{4}{5}$

Strona 2 z22

MMA-IR





Odpowiedzí :
\begin{center}
\includegraphics[width=82.044mm,height=17.784mm]{./F2_M_PR_M2019_page10_images/image001.eps}
\end{center}
Wypelnia

egzaminator

Nr zadania

Maks. liczba kt

10.

4

Uzyskana liczba pkt

MMA-IR

Strona ll z22





Zadanie 11. (0-6)

Dane są okręgi o równaniach $x^{2}+y^{2}-12x-8y+43=0 \mathrm{i} x^{2}+y^{2}-2ax+4y+a^{2}-77=0.$

Wyznacz wszystkie wartości parametru $a$, dla których te okręgi mają dokładnie jeden punkt

wspólny. Rozwaz wszystkie przypadki.

Strona 12 z22

MMA-IR





Odpowiedzí :
\begin{center}
\includegraphics[width=82.044mm,height=17.784mm]{./F2_M_PR_M2019_page12_images/image001.eps}
\end{center}
Wypelnia

egzaminator

Nr zadania

Maks. liczba kt

11.

Uzyskana liczba pkt

MMA-IR

Strona 13 z22





Zadanie 12. (0-6)

Trzywyrazowy ciąg $(a,b,c)$ o wyrazach dodatnich jest arytmetyczny, natomiast ciąg

$(\displaystyle \frac{1}{a},\frac{2}{3b},\frac{1}{2a+2b+c})$ jest geometryczny. Oblicz iloraz ciągu geometrycznego.

Strona 14 z22

MMA-IR





Odpowiedzí :
\begin{center}
\includegraphics[width=82.044mm,height=17.784mm]{./F2_M_PR_M2019_page14_images/image001.eps}
\end{center}
Wypelnia

egzaminator

Nr zadania

Maks. liczba kt

12.

Uzyskana liczba pkt

MMA-IR

Strona 15 z22





Zadanie 13. (0-6)

Wielomian określony

wzorem

$W(x)=2x^{3}+(m^{3}+2)x^{2}-11x-2(2m+1)$ jest podzielny

przez dwumian $(x-2)$ oraz przy dzieleniu przez dwumian $(x+1)$ daje resztę 6. Ob1icz $m$

i dla wyznaczonej wartości $m$ rozwiąz nierówność $W(x)\leq 0.$

Strona 16 z22

MMA-IR





Odpowied $\acute{\mathrm{z}}$:
\begin{center}
\includegraphics[width=82.044mm,height=17.832mm]{./F2_M_PR_M2019_page16_images/image001.eps}
\end{center}
Wypelnia

egzaminator

Nr zadania

Maks. liczba kt

13.

Uzyskana liczba pkt

MMA-IR

Strona 17 z22





Zadanie 14. (0-4)

Rozwiąz równanie $(\displaystyle \cos x)[\sin(x-\frac{\pi}{3})+\sin(x+\frac{\pi}{3})]=\frac{1}{2}\sin x.$

Strona 18 z22

$\mathrm{M}\mathrm{M}\mathrm{A}_{-}1l$





Odpowiedzí :
\begin{center}
\includegraphics[width=82.044mm,height=17.784mm]{./F2_M_PR_M2019_page18_images/image001.eps}
\end{center}
Wypelnia

egzaminator

Nr zadania

Maks. liczba kt

14.

4

Uzyskana liczba pkt

MMA-IR

Strona 19 z22





Zadanie 15. (0-7)

Rozwazmy wszystkie graniastosłupy prawidłowe trójkątne o objętości $V=2$. Wyznacz

długości krawędzi tego z rozwazanych graniastosłupów, którego pole powierzchni całkowitej

jest najmniejsze. Oblicz to najmniejsze pole.

Strona 20 z22

MMA-IR





BRUDNOPIS

1R

Strona 3 z22





Odpowiedzí :
\begin{center}
\includegraphics[width=82.044mm,height=17.784mm]{./F2_M_PR_M2019_page20_images/image001.eps}
\end{center}
Wypelnia

egzaminator

Nr zadania

Maks. liczba kt

15.

7

Uzyskana liczba pkt

MMA-IR

Strona 21 z22





{\it BRUDNOPIS} ({\it nie podlega ocenie})

Strona 22 z22

$\mathrm{M}\mathrm{M}\mathrm{A}_{-}1l$





Zadanie 5. (0-2)

Oblicz granicę

$\displaystyle \lim_{n\rightarrow\infty}(\frac{9n^{3}+11n^{2}}{7n^{3}+5n^{2}+3n+1}-\frac{n^{2}}{3n^{2}+1})$

Wpisz w ponizsze kratki-od lewej do prawej- trzy kolejne cyfry po przecinku rozwinięcia

dziesiętnego otrzymanego wyniku.
\begin{center}
\includegraphics[width=22.500mm,height=10.872mm]{./F2_M_PR_M2019_page3_images/image001.eps}
\end{center}
Strona 4 z22

MMA-IR





Zadanie 6. (0-3)

Rozwazamy wszystkie liczby naturalne pięciocyfrowe zapisane przy uzyciu cyfr 1, 3, 5, 7, 9,

bez powtarzaniajakiejkolwiek cyfry. Oblicz sumę wszystkich takich liczb.

Odpowiedzí:
\begin{center}
\includegraphics[width=96.012mm,height=17.784mm]{./F2_M_PR_M2019_page4_images/image001.eps}
\end{center}
Wypelnia

egzamÍnator

Nr zadania

Maks. liczba kt

5.

2

3

Uzyskana liczba pkt

MMA-IR

Strona 5 z22





Zadanie 7. (0-2)

Punkt $P=(10,$ 2429$)$ lezy na paraboli o równaniu $y=2x^{2}+x+2219$. Prosta o równaniu

kierunkowym $y=ax+b$ jest styczna do tej paraboli w punkcie $P$. Oblicz współczynnik $b.$

Odpowied $\acute{\mathrm{z}}$:

Strona 6 z22

MMA-IR





Zadanie 8. (0-3)

Udowodnij, $\dot{\mathrm{z}}\mathrm{e}$ dla dowolnych dodatnich liczb rzeczywistych $x\mathrm{i}y$, takich $\dot{\mathrm{z}}\mathrm{e}x<y$, i dowolnej

dodatniej liczby rzeczywistej $a$, prawdziwajest nierówność $\displaystyle \frac{x+a}{y+a}+\frac{y}{x}>2.$
\begin{center}
\includegraphics[width=96.012mm,height=17.832mm]{./F2_M_PR_M2019_page6_images/image001.eps}
\end{center}
Wypelnia

egzaminator

Nr zadania

Maks. liczba kt

7.

2

8.

3

Uzyskana liczba pkt

MMA-IR

Strona 7 z22





Zadanie 9. (0-3)

Dany jest trójkąt równoramienny $ABC$, w którym $|AC|=|BC|$. Na ramieniu $AC$ tego trójkąta

wybrano punkt $M(M\neq A\mathrm{i}M\neq C)$, a na ramieniu $BC$ wybrano punkt $N$, w taki sposób, $\dot{\mathrm{z}}\mathrm{e}$

$|AM|=|CN|$. Przez punkty $M\mathrm{i}N$ poprowadzono proste prostopadłe do podstawy $AB$ tego

trójkąta, które wyznaczają na niej punkty $S\mathrm{i}T$. Udowodnij, $\displaystyle \dot{\mathrm{z}}\mathrm{e}|ST|=\frac{1}{2}|AB|.$

Strona 8 z22

MMA-IR




\begin{center}
\includegraphics[width=82.044mm,height=17.784mm]{./F2_M_PR_M2019_page8_images/image001.eps}
\end{center}
Wypelnia

egzaminator

Nr zadania

Maks. liczba kt

3

Uzyskana liczba pkt

1R

Strona 9 z22





Zadanie 10. (0-4)

Punkt $D$ lezy na boku $AB$ trójkąta $ABC$ oraz $|AC|=16, |AD|=6, |CD|=14 \mathrm{i} |BC|=|BD|.$

Oblicz obwód trójkąta $ABC.$

Strona 10 z22

MMA-IR







$\mathrm{g}_{\mathrm{E}\mathrm{G}\mathrm{Z}\mathrm{A}\mathrm{M}\mathrm{I}\mathrm{N}\mathrm{A}\mathrm{C}\mathrm{Y}\mathrm{J}\mathrm{N}\mathrm{A}}^{\mathrm{C}\mathrm{E}\mathrm{N}\mathrm{T}\mathrm{R}\mathrm{A}\mathrm{L}\mathrm{N}\mathrm{A}}$KOMISJA

Arkusz zawiera informacje

prawnie chronione do momentu

rozpoczęcia egzaminu.

WYPELNIA ZDAJACY

{\it miejsce}

{\it na naklejkę}
\begin{center}
\includegraphics[width=21.900mm,height=16.104mm]{./F2_M_PR_M2020_page0_images/image001.eps}
\end{center}
KOD
\begin{center}
\includegraphics[width=79.608mm,height=16.104mm]{./F2_M_PR_M2020_page0_images/image002.eps}
\end{center}
PESEL
\begin{center}
\includegraphics[width=193.548mm,height=268.584mm]{./F2_M_PR_M2020_page0_images/image003.eps}
\end{center}
EGZAMIN MATU  LNY

Z MATEMATY

POZIOM ROZSZE ONY

DATA: 7 maja 2020 r.

CZAS P CY:180 minut

LICZBA P KTÓW DO UZYS NIA: 50

Instrukcja dla zdającego

1.

2.

3.

Sprawdzí, czy arkusz egzaminacyjny zawiera 22 strony (zadania $1-15$).

Ewentualny brak zgłoś przewodniczącemu zespo nadzorującego

egzamin.

Rozwiązania zadań i odpowiedzi wpisuj w miejscu na to przeznaczonym.

Odpowiedzi do zadań za iętych (l ) zaznacz na karcie odpowiedzi

w części ka przeznaczonej dla zdającego. Zamaluj $\blacksquare$ pola do tego

4.

5.

6.

przeznaczone. Błędne zaznaczenie otocz kółkiem i zaznacz właściwe.

$\mathrm{W}$ zadaniu 5. wpisz odpowiednie cyf w atki pod treścią zadania.

Pamiętaj, $\dot{\mathrm{z}}\mathrm{e}$ pominięcie argumentacji lub istotnych obliczeń

w rozwiązaniu zadania o a ego (6-15) $\mathrm{m}\mathrm{o}\dot{\mathrm{z}}\mathrm{e}$ spowodować, $\dot{\mathrm{z}}\mathrm{e}$ za to

rozwiązanie nie otr masz pełnej liczby pu tów.

Pisz czytelnie i $\mathrm{u}\dot{\mathrm{z}}$ aj tylko $\mathrm{d}$ gopisu lub pióra z czatnym tuszem lub

atramentem.

7. Nie $\mathrm{u}\dot{\mathrm{z}}$ aj korektora, a błędne zapisy $\mathrm{r}\mathrm{a}\acute{\mathrm{z}}\mathrm{n}\mathrm{i}\mathrm{e}$ prze eśl.

8. Pamiętaj, $\dot{\mathrm{z}}\mathrm{e}$ zapisy w brudnopisie nie będą oceniane.

9. $\mathrm{M}\mathrm{o}\dot{\mathrm{z}}$ esz korzystać z zesta wzorów matema cznych, cyrkla i linijki oraz

kalkulatora prostego.

10. Na tej stronie oraz na karcie odpowiedzi wpisz swój numer PESEL

i przyklej naklejkę z kodem.

ll. Nie wpisuj $\dot{\mathrm{z}}$ adnych znaków w części przeznaczonej dla egzaminatora.

$\Vert\Vert\Vert\Vert\Vert\Vert\Vert\Vert\Vert\Vert\Vert\Vert\Vert\Vert\Vert\Vert\Vert\Vert\Vert\Vert\Vert\Vert\Vert\Vert|$

$\mathrm{M}\mathrm{M}\mathrm{A}-\mathrm{R}1_{-}1\mathrm{P}-202$

Układ graficzny

\copyright CKE 2015

$| 1$




{\it Wkazdym z zadań od l. do 4. wybierz i zaznacz na karcie odpowiedzi poprawnq odpowiedzí}.

Zadanie 1. (0-1)

Wielomian $W$ określony wzorem $W(x)=x^{2019}-3x^{2000}+2x+6$

A. jest podzielny przez $(x-1)$ i z dzielenia przez $(x+1)$ daje resztę równą 6.

B. jest podzielny przez $(x+1)$ i z dzielenia przez $(x-1)$ daje resztę równą 6.

C. jest podzielny przez $(x-1)$ ijest podzielny przez $(x+1).$

D. niejest podzielny ani przez $(x-1)$, ani przez $(x+1).$

Zadanie 2. (0-1)

Ciąg $(a_{n})$ jest określony wzorem $a_{n}=\displaystyle \frac{3n^{2}+7n-5}{11-5n+5n^{2}}$ dla $\mathrm{k}\mathrm{a}\dot{\mathrm{z}}$ dej liczby naturalnej $n\geq 1.$

Granica tego ciągu jest równa

A. 3

B.

-51

C.

-53

D.

$-\displaystyle \frac{5}{11}$

Zadanie 3. (0-1)

Mamy dwie urny. $\mathrm{W}$ pierwszej są 3 ku1e białe i 7 ku1 czamych, w drugiej jestjedna ku1a biała

$\mathrm{i}9$ kul czarnych. Rzucamy symetryczną sześcienną kostką do gry, która na $\mathrm{k}\mathrm{a}\dot{\mathrm{z}}$ dej ściance ma

inną liczbę oczek, odjednego oczka do sześciu oczek. Jeśli w wyniku rzutu otrzymamy ściankę

z jednym oczkiem, to losujemy jedną kulę z pierwszej umy, w przeciwnym przypadku-

losujemy jedną kulę z drugiej umy. Wtedy prawdopodobieństwo wylosowania kuli białej jest

równe

A.

$\displaystyle \frac{2}{15}$

B.

-51

C.

-45

D.

$\displaystyle \frac{13}{15}$

Zadanie 4. (0-1)

Po przekształceniu wyrazenia algebraicznego

$ax^{4}+bx^{3}y+cx^{2}y^{2}+dxy^{3}+ey^{4}$ współczynnik $c$ jest równy

$(\sqrt{2}\sqrt{3})^{4}$

do postaci

A. 6

B. 36

C. $8\sqrt{6}$

D. $12\sqrt{6}$

Strona 2 z22

MMA-IR





Odpowiedzí:
\begin{center}
\includegraphics[width=82.044mm,height=17.832mm]{./F2_M_PR_M2020_page10_images/image001.eps}
\end{center}
Wypelnia

egzaminator

Nr zadania

Maks. liczba kt

10.

5

Uzyskana liczba pkt

MMA-IR

Strona ll z22





Zadanie 11. (0-4)

Dane jest równanie kwadratowe $x^{2}-(3m+2)x+2m^{2}+7m-15=0$ z niewiadomą $x$. Wyznacz

wszystkie wartości parametru $m$, dla których rózne rozwiązania

i spełniają warunek

$2x_{1}^{2}+5x_{1}x_{2}+2x_{2}^{2}=2.$

$x_{1}$ i $x_{2}$ tego równania istnieją

Strona 12 z22

MMA-IR





Odpowiedzí:
\begin{center}
\includegraphics[width=82.044mm,height=17.784mm]{./F2_M_PR_M2020_page12_images/image001.eps}
\end{center}
Wypelnia

egzamÍnator

Nr zadania

Maks. liczba kt

11.

4

Uzyskana liczba pkt

MMA-IR

Strona 13 z22





Zadanie 12. (0-5)

Prosta o równaniu

$x+y-10=0$ przecina

okrąg o

równaniu $x^{2}+y^{2}-8x-6y+8=0$

wpunktach $K\mathrm{i}L$. Punkt $S$ jest środkiem cięciwy $KL$. Wyznacz równanie obrazu tego okręgu

wjednokładności o środku $S$ i skali $k=-3.$

Strona 14 z22

MMA-IR





Odpowiedzí:
\begin{center}
\includegraphics[width=82.044mm,height=17.784mm]{./F2_M_PR_M2020_page14_images/image001.eps}
\end{center}
Wypelnia

egzamÍnator

Nr zadania

Maks. liczba kt

12.

5

Uzyskana liczba pkt

MMA-IR

Strona 15 z22





Zadanie 13. (0-4)

Oblicz, ilejest wszystkich siedmiocyfrowych liczb naturalnych, w których zapisie dziesiętnym

występują dokładnie trzy cyfry l i dokładnie dwie cyfry 2.

Strona 16 z22

MMA-IR





Odpowiedzí:
\begin{center}
\includegraphics[width=82.044mm,height=17.784mm]{./F2_M_PR_M2020_page16_images/image001.eps}
\end{center}
Wypelnia

egzamÍnator

Nr zadania

Maks. liczba kt

13.

4

Uzyskana liczba pkt

MMA-IR

Strona 17 z22





Zadanie 14. (0-6)

Podstawą ostrosłupa czworokątnego ABCDS jest trapez ABCD (AB $||$ CD). Ramiona tego

trapezu mają długości $|AD|=10 \mathrm{i}|BC|=16$, a miara kąta $ABC$ jest równa $30^{\mathrm{o}}. \mathrm{K}\mathrm{a}\dot{\mathrm{z}}$ da ściana

boczna tego ostrosłupa tworzy z płaszczyzną podstawy kąt $\alpha$, taki, $\dot{\mathrm{z}}\mathrm{e}$ tg $\displaystyle \alpha=\frac{9}{2}$. Oblicz objętość

tego ostrosłupa.

Strona 18 z22

MMA-IR





Odpowiedzí:
\begin{center}
\includegraphics[width=82.044mm,height=17.784mm]{./F2_M_PR_M2020_page18_images/image001.eps}
\end{center}
Wypelnia

egzamÍnator

Nr zadania

Maks. liczba kt

14.

Uzyskana liczba pkt

MMA-IR

Strona 19 z22





Zadanie 15. (0-7)

Nalez$\mathrm{y}$ zaprojektować wymiary prostokątnego ekranu smartfona, tak aby odległości tego

ekranu od krótszych brzegów smartfona były równe 0,5 cm $\mathrm{k}\mathrm{a}\dot{\mathrm{z}}$ da, a odległości tego ekranu

od dłuzszych brzegów smartfona były równe 0,3 cm $\mathrm{k}\mathrm{a}\dot{\mathrm{z}}$ da (zobacz rysunek- ekran zaznaczono

kolorem szarym). Sam ekran ma mieć powierzchnię 60 $\mathrm{c}\mathrm{m}^{2}$. Wyznacz takie wymiary ekranu

smartfona, przy których powierzchnia ekranu wraz z obramowaniemjest najmniejsza.
\begin{center}
\includegraphics[width=103.584mm,height=116.388mm]{./F2_M_PR_M2020_page19_images/image001.eps}
\end{center}
0,5 cm

e an

0,5 cm

obramowanie

brzeg

Strona 20 z22

MMA-IR





BRUDNOPIS

1R

Strona 3 z22





Odpowiedzí:
\begin{center}
\includegraphics[width=82.044mm,height=17.784mm]{./F2_M_PR_M2020_page20_images/image001.eps}
\end{center}
Wypelnia

egzamÍnator

Nr zadania

Maks. liczba kt

15.

7

Uzyskana liczba pkt

MMA-IR

Strona 21 z22





{\it BRUDNOPIS} ({\it nie podlega ocenie})

Strona 22 z22

$\mathrm{M}\mathrm{M}\mathrm{A}_{-}1l$















Zadanie 5. (0-2)

$\mathrm{W}$ trójkącie $ABC$ bok $AB$ jest 3 razy dłuzszy od boku $AC$, a długość boku $BC$ stanowi $\displaystyle \frac{4}{5}$

długości boku $AB$. Oblicz cosinus najmniejszego kąta trójkąta $ABC.$

$\mathrm{W}$ kratki ponizej wpisz kolejno-od lewej do prawej -pierwszą, drugą oraz trzecią cyfrę

po przecinku nieskończonego rozwinięcia dziesiętnego otrzymanego wyniku.
\begin{center}
\includegraphics[width=22.500mm,height=10.872mm]{./F2_M_PR_M2020_page3_images/image001.eps}
\end{center}
Strona 4 z22

MMA-IR





Zadanie 6. (0-3)

Wyznacz wszystkie wartości parametru $a$, dla których równanie $|x-5|=(a-1)^{2}-4$ ma dwa

rózne rozwiązania dodatnie.

Odpowiedzí:
\begin{center}
\includegraphics[width=96.012mm,height=17.832mm]{./F2_M_PR_M2020_page4_images/image001.eps}
\end{center}
Wypelnia

egzaminator

Nr zadania

Maks. liczba kt

5.

2

3

Uzyskana liczba pkt

MMA-IR

Strona 5 z22





Zadanie 7. (0-3)

Dany jest trójkąt równoramienny $ABC$, w którym $|AC|=|BC|=6$, a punkt $D$ jest środkiem

podstawy $AB$. Okrąg o środku $D$ jest styczny do prostej $AC$ w punkcie $M$. Punkt $K$ lezy na boku

$AC$, punkt $L$ lezy na boku $BC$, odcinek $KL$ jest styczny do rozwazanego okręgu oraz $|KC|=|LC|=2$

(zobacz rysunek).
\begin{center}
\includegraphics[width=101.448mm,height=64.260mm]{./F2_M_PR_M2020_page5_images/image001.eps}
\end{center}
{\it C}

{\it K  L}

{\it M}

{\it A  D  B}

Wykaz, $\displaystyle \dot{\mathrm{z}}\mathrm{e}\frac{|AM|}{|MC|}=\frac{4}{5}$

Strona 6 z22

MMA-IR




\begin{center}
\includegraphics[width=82.044mm,height=17.784mm]{./F2_M_PR_M2020_page6_images/image001.eps}
\end{center}
Wypelnia

egzamÍnator

Nr zadania

Maks. liczba kt

7.

3

Uzyskana liczba pkt

1R

Strona 7 z22





Zadanie 8. (0-3)

Liczby dodatnie $a\mathrm{i}b$ spełniają równość $a^{2}+2a=4b^{2}+4b$. Wykaz, $\dot{\mathrm{z}}\mathrm{e}a=2b.$

Strona 8 z22

MMA-I]





Zadanie 9. (0-4)

Rozwiąz równanie 3 $\cos 2x+10\cos^{2}x=24\sin x-3$ dla $x\in\langle 0, 2\pi\rangle.$

Odpowiedzí:
\begin{center}
\includegraphics[width=96.012mm,height=17.784mm]{./F2_M_PR_M2020_page8_images/image001.eps}
\end{center}
WypelnÍa

egzaminator

Nr zadania

Maks. lÍczba kt

8.

3

4

Uzyskana liczba pkt

MMA-IR

Strona 9 z22





Zadanie 10. (0-5)

$\mathrm{W}$ trzywyrazowym ciągu geometrycznym $(a_{1},a_{2},a_{3})$ spełnionajest równość $a_{1}+a_{2}+a_{3}=\displaystyle \frac{21}{4}.$

Wyrazy $a_{1}, a_{2}, a_{3}$ są- odpowiednio-czwartym, drugim i pierwszym wyrazem rosnącego

ciągu arytmetycznego. Oblicz $a_{1}.$

Strona 10 z22

MMA-IR







CENTRALNA

KOMISJA

EGZAMINACYJNA

Arkusz zawiera informacje prawnie chronione

do momentu rozpoczecia egzaminu.

KOD

WYPELNIA ZDAJACY

PESEL

{\it Miejsce na naklejke}.

{\it Sprawdz}', {\it czy kod na naklejce to}

E-100.
\begin{center}
\includegraphics[width=21.900mm,height=10.212mm]{./F2_M_PR_M2021_page0_images/image001.eps}

\includegraphics[width=79.608mm,height=10.212mm]{./F2_M_PR_M2021_page0_images/image002.eps}
\end{center}
$J\mathrm{e}\dot{\mathrm{z}}$ {\it eli tak}- {\it przyklej naklejkq}.

{\it lezeli nie}- {\it zgtoś to nauczycielowi}.

EGZAMIN MATURALNY Z MATEMATYKI

POZIOM ROZSZERZONY

WYPELNIA ZESP6L NADZORUJACY

DAT$\mathrm{A}^{\cdot}$ ll maja 2021 $\mathrm{r}.$

GODZINA ROZPOCZeClA: 9:00

CZAS PRACY: $180 \displaystyle \min \mathrm{u}\mathrm{t}$

LICZBA PUNKTÓW DO UZYSKANIA 50

Uprawnienia zdajacego do:

\fbox{} dostosowania zasad oceniania

\fbox{} dostosowania w zw. z dyskalkuliq

\fbox{} nieprzenoszenia zaznaczeń na kart9.

$\Vert\Vert\Vert\Vert\Vert\Vert\Vert\Vert\Vert\Vert\Vert\Vert\Vert\Vert\Vert\Vert\Vert\Vert\Vert\Vert\Vert\Vert\Vert\Vert\Vert\Vert\Vert\Vert\Vert\Vert|$

EMAP-R0-100-2105

lnstrukcja dla zdajqcego

l. Sprawdz', czy arkusz egzaminacyjny zawiera 27 stron (zadania $1-15$).

Ewentualny brak zgloś przewodniczqcemu zespolu nadzorujqcego egzamin.

2. Na tej stronie oraz na karcie odpowiedzi wpisz swój numer PESEL i przyklej naklejk9

z kodem.

3. Nie wpisuj $\dot{\mathrm{z}}$ adnych znaków w cześci przeznaczonej dla egzaminatora.

4. Rozwiqzania zadań i odpowiedzi wpisuj w miejscu na to przeznaczonym.

5. Odpowiedzi do zadań zamknietych $(1-4)$ zaznacz na karcie odpowiedzi w części karty

przeznaczonej dla zdajqcego. Zamaluj $\blacksquare$ pola do tego przeznaczone. $\mathrm{B}_{9}\mathrm{d}\mathrm{n}\mathrm{e}$

zaznaczenie otocz kólkiem @ i zaznacz wlaściwe.

6. $\mathrm{W}$ zadaniu 5. wpisz odpowiednie cyfry w kratki pod treściq zadania.

7. Pamiptaj, $\dot{\mathrm{z}}\mathrm{e}$ pominipcie argumentacji lub istotnych obliczeń w rozwiqzaniu zadania

otwartego (6-15) $\mathrm{m}\mathrm{o}\dot{\mathrm{z}}\mathrm{e}$ spowodować, $\dot{\mathrm{z}}\mathrm{e}$ za to rozwiqzanie nie otrzymasz pelnej liczby

punktów.

8. Pisz czytelnie i $\mathrm{u}\dot{\mathrm{z}}$ ywaj tylko dlugopisu lub pióra z czarnym tuszem lub atramentem.

9. Nie $\mathrm{u}\dot{\mathrm{z}}$ ywaj korektora, a bledne zapisy wyra $\acute{\mathrm{z}}$ nie przekreśl.

10. $\mathrm{P}\mathrm{a}\mathrm{m}\mathrm{i}_{9}\mathrm{t}\mathrm{a}\mathrm{j}, \dot{\mathrm{z}}\mathrm{e}$ zapisy w brudnopisie nie bedq oceniane.

11. $\mathrm{M}\mathrm{o}\dot{\mathrm{z}}$ esz korzystač z zestawu wzorów matematycznych, cyrkla i linijki oraz kalkulatora

prostego.

Uklad graficzny

\copyright CKE 2021




{\it W kazdym z zadań od f. do 4. wybierz i zaznacz na karcie odpowiedzi poprawnq odpowiedz}'.

Zadanie 1. (0-1)

Róznica $\cos^{2}165^{\mathrm{o}}-\sin^{2}165^{\mathrm{o}}$ jest równa

A. $-1$

B. $-\displaystyle \frac{\sqrt{3}}{2}$

C. - -21

D. $\displaystyle \frac{\sqrt{3}}{2}$

Zadanie 2. $\{0-l\mathrm{I}$

Na rysunku przedstawiono fragment

rzeczywistej $x.$

wykresu funkcji

f określonej

dla $\mathrm{k}\mathrm{a}\dot{\mathrm{z}}$ dej liczby

Jeden spośród podanych ponizej wzorów jest wzorem tej funkcji. Wskaz wzór funkcji f.

A. $f(x)=\displaystyle \frac{\cos x+1}{|\cos x|+1}$

B. $f(x)=\displaystyle \frac{\sin x+1}{|\sin x|+1}$

C. $f(x)=\displaystyle \frac{|\cos x|-2}{\cos x-2}$

D. $f(x)=\displaystyle \frac{|\sin x|-2}{\sin x-2}$

Zadanie 3. (0-1)

Wielomian $W(x)=x^{4}+81$ jest podzielny przez

A. $x-3$

B. $x^{2}+9$

C. $x^{2}-3\sqrt{2}x+9$

D. $x^{2}+3\sqrt{2}x-9$

Zadanie 4. (0-1)

Liczba róznych pierwiastków równania $3x+|x-4|=0$ jest równa

A. 0

B. l

C. 2

D. 3

Strona 2 z27

$\mathrm{E}\mathrm{M}\mathrm{A}\mathrm{P}-\mathrm{R}0_{-}100$




\begin{center}
\includegraphics[width=192.840mm,height=258.876mm]{./F2_M_PR_M2021_page10_images/image001.eps}
\end{center}
$\mathrm{O}\mathrm{d}\mathrm{p}\mathrm{o}\mathrm{w}\mathrm{i}\mathrm{e}\mathrm{d}\acute{\mathrm{z}}$:$\ldots\ldots\ldots\ldots\ldots\ldots\ldots\ldots\ldots\ldots\ldots\ldots\ldots\ldots\ldots\ldots\ldots\ldots\ldots\ldots\ldots\ldots\ldots\ldots\ldots\ldots\ldots\ldots\ldots\ldots\ldots\ldots\ldots\ldots\ldots\ldots\ldots\ldots\ldots\ldots\ldots\ldots$

Wypelnia

egzaminator

Nr zadania

Maks. liczba pkt

Uzyskana liczba pkt

9.

4

$\mathrm{E}\mathrm{M}\mathrm{A}\mathrm{P}-\mathrm{R}0_{-}100$

Strona ll z27





$\mathrm{Z}\text{à} \mathrm{d}^{\backslash }\cdot \mathrm{a}\mathfrak{n}1\mathrm{e}10. 1(0-4l1$

Prosta przechodzqca przez

punkty $A = (8,-6)$

i

$B = (5,15)$ jest styczna

do

okregu

o środku w punkcie $0 =$

(0,0). Oblicz promień tego okregu i wspólrzedne punktu styczności tego

okrpgu z prostq

{\it AB}.
\begin{center}
\includegraphics[width=192.840mm,height=258.876mm]{./F2_M_PR_M2021_page11_images/image001.eps}
\end{center}
Strona 12 z27

$\mathrm{E}\mathrm{M}\mathrm{A}\mathrm{P}-\mathrm{R}0_{-}100$




\begin{center}
\includegraphics[width=192.840mm,height=258.876mm]{./F2_M_PR_M2021_page12_images/image001.eps}
\end{center}
$\mathrm{O}\mathrm{d}\mathrm{p}\mathrm{o}\mathrm{w}\mathrm{i}\mathrm{e}\mathrm{d}\acute{\mathrm{z}}$:$\ldots\ldots\ldots\ldots\ldots\ldots\ldots\ldots\ldots\ldots\ldots\ldots\ldots\ldots\ldots\ldots\ldots\ldots\ldots\ldots\ldots\ldots\ldots\ldots\ldots\ldots\ldots\ldots\ldots\ldots\ldots\ldots\ldots\ldots\ldots\ldots\ldots\ldots\ldots\ldots\ldots\ldots$

Wypelnia

egzaminator

Nr zadania

Maks. liczba pkt

Uzyskana liczba pkt

10.

4

$\mathrm{E}\mathrm{M}\mathrm{A}\mathrm{P}-\mathrm{R}0_{-}100$

Strona 13 z27





$\mathrm{Z}\text{à} \mathrm{d}^{\backslash }\cdot \mathrm{a}\mathfrak{n}1\mathrm{e}11^{\backslash }1.(0-5\}$

Wyznacz wszystkie wartości parametru $m$, dla których trójmian kwadratowy

$4x^{2}-2(m+1)x+m$

ma dwa rózne pierwiastki rzeczywiste

$\chi_{1}$

oraz $x_{2}$, spelniajqce warunki

$\chi_{1} \neq 0,$

$\chi_{2}$

$\neq 0$

oraz

$\chi_{1} +x_{2} \leq_{\chi_{1}\chi_{2}}^{1_{+}1}$
\begin{center}
\includegraphics[width=192.840mm,height=234.852mm]{./F2_M_PR_M2021_page13_images/image001.eps}
\end{center}
Strona 14 z27

$\mathrm{E}\mathrm{M}\mathrm{A}\mathrm{P}-\mathrm{R}0_{-}100$




\begin{center}
\includegraphics[width=192.840mm,height=258.876mm]{./F2_M_PR_M2021_page14_images/image001.eps}
\end{center}
$\mathrm{O}\mathrm{d}\mathrm{p}\mathrm{o}\mathrm{w}\mathrm{i}\mathrm{e}\mathrm{d}\acute{\mathrm{z}}$:$\ldots\ldots\ldots\ldots\ldots\ldots\ldots\ldots\ldots\ldots\ldots\ldots\ldots\ldots\ldots\ldots\ldots\ldots\ldots\ldots\ldots\ldots\ldots\ldots\ldots\ldots\ldots\ldots\ldots\ldots\ldots\ldots\ldots\ldots\ldots\ldots\ldots\ldots\ldots\ldots\ldots\ldots$

Wypelnia

egzaminator

Nr zadania

Maks. liczba pkt

Uzyskana liczba pkt

11.

5

$\mathrm{E}\mathrm{M}\mathrm{A}\mathrm{P}-\mathrm{R}0_{-}100$

Strona 15 z27





$\mathrm{Z}\text{à} \mathrm{d}^{\backslash }\cdot \mathrm{a}\mathfrak{n}1\mathrm{e}12^{\backslash }1.(0-5\}$

Rozwiqz równanie

$\cos 2x =\displaystyle \frac{\sqrt{2}}{2}(\cos x-\sin x)$

w przedziale

$\langle 0, \pi\rangle.$
\begin{center}
\includegraphics[width=192.840mm,height=270.912mm]{./F2_M_PR_M2021_page15_images/image001.eps}
\end{center}
Strona 16 z27

$\mathrm{E}\mathrm{M}\mathrm{A}\mathrm{P}-\mathrm{R}0_{-}100$




\begin{center}
\includegraphics[width=192.840mm,height=258.876mm]{./F2_M_PR_M2021_page16_images/image001.eps}
\end{center}
$\mathrm{O}\mathrm{d}\mathrm{p}\mathrm{o}\mathrm{w}\mathrm{i}\mathrm{e}\mathrm{d}\acute{\mathrm{z}}$:$\ldots\ldots\ldots\ldots\ldots\ldots\ldots\ldots\ldots\ldots\ldots\ldots\ldots\ldots\ldots\ldots\ldots\ldots\ldots\ldots\ldots\ldots\ldots\ldots\ldots\ldots\ldots\ldots\ldots\ldots\ldots\ldots\ldots\ldots\ldots\ldots\ldots\ldots\ldots\ldots\ldots\ldots$

Wypelnia

egzaminator

Nr zadania

Maks. liczba pkt

Uzyskana liczba pkt

12.

5

$\mathrm{E}\mathrm{M}\mathrm{A}\mathrm{P}-\mathrm{R}0_{-}100$

Strona

17 z27





$\mathrm{Z}\text{à} \mathrm{d}^{\backslash }\cdot \mathrm{a}\mathrm{n}1\mathrm{e}1 13^{\backslash }. (0-4$

Dany jest trójkqt prostokqtny

{\it ABC}.

Promień okregu wpisanego w ten trójkqt jest pieć razy

krótszy od przeciwprostokqtnej tego trójkqta. Oblicz sinus tego z kqtów ostrych trójkqta

{\it ABC},

który ma wipkszq miar9.
\begin{center}
\includegraphics[width=192.840mm,height=258.924mm]{./F2_M_PR_M2021_page17_images/image001.eps}
\end{center}
Strona 18 z27

$\mathrm{E}\mathrm{M}\mathrm{A}\mathrm{P}-\mathrm{R}0_{-}100$




\begin{center}
\includegraphics[width=192.840mm,height=258.876mm]{./F2_M_PR_M2021_page18_images/image001.eps}
\end{center}
$\mathrm{O}\mathrm{d}\mathrm{p}\mathrm{o}\mathrm{w}\mathrm{i}\mathrm{e}\mathrm{d}\acute{\mathrm{z}}$:$\ldots\ldots\ldots\ldots\ldots\ldots\ldots\ldots\ldots\ldots\ldots\ldots\ldots\ldots\ldots\ldots\ldots\ldots\ldots\ldots\ldots\ldots\ldots\ldots\ldots\ldots\ldots\ldots\ldots\ldots\ldots\ldots\ldots\ldots\ldots\ldots\ldots\ldots\ldots\ldots\ldots\ldots$

Wypelnia

egzaminator

Nr zadania

Maks. liczba pkt

Uzyskana liczba pkt

13.

4

$\mathrm{E}\mathrm{M}\mathrm{A}\mathrm{P}-\mathrm{R}0_{-}100$

Strona 19 z27





Zadänie $l4. (0-6)$

Dane sq parabola o równaniu $y=x^{2}$ oraz punkty $A=(0,2) \mathrm{i} B=(1,3)$ (zobacz rysunek).

Rozpatrujemy wszystkie trójkqty $ABC$, których wierzcholek $C \mathrm{l}\mathrm{e}\dot{\mathrm{z}}\mathrm{y}$ na tej paraboli. Niech $m$

oznacza pierwszq wspólrzedna punktu $C.$

a) Wyznacz pole $P$ trójkqta $ABC$ jako funkcj9 zmiennej $m.$

b) Wyznacz wszystkie wartości $m$, dla których trójkqt $ABC$ jest ostrokqtny.
\begin{center}
\includegraphics[width=192.840mm,height=132.636mm]{./F2_M_PR_M2021_page19_images/image001.eps}
\end{center}
1

1

$1$

$\Gamma^{\neg}111$

$111\overline{1}$

1

$-|$

$1111$

$-|$

1

$1^{1} \mathrm{T}$

$1^{-}1$

I

$\mathrm{i}$

1

1

11

111

Jll

$\rfloor 1$

$\lceil$ tl i$|$- $| \dagger|$1l

$\mapsto_{1}\mathrm{I}1$

L

1

11-

1

$\leftarrow \mathrm{I}\mathrm{I}\mathrm{l}\mathrm{l}\mathrm{I}\mathrm{T}11$

ll

$-1$

$\lfloor 1$ 11

1

L$\lfloor+ +$Ill$| \iota|\lfloor$----$|$--lllll

11

11

11

Il

L

1

1

1

$- \lrcorner 1 \mathrm{L}\mathrm{l}1$

$\mathrm{I}1$

11

$-111 \dagger^{1} \rightarrow$

1

$11$ Il

-i $\dot{1}$

1

$1$ 1

$|$l.i

$1$

$111$

$\lrcorner 1$

$\leftarrow 1111$

$\downarrow$

$\dashv-1_{1}^{\mathrm{t}}1$

1

$1$

$1111$

1

1

L

1

$1$

$\mathrm{t}$

$|$ll $|$l--l

$1$

$1$

$1$

L

1

$\perp$

11

1

1!

$!$ : -$\vdash 11 1\mathrm{I}1$

$1$

T

I

1

$1$

$\lceil$

$11\mathrm{I}1$

$-1-$

$\mathrm{r}111$

$111$

$1111$

$11$

ll

1

$1_{-}1$

11

11

ll

$-1$

$1$

I

$111$

$+$

$\neg 1$ -T llll

$\tau \iota$l $|| | |$--l1lll

$\ulcorner 1$

1

$-1_{1}1^{-\mapsto}$

$11$

$+^{1}$

T

$-111$

1

11

Il $1 1$

Il

$\mathrm{L}\mathrm{l}1$

$\rfloor$

$1$

$1$

$1$

-l$|$

Strona 20 z27

$\mathrm{E}\mathrm{M}\mathrm{A}\mathrm{P}-\mathrm{R}0_{-}100$





- RUDNOPIS

$\{m^{\bullet}\mathrm{e}$ {\it podlega} $oc\mathrm{e}m^{\bullet}\mathrm{e}\}$
\begin{center}
\includegraphics[width=192.840mm,height=282.960mm]{./F2_M_PR_M2021_page2_images/image001.eps}
\end{center}
$\mathrm{E}\mathrm{M}\mathrm{A}\mathrm{P}-\mathrm{R}0_{-}100$

Strona 3 z27




\begin{center}
\includegraphics[width=192.840mm,height=294.948mm]{./F2_M_PR_M2021_page20_images/image001.eps}
\end{center}
$\mathrm{E}\mathrm{M}\mathrm{A}\mathrm{P}-\mathrm{R}0_{-}100$

Strona 21 z27




\begin{center}
\includegraphics[width=192.840mm,height=294.948mm]{./F2_M_PR_M2021_page21_images/image001.eps}
\end{center}
Strona 22 z27

$\mathrm{E}\mathrm{M}\mathrm{A}\mathrm{P}-\mathrm{R}0_{-}100$




\begin{center}
\includegraphics[width=192.840mm,height=258.876mm]{./F2_M_PR_M2021_page22_images/image001.eps}
\end{center}
$\mathrm{O}\mathrm{d}\mathrm{p}\mathrm{o}\mathrm{w}\mathrm{i}\mathrm{e}\mathrm{d}\acute{\mathrm{z}}$:$\ldots\ldots\ldots\ldots\ldots\ldots\ldots\ldots\ldots\ldots\ldots\ldots\ldots\ldots\ldots\ldots\ldots\ldots\ldots\ldots\ldots\ldots\ldots\ldots\ldots\ldots\ldots\ldots\ldots\ldots\ldots\ldots\ldots\ldots\ldots\ldots\ldots\ldots\ldots\ldots\ldots\ldots$

Wypelnia

egzaminator

Nr zadania

Maks. liczba pkt

Uzyskana liczba pkt

14.

6

$\mathrm{E}\mathrm{M}\mathrm{A}\mathrm{P}-\mathrm{R}0_{-}100$

Strona 23 z27





Zadänie 15. $(0-7\displaystyle \int$

Pewien zaklad otrzymal zamówienie na wykonanie prostopadlościennego zbiornika

(calkowicie otwartego od góry) o pojemności 144 $\mathrm{m}^{3}$ Dno zbiornika ma byč kwadratem.

$\dot{\mathrm{Z}}$ aden z wymiarów zbiornika (krawedzi prostopadlościanu) nie $\mathrm{m}\mathrm{o}\dot{\mathrm{z}}\mathrm{e}$ przekraczač 9 metrów.

Calkowity koszt wykonania zbiornika ustalono w nastepujqcy sposób:

- 100 zl za l $\mathrm{m}^{2}$ dna

- 75 zl za l $\mathrm{m}^{2}$ ściany bocznej.

Oblicz wymiary zbiornika, dla którego tak ustalony koszt wykonania będzie najmniejszy.
\begin{center}
\includegraphics[width=192.840mm,height=228.852mm]{./F2_M_PR_M2021_page23_images/image001.eps}
\end{center}
$1^{\cdot}$

1

11

$1$

$\underline{!}$

1

$1 \rfloor 1$

$1$

$1$

1111

$+$

$\mathrm{f}-1$

1

$\iota$

$1$

IlIl

1

$1$

$\mathrm{I}11$

$\iota^{1}$

$1$

$-\mathrm{I}1$

$1$

$1-\mathrm{I}\mathrm{I}1$

1

$!1$

$\rfloor$

l

$\dagger$lI

t

-ll

l

1

$1 |$

$\mathrm{r} !$

$1 |$

1

$+^{1}11$

$1^{-}11$

$\mathrm{r}\mathrm{l}1$

$\lrcorner\rfloor$

$1$

$\lceil$

$\mathrm{J}1$

$\rfloor|1$

1

1

.llI :$\neg\urcorner|$

$\mathrm{r}1$

$\vdash$IlllIl--]$+\dashv\neg$

11

I

$\mathrm{i}$

T

l

$|$

ll

$|$

$|$

I

1l$|$

11

1

$1$

1

$1_{-} \mathrm{T}$

$1$

$\downarrow$

$1$

11

$11$

- $\mathrm{r}$

ll

$\}$

1

1

1

I

1

1 $\mathrm{L} \mathrm{T}$

1

Il

$1$

$\mathrm{i}$

1

Tl$|$

ll

11

1

11

$1$

$\mathrm{I}\mathrm{l}\mathrm{I}|$

ll$|$l

$1\displaystyle \frac{!}{1}1|\mathrm{I}1$

$\rightarrow$ll

$11$

1

1

1

1

$\mapsto 11$

ll

$1$

$1$ Il

$\iota_{1}$ -

$1 \mathrm{i}$

1

1

$1$

$\mathrm{J}^{1}$

$\lrcorner 11\neg 1$

$1$

í $\neg$llLI L$\dagger$I--$| |$lll f $\ulcorner$ -$|$lllll $\perp+$lIl t

$\dagger 1$

-$\dagger||$ lll--l.

I

$\mathrm{l}\mathrm{r}\ulcorner 1$

1

$1$ 1

$-1$

$+$

1

$11111$

11

$1^{-}$

$\mathrm{L}\mathrm{I}\mathrm{I}$

$1$

ll

11

$\ulcorner$

$\dashv 11-11\mathrm{I}1\urcorner$

1

$1$ 11

11

$1$

$\wedge 1$

Il

$-11$

ll

$1$

$1$

ll

11

$\urcorner$

1

$11$

$\downarrow$-

1

$\mathrm{t} |^{-}1$

1

$!$

1

$\leftarrow 11$

il

1

--ll$|$

$1$

$1$

$|^{-}1$

$\dot{\mathrm{i}}|111$

$\dagger$1 $1^{-}1 \leftarrow$

$\dagger$

1

1

$11$

$\rfloor\iota_{\mathrm{I}}$

$1$

1

$1$

1

ll

- $11$

1

$11$

$\dagger 1$

1

$\dashv 1$

$-111^{-}1$

11

$11$

$1111$

ll

1 $|^{-}1$

$1$

$1$

$\mathrm{t}^{1}$

LlI

1

I

ll

1

\{1

l

$1 1$

$11$

1

$1$ 11

11

$\leftarrow 1$

T

$| +$llll$\lrcorner$1$|$

$\}1$

$\rfloor 1$

$1$

$\rfloor$

-$||$

-1

1

$\llcorner 1 \rfloor$

Strona 24 z27

$\mathrm{E}\mathrm{M}\mathrm{A}\mathrm{P}-\mathrm{R}0_{-}100$




\begin{center}
\includegraphics[width=192.840mm,height=294.948mm]{./F2_M_PR_M2021_page24_images/image001.eps}
\end{center}
$\mathrm{E}\mathrm{M}\mathrm{A}\mathrm{P}-\mathrm{R}0_{-}100$

Strona 25 z27




\begin{center}
\includegraphics[width=192.840mm,height=258.876mm]{./F2_M_PR_M2021_page25_images/image001.eps}
\end{center}
$\mathrm{O}\mathrm{d}\mathrm{p}\mathrm{o}\mathrm{w}\mathrm{i}\mathrm{e}\mathrm{d}\acute{\mathrm{z}}$:$\ldots\ldots\ldots\ldots\ldots\ldots\ldots\ldots\ldots\ldots\ldots\ldots\ldots\ldots\ldots\ldots\ldots\ldots\ldots\ldots\ldots\ldots\ldots\ldots\ldots\ldots\ldots\ldots\ldots\ldots\ldots\ldots\ldots\ldots\ldots\ldots\ldots\ldots\ldots\ldots\ldots\ldots$

Wypelnia

egzaminator

Nr zadania

Maks. liczba pkt

Uzyskana liczba pkt

15.

7

Strona 26 z27

$\mathrm{E}\mathrm{M}\mathrm{A}\mathrm{P}-\mathrm{R}0_{-}100$





: RU DNOPIS

\{$m^{\bullet}\mathrm{e}$ {\it podlega ocenie}\}
\begin{center}
\includegraphics[width=192.840mm,height=276.960mm]{./F2_M_PR_M2021_page26_images/image001.eps}
\end{center}
$\mathrm{E}\mathrm{M}\mathrm{A}\mathrm{P}-\mathrm{R}0_{-}100$

Strona 27 z27










Zadänie 5. $\langle 0-2l$

Oblicz granice $\displaystyle \lim_{n\rightarrow\infty}\frac{(3n+2)^{2}-(1-2n)^{2}}{(2n-1)^{2}}$

W ponizsze kratki wpisz kolejno-od lewej do prawej- cyfre jedności i pierwsze dwie cyfry po

przecinku skończonego rozwiniecia dziesietnego otrzymanego wyniku.
\begin{center}
\includegraphics[width=25.452mm,height=12.240mm]{./F2_M_PR_M2021_page3_images/image001.eps}

\includegraphics[width=192.840mm,height=222.864mm]{./F2_M_PR_M2021_page3_images/image002.eps}
\end{center}
Strona 4 z27

$\mathrm{E}\mathrm{M}\mathrm{A}\mathrm{P}-\mathrm{R}0_{-}100$





$\mathrm{Z}\text{à} \mathrm{d}^{\backslash }\cdot \mathrm{a}\mathfrak{n}1\mathrm{e}61. \langle 0-3)^{\backslash }$

Niech log218 $= \mathrm{c}$. Wykaz, $\dot{\mathrm{z}}\mathrm{e}$

$\log_{3}4 =\displaystyle \frac{4}{\mathrm{c}-1}$
\begin{center}
\includegraphics[width=192.840mm,height=246.888mm]{./F2_M_PR_M2021_page4_images/image001.eps}

\begin{tabular}{|l|l|l|l|}
\cline{2-4}
&	\multicolumn{1}{|l|}{Nr zadania}&	\multicolumn{1}{|l|}{$5.$}&	\multicolumn{1}{|l|}{ $6.$}	\\
\cline{2-4}
&	\multicolumn{1}{|l|}{Maks. liczba pkt}&	\multicolumn{1}{|l|}{$2$}&	\multicolumn{1}{|l|}{ $3$}	\\
\cline{2-4}
\multicolumn{1}{|l|}{egzaminator}&	\multicolumn{1}{|l|}{Uzyskana liczba pkt}&	\multicolumn{1}{|l|}{}&	\multicolumn{1}{|l|}{}	\\
\hline
\end{tabular}

\end{center}
$\mathrm{E}\mathrm{M}\mathrm{A}\mathrm{P}-\mathrm{R}0_{-}100$

Strona 5 z27





$\mathrm{Z}\text{à} \mathrm{d}^{\backslash }\cdot \mathrm{a}\mathfrak{n}1\mathrm{e}_{1}7_{1}1. (0-3)^{\backslash }$

Rozwiqz nierównośč:

$\displaystyle \frac{2x-1}{1-x}\leq\frac{2+2x}{5x}$
\begin{center}
\includegraphics[width=192.840mm,height=264.924mm]{./F2_M_PR_M2021_page5_images/image001.eps}
\end{center}
Strona 6 z27

$\mathrm{E}\mathrm{M}\mathrm{A}\mathrm{P}-\mathrm{R}0_{-}100$




\begin{center}
\includegraphics[width=192.840mm,height=258.876mm]{./F2_M_PR_M2021_page6_images/image001.eps}
\end{center}
$\mathrm{O}\mathrm{d}\mathrm{p}\mathrm{o}\mathrm{w}\mathrm{i}\mathrm{e}\mathrm{d}\acute{\mathrm{z}}$:$\ldots\ldots\ldots\ldots\ldots\ldots\ldots\ldots\ldots\ldots\ldots\ldots\ldots\ldots\ldots\ldots\ldots\ldots\ldots\ldots\ldots\ldots\ldots\ldots\ldots\ldots\ldots\ldots\ldots\ldots\ldots\ldots\ldots\ldots\ldots\ldots\ldots\ldots\ldots\ldots\ldots\ldots$

Wypelnia

egzaminator

Nr zadania

Maks. liczba pkt

Uzyskana liczba pkt

7.

3

$\mathrm{E}\mathrm{M}\mathrm{A}\mathrm{P}-\mathrm{R}0_{-}100$

Strona 7 z27





$\mathrm{Z}\text{à} \mathrm{d}^{\backslash }\cdot \mathrm{a}\mathfrak{n}1\mathrm{e}1$ @. $(0-3)^{\backslash }$

Danyjest trójkqt równoboczny $ABC$. Na bokach AB $\mathrm{i} AC$ wybrano punkty- odpowiednio-

{\it D} $\mathrm{i} E$ takie, $\dot{\mathrm{z}}\mathrm{e} |BD| = |AE| =\displaystyle \frac{1}{3}|AB|$. Odcinki CD $\mathrm{i}$ BE przecinajq si9 w punkcie $P$

(zobacz rysunek).

{\it C}
\begin{center}
\includegraphics[width=72.852mm,height=62.784mm]{./F2_M_PR_M2021_page7_images/image001.eps}
\end{center}
{\it E}

{\it P}

{\it A}

{\it D}

{\it B}

Wykaz, $\dot{\mathrm{z}}\mathrm{e}$ pole trójkata

{\it DBP}

jest 21

razy mniejsze od pola trójkqta

{\it ABC}.
\begin{center}
\includegraphics[width=192.840mm,height=162.708mm]{./F2_M_PR_M2021_page7_images/image002.eps}
\end{center}
Strona 8 z27

$\mathrm{E}\mathrm{M}\mathrm{A}\mathrm{P}-\mathrm{R}0_{-}100$




\begin{center}
\includegraphics[width=192.840mm,height=270.912mm]{./F2_M_PR_M2021_page8_images/image001.eps}
\end{center}
Wypelnia

egzaminator

Nr zadania

Maks. liczba pkt

Uzyskana liczba pkt

8.

3

$\mathrm{E}\mathrm{M}\mathrm{A}\mathrm{P}-\mathrm{R}0_{-}100$

Strona 9 z27





$\mathrm{Z}\text{à} \mathrm{d}^{\backslash }\cdot \mathrm{a}\mathfrak{n}\mathrm{i}_{1}\mathrm{e}1\mathrm{g}1. (0-\backslash 4)^{\backslash }\backslash \cdot$

Ze zbioru wszystkich

liczb naturalnych

czterocyfrowych losujemy

jednq liczb9.

Oblicz

prawdopodobieństwo zdarzenia polegajqcego na tym, $\dot{\mathrm{z}}\mathrm{e}$ wylosowana liczba jest podzielna

przez

15, jeśli wiadomo, $\dot{\mathrm{z}}\mathrm{e}$ jest ona podzielna przez 18.
\begin{center}
\includegraphics[width=192.840mm,height=264.924mm]{./F2_M_PR_M2021_page9_images/image001.eps}
\end{center}
Strona 10 z27

$\mathrm{E}\mathrm{M}\mathrm{A}\mathrm{P}-\mathrm{R}0_{-}100$







CENTRALNA

KOMISJA

EGZAMINACYJNA

Arkusz zawiera informacje prawnie chronione

do momentu rozpoczecia egzaminu.

WYPELNIA ZDAJACY

{\it Miejsce na naklejke}.

{\it Sprawdz}', {\it czy kod na naklejce to}

e-100.
\begin{center}
\includegraphics[width=21.900mm,height=16.260mm]{./F2_M_PR_M2022_page0_images/image001.eps}
\end{center}
KOD
\begin{center}
\includegraphics[width=79.656mm,height=16.260mm]{./F2_M_PR_M2022_page0_images/image002.eps}
\end{center}
PESEL

{\it Jezeli tak}- {\it przyklej naklejkq}.

{\it Jezeli nie}- {\it zgtoś to nauczycielowi}.

EGZAMIN MATURALNY Z MATEMATYKI

POZIOM ROZSZERZONY

{\it wr}PELNlA ZESPÓL $\mathrm{N}\mathrm{A}\mathrm{D}\mathrm{Z}\mathrm{O}\mathrm{R}\mathrm{U}\mathrm{J}\wedge \mathrm{C}Y$

DATA: \{$\{$ maja 2022 $\mathrm{r}.$

GODZINA ROZPOCZGCIA: 9: 00

CZAS PRACY: $\{80 \displaystyle \min \mathrm{u}\mathrm{t}$

LICZBA PUNKTÓW DO UZYSKANIA: 50

Uprawnienia zdaj\S cego do:

\fbox{} nieprzenoszenia zaznaczeń na karte

\fbox{} dostosowania zasad oceniania

\fbox{} dostosowania w zw. z dyskalkuliq.

$\Vert\Vert\Vert\Vert\Vert\Vert\Vert\Vert\Vert\Vert\Vert\Vert\Vert\Vert\Vert\Vert\Vert\Vert\Vert\Vert\Vert\Vert\Vert\Vert\Vert\Vert\Vert\Vert\Vert\Vert|$

EMAP-R0-100-2205

lnstrukcja dla zdajqcego

l. Sprawdz', czy arkusz egzaminacyjny zawiera 26 stron (zadania $1-15$).

Ewentualny brak zgloś przewodniczacemu zespolu nadzorujacego egzamin.

2. Na tej stronie oraz na karcie odpowiedzi wpisz swój numer PESEL i przyklej naklejke

z kodem.

3. Nie wpisuj $\dot{\mathrm{z}}$ adnych znaków w cz9ści przeznaczonej d1a egzaminatora.

4. Rozwiqzania zadań i odpowiedzi wpisuj w miejscu na to przeznaczonym.

5. Odpowiedzi do zadań zamknietych $(1-4)$ zaznacz na karcie odpowiedzi w cześci karty

przeznaczonej dla zdajqcego. Zamaluj $\blacksquare$ pola do tego przeznaczone. $\mathrm{B}_{9}\mathrm{d}\mathrm{n}\mathrm{e}$

zaznaczenie otocz kólkiem \copyright i zaznacz wlaściwe.

6. $\mathrm{W}$ zadaniu 5. wpisz odpowiednie cyfry w kratki pod treściq zadania.

7. Pamietaj, $\dot{\mathrm{z}}\mathrm{e}$ pominiecie argumentacji lub istotnych obliczeń w rozwiqzaniu zadania

otwartego (6-15) $\mathrm{m}\mathrm{o}\dot{\mathrm{z}}\mathrm{e}$ spowodowač, $\dot{\mathrm{z}}\mathrm{e}$ za to rozwiqzanie nie otrzymasz pelnej liczby

punktów.

8. Pisz czytelnie i $\mathrm{u}\dot{\mathrm{z}}$ ywaj tylko dlugopisu lub pióra z czarnym tuszem lub atramentem.

9. Nie $\mathrm{u}\dot{\mathrm{z}}$ ywaj korektora, a blędne zapisy wyraz'nie przekreśl.

10. Pamietaj, $\dot{\mathrm{z}}\mathrm{e}$ zapisy w brudnopisie nie bpdq oceniane.

11. $\mathrm{M}\mathrm{o}\dot{\mathrm{z}}$ esz korzystač z zestawu wzorów matematycznych, cyrkla i linijki oraz kalkulatora

prostego.

Uk\}ad graficzny

\copyright CKE 2021




{\it W kazdym z zadań od f. do 4. wybierz i zaznacz na karcie odpowiedzi poprawnq odpowiedz}'.

Zadanie l. $(0-1$\}

Liczba $\log_{3}\sqrt{27}-\log_{27}\sqrt{3}$ jest równa

A. -43

B. -21

C. $\displaystyle \frac{11}{12}$

D. 3

Zadanie 2. $\{0-1$)

Funkcja $f$ jest określona wzorem $f(x)=\displaystyle \frac{x^{3}-8}{x-2}$ dla $\mathrm{k}\mathrm{a}\dot{\mathrm{z}}$ dej liczby rzeczywistej $x\neq 2.$

Wartośč pochodnej tej funkcji dla argumentu $x=\displaystyle \frac{1}{2}$ jest równa

A. -43

B. -49

C. 3

D. $\displaystyle \frac{54}{8}$

Zadanie 3. $\langle 0-1$)

$\mathrm{J}\mathrm{e}\dot{\mathrm{z}}$ eli $\displaystyle \cos\beta=-\frac{1}{3} \mathrm{i} \displaystyle \beta\in(\pi,\frac{3}{2}\pi)$, to wartośč wyrazenia $\displaystyle \sin(\beta-\frac{1}{3}\pi)$ jest równa

A. $\displaystyle \frac{-2\sqrt{2}+\sqrt{3}}{6}$

B. $\displaystyle \frac{2\sqrt{6}+1}{6}$

C. $\displaystyle \frac{2\sqrt{2}+\sqrt{3}}{6}$

D. $\displaystyle \frac{1-2\sqrt{6}}{6}$

Zadanie 4. (0-1)

Dane sa dwie urny z kulami. $\mathrm{W}\mathrm{k}\mathrm{a}\dot{\mathrm{z}}$ dej z urn jest siedem kul. $\mathrm{W}$ pierwszej urnie sq jedna kula

biala i sześč kul czarnych, w drugiej urnie sa cztery kule biale i trzy kule czarne.

Rzucamyjeden raz symetryczna moneta. $\mathrm{J}\mathrm{e}\dot{\mathrm{z}}$ eli wypadnie reszka, to losujemyjedna kule

z pierwszej urny, w przeciwnym przypadku-jedna $\mathrm{k}\mathrm{u}\mathrm{l}9$ z drugiej urny.

Prawdopodobieństwo zdarzenia polegajqcego na tym, $\dot{\mathrm{z}}\mathrm{e}$ wylosujemy ku19 bia1a w tym

doświadczeniu, jest równe

A. $\displaystyle \frac{5}{14}$

B. $\displaystyle \frac{9}{14}$

C. -75

D. -67

Strona 2 z26

$\mathrm{E}\mathrm{M}\mathrm{A}\mathrm{P}-\mathrm{R}0_{-}100$





Wypelnia

egzaminator

Nr zadania

Maks. liczba pkt

Uzyskana liczba pkt

9.

4

-RO-100

Strona ll z26





Zadanie $10_{\mathrm{h}}\{0-4$)

Ciqg $(a_{n})$, określony dla $\mathrm{k}\mathrm{a}\dot{\mathrm{z}}$ dej liczby naturalnej $n\geq 1$, jest geometryczny i ma wszystkie

wyrazy dodatnie. Ponadto $a_{1}=675 \mathrm{i} a_{22}=\displaystyle \frac{5}{4}a_{23}+\frac{1}{5}a_{21}.$

Ciqg $(b_{n})$, określony dla $\mathrm{k}\mathrm{a}\dot{\mathrm{z}}$ dej liczby naturalnej $n\geq 1$, jest arytmetyczny.

Suma wszystkich wyrazów ciqgu $(a_{n})$ jest równa sumie dwudziestu pipciu poczqtkowych

kolejnych wyrazów ciqgu $(b_{n})$. Ponadto $a_{3}=b_{4}$. Oblicz $b_{1}.$

Strona 12 z26

$\mathrm{E}\mathrm{M}\mathrm{A}\mathrm{P}-\mathrm{R}0_{-}100$





Wypelnia

egzaminator

Nr zadania

Maks. liczba pkt

Uzyskana liczba pkt

10.

4

-RO-100

Strona 13 z26





Zadanie ll. $\{0-4$)

Rozwiqz równanie $\sin x+\sin 2x+\sin 3x=0$ w przedziale $\langle 0, \pi\rangle.$

Strona 14 z26

$\mathrm{E}\mathrm{M}\mathrm{A}\mathrm{P}-\mathrm{R}0_{-}10$





Wypelnia

egzaminator

Nr zadania

Maks. liczba pkt

Uzyskana liczba pkt

11.

4

-RO-100

Strona 15 z26





Zadarie 12. $\{0-5$)

Wyznacz wszystkie wartości parametru $m$, dla których równanie

$x^{2}-(m+1)x+m=0$

ma dwa rózne rozwiqzania rzeczywiste $x_{1}$ oraz $x_{2}, \mathrm{s}\mathrm{p}\mathrm{e}$niajqce warunki:

$\chi_{1}\neq 0, \chi_{2}\neq 0$

oraz

$\displaystyle \frac{1}{\chi_{1}}+\frac{1}{\chi_{2}}+2=\frac{1}{x_{1}^{2}}+\frac{1}{x_{2}^{2}}$

Strona 16 z26

$\mathrm{E}\mathrm{M}\mathrm{A}\mathrm{P}-\mathrm{R}0_{-}10$





Wypelnia

egzaminator

Nr zadania

Maks. liczba pkt

Uzyskana liczba pkt

12.

5

-RO-100

Strona 17 z26





Zadanie 13. $(0-5$\}

Danyjest $\mathrm{g}\mathrm{r}\mathrm{a}\mathrm{n}\mathrm{i}\mathrm{a}\mathrm{s}\mathrm{t}\mathrm{o}\mathrm{s}\dagger \mathrm{u}\mathrm{p}$ prosty ABCDEFGH o podstawie prostokqtnej ABCD. Przekatne

$AH \mathrm{i} AF$ ścian bocznych tworzq kqt ostry o mierze $\alpha$ takiej, $\dot{\mathrm{z}}\mathrm{e} \displaystyle \sin\alpha=\frac{12}{13}$ (zobacz

rysunek). Pole trójk ta $AFH$ jest równe 26,4

Oblicz wysokośč $h$ tego graniastoslupa.
\begin{center}
\includegraphics[width=60.660mm,height=83.316mm]{./F2_M_PR_M2022_page17_images/image001.eps}
\end{center}
{\it G}

I

I

I

I

{\it H} $1 E$

I

I

I

I

I

I

I

{\it F}

I

I

I

I

$--- B$

$\alpha$

{\it D  A}

{\it h}

Strona 18 z26

$\mathrm{E}\mathrm{M}\mathrm{A}\mathrm{P}-\mathrm{R}0_{-}100$





Wypelnia

egzaminator

Nr zadania

Maks. liczba pkt

Uzyskana liczba pkt

13.

5

-RO-100

Strona 19 z26





Zadarie 14. (0-6)

Punkt $A=(-3,2)$ jest wierzcholkiem trójkata równoramiennego $ABC$, w którym $|AC|=|BC|.$

Pole tego trójkqta jest równe 15. Bok $BC$ zawarty jest w prostej o równaniu $y=x-1.$

Oblicz wspólrzedne wierzcholków $B \mathrm{i} C$ tego trójkata.

Strona 20 z26

$\mathrm{E}\mathrm{M}\mathrm{A}\mathrm{P}-\mathrm{R}0_{-}100$





: {\it RU DNOPIS} \{{\it nie podlega ocenie}\}

$\mathrm{h}\mathrm{P}-\mathrm{R}0_{-}100$

Strona 3z 26





Wypelnia

egzaminator

Nr zadania

Maks. liczba pkt

Uzyskana liczba pkt

14.

6

-RO-100

Strona 21 z26





Zadarie 15. $\{0-7\}$

Rozpatrujemy wszystkie trójkqty równoramienne o obwodzie równym 18.

a) Wykaz, $\dot{\mathrm{z}}\mathrm{e}$ pole $P \mathrm{k}\mathrm{a}\dot{\mathrm{z}}$ dego z tych trójkqtów, jako funkcja dlugości $b$ ramienia, wyraza si9

wzorem $P(b)=\displaystyle \frac{(18-2b)\cdot\sqrt{18b-81}}{2}$

b) Wyznacz dziedzin9 funkcji P.

c) Oblicz dlugości boków tego z rozpatrywanych trójkatów, który ma najwipksze pole.

Strona 22 z26

$\mathrm{E}\mathrm{M}\mathrm{A}\mathrm{P}-\mathrm{R}0_{-}100$





$1)0_{-}100$

Strona 23 z26





Wypelnia

egzaminator

Nr zadania

Maks. liczba pkt

Uzyskana liczba pkt

15.

7

Strona 24 z26

$\mathrm{E}\mathrm{M}\mathrm{A}\mathrm{P}-\mathrm{R}0_{-}10$





: {\it RU DNOPIS} \{{\it nie podlega ocenie}\}

$\mathrm{h}\mathrm{P}-\mathrm{R}0_{-}100$

Strona 25 z26





Strona 26 z26

$\mathrm{E}\mathrm{M}\mathrm{A}\mathrm{P}-\mathrm{R}0_{-}10$















$\mathrm{Z}\mathrm{a}\mathrm{d}\mathrm{a}*\mathrm{i}\mathrm{e}5. (0-2\}$

Ciqg $(a_{n})$ jest określony dla $\mathrm{k}\mathrm{a}\dot{\mathrm{z}}\mathrm{d}\mathrm{e}\mathrm{j}$ liczby naturalnej $n\geq 1$ wzorem $a_{n}=\displaystyle \frac{(7p-1)n^{3}+5pn-3}{(p+1)n^{3}+n^{2}+p}$

gdzie $p$ jest liczbq rzeczywistq dodatniq.

Oblicz wartośč $p$, dla której granica ciagu $(a_{n})$ jest równa $\displaystyle \frac{4}{3}$

W ponizsze kratki wpisz kolejno-od lewej do prawej-pierwsza, drugq oraz trzeciq cyfre po

przecinku nieskończonego rozwiniecia dziesiptnego otrzymanego wyniku.
\begin{center}
\includegraphics[width=25.452mm,height=12.240mm]{./F2_M_PR_M2022_page3_images/image001.eps}
\end{center}
: {\it RU DNOPIS} \{{\it nie podlega ocenie}\}

Strona 4 z26

$\mathrm{E}\mathrm{M}\mathrm{A}\mathrm{P}-\mathrm{R}0_{-}100$





Zadarie 6. $\{0-3\}$

Wykaz, $\dot{\mathrm{z}}\mathrm{e}$ dla $\mathrm{k}\mathrm{a}\dot{\mathrm{z}}$ dej liczby rzeczywistej $x$ i dla $\mathrm{k}\mathrm{a}\dot{\mathrm{z}}$ dej liczby rzeczywistej $y$ takich, $\dot{\mathrm{z}}\mathrm{e}$

$2x>\mathrm{y}$, spelniona jest nierównośč

$7x^{3}+4x^{2}y\geq y^{3}+2xy^{2}-x^{3}$
\begin{center}
\begin{tabular}{|l|l|l|l|}
\cline{2-4}
&	\multicolumn{1}{|l|}{Nr zadania}&	\multicolumn{1}{|l|}{$5.$}&	\multicolumn{1}{|l|}{ $6.$}	\\
\cline{2-4}
&	\multicolumn{1}{|l|}{Maks. liczba pkt}&	\multicolumn{1}{|l|}{$2$}&	\multicolumn{1}{|l|}{ $3$}	\\
\cline{2-4}
\multicolumn{1}{|l|}{egzaminator}&	\multicolumn{1}{|l|}{Uzyskana liczba pkt}&	\multicolumn{1}{|l|}{}&	\multicolumn{1}{|l|}{}	\\
\hline
\end{tabular}

\end{center}
$\mathrm{E}\mathrm{M}\mathrm{A}\mathrm{P}-\mathrm{R}0_{-}100$

Strona 5 z26





Zadarie 7. $\{0-3\}$

Rozwiqz równanie:

$|x-3|=2x+11$

Strona 6 z26

$\mathrm{E}\mathrm{M}\mathrm{A}\mathrm{P}-\mathrm{R}0_{-}10$





Wypelnia

egzaminator

Nr zadania

Maks. liczba pkt

Uzyskana liczba pkt

7.

3

-RO-100

Strona 7 z26





Zadarie 8. $\{0-3\}$

Punkt $P$ jest punktem $\mathrm{p}\mathrm{r}\mathrm{z}\mathrm{e}\mathrm{c}\mathrm{i}_{9}\mathrm{c}\mathrm{i}\mathrm{a}$ przekqtnych trapezu ABCD. Dlugośč podstawy $CD$ jest

$0 2$ mniejsza od dlugości podstawy $AB$. Promień okregu opisanego na trójkacie

ostrokatnvm $CPD$ jest o 3 mniejszy od promienia okregu opisanego na trójkqcie $APB.$

Wykaz, $\dot{\mathrm{z}}\mathrm{e}$ spelnionyjest warunek $|DP|^{2}+|CP|^{2}-|CD|^{2}=\displaystyle \frac{4\sqrt{2}}{3}\cdot|DP| |CP|.$

Strona 8 z26

$\mathrm{E}\mathrm{M}\mathrm{A}\mathrm{P}-\mathrm{R}0_{-}100$





Wypelnia

egzaminator

Nr zadania

Maks. liczba pkt

Uzyskana liczba pkt

8.

3

-RO-100

Strona 9 z26





Zadanie $\mathrm{g}. \{0-4$)

Reszta z dzielenia wielomianu $W(x)=4x^{3}-6x^{2}-(5m+1)x-2m$ przez dwumian $x+2$

jest równa $(-30).$

Oblicz $m$ i dla wyznaczonej wartości $m$ rozwiqz nierównośč $W(x)\geq 0.$

Strona 10 z26

$\mathrm{E}\mathrm{M}\mathrm{A}\mathrm{P}-\mathrm{R}0_{-}100$







CENTRALNA

KOMISJA

EGZAMINACYJNA

KOD

WYPELNIA ZDAJACY

PESEL
\begin{center}
\includegraphics[width=21.900mm,height=10.164mm]{./F2_M_PR_M2023_page0_images/image001.eps}

\includegraphics[width=79.656mm,height=10.164mm]{./F2_M_PR_M2023_page0_images/image002.eps}
\end{center}
Egzamin maturalny

DATA: 12 maja 2023 r.

GODZINA R0ZP0CZECIA: 9:00

CZAS TRWANIA: $180 \displaystyle \min$ ut

Arkusz zawiera informacje prawnie chronione

do momentu rozpoczecia egzaminu.

{\it Miejsce na naklejke}.

{\it Sprawdz}', {\it czy kod na naklejce to}

e-100.

/{\it ezeli tak}- {\it przyklej naklejke}.

/{\it ezeli nie}- {\it zgtoś to nauczycielowi}.

MAP-R0-100-2305

$\Re \mathrm{V}\Psi\S \mathrm{L}\mathrm{N}\Re$ 2B@P A $\mathrm{N}\mathrm{A}\emptyset.\mathrm{Z}\otimes\Re \mathrm{U}\mathrm{d}.\mathrm{A}\otimes Y,$

Uprawnienia $\mathrm{z}\mathrm{d}\mathrm{a}\mathrm{j}_{8}$cego do:

\fbox{} dostosowania zasad oceniania

\fbox{} dostosowania w zw. z dyskalkuliq

\fbox{} nieprzenoszenia zaznaczeń na karte.

LICZBA PUNKTÓW DO UZYSKANIA 50

Przed rozpoczeciem pracy z arkuszem egzaminacyjnym

1.

Sprawd $\acute{\mathrm{z}}$, czy nauczyciel przekazal Ci wlaściwy arkusz egzaminacyjny,

tj. arkusz we wlaściwej formule, z w[aściwego przedmiotu na wlaściwym

poziomie.

2.

$\mathrm{J}\mathrm{e}\dot{\mathrm{z}}$ eli przekazano Ci niew[aściwy arkusz- natychmiast zgloś to nauczycielowi.

Nie rozrywaj banderol.

3. $\mathrm{J}\mathrm{e}\dot{\mathrm{z}}$ eli przekazano Ci w[aściwy arkusz- rozerwij banderole po otrzymaniu

takiego polecenia od nauczyciela. Zapoznaj $\mathrm{s}\mathrm{i}\mathrm{e}$ z instrukcjq na stronie 2.

Uk\}ad graficzny

\copyright CKE 2022

$\Vert\Vert\Vert\Vert\Vert\Vert\Vert\Vert\Vert\Vert\Vert\Vert\Vert\Vert\Vert\Vert\Vert\Vert\Vert\Vert\Vert\Vert\Vert\Vert\Vert\Vert\Vert\Vert\Vert\Vert|$




lnstrukcja dla zdajqcego

l. Sprawdz', czy arkusz egzaminacyjny zawiera 29 stron (zadania $1-16$).

Ewentualny brak zgloś przewodniczqcemu zespolu nadzorujqcego egzamin.

2. Na pierwszej stronie arkusza oraz na karcie odpowiedzi wpisz swój numer PESEL

i przyklej naklejke z kodem.

3. Odpowiedzi do zadań $\mathrm{z}\mathrm{a}\mathrm{m}\mathrm{k}\mathrm{n}\mathrm{i}9$tych ($1-4)$ zaznacz na karcie odpowiedzi w cześci karty

przeznaczonej dla zdajacego. Zamaluj $\blacksquare$ pola do tego przeznaczone. $\mathrm{B}9\mathrm{d}\mathrm{n}\mathrm{e}$

zaznaczenie otocz kólkiem\copyright izaznacz wlaściwe.

4. $\mathrm{W}$ zadaniu 5. wpisz odpowiednie cyfry w kratki pod treściq zadania.

5. Pamiptaj, $\dot{\mathrm{z}}\mathrm{e}$ pominiecie argumentacji lub istotnych obliczeń w rozwiqzaniu zadania

otwartego (6-16) $\mathrm{m}\mathrm{o}\dot{\mathrm{z}}\mathrm{e}$ spowodowač, $\dot{\mathrm{z}}\mathrm{e}$ za to rozwiqzanie nie otrzymasz pelnej liczby

punktów.

6. Rozwiqzania zadań i odpowiedzi wpisuj w miejscu na to przeznaczonym.

7. Pisz czytelnie i $\mathrm{u}\dot{\mathrm{z}}$ ywaj tylko dlugopisu lub pióra z czarnym tuszem lub atramentem.

8. Nie $\mathrm{u}\dot{\mathrm{z}}$ ywaj korektora, a bledne zapisy wyra $\acute{\mathrm{z}}$ nie przekreśl.

9. Nie wpisuj $\dot{\mathrm{z}}$ adnych znaków w cześci przeznaczonej dla egzaminatora.

10. Pamietaj, $\dot{\mathrm{z}}\mathrm{e}$ zapisy w brudnopisie nie beda oceniane.

11. $\mathrm{M}\mathrm{o}\dot{\mathrm{z}}$ esz korzystač z Wybranych wzoróvv matematycznych, cyrkla i linijki oraz kalkulatora

prostego. Upewnij $\mathrm{s}\mathrm{i}\mathrm{e}$, czy przekazano Ci broszur9 z ok1adka takq jak widoczna ponizej.

$mb\uparrow h_{a_{\Delta>0}}\ldots.,.$

[‡M]$\xi$

{\$} $\mathrm{r}..\iota \mathrm{u}\mathrm{p}z\mathrm{n}\backslash \sim-\wedge\cdot\cdot \backslash \mathrm{t}n\triangleright\tau \mathrm{r}\dot{\alpha}\mathrm{c}\backslash $

$\overline{\infty\epsilon \mathrm{w}\mathrm{r}}$'' $-\underline{\overline{.\backslash -}}\bullet$

Strona 2 z29

$\mathrm{E}\mathrm{M}\mathrm{A}\mathrm{P}-\mathrm{R}0_{-}100$





Zadanie 9. (0-3)

Funkcja $f$ jest określona wzorem $f(x)=\displaystyle \frac{3x^{2}-2x}{x^{2}+2x+8}$ dla $\mathrm{k}\mathrm{a}\dot{\mathrm{z}}$ dej liczby rzeczywistej $x.$

Punkt $P=(x_{0}$, 3$)$ nalez $\mathrm{y}$ do wykresu funkcji $f$. Oblicz $x_{0}$ oraz wyznacz równanie

stycznej do wykresu funkcji $f$ w punkcie $P.$
\begin{center}
\begin{tabular}{|l|l|l|l|}
\cline{2-4}
&	\multicolumn{1}{|l|}{Nr zadania}&	\multicolumn{1}{|l|}{$8.$}&	\multicolumn{1}{|l|}{ $9.$}	\\
\cline{2-4}
&	\multicolumn{1}{|l|}{Maks. liczba pkt}&	\multicolumn{1}{|l|}{$3$}&	\multicolumn{1}{|l|}{ $3$}	\\
\cline{2-4}
\multicolumn{1}{|l|}{egzaminator}&	\multicolumn{1}{|l|}{Uzyskana liczba pkt}&	\multicolumn{1}{|l|}{}&	\multicolumn{1}{|l|}{}	\\
\hline
\end{tabular}

\end{center}
$\mathrm{E}\mathrm{M}\mathrm{A}\mathrm{P}-\mathrm{R}0_{-}100$

Strona ll z29





Zadanie 10. $(0-4$\}

Rozwiqz nierównośč

$\displaystyle \sqrt{x^{2}+4x+4}<\frac{25}{3}-\sqrt{x^{2}-6x+9}$

{\it Wskazówka}: {\it skorzystaj z tego, ze} $\sqrt{a^{2}}=|a|$ {\it dla kazdei liczby rzeczywistei} $a.$

Strona 12 z29

$\mathrm{E}\mathrm{M}\mathrm{A}\mathrm{P}-\mathrm{R}0_{-}10$





Wypelnia

egzaminator

Nr zadania

Maks. liczba pkt

Uzyskana liczba pkt

10.

4

-RO-100

Strona 13 z29





Zadanie lt. $\langle 0-4$\}

Określamy kwadraty $K_{1}, K_{2}, K_{3}$, następujqco:

$\bullet K_{1}$ jest kwadratem o boku d\}ugości $a$

$\bullet K_{2}$ jest kwadratem, którego $\mathrm{k}\mathrm{a}\dot{\mathrm{z}}\mathrm{d}\mathrm{y}$ wierzcholek $\mathrm{l}\mathrm{e}\dot{\mathrm{z}}\mathrm{y}$ na innym boku kwadratu $K_{1}$

ten bok w stosunku 1 : 3

i dzieli

$\bullet K_{3}$ jest kwadratem, którego $\mathrm{k}\mathrm{a}\dot{\mathrm{z}}\mathrm{d}\mathrm{y}$ wierzcholek $\mathrm{l}\mathrm{e}\dot{\mathrm{z}}\mathrm{y}$ na innym boku kwadratu $K_{2}$ i dzieli

ten bok w stosunku 1 : 3

i ogólnie, dla $\mathrm{k}\mathrm{a}\dot{\mathrm{z}}$ dej liczby naturalnej $n\geq 2,$

$\bullet K_{n}$ jest kwadratem, którego $\mathrm{k}\mathrm{a}\dot{\mathrm{z}}\mathrm{d}\mathrm{y}$ wierzcholek $\mathrm{l}\mathrm{e}\dot{\mathrm{z}}\mathrm{y}$ na innym boku kwadratu $K_{n-1}$

i dzieli ten bok w stosunku 1 : 3.

Obwody wszystkich kwadratów określonych powyzej tworzq nieskończony ciqg

geometryczny.

Na rysunku przedstawiono kwadraty utworzone w sposób opisany powyzej.

{\it a}
\begin{center}
\includegraphics[width=58.824mm,height=58.872mm]{./F2_M_PR_M2023_page13_images/image001.eps}
\end{center}
{\it a}

Oblicz sume wszystkich wyrazów tego nieskończonego ciagu.

Strona 14 z29

$\mathrm{E}\mathrm{M}\mathrm{A}\mathrm{P}-\mathrm{R}0_{-}100$





Wypelnia

egzaminator

Nr zadania

Maks. liczba pkt

Uzyskana liczba pkt

11.

4

-RO-100

Strona 15 z29





Zadanie 12. $(0-4$\}

Rozwiqz równanie 3 $\sin^{2}x-\sin^{2}(2x)=0$ w przedziale $\langle\pi, 2\pi\rangle.$

Strona 16 z29

$\mathrm{E}\mathrm{M}\mathrm{A}\mathrm{P}-\mathrm{R}0_{-}10$





Wypelnia

egzaminator

Nr zadania

Maks. liczba pkt

Uzyskana liczba pkt

12.

4

-RO-100

Strona 17 z29





Zadanie 13. $(0-4$\}

Czworokqt ABCD, w którym $|BC|=4 \mathrm{i} |CD|=5$, jest opisany na okregu. Przekatna $AC$

tego czworokata tworzy z bokiem $BC$ kqt o mierze $60^{\mathrm{o}}$, natomiast z bokiem $AB-$ kqt ostry,

którego sinus jest równy $\displaystyle \frac{1}{4}$. Oblicz obwód $\mathrm{c}\mathrm{z}\mathrm{w}\mathrm{o}\mathrm{r}\mathrm{o}\mathrm{k}_{\mathrm{c}}$]$\mathrm{t}\mathrm{a}$ {\it ABCD}.

Strona 18 z29

$\mathrm{E}\mathrm{M}\mathrm{A}\mathrm{P}-\mathrm{R}0_{-}100$





Wypelnia

egzaminator

Nr zadania

Maks. liczba pkt

Uzyskana liczba pkt

13.

4

-RO-100

Strona 19 z29





Zadanie 14. $(0-4$\}

Danyjest sześcian ABCDEFGH o krawpdzi

dlugości 6. Punkt $S$ jest punktem przeciecia

przekqtnych $AH \mathrm{i}$ DE ściany bocznej ADHE

(zobacz rysunek).

Oblicz wysokośč trójkata SBH poprowadzona z punktu S na bok BH tego trójkata.

Strona 20 z29

$\mathrm{E}\mathrm{M}\mathrm{A}\mathrm{P}-\mathrm{R}0_{-}100$





Zadania egzaminacyine sq wydrukowane

na nastepnych stronach.

$\mathrm{E}\mathrm{M}\mathrm{A}\mathrm{P}-\mathrm{R}0_{-}100$

Strona 3 z29





Wypelnia

egzaminator

Nr zadania

Maks. liczba pkt

Uzyskana liczba pkt

14.

4

-RO-100

Strona 21 z29





Zadanie 15. $(0-5$\}

Wyznacz wszystkie wartości parametru $m\neq 2$, dla których równanie

$x^{2}+4x-\displaystyle \frac{m-3}{m-2}=0$

ma dwa rózne rozwiqzania rzeczywiste $x_{1}, x_{2}$ spelniajace warunek $x_{1}^{3}+x_{2}^{3}>-28.$

Strona 22 z29

$\mathrm{E}\mathrm{M}\mathrm{A}\mathrm{P}-\mathrm{R}0_{-}100$





Wypelnia

egzaminator

Nr zadania

Maks. liczba pkt

Uzyskana liczba pkt

15.

5

-RO-100

Strona 23 z29





Zadanie 16. (0-7)

Rozwazamy trójkqty $ABC$, w których $A=(0,0), B=(m,0)$, gdzie $m\in(4,+\infty),$

a wierzcholek $C \mathrm{l}\mathrm{e}\dot{\mathrm{z}}\mathrm{y}$ na prostej o równaniu $y=-2x$. Na boku $BC$ tego trójkqta $\mathrm{l}\mathrm{e}\dot{\mathrm{z}}\mathrm{y}$ punkt

$D=(3,2).$

a) Wykaz, $\dot{\mathrm{z}}\mathrm{e}$ dla $m\in(4,+\infty)$ pole $P$ trójkqta $ABC$, jako funkcja zmiennej $m$, wyraza $\mathrm{s}\mathrm{i}\mathrm{e}$

wzorem

$P(m)=\displaystyle \frac{m^{2}}{m-4}$

b) Oblicz t9 wartośč m, d1a której funkcja P osiaga wartośč najmniejszq. Wyznacz

równanie prostej BC, przy której funkcja F osiaga t9 najmniejszq wartośč.

Strona 24 z29

$\mathrm{E}\mathrm{M}\mathrm{A}\mathrm{P}-\mathrm{R}0_{-}100$





$1)0_{-}100$

Strona 25 z29





Wypelnia

egzaminator

Nr zadania

Maks. liczba pkt

Uzyskana liczba pkt

16.

7

Strona 26 z29

$\mathrm{E}\mathrm{M}\mathrm{A}\mathrm{P}-\mathrm{R}0_{-}10$





: {\it RU DNOPIS} \{{\it nie podlega ocenie}\}

$\mathrm{h}\mathrm{P}-\mathrm{R}0_{-}100$

Strona 27 z29





Strona 28 z29

$\mathrm{E}\mathrm{M}\mathrm{A}\mathrm{P}-\mathrm{R}0_{-}10$





$1)0_{-}100$

Strona 29 z29










$W$ {\it kazdym z zadań od} $f.$ {\it do 4. wybierz i zaznacz na karcie odpowiedzi poprawnq odpowiedz}'.

Zadanie $1_{p}(0-1)$

Granica $\displaystyle \lim_{x\rightarrow 1}\frac{x^{3}-1}{(x-1)(x+2)}$ jest równa

A. $(-1)$

B. 0

C. -31

D. l

Zadanie 2. (0-1)

Dane sq wektory $\vec{u}=[4,-5]$ oraz $\vec{v}=[-1,-5]$. Dlugośč wektora $\vec{u}-4\vec{v}$ jest równa

A. 7

B. 15

C. 17

D. 23

Zadanie 3. $(0-l\displaystyle \int$

Punkty $A, B, C, D, E \mathrm{l}\mathrm{e}\dot{\mathrm{z}}$ a na okregu o środku $S$. Miara $\ltimes \mathrm{a}\mathrm{t}\mathrm{a} BCD$ jest równa $110^{\mathrm{o}},$

a miara kqta $BDA$ jest równa $35^{\mathrm{o}}$ (zobacz rysunek).
\begin{center}
\includegraphics[width=77.316mm,height=77.160mm]{./F2_M_PR_M2023_page3_images/image001.eps}
\end{center}
{\it D  C}

$110^{\mathrm{o}}$

$35^{\mathrm{o}}$

$S_{\bullet}$

{\it E  B}

{\it A}

Wtedy kqt DEA ma miare równq

A. $100^{\mathrm{o}}$

B. $105^{\mathrm{o}}$

C. $110^{\mathrm{o}}$

D. $115^{\mathrm{o}}$

Zadanie 4. $\{0-1\}$

Dany jest zbiór trzynastu liczb \{1, 2, 3, 4, 5, 6, 7, 8, 9, 10, 11, 12, 13\}, z którego 1osujemy

jednocześnie dwie liczby. Wszystkich róznych sposobów wylosowania z tego zbioru dwóch

liczb, których iloczyn jest liczbq parzystq, jest

A. $\left(\begin{array}{l}
7\\
2
\end{array}\right)+49$

B. $\left(\begin{array}{l}
6\\
1
\end{array}\right)\cdot\left(\begin{array}{l}
7\\
1
\end{array}\right)+49$

C. $\left(\begin{array}{l}
13\\
2
\end{array}\right)-(_{2}^{7})$

D. $\left(\begin{array}{l}
13\\
2
\end{array}\right)-(_{2}^{6})$

Strona 4 z29

$\mathrm{E}\mathrm{M}\mathrm{A}\mathrm{P}-\mathrm{R}0_{-}100$















: {\it RU DNOPIS} \{{\it nie podlega ocenie}\}

$\mathrm{h}\mathrm{P}-\mathrm{R}0_{-}100$

Strona 5 z29





Zadanie 5. $(0-2$\}

Wielomian $W(x)=7x^{3}-9x^{2}+9x-2$ ma dokladnie jeden pierwiastek rzeczywisty.

Oblicz ten pierwiastek.

$\mathrm{W}$ ponizsze kratki wpisz kolejno-od lewej do prawej-pierwsza, drugq oraz trzeciq cyfr9 po

przecinku nieskończonego rozwiniecia dziesiptnego otrzymanego wyniku.
\begin{center}
\includegraphics[width=25.452mm,height=12.240mm]{./F2_M_PR_M2023_page5_images/image001.eps}
\end{center}
: {\it RU DNOPIS} \{{\it nie podlega ocenie}\}

Strona 6 z29

$\mathrm{E}\mathrm{M}\mathrm{A}\mathrm{P}-\mathrm{R}0_{-}100$





Zadanie 6. $\{0-3$)

Liczby rzeczywiste $x$ oraz $y$ spelniajqjednocześnie równanie $x+y=4$ i nierównośč

$x^{3}-x^{2}\mathrm{y}\leq x\mathrm{y}^{2}-y^{3}$. Wykaz, $\dot{\mathrm{z}}\mathrm{e} x=2$ oraz $y=2.$
\begin{center}
\begin{tabular}{|l|l|l|l|}
\cline{2-4}
&	\multicolumn{1}{|l|}{Nr zadania}&	\multicolumn{1}{|l|}{$5.$}&	\multicolumn{1}{|l|}{ $6.$}	\\
\cline{2-4}
&	\multicolumn{1}{|l|}{Maks. liczba pkt}&	\multicolumn{1}{|l|}{$2$}&	\multicolumn{1}{|l|}{ $3$}	\\
\cline{2-4}
\multicolumn{1}{|l|}{egzaminator}&	\multicolumn{1}{|l|}{Uzyskana liczba pkt}&	\multicolumn{1}{|l|}{}&	\multicolumn{1}{|l|}{}	\\
\hline
\end{tabular}

\end{center}
$\mathrm{E}\mathrm{M}\mathrm{A}\mathrm{P}-\mathrm{R}0_{-}100$

Strona 7 z29





Zadanie 7. (0-3)

Danyjest trójkqt prostokqtny $ABC$, w którym $|4ABC|=90^{\mathrm{o}}$ oraz $|4\mathrm{C}AB|=60^{\mathrm{o}}$ Punkty

$K \mathrm{i} L \mathrm{l}\mathrm{e}\dot{\mathrm{z}}$ a na bokach- odpowiednio -$AB \mathrm{i} BC$ tak, $\dot{\mathrm{z}}\mathrm{e} |BK|=|BL|=1$ (zobacz

rysunek). Odcinek $KL$ przecina wysokośč $BD$ tego trójkqta w punkcie $N$, a ponadto

$|AD|=2.$
\begin{center}
\includegraphics[width=133.704mm,height=81.432mm]{./F2_M_PR_M2023_page7_images/image001.eps}
\end{center}
{\it A}

$60^{\mathrm{o}}$  2

{\it D}

{\it K}

{\it N}

1

{\it C}

{\it B} l $L$

Wykaz, $\dot{\mathrm{z}}\mathrm{e} |ND|=\sqrt{3}+1.$

Strona 8 z29

$\mathrm{E}\mathrm{M}\mathrm{A}\mathrm{P}-\mathrm{R}0_{-}100$





Wypelnia

egzaminator

Nr zadania

Maks. liczba pkt

Uzyskana liczba pkt

7.

3

-RO-100

Strona 9 z29





Zadanie 8. (0-3)

$\mathrm{W}$ pojemniku jest siedem kul: pi9č ku1 bia1ych i dwie ku1e czarne. $\mathrm{Z}$ tego pojemnika losujemy

jednocześnie dwie kule bez zwracania. Nastppnie-z kul pozostalych w pojemniku-

losujemy jeszcze $\mathrm{j}\mathrm{e}\mathrm{d}\mathrm{h}_{\mathrm{c}1}$ ku19. Ob1icz prawdopodobieństwo wy1osowania ku1i czarnej w drugim

losowaniu.

Strona 10 z29

$\mathrm{E}\mathrm{M}\mathrm{A}\mathrm{P}-\mathrm{R}0_{-}100$







CENTRALNA

KOMISJA

EGZAMINACYJNA

KOD

WYPELNIA ZDAJACY

PESEL
\begin{center}
\includegraphics[width=21.900mm,height=10.164mm]{./F2_M_PR_M2024_page0_images/image001.eps}

\includegraphics[width=79.656mm,height=10.164mm]{./F2_M_PR_M2024_page0_images/image002.eps}
\end{center}
Egzamin maturalny

DATA: 15 maja 2024 r.

GODZINA R0ZP0CZECIA: 9:00

CZAS TRWANIA: $180 \displaystyle \min$ ut

Arkusz zawiera informacje prawnie chronione

do momentu rozpoczecia egzaminu.

{\it Miejsce na naklejke}.

{\it Sprawdz}', {\it czy kod na naklejce to}

e-100.

/{\it ezeli tak}- {\it przyklej naklejkq}.

/{\it ezeli nie}- {\it zgtoś to nauczycielowi}.

MAP-R0-100-2405

$\mathrm{W}\mathrm{Y}\Re$gLMlA 8$\mathrm{E}8\mathrm{P}\mathfrak{H}^{\aleph}\mathrm{L}\mathrm{M}\mathrm{A}\mathrm{D}\mathrm{Z}\mathrm{Q}\Re \mathrm{U}\mathrm{J}h\vartheta Y$

Uprawnienia $\mathrm{z}\mathrm{d}\mathrm{a}1\varepsilon$cego do:

\fbox{} dostosowania zasad oceniania

\fbox{} nieprzenoszenia odpowiedzi na karte.

LICZBA PUNKTÓW DO UZYSKANIA 50

Przed rozpoczeciem pracy z arkuszem egzaminacyjnym

1.

Sprawd $\acute{\mathrm{z}}$, czy nauczyciel przekazal Ci wlaściwy arkusz egzaminacyjny,

tj. arkusz we wlaściwej formule, z w[aściwego przedmiotu na wlaściwym

poziomie.

2.

$\mathrm{J}\mathrm{e}\dot{\mathrm{z}}$ eli przekazano Ci niew[aściwy arkusz- natychmiast zgloś to nauczycielowi.

Nie rozrywaj banderol.

3. $\mathrm{J}\mathrm{e}\dot{\mathrm{z}}$ eli przekazano Ci w[aściwy arkusz- rozerwij banderole po otrzymaniu

takiego polecenia od nauczyciela. Zapoznaj $\mathrm{s}\mathrm{i}\mathrm{e}$ z instrukcjq na stronie 2.

Uk\}ad graficzny

\copyright CKE 2022

$\Vert\Vert\Vert\Vert\Vert\Vert\Vert\Vert\Vert\Vert\Vert\Vert\Vert\Vert\Vert\Vert\Vert\Vert\Vert\Vert\Vert\Vert\Vert\Vert\Vert\Vert\Vert\Vert\Vert\Vert|$




lnstrukcja dla zdajqcego

l. Sprawdz', czy arkusz egzaminacyjny zawiera 29 stron (zadania $1-16$).

Ewentualny brak zgloś przewodniczqcemu zespolu nadzorujqcego egzamin.

2. Na pierwszej stronie arkusza oraz na karcie odpowiedzi wpisz swój numer PESEL

i przyklej naklejke z kodem.

3. Odpowiedzi do zadań $\mathrm{z}\mathrm{a}\mathrm{m}\mathrm{k}\mathrm{n}\mathrm{i}9$tych ($1-4)$ zaznacz na karcie odpowiedzi w cześci karty

przeznaczonej dla zdajacego. Zamaluj $\blacksquare$ pola do tego przeznaczone. $\mathrm{B}9\mathrm{d}\mathrm{n}\mathrm{e}$

zaznaczenie otocz kólkiem\copyright izaznacz wlaściwe.

4. $\mathrm{W}$ zadaniu 5. wpisz odpowiednie cyfry w kratki pod treścia zadania.

5. $\mathrm{P}\mathrm{a}\mathrm{m}\mathrm{i}9\mathrm{t}\mathrm{a}\mathrm{j}, \dot{\mathrm{z}}\mathrm{e}$ pominiecie argumentacji lub istotnych obliczeń w rozwiqzaniu zadania

otwartego (6-16) $\mathrm{m}\mathrm{o}\dot{\mathrm{z}}\mathrm{e}$ spowodowač, $\dot{\mathrm{z}}\mathrm{e}$ za to rozwiqzanie nie otrzymasz pelnej liczby

punktów.

6. Rozwiqzania zadań i odpowiedzi wpisuj w miejscu na to przeznaczonym.

7. Pisz czytelnie i $\mathrm{u}\dot{\mathrm{z}}$ ywaj tylko dlugopisu lub pióra z czarnym tuszem lub atramentem.

8. Nie $\mathrm{u}\dot{\mathrm{z}}$ ywaj korektora, a bledne zapisy wyra $\acute{\mathrm{z}}$ nie przekreśl.

9. Nie wpisuj $\dot{\mathrm{z}}$ adnych znaków w cześci przeznaczonej dla egzaminatora.

10. Pamietaj, $\dot{\mathrm{z}}\mathrm{e}$ zapisy w brudnopisie nie bedq oceniane.

11. $\mathrm{M}\mathrm{o}\dot{\mathrm{z}}$ esz korzystač z Wybranych wzoróvv matematycznych, cyrkla i linijki oraz kalkulatora

prostego. Upewnij $\mathrm{s}\mathrm{i}\mathrm{e}$, czy przekazano Ci broszur9 z ok1adka takq jak widoczna ponizej.

$\text{{\it á}}_{-,\rightarrow f'(^{\wedge}x_{0})}^{n_{è\mathrm{A}\cdot\alpha}h}$

$\rightarrow$2'$|.(${\it ra}$\vartheta\eta\hat{}\tilde{}\hat{}${\it h}A$+$`{\it r}$\grave{}|${\it ua}.$\approx\acute{}${\it g}.`{\it u}..'$|\Delta${\it h}A$\sqrt{}>${\it u}$\acute{}$30-('

$-\rightarrow 3$

$\mathrm{q},1\cdots\cdot 1\cup \mathrm{R} \varsigma..\vee\prime:\tilde{\mathrm{v}}\mathrm{k}r.7k\cdot(\mathrm{n}\rightarrow\prime.$

$\overline{\mathrm{w}u\mathrm{r}}$[‡@]$\mathrm{r}\mathrm{w} --\overline{\underline{\mathrm{R}\infty-}},\bullet$

Strona 2 z29

$\mathrm{E}\mathrm{M}\mathrm{A}\mathrm{P}-\mathrm{R}0_{-}100$





Zadanie 9. (0-3)

Funkcja f jest określona wzorem

$f(x)=\displaystyle \frac{x^{3}-3x+2}{\chi}$

dla $\mathrm{k}\mathrm{a}\dot{\mathrm{z}}$ dej liczby rzeczywistej $x$ róznej od zera. Punkt $P$, o pierwszej wspólrz9dnej

równej 2, na1ez $\mathrm{y}$ do wykresu funkcji $f$. Prosta o równaniu $y=ax+b$ jest styczna do

wykresu funkcji $f$ w punkcie $P.$

Oblicz wspólczynniki $a$ oraz $b$ w równaniu tej stycznej.
\begin{center}
\begin{tabular}{|l|l|l|l|}
\cline{2-4}
&	\multicolumn{1}{|l|}{Nr zadania}&	\multicolumn{1}{|l|}{$8.$}&	\multicolumn{1}{|l|}{ $9.$}	\\
\cline{2-4}
&	\multicolumn{1}{|l|}{Maks. liczba pkt}&	\multicolumn{1}{|l|}{$3$}&	\multicolumn{1}{|l|}{ $3$}	\\
\cline{2-4}
\multicolumn{1}{|l|}{egzaminator}&	\multicolumn{1}{|l|}{Uzyskana liczba pkt}&	\multicolumn{1}{|l|}{}&	\multicolumn{1}{|l|}{}	\\
\hline
\end{tabular}

\end{center}
$\mathrm{E}\mathrm{M}\mathrm{A}\mathrm{P}-\mathrm{R}0_{-}100$

Strona ll z29





Zadanie 10. (0-3)

Spośród wszystkich liczb naturalnych sześciocyfrowych, których wszystkie cyfry naleza do

zbioru \{1, 2, 3, 4, 5, 6, 7, 8\}, 1osujemy jednq. Wy1osowanie $\mathrm{k}\mathrm{a}\dot{\mathrm{z}}$ dej z tych liczb jest jednakowo

prawdopodobne.

Oblicz prawdopodobieństwo zdarzenia polegajqcego na tym, $\dot{\mathrm{z}}\mathrm{e}$ wylosujemy liczbe, która

ma nastppujqca wlasnośč: kolejne cyfry tej liczby (liczqc od lewej strony) $\mathrm{t}\mathrm{w}\mathrm{o}\mathrm{r}\mathrm{Z}_{\mathrm{c}1}-\mathrm{w}$ podanej

kolejności- sześciowyrazowy ciqg malejqcy.

Strona 12 z29

$\mathrm{E}\mathrm{M}\mathrm{A}\mathrm{P}-\mathrm{R}0_{-}100$





Wypelnia

egzaminator

Nr zadania

Maks. liczba pkt

Uzyskana liczba pkt

10.

3

-RO-100

Strona 13 z29





Zadanie tl. $(0-4$\}

Trzywyrazowy ciag $(x,y,z)$ jest geometryczny i rosnqcy. Suma wyrazów tego ciqgu jest

równa 105. Liczby $x, y$ oraz $z$ sq- odpowiednio-pierwszym, drugim oraz szóstym

wyrazem ciqgu arytmetycznego $(a_{n})$, określonego dla $\mathrm{k}\mathrm{a}\dot{\mathrm{z}}$ dej liczby naturalnej $n\geq 1.$

Oblicz $x, \mathrm{y}$ oraz $z.$

Strona 14 z29

$\mathrm{E}\mathrm{M}\mathrm{A}\mathrm{P}-\mathrm{R}0_{-}100$





Wypelnia

egzaminator

Nr zadania

Maks. liczba pkt

Uzyskana liczba pkt

11.

4

-RO-100

Strona 15 z29





Zadanie 12. $(0-4$\}

Rozwiqz równanie

$\sin(2x)+\cos(2x)=1+\sin x-\cos x$

w zbiorze $\langle 0,2\pi\rangle.$

Strona 16 z29

$\mathrm{E}\mathrm{M}\mathrm{A}\mathrm{P}-\mathrm{R}0_{-}10$





Wypelnia

egzaminator

Nr zadania

Maks. liczba pkt

Uzyskana liczba pkt

12.

4

-RO-100

Strona 17 z29





Zadanie 13. $(0-4$\}

Promień okregu opisanego na trójkqcie $ABC$ jest równy 17. Najdlu $\dot{\mathrm{z}}$ szym bokiem tego

trójkata jest bok $AC$, a dlugości dwóch pozostalych boków sq równe $|AB|=30$ oraz

$|BC|=17$. Oblicz miar9 kqta $BAC$ oraz dlugośč boku $AC$ tego trójkqta.

Strona 18 z29

$\mathrm{E}\mathrm{M}\mathrm{A}\mathrm{P}-\mathrm{R}0_{-}100$





Wypelnia

egzaminator

Nr zadania

Maks. liczba pkt

Uzyskana liczba pkt

13.

4

-RO-100

Strona 19 z29





Zadanie 14. $(0-5$\}

$\acute{\mathrm{S}}$ rodek $S$ okregu o promieniu $\sqrt{5} \mathrm{l}\mathrm{e}\dot{\mathrm{z}}\mathrm{y}$ na prostej o równaniu $y=x+1$. Przez punkt

$A=(1,2)$, którego odleglośč od punktu $S$ jest wieksza od $\sqrt{5}$, poprowadzono dwie proste

styczne do tego okregu w punktach- odpowiednio -$B \mathrm{i} C$. Pole czworokata ABSC jest

równe 15.

Oblicz wspólrzedne punktu $S$. Rozwaz wszystkie przypadki.

Strona 20 z29

$\mathrm{E}\mathrm{M}\mathrm{A}\mathrm{P}-\mathrm{R}0_{-}100$





Zadania egzaminacyine sq wydrukowane

na nastepnych stronach.

$\mathrm{E}\mathrm{M}\mathrm{A}\mathrm{P}-\mathrm{R}0_{-}100$

Strona 3 z29





Wypelnia

egzaminator

Nr zadania

Maks. liczba pkt

Uzyskana liczba pkt

14.

5

-RO-100

Strona 21 z29





Zadanie 15. (0-6)

Wyznacz wszystkie wartości parametru $m$, dla których równanie

$x^{2}-(3m+1)\cdot x+2m^{2}+m+1=0$

ma dwa rózne rozwiazania rzeczywiste $x_{1}, x_{2}$ spelniajace warunek

$x_{1}^{3}+x_{2}^{3}+3\cdot x_{1}\cdot x_{2}\cdot(x_{1}+x_{2}-3)\leq 3m-7$

Strona 22 z29

$\mathrm{E}\mathrm{M}\mathrm{A}\mathrm{P}-\mathrm{R}0_{-}10$





$1)0_{-}100$

Strona 23 z29





Wypelnia

egzaminator

Nr zadania

Maks. liczba pkt

Uzyskana liczba pkt

15.

6

Strona 24 z29

$\mathrm{E}\mathrm{M}\mathrm{A}\mathrm{P}-\mathrm{R}0_{-}10$





Zadanie 16. (0-6)

Rozwazamy wszystkie graniastoslupy prawidlowe trójkqtne o objetości 3456, których

$\mathrm{k}\mathrm{r}\mathrm{a}\mathrm{w}9^{\mathrm{d}\acute{\mathrm{Z}}}$ podstawy ma dlugośč nie wipkszq $\mathrm{n}\mathrm{i}\dot{\mathrm{z}} 8\sqrt{3}.$

a)

Wykaz, $\dot{\mathrm{z}}\mathrm{e}$ pole $P$ powierzchni calkowitej graniastoslupa w zalezności od dlugości $a$

$\mathrm{k}\mathrm{r}\mathrm{a}\mathrm{w}9^{\mathrm{d}\mathrm{z}\mathrm{i}}$ podstawy $\mathrm{g}\mathrm{r}\mathrm{a}\mathrm{n}\mathrm{i}\mathrm{a}\mathrm{s}\mathrm{t}\mathrm{o}\mathrm{s}\mathrm{u}\mathrm{p}\mathrm{a}$ jest określone wzorem

$P(a)=\displaystyle \frac{a^{2}\cdot\sqrt{3}}{2}+\frac{13824\sqrt{3}}{a}$

b) Pole $P$ powierzchni calkowitej graniastoslupa w zalezności od d$\dagger$ugości $a$ krawedzi

podstawy graniastoslupa jest określone wzorem

$P(a)=\displaystyle \frac{a^{2}\cdot\sqrt{3}}{2}+\frac{13824\sqrt{3}}{a}$

dla $a\in(0,8\sqrt{3}\rangle.$

Wyznacz dlugość krawedzi podstawy tego z rozwazanych graniastoslupów, którego pole

powierzchni calkowitej jest najmniejsze. Oblicz to najmniejsze pole.

$\mathrm{E}\mathrm{M}\mathrm{A}\mathrm{P}-\mathrm{R}0_{-}100$

Strona 25 z29





Strona 26 z29

$\mathrm{E}\mathrm{M}\mathrm{A}\mathrm{P}-\mathrm{R}0_{-}10$





Wypelnia

egzaminator

Nr zadania

Maks. liczba pkt

Uzyskana liczba pkt

16.

6

-RO-100

Strona 27 z29





: {\it RU DNOPIS} \{{\it nie podlega ocenie}\}

Strona 28z 29

$\mathrm{E}\mathrm{M}\mathrm{A}\mathrm{P}-\mathrm{R}0_{-}10$





$1)0_{-}100$

Strona 29 z29










{\it W kazdym z zadań od f. do 4. wybierz i zaznacz na karcie odpowiedzi poprawnq odpowiedz}'.

Zadanie $1_{p}(0-1)$

Odleglośč punktu $A=(6,2)$ od prostej o równaniu $5x-12y+1=0$ jest równa

A. $\displaystyle \frac{7}{13}$

B. $\displaystyle \frac{7}{12}$

C. $\displaystyle \frac{5}{12}$

D. $\displaystyle \frac{12}{13}$

Zadanie 2. (0-1)

Równanie $|2x-4|=3x+1$ w zbiorze liczb rzeczywistych

A. nie ma rozwiazań.

B. ma dokladnie jedno rozwiazanie.

C. ma dokladnie dwa rozwiqzania.

D. ma dokladnie cztery rozwiazania.

Zadanie 3. $(0-l\displaystyle \int$

Funkcja $f$ jest określona wzorem $f(x)=|-(x+2)^{3}+5|$ dla $\mathrm{k}\mathrm{a}\dot{\mathrm{z}}$ dej liczby

rzeczywistej $x$. Zbiorem wartości funkcji $f$ jest przedzial

A. $\langle-2, +\infty)$

B. $\langle 0, +\infty)$

C. $\langle 3, +\infty)$

D. $\langle 5, +\infty)$

Zadanie 4. $\{0-1\}$

Granica $\displaystyle \lim_{\chi\rightarrow+\infty}\frac{1+3a+2ax+ax^{3}}{3+4x+5x^{2}+5x^{3}}$ jest równa 3. Wtedy

A. $a=3$

B. $a=9$

C. $a=15$

D. $a=21$

Strona 4 z29

$\mathrm{E}\mathrm{M}\mathrm{A}\mathrm{P}-\mathrm{R}0_{-}100$















: {\it RU DNOPIS} \{{\it nie podlega ocenie}\}

$\mathrm{h}\mathrm{P}-\mathrm{R}0_{-}100$

Strona 5 z29





Zadanie 5. $(0-2$\}

Wielomian $W(x)=8x^{3}+14x^{2}+5x+3$ jest iloczynem wielomianów $P(x)=2x+3$

oraz $Q(x)=ax^{2}+bx+c.$

$\mathrm{W}$ ponizsze kratki wpisz kolejno-od lewej do prawej- wartości wspólczynników: $a, b$

oraz $c.$
\begin{center}
\includegraphics[width=25.452mm,height=12.240mm]{./F2_M_PR_M2024_page5_images/image001.eps}
\end{center}
: {\it RU DNOPIS} \{{\it nie podlega ocenie}\}

Strona 6 z29

$\mathrm{E}\mathrm{M}\mathrm{A}\mathrm{P}-\mathrm{R}0_{-}100$





Zadanie 6. $\{0-3$)

Wykaz, $\dot{\mathrm{z}}\mathrm{e}\mathrm{j}\mathrm{e}\dot{\mathrm{z}}$ eli log54$=a$ oraz log43 $=b$, to log1280$=\displaystyle \frac{2a+1}{a\cdot(1+b)}$
\begin{center}
\begin{tabular}{|l|l|l|l|}
\cline{2-4}
&	\multicolumn{1}{|l|}{Nr zadania}&	\multicolumn{1}{|l|}{$5.$}&	\multicolumn{1}{|l|}{ $6.$}	\\
\cline{2-4}
&	\multicolumn{1}{|l|}{Maks. liczba pkt}&	\multicolumn{1}{|l|}{$2$}&	\multicolumn{1}{|l|}{ $3$}	\\
\cline{2-4}
\multicolumn{1}{|l|}{egzaminator}&	\multicolumn{1}{|l|}{Uzyskana liczba pkt}&	\multicolumn{1}{|l|}{}&	\multicolumn{1}{|l|}{}	\\
\hline
\end{tabular}

\end{center}
$\mathrm{E}\mathrm{M}\mathrm{A}\mathrm{P}-\mathrm{R}0_{-}100$

Strona 7 z29





Zadanie 7. $(0-3$\}

Danyjest $\mathrm{c}\mathrm{z}\mathrm{w}\mathrm{o}\mathrm{r}\mathrm{o}\mathrm{k}_{\mathrm{c}}$]$\mathrm{t}$ wypukly ABCD. $\mathrm{P}\mathrm{r}\mathrm{z}\mathrm{e}\mathrm{k}_{\mathrm{c}}$]$\mathrm{t}\mathrm{n}\mathrm{e} AC$ oraz $BD$ tego czworokqta przecinajq

si9 w punkcie $S.$

Wykaz, $\dot{\mathrm{z}}\mathrm{e}\mathrm{j}\mathrm{e}\dot{\mathrm{z}}$ eli $\displaystyle \frac{|AS|}{|DS|}=\frac{|BS|}{|CS|}$, to na czworokqcie ABCD $\mathrm{m}\mathrm{o}\dot{\mathrm{z}}$ na opisač okrqg.

Strona 8 z29

$\mathrm{E}\mathrm{M}\mathrm{A}\mathrm{P}-\mathrm{R}0_{-}100$





Wypelnia

egzaminator

Nr zadania

Maks. liczba pkt

Uzyskana liczba pkt

7.

3

-RO-100

Strona 9 z29





Zadanie 8. (0-3)

Rozwazamy wszystkie liczby naturalne, w których zapisie dziesietnym nie powtarza sie

jakakolwiek cyfra oraz dokladnie trzy cyfry sq nieparzyste i dokladnie dwie cyfry sq parzyste.

Oblicz, ile jest wszystkich takich liczb.

Strona 10 z29

$\mathrm{E}\mathrm{M}\mathrm{A}\mathrm{P}-\mathrm{R}0_{-}100$







CENTRALNA

KOMISJA

EGZAMINACYJNA

Arkusz zawiera informacje prawnie chronione

do momentu rozpoczecia egzaminu.

KOD

WYPELNIA ZOAJACY

PESEL

{\it Miejsce na naklejke}.

{\it Sprawdz}', {\it czy kod na naklejce to}

M-100.
\begin{center}
\includegraphics[width=21.900mm,height=10.164mm]{./F3_M_PP_M2023_page0_images/image001.eps}

\includegraphics[width=79.656mm,height=10.164mm]{./F3_M_PP_M2023_page0_images/image002.eps}
\end{center}
/{\it ezeli tak}- {\it przyklej naklejkq}.

/{\it ezeli nie}- {\it zgtoś to nauczycielowi}.

Egzamin maturalny

$\displaystyle \int$
\begin{center}
\includegraphics[width=193.344mm,height=75.792mm]{./F3_M_PP_M2023_page0_images/image003.eps}
\end{center}
Poziom  podstawowy

{\it Symbo arkusza}

MMAP-P0-100-2305

DATA: 8 maja 2023 r.

GODZINA R0ZP0CZECIA: 9:00

CZAS TRWANIA: $180 \displaystyle \min$ ut

WYPELNIA ZESPÓL NADZORUJACY

Uprawnienia $\mathrm{z}\mathrm{d}\mathrm{a}\mathrm{j}_{8}$cego do:

\fbox{} dostosowania zasad oceniania

\fbox{} dostosowania w zw. z dyskalkulia

\fbox{} nieprzenoszenia zaznaczeń na karte.

LICZBA PUNKTÓW DO UZYSKANIA 46

Przed rozpoczeciem pracy z arkuszem egzaminacyjnym

1.

Sprawd $\acute{\mathrm{z}}$, czy nauczyciel przekazal Ci wlaściwy arkusz egzaminacyjny,

tj. arkusz we wlaściwej formule, z w[aściwego przedmiotu na wlaściwym

poziomie.

2.

$\mathrm{J}\mathrm{e}\dot{\mathrm{z}}$ eli przekazano Ci niew[aściwy arkusz- natychmiast zgloś to nauczycielowi.

Nie rozrywaj banderol.

3. $\mathrm{J}\mathrm{e}\dot{\mathrm{z}}$ eli przekazano Ci w[aściwy arkusz- rozerwij banderole po otrzymaniu

takiego polecenia od nauczyciela. Zapoznaj $\mathrm{s}\mathrm{i}\mathrm{e}$ z instrukcjq na stronie 2.

$\mathrm{U}\mathrm{k}\}\mathrm{a}\mathrm{d}$ graficzny

\copyright CKE 2022 $\bullet$

$\Vert\Vert\Vert\Vert\Vert\Vert\Vert\Vert\Vert\Vert\Vert\Vert\Vert\Vert\Vert\Vert\Vert\Vert\Vert\Vert\Vert\Vert\Vert\Vert\Vert\Vert\Vert\Vert\Vert\Vert|$




lnstrukcja dla zdajqcego

l. Sprawdz', czy arkusz egzaminacyjny zawiera 31 stron (zadania $1-31$).

Ewentualny brak zgloś przewodniczqcemu zespolu nadzorujqcego egzamin.

2. Na pierwszej stronie arkusza oraz na karcie odpowiedzi wpisz swój numer PESEL

i przyklej naklejke z kodem.

3. Symbol ${}_{1\mathrm{g}}P$ zamieszczony w naglówku zadania oznacza, $\dot{\mathrm{z}}\mathrm{e}$ rozwiqzanie zadania

zamknietego musisz przenieśč na karte odpowiedzi.

4. Odpowiedzi do zadań zamknietych zaznacz na karcie odpowiedzi w cześci karty

przeznaczonej dla zdajqcego. Zamaluj $\blacksquare$ pola do tego przeznaczone. Bledne

zaznaczenie otocz kólkiem \copyright i zaznacz wlaściwe.

5. $\mathrm{P}\mathrm{a}\mathrm{m}\mathrm{i}_{9}\mathrm{t}\mathrm{a}\mathrm{j}, \dot{\mathrm{z}}\mathrm{e}$ pominiecie argumentacji lub istotnych obliczeń w rozwiqzaniu zadania

otwartego $\mathrm{m}\mathrm{o}\dot{\mathrm{z}}\mathrm{e}$ spowodować, $\dot{\mathrm{z}}\mathrm{e}$ za to rozwiazanie nie otrzymasz pelnej liczby punktów.

6. Rozwiqzania zadań i odpowiedzi wpisuj w miejscu na to przeznaczonym.

7. Pisz czytelnie i $\mathrm{u}\dot{\mathrm{z}}$ ywaj tylko dlugopisu lub pióra z czarnym tuszem lub atramentem.

8. Nie $\mathrm{u}\dot{\mathrm{z}}$ ywaj korektora, a bledne zapisy wyra $\acute{\mathrm{z}}$ nie przekreśl.

9. Nie wpisuj $\dot{\mathrm{z}}$ adnych znaków w tabelkach przeznaczonych dla egzaminatora.

Tabelki umieszczone sa na marginesie przy odpowiednich zadaniach.

10. Pamietaj, $\dot{\mathrm{z}}\mathrm{e}$ zapisy w brudnopisie nie $\mathrm{b}9\mathrm{d}\mathrm{a}$ oceniane.

11. $\mathrm{M}\mathrm{o}\dot{\mathrm{z}}$ esz korzystač z Wybranych wzorów matematycznych, cyrkla i linijki oraz kalkulatora

prostego. Upewnij si9, czy przekazano Ci broszure z ok1adka taka jak widoczna ponizej.

Strona 2 z31

$\mathrm{M}\mathrm{M}\mathrm{A}\mathrm{P}-\mathrm{P}0_{-}100$





Zadanie Y\S$*$(0-2)

Danyjest $\mathrm{p}\mathrm{r}\mathrm{o}\mathrm{s}\mathrm{t}\mathrm{o}\mathrm{k}_{\mathrm{c}1}\mathrm{t}$ o bokach dlugości a $\mathrm{i} b$, gdzie $a>b$. Obwód tego $\mathrm{p}\mathrm{r}\mathrm{o}\mathrm{s}\mathrm{t}\mathrm{o}\mathrm{k}_{\mathrm{c}}$]$\mathrm{t}\mathrm{a}$ jest

równy 30. Jeden z boków prostokqta jest o 5 krótszy od drugiego.

Uzupe[nij zdanie. Wybierz dwie w[aściwe odpowiedzi spośród oznaczonych literami

A-F i wpisz te litery w wykropkowanych miejscach.

Zalezności miedzy dlugościami boków tego prostokqta zapisano w ukladach równań

oznaczonych literami:

oraz

A. 

B. 

C. 

D. 

E. 

F. 

{\it Brud}$\underline{no}\underline{\sqrt{}is}_{-} -$

$\mathrm{M}\mathrm{M}\mathrm{A}\mathrm{P}-\mathrm{P}0_{-}100$

Strona ll z31





Zadanie 82.

$\mathrm{W}$ kartezjańskim ukladzie wspólrzednych $(x,y)$ narysowano wykres funkcji $y=f(x)$

(zobacz rysunek).

Zadanie 02.1. $\{0-8\} \beta$

Dokończ zdanie. Wybierz w[aściwq odpowied $\acute{\mathrm{z}}$ spośród podanych.

Dziedzinq funkcji $f$ jest zbiór

A. [-6, 5]

B. $(-6,5)$

C. $(-3,5]$

D. [-3, 5]

{\it Brudnopis}

Zadanie 82.2. (0-\S) ff

Dokończ zdanie. Wybierz w[aściwq odpowied $\acute{\mathrm{z}}$ spośród podanych.

Najwi9ksza wartośč funkcji f w przedzia1e [-4, 1] jest równa

A. 0

B. l

C. 2

D. 5

{\it Brudnopis}

Strona 12 z31

$\mathrm{M}\mathrm{M}\mathrm{A}\mathrm{P}-\mathrm{P}0_{-}100$





Zädanie 12,3. (0-\S)mp

Dokończ zdanie. Wybierz w[aściwq odpowied $\acute{\mathrm{z}}$ spośród podanych.

Funkcja f jest malejaca w zbiorze

A. $[-6,-3)$

B. [-3, 1]

{\it Brudnopis}

C. (1, 2]

D. [2, 5]

$\mathrm{Z}\mathrm{a}\mathrm{d}\mathrm{a}\mathrm{n}\dot{\mathrm{l}}\mathrm{e}13_{*}(0-9) \beta$

Funkcja liniowa $f$ jest określona wzorem

$f(x)=ax+b$, gdzie $a \mathrm{i} b$ sa pewnymi

liczbami rzeczywistymi. Na rysunku obok

przedstawiono fragment wykresu funkcji $f$

w kartezjańskim ukladzie wspólrzednych $(x,y).$
\begin{center}
\includegraphics[width=82.908mm,height=69.540mm]{./F3_M_PP_M2023_page12_images/image001.eps}
\end{center}
{\it y}

1

0 1  $\chi$

$y=f(x)$

Dokończ zdanie. Wybierz w[aściwq odpowied $\acute{\mathrm{z}}$ spośród podanych.

Liczba a oraz liczba b we wzorze funkcji f spelniaja warunki:

A. $a>0 \mathrm{i} b>0.$

B. $a>0 \mathrm{i} b<0.$

C. $a<0 \mathrm{i} b>0.$

D. $a<0 \mathrm{i} b<0.$

{\it Brudnopis}

$\mathrm{M}\mathrm{M}\mathrm{A}\mathrm{P}-\mathrm{P}0_{-}100$

Strona 13 z31





Zädanie 14. \{0-\S)

$\beta$

Jednym z miejsc zerowych funkcji kwadratowej $f$ jest liczba $(-5)$. Pierwsza wspólrz9dna

wierzcholka paraboli, $\mathrm{b}_{9}$dacej wykresem funkcji $f$, jest równa 3.

Dokończ zdanie. Wybierz w[aściwq odpowied $\acute{\mathrm{z}}$ spośród podanych.

Drugim miejscem zerowym funkcji f jest liczba

A. ll

B. l

C. $(-1)$

D. $(-13)$

{\it Brudnopis}

$| 1$

Strona 14 z31

$\mathrm{M}\mathrm{M}\mathrm{A}\mathrm{P}-\mathrm{P}0_{-}100$





Zädanie 95. (0-\S) $\beta$

Ciqg $(a_{n})$ jest określony wzorem $a_{n}=2^{n}$

$(n+1)$ dla $\mathrm{k}\mathrm{a}\dot{\mathrm{z}}$ dej liczby naturalnej $n\geq 1.$

Dokończ zdanie. Wybierz wlaściwq odpowied $\acute{\mathrm{z}}$ spośród podanych.

Wyraz $a_{4}$ jest równy

A. 64

B. 40

C. 48

D. 80

{\it Brudnopis}

$1 -$

Zadanie 86. (0-1) $\beta$

Trzywyrazowy ciag $($27, 9, $a-1)$ jest geometryczny.

Dokończ zdanie. Wybierz w[aściwq odpowied $\acute{\mathrm{z}}$ spośród podanych.

Liczba a jest równa

A. 3

B. 0

{\it Brudnopis}

4

D. 2

$\mathrm{M}\mathrm{M}\mathrm{A}\mathrm{P}-\mathrm{P}0_{-}100$

Strona 15 z31





Zadanie $\mathrm{t}7. (0-2)$

Pan Stanislaw splacil pozyczk9 w wysokości 8910 z1 w osiemnastu ratach. $\mathrm{K}\mathrm{a}\dot{\mathrm{z}}$ da kolejna

rata byla mniejsza od poprzedniej o 30 z1.

Oblicz kwote pierwszei raty. Zapisz obliczenia.

1

1

Strona 16 z31

$\mathrm{M}\mathrm{M}\mathrm{A}\mathrm{P}-\mathrm{P}0_{-}100$





Zädanie 1\S. \{0-\S) $\beta$

$\mathrm{W}$ kartezjańskim ukladzie wspólrzednych $(x,y)$ zaznaczono kqt $\alpha$ o wierzcholku

w punkcie $0=(0,0)$. Jedno z ramion tego kqta pokrywa si9 z dodatniq pó1osiq $0x,$

a drugie przechodzi przez punkt $P=(-3,1)$ (zobacz rysunek).
\begin{center}
\includegraphics[width=78.228mm,height=48.768mm]{./F3_M_PP_M2023_page16_images/image001.eps}
\end{center}
{\it y}

$P=(-3,1)$

$\alpha$

{\it 0}

$-3 -2 -1$

$-1$

1 2  3  $\chi$

Dokończ zdanie. Wybierz wlaściwq odpowied $\acute{\mathrm{z}}$ spośród podanych.

Tangens kqta $\alpha$ jest równy

A. -$\sqrt{}$110

B. $(-\displaystyle \frac{3}{\sqrt{10}})$

C. $(-\displaystyle \frac{3}{1})$

D. $(-\displaystyle \frac{1}{3})$

$Brudno\sqrt{}is$

-

Zadanie 19$*$(0-\S) $p$

Dokończ zdanie. Wybierz w[aściwq odpowied $\acute{\mathrm{z}}$ spośród podanych.

Dla $\mathrm{k}\mathrm{a}\dot{\mathrm{z}}$ dego kqta ostrego $\alpha$ wyrazenie $\sin^{4}\alpha +\sin^{2}\alpha\cdot\cos^{2}\alpha$ jest równe

A. $\sin^{2}\alpha$

B. $\sin^{6}\alpha\cdot\cos^{2}\alpha$

C. $\sin^{4}\alpha+1$

D. $\sin^{2}\alpha\cdot(\sin\alpha+\cos\alpha)\cdot(\sin\alpha-\cos\alpha)$

{\it Brudnopis}

$\mathrm{M}\mathrm{M}\mathrm{A}\mathrm{P}-\mathrm{P}0_{-}100$

Strona 17 z31





Zädanie 20. (0-\S) $\beta$

$\mathrm{W}$ rombie o boku dlugości $6\sqrt{2}$ kqt rozwarty ma miar9 $150^{\mathrm{o}}$

Dokończ zdanie. Wybierz wlaściwq odpowied $\acute{\mathrm{z}}$ spośród podanych.

lloczyn dlugości przekqtnych tego rombu jest równy

A. 24

B. 72

C. 36

D. $36\sqrt{2}$

{\it Brudnopis}

Zadan$\mathrm{e}2l. (0\infty 1)$

H $\beta$

Punkty $A, B, C \mathrm{l}\mathrm{e}\dot{\mathrm{z}}$ a na okr gu o środku w punkcie 0.

$\mathrm{K} \mathrm{t} AC0$ ma miar $70^{\mathrm{o}}$ (zobacz rysunek).
\begin{center}
\includegraphics[width=68.124mm,height=73.104mm]{./F3_M_PP_M2023_page17_images/image001.eps}
\end{center}
{\it B}

{\it 0}

{\it C}

{\it A}

odpowied $\acute{\mathrm{z}}$ spośród podanych.

Dokończ zdanie.

Wybierz w[aściw

Miara kata ostrego ABC jest równa

A. $10^{\mathrm{o}}$

B. $20^{\mathrm{o}}$

C. $35^{\mathrm{o}}$

D. $40^{\mathrm{o}}$

{\it Brudnopis}

Strona 18 z31

$\mathrm{M}\mathrm{M}\mathrm{A}\mathrm{P}-\mathrm{P}0_{-}100$





Zadanie 22. (0-2)

Trójkqty prostokatne $T_{1}$ i $T_{2}$ sq podobne. Przyprostokqtne trójkqta $T_{1}$ maja

dlugości 5 $\mathrm{i} 12$. Przeciwprostokatna trójkata $T_{2}$ ma dlugośč 26.

Oblicz pole tróikqta $T_{2}$. Zapisz obliczenia.

$\overline{1}$

$1-$

$\mathrm{M}\mathrm{M}\mathrm{A}\mathrm{P}-\mathrm{P}0_{-}100$

Strona 19 z31





Zädanie 23. \{0-\S) $\beta$

$\mathrm{W}$ kartezjańskim ukladzie wspólrzednych $(x,y)$ dane sa proste $k$ oraz $l$ o równaniach

{\it k}:

{\it y}$=$ -32 $\chi$

{\it l}:

$y=-\displaystyle \frac{3}{2}x+13$

Dokończ zdanie. Wybierz odpowied $\acute{\mathrm{z}}$ A albo $\mathrm{B}$ oraz $\mathrm{o}\mathrm{d}\mathrm{p}\mathrm{o}\mathrm{w}\mathrm{i}\mathrm{e}\mathrm{d}\acute{\mathrm{z}}1.$, 2. albo 3.

Proste $k$ oraz $l$

A. sq prostopadle

1. $(-6,-4)$

i przecinajq si9 w punkcie P o wspó1rzednych 2.

(6, 4)

B.

nie sq

prostopadle

3. $(-6,4)$

{\it Brud}$\underline{no}\underline{\sqrt{}is}_{-} -$

-

Strona 20 z31

$\mathrm{M}\mathrm{M}\mathrm{A}\mathrm{P}-\mathrm{P}0_{-}100$





Zadania egzaminacyjne sq wydrukowane

na nastepnych stronach.

$\mathrm{M}\mathrm{M}\mathrm{A}\mathrm{P}-\mathrm{P}0_{-}100$

Strona 3 z31





Zädanie 24. \{0-\S) $\beta$

$\mathrm{W}$ kartezjańskim ukladzie wspólrzednych $(x,y)$ dana jest prosta $k$ o równaniu

$y=-\displaystyle \frac{1}{3}x+2$

Dokończ zdanie. Wybierz w[aściwq odpowied $\acute{\mathrm{z}}$ spośród podanych.

Prosta o równaniu $y=ax+b$ jest równolegla do prostej $k$ i przechodzi przez

punkt $P=(3,5)$, gdy

A. $a=3 \mathrm{i} b=4.$

B. $a=-\displaystyle \frac{1}{3} \mathrm{i} b=4.$

C. $a=3 \mathrm{i} b=-4.$

D. $a=-\displaystyle \frac{1}{3} \mathrm{i} b=6.$

{\it Brud}$\underline{no}\underline{\sqrt{}is}_{-} -$

-

Zadanie 25. (0-{\$}) $\mathrm{R} \beta$

Dany jest graniastoslup prawidlowy czworokqtny, w którym krawedz' podstawy ma

dlugośč 15. Przekqtna graniastos1upa jest nachy1ona do p1aszczyzny podstawy pod

kqtem $\alpha$ takim, $\dot{\mathrm{z}}\mathrm{e} \displaystyle \cos\alpha=\frac{\sqrt{2}}{3}$

Dokończ zdanie. Wybierz w[aściwq odpowied $\acute{\mathrm{z}}$ spośród podanych.

Dlugośč przekqtnej tego graniastoslupa jest równa

A. $15\sqrt{2}$

B. 45

C. $5\sqrt{2}$

D. 10

{\it Brudnopis}

$\mathrm{M}\mathrm{M}\mathrm{A}\mathrm{P}-\mathrm{P}0_{-}100$

Strona 21 z31





Zadanie 26. (0-4)

Dany jest ostroslup prawidlowy czworokatny. Wysokośč ściany bocznej tego ostroslupa jest

nachylona do plaszczyzny podstawy pod $\mathrm{k}_{\mathrm{c}}$]$\mathrm{t}\mathrm{e}\mathrm{m} 30^{\mathrm{o}}$ i ma dlugośč równa 6 (zobacz rysunek).

Oblicz objetośč i pole powierzchni calkowitej tego ostros[upa. Zapisz obliczenia.

1

Strona 22 z31

$\mathrm{M}\mathrm{M}\mathrm{A}\mathrm{P}-\mathrm{P}0_{-}100$





1

$\overline{11}-$

-

$0_{-}100$

Strona 23 z31





Zädanie 27, \{0-\S)

$\beta$

$\mathrm{W}$ pewnym ostroslupie prawidlowym stosunek liczby $W$ wszystkich wierzcholków do

liczby $K$ wszystkich krawedzi jest równy $\displaystyle \frac{W}{K}=\frac{3}{5}$

Dokończ zdanie. Wybierz w[aściwq odpowied $\acute{\mathrm{z}}$ spośród podanych.

Podstawq tego ostroslupa jest

A. kwadrat.

B. $\mathrm{p}\mathrm{i}_{9}$ciokqt foremny.

C. sześciokqt foremny.

D. siedmiokqt foremny.

{\it Brudnopis}

Zadanie 2@. $(0\leftrightarrow 1) \mathrm{E} \beta$

Dokończ zdanie. Wybierz w[aściwq odpowied $\acute{\mathrm{z}}$ spośród podanych.

Wszystkich liczb naturalnych pieciocyfrowych, w których zapisie dziesietnym wystepujq tylko

cyfry 0, 5, 7 (np. 57075, 55555), jest

A. $5^{3}$

B. $2\cdot 4^{3}$

C. $2\cdot 3^{4}$

D. $3^{5}$

{\it Brudnopis}

Strona 24 z31

$\mathrm{M}\mathrm{M}\mathrm{A}\mathrm{P}-\mathrm{P}0_{-}100$





Zadanie 29. (0-2)

Na diagramie ponizej przedstawiono ceny pomidorów w szesnastu wybranych sklepach.

6

5

4

liczba

sklepów 3
\begin{center}
\includegraphics[width=154.176mm,height=80.364mm]{./F3_M_PP_M2023_page24_images/image001.eps}
\end{center}
2

1

0

5,05

5,60

5,70

6,00

6,30

cena za l kg pomidorów (w zl)

Uzupe[nij tabele. Wpisz w $\mathrm{k}\mathrm{a}\dot{\mathrm{z}}$ dq pustq komórke tabeli w[aściwq odpowied $\acute{\mathrm{z}}$, wybranq

spośród oznaczonych literami A-E.
\begin{center}
\begin{tabular}{|l|l|l|}
\hline
\multicolumn{1}{|l|}{$29.1.$}&	\multicolumn{1}{|l|}{$\begin{array}{l}\mbox{Mediana ceny kilograma pomidorów w tych wybranych sklepach jest}	\\	\mbox{równa}	\end{array}$}&	\multicolumn{1}{|l|}{}	\\
\hline
\multicolumn{1}{|l|}{ $29.2.$}&	\multicolumn{1}{|l|}{$\begin{array}{l}\mbox{ $\acute{\mathrm{S}}$ rednia cena kilograma pomidorów w tych wybranych sklepach jest}	\\	\mbox{równa}	\end{array}$}&	\multicolumn{1}{|l|}{}	\\
\hline
\end{tabular}

\end{center}
A. 5,80 z1

B. 5,73 z1

C. 5,85 z1

D. 6,00 z1

E. 5,70 z1

{\it Brudnopis}

$\mathrm{M}\mathrm{M}\mathrm{A}\mathrm{P}-\mathrm{P}0_{-}100$

Strona 25 z31





Zadanie $30_{\mathrm{L}}\{0-2$)

Ze zbioru ośmiu liczb \{2, 3, 4, 5, 6, 7, 8, 9\} 1osujemy ze zwracaniem ko1ejno dwa razy po

jednej liczbie.

Oblicz prawdopodobieństwo zdarzenia $A$ polegajqcego na tym, $\dot{\mathrm{z}}\mathrm{e}$ iloczyn

wylosowanych liczb jest podzielny przez 15. Zapisz ob1iczenia.

$-|1$

1

1

$1-$

1

Strona 26 z31

$\mathrm{M}\mathrm{M}\mathrm{A}\mathrm{P}-\mathrm{P}0_{-}100$





Zadanie 38.

Wlaściciel pewnej apteki przeanalizowal dane dotyczqce liczby obslugiwanych klientów

$\mathrm{z} 30$ kolejnych dni. Przyjmijmy, $\dot{\mathrm{z}}\mathrm{e}$ liczbe $L$ obslugiwanych klientów $n$-tego dnia opisuje

funkcja

$L(n)=-n^{2}+22n+279$

gdzie $n$ jest liczbq naturalnq$\mathrm{s}\mathrm{p}\mathrm{e}$niajqcq warunki $n\geq 1 \mathrm{i} n\leq 30.$

Zadanie $38_{\mathrm{r}}\S. (0-\not\in)\mathrm{E} p$

Oceń prawdziwośč ponizszych stwierdzeń. Wybierz $\mathrm{P}$, jeśli stwierdzenie jest

prawdziwe, albo $\mathrm{F}$ -jeśli jest fa[szywe.
\begin{center}
\begin{tabular}{|l|l|l|}
\hline
\multicolumn{1}{|l|}{ $\begin{array}{l}\mbox{Laczna liczba klientów obsluzonych w czasie wszystkich analizowanych dni}	\\	\mbox{jest równa $L(30).$}	\end{array}$}&	\multicolumn{1}{|l|}{P}&	\multicolumn{1}{|l|}{F}	\\
\hline
\multicolumn{1}{|l|}{$\mathrm{W}$ trzecim dniu analizowanego okresu obsluzono 336 klientów.}&	\multicolumn{1}{|l|}{P}&	\multicolumn{1}{|l|}{F}	\\
\hline
\end{tabular}

\end{center}
$B_{\Gamma}udno\sqrt{}is$

Zadanie 38.2. $(0-2J$

Którego dnia analizowanego okresu w aptece obslu $\dot{\mathrm{z}}$ ono najwiekszq liczbe klientów?

Oblicz liczbe klientów obslu $\dot{\mathrm{z}}$ onych tego dnia. Zapisz obliczenia.

$\mathrm{M}\mathrm{M}\mathrm{A}\mathrm{P}-\mathrm{P}0_{-}100$

Strona 27 z31





1

$\overline{11}-$

-

Strona 28 z31

$\mathrm{M}\mathrm{M}\mathrm{A}\mathrm{P}-\mathrm{P}0_{-}10$





BRUDNOPIS (nie podlega ocenie)

1

-PO-100

Strona 29 z31





$| 1$

Strona 30 z31

$\mathrm{M}\mathrm{M}\mathrm{A}\mathrm{P}-\mathrm{P}0_{-}10$





Zädanie 1. (0-t) $\beta$

Na osi liczbowej zaznaczono sum9 przedzia1ów.
\begin{center}
\includegraphics[width=143.964mm,height=11.424mm]{./F3_M_PP_M2023_page3_images/image001.eps}
\end{center}
$-2$  5  $\chi$

Dokończ zdanie. Wybierz w[aściwq odpowied $\acute{\mathrm{z}}$ spośród podanych.

Zbiór zaznaczony na osi jest zbiorem wszystkich rozwiqzań nierówności

A. $|x-3,5|\geq 1,5$

B. $|x-1,5|\geq 3,5$

C. $|x-3,5|\leq 1,5$

D. $|x-1,5|\leq 3,5$

$\underline{Brudno\sqrt{}is}$

$1 -$

Zadanie $2_{\mathrm{Y}}$ (0-\S\} $\bullet \beta$

Dokończ zdanie. Wybierz w[aściwq odpowied $\acute{\mathrm{z}}$ spośród podanych.

Liczba $\sqrt[3]{-\frac{27}{16}}\cdot\sqrt[3]{2}$ jest równa

A. $(-\displaystyle \frac{3}{2})$

B. -23

C. -32

D. $(-\displaystyle \frac{2}{3})$

{\it Brudnopis}

Strona 4 z31

$\mathrm{M}\mathrm{M}\mathrm{A}\mathrm{P}-\mathrm{P}0_{-}100$





$0_{-}100$

$| 1$

Strona 31 z31










Zadanie 3. $(0-2$\}

Wykaz, $\dot{\mathrm{z}}\mathrm{e}$ dla $\mathrm{k}\mathrm{a}\dot{\mathrm{z}}$ dej liczby naturalnej $n\geq 1$ liczba $(2n+1)^{2}-1$ jest podzielna

przez 8.

$\mathrm{M}\mathrm{M}\mathrm{A}\mathrm{P}-\mathrm{P}0_{-}100$

Strona 5 z31





Zädanie 4. (0-\S) $\beta$

Dokończ zdanie. Wybierz wlaściwq odpowied $\acute{\mathrm{z}}$ spośród podanych.

Liczba $\log_{9}27+\log_{9}3$ jest równa

A. 81

B. 9

{\it Brudnopis}

$1-$

4

D. 2

$\mathrm{Z}\mathrm{a}\mathrm{d}\mathrm{a}\mathrm{n}\dot{\mathrm{l}}\mathrm{e}5. (0-9) \beta$

Dokończ zdanie. Wybierz w[aściwq odpowied $\acute{\mathrm{z}}$ spośród podanych.

Dla $\mathrm{k}\mathrm{a}\dot{\mathrm{z}}$ dej liczby rzeczywistej $a$ wyrazenie $(2a-3)^{2}-(2a+3)^{2}$ jest równe

A. $-24a$

B. 0

[‡A][D]18

D. $16a^{2}-24a$

{\it Brudnopis}

Strona 6 z31

$\mathrm{M}\mathrm{M}\mathrm{A}\mathrm{P}-\mathrm{P}0_{-}100$





Zädanie 6. (0-t) $\beta$

Dokończ zdanie. Wybierz wlaściwq odpowied $\acute{\mathrm{z}}$ spośród podanych.

Zbiorem wszystkich rozwiqzań nierówności

$-2(x+3)\displaystyle \leq\frac{2-x}{3}$

jest przedzial

A. $(-\infty,-4]$

B. $(-\infty,4]$

C. $[-4,\infty)$

D. [4, $\infty)$

{\it Brudnopis}

$-||\mathrm{i}1$

1

$| 1$

-

Zadanie 7. (0-t) $\mathrm{w} p$

Dokończ zdanie. Wybierz w[aściwq odpowied $\acute{\mathrm{z}}$ spośród podanych.

Jednym z rozwiqzań równania $\sqrt{3}(x^{2}-2)(x+3)=0$ jest liczba

A. 3

B. 2

C. $\sqrt{3}$

D. $\sqrt{2}$

{\it Brudnopis}

$\mathrm{M}\mathrm{M}\mathrm{A}\mathrm{P}-\mathrm{P}0_{-}100$

Strona 7 z31





Zadanie 8. (0-t) $\beta$

Dokończ zdanie. Wybierz wlaściwq odpowied $\acute{\mathrm{z}}$ spośród podanych.

Równanie $\displaystyle \frac{(x+1)(x-1)^{2}}{(x-1)(x+1)^{2}}=0$ w zbiorze liczb rzeczywistych

A. nie ma rozwiqzania.

B. ma dokladnie jedno rozwiqzanie: $-1.$

C. ma dokladnie jedno rozwiqzanie: l.

D. ma dokladnie dwa rozwiqzania: $-1$ oraz l.

{\it Brudnopis} 

1

1

$1-$

Zadanie $\mathrm{g}. (0-3$\}

Rozwiqz równanie

$3x^{3}-2x^{2}-12x+8=0$

Zapisz obliczenia.

Strona 8 z31

$\mathrm{M}\mathrm{M}\mathrm{A}\mathrm{P}-\mathrm{P}0_{-}10$





1

$\overline{11}-$

-

$0_{-}100$

Strona 9 z31





Zadanie \S 0. (0-\S J $\mathrm{E}\mathrm{H}\mathrm{B}^{\beta}$

Na rysunku przedstawiono interpretacj9 $\displaystyle \mathrm{g}\mathrm{e}\mathrm{o}\mathrm{m}\mathrm{e}\mathrm{t}\mathrm{r}\mathrm{y}\mathrm{c}\mathrm{z}\bigcap_{\mathrm{c}1}$ w kartezjańskim ukladzie

wspólrz9dnych $(x,y)$ jednego z $\mathrm{n}\mathrm{i}\dot{\mathrm{z}}$ ej zapisanych ukladów równań A-D.

Dokończ zdanie. Wybierz w[aściwq odpowied $\acute{\mathrm{z}}$ spośród podanych.

Ukladem równań, którego interpretacje geometrycznq przedstawiono na rysunku, jest

A. 

B. 

C. 

D. 

{\it Brudnopis}

Strona 10 z31

$\mathrm{M}\mathrm{M}\mathrm{A}\mathrm{P}-\mathrm{P}0_{-}100$







CENTRALNA

KOMISJA

EGZAMINACYJNA

Arkusz zawiera informacje prawnie chronione

do momentu rozpoczecia egzaminu.

KOD

WYPELNIA ZOAJACY

PESEL

{\it Miejsce na naklejke}.

{\it Sprawdz}', {\it czy kod na naklejce to}

M-100.
\begin{center}
\includegraphics[width=21.900mm,height=10.164mm]{./F3_M_PP_M2024_page0_images/image001.eps}

\includegraphics[width=79.656mm,height=10.164mm]{./F3_M_PP_M2024_page0_images/image002.eps}
\end{center}
/{\it ezeli tak}- {\it przyklej naklejkq}.

/{\it ezeli nie}- {\it zgtoś to nauczycielowi}.

Egzamin maturalny

$\displaystyle \int$
\begin{center}
\includegraphics[width=193.344mm,height=75.792mm]{./F3_M_PP_M2024_page0_images/image003.eps}
\end{center}
$\mathrm{P}\mathrm{o}\mathrm{z}\mathrm{i}$\fcircle$\mathrm{m}$  podstawowy

{\it Symbo arkusza}

MMAP-P0-100-2405

DATA: 8 maja 2024 r.

GODZINA R0ZP0CZECIA: 9:00

CZAS TRWANIA: $180 \displaystyle \min$ ut

WYPetNIA ZESPÓL NAOZORUJACY

Uprawnienia $\mathrm{z}\mathrm{d}\mathrm{a}\mathrm{j}_{8}$cego do:

\fbox{} dostosowania zasad oceniania

\fbox{} dostosowania w zw. z dyskalkuliq

\fbox{} nieprzenoszenia odpowiedzi na karte.

LICZBA PUNKTÓW DO UZYSKANIA 46

Przed rozpoczeciem pracy z arkuszem egzaminacyjnym

1.

Sprawd $\acute{\mathrm{z}}$, czy nauczyciel przekazal Ci wlaściwy arkusz egzaminacyjny,

tj. arkusz we wlaściwej formule, z w[aściwego przedmiotu na wlaściwym

poziomie.

2.

$\mathrm{J}\mathrm{e}\dot{\mathrm{z}}$ eli przekazano Ci niew[aściwy arkusz- natychmiast zgloś to nauczycielowi.

Nie rozrywaj banderol.

3. $\mathrm{J}\mathrm{e}\dot{\mathrm{z}}$ eli przekazano Ci w[aściwy arkusz- rozerwij banderole po otrzymaniu

takiego polecenia od nauczyciela. Zapoznaj $\mathrm{s}\mathrm{i}\mathrm{e}$ z instrukcjq na stronie 2.

Uk\}ad graficzny

\copyright CKE 2022 O

$\Vert\Vert\Vert\Vert\Vert\Vert\Vert\Vert\Vert\Vert\Vert\Vert\Vert\Vert\Vert\Vert\Vert\Vert\Vert\Vert\Vert\Vert\Vert\Vert\Vert\Vert\Vert\Vert\Vert\Vert|$




lnstrukcja dla zdajqcego

l. Sprawdz', czy arkusz egzaminacyjny zawiera 30 stron (zadania $1-31$).

Ewentualny brak zgloś przewodniczacemu zespolu nadzorujacego egzamin.

2. Na pierwszej stronie arkusza oraz na karcie odpowiedzi wpisz swój numer PESEL

i przyklej naklejke z kodem.

3. Symbol $\overline{\mathrm{L}^{\mathrm{g}}\Leftrightarrow \mathrm{g}}\nearrow$zamieszczony w naglówku zadania oznacza, $\dot{\mathrm{z}}\mathrm{e}$ rozwiqzanie zadania

zamknietego musisz przenieśč na karte odpowiedzi. Ocenie podlegajq$\mathrm{w}\mathrm{y}$qcznie

odpowiedzi zaznaczone na karcie odpowiedzi.

4. Odpowiedzi do zadań $\mathrm{z}\mathrm{a}\mathrm{m}\mathrm{k}\mathrm{n}\mathrm{i}_{9}$tych zaznacz na karcie odpowiedzi w cześci karty

przeznaczonej dla zdajqcego. Zamaluj $\blacksquare$ pola do tego przeznaczone. $\mathrm{B}_{9}\mathrm{d}\mathrm{n}\mathrm{e}$

zaznaczenie otocz kólkiem \copyright i zaznacz wlaściwe.

5. Pamietaj, $\dot{\mathrm{z}}\mathrm{e}$ pominiecie argumentacji lub istotnych obliczeń w rozwiqzaniu zadania

otwartego $\mathrm{m}\mathrm{o}\dot{\mathrm{z}}\mathrm{e}$ spowodować, $\dot{\mathrm{z}}\mathrm{e}$ za to rozwiazanie nie otrzymasz pelnej liczby punktów.

6. Rozwiqzania zadań i odpowiedzi wpisuj w miejscu na to przeznaczonym.

7. Pisz czytelnie i $\mathrm{u}\dot{\mathrm{z}}$ ywaj tylko dlugopisu lub pióra z czarnym tuszem lub atramentem.

8. Nie $\mathrm{u}\dot{\mathrm{z}}$ ywaj korektora, a bledne zapisy wyra $\acute{\mathrm{z}}$ nie przekreśl.

9. Nie wpisuj $\dot{\mathrm{z}}$ adnych znaków w tabelkach przeznaczonych dla egzaminatora.

Tabelki umieszczone sa na marginesie przy odpowiednich zadaniach.

10. Pamietaj, $\dot{\mathrm{z}}\mathrm{e}$ zapisy w brudnopisie nie beda oceniane.

11. $\mathrm{M}\mathrm{o}\dot{\mathrm{z}}$ esz korzystač z Wybranych wzorów matematycznych, cyrkla i linijki oraz kalkulatora

prostego. Upewnij $\mathrm{s}\mathrm{i}\mathrm{e}$, czy przekazano Ci broszure z $\mathrm{o}\mathrm{k}$adkq takQ jak widoczna ponizej.

Strona 2 z30

$\mathrm{M}\mathrm{M}\mathrm{A}\mathrm{P}-\mathrm{P}0_{-}100$





Zadanie 9{\$}. (0-{\$}) $\overline{\mathrm{L}\mathfrak{B}\mathfrak{B}}$'

Na rysunku, w kartezjańskim ukladzie wspólrzednych $(x,y)$, przedstawiono dwie proste

równolegle, które sq interpretacjq geometrycznq jednego z ponizszych ukladów równań A-D.
\begin{center}
\includegraphics[width=97.176mm,height=100.884mm]{./F3_M_PP_M2024_page10_images/image001.eps}
\end{center}
{\it y}

1

0  1  $\chi$

Dokończ zdanie. Wybierz w[aściwq odpowied $\acute{\mathrm{z}}$ spośród podanych.

Ukladem równań, którego interpretacj9 geometrycznq przedstawiono na rysunku, jest

A. 

B. 

C. 

D. 

{\it Brudnopis}

$\mathrm{M}\mathrm{M}\mathrm{A}\mathrm{P}-\mathrm{P}0_{-}100$

Strona ll z30





$\mathrm{Z}\mathrm{a}\mathrm{d}\mathrm{a}\mathrm{n}\ddagger \mathrm{e}$ \S 2. $(0-1\} \overline{\mathrm{L}\mathfrak{B}\mathfrak{B}}$'

Funkcja liniowa $f$ jest określona wzorem $f(x)=(-2k+3)x+k-1$, gdzie $k\in \mathbb{R}.$

Dokończ zdanie. Wybierz wlaściwq odpowied $\acute{\mathrm{z}}$ spośród podanych.

Funkcja $f$ jest malejaca dla $\mathrm{k}\mathrm{a}\dot{\mathrm{z}}$ dej liczby $k$ nalezqcej do przedzialu

A. $(-\infty,1)$

B. $(-\displaystyle \infty,-\frac{3}{2})$

C. $(1,+\infty)$

D. $(\displaystyle \frac{3}{2},+\infty)$

{\it Brudnopis}

1

Zadanie \S 3$*$\{0$\infty$9) $\square \sqsupset\sqsupset\sqsupset$'

Funkcje liniowe $f$ oraz $g$, określone wzorami $f(x)=3x+6$ oraz $g(x)=ax+7$, maja

to samo miejsce zerowe.

Dokończ zdanie. Wybierz w[aściwq odpowied $\acute{\mathrm{z}}$ spośród podanych.

Wspólczynnik a we wzorze funkcji g jest równy

A. $(-\displaystyle \frac{7}{2})$

B. $(-\displaystyle \frac{2}{7})$

C. -72

D. -27

{\it Brudnopis}

Strona 12 z30

$\mathrm{M}\mathrm{M}\mathrm{A}\mathrm{P}-\mathrm{P}0_{-}100$





Zadanie t4.

W kartezjańskim ukladzie wspólrzednych (x, y) przedstawiono fragment paraboli, która jest

wykresem funkcji kwadratowej f (zobacz rysunek). Wierzcholek tej paraboli oraz punkty

przecipcia paraboli z osiami ukladu wspólrzednych maja obie wspólrzedne calkowite.

Zadanie \S 4.a. $(0-8$\}

Uzupe[nij ponizsze zdanie. Wpisz odpowiedni przedzia[w wykropkowanym miejscu

tak, aby zdanie bylo prawdziwe.

Zbiorem wszystkich rozwiazań nierówności $f(x)\geq 0$ jest przedzial

{\it Brudnopis}

Zadanie 94.2. (0-\S\} $\overline{1^{\mathrm{n}}\Delta \mathrm{g}\mathrm{g}}1$

Dokończ zdanie. Wybierz w[aściwq odpowied $\acute{\mathrm{z}}$ spośród podanych.

Funkcja kwadratowa f jest określona wzorem

A. $f(x)=-(x+1)^{2}-9$

B. $f(x)=-(x-1)^{2}+9$

C. $f(x)=-(x-1)^{2}-9$

D. $f(x)=-(x+1)^{2}+9$

{\it Brudnopis}

$\mathrm{M}\mathrm{M}\mathrm{A}\mathrm{P}-\mathrm{P}0_{-}100$

Strona 13 z30





Zadanie 14.3. (0-\S\} $\overline{\mathrm{L}\mathfrak{W}\mathfrak{B}}$;

Dokończ zdanie. Wybierz w[aściwq odpowied $\acute{\mathrm{z}}$ spośród podanych.

Dla funkcji f prawdziwa jest równośč

A. $f(-4)=f(6)$

B. $f(-4)=f(5)$

C. $f(-4)=f(4)$

D. $f(-4)=f(7)$

{\it Brudnopis}

$\mathrm{Z}\mathrm{a}\mathrm{d}\mathrm{a}\mathfrak{n}i\mathrm{e}*4.4_{*}\{0\infty 2)$

Funkcje kwadratowe $g$ oraz $h$ sa określone za pomocq funkcji $f$ (zobacz rysunek na

stronie 13) nastepujqco: $g(x)=f(x+3), h(x)=f(-x).$

Na rysunkach A-F przedstawiono, w kartezjańskim uk[adzie wspólrz9dnych $(x,y),$

fragmenty wykresów róznych funkcji-w tym fragment wykresu funkcji $g$ oraz fragment

wykresu funkcji $h.$

Uzupelnij tabele. $\mathrm{K}\mathrm{a}\dot{\mathrm{z}}$ dej z funkcji $g$ oraz $h$ przyporzqdkuj fragmentjej wykresu.

Wpisz w $\mathrm{k}\mathrm{a}\dot{\mathrm{z}}$ dq pustq komórke tabeli w[aściwq odpowied $\acute{\mathrm{z}}$, wybranq spośród

oznaczonych literami A-F.
\begin{center}
\begin{tabular}{|l|l|}
\hline
\multicolumn{1}{|l|}{Fragment wykresu funkcji $y=g(x)$ przedstawiono na rysunku}&	\multicolumn{1}{|l|}{}	\\
\hline
\multicolumn{1}{|l|}{Fragment wykresu funkcji $y=h(x)$ przedstawiono na rysunku}&	\multicolumn{1}{|l|}{}	\\
\hline
\end{tabular}

\end{center}
Strona 14 z30

$\mathrm{M}\mathrm{M}\mathrm{A}\mathrm{P}-\mathrm{P}0_{-}100$





{\it Brudnopis}

$\mathrm{M}\mathrm{M}\mathrm{A}\mathrm{P}-\mathrm{P}0_{-}100$

Strona 15 z30





Zadanie 15. $(0-1$\} $\overline{\mathrm{L}\mathfrak{B}\mathfrak{B}}$'

Ciqg $(a_{n})$ jest określony wzorem $a_{n}=(-1)^{n}\cdot(n-5)$ dla $\mathrm{k}\mathrm{a}\dot{\mathrm{z}}$ dej liczby

naturalnej $n\geq 1.$

Oceń prawdziwośč ponizszych stwierdzeń. Wybierz P, jeśli stwierdzenie jest

prawdziwe, albo F -jeśli jest fa[szywe.
\begin{center}
\begin{tabular}{|l|l|l|}
\hline
\multicolumn{1}{|l|}{$\begin{array}{l}\mbox{Pierwszy wyraz ciqgu $(a_{n})$ jest dwa razy wipkszy od trzeciego wyrazu tego}	\\	\mbox{ciqgu.}	\end{array}$}&	\multicolumn{1}{|l|}{P}&	\multicolumn{1}{|l|}{F}	\\
\hline
\multicolumn{1}{|l|}{Wszystkie wyrazy ciqgu $(a_{n})$ sq dodatnie.}&	\multicolumn{1}{|l|}{P}&	\multicolumn{1}{|l|}{F}	\\
\hline
\end{tabular}

\end{center}
{\it Brudnopis}

-

1

Zadanie k6. $\{0\infty \mathrm{B}$) $\overline{\mathrm{L}\emptyset\infty \mathrm{g}}$;

Trzywyrazowy ciag $(12,6,2m-1)$ jest geometryczny.

Dokończ zdanie. Wybierz odpowied $\acute{\mathrm{z}}$ A albo $\mathrm{B}$ oraz $\mathrm{o}\mathrm{d}\mathrm{p}\mathrm{o}\mathrm{w}\mathrm{i}\mathrm{e}\mathrm{d}\acute{\mathrm{z}}1., 2$. albo 3.

Ten ciqg jest

A.

rosnacy

1.

$m=\displaystyle \frac{1}{2}$

oraz

2.

$m=2$

B.

malejqcy

3.

$m=3$

{\it Brudnopis}

Strona 16 z30

$\mathrm{M}\mathrm{M}\mathrm{A}\mathrm{P}-\mathrm{P}0_{-}100$





Zadanie $\mathrm{t}7. (0-2)$

Ciqg arytmetyczny $(a_{n})$ jest określony dla $\mathrm{k}\mathrm{a}\dot{\mathrm{z}}$ dej liczby naturalnej $n\geq 1$. Trzeci wyraz

tego ciqgu jest równy $(-1)$, a suma piptnastu poczqtkowych kolejnych wyrazów tego ciqgu

jest równa $(-165).$

Oblicz róznice tego ciqgu. Zapisz obliczenia.

1

$1-$

1

$\mathrm{M}\mathrm{M}\mathrm{A}\mathrm{P}-\mathrm{P}0_{-}100$

Strona 17 z30





Zadanie $\mathrm{f}8_{*}(0-2)$

$\mathrm{W}$ kartezjańskim ukladzie wspólrzednych $(x,y)$ zaznaczono kqt o mierze $\alpha$ taki, $\dot{\mathrm{z}}\mathrm{e}$

tg $\alpha=-3$ oraz $90^{\mathrm{o}}<\alpha<180^{\mathrm{o}}$ (zobacz rysunek).

Uzupe[nij zdanie. Wybierz dwie w[aściwe odpowiedzi spośród oznaczonych literami

A-F i wpisz te litery w wykropkowanych miejscach.

Prawdziwe sq zalezności:

oraz

A. $\sin\alpha<0$

B. $\sin\alpha\cdot\cos\alpha<0$

C. $\sin\alpha\cdot\cos\alpha>0$

D. $\cos\alpha>0$

E. $\displaystyle \sin\alpha=-\frac{1}{3}\cos\alpha$

$\mathrm{F}.\ \sin\alpha=-3\cos\alpha$

$Brudno\sqrt{}is -$

Zadanie \S 9. $(0-\not\in) \overline{\llcorner \mathfrak{B}\mathrm{g}}$;

Dokończ zdanie. Wybierz w[aściwq odpowied $\acute{\mathrm{z}}$ spośród podanych.

Liczba $\sin^{3}20^{\mathrm{o}}+\cos^{2}20^{\mathrm{o}}\cdot\sin 20^{\mathrm{o}}$ jest równa

A. $\cos 20^{\mathrm{o}}$

B. $\sin 20^{\mathrm{o}}$

C. $\mathrm{t}\mathrm{g}20^{\mathrm{o}}$

$\mathrm{D}.\ \sin 20^{\mathrm{o}}\cdot\cos 20^{\mathrm{o}}$

{\it Brudnopis}

Strona 18 z30

$\mathrm{M}\mathrm{M}\mathrm{A}\mathrm{P}-\mathrm{P}0_{-}100$





Zadanie 20. $\{0-1$) $\overline{\mathrm{L}\mathrm{E}\mathrm{g}\mathrm{g}}$;

Danyjest trójkqt $KLM$, w którym $|KM|=a, |LM|=b$ oraz $a\neq b$. Dwusieczna kqta

$KML$ przecina bok $KL$ w punkcie $N$ takim, $\dot{\mathrm{z}}\mathrm{e} |KN|=c, |NL|=d$ oraz $|MN|=e$

(zobacz rysunek).
\begin{center}
\includegraphics[width=83.772mm,height=48.864mm]{./F3_M_PP_M2024_page18_images/image001.eps}
\end{center}
{\it M}

{\it a e  b}

{\it K N}

{\it c  d L}

Dokończ zdanie. Wybierz wlaściwq odpowied $\acute{\mathrm{z}}$ spośród podanych.

$\mathrm{W}$ trójkqcie $KLM$ prawdziwa jest równośč

A. $a\cdot b=c\cdot d$

B. $a\cdot d=b\cdot c$

C. $a\cdot c=b\cdot d$

D. $a\cdot b=e\cdot e$

$Brudno\sqrt{}is$

I I I $1 -$

ZadanIe 2{\$}. $(0-9$\}

$\square \sqsupset\sqsupset\sim\Sigma*$'

Danyjest równoleglobok o bokach dlugości 3 $\mathrm{i} 4$ oraz o kacie miedzy nimi o mierze $120^{\mathrm{o}}$

Dokończ zdanie. Wybierz w[aściwq odpowied $\acute{\mathrm{z}}$ spośród podanych.

Pole tego równolegloboku jest równe

A. 12

B. $12\sqrt{3}$

C. 6

D. $6\sqrt{3}$

{\it Brudnopis}

$\mathrm{M}\mathrm{M}\mathrm{A}\mathrm{P}-\mathrm{P}0_{-}100$

Strona 19 z30





Zadanie 22. (0-1)

$\overline{\mathrm{L}\mathrm{E}\mathrm{g}\mathrm{g}}$;

$\mathrm{W}$ trójkqcie $ABC$, wpisanym w $\mathrm{o}\mathrm{k}\mathrm{r}_{\mathrm{c}}$]$\mathrm{g}$ o środku w punkcie $S$, kqt $ACB$ ma miar9 $42^{\mathrm{o}}$

(zobacz rysunek).
\begin{center}
\includegraphics[width=67.968mm,height=65.076mm]{./F3_M_PP_M2024_page19_images/image001.eps}
\end{center}
{\it C}

$42^{\mathrm{o}}$  {\it S}

{\it A}

{\it B}

Dokończ zdanie. Wybierz w[aściwq odpowied $\acute{\mathrm{z}}$ spośród podanych.

Miara kqta ostrego BAS jest równa

A. $42^{\mathrm{o}}$

B. $45^{\mathrm{o}}$

C. $48^{\mathrm{o}}$

D. $69^{\mathrm{o}}$

$Brudno\sqrt{}is -$

Zadanie 23. (0-\S) $\overline{[similar]\alpha \mathrm{g}\mathrm{u}}$'

$\mathrm{W}$ kartezjańskim ukladzie wspólrzednych $(x,y)$ proste $k$ oraz $l$ sq określone równaniami

{\it k}:

$y=(m+1)x+7$

{\it l}:

$y=-2x+7$

Dokończ zdanie. Wybierz w[aściwq odpowied $\acute{\mathrm{z}}$ spośród podanych.

Proste k oraz l sa prostopadle, gdy liczba m jest równa

A. $(-\displaystyle \frac{1}{2})$

B. -21

C. $(-3)$

D. l

{\it Brudnopis}

Strona 20 z30

$\mathrm{M}\mathrm{M}\mathrm{A}\mathrm{P}-\mathrm{P}0_{-}100$





Zadania egzaminacyjne sq wydrukowane

na nastepnych stronach.

$\mathrm{M}\mathrm{M}\mathrm{A}\mathrm{P}-\mathrm{P}0_{-}100$

Strona 3 z30





Zadanie 24. (0-2)

$\mathrm{W}$ kartezjańskim ukladzie wspólrzednych $(x,y)$ danyjest równoleglobok ABCD, w którym

$A=(-2,6)$ oraz $B=(10,2)$. Przekqtne $AC$ oraz $BD$ tego równolegloboku przecinajq

sipw punkcie $P=(6,7).$

Oblicz dlugośč boku BC tego równoleg[oboku. Zapisz obliczenia.

1

1

$\mathrm{M}\mathrm{M}\mathrm{A}\mathrm{P}-\mathrm{P}0_{-}100$

Strona 21 z30





Zadanie $25_{\mathrm{r}}$

Wysokośč graniastoslupa prawidlowego sześciokatnego jest równa 6 (zobacz rysunek).

Pole podstawy tego graniastoslupa jest równe $15\sqrt{3}.$
\begin{center}
\includegraphics[width=62.328mm,height=71.016mm]{./F3_M_PP_M2024_page21_images/image001.eps}
\end{center}
I

I

I

I

I

I

I

I

I

I

I

I

I

$\underline{\mathrm{I}}$

I

I

I

I

I

I

I

I

I

I

I

I

6

Zadanie 25.1. $(\emptyset\infty \mathrm{e}$\} $\overline{\llcorner \mathfrak{B}\mathrm{g}}$;

Dokończ zdanie. Wybierz w[aściwq odpowied $\acute{\mathrm{z}}$ spośród podanych.

Pole $|\mathrm{e}\mathrm{d}\mathrm{n}\mathrm{e}1$ ściany bocznej tego graniastoslupa jest równe

A. $36\sqrt{10}$

B. 60

C. $6\sqrt{10}$

D. 360

$Brudno\sqrt{}is$

Strona 22 z30

$\mathrm{M}\mathrm{M}\mathrm{A}\mathrm{P}-\mathrm{P}0_{-}100$





Zadanie $25_{*}2_{\alpha}$ (0-\S) $\overline{\mathrm{L}\mathfrak{W}\mathfrak{B}}$;

Dokończ zdanie. Wybierz wlaściwq odpowied $\acute{\mathrm{z}}$ spośród podanych.

Kqt nachylenia najdlu $\dot{\mathrm{z}}$ szej przekqtnej graniastoslupa prawidlowego sześciokqtnego do

plaszczyzny podstawy jest zaznaczony na rysunku

A.
\begin{center}
\includegraphics[width=57.912mm,height=70.968mm]{./F3_M_PP_M2024_page22_images/image001.eps}
\end{center}
I

I

I

I

I

I

I

I

I

I

I

I

I

$\underline{\mathrm{I}}$

I

I

I

I

I

I

I

I

I

I

I

I

C.
\begin{center}
\includegraphics[width=57.816mm,height=70.920mm]{./F3_M_PP_M2024_page22_images/image002.eps}
\end{center}
I

I

I

I

I

I

I

I

I

I

I

I

I

$\prime\underline{\mathrm{I}}$

I

I

I

I

I

I

I

I

I

I

I

I

{\it Brudnopis}

B.
\begin{center}
\includegraphics[width=57.960mm,height=70.968mm]{./F3_M_PP_M2024_page22_images/image003.eps}
\end{center}
I

I

I

I

I

I

I

I

I

I

I

I

I

$\underline{\mathrm{I}}$

I

I

I

I

I

I

I

I

I

I

D.
\begin{center}
\includegraphics[width=57.912mm,height=70.920mm]{./F3_M_PP_M2024_page22_images/image004.eps}
\end{center}
I

I

I

I

I

I

I

I

I

I

I

I

I

$\underline{\mathrm{I}}-$

I

I

I

I

I

I

I

I

I

I

I

I

$-1$

$\mathrm{M}\mathrm{M}\mathrm{A}\mathrm{P}-\mathrm{P}0_{-}100$

Strona 23 z30





Zadanie 26. $(0-\not\in)$

Ostroslup $F_{1}$ jest podobny do ostroslupa $F_{2}.$

Obj9tośč ostros1upa $F_{1}$ jest równa 64.

Obj9tośč ostros1upa $F_{2}$ jest równa 512.

Uzupe[nij ponizsze zdanie. Wpisz odpowiedniq liczbe w wykropkowanym miejscu tak,

aby zdanie by[o prawdziwe.

Stosunek pola powierzchni calkowitej ostroslupa $F_{2}$ do pola powierzchni calkowitej

ostroslupa $F_{1}$ jest równy

{\it Brudnopis}

$-|1\mathrm{i}\mathrm{i}-$

$| 1$

$\mathrm{Z}\mathrm{a}\mathrm{d}\mathrm{a}\mathrm{n}\dot{\mathfrak{x}}\mathrm{e}Z7$. (0-{\$}) $\square \sqsupset\sqsupset 2$;

Rozwazamy wszystkie kody czterocyfrowe utworzone tylko z cyfr 1, 3, 6, 8, przy czym

w $\mathrm{k}\mathrm{a}\dot{\mathrm{z}}$ dym kodzie $\mathrm{k}\mathrm{a}\dot{\mathrm{z}}$ da z tych cyfr wystppuje dokladnie jeden raz.

Dokończ zdanie. Wybierz w[aściwq odpowied $\acute{\mathrm{z}}$ spośród podanych.

Liczba wszystkich takich kodówjest równa

A. 4

B. 10

C. 24

D. 16

{\it Brudnopis}

Strona 24 z30

$\mathrm{M}\mathrm{M}\mathrm{A}\mathrm{P}-\mathrm{P}0_{-}100$





Zadanie 28, (0-{\$}) $\overline{\mathrm{L}\mathfrak{B}\mathfrak{B}}$'

$\acute{\mathrm{S}}$ rednia arytmetyczna trzech liczb: $a, b, c$, jest równa 9.

Dokończ zdanie. Wybierz w[aściwq odpowied $\acute{\mathrm{z}}$ spośród podanych.

$\acute{\mathrm{S}}$ rednia arytmetyczna sześciu liczb: $a, a, b, b, c, c$, jest równa

A. 9

B. 6

C. 4,5

D. 18

{\it Brudnopis}

$\mathrm{Z}\mathrm{a}\mathrm{d}\mathrm{a}\mathrm{n}\mathrm{i}^{\vee}\mathrm{e}29. (0\infty 1\} \square \sqsupset\supset\supset 1$

Na diagramie przedstawiono wyniki sprawdzianu z matematyki w pewnej klasie maturalnej.

Na osi poziomej podano oceny, które uzyskali uczniowie tej klasy, a na osi pionowej podano

liczbe uczniów, którzy otrzymali danq ocen9.

8

7

6

liczba

uczniów

5

4
\begin{center}
\includegraphics[width=105.060mm,height=57.144mm]{./F3_M_PP_M2024_page24_images/image001.eps}
\end{center}
3

2

1

0

1 2

3 4

5 6

ocena

Dokończ zdanie. Wybierz w[aściwq odpowied $\acute{\mathrm{z}}$ spośród podanych.

Mediana ocen uzyskanych z tego sprawdzianu przez uczniów tej klasy jest równa

A. 4,5

B. 4

C. $3_{r}5$

D. 3

{\it Brudnopis}

$\mathrm{M}\mathrm{M}\mathrm{A}\mathrm{P}-\mathrm{P}0_{-}100$

Strona 25 z30





Zadanie $30_{\mathrm{L}}\{0-2$)

Dany jest piecioelementowy zbiór $K=\{5$, 6, 7, 8, 9$\}$. Wylosowanie $\mathrm{k}\mathrm{a}\dot{\mathrm{z}}$ dej liczby z tego

zbioru jestjednakowo prawdopodobne. Ze zbioru $K$ losujemy ze zwracaniem kolejno dwa

razy po jednej liczbie i zapisujemy je w kolejności losowania.

Oblicz prawdopodobieństwo zdarzenia $A$ polegajqcego na tym, $\dot{\mathrm{z}}\mathrm{e}$ suma

wylosowanych liczb jest liczbq parzystq. Zapisz obliczenia.

1

1

Strona 26 z30

$\mathrm{M}\mathrm{M}\mathrm{A}\mathrm{P}-\mathrm{P}0_{-}100$





Zadanie $38_{*}(0-4)$

$\mathrm{W}$ schronisku dla zwierzat, na $\mathrm{p}$\}askiej powierzchni, nale $\dot{\mathrm{z}}\mathrm{y}$ zbudowač ogrodzenie z siatki

wydzielajqce trzy identyczne wybiegi o $\underline{\mathrm{w}\mathrm{s}\mathrm{p}\text{ó} \mathrm{l}\mathrm{n}\mathrm{v}\mathrm{c}\mathrm{h}}$ ścianach wewnptrznych.

Podstawq$\mathrm{k}\mathrm{a}\dot{\mathrm{z}}$ dego z tych trzech wybiegów jest $\mathrm{p}\mathrm{r}\mathrm{o}\mathrm{s}\mathrm{t}\mathrm{o}\mathrm{k}_{\mathrm{c}1}\mathrm{t}$ (jak pokazano na rysunku).

Do wykonania tego ogrodzenia nale $\dot{\mathrm{z}}\mathrm{y}\mathrm{z}\mathrm{u}\dot{\mathrm{z}}$ yč 36 metrów biezqcych siatki.

Schematyczny rysunek trzech wybiegów (widok z góry).

Liniq przerywanq zaznaczono siatk9.
\begin{center}
\includegraphics[width=107.952mm,height=6.804mm]{./F3_M_PP_M2024_page26_images/image001.eps}
\end{center}
{\it y y  y}

$\chi$

$\iota \Gamma$ 1

I

I I I

I

$1 \mathrm{I}$ 1

I

I I I

I

I I I

I

I I I

I

I bi l bi 2 I bi 3 I

wybieg l. 1 wybieg 2. wybieg 3.

I I I

I

1 I I

1

I I I

I

I I I

I

1 I I

I

I I 1

I

$1 \llcorner$ 1

Oblicz wymiary $x$ oraz $y$ jednego wybiegu, przy których suma pól podstaw tych

trzech wybiegów bedzie najwieksza. $\mathrm{W}$ obliczeniach pomiń szerokośč wejścia na

$\mathrm{k}\mathrm{a}\dot{\mathrm{z}}\mathrm{d}\mathrm{y}$ z wybiegów. Zapisz obliczenia.

$\mathrm{M}\mathrm{M}\mathrm{A}\mathrm{P}-\mathrm{P}0_{-}100$

Strona 27 z30





1

$\overline{11}-$

-

Strona 28 z30

$\mathrm{M}\mathrm{M}\mathrm{A}\mathrm{P}-\mathrm{P}0_{-}10$





BRUDNOPIS (nie podlega ocenie)

1

-PO-100

Strona 29 z30





$| 1$

Strona 30 z30

$\mathrm{M}\mathrm{M}\mathrm{A}\mathrm{P}-\mathrm{P}0_{-}10$





Zadanie 8. (0-8) $\overline{\mathrm{L}\mathfrak{W}\mathfrak{B}}$;

Dana jest nierównośč

$|x-1|\geq 3$

Na którym rysunku poprawnie zaznaczono na osi liczbowej zbiór wszystkich liczb

rzeczywistych spe[niajqcych powy $\dot{\mathrm{z}}$ szq nierównośč? Wybierz w[aściwq $\mathrm{o}\mathrm{d}\mathrm{p}\mathrm{o}\mathrm{w}\mathrm{i}\mathrm{e}\mathrm{d}\acute{\mathrm{z}}$

spośród podanych.

A.

B.
\begin{center}
\includegraphics[width=83.820mm,height=10.764mm]{./F3_M_PP_M2024_page3_images/image001.eps}
\end{center}
$-2$  4  $\chi$
\begin{center}
\includegraphics[width=83.820mm,height=10.764mm]{./F3_M_PP_M2024_page3_images/image002.eps}
\end{center}
$-2$  4  $\chi$

C.

D.
\begin{center}
\includegraphics[width=83.820mm,height=10.764mm]{./F3_M_PP_M2024_page3_images/image003.eps}
\end{center}
$-2$  4  $\chi$
\begin{center}
\includegraphics[width=83.820mm,height=10.764mm]{./F3_M_PP_M2024_page3_images/image004.eps}
\end{center}
$-2$  4  $\chi$

$Brudno\sqrt{}is -$

1

Zadanie 2. (0-\S) $\square \sqsupset\infty\infty$;

Dokończ zdanie. Wybierz w[aściwq odpowied $\acute{\mathrm{z}}$ spośród podanych.

Liczba $(\displaystyle \frac{1}{16})^{8}\cdot 8^{16}$ jest równa

A. $2^{24}$

B. $2^{16}$

C. $2^{12}$

D. $2^{8}$

{\it Brudnopis}

Strona 4 z30

$\mathrm{M}\mathrm{M}\mathrm{A}\mathrm{P}-\mathrm{P}0_{-}100$















Zadanie 3. $(0-2$\}

Wykaz, $\dot{\mathrm{z}}\mathrm{e}$ dla $\mathrm{k}\mathrm{a}\dot{\mathrm{z}}$ dej liczby naturalnej $n\geq 1$ liczba $n^{2}+(n+1)^{2}+(n+2)^{2}$ przy

dzieleniu przez 3 daje reszte 2.

-

1

1

$\mathrm{M}\mathrm{M}\mathrm{A}\mathrm{P}-\mathrm{P}0_{-}100$

Strona 5 z30





Zadanie 4. \{0-\S) $\overline{\mathrm{L}\mathfrak{W}\mathfrak{B}}$;

Dokończ zdanie. Wybierz wlaściwq odpowied $\acute{\mathrm{z}}$ spośród podanych.

Liczba $\log_{\sqrt{3}}9$ jest równa

A. 2

B. 3

{\it Brudnopis}

4

D. 9

Zadanie 5. (0-P) $\overline{\omega \mathrm{D}\mathrm{B}\mathrm{B}}1$

Dokończ zdanie. Wybierz w[aściwq odpowied $\acute{\mathrm{z}}$ spośród podanych.

Dla $\mathrm{k}\mathrm{a}\dot{\mathrm{z}}$ dej liczby rzeczywistej $a$ i dla $\mathrm{k}\mathrm{a}\dot{\mathrm{z}}$ dej liczby rzeczywistej $b$ wartośč wyrazenia

$(2a+b)^{2}-(2a-b)^{2}$ jest równa wartości wyra $\dot{\mathrm{z}}$ enia

A. $8a^{2}$

B. 8ab

C. $-8ab$

D. $2b^{2}$

{\it Brudnopis}

Strona 6 z30

$\mathrm{M}\mathrm{M}\mathrm{A}\mathrm{P}-\mathrm{P}0_{-}100$





Zadanie 6. (0-8) $\overline{\mathrm{L}\mathfrak{W}\mathfrak{B}}$;

Dokończ zdanie. Wybierz wlaściwq odpowied $\acute{\mathrm{z}}$ spośród podanych.

Zbiorem wszystkich rozwiqzań nierówności

$1-\displaystyle \frac{3}{2}x<\frac{2}{3}-x$

jest przedzial

A. $(-\displaystyle \infty,-\frac{2}{3})$

B.(-$\infty$,-23)

C. $(-\displaystyle \frac{2}{3},+\infty)$

D. $(\displaystyle \frac{2}{3},+\infty)$

{\it Brudnopis}

-

Zadanie $7_{\mathrm{r}}\{\emptyset\infty 4$) $\square \sqsupset\infty 3*\nearrow$

Dokończ zdanie. Wybierz w[aściwq odpowied $\acute{\mathrm{z}}$ spośród podanych.

Równanie $\displaystyle \frac{x+1}{(x+2)(x-3)}=0$ w zbiorze liczb rzeczywistych

A. nie ma rozwiqzania.

B. ma dokladnie jedno rozwiqzanie: $(-1).$

C. ma dokladnie dwa rozwiqzania: $(-2)$ oraz 3.

D. ma dokladnie trzy rozwiqzania: $(-1), (-2)$ oraz 3.

{\it Brudnopis}

$\mathrm{M}\mathrm{M}\mathrm{A}\mathrm{P}-\mathrm{P}0_{-}100$

Strona 7 z30





Zadanie @$*$(0-\S) $\overline{\mathrm{L}\mathfrak{W}\mathfrak{B}}$;

Danyjest wielomian $W(x)=3x^{3}+6x^{2}+9x.$

Oceń prawdziwośč ponizszych stwierdzeń. Wybierz P, jeśli stwierdzenie jest

prawdziwe, albo F -jeśli jest fa[szywe.
\begin{center}
\begin{tabular}{|l|l|l|}
\hline
\multicolumn{1}{|l|}{Wielomian $W$ jest iloczynem wielomianów $F(x)=3x \mathrm{i} G(x)=\chi^{2}+2x+3.$}&	\multicolumn{1}{|l|}{P}&	\multicolumn{1}{|l|}{F}	\\
\hline
\multicolumn{1}{|l|}{Liczba $(-1)$ jest rozwiqzaniem równania $W(x)=0.$}&	\multicolumn{1}{|l|}{P}&	\multicolumn{1}{|l|}{F}	\\
\hline
\end{tabular}

\end{center}
{\it Brudnopis}

1

$| 1$

$1$

Zadanie $\mathrm{g}. (0-3\rangle$

Rozwiqz równanie

$x^{3}-2x^{2}-3x+6=0$

Zapisz obliczenia.

Strona 8 z30

$\mathrm{M}\mathrm{M}\mathrm{A}\mathrm{P}-\mathrm{P}0_{-}100$





1

$\overline{11}-$

-

$0_{-}100$

Strona 9 z30





Zadanie 10. (0-{\$}) $\overline{\mathrm{L}\mathfrak{B}\mathfrak{B}}$'

$\mathrm{W}$ paz'dzierniku 2022 roku za1ozono dwa sady, w których posadzono 1qcznie 1960 drzew.

Po roku stwierdzono, $\dot{\mathrm{z}}\mathrm{e}$ uschlo 5\% drzew w pierwszym sadzie i 10\% drzew w drugim

sadzie. Uschnipte drzewa usunieto, a nowych nie dosadzano.

Liczba drzew, które pozostaly w drugim sadzie, stanowila 60\% 1iczby drzew, które

pozostaly w pierwszym sadzie.

Niech $x$ oraz $\mathrm{y}$ oznaczaja liczby drzew posadzonych- odpowiednio-w pierwszym

i drugim sadzie.

Dokończ zdanie. Wybierz wlaściwq odpowied $\acute{\mathrm{z}}$ spośród podanych.

Ukladem równań, którego poprawne rozwiqzanie prowadzi do obliczenia liczby $x$ drzew

posadzonych w pierwszym sadzie oraz liczby $y$ drzew posadzonych w drugim sadzie, jest

A. 

B. 

C. 

D. 

$Brudno\sqrt{}is$

Strona 10 z30

$\mathrm{M}\mathrm{M}\mathrm{A}\mathrm{P}-\mathrm{P}0_{-}100$







CENTRALNA

KOMISJA

EGZAMINACYJNA

Arkusz zawiera informacje prawnie chronione

do momentu rozpoczecia egzaminu.

KOD

WYPELNIA ZDAJACY

PESEL

{\it Miejsce na naklejke}.

{\it Sprawdz}', {\it czy kod na naklejce to}

M-100.
\begin{center}
\includegraphics[width=21.900mm,height=10.164mm]{./F3_M_PR_M2023_page0_images/image001.eps}

\includegraphics[width=79.656mm,height=10.164mm]{./F3_M_PR_M2023_page0_images/image002.eps}
\end{center}
/{\it ezeli tak}- {\it przyklej naklejkq}.

/{\it ezeli nie}- {\it zgtoś to nauczycielowi}.

Egzamin maturalny

$\displaystyle \int$
\begin{center}
\includegraphics[width=193.344mm,height=78.180mm]{./F3_M_PR_M2023_page0_images/image003.eps}
\end{center}
Poziom  rozszerzony

{\it Symbol arkusza}

MMAP-R0-100-2305

DATA: 12 maja 2023 r.
\begin{center}
\begin{tabular}{|l|}
\hline
\multicolumn{1}{|l|}{P LNIA $\mathrm{E}\mathrm{S}\mathrm{P}6\mathrm{L}$ NADZORUJACY}	\\
\hline
\multicolumn{1}{|l|}{$\begin{array}{l}\mbox{Uprawnienia zdaj cego do:}	\\	\mbox{dostosowania zasad oceniania}	\\	\mbox{dostosowania w zw. z dyskalkuli}	\end{array}$}	\\
\hline
\end{tabular}

\end{center}
GODZINA R0ZP0CZECIA: 9:00

CZAS TRWANIA: $180 \displaystyle \min$ ut

LICZBA PUNKTÓW DO UZYSKANIA 50

Przed rozpoczeciem pracy z arkuszem egzaminacyjnym

1.

Sprawd $\acute{\mathrm{z}}$, czy nauczyciel przekazal Ci wlaściwy arkusz egzaminacyjny,

tj. arkusz we wlaściwej formule, z w[aściwego przedmiotu na wlaściwym

poziomie.

2.

$\mathrm{J}\mathrm{e}\dot{\mathrm{z}}$ eli przekazano Ci niew[aściwy arkusz- natychmiast zgloś to nauczycielowi.

Nie rozrywaj banderol.

3.

$\mathrm{J}\mathrm{e}\dot{\mathrm{z}}$ eli przekazano Ci w[aściwy arkusz- rozerwij banderole po otrzymaniu

takiego polecenia od nauczyciela. Zapoznaj $\mathrm{s}\mathrm{i}\mathrm{e}$ z instrukcjq na stronie 2.

$\mathrm{U}\mathrm{k}\}\mathrm{a}\mathrm{d}$ graficzny

\copyright CKE 2022 $\blacksquare$

$\Vert\Vert\Vert\Vert\Vert\Vert\Vert\Vert\Vert\Vert\Vert\Vert\Vert\Vert\Vert\Vert\Vert\Vert\Vert\Vert\Vert\Vert\Vert\Vert\Vert\Vert\Vert\Vert\Vert\Vert|$




lnstrukcja dla zdajqcego

1.

2.

3.

4.

5.

6.

7.

8.

9.

Sprawd $\acute{\mathrm{z}}$, czy arkusz egzaminacyjny zawiera 27 stron (zadania $1-13$).

Ewentualny brak zgloś przewodniczqcemu zespolu nadzorujqcego egzamin.

Na pierwszej stronie arkusza oraz na karcie odpowiedzi wpisz swój numer PESEL

i przyklej naklejke z kodem.

$\mathrm{P}\mathrm{a}\mathrm{m}\mathrm{i}_{9}\mathrm{t}\mathrm{a}\mathrm{j}, \dot{\mathrm{z}}\mathrm{e}$ pominiecie argumentacji lub istotnych obliczeń w rozwiqzaniu zadania

otwartego $\mathrm{m}\mathrm{o}\dot{\mathrm{z}}\mathrm{e}$ spowodować, $\dot{\mathrm{z}}\mathrm{e}$ za to rozwiazanie nie otrzymasz pelnej liczby punktów.

Rozwiqzania zadań i odpowiedzi wpisuj w miejscu na to przeznaczonym.

Pisz czytelnie i $\mathrm{u}\dot{\mathrm{z}}$ ywaj tylko dlugopisu lub pióra z czarnym tuszem lub atramentem.

Nie $\mathrm{u}\dot{\mathrm{z}}$ ywaj korektora, a bledne zapisy wyra $\acute{\mathrm{z}}$ nie przekreśl.

Nie wpisuj $\dot{\mathrm{z}}$ adnych znaków w tabelkach przeznaczonych dla egzaminatora. Tabelki

umieszczone sq na marginesie przy $\mathrm{k}\mathrm{a}\dot{\mathrm{z}}$ dym zadaniu.

$\mathrm{P}\mathrm{a}\mathrm{m}\mathrm{i}_{9}\mathrm{t}\mathrm{a}\mathrm{j}, \dot{\mathrm{z}}\mathrm{e}$ zapisy w brudnopisie nie beda oceniane.

$\mathrm{M}\mathrm{o}\dot{\mathrm{z}}$ esz korzystač z Wybranych wzorów matematycznych, cyrkla i linijki oraz kalkulatora

prostego. Upewnij $\mathrm{s}\mathrm{i}\mathrm{e}$, czy przekazano Ci broszur9 z ok1adka taka jak widoczna ponizej.

Strona 2 z27

$\mathrm{M}\mathrm{M}\mathrm{A}\mathrm{P}-\mathrm{R}0_{-}100$





Zadanie 7. $(0-4$\}

Danyjest sześcian ABCDEFGH o krawpdzi

dlugości 6. Punkt $S$ jest punktem przeciecia

przekqtnych $AH \mathrm{i}$ DE ściany bocznej ADHE

(zobacz rysunek).

Oblicz wysokośč tróikqta SBH poprowadzonq z punktu S na bok BH tego trójkqta.

Zapisz obliczenia.

$\mathrm{M}\mathrm{M}\mathrm{A}\mathrm{P}-\mathrm{R}0_{-}100$

Strona ll z27





Zadanie 8. $(0-4$\}

Czworokqt ABCD, w którym $|BC|=4 \mathrm{i} |CD|=5$, jest opisany na okregu. Przekatna $AC$

tego czworokata tworzy z bokiem $BC$ kqt o mierze $60^{\mathrm{o}}$, natomiast z bokiem $AB-$ kqt ostry,

którego sinus jest równy $\displaystyle \frac{1}{4}.$

Oblicz obwód czworokqta ABCD. Zapisz obliczenia.

Strona 12 z27

$\mathrm{M}\mathrm{M}\mathrm{A}\mathrm{P}-\mathrm{R}0_{-}100$





RO-100

Strona 13 z27





Zadanie $\mathrm{g}. (0-4$\}

Rozwiqz nierównośč

$\displaystyle \sqrt{x^{2}+4x+4}<\frac{25}{3}-\sqrt{x^{2}-6x+9}$

Zapisz obliczenia.

{\it Wskazówka}: {\it skorzystaj z tego, ze} $\sqrt{a^{2}}=|a|$ {\it dla kazdej liczby} $ rz\mathrm{e}\mathrm{c}z\gamma${\it wi}s{\it t}e{\it j} $a.$

Strona 14 z27

$\mathrm{M}\mathrm{M}\mathrm{A}\mathrm{P}-\mathrm{R}0_{-}10$





RO-100

Strona 15 z27





Zadanie $\mathrm{f}0_{\mathrm{L}}\{0-4$)

Określamy kwadraty $K_{1}, K_{2}, K_{3}$, następujqco:

$\bullet K_{1}$ jest kwadratem o boku dlugości $a$

$\bullet K_{2}$ jest kwadratem, którego $\mathrm{k}\mathrm{a}\dot{\mathrm{z}}\mathrm{d}\mathrm{y}$ wierzcholek $\mathrm{l}\mathrm{e}\dot{\mathrm{z}}\mathrm{y}$ na innym boku kwadratu $K_{1}$

ten bok w stosunku 1 : 3

i dzieli

$\bullet K_{3}$ jest kwadratem, którego $\mathrm{k}\mathrm{a}\dot{\mathrm{z}}\mathrm{d}\mathrm{y}$ wierzcholek $\mathrm{l}\mathrm{e}\dot{\mathrm{z}}\mathrm{y}$ na innym boku kwadratu $K_{2}$ i dzieli

ten bok w stosunku 1 : 3

i ogólnie, dla $\mathrm{k}\mathrm{a}\dot{\mathrm{z}}$ dej liczby naturalnej $n\geq 2,$

$\bullet K_{n}$ jest kwadratem, którego $\mathrm{k}\mathrm{a}\dot{\mathrm{z}}\mathrm{d}\mathrm{y}$ wierzcholek $\mathrm{l}\mathrm{e}\dot{\mathrm{z}}\mathrm{y}$ na innym boku kwadratu $K_{n-1}$

i dzieli ten bok w stosunku 1 : 3.

Obwody wszystkich kwadratów określonych powyzej tworzq nieskończony ciqg

geometryczny.

Na rysunku przedstawiono kwadraty utworzone w sposób opisany powyzej.

{\it a}
\begin{center}
\includegraphics[width=58.824mm,height=58.872mm]{./F3_M_PR_M2023_page15_images/image001.eps}
\end{center}
{\it a}

Oblicz sume wszystkich wyrazów tego nieskończonego ciqgu. Zapisz obliczenia.

$\dagger$

Strona 16 z27

$\mathrm{M}\mathrm{M}\mathrm{A}\mathrm{P}-\mathrm{R}0_{-}100$





RO-100

Strona 17 z27





Zadanie Y\S$*$(0-5)

Wyznacz wszystkie wartości parametru $m\neq 2$, dla których równanie

$x^{2}+4x-\displaystyle \frac{m-3}{m-2}=0$

ma dwa rózne rozwiqzania rzeczywiste $x_{1}, x_{2}$ spelniajqce warunek $x_{1}^{3}+x_{2}^{3}>-28.$

Zapisz obliczenia.

Strona 18 z27

$\mathrm{M}\mathrm{M}\mathrm{A}\mathrm{P}-\mathrm{R}0_{-}100$





RO-100

Strona 19 z27





Zadanie 82.

Funkcja $f$ jest określona wzorem $f(x)=81^{\log_{3}x}+\displaystyle \frac{2\cdot\log_{2}\sqrt{27}\cdot\log_{3}2}{3}\cdot x^{2}-6x$ dla

$\mathrm{k}\mathrm{a}\dot{\mathrm{z}}$ dej liczby dodatniel $x.$

Zadanie \S 2.a. $\{0-2\}$

Wykaz, $\dot{\mathrm{z}}\mathrm{e}$ dla $\mathrm{k}\mathrm{a}\dot{\mathrm{z}}\mathrm{d}\mathrm{e}\mathrm{i}$ liczby dodatniej $x$ wyra $\dot{\mathrm{z}}$ enie

$81^{\log_{3}x}+\displaystyle \frac{2\cdot\log_{2}\sqrt{27}\cdot\log_{3}2}{3}\cdot x^{2}-6x$

$\mathrm{m}\mathrm{o}\dot{\mathrm{z}}$ na równowaznie przekszta[cič do postaci $x^{4}+x^{2}-6x.$

Strona 20 z27

$\mathrm{M}\mathrm{M}\mathrm{A}\mathrm{P}-\mathrm{R}0_{-}100$





Zadania egzaminacyine sq wydrukowane

na nastepnych stronach.

$\mathrm{M}\mathrm{M}\mathrm{A}\mathrm{P}-\mathrm{R}0_{-}100$

Strona 3 z27





Zadanie n2.2. (0-4)

Oblicz najmniejszq wartośč funkcji $f$ określonej dla $\mathrm{k}\mathrm{a}\dot{\mathrm{z}}$ dej liczby dodatniei $x.$

Zapisz obliczenia.

{\it Wskazówka}: {\it przyjmij, ze wzór funkcii} $f$ {\it mozna przedstawič w postaci} $f(x)=x^{4}+x^{2}-6x.$

$\mathrm{M}\mathrm{M}\mathrm{A}\mathrm{P}-\mathrm{R}0_{-}100$

Strona 21 z27





Zadanie 83. (0-6)

$\mathrm{W}$ kartezjańskim ukladzie wspólrzednych $(x,y)$ prosta $l$ o równaniu $x-y-2=0$

przecina parabo19 o równaniu $y=4x^{2}-7x+1$ w punktach $A$ oraz $B$. Odcinek $AB$ jest

średnicq okrpgu $O$. Punkt $C \mathrm{l}\mathrm{e}\dot{\mathrm{z}}\mathrm{y}$ na okrpgu $O$ nad prostq $l$, a kqt $BAC$ jest ostry i ma

miar9 $\alpha$ takq, $\dot{\mathrm{z}}\mathrm{e} \displaystyle \mathrm{t}\mathrm{g}\alpha=\frac{1}{3}$ (zobacz rysunek).
\begin{center}
\includegraphics[width=122.580mm,height=132.024mm]{./F3_M_PR_M2023_page21_images/image001.eps}
\end{center}
{\it y}

1  $y=4x^{2}-7x+1$

{\it l}

$x-y-2=0$

1 {\it C}  $\chi$

{\it B}

$\alpha$

{\it A}

Oblicz wspó[rzedne punktu C. Zapisz obliczenia.

$\rfloor$

$\rceil_{1}$

$\rfloor$

$\rfloor$

$\rfloor$

$\rfloor$

$i$

$\mathrm{t}^{:}$

Strona 22 z27

$\mathrm{M}\mathrm{M}\mathrm{A}\mathrm{P}-\mathrm{R}0_{-}100$





RO-100

Strona 23 z27





Strona 24 z27

$\mathrm{M}\mathrm{M}\mathrm{A}\mathrm{P}-\mathrm{R}0_{-}10$





: RUDNOPIS (nie podlega ocenie)

$\Psi-\mathrm{R}0_{-}100$

Strona 25 z27





1

-$|\mathfrak{l} \mathfrak{l} \mathfrak{l}|$ -

Strona 26 z27

$\mathrm{M}\mathrm{M}\mathrm{A}\mathrm{P}-\mathrm{R}0_{-}10$





1

-$|\mathfrak{l} \mathfrak{l} \mathfrak{l}|$ -

RO-100

Strona 27 z27










Zadanie 8. $(0-2$\}

$\mathrm{W}$ chwili poczqtkowej $(t=0)$ masa substancji jest równa 4 gramom. Wskutek rozpadu

czqsteczek tej substancji jej masa si9 zmniejsza. Po $\mathrm{k}\mathrm{a}\dot{\mathrm{z}}$ dej kolejnej dobie ubywa

19\% masy, jaka byla na koniec doby poprzedniej. Dla $\mathrm{k}\mathrm{a}\dot{\mathrm{z}}$ dej liczby calkowitej $t\geq 0$

funkcja $m(t)$ określa mase substancji w gramach po $t$ pelnych dobach (czas liczymy od

chwili poczatkowej).

Wyznacz wzór funkcji m(t). Oblicz, po ilu pe[nych dobach masa tej substancji bedzie

po raz pierwszy mniejsza od 1, 5 grama.

Zapisz obliczenia.

$1 1$

Strona 4 z27

$\mathrm{M}\mathrm{M}\mathrm{A}\mathrm{P}-\mathrm{R}0_{-}100$





Zadanie 2. $(0-3$\}

Tomek i Romek postanowili rozegrač między sobq pieč partii szachów. Prawdopodobieństwo

wygrania pojedynczej partii przez Tomka jest równe $\displaystyle \frac{1}{4}.$

Oblicz prawdopodobieństwo wygrania przez Tomka co najmniej czterech z pieciu

partii. Wynik podaj w postaci ulamka zwyk[ego nieskracalnego. Zapisz obliczenia.

$\mathrm{M}\mathrm{M}\mathrm{A}\mathrm{P}-\mathrm{R}0_{-}100$

Strona 5 z27





Zadanie 3. $(0-3$\}

Funkcja $f$ jest określona wzorem $f(x)=\displaystyle \frac{3x^{2}-2x}{x^{2}+2x+8}$ dla $\mathrm{k}\mathrm{a}\dot{\mathrm{z}}$ dej liczby rzeczywistej $x.$

Punkt $P=(x_{0}$, 3$)$ nalez $\mathrm{y}$ do wykresu funkcji $f.$

Oblicz $x_{0}$ oraz wyznacz równanie stycznej do wykresu funkcji $f$ w punkcie $P.$

Zapisz obliczenia.

$-\mathrm{i}1$

$1 -$

Strona 6 z27

$\mathrm{M}\mathrm{M}\mathrm{A}\mathrm{P}-\mathrm{R}0_{-}100$





Zadanie 4. $(0-3$\}

Liczby rzeczywiste $x$ oraz $y$ spelniajqjednocześnie równanie $x+y=4$ i nierównośč

$x^{3}-x^{2}\mathrm{y}\leq x\mathrm{y}^{2}-y^{3}$

Wykaz, $\dot{\mathrm{z}}\mathrm{e} x=2$ oraz $y=2.$

$\mathrm{M}\mathrm{M}\mathrm{A}\mathrm{P}-\mathrm{R}0_{-}100$

Strona 7 z27





Zadanie 5. $(0-3$\}

Danyjest trójkqt prostokqtny $ABC$, w którym $|4ABC|=90^{\mathrm{o}}$ oraz $|4\mathrm{C}AB|=60^{\mathrm{o}}$ Punkty

$K \mathrm{i} L \mathrm{l}\mathrm{e}\dot{\mathrm{z}}$ a na bokach- odpowiednio -$AB \mathrm{i} BC$ tak, $\dot{\mathrm{z}}\mathrm{e} |BK|=|BL|=1$ (zobacz

rysunek). Odcinek $KL$ przecina wysokośč $BD$ tego trójkqta w punkcie $N$, a ponadto

$|AD|=2.$
\begin{center}
\includegraphics[width=133.404mm,height=81.480mm]{./F3_M_PR_M2023_page7_images/image001.eps}
\end{center}
{\it A}

$60^{\mathrm{o}}$

2

{\it D}

{\it K}

{\it N}

1

{\it B 1 L  C}

Wykaz, $\dot{\mathrm{z}}\mathrm{e} |ND|=\sqrt{3}+1.$

Strona 8 z27

$\mathrm{M}\mathrm{M}\mathrm{A}\mathrm{P}-\mathrm{R}0_{-}100$





RO-100

Strona 9 z27





Zadanie 6. $(0-3$\}

Rozwiqz równanie

$4\sin(4x)\cos(6x)=2\sin(10x)+1$

Zapisz obliczenia.

Strona 10 z27

$\mathrm{M}\mathrm{M}\mathrm{A}\mathrm{P}-\mathrm{R}0_{-}10$







CENTRALNA

KOMISJA

EGZAMINACYJNA

Arkusz zawiera informacje prawnie chronione

do momentu rozpoczecia egzaminu.

KOD

WYPELNIA ZDAJACY

PESEL

{\it Miejsce na naklejke}.

{\it Sprawdz}', {\it czy kod na naklejce to}

M-100.
\begin{center}
\includegraphics[width=21.900mm,height=10.164mm]{./F3_M_PR_M2024_page0_images/image001.eps}

\includegraphics[width=79.656mm,height=10.164mm]{./F3_M_PR_M2024_page0_images/image002.eps}
\end{center}
/{\it ezeli tak}- {\it przyklej naklejkq}.

/{\it ezeli nie}- {\it zgtoś to nauczycielowi}.

Egzamin maturalny

$\displaystyle \int$
\begin{center}
\includegraphics[width=193.344mm,height=78.180mm]{./F3_M_PR_M2024_page0_images/image003.eps}
\end{center}
Poziom  rozszerzony

{\it Symbol arkusza}

MMAP-R0-100-2405

DATA: 15 maja 2024 r.
\begin{center}
\begin{tabular}{|l|}
\hline
\multicolumn{1}{|l|}{WYP N1A $\mathrm{S}\mathrm{P}6$ NADZORUJACY}	\\
\hline
\multicolumn{1}{|l|}{$\begin{array}{l}\mbox{Uprawnienia zdaj cego do:}	\\	\mbox{dostosowania zasad oceniania.}	\end{array}$}	\\
\hline
\end{tabular}

\end{center}
GODZINA R0ZP0CZECIA: 9:00

CZAS TRWANIA: $180 \displaystyle \min$ ut

LICZBA PUNKTÓW DO UZYSKANIA 50

Przed rozpoczeciem pracy z arkuszem egzaminacyjnym

1.

Sprawd $\acute{\mathrm{z}}$, czy nauczyciel przekazal Ci wlaściwy arkusz egzaminacyjny,

tj. arkusz we wlaściwej formule, z w[aściwego przedmiotu na wlaściwym

poziomie.

2.

$\mathrm{J}\mathrm{e}\dot{\mathrm{z}}$ eli przekazano Ci niew[aściwy arkusz- natychmiast zgloś to nauczycielowi.

Nie rozrywaj banderol.

3.

$\mathrm{J}\mathrm{e}\dot{\mathrm{z}}$ eli przekazano Ci w[aściwy arkusz- rozerwij banderole po otrzymaniu

takiego polecenia od nauczyciela. Zapoznaj $\mathrm{s}\mathrm{i}\mathrm{e}$ z instrukcjq na stronie 2.

$\mathrm{U}\mathrm{k}\}\mathrm{a}\mathrm{d}$ graficzny

\copyright CKE 2022 $\bullet 1$

$\Vert\Vert\Vert\Vert\Vert\Vert\Vert\Vert\Vert\Vert\Vert\Vert\Vert\Vert\Vert\Vert\Vert\Vert\Vert\Vert\Vert\Vert\Vert\Vert\Vert\Vert\Vert\Vert\Vert\Vert|$




lnstrukcja dla zdajqcego

1.

2.

3.

4.

5.

6.

7.

8.

9.

Sprawd $\acute{\mathrm{z}}$, czy arkusz egzaminacyjny zawiera 27 stron (zadania $1-13$).

Ewentualny brak zgloś przewodniczqcemu zespolu nadzorujqcego egzamin.

Na pierwszej stronie arkusza oraz na karcie odpowiedzi wpisz swój numer PESEL

i przyklej naklejke z kodem.

$\mathrm{P}\mathrm{a}\mathrm{m}\mathrm{i}_{9}\mathrm{t}\mathrm{a}\mathrm{j}, \dot{\mathrm{z}}\mathrm{e}$ pominiecie argumentacji lub istotnych obliczeń w rozwiqzaniu zadania

otwartego $\mathrm{m}\mathrm{o}\dot{\mathrm{z}}\mathrm{e}$ spowodować, $\dot{\mathrm{z}}\mathrm{e}$ za to rozwiazanie nie otrzymasz pelnej liczby punktów.

Rozwiqzania zadań i odpowiedzi wpisuj w miejscu na to przeznaczonym.

Pisz czytelnie i $\mathrm{u}\dot{\mathrm{z}}$ ywaj tylko dlugopisu lub pióra z czarnym tuszem lub atramentem.

Nie $\mathrm{u}\dot{\mathrm{z}}$ ywaj korektora, a bledne zapisy wyra $\acute{\mathrm{z}}$ nie przekreśl.

Nie wpisuj $\dot{\mathrm{z}}$ adnych znaków w tabelkach przeznaczonych dla egzaminatora. Tabelki

umieszczone $\mathrm{s}_{\mathrm{c}}1$ na marginesie przy $\mathrm{k}\mathrm{a}\dot{\mathrm{z}}$ dym zadaniu.

$\mathrm{P}\mathrm{a}\mathrm{m}\mathrm{i}_{9}\mathrm{t}\mathrm{a}\mathrm{j}, \dot{\mathrm{z}}\mathrm{e}$ zapisy w brudnopisie nie bedq oceniane.

$\mathrm{M}\mathrm{o}\dot{\mathrm{z}}$ esz korzystač z Wybranych wzorów matematycznych, cyrkla i linijki oraz kalkulatora

prostego. Upewnij $\mathrm{s}\mathrm{i}\mathrm{e}$, czy przekazano Ci broszur9 z ok1adka taka jak widoczna ponizej.

Strona 2 z27

$\mathrm{M}\mathrm{M}\mathrm{A}\mathrm{P}-\mathrm{R}0_{-}100$





RO-100

Strona ll z27





Zadanie 8. $(0-4$\}

Danyjest trójkqt $ABC$, który nie jest równoramienny. $\mathrm{W}$ tym trójkqcie miara kqta $ABC$ jest

dwa razy wieksza od miary kqta $BAC.$

Wykaz, $\dot{\mathrm{z}}\mathrm{e}$ dlugości boków tego trójkqta spe[niajq warunek

$|AC|^{2}=|BC|^{2}+|AB|$

$|BC|$

Strona 12 z27

$\mathrm{M}\mathrm{M}\mathrm{A}\mathrm{P}-\mathrm{R}0_{-}100$





RO-100

Strona 13 z27





Zadanie $\mathrm{g}. (0-4$\}

Danyjest kwadrat ABCD o boku dlugości $a$. Punkt $E$ jest środkiem boku $CD$. Przekatna

$BD$ dzieli trójkat ACE na dwie figury: $AGF$ oraz CEFG (zobacz rysunek).
\begin{center}
\includegraphics[width=66.096mm,height=70.668mm]{./F3_M_PR_M2024_page13_images/image001.eps}
\end{center}
{\it D E  C}

{\it F}

{\it G}

{\it A  a  B}

Oblicz pola figur AGF oraz CEFG. Zapisz obliczenia.

Strona 14 z27

$\mathrm{M}\mathrm{M}\mathrm{A}\mathrm{P}-\mathrm{R}0_{-}100$





RO-100

Strona 15 z27





Zadanie $\mathrm{f}0. (0-5)$

Rozwiqz równanie

$\sin(4x)-\sin(2x)=4\cos^{2}x-3$

w zbiorze $[0,2\pi]$. Zapisz obliczenia.

Strona 16 z27

$\mathrm{M}\mathrm{M}\mathrm{A}\mathrm{P}-\mathrm{R}0_{-}10$





RO-100

Strona 17 z27





Zadanie Y\S$*$(0-5)

$\mathrm{W}$ kartezjańskim ukladzie wspólrzednych $(x,y)$ środek $S$ okregu o promieniu $\sqrt{5} \mathrm{l}\mathrm{e}\dot{\mathrm{z}}\mathrm{y}$ na

prostej o równaniu $y=x+1$. Przez punkt $A=(1,2)$, którego odleglośč od punktu $S$ jest

wipksza od $\sqrt{5}$, poprowadzono dwie proste styczne do tego okregu w punktach-

odpowiednio- $B \mathrm{i} C$. Pole czworokqta ABSC jest równe 15.

Oblicz wspólrzqdne punktu $S.$ Rozwa $\dot{\mathrm{z}}$ wszystkie przypadki. Zapisz obliczenia.

Strona 18 z27

$\mathrm{M}\mathrm{M}\mathrm{A}\mathrm{P}-\mathrm{R}0_{-}100$





RO-100

Strona 19 z27





Zadanie 82. (0-6)

Wyznacz wszystkie wartości parametru $m$, dla których równanie

$x^{2}-(3m+1)\cdot x+2m^{2}+m+1=0$

ma dwa rózne rozwiqzania rzeczywiste $x_{1}, x_{2}$ spelniajqce warunek

$x_{1}^{3}+x_{2}^{3}+3\cdot x_{1}\cdot x_{2}$

$(x_{1}+x_{2}-3)\leq 3m-7$

Zapisz obliczenia.

Strona 20 z27

$\mathrm{M}\mathrm{M}\mathrm{A}\mathrm{P}-\mathrm{R}0_{-}10$





Zadania egzaminacyine sq wydrukowane

na nastepnych stronach.

$\mathrm{M}\mathrm{M}\mathrm{A}\mathrm{P}-\mathrm{R}0_{-}100$

Strona 3 z27





RO-100

Strona 21 z27





Strona 22 z27

$\mathrm{M}\mathrm{M}\mathrm{A}\mathrm{P}-\mathrm{R}0_{-}10$





Zadanie 83

Rozwazamy wszystkie graniastoslupy prawidlowe trójkqtne o objetości 3456, których

$\mathrm{k}\mathrm{r}\mathrm{a}\mathrm{w}9^{\mathrm{d}\acute{\mathrm{Z}}}$ podstawy ma dlugośč nie większq $\mathrm{n}\mathrm{i}\dot{\mathrm{z}} 8\sqrt{3}.$

Zadanie $83.9_{1}(0-2)$

Wykaz, $\dot{\mathrm{z}}\mathrm{e}$ pole $P$ powierzchni ca[kowitej graniastoslupa w zale $\dot{\mathrm{z}}$ ności od d[ugości $a$

krawedzi podstawy graniastos[upa jest określone wzorem

$P(a)=\displaystyle \frac{a^{2}\cdot\sqrt{3}}{2}+\frac{13824\sqrt{3}}{a}$

$\mathrm{M}\mathrm{M}\mathrm{A}\mathrm{P}-\mathrm{R}0_{-}100$

Strona 23 z27





Zadanie 83[‡C]2. (0-4)

Pole $P$ powierzchni calkowitej graniastoslupa w zalezności od d$\dagger$ugości $a$ krawedzi

podstawy graniastoslupa jest określone wzorem

$P(a)=\displaystyle \frac{a^{2}\cdot\sqrt{3}}{2}+\frac{13824\sqrt{3}}{a}$

dla $a\in(0,8\sqrt{3}].$

Wyznacz d[ugośč krawedzi podstawy tego z rozwa $\dot{\mathrm{z}}$ anych graniastos[upów, którego

pole powierzchni calkowitej jest najmniejsze. Oblicz to najmniejsze pole. Zapisz

obliczenia.

Strona 24 z27

$\mathrm{M}\mathrm{M}\mathrm{A}\mathrm{P}-\mathrm{R}0_{-}100$





RO-100

Strona 25 z27





: RUDNOPIS (nie podlega ocenie)

Strona 26 z27

$\mathrm{M}\mathrm{M}\mathrm{A}\mathrm{P}-\mathrm{R}0_{-}10$





1

-$|\mathfrak{l} \mathfrak{l} \mathfrak{l}|$ -

RO-100

Strona 27 z27










Zadanie 8. $(0-2$\}

$\mathrm{W}$ chwili poczqtkowej$(t=0)$ filizanka z goracq kawq znajduje si9 w pokoju, a temperatura

tej kawy jest równa $80^{\mathrm{o}}\mathrm{C}$. Temperatura w pokoju (temperatura otoczenia)jest stala

i równa $20^{\mathrm{o}}\mathrm{C}$. Temperatura $T$ tej kawy zmienia si9 w czasie zgodnie z za1eznościq

$T(t)=(T_{p}-T_{Z})\cdot k^{-r}+T_{Z}$ dla

$r\geq 0$

gdzie:

T - temperatura kawy wyrazona w stopniach Celsjusza,

$t -$ czas wyrazony w minutach, liczony od chwili poczqtkowej,

$T_{\mathrm{P}}-$ temperatura poczqtkowa kawy wyrazona w stopniach Celsjusza,

$T_{Z}-$ temperatura otoczenia wyrazona w stopniach Celsjusza,

$k -$ stala charakterystyczna dla danej cieczy.

Po 10 minutach, 1iczqc od chwi1i poczatkowej, kawa ostyg1a do temperatury 65 $\mathrm{o}\mathrm{C}.$

Oblicz temperature tej kawy po nastepnych pieciu minutach. Wynik podaj w stopniach

Celsjusza, w zaokrqgleniu do jedności. Zapisz obliczenia.

Strona 4 z27

$\mathrm{M}\mathrm{M}\mathrm{A}\mathrm{P}-\mathrm{R}0_{-}100$





Zadanie 2. $(0-2$\}

Oblicz granice

Zapisz obliczenia.

$\mathrm{M}\mathrm{M}\mathrm{A}\mathrm{P}-\mathrm{R}0_{-}100$

$\chi$li$\rightarrow$m2--($\chi\chi$3--28)2

Strona 5 z27





Zadanie 3. $(0-3$\}

$\mathrm{W}$ pewnym zakladzie mleczarskim śmietana produkowana jest w 200-gramowych

opakowaniach. Prawdopodobieństwo zdarzenia, $\dot{\mathrm{z}}\mathrm{e}$ w losowo wybranym opakowaniu

śmietana zawiera mniej $\mathrm{n}\mathrm{i}\dot{\mathrm{z}}$ 36\% tluszczu, jest równe 0,0l. Kontroli poddajemy l0 losowo

wybranych opakowań ze śmietanq.

Oblicz prawdopodobieństwo zdarzenia polegajqcego na tym, $\dot{\mathrm{z}}\mathrm{e}$ wśród opakowań

poddanych $\mathrm{t}\mathrm{e}\mathrm{i}$ kontroli bedzie co najwy $\dot{\mathrm{z}}$ ej jedno opakowanie ze śmietanq, która

zawiera mniej $\mathrm{n}\mathrm{i}\dot{\mathrm{z}}$ 36\% tluszczu. Wynik zapisz w postaci ulamka dziesietnego

w zaokrqgleniu do cześci tysiecznych. Zapisz obliczenia.

Strona 6 z27

$\mathrm{M}\mathrm{M}\mathrm{A}\mathrm{P}-\mathrm{R}0_{-}100$





Zadaníe 4. $(0-3$\}

Funkcja $f$ jest określona wzorem

$f(x)=\displaystyle \frac{x^{3}-3x+2}{\chi}$

dla $\mathrm{k}\mathrm{a}\dot{\mathrm{z}}$ dej liczby rzeczywistej $x$ róznej od zera. $\mathrm{W}$ kartezjańskim ukladzie wspólrz9dnych

$(x,\mathrm{y})$ punkt $P$, o pierwszej wspólrzednej równej 2, na1ez $\mathrm{y}$ do wykresu funkcji $f.$

Prosta o równaniu $y=ax+b$ jest styczna do wykresu funkcji $f$ w punkcie $P.$

Oblicz wspólczynniki a oraz b w równaniu tei stycznej. Zapisz obliczenia.

$\mathrm{M}\mathrm{M}\mathrm{A}\mathrm{P}-\mathrm{R}0_{-}100$

Strona 7 z27





Zadanie $5*(0-3$\}

Wyka $\dot{\mathrm{z}}, \dot{\mathrm{z}}\mathrm{e}\mathrm{j}\mathrm{e}\dot{\mathrm{z}}$ eli $\log_{5}4=a$ oraz log43 $=b$, to log1280$=\displaystyle \frac{2a+1}{a\cdot(1+b)}$

Strona 8 z27

$\mathrm{M}\mathrm{M}\mathrm{A}\mathrm{P}-\mathrm{R}0_{-}10$





Zadanie 6, $(0-3$\}

Rozwazamy wszystkie liczby naturalne, w których zapisie $\mathrm{d}\mathrm{z}\mathrm{i}\mathrm{e}\mathrm{s}\mathrm{i}9$tnym nie powtarza si9

jakakolwiek cyfra oraz dokladnie trzy cyfry sq nieparzyste i dokladnie dwie cyfry sq parzyste.

Oblicz, ile jest wszystkich takich liczb. Zapisz obliczenia.

$\mathrm{M}\mathrm{M}\mathrm{A}\mathrm{P}-\mathrm{R}0_{-}100$

Strona 9 z27





Zadanie 7. $(0-4$\}

Trzywyrazowy ciag $(x,y,z)$ jest geometryczny i rosnqcy. Suma wyrazów tego ciqgu jest

równa 105. Liczby $x, y$ oraz $z$ sq- odpowiednio-pierwszym, drugim oraz szóstym

wyrazem ciqgu arytmetycznego $(a_{n})$, określonego dla $\mathrm{k}\mathrm{a}\dot{\mathrm{z}}$ dej liczby naturalnej $n\geq 1.$

Oblicz x, y oraz z. Zapisz obliczenia.

Strona 10 z27

$\mathrm{M}\mathrm{M}\mathrm{A}\mathrm{P}-\mathrm{R}0_{-}100$







Centralna Komisja Egzaminacyjna

Arkusz zawiera informacje prawnie chronione do momentu rozpoczęcia egzaminu.

WPISUJE ZDAJACY

KOD PESEL

{\it Miejsce}

{\it na naklejkę}

{\it z kodem}
\begin{center}
\includegraphics[width=21.432mm,height=9.804mm]{./F1_M_PP_C2012_page0_images/image001.eps}

\includegraphics[width=82.092mm,height=9.804mm]{./F1_M_PP_C2012_page0_images/image002.eps}
\end{center}
\fbox{} dysleksja
\begin{center}
\includegraphics[width=204.060mm,height=216.048mm]{./F1_M_PP_C2012_page0_images/image003.eps}
\end{center}
EGZAMIN MATU LNY

Z MATEMATYKI

CZERWIEC 2012

POZIOM PODSTAWOWY

1. Sprawd $\acute{\mathrm{z}}$, czy arkusz egzaminacyjny zawiera 18 stron

(zadania $1-34$). Ewentualny brak zgłoś przewodniczącemu

zespo nadzorującego egzamin.

2. Rozwiązania zadań i odpowiedzi wpisuj w miejscu na to

przeznaczonym.

3. Odpowiedzi do zadań za niętych (l-24) przenieś

na ka ę odpowiedzi, zaznaczając je w części ka $\mathrm{y}$

przeznaczonej dla zdającego. Zamaluj $\blacksquare$ pola do tego

przeznaczone. Błędne zaznaczenie otocz kółkiem \fcircle$\bullet$

i zaznacz właściwe.

4. Pamiętaj, $\dot{\mathrm{z}}\mathrm{e}$ pominięcie argumentacji lub istotnych

obliczeń w rozwiązaniu zadania otwa ego (25-34) $\mathrm{m}\mathrm{o}\dot{\mathrm{z}}\mathrm{e}$

spowodować, $\dot{\mathrm{z}}\mathrm{e}$ za to rozwiązanie nie będziesz mógł

dostać pełnej liczby punktów.

5. Pisz czytelnie i uzywaj tvlko długopisu lub -Dióra

z czarnym tuszem lub atramentem.

6. Nie uzywaj korektora, a błędne zapisy wyrazínie prze eśl.

7. Pamiętaj, $\dot{\mathrm{z}}\mathrm{e}$ zapisy w brudnopisie nie będą oceniane.

8. $\mathrm{M}\mathrm{o}\dot{\mathrm{z}}$ esz korzystać z zestawu wzorów matematycznych,

cyrkla i linijki oraz kalkulatora.

9. Na tej stronie oraz na karcie odpowiedzi wpisz swój

numer PESEL i przyklej naklejkę z kodem.

10. Nie wpisuj $\dot{\mathrm{z}}$ adnych znaków w części przeznaczonej

dla egzaminatora.

Czas pracy:

170 minut

Liczba punktów

do uzyskania: 50

$\Vert\Vert\Vert\Vert\Vert\Vert\Vert\Vert\Vert\Vert\Vert\Vert\Vert\Vert\Vert\Vert\Vert\Vert\Vert\Vert\Vert\Vert\Vert\Vert|  \mathrm{M}\mathrm{M}\mathrm{A}-\mathrm{P}1_{-}1\mathrm{P}-123$




{\it 2}

{\it Egzamin maturalny z matematyki}

{\it Poziom podstawowy}

ZADANIA ZAMKNIĘTE

{\it Wzadaniach} $\theta d1.$ {\it do 24. wybierz i zaznacz na karcie odpowiedzipoprawnq odpowiedzí}.

Zadanie l. $(1pkt)$

Ułamek $\displaystyle \frac{\sqrt{5}+2}{\sqrt{5}-2}$ jest równy

A. 1 B. $-1$

C. $7+4\sqrt{5}$

D. $9+4\sqrt{5}$

Zadanie 2. $(1pkt)$

Liczbami spełniającymi równanie $|2x+3|=5$ są

A. $1\mathrm{i}-4$

B. l i 2

C. $-1\mathrm{i}4$

D. $-2\mathrm{i}2$

Zadanie 3. $(1pkt)$

Równanie $(x+5)(x-3)(x^{2}+1)=0$ ma

A.

B.

C.

D.

dwa rozwiązania: $x=-5, x=3.$

dwa rozwiązania: $x=-3, x=5.$

cztery rozwiązania: $x=-5, x=-1, x=1, x=3.$

cztery rozwiązania: $x=-3, x=-1, x=1, x=5.$

Zadanie 4. (1pkt)

Marza równa 1,5\% kwoty pozyczonego kapitału była równa 3000 zł.

pozyczono

Wynika stąd, $\dot{\mathrm{z}}\mathrm{e}$

A. 45 zł

B. 2000 zł

C. 200000 zł

D. 450000 zł

Zadanie 5. $(1pkt)$

Najednym z ponizszych rysunków przedstawiono fragment wykresu funkcji $y=x^{2}+2x-3.$

Wskaz ten rysunek.
\begin{center}
\includegraphics[width=4.932mm,height=22.812mm]{./F1_M_PP_C2012_page1_images/image001.eps}

\begin{tabular}{|l|l|}
\hline
\multicolumn{1}{|l|}{ $\begin{array}{l}\mbox{$4$}	\\	\mbox{ $3$}	\\	\mbox{ $2$}	\\	\mbox{ $1$}	\end{array}$}&	\multicolumn{1}{|l|}{ $\mathrm{y}$}	\\
\hline
\multicolumn{1}{|l|}{ $\begin{array}{l}\mbox{ $-4-2-1$}	\\	\mbox{ $-1$}	\\	\mbox{ $-2$}	\\	\mbox{ $-3$}	\\	\mbox{ $-4$}	\end{array}$}&	\multicolumn{1}{|l|}{ $234$}	\\
\hline
\end{tabular}


\begin{tabular}{|l|l|}
\hline
\multicolumn{1}{|l|}{ $\begin{array}{l}\mbox{$4$}	\\	\mbox{ $3$}	\\	\mbox{ $1$}	\end{array}$}&	\multicolumn{1}{|l|}{ $\mathrm{y}$}	\\
\hline
\multicolumn{1}{|l|}{ $-4-3-21^{1}4321$}&	\multicolumn{1}{|l|}{ $124$}	\\
\hline
\end{tabular}


\begin{tabular}{|l|l|}
\hline
\multicolumn{1}{|l|}{ $\begin{array}{l}\mbox{$4$}	\\	\mbox{ $3$}	\\	\mbox{ $2$}	\\	\mbox{ $1$}	\end{array}$}&	\multicolumn{1}{|l|}{ $\mathrm{y}$}	\\
\hline
\multicolumn{1}{|l|}{ $\begin{array}{l}\mbox{ $-43-2-1$}	\\	\mbox{ $-1$}	\\	\mbox{ $-2$}	\\	\mbox{ $-3$}	\\	\mbox{ $-4$}	\end{array}$}&	\multicolumn{1}{|l|}{ $234$}	\\
\hline
\end{tabular}


\includegraphics[width=4.932mm,height=22.812mm]{./F1_M_PP_C2012_page1_images/image002.eps}

\begin{tabular}{|l|l|}
\hline
\multicolumn{1}{|l|}{ $\begin{array}{l}\mbox{$4$}	\\	\mbox{ $3$}	\\	\mbox{ $2$}	\\	\mbox{ $1$}	\end{array}$}&	\multicolumn{1}{|l|}{ $\mathrm{y}$}	\\
\hline
\multicolumn{1}{|l|}{ $\begin{array}{l}\mbox{ $-4-3-2$}	\\	\mbox{ $-1$}	\\	\mbox{ $-3$}	\\	\mbox{ $-4$}	\end{array}$}&	\multicolumn{1}{|l|}{ $124$}	\\
\hline
\end{tabular}


\includegraphics[width=5.232mm,height=22.860mm]{./F1_M_PP_C2012_page1_images/image003.eps}
\end{center}
A.

B.

C.

D.





{\it Egzamin maturalny z matematyki}

{\it Poziom podstawowy}

{\it 11}

Zadanie 27. (2pkt)

Podstawy trapezu prostokątnego mają długości 6 i 10 oraz tangens jego kąta ostrego jest

równy 3. Ob1icz po1e tego trapezu.

Odpowiedzí :

Zadanie 28. $(2pkt)$

Uzasadnij, $\dot{\mathrm{z}}$ ejezeli $\alpha$ jest kątem ostrym, to $\sin^{4}\alpha+\cos^{2}\alpha=\sin^{2}\alpha+\cos^{4}\alpha.$





{\it 12}

{\it Egzamin maturalny z matematyki}

{\it Poziom podstawowy}

Zadanie 29. $(2pkt)$

Uzasadnij, $\dot{\mathrm{z}}\mathrm{e}$ suma kwadratów trzech kolejnych liczb całkowitych przy dzieleniu przez 3 daje

resztę 2.

Zadanie 30. $(2pkt)$

Suma $S_{n}=a_{1}+a_{2}+\ldots+a_{n}$ początkowych $n$ wyrazów pewnego ciągu arytmetycznego $(a_{n})$

jest określona wzorem $S_{n}=n^{2}-2n$ dla $n\geq 1$. Wyznacz wzór na n-ty wyraz tego ciągu.

Odpowied $\acute{\mathrm{z}}$:





{\it Egzamin maturalny z matematyki}

{\it Poziom podstawowy}

{\it 13}

Zadanie 31. $(2pkt)$

Dany jest romb, którego kąt ostry ma miarę $45^{\mathrm{o}}$, a jego pole jest równe $50\sqrt{2}$. Oblicz

wysokość tego rombu.

Odpowied $\acute{\mathrm{z}}$:





{\it 14}

{\it Egzamin maturalny z matematyki}

{\it Poziom podstawowy}

Zadanie 32. $(4pkt)$

Punkty $A=(2,11), B=(8,23), C=(6,14)$ są wierzchołkami trójkąta. Wysokość trójkąta

poprowadzona z wierzchołka $C$ przecina prostą AB w punkcie $D$. Oblicz współrzędne punktu $D.$

Odpowied $\acute{\mathrm{z}}$:





{\it Egzamin maturalny z matematyki}

{\it Poziom podstawowy}

{\it 15}

Zadanie 33. (4pkt)

Oblicz, ile jest liczb naturalnych pięciocyfrowych, w zapisie których nie występuje zero, jest

dokładniejedna cyfra 7 i dokładniejedna cyfra parzysta.

Odpowiedzí :





{\it 16}

{\it Egzamin maturalny z matematyki}

{\it Poziom podstawowy}

Zadanie 34. (4pkt)

Dany jest graniastosłup prawidłowy trójkątny ABCDEF o podstawach ABC i DEF

i krawędziach bocznych AD, BE iCF (zobacz rysunek). Długość krawędzi podstawy AB jest

równa 8, a po1e trójkąta ABFjest równe 52. Ob1icz objętość tego graniastosłupa.





{\it Egzamin maturalny z matematyki}

{\it Poziom podstawowy}

{\it 1}7

Odpowied $\acute{\mathrm{z}}$:





{\it 18}

{\it Egzamin maturalny z matematyki}

{\it Poziom podstawowy}

BRUDNOPIS





{\it Egzamin maturalny z matematyki}

{\it Poziom podstawowy}

{\it 3}

BRUDNOPIS





{\it 4}

{\it Egzamin maturalny z matematyki}

{\it Poziom podstawowy}

Zadanie 6. $(1pkt)$

Wierzchołkiem paraboli będącej wykresem ffinkcji określonej wzorem $f(x)=x^{2}-4x+4$

jest punkt o współrzędnych

A. (0,2)

B. $(0,-2)$

C. $(-2,0)$

D. (2, 0)

Zadanie 7. $(1pkt)$

Jeden kąt trójkąta ma miarę $54^{\mathrm{o}} \mathrm{Z}$ pozostałych dwóch kątów tego trójkątajedenjest 6 razy

większy od drugiego. Miary pozostałych kątów są równe

A. $21^{\mathrm{o}}$ i $105^{\mathrm{o}}$

B. $11^{\mathrm{o}}$ i $66^{\mathrm{o}}$

C. $18^{\mathrm{o}}$ i $108^{\mathrm{o}}$

D. $16^{\mathrm{o}}\mathrm{i}96^{\mathrm{o}}$

Zadanie 8. $(1pkt)$

Krótszy bok prostokąta ma długość 6. Kąt między przekątną prostokąta i dłuzszym bokiem

ma miarę $30^{\mathrm{o}}$. Dłuzszy bok prostokąta ma długość

A. $2\sqrt{3}$

B. $4\sqrt{3}$

C. $6\sqrt{3}$

D. 12

Zadanie 9. (1pkt)

Cięciwa okręgu ma długość 8 cm ijest odda1ona odjego środka o 3 cm. Promień tego okręgu

ma długość

A. 3 cm

B. 4 cm

C. 5 cm

D. 8 cm

Zadanie 10. (1pkt)

Punkt O jest środkiem okręgu. Kąt wpisany BAD ma miarę

A. $150^{\mathrm{o}}$
\begin{center}
\includegraphics[width=44.040mm,height=46.380mm]{./F1_M_PP_C2012_page3_images/image001.eps}
\end{center}
{\it D  C}

$130^{\circ}$

{\it O}

$60^{\circ}$

{\it B}

{\it A}

$115^{\mathrm{o}}$

$120^{\mathrm{o}}$

C.

B.

D. $85^{\mathrm{o}}$

Zadanie ll. (lpkt)

Pięciokąt ABCDE jest foremny. Wskaz trójkąt przystający do trójkąta ECD

A.

$\Delta ABF$

B.

$\Delta CAB$
\begin{center}
\includegraphics[width=55.884mm,height=50.088mm]{./F1_M_PP_C2012_page3_images/image002.eps}
\end{center}
{\it D}

{\it E  I H  C}

{\it J  G}

{\it F}

{\it A B}

$\Delta ABD$

D.

$\Delta IHD$

C.





{\it Egzamin maturalny z matematyki}

{\it Poziom podstawowy}

{\it 5}

BRUDNOPIS





{\it 6}

{\it Egzamin maturalny z matematyki}

{\it Poziom podstawowy}

Zadanie 12. (1pkt)

Punkt O jest środkiem okręgu przedstawionego na rysunku. Równanie tego okręgu ma postać:

A.

B.
\begin{center}
\includegraphics[width=65.436mm,height=64.104mm]{./F1_M_PP_C2012_page5_images/image001.eps}
\end{center}
y

4

2

{\it o}

$-1$  1 2  3 4

x

5

$-2$

D.

C.

Zadanie 13. $(1pkt)$

Wyra $\dot{\mathrm{z}}$ enie $\displaystyle \frac{3x+1}{x-2}-\frac{2x-1}{x+3}$ jest równe

A.

-({\it xx}2-$+$21)5({\it xx} $++$31)

B.

$\displaystyle \frac{x+2}{(x-2)(x+3)}$

$(x-2)^{2}+(y-1)^{2}=9$

$(x-2)^{2}+(y-1)^{2}=3$

$(x+2)^{2}+(y+1)^{2}=9$

$(x+2)^{2}+(y+1)^{2}=3$

C. $\displaystyle \frac{x}{(x-2)(x+3)}$

D.

$\displaystyle \frac{x+2}{-5}$

Zadanie 14. $(1pkt)$

Ciąg $(a_{n})$ jest określony wzorem $a_{n}=\sqrt{2n+4}$ dla $n\geq 1$. Wówczas

A. $a_{8}=2\sqrt{5}$

B. $a_{8}=8$

C. $a_{8}=5\sqrt{2}$

D. $a_{8}=\sqrt{12}$

Zadanie 15. $(1pkt)$

Ciąg $(2\sqrt{2},4,a)$ jest geometryczny. Wówczas

A. $a=8\sqrt{2}$

B. $a=4\sqrt{2}$

C. $a=8-2\sqrt{2}$

D. $a=8+2\sqrt{2}$

Zadanie 16. $(1pkt)$

Kąt $\alpha$ jest ostry i $\mathrm{t}\mathrm{g}\alpha=1$. Wówczas

A. $\alpha<30^{\mathrm{o}}$

B. $\alpha=30^{\mathrm{o}}$

C. $\alpha=45^{\mathrm{o}}$

D. $\alpha>45^{\mathrm{o}}$

Zadanie 17. (1pkt)

Wiadomo, $\dot{\mathrm{z}}\mathrm{e}$ dziedziną funkcji

$(-\infty,2)\cup(2,+\infty)$. Wówczas

$f$ określonej wzorem $f(x)=\displaystyle \frac{x-7}{2x+a}$ jest zbiór

A. $a=2$

B. $a=-2$

C. $a=4$

D. $a=-4$





{\it Egzamin maturalny z matematyki}

{\it Poziom podstawowy}

7

BRUDNOPIS





{\it 8}

{\it Egzamin maturalny z matematyki}

{\it Poziom podstawowy}

Zadanie 18. $(1pkt)$

Jeden z rysunków przedstawia wykres ffinkcji liniowej $f(x)=ax+b$, gdzie $a>0\mathrm{i}b<0$. Wskaz

ten wykres.
\begin{center}
\includegraphics[width=45.108mm,height=45.108mm]{./F1_M_PP_C2012_page7_images/image001.eps}
\end{center}
$\gamma$

{\it x}

0
\begin{center}
\includegraphics[width=45.012mm,height=51.864mm]{./F1_M_PP_C2012_page7_images/image002.eps}
\end{center}
$\gamma$

{\it x}

0

D.

A.
\begin{center}
\includegraphics[width=45.156mm,height=51.864mm]{./F1_M_PP_C2012_page7_images/image003.eps}
\end{center}
$\gamma$

{\it x}

0

B.
\begin{center}
\includegraphics[width=44.904mm,height=51.912mm]{./F1_M_PP_C2012_page7_images/image004.eps}
\end{center}
$\gamma$

{\it x}

0

C.

Zadanie 19. $(1pkt)$

Punkt $S=(2,7)$ jest środkiem odcinka $AB$, w którym $A=(-1,3)$. Punkt $B$ ma współrzędne:

A. $B=(5,11)$ B. $B=(\displaystyle \frac{1}{2},2)$ C. $B=(-\displaystyle \frac{3}{2},-5)$ D. $B=(3,11)$

Zadanie 20. $(1pkt)$

$\mathrm{W}$ kolejnych sześciu rzutach kostką otrzymano następujące wyniki: 6, 3, 1, 2, 5, 5. Mediana

tych wynikówjest równa:

A. 3

B. 3,5

C. 4

D. 5

Zadanie 21. $(1pkt)$

RównoŚć $(a+2\sqrt{2})^{2}=a^{2}+28\sqrt{2}+8$ zachodzi dla

A. $a=14$ B. $a=7\sqrt{2}$ C.

$a=7$

D. $a=2\sqrt{2}$

Zadanie 22. (1pkt)

Trójkąt prostokątny o przyprostokątnych 4 i 6 obracamy wokół dłuzszej przyprostokątnej.

Objętość powstałego stozkajest równa

A. $ 96\pi$

B. $ 48\pi$

C. $ 32\pi$

D. $ 8\pi$

Zadanie 23. $(1pkt)$

$\mathrm{J}\mathrm{e}\dot{\mathrm{z}}$ eli $A \mathrm{i} B$ są zdarzeniami losowymi, $B'$ jest zdarzeniem przeciwnym do $B, P(A)=0,3,$

$P(B')=0,4$ oraz $ A\cap B=\emptyset$, to $P(A\cup B)$ jest równe

A. 0,12

B. 0,18

C. 0,6

D. 0,9

Zadanie 24. $(1pkt)$

Przekrój osiowy walca jest kwadratem o boku $a. \mathrm{J}\mathrm{e}\dot{\mathrm{z}}$ eli $r$ oznacza promień podstawy walca,

$h$ oznacza wysokość walca, to

A. $r+h=a$

B.

$h-r=\displaystyle \frac{a}{2}$

C.

{\it r-h}$=$ -{\it a}2

D. $r^{2}+h^{2}=a^{2}$





{\it Egzamin maturalny z matematyki}

{\it Poziom podstawowy}

{\it 9}

BRUDNOPIS





$ 1\theta$

{\it Egzamin maturalny z matematyki}

{\it Poziom podstawowy}

ZADANIA OTWARTE

{\it Rozwiqzania zadań o numerach od 25. do 34. nalezy zapisać w} $wyznacz\theta nych$ {\it miejscach}

{\it pod treściq zadania}.

Zadanie 25. $(2pkt)$

Rozwiąz nierówność $x^{2}-3x-10<0.$

Odpowiedz:

Zadanie 26. (2pkt)

Średnia wieku w pewnej giupie studentówjest równa 231ata. Średnia wieku tych studentów

i ich opiekunajest równa 241ata. Opiekun ma 391at. Ob1icz, i1u studentówjest w tej giupie.

Odpowiedzí:





\begin{center}
\begin{tabular}{l|l}
\multicolumn{1}{l|}{{\it dysleksja}}&	\multicolumn{1}{|l}{}	\\
\hline
\multicolumn{1}{l|}{ $\begin{array}{l}\mbox{MATERIAL DIAGNOSTYCZNY}	\\	\mbox{Z MATEMATYKI}	\\	\mbox{Arkusz I}	\\	\mbox{POZIOM PODSTAWOWY}	\\	\mbox{Czas pracy 120 minut}	\\	\mbox{Instrukcja dla ucznia}	\\	\mbox{1. $\mathrm{S}\mathrm{p}\mathrm{r}\mathrm{a}\mathrm{w}\mathrm{d}\acute{\mathrm{z}}$, czy arkusz zawiera 12 ponumerowanych stron.}	\\	\mbox{Ewentualny brak zgłoś przewodniczącemu zespo}	\\	\mbox{nadzorującego badanie.}	\\	\mbox{2. Rozwiązania i odpowiedzi zapisz w miejscu na to}	\\	\mbox{przeznaczonym.}	\\	\mbox{3. $\mathrm{W}$ rozwiązaniach zadań przedstaw tok rozumowania}	\\	\mbox{prowadzący do ostatecznego wyniku.}	\\	\mbox{4. Pisz czytelnie. Uzywaj długopisu pióra tylko z czamym}	\\	\mbox{tusze atramentem.}	\\	\mbox{5. Nie uzywaj korektora, a błędne zapisy $\mathrm{w}\mathrm{y}\mathrm{r}\mathrm{a}\acute{\mathrm{z}}\mathrm{n}\mathrm{i}\mathrm{e}$ prze eśl.}	\\	\mbox{6. Pamiętaj, $\dot{\mathrm{z}}\mathrm{e}$ zapisy w brudnopisie nie podlegają ocenie.}	\\	\mbox{7. $\mathrm{M}\mathrm{o}\dot{\mathrm{z}}$ esz korzystać z zestawu wzorów matematycznych, cyrkla}	\\	\mbox{i linijki oraz kalkulatora.}	\\	\mbox{8. Wypełnij tę część ka $\mathrm{y}$ odpowiedzi, którą koduje uczeń. Nie}	\\	\mbox{wpisuj $\dot{\mathrm{z}}$ adnych znaków w części przeznaczonej dla}	\\	\mbox{oceniającego.}	\\	\mbox{9. Na karcie odpowiedzi wpisz swoją datę urodzenia i PESEL.}	\\	\mbox{Zamaluj $\blacksquare$ pola odpowiadające cyfrom numeru PESEL. Błędne}	\\	\mbox{zaznaczenie otocz kółkiem \fcircle i zaznacz właściwe.}	\\	\mbox{{\it Zyczymy} $p\theta wodzenia'$}	\end{array}$}&	\multicolumn{1}{|l}{$\begin{array}{l}\mbox{ARKUSZ I}	\\	\mbox{GRUDZIEN}	\\	\mbox{ROK 2005}	\\	\mbox{Za rozwiązanie}	\\	\mbox{wszystkich zadań}	\\	\mbox{mozna otrzymać}	\\	\mbox{łącznie}	\\	\mbox{50 punktów}	\end{array}$}	\\
\hline
\multicolumn{1}{l|}{$\begin{array}{l}\mbox{W ełnia uczeń rzed roz oczęciem rac}	\\	\mbox{PESEL UCZNIA}	\end{array}$}&	\multicolumn{1}{|l}{$\begin{array}{l}\mbox{Wypełnia uczeń}	\\	\mbox{przed rozpoczęciem}	\\	\mbox{pracy}	\\	\mbox{KOD UCZNIA}	\end{array}$}
\end{tabular}


\includegraphics[width=80.724mm,height=12.756mm]{./F1_M_PP_G2005_page0_images/image001.eps}

\includegraphics[width=23.616mm,height=9.852mm]{./F1_M_PP_G2005_page0_images/image002.eps}
\end{center}



{\it 2}

{\it Materialpomocniczy do doskonalenia nauczycieli w zakresie diagnozowania, oceniania i egzaminowania}

{\it Matematyka}- {\it grudzień 2005 r}.

Zadanie l. $(4pkt)$

Wielomian $P(x)=x^{3}-21x+20$ rozłóz na czynniki liniowe, to znaczy zapisz go w postaci

iloczynu trzech wielomianów stopnia pierwszego.





{\it Materialpomocniczy do doskonalenia nauczycieli w zakresie diagnozowania, oceniania i egzaminowania ll}

{\it Matematyka}- {\it grudzień 2005 r}.

Zadanie 10. $(7pkt)$

Pole powierzchni całkowitej prawidłowego ostrosłupa trójkątnego równa

a polejego powierzchni bocznej $96\sqrt{3}$. Oblicz objętość tego ostrosłupa.

się $144\sqrt{3},$





{\it 12 Materiatpomocniczy do doskonalenia nauczycieli w zakresie diagnozowania, oceniania i egzaminowania}

{\it Matematyka}- {\it grudzień 2005 r}.

BRUDNOPIS





{\it Materialpomocniczy do doskonalenia nauczycieli w zakresie diagnozowania, oceniania i egzaminowania}

{\it Matematyka}- {\it grudzień 2005 r}.

{\it 3}

Zadanie 2. (4pkt)

W roku 2005 na uroczystości urodzin zapytano jubi1ata, i1e ma 1at.

Jubilat odpowiedział:,,Jeśli swój wiek sprzed 101at pomnozę przez swój wiek za 111at,

to otrzymam rok mojego urodzenia'' Ułóz odpowiednie równanie, rozwiąz je i zapisz,

w którym roku urodził się tenjubilat.





{\it 4}

{\it Materialpomocniczy do doskonalenia nauczycieli w zakresie diagnozowania, oceniania i egzaminowania}

{\it Matematyka}- {\it grudzień 2005 r}.

Zadanie 3. $(5pkt)$

Funkcja $f(x)$ jest określona wzorem: $f(x)=$

a) Sprawd $\acute{\mathrm{z}}$, czy liczba $a=(0,25)^{-0,5}$ nalezy do dziedziny funkcji $f(x).$

b) Oblicz $f(2)$ oraz $f(3).$

c) Sporządz$\acute{}$ wykres funkcji $f(x).$

d) Podaj rozwiązanie równania $f(x)=0.$

e) Zapisz zbiór wartości funkcji $f(x).$





{\it Materialpomocniczy do doskonalenia nauczycieli w zakresie diagnozowania, oceniania i egzaminowania}

{\it Matematyka}- {\it grudzień 2005 r}.

{\it 5}

Zadanie 4. $(6pkt)$

$\mathrm{W}$ układzie współrzędnych są dane dwa punkty: $A=(-2,2)\mathrm{i}B=(4,4).$

a) Wyznacz równanie prostej $AB.$

b) Prosta $AB$ oraz prosta o równaniu $9x-6y-26=0$ przecinają się w punkcie

Oblicz współrzędne punktu $C.$

c) Wyznacz równanie symetralnej odcinka $AB.$

{\it C}.





{\it 6}

{\it Materialpomocniczy do doskonalenia nauczycieli w zakresie diagnozowania, oceniania i egzaminowania}

{\it Matematyka}- {\it grudzień 2005 r}.

Zadanie 5. $(5pkt)$

Nieskończony ciąg liczbowy $(a_{n})$ jest określony wzorem $a_{n}=4n-31, n=1,2,3,\ldots.$

Wyrazy $a_{k}, a_{k+1}, a_{k+2}$ danego ciągu $(a_{n})$, wzięte w takim porządku, powiększono: wyraz

$a_{k} 01$, wyraz $a_{k+1} 03$ oraz wyraz $a_{k+2}023. \mathrm{W}$ ten sposób otrzymano trzy pierwsze wyrazy

pewnego ciągu geometrycznego. Wyznacz $k$ oraz czwarty wyraz tego ciągu geometrycznego.





{\it Materialpomocniczy do doskonalenia nauczycieli w zakresie diagnozowania, oceniania i egzaminowania}

{\it Matematyka}- {\it grudzień 2005 r}.

7

Zadanie 6. $(4pkt)$

Do szkolnych zawodów szachowych zgłosiło się 16 uczniów, wśród których było dwóch

faworytów. Organizatorzy zawodów zamierzają losowo podzielić szachistów na dwie

jednakowo liczne grupy eliminacyjne, Niebieską i Zółtą. Oblicz prawdopodobieństwo

zdarzenia polegającego na tym, $\dot{\mathrm{z}}\mathrm{e}$ faworyci tych zawodów nie znajdą się w tej samej grupie

eliminacyjnej. Końcowy wynik obliczeń zapisz w postaci ułamka nieskracalnego.





{\it 8}

{\it Materialpomocniczy do doskonalenia nauczycieli w zakresie diagnozowania, oceniania i egzaminowania}

{\it Matematyka}- {\it grudzień 2005 r}.

Zadanie 7. $(3pkt)$

Aby wyznaczyć wszystkie liczby całkowite $c$, dla których liczba postaci $\displaystyle \frac{c-3}{c-5}$ jest takz $\mathrm{e}$

liczbą całkowitą mozna postąpić w następujący sposób:

a) Wyrazenie w liczniku ułamka zapisujemy w postaci sumy, której jednym

ze składnikówjest wyrazenie z mianownika:

$\displaystyle \frac{c-3}{c-5}=\frac{(c-5)+2}{c-5}$

b) Zapisujemy powyzszy ułamek w postaci sumy liczby l oraz pewnego ułamka:

$\displaystyle \frac{c-5+2}{c-5}=\frac{c-5}{c-5}+\frac{2}{c-5}=1+\frac{2}{c-5}$

c) Zauwazamy, $\dot{\mathrm{z}}\mathrm{e}$ ułamek $\displaystyle \frac{2}{c-5}$ jest liczbą całkowitą wtedy i tylko wtedy, gdy liczba

$(c-5)$ jest całkowitym dzielnikiem liczby 2, czy1i $\dot{\mathrm{z}}\mathrm{e}(c-5)\in\{-1,1,-2,2\}.$

d) Rozwiązujemy kolejno równania $c-5=-1, c-5=1, c-5=-2, c-5=2,$

i otrzymujemy odpowiedzí: liczba postaci $\displaystyle \frac{c-3}{c-5}$ jest całkowita dla:

$c=4,c=6,c=3,c=7.$

Rozumując analogicznie, wyznacz wszystkie liczby całkowite $x$, dla których liczba postaci

$\displaystyle \frac{x}{x-3}$ jest liczbą całkowitą.





{\it Materialpomocniczy do doskonalenia nauczycieli w zakresie diagnozowania, oceniania i egzaminowania}

{\it Matematyka}- {\it grudzień 2005 r}.

{\it 9}

Zadanie 8. $(5pkt)$

$\mathrm{W}$ kwadrat ABCD wpisano kwadrat EFGH, jak pokazano na ponizszym rysunku. Wiedząc,

$\dot{\mathrm{z}}\mathrm{e}|AB|=1$ oraz tangens kąta $AEH$ równa się $\displaystyle \frac{2}{5}$, oblicz pole kwadratu EFGH.

{\it A}
\begin{center}
\includegraphics[width=95.304mm,height=93.168mm]{./F1_M_PP_G2005_page8_images/image001.eps}
\end{center}
{\it D  G  C}

{\it F}

{\it H}

{\it E B}





$ 1\theta$ {\it Materiatpomocniczy do doskonalenia nauczycieli w zakresie diagnozowania, oceniania i egzaminowania}

{\it Matematyka}- {\it grudzień 2005} $r.$

Zadanie 9. $(7pkt)$

Liczbę naturalną $t_{n}$ nazywamy $n$ -tą liczbą trójkątn\% $\mathrm{j}\mathrm{e}\dot{\mathrm{z}}$ eli jest ona sumą $n$

kolejnych,

początkowych liczb naturalnych. Liczbami trójkątnymi są zatem: $t_{1}=1, t_{2}=1+2=3,$

$t_{3}=1+2+3=6, t_{4}=1+2+3+4=10, t_{5}=1+2+3+4+5=15$. Stosując tę definicję:

a) wyznacz liczbę $t_{17}.$

b) ułóz odpowiednie równanie i zbadaj, czy liczba $7626$jest liczbą trójkątną.

c) wyznacz największą czterocyfrową liczbę trójkątną.






\begin{center}
\begin{tabular}{l|l}
\multicolumn{1}{l|}{$\begin{array}{l}\mbox{{\it dysleksja}}	\\	\mbox{Miejsce}	\\	\mbox{na na ejkę}	\\	\mbox{z kodem szkoly}	\end{array}$}&	\multicolumn{1}{|l}{}	\\
\hline
\multicolumn{1}{l|}{ $\begin{array}{l}\mbox{PRÓBNY EGZAMIN}	\\	\mbox{MATURALNY}	\\	\mbox{Z MATEMATYKI}	\\	\mbox{POZIOM PODSTAWOWY}	\\	\mbox{Czas pracy 120 minut}	\\	\mbox{Instrukcja dla zdającego}	\\	\mbox{1. $\mathrm{S}\mathrm{p}\mathrm{r}\mathrm{a}\mathrm{w}\mathrm{d}\acute{\mathrm{z}}$, czy arkusz egzaminacyjny zawiera 15 stron}	\\	\mbox{(zadania $1-11$). Ewentualny brak zgłoś przewodniczącemu}	\\	\mbox{zespo nadzorującego egzamin.}	\\	\mbox{2. Rozwiązania zadań i odpowiedzi zamieść w miejscu na to}	\\	\mbox{przeznaczonym.}	\\	\mbox{3. $\mathrm{W}$ rozwiązaniach zadań przedstaw tok rozumowania}	\\	\mbox{prowadzący do ostatecznego wyniku.}	\\	\mbox{4. Pisz czytelnie. Uzywaj długopisu pióra tylko z czamym}	\\	\mbox{tusze atramentem.}	\\	\mbox{5. Nie uzywaj korektora, a błędne zapisy prze eśl.}	\\	\mbox{6. Pamiętaj, $\dot{\mathrm{z}}\mathrm{e}$ zapisy w brudnopisie nie podlegają ocenie.}	\\	\mbox{7. $\mathrm{M}\mathrm{o}\dot{\mathrm{z}}$ esz korzystać z zestawu wzorów matematycznych, cyrkla}	\\	\mbox{i linijki oraz kalkulatora.}	\\	\mbox{8. Wypełnij tę część ka $\mathrm{y}$ odpowiedzi, którą koduje zdający.}	\\	\mbox{Nie wpisuj $\dot{\mathrm{z}}$ adnych znaków w części przeznaczonej dla}	\\	\mbox{egzaminatora.}	\\	\mbox{9. Na karcie odpowiedzi wpisz swoją datę urodzenia i PESEL.}	\\	\mbox{Zamaluj $\blacksquare$ pola odpowiadające cyfrom numeru PESEL. Błędne}	\\	\mbox{zaznaczenie otocz kółkiem $\mathrm{O}$ i zaznacz właściwe.}	\\	\mbox{{\it Zyczymy} $p\theta wodzenia'$}	\end{array}$}&	\multicolumn{1}{|l}{$\begin{array}{l}\mbox{LISTOPAD}	\\	\mbox{ROK 2006}	\\	\mbox{Za rozwiązanie}	\\	\mbox{wszystkich zadań}	\\	\mbox{mozna otrzymać}	\\	\mbox{łącznie}	\\	\mbox{50 punktów}	\end{array}$}	\\
\hline
\multicolumn{1}{l|}{$\begin{array}{l}\mbox{Wypelnia zdający przed}	\\	\mbox{roz oczęciem racy}	\\	\mbox{PESEL ZDAJACEGO}	\end{array}$}&	\multicolumn{1}{|l}{$\begin{array}{l}\mbox{KOD}	\\	\mbox{ZDAJACEGO}	\end{array}$}
\end{tabular}


\includegraphics[width=21.840mm,height=9.852mm]{./F1_M_PP_L2006_page0_images/image001.eps}

\includegraphics[width=78.792mm,height=13.356mm]{./F1_M_PP_L2006_page0_images/image002.eps}
\end{center}



{\it 2}

{\it Próbny egzamin maturalny z matematyki}

{\it Poziom podstawowy}

Zadanie 1. (3pkt)

Wzrost kursu euro w stosunku do złotego spowodował podwyzkę ceny wycieczki

zagranicznej o 5\%. Poniewaz nowa cena nie była zachęcająca, postanowiono obnizyć ją

0 8\%, ustalając cenę promocyjną równą l449 zł. Oblicz pierwotną cenę wycieczki dla

jednego uczestnika.





{\it Próbny egzamin maturalny z matematyki}

{\it Poziom podstawowy}

{\it 11}

Zadanie 9. $(4pkt)$

Nauczyciele informatyki, chcąc wyłonić reprezentację szkoły na wojewódzki konkurs

informatyczny, przeprowadzili w klasach I A i I $\mathrm{B}$ test z zakresu poznanych wiadomości.

$\mathrm{K}\mathrm{a}\dot{\mathrm{z}}\mathrm{d}\mathrm{y}$ z nich przygotował zestawienie wyników swoich uczniów w innej formie.

Na podstawie analizy przedstawionych ponizej wyników obu klas:

a) oblicz średni wynik z testu $\mathrm{k}\mathrm{a}\dot{\mathrm{z}}$ dej klasy,

b) oblicz, ile procent uczniów klasy I $\mathrm{B}$ uzyskało wynik $\mathrm{w}\mathrm{y}\dot{\mathrm{z}}$ szy $\mathrm{n}\mathrm{i}\dot{\mathrm{z}}$ średni w swojej klasie,

c) podaj medianę wyników uzyskanych w klasie I A.

$\mathrm{w}\mathfrak{n}\mathrm{i}\mathrm{k}\mathrm{l}$ testu $\displaystyle \inf \mathrm{u}-\mathrm{i}\acute{\mathrm{r}}$ kl. lA
\begin{center}
\includegraphics[width=98.244mm,height=74.832mm]{./F1_M_PP_L2006_page10_images/image001.eps}
\end{center}
5

4

1

0

0 1

2 3 4 5 6 7 8

1J
\begin{center}
\begin{tabular}{|l|l|}
\hline
\multicolumn{1}{|l|}{Liczba punktów}&	\multicolumn{1}{|l|}{Liczba uczniów}	\\
\hline
\multicolumn{1}{|l|}{$0$}&	\multicolumn{1}{|l|}{ $1$}	\\
\hline
\multicolumn{1}{|l|}{ $1$}&	\multicolumn{1}{|l|}{ $2$}	\\
\hline
\multicolumn{1}{|l|}{ $2$}&	\multicolumn{1}{|l|}{ $1$}	\\
\hline
\multicolumn{1}{|l|}{ $3$}&	\multicolumn{1}{|l|}{ $2$}	\\
\hline
\multicolumn{1}{|l|}{ $4$}&	\multicolumn{1}{|l|}{ $1$}	\\
\hline
\multicolumn{1}{|l|}{ $5$}&	\multicolumn{1}{|l|}{ $2$}	\\
\hline
\multicolumn{1}{|l|}{ $6$}&	\multicolumn{1}{|l|}{ $4$}	\\
\hline
\multicolumn{1}{|l|}{ $7$}&	\multicolumn{1}{|l|}{ $4$}	\\
\hline
\multicolumn{1}{|l|}{ $8$}&	\multicolumn{1}{|l|}{ $1$}	\\
\hline
\multicolumn{1}{|l|}{ $9$}&	\multicolumn{1}{|l|}{ $2$}	\\
\hline
\multicolumn{1}{|l|}{ $10$}&	\multicolumn{1}{|l|}{ $5$}	\\
\hline
\end{tabular}

\end{center}
Wyniki testu informatycznego

uczniów kl. l B.
\begin{center}
\includegraphics[width=195.168mm,height=145.488mm]{./F1_M_PP_L2006_page10_images/image002.eps}
\end{center}




{\it 12}

{\it Próbny egzamin maturalny z matematyki}

{\it Poziom podstawowy}

Zadanie 10. $(6pkt)$

Dane są zbiory:

$A=\{x\in R:|5-x|\geq 3\}, B=\{x\in R:x^{2}-9\geq 0\} \mathrm{i} C=\displaystyle \{x\in R:\frac{x+1}{x-1}\leq 1\}.$

a) Zaznacz na osi liczbowej zbiory $A, B \mathrm{i}C.$

b) Wyznacz i zapisz za pomocą przedziału liczbowego zbiór $C\backslash (A\cap B).$
\begin{center}
\includegraphics[width=192.072mm,height=72.240mm]{./F1_M_PP_L2006_page11_images/image001.eps}
\end{center}
zbiór A
\begin{center}
\includegraphics[width=192.072mm,height=72.288mm]{./F1_M_PP_L2006_page11_images/image002.eps}
\end{center}
zbiór B
\begin{center}
\includegraphics[width=192.072mm,height=72.240mm]{./F1_M_PP_L2006_page11_images/image003.eps}
\end{center}
zbiór C





{\it Próbny egzamin maturalny z matematyki}

{\it Poziom podstawowy}

{\it 13}
\begin{center}
\includegraphics[width=195.168mm,height=290.784mm]{./F1_M_PP_L2006_page12_images/image001.eps}
\end{center}




{\it 14}

{\it Próbny egzamin maturalny z matematyki}

{\it Poziom podstawowy}

Zadanie ll. $(4pkt)$

Funkcja $f$ przyporządkowuje $\mathrm{k}\mathrm{a}\dot{\mathrm{z}}$ dej liczbie rzeczywistej $x$ z przedziału $\langle-4,-2\rangle$ połowę

kwadratu tej liczby pomniejszoną o 8.

a) Podaj wzór tej funkcji.

b) Wyznacz najmniejszą wartość funkcji $f$ w podanym przedziale.
\begin{center}
\includegraphics[width=195.168mm,height=254.412mm]{./F1_M_PP_L2006_page13_images/image001.eps}
\end{center}




{\it Próbny egzamin maturalny z matematyki}

{\it Poziom podstawowy}

{\it 15}

BRUDNOPIS





{\it Próbny egzamin maturalny z matematyki}

{\it Poziom podstawowy}

{\it 3}

Zadanie 2. $(4pkt)$

Dany jest kwadrat o boku długości $a. \mathrm{W}$ prostokącie ABCD bok $AB$ jest dwa razy dłuzszy $\mathrm{n}\mathrm{i}\dot{\mathrm{z}}$

bok kwadratu, a bok $AD$ jest o 2 cm krótszy od boku kwadratu. Po1e tego prostokąta jest

$012\mathrm{c}\mathrm{m}^{2}$ większe od pola kwadratu. Oblicz długość boku kwadratu.
\begin{center}
\includegraphics[width=195.168mm,height=266.544mm]{./F1_M_PP_L2006_page2_images/image001.eps}
\end{center}




{\it 4}

{\it Próbny egzamin maturalny z matematyki}

{\it Poziom podstawowy}

Zadanie 3. (5pkt)

Z prostokąta o szerokości 60 cm wycina się deta1e w kształcie półko1a o promieniu 60 cm.

Sposób wycinania detali ilustruje ponizszy rysunek.

Oblicz najmniejszą długość prostokąta potrzebnego do wycięcia dwóch takich detali. Wynik

zaokrąglij do pełnego centymetra.
\begin{center}
\includegraphics[width=195.168mm,height=218.136mm]{./F1_M_PP_L2006_page3_images/image001.eps}
\end{center}




{\it Próbny egzamin maturalny z matematyki}

{\it Poziom podstawowy}

{\it 5}

Zadanie 4. $(3pkt)$

Wielomian $W(x)=-2x^{4}+5x^{3}+9x^{2}-15x-9$

Wyznacz pierwiastki tego wielomianu.

jest podzielny przez

dwumian $(2x+1).$
\begin{center}
\includegraphics[width=195.168mm,height=266.544mm]{./F1_M_PP_L2006_page4_images/image001.eps}
\end{center}




{\it 6}

{\it Próbny egzamin maturalny z matematyki}

{\it Poziom podstawowy}

Zadanie 5. $(5pkt)$

Dane sąproste o równaniach $2x-y-3=0\mathrm{i}2x-3y-7=0.$

a) Zaznacz w prostokątnym układzie współrzędnych na płaszczyzínie kąt

układem nierówności 

b) Oblicz odległość punktu przecięcia się tych prostych od punktu $S=(3,-8).$

opisany
\begin{center}
\includegraphics[width=165.204mm,height=151.536mm]{./F1_M_PP_L2006_page5_images/image001.eps}
\end{center}
7 J

5

4

3

2

1

{\it x}

$-7  -5$ -$4  -3$ -$2  -1 0 1$ 2  1 2 3 4 5  7

$-1$

$-2$

$-3$

$-4$

$-5$

$-7$
\begin{center}
\includegraphics[width=195.168mm,height=97.080mm]{./F1_M_PP_L2006_page5_images/image002.eps}
\end{center}




{\it Próbny egzamin maturalny z matematyki}

{\it Poziom podstawowy}

7
\begin{center}
\includegraphics[width=195.168mm,height=290.784mm]{./F1_M_PP_L2006_page6_images/image001.eps}
\end{center}




{\it 8}

{\it Próbny egzamin maturalny z matematyki}

{\it Poziom podstawowy}

Zadanie 6. $(5pkt)$

$\mathrm{W}$ utnie znajdują się kule z kolejnymi liczbami 10, 11, 12, 13, 50, przy czym ku1

z liczbą 10 jest 10, ku1 z 1iczbą 11 jest 11 itd., a ku1 z 1iczbą $50$jest 5$0. \mathrm{Z}$ umy tej losujemy

jedną kulę. Oblicz prawdopodobieństwo, $\dot{\mathrm{z}}\mathrm{e}$ wylosujemy kulę z liczbą parzystą.





{\it Próbny egzamin maturalny z matematyki}

{\it Poziom podstawowy}

{\it 9}

Zadanie 7. $(6pkt)$

$\mathrm{W}$ graniastosłupie prawidłowym czworokątnym przekątna podstawy ma długość 8 cm

i tworzy z przekątną ściany bocznej, z którą ma wspólny wierzchołek kąt, którego cosinus

jest równy $\displaystyle \frac{2}{3}$. Oblicz objętość i pole powierzchni całkowitej tego graniastosłupa.
\begin{center}
\includegraphics[width=195.168mm,height=254.460mm]{./F1_M_PP_L2006_page8_images/image001.eps}
\end{center}




$ 1\theta$

{\it Próbny egzamin maturalny z matematyki}

{\it Poziom podstawowy}

Zadanie 8. $(5pkt)$

Dany jest wykres funkcji $y=f(x)$ określonej dla $x\in\langle-6, 6\rangle.$
\begin{center}
\begin{tabular}{|l|l|}
\hline
\multicolumn{1}{|l|}{ $\begin{array}{l}\mbox{$7$}	\\	\mbox{ $6$}	\\	\mbox{ $5$}	\\	\mbox{ $4$}	\\	\mbox{ $3$}	\\	\mbox{ $2$}	\end{array}$}&	\multicolumn{1}{|l|}{ $\mathrm{y}$}	\\
\hline
\multicolumn{1}{|l|}{ $\begin{array}{l}\mbox{-f $-8 -7 -6 -4 -3 -2$ 1}	\\	\mbox{$-1$}	\\	\mbox{ $-2$}	\\	\mbox{ $-3$}	\\	\mbox{ $-4$}	\\	\mbox{ $-5$}	\\	\mbox{ $-6$}	\\	\mbox{ $-7$}	\end{array}$}&	\multicolumn{1}{|l|}{ $2346789$}	\\
\hline
\end{tabular}


\includegraphics[width=35.508mm,height=42.576mm]{./F1_M_PP_L2006_page9_images/image001.eps}

\includegraphics[width=35.760mm,height=42.576mm]{./F1_M_PP_L2006_page9_images/image002.eps}
\end{center}
Korzystając z wykresu ffinkcji zapisz:

a) maksymalne przedziały, w których funkcjajest rosnąca,

b) zbiór argumentów, dla których ffinkcja przyjmuje wartości dodatnie,

c) największąwartość ffinkcji $f$ w przedziale $\langle-5, 5\rangle,$

d) miejsca zerowe ffinkcji $g(x)=f(x-1),$

e) najmniejszą wartość funkcji $h(x)=f(x)+2.$






\begin{center}
\includegraphics[width=7.212mm,height=14.220mm]{./F1_M_PP_L2009_page0_images/image001.eps}

\includegraphics[width=25.140mm,height=9.900mm]{./F1_M_PP_L2009_page0_images/image002.eps}
\end{center}
Centralna

Komisja

Egzaminacyjna

APiTAtL z l

NA O OWAST A Eclk l

Materiał współfmansowany ze środków Unii Europejskiej

w ramach Europejskiego Funduszu Społecznego

$\displaystyle \mathrm{F}\cup \mathrm{N}\mathrm{O}\cup \mathrm{s}\mathrm{z}\mathrm{o}r\mathrm{s}_{\mathrm{n}}\bigcup_{\mathrm{S}\mathrm{P}}^{\mathrm{N}1\mathrm{A}\mathrm{E}\cup}\mathrm{R}\mathrm{O}_{\varsigma \mathrm{z}\mathrm{N}?}$
\begin{center}
\includegraphics[width=20.772mm,height=13.920mm]{./F1_M_PP_L2009_page0_images/image003.eps}

\begin{tabular}{|l|l|l}
\cline{1-1}
\multicolumn{1}{|l|}{$\begin{array}{l}\mbox{Miejsce}	\\	\mbox{na na ejkę}	\end{array}$}&	\multicolumn{1}{|l|}{$\begin{array}{l}\mbox{{\it ARKUSZ ZA WIE}}	\\	\mbox{{\it INFO ACJE}}	\\	\mbox{{\it P WNIE CHRONIONE}}	\\	\mbox{{\it DO MOMENTU}}	\\	\mbox{{\it ROZPOCZĘCIA}}	\\	\mbox{{\it EGZAMINU}.'}	\end{array}$}&	\multicolumn{1}{|l}{ $\mathrm{M}\mathrm{M}\mathrm{A}-\mathrm{P}1_{-}1\mathrm{P}-095$}	\\
\hline
&	\multicolumn{1}{|l}{$\begin{array}{l}\mbox{LISTOPAD}	\\	\mbox{ROK 2009}	\\	\mbox{Za rozwiązanie}	\\	\mbox{wszystkich zadań}	\\	\mbox{mozna otrzymać}	\\	\mbox{łącznie}	\\	\mbox{50 punktów}	\end{array}$}	\\
\cline{3-3}
&	\multicolumn{1}{|l}{$\begin{array}{l}\mbox{KOD}	\\	\mbox{ZDAJACEGO}	\end{array}$}
\end{tabular}


\includegraphics[width=21.840mm,height=9.852mm]{./F1_M_PP_L2009_page0_images/image004.eps}

\includegraphics[width=78.744mm,height=13.308mm]{./F1_M_PP_L2009_page0_images/image005.eps}
\end{center}



{\it 2}

{\it Próbny egzamin maturalny z matematyki}

{\it Poziom podstawowy}

ZADANIA ZAMKNIĘTE

$W$ {\it zadaniach} $\theta d1. d_{\theta}25$. {\it wybierz i zaznacz na karcie} $\theta dp\theta${\it wiedzijednq}

{\it poprawnq odpowied} $\acute{z}.$

Zadanie l. $(1pkt)$

Wskaz nierówność, która opisuje sumę przedziałów zaznaczonych na osi liczbowej.
\begin{center}
\includegraphics[width=174.852mm,height=13.416mm]{./F1_M_PP_L2009_page1_images/image001.eps}
\end{center}
$-2$  6  {\it x}

A. $|x-2|>4$

B. $|x-2|<4$

C. $|x-4|<2$

D. $|x-4|>2$

Zadanie 2. (1pkt)

Na seans filmowy sprzedano 280 bi1etów, w tym 126 u1gowych. Jaki procent sprzedanych

biletów stanowiły bilety ulgowe?

A. 22\%

B. 33\%

Zadanie 3. (1pkt)

6\% 1iczby x jest równe 9. Wtedy

A. $x=240$

B. $x=150$

Zadanie 4. $(1pkt)$

Iloraz $32^{-3}$ : $(\displaystyle \frac{1}{8})^{4}$ jest równy

A. $2^{-27}$ B. $2^{-3}$

Zadanie 5. $(1pkt)$

$\mathrm{O}$ liczbie $x$ wiadomo, $\dot{\mathrm{z}}\mathrm{e}\log_{3}x=9$. Zatem

A. {\it x}$=$2 B. {\it x}$=- 21$

Zadanie 6. $(1pkt)$

Wyrazenie $27x^{3}+y^{3}$ jest równe iloczynowi

A.

B.

C.

D.

$(3x+y)(9x^{2}-3xy+y^{2})$

$(3x+y)(9x^{2}+3xy+y^{2})$

$(3x-y)(9x^{2}+3xy+y^{2})$

$(3x-y)(9x^{2}-3xy+y^{2})$

C. 45\%

D. 63\%

C. $x=24$

D. $x=15$

C. $2^{3}$

D. $2^{27}$

C. $x=3^{9}$

D. $x=9^{3}$

Zadanie 7. $(1pkt)$

Dane sąwielomiany: $W(x)=x^{3}-3x+1$ oraz $V(x)=2x^{3}$. Wielomian $W(x)\cdot\nabla(x)$ jest równy

A. $2x^{5}-6x^{4}+2x^{3}$

B. $2x^{6}-6x^{4}+2x^{3}$

C. $2x^{5}+3x+1$

D. $2x^{5}+6x^{4}+2x^{3}$





{\it Próbny egzamin maturalny z matematyki}

{\it Poziom podstawowy}

{\it 11}

Zadanie 28. $(2pkt)$

$\mathrm{W}$ układzie współrzędnych na płaszczyzínie punkty $A=(2,5)$ i $\mathrm{C}=(6,7)$ są przeciwległymi

wierzchołkami kwadratu ABCD. Wyznacz równanie prostej $BD.$

Odpowied $\acute{\mathrm{z}}$:

Zadanie 29. $(2pkt)$

Kąt $a$ jest ostry i $\displaystyle \mathrm{t}\mathrm{g}\alpha=\frac{4}{3}$. Oblicz $\sin\alpha+\cos\alpha.$

Odpowiedzí :





{\it 12}

{\it Próbny egzamin maturalny z matematyki}

{\it Poziom podstawowy}

Zadanie 30. $(2pkt)$

Wykaz, $\dot{\mathrm{z}}\mathrm{e}$ dla $\mathrm{k}\mathrm{a}\dot{\mathrm{z}}$ dego $m$ ciąg $(\displaystyle \frac{m+1}{4},\frac{m+3}{6},\frac{m+9}{12})$ jest arytmetyczny.





{\it Próbny egzamin maturalny z matematyki}

{\it Poziom podstawowy}

{\it 13}

Zadanie 31. $(2pkt)$

Trójkąty $ABC\mathrm{i}CDE$ są równoboczne. Punkty $A, C\mathrm{i}E$ lez$\cdot$ą najednej prostej. Punkty $K, L\mathrm{i}M$

są środkami odcinków $AC$, {\it CE} $\mathrm{i} BD$ (zobacz rysunek). Wykaz, $\dot{\mathrm{z}}\mathrm{e}$ punkty $K, L \mathrm{i} M$

są wierzchołkami trójkąta równobocznego.
\begin{center}
\includegraphics[width=116.640mm,height=65.124mm]{./F1_M_PP_L2009_page12_images/image001.eps}
\end{center}
{\it D}

{\it M}

{\it B}

{\it A  E}

{\it K C  L}





{\it 14}

{\it Próbny egzamin maturalny z matematyki}

{\it Poziom podstawowy}

Zadanie 32. $(5pkt)$

Uczeń przeczytał ksiązkę liczącą480 stron, przy czym $\mathrm{k}\mathrm{a}\dot{\mathrm{z}}$ dego dnia czytał jednakową liczbę

stron. Gdyby czytał $\mathrm{k}\mathrm{a}\dot{\mathrm{z}}$ dego dnia o 8 stron więcej, to przeczytałby tę ksiązkę o 3 dni

wcześniej. Oblicz, ile dni uczeń czytał tę ksiązkę.

Odpowiedzí:





{\it Próbny egzamin maturalny z matematyki}

{\it Poziom podstawowy}

{\it 15}

Zadanie 33. $(4pkt)$

Punkty $A=(2,0) \mathrm{i} B=(12,0)$ są wierzchołkami trójkąta prostokątnego $ABC$

o przeciwprostokątnej $AB$. Wierzchołek $C$ lezy na prostej o równaniu $y=x$. Oblicz

współrzędne punktu $C.$

Odpowiedzí :





{\it 16}

{\it Próbny egzamin maturalny z matematyki}

{\it Poziom podstawowy}

Zadanie 34. $(4pkt)$

Pole trójkąta prostokątnego jest równe 60 $\mathrm{c}\mathrm{m}^{2}$ Jedna przyprostokątna jest o 7 cm diuzsza

od drugiej. Oblicz długość przeciwprostokątnej tego trójkąta.

Odpowiedzí:





{\it Próbny egzamin maturalny z matematyki}

{\it Poziom podstawowy}

{\it 1}7

BRUDNOPIS





{\it Próbny egzamin maturalny z matematyki}

{\it Poziom podstawowy}

{\it 3}

BRUDNOPIS





{\it 4}

{\it Próbny egzamin maturalny z matematyki}

{\it Poziom podstawowy}

Zadanie 8. $(1pkt)$

Wierzchołek paraboli o równaniu $y=-3(x+1)^{2}$ ma współrzędne

A. $(-1,0)$ B. $(0,-1)$ C. $($1, $0)$

D. (0,1)

Zadanie 9. $(1pkt)$

Do wykresu funkcji $f(x)=x^{2}+x-2$ nalezy punkt

A. $(-1,-4)$

B. $(-1,1)$

C. $(-1,-1)$

D. $(-1,-2)$

Zadanie 10. $(1pkt)$

Rozwiązaniem równania $\displaystyle \frac{x-5}{x+3}=\frac{2}{3}$ jest liczba

A. 21 B. 7

C.

$\displaystyle \frac{17}{3}$

D. 0

Zadanie ll. $(1pkt)$

Zbiór rozwiązań nierównoŚci $(x+1)(x-3)>0$ przedstawionyjest na rysunku
\begin{center}
\includegraphics[width=170.988mm,height=15.804mm]{./F1_M_PP_L2009_page3_images/image001.eps}
\end{center}
$-1$  3  {\it x}

A.
\begin{center}
\includegraphics[width=171.756mm,height=13.716mm]{./F1_M_PP_L2009_page3_images/image002.eps}
\end{center}
{\it x}

1

$-3$

B.
\begin{center}
\includegraphics[width=171.048mm,height=15.852mm]{./F1_M_PP_L2009_page3_images/image003.eps}
\end{center}
$-1$  3  {\it x}

C.
\begin{center}
\includegraphics[width=171.756mm,height=13.716mm]{./F1_M_PP_L2009_page3_images/image004.eps}
\end{center}
{\it x}

1

$-3$

D.

Zadanie 12. $(1pkt)$

Dla $ n=1,2,3,\ldots$ ciąg $(a_{n})$ jest określony wzorem: $a_{n}=(-1)^{n}\cdot(3-n)$. Wtedy

A. $a_{3}<0$

B. $a_{3}=0$

C. $a_{3}=1$

D. $a_{3}>1$

Zadanie 13. (1pkt)

W ciągu arytmetycznym trzeci wyraz jest równy 14, ajedenasty jest równy 34. Róznica tego

ciągu jest równa

A. 9 B. -25 C. 2 D. -25

Zadanie 14. $(1pkt)$

$\mathrm{W}$ ciągu geometrycznym $(a_{n})$ dane są: $a_{1}=32 \mathrm{i}a_{4}=-4$. Iloraz tego ciągujest równy

A. 12 B. $\displaystyle \frac{1}{2}$ C. $-\displaystyle \frac{1}{2}$ D. $-12$





{\it Próbny egzamin maturalny z matematyki}

{\it Poziom podstawowy}

{\it 5}

BRUDNOPIS





{\it 6}

{\it Próbny egzamin maturalny z matematyki}

{\it Poziom podstawowy}

Zadanie 15. $(1pkt)$

Kąt $\alpha$ jest ostry i $\displaystyle \sin\alpha=\frac{8}{9}$. Wtedy $\cos\alpha$ jest równy

A. -91 B. -98 C. --$\sqrt{}$917

D.

$\displaystyle \frac{\sqrt{65}}{9}$

Zadanie 16. $(1pkt)$

Danyjest trójkąt prostokątny (patrz rysunek). Wtedy tg $\alpha$ jest równy
\begin{center}
\includegraphics[width=57.660mm,height=36.420mm]{./F1_M_PP_L2009_page5_images/image001.eps}
\end{center}
$\sqrt{3}$

1

$\alpha$

A. $\sqrt{2}$

B.

$\sqrt{2}$

$\sqrt{3}$

$\sqrt{2}$

C.

$\sqrt{3}$

$\sqrt{2}$

D.

-$\sqrt{}$12

Zadanie 17. (1pkt)

W trójkącie równoramiennym ABC dane są

opuszczona z wierzchołka C jest równa

$|AC|=|BC|=7$

oraz

$|AB|=12.$

Wysokość

A. $\sqrt{13}$

B. $\sqrt{5}$

C. l

D. 5

Zadanie 18. $(1pkt)$

Oblicz $\mathrm{d}$ gość odcinka $AE$ wiedząc, $\dot{\mathrm{z}}\mathrm{e}AB||CD \mathrm{i} AB=6, AC=4, CD=8.$
\begin{center}
\includegraphics[width=102.108mm,height=50.040mm]{./F1_M_PP_L2009_page5_images/image002.eps}
\end{center}
{\it D}

{\it B}

8

6

{\it E  A}  4  {\it C}

A.

$|AE|=2$

B.

$|AE|=4$

C.

$|AE|=6$

D.

$|AE|=12$

Zadanie 19. $(1pkt)$

Dane sąpunkty $A=(-2,3)$ oraz $B=(4,6)$. Długość odcinka $AB$ jest równa

A. $\sqrt{208}$

B. $\sqrt{52}$

C. $\sqrt{45}$

D. $\sqrt{40}$

Zadanie 20. $(1pkt)$

Promień okręgu o równaniu $(x-1)^{2}+y^{2}=16$ jest równy

A. l

B. 2

C. 3

D. 4





{\it Próbny egzamin maturalny z matematyki}

{\it Poziom podstawowy}

7

BRUDNOPIS





{\it 8}

{\it Próbny egzamin maturalny z matematyki}

{\it Poziom podstawowy}

Zadanie 21. $(1pkt)$

Wykres ffinkcji liniowej określonej wzorem $f(x)=3x+2$ jest prostą prostopadłą do prostej

o równaniu:

A. $y=-\displaystyle \frac{1}{3}x-1$ B. $y=\displaystyle \frac{1}{3}x+1$ C. $y=3x+1$ D. $y=3x-1$

Zadanie 22. $(1pkt)$

Prosta o równaniu $y=-4x+(2m-7)$ przechodzi przez punkt $A=(2,-1)$. Wtedy

A. $m=7$

B.

{\it m}$=$2 -21

C.

{\it m}$=$ - -21

D. $m=-17$

Zadanie 23. $(1pkt)$

Pole powierzchni całkowitej sześcianu jest równe 150 $\mathrm{c}\mathrm{m}^{2}$ Długość krawędzi tego sześcianu

jest równa

A. 3,5 cm

B. 4 cm

C. 4,5 cm

D. 5 cm

Zadanie 24. (1pkt)

Średnia arytmetyczna pięciu liczb: 5, x, 1, 3, 1 jest równa 3. Wtedy

A. $x=2$

B. $x=3$

C. $x=4$

D. $x=5$

Zadanie 25. $(1pkt)$

Wybieramy liczbę $a$ ze zbioru $A=\{2,3,4,5\}$ oraz liczbę $b$ ze zbioru $B=\{1,4\}$. Ilejest takich par

$(a,b), \dot{\mathrm{z}}\mathrm{e}$ iloczyn $a\cdot b$ jest liczbą nieparzystą?

A. 2

B. 3

C. 5

D. 20





{\it Próbny egzamin maturalny z matematyki}

{\it Poziom podstawowy}

{\it 9}

BRUDNOPIS





$ 1\theta$

{\it Próbny egzamin maturalny z matematyki}

{\it Poziom podstawowy}

ZADANIA OTWARTE

{\it Rozwiqzania zadań o numerach od 26. do 34. nalezy zapisać w} $wyznacz\theta nych$ {\it miejscach}

{\it pod treściq zadania}.

Zadanie 26. $(2pkt)$

Rozwiąz nierówność $x^{2}-3x+2\leq 0.$

Odpowiedzí:

Zadanie 27. $(2pkt)$

Rozwiąz równanie $x^{3}-7x^{2}+2x-14=0.$

Odpowied $\acute{\mathrm{z}}$:







$\mathrm{g}$ NARODOWASTRATECIASPóJNOS$\subseteq$lKAPITALL$\cup$DZKl Centralna Komisja Egzaminacyjna $\mathrm{F}\cup \mathrm{N}\mathrm{D}\cup \mathrm{s}\mathrm{z}\mathrm{S}\mathrm{P}\mathrm{O}\mathrm{L}\mathrm{E}\mathrm{C}\mathrm{Z}\mathrm{N}\mathrm{Y}\cup \mathrm{N}\mathrm{l}\mathrm{A}\mathrm{E}\cup \mathrm{R}\mathrm{O}\mathrm{p}\mathrm{E}\mathrm{J}\mathrm{S}\mathrm{K}\mathrm{A}\mathrm{E}\cup \mathrm{R}\mathrm{O}\mathrm{P}\mathrm{E}\rfloor 5\mathrm{K}\mathrm{l}$\fbox{}

Materiał współfinansowany ze środków Unii Europejskiej w ramach Europejskiego Funduszu Społecznego.

Arkusz zawiera informacje prawnie chronione do momentu rozpoczęcia egzaminu.

WPISUJE ZDAJACY

KOD PESEL

{\it Miejsce}

{\it na naklejkę}

{\it z kodem}
\begin{center}
\includegraphics[width=21.432mm,height=9.804mm]{./F1_M_PP_L2010_page0_images/image001.eps}

\includegraphics[width=82.092mm,height=9.804mm]{./F1_M_PP_L2010_page0_images/image002.eps}

\includegraphics[width=204.012mm,height=216.048mm]{./F1_M_PP_L2010_page0_images/image003.eps}
\end{center}
PRÓBNY EGZAMIN MATU

Z MATEMATY

LNY

POZIOM PODSTAWOWY  LISTOPAD 2010

1.

2.

3.

Sprawdzí, czy arkusz egzaminacyjny zawiera 19 stron

(zadania $1-34$). Ewentualny brak zgłoś przewodniczącemu

zespo nadzorującego egzamin.

Rozwiązania zadań i odpowiedzi wpisuj w miejscu na to

przeznaczonym.

Odpowiedzi do zadań zamkniętych (1-25) przenieś

na ka ę odpowiedzi, zaznaczając je w części ka $\mathrm{y}$

przeznaczonej dla zdającego. Zamaluj $\blacksquare$ pola do tego

przeznaczone. Błędne zaznaczenie otocz kółkiem

i zaznacz właściwe.

4. Pamiętaj, $\dot{\mathrm{z}}\mathrm{e}$ pominięcie argumentacji lub istotnych

obliczeń w rozwiązaniu zadania otwa ego (26-34) $\mathrm{m}\mathrm{o}\dot{\mathrm{z}}\mathrm{e}$

spowodować, $\dot{\mathrm{z}}\mathrm{e}$ za to rozwiązanie nie będziesz mógł

dostać pełnej liczby punktów.

5. Pisz cz elnie i $\mathrm{u}\dot{\mathrm{z}}$ aj tvlko długopisu lub -Dióra

z czarnym tuszem lub atramentem.

6. Nie $\mathrm{u}\dot{\mathrm{z}}$ aj korektora, a błędne zapisy wyrazínie prze eśl.

7. Pamiętaj, $\dot{\mathrm{z}}\mathrm{e}$ zapisy w brudnopisie nie będą oceniane.

8. $\mathrm{M}\mathrm{o}\dot{\mathrm{z}}$ esz korzystać z zestawu wzorów matematycznych,

cyrkla i linijki oraz kalkulatora.

9. Na karcie odpowiedzi wpisz i zakoduj swój numer

PESEL.

10. Nie wpisuj $\dot{\mathrm{z}}$ adnych znaków w części przeznaczonej dla

egzaminatora.

Czas pracy:

170 minut

Liczba punktów

do uzyskania: 50

$\Vert\Vert\Vert\Vert\Vert\Vert\Vert\Vert\Vert\Vert\Vert\Vert\Vert\Vert\Vert\Vert\Vert\Vert\Vert\Vert\Vert\Vert\Vert\Vert|  \mathrm{M}\mathrm{M}\mathrm{A}-\mathrm{P}1_{-}1\mathrm{P}-105$




{\it 2}

{\it Próbny egzamin maturalny z matematyki}

{\it Poziom podstawowy}

ZADANIA ZAMKNIĘTE

{\it Wzadaniach od l. do 25. wybierz i zaznacz na karcie odpowiedzijednq}

{\it poprawnq odpowied} $\acute{z}.$

Zadanie l. $(1pkt)$

Liczba $|5-7|-|-3+4|$ jest równa

A. $-3$ B. $-5$

C. l

D. 3

Zadanie 2. $(1pkt)$

Wskaz rysunek, na którym jest przedstawiony zbiór rozwiązań nierówności $|x-2|\geq 3.$
\begin{center}
\includegraphics[width=173.280mm,height=14.532mm]{./F1_M_PP_L2010_page1_images/image001.eps}
\end{center}
$-1$  5  {\it x}

A.
\begin{center}
\includegraphics[width=172.716mm,height=15.648mm]{./F1_M_PP_L2010_page1_images/image002.eps}
\end{center}
$-1$  5  {\it x}

B.
\begin{center}
\includegraphics[width=171.756mm,height=13.104mm]{./F1_M_PP_L2010_page1_images/image003.eps}
\end{center}
3  {\it x}

C.
\begin{center}
\includegraphics[width=172.716mm,height=15.648mm]{./F1_M_PP_L2010_page1_images/image004.eps}
\end{center}
5  {\it x}

D.

Zadanie 3. (1pkt)

Samochód kosztował 30000 zł. Jego cenę obnizono o 10\%, a następnie cenę po tej obnizce

ponownie obnizono o 10\%. Po tych obnizkach samochód kosztował

A. 24400 zł

B. 24700 zł

C. 24000 zł

D. 24300 zł

Zadanie 4. $(1pkt)$

Danajest liczba $x=63^{2}\displaystyle \cdot(\frac{1}{3})^{4}$. Wtedy

A. $x=7^{2}$ B. $x=7^{-2}$

C. $x=3^{8}\cdot 7^{2}$

D. $x=3\cdot 7$

Zadanie 5. $(1pkt)$

Kwadrat liczby $x=5+2\sqrt{3}$ jest równy

A. 37 B. $25+4\sqrt{3}$

C. $37+20\sqrt{3}$

D. 147

Zadanie 6. $(1pkt)$

Liczba $\log_{5}5-\log_{5}125$ jest równa

A. $-2$ B. $-1$

C.

$\displaystyle \frac{1}{25}$

D. 4





{\it Próbny egzamin maturalny z matematyki}

{\it Poziom podstawowy}

{\it 11}

BRUDNOPIS





{\it 12}

{\it Próbny egzamin maturalny z matematyki}

{\it Poziom podstawowy}

ZADANIA OTWARTE

{\it Rozwiqzania zadań o numerach od 26. do 34. nalezy zapisać w wyznaczonych miejscach}

{\it pod treściq zadania}.

Zadanie 26. $(2pkt)$

Rozwiąz nierówność $x^{2}+11x+30\leq 0.$

Odpowiedzí:

Zadanie 27. $(2pkt)$

Rozwiąz równanie $x^{3}+2x^{2}-5x-10=0.$

Odpowiedzí:





{\it Próbny egzamin maturalny z matematyki}

{\it Poziom podstawowy}

{\it 13}

Zadanie 28. (2pkt)

Przeciwprostokątna trójkąta prostokątnego jest dłuzsza od jednej przyprostokątnej o l cm

i od drugiej przyprostokątnej o 32 cm. Ob1icz długości boków tego trójkąta.

Odpowiedzí:





{\it 14}

{\it Próbny egzamin maturalny z matematyki}

{\it Poziom podstawowy}

Zadanie 29. (2pkt)

Dany jest prostokąt ABCD. Okręgi o średnicach AB $\mathrm{i}$ AD przecinają się w punktach $A\mathrm{i}P$

(zobacz rysunek). Wykaz, $\dot{\mathrm{z}}\mathrm{e}$ punkty $B, P\mathrm{i}D$ lez$\cdot$ą najednej prostej.
\begin{center}
\includegraphics[width=81.228mm,height=71.628mm]{./F1_M_PP_L2010_page13_images/image001.eps}
\end{center}
{\it D  C}

{\it P}

{\it A  B}





{\it Próbny egzamin maturalny z matematyki}

{\it Poziom podstawowy}

{\it 15}

Zadanie 30. $(2pkt)$

Uzasadnij, $\dot{\mathrm{z}}$ ejeśli $(a^{2}+b^{2})(c^{2}+d^{2})=(ac+bd)^{2}$, to {\it ad}$=bc.$

Zadanie 31. (2pkt)

Oblicz, ile jest liczb naturalnych czterocyfrowych, w których zapisie pierwsza cyfra jest

parzysta, a pozostałe nieparzyste.

Odpowiedzí:





{\it 16}

{\it Próbny egzamin maturalny z matematyki}

{\it Poziom podstawowy}

Zadanie 32. $(4pkt)$

Ciąg $(1,x,y-1)$ jest arytmetyczny, natomiast

Oblicz $x$ oraz $y$ i podaj ten ciąg geometryczny.

ciąg (x, y, 12)

jest geometryczny.

Odpowiedzí:





{\it Próbny egzamin maturalny z matematyki}

{\it Poziom podstawowy}

{\it 1}7

Zadanie 33. $(4pkt)$

Punkty $A=(1,5), B=(14,31), C=(4,31)$ są wierzchołkami trójkąta. Prosta zawierająca

wysokość tego trójkąta poprowadzona z wierzchołka $C$ przecina prostą AB w punkcie $D.$

Oblicz długość odcinka $BD.$

Odpowiedzí:





{\it 18}

{\it Próbny egzamin maturalny z matematyki}

{\it Poziom podstawowy}

Zadanie 34. $(5pkt)$

Droga z miasta A do miasta $\mathrm{B}$ ma długość 474 km. Samochódjadący z miasta A do miasta $\mathrm{B}$

wyrusza godzinę pózíniej $\mathrm{n}\mathrm{i}\dot{\mathrm{z}}$ samochód z miasta $\mathrm{B}$ do miasta A. Samochody te spotykają się

w odległości 300 km od miasta B. Średnia prędkość samochodu, który wyjechał z miasta $\mathrm{A},$

liczona od chwili wyjazdu z A do momentu spotkania, była o 17 $\mathrm{k}\mathrm{m}/\mathrm{h}$ mniejsza od średniej

prędkości drugiego samochodu liczonej od chwili wyjazdu z $\mathrm{B}$ do chwili spotkania. Oblicz

średniąprędkość $\mathrm{k}\mathrm{a}\dot{\mathrm{z}}$ dego samochodu do chwili spotkania.

Odpowiedzí:





{\it Próbny egzamin maturalny z matematyki}

{\it Poziom podstawowy}

{\it 19}

BRUDNOPIS





{\it Próbny egzamin maturalny z matematyki}

{\it Poziom podstawowy}

{\it 3}

BRUDNOPIS





{\it 4}

{\it Próbny egzamin maturalny z matematyki}

{\it Poziom podstawowy}

{\it W zadaniach 7, 8 i9 wykorzystaj przedstawiony ponizej wykres funkcji f}
\begin{center}
\begin{tabular}{|l|l|l|l|l|l|l|l|l|l|l|l|l|l|l|l|l|l|}
\hline
\multicolumn{1}{|l|}{}&	\multicolumn{1}{|l|}{}&	\multicolumn{1}{|l|}{}&	\multicolumn{1}{|l|}{}&	\multicolumn{1}{|l|}{}&	\multicolumn{1}{|l|}{}&	\multicolumn{1}{|l|}{}&	\multicolumn{1}{|l|}{$y$}&	\multicolumn{1}{|l|}{}&	\multicolumn{1}{|l|}{}&	\multicolumn{1}{|l|}{}&	\multicolumn{1}{|l|}{}&	\multicolumn{1}{|l|}{}&	\multicolumn{1}{|l|}{}&	\multicolumn{1}{|l|}{}&	\multicolumn{1}{|l|}{}&	\multicolumn{1}{|l|}{}&	\multicolumn{1}{|l|}{}	\\
\hline
\multicolumn{1}{|l|}{}&	\multicolumn{1}{|l|}{}&	\multicolumn{1}{|l|}{}&	\multicolumn{1}{|l|}{}&	\multicolumn{1}{|l|}{}&	\multicolumn{1}{|l|}{}&	\multicolumn{1}{|l|}{}&	\multicolumn{1}{|l|}{}&	\multicolumn{1}{|l|}{}&	\multicolumn{1}{|l|}{}&	\multicolumn{1}{|l|}{}&	\multicolumn{1}{|l|}{}&	\multicolumn{1}{|l|}{}&	\multicolumn{1}{|l|}{}&	\multicolumn{1}{|l|}{}&	\multicolumn{1}{|l|}{}&	\multicolumn{1}{|l|}{}&	\multicolumn{1}{|l|}{}	\\
\hline
\multicolumn{1}{|l|}{}&	\multicolumn{1}{|l|}{}&	\multicolumn{1}{|l|}{}&	\multicolumn{1}{|l|}{}&	\multicolumn{1}{|l|}{}&	\multicolumn{1}{|l|}{}&	\multicolumn{1}{|l|}{}&	\multicolumn{1}{|l|}{}&	\multicolumn{1}{|l|}{}&	\multicolumn{1}{|l|}{}&	\multicolumn{1}{|l|}{}&	\multicolumn{1}{|l|}{}&	\multicolumn{1}{|l|}{}&	\multicolumn{1}{|l|}{}&	\multicolumn{1}{|l|}{}&	\multicolumn{1}{|l|}{}&	\multicolumn{1}{|l|}{}&	\multicolumn{1}{|l|}{}	\\
\hline
\multicolumn{1}{|l|}{}&	\multicolumn{1}{|l|}{}&	\multicolumn{1}{|l|}{}&	\multicolumn{1}{|l|}{}&	\multicolumn{1}{|l|}{}&	\multicolumn{1}{|l|}{}&	\multicolumn{1}{|l|}{}&	\multicolumn{1}{|l|}{}&	\multicolumn{1}{|l|}{}&	\multicolumn{1}{|l|}{}&	\multicolumn{1}{|l|}{}&	\multicolumn{1}{|l|}{}&	\multicolumn{1}{|l|}{}&	\multicolumn{1}{|l|}{}&	\multicolumn{1}{|l|}{}&	\multicolumn{1}{|l|}{}&	\multicolumn{1}{|l|}{}&	\multicolumn{1}{|l|}{}	\\
\hline
\multicolumn{1}{|l|}{}&	\multicolumn{1}{|l|}{}&	\multicolumn{1}{|l|}{}&	\multicolumn{1}{|l|}{}&	\multicolumn{1}{|l|}{}&	\multicolumn{1}{|l|}{}&	\multicolumn{1}{|l|}{}&	\multicolumn{1}{|l|}{}&	\multicolumn{1}{|l|}{}&	\multicolumn{1}{|l|}{}&	\multicolumn{1}{|l|}{}&	\multicolumn{1}{|l|}{}&	\multicolumn{1}{|l|}{}&	\multicolumn{1}{|l|}{}&	\multicolumn{1}{|l|}{}&	\multicolumn{1}{|l|}{}&	\multicolumn{1}{|l|}{}&	\multicolumn{1}{|l|}{}	\\
\hline
\multicolumn{1}{|l|}{}&	\multicolumn{1}{|l|}{}&	\multicolumn{1}{|l|}{}&	\multicolumn{1}{|l|}{}&	\multicolumn{1}{|l|}{}&	\multicolumn{1}{|l|}{}&	\multicolumn{1}{|l|}{}&	\multicolumn{1}{|l|}{}&	\multicolumn{1}{|l|}{}&	\multicolumn{1}{|l|}{}&	\multicolumn{1}{|l|}{}&	\multicolumn{1}{|l|}{}&	\multicolumn{1}{|l|}{}&	\multicolumn{1}{|l|}{}&	\multicolumn{1}{|l|}{}&	\multicolumn{1}{|l|}{}&	\multicolumn{1}{|l|}{}&	\multicolumn{1}{|l|}{}	\\
\hline
\multicolumn{1}{|l|}{}&	\multicolumn{1}{|l|}{}&	\multicolumn{1}{|l|}{}&	\multicolumn{1}{|l|}{}&	\multicolumn{1}{|l|}{}&	\multicolumn{1}{|l|}{}&	\multicolumn{1}{|l|}{}&	\multicolumn{1}{|l|}{}&	\multicolumn{1}{|l|}{}&	\multicolumn{1}{|l|}{}&	\multicolumn{1}{|l|}{}&	\multicolumn{1}{|l|}{}&	\multicolumn{1}{|l|}{}&	\multicolumn{1}{|l|}{}&	\multicolumn{1}{|l|}{}&	\multicolumn{1}{|l|}{}&	\multicolumn{1}{|l|}{}&	\multicolumn{1}{|l|}{ $x$}	\\
\hline
\multicolumn{1}{|l|}{}&	\multicolumn{1}{|l|}{}&	\multicolumn{1}{|l|}{}&	\multicolumn{1}{|l|}{}&	\multicolumn{1}{|l|}{}&	\multicolumn{1}{|l|}{}&	\multicolumn{1}{|l|}{}&	\multicolumn{1}{|l|}{}&	\multicolumn{1}{|l|}{}&	\multicolumn{1}{|l|}{}&	\multicolumn{1}{|l|}{}&	\multicolumn{1}{|l|}{}&	\multicolumn{1}{|l|}{}&	\multicolumn{1}{|l|}{}&	\multicolumn{1}{|l|}{}&	\multicolumn{1}{|l|}{}&	\multicolumn{1}{|l|}{ $1$}&	\multicolumn{1}{|l|}{}	\\
\hline
\multicolumn{1}{|l|}{}&	\multicolumn{1}{|l|}{}&	\multicolumn{1}{|l|}{}&	\multicolumn{1}{|l|}{}&	\multicolumn{1}{|l|}{}&	\multicolumn{1}{|l|}{}&	\multicolumn{1}{|l|}{}&	\multicolumn{1}{|l|}{}&	\multicolumn{1}{|l|}{}&	\multicolumn{1}{|l|}{}&	\multicolumn{1}{|l|}{}&	\multicolumn{1}{|l|}{}&	\multicolumn{1}{|l|}{}&	\multicolumn{1}{|l|}{}&	\multicolumn{1}{|l|}{}&	\multicolumn{1}{|l|}{}&	\multicolumn{1}{|l|}{}&	\multicolumn{1}{|l|}{}	\\
\hline
\multicolumn{1}{|l|}{}&	\multicolumn{1}{|l|}{}&	\multicolumn{1}{|l|}{}&	\multicolumn{1}{|l|}{}&	\multicolumn{1}{|l|}{}&	\multicolumn{1}{|l|}{}&	\multicolumn{1}{|l|}{}&	\multicolumn{1}{|l|}{}&	\multicolumn{1}{|l|}{}&	\multicolumn{1}{|l|}{}&	\multicolumn{1}{|l|}{}&	\multicolumn{1}{|l|}{}&	\multicolumn{1}{|l|}{}&	\multicolumn{1}{|l|}{}&	\multicolumn{1}{|l|}{}&	\multicolumn{1}{|l|}{}&	\multicolumn{1}{|l|}{}&	\multicolumn{1}{|l|}{}	\\
\hline
\multicolumn{1}{|l|}{}&	\multicolumn{1}{|l|}{}&	\multicolumn{1}{|l|}{}&	\multicolumn{1}{|l|}{}&	\multicolumn{1}{|l|}{}&	\multicolumn{1}{|l|}{}&	\multicolumn{1}{|l|}{}&	\multicolumn{1}{|l|}{}&	\multicolumn{1}{|l|}{}&	\multicolumn{1}{|l|}{}&	\multicolumn{1}{|l|}{}&	\multicolumn{1}{|l|}{}&	\multicolumn{1}{|l|}{}&	\multicolumn{1}{|l|}{}&	\multicolumn{1}{|l|}{}&	\multicolumn{1}{|l|}{}&	\multicolumn{1}{|l|}{}&	\multicolumn{1}{|l|}{}	\\
\hline
\end{tabular}

\end{center}
Zadanie 7. (1pkt)

Zbiorem wartości ffinkcjifjest

A. $\langle-2,5\rangle$

B. $\langle-4,8\rangle$

C. $\langle-1,4\rangle$

D. $\langle$5, $ 8\rangle$

Zadanie 8. (1pkt)

Korzystając z wykresu ffinkcjif, wskaz nierówność prawdziwą.

A. $f(-1)<f(1)$

B. $f(1)<f(3)$

C. $f(-1)<f(3)$

D. $f(3)<f(0)$

Zadanie 9. $(1pkt)$

Wykres ffinkcji $g$ określonej wzorem $g(x)=f(x)+2$ jest przedstawiony na rysunku

A. B.





{\it Próbny egzamin maturalny z matematyki}

{\it Poziom podstawowy}

{\it 5}

BRUDNOPIS





{\it 6}

{\it Próbny egzamin maturalny z matematyki}

{\it Poziom podstawowy}

Zadanie 10. $(1pkt)$

Liczby $x_{1}$ i $x_{2}$ sąpierwiastkami równania $x^{2}+10x-24=0\mathrm{i}x_{1}<x_{2}$. Oblicz $2x_{1}+x_{2}.$

A. $-22$

B. $-17$

C. 8

D. 13

Zadanie ll. (lpkt)

Liczba 2 jest pierwiastkiem wie1omianu

równy

$W(x)=x^{3}+ax^{2}+6x-4$. Współczynnik $a$ jest

A. 2

B. $-2$

C. 4

D. $-4$

Zadanie 12. $(1pkt)$

Wskaz $m$, dla którego ffinkcja liniowa określona wzorem $f(x)=(m-1)x+3$ jest stała.

A. $m=1$

B. $m=2$

C. $m=3$

D. $m=-1$

Zadanie 13. $(1pkt)$

Zbiorem rozwiązań nierówności $(x-2)(x+3)\geq 0$ jest

A.

B.

C.

D.

$\langle-2,3\rangle$

$\langle-3,2\rangle$

$(-\infty,-3\rangle\cup\langle 2,+\infty)$

$(-\infty,-2\rangle\cup\langle 3,+\infty)$

Zadanie 14. $(1pkt)$

$\mathrm{W}$ ciągu geometrycznym $(a_{n})$ dane są: $a_{1}=2\mathrm{i}a_{2}=12$. Wtedy

A. $a_{4}=26$

B. $a_{4}=432$

C. $a_{4}=32$

D. $a_{4}=2592$

Zadanie 15. $(1pkt)$

$\mathrm{W}$ ciągu arytmetycznym $a_{1}=3$ oraz $a_{20}=7$. Wtedy suma $S_{20}=a_{1}+a_{2}+\ldots+a_{19}+a_{20}$ jest

równa

A. 95

B. 200

C. 230

D. 100

Zadanie 16. $(1pkt)$

Na rysunku zaznaczono długości boków i kąt $\alpha$ trójkąta prostokątnego (zobacz rysunek). Wtedy
\begin{center}
\includegraphics[width=87.984mm,height=32.868mm]{./F1_M_PP_L2010_page5_images/image001.eps}
\end{center}
13

5

12

A.

$\displaystyle \cos\alpha=\frac{5}{13}$

B.

$\displaystyle \mathrm{t}\mathrm{g}\alpha=\frac{13}{12}$

C.

$\displaystyle \cos\alpha=\frac{12}{13}$

D.

$\displaystyle \mathrm{t}\mathrm{g}\alpha=\frac{12}{5}$





{\it Próbny egzamin maturalny z matematyki}

{\it Poziom podstawowy}

7

BRUDNOPIS





{\it 8}

{\it Próbny egzamin maturalny z matematyki}

{\it Poziom podstawowy}

Zadanie 17. (1pkt)

Ogród ma kształt prostokąta o bokach długości 20 m i 40 m. Na dwóch końcach przekątnej

tego prostokąta wbito słupki. Odległość między tymi słupkamijest

A.

B.

C.

D.

równa 40 $\mathrm{m}$

większa $\mathrm{n}\mathrm{i}\dot{\mathrm{z}}50\mathrm{m}$

większa $\mathrm{n}\mathrm{i}\dot{\mathrm{z}}40\mathrm{m}$ i mniejsza $\mathrm{n}\mathrm{i}\dot{\mathrm{z}}45\mathrm{m}$

większa $\mathrm{n}\mathrm{i}\dot{\mathrm{z}}45\mathrm{m}$ i mniejsza $\mathrm{n}\mathrm{i}\dot{\mathrm{z}}50\mathrm{m}$

Zadanie 18. (1pkt)

Pionowy słupek o wysokości 90 cm rzuca cień o długości 60 cm. W tej samej chwi1i stojąca

obok wieza rzuca cień długości 12 m. Jakajest wysokość wiezy?

A. 18 m

B. 8m

C. 9m

D. 16 m

Zadanie 19. $(1pkt)$

Punkty $A, B \mathrm{i} C$ lez$\cdot$ą na okręgu o środku $S$ (zobacz rysunek). Miara zaznaczonego kąta

wpisanego $ACB$ jest równa
\begin{center}
\includegraphics[width=53.796mm,height=52.728mm]{./F1_M_PP_L2010_page7_images/image001.eps}
\end{center}
{\it C}

{\it A  B}

{\it S}

$230^{\mathrm{o}}$

A. $65^{\mathrm{o}}$

B. $100^{\mathrm{o}}$

C. $115^{\mathrm{o}}$

D. $130^{\mathrm{o}}$

Zadanie 20. $(1pkt)$

Dane sąpunkty $S=(2,1), M=(6,4)$. Równanie okręgu o środku $S$ i przechodzącego przez

punkt $M$ ma postać

A.

B.

C.

D.

$(x-2)^{2}+(y-1)^{2}=5$

$(x-2)^{2}+(y-1)^{2}=25$

$(x-6)^{2}+(y-4)^{2}=5$

$(x-6)^{2}+(y-4)^{2}=25$





{\it Próbny egzamin maturalny z matematyki}

{\it Poziom podstawowy}

{\it 9}

BRUDNOPIS





$ 1\theta$

{\it Próbny egzamin maturalny z matematyki}

{\it Poziom podstawowy}

Zadanie 21. $(1pkt)$

Proste o równaniach $y=2x+3$ oraz $y=-\displaystyle \frac{1}{3}x+2$

A. są równoległe i rózne

B. sąprostopadłe

C. przecinają się pod kątem innym $\mathrm{n}\mathrm{i}\dot{\mathrm{z}}$ prosty

D. pokrywają się

Zadanie 22. $(1pkt)$

Wskaz równanie prostej, którajest osią symetrii paraboli o równaniu $y=x^{2}-4x+2010.$

A. $x=4$

B. $x=-4$

C. $x=2$

D. $x=-2$

Zadanie 23. $(1pkt)$

Kąt $\alpha$ jest ostry i $\displaystyle \cos\alpha=\frac{3}{7}$. Wtedy

A.

$\displaystyle \sin\alpha=\frac{2\sqrt{10}}{7}$

B.

$\displaystyle \sin\alpha=\frac{\sqrt{10}}{7}$

C.

$\displaystyle \sin\alpha=\frac{4}{7}$

D.

$\displaystyle \sin\alpha=\frac{3}{4}$

Zadanie 24. (1pkt)

W karcie dań jest 5 zup i 4 drugie dania. Na i1e sposobów mozna zamówić obiad s$\mathbb{H}$adający się

zjednej zupy ijednego drugiego dania?

A. 25

B. 20

C. 16

D. 9

Zadanie 25. (1pkt)

W czterech rzutach sześcienną kostką do gry otrzymano następujące liczby oczek: 6, 3, 1, 4.

Mediana tych danychjest równa

A. 2

B. 2,5

C. 5

D. 3,5






\begin{center}
\begin{tabular}{l|l}
\multicolumn{1}{l|}{$\begin{array}{l}\mbox{{\it dysleksja}}	\\	\mbox{Miejsce}	\\	\mbox{na na ejkę}	\\	\mbox{z kodem szkoly}	\end{array}$}&	\multicolumn{1}{|l}{MMA-PIAIP-052}	\\
\hline
\multicolumn{1}{l|}{$\begin{array}{l}\mbox{EGZAMIN MATURALNY}	\\	\mbox{Z MATEMATYKI}	\\	\mbox{Arkusz I}	\\	\mbox{POZIOM PODSTAWOWY}	\\	\mbox{Czas pracy 120 minut}	\\	\mbox{Instrukcja dla zdającego}	\\	\mbox{1. $\mathrm{S}\mathrm{p}\mathrm{r}\mathrm{a}\mathrm{w}\mathrm{d}\acute{\mathrm{z}}$, czy arkusz egzaminacyjny zawiera 13 stron.}	\\	\mbox{Ewentualny brak zgłoś przewodniczącemu zespo}	\\	\mbox{nadzorującego egzamin.}	\\	\mbox{2. Rozwiązania zadań i odpowiedzi zamieść w miejscu na to}	\\	\mbox{przeznaczonym.}	\\	\mbox{3. $\mathrm{W}$ rozwiązaniach zadań przedstaw tok rozumowania}	\\	\mbox{prowadzący do ostatecznego wyniku.}	\\	\mbox{4. Pisz czytelnie. Uzywaj długopisu pióra tylko z czatnym}	\\	\mbox{tusze atramentem.}	\\	\mbox{5. Nie uzywaj korektora. Błędne zapisy prze eśl.}	\\	\mbox{6. Pamiętaj, $\dot{\mathrm{z}}\mathrm{e}$ zapisy w $\mathrm{b}$ dnopisie nie podlegają ocenie.}	\\	\mbox{7. Obok $\mathrm{k}\mathrm{a}\dot{\mathrm{z}}$ dego zadania podanajest maksymalna liczba punktów,}	\\	\mbox{którą mozesz uzyskać zajego poprawne rozwiązanie.}	\\	\mbox{8. $\mathrm{M}\mathrm{o}\dot{\mathrm{z}}$ esz korzystać z zestawu wzorów matematycznych, cyrkla}	\\	\mbox{i linijki oraz kalkulatora.}	\\	\mbox{9. Wypełnij tę część ka $\mathrm{y}$ odpowiedzi, którą koduje zdający.}	\\	\mbox{Nie wpisuj $\dot{\mathrm{z}}$ adnych znaków w części przeznaczonej}	\\	\mbox{dla egzaminatora.}	\\	\mbox{10. Na karcie odpowiedzi wpisz swoją datę urodzenia i PESEL.}	\\	\mbox{Zamaluj $\blacksquare$ pola odpowiadające cyfrom numeru PESEL. Błędne}	\\	\mbox{zaznaczenie otocz kółkiem i zaznacz właściwe.}	\\	\mbox{{\it Zyczymy powodzenia}.'}	\end{array}$}&	\multicolumn{1}{|l}{$\begin{array}{l}\mbox{ARKUSZ I}	\\	\mbox{MAJ}	\\	\mbox{ROK 2005}	\\	\mbox{Za rozwiązanie}	\\	\mbox{wszystkich zadań}	\\	\mbox{mozna otrzymać}	\\	\mbox{łącznie}	\\	\mbox{50 punktów}	\end{array}$}	\\
\hline
\multicolumn{1}{l|}{$\begin{array}{l}\mbox{Wypelnia zdający przed}	\\	\mbox{roz oczęciem racy}	\\	\mbox{PESEL ZDAJACEGO}	\end{array}$}&	\multicolumn{1}{|l}{$\begin{array}{l}\mbox{tylko}	\\	\mbox{O Kraków,}	\\	\mbox{OKE Wroclaw}	\\	\mbox{KOD}	\\	\mbox{ZDAJACEGO}	\end{array}$}
\end{tabular}


\includegraphics[width=78.792mm,height=13.356mm]{./F1_M_PP_M2005_page0_images/image001.eps}

\includegraphics[width=21.840mm,height=9.804mm]{./F1_M_PP_M2005_page0_images/image002.eps}
\end{center}



{\it 2}

{\it Egzamin maturalny z matematyki}

{\it Arkusz I}

Zadanie 1. (3pkt)

W pudełku są trzy kule białe i pięć kul czarnych. Do pudełka mozna albo dołozyć jedną kulę

białą albo usunąč z niegojedną kulę czarn4 a następnie wy1osować z tego pudełkajedną ku1ę.

W którym z tych przypadków wylosowanie kuli białej jest bardziej prawdopodobne?

Wykonaj odpowiednie obliczenia.
\begin{center}
\includegraphics[width=192.588mm,height=252.684mm]{./F1_M_PP_M2005_page1_images/image001.eps}
\end{center}




{\it Egzamin maturalny z matematyki}

{\it Arkusz I}

{\it 11}

Zadanie 10. $(7pkt)$

$\mathrm{W}$ ostrosłupie czworokątnym prawidłowym wysokości przeciwległych ścian bocznych

poprowadzone z wierzchołka ostrosłupa mają długości $h$ i tworzą kąt o mierze $ 2\alpha$. Oblicz

objętość tego ostrosłupa.
\begin{center}
\includegraphics[width=192.588mm,height=258.720mm]{./F1_M_PP_M2005_page10_images/image001.eps}
\end{center}




{\it 12}

{\it Egzamin maturalny z matematyki}

{\it Arkusz I}

BRUDNOPIS





{\it Egzamin maturalny z matematyki}

{\it Arkusz I}

{\it 13}





{\it Egzamin maturalny z matematyki}

{\it Arkusz I}

{\it 3}
\begin{center}
\includegraphics[width=193.548mm,height=290.220mm]{./F1_M_PP_M2005_page2_images/image001.eps}
\end{center}
Zadanie 2. $(4pkt)$

Dany jest ciąg $(a_{n})$, gdzie $a_{n}=\displaystyle \frac{n+2}{3n+1}$ dla $ n=1,2,3\ldots$ Wyznacz wszystkie wyrazy tego ciągu

większe od $\displaystyle \frac{1}{2}$





{\it 4}

{\it Egzamin maturalny z matematyki}

{\it Arkusz I}

Zadanie 3. (4pkt)

Dany jest wielomian $W(x)=x^{3}+kx^{2}-4.$

a) Wyznacz współczynnik $k$ tego wielomianu wiedząc, $\dot{\mathrm{z}}\mathrm{e}$ wielomian ten jest podzielny

przez dwumian $x+2.$

b) Dla wyznaczonej wartości $k$ rozłóz wielomian na czynniki i podaj wszystkie jego

pierwiastki.
\begin{center}
\includegraphics[width=192.588mm,height=240.696mm]{./F1_M_PP_M2005_page3_images/image001.eps}
\end{center}




{\it Egzamin maturalny z matematyki}

{\it Arkusz I}

{\it 5}

Zadanie 4. $(5pkt)$

Na trzech półkach ustawiono 76 płyt kompaktowych. Okazało się, $\dot{\mathrm{z}}\mathrm{e}$ liczby płyt na półkach

gótnej, środkowej i dolnej tworzą rosnący ciąg geometryczny. Na środkowej półce stoją

24 płyty. Oblicz, ile płyt stoi na półce gótnej, a ile płyt stoi na półce dolnej.
\begin{center}
\includegraphics[width=192.588mm,height=258.720mm]{./F1_M_PP_M2005_page4_images/image001.eps}
\end{center}




{\it 6}

{\it Egzamin maturalny z matematyki}

{\it Arkusz I}

Zadanie 5. $(4pkt)$

Sklep sprowadza z hurtowni kurtki płacąc po 100 zł za sztukę i sprzedaje średnio 40 sztuk

miesięcznie po 160 zł. Zaobserwowano, $\dot{\mathrm{z}}\mathrm{e} \mathrm{k}\mathrm{a}\dot{\mathrm{z}}$ da kolejna obnizka ceny sprzedaz$\mathrm{y}$ kurtki

$01$ zł zwiększa sprzedaz miesięczną o l sztukę. Jaką cenę kurtki powinien ustalić

sprzedawca, abyjego miesięczny zysk był największy?
\begin{center}
\includegraphics[width=192.588mm,height=252.684mm]{./F1_M_PP_M2005_page5_images/image001.eps}
\end{center}




{\it Egzamin maturalny z matematyki}

{\it Arkusz I}

7

Zadanie 6. (6pkt)

Dane są zbiory liczb rzeczywistych:

$A=\{x:|x+2|\langle 3\}$

$B=\{x:(2x-1)^{3}\leq 8x^{3}-13x^{2}+6x+3\}$

Zapisz w postaci przedziałów liczbowych zbiory $A, B, A\cap B$ oraz $B-A.$
\begin{center}
\includegraphics[width=192.588mm,height=240.696mm]{./F1_M_PP_M2005_page6_images/image001.eps}
\end{center}




{\it 8}

{\it Egzamin maturalny z matematyki}

{\it Arkusz I}

Zadanie 7. (5pkt)

W ponizszej tabeli przedstawiono wyniki sondazu przeprowadzonego w grupie uczniów,

dotyczącego czasu przeznaczanego dziennie na przygotowanie zadań domowych.
\begin{center}
\begin{tabular}{|l|l|l|l|l|}
\hline
\multicolumn{1}{|l|}{$\begin{array}{l}\mbox{Czas}	\\	\mbox{(w godzinach)}	\end{array}$}&	\multicolumn{1}{|l|}{ $1$}&	\multicolumn{1}{|l|}{ $2$}&	\multicolumn{1}{|l|}{ $3$}&	\multicolumn{1}{|l|}{ $4$}	\\
\hline
\multicolumn{1}{|l|}{$\begin{array}{l}\mbox{Liczba}	\\	\mbox{uczniów}	\end{array}$}&	\multicolumn{1}{|l|}{ $5$}&	\multicolumn{1}{|l|}{ $10$}&	\multicolumn{1}{|l|}{ $15$}&	\multicolumn{1}{|l|}{ $10$}	\\
\hline
\end{tabular}

\end{center}
a) Naszkicuj diagram s

wyniki tego sondazu.

pkowy ilustrujący

b) Oblicz średnią liczbę godzin, jaką

uczniowie przeznaczają dziennie na

przygotowanie zadań domowych.
\begin{center}
\includegraphics[width=96.972mm,height=96.924mm]{./F1_M_PP_M2005_page7_images/image001.eps}
\end{center}
c)

Oblicz wariancję i odchylenie

standardowe czasu przeznaczonego

dziennie na przygotowanie zadań

domowych. Wynik podaj z dokładnością

do 0,01.
\begin{center}
\includegraphics[width=192.588mm,height=126.492mm]{./F1_M_PP_M2005_page7_images/image002.eps}
\end{center}




{\it Egzamin maturalny z matematyki}

{\it Arkusz I}

{\it 9}

Zadanie 8. (6pkt)

Z kawałka materiału o kształcie i wymiarach

czworokąta ABCD (patrz na rysunek obok)

wycięto okrągłą serwetkę o promieniu 3 dm.

Oblicz, ile procent całego materiału stanowi

jego niewykorzystana część. Wynik podaj

z dokładnością do 0,01 procenta.
\begin{center}
\includegraphics[width=71.376mm,height=82.092mm]{./F1_M_PP_M2005_page8_images/image001.eps}
\end{center}
{\it c}

{\it D}

10

{\it o}

3
\begin{center}
\includegraphics[width=192.588mm,height=204.624mm]{./F1_M_PP_M2005_page8_images/image002.eps}
\end{center}




$ 1\theta$

{\it Egzamin maturalny z matematyki}

{\it Arkusz I}

Zadanie 9. (6pkt)
\begin{center}
\includegraphics[width=193.644mm,height=280.620mm]{./F1_M_PP_M2005_page9_images/image001.eps}
\end{center}
Rodzeństwo w wieku 8 $\mathrm{i} 10$ lat otrzymało razem w spadku 84100 zł. Kwotę tę złozono

w banku, który stosuje kapitalizację roczną przy rocznej stopie procentowej 5\%. $\mathrm{K}\mathrm{a}\dot{\mathrm{z}}$ de

z dzieci otrzyma swoją część spadku z chwilą osiągnięcia wieku 211at. $\dot{\mathrm{Z}}$ yczeniem

spadkodawcy było takie podzielenie kwoty spadku, aby w przyszłości obie wypłacone części

spadku zaokrąglone do l zł były równe. Jak nalez$\mathrm{y}$ podzielić kwotę 84100 zł między

rodzeńs $0$? Za isz wszystkie wykon ane obliczenia.






\begin{center}
\begin{tabular}{l|l}
\multicolumn{1}{l|}{$\begin{array}{l}\mbox{{\it dysleksja}}	\\	\mbox{Miejsce}	\\	\mbox{na na ejkę}	\\	\mbox{z kodem szkoly}	\end{array}$}&	\multicolumn{1}{|l}{MMA-PIAIP-062}	\\
\hline
\multicolumn{1}{l|}{ $\begin{array}{l}\mbox{EGZAMIN MATURALNY}	\\	\mbox{Z MATEMATYKI}	\\	\mbox{Arkusz I}	\\	\mbox{POZIOM PODSTAWOWY}	\\	\mbox{Czas pracy 120 minut}	\\	\mbox{Instrukcja dla zdającego}	\\	\mbox{1. Sprawdzí, czy arkusz egzaminacyjny zawiera 14 stron (zadania}	\\	\mbox{$1-11)$. Ewentualny brak zgłoś przewodniczącemu zespo}	\\	\mbox{nadzorującego egzamin.}	\\	\mbox{2. Rozwiązania zadań i odpowiedzi zamieść w miejscu na to}	\\	\mbox{przeznaczonym.}	\\	\mbox{3. $\mathrm{W}$ rozwiązaniach zadań przedstaw tok rozumowania}	\\	\mbox{prowadzący do ostatecznego wyniku.}	\\	\mbox{4. Pisz czytelnie. $\mathrm{U}\dot{\mathrm{z}}$ aj długopisu pióra tylko z czarnym}	\\	\mbox{tusze atramentem.}	\\	\mbox{5. Nie uzywaj korektora, a błędne zapisy prze eśl.}	\\	\mbox{6. Pamiętaj, $\dot{\mathrm{z}}\mathrm{e}$ zapisy w $\mathrm{b}$ dnopisie nie podlegają ocenie.}	\\	\mbox{7. Obok $\mathrm{k}\mathrm{a}\dot{\mathrm{z}}$ dego zadania podanajest maksymalna liczba punktów,}	\\	\mbox{którą mozesz uzyskać zajego poprawne rozwiązanie.}	\\	\mbox{8. $\mathrm{M}\mathrm{o}\dot{\mathrm{z}}$ esz korzystać z zestawu wzorów matematycznych, cyrkla}	\\	\mbox{i linijki oraz kalkulatora.}	\\	\mbox{9. Wypełnij tę część ka $\mathrm{y}$ odpowiedzi, którą koduje zdający.}	\\	\mbox{Nie wpisuj $\dot{\mathrm{z}}$ adnych znaków w części przeznaczonej dla}	\\	\mbox{egzaminatora.}	\\	\mbox{10. Na karcie odpowiedzi wpisz swoją datę urodzenia i PESEL.}	\\	\mbox{Zamaluj $\blacksquare$ pola odpowiadające cyfrom numeru PESEL. Błędne}	\\	\mbox{zaznaczenie otocz kółkiem $\mathrm{O}$ i zaznacz właściwe.}	\\	\mbox{{\it Zyczymy} $p\theta wodzenia'$}	\end{array}$}&	\multicolumn{1}{|l}{$\begin{array}{l}\mbox{ARKUSZ I}	\\	\mbox{MAJ}	\\	\mbox{ROK 2006}	\\	\mbox{Za rozwiązanie}	\\	\mbox{wszystkich zadań}	\\	\mbox{mozna otrzymać}	\\	\mbox{łącznie}	\\	\mbox{50 punktów}	\end{array}$}	\\
\hline
\multicolumn{1}{l|}{$\begin{array}{l}\mbox{Wypelnia zdający przed}	\\	\mbox{roz oczęciem racy}	\\	\mbox{PESEL ZDAJACEGO}	\end{array}$}&	\multicolumn{1}{|l}{$\begin{array}{l}\mbox{KOD}	\\	\mbox{ZDAJACEGO}	\end{array}$}
\end{tabular}


\includegraphics[width=21.840mm,height=9.852mm]{./F1_M_PP_M2006_page0_images/image001.eps}

\includegraphics[width=78.792mm,height=13.356mm]{./F1_M_PP_M2006_page0_images/image002.eps}
\end{center}



{\it 2}

{\it Egzamin maturalny z matematyki}

{\it Arkusz I}

Zadanie l. $(3pkt)$

Dane są zbiory: $A=\{x\in R:|x-4|\geq 7\}, B=\{x\in R$:

a) zbiór $A,$

b) zbiór $B,$

c) zbiór $C=B\backslash A.$

$x^{2}>0$. Zaznacz na osi liczbowej:

a)

b)

c)
\begin{center}
\includegraphics[width=192.072mm,height=72.240mm]{./F1_M_PP_M2006_page1_images/image001.eps}

\includegraphics[width=192.072mm,height=72.240mm]{./F1_M_PP_M2006_page1_images/image002.eps}

\includegraphics[width=192.072mm,height=72.240mm]{./F1_M_PP_M2006_page1_images/image003.eps}

\includegraphics[width=109.932mm,height=17.580mm]{./F1_M_PP_M2006_page1_images/image004.eps}
\end{center}
WypelnÍa

egzaminator!

Nr czynności

Maks. liczba kt

1.1.

1

1.2.

1.3.

1

Uzyskana liczba pkt





{\it Egzamin maturalny z matematyki}

{\it Arkusz I}

{\it 11}

Zadanie 10. $(6pkt)$

Liczby 3 $\mathrm{i}-1$ sąpierwiastkami wielomianu $W(x)=2x^{3}+ax^{2}+bx+30.$

a) Wyznacz wartości współczynników a $\mathrm{i}b.$

b) Oblicz trzeci pierwiastek tego wielomianu.
\begin{center}
\includegraphics[width=192.276mm,height=260.508mm]{./F1_M_PP_M2006_page10_images/image001.eps}

\includegraphics[width=151.788mm,height=17.580mm]{./F1_M_PP_M2006_page10_images/image002.eps}
\end{center}
Wypelnia

egzamÍnator!

Nr czynności

Maks. liczba kt

10.1.

1

10.2.

1

10.3.

1

10.4.

1

10.5.

1

1

Uzyskana liczba pkt





{\it 12}

{\it Egzamin maturalny z matematyki}

{\it Arkusz I}

Zadanie ll. $(3pkt)$

Sumę $S=\displaystyle \frac{3}{1\cdot 4}+\frac{3}{4\cdot 7}+\frac{3}{7\cdot 10}+\ldots+\frac{3}{301\cdot 304}+\frac{3}{304\cdot 307}$ mozna obliczyć w następujący sposób:

a) sumę $S$ zapisujemy w postaci

$S=\displaystyle \frac{4-1}{4\cdot 1}+\frac{7-4}{7\cdot 4}+\frac{10-7}{10\cdot 7}+\ldots+\frac{304-301}{304\cdot 301}+\frac{307-304}{307\cdot 304}$

b) $\mathrm{k}\mathrm{a}\dot{\mathrm{z}}\mathrm{d}\mathrm{y}$ składnik tej sumy przedstawiamy jako róznicę ułamków

$S=(\displaystyle \frac{4}{4\cdot 1}-\frac{1}{4\cdot 1})+(\frac{7}{7\cdot 4}-\frac{4}{7\cdot 4})+(\frac{10}{10\cdot 7}-\frac{7}{10\cdot 7})+\ldots+(\frac{304}{304\cdot 301}-\frac{301}{304\cdot 301})+(\frac{307}{307\cdot 304}-\frac{304}{307\cdot 304})$

stąd $S=(1-\displaystyle \frac{1}{4})+(\frac{1}{4}-\frac{1}{7})+(\frac{1}{7}-\frac{1}{10})+\ldots+(\frac{1}{301}-\frac{1}{304})+(\frac{1}{304}-\frac{1}{307})$

więc $S=1-\displaystyle \frac{1}{4}+\frac{1}{4}-\frac{1}{7}+\frac{1}{7}-\frac{1}{10}+\ldots+\frac{1}{301}-\frac{1}{304}+\frac{1}{304}-\frac{1}{307}$

c) obliczamy sumę, redukując parami wyrazy sąsiednie, poza pierwszym i ostatnim

$S=1-\displaystyle \frac{1}{307}=\frac{306}{307}.$

Postępując w analogiczny sposób, oblicz sumę $S_{1}=\displaystyle \frac{4}{1\cdot 5}+\frac{4}{5\cdot 9}+\frac{4}{9\cdot 13}+\ldots+\frac{4}{281\cdot 285}$
\begin{center}
\includegraphics[width=192.228mm,height=193.956mm]{./F1_M_PP_M2006_page11_images/image001.eps}
\end{center}




{\it Egzamin maturalny z matematyki}

{\it Arkusz I}

{\it 13}
\begin{center}
\includegraphics[width=192.276mm,height=290.784mm]{./F1_M_PP_M2006_page12_images/image001.eps}

\includegraphics[width=109.980mm,height=17.580mm]{./F1_M_PP_M2006_page12_images/image002.eps}
\end{center}
Nr czynno\S ci

WypelnÍa Maks. liczba kt

egzaminator! Uzyskana liczba pkt

11.1.

1

11.2.

11.3.

1





{\it 14}

{\it Egzamin maturalny z matematyki}

{\it Arkusz I}

BRUDNOPIS





{\it Egzamin maturalny z matematyki}

{\it Arkusz I}

{\it 3}
\begin{center}
\includegraphics[width=192.276mm,height=289.200mm]{./F1_M_PP_M2006_page2_images/image001.eps}
\end{center}
Zadanie 2. $(3pkt)$

$\mathrm{W}$ wycieczce szkolnej bierze udział 16 uczniów, wśród których ty1ko czworo zna oko1icę.

Wychowawca chce wybrać w sposób losowy 3 osoby, które mają pójść do sk1epu. Ob1icz

prawdopodobieństwo tego, $\dot{\mathrm{z}}\mathrm{e}$ wśród wybranych trzech osób będą dokładnie dwie znające

okolicę.
\begin{center}
\includegraphics[width=109.980mm,height=17.580mm]{./F1_M_PP_M2006_page2_images/image002.eps}
\end{center}
Nr czynno\S ci

WypelnÍa Maks. liczba $\llcorner\prime \mathrm{t}$

egzaminator! Uzyskana liczba pkt

2.1.

1

2.2.

2.3.

1





{\it 4}

{\it Egzamin maturalny z matematyki}

{\it Arkusz I}

Zadanie 3. (5pkt)

Kostka masła produkowanego przez pewien zakład mleczarski ma nominalną masę

20 dag. W czasie kontroli zakładu zwazono l50 losowo wybranych kostek masła. Wyniki

badań przedstawiono w tabeli.
\begin{center}
\begin{tabular}{|l|l|l|l|l|l|l|}
\hline
\multicolumn{1}{|l|}{Masa kostki masła (w dag)}&	\multicolumn{1}{|l|}{$16$}&	\multicolumn{1}{|l|}{ $18$}&	\multicolumn{1}{|l|}{ $19$}&	\multicolumn{1}{|l|}{ $20$}&	\multicolumn{1}{|l|}{ $21$}&	\multicolumn{1}{|l|}{ $22$}	\\
\hline
\multicolumn{1}{|l|}{Liczba kostek masła}&	\multicolumn{1}{|l|}{$1$}&	\multicolumn{1}{|l|}{ $15$}&	\multicolumn{1}{|l|}{ $24$}&	\multicolumn{1}{|l|}{ $68$}&	\multicolumn{1}{|l|}{ $26$}&	\multicolumn{1}{|l|}{ $16$}	\\
\hline
\end{tabular}

\end{center}
a) Na podstawie danych przedstawionych w tabeli oblicz średnią arytmetyczną oraz

odchylenie standardowe masy kostki masła.

b) Kontrola wypada pozytywnie, jeśli średnia masa kostki masła jest równa masie

nominalnej i odchylenie standardowe nie przekracza l dag. Czy kontrola zakładu

wypadła pozytywnie? Odpowiedzí uzasadnij.
\begin{center}
\includegraphics[width=192.228mm,height=212.088mm]{./F1_M_PP_M2006_page3_images/image001.eps}

\includegraphics[width=109.932mm,height=17.580mm]{./F1_M_PP_M2006_page3_images/image002.eps}
\end{center}
Wypelnia

egzaminator!

Nr czynnoŚci

Maks. liczba kt

3.1.

2

3.2.

2

3.3.

Uzyskana liczba pkt





{\it Egzamin maturalny z matematyki}

{\it Arkusz I}

{\it 5}
\begin{center}
\includegraphics[width=192.276mm,height=286.668mm]{./F1_M_PP_M2006_page4_images/image001.eps}
\end{center}
Zadanie 4. $(4pkt)$

Dany jest rosnący ciąg geometryczny, w którym $a_{1}=12, a_{3}=27.$

a) Wyznacz iloraz tego ciągu.

b) Zapisz wzór, na podstawie którego mozna obliczyć wyraz $a_{n}$, dla $\mathrm{k}\mathrm{a}\dot{\mathrm{z}}$ dej liczby naturalnej

$n\geq 1.$

c) Oblicz wyraz $a_{6}.$
\begin{center}
\includegraphics[width=109.980mm,height=17.580mm]{./F1_M_PP_M2006_page4_images/image002.eps}
\end{center}
Nr czynności

Wypelnia Maks. liczba $\llcorner\prime \mathrm{t}$

egzaminator! Uzyskana liczba pkt

4.1.

2

4.2.

1

4.3.





{\it 6}

{\it Egzamin maturalny z matematyki}

{\it Arkusz I}

Zadanie 5. $(3pkt)$

Wiedząc, $\dot{\mathrm{z}}\mathrm{e}0^{\mathrm{o}}\leq\alpha\leq 360^{\mathrm{o}}, \sin\alpha<0$ oraz 4 tg $\alpha=3\sin^{2}\alpha+3\cos^{2}\alpha$

a) oblicz $\mathrm{t}\mathrm{g}\alpha,$

b) zaznacz w układzie współrzędnych kąt $\alpha$ i podaj współrzędne dowolnego punktu,

róznego od początku układu współrzędnych, który lezy na końcowym ramieniu tego
\begin{center}
\includegraphics[width=84.024mm,height=108.408mm]{./F1_M_PP_M2006_page5_images/image001.eps}
\end{center}
kąta.
\begin{center}
\includegraphics[width=90.732mm,height=145.488mm]{./F1_M_PP_M2006_page5_images/image002.eps}

\includegraphics[width=109.932mm,height=17.628mm]{./F1_M_PP_M2006_page5_images/image003.eps}
\end{center}
Wypelnia

egzaminator!

Nr czynnoŚci

Maks. liczba kl

5.1.

1

5.2.

1

5.3.

Uzyskana liczba pkt





{\it Egzamin maturalny z matematyki}

{\it Arkusz I}

7

Zadanie 6. $(7pkt)$

Państwo Nowakowie przeznaczyli 26000 zł na zakup działki. Do jednej z ofert dołączono

rysunek dwóch przylegających do siebie działek w skali 1:1000. Jeden metr kwadratowy

gruntu w tej ofercie kosztuje 35 zł. Ob1icz, czy przeznaczona przez państwa Nowaków kwota

wystarczy na zakup działki $\mathrm{P}_{2}.$

E
\begin{center}
\includegraphics[width=126.240mm,height=55.020mm]{./F1_M_PP_M2006_page6_images/image001.eps}
\end{center}
D

$\mathrm{P}_{1}$

$\mathrm{P}_{2}$

A  B  C

AE $=5$ cm,

EC $=13$ cm,

BC $=6,5$ cm.
\begin{center}
\includegraphics[width=193.044mm,height=193.908mm]{./F1_M_PP_M2006_page6_images/image002.eps}

\includegraphics[width=165.756mm,height=17.580mm]{./F1_M_PP_M2006_page6_images/image003.eps}
\end{center}
Wypelnia

egzaminator!

Nr czynnoŚci

Maks. liczba kt

1

1

1

1

1

1

Uzyskana liczba pkt





{\it 8}

{\it Egzamin maturalny z matematyki}

{\it Arkusz I}

Zadanie 7. $(5pkt)$

Szkic przedstawia kanał ciepłowniczy, którego przekrój poprzeczny jest prostokątem.

Wewnątrz kanału znajduje się rurociąg składający się z trzech rur, $\mathrm{k}\mathrm{a}\dot{\mathrm{z}}$ da o średnicy

zewnętrznej l $\mathrm{m}$. Oblicz wysokość i szerokość kanału ciepłowniczego. Wysokość zaokrąglij

do 0,01 $\mathrm{m}.$
\begin{center}
\includegraphics[width=192.636mm,height=97.380mm]{./F1_M_PP_M2006_page7_images/image001.eps}

\includegraphics[width=192.228mm,height=157.632mm]{./F1_M_PP_M2006_page7_images/image002.eps}

\includegraphics[width=123.900mm,height=17.580mm]{./F1_M_PP_M2006_page7_images/image003.eps}
\end{center}
Wypelnia

egzamÍnator!

Nr czynności

Maks. liczba kt

7.1.

1

7.2.

1

7.3.

2

7.4.

1

Uzyskana liczba pkt





{\it Egzamin maturalny z matematyki}

{\it Arkusz I}

{\it 9}

Zadanie 8. $(5pkt)$

Dana jest ffinkcja $f(x) =-x^{2} +6x-5.$

a) Naszkicuj wykres funkcji $f$ i podaj jej zbiór wartości.

b) Podaj rozwiązanie nierówności $f(x)\geq 0.$
\begin{center}
\includegraphics[width=176.940mm,height=103.476mm]{./F1_M_PP_M2006_page8_images/image001.eps}

\includegraphics[width=192.276mm,height=151.584mm]{./F1_M_PP_M2006_page8_images/image002.eps}

\includegraphics[width=137.820mm,height=17.580mm]{./F1_M_PP_M2006_page8_images/image003.eps}
\end{center}
WypelnÍa

egzaminator!

Nr czynności

Maks. liczba kt

8.1.

1

8.2.

1

8.3.

1

8.4.

1

8.5.

1

Uzyskana liczba pkt





$ 1\theta$

{\it Egzamin maturalny z matematyki}

{\it Arkusz I}

Zadanie 9. $(6pkt)$

Dach wiez$\mathrm{y}$ ma kształt powierzchni bocznej ostrosłupa prawidłowego czworokątnego,

którego krawędzí podstawy ma długość 4 $\mathrm{m}$. Ściana boczna tego ostrosłupajest nachylona do

płaszczyzny podstawy pod kątem $60^{\mathrm{o}}$

a) Sporządz$\acute{}$ pomocniczy rysunek i zaznacz na nim podane w zadaniu wielkości.

b) Oblicz, ile sztuk dachówek nalez$\mathrm{y}$ kupić, aby pokryć ten dach, wiedząc, $\dot{\mathrm{z}}\mathrm{e}$ do pokrycia

$1\mathrm{m}^{2}$ potrzebne są24 dachówki. Przy zakupie na1ez$\mathrm{y}$ doliczyć 8\% dachówek na zapas.
\begin{center}
\includegraphics[width=192.228mm,height=242.364mm]{./F1_M_PP_M2006_page9_images/image001.eps}

\includegraphics[width=137.868mm,height=17.628mm]{./F1_M_PP_M2006_page9_images/image002.eps}
\end{center}
Nr czynności

Wypelnia Maks. liczba kt

egzaminator! Uzyskana liczba pkt

1

1

1

2

1






\begin{center}
\begin{tabular}{l|l}
\multicolumn{1}{l|}{$\begin{array}{l}\mbox{{\it dysleksja}}	\\	\mbox{Miejsce}	\\	\mbox{na naklejkę}	\\	\mbox{z kodem szkoly}	\end{array}$}&	\multicolumn{1}{|l}{ $\mathrm{M}\mathrm{M}\mathrm{A}-\mathrm{P}1_{-}1\mathrm{P}-072$}	\\
\hline
\multicolumn{1}{l|}{$\begin{array}{l}\mbox{EGZAMIN MATURALNY}	\\	\mbox{Z MATEMATYKI}	\\	\mbox{POZIOM PODSTAWOWY}	\\	\mbox{Czas pracy 120 minut}	\\	\mbox{Instrukcja dla zdającego}	\\	\mbox{1. Sprawdzí, czy arkusz egzaminacyjny zawiera 15 stron (zadania}	\\	\mbox{$1-11)$. Ewentualny brak zgłoś przewodniczącemu zespołu}	\\	\mbox{nadzorującego egzamin.}	\\	\mbox{2. Rozwiązania zadań i odpowiedzi zamieść w miejscu na to}	\\	\mbox{przeznaczonym.}	\\	\mbox{3. $\mathrm{W}$ rozwiązaniach zadań przedstaw tok rozumowania}	\\	\mbox{prowadzący do ostatecznego wyniku.}	\\	\mbox{4. Pisz czytelnie. Uzywaj $\mathrm{d}$ gopisu pióra tylko z czatnym}	\\	\mbox{tusze atramentem.}	\\	\mbox{5. Nie uzywaj korektora, a błędne zapisy prze eśl.}	\\	\mbox{6. Pamiętaj, $\dot{\mathrm{z}}\mathrm{e}$ zapisy w brudnopisie nie podlegają ocenie.}	\\	\mbox{7. Obok $\mathrm{k}\mathrm{a}\dot{\mathrm{z}}$ dego zadania podanajest maksymalna liczba punktów,}	\\	\mbox{którą mozesz uzyskać zajego poprawne rozwiązanie.}	\\	\mbox{8. $\mathrm{M}\mathrm{o}\dot{\mathrm{z}}$ esz korzystać z zestawu wzorów matematycznych, cyrkla}	\\	\mbox{i linijki oraz kalkulatora.}	\\	\mbox{9. Wypełnij tę część ka $\mathrm{y}$ odpowiedzi, którą koduje zdający.}	\\	\mbox{Nie wpisuj $\dot{\mathrm{z}}$ adnych znaków w części przeznaczonej dla}	\\	\mbox{egzaminatora.}	\\	\mbox{10. Na karcie odpowiedzi wpisz swoją datę urodzenia i PESEL.}	\\	\mbox{Zamaluj $\blacksquare$ pola odpowiadające cyfrom numeru PESEL. Błędne}	\\	\mbox{zaznaczenie otocz kółkiem $\mathrm{O}$ i zaznacz właściwe.}	\\	\mbox{{\it Zyczymy powodzenia}.'}	\end{array}$}&	\multicolumn{1}{|l}{$\begin{array}{l}\mbox{MAJ}	\\	\mbox{ROK 2007}	\\	\mbox{Za rozwiązanie}	\\	\mbox{wszystkich zadań}	\\	\mbox{mozna otrzymać}	\\	\mbox{łącznie}	\\	\mbox{50 punktów}	\end{array}$}	\\
\hline
\multicolumn{1}{l|}{$\begin{array}{l}\mbox{Wypelnia zdający}	\\	\mbox{rzed roz oczęciem racy}	\\	\mbox{PESEL ZDAJACEGO}	\end{array}$}&	\multicolumn{1}{|l}{$\begin{array}{l}\mbox{KOD}	\\	\mbox{ZDAJACEGO}	\end{array}$}
\end{tabular}


\includegraphics[width=21.840mm,height=9.852mm]{./F1_M_PP_M2007_page0_images/image001.eps}

\includegraphics[width=78.792mm,height=13.356mm]{./F1_M_PP_M2007_page0_images/image002.eps}
\end{center}



{\it 2}

{\it Egzamin maturalny z matematyki}

{\it Poziom podstawowy}

Zadanie 1. (5pkt)

Znajdzí wzór funkcji kwadratowej $y=f(x)$, której wykresem jest parabola o wierzchołku

$(1,-9)$ przechodząca przez punkt o współrzędnych $(2,-8)$. Otrzymaną funkcję przedstaw

w postaci kanonicznej. Obliczjej miejsca zerowe i naszkicuj wykres.
\begin{center}
\includegraphics[width=137.868mm,height=17.628mm]{./F1_M_PP_M2007_page1_images/image001.eps}
\end{center}
Nr czynnoŚci

Wypelnia Maks. liczba kt

egzaminator! Uzyskana liczba pkt

1.1.

1

1.2.

1

1.3.

1

1.4.

1

1.5.

1





{\it Egzamin maturalny z matematyki}

{\it Poziom podstawowy}

{\it 11}
\begin{center}
\includegraphics[width=151.788mm,height=17.580mm]{./F1_M_PP_M2007_page10_images/image001.eps}
\end{center}
WypelnÍa

egzaminator!

Nr czynności

Maks. lÍczba kt

1

1

1

1

1

Uzyskana liczba pkt





{\it 12}

{\it Egzamin maturalny z matematyki}

{\it Poziom podstawowy}

Zadanie 10. (5pkt)

Dany jest graniastosłup czworokątny prosty ABCDEFGH o podstawach ABCD $\mathrm{i}$ {\it EFGH oraz}

krawędziach bocznych $AE, BF, CG, DH$. Podstawa ABCD graniastosłupajest rombem o boku

długości 8 cm i kątach ostrych $A \mathrm{i} C$ o mierze $60^{\circ}$ Przekątna graniastosłupa $CE$ jest

nachylona do płaszczyzny podstawy pod kątem $60^{\circ}$ Sporządz$\acute{}$ rysunek pomocniczy i zaznacz

na nim wymienione w zadaniu kąty. Oblicz objętość tego graniastosłupa.





{\it Egzamin maturalny z matematyki}

{\it Poziom podstawowy}

{\it 13}
\begin{center}
\includegraphics[width=137.820mm,height=17.580mm]{./F1_M_PP_M2007_page12_images/image001.eps}
\end{center}
Wypelnia

egzaminator!

Nr czynno\S ci

Maks. liczba kt

10.1.

1

10.2.

1

10.3.

1

10.4.

1

10.5.

1

Uzyskana liczba pkt





{\it 14}

{\it Egzamin maturalny z matematyki}

{\it Poziom podstawowy}

Zadanie 11. (4pkt)

Dany jest rosnący ciąg geometryczny $(a_{n})$ dla

Oblicz $x$ oraz $y, \mathrm{j}\mathrm{e}\dot{\mathrm{z}}$ eli wiadomo, $\dot{\mathrm{z}}\mathrm{e}x+y=35$

$n\geq 1$, w którym $a_{1}=x, a_{2}=14, a_{3}=y.$
\begin{center}
\includegraphics[width=123.900mm,height=17.628mm]{./F1_M_PP_M2007_page13_images/image001.eps}
\end{center}
Wypelnia

egzaminator!

Nr czynności

Maks. liczba kt

11.1.

1

1

11.4.

1

Uzyskana liczba pkt





{\it Egzamin maturalny z matematyki}

{\it Poziom podstawowy}

{\it 15}

BRUDNOPIS





{\it Egzamin maturalny z matematyki}

{\it Poziom podstawowy}

{\it 3}

Zadanie 2. (3pkt)

Wysokość prowizji, którą klient płaci w pewnym biurze maklerskim przy $\mathrm{k}\mathrm{a}\dot{\mathrm{z}}$ dej zawieranej

transakcji kupna lub sprzedaz$\mathrm{y}$ akcji jest uzalezniona od wartości transakcji. Zalezność ta

została przedstawiona w tabeli:
\begin{center}
\begin{tabular}{|l|l|}
\hline
\multicolumn{1}{|l|}{Wartość transakcji}&	\multicolumn{1}{|l|}{Wysokość rowizji}	\\
\hline
\multicolumn{1}{|l|}{do 500 zł}&	\multicolumn{1}{|l|}{15 zł}	\\
\hline
\multicolumn{1}{|l|}{od 500,01 zł do 3000 zł}&	\multicolumn{1}{|l|}{2\% wartości $\mathrm{t}\mathrm{r}\mathrm{a}\mathrm{n}\mathrm{s}\mathrm{a}\mathrm{k}\mathrm{c}\mathrm{j}\mathrm{i}+5$ zł}	\\
\hline
\multicolumn{1}{|l|}{od 3000,01 zł do 8000 zł}&	\multicolumn{1}{|l|}{1,5\% wartości $\mathrm{t}\mathrm{r}\mathrm{a}\mathrm{n}\mathrm{s}\mathrm{a}\mathrm{k}\mathrm{c}\mathrm{j}\mathrm{i}+20$ zł}	\\
\hline
\multicolumn{1}{|l|}{od 8000,01 zł do 15000 zł}&	\multicolumn{1}{|l|}{1\% wartości $\mathrm{t}\mathrm{r}\mathrm{a}\mathrm{n}\mathrm{s}\mathrm{a}\mathrm{k}\mathrm{c}\mathrm{j}\mathrm{i}+60$ zł}	\\
\hline
\multicolumn{1}{|l|}{powyzej 15000 zł}&	\multicolumn{1}{|l|}{0,7\% wartości $\mathrm{t}\mathrm{r}\mathrm{a}\mathrm{n}\mathrm{s}\mathrm{a}\mathrm{k}\mathrm{c}\mathrm{j}\mathrm{i}+105$ zł}	\\
\hline
\end{tabular}

\end{center}
Klient zakupił za pośrednictwem tego biura maklerskiego 530 akcji w cenie 25 zł za jedną

akcję. Po roku sprzedał wszystkie kupione akcje po 45 zł zajedną sztukę. Ob1icz, i1e zarobił

na tych transakcjach po uwzględnieniu prowizji, które zapłacił.
\begin{center}
\includegraphics[width=109.980mm,height=17.580mm]{./F1_M_PP_M2007_page2_images/image001.eps}
\end{center}
Nr czynności

Wypelnia Maks. liczba kt

egzaminator! Uzyskana liczba pkt

2.1.

1

2.2.

1

2.3.

1





{\it 4}

{\it Egzamin maturalny z matematyki}

{\it Poziom podstawowy}

Zadanie 3. (4pkt)

Korzystając z danych przedstawionych na rysunku, oblicz wartość wyrazenia:

$\mathrm{t}\mathrm{g}^{2}\beta-5\sin\beta$. ctg $\alpha+\sqrt{1-\cos^{2}\alpha}.$

{\it C}
\begin{center}
\includegraphics[width=81.684mm,height=35.256mm]{./F1_M_PP_M2007_page3_images/image001.eps}
\end{center}
8  6

$\beta$

{\it A}  $\alpha$  {\it B}
\begin{center}
\includegraphics[width=123.900mm,height=17.628mm]{./F1_M_PP_M2007_page3_images/image002.eps}
\end{center}
Wypelnia

egzaminator!

Nr czynności

Maks. liczba kt

3.1.

1

3.2.

1

3.3.

1

3.4.

1

Uzyskana liczba pkt





{\it Egzamin maturalny z matematyki}

{\it Poziom podstawowy}

{\it 5}

Zadanie 4. (5pkt)

Samochód przebył w pewnym czasie 210 km. Gdybyjechał ze średnią prędkością o 10 km/h

większib to czas przejazdu skróciłby się o pół godziny. Oblicz, z jaką średnią prędkością

jechał ten samochód.
\begin{center}
\includegraphics[width=137.820mm,height=17.580mm]{./F1_M_PP_M2007_page4_images/image001.eps}
\end{center}
Wypelnia

egzaminator!

Nr czynno\S ci

Maks. liczba kt

4.1.

1

4.2.

1

4.3.

1

4.4.

1

4.5.

1

Uzyskana liczba pkt





{\it 6}

{\it Egzamin maturalny z matematyki}

{\it Poziom podstawowy}

Zadanie 5. (5pkt)

Dany jest ciąg arytmetyczny $(a_{n})$, gdzie $n\geq 1$. Wiadomo, $\dot{\mathrm{z}}\mathrm{e}$ dla $\mathrm{k}\mathrm{a}\dot{\mathrm{z}}$ dego $n\geq 1$

$n$ początkowych wyrazów $S_{n}=a_{1}+a_{2}+\ldots+a_{n}$ wyraza się wzorem: $S_{n}=-n^{2}+13n.$

a) Wyznacz wzór na $n-\mathrm{t}\mathrm{y}$ wyraz ciągu $(a_{n}).$

b) Oblicz a200$7^{\cdot}$

c) Wyznacz liczbę $n$, dla której $a_{n}=0.$

suma
\begin{center}
\includegraphics[width=137.868mm,height=17.580mm]{./F1_M_PP_M2007_page5_images/image001.eps}
\end{center}
Nr czynności

Wypelnia Maks. liczba kt

egzaminator! Uzyskana liczba pkt

5.1.

5.2.

1

5.3.

1

5.4.

1

5.5.

1





{\it Egzamin maturalny z matematyki}

{\it Poziom podstawowy}

7

Zadanie 6. (4pkt)

Dany jest wielomian $W(x)=2x^{3}+ax^{2}-14x+b.$

a) Dla $a=0 \mathrm{i} b=0$ otrzymamy wielomian $W(x)=2x^{3}-14x$. Rozwiąz równanie

$2x^{3}-14x=0.$

b) Dobierz wartości $a\mathrm{i}b$ tak, aby wielomian $W(x)$ był podzielny jednocześnie przez $x-2$

oraz przez $x+3.$
\begin{center}
\includegraphics[width=123.900mm,height=17.580mm]{./F1_M_PP_M2007_page6_images/image001.eps}
\end{center}
Nr czynności

Wypelnia Maks. liczba kt

egzamÍnator! Uzyskana liczba pkt

1

1

1





{\it 8}

{\it Egzamin maturalny z matematyki}

{\it Poziom podstawowy}

Zadanie 7. (5pkt)

Dany jest punkt $C=(2,3)$ i prosta o równaniu $y=2x-8$ będąca symetralną odcinka $BC.$

Wyznacz współrzędne punktu $B$. Wykonaj obliczenia uzasadniające odpowiedz.
\begin{center}
\includegraphics[width=137.868mm,height=17.628mm]{./F1_M_PP_M2007_page7_images/image001.eps}
\end{center}
Nr czynnoŚci

Wypelnia Maks. liczba kt

egzaminator! Uzyskana liczba pkt

7.1.

7.2.

1

7.3.

1

7.4.

1

7.5.

1





{\it Egzamin maturalny z matematyki}

{\it Poziom podstawowy}

{\it 9}

Zadanie 8. (4pkt)

Na stole $\mathrm{l}\mathrm{e}\dot{\mathrm{z}}$ ało 14 banknotów: 2 banknoty o nomina1e 100 zł, 2 banknoty o nomina1e 50 zł

$\mathrm{i} 10$ banknotów o nominale 20 zł. Wiatr zdmuchnął na podłogę 5 banknotów. Ob1icz

prawdopodobieństwo tego, $\dot{\mathrm{z}}\mathrm{e}$ na podłodze lezy dokładnie 130 zł. Odpowied $\acute{\mathrm{z}}$ podaj w postaci

ułamka nieskracalnego.
\begin{center}
\includegraphics[width=123.900mm,height=17.628mm]{./F1_M_PP_M2007_page8_images/image001.eps}
\end{center}
Nr czynnoŚci

Wypelnia Maks. liczba kt

egzaminator! Uzyskana liczba pkt

8.1.

1

8.2.

8.3.

1

8.4.

1





$ 1\theta$

{\it Egzamin maturalny z matematyki}

{\it Poziom podstawowy}

Zadanie 9. (6pkt)

Oblicz pole czworokąta wypukłego ABCD, w którym kąty wewnętrzne mają odpowiednio

miary: $4A=90^{\circ}, \triangleleft B=75^{\circ}, \triangleleft C=60^{\circ}, \triangleleft D=135^{\circ}$, a boki AB $\mathrm{i}$ AD mają długość 3 cm.

Sporządzí rysunek pomocniczy.







{\it ARKUSZ ZA WIERA INFORMACJE} $PRA$ {\it WNIE CHRONIONE}

{\it DO MOMENTU ROZPOCZĘCIA EGZAMINU}.$\displaystyle \int$
\begin{center}
\begin{tabular}{|l|l|l}
\cline{1-1}
\multicolumn{1}{|l|}{$\begin{array}{l}\mbox{Miejsce}	\\	\mbox{na na ejkę}	\end{array}$}&	\multicolumn{1}{|l|}{}&	\multicolumn{1}{|l}{ $\mathrm{M}\mathrm{M}\mathrm{A}-\mathrm{P}1_{-}1\mathrm{P}-082$}	\\
\hline
&	\multicolumn{1}{|l}{$\begin{array}{l}\mbox{MAJ}	\\	\mbox{ROK 2008}	\\	\mbox{Za rozwiązanie}	\\	\mbox{wszystkich zadań}	\\	\mbox{mozna otrzymać}	\\	\mbox{łącznie}	\\	\mbox{50 punktów}	\end{array}$}	\\
\cline{3-3}
&	\multicolumn{1}{|l}{$\begin{array}{l}\mbox{KOD}	\\	\mbox{ZDAJACEGO}	\end{array}$}
\end{tabular}


\includegraphics[width=21.840mm,height=9.852mm]{./F1_M_PP_M2008_page0_images/image001.eps}

\includegraphics[width=78.792mm,height=13.356mm]{./F1_M_PP_M2008_page0_images/image002.eps}
\end{center}



{\it 2 Egzamin maturalny z matematyki}

{\it Poziom podstawowy}

Zadanie l. $(4pkt)$

Na ponizszym rysunku przedstawiono łamaną ABCD, którajest wykresem ffinkcji $y=f(x).$
\begin{center}
\includegraphics[width=108.108mm,height=107.952mm]{./F1_M_PP_M2008_page1_images/image001.eps}
\end{center}
{\it y}

{\it C  D}

3

1

$-3 -2  -1  0 1_{1} 2_{1}$ 3  $1_{1} 2_{1}$  4  {\it x}

1

$-2$

{\it A  B}  $-4$

Korzystając z tego wykresu:

a) zapisz w postaci przedziału zbiór wartości funkcji $f,$

b) podaj wartość funkcji $f$ dla argumentu $x=1-\sqrt{10},$

c) wyznacz równanie prostej $BC,$

d) oblicz długość odcinka $BC.$





{\it Egzamin maturalny z matematyki ll}

{\it Poziom podstawowy}
\begin{center}
\includegraphics[width=123.900mm,height=17.832mm]{./F1_M_PP_M2008_page10_images/image001.eps}
\end{center}
Nr zadania

Wypelnia Maks. liczba kt

egzaminator! Uzyskana liczba pkt

7.1

1

7.2

1

7.3

1

7.4

1





{\it 12 Egzamin maturalny z matematyki}

{\it Poziom podstawowy}

Zadanie 8. $(4pkt)$

Dany jest wielomian $W(x)=x^{3}-5x^{2}-9x+45.$

a) Sprawdzí, czy punkt $A=(1$, 30$)$ nalezy do wykresu tego wielomianu.

b) Zapisz wielomian $W$ w postaci iloczynu trzech wielomianów stopnia pierwszego.
\begin{center}
\includegraphics[width=123.948mm,height=17.784mm]{./F1_M_PP_M2008_page11_images/image001.eps}
\end{center}
Nr zadania

Wypelnia Maks. liczba kt

egzaminator! Uzyskana lÍczba pkt

8.1

1

8.2

1

8.3

1

8.4

1





{\it Egzamin maturalny z matematyki 13}

{\it Poziom podstawowy}

Zadanie 9. (5pkt)

Oblicz najmniejszą i

w przedziale $\langle-2, 2\rangle.$

największą wartość

ffinkcji kwadratowej

$f(x)=(2x+1)(x-2)$
\begin{center}
\includegraphics[width=137.928mm,height=17.832mm]{./F1_M_PP_M2008_page12_images/image001.eps}
\end{center}
Wypelnia

egzaminator!

Nr zadania

Maks. liczba kt

1

1

Uzyskana liczba pkt





{\it 14 Egzamin maturalny z matematyki}

{\it Poziom podstawowy}

Zadanie 10. $(3pkt)$

Rysunek przedstawia fragment wykresu funkcji $h$, określonej wzorem $h(x)=\displaystyle \frac{a}{x}$ dla $x\neq 0.$

Wiadomo, $\dot{\mathrm{z}}\mathrm{e}$ do wykresu ffinkcji $h$ nalezy punkt $P=(2,5).$

a) Oblicz wartość współczynnika $a.$

b) Ustal, czy liczba $h(\pi)-h(-\pi)$ jest dodatnia czy ujemna.

c) Rozwiąz nierówność $h(x)>5.$
\begin{center}
\includegraphics[width=140.664mm,height=112.824mm]{./F1_M_PP_M2008_page13_images/image001.eps}
\end{center}
{\it y}

1

1  {\it x}





{\it Egzamin maturalny z matematyki 15}

{\it Poziom podstawowy}
\begin{center}
\includegraphics[width=109.980mm,height=17.832mm]{./F1_M_PP_M2008_page14_images/image001.eps}
\end{center}
Nr zadania

Wypelnia Maks. liczba kt

egzaminator! Uzyskana liczba pkt

10.1

10.2

10.3

1





{\it 16 Egzamin maturalny z matematyki}

{\it Poziom podstawowy}

Zadanie ll. $(5pkt)$

Pole powierzchni bocznej ostrosłupa prawidłowego trójkątnego równa się $\displaystyle \frac{a^{2}\sqrt{15}}{4}$, gdzie

$a$ oznacza długość krawędzi podstawy tego ostrosłupa. Zaznacz na ponizszym rysunku kąt

nachylenia ściany bocznej ostrosłupa do płaszczyzny jego podstawy. Miarę tego kąta oznacz

symbolem $\beta$. Oblicz $\cos\beta$ i korzystając z tablic funkcji trygonometrycznych odczytaj

przyblizoną wartość $\beta$ z dokładnością do $1^{\mathrm{o}}$





{\it Egzamin maturalny z matematyki 17}

{\it Poziom podstawowy}
\begin{center}
\includegraphics[width=137.928mm,height=17.832mm]{./F1_M_PP_M2008_page16_images/image001.eps}
\end{center}
Wypelnia

egzaminator!

Nr zadania

Maks. liczba kt

1

11.2

1

11.3

1

11.4

1

11.5

Uzyskana liczba pkt





{\it 18 Egzamin maturalny z matematyki}

{\it Poziom podstawowy}

Zadanie 12. $(4pkt)$

Rzucamy dwa razy symetryczną sześcienną kostką do gry. Oblicz prawdopodobieństwo

$\mathrm{k}\mathrm{a}\dot{\mathrm{z}}$ dego z następujących zdarzeń:

a) $A-\mathrm{w}\mathrm{k}\mathrm{a}\dot{\mathrm{z}}$ dym rzucie wypadnie nieparzysta liczba oczek.

b) $B-$ suma oczek otrzymanych w obu rzutachjest liczbą większą od 9.

c) $C-$ suma oczek otrzymanych w obu rzutachjest liczbą nieparzystą i większą od 9.
\begin{center}
\includegraphics[width=123.948mm,height=17.832mm]{./F1_M_PP_M2008_page17_images/image001.eps}
\end{center}
Wypelnia

egzaminator!

Nr zadania

Maks. liczba kt

12.1

1

12.2

1

12.3

1

12.4

1

Uzyskana liczba pkt





{\it Egzamin maturalny z matematyki 19}

{\it Poziom podstawowy}

BRUDNOPIS





{\it Egzamin maturalny z matematyki 3}

{\it Poziom podstawowy}
\begin{center}
\includegraphics[width=123.900mm,height=17.832mm]{./F1_M_PP_M2008_page2_images/image001.eps}
\end{center}
Nr zadania

Wypelnia Maks. liczba kt

egzaminator! Uzyskana liczba pkt

1.1

1

1.2

1

1.3

1

1.4

1





{\it 4 Egzamin maturalny z matematyki}

{\it Poziom podstawowy}

Zadanie 2. (4pkt)

Liczba przekątnych wielokąta wypukłego, w którymjest $n$ boków i $n\geq 3$ wyraza się wzorem

$P(n)=\displaystyle \frac{n(n-3)}{2}.$

Wykorzystując ten wzór:

a) oblicz liczbę przekątnych w dwudziestokącie wypukłym.

b) oblicz, ile boków ma wielokąt wypukły, w którym liczba przekątnych jest pięć razy

większa od liczby boków.

c) sprawd $\acute{\mathrm{z}}$, czy jest prawdziwe następujące stwierdzenie:

{\it Kazdy wielokqt wypukly o parzystej liczbie boków ma parzystq liczbę przekqtnych}.

Odpowied $\acute{\mathrm{z}}$ uzasadnij.
\begin{center}
\includegraphics[width=123.948mm,height=17.784mm]{./F1_M_PP_M2008_page3_images/image001.eps}
\end{center}
Nr zadania

Wypelnia Maks. liczba kt

egzamÍnator! Uzyskana lÍczba pkt

2.1

1

2.2

1

2.3

1

2.4

1





{\it Egzamin maturalny z matematyki 5}

{\it Poziom podstawowy}

Zadanie 3. $(4pkt)$

Rozwiąz. równanie $4^{23}x-32^{9}x=16^{4}\cdot(4^{4})^{4}$

Zapisz rozwiązanie tego równania w postaci $2^{k}$, gdzie kjest liczbą całkowitą.
\begin{center}
\includegraphics[width=123.900mm,height=17.784mm]{./F1_M_PP_M2008_page4_images/image001.eps}
\end{center}
Nr zadania

Wypelnia Maks. liczba kt

egzaminator! Uzyskana liczba pkt

3.1

1

3.2

1

3.3

1

3.4

1





{\it 6 Egzamin maturalny z matematyki}

{\it Poziom podstawowy}

Zadanie 4. (3pkt)

Koncetn paliwowy podnosił dwukrotnie w jednym tygodniu cenę benzyny, pierwszy raz

010\%, a drugi raz o 5\%. Po obu tych podwyzkachjeden litr benzyny, wyprodukowanej przez

ten koncern, kosztuje 4,62 zł. Ob1icz cenę jednego 1itra benzyny przed omawianymi

podwyzkami.
\begin{center}
\includegraphics[width=109.932mm,height=17.832mm]{./F1_M_PP_M2008_page5_images/image001.eps}
\end{center}
Wypelnia

egzaminator!

Nr zadania

Maks. liczba kt

4.2

4.3

1

Uzyskana liczba pkt





{\it Egzamin maturalny z matematyki 7}

{\it Poziom podstawowy}

Zadanie 5. $(5pkt)$

Nieskończony ciąg liczbowy $(a_{n})$ jest określony wzorem $a_{n}=2-\displaystyle \frac{1}{n}, n=1$, 2, 3,$\ldots.$

a) Oblicz, ile wyrazów ciągu $(a_{n})$ jest mniejszych od 1,975.

b) Dla pewnej liczby $x$ trzywyrazowy ciąg $(a_{2},a_{7},x)$ jest arytmetyczny. Oblicz $x.$
\begin{center}
\includegraphics[width=137.928mm,height=17.784mm]{./F1_M_PP_M2008_page6_images/image001.eps}
\end{center}
Nr zadania

Wypelnia Maks. liczba $\mathrm{k}\iota$

egzaminator! Uzyskana lÍczba pkt

5.1

1

5.2

1

5.3

1

5.4

5.5

1





{\it 8 Egzamin maturalny z matematyki}

{\it Poziom podstawowy}

Zadanie 6. $(5pkt)$

Prosta o równaniu $5x+4y-10=0$ przecina oś $Ox$ układu współrzędnych w punkcie $A$ oraz

oś $Oy$ w punkcie $B$. Oblicz współrzędne wszystkich punktów $C$ lez$\cdot$ących na osi $Ox$ i takich,

$\dot{\mathrm{z}}\mathrm{e}$ trójkąt $ABC$ ma pole równe 35.





{\it Egzamin maturalny z matematyki 9}

{\it Poziom podstawowy}
\begin{center}
\includegraphics[width=137.928mm,height=17.832mm]{./F1_M_PP_M2008_page8_images/image001.eps}
\end{center}
Wypelnia

egzaminator!

Nr zadania

Maks. liczba kt

1

1

1

1

Uzyskana liczba pkt





$ 1\theta$ {\it Egzamin maturalny z matematyki}

{\it Poziom podstawowy}

Zadanie 7. $(4pkt)$

Dany jest trapez, w którym podstawy mają długość 4 cm i 10 cm oraz ramiona tworzą

z dłuzsząpodstawą kąty o miarach $30^{\mathrm{o}}$ i $45^{\mathrm{o}}$. Oblicz wysokość tego trapezu.







{\it ARKUSZ ZA WIERA INFORMACJE} $PRA$ {\it WNIE CHRONIONE}

{\it DO MOMENTU ROZPOCZĘCIA EGZAMINU}.$\displaystyle \int$
\begin{center}
\includegraphics[width=192.024mm,height=288.084mm]{./F1_M_PP_M2009_page0_images/image001.eps}
\end{center}
Miejsce

na na ejkę

MMA-PI IP-092

EGZAMIN MATURALNY

MAJ

Z MATEMATYKI

POZIOM PODSTAWOWY

Czas pracy 120 minut

Instrukcja dla zdającego

1.

2.

3.

4.

5.

6.

7.

8.

9.

Sprawd $\acute{\mathrm{z}}$, czy arkusz egzaminacyjny zawiera 16 stron (zadania

$1-11)$. Ewentualny brak zgłoś przewodniczącemu zespołu

nadzorującego egzamin.

Rozwiązania zadań i odpowiedzi zamieść w miejscu na to

przeznaczonym.

W rozwiązaniach zadań przedstaw tok rozumowania

prowadzący do ostatecznego wyniku.

Pisz czytelnie. Uzywaj $\mathrm{d}$ gopisu pióra tylko z czatnym

tusze atramentem.

Nie uzywaj korektora, a błędne zapisy prze eśl.

Pamiętaj, $\dot{\mathrm{z}}\mathrm{e}$ zapisy w brudnopisie nie podlegają ocenie.

Obok $\mathrm{k}\mathrm{a}\dot{\mathrm{z}}$ dego zadania podanajest maksymalna liczba punktów,

którą $\mathrm{m}\mathrm{o}\dot{\mathrm{z}}$ esz uzyskać zajego poprawne rozwiązanie.

$\mathrm{M}\mathrm{o}\dot{\mathrm{z}}$ esz korzystać z zesta wzorów matematycznych, cyrkla

i linijki oraz kalkulatora.

Na karcie odpowiedzi wpisz swoją datę urodzenia i PESEL.

Nie wpisuj $\dot{\mathrm{z}}$ adnych znaków w części przeznaczonej

dla egzaminatora.

Za rozwiązanie

wszystkich zadań

mozna otrzymać

łącznie

50 punktów

{\it Zyczymy} $pow\theta dzenia'$

Wypelnia zdający

rzed roz oczęciem racy

PESEL ZDAJACEGO

KOD

ZDAJACEGO




{\it 2}

{\it Egzamin maturalny z matematyki}

{\it Poziom podstawowy}

Zadanie l. $(5pkt)$

Funkcja $f$ określona jest wzorem $f(x)=$

a) Uzupełnij tabelę:

dla $x<2$

dla $2\leq x\leq 4$
\begin{center}
\begin{tabular}{|l|l|l|l|}
\hline
\multicolumn{1}{|l|}{$x$}&	\multicolumn{1}{|l|}{ $-3$}&	\multicolumn{1}{|l|}{ $3$}&	\multicolumn{1}{|l|}{}	\\
\hline
\multicolumn{1}{|l|}{ $f(x)$}&	\multicolumn{1}{|l|}{}&	\multicolumn{1}{|l|}{}&	\multicolumn{1}{|l|}{ $0$}	\\
\hline
\end{tabular}

\end{center}
b) Narysuj wykres funkcji $f.$

c) Podaj wszystkie liczby całkowite $x$, spełniające nierówność $f(x)\geq-6.$
\begin{center}
\includegraphics[width=137.868mm,height=17.832mm]{./F1_M_PP_M2009_page1_images/image001.eps}
\end{center}
Nr zadania

Wypelnia Maks. liczba kt

egzaminator! Uzyskana liczba pkt

1.1

1

1

1.3

1

1.4

1

1.5





{\it Egzamin maturalny z matematyki}

{\it Poziom podstawowy}

{\it 11}
\begin{center}
\includegraphics[width=192.276mm,height=290.784mm]{./F1_M_PP_M2009_page10_images/image001.eps}

\includegraphics[width=123.900mm,height=17.784mm]{./F1_M_PP_M2009_page10_images/image002.eps}
\end{center}
Nr zadania

Wypelnia Maks. liczba kt

egzamÍnator! Uzyskana liczba pkt

8.1

1

8.2

1

8.3

1

8.4

1





{\it 12}

{\it Egzamin maturalny z matematyki}

{\it Poziom podstawowy}

Zadanie 9. $(4pkt)$

Punkty $B=(0,10) \mathrm{i} O=(0,0)$ są wierzchołkami trójkąta prostokątnego $OAB$, w którym

$|\neq OAB|=90^{\mathrm{o}}$ Przyprostokątna $OA$ zawiera się w prostej o równaniu

współrzędne punktu $A$ i długość przyprostokątnej $OA.$

$y=\displaystyle \frac{1}{2}x$. Oblicz
\begin{center}
\includegraphics[width=192.228mm,height=254.460mm]{./F1_M_PP_M2009_page11_images/image001.eps}

\includegraphics[width=123.948mm,height=17.784mm]{./F1_M_PP_M2009_page11_images/image002.eps}
\end{center}
Nr zadania

Wypelnia Maks. liczba kt

egzamÍnator! Uzyskana lÍczba pkt

1

1

1

1





{\it Egzamin maturalny z matematyki}

{\it Poziom podstawowy}

{\it 13}

Zadanie 10. $(5pkt)$

Tabela przedstawia wyniki części teoretycznej egzaminu na prawo jazdy. Zdający uzyskał

wynik pozytywny, $\mathrm{j}\mathrm{e}\dot{\mathrm{z}}$ eli popełnił co najwyzej dwa błędy.
\begin{center}
\begin{tabular}{|l|l|l|l|l|l|l|l|l|l|}
\hline
\multicolumn{1}{|l|}{liczba błędów}&	\multicolumn{1}{|l|}{$0$}&	\multicolumn{1}{|l|}{ $1$}&	\multicolumn{1}{|l|}{ $2$}&	\multicolumn{1}{|l|}{ $3$}&	\multicolumn{1}{|l|}{ $4$}&	\multicolumn{1}{|l|}{ $5$}&	\multicolumn{1}{|l|}{ $6$}&	\multicolumn{1}{|l|}{ $7$}&	\multicolumn{1}{|l|}{ $8$}	\\
\hline
\multicolumn{1}{|l|}{liczba zdających}&	\multicolumn{1}{|l|}{$8$}&	\multicolumn{1}{|l|}{ $5$}&	\multicolumn{1}{|l|}{ $8$}&	\multicolumn{1}{|l|}{ $5$}&	\multicolumn{1}{|l|}{ $2$}&	\multicolumn{1}{|l|}{ $1$}&	\multicolumn{1}{|l|}{ $0$}&	\multicolumn{1}{|l|}{ $0$}&	\multicolumn{1}{|l|}{ $1$}	\\
\hline
\end{tabular}

\end{center}
a) Oblicz średnią arytmetyczną liczby błędów popełnionych przez zdających ten egzamin.

Wynik podaj w zaokrągleniu do całości.

b) Oblicz prawdopodobieństwo, $\dot{\mathrm{z}}\mathrm{e}$ wśród dwóch losowo wybranych zdających tylko jeden

uzyskał wynik pozytywny. Wynik zapisz w postaci ułamka zwykłego nieskracalnego.
\begin{center}
\includegraphics[width=192.276mm,height=218.136mm]{./F1_M_PP_M2009_page12_images/image001.eps}

\includegraphics[width=137.928mm,height=17.832mm]{./F1_M_PP_M2009_page12_images/image002.eps}
\end{center}
Nr zadania

Wypelnia Maks. liczba kt

egzaminator! Uzyskana liczba pkt

10.1

1

10.2

1

10.3

1

10.4

1

10.5





{\it 14}

{\it Egzamin maturalny z matematyki}

{\it Poziom podstawowy}

Zadanie ll. $(5pkt)$

Powierzchnia boczna walca po rozwinięciu na płaszczyznę jest prostokątem. Przekątna tego

prostokąta ma długość 12 i tworzy z bokiem, którego długość jest równa wysokości wa1ca,

kąt o mierze $30^{\circ}$

a) Oblicz pole powierzchni bocznej tego walca.

b) Sprawdzí, czy objętość tego walcajest większa od $18\sqrt{3}$. Odpowiedzí uzasadnij.
\begin{center}
\includegraphics[width=192.228mm,height=272.640mm]{./F1_M_PP_M2009_page13_images/image001.eps}
\end{center}




{\it Egzamin maturalny z matematyki}

{\it Poziom podstawowy}

{\it 15}
\begin{center}
\includegraphics[width=192.276mm,height=290.784mm]{./F1_M_PP_M2009_page14_images/image001.eps}

\includegraphics[width=137.928mm,height=17.784mm]{./F1_M_PP_M2009_page14_images/image002.eps}
\end{center}
Nr zadania

Wypelnia Maks. liczba kt

egzaminator! Uzyskana lÍczba pkt

11.1

1

11.2

1

11.3

11.4

11.5

1





{\it 16}

{\it Egzamin maturalny z matematyki}

{\it Poziom podstawowy}

BRUDNOPIS





{\it Egzamin maturalny z matematyki}

{\it Poziom podstawowy}

{\it 3}

Zadanie 2. $(3pkt)$

Dwaj rzemieślnicy przyjęli zlecenie wykonania wspólnie 980 deta1i. Zap1anowa1i, $\dot{\mathrm{z}}\mathrm{e}$

$\mathrm{k}\mathrm{a}\dot{\mathrm{z}}$ dego dnia pierwszy z nich wykona $m$, a drugi $n$ detali. Obliczyli, $\dot{\mathrm{z}}\mathrm{e}$ razem wykonają

zlecenie w ciągu 7 dni. Po pierwszym dniu pracy pierwszy z rzemieś1ników rozchorował się

i wtedy drugi, aby wykonać całe zlecenie, musiał pracować o 8 dni dłuzej $\mathrm{n}\mathrm{i}\dot{\mathrm{z}}$ planował, (nie

zmieniając liczby wykonywanych codziennie detali). Oblicz $m \mathrm{i} n.$
\begin{center}
\includegraphics[width=192.276mm,height=248.364mm]{./F1_M_PP_M2009_page2_images/image001.eps}

\includegraphics[width=109.980mm,height=17.784mm]{./F1_M_PP_M2009_page2_images/image002.eps}
\end{center}
Nr zadania

Wypelnia Maks. liczba kt

egzaminator! Uzyskana lÍczba pkt

2.1

2.2

1

2.3

1





{\it 4}

{\it Egzamin maturalny z matematyki}

{\it Poziom podstawowy}

Zadanie 3. $(5pkt)$

Wykres funkcji $f$ danej wzorem $f(x)=-2x^{2}$ przesunięto wzdłuz osi $Ox 0 3$ jednostki

w prawo oraz wzdłuz osi $Oy\mathrm{o}$ 8jednostek w górę, otrzymując wykres funkcji $g.$

a) Rozwiąz nierówność $f(x)+5<3x.$

b) Podaj zbiór wartości funkcji $g.$

c) Funkcja $g$ określonajest wzorem $g(x)=-2x^{2}+bx+c$. Oblicz $b\mathrm{i}c.$
\begin{center}
\includegraphics[width=192.228mm,height=260.508mm]{./F1_M_PP_M2009_page3_images/image001.eps}
\end{center}




{\it Egzamin maturalny z matematyki}

{\it Poziom podstawowy}

{\it 5}
\begin{center}
\includegraphics[width=192.276mm,height=290.784mm]{./F1_M_PP_M2009_page4_images/image001.eps}

\includegraphics[width=137.928mm,height=17.784mm]{./F1_M_PP_M2009_page4_images/image002.eps}
\end{center}
Nr zadanÍa

Wypelnia Maks. liczba kt

egzaminator! Uzyskana lÍczba pkt

1

3.2

1

3.3

3.4

3.5

1





{\it 6}

{\it Egzamin maturalny z matematyki}

{\it Poziom podstawowy}

Zadanie 4. $(3pkt)$

Wykaz, $\dot{\mathrm{z}}\mathrm{e}$ liczba $3^{54}$jest rozwiązaniem równania $243^{11}-81^{14}+7x=9^{27}$
\begin{center}
\includegraphics[width=109.932mm,height=17.832mm]{./F1_M_PP_M2009_page5_images/image001.eps}
\end{center}
Wypelnia

egzaminator!

Nr zadania

Maks. liczba kt

4.2

4.3

1

Uzyskana liczba pkt





{\it Egzamin maturalny z matematyki}

{\it Poziom podstawowy}

7

Zadanie 5. $(5pkt)$

Wielomian $W$ dany jest wzorem $W(x)=x^{3}+ax^{2}-4x+b.$

a) Wyznacz $a, b$ oraz $c$ tak, aby wielomian $W$ był równy wielomianowi $P$, gdy

$P(x)=x^{3}+(2a+3)x^{2}+(a+b+c)x-1.$

b) Dla $a=3 \mathrm{i} b=0$ zapisz wielomian $W$ w postaci iloczynu trzech wielomianów stopnia

pierwszego.
\begin{center}
\includegraphics[width=137.928mm,height=17.832mm]{./F1_M_PP_M2009_page6_images/image001.eps}
\end{center}
Nr zadania

Wypelnia Maks. liczba kt

egzaminator! Uzyskana liczba pkt

5.1

1

5.2

1

5.3

1

5.4

1

5.5





{\it 8}

{\it Egzamin maturalny z matematyki}

{\it Poziom podstawowy}

Zadanie 6. $(5pkt)$

Miarajednego z kątów ostrych w trójkącie prostokątnymjest równa $\alpha.$

a) Uzasadnij, ze spełnionajest nierówność $\sin\alpha-\mathrm{t}\mathrm{g}\alpha<0.$

b) Dla $\displaystyle \sin\alpha=\frac{\mathrm{z}\sqrt{2}}{3}$ oblicz wartość wyrazenia $\cos^{3}\alpha+\cos\alpha\cdot\sin^{2}\alpha.$
\begin{center}
\includegraphics[width=192.228mm,height=254.460mm]{./F1_M_PP_M2009_page7_images/image001.eps}

\includegraphics[width=137.868mm,height=17.832mm]{./F1_M_PP_M2009_page7_images/image002.eps}
\end{center}
Nr zadania

Wypelnia Maks. liczba kt

egzaminator! Uzyskana liczba pkt

1

1





{\it Egzamin maturalny z matematyki}

{\it Poziom podstawowy}

{\it 9}

Zadanie 7. $(6pkt)$

Dany jest ciąg arytmetyczny $(a_{n})$ dla $n\geq 1$, w którym $a_{7}=1, a_{11}=9.$

a) Oblicz pierwszy wyraz $a_{1}$ i róznicę $r$ ciągu $(a_{n}).$

b) Sprawdzí, czy ciąg $(a_{7},a_{8},a_{11})$ jest geometryczny.

c) Wyznacz takie $n$, aby suma $n$ początkowych wyrazów ciągu

najmniejszą.

$(a_{n})$

miała wartość
\begin{center}
\includegraphics[width=192.276mm,height=242.316mm]{./F1_M_PP_M2009_page8_images/image001.eps}

\includegraphics[width=151.836mm,height=17.784mm]{./F1_M_PP_M2009_page8_images/image002.eps}
\end{center}
Wypelnia

egzaminator!

Nr zadania

Maks. liczba kt

7.1

1

7.2

1

7.3

1

7.4

1

7.5

1

1

Uzyskana liczba pkt





$ 1\theta$

{\it Egzamin maturalny z matematyki}

{\it Poziom podstawowy}

Zadanie 8. (4pkt)

W trapezie ABCD długość podstawy CD jest równa 18, a długości ramion trapezu AD iBC

są odpowiednio równe 25 i 15. Kąty ADBi DCB, zaznaczone na rysunku, mają równe miary.

Oblicz obwód tego trapezu.
\begin{center}
\includegraphics[width=136.704mm,height=55.680mm]{./F1_M_PP_M2009_page9_images/image001.eps}
\end{center}
{\it D  C}

{\it A  B}







Centralna Komisja Egzaminacyjna

Arkusz zawiera informacje prawnie chronione do momentu rozpoczęcia egzaminu.

WPISUJE ZDAJACY

KOD PESEL

{\it Miejsce}

{\it na naklejkę}

{\it z kodem}
\begin{center}
\includegraphics[width=21.432mm,height=9.852mm]{./F1_M_PP_M2010_page0_images/image001.eps}

\includegraphics[width=82.092mm,height=9.852mm]{./F1_M_PP_M2010_page0_images/image002.eps}

\includegraphics[width=204.060mm,height=216.048mm]{./F1_M_PP_M2010_page0_images/image003.eps}
\end{center}
EGZAMIN MATU

Z MATEMATY

LNY

POZIOM PODSTAWOWY  MAJ 2010

l. Sprawdzí, czy arkusz egzaminacyjny zawiera 20 stron

(zadania $1-34$). Ewentualny brak zgłoś przewodniczącemu

zespo nadzo jącego egzamin.

2. Rozwiązania zadań i odpowiedzi wpisuj w miejscu na to

przeznaczonym.

3. Odpowiedzi do zadań za iętych (l-25) przenieś

na ka ę odpowiedzi, zaznaczając je w części ka $\mathrm{y}$

przeznaczonej dla zdającego. Zamaluj $\blacksquare$ pola do tego

przeznaczone. Błędne zaznaczenie otocz kółkiem \fcircle$\bullet$

i zaznacz właściwe.

4. Pamiętaj, $\dot{\mathrm{z}}\mathrm{e}$ pominięcie argumentacji lub istotnych

obliczeń w rozwiązaniu zadania otwa ego (26-34) $\mathrm{m}\mathrm{o}\dot{\mathrm{z}}\mathrm{e}$

spowodować, $\dot{\mathrm{z}}\mathrm{e}$ za to rozwiązanie nie będziesz mógł

dostać pełnej liczby punktów.

5. Pisz czytelnie i $\mathrm{u}\dot{\mathrm{z}}$ aj tvlko $\mathrm{d}$ gopisu lub -Dióra

z czamym tuszem lub atramentem.

6. Nie uzywaj korektora, a błędne zapisy wyra $\acute{\mathrm{z}}\mathrm{n}\mathrm{i}\mathrm{e}$ prze eśl.

7. Pamiętaj, $\dot{\mathrm{z}}\mathrm{e}$ zapisy w brudnopisie nie będą oceniane.

8. $\mathrm{M}\mathrm{o}\dot{\mathrm{z}}$ esz korzystać z zesta wzorów matematycznych,

cyrkla i linijki oraz kalkulatora.

9. Na karcie odpowiedzi wpisz swój numer PESEL i przyklej

naklejkę z kodem.

10. Nie wpisuj $\dot{\mathrm{z}}$ adnych znaków w części przeznaczonej dla

egzaminatora.

Czas pracy:

170 minut

Liczba punktów

do uzyskania: 50

$\Vert\Vert\Vert\Vert\Vert\Vert\Vert\Vert\Vert\Vert\Vert\Vert\Vert\Vert\Vert\Vert\Vert\Vert\Vert\Vert\Vert\Vert\Vert\Vert|  \mathrm{M}\mathrm{M}\mathrm{A}-\mathrm{P}1_{-}1\mathrm{P}-102$




{\it 2}

{\it Egzamin maturalny z matematyki}

{\it Poziom podstawowy}

ZADANIA ZAMKNIĘTE

{\it Wzadaniach} $\theta d1.$ {\it do 25. wybierz i zaznacz na karcie odpowiedzipoprawnq odpowied} $\acute{z}.$

Zadanie l. $(1pkt)$

Wskaz rysunek, na którymjest przedstawiony zbiór rozwiązań nierówności $|x+7|>5.$
\begin{center}
\includegraphics[width=173.280mm,height=13.212mm]{./F1_M_PP_M2010_page1_images/image001.eps}
\end{center}
$-12$  2  {\it x}

A.
\begin{center}
\includegraphics[width=175.008mm,height=13.812mm]{./F1_M_PP_M2010_page1_images/image002.eps}
\end{center}
2  12  {\it x}

B.
\begin{center}
\includegraphics[width=173.280mm,height=13.260mm]{./F1_M_PP_M2010_page1_images/image003.eps}
\end{center}
$-12  -2$  {\it x}

C.
\begin{center}
\includegraphics[width=171.756mm,height=13.104mm]{./F1_M_PP_M2010_page1_images/image004.eps}
\end{center}
$-2$  12  {\it x}

D.

Zadanie 2. (1pkt)

Spodnie po obnizce ceny o 30\% kosztują 126 zł. I1e kosztowały spodnie przed obnizką?

A. 163,80 zł

B. 180 zł

C. 294 zł

D. 420 zł

Zadanie 3. $(1pkt)$

Liczba $(\displaystyle \frac{2^{-2}\cdot 3^{-1}}{2^{-1}3^{-2}})^{0}$ jest równa

A. I B. 4

C. 9

D. 36

Zadanie 4. (1pkt)

Liczba $\log_{4}8+\log_{4}2$ jest równa

A. l

B. 2

C. $\log_{4}6$

D. log410

Zadanie 5. $(1pkt)$

Dane są wielomiany $W(x)=-2x^{3}+5x^{2}-3$ oraz $P(x)=2x^{3}+12x$. Wielomian $W(x)+P(x)$

jest równy

A. $5x^{2}+12x-3$

B. $4x^{3}+5x^{2}+12x-3$

C. $4x^{6}+5x^{2}+12x-3$

D. $4x^{3}+12x^{2}-3$





{\it Egzamin maturalny z matematyki}

{\it Poziom podstawowy}

{\it 11}

Zadanie 28. (2pkt)

Trójkąty prostokątne równoramienne $ABC\mathrm{i}CDE$ są połozone tak, jak na ponizszym rysunku

(w obu trójkątach kąt przy wierzchołku $C$ jest prosty). Wykaz, $\dot{\mathrm{z}}\mathrm{e}|AD|=|BE|.$

{\it C}
\begin{center}
\includegraphics[width=76.404mm,height=36.072mm]{./F1_M_PP_M2010_page10_images/image001.eps}
\end{center}
{\it E}

{\it D}

{\it A  B}
\begin{center}
\includegraphics[width=109.980mm,height=17.832mm]{./F1_M_PP_M2010_page10_images/image002.eps}
\end{center}
Nr zadani,`

Wypelnia Maks. liczba kt

egzaminator

Uzyskana liczba pkt

2

27.

2

28.

2





{\it 12}

{\it Egzamin maturalny z matematyki}

{\it Poziom podstawowy}

Zadanie 29. $(2pkt)$

Kąt $\alpha$ jest ostry i $\displaystyle \mathrm{t}\mathrm{g}\alpha=\frac{5}{12}$. Oblicz $\cos\alpha.$

Odpowied $\acute{\mathrm{z}}$:

Zadanie 30. $(2pkt)$

Wyka $\dot{\mathrm{z}}$, ze jeśli $a>0$, to $\displaystyle \frac{a^{2}+1}{a+1}\geq\frac{a+1}{2}.$





{\it Egzamin maturalny z matematyki}

{\it Poziom podstawowy}

{\it 13}

Zadanie 31. (2pkt)

W trapezie prostokątnym krótsza przekątna dzieli go na trójkąt prostokątny

równoboczny. Dłuzsza podstawa trapezujest równa 6. Ob1icz obwód tego trapezu.

i trójkąt

Odpowiedzí :
\begin{center}
\includegraphics[width=109.980mm,height=17.784mm]{./F1_M_PP_M2010_page12_images/image001.eps}
\end{center}
Nr zadania

Wypelnia Maks. liczba kt

egzaminator

Uzyskana lÍczba pkt

2

30.

2

31.

2





{\it 14}

{\it Egzamin maturalny z matematyki}

{\it Poziom podstawowy}

Zadanie 32. $(4pkt)$

Podstawą ostrosłupa ABCD jest trójkąt $ABC$. Krawędzí AD jest wysokością ostrosłupa (zobacz

rysunek). Oblicz objętość ostrostupa ABCD, jeśli wiadomo, $\dot{\mathrm{z}}\mathrm{e} |AD|=12, |BC|=6,$

$|BD|=|CD|=13.$

{\it D}





{\it Egzamin maturalny z matematyki}

{\it Poziom podstawowy}

{\it 15}

Odpowiedzí :
\begin{center}
\includegraphics[width=82.044mm,height=17.832mm]{./F1_M_PP_M2010_page14_images/image001.eps}
\end{center}
Wypelnia

egzaminator

Nr zadania

Maks. liczba kt

32.

4

Uzyskana liczba pkt





{\it 16}

{\it Egzamin maturalny z matematyki}

{\it Poziom podstawowy}

Zadanie 33. $(4pkt)$

Doświadczenie losowe polega na dwukrotnym rzucie symetryczną sześcienną kostką do gry.

Oblicz prawdopodobieństwo zdarzenia $A$ polegającego na tym, $\dot{\mathrm{z}}\mathrm{e}$ w pierwszym rzucie

otrzymamy parzystą liczbę oczek i iloczyn liczb oczek w obu rzutach będzie podzielny przez 12.

Wynik przedstaw w postaci ułamka zwykłego nieskracalnego.





{\it Egzamin maturalny z matematyki}

{\it Poziom podstawowy}

{\it 1}7

Odpowiedzí :
\begin{center}
\includegraphics[width=82.044mm,height=17.784mm]{./F1_M_PP_M2010_page16_images/image001.eps}
\end{center}
Wypelnia

egzaminator

Nr zadania

Maks. lÍczba kt

33.

4

Uzyskana lÍczba pkt





{\it 18}

{\it Egzamin maturalny z matematyki}

{\it Poziom podstawowy}

Zadanie 34. $(5pkt)$

$\mathrm{W}$ dwóch hotelach wybudowano prostokątne baseny. Basen w pierwszym hotelu

ma powierzchnię 240 $\mathrm{m}^{2}$ Basen w drugim hotelu ma powierzchnię 350 $\mathrm{m}^{2}$ oraz jest o 5 $\mathrm{m}$

dłuzszy i 2 $\mathrm{m}$ szerszy $\mathrm{n}\mathrm{i}\dot{\mathrm{z}}$ w pierwszym hotelu. Oblicz, jakie wymiary mogą mieć baseny

w obu hotelach. Podaj wszystkie $\mathrm{m}\mathrm{o}\dot{\mathrm{z}}$ liwe odpowiedzi.





{\it Egzamin maturalny z matematyki}

{\it Poziom podstawowy}

{\it 19}

Odpowiedzí :
\begin{center}
\includegraphics[width=82.044mm,height=17.832mm]{./F1_M_PP_M2010_page18_images/image001.eps}
\end{center}
Wypelnia

egzaminator

Nr zadania

Maks. liczba kt

34.

5

Uzyskana liczba pkt





$ 2\theta$

{\it Egzamin maturalny z matematyki}

{\it Poziom podstawowy}

BRUDNOPIS





{\it Egzamin maturalny z matematyki}

{\it Poziom podstawowy}

{\it 3}

BRUDNOPIS





{\it 4}

{\it Egzamin maturalny z matematyki}

{\it Poziom podstawowy}

Zadanie 6. $(1pkt)$

Rozwiązaniem równania $\displaystyle \frac{3x-1}{7x+1}=\frac{2}{5}$ jest

A. 1 B. -73

C.

-47

D. 7

Zadanie 7. $(1pkt)$

Do zbioru rozwiązań nierównoŚci $(x-2)(x+3)<0$ nalezy liczba

A. 9 B. 7 C. 4

D. l

Zadanie 8. $(1pkt)$

Wykresem funkcji kwadratowej $f(x)=-3x^{2}+3$ jest parabola o wierzchołku w punkcie

A. $($3, $0)$ B. $(0,3)$ C. $(-3,0)$ D. $(0,-3)$

Zadanie 9. $(1pkt)$

Prosta o równaniu $y=-2x+(3m+3)$ przecina w układzie współrzędnych oś $Oy$ w punkcie

(0,2). Wtedy

A. {\it m}$=$--23

B.

{\it m}$=$- -31

C.

{\it m}$=$ -31

D.

{\it m}$=$ -35

Zadanie 10. $(1pkt)$

Na rysunku jest przedstawiony wykres funkcji $y=f(x).$
\begin{center}
\begin{tabular}{|l|l|l|l|l|l|l|l|l|l|l|l|l|l|l|}
\hline
\multicolumn{1}{|l|}{}&	\multicolumn{1}{|l|}{}&	\multicolumn{1}{|l|}{}&	\multicolumn{1}{|l|}{$y$}&	\multicolumn{1}{|l|}{}&	\multicolumn{1}{|l|}{}&	\multicolumn{1}{|l|}{}&	\multicolumn{1}{|l|}{}&	\multicolumn{1}{|l|}{}&	\multicolumn{1}{|l|}{}&	\multicolumn{1}{|l|}{}&	\multicolumn{1}{|l|}{}&	\multicolumn{1}{|l|}{}&	\multicolumn{1}{|l|}{}&	\multicolumn{1}{|l|}{}	\\
\hline
\multicolumn{1}{|l|}{}&	\multicolumn{1}{|l|}{}&	\multicolumn{1}{|l|}{}&	\multicolumn{1}{|l|}{}&	\multicolumn{1}{|l|}{}&	\multicolumn{1}{|l|}{}&	\multicolumn{1}{|l|}{}&	\multicolumn{1}{|l|}{}&	\multicolumn{1}{|l|}{}&	\multicolumn{1}{|l|}{}&	\multicolumn{1}{|l|}{}&	\multicolumn{1}{|l|}{}&	\multicolumn{1}{|l|}{}&	\multicolumn{1}{|l|}{}&	\multicolumn{1}{|l|}{}	\\
\hline
\multicolumn{1}{|l|}{}&	\multicolumn{1}{|l|}{}&	\multicolumn{1}{|l|}{}&	\multicolumn{1}{|l|}{}&	\multicolumn{1}{|l|}{}&	\multicolumn{1}{|l|}{}&	\multicolumn{1}{|l|}{}&	\multicolumn{1}{|l|}{}&	\multicolumn{1}{|l|}{}&	\multicolumn{1}{|l|}{}&	\multicolumn{1}{|l|}{}&	\multicolumn{1}{|l|}{}&	\multicolumn{1}{|l|}{}&	\multicolumn{1}{|l|}{}&	\multicolumn{1}{|l|}{}	\\
\hline
\multicolumn{1}{|l|}{}&	\multicolumn{1}{|l|}{}&	\multicolumn{1}{|l|}{}&	\multicolumn{1}{|l|}{}&	\multicolumn{1}{|l|}{}&	\multicolumn{1}{|l|}{}&	\multicolumn{1}{|l|}{}&	\multicolumn{1}{|l|}{}&	\multicolumn{1}{|l|}{}&	\multicolumn{1}{|l|}{}&	\multicolumn{1}{|l|}{}&	\multicolumn{1}{|l|}{}&	\multicolumn{1}{|l|}{}&	\multicolumn{1}{|l|}{}&	\multicolumn{1}{|l|}{}	\\
\hline
\multicolumn{1}{|l|}{}&	\multicolumn{1}{|l|}{}&	\multicolumn{1}{|l|}{}&	\multicolumn{1}{|l|}{}&	\multicolumn{1}{|l|}{}&	\multicolumn{1}{|l|}{}&	\multicolumn{1}{|l|}{}&	\multicolumn{1}{|l|}{}&	\multicolumn{1}{|l|}{}&	\multicolumn{1}{|l|}{}&	\multicolumn{1}{|l|}{}&	\multicolumn{1}{|l|}{}&	\multicolumn{1}{|l|}{}&	\multicolumn{1}{|l|}{}&	\multicolumn{1}{|l|}{}	\\
\hline
\multicolumn{1}{|l|}{}&	\multicolumn{1}{|l|}{}&	\multicolumn{1}{|l|}{}&	\multicolumn{1}{|l|}{}&	\multicolumn{1}{|l|}{}&	\multicolumn{1}{|l|}{}&	\multicolumn{1}{|l|}{}&	\multicolumn{1}{|l|}{}&	\multicolumn{1}{|l|}{}&	\multicolumn{1}{|l|}{}&	\multicolumn{1}{|l|}{}&	\multicolumn{1}{|l|}{}&	\multicolumn{1}{|l|}{}&	\multicolumn{1}{|l|}{}&	\multicolumn{1}{|l|}{}	\\
\hline
\multicolumn{1}{|l|}{}&	\multicolumn{1}{|l|}{}&	\multicolumn{1}{|l|}{}&	\multicolumn{1}{|l|}{}&	\multicolumn{1}{|l|}{}&	\multicolumn{1}{|l|}{}&	\multicolumn{1}{|l|}{}&	\multicolumn{1}{|l|}{}&	\multicolumn{1}{|l|}{}&	\multicolumn{1}{|l|}{}&	\multicolumn{1}{|l|}{}&	\multicolumn{1}{|l|}{}&	\multicolumn{1}{|l|}{}&	\multicolumn{1}{|l|}{}&	\multicolumn{1}{|l|}{}	\\
\hline
\multicolumn{1}{|l|}{}&	\multicolumn{1}{|l|}{}&	\multicolumn{1}{|l|}{}&	\multicolumn{1}{|l|}{}&	\multicolumn{1}{|l|}{}&	\multicolumn{1}{|l|}{}&	\multicolumn{1}{|l|}{}&	\multicolumn{1}{|l|}{}&	\multicolumn{1}{|l|}{}&	\multicolumn{1}{|l|}{}&	\multicolumn{1}{|l|}{}&	\multicolumn{1}{|l|}{}&	\multicolumn{1}{|l|}{}&	\multicolumn{1}{|l|}{}&	\multicolumn{1}{|l|}{}	\\
\hline
\multicolumn{1}{|l|}{}&	\multicolumn{1}{|l|}{}&	\multicolumn{1}{|l|}{}&	\multicolumn{1}{|l|}{}&	\multicolumn{1}{|l|}{}&	\multicolumn{1}{|l|}{}&	\multicolumn{1}{|l|}{}&	\multicolumn{1}{|l|}{}&	\multicolumn{1}{|l|}{}&	\multicolumn{1}{|l|}{}&	\multicolumn{1}{|l|}{}&	\multicolumn{1}{|l|}{}&	\multicolumn{1}{|l|}{}&	\multicolumn{1}{|l|}{}&	\multicolumn{1}{|l|}{ $x$}	\\
\hline
\multicolumn{1}{|l|}{}&	\multicolumn{1}{|l|}{}&	\multicolumn{1}{|l|}{ $0$}&	\multicolumn{1}{|l|}{}&	\multicolumn{1}{|l|}{}&	\multicolumn{1}{|l|}{}&	\multicolumn{1}{|l|}{}&	\multicolumn{1}{|l|}{}&	\multicolumn{1}{|l|}{}&	\multicolumn{1}{|l|}{}&	\multicolumn{1}{|l|}{}&	\multicolumn{1}{|l|}{}&	\multicolumn{1}{|l|}{ $1$}&	\multicolumn{1}{|l|}{O 1}&	\multicolumn{1}{|l|}{$1$}	\\
\hline
\multicolumn{1}{|l|}{}&	\multicolumn{1}{|l|}{}&	\multicolumn{1}{|l|}{}&	\multicolumn{1}{|l|}{}&	\multicolumn{1}{|l|}{}&	\multicolumn{1}{|l|}{}&	\multicolumn{1}{|l|}{}&	\multicolumn{1}{|l|}{}&	\multicolumn{1}{|l|}{}&	\multicolumn{1}{|l|}{}&	\multicolumn{1}{|l|}{}&	\multicolumn{1}{|l|}{}&	\multicolumn{1}{|l|}{}&	\multicolumn{1}{|l|}{}&	\multicolumn{1}{|l|}{}	\\
\hline
\end{tabular}

\end{center}
Które równanie ma dokładnie trzy rozwiązania?

A. $f(x)=0$

B. $f(x)=1$

C. $f(x)=2$

D. $f(x)=3$

Zadanie ll. $(1pkt)$

$\mathrm{W}$ ciągu arytmetycznym $(a_{n})$ dane są: $a_{3}=13\mathrm{i}a_{5}=39$. Wtedy wyraz $a_{1}$ jest równy

A. 13

B. 0

C. $-13$

D. $-26$

Zadanie 12. $(1pkt)$

$\mathrm{W}$ ciągu geometrycznym $(a_{n})$ dane są: $a_{1}=3\mathrm{i}a_{4}=24$. Iloraz tego ciągujest równy

A. 8 B. 2 C. -81 D. --21





{\it Egzamin maturalny z matematyki}

{\it Poziom podstawowy}

{\it 5}

BRUDNOPIS





{\it 6}

{\it Egzamin maturalny z matematyki}

{\it Poziom podstawowy}

Zadanie 13. (1pkt)

Liczba przekątnych siedmiokąta foremnegojest równa

A. 7

B. 14

C. 21

D. 28

Zadanie 14. $(1pkt)$

Kąt $\alpha$ jest ostry i $\displaystyle \sin\alpha=\frac{3}{4}$. Wartość wyrazenia $ 2-\cos^{2}\alpha$ jest równa

A. --2165 B. -23 C. --1176 D.

$\displaystyle \frac{31}{16}$

Zadanie 15. (1pkt)

Okrąg opisany na kwadracie ma promień 4. Długość boku tego kwadratujest równa

A. $4\sqrt{2}$

B. $2\sqrt{2}$

C. 8

D. 4

Zadanie 16. (1pkt)

Podstawa trójkąta równoramiennego ma długość 6, a ramię ma długość 5.

opuszczona na podstawę ma długość

Wysokość

A. 3

B. 4

C. $\sqrt{34}$

D. $\sqrt{61}$

Zadanie 17. (1pkt)

Odcinki AB i DE są równoległe. Długości odcinków CD, DE i AB są odpowiednio równe

1, 3 i 9. Długość odcinka AD jest równa
\begin{center}
\includegraphics[width=90.624mm,height=47.652mm]{./F1_M_PP_M2010_page5_images/image001.eps}
\end{center}
{\it C}

1

{\it D E}

3

{\it A}  9  {\it B}

A. 2

B. 3

C. 5

D. 6

Zadanie 18. $(1pkt)$

Punkty $A, B, C$ lez$\cdot$ące na okręgu o środku $S$ są wierzchołkami trójkąta równobocznego. Miara

zaznaczonego na rysunku kąta środkowego $ASB$ jest równa
\begin{center}
\includegraphics[width=65.436mm,height=70.968mm]{./F1_M_PP_M2010_page5_images/image002.eps}
\end{center}
{\it C}

{\it S}

{\it A  B}

B. $90^{\mathrm{o}}$  C. $60^{\mathrm{o}}$

A. $120^{\mathrm{o}}$

D. $30^{\mathrm{o}}$





{\it Egzamin maturalny z matematyki}

{\it Poziom podstawowy}

7

BRUDNOPIS





{\it 8}

{\it Egzamin maturalny z matematyki}

{\it Poziom podstawowy}

Zadanie 19. (1pkt)

Latawiec ma wymiary podane na

zacieniowanego trójkątajest równa

rysunku. Powierzchnia

A. 3200 $\mathrm{c}\mathrm{m}^{2}$

B. 6400 $\mathrm{c}\mathrm{m}^{2}$
\begin{center}
\includegraphics[width=33.480mm,height=80.676mm]{./F1_M_PP_M2010_page7_images/image001.eps}
\end{center}
30

1600 $\mathrm{c}\mathrm{m}^{2}$

800 $\mathrm{c}\mathrm{m}^{2}$

C.

D.

Zadanie 20. $(1pkt)$

Współczynnik kierunkowy prostej równoległej do prostej o równaniu $y=-3x+5$ jest równy:

A.

- -31

B. $-3$

C.

-31

D. 3

Zadanie 21. (1pkt)

Wskaz równanie okręgu o promieniu 6.

A. $x^{2}+y^{2}=3$

B. $x^{2}+y^{2}=6$

C. $x^{2}+y^{2}=12$

D. $x^{2}+y^{2}=36$

Zadanie 22. $(1pkt)$

Punkty $A=(-5,2) \mathrm{i} B=(3,-2)$ są wierzchołkami trójkąta równobocznego $ABC$. Obwód

tego trójkątajest równy

A. 30

B. $4\sqrt{5}$

C. $12\sqrt{5}$

D. 36

Zadanie 23. $(1pkt)$

Pole powierzchni całkowitej prostopadłoŚcianu o wymiarach $5\times 3\times 4$ jest równe

A. 94

B. 60

C. 47

D. 20

Zadanie 24. (1pkt)

Ostrosłup ma 18 wierzchołków. Liczba wszystkich krawędzi tego ostrosłupajest równa

A. ll

B. 18

C. 27

D. 34

Zadanie 25. (1pkt)

Średnia arytmetyczna dziesięciu liczb x, 3, 1, 4, 1, 5, 1, 4, 1, 5jest równa 3. Wtedy

A. $x=2$

B. $x=3$

C. $x=4$

D. $x=5$





{\it Egzamin maturalny z matematyki}

{\it Poziom podstawowy}

{\it 9}

BRUDNOPIS





$ 1\theta$

{\it Egzamin maturalny z matematyki}

{\it Poziom podstawowy}

ZADANIA OTWARTE

{\it Rozwiqzania zadań o numerach od 26. do 34. nalezy zapisać w} $wyznacz\theta nych$ {\it miejscach}

{\it pod treściq zadania}.

Zadanie 26. $(2pkt)$

Rozwiąz nierówność $x^{2}-x-2\leq 0.$

Odpowied $\acute{\mathrm{z}}$:

Zadanie 27. $(2pkt)$

Rozwiąz równanie $x^{3}-7x^{2}-4x+28=0.$

Odpowied $\acute{\mathrm{z}}$:







Centralna Komisja Egzaminacyjna

Arkusz zawiera informacje prawnie chronione do momentu rozpoczęcia egzaminu.

WPISUJE ZDAJACY

KOD PESEL

{\it Miejsce}

{\it na naklejkę}

{\it z kodem}
\begin{center}
\includegraphics[width=21.432mm,height=9.804mm]{./F1_M_PP_M2011_page0_images/image001.eps}

\includegraphics[width=82.092mm,height=9.804mm]{./F1_M_PP_M2011_page0_images/image002.eps}

\includegraphics[width=204.060mm,height=216.048mm]{./F1_M_PP_M2011_page0_images/image003.eps}
\end{center}
EGZAMIN MATU

Z MATEMATY

LNY

POZIOM PODSTAWOWY  MAJ 2011

1. Sprawd $\acute{\mathrm{z}}$, czy arkusz egzaminacyjny zawiera 20 stron

(zadania $1-33$). Ewentualny brak zgłoś przewodniczącemu

zespo nadzorującego egzamin.

2. Rozwiązania zadań i odpowiedzi wpisuj w miejscu na to

przeznaczonym.

3. Odpowiedzi do zadań za iętych (l-23) przenieś

na ka ę odpowiedzi, zaznaczając je w części ka $\mathrm{y}$

przeznaczonej dla zdającego. Zamaluj $\blacksquare$ pola do tego

przeznaczone. Błędne zaznaczenie otocz kółkiem \fcircle$\bullet$

i zaznacz właściwe.

4. Pamiętaj, $\dot{\mathrm{z}}\mathrm{e}$ pominięcie argumentacji lub istotnych

obliczeń w rozwiązaniu zadania otwa ego (24-33) $\mathrm{m}\mathrm{o}\dot{\mathrm{z}}\mathrm{e}$

spowodować, $\dot{\mathrm{z}}\mathrm{e}$ za to rozwiązanie nie będziesz mógł

dostać pełnej liczby punktów.

5. Pisz czytelnie i $\mathrm{u}\dot{\mathrm{z}}$ aj tvlko $\mathrm{d}$ gopisu lub -Dióra

z czatnym tuszem lub atramentem.

6. Nie uzywaj korektora, a błędne zapisy wyrazínie prze eśl.

7. Pamiętaj, $\dot{\mathrm{z}}\mathrm{e}$ zapisy w brudnopisie nie będą oceniane.

8. $\mathrm{M}\mathrm{o}\dot{\mathrm{z}}$ esz korzystać z zestawu wzorów matematycznych,

cyrkla i linijki oraz kalkulatora.

9. Na karcie odpowiedzi wpisz swój numer PESEL i przyklej

naklejkę z kodem.

10. Nie wpisuj $\dot{\mathrm{z}}$ adnych znaków w części przeznaczonej dla

egzaminatora.

Czas pracy:

170 minut

Liczba punktów

do uzyskania: 50

$\Vert\Vert\Vert\Vert\Vert\Vert\Vert\Vert\Vert\Vert\Vert\Vert\Vert\Vert\Vert\Vert\Vert\Vert\Vert\Vert\Vert\Vert\Vert\Vert|  \mathrm{M}\mathrm{M}\mathrm{A}-\mathrm{P}1_{-}1\mathrm{P}-112$




{\it 2}

{\it Egzamin maturalny z matematyki}

{\it Poziom podstawowy}

ZADANIA ZAMKNIĘTE

{\it Wzadaniach} $\theta d1.$ {\it do 23. wybierz i zaznacz na karcie odpowiedzipoprawnq odpowied} $\acute{z}.$

Zadanie l. $(1pkt)$

Wska $\dot{\mathrm{z}}$ nierówność, którą spełnia liczba $\pi.$

A. $|x+1|>5$ B. $|x-1|<2$

C.

$|x+\displaystyle \frac{2}{3}|\leq 4$

D.

$|x-\displaystyle \frac{1}{3}|\geq 3$

Zadanie 2. (1pkt)

Pierwsza rata, która stanowi 9\% ceny roweru, jest równa 189zł. Rower kosztuje

A. 1701 zł.

B. 2100 zł.

C. 1890 zł.

D. 2091 zł.

Zadanie 3. $(1pkt)$

Wyrazenie $5a^{2}-10ab+15a$ jest równe iloczynowi

A. $5a^{2}(1-10b+3)$

B. $5a(a-2b+3)$

C. $5a(a-10b+15)$

D. $5(a-2b+3)$

Zadanie 4. (1pkt)

Układ równań 

A. $a=-1$

B. $a=0$

C. $a=2$

D. $a=3$

Zadanie 5. $(1pkt)$

Rozwiązanie równania $x(x+3)-49=x(x-4)$ nalezy do przedziału

A.

$(-\infty,3)$

B. $(10,+\infty)$

C. $(-5,-1)$

D. $(2,+\infty)$

Zadanie 6. $(1pkt)$

Najmniejszą liczbą całkowitą nalezącą do zbioru rozwiązań nierówności $\displaystyle \frac{3}{8}+\frac{x}{6}<\frac{5x}{12}$ jest

A. l

B. 2

C. $-1$

D. $-2$

Zadanie 7. $(1pkt)$

Wskaz, który zbiór przedstawiony na osi liczbowej jest zbiorem liczb spełniających

jednocześnie następujące nierówności: 3 $(x-1)(x-5)\leq 0 \mathrm{i} x>1.$
\begin{center}
\includegraphics[width=45.264mm,height=7.320mm]{./F1_M_PP_M2011_page1_images/image001.eps}
\end{center}
A.

B.

1

$\check{}$6

$\underline{x}$
\begin{center}
\includegraphics[width=36.168mm,height=7.320mm]{./F1_M_PP_M2011_page1_images/image002.eps}

\includegraphics[width=84.228mm,height=14.172mm]{./F1_M_PP_M2011_page1_images/image003.eps}
\end{center}
{\it x}

1 5

D.

$\underline{x}$

-$\check{}$5

C.

1





{\it Egzamin maturalny z matematyki}

{\it Poziom podstawowy}

{\it 11}

Zadanie 26. (2pkt)

Na rysunku przedstawiono wykres funkcjif.
\begin{center}
\includegraphics[width=154.836mm,height=87.372mm]{./F1_M_PP_M2011_page10_images/image001.eps}
\end{center}
Odczytaj z wykresu i zapisz:

a) zbiór wartości funkcjif,

b) przedział maksymalnej długości, w którym funkcja f jest malejąca.

Odpowied $\acute{\mathrm{z}}$:
\begin{center}
\includegraphics[width=109.980mm,height=17.784mm]{./F1_M_PP_M2011_page10_images/image002.eps}
\end{center}
Nr zadania

Wypelnia Maks. liczba kt

egzaminator

Uzyskana lÍczba pkt

24.

2

25.

2

2





{\it 12}

{\it Egzamin maturalny z matematyki}

{\it Poziom podstawowy}

Zadanie 27. $(2pkt)$

Liczby $x, y$, 19 w podanej kolejności tworzą ciąg arytmetyczny, przy czym $x+y=8.$

Oblicz $x\mathrm{i}y.$

Odpowied $\acute{\mathrm{z}}$:

Zadanie 28. $(2pkt)$

Kąt $\alpha$ jest ostry i $\displaystyle \frac{\sin\alpha}{\cos\alpha}+\frac{\cos\alpha}{\sin\alpha}=2$. Oblicz wartość wyrazenia $\sin\alpha\cdot\cos\alpha.$

Odpowiedzí:





{\it Egzamin maturalny z matematyki}

{\it Poziom podstawowy}

{\it 13}

Zadanie 29. $(2pkt)$

Dany jest czworokąt ABCD, w którym AB $\Vert$ CD. Na boku $BC$ wybrano taki punkt $E,$

$\dot{\mathrm{z}}\mathrm{e}|EC|=|CD|\mathrm{i}|EB|=|BA|$. Wykaz$\cdot, \dot{\mathrm{z}}\mathrm{e}$ kąt $AED$ jest prosty.

Odpowiedzí :
\begin{center}
\includegraphics[width=109.980mm,height=17.832mm]{./F1_M_PP_M2011_page12_images/image001.eps}
\end{center}
Nr zadania

Wypelnia Maks. liczba kt

egzaminator

Uzyskana liczba pkt

27.

2

28.

2

2





{\it 14}

{\it Egzamin maturalny z matematyki}

{\it Poziom podstawowy}

Zadanie 30. (2pkt)

Ze zbioru liczb \{1, 2, 3 7\} 1osujemy ko1ejno dwa razy po jednej 1iczbie ze zwracaniem.

Oblicz prawdopodobieństwo wylosowania liczb, których sumajest podzielna przez 3.

Odpowied $\acute{\mathrm{z}}$:





{\it Egzamin maturalny z matematyki}

{\it Poziom podstawowy}

{\it 15}

Zadanie 31. $(4pkt)$

Okrąg o środku w punkcie $S=(3,7)$ jest styczny do prostej o równaniu $y=2x-3$. Oblicz

współrzędne punktu styczności.

Odpowied $\acute{\mathrm{z}}$:
\begin{center}
\includegraphics[width=96.012mm,height=17.784mm]{./F1_M_PP_M2011_page14_images/image001.eps}
\end{center}
WypelnÍa

egzaminator

Nr zadania

Maks. liczba kt

30.

2

31.

4

Uzyskana liczba pkt





{\it 16}

{\it Egzamin maturalny z matematyki}

{\it Poziom podstawowy}

Zadanie 32. $(5pkt)$

Pewien turysta pokonał trasę 112 km, przechodząc $\mathrm{k}\mathrm{a}\dot{\mathrm{z}}$ dego dnia tę samą liczbę kilometrów.

Gdyby mógł przeznaczyć na tę wędrówkę o 3 dni więcej, to w ciągu $\mathrm{k}\mathrm{a}\dot{\mathrm{z}}$ dego dnia mógłby

przechodzić o 12 km mniej. Ob1icz, i1e ki1ometrów dziennie przechodził ten turysta.





{\it Egzamin maturalny z matematyki}

{\it Poziom podstawowy}

17

Odpowiedzí :
\begin{center}
\includegraphics[width=82.044mm,height=17.832mm]{./F1_M_PP_M2011_page16_images/image001.eps}
\end{center}
Wypelnia

egzaminator

Nr zadania

Maks. liczba kt

32.

5

Uzyskana liczba pkt





{\it 18}

{\it Egzamin maturalny z matematyki}

{\it Poziom podstawowy}

Zadanie 33. (4pkt)

Punkty K, L iM są środkami krawędzi BC, GHi AE szeŚcianu ABCDEFGH o krawędzi

długości l (zobacz rysunek). Oblicz pole trójkąta KLM.





{\it Egzamin maturalny z matematyki}

{\it Poziom podstawowy}

{\it 19}

Odpowiedzí :
\begin{center}
\includegraphics[width=82.044mm,height=17.832mm]{./F1_M_PP_M2011_page18_images/image001.eps}
\end{center}
Wypelnia

egzaminator

Nr zadania

Maks. liczba kt

33.

4

Uzyskana liczba pkt





$ 2\theta$

{\it Egzamin maturalny z matematyki}

{\it Poziom podstawowy}

BRUDNOPIS





{\it Egzamin maturalny z matematyki}

{\it Poziom podstawowy}

{\it 3}

BRUDNOPIS





$\blacksquare$

$\blacksquare$

$\Vert\Vert\Vert\Vert\Vert\Vert\Vert\Vert\Vert\Vert\Vert\Vert\Vert\Vert\Vert\Vert\Vert\Vert\Vert\Vert\Vert\Vert\Vert\Vert|$
\begin{center}
\includegraphics[width=79.452mm,height=15.804mm]{./F1_M_PP_M2011_page20_images/image001.eps}
\end{center}
PESEL

$\mathrm{M}\mathrm{M}\mathrm{A}-\mathrm{P}1_{-}1$ P-112

WYPELNIA ZDAJACY
\begin{center}
\begin{tabular}{|l|l|l|l|l|}
\cline{1-1}
\multicolumn{1}{|l|}{$\begin{array}{l}\mbox{Nr}	\\	\mbox{zad.}	\end{array}$}	\\
\cline{1-1}
\multicolumn{1}{|l|}{ $1$}&	\multicolumn{1}{|l|}{ $\fbox{$\mathrm{A}$}$}&	\multicolumn{1}{|l|}{ $\fbox{$\mathrm{B}$}$}&	\multicolumn{1}{|l|}{ $\underline{\mathrm{H}\mathrm{c}}-$}&	\multicolumn{1}{|l|}{ $\Gamma \mathrm{D}\lrcorner$}	\\
\hline
\multicolumn{1}{|l|}{ $2$}&	\multicolumn{1}{|l|}{ $\fbox{$\mathrm{A}$}$}&	\multicolumn{1}{|l|}{[‡L]}&	\multicolumn{1}{|l|}{$\fbox{$\mathrm{c}$}$}&	\multicolumn{1}{|l|}{ $\fbox{$\mathrm{D}$}$}	\\
\hline
\multicolumn{1}{|l|}{ $3$}&	\multicolumn{1}{|l|}{ $\fbox{$\mathrm{A}$}$}&	\multicolumn{1}{|l|}{ $\fbox{$\mathrm{B}$}$}&	\multicolumn{1}{|l|}{ $\underline{\mathrm{H}\mathrm{c}}-$}&	\multicolumn{1}{|l|}{ $\Gamma \mathrm{D}\lrcorner$}	\\
\hline
\multicolumn{1}{|l|}{ $4$}&	\multicolumn{1}{|l|}{ $\fbox{$\mathrm{A}$}$}&	\multicolumn{1}{|l|}{ $\fbox{$\mathrm{B}$}$}&	\multicolumn{1}{|l|}{ $\fbox{$\mathrm{c}$}$}&	\multicolumn{1}{|l|}{ $\fbox{$\mathrm{D}$}$}	\\
\hline
\multicolumn{1}{|l|}{ $5$}&	\multicolumn{1}{|l|}{ $\displaystyle \prod$}&	\multicolumn{1}{|l|}{ $\fbox{$\mathrm{B}$}$}&	\multicolumn{1}{|l|}{ $\fbox{$\mathrm{c}$}$}&	\multicolumn{1}{|l|}{ $\mathrm{g}$}	\\
\hline
\multicolumn{1}{|l|}{ $6$}&	\multicolumn{1}{|l|}{ $\fbox{$\mathrm{A}$}$}&	\multicolumn{1}{|l|}{ $\fbox{$\mathrm{B}$}$}&	\multicolumn{1}{|l|}{ $\fbox{$\mathrm{c}$}$}&	\multicolumn{1}{|l|}{ $\fbox{$\mathrm{D}$}$}	\\
\hline
\multicolumn{1}{|l|}{ $7$}&	\multicolumn{1}{|l|}{ $\fbox{$\mathrm{A}$}$}&	\multicolumn{1}{|l|}{ $\fbox{$\mathrm{B}$}$}&	\multicolumn{1}{|l|}{ $\fbox{$\mathrm{c}$}$}&	\multicolumn{1}{|l|}{ $\fbox{$\mathrm{D}$}$}	\\
\hline
\multicolumn{1}{|l|}{ $8$}&	\multicolumn{1}{|l|}{ $\cap$}&	\multicolumn{1}{|l|}{ $\fbox{$\mathrm{B}$}$}&	\multicolumn{1}{|l|}{ $\fbox{$\mathrm{c}$}$}&	\multicolumn{1}{|l|}{ $\fbox{$\mathrm{D}$}$}	\\
\hline
\multicolumn{1}{|l|}{ $9$}&	\multicolumn{1}{|l|}{ $\fbox{$\mathrm{A}$}$}&	\multicolumn{1}{|l|}{ $\fbox{$\mathrm{B}$}$}&	\multicolumn{1}{|l|}{ $\fbox{$\mathrm{c}$}$}&	\multicolumn{1}{|l|}{ $\fbox{$\mathrm{D}$}$}	\\
\hline
\multicolumn{1}{|l|}{ $10$}&	\multicolumn{1}{|l|}{ $\fbox{$\mathrm{A}$}$}&	\multicolumn{1}{|l|}{ $\fbox{$\mathrm{B}$}$}&	\multicolumn{1}{|l|}{ $\fbox{$\mathrm{c}$}$}&	\multicolumn{1}{|l|}{ $\fbox{$\mathrm{D}$}$}	\\
\hline
\multicolumn{1}{|l|}{ $11$}&	\multicolumn{1}{|l|}{ $\fbox{$\mathrm{A}$}$}&	\multicolumn{1}{|l|}{ $\fbox{$\mathrm{B}$}$}&	\multicolumn{1}{|l|}{ $\underline{\mathrm{H}\mathrm{c}}-$}&	\multicolumn{1}{|l|}{ $\fbox{$\mathrm{D}$}$}	\\
\hline
\multicolumn{1}{|l|}{ $12$}&	\multicolumn{1}{|l|}{ $\fbox{$\mathrm{A}$}$}&	\multicolumn{1}{|l|}{ $\fbox{$\mathrm{B}$}$}&	\multicolumn{1}{|l|}{ $\fbox{$\mathrm{c},$}$}&	\multicolumn{1}{|l|}{ $\fbox{$\mathrm{D}$}$}	\\
\hline
\multicolumn{1}{|l|}{ $13$}&	\multicolumn{1}{|l|}{ $\displaystyle \prod$}&	\multicolumn{1}{|l|}{ $\fbox{$\mathrm{B}$}$}&	\multicolumn{1}{|l|}{ $\fbox{$\mathrm{c}$}$}&	\multicolumn{1}{|l|}{ $\fbox{$\mathrm{D}$}$}	\\
\hline
\multicolumn{1}{|l|}{ $14$}&	\multicolumn{1}{|l|}{ $\fbox{$\mathrm{A}$}$}&	\multicolumn{1}{|l|}{[‡N]}&	\multicolumn{1}{|l|}{$\fbox{$\zeta \mathrm{i},$}$}&	\multicolumn{1}{|l|}{ $\fbox{$\mathrm{D}$}$}	\\
\hline
\multicolumn{1}{|l|}{ $15$}&	\multicolumn{1}{|l|}{ $\fbox{$\mathrm{A}$}$}&	\multicolumn{1}{|l|}{ $\fbox{$\mathrm{B}$}$}&	\multicolumn{1}{|l|}{ $\mathrm{H}$-c-}&	\multicolumn{1}{|l|}{$\fbox{$\mathrm{D}$}$}	\\
\hline
\multicolumn{1}{|l|}{ $16$}&	\multicolumn{1}{|l|}{ $\fbox{$\mathrm{A}$}$}&	\multicolumn{1}{|l|}{[‡N]}&	\multicolumn{1}{|l|}{$\fbox{$\mathrm{c}$}$}&	\multicolumn{1}{|l|}{ $\fbox{$\mathrm{D}$}$}	\\
\hline
\multicolumn{1}{|l|}{ $17$}&	\multicolumn{1}{|l|}{ $\fbox{$\mathrm{A}$}$}&	\multicolumn{1}{|l|}{ $\fbox{$\mathrm{B}$}$}&	\multicolumn{1}{|l|}{ $\underline{\mathrm{H}\mathrm{c}}-$}&	\multicolumn{1}{|l|}{ $\fbox{$\ulcorner)$}$}	\\
\hline
\multicolumn{1}{|l|}{ $18$}&	\multicolumn{1}{|l|}{ $\fbox{$\mathrm{A}$}$}&	\multicolumn{1}{|l|}{[‡N]}&	\multicolumn{1}{|l|}{$\fbox{$\mathrm{c}$}$}&	\multicolumn{1}{|l|}{ $\fbox{$\mathrm{D}$}$}	\\
\hline
\multicolumn{1}{|l|}{ $19$}&	\multicolumn{1}{|l|}{ $\fbox{$\mathrm{A}$}$}&	\multicolumn{1}{|l|}{ $\fbox{$\mathrm{B}$}$}&	\multicolumn{1}{|l|}{ $\fbox{$\mathrm{c}$}$}&	\multicolumn{1}{|l|}{ $\fbox{$\ulcorner)$}$}	\\
\hline
\multicolumn{1}{|l|}{ $20$}&	\multicolumn{1}{|l|}{ $\fbox{$\mathrm{A}$}$}&	\multicolumn{1}{|l|}{ $\fbox{$\mathrm{B}$}$}&	\multicolumn{1}{|l|}{ $\fbox{$\mathrm{c}$}$}&	\multicolumn{1}{|l|}{ $\fbox{$\mathrm{D}$}$}	\\
\hline
\multicolumn{1}{|l|}{ $21$}&	\multicolumn{1}{|l|}{ $\displaystyle \prod$}&	\multicolumn{1}{|l|}{ $\fbox{$\mathrm{B}$}$}&	\multicolumn{1}{|l|}{ $\fbox{$\mathrm{c}$}$}&	\multicolumn{1}{|l|}{ $\ulcorner \mathrm{D}\rfloor$}	\\
\hline
\multicolumn{1}{|l|}{ $22$}&	\multicolumn{1}{|l|}{ $\fbox{$\mathrm{A}$}$}&	\multicolumn{1}{|l|}{ $\fbox{$\mathrm{B}$}$}&	\multicolumn{1}{|l|}{ $\fbox{$\mathrm{c}$}$}&	\multicolumn{1}{|l|}{ $\fbox{$\mathrm{D}$}$}	\\
\hline
\multicolumn{1}{|l|}{ $23$}&	\multicolumn{1}{|l|}{ $\fbox{$\mathrm{A}$}$}&	\multicolumn{1}{|l|}{ $\fbox{$\mathrm{B}$}$}&	\multicolumn{1}{|l|}{ $\fbox{$\mathrm{c}$}$}&	\multicolumn{1}{|l|}{ $\fbox{$ 1\supset$}$}	\\
\hline
\end{tabular}

\end{center}
Miejsce na naKlej$\kappa$e

z rr PESE-

WYPELNIA EGZAMINATOR
\begin{center}
\begin{tabular}{|l|l|l|l|l|l|l|}
	\\
&	\multicolumn{1}{|l|}{$0$}&	\multicolumn{1}{|l|}{ $1$}&	\multicolumn{1}{|l|}{ $2$}&	\multicolumn{1}{|l|}{ $3$}&	\multicolumn{1}{|l|}{ $4$}&	\multicolumn{1}{|l|}{ $5$}	\\
\cline{2-7}
\multicolumn{1}{|l|}{ $24$}&	\multicolumn{1}{|l|}{ $\square $}&	\multicolumn{1}{|l|}{ $\square $}&	\multicolumn{1}{|l|}{ $\square $}&	\multicolumn{1}{|l|}{}&	\multicolumn{1}{|l|}{}&	\multicolumn{1}{|l|}{}	\\
\hline
\multicolumn{1}{|l|}{ $25$}&	\multicolumn{1}{|l|}{ $\square $}&	\multicolumn{1}{|l|}{ $\square $}&	\multicolumn{1}{|l|}{ $\square $}&	\multicolumn{1}{|l|}{}&	\multicolumn{1}{|l|}{}&	\multicolumn{1}{|l|}{}	\\
\hline
\multicolumn{1}{|l|}{ $26$}&	\multicolumn{1}{|l|}{ $\square $}&	\multicolumn{1}{|l|}{ $\square $}&	\multicolumn{1}{|l|}{ $\square $}&	\multicolumn{1}{|l|}{}&	\multicolumn{1}{|l|}{}&	\multicolumn{1}{|l|}{}	\\
\hline
\multicolumn{1}{|l|}{ $27$}&	\multicolumn{1}{|l|}{ $\square $}&	\multicolumn{1}{|l|}{ $\square $}&	\multicolumn{1}{|l|}{ $\square $}&	\multicolumn{1}{|l|}{}&	\multicolumn{1}{|l|}{}&	\multicolumn{1}{|l|}{}	\\
\hline
\multicolumn{1}{|l|}{ $28$}&	\multicolumn{1}{|l|}{ $\square $}&	\multicolumn{1}{|l|}{ $\square $}&	\multicolumn{1}{|l|}{ $\square $}&	\multicolumn{1}{|l|}{}&	\multicolumn{1}{|l|}{}&	\multicolumn{1}{|l|}{}	\\
\hline
\multicolumn{1}{|l|}{ $29$}&	\multicolumn{1}{|l|}{ $\square $}&	\multicolumn{1}{|l|}{ $\square $}&	\multicolumn{1}{|l|}{ $\square $}&	\multicolumn{1}{|l|}{}&	\multicolumn{1}{|l|}{}&	\multicolumn{1}{|l|}{}	\\
\hline
\multicolumn{1}{|l|}{ $30$}&	\multicolumn{1}{|l|}{ $\square $}&	\multicolumn{1}{|l|}{ $\square $}&	\multicolumn{1}{|l|}{ $\square $}&	\multicolumn{1}{|l|}{}&	\multicolumn{1}{|l|}{}&	\multicolumn{1}{|l|}{}	\\
\hline
\multicolumn{1}{|l|}{ $31$}&	\multicolumn{1}{|l|}{ $\square $}&	\multicolumn{1}{|l|}{ $\square $}&	\multicolumn{1}{|l|}{ $\square $}&	\multicolumn{1}{|l|}{ $\square $}&	\multicolumn{1}{|l|}{ $\square $}&	\multicolumn{1}{|l|}{}	\\
\hline
\multicolumn{1}{|l|}{ $32$}&	\multicolumn{1}{|l|}{ $\square $}&	\multicolumn{1}{|l|}{ $\square $}&	\multicolumn{1}{|l|}{ $\square $}&	\multicolumn{1}{|l|}{ $\square $}&	\multicolumn{1}{|l|}{ $\square $}&	\multicolumn{1}{|l|}{ $\square $}	\\
\hline
\multicolumn{1}{|l|}{ $33$}&	\multicolumn{1}{|l|}{ $\square $}&	\multicolumn{1}{|l|}{ $\square $}&	\multicolumn{1}{|l|}{ $\square $}&	\multicolumn{1}{|l|}{ $\square $}&	\multicolumn{1}{|l|}{ $\square $}&	\multicolumn{1}{|l|}{}	\\
\hline
\end{tabular}


\includegraphics[width=14.580mm,height=9.852mm]{./F1_M_PP_M2011_page20_images/image002.eps}
\end{center}
$\blacksquare$

$\blacksquare$

SUMA

PUNKTÓW

D \square  \square  \square  \square  \square  \square  \square  \square  \square  \square 

J

0 1 2 3 4 5 6 7 8 9

0 1 2 3 4 5 6 7 8 9

$\blacksquare$




\begin{center}
\includegraphics[width=73.152mm,height=11.028mm]{./F1_M_PP_M2011_page21_images/image001.eps}
\end{center}
KOD EGZAMINATORA

Czytelny podpis egzaminatora
\begin{center}
\includegraphics[width=21.840mm,height=9.852mm]{./F1_M_PP_M2011_page21_images/image002.eps}
\end{center}
KOD ZDAJACEGO





{\it 4}

{\it Egzamin maturalny z matematyki}

{\it Poziom podstawowy}

Zadanie 8. $(1pkt)$

Wyrazenie $\log_{4}(2x-1)$ jest określone dla wszystkich liczb $x$ spełniających warunek

A.

$x\displaystyle \leq\frac{1}{2}$

B.

$x>\displaystyle \frac{1}{2}$

C. $x\leq 0$

D. $x>0$

Zadanie 9. $(1pkt)$

Dane są funkcje liniowe $f(x)=x-2$ oraz $g(x)=x+4$ określone dla wszystkich liczb

rzeczywistych $x$. Wskaz, który z ponizszych wykresów jest wykresem funkcji

$h(x)=f(x)\cdot g(x).$
\begin{center}
\includegraphics[width=29.868mm,height=49.380mm]{./F1_M_PP_M2011_page3_images/image001.eps}
\end{center}
{\it y}

{\it x}

$-4$  2
\begin{center}
\includegraphics[width=30.024mm,height=49.380mm]{./F1_M_PP_M2011_page3_images/image002.eps}
\end{center}
{\it y}

$-2$

{\it x}

4
\begin{center}
\includegraphics[width=29.868mm,height=49.380mm]{./F1_M_PP_M2011_page3_images/image003.eps}
\end{center}
{\it y}

{\it x}

$-4$  2
\begin{center}
\includegraphics[width=29.868mm,height=49.380mm]{./F1_M_PP_M2011_page3_images/image004.eps}
\end{center}
{\it y}

$-2$

{\it X}

4

A.

B.

C.

D.

Zadanie 10 $(1pkt)$

Funkcja liniowa określona jest wzorem $f(x)=-\sqrt{2}x+4$. Miejscem zerowym tej funkcjijest

liczba

A. $-2\sqrt{2}$

B.

-$\sqrt{}$22

C.

- -$\sqrt{}$22

D. $2\sqrt{2}$

Zadanie ll. $(1pkt)$

Danyjest nieskończony ciąg geometryczny $(a_{n})$, w którym $a_{3}=1 \displaystyle \mathrm{i}a_{4}=\frac{2}{3}$. Wtedy

A. {\it a}1$=- 23$ B. {\it a}1$=- 49$ C. {\it a}1$=$-23 D. {\it a}1$=$-49

Zadanie 12. $(1pkt)$

Danyjest nieskończony rosnący ciąg arytmetyczny $(a_{n})$ o wyrazach dodatnich. Wtedy

A. $a_{4}+a_{7}=a_{10}$

B. $a_{4}+a_{6}=a_{3}+a_{8}$

C. $a_{2}+a_{9}=a_{3}+a_{8}$

D. $a_{5}+a_{7}=2a_{8}$

Zadanie 13. $(1pkt)$

Kąt $\alpha$ jest ostry i $\displaystyle \cos\alpha=\frac{5}{13}$. Wtedy

A. $\displaystyle \sin\alpha=\frac{12}{13}$ oraz $\displaystyle \mathrm{t}\mathrm{g}\alpha=\frac{12}{5}$

C. $\displaystyle \sin\alpha=\frac{12}{5}$ oraz $\displaystyle \mathrm{t}\mathrm{g}\alpha=\frac{12}{13}$

B. $\displaystyle \sin\alpha=\frac{12}{13}$ oraz $\displaystyle \mathrm{t}\mathrm{g}\alpha=\frac{5}{12}$

D. $\displaystyle \sin\alpha=\frac{5}{12}$ oraz $\displaystyle \mathrm{t}\mathrm{g}\alpha=\frac{12}{13}$





{\it Egzamin maturalny z matematyki}

{\it Poziom podstawowy}

{\it 5}

BRUDNOPIS





{\it 6}

{\it Egzamin maturalny z matematyki}

{\it Poziom podstawowy}

Zadanie 14. $(1pkt)$

Wartość wyrazenia $\displaystyle \frac{\sin^{2}38^{\mathrm{o}}+\cos^{2}38^{\mathrm{o}}-1}{\sin^{2}52^{\mathrm{o}}+\cos^{2}52^{\mathrm{o}}+1}$ jest równa

A.

-21

B. 0

C.

- -21

D. l

Zadanie 15. $(1pkt)$

$\mathrm{W}$ prostopadłoŚcianie ABCDEFGH mamy: $|AB|=5, |AD|=4, |AE|=3$. Który z odcinków

{\it AB}, $BG, GE, EB$ jest najdłuzszy?

A.

{\it AB}

B.

{\it BG}

C.

{\it GE}

{\it D. EB}

Zadanie 16. $(1pkt)$

Punkt $O$ jest środkiem okręgu. Kąt wpisany $\alpha$ ma miarę
\begin{center}
\includegraphics[width=66.348mm,height=60.912mm]{./F1_M_PP_M2011_page5_images/image001.eps}
\end{center}
{\it B}

$\alpha$

{\it A}

$160^{\mathrm{o}}$  {\it C}

{\it O}

A. $80^{\mathrm{o}}$

B. $100^{\mathrm{o}}$

C. $110^{\mathrm{o}}$

D. $120^{\mathrm{o}}$

Zadanie 17. $(1pkt)$

Wysokość rombu o boku długości 6 i kącie ostrym $60^{\mathrm{o}}$ jest równa

A. $3\sqrt{3}$

B. 3

C. $6\sqrt{3}$

D. 6

Zadanie 18. $(1pkt)$

Prosta $k$ ma równanie $y=2x-3$. Wskaz równanie prostej $l$ równoległej do prostej $k$

i przechodzącej przez punkt $D$ o współrzędnych $(-2,1).$

A. $y=-2x+3$

B. $y=2x+1$

C. $y=2x+5$

D. $y=-x+1$





{\it Egzamin maturalny z matematyki}

{\it Poziom podstawowy}

7

BRUDNOPIS





{\it 8}

{\it Egzamin maturalny z matematyki}

{\it Poziom podstawowy}

Zadanie 19. $(1pkt)$

Styczną do okręgu $(x-1)^{2}+y^{2}-4=0$ jest prosta o równaniu

A. $x=1$

B. $x=3$

C. $y=0$

D. $y=4$

Zadanie 20. (1pkt)

Pole powierzchni całkowitej sześcianu jest równe 54. Długość przekątnej tego sześcianu jest

równa

A. $\sqrt{6}$

B. 3

C. 9

D. $3\sqrt{3}$

Zadanie 21. (1pkt)

Objętość stozka o wysokości 8 i średnicy podstawy 12jest równa

A. $ 124\pi$

B. $ 96\pi$

C. $ 64\pi$

D. $ 32\pi$

Zadanie 22. (1pkt)

Rzucamy dwa razy symetryczną sześcienną kostką do gry. Prawdopodobieństwo otrzymania

sumy oczek równej trzy wynosi

A.

-61

B.

-91

C.

$\displaystyle \frac{1}{12}$

D.

$\displaystyle \frac{1}{18}$

Zadanie 23. (1pkt)

Uczniowie pewnej klasy zostali poproszeni o odpowiedzí na pytanie:,,Ile osób liczy twoja

rodzina?'' Wyniki przedstawiono w tabeli:
\begin{center}
\begin{tabular}{|l|l|}
\hline
\multicolumn{1}{|l|}{$\begin{array}{l}\mbox{Liczba osób}	\\	\mbox{w rodzinie}	\end{array}$}&	\multicolumn{1}{|l|}{$\begin{array}{l}\mbox{liczba}	\\	\mbox{uczniów}	\end{array}$}	\\
\hline
\multicolumn{1}{|l|}{ $3$}&	\multicolumn{1}{|l|}{ $6$}	\\
\hline
\multicolumn{1}{|l|}{ $4$}&	\multicolumn{1}{|l|}{ $12$}	\\
\hline
\multicolumn{1}{|l|}{ $x$}&	\multicolumn{1}{|l|}{ $2$}	\\
\hline
\end{tabular}

\end{center}
Średnia liczba osób w rodzinie dla uczniów tej klasyjest równa 4. Wtedy 1iczba x jest równa

A. 3

B. 4

C. 5

D. 7





{\it Egzamin maturalny z matematyki}

{\it Poziom podstawowy}

{\it 9}

BRUDNOPIS





$ 1\theta$

{\it Egzamin maturalny z matematyki}

{\it Poziom podstawowy}

ZADANIA OTWARTE

{\it Rozwiqzania zadań o numerach od 24. do 33. nalezy zapisać w} $wyznacz\theta nych$ {\it miejscach}

{\it pod treściq zadania}.

Zadanie 24. $(2pkt)$

Rozwiąz nierówność $3x^{2}-10x+3\leq 0.$

Odpowiedzí:

Zadanie 25. $(2pkt)$

Uzasadnij, $\dot{\mathrm{z}}\mathrm{e}\mathrm{j}\mathrm{e}\dot{\mathrm{z}}$ eli $a+b=1$

$\mathrm{i} a^{2}+b^{2}=7$, to $a^{4}+b^{4}=31.$







$1-$

$-1\cup 1$

$-\mapsto 1$

$\mathrm{r}--$

Centralna Komisja Egzaminacyjna

Arkusz zawiera informacje prawnie chronione do momentu rozpoczęcia egzaminu.

WPISUJE ZDAJACY

KOD PESEL

{\it Miejsce}

{\it na naklejkę}

{\it z kodem}
\begin{center}
\includegraphics[width=21.432mm,height=9.804mm]{./F1_M_PP_M2012_page0_images/image001.eps}

\includegraphics[width=82.092mm,height=9.804mm]{./F1_M_PP_M2012_page0_images/image002.eps}
\end{center}
\fbox{} dysleksja
\begin{center}
\includegraphics[width=204.060mm,height=216.048mm]{./F1_M_PP_M2012_page0_images/image003.eps}
\end{center}
EGZAMIN MATU LNY

Z MATEMATYKI

MAJ 2012

POZIOM PODSTAWOWY

1. Sprawd $\acute{\mathrm{z}}$, czy arkusz egzaminacyjny zawiera 18 stron

(zadania $1-34$). Ewentualny brak zgłoś przewodniczącemu

zespo nadzorującego egzamin.

2. Rozwiązania zadań i odpowiedzi wpisuj w miejscu na to

przeznaczonym.

3. Odpowiedzi do zadań za niętych (l-25) przenieś

na ka ę odpowiedzi, zaznaczając je w części ka $\mathrm{y}$

przeznaczonej dla zdającego. Zamaluj $\blacksquare$ pola do tego

przeznaczone. Błędne zaznaczenie otocz kółkiem \fcircle$\bullet$

i zaznacz właściwe.

4. Pamiętaj, $\dot{\mathrm{z}}\mathrm{e}$ pominięcie argumentacji lub istotnych

obliczeń w rozwiązaniu zadania otwa ego (26-34) $\mathrm{m}\mathrm{o}\dot{\mathrm{z}}\mathrm{e}$

spowodować, $\dot{\mathrm{z}}\mathrm{e}$ za to rozwiązanie nie będziesz mógł

dostać pełnej liczby punktów.

5. Pisz czytelnie i uzywaj tvlko długopisu lub -Dióra

z czarnym tuszem lub atramentem.

6. Nie uzywaj korektora, a błędne zapisy wyrazínie prze eśl.

7. Pamiętaj, $\dot{\mathrm{z}}\mathrm{e}$ zapisy w brudnopisie nie będą oceniane.

8. $\mathrm{M}\mathrm{o}\dot{\mathrm{z}}$ esz korzystać z zestawu wzorów matematycznych,

cyrkla i linijki oraz kalkulatora.

9. Na tej stronie oraz na karcie odpowiedzi wpisz swój

numer PESEL i przyklej naklejkę z kodem.

10. Nie wpisuj $\dot{\mathrm{z}}$ adnych znaków w części przeznaczonej

dla egzaminatora.

Czas pracy:

170 minut

Liczba punktów

do uzyskania: 50

$\Vert\Vert\Vert\Vert\Vert\Vert\Vert\Vert\Vert\Vert\Vert\Vert\Vert\Vert\Vert\Vert\Vert\Vert\Vert\Vert\Vert\Vert\Vert\Vert|  \mathrm{M}\mathrm{M}\mathrm{A}-\mathrm{P}1_{-}1\mathrm{P}-122$




{\it 2}

{\it Egzamin maturalny z matematyki}

{\it Poziom podstawowy}

ZADANIA ZAMKNIĘTE

{\it Wzadaniach} $\theta d1.$ {\it do 25. wybierz i zaznacz na karcie odpowiedzipoprawnq odpowied} $\acute{z}.$

Zadanie l. (lpkt)

Cenę nart obnizono o 20\%, a po miesiącu nową cenę obnizono o da1sze 30\%. W wyniku obu

obnizek cena nart zmniejszyła się o

A. 44\%

B. 50\%

C. 56\%

D. 60\%

Zadanie 2. $(1pkt)$

3

Liczba $\sqrt[3]{(-8)^{-1}}\cdot 16^{\overline{4}}$ jest równa

A. $-8$

B. $-4$

C. 2

D. 4

Zadanie 3. $(1pkt)$

Liczba $(3-\sqrt{2})^{2}+4(2-\sqrt{2})$ jest równa

A. $19-10\sqrt{2}$

B. $17-4\sqrt{2}$

C. $15+14\sqrt{2}$

D. $19+6\sqrt{2}$

Zadanie 4. $(1pkt)$

Iloczyn 2$\cdot\log_{1}9$ jest równy

-3

A. $-6$ B. $-4$

C. $-1$

D. l

Zadanie 5. $(1pkt)$

Wska $\dot{\mathrm{z}}$ liczbę, która spełnia równanie $|3x+1|=4x.$

A. $x=-1$

B. $x=1$

C. $x=2$

D. $x=-2$

Zadanie 6. $(1pkt)$

Liczby $x_{1}, x_{2}$ sąróz$\cdot$nymi rozwiązaniami równania $2x^{2}+3x-7=0$. Suma $x_{1}+x_{2}$ jest równa

A.

- -27

B.

- -47

C.

- -23

D.

- -43

Zadanie 7. $(1pkt)$

Miejscami zerowymi ffinkcji kwadratowej $y=-3(x-7\mathrm{X}x+2)$ są

A. $x=7, x=-2$

B. $x=-7, x=-2$

C. $x=7, x=2$

D. $x=-7, x=2$

Zadanie 8. $(1pkt)$

Funkcja liniowafjest określona wzorem $f(x)=ax+6$, gdzie $a>0$. Wówczas spełniony jest

warunek

A. $f(1)>1$

B. $f(2)=2$

C. $f(3)<3$

D. $f(4)=4$





{\it Egzamin maturalny z matematyki}

{\it Poziom podstawowy}

{\it 11}

Zadanie 28. $(2pkt)$

Liczby $x_{1}=-4 \mathrm{i} x_{2}=3$ są pierwiastkami

trzeci pierwiastek tego wielomianu.

Odpowiedzí :

wielomianu $W(x)=x^{3}+4x^{2}-9x-36$. Oblicz

Zadanie 29. $(2pkt)$

Wyznacz równanie symetralnej odcinka o końcach $A=(-2,2)\mathrm{i}B=(2,10).$

Odpowiedzí :
\begin{center}
\includegraphics[width=123.900mm,height=17.832mm]{./F1_M_PP_M2012_page10_images/image001.eps}
\end{center}
Nr zadania

Wypelnia Maks. liczba kt

egzaminator

Uzyskana liczba pkt

2

27.

2

28.

2

2





{\it 12}

{\it Egzamin maturalny z matematyki}

{\it Poziom podstawowy}

Zadanie 30. $(2pkt)$

$\mathrm{W}$ trójkącie $ABC$ poprowadzono dwusieczne kątów A $\mathrm{i}B$. Dwusieczne te przecinają się

w punkcie $P$. Uzasadnij, $\dot{\mathrm{z}}\mathrm{e}$ kąt $APB$ jest rozwarty.





{\it Egzamin maturalny z matematyki}

{\it Poziom podstawowy}

{\it 13}

Zadanie 31. (2pkt)

Ze zbioru liczb \{1,2,3,4,5,6,7\} 1osujemy dwa razy po jednej 1iczbie ze zwracaniem. Ob1icz

prawdopodobieństwo zdarzenia A, polegającego na wylosowaniu liczb, których iloczyn jest

podzielny przez 6.

Odpowied $\acute{\mathrm{z}}$:
\begin{center}
\includegraphics[width=95.964mm,height=17.784mm]{./F1_M_PP_M2012_page12_images/image001.eps}
\end{center}
Wypelnia

egzaminator

Nr zadania

Maks. liczba kt

30.

2

31.

2

Uzyskana liczba pkt





{\it 14}

{\it Egzamin maturalny z matematyki}

{\it Poziom podstawowy}

Zadanie 32. (4pkt)

Ciąg (9, x,19) jest arytmetyczny, a ciąg (x,42,y,z) jest geometryczny. Ob1icz x, y oraz z.

Odpowiedzí:





{\it Egzamin maturalny z matematyki}

{\it Poziom podstawowy}

{\it 15}

Zadanie 33. $(4pkt)$

$\mathrm{W}$ graniastosłupie prawidłowym czworokątnym ABCDEFGH przekątna $AC$ podstawy

ma długość 4. Kąt ACE jest równy $60^{\mathrm{o}}$. Oblicz objętość ostrosłupa ABCDE przedstawionego

na ponizszym rysunku.

Odpowiedzí :
\begin{center}
\includegraphics[width=95.964mm,height=17.784mm]{./F1_M_PP_M2012_page14_images/image001.eps}
\end{center}
Wypelnia

egzaminator

Nr zadania

Maks. liczba kt

32.

4

33.

4

Uzyskana liczba pkt





{\it 16}

{\it Egzamin maturalny z matematyki}

{\it Poziom podstawowy}

Zadanie 34. $(5pkt)$

Miasto $A$ i miasto $B$ łączy linia kolejowa długości 210 km. Średnia prędkość pociągu

pospiesznego na tej trasie jest o 24 $\mathrm{k}\mathrm{m}/\mathrm{h}$ większa od średniej prędkości pociągu osobowego.

Pociąg pospieszny pokonuje tę trasę o l godzinę krócej $\mathrm{n}\mathrm{i}\dot{\mathrm{z}}$ pociąg osobowy. Oblicz czas

pokonania tej drogi przez pociąg pospieszny.





{\it Egzamin maturalny z matematyki}

{\it Poziom podstawowy}

{\it 1}7

Odpowied $\acute{\mathrm{z}}$:
\begin{center}
\includegraphics[width=82.044mm,height=17.832mm]{./F1_M_PP_M2012_page16_images/image001.eps}
\end{center}
Wypelnia

egzaminator

Nr zadania

Maks. liczba kt

34.

5

Uzyskana liczba pkt





{\it 18}

{\it Egzamin maturalny z matematyki}

{\it Poziom podstawowy}

BRUDNOPIS





{\it Egzamin maturalny z matematyki}

{\it Poziom podstawowy}

{\it 3}

BRUDNOPIS





{\it 4}

{\it Egzamin maturalny z matematyki}

{\it Poziom podstawowy}

Zadanie 9. $(1pkt)$

Wskaz wykres funkcji, która w przedziale $\langle-4,4\rangle$ ma dokładniejedno miejsce zerowe.

A.
\begin{center}
\includegraphics[width=57.408mm,height=58.368mm]{./F1_M_PP_M2012_page3_images/image001.eps}
\end{center}
4

3

y

2

1

$-4$ -$3  -2$

$-1$

$-1$

1 2

$\mathrm{x}$

$3\backslash ^{4}$

$-2$

$-3$

$-4$

C.
\begin{center}
\includegraphics[width=58.668mm,height=58.980mm]{./F1_M_PP_M2012_page3_images/image002.eps}
\end{center}
y

3

2

1

x

$-4$ -$3  -2  -1$  1 2 3 4

$-2$

$-3$

Zadanie 10. $(1pkt)$

Liczba tg $30^{\mathrm{o}}-\sin 30^{\mathrm{o}}$ jest równa

A. $\sqrt{3}-1$

B.

- -$\sqrt{}$63

B.
\begin{center}
\includegraphics[width=57.144mm,height=58.164mm]{./F1_M_PP_M2012_page3_images/image003.eps}
\end{center}
4  y

2

1

$-4$ -$3  -2$

$-1$

$-1$

1 2  3 4

$-2$

$-3$

$-4$

D.
\begin{center}
\includegraphics[width=56.088mm,height=57.300mm]{./F1_M_PP_M2012_page3_images/image004.eps}
\end{center}
4  y

2

1

$-4  -2$

$-1$

$-1$

1 3  4

$-2$

$-3$

$-4$

C.

$\displaystyle \frac{\sqrt{3}-1}{6}$

D.

$\displaystyle \frac{2\sqrt{3}-3}{6}$

Zadanie ll. $(1pkt)$

$\mathrm{W}$ trójkącie prostokątnym $ABC$ odcinek $AB$ jest przeciwprostokątną

$|BC|=12$. Wówczas sinus kąta ABCjest równy

i

$|AB|=13$

oraz

A.

-1123

B.

$\displaystyle \frac{5}{13}$

C.

$\displaystyle \frac{5}{12}$

D.

$\displaystyle \frac{13}{12}$

Zadanie 12. (1pkt)

W trójkącie równoramiennym ABC dane

Podstawa AB tego trójkąta ma długość

są

$|AC|=|BC|=5$

oraz wysokość

$|CD|=2.$

A. 6

B. $\mathrm{z}\sqrt{21}$

C. $\mathrm{z}\sqrt{29}$

D. 14





{\it Egzamin maturalny z matematyki}

{\it Poziom podstawowy}

{\it 5}

BRUDNOPIS





{\it 6}

{\it Egzamin maturalny z matematyki}

{\it Poziom podstawowy}

Zadanie 13. $(1pkt)$

$\mathrm{W}$ trójkącie prostokątnym dwa dłuzsze boki mają długości 5 $\mathrm{i}7$. Obwód tego trójkąta jest

równy

A. $16\sqrt{6}$ B. $14\sqrt{6}$ C. $12+4\sqrt{6}$ D. $12+2\sqrt{6}$

Zadanie 14. $(1pkt)$

Odcinki AB $\mathrm{i}$ CD są równoległe i $|AB|=5, |AC|=2, |CD|=7$ (zobacz rysunek). Długość

odcinka $AE$ jest równa

A.

$\displaystyle \frac{10}{7}$

B.

$\displaystyle \frac{14}{5}$
\begin{center}
\includegraphics[width=68.628mm,height=61.116mm]{./F1_M_PP_M2012_page5_images/image001.eps}
\end{center}
{\it D}

{\it B}

7

5

{\it E  A} 2  {\it C}

5

C. 3

D. 5

Zadanie 15. (1pkt)

Pole kwadratu wpisanego w okrąg o promieniu 5jest równe

A. 25

B. 50

C. 75

D. 100

Zadanie 16. $(1pkt)$

Punkty $A, B, C, D$ dzielą okrąg na 4 równe łuki. Miara zaznaczonego na rysunku kąta

wpisanego $ACD$ jest równa

A. $90^{\mathrm{o}}$

B. $60^{\mathrm{o}}$
\begin{center}
\includegraphics[width=50.388mm,height=50.388mm]{./F1_M_PP_M2012_page5_images/image002.eps}
\end{center}
{\it C}

$D$

{\it B}

{\it A}

D. $30^{\mathrm{o}}$

C. $45^{\mathrm{o}}$

Zadanie 17. (1pkt)

Miary kątów czworokąta tworzą ciąg arytmetyczny o róznicy

czworokąta ma miarę

$20^{\mathrm{o}}$ Najmniejszy kąt tego

A. $40^{\mathrm{o}}$

B. $50^{\mathrm{o}}$

C. $60^{\mathrm{o}}$

D. $70^{\mathrm{o}}$

Zadanie 18. $(1pkt)$

Dany jest ciąg $(a_{n})$ określony wzorem $a_{n}=(-1)^{n}\displaystyle \cdot\frac{2-n}{n^{2}}$ dla $n\geq 1$. Wówczas wyraz $a_{5}$ tego

ciągujest równy

A. - $\displaystyle \frac{3}{25}$ B. $\displaystyle \frac{3}{25}$ C. - $\displaystyle \frac{7}{25}$ D. $\displaystyle \frac{7}{25}$





{\it Egzamin maturalny z matematyki}

{\it Poziom podstawowy}

7

BRUDNOPIS





{\it 8}

{\it Egzamin maturalny z matematyki}

{\it Poziom podstawowy}

Zadanie 19. (1pkt)

Pole powierzchni jednej ściany sześcianujest równe 4. Objętość tego sześcianujest równa

A. 6

B. 8

C. 24

D. 64

Zadanie 20. $(1pkt)$

Tworząca stozka ma długość 4 i jest nachy1ona do płaszczyzny podstawy pod kątem $45^{\mathrm{o}}$

Wysokość tego stozkajest równa

A. $2\sqrt{2}$

B. $ 16\pi$

C. $4\sqrt{2}$

D. $ 8\pi$

Zadanie 21. $(1pkt)$

Wskaz równanie prostej równoległej do prostej o równaniu $3x-6y+7=0.$

A. {\it y}$=$-21{\it x} B. {\it y}$=$--21{\it x} C. {\it y}$=$2{\it x} D. {\it y}$=- 2x$

Zadanie 22. (1pkt)

Punkt A ma współrzędne (5,2012). Punkt B jest symetryczny do punktu A wzg1ędem osi Ox,

a punkt Cjest symetryczny do punktu B względem osi Oy. Punkt C ma współrzędne

A. $(-5,-2012)$

B. $(-2012,-5)$

C. $(-5$, 2012$)$

D. $(-2012,5)$

Zadanie 23. $(1pkt)$

Na okręgu o równaniu $(x-2)^{2}+(y+7)^{2}=4\mathrm{l}\mathrm{e}\dot{\mathrm{z}}\mathrm{y}$ punkt

A. $A=(-2,5)$

B. $B=(2,-5)$

C. $C=(2,-7)$

D. $D=(7,-2)$

Zadanie 24. (1pkt)

Flagę, takąjak pokazano na rysunku, nalezy zszyć

z trzech jednakowej szerokości pasów kolorowej

tkaniny. Oba pasy zewnętrzne mają być tego

samego koloru, a pas znajdujący się między nimi

ma być innego koloru.

Liczba róznych takich flag, które mozna uszyć,

mając do dyspozycji tkaniny w 10 ko1orach, jest

równa

A. 100

B. 99

C. 90

D. 19

Zadanie 25. (1pkt)

Średnia arytmetyczna cen sześciu akcji na giełdzie jest równa 500 zł. Za pięć z tych akcji

zapłacono 2300 zł. Cena szóstej akcjijest równa

A. 400 zł

B. 500 zł

C. 600 zł

D. 700 zł





{\it Egzamin maturalny z matematyki}

{\it Poziom podstawowy}

{\it 9}

BRUDNOPIS





$ 1\theta$

{\it Egzamin maturalny z matematyki}

{\it Poziom podstawowy}

ZADANIA OTWARTE

{\it Rozwiqzania zadań o numerach od 26. do 34. nalezy zapisać w} $wyznacz\theta nych$ {\it miejscach}

{\it pod treściq zadania}.

Zadanie 26. $(2pkt)$

Rozwiąz nierówność $x^{2}+8x+15>0.$

Odpowiedzí:

Zadanie 27. $(2pkt)$

Uzasadnij, $\dot{\mathrm{z}}\mathrm{e}$ jeśli liczby rzeczywiste $a,$

$\displaystyle \frac{a+b+c}{3}>\frac{a+b}{2}.$

$b, c$ spełniają nierówności $0<a<b<c$, to







Centralna Komisja Egzaminacyjna

Arkusz zawiera informacje prawnie chronione do momentu rozpoczęcia egzaminu.

WPISUJE ZDAJACY

KOD PESEL

{\it Miejsce}

{\it na naklejkę}

{\it z kodem}
\begin{center}
\includegraphics[width=21.432mm,height=9.804mm]{./F1_M_PP_M2013_page0_images/image001.eps}

\includegraphics[width=82.092mm,height=9.804mm]{./F1_M_PP_M2013_page0_images/image002.eps}
\end{center}
\fbox{} dysleksja
\begin{center}
\includegraphics[width=204.060mm,height=216.048mm]{./F1_M_PP_M2013_page0_images/image003.eps}
\end{center}
EGZAMIN MATU LNY

Z MATEMATYKI

MAJ 2013

POZIOM PODSTAWOWY

1.

2.

3.

Sprawd $\acute{\mathrm{z}}$, czy ar sz egzaminacyjny zawiera 22 strony

(zadania $1-34$). Ewentualny brak zgłoś przewodniczącemu

zespo nadzorującego egzamin.

Rozwiązania zadań i odpowiedzi wpisuj w miejscu na to

przeznaczonym.

Odpowiedzi do zadań za iętych (1-25) przenieś

na ka ę odpowiedzi, zaznaczając je w części ka $\mathrm{y}$

przeznaczonej dla zdającego. Zamaluj $\blacksquare$ pola do tego

przeznaczone. Błędne zaznaczenie otocz kółkiem

i zaznacz właściwe.

4. Pamiętaj, $\dot{\mathrm{z}}\mathrm{e}$ pominięcie argumentacji lub istotnych

obliczeń w rozwiązaniu zadania otwa ego (26-34) $\mathrm{m}\mathrm{o}\dot{\mathrm{z}}\mathrm{e}$

spowodować, $\dot{\mathrm{z}}\mathrm{e}$ za to rozwiązanie nie będziesz mógł

dostać pełnej liczby punktów.

5. Pisz czytelnie i uzywaj tvlko długopisu lub -Dióra

z czamym tuszem lub atramentem.

6. Nie uzywaj korektora, a błędne zapisy wyra $\acute{\mathrm{z}}\mathrm{n}\mathrm{i}\mathrm{e}$ przekreśl.

7. Pamiętaj, $\dot{\mathrm{z}}\mathrm{e}$ zapisy w brudnopisie nie będą oceniane.

8. $\mathrm{M}\mathrm{o}\dot{\mathrm{z}}$ esz korzystać z zestawu wzorów matematycznych,

cyrkla i linijki oraz kalkulatora.

9. Na tej stronie oraz na karcie odpowiedzi wpisz swój

numer PESEL i przyklej naklejkę z kodem.

10. Nie wpisuj $\dot{\mathrm{z}}$ adnych znaków w części przeznaczonej

dla egzaminatora.

Czas pracy:

170 minut

Liczba punktów

do uzyskania: 50

$\Vert\Vert\Vert\Vert\Vert\Vert\Vert\Vert\Vert\Vert\Vert\Vert\Vert\Vert\Vert\Vert\Vert\Vert\Vert\Vert\Vert\Vert\Vert\Vert|  \mathrm{M}\mathrm{M}\mathrm{A}-\mathrm{P}1_{-}1\mathrm{P}-132$




{\it 2}

{\it Egzamin maturalny z matematyki}

{\it Poziom podstawowy}

ZADANIA ZAMKNIĘTE

{\it Wzadaniach l-25 wybierz i zaznacz na karcie odpowiedzipoprawnq odpowiedzí}.

Zadanie l. $(1pkt)$

Wskaz rysunek, na którym zaznaczony

spełniających nierówność $|x+4|<5.$

jest zbiór

wszystkich liczb rzeczywistych
\begin{center}
\includegraphics[width=165.552mm,height=12.240mm]{./F1_M_PP_M2013_page1_images/image001.eps}
\end{center}
A.
\begin{center}
\includegraphics[width=165.612mm,height=17.832mm]{./F1_M_PP_M2013_page1_images/image002.eps}
\end{center}
$-9  -4$  1  {\it X}

B.
\begin{center}
\includegraphics[width=165.552mm,height=18.036mm]{./F1_M_PP_M2013_page1_images/image003.eps}
\end{center}
$-1$  4 9  {\it X}

C.
\begin{center}
\includegraphics[width=165.552mm,height=17.784mm]{./F1_M_PP_M2013_page1_images/image004.eps}
\end{center}
$-9  -5  -1$  {\it X}

1 5  9  {\it X}

D.

Zadanie 2. $(1pkt)$

Liczby $a\mathrm{i}b$ są dodatnie oraz 12\% 1iczby $a$ jest równe 15\% 1iczby $b$. Stąd wynika, $\dot{\mathrm{z}}\mathrm{e}a$ jest

równe

A. 103\% 1iczby $b$ B. 125\% 1iczby $b$ C. 150\% 1iczby $b$ D. 153\% 1iczby $b$

Zadanie 3. $(1pkt)$

Liczba $\log 100-\log_{2}8$ jest równa

A. $-2$

B. $-1$

C. 0

D. l

Zadanie 4. $(1pkt)$

Rozwiązaniem układu równań 

A. $x=-3 \mathrm{i}y=4$

B. $x=-3 \mathrm{i}y=6$

C. $x=3 \mathrm{i}y=-4$

D. $x=9 \mathrm{i}y=4$

Zadanie 5. $(1pkt)$

Punkt $A=(0,1)$ lezy na wykresie ffinkcji liniowej $f(x)=(m-2)x+m-3$. Stąd wynika, $\dot{\mathrm{z}}\mathrm{e}$

A. $m=1$

B. $m=2$

C. $m=3$

D. $m=4$

Zadanie 6. $(1pkt)$

Wierzchołkiem paraboli o równaniu $y=-3(x-2)^{2}+4$ jest punkt o współrzędnych

A. $(-2,-4)$

B. $(-2,4)$

C. $(2,-4)$

D. (2, 4)

Zadanie 7. $(1pkt)$

Dla $\mathrm{k}\mathrm{a}\dot{\mathrm{z}}$ dej liczby rzeczywistej $x$, wyrazenie $4x^{2}-12x+9$ jest równe

A. $(4x+3)(x+3)$

B. $(2x-3)(2x+3)$

C. $(2x-3)(2x-3)$

D. $(x-3)(4x-3)$





{\it Egzamin maturalny z matematyki}

{\it Poziom podstawowy}

{\it 11}

Zadanie 27. $(2pkt)$

Kąt $\alpha$ jest ostry i $\displaystyle \sin\alpha=\frac{\sqrt{3}}{2}$. Oblicz wartość wyrazenia $\sin^{2}\alpha-3\cos^{2}\alpha.$

Odpowied $\acute{\mathrm{z}}$:
\begin{center}
\includegraphics[width=95.964mm,height=17.832mm]{./F1_M_PP_M2013_page10_images/image001.eps}
\end{center}
Wypelnia

egzaminator

2

27.

2

Uzyskana liczba pkt





{\it 12}

{\it Egzamin maturalny z matematyki}

{\it Poziom podstawowy}

Zadanie 28. $(2pkt)$

Udowodnij, $\dot{\mathrm{z}}\mathrm{e}$ dla dowolnych liczb rzeczywistych $x, y, z$ takich, $\dot{\mathrm{z}}\mathrm{e}x+y+z=0$, prawdziwa

jest nierówność $xy+yz+zx\leq 0.$

$\mathrm{M}\mathrm{o}\dot{\mathrm{z}}$ esz skorzystać z $\mathrm{t}\mathrm{o}\dot{\mathrm{z}}$ samości $(x+y+z)^{2}=x^{2}+y^{2}+z^{2}+2xy+2xz+2yz.$





{\it Egzamin maturalny z matematyki}

{\it Poziom podstawowy}

{\it 13}

Zadanie 29. $(2pkt)$

Na rysunku przedstawiony jest wykres funkcji $f(x)$ określonej dla $x\in\langle-7,8\rangle.$
\begin{center}
\includegraphics[width=162.564mm,height=98.292mm]{./F1_M_PP_M2013_page12_images/image001.eps}
\end{center}
Odczytaj z wykresu i zapisz:

a) największą wartość funkcji f,

b) zbiór rozwiązań nierówności $f(x)<0.$
\begin{center}
\includegraphics[width=96.012mm,height=17.832mm]{./F1_M_PP_M2013_page12_images/image002.eps}
\end{center}
Wypelnia

egzaminator

Nr zadania

Maks. liczba kt

28.

2

2

Uzyskana liczba pkt





{\it 14}

{\it Egzamin maturalny z matematyki}

{\it Poziom podstawowy}

Zadanie 30. $(2pkt)$

Rozwiąz nierówność $2x^{2}-7x+5\geq 0.$

Odpowiedzí:





{\it Egzamin maturalny z matematyki}

{\it Poziom podstawowy}

{\it 15}

Zadanie 31. $(2pkt)$

Wykaz, $\dot{\mathrm{z}}\mathrm{e}$ liczba $6^{100}-2\cdot 6^{99}+10\cdot 6^{98}$ jest podzielna przez 17.
\begin{center}
\includegraphics[width=95.964mm,height=17.784mm]{./F1_M_PP_M2013_page14_images/image001.eps}
\end{center}
Wypelnia

egzaminator

Nr zadania

Maks. liczba kt

30.

2

31.

2

Uzyskana liczba pkt





{\it 16}

{\it Egzamin maturalny z matematyki}

{\it Poziom podstawowy}

Zadanie 32. (4pkt)

Punkt S jest środkiem okręgu opisanego na trójkącie ostrokątnym ABC. Kąt ACS jest trzy razy

większy od kąta BAS, a kąt CBSjest dwa razy większy od kąta BAS. Oblicz kąty trójkąta ABC.
\begin{center}
\includegraphics[width=93.828mm,height=90.168mm]{./F1_M_PP_M2013_page15_images/image001.eps}
\end{center}
{\it C}

{\it S}

{\it A  B}





{\it Egzamin maturalny z matematyki}

{\it Poziom podstawowy}

17

Odpowied $\acute{\mathrm{z}}$:
\begin{center}
\includegraphics[width=82.044mm,height=17.832mm]{./F1_M_PP_M2013_page16_images/image001.eps}
\end{center}
Wypelnia

egzaminator

Nr zadania

Maks. liczba kt

32.

4

Uzyskana liczba pkt





{\it 18}

{\it Egzamin maturalny z matematyki}

{\it Poziom podstawowy}

Zadanie 33. $(4pkt)$

Pole podstawy ostrosłupa prawidłowego czworokątnego jest równe 100

pole powierzchni bocznej jest równe 260 $\mathrm{c}\mathrm{m}^{2}$. Oblicz objętość tego ostrosłupa.

$\mathrm{c}\mathrm{m}^{2}$, a jego





{\it Egzamin maturalny z matematyki}

{\it Poziom podstawowy}

{\it 19}

Odpowiedzí :
\begin{center}
\includegraphics[width=82.044mm,height=17.832mm]{./F1_M_PP_M2013_page18_images/image001.eps}
\end{center}
Wypelnia

egzaminator

Nr zadania

Maks. liczba kt

33.

4

Uzyskana liczba pkt





$ 2\theta$

{\it Egzamin maturalny z matematyki}

{\it Poziom podstawowy}

Zadanie 34. $(5pkt)$

Dwa miasta łączy linia kolejowa o długości 336 ki1ometrów. Pierwszy pociąg przebył tę trasę

w czasie o 40 minut krótszym $\mathrm{n}\mathrm{i}\dot{\mathrm{z}}$ drugi pociąg. Średnia prędkość pierwszego pociągu na tej

trasie była o 9 $\mathrm{k}\mathrm{n}\vee \mathrm{h}$ większa od średniej prędkości drugiego pociągu. Oblicz średnią

prędkość $\mathrm{k}\mathrm{a}\dot{\mathrm{z}}$ dego z tych pociągów na tej trasie.





{\it Egzamin maturalny z matematyki}

{\it Poziom podstawowy}

{\it 3}

BRUDNOPIS





{\it Egzamin maturalny z matematyki}

{\it Poziom podstawowy}

{\it 21}

Odpowiedzí :
\begin{center}
\includegraphics[width=82.044mm,height=17.832mm]{./F1_M_PP_M2013_page20_images/image001.eps}
\end{center}
Wypelnia

egzaminator

Nr zadania

Maks. liczba kt

34.

5

Uzyskana liczba pkt





{\it 22}

{\it Egzamin maturalny z matematyki}

{\it Poziom podstawowy}

BRUDNOPIS





{\it 4}

{\it Egzamin maturalny z matematyki}

{\it Poziom podstawowy}

Zadanie 8. $(1pkt)$

Prosta o równaniu $y=\displaystyle \frac{2}{m}x+1$ jest prostopadła do prostej o równaniu $y=-\displaystyle \frac{3}{2}x-1$. Stąd

wynika, $\dot{\mathrm{z}}\mathrm{e}$

A. $m=-3$

B.

{\it m}$=$ -23

C.

{\it m}$=$ -23

D. $m=3$

Zadanie 9. $(1pkt)$

Na rysunku przedstawiony jest fragment wykresu pewnej funkcji liniowej $y=ax+b.$
\begin{center}
\includegraphics[width=66.036mm,height=50.748mm]{./F1_M_PP_M2013_page3_images/image001.eps}
\end{center}
$y$

0  {\it x}

Jakie znaki mają współczynniki a ib?

A. $a<0 \mathrm{i}b<0$

B. $a<0 \mathrm{i}b>0$

C. $a>0 \mathrm{i}b<0$

D. $a>0\mathrm{i}b>0$

Zadanie 10. (1pkt)

Najmniejszą liczbą całkowitą spełniającą nierówność $\displaystyle \frac{x}{2}\leq\frac{2x}{3}+\frac{1}{4}$ jest

A. $-2$

B. $-1$

C. 0

D. l

Zadanie ll. $(1pkt)$

Na rysunku l przedstawiony jest wykres funkcji $y=f(x)$ określonej dla $x\in\langle-7,4\rangle.$
\begin{center}
\includegraphics[width=184.500mm,height=59.280mm]{./F1_M_PP_M2013_page3_images/image002.eps}
\end{center}
Rysunek 2 przedstawia wykres ffinkcji

A. $y=f(x+2)$ B. $y=f(x)-2$

C. $y=f(x-2)$

D. $y=f(x)+2$

Zadanie 12. $(1pkt)$

Ciąg $($27, 18, $x+5)$ jest geometryczny. Wtedy

A. $x=4$

B. $x=5$

C. $x=7$

D. $x=9$





{\it Egzamin maturalny z matematyki}

{\it Poziom podstawowy}

{\it 5}

BRUDNOPIS





{\it 6}

{\it Egzamin maturalny z matematyki}

{\it Poziom podstawowy}

Zadanie 13. $(1pkt)$

Ciąg $(a_{n})$ określony dla $n\geq 1$ jest arytmetyczny oraz $a_{3}=10 \mathrm{i}a_{4}=14$. Pierwszy wyraz tego

ciągu jest równy

A. $a_{1}=-2$ B. $a_{1}=2$ C. $a_{1}=6$ D. $a_{1}=12$

Zadanie 14. $(1pkt)$

Kąt $\alpha$ jest ostry i $\displaystyle \sin\alpha=\frac{\sqrt{3}}{2}$. Wartość wyrazenia $\cos^{2}\alpha-2$ jest równa

A.

- -47

B.

- -41

C.

-21

D.

-$\sqrt{}$23

Zadanie 15. $(1pkt)$

Średnice AB $\mathrm{i}$ CD okręgu o środku $S$ przecinają się pod kątem $50^{\mathrm{o}}$ (takjak na rysunku).
\begin{center}
\includegraphics[width=65.124mm,height=65.628mm]{./F1_M_PP_M2013_page5_images/image001.eps}
\end{center}
{\it B}

{\it D}

$\alpha$

{\it S  M}

$50^{\mathrm{o}}$

{\it C}

{\it A}

Miara kąta $\alpha$ jest równa

A. $25^{\mathrm{o}}$

B. $30^{\mathrm{o}}$

C. $40^{\mathrm{o}}$

D. $50^{\mathrm{o}}$

Zadanie 16. $(1pkt)$

Liczba rzeczywistych rozwiązań równania $(x+1)(x+2)(x^{2}+3)=0$ jest równa

A. 0

B. l

C. 2

D. 4

Zadanie 17. $(1pkt)$

Punkty $A=(-1,2) \mathrm{i}B=(5,-2)$ są dwoma sąsiednimi wierzchołkami rombu ABCD. Obwód

tego rombujest równy

A. $\sqrt{13}$

B. 13

C. 676

D. $8\sqrt{13}$

Zadanie 18. $(1pkt)$

Punkt $S=(-4,7)$ jest środkiem odcinka

współrzędne

$PQ$, gdzie $Q=(17,12)$. Zatem punkt $P$ ma

A. $P=(2,-25)$

B. $P=(38,17)$

C. $P=(-25,2)$

D. $P=(-12,4)$





{\it Egzamin maturalny z matematyki}

{\it Poziom podstawowy}

7

BRUDNOPIS





{\it 8}

{\it Egzamin maturalny z matematyki}

{\it Poziom podstawowy}

Zadanie 19. $(1pkt)$

Odległość między środkami okręgów o równaniach $(x+1)^{2}+(y-2)^{2}=9$ oraz $x^{2}+y^{2}=10$

jest równa

A. $\sqrt{5}$

B. $\sqrt{10}-3$

C. 3

D. 5

Zadanie 20. $(1pkt)$

Liczba wszystkich krawędzi graniastosłupajest o 10 większa od 1iczby wszystkichjego ścian

bocznych. Stąd wynika, $\dot{\mathrm{z}}\mathrm{e}$ podstawą tego graniastosłupajest

A. czworokąt

B. pięciokąt

C. sześciokąt

D. dziesięciokąt

Zadanie 21. (1pkt)

Pole powierzchni bocznej stozka o wysokości 4 i promieniu podstawy 3 jest równe

A. $ 9\pi$

B. $ 12\pi$

C. $ 15\pi$

D. $ 16\pi$

Zadanie 22. $(1pkt)$

Rzucamy dwa razy symetryczną sześcienną kostką do gry. Niech $p$ oznacza

prawdopodobieństwo zdarzenia, $\dot{\mathrm{z}}\mathrm{e}$ iloczyn liczb wyrzuconych oczekjest równy 5. Wtedy

A.

$p=\displaystyle \frac{1}{36}$

B.

$p=\displaystyle \frac{1}{18}$

C.

$p=\displaystyle \frac{1}{12}$

D.

{\it p}$=$ -91

Zadanie 23. $(1pkt)$

Liczba $\displaystyle \frac{\sqrt{50}-\sqrt{18}}{\sqrt{2}}$ jest równa

A. $2\sqrt{2}$ B. 2

C. 4

D. $\sqrt{10}-\sqrt{6}$

Zadanie 24. (1pkt)

Mediana uporządkowanego niemalejąco zestawu sześciu liczb:

Wtedy

1, 2, 3, x, 5, 8 jest równa 4.

A. $x=2$

B. $x=3$

C. $x=4$

D. $x=5$

Zadanie 25. $(1pkt)$

Objętość graniastosłupa prawidłowego trójkątnego o wysokości $7$jest równa $28\sqrt{3}$. Długość

krawędzi podstawy tego graniastosłupajest równa

A. 2

B. 4

C. 8

D. 16





{\it Egzamin maturalny z matematyki}

{\it Poziom podstawowy}

{\it 9}

BRUDNOPIS





$ 1\theta$

{\it Egzamin maturalny z matematyki}

{\it Poziom podstawowy}

ZADANIA OTWARTE

{\it Rozwiqzania zadań} $26-34$ {\it nalezy zapisać w wyznaczonych miejscach} $p\theta d$ {\it treściq zadania}.

Zadanie 26. $(2pkt)$

Rozwiąz równanie $x^{3}+2x^{2}-8x-16=0$

Odpowiedzí:







$--\cdot\backslash \tau^{\mathrm{l}}\cdots-\cdot i1\dot{\text{‡C}}$

$:_{}^{\prime:=_{1\text{‡@}}^{1}}\overline{\iota}_{:\dot{!^{\mathrm{f}}:}!_{\vee}}^{-1}..$

-r$\equiv$:$\grave{}=$-J-$=$.-$\acute{}$z--,-.-[‡@]n-w$\Omega$-.:-.R-.J--n-$\llcorner\ulcorner-\simeq-\breve{}$.---$\lrcorner$.--[‡@]R-[‡@]-.--{\it l}-$\iota$-

Arkusz zawiera informacje prawnie chronione do momentu rozpoczęcia egzaminu.

WPISUJE ZDAJACY

KOD PESEL

{\it Miejsce}

{\it na naklejkę}

{\it z kodem}
\begin{center}
\includegraphics[width=21.432mm,height=9.852mm]{./F1_M_PP_M2014_page0_images/image001.eps}

\includegraphics[width=82.044mm,height=9.852mm]{./F1_M_PP_M2014_page0_images/image002.eps}
\end{center}
\fbox{} dysleksja
\begin{center}
\includegraphics[width=204.060mm,height=219.000mm]{./F1_M_PP_M2014_page0_images/image003.eps}
\end{center}
EGZAMIN MATU LNY

Z MATEMATYKI

MAJ 2014

POZIOM PODSTAWOWY

1. Sprawd $\acute{\mathrm{z}}$, czy arkusz egzaminacyjny zawiera 19 stron

(zadania $1-34$). Ewentualny brak zgłoś przewodniczącemu

zespo nadzo jącego egzamin.

2. Rozwiązania zadań i odpowiedzi wpisuj w miejscu na to

przeznaczonym.

3. Odpowiedzi do zadań za niętych (l-25) przenieś

na ka ę odpowiedzi, zaznaczając je w części ka $\mathrm{y}$

przeznaczonej dla zdającego. Zamaluj $\blacksquare$ pola do tego

przeznaczone. Błędne zaznaczenie otocz kółkiem \fcircle$\bullet$

i zaznacz właściwe.

4. Pamiętaj, $\dot{\mathrm{z}}\mathrm{e}$ pominięcie argumentacji lub istotnych

obliczeń w rozwiązaniu zadania otwa ego (26-34) $\mathrm{m}\mathrm{o}\dot{\mathrm{z}}\mathrm{e}$

spowodować, $\dot{\mathrm{z}}\mathrm{e}$ za to rozwiązanie nie otrzymasz pełnej

liczby punktów.

5. Pisz czytelnie i uzywaj tvlko długopisu lub -Dióra

z czatnym tuszem lub atramentem.

6. Nie uzywaj korektora, a błędne zapisy wyrazínie prze eśl.

7. Pamiętaj, $\dot{\mathrm{z}}\mathrm{e}$ zapisy w brudnopisie nie będą oceniane.

8. $\mathrm{M}\mathrm{o}\dot{\mathrm{z}}$ esz korzystać z zestawu wzorów matematycznych,

cyrkla i linijki oraz kalkulatora.

9. Na tej stronie oraz na karcie odpowiedzi wpisz swój

numer PESEL i przyklej naklejkę z kodem.

10. Nie wpisuj $\dot{\mathrm{z}}$ adnych znaków w części przeznaczonej

dla egzaminatora.

Czas pracy:

170 minut

Liczba punktów

do uzyskania: 50

$\Vert\Vert\Vert\Vert\Vert\Vert\Vert\Vert\Vert\Vert\Vert\Vert\Vert\Vert\Vert\Vert\Vert\Vert\Vert\Vert\Vert\Vert\Vert\Vert|  \mathrm{M}\mathrm{M}\mathrm{A}-\mathrm{P}1_{-}1\mathrm{P}-142$




{\it 2}

{\it Egzamin maturalny z matematyki}

{\it Poziom podstawowy}

ZADANIA ZAMKNIĘTE

{\it Wzadaniach od l. do 25. wybierz i zaznacz na karcie odpowiedzipoprawnq} $odp\theta wied\acute{z}.$

Zadanie l. $(1pkt)$

Na rysunku przedstawiono geometryczną interpretację jednego z $\mathrm{n}\mathrm{i}\dot{\mathrm{z}}$ ej zapisanych układów

równań.
\begin{center}
\includegraphics[width=62.844mm,height=50.340mm]{./F1_M_PP_M2014_page1_images/image001.eps}
\end{center}
4  {\it y}

$-3$

2

$-2$ -$1$

0

$-1$

1 2 3  {\it x}

Wskaz ten układ.

A.

$\left\{\begin{array}{l}
y=x+1\\
y=-2x+4
\end{array}\right.$

B.

$\left\{\begin{array}{l}
y=x-1\\
y=2x+4
\end{array}\right.$

C.

$\left\{\begin{array}{l}
y=x-1\\
y=-2x+4
\end{array}\right.$

D.

$\left\{\begin{array}{l}
y=x+1\\
y=2x+4
\end{array}\right.$

Zadanie 2. $(1pkt)$

$\mathrm{J}\mathrm{e}\dot{\mathrm{z}}$ eli liczba $78$jest o 50\% większa od 1iczby $c$, to

A. $c=60$

B. $c=52$

C. $c=48$

D. $c=39$

Zadanie 3. $(1pkt)$

Wartość wyrazenia $\displaystyle \frac{2}{\sqrt{3}-1}-\frac{2}{\sqrt{3}+1}$ jest równa

A. $-2$ B. $-2\sqrt{3}$

C. 2

D. $2\sqrt{3}$

Zadanie $4.(1pkt)$

Suma $\log_{8}16+1$ jest równa

A. 3

B.

-23

C. log817

D.

-73

Zadanie 5. $(1pkt)$

Wspólnym pierwiastkiem równań $(x^{2}-1)(x-10)(x-5)=0$ oraz $\displaystyle \frac{2x-10}{x-1}=0$ jest liczba

A. $-1$

B. l

C. 5

D. 10





{\it Egzamin maturalny z matematyki}

{\it Poziom podstawowy}

{\it 11}

Zadanie 27. $(2pkt)$

Rozwiąz równanie $9x^{3}+18x^{2}-4x-8=0.$

Odpowiedzí :
\begin{center}
\includegraphics[width=90.372mm,height=17.580mm]{./F1_M_PP_M2014_page10_images/image001.eps}
\end{center}
Wypelnia

egzamÍnator

Nr zadania

Maks. liczba kt

2

27.

2

Uzyskana liczba pkt





{\it 12}

{\it Egzamin maturalny z matematyki}

{\it Poziom podstawowy}

Zadanie 28. $(2pkt)$

Udowodnij, $\dot{\mathrm{z}}\mathrm{e}\mathrm{k}\mathrm{a}\dot{\mathrm{z}}$ da liczba całkowita $k$, która przy dzieleniu przez 7 daje resztę 2, ma tę

własność, $\dot{\mathrm{z}}\mathrm{e}$ reszta z dzielenia liczby $3k^{2}$ przez $7$jest równa 5.





{\it Egzamin maturalny z matematyki}

{\it Poziom podstawowy}

{\it 13}

Zadanie 29. $(2pkt)$

Na rysunku przedstawiono fragment wykresu ffinkcji $f$, który powstał w wyniku przesunięcia

wykresu funkcji określonej wzorem $y=\displaystyle \frac{1}{x}$ dla $\mathrm{k}\mathrm{a}\dot{\mathrm{z}}$ dej liczby rzeczywistej $x\neq 0.$
\begin{center}
\includegraphics[width=98.760mm,height=87.372mm]{./F1_M_PP_M2014_page12_images/image001.eps}
\end{center}
a) Odczytaj z wykresu i zapisz zbiór tych wszystkich argumentów, dla których wartości

funkcji $f$ są większe od 0.

b) Podaj miejsce zerowe funkcji $g$ określonej wzorem $g(x)=f(x-3).$

Odpowied $\acute{\mathrm{z}}:\mathrm{a})$

b)
\begin{center}
\includegraphics[width=90.372mm,height=17.580mm]{./F1_M_PP_M2014_page12_images/image002.eps}
\end{center}
Wypelnia

egzamÍnator

Nr zadania

Maks. liczba kt

28.

2

2

Uzyskana liczba pkt





{\it 14}

{\it Egzamin maturalny z matematyki}

{\it Poziom podstawowy}

Zadanie 30. (2pkt)

Ze zbioiu liczb \{1, 2, 3, 4, 5, 6, 7, 8\} 1osujemy dwa razy po jednej 1iczbie ze zwracaniem.

Oblicz prawdopodobieństwo zdarzenia A, polegającego na wylosowaniu liczb, z których

pierwszajest większa od drugiej o 41ub 6.

Odpowied $\acute{\mathrm{z}}$:





{\it Egzamin maturalny z matematyki}

{\it Poziom podstawowy}

{\it 15}

Zadanie 31. (2pkt)

Środek S okręgu opisanego na trójkącie równoramiennym ABC, o ramionach ACiBC, lezy

wewnątrz tego trójkąta (zobacz rysunek).
\begin{center}
\includegraphics[width=60.708mm,height=65.076mm]{./F1_M_PP_M2014_page14_images/image001.eps}
\end{center}
{\it C}

{\it S}

{\it A  B}

{\it ASB}

kąta wypukłego

Wykaz, $\dot{\mathrm{z}}\mathrm{e}$ miara

wypukłego $SBC.$

est cztery

razy większa

od miary kąta
\begin{center}
\includegraphics[width=90.372mm,height=17.580mm]{./F1_M_PP_M2014_page14_images/image002.eps}
\end{center}
Wypelnia

egzamÍnator

Nr zadania

Maks. liczba kt

30.

2

31.

2

Uzyskana liczba pkt





{\it 16}

{\it Egzamin maturalny z matematyki}

{\it Poziom podstawowy}

Zadanie 32. (4pkt)

Pole powierzchni całkowitej prostopadłościanu jest równe 198. Stosunki długości krawędzi

prostopadłościanu wychodzących z tego samego wierzchołka prostopadłościanu to 1: 2: 3.

Oblicz długość przekątnej tego prostopadłościanu.

Odpowied $\acute{\mathrm{z}}$:





{\it Egzamin maturalny z matematyki}

{\it Poziom podstawowy}

{\it 1}7

Zadanie 33. $(5pkt)$

Turysta zwiedzał zamek stojący na wzgórzu. Droga łącząca parking z zamkiem ma długość

2,1 km. Lączny czas wędrówki turysty z parkingu do zamku i z powrotem, nie licząc czasu

poświęconego na zwiedzanie, był równy l godzinę i 4 minuty. Ob1icz, z jaką średnią

prędkością turysta wchodził na wzgórze, $\mathrm{j}\mathrm{e}\dot{\mathrm{z}}$ eli prędkość ta była o $1 \displaystyle \frac{\mathrm{k}\mathrm{m}}{\mathrm{h}}$ mniejsza od średniej

prędkości, zjaką schodził ze wzgórza.

Odpowied $\acute{\mathrm{z}}$:
\begin{center}
\includegraphics[width=90.372mm,height=17.628mm]{./F1_M_PP_M2014_page16_images/image001.eps}
\end{center}
Wypelnia

egzaminator

Nr zadania

Maks. liczba kt

32.

4

33.

5

Uzyskana liczba pkt





{\it 18}

{\it Egzamin maturalny z matematyki}

{\it Poziom podstawowy}

Zadanie 34. $(4pkt)$

Kąt CAB trójkąta prostokątnego $ACB$ ma miarę $30^{\mathrm{o}}$. Pole kwadratu DEFG, wpisanego w ten

trójkąt (zobacz rysunek), jest równe 4. Ob1icz po1e trójkąta $ACB.$
\begin{center}
\includegraphics[width=68.880mm,height=43.692mm]{./F1_M_PP_M2014_page17_images/image001.eps}
\end{center}
{\it B}

{\it F}

{\it E}

{\it G}

$30^{\mathrm{o}}$

{\it C D  A}

Odpowiedzí :
\begin{center}
\includegraphics[width=78.840mm,height=17.580mm]{./F1_M_PP_M2014_page17_images/image002.eps}
\end{center}
Wypelnia

egzaminator

Nr zadania

Maks. liczba kt

34.

4

Uzyskana liczba pkt





{\it Egzamin maturalny z matematyki}

{\it Poziom podstawowy}

{\it 19}

BRUDNOPIS





{\it Egzamin maturalny z matematyki}

{\it Poziom podstawowy}

{\it 3}

BRUDNOPIS





{\it 4}

{\it Egzamin maturalny z matematyki}

{\it Poziom podstawowy}

Zadanie 6. $(1pkt)$

Funkcja liniowa $f(x)=(m^{2}-4)x+2$ jest malejąca, gdy

A. $m\in\{-2,2\}$

B. $m\in(-2,2)$

C.

$m\in(-\infty,-2)$

D. $m\in(2,+\infty)$

Zadanie 7. (1pkt)

Na rysunku przedstawiono fragment wykresu funkcji kwadratowej f
\begin{center}
\includegraphics[width=56.436mm,height=49.788mm]{./F1_M_PP_M2014_page3_images/image001.eps}
\end{center}
{\it y}

{\it x}

0

Funkcjafjest określona wzorem

A.

C.

$f(x)=\displaystyle \frac{1}{2}(x+3)(x-1)$

$f(x)=-\displaystyle \frac{1}{2}(x+3)(x-1)$

B.

D.

$f(x)=\displaystyle \frac{1}{2}(x-3)(x+1)$

$f(x)=-\displaystyle \frac{1}{2}(x-3)(x+1)$

Zadanie 8. $(1pkt)$

Punkt $C=(0,2)$ jest wierzchołkiem trapezu ABCD, którego podstawa $AB$ jest zawarta

w prostej o równaniu $y=2x-4$. Wskaz równanie prostej zawierającej podstawę CD.

A. $y=\displaystyle \frac{1}{2}x+2$ B. $y=-2x+2$ C. $y=-\displaystyle \frac{1}{2}x+2$ D. $y=2x+2$

Zadanie 9. $(1pkt)$

Dla $\mathrm{k}\mathrm{a}\dot{\mathrm{z}}$ dej liczby $x$, spełniającej warunek-3$<x<0$, wyrazenie $\displaystyle \frac{|x+3|-x+3}{x}$ jest równe

A. 2 B. 3 C. --{\it x}6 D. -{\it x}6

Zadanie 10. $(1pkt)$

Pierwiastki $x_{1}, x_{2}$ równania $2(x+2)(x-2)=0$ spełniają warunek

A.

$\underline{1}\underline{1}+=-1$

$x_{1} x_{2}$

B.

$\underline{1}+\underline{1}=0$

$x_{1} x_{2}$

C.

-{\it x}1  1 $+$ -{\it x}12 $=$ -41

D.

-{\it x}1  1 $+$ -{\it x}12 $=$ -21

Zadanie ll. $(1pkt)$

Liczby $2, -1, -4$ są trzema początkowymi wyrazami ciągu arytmetycznego

określonego dla liczb naturalnych $n\geq 1$. Wzór ogólny tego ciągu ma postać

A. $a_{n}=-3n+5$ B. $a_{n}=n-3$ C. $a_{n}=-n+3$ D. $a_{n}=3n-5$

$(a_{n}),$





{\it Egzamin maturalny z matematyki}

{\it Poziom podstawowy}

{\it 5}

BRUDNOPIS





{\it 6}

{\it Egzamin maturalny z matematyki}

{\it Poziom podstawowy}

Zadanie 12. $(1pkt)$

$\mathrm{J}\mathrm{e}\dot{\mathrm{z}}$ eli trójkąty $ABC \mathrm{i} A'B'C'$ są podobne, a ich pola $\mathrm{S}i\mathrm{L}$ odpowiednio, równe 25 $\mathrm{c}\mathrm{m}^{2}$

$\mathrm{i}50\mathrm{c}\mathrm{m}^{2}$, to skala podobieństwa $\displaystyle \frac{A'B'}{AB}$ jest równa

A. 2 B. -21 C. $\sqrt{}$2 D. --$\sqrt{}$22

Zadanie 13. $(1pkt)$

Liczby: $x-2$, 6, 12, w podanej kolejności,

geometrycznego. Liczba $x$ jest równa

są trzema

kolejnymi wyrazami

ciągu

A. 0

B. 2

C. 3

D. 5

Zadanie 14. $(1pkt)$

$\mathrm{J}\mathrm{e}\dot{\mathrm{z}}$ eli $\alpha$ jest kątem ostrym oraz $\displaystyle \mathrm{t}\mathrm{g}\alpha=\frac{2}{5}$, to wartość wyrazenia $\displaystyle \frac{3\cos\alpha-2\sin\alpha}{\sin\alpha-5\cos\alpha}$ jest równa

A.

$-\displaystyle \frac{11}{23}$

B.

$\displaystyle \frac{24}{5}$

C.

- -2131

D.

$\displaystyle \frac{5}{24}$

Zadanie 15. (1pkt)

Liczba punktów wspólnych okręgu

współrzędnychjest równa

A. 0 B. 1

o równaniu $(x+2)^{2}+(y-3)^{2}=4$

C. 2 D.

z osiami układu

4

Zadanie 16. $(1pkt)$

Wysokość trapezu równoramiennego o kącie ostrym $60^{\mathrm{o}}$ i ramieniu długości $2\sqrt{3}$ jest równa

A. $\sqrt{3}$ B. 3 C. $2\sqrt{3}$ D. 2

Zadanie 17. $(1pkt)$

Kąt środkowy oparty na iuku, którego diugośćjest równa $\displaystyle \frac{4}{9}$ diugości okręgu, ma miarę

A. $160^{\mathrm{o}}$

B. $80^{\mathrm{o}}$

C. $40^{\mathrm{o}}$

D. $20^{\mathrm{o}}$

Zadanie 18. $(1pkt)$

$\mathrm{O}$ funkcji liniowej $f$ wiadomo, $\dot{\mathrm{z}}\mathrm{e}f(1)=2$. Do wykresu tej funkcji nalez$\mathrm{y}$ punkt $P=(-2,3).$

Wzór funkcji $f$ to

A. $f(x)=-\displaystyle \frac{1}{3}x+\frac{7}{3}$ B. $f(x)=-\displaystyle \frac{1}{2}x+2$ C. $f(x)=-3x+7$ D. $f(x)=-2x+4$

Zadanie 19. $(1pkt)$

$\mathrm{J}\mathrm{e}\dot{\mathrm{z}}$ eli ostrosłup ma 10 krawędzi, to 1iczba ścian bocznychjest równa

A. 5

B. 7

C. 8

D. 10





{\it Egzamin maturalny z matematyki}

{\it Poziom podstawowy}

7

BRUDNOPIS





{\it 8}

{\it Egzamin maturalny z matematyki}

{\it Poziom podstawowy}

Zadanie 20. (1pkt)

Stozek i walec mają takie same podstawy i równe pola powierzchni bocznych. Wtedy

tworząca stozka jest

A. sześć razy dłuzsza od wysokości walca.

B. trzy razy dłuzsza od wysokości walca.

C. dwa razy dłuzsza od wysokości walca.

D. równa wysokości walca.

Zadanie 21. $(1pkt)$

Liczba $(\displaystyle \frac{1}{(\sqrt[3]{729}+\sqrt[4]{256}+2)^{0}})^{-2}$ jest równa

A. $\displaystyle \frac{1}{225}$ B. $\displaystyle \frac{1}{15}$

C. l

D. 15

Zadanie 22. $(1pkt)$

Do wykresu ffinkcji, określonej dla wszystkich liczb rzeczywistych wzorem $y=-2^{x-2}$, nalez$\mathrm{y}$

punkt

A. $A=(1,-2)$ B. $B=(2,-1)$ C. $C=(1,\displaystyle \frac{1}{2})$ D. $D=(4,4)$

Zadanie 23. $(1pkt)$

$\mathrm{J}\mathrm{e}\dot{\mathrm{z}}$ eli $A$ jest zdarzeniem losowym, $\mathrm{a}$

zachodzi równość $P(A)=2\cdot P(A^{\uparrow})$, to

A. $P(A)=\displaystyle \frac{2}{3}$ B. $P(A)=\displaystyle \frac{1}{2}$

A ` -zdarzeniem przeciwnym do zdarzenia A oraz

C. $P(A)=\displaystyle \frac{1}{3}$ D. $P(A)=\displaystyle \frac{1}{6}$

Zadanie 24. (1pkt)

Na ile sposobów mozna wybrać dwóch graczy spośród 10 zawodników?

A. 100 B. 90 C. 45 D.

20

Zadanie 25. $(1pkt)$

Mediana zestawu danych 2, 12, $a$, 10, 5, 3 jest równa 7. Wówczas

A. $a=4$ B. $a=6$ C. $a=7$

D. $a=9$





{\it Egzamin maturalny z matematyki}

{\it Poziom podstawowy}

{\it 9}

BRUDNOPIS





$ 1\theta$

{\it Egzamin maturalny z matematyki}

{\it Poziom podstawowy}

ZADANIA OTWARTE

{\it Rozwiqzania zadań o numerach od 26. do 34. nalezy zapisać}

{\it w wyznaczonych miejscach} $p\theta d$ {\it treściq zadania}.

Zadanie 26. $(2pkt)$

Wykresem funkcji kwadratowej $f(x)=2x^{2}+bx+c$ jest parabola, której wierzchołkiemjest

punkt $W=(4,0)$. Oblicz wartości współczynników $b\mathrm{i}c.$

Odpowied $\acute{\mathrm{z}}$:







Arkusz zawiera informacje prawnie chronione do momentu rozpoczęcia egzaminu.

UZUPELNIA ZDAJACY

KOD PESEL

{\it Miejsce}

{\it na naklejkę}

{\it z kodem}
\begin{center}
\includegraphics[width=21.432mm,height=9.852mm]{./F1_M_PP_M2015_page0_images/image001.eps}

\includegraphics[width=82.092mm,height=9.852mm]{./F1_M_PP_M2015_page0_images/image002.eps}
\end{center}
\fbox{} dysleksja
\begin{center}
\includegraphics[width=204.060mm,height=216.048mm]{./F1_M_PP_M2015_page0_images/image003.eps}
\end{center}
EGZAMIN MATU LNY

Z MATEMATYKI

POZIOM PODSTAWOWY  5 MAJA 20I5

Instrukcja dla zdającego

l. Sprawdzí, czy arkusz egzaminacyjny zawiera 24 strony

(zadania $1-34$). Ewentualny brak zgłoś przewodniczącemu

zespo nadzorującego egzamin.

2. Rozwiązania zadań i odpowiedzi wpisuj w miejscu na to

przeznaczonym.

3. Odpowiedzi do zadań za niętych (l-25) przenieś

na ka ę odpowiedzi, zaznaczając je w części ka $\mathrm{y}$

przeznaczonej dla zdającego. Zamaluj $\blacksquare$ pola do tego

przeznaczone. Błędne zaznaczenie otocz kółkiem \fcircle$\bullet$

i zaznacz właściwe.

4. Pamiętaj, $\dot{\mathrm{z}}\mathrm{e}$ pominięcie argumentacji lub istotnych

obliczeń w rozwiązaniu zadania otwartego (26-34) $\mathrm{m}\mathrm{o}\dot{\mathrm{z}}\mathrm{e}$

spowodować, $\dot{\mathrm{z}}\mathrm{e}$ za to rozwiązanie nie będziesz mógł

dostać pełnej liczby punktów.

5. Pisz czytelnie i $\mathrm{u}\dot{\mathrm{z}}$ aj tvlko długopisu lub -Dióra

z czamym tuszem lub atramentem.

6. Nie uzywaj korektora, a błędne zapisy wyra $\acute{\mathrm{z}}\mathrm{n}\mathrm{i}\mathrm{e}$ prze eśl.

7. Pamiętaj, $\dot{\mathrm{z}}\mathrm{e}$ zapisy w brudnopisie nie będą oceniane.

8. $\mathrm{M}\mathrm{o}\dot{\mathrm{z}}$ esz korzystać z zestawu wzorów matematycznych,

cyrkla i linijki oraz kalkulatora prostego.

9. Na tej stronie oraz na karcie odpowiedzi wpisz swój

numer PESEL i przyklej naklejkę z kodem.

10. Nie wpisuj $\dot{\mathrm{z}}$ adnych znaków w części przeznaczonej dla

egzaminatora.

Godzina rozpoczęcia:

Czas pracy:

170 minut

Liczba punktów

do uzyskania: 50

$\Vert\Vert\Vert\Vert\Vert\Vert\Vert\Vert\Vert\Vert\Vert\Vert\Vert\Vert\Vert\Vert\Vert\Vert\Vert\Vert\Vert\Vert\Vert\Vert|  \mathrm{M}\mathrm{M}\mathrm{A}-\mathrm{P}1_{-}1\mathrm{P}-152$




{\it Egzamin maturalny z matematyki}

{\it Poziom podstawowy}

{\it Wzadaniach od l. do 25. wybierz i zaznacz na karcie odpowiedzi poprawnq odpowiedzí}.

Zadanie l. (lpkt)

Cena pewnego towaru wraz z 7-procentowym podatkiem VAT jest równa 34347 zł. Cena

tego samego towaru wraz z 23-procentowym podatkiem VAT będzie równa

A. 37236 zł

B. 39842, 52 zł

C. 39483 zł

D. 42246, 81 zł

Zadanie 2. $(1pkt)$

Najmniejszą liczbą całkowitą dodatnią spełniającą nierówność $|x+4,5|\geq 6$ jest

A. $x=1$

B. $x=2$

C. $x=3$

D. $x=6$

Zadanie 3. $(1pkt)$

Liczba $2^{\frac{4}{3}}. \sqrt[3]{2^{5}}$ jest równa

A.

$2^{\frac{20}{3}}$

B. 2

C.

2-45

D. $2^{3}$

Zadanie 4. $(1pkt)$

Liczba 2 $\log_{5}10-\log_{5}4$ jest równa

A. 2 B. 1og596

C. $2\log_{5}6$

D. 5

$\mathrm{Z}\mathrm{a}\mathrm{d}\mathrm{a}\mathrm{n}\mathrm{i}\varepsilon 5. (1pkt)$

Zbiór wszystkich liczb rzeczywistych spełniających nierówność $\displaystyle \frac{3}{5}-\frac{2x}{3}\geq\frac{x}{6}$ jest przedziałem

A.

$\displaystyle \langle\frac{9}{15},+\infty)$

B.

$(-\displaystyle \infty,\frac{18}{25}\}$

C.

$\displaystyle \{\frac{1}{30},+\infty)$

D.

(-$\infty$ , -95$\rangle$

Zadanie 6. $(1pkt)$

Dziedziną funkcji $f$ określonej wzorem $f(x)=\displaystyle \frac{x+4}{x^{2}-4x}\mathrm{m}\mathrm{o}\dot{\mathrm{z}}\mathrm{e}$ być zbiór

A. wszystkich liczb rzeczywistych róznych od 0 i od 4.

B. wszystkich liczb rzeczywistych róznych od $-4$ i od 4.

C. wszystkich liczb rzeczywistych róznych od -A i od 0.

D. wszystkich liczb rzeczywistych.

$\mathrm{Z}\mathrm{a}\mathrm{d}\mathrm{a}\mathrm{n}\mathrm{i}\varepsilon 7. (1pkt)$

Rozwiązaniem równania $\displaystyle \frac{2x-4}{3-x}=\frac{4}{3}$ jest liczba

A. $x=0$

B.

$x=\displaystyle \frac{12}{5}$

C. $x=2$

Strona 2 z24

D.

{\it x}$=$ -2151

MMA-IP





{\it Egzamin maturalny z matematyki}

{\it Poziom podstawowy}

{\it BRUDNOPIS} ({\it nie podlega ocenie})

MMA-IP

Strona ll z24





{\it Egzamin maturalny z matematyki}

{\it Poziom podstawowy}

Zadanie $2\not\in. (2pki)$

Wykaz, $\dot{\mathrm{z}}\mathrm{e}$ dla $\mathrm{k}\mathrm{a}\dot{\mathrm{z}}$ dej liczby rzeczywistej $x$ i dla $\mathrm{k}\mathrm{a}\dot{\mathrm{z}}$ dej liczby rzeczywistej $y$ prawdziwajest

nierówność $4x^{2}-8xy+5y^{2}\geq 0.$

Strona 12 z24

MMA-IP





{\it Egzamin maturalny z matematyki}

{\it Poziom podstawowy}

Zadanie 27. $(2pkt)$

Rozwiąz nierówność $2x^{2}-4x\geq x-2.$

Odpowied $\acute{\mathrm{z}}$:
\begin{center}
\includegraphics[width=96.012mm,height=17.784mm]{./F1_M_PP_M2015_page12_images/image001.eps}
\end{center}
Wypelnia

egzaminator

Nr zadania

Maks. liczba kt

2

27.

2

Uzyskana liczba pkt

MMA-IP

Strona 13 z24





{\it Egzamin maturalny z matematyki}

{\it Poziom podstawowy}

Zadanie $2\mathrm{S}. (2pkt)$

Rozwiąz równanie $4x^{3}+4x^{2}-x-1=0.$

Odpowiedzí:

Strona 14 z24

MMA-IP





{\it Egzamin maturalny z matematyki}

{\it Poziom podstawowy}

Zadanie $29_{n}(2pkt)$

Na rysunku przedstawiono wykres funkcji $f.$
\begin{center}
\includegraphics[width=143.052mm,height=110.136mm]{./F1_M_PP_M2015_page14_images/image001.eps}
\end{center}
$y$

5

4

3

2

1

{\it x}

$-4$ -$3  -2$ -$1 0$  1 2  3 4  5 6

$-1$

$-2$

Funkcja $h$ określona jest dla $x\in\langle-3,  5\rangle$ wzorem $h(x)=f(x)+q$, gdzie $q$ jest pewną liczbą

rzeczywistą. Wiemy, $\dot{\mathrm{z}}$ ejednym z miejsc zerowych funkcji $h$ jest liczba $x_{0}=-1.$

a) Wyznacz q.

b) Podaj wszystkie pozostałe miejsca zerowe funkcji h.

Odpowiedzí :
\begin{center}
\includegraphics[width=96.012mm,height=23.676mm]{./F1_M_PP_M2015_page14_images/image002.eps}
\end{center}
Wypelnia

egzaminator

Nr zadania

Maks. liczba kt

28.

2

2

Uzyskana liczba pkt

Strona 15 z24

MMA-IP





{\it Egzamin maturalny z matematyki}

{\it Poziom podstawowy}

Zadanie 30. $(2pki)$

Dany jest skończony ciąg, w którym pierwszy wyraz jest równy 444, a ostatni jest

równy 653. $\mathrm{K}\mathrm{a}\dot{\mathrm{z}}\mathrm{d}\mathrm{y}$ wyraz tego ciągu, począwszy od drugiego, jest o ll większy od wyrazu

bezpośrednio go poprzedzającego. Oblicz sumę wszystkich wyrazów tego ciągu.

Odpowiedzí:

Strona 16 z24

MMA-IP





{\it Egzamin maturalny z matematyki}

{\it Poziom podstawowy}

Zadanie 31. $(2pkt)$

Dany jest okrąg o środku w punkcie $O$. Prosta $KL$ jest styczna do tego okręgu w punkcie $L,$

a środek $O$ tego okręgu lezy na odcinku KM (zob. rysunek). Udowodnij, $\dot{\mathrm{z}}\mathrm{e}$ kąt $KML$ ma

miarę $31^{\mathrm{o}}$
\begin{center}
\includegraphics[width=112.320mm,height=75.180mm]{./F1_M_PP_M2015_page16_images/image001.eps}
\end{center}
{\it L}

{\it M} ?

{\it O}

$28^{\mathrm{o}}$

{\it K}
\begin{center}
\includegraphics[width=96.012mm,height=17.784mm]{./F1_M_PP_M2015_page16_images/image002.eps}
\end{center}
Wypelnia

egzaminator

Nr zadania

Maks. liczba kt

30.

2

31.

2

Uzyskana liczba pkt

MMA-IP

Strona 17 z24





{\it Egzamin maturalny z matematyki}

{\it Poziom podstawowy}

Zadanie 32. $(4pki)$

Wysokość graniastosłupa prawidłowego czworokątnego jest równa 16. Przekątna graniastosłupa

jest nachylona do płaszczyzny jego podstawy pod kątem, którego cosinus jest równy $\displaystyle \frac{3}{5}$. Oblicz

pole powierzchni całkowitej tego graniastosłupa.

Strona 18 z24

MMA-IP





{\it Egzamin maturalny z matematyki}

{\it Poziom podstawowy}

Odpowied $\acute{\mathrm{z}}$:
\begin{center}
\includegraphics[width=82.044mm,height=17.832mm]{./F1_M_PP_M2015_page18_images/image001.eps}
\end{center}
Nr zadania

Wypelnia Maks. liczba kt

egzaminator

Uzyskana liczba pkt

32.

4

MMA-IP

Strona 19 z24





{\it Egzamin maturalny z matematyki}

{\it Poziom podstawowy}

Zadanie 33. (4pkt)

Wśród 115 osób przeprowadzono badania ankietowe, związane z zakupami w pewnym

kiosku. W ponizszej tabeli przedstawiono informacje o tym, ile osób kupiło bilety

tramwajowe ulgowe oraz ile osób kupiło bilety tramwajowe normalne.
\begin{center}
\begin{tabular}{|l|l|}
\hline
\multicolumn{1}{|l|}{$\begin{array}{l}\mbox{Rodzaj kupionych}	\\	\mbox{biletów}	\end{array}$}&	\multicolumn{1}{|l|}{Liczba osób}	\\
\hline
\multicolumn{1}{|l|}{ulgowe}&	\multicolumn{1}{|l|}{$76$}	\\
\hline
\multicolumn{1}{|l|}{normalne}&	\multicolumn{1}{|l|}{$41$}	\\
\hline
\end{tabular}

\end{center}
Uwaga! 27 osób spośród ankietowanych kupiło oba rodzaje bi1etów.

Oblicz prawdopodobieństwo zdarzenia polegającego na tym, $\dot{\mathrm{z}}\mathrm{e}$ osoba losowo wybrana

spośród ankietowanych nie kupiła $\dot{\mathrm{z}}$ adnego biletu. Wynik przedstaw w formie nieskracalnego

ułamka.

Strona 20 z24

MMA-IP





{\it Egzamin maturalny z matematyki}

{\it Poziom podstawowy}

{\it BRUDNOPIS} ({\it nie podlega ocenie})

MMA-IP

Strona 3 z24





{\it Egzamin maturalny z matematyki}

{\it Poziom podstawowy}

Odpowied $\acute{\mathrm{z}}$:
\begin{center}
\includegraphics[width=82.044mm,height=17.832mm]{./F1_M_PP_M2015_page20_images/image001.eps}
\end{center}
Wypelnia

egzaminator

Nr zadania

Maks. liczba kt

33.

4

Uzyskana liczba pkt

MMA-IP

Strona 21 z24





{\it Egzamin maturalny z matematyki}

{\it Poziom podstawowy}

Zadanie 34. $\beta 5pkt$)

Biegacz narciarski Borys wyruszył na trasę biegu o 10 minut pózíniej $\mathrm{n}\mathrm{i}\dot{\mathrm{z}}$ inny zawodnik,

Adam. Metę zawodów, po przebyciu 15-ki1ometrowej trasy biegu, obaj zawodnicy pokona1i

równocześnie. Okazało się, $\dot{\mathrm{z}}\mathrm{e}$ wartość średniej prędkości na całej trasie w przypadku Borysa

była o $4,5 \displaystyle \frac{\mathrm{k}\mathrm{m}}{\mathrm{h}}$ większa $\mathrm{n}\mathrm{i}\dot{\mathrm{z}}$ w przypadku Adama. Oblicz, wjakim czasie Adam pokonał całą

trasę biegu.

Strona 22 z24

MMA-IP





{\it Egzamin maturalny z matematyki}

{\it Poziom podstawowy}

Odpowied $\acute{\mathrm{z}}$:
\begin{center}
\includegraphics[width=82.044mm,height=17.832mm]{./F1_M_PP_M2015_page22_images/image001.eps}
\end{center}
Nr zadania

Wypelnia Maks. liczba kt

egzaminator

Uzyskana liczba pkt

34.

5

MMA-IP

Strona 23 z24





{\it Egzamin maturalny z matematyki}

{\it Poziom podstawowy}

{\it BRUDNOPIS} ({\it nie podlega ocenie})

Strona 24 z24

MMA-IP





{\it Egzamin maturalny z matematyki}

{\it Poziom podstawowy}

Zadanie 8. $(1pkt)$

Miejscem zerowym funkcji liniowej określonej wzorem $f(x)=-\displaystyle \frac{2}{3}x+4$ jest

A. 0

B. 6

C. 4

D. $-6$

ZadanÍe 9. $(1pkt)$

Punkt $M=(\displaystyle \frac{1}{2},3)$

nalezy do

wykresu funkcji

liniowej określonej

wzorem

$f(x)=(3-2a)x+2$. Wtedy

A.

{\it a}$=$- -21

B. $a=2$

C.

{\it a}$=$ -21

D. $a=-2$

Zadanie $l0. (1pkt)$

Na rysunku przedstawiono fragment prostej o równaniu $y=ax+b.$
\begin{center}
\includegraphics[width=125.328mm,height=84.840mm]{./F1_M_PP_M2015_page3_images/image001.eps}
\end{center}
{\it y}

7

6

5

$P=(2,5)$

4  $Q=(5,3)$

3

2

1

{\it x}

$-1$

0

$-1$

1 2 3 4  5 6 7 8  9 1

Współczynnik kierunkowy tej prostej jest równy

A.

{\it a}$=$- -23

B.

{\it a}$=$- -23

C.

{\it a}$=$- -25

D.

{\it a}$=$- -53

Zadanie ll. (lpkt)

$\mathrm{W}$ ciągu arytmetycznym $(a_{n})$ określonym dla

wyrazem tego ciągujest liczba 156?

$n\geq 1$ dane są $a_{1}=-4$

i

$r=2$. Którym

A. 81.

B. 80.

C. 76.

D. 77.

Zadanie 12. (1pkt)

W rosnącym ciągu geometrycznym

$(a_{n})$, określonym dla $n\geq 1$, spełniony jest warunek

$a_{4}=3a_{1}$. Iloraz $q$ tego ciągu jest równy

A.

{\it q}$=$ -31

B.

{\it q}$=$ -$\sqrt{}$313

C. $q=\sqrt[3]{3}$

Strona 4 $\mathrm{z}24$

D. $q=3$

MMA-IP





{\it Egzamin maturalny z matematyki}

{\it Poziom podstawowy}

{\it BRUDNOPIS} ({\it nie podlega ocenie})

MMA-IP

Strona 5 z24





{\it Egzamin maturalny z matematyki}

{\it Poziom podstawowy}

Zadanie 13. (1pkt)

Drabinę o długości 4 metrów oparto o pionowy mur,

w odległości 1,30 m od tego muru (zobacz rysunek).

a jej podstawę umieszczono
\begin{center}
\includegraphics[width=27.072mm,height=53.388mm]{./F1_M_PP_M2015_page5_images/image001.eps}
\end{center}
4m

$\alpha$

1,30 $\mathrm{m}$

Kąt $\alpha$, podjakim ustawiono drabinę, spełnia warunek

A. $0^{\mathrm{o}}<\alpha<30^{\mathrm{o}}$

B. $30^{\mathrm{o}}<\alpha<45^{\mathrm{o}}$

C. $45^{\mathrm{o}}<\alpha<60^{\mathrm{o}}$

D. $60^{\mathrm{o}}<\alpha<90^{\mathrm{o}}$

Zadanie 14. $(1pkt)$

Kąt $\alpha$jest ostry i $\displaystyle \sin\alpha=\frac{2}{5}$. Wówczas $\cos\alpha$ jest równy

A. -25 B. --$\sqrt{}$421 C. -53

D.

$\displaystyle \frac{\sqrt{21}}{5}$

Zadanie $15_{\mathfrak{v}}(1pkt)$

$\mathrm{W}$ trójkącie równoramiennym $ABC$ spełnione są warunki: $|AC|=|BC|, |\neq CAB|=50^{\mathrm{o}}$

Odcinek $BD$ jest dwusieczną kąta $ABC$, a odcinek $BE$ jest wysokoŚcią opuszczoną

z wierzchołka $B$ na bok $AC$. Miara kąta $EBD$ jest równa
\begin{center}
\includegraphics[width=136.200mm,height=91.392mm]{./F1_M_PP_M2015_page5_images/image002.eps}
\end{center}
{\it C}

{\it E}

{\it D}

?

$50^{\mathrm{o}}$

{\it A  B}

B. 12, $5^{\mathrm{o}}$

A. $10^{\mathrm{o}}$

C. 13, $5^{\mathrm{o}}$

D. $15^{\mathrm{o}}$

Strona 6 z24

MMA-IP





{\it Egzamin maturalny z matematyki}

{\it Poziom podstawowy}

{\it BRUDNOPIS} ({\it nie podlega ocenie})

MMA-IP

Strona 7 z24





{\it Egzamin maturalny z matematyki}

{\it Poziom podstawowy}

Zadanie $l\not\in. (1pki)$

Przedstawione na rysunku trójkąty są podobne.
\begin{center}
\includegraphics[width=60.912mm,height=31.092mm]{./F1_M_PP_M2015_page7_images/image001.eps}
\end{center}
{\it a}

4

$\alpha  \beta$
\begin{center}
\includegraphics[width=121.416mm,height=61.980mm]{./F1_M_PP_M2015_page7_images/image002.eps}
\end{center}
{\it b}

$\alpha  \beta$

6

15

12

Wówczas

A. $a=13, b=17$

B. $a=10, b=18$

C. $a=9, b=19$

D. $a=11, b=13$

Zadauie 17. $(1pkt)$

Proste o równaniach: $y=2mx-m^{2}-1$ oraz $y=4m^{2}x+m^{2}+1$ są prostopadłe dla

A. {\it m}$=$--21 B. {\it m}$=$-21 C. {\it m}$=$1 D. {\it m}$=$2

Zadanie 18. (1pkt)

Dane są punkty $M=(3,-5)$ oraz $N=(-1,7)$. Prosta przechodząca przez te punkty ma

równanie

A. $y=-3x+4$

B. $y=3x-4$

C.

$y=-\displaystyle \frac{1}{3}x+4$

D. $y=3x+4$

ZadaBie 19. $(1pkt)$

Dane są punkty: $P=(-2,-2), Q=(3$, 3$)$. Odległość punktu $P$ od punktu $Q$ jest równa

A. I B. 5 C. $5\sqrt{2}$ D. $2\sqrt{5}$

Zadanie 20. $(1pkt)$

Punkt $K=(-4,4)$ jest końcem odcinka $KL$, punkt $L$ lezy na osi $Ox$, a środek $S$ tego odcinka

lezy na osi $Oy$. Wynika stąd, $\dot{\mathrm{z}}\mathrm{e}$

A. $S=(0,2)$

B. $S=(-2,0)$

C. $S=(4,0)$

D. $S=(0,4)$

Strona 8 z24

MMA-IP





{\it Egzamin maturalny z matematyki}

{\it Poziom podstawowy}

{\it BRUDNOPIS} ({\it nie podlega ocenie})

MMA-IP

Strona 9 z24





{\it Egzamin maturalny z matematyki}

{\it Poziom podstawowy}

Zadanie 21. $(1pki)$

Okrąg przedstawiony na rysunku ma środek w punkcie $O=(3,1)$ i przechodzi przez punkty

$S=(0,4)\mathrm{i}T=(0,-2)$. Okrąg tenjest opisany przez równanie
\begin{center}
\includegraphics[width=99.420mm,height=89.460mm]{./F1_M_PP_M2015_page9_images/image001.eps}
\end{center}
{\it y}

6

5

4 {\it S}

3

2

1

{\it O}

{\it x}

0

1

1 2  3 4 5 6  8

$-2$  {\it T}

A. $(x+3)^{2}+(y+1)^{2}=18$

B. $(x-3)^{2}+(y+1)^{2}=18$

C. $(x-3)^{2}+(y-1)^{2}=18$

D. $(x+3)^{2}+(y-1)^{2}=18$

Zadanie 22. (1pkt)

Przekątna ściany sześcianu ma długość 2. Po1e powierzchni całkowitej tego sześcianu jest

równe

A. 24

B. $12\sqrt{2}$

C. 12

D. $16\sqrt{2}$

Zadanie 23. $(1pkt)$

Kula o promieniu 5 cm i stozek o promieniu podstawy

Wysokość stozkajest równa

A. $\displaystyle \frac{25}{\pi}$ cm B. $10\mathrm{c}\mathrm{m}$ C. $\displaystyle \frac{10}{\pi}$ cm

10 cm mają równe objętości.

D. 5 cm

Zadanie 24. (1pki)

Średnia arytmetyczna zestawu danych:

2, 4, 7, 8, 9

jest taka sama jak średnia arytmetyczna zestawu danych:

2, 4, 7, 8, 9, $x.$

Wynika stąd, $\dot{\mathrm{z}}\mathrm{e}$

A. $x=0$

B. $x=3$

C. $x=5$

D. $x=6$

Zadanie $25_{\mathfrak{v}}(1pkt)$

$\mathrm{W}$ pewnej klasie stosunek liczby dziewcząt do liczby chłopców jest równy 4: 5. Losujemy

jedną osobę z tej klasy. Prawdopodobieństwo tego, $\dot{\mathrm{z}}\mathrm{e}$ będzie to dziewczyna, jest równe

A. -45 B. -49 C. -41 D. -91 MMA-1P

Strona 10 $\mathrm{z}24$







Arkusz zawiera informacje prawnie chronione do momentu rozpoczęcia egzaminu.

UZUPELNIA ZDAJACY

KOD PESEL

{\it miejsce}

{\it na naklejkę}
\begin{center}
\includegraphics[width=21.432mm,height=9.852mm]{./F1_M_PP_M2016_page0_images/image001.eps}

\includegraphics[width=82.092mm,height=9.852mm]{./F1_M_PP_M2016_page0_images/image002.eps}
\end{center}
\square  dyskalkulia

\fbox{} dysleksja
\begin{center}
\includegraphics[width=204.060mm,height=216.048mm]{./F1_M_PP_M2016_page0_images/image003.eps}
\end{center}
EGZAMIN MATU

Z MATEMATY

LNY

POZIOM PODSTAWOWY  5 MAJA 20I

Instrukcja dla zdającego

l. Sprawdzí, czy arkusz egzaminacyjny zawiera 24 strony

(zadania $1-34$). Ewentualny brak zgłoś przewodniczącemu

zespo nadzorującego egzamin.

2. Rozwiązania zadań i odpowiedzi wpisuj w miejscu na to

przeznaczonym.

3. Odpowiedzi do zadań zamkniętych $(1-25)$ zaznacz

na karcie odpowiedzi, w części ka $\mathrm{y}$ przeznaczonej dla

zdającego. Zamaluj $\blacksquare$ pola do tego przeznaczone. Błędne

zaznaczenie otocz kółkiem $\mathrm{O}$ i zaznacz właściwe.

4. Pamiętaj, $\dot{\mathrm{z}}\mathrm{e}$ pominięcie argumentacji lub istotnych

obliczeń w rozwiązaniu zadania otwa ego (26-34) $\mathrm{m}\mathrm{o}\dot{\mathrm{z}}\mathrm{e}$

spowodować, $\dot{\mathrm{z}}\mathrm{e}$ za to rozwiązanie nie będziesz mógł

dostać pełnej liczby punktów.

5. Pisz czytelnie i $\mathrm{u}\dot{\mathrm{z}}$ aj tylko $\mathrm{d}$ gopisu lub pióra

z czarnym tuszem lub atramentem.

6. Nie $\mathrm{u}\dot{\mathrm{z}}$ aj korektora, a błędne zapisy wyra $\acute{\mathrm{z}}\mathrm{n}\mathrm{i}\mathrm{e}$ prze eśl.

7. Pamiętaj, $\dot{\mathrm{z}}\mathrm{e}$ zapisy w brudnopisie nie będą oceniane.

8. $\mathrm{M}\mathrm{o}\dot{\mathrm{z}}$ esz korzystać z zestawu wzorów matematycznych,

cyrkla i linijki oraz kalkulatora prostego.

9. Na tej stronie oraz na karcie odpowiedzi wpisz swój

numer PESEL i przyklej naklejkę z kodem.

10. Nie wpisuj $\dot{\mathrm{z}}$ adnych znaków w części przeznaczonej dla

egzaminatora.

Godzina rozpoczęcia:

Czas pracy:

170 minut

Liczba punktów

do uzyskania: 50

$\Vert\Vert\Vert\Vert\Vert\Vert\Vert\Vert\Vert\Vert\Vert\Vert\Vert\Vert\Vert\Vert\Vert\Vert\Vert\Vert\Vert\Vert\Vert\Vert|  \mathrm{M}\mathrm{M}\mathrm{A}-\mathrm{P}1_{-}1\mathrm{P}-162$




{\it Egzamin maturalny z matematyki}

{\it Poziom podstawowy}

ZADANIA ZAMKNIĘTE

{\it Wzadaniach od l. do 25. wybierz i zaznacz na karcie odpowiedzipoprawnq} $odp\theta wied\acute{z}.$

Zadanie l, (l pkţ)

Dla $\mathrm{k}\mathrm{a}\dot{\mathrm{z}}$ dej dodatniej liczby $a$ iloraz $\displaystyle \frac{a^{-2,6}}{a^{1,3}}$ jest równy

A.

$a^{-3,9}$

B.

$a^{-2}$

C.

$a^{-1,3}$

D.

$a^{1,3}$

Zadanie 2. $(1pkt)$

Liczba $\log_{\sqrt{2}}(2\sqrt{2})$ jest równa

A.

-23

B. 2

C.

-25

D. 3

Zadanie 3. $(1pkt)$

Liczby $a\mathrm{i}c$ są dodatnie. Liczba $b$ stanowi 48\% 1iczby $a$ oraz 32\% 1iczby $c$. Wynika stąd, $\dot{\mathrm{z}}\mathrm{e}$

A. $c=1,5a$

B. $c=1,6a$

C. $c=0,8a$

D. $c=0,16a$

ZadanÍe 4. $(1pkt)$

RównoŚć $(2\sqrt{2}-a)^{2}=17-12\sqrt{2}$ jest prawdziwa dla

A. $a=3$

B. $a=1$

C. $a=-2$

D. $a=-3$

Zadanie 5. $(1pktJ$

Jedną z liczb, które spełniają nierówność $-x^{5}+x^{3}-x<-2$, jest

A. l

B. $-1$

C. 2

D. $-2$

Zadanie $\epsilon. (1pkt)$

Proste o równaniach $2x-3y=4\mathrm{i}5x-6y=7$ przecinają się w punkcie $P$. Stąd wynika, $\dot{\mathrm{z}}\mathrm{e}$

A. $P=(1,2)$

B. $P=(-1,2)$

C. $P=(-1,-2)$

D. $P=(1,-2)$

ZadanÍe 7. (1pkt)

Punkty ABCD $\mathrm{l}\mathrm{e}\dot{\mathrm{z}}$ ą na o ęgu o środku $S$ (zobacz

Miara kąta $BDC$ jest równa

A. $91^{\mathrm{o}}$

sunek).

B. $72,5^{\mathrm{o}}$
\begin{center}
\includegraphics[width=78.132mm,height=79.452mm]{./F1_M_PP_M2016_page1_images/image001.eps}
\end{center}
{\it D}

{\it C}

$27^{\mathrm{o}}$

{\it S}

18

{\it B}

{\it A}

Strona 2 z24

D. $32^{\mathrm{o}}$

C. $18^{\mathrm{o}}$

MMA-IP





{\it Egzamin maturalny z matematyki}

{\it Poziom podstawowy}

{\it BRUDNOPIS} ({\it nie podlega ocenie})

MMA-IP

Strona ll z24





{\it Egzamin maturalny z matematyki}

{\it Poziom podstawowy}

ZADANIA OTWARTE

{\it Rozwiqzania zadań o numerach od 26. do 34. nalezy zapisać w wyznaczonych miejscach pod treściq}

{\it zadania}.

Zadanie 26. (2pkt)

Rozwiąz nierównoŚć $2x^{2}+5x-3>0.$

Odpowied $\acute{\mathrm{z}}$:

Strona 12 $\mathrm{z}24$

MMA-IP





{\it Egzamin maturalny z matematyki}

{\it Poziom podstawowy}

Zadanie 27. (2pkt)

Rozwiąz równanie $x^{3}+3x^{2}+2x+6=0.$

Odpowied $\acute{\mathrm{z}}$:
\begin{center}
\includegraphics[width=96.012mm,height=17.832mm]{./F1_M_PP_M2016_page12_images/image001.eps}
\end{center}
Wypelnia

egzaminator

Nr zadania

Maks. liczba kt

2

27.

2

Uzyskana liczba pkt

MMA-IP

Strona 13 z24





{\it Egzamin maturalny z matematyki}

{\it Poziom podstawowy}

Zadanie 28. (2pktJ

Kąt $\alpha$ jest ostry $\displaystyle \mathrm{i}(\sin\alpha+\cos\alpha)^{2}=\frac{3}{2}$. Oblicz wartość wyrazenia $\sin\alpha\cdot\cos\alpha.$

Odpowiedzí :

Strona 14 z24

MMA-IP





{\it Egzamin maturalny z matematyki}

{\it Poziom podstawowy}

Zadanie 29. (2pkt)

Dany jest trójkąt prostokątny $ABC$. Na przyprostokątnych $AC\mathrm{i}$ AB tego trójkąta obrano

odpowiednio punkty $D\mathrm{i}G$. Na przeciwprostokątnej $BC$ wyznaczono punkty $E\mathrm{i}F$ takie, $\dot{\mathrm{z}}\mathrm{e}$

$|\wedge DEC|=|\triangleleft BGF|=90^{\mathrm{o}}$ (zobacz rysunek). Wykaz, $\dot{\mathrm{z}}\mathrm{e}$ trójkąt $CDE$ jest podobny do

trójkąta $FBG.$
\begin{center}
\includegraphics[width=87.528mm,height=55.476mm]{./F1_M_PP_M2016_page14_images/image001.eps}
\end{center}
{\it C}

{\it E}

{\it F}

{\it D}

{\it A  G B}
\begin{center}
\includegraphics[width=96.012mm,height=17.784mm]{./F1_M_PP_M2016_page14_images/image002.eps}
\end{center}
Wypelnia

egzaminator

Nr zadania

Maks. liczba kt

28.

2

2

Uzyskana liczba pkt

MMA-IP

Strona 15 z24





{\it Egzamin maturalny z matematyki}

{\it Poziom podstawowy}

Zadanie 30. (2pktJ

Ciąg $(a_{n})$ jest określony wzorem $a_{n}=2n^{2}+2n$ dla $n\geq 1$. Wykaz, $\dot{\mathrm{z}}\mathrm{e}$ suma $\mathrm{k}\mathrm{a}\dot{\mathrm{z}}$ dych dwóch

kolejnych wyrazów tego ciągu jest kwadratem liczby naturalnej.

Strona 16 z24

MMA-IP





{\it Egzamin maturalny z matematyki}

{\it Poziom podstawowy}

{\it Zadanie 3l}. ({\it 2pktJ}

$\mathrm{W}$ skończonym ciągu arytmetycznym $(a_{n})$ pierwszy wyraz $a_{1}$ jest równy 7 oraz ostatni

wyraz $a_{n}$ jest równy 89. Suma wszystkich wyrazów tego ciągujest równa 2016.

Oblicz, ile wyrazów ma ten ciąg.

Odpowied $\acute{\mathrm{z}}$:
\begin{center}
\includegraphics[width=96.012mm,height=17.832mm]{./F1_M_PP_M2016_page16_images/image001.eps}
\end{center}
Wypelnia

egzaminator

Nr zadania

Maks. liczba kt

30.

2

31.

2

Uzyskana liczba pkt

MMA-IP

Strona 17 z24





{\it Egzamin maturalny z matematyki}

{\it Poziom podstawowy}

Zadanie 32. (4pktJ

Jeden z kątów trójkąta jest trzy razy większy od mniejszego z dwóch pozostałych kątów,

które róznią się o $50^{\mathrm{o}}$. Oblicz kąty tego trójkąta.

Strona 18 z24

MMA-IP





{\it Egzamin maturalny z matematyki}

{\it Poziom podstawowy}

Odpowiedzí :
\begin{center}
\includegraphics[width=82.044mm,height=17.784mm]{./F1_M_PP_M2016_page18_images/image001.eps}
\end{center}
Nr zadanÍa

WypelnÍa Maks. liczba kt

egzaminator

Uzyskana liczba pkt

32.

4

MMA-IP

Strona 19 z24





{\it Egzamin maturalny z matematyki}

{\it Poziom podstawowy}

Zadanie 33. (5pktJ

Grupa znajomych wyjez $\mathrm{d}\dot{\mathrm{z}}$ ających na biwak wynajęła bus. Koszt wynajęcia busa jest równy

960 złotych i tę kwotę rozłozono po równo pomiędzy uczestników wyjazdu. Do grupy

wyjez $\mathrm{d}\dot{\mathrm{z}}$ ających dołączyło w ostatniej chwili dwóch znajomych. Wtedy koszt wyjazdu

przypadający na jednego uczestnika zmniejszył się o 16 złotych. Ob1icz, i1e osób wyjechało

na biwak.

Strona 20 z24

MMA-IP





{\it Egzamin maturalny z matematyki}

{\it Poziom podstawowy}

{\it BRUDNOPIS} ({\it nie podlega ocenie})

MMA-IP





{\it Egzamin maturalny z matematyki}

{\it Poziom podstawowy}
\begin{center}
\includegraphics[width=82.044mm,height=17.832mm]{./F1_M_PP_M2016_page20_images/image001.eps}
\end{center}
WypelnÍa

egzaminator

Nr zadanÍa

Maks. liczba kt

33.

5

Uzyskana liczba pkt

MMA-IP

Strona 21 z24





{\it Egzamin maturalny z matematyki}

{\it Poziom podstawowy}

Zadanie 34. (4pktJ

Ze zbioru wszystkich liczb naturalnych dwucyfrowych losujemy kolejno dwa razy po jednej

liczbie bez zwracania. Oblicz prawdopodobieństwo zdarzenia polegającego na tym, $\dot{\mathrm{z}}\mathrm{e}$ suma

wylosowanych liczb będzie równa 30. Wynik zapisz w postaci ułamka zwykłego

nieskracalnego.

Strona 22 z24

MMA-IP





{\it Egzamin maturalny z matematyki}

{\it Poziom podstawowy}

Odpowiedzí :
\begin{center}
\includegraphics[width=82.044mm,height=17.832mm]{./F1_M_PP_M2016_page22_images/image001.eps}
\end{center}
Wypelnia

egzaminator

Nr zadania

Maks. liczba kt

34.

4

Uzyskana liczba pkt

MMA-IP

Strona 23 z24





{\it Egzamin maturalny z matematyki}

{\it Poziom podstawowy}

{\it BRUDNOPIS} ({\it nie podlega ocenie})

Strona 24 z24

MMA-IP





{\it Egzamin maturalny z matematyki}

{\it Poziom podstawowy}

Zadam$\mathrm{e}8. (1pkt)$

Danajest ffinkcja liniowa $f(x)=\displaystyle \frac{3}{4}x+6$. Miejscem zerowym tej funkcjijest liczba

A. 8

B. 6

C. $-6$

D. $-8$

Zadanie $g. (1pktJ$

Równanie wymierne $\displaystyle \frac{3x-1}{x+5}=3$, gdzie $x\neq-5,$

A.

B.

C.

D.

nie ma rozwiązań rzeczywistych.

ma dokładniejedno rozwiązanie rzeczywiste.

ma dokładnie dwa rozwiązania rzeczywiste.

ma dokładnie trzy rozwiązania rzeczywiste.

Informacja do zadań 10. $\mathrm{i}l1.$

Na rysunku przedstawiony jest fragment paraboli będącej wykresem funkcji kwadratowej $f.$

Wierzchołkiem tej parabolijest punkt $W=(1,9)$. Liczby $-2\mathrm{i}4$ to miejsca zerowe funkcji $f.$
\begin{center}
\includegraphics[width=192.228mm,height=118.104mm]{./F1_M_PP_M2016_page3_images/image001.eps}
\end{center}
Zadanie 10. (1pkt)

Zbiorem wartości funkcji f jest przedział

A.

$(-\infty'-2\rangle$

B. $\langle-2,  4\rangle$

C.

$\langle 4,+\infty)$

D. $(-\infty$' $ 9\rangle$

Zadanie $ll. (1pkt)$

Najmniejsza wartość funkcji $f$ w przedziale $\langle-1,2\rangle$ jest równa

A. 2

B. 5

C. 8

D. 9

Strona 4 z24

MMA-IP





{\it Egzamin maturalny z matematyki}

{\it Poziom podstawowy}

{\it BRUDNOPIS} ({\it nie podlega ocenie})

MMA-IP

Strona 5 z24





{\it Egzamin maturalny z matematyki}

{\it Poziom podstawowy}

Zadanie $l2. (1pkt)$

Funkcja $f$ określona jest wzorem $f(x)=\displaystyle \frac{2x^{3}}{x^{6}+1}$ dla kazdej liczby rzeczywistej $x$. Wtedy

$f(-\sqrt[3]{3})$ jest równa

A.

$-\displaystyle \frac{\sqrt[3]{9}}{2}$

B.

- -53

C.

-53

D.

$\displaystyle \frac{\sqrt[3]{3}}{2}$

Zadanie 13. $(1pktJ$

$\mathrm{W}$ okręgu o środku w punkcie $S$ poprowadzono cięciwę AB, która utworzyła z promieniem

$AS$ kąt o mierze $31^{\mathrm{o}}$ (zobacz rysunek). Promień tego okręgu ma długość 10. Od1egłość punktu

$S$ od cięciwy $AB$ jest liczbą z przedziału

A. $\displaystyle \{\frac{9}{2},\frac{11}{2}\}$

B. $\displaystyle \frac{11}{2}, \displaystyle \frac{13}{2}$

C. $\displaystyle \frac{13}{2}, \displaystyle \frac{19}{2}$
\begin{center}
\includegraphics[width=72.588mm,height=76.200mm]{./F1_M_PP_M2016_page5_images/image001.eps}
\end{center}
$B$

{\it K}

{\it S}

31

{\it A}

$\displaystyle \frac{19}{2}, \displaystyle \frac{37}{2}$

D.

Zadanie 14. $(1pkt)$

Czternasty wyraz ciągu arytmetycznegojest równy 8, a róznica tego ciągujest równa $(-\displaystyle \frac{3}{2}).$

Siódmy wyraz tego ciągu jest równy

A.

$\displaystyle \frac{37}{2}$

B.

$-\displaystyle \frac{37}{2}$

C.

- -25

D.

-25

Zadanie 15. $(1pki)$

Ciąg $(x,2x+3,4x+3)$ jest geometryczny. Pierwszy wyraz tego ciągujest równy

A. $-4$

B. l

C. 0

D. $-1$

Zadanie 16. (1pkt)

Przedstawione na rysunku trójkąty ABCi PQR są podobne. Bok AB trójkąta ABC ma długość

A. 8

B. 8,5

C. 9,5
\begin{center}
\includegraphics[width=105.660mm,height=60.504mm]{./F1_M_PP_M2016_page5_images/image002.eps}
\end{center}
18

{\it Q}  $62^{\mathrm{o}}$  {\it R}

{\it C}

17

9

$70^{\mathrm{o}}$

$70^{\mathrm{o}}  48^{\mathrm{o}}$

{\it A B}

{\it x  P}

D. 10

Strona 6 z24

MMA-IP





{\it Egzamin maturalny z matematyki}

{\it Poziom podstawowy}

{\it BRUDNOPIS} ({\it nie podlega ocenie})

MMA-IP

Strona 7 z24





{\it Egzamin maturalny z matematyki}

{\it Poziom podstawowy}

Zadanie 17. $(1pkt)$

Kąt $\alpha$ jest ostry i $\displaystyle \mathrm{t}\mathrm{g}\alpha=\frac{2}{3}$. Wtedy

A.

$\displaystyle \sin\alpha=\frac{3\sqrt{13}}{26}$

B.

$\displaystyle \sin\alpha=\frac{\sqrt{13}}{13}$

C.

$\mathrm{s}$i$\displaystyle \mathrm{n}\alpha=\frac{2\sqrt{13}}{13}$

D.

$\displaystyle \sin\alpha=\frac{3\sqrt{13}}{13}$

Zadanie $l\mathrm{S}. (1pkt)$

$\mathrm{Z}$ odcinków o długościach: 5, $2a+1, a-1$ mozna zbudować trójkąt równoramienny. Wynika

stąd, $\dot{\mathrm{z}}\mathrm{e}$

A. $a=6$

B. $a=4$

C. $a=3$

D. $a=2$

Zadanie 19. (1pRt)

Okręgi o promieniach 3 i 4 są styczne zewnętrznie. Prosta styczna do okręgu

o promieniu 4 w punkcie P przechodzi przez środek okręgu o promieniu 3 (zobacz rysunek).
\begin{center}
\includegraphics[width=171.504mm,height=116.184mm]{./F1_M_PP_M2016_page7_images/image001.eps}
\end{center}
{\it P}

$O_{1}$  3 4  $O_{2}$

Pole trójkąta, którego wierzchołkami są środki okręgów i punkt styczności P, jest równe

A. 14

B. $2\sqrt{33}$

C. $4\sqrt{33}$

D. 12

Zadanie 20. $(1pkt)$

Proste opisane równaniami $y=\displaystyle \frac{2}{m-1}x+m-2$ oraz $y=mx+\displaystyle \frac{1}{m+1}$ są prostopadłe, gdy

A. $m=2$

B.

{\it m}$=$ -21

C.

{\it m}$=$ -31

D. $m=-2$

Strona 8 z24

MMA-IP





{\it Egzamin maturalny z matematyki}

{\it Poziom podstawowy}

{\it BRUDNOPIS} ({\it nie podlega ocenie})

MMA-IP

Strona 9 z24





{\it Egzamin maturalny z matematyki}

{\it Poziom podstawowy}

Zadanie $2l. (1pkt)$

$\mathrm{W}$ układzie współrzędnych dane są punkty $A=(a,6)$ oraz $B=(7,b)$. Środkiem odcinka $AB$

jest punkt $M=(3,4)$. Wynika stąd, $\dot{\mathrm{z}}\mathrm{e}$

A. $a=5 \mathrm{i}b=5$

B. $a=-1 \mathrm{i}b=2$

C. $a=4\mathrm{i}b=10$

D. $a=-4 \mathrm{i}b=-2$

Zadanie 22. (Ipkt)

Rzucamy trzy razy symetryczną monetą. Niech p oznacza prawdopodobieństwo otrzymania

dokładnie dwóch orłów w tych trzech rzutach. Wtedy

A. $0\leq p<0,2$

B. $0,2\leq p\leq 0,35$

C. $0,35<p\leq 0,5$

D. $0,5<p\leq 1$

Zadanie 23. $(1pki)$

Kąt rozwarcia stozka ma miarę $120^{\mathrm{o}}$, a tworząca tego stozka ma długość 4. Objętość tego

stozkajest równa

A. $ 36\pi$

B. $ 18\pi$

C. $ 24\pi$

D. $ 8\pi$

Zadanie 24. (1pki)

Przekątna podstawy graniastosłupa prawidłowego czworokątnego jest dwa razy dłuzsza od

wysokości graniastosłupa. Graniastosłup przecięto płaszczyzną przechodzącą przez przekątną

podstawy ijeden wierzchołek drugiej podstawy (patrz rysunek).

Płaszczyzna przekroju tworzy z podstawą graniastosłupa kąt $\alpha$ o mierze

A. $30^{\mathrm{o}}$

B. $45^{\mathrm{o}}$

C. $60^{\mathrm{o}}$

D. $75^{\mathrm{o}}$

Zadanie 25. $(1pki)$

Średnia arytmetyczna sześciu liczb naturalnych: 31, 16, 25, 29, 27, $x$, jest równa $\displaystyle \frac{x}{2}$. Mediana

tych liczb jest równa

A. 26

B. 27

C. 28

D. 29

Strona 10 z24

MMA-IP







Arkusz zawiera informacje prawnie chronione do momentu rozpoczęcia egzaminu.

UZUPELNIA ZDAJACY

KOD PESEL

{\it miejsce}

{\it na naklejkę}
\begin{center}
\includegraphics[width=21.432mm,height=9.852mm]{./F1_M_PP_M2017_page0_images/image001.eps}

\includegraphics[width=82.092mm,height=9.852mm]{./F1_M_PP_M2017_page0_images/image002.eps}

\includegraphics[width=204.060mm,height=216.048mm]{./F1_M_PP_M2017_page0_images/image003.eps}
\end{center}
EGZAMIN MATU

Z MATEMATY

LNY

POZIOM PODSTAWOWY

Instrukcja dla zdającego

l. Sprawdzí, czy arkusz egzaminacyjny zawiera 26 stron

(zadania $1-34$). Ewentualny brak zgłoś przewodniczącemu

zespo nadzorującego egzamin.

2. Rozwiązania zadań i odpowiedzi wpisuj w miejscu na to

przeznaczonym.

3. Odpowiedzi do zadań zamkniętych $(1-25)$ zaznacz

na karcie odpowiedzi, w części ka $\mathrm{y}$ przeznaczonej dla

zdającego. Zamaluj $\blacksquare$ pola do tego przeznaczone. Błędne

zaznaczenie otocz kółkiem $\mathrm{O}$ i zaznacz właściwe.

4. Pamiętaj, $\dot{\mathrm{z}}\mathrm{e}$ pominięcie argumentacji lub istotnych

obliczeń w rozwiązaniu zadania otwa ego (26-34) $\mathrm{m}\mathrm{o}\dot{\mathrm{z}}\mathrm{e}$

spowodować, $\dot{\mathrm{z}}\mathrm{e}$ za to rozwiązanie nie otrzymasz pełnej

liczby punktów.

5. Pisz czytelnie i $\mathrm{u}\dot{\mathrm{z}}$ aj tylko $\mathrm{d}$ gopisu lub pióra

z czarnym tuszem lub atramentem.

6. Nie $\mathrm{u}\dot{\mathrm{z}}$ aj korektora, a błędne zapisy wyra $\acute{\mathrm{z}}\mathrm{n}\mathrm{i}\mathrm{e}$ prze eśl.

7. Pamiętaj, $\dot{\mathrm{z}}\mathrm{e}$ zapisy w brudnopisie nie będą oceniane.

8. $\mathrm{M}\mathrm{o}\dot{\mathrm{z}}$ esz korzystać z zestawu wzorów matematycznych,

cyrkla i linijki oraz kalkulatora prostego.

9. Na tej stronie oraz na karcie odpowiedzi wpisz swój

numer PESEL i przyklej naklejkę z kodem.

10. Nie wpisuj $\dot{\mathrm{z}}$ adnych znaków w części przeznaczonej dla

egzaminatora.

5 MAJA 20I7

Godzina rozpoczęcia:

9:00

Czas pracy:

170 minut

Liczba punktów

do uzyskania: 50

$\Vert\Vert\Vert\Vert\Vert\Vert\Vert\Vert\Vert\Vert\Vert\Vert\Vert\Vert\Vert\Vert\Vert\Vert\Vert\Vert\Vert\Vert\Vert\Vert|  \mathrm{M}\mathrm{M}\mathrm{A}-\mathrm{P}1_{-}1\mathrm{P}-172$




{\it Egzamin maturalny z matematyki}

{\it Poziom podstawowy}

ZADANIA ZAMKNIĘTE

{\it Wzadaniach od l. do 25. wybierz i zaznacz na karcie odpowiedzipoprawnq} $odp\theta wied\acute{z}.$

Zadanie l. $(1pktJ$

Liczba $5^{8}.16^{-2}$ jest równa

A. $(\displaystyle \frac{5}{2})^{8}$ B.

-25

Zadanie 2. $(1pktJ$

Liczba $\sqrt[3]{54}-\sqrt[3]{2}$ jest równa

A. $\sqrt[3]{52}$

B. 3

Zadanie 3. $(1pktJ$

Liczba 2 $\log_{2}3-2\log_{2}5$ jest równa

A.

$\displaystyle \log_{2}\frac{9}{25}$

B.

$\log_{2} \displaystyle \frac{3}{5}$

C. $10^{8}$

D. 10

C. $\mathrm{z}\sqrt[3]{2}$

D. 2

C.

$\log_{2} \displaystyle \frac{9}{5}$

D.

$\displaystyle \log_{2}\frac{6}{25}$

Zadanie 4. (1pktJ

Liczba osobników pewnego zagrozonego wyginięciem gatunku zwierząt wzrosła w stosunku

do liczby tych zwierząt z 31 grudnia 2011 r. 0120\% i obecnie jest równa 8910. I1e zwierząt

liczyła populacja tego gatunku w ostatnim dniu 2011 roku?

A. 4050

B. 1782

C. 7425

D. 7128

Zadame 5. $(1pkt)$

Równość $(x\sqrt{2}-2)^{2}=(2+\sqrt{2})^{2}$ jest

A. prawdziwa dla $x=-\sqrt{2}.$

B. prawdziwa dla $x=\sqrt{2}.$

C. prawdziwa dla $x=-1.$

D. fałszywa dla $\mathrm{k}\mathrm{a}\dot{\mathrm{z}}$ dej liczby $x.$

Strona 2 z 26

MMA-IP





{\it Egzamin maturalny z matematyki}

{\it Poziom podstawowy}

{\it BRUDNOPIS} ({\it nie podlega ocenie})

MMA-IP

Strona ll z 26





{\it Egzamin maturalny z matematyki}

{\it Poziom podstawowy}

Zadanie 18. $(1pkt)$

Na rysunku przedstawiona jest prosta $k$ o równaniu $y=ax$, przechodząca przez punkt

$A=(2,-3)$ i przez początek układu współrzędnych, oraz zaznaczony jest kąt $\alpha$ nachylenia

tej prostej do osi $Ox.$
\begin{center}
\includegraphics[width=70.716mm,height=67.512mm]{./F1_M_PP_M2017_page11_images/image001.eps}
\end{center}
{\it k}

{\it y}

5

4

3

2

1

$\alpha$

{\it x}

$-5$ -$4  -3$ -$2$

$-1 0$ 1

$-1$

2 3  4 5

$-2$

$-3  -A$

$-4$

Zatem

A.

{\it a}$=$- -23

B.

{\it a}$=$- -23

C.

{\it a}$=$ -23

D.

{\it a}$=$ -23

Zadanie $l9*(1pkt)$

Na płaszczyz$\acute{}$nie z układem współrzędnych proste $k\mathrm{i} l$ przecinają się pod kątem prostym

w punkcie $A=(-2,4)$. Prosta $k$ jest określona równaniem $y=-\displaystyle \frac{1}{4}x+\frac{7}{2}$ Zatem prostą $l$

opisuje równanie

A.

{\it y}$=$ -41 {\it x}$+$ -27

B.

{\it y}$=$- -41 {\it x}- -27

C. $y=4x-12$

D. $y=4x+12$

Zadanie 20. $(1pkt)$

Dany jest okrąg o środku $S=(2,3)$ i promieniu $r=5$. Który z podanych punktów lezy na

tym okręgu?

A. $A=(-1,7)$

B. $B=(2,-3)$

C. $C=(3,2)$

D. $D=(5,3)$

Zadanie 21. (1pkt)

Pole powierzchni całkowitej graniastosiupa prawidłowego czworokątnego, w którym

wysokość jest 3 razy dłuzsza od krawędzi podstawy, jest równe 140. Zatem krawędz$\acute{}$

podstawy tego graniastosłupajest równa

A. $\sqrt{10}$

B.

$3\sqrt{10}$

C. $\sqrt{42}$

D. $3\sqrt{42}$

Strona 12 z 26

MMA-IP





{\it Egzamin maturalny z matematyki}

{\it Poziom podstawowy}

{\it BRUDNOPIS} ({\it nie podlega ocenie})

MMA-IP

Strona 13 z 26





{\it Egzamin maturalny z matematyki}

{\it Poziom podstawowy}

Zadanie 22. (1pkt)

Promień AS podstawy walca jest równy wysokości OS tego walca. Sinus kąta OAS (zobacz

rysunek) jest równy
\begin{center}
\includegraphics[width=49.272mm,height=39.984mm]{./F1_M_PP_M2017_page13_images/image001.eps}
\end{center}
{\it O}

{\it S}

{\it A}

A.

-21

B.

-$\sqrt{}$22

C.

-$\sqrt{}$23

D. l

Zadanie 23. (1pkt)

Dany jest stozek o wysokości 4 i średnicy podstawy 12. Objętość tego stozkajest równa

A. $ 576\pi$

B. $ 192\pi$

C. $ 144\pi$

D. $ 48\pi$

Zadanie 24. (1pkt)

Średnia arytmetyczna ośmiu liczb: 3, 5, 7, 9, x, 15, 17, 19jest równa 11. Wtedy

A. $x=1$

B. $x=2$

C. $x=11$

D. $x=13$

Zadanie 25. $(1pkt)$

Ze zbioru dwudziesm czterech kolejnych liczb naturalnych od l do 241osujemy jedną 1iczbę.

Niech $A$ oznacza zdarzenie, $\dot{\mathrm{z}}\mathrm{e}$ wylosowana liczba będzie dzielnikiem liczby 24. Wtedy

prawdopodobieństwo zdarzenia $A$ jest równe

A.

-41

B.

-31

C.

-81

D.

-61

Strona 14 z26

MMA-IP





{\it Egzamin maturalny z matematyki}

{\it Poziom podstawowy}

{\it BRUDNOPIS} ({\it nie podlega ocenie})

MMA-IP

Strona 15 z 26





{\it Egzamin maturalny z matematyki}

{\it Poziom podstawowy}

Zadanie 26. $(2pktJ$

Rozwiąz nierówność $8x^{2}-72x\leq 0.$

Odpowied $\acute{\mathrm{z}}$:

Strona 16 $\mathrm{z}26$

MMA-IP





{\it Egzamin maturalny z matematyki}

{\it Poziom podstawowy}

Zadanie 27, $(2pktJ$

Wykaz, $\dot{\mathrm{z}}\mathrm{e}$ liczba $4^{2017}+4^{2018}+4^{2019}+4^{2020}$ jest podzielna przez 17.
\begin{center}
\includegraphics[width=96.012mm,height=17.784mm]{./F1_M_PP_M2017_page16_images/image001.eps}
\end{center}
Wypelnia

egzamÍnator

Nr zadania

Maks. liczba kt

2

27.

2

Uzyskana liczba pkt

MMA-IP

Strona 17 z26





{\it Egzamin maturalny z matematyki}

{\it Poziom podstawowy}

Zadanie 2{\$}. $(2pktJ$

Dane są dwa okręgi o środkach w punktach $P \mathrm{i} R$, styczne zewnętrznie w punkcie $C.$

Prosta $AB$ jest styczna do obu okręgów odpowiednio w punktach $A \mathrm{i}B$ oraz $|\triangleleft APC|=\alpha$

$\mathrm{i}|<ABC|=\beta$ (zobacz rysunek). Wykaz, $\dot{\mathrm{z}}\mathrm{e}\alpha=180^{\mathrm{o}}-2\beta.$
\begin{center}
\includegraphics[width=190.908mm,height=44.904mm]{./F1_M_PP_M2017_page17_images/image001.eps}
\end{center}
{\it P}

$\alpha$  {\it C  R}

$(\beta$

{\it A}  -{\it B}

Strona 18 z26

MMA-IP





{\it Egzamin maturalny z matematyki}

{\it Poziom podstawowy}

Zadanie 29. $(4pkt)$

Funkcja kwadratowa $f$ jest określona dla wszystkich liczb rzeczywistych $x$ wzorem

$f(x)=ax^{2}+bx+c$. Największa wartość funkcji $f$ jest równa 6 oraz $f(-6)=f(0)=\displaystyle \frac{3}{2}.$

Oblicz wartość współczynnika $a.$

Odpowiedzí :
\begin{center}
\includegraphics[width=96.012mm,height=17.784mm]{./F1_M_PP_M2017_page18_images/image001.eps}
\end{center}
Wypelnia

egzamÍnator

Nr zadani,

Maks. liczba kt

28.

2

4

Uzyskana liczba pkt

MMA-IP

Strona 19 z26





{\it Egzamin maturalny z matematyki}

{\it Poziom podstawowy}

Zadanie 30. (2pkt)

Przeciwprostokątna trójkąta prostokątnego ma długość 26 cm, a jedna z przyprostokątnych

jest o 14 cm dłuzsza od drugiej. Ob1icz obwód tego trójkąta.

Odpowied $\acute{\mathrm{z}}$:

Strona 20 $\mathrm{z}26$

MMA-IP





{\it Egzamin maturalny z matematyki}

{\it Poziom podstawowy}

{\it BRUDNOPIS} ({\it nie podlega ocenie})

MMA-IP

Strona 3 z 26





{\it Egzamin maturalny z matematyki}

{\it Poziom podstawowy}

Zadanie 31. (2pkt)

$\mathrm{W}$ ciągu arytmetycznym $(a_{n})$, określonym dla $n\geq 1$, dane są: wyraz $a_{1}=8$ i suma trzech

początkowych wyrazów tego ciągu $S_{3}=33$. Oblicz róznicę $a_{16}-a_{13}.$

Odpowiedzí :
\begin{center}
\includegraphics[width=96.012mm,height=17.784mm]{./F1_M_PP_M2017_page20_images/image001.eps}
\end{center}
Wypelnia

egzamÍnator

Nr zadani,

Maks. liczba kt

30.

2

31.

2

Uzyskana liczba pkt

MMA-IP

Strona 21 z26





{\it Egzamin maturalny z matematyki}

{\it Poziom podstawowy}

Zadanie 32. $(SpktJ$

Dane są punkty $A=(-4,0) \mathrm{i}M=(2,9)$ oraz prosta $k$ o równaniu $y=-2x+10$. Wierzchołek

$B$ trójkąta $ABC$ to punkt przecięcia prostej $k$ z osią $Ox$ układu współrzędnych, a wierzchołek

$C$ jest punktem przecięcia prostej $k$ z prostą AM. Oblicz pole trójkąta $ABC.$

Odpowied $\acute{\mathrm{z}}$:

Strona 22 $\mathrm{z}26$

MMA-IP





{\it Egzamin maturalny z matematyki}

{\it Poziom podstawowy}

Zadanie 33. $(2pkt)$

Ze zbioru wszystkich liczb naturalnych dwucyfrowych losujemy jedną liczbę. Oblicz

prawdopodobieństwo zdarzenia, $\dot{\mathrm{z}}\mathrm{e}$ wylosujemy liczbę, która jest równocześnie mniejsza od

40 i podzielna przez 3. Wynik zapisz w postaci ułamka zwykłego nieskracalnego.

Odpowiedzí:
\begin{center}
\includegraphics[width=96.012mm,height=17.784mm]{./F1_M_PP_M2017_page22_images/image001.eps}
\end{center}
Wypelnia

egzamÍnator

Nr zadania

Maks. liczba kt

32.

5

33.

2

Uzyskana liczba pkt

MMA-IP

Strona 23 z 26





{\it Egzamin maturalny z matematyki}

{\it Poziom podstawowy}

Zadanie 34. $(4pktJ$

$\mathrm{W}$ ostrosłupie prawidłowym trójkątnym wysokość ściany bocznej prostopadła do krawędzi

podstawy ostrosłupa jest równa $\displaystyle \frac{5\sqrt{3}}{4}$, a pole powierzchni bocznej tego ostrosłupa jest

równe $\displaystyle \frac{15\sqrt{3}}{4}$. Oblicz objętość tego ostrosłupa.

Strona 24 z26

MMA-IP





{\it Egzamin maturalny z matematyki}

{\it Poziom podstawowy}

Odpowied $\acute{\mathrm{z}}$:
\begin{center}
\includegraphics[width=82.044mm,height=17.832mm]{./F1_M_PP_M2017_page24_images/image001.eps}
\end{center}
Wypelnia

egzaminator

Nr zadanÍa

Maks. lÍczba kt

34.

4

Uzyskana liczba pkt

MMA-IP

Strona 25 z26





{\it Egzamin maturalny z matematyki}

{\it Poziom podstawowy}

{\it BRUDNOPIS} ({\it nie podlega ocenie})

Strona 26 z26

MMA-IP





{\it Egzamin maturalny z matematyki}

{\it Poziom podstawowy}

Zadanie 6. $(1pkt)$

Do zbioru rozwiązań nierówności $(x^{4}+1)(2-x)>0$ nie nalez$\mathrm{v}$ liczba

A. $-3$

B. $-1$

C. l

D. 3

Zadam$\mathrm{e}7. (1pkt)$

Wskaz rysunek, na którym jest przedstawiony zbiór wszystkich rozwiązań nierówności

$2-3x\geq 4.$

A.
\begin{center}
\includegraphics[width=167.940mm,height=17.676mm]{./F1_M_PP_M2017_page3_images/image001.eps}
\end{center}
-23  {\it x}

B.
\begin{center}
\includegraphics[width=167.940mm,height=17.784mm]{./F1_M_PP_M2017_page3_images/image002.eps}
\end{center}
-23  {\it x}

C.
\begin{center}
\includegraphics[width=168.000mm,height=17.832mm]{./F1_M_PP_M2017_page3_images/image003.eps}
\end{center}
- -23  {\it x}

D.
\begin{center}
\includegraphics[width=168.048mm,height=17.832mm]{./F1_M_PP_M2017_page3_images/image004.eps}
\end{center}
- -23  {\it x}

Zadanie $S, (1pktJ$

Równanie $x(x^{2}-4)(x^{2}+4)=0$ z niewiadomą $x$

A. nie ma rozwiązań w zbiorze liczb rzeczywistych.

B. ma dokładnie dwa rozwiązania w zbiorze liczb rzeczywistych.

C. ma dokładnie trzy rozwiązania w zbiorze liczb rzeczywistych.

D. ma dokładnie pięć rozwiązań w zbiorze liczb rzeczywistych.

{\it Zadanie g}. ({\it lpkt})

Miejscem zerowym funkcji liniowej

$f(x)=\sqrt{3}(x+1)-12$ jest liczba

A. $\sqrt{3}-4$

B. $-2\sqrt{3}+1$

C. $4\sqrt{3}-1$

D. $-\sqrt{3}+12$

Strona 4 z 26

MMA-IP





{\it Egzamin maturalny z matematyki}

{\it Poziom podstawowy}

{\it BRUDNOPIS} ({\it nie podlega ocenie})

MMA-IP

Strona 5 z 26





{\it Egzamin maturalny z matematyki}

{\it Poziom podstawowy}

Zadanie 10. $(1pktJ$

Na rysunku przedstawiono fragment wykresu funkcji

o miejscach zerowych: $-3 \mathrm{i}1.$

kwadratowej $f(x)=ax^{2}+bx+c,$
\begin{center}
\includegraphics[width=86.004mm,height=100.380mm]{./F1_M_PP_M2017_page5_images/image001.eps}
\end{center}
{\it 5y}

)4

3

2

1

{\it X}

$\rightarrow 2$

$-4$

$-5$

Współczynnik c we wzorze funkcji f jest równy

A. l

B. 2

C. 3

D. 4

Zadanie ll. $(Ipkt)$

Na rysunku przedstawiono fragment wykresu funkcji wykładniczej $f$ określonej wzorem

$f(x)=a^{x}$. Punkt $A=(1,2)$ nalezy do tego wykresu funkcji.
\begin{center}
\includegraphics[width=143.052mm,height=75.588mm]{./F1_M_PP_M2017_page5_images/image002.eps}
\end{center}
Podstawa a potęgijest równa

A.

- -21

B.

-21

C. $-2$

D. 2

Strona 6 z 26

MMA-IP





{\it Egzamin maturalny z matematyki}

{\it Poziom podstawowy}

{\it BRUDNOPIS} ({\it nie podlega ocenie})

MMA-IP

Strona 7 z 26





{\it Egzamin maturalny z matematyki}

{\it Poziom podstawowy}

Zadanie 12. $(1pkt)$

$\mathrm{W}$ ciągu arytmetycznym $(a_{n})$, określonym dla $n\geq 1$, dane są: $a_{1}=5, a_{2}=11$. Wtedy

A. $a_{14}=71$

B. $a_{12}=71$

C. $a_{11}=71$

D. $a_{10}=71$

Zadanie 13. $(1pkt)$

Dany jest trzywyrazowy ciąg geometryczny $($24, 6, $a-1)$. Stąd wynika, $\dot{\mathrm{z}}\mathrm{e}$

A.

{\it a}$=$ -25

B.

{\it a}$=$ -25

C.

{\it a}$=$ -23

D.

{\it a}$=$ -23

Zadanie 14. $(1pkt)$

Jeśli $m=\sin 50^{\mathrm{o}}$, to

A.

$m=\sin 40^{\mathrm{o}}$

B. $m=\cos 40^{\mathrm{o}}$

C. $m=\cos 50^{\mathrm{o}}$

D. $m=\mathrm{t}\mathrm{g}50^{\mathrm{o}}$

Zadanie 15. (I pkt)

Na okręgu o środku w punkcie O lezy punkt C (zobacz rysunek). Odcinek AB jest średnicą

tego okręgu. Zaznaczony na rysunku kąt środkowy a ma miarę
\begin{center}
\includegraphics[width=70.260mm,height=66.552mm]{./F1_M_PP_M2017_page7_images/image001.eps}
\end{center}
{\it C}

$56^{\mathrm{o}}$

{\it A}

$\alpha$

{\it O}

{\it B}

A. $116^{\mathrm{o}}$

B. $114^{\mathrm{o}}$

C. $112^{\mathrm{o}}$

D. $110^{\mathrm{o}}$

Strona 8 z 26

MMA-IP





{\it Egzamin maturalny z matematyki}

{\it Poziom podstawowy}

{\it BRUDNOPIS} ({\it nie podlega ocenie})

MMA-IP

Strona 9 z 26





{\it Egzamin maturalny z matematyki}

{\it Poziom podstawowy}

Zadanie 16. $(1pktJ$

$\mathrm{W}$ trójkącie $ABC$ punkt $D$ lezy na boku $BC$, a punkt $E$ lezy na boku $AB$. Odcinek $DE$ jest

równoległy do boku $AC$, a ponadto $|BD|=10, |BC|=12 \mathrm{i}|AC|=24$ (zobacz rysunek).
\begin{center}
\includegraphics[width=117.756mm,height=49.020mm]{./F1_M_PP_M2017_page9_images/image001.eps}
\end{center}
{\it B}

10

{\it D}

2

{\it C}

{\it E}

{\it A}

24

Długość odcinka DE jest równa

A. 22 B. 20

C. 12

D. ll

{\it Zadanie l7}. ({\it lpktJ}

Obwód trójkąta ABC, przedstawionego na rysunku, jest równy

A. $(3+\displaystyle \frac{\sqrt{3}}{2})a$
\begin{center}
\includegraphics[width=78.132mm,height=48.816mm]{./F1_M_PP_M2017_page9_images/image002.eps}
\end{center}
{\it C}

{\it a}

$30^{\mathrm{o}}$

{\it A  B}

C. $(3+\sqrt{3})a$

B. $(2+\displaystyle \frac{\sqrt{2}}{2})a$

D. $(2+\sqrt{2})a$

Strona 10 z 26

MMA-IP







CENTRALNA

KOMISJA

EGZAMINACYJNA

Arkusz zawiera informacje prawnie chronione do momentu rozpoczęcia egzaminu.

UZUPELNIA ZDAJACY

KOD PESEL

{\it miejsce}

{\it na naklejkę}
\begin{center}
\includegraphics[width=21.432mm,height=9.852mm]{./F1_M_PP_M2018_page0_images/image001.eps}

\includegraphics[width=82.140mm,height=9.852mm]{./F1_M_PP_M2018_page0_images/image002.eps}

\includegraphics[width=204.060mm,height=216.048mm]{./F1_M_PP_M2018_page0_images/image003.eps}
\end{center}
EGZAMIN MATU LNY

Z MATEMATYKI

POZIOM PODSTAWOWY

Instrukcja dla zdającego

1. Sprawd $\acute{\mathrm{z}}$, czy arkusz egzaminacyjny zawiera 26 stron

(zadania $1-34$). Ewentualny brak zgłoś przewodniczącemu

zespo nadzorującego egzamin.

2. Rozwiązania zadań i odpowiedzi wpisuj w miejscu na to

przeznaczonym.

3. Odpowiedzi do zadań zam iętych $(1-25)$ zaznacz

na karcie odpowiedzi, w części ka $\mathrm{y}$ przeznaczonej dla

zdającego. Zamaluj $\blacksquare$ pola do tego przeznaczone. Błędne

zaznaczenie otocz kółkiem $\mathrm{O}$ i zaznacz właściwe.

4. Pamiętaj, $\dot{\mathrm{z}}\mathrm{e}$ pominięcie argumentacji lub istotnych

obliczeń w rozwiązaniu zadania otwa ego (26-34) $\mathrm{m}\mathrm{o}\dot{\mathrm{z}}\mathrm{e}$

spowodować, $\dot{\mathrm{z}}\mathrm{e}$ za to rozwiązanie nie otrzymasz pełnej

liczby punktów.

5. Pisz czytelnie i uzywaj tylko długopisu lub pióra

z czarnym tuszem lub atramentem.

6. Nie uzywaj korektora, a błędne zapisy wyrazínie prze eśl.

7. Pamiętaj, $\dot{\mathrm{z}}\mathrm{e}$ zapisy w brudnopisie nie będą oceniane.

8. $\mathrm{M}\mathrm{o}\dot{\mathrm{z}}$ esz korzystać z zestawu wzorów matematycznych,

cyrkla i linijki oraz kalkulatora prostego.

9. Na tej stronie oraz na karcie odpowiedzi wpisz swój

numer PESEL i przyklej naklejkę z kodem.

10. Nie wpisuj $\dot{\mathrm{z}}$ adnych znaków w części przeznaczonej dla

egzaminatora.

7 MAJA 20I8

Godzina rozpoczęcia:

Czas pracy:

170 minut

Liczba punktów

do uzyskania: 50

$\Vert\Vert\Vert\Vert\Vert\Vert\Vert\Vert\Vert\Vert\Vert\Vert\Vert\Vert\Vert\Vert\Vert\Vert\Vert\Vert\Vert\Vert\Vert\Vert|  \mathrm{M}\mathrm{M}\mathrm{A}-\mathrm{P}1_{-}1\mathrm{P}-182$




{\it Egzamin maturalny z matematyki}

{\it Poziom podstawowy}

ZADANIA ZAMKNIĘTE

$W$ {\it kazdym z zadań od l. do 25. wybierz i zaznacz na karcie odpowiedzipoprawnq odpowied} $\acute{z}.$

Zadanie l. $(1pktJ$

Liczba 2 $\log_{3}6-\log_{3}4$ jest równa

A. 4

B. 2

Zadanie 2. $(1pkt)$

Liczba $\sqrt[3]{\frac{7}{3}}\cdot\sqrt[3]{\frac{81}{56}}$ jest równa

A.

-$\sqrt{}$23

B.

$\displaystyle \frac{3}{2\sqrt[3]{21}}$

C. $2\log_{3}2$

D. $\log_{3}8$

C.

-23

D.

-49

Zadanie 3. $(1pkt)$

Dane są liczby $a=3,6\cdot 10^{-12}$ oraz $b=2,4\cdot 10^{-20}$. Wtedy iloraz $\displaystyle \frac{a}{b}$ jest równy

A. $8,64\cdot 10^{-32}$

B. $1,5\cdot 10^{-8}$

C. $1,5\cdot 10^{8}$

D. $8,64\cdot 10^{32}$

Zadame4. (1pkt)

Cena roweru po obnizce o 15\% była równa 850 zł. Przed tą obnizką rower ten kosztował

A. 865,00 zł

B. 850,15 zł

C. 1000,00 zł

D. 977,50 zł

Zadanie 5. $(1pkt)$

Zbiorem wszystkich rozwiązań nierówności $\displaystyle \frac{1-2x}{2}>\frac{1}{3}$ jest przedział

A.

(-$\infty$' -61)

B.

(-$\infty$' -23)

C.

$(\displaystyle \frac{1}{6},+\infty)$

D.

$(\displaystyle \frac{2}{3},+\infty)$

{\it Zadanie 6}. ({\it lpkt})

Funkcja kwadratowa określona jest wzorem

róznymi miejscami zerowymi ffinkcjif. Zatem

$f(x)=-2(x+3)(x-5)$. Liczby

$x_{1}, x_{2}$

są

A. $x_{1}+x_{2}=-8$

B. $x_{1}+x_{2}=-2$

C. $x_{1}+x_{2}=2$

D. $x_{1}+x_{2}=8$

Strona 2 z 26

MMA-IP





{\it Egzamin maturalny z matematyki}

{\it Poziom podstawowy}

{\it BRUDNOPIS} ({\it nie podlega ocenie})

MMA-IP

Strona ll z 26





{\it Egzamin maturalny z matematyki}

{\it Poziom podstawowy}

Zadanie 23. $(1pktJ$

$\mathrm{W}$ zestawie $\displaystyle \frac{2,2,2,\ldots,2}{m1\mathrm{i}\mathrm{c}\mathrm{z}\mathrm{b}}\frac{4,4,4,\ldots,4}{m1\mathrm{i}\mathrm{c}\mathrm{z}\mathrm{b}}$ jest $2m$ liczb $(m\geq 1)$ ` w tym $m$ liczb 2 $\mathrm{i} m$ liczb 4.

Odchylenie standardowe tego zestawu liczb jest równe

A. 2

B. l

C.

-$\sqrt{}$12

D. $\sqrt{2}$

Zadanie 24. $(1pktJ$

Ile jest wszystkich liczb naturalnych czterocyfrowych mniejszych $\mathrm{n}\mathrm{i}\dot{\mathrm{z}}$ 2018 i podzielnych

przez 5?

A. 402

B. 403

C. 203

D. 204

Zadanie 25, $(1pktJ$

$\mathrm{W}$ pudełku jest 50 kuponów, wśród których jest 15 kuponów przegrywających, a pozostałe

kupony są wygrywające. $\mathrm{Z}$ tego pudełka w sposób losowy wyciągamy jeden kupon.

Prawdopodobieństwo zdarzenia polegającego na tym, $\dot{\mathrm{z}}\mathrm{e}$ wyciągniemy kupon wygrywający, jest

równe

A.

$\displaystyle \frac{15}{35}$

B.

$\displaystyle \frac{1}{50}$

C.

$\displaystyle \frac{15}{50}$

D.

$\displaystyle \frac{35}{50}$

Strona 12 z 26

MMA-IP





{\it Egzamin maturalny z matematyki}

{\it Poziom podstawowy}

{\it BRUDNOPIS} ({\it nie podlega ocenie})

MMA-IP

Strona 13 z 26





{\it Egzamin maturalny z matematyki}

{\it Poziom podstawowy}

Zadanie 26. $(2pktJ$

Rozwiąz nierówność $2x^{2}-3x>5.$

Odpowiedzí :

Strona 14 z 26

MMA-IP





{\it Egzamin maturalny z matematyki}

{\it Poziom podstawowy}

Zadanie 27, $(2pktJ$

Rozwiąz równanie $x^{3}-7x^{2}-4x+28=0.$

Odpowiedzí :
\begin{center}
\includegraphics[width=96.012mm,height=17.832mm]{./F1_M_PP_M2018_page14_images/image001.eps}
\end{center}
Wypelnia

egzaminator

Nr zadania

Maks. liczba kt

2

27.

2

Uzyskana liczba pkt

MMA-IP

Strona 15 z 26





{\it Egzamin maturalny z matematyki}

{\it Poziom podstawowy}

Zadanie 2{\$}. $(2pktJ$

Udowodnij, $\dot{\mathrm{z}}\mathrm{e}$ dla dowolnych liczb dodatnich $a, b$ prawdziwajest nierówność

$\displaystyle \frac{1}{2a}+\frac{1}{2f_{i}}\geq\frac{2}{a+b}.$

Strona 16 z 26

MMA-IP





{\it Egzamin maturalny z matematyki}

{\it Poziom podstawowy}

Zadanie 29. $(2pktJ$

Okręgi o środkach odpowiednio $A\mathrm{i}B$ są styczne zewnętrznie i $\mathrm{k}\mathrm{a}\dot{\mathrm{z}}\mathrm{d}\mathrm{y}$ z nichjest styczny do obu

ramion danego kąta prostego (zobacz rysunek). Promień okręgu o środku $A$ jest równy 2.

{\it A}.

{\it B}.

Uzasadnij, $\dot{\mathrm{z}}\mathrm{e}$ promień okręgu o środku $B$ jest mniejszy od $\sqrt{2}-1.$
\begin{center}
\includegraphics[width=96.012mm,height=17.784mm]{./F1_M_PP_M2018_page16_images/image001.eps}
\end{center}
Wypelnia

egzaminator

Nr zadania

Maks. liczba kt

28.

2

2

Uzyskana liczba pkt

MMA-IP

Strona 17 z 26





{\it Egzamin maturalny z matematyki}

{\it Poziom podstawowy}

Zadanie 30. $(2pkt)$

Do wykresu funkcji wykładniczej, określonej

$f(x)=a^{x}$ (gdzie $a>0 \mathrm{i} a\neq 1$), nalezy punkt

ffinkcji $g$, określonej wzorem $g(x)=f(x)-2$

dla $\mathrm{k}\mathrm{a}\dot{\mathrm{z}}$ dej liczby rzeczywistej $x$ wzorem

$P=(2,9)$. Oblicz $a$ i zapisz zbiór wartości

Odpowiedzí :

Strona 18 z 26

MMA-IP





{\it Egzamin maturalny z matematyki}

{\it Poziom podstawowy}

Zadanie 31. $(2pktJ$

Dwunasty wyraz ciągu arytmetycznego $(a_{n})$, określonego dla $n\geq 1$, jest równy 30, a sumajego

dwunastu początkowych wyrazówjest równa 162. Ob1icz pierwszy wyraz tego ciągu.

Odpowiedzí :
\begin{center}
\includegraphics[width=96.012mm,height=17.832mm]{./F1_M_PP_M2018_page18_images/image001.eps}
\end{center}
Wypelnia

egzaminator

Nr zadania

Maks. liczba kt

30.

2

31.

2

Uzyskana liczba pkt

MMA-IP

Strona 19 z 26





{\it Egzamin maturalny z matematyki}

{\it Poziom podstawowy}

Zadanie 32. $(SpktJ$

$\mathrm{W}$ układzie współrzędnych punkty $A=(4,3) \mathrm{i} B=(10,5)$ są wierzchołkami trójkąta $ABC.$

Wierzchołek $C$ lezy na prostej o równaniu $y=2x+3$. Oblicz współrzędne punktu $C$, dla którego

kąt $ABC$ jest prosty.

Strona 20 z 26

MMA-IP





{\it Egzamin maturalny z matematyki}

{\it Poziom podstawowy}

{\it BRUDNOPIS} ({\it nie podlega ocenie})

MMA-IP

Strona 3 z 26





{\it Egzamin maturalny z matematyki}

{\it Poziom podstawowy}

Odpowiedzí :
\begin{center}
\includegraphics[width=82.044mm,height=17.832mm]{./F1_M_PP_M2018_page20_images/image001.eps}
\end{center}
Wypelnia

egzaminator

Nr zadania

Maks. liczba kt

32.

5

Uzyskana liczba pkt

MMA-IP

Strona 21 z 26





{\it Egzamin maturalny z matematyki}

{\it Poziom podstawowy}

Zadanie 33. $(4pktJ$

Dane są dwa zbiory: $A=\{100$, 200, 300, 400, 500, 600, 700$\} \mathrm{i} B=\{10$, 11, 12, 13, 14, 15, 16$\}.$

$\mathrm{Z}\mathrm{k}\mathrm{a}\dot{\mathrm{z}}$ dego z nich losujemyjedną liczbę. Oblicz prawdopodobieństwo zdarzenia polegającego na

tym, $\dot{\mathrm{z}}\mathrm{e}$ suma wylosowanych liczb będzie podzielna przez 3. Ob1iczone prawdopodobieństwo

zapisz w postaci nieskracalnego ułamka zwykłego.

Strona 22 z 26

MMA-IP





{\it Egzamin maturalny z matematyki}

{\it Poziom podstawowy}

Odpowiedzí :
\begin{center}
\includegraphics[width=82.044mm,height=17.832mm]{./F1_M_PP_M2018_page22_images/image001.eps}
\end{center}
Wypelnia

egzaminator

Nr zadania

Maks. liczba kt

33.

4

Uzyskana liczba pkt

MMA-IP

Strona 23 z 26





{\it Egzamin maturalny z matematyki}

{\it Poziom podstawowy}

Zadanie 34. $(4pktJ$

Dany jest graniastosłup prawidłowy trójkątny (zobacz rysunek). Pole powierzchni całkowitej

tego graniastosłupa jest równe $45\sqrt{3}$. Pole podstawy graniastosłupa jest równe polu jednej

ściany bocznej. Oblicz objętość tego graniastosłupa.
\begin{center}
\includegraphics[width=61.980mm,height=42.828mm]{./F1_M_PP_M2018_page23_images/image001.eps}
\end{center}
{\it F}

{\it E}

{\it C  D}

{\it B}

{\it A}

Strona 24 z 26

MMA-IP





{\it Egzamin maturalny z matematyki}

{\it Poziom podstawowy}

Odpowiedzí :
\begin{center}
\includegraphics[width=82.044mm,height=17.832mm]{./F1_M_PP_M2018_page24_images/image001.eps}
\end{center}
Wypelnia

egzaminator

Nr zadania

Maks. liczba kt

34.

4

Uzyskana liczba pkt

MMA-IP

Strona 25 z 26





{\it Egzamin maturalny z matematyki}

{\it Poziom podstawowy}

{\it BRUDNOPIS} ({\it nie podlega ocenie})

Strona 26 z 26

MMA-IP





{\it Egzamin maturalny z matematyki}

{\it Poziom podstawowy}

Zadanie 7. $(1pkt)$

Równanie $\displaystyle \frac{x^{2}+2x}{x^{2}-4}=0$

A. ma trzy rozwiązania: $x=-2, x=0, x=2$

B. ma dwa rozwiązania: $x=0, x=2$

C. ma dwa rozwiązania: $x=-2, x=2$

D. majedno rozwiązanie: $x=0$

Zadanie S, (lpkt)

Funkcja liniowa f określona jest wzorem

rzeczywistych x. Wskaz zdanie prawdziwe.

$f(x)=\displaystyle \frac{1}{3}x-1,$

dla wszystkich

liczb

A. Funkcja $f$ jest malejącaijej wykres przecina oś $oy$ w punkcie $P=(0,\displaystyle \frac{1}{3}).$

B. Funkcja $f$ jest malejącaijej wykres przecina oś $Oy$ w punkcie $P=(0,-1).$

C. Funkcja $f$ jest rosnąca ijej wykres przecina oś $oy$ w punkcie $P=(0,\displaystyle \frac{1}{3}).$

D. Funkcja $f$ jest rosnącaijej wykres przecina oś $Oy$ w punkcie $P=(0,-1).$

Zadam$\mathrm{e}9. (1pkt)$

Wykresem funkcji kwadratowej $f(x)=x^{2}-6x-3$ jest parabola, której wierzchołkiem jest

punkt o współrzędnych

A. $(-6,-3)$

B. $(-6,69)$

C. $(3,-12)$

D. $(6,-3)$

Zadanie $l0. (1pkt)$

Liczba l jest miejscem zerowym funkcji liniowej $f(x)=ax+b$, a punkt $M=(3,-2)$ nalezy

do wykresu tej funkcji. Współczynnik $a$ we wzorze tej funkcjijest równy

A. l

B.

-23

C.

- -23

D. $-1$

Zadanie ll. $(1pktJ$

Dany jest ciąg $(a_{n})$ jest określony wzorem $a_{n}=\displaystyle \frac{5-2n}{6}$ dla $n\geq 1$. Ciąg tenjest

A.

B.

C.

D.

arytmetyczny ijego róznicajest równa $r=-\displaystyle \frac{1}{3}$

arytmetyczny ijego róznicajest równa $r=-2.$

geometryczny ijego iloraz jest równy $q=-\displaystyle \frac{1}{3}.$

geometryczny ijego iloraz jest równy $q=\displaystyle \frac{5}{6}$

Strona 4 $\mathrm{z}26$

MMA-IP





{\it Egzamin maturalny z matematyki}

{\it Poziom podstawowy}

{\it BRUDNOPIS} ({\it nie podlega ocenie})

MMA-IP

Strona 5 z 26





{\it Egzamin maturalny z matematyki}

{\it Poziom podstawowy}

Zadanie 12. $(1pktJ$

Dla ciągu arytmetycznego $(a_{n})$, określonego dla $n\geq 1$, jest spetniony wamnek $a_{4}+a_{5}+a_{6}=12.$

Wtedy

A. $a_{5}=4$

B. $a_{5}=3$

C. $a_{5}=6$

D. $a_{5}=5$

Zadanie $l3. (1pktJ$

Dany jest ciąg geometryczny $(a_{n})$, określony dla $n\geq 1$, w którym $a_{1}=\sqrt{2},$

$a_{3}=4\sqrt{2}$. Wzór na n-ty wyraz tego ciągu ma postać

$a_{2}=2\sqrt{2},$

A. $a_{n}=(\sqrt{2})^{n}$

B.

{\it an}$=$ -$\sqrt{}$22{\it n}

C.

{\it an}$=$(-$\sqrt{}$22){\it n}

D.

$a_{n}=\displaystyle \frac{(\sqrt{2})}{2}n$

Zadanie 14. (1pkt)

Przyprostokątna LM trójkąta prostokątnego KLM ma długość 3, a przeciwprostokątna KL ma

długość 8 (zobacz rysunek).

3
\begin{center}
\includegraphics[width=82.656mm,height=35.808mm]{./F1_M_PP_M2018_page5_images/image001.eps}
\end{center}
{\it L}

8

$\alpha$

{\it M  K}

Wówczas miara $\alpha$ kąta ostrego $LMK$ tego trójkąta spełnia waiunek

A. $27^{\mathrm{o}}<\alpha\leq 30^{\mathrm{o}}$

B. $24^{\mathrm{o}}<\alpha\leq 27^{\mathrm{o}}$

C. $21^{\mathrm{o}}<\alpha\leq 24^{\mathrm{o}}$

D. $18^{\mathrm{o}}<\alpha\leq 21^{\mathrm{o}}$

Zadanie 15. $(1pkt)$

Dany jest trójkąt o bokach długości: $2\sqrt{5}, 3\sqrt{5}, 4\sqrt{5}$. Trójkątem podobnym do tego trójkąta

jest trójkąt, którego boki mają długości

A. 10, 15, 20

B. 20, 45, 80

C. $\sqrt{2}, \sqrt{3}, \sqrt{4}$

D. $\sqrt{5}, 2\sqrt{5}, 3\sqrt{5}$

Strona 6 z 26

MMA-IP





{\it Egzamin maturalny z matematyki}

{\it Poziom podstawowy}

{\it BRUDNOPIS} ({\it nie podlega ocenie})

MMA-IP

Strona 7 z 26





{\it Egzamin maturalny z matematyki}

{\it Poziom podstawowy}

Zadanie 16. $(1pkt)$

Dany jest okrąg o środku $S$. Punkty $K, L\mathrm{i}M$ lez$\cdot$ą na tym okręgu. Na łuku $KL$ tego okręgu są

oparte kąty $KSL \mathrm{i} KML$ (zobacz rysunek), których miary a $\mathrm{i} \beta$, spełniają warunek

$\alpha+\beta=111^{\mathrm{o}}$. Wynika stąd, $\dot{\mathrm{z}}\mathrm{e}$
\begin{center}
\includegraphics[width=64.368mm,height=61.620mm]{./F1_M_PP_M2018_page7_images/image001.eps}
\end{center}
{\it M}

{\it K  L}

A. $\alpha=74^{\mathrm{o}}$

B. $\alpha=76^{\mathrm{o}}$

C. $\alpha=70^{\mathrm{o}}$

D. $\alpha=72^{\mathrm{o}}$

Zadanie 17. $(1pkt)$

Dany jest trapez prostokątny KLMN, którego podstawy mają długoŚci $|KL|=a, |MN|=b,$

$a>b$. Kąt $KLM$ ma miarę $60^{\mathrm{o}}$. DługoŚć ramienia $LM$ tego trapezujest równa

{\it N b}

{\it M}
\begin{center}
\includegraphics[width=89.712mm,height=41.556mm]{./F1_M_PP_M2018_page7_images/image002.eps}
\end{center}
{\it K  L}

{\it a}

A. $a-b$

B. $2(a-b)$

C.

$a+\displaystyle \frac{1}{2}b$

D.

-{\it a} $+$2 {\it b}

Zadanie $l8.(1pkt)$

Średnicą okręgu jest odcinek $Kl$, gdzie $K=(6,8), L=(-6,-8)$. Równanie tego okręgu ma

postać

A. $x^{2}+y^{2}=200$

B. $x^{2}+y^{2}=100$

C. $x^{2}+y^{2}=400$

D. $x^{2}+y^{2}=300$

Zadanie $1\vartheta. (1pkt)$

Proste o równaniach $y=(m+2)x+3$ oraz $y=(2m-1)x-3$ są równoległe, gdy

A. $m=2$

B. $m=3$

C. $m=0$

D. $m=1$

Strona 8 z 26

MMA-IP





{\it Egzamin maturalny z matematyki}

{\it Poziom podstawowy}

{\it BRUDNOPIS} ({\it nie podlega ocenie})

MMA-IP

Strona 9 z 26





{\it Egzamin maturalny z matematyki}

{\it Poziom podstawowy}

Zadanie 20. $(1pktJ$

Podstawą ostrosłupa jest kwadrat KLMN o boku długości 4. Wysokością tego ostrosłupajest

krawędzí $NS$, ajej długość $\mathrm{t}\mathrm{e}\dot{\mathrm{z}}$ jest równa 4 (zobacz rysunek).

Kąt $\alpha$, jaki tworzą krawędzie $KS\mathrm{i}MS$, spełnia warunek

A. $\alpha=45^{\mathrm{o}}$

B. $45^{\mathrm{o}}<\alpha<60^{\mathrm{o}}$

C. $a>60^{\mathrm{o}}$

D. $\alpha=60^{\mathrm{o}}$

Zadanie 21. $(1pkt)$

Podstawą graniastosłupa prostegojest prostokąt o bokach długości 3 $\mathrm{i}4$. Kąt $\alpha$, jaki przekątna

tego graniastosłupa tworzy zjego podstawą, jest równy $45^{\mathrm{o}}$ (zobacz rysunek).

Wysokość graniastosłupa jest równa

A. 5

B. $3\sqrt{2}$

C. $5\sqrt{2}$

D.

$\displaystyle \frac{5\sqrt{3}}{3}$

Zadanie 22. (1pkt)

Na rysunku przedstawiono bryłę zbudowaną z walca i półkuli. Wysokość walcajest równa r

ijest taka samajak promień półkuli oraz taka sama jak promień podstawy walca.

Objętość tej bryłyjest równa

A.

$\displaystyle \frac{5}{3}\pi r^{3}$

B.

$\displaystyle \frac{4}{3}\pi r^{3}$

C.

$\displaystyle \frac{2}{3}\pi r^{3}$

D.

$\displaystyle \frac{1}{3}\pi r^{3}$

Strona 10 z 26

MMA-IP







CENTRALNA

KOMISJA

EGZAMiNACYJNA

Arkusz zawiera informacje prawnie chronione do momentu rozpoczęcia egzaminu.

UZUPELNIA ZDAJACY

KOD PESEL

{\it miejsce}

{\it na naklejkę}
\begin{center}
\includegraphics[width=21.432mm,height=9.852mm]{./F1_M_PP_M2019_page0_images/image001.eps}

\includegraphics[width=82.140mm,height=9.852mm]{./F1_M_PP_M2019_page0_images/image002.eps}

\includegraphics[width=204.060mm,height=216.108mm]{./F1_M_PP_M2019_page0_images/image003.eps}
\end{center}
EGZAMIN MATU LNY

Z MATEMATYKI

POZIOM PODSTAWOWY

Instrukcja dla zdającego

1. Sprawd $\acute{\mathrm{z}}$, czy arkusz egzaminacyjny zawiera 26 stron

(zadania $1-34$). Ewentualny brak zgłoś przewodniczącemu

zespo nadzorującego egzamin.

2. Rozwiązania zadań i odpowiedzi wpisuj w miejscu na to

przeznaczonym.

3. Odpowiedzi do zadań zam iętych $(1-25)$ zaznacz

na karcie odpowiedzi, w części ka $\mathrm{y}$ przeznaczonej dla

zdającego. Zamaluj $\blacksquare$ pola do tego przeznaczone. Błędne

zaznaczenie otocz kółkiem \copyright i zaznacz właściwe.

4. Pamiętaj, $\dot{\mathrm{z}}\mathrm{e}$ pominięcie argumentacji lub istotnych

obliczeń w rozwiązaniu zadania otwa ego (26-34) $\mathrm{m}\mathrm{o}\dot{\mathrm{z}}\mathrm{e}$

spowodować, $\dot{\mathrm{z}}\mathrm{e}$ za to rozwiązanie nie otrzymasz pełnej

liczby punktów.

5. Pisz czytelnie i uzywaj tylko długopisu lub pióra

z czamym tuszem lub atramentem.

6. Nie uzywaj korektora, a błędne zapisy wyra $\acute{\mathrm{z}}\mathrm{n}\mathrm{i}\mathrm{e}$ prze eśl.

7. Pamiętaj, $\dot{\mathrm{z}}\mathrm{e}$ zapisy w brudnopisie nie będą oceniane.

8. $\mathrm{M}\mathrm{o}\dot{\mathrm{z}}$ esz korzystać z zestawu wzorów matematycznych,

cyrkla i linijki oraz kalkulatora prostego.

9. Na tej stronie oraz na karcie odpowiedzi wpisz swój

numer PESEL i przyklej naklejkę z kodem.

10. Nie wpisuj $\dot{\mathrm{z}}$ adnych znaków w części przeznaczonej dla

egzaminatora.

Godzina rozpoczęcia:

9:00

Czas pracy:

170 minut

Liczba punktów

do uzyskania: 50

$\Vert\Vert\Vert\Vert\Vert\Vert\Vert\Vert\Vert\Vert\Vert\Vert\Vert\Vert\Vert\Vert\Vert\Vert\Vert\Vert\Vert\Vert\Vert\Vert|  \mathrm{M}\mathrm{M}\mathrm{A}-\mathrm{P}1_{-}1\mathrm{P}-192$




{\it Egzamin maturalny z matematyki}

{\it Poziom podstawowy}

ZADANIA ZAMKNIĘTE

$W$ {\it kazdym z zadań} $\theta d1.$ {\it do 25. wybierz i zaznacz na karcie} $\theta owiedipprawnq$ {\it odpowiedz}$\acute{}$.{\it í}

Zadanie l. $(1pkt)$

Liczba $\log_{\sqrt{2}}2$ jest równa

A. 2

B. 4

C.

$\sqrt{2}$

D.

-21

Zadanie 2. $(1pkt)$

Liczba naturalna $n=2^{14}\cdot 5^{15}$ w zapisie dziesiętnym ma

A. 14 cyfr

B. 15 cyfr

C. 16 cyfr

D. 30 cyfr

$\mathrm{Z}_{\vartheta}\mathrm{d}\mathrm{a}\mathrm{n}\mathrm{i}\S 3. (1pkt)$

$\mathrm{W}$ pewnym banku prowizja od udzielanych kredytów hipotecznych przez cały styczeń była

równa 4\%. Na początku 1utego ten bank obnizył wysokość prowizji od wszystkich kredytów

$0 1$ punkt procentowy. Oznacza to, $\dot{\mathrm{z}}\mathrm{e}$ prowizja od kredytów hipotecznych w tym banku

zmniejszyła się o

A. l\%

B. 25\%

C. 33\%

D. 75\%

Zadanie 4, $(1pkt)$

Równość $\displaystyle \frac{1}{4}+\frac{1}{5}+\frac{1}{a}=1$ jest prawdziwa dla

A.

$a=\displaystyle \frac{11}{20}$

B.

{\it a}$=$ -98

C.

{\it a}$=$ -98

D.

{\it a}$=$ -2101

Zadanie 5. $(1pkt)$

Para liczb $x=2 \mathrm{i}y=2$ jest rozwiązaniem układu równań 

A. $a=-1$

B. $a=1$

C. $a=-2$

D. $a=2$

Zadanie $\epsilon. (1pkt)$

Równanie $\displaystyle \frac{(x-1)(x+2)}{x-3}=0$

A. ma trzy rózne rozwiązania: $x=1, x=3, x=-2.$

B. ma trzy rózne rozwiązania: $x=-1, x=-3, x=2.$

C. ma dwa rózne rozwiązania: $x=1, x=-2.$

D. ma dwa rózne rozwiązania: $x=-1, x=2.$

Strona 2 z26

MMA-IP





{\it Egzamin maturalny z matematyki}

{\it Poziom podstawowy}

{\it BRUDNOPIS} ({\it nie podlega ocenie})

Strona ll z 26





{\it Egzamin maturalny z matematyki}

{\it Poziom podstawowy}

Zadanie 22. (1pktJ

Podstawą ostrosłupa prawidłowego czworokątnego ABCDS jest kwadrat ABCD. Wszystkie

ściany boczne tego ostrosłupa są trójkątami równobocznymi.

Miara kąta SAC jest równa

A. $90^{\mathrm{o}}$

B. $75^{\mathrm{o}}$

C. $60^{\mathrm{o}}$

D. $45^{\mathrm{o}}$

Zadanie 23. (1pkt)

Mediana zestawu sześciu danych liczb: 4, 8, 21, a, 16, 25, jest równa 14. Zatem

A. $a=7$

B. $a=12$

C. $a=14$

D. $a=20$

Zadanie 24. (1pkt)

Wszystkich liczb pięciocyfrowych, w których występują wyłącznie cyfry 0, 2, 5, jest

A.

12

B. 36

C. 162

D. 243

Zadanie 25. $(1pkt)$

$\mathrm{W}$ pudełku jest 40 ku1. Wśród nich jest 35 ku1 białych, a pozostałe to ku1e czerwone.

Prawdopodobieństwo wylosowania $\mathrm{k}\mathrm{a}\dot{\mathrm{z}}$ dej kulijest takie samo. $\mathrm{Z}$ pudełka losujemyjedną kulę.

Prawdopodobieństwo zdarzenia polegającego na tym, $\dot{\mathrm{z}}\mathrm{e}$ otrzymamy kulę czerwoną, jest równe

A.

-81

B.

-51

C.

$\displaystyle \frac{1}{40}$

D.

$\displaystyle \frac{1}{35}$

Strona 12 z 26

MMA-IP





{\it Egzamin maturalny z matematyki}

{\it Poziom podstawowy}

{\it BRUDNOPIS} ({\it nie podlega ocenie})

Strona 13 z 26





{\it Egzamin maturalny z matematyki}

{\it Poziom podstawowy}

Zadanie 26. $(2pktJ$

Rozwiąz równanie $x^{3}-5x^{2}-9x+45=0.$

Odpowied $\acute{\mathrm{z}}$:

Strona 14 $\mathrm{z}26$





{\it Egzamin maturalny z matematyki}

{\it Poziom podstawowy}

Zadanie 27, $(2pktJ$

Rozwiąz nierównoŚć $3x^{2}-16x+16>0.$

Odpowiedzí :
\begin{center}
\includegraphics[width=96.012mm,height=17.784mm]{./F1_M_PP_M2019_page14_images/image001.eps}
\end{center}
Wypelnia

egzaminator

Nr zadania

Maks. liczba kt

2

27.

2

Uzyskana liczba pkt

MMA-IP

Strona 15 z 26





{\it Egzamin maturalny z matematyki}

{\it Poziom podstawowy}

Zadanie 2{\$}. $(2pktJ$

Wykaz, $\dot{\mathrm{z}}\mathrm{e}$ dla dowolnych liczb rzeczywistych $a\mathrm{i}b$ prawdziwajest nierówność

$3a^{2}-2ab+3b^{2}\geq 0.$

Strona 16 z 26





{\it Egzamin maturalny z matematyki}

{\it Poziom podstawowy}

Zadanie 29. $(2pktJ$

Danyjest okrąg o środku w punkcie $S$ i promieniu $r$. Na przedłuzeniu cięciwy $AB$ poza punkt $B$

odłozono odcinek $BC$ równy promieniowi danego okręgu. Przez punkty $C\mathrm{i}S$ poprowadzono

prostą. Prosta $CS$ przecina dany okrąg w punktach $D\mathrm{i}E$ (zobacz rysunek). Wykaz, $\dot{\mathrm{z}}$ ejezeli

miara kąta ACSjest równa $\alpha$, to miara kąta $ASD$ jest równa $3\alpha.$
\begin{center}
\includegraphics[width=96.924mm,height=62.580mm]{./F1_M_PP_M2019_page16_images/image001.eps}
\end{center}
{\it D  r}

{\it S}

{\it r}

{\it E}

{\it r  r}

{\it C}

{\it A  B}
\begin{center}
\includegraphics[width=96.012mm,height=17.832mm]{./F1_M_PP_M2019_page16_images/image002.eps}
\end{center}
Wypelnia

egzaminator

Nr zadania

Maks. liczba kt

28.

2

2

Uzyskana liczba pkt

MMA-IP

Strona 17 z 26





{\it Egzamin maturalny z matematyki}

{\it Poziom podstawowy}

Zadanie 30. (2pktJ

Ze zbioru liczb \{1, 2, 3, 4, 5\} 1osujemy dwa razy po jednej 1iczbie ze zwracaniem. Ob1icz

prawdopodobieństwo zdarzenia A polegającego na wylosowaniu liczb, których iloczyn jest

liczbą nieparzystą.

Odpowied $\acute{\mathrm{z}}$:

Strona 18 $\mathrm{z}26$

MMA-IP





{\it Egzamin maturalny z matematyki}

{\it Poziom podstawowy}

Zadanie 31. $(2pktJ$

$\mathrm{W}$ trapezie prostokątnym ABCD dłuzsza podstawa $AB$ ma długość 8. Przekątna $AC$ tego trapezu

ma długość 4 i tworzy z krótszą podstawą trapezu kąt o mierze $30^{\mathrm{o}}$ (zobacz rysunek). Oblicz

długość przekątnej $BD$ tego trapezu.
\begin{center}
\includegraphics[width=106.728mm,height=35.352mm]{./F1_M_PP_M2019_page18_images/image001.eps}
\end{center}
{\it D  C}

4

{\it A}  8  {\it B}

Odpowied $\acute{\mathrm{z}}$:
\begin{center}
\includegraphics[width=96.012mm,height=17.832mm]{./F1_M_PP_M2019_page18_images/image002.eps}
\end{center}
Wypelnia

egzaminator

Nr zadania

Maks. liczba kt

30.

2

31.

2

Uzyskana liczba pkt

MMA-IP

Strona 19 z 26





{\it Egzamin maturalny z matematyki}

{\it Poziom podstawowy}

Zadanie 32. $(4pktJ$

Ciąg arytmetyczny $(a_{n})$ jest określony dla $\mathrm{k}\mathrm{a}\dot{\mathrm{z}}$ dej liczby naturalnej $n\geq 1$. Róznicą tego

ciągujest liczba $r=-4$, a średnia arytmetyczna początkowych sześciu wyrazów tego ciągu:

$a_{1}, a_{2}, a_{3}, a_{4}, a_{5}, a_{6}$, jest równa 16.

a) Oblicz pierwszy wyraz tego ciągu.

b) Oblicz liczbę $k$, dla której $a_{k}=-78.$

Strona 20 z 26

MMA-IP





{\it Egzamin maturalny z matematyki}

{\it Poziom podstawowy}

{\it BRUDNOPIS} ({\it nie podlega ocenie})

Strona 3 z26





Odpowiedzí :

{\it Egzamin maturalny z matematyki}

{\it Poziom podstawowy}
\begin{center}
\includegraphics[width=82.044mm,height=17.832mm]{./F1_M_PP_M2019_page20_images/image001.eps}
\end{center}
Wypelnia

egzaminator

Nr zadania

Maks. liczba kt

32.

4

Uzyskana liczba pkt

MMA-IP

Strona 21 z 26





{\it Egzamin maturalny z matematyki}

{\it Poziom podstawowy}

Zadanie 33. $(4pktJ$

Danyjest punkt $A=(-18,10)$. Prosta o równaniu $y=3x$ jest symetralną odcinka $AB$. Wyznacz

współrzędne punktu $B.$

Strona 22 z 26

MMA-IP





{\it Egzamin maturalny z matematyki}

{\it Poziom podstawowy}

Odpowied $\acute{\mathrm{z}}$:
\begin{center}
\includegraphics[width=82.044mm,height=17.832mm]{./F1_M_PP_M2019_page22_images/image001.eps}
\end{center}
Nr zadania

Wypelnia Maks. liczba kt

egzaminator

Uzyskana liczba pkt

33.

4

MMA-IP

Strona 23 z 26





{\it Egzamin maturalny z matematyki}

{\it Poziom podstawowy}

Zadanie 34. $(SpktJ$

Długość krawędzi podstawy ostrosłupa prawidłowego czworokątnego jest równa 6. Po1e

powierzchni całkowitej tego ostrosłupajest cztery razy większe od polajego podstawy. Kąt $\alpha$

jest kątem nachylenia krawędzi bocznej tego ostrosłupa do płaszczyzny podstawy (zobacz

rysunek). Oblicz cosinus kąta $\alpha.$

Strona 24 z 26

MMA-IP





{\it Egzamin maturalny z matematyki}

{\it Poziom podstawowy}

Odpowied $\acute{\mathrm{z}}$:
\begin{center}
\includegraphics[width=82.044mm,height=17.832mm]{./F1_M_PP_M2019_page24_images/image001.eps}
\end{center}
Nr zadania

Wypelnia Maks. liczba kt

egzaminator

Uzyskana liczba pkt

34.

5

MMA-IP

Strona 25 z 26





{\it Egzamin maturalny z matematyki}

{\it Poziom podstawowy}

{\it BRUDNOPIS} ({\it nie podlega ocenie})

Strona 26 z 26





{\it Egzamin maturalny z matematyki}

{\it Poziom podstawowy}

Zadanie 7. $(1pkt)$

Miejscem zerowym funkcji liniowej $f$ określonej wzorem $f(x)=3(x+1)-6\sqrt{3}$ jest liczba

A. $3-6\sqrt{3}$

B.

$1-6\sqrt{3}$

C. $2\sqrt{3}-1$

D.

$2\displaystyle \sqrt{3}-\frac{1}{3}$

Informacja do zadań S.-10.

Na rysunku przedstawiony jest fragment paraboli będącej wykresem funkcji kwadratowej $f.$

Wierzchołkiem tej parabolijest punkt $W=(2,-4)$. Liczby 0 $\mathrm{i}4$ to miejsca zerowe funkcji $f.$
\begin{center}
\includegraphics[width=127.248mm,height=105.060mm]{./F1_M_PP_M2019_page3_images/image001.eps}
\end{center}
{\it y}

4

3

1

{\it x}

$-3 -2$

$-1 0$

$-1$

1 2 3 4  5 6

$-2$

$-3$

{\it W}

$\langle 0,  4\rangle$

B.

$(-\infty,  0\rangle$

A.

Zadanie 8. (1pkt)

Zbiorem wartości funkcji f jest przedział

C.

$\langle-4, +\infty)$

D. $\langle 4, +\infty)$

Zadam$\mathrm{e}9\cdot(1pkt)$

Największa wartość funkcji $f$ w przedziale $\langle$1, $ 4\rangle$ jest równa

A. $-3$

B. $-4$

C. 4

D. 0

Zadanie 10. (1pkt)

Osią symetrii wykresu funkcji f jest prosta o równaniu

A. $y=-4$

B. $x=-4$

C. $y=2$

D. $x=2$

Strona 4 z26

MMA-IP





{\it Egzamin maturalny z matematyki}

{\it Poziom podstawowy}

{\it BRUDNOPIS} ({\it nie podlega ocenie})

Strona 5 z 26





{\it Egzamin maturalny z matematyki}

{\it Poziom podstawowy}

Zadanie ll. $(1pktJ$

$\mathrm{W}$ ciągu arytmetycznym $(a_{n})$, określonym dla $n\geq 1$, dane są dwa wyrazy: $a_{1}=7\mathrm{i}a_{8}=-49.$

Suma ośmiu początkowych wyrazów tego ciągujest równa

A. $-168$

B. $-189$

C. $-21$

D. $-42$

Zadanie $l2. (1pkt)$

Dany jest ciąg geometryczny $(a_{n})$, określony dla $n\geq 1$. Wszystkie wyrazy tego ciągu są

dodatnie i spełnionyjest watunek $\displaystyle \frac{a_{5}}{a_{3}}=\frac{1}{9}$. Iloraz tego ciągujest równy

A.

-31

B.

-$\sqrt{}$13

C. 3

D. $\sqrt{3}$

Zadanie 13. $(1pktJ$

Sinus kąta ostrego $\alpha$ jest równy $\displaystyle \frac{4}{5}$. Wtedy

A.

$\displaystyle \cos\alpha=\frac{5}{4}$

B.

$\displaystyle \cos\alpha=\frac{1}{5}$

C.

$\displaystyle \cos\alpha=\frac{9}{25}$

D.

$\displaystyle \cos\alpha=\frac{3}{5}$

Zadanie 14. $(1pktJ$

Punkty $D\mathrm{i}E$ lez$\cdot$ą na okręgu opisanym na trójkącie równobocznym $ABC$ (zobacz rysunek).

Odcinek $CD$ jest średnicą tego okręgu. Kąt wpisany $DEB$ ma miarę $\alpha.$
\begin{center}
\includegraphics[width=46.788mm,height=52.680mm]{./F1_M_PP_M2019_page5_images/image001.eps}
\end{center}
{\it C}

{\it E}

$\alpha$

{\it A  B}

{\it D}

Zatem

A. $\alpha=30^{\mathrm{o}}$

B. $\alpha<30^{\mathrm{o}}$

C. $\alpha>45^{\mathrm{o}}$

D. $\alpha=45^{\mathrm{o}}$

Strona 6 z26

MMA-IP





{\it Egzamin maturalny z matematyki}

{\it Poziom podstawowy}

{\it BRUDNOPIS} ({\it nie podlega ocenie})

Strona 7 z 26





{\it Egzamin maturalny z matematyki}

{\it Poziom podstawowy}

Zadanie 15. (1pktJ

Dane są dwa okręgi: okrąg o środku w punkcie O i promieniu 5 oraz okrąg o środku

w punkcie P i promieniu 3. Odcinek OP ma długość 16. Prosta AB jest styczna do tych okręgów

w punktach A iB. Ponadto prosta AB przecina odcinek OP w punkcie K(zobacz rysunek).
\begin{center}
\includegraphics[width=155.292mm,height=65.436mm]{./F1_M_PP_M2019_page7_images/image001.eps}
\end{center}
{\it B}

{\it O  K}

{\it P}

{\it A}

Wtedy

A.

$|OK|=6$

B.

$|OK|=8$

C.

$|OK|=10$

D.

$|OK|=12$

Zadanie $1\mathrm{f}\cdot(1pkt)$

Dany jest romb o boku długości 4 i kącie rozwartym $150^{\mathrm{o}}$. Pole tego rombujest równe

A. 8

B. 12

C. $8\sqrt{3}$

D. 16

Zadanie $l7. (1pktJ$

Proste o równaniach $y=(2m+2)x-2019$ oraz $y=(3m-3)x+2019$ są równoległe, gdy

A. $m=-1$

B. $m=0$

C. $m=1$

D. $m=5$

Zadanie 18. (1pktJ

Prosta o równaniu $y=ax+b$ jest prostopadła do prostej o równaniu $y=-4x+1$ i przechodzi

przez punkt $P=(\displaystyle \frac{1}{2},0)$, gdy

A. $a=-4\mathrm{i}b=-2$

B. {\it a}$=$-41i{\it b}$=$--81

C. $a=-4\mathrm{i}b=2$

D. {\it a}$=$-41i{\it b}$=$-21

Strona 8 z 26

MMA-IP





{\it Egzamin maturalny z matematyki}

{\it Poziom podstawowy}

{\it BRUDNOPIS} ({\it nie podlega ocenie})

Strona 9 z 26





{\it Egzamin maturalny z matematyki}

{\it Poziom podstawowy}

Zadanie 19. $(1pktJ$

Na rysunku przedstawiony jest fragment wykresu funkcji liniowej $f$ Na wykresie tej ffinkcji

$\mathrm{l}\mathrm{e}\dot{\mathrm{z}}$ ą punkty $A=(0,4)\mathrm{i}B=(2,2).$
\begin{center}
\includegraphics[width=65.376mm,height=67.920mm]{./F1_M_PP_M2019_page9_images/image001.eps}
\end{center}
$y$

$5$

-$4^{A}$

3

2

$B1$

1

{\it x}

$-4  -3$ -$2$ -$1$ -$10$  1 2 3 4  $-5$

$-2$

$-3$

$-4$

Obrazem prostej AB w symetrii względem początku układu współrzędnych jest wykres

funkcji g określonej wzorem

A. $g(x)=x+4$

B. $g(x)=x-4$

C. $g(x)=-x-4$

D. $g(x)=-x+4$

Zadanie 20. $(1pktJ$

Dane są punkty o współrzędnych $A=(-2,5)$ oraz $B=(4,-1)$. Średnica okręgu wpisanego

w kwadrat o boku $AB$ jest równa

A. 12

B. 6

C.

$6\sqrt{2}$

D. $2\sqrt{6}$

Zadanie 21. (1pkt)

Promień AS podstawy walca jest równy połowie wysokości OS tego walca. Sinus kąta OAS

(zobacz rysunek) jest równy
\begin{center}
\includegraphics[width=47.904mm,height=76.404mm]{./F1_M_PP_M2019_page9_images/image002.eps}
\end{center}
{\it O}

{\it S}

{\it A}

A.

-$\sqrt{}$25

B.

$\displaystyle \frac{2\sqrt{5}}{5}$

C.

-21

D. l

Strona 10 z 26

MMA-IP







CENTRALNA

KOMISJA

EGZAMINACYJNA

Arkusz zawiera informacje prawnie chronione do momentu rozpoczęcia egzaminu.

UZUPELNIA ZDAJACY

KOD PESEL

{\it miejsce}

{\it na naklejkę}
\begin{center}
\includegraphics[width=21.432mm,height=9.852mm]{./F1_M_PP_M2020_page0_images/image001.eps}

\includegraphics[width=82.140mm,height=9.852mm]{./F1_M_PP_M2020_page0_images/image002.eps}

\includegraphics[width=204.060mm,height=216.048mm]{./F1_M_PP_M2020_page0_images/image003.eps}
\end{center}
EGZAMIN MATU LNY

Z MATEMATYKI

POZIOM PODSTAWOWY

Instrukcja dla zdającego

1. Sprawd $\acute{\mathrm{z}}$, czy arkusz egzaminacyjny zawiera 26 stron

(zadania $1-34$). Ewentualny brak zgłoś przewodniczącemu

zespo nadzo jącego egzamin.

2. Rozwiązania zadań i odpowiedzi wpisuj w miejscu na to

przeznaczonym.

3. Odpowiedzi do zadań zamkniętych $(1-25)$ zaznacz

na karcie odpowiedzi, w części ka $\mathrm{y}$ przeznaczonej dla

zdającego. Zamaluj $\blacksquare$ pola do tego przeznaczone. Błędne

zaznaczenie otocz kółkiem $\mathrm{O}$ i zaznacz właściwe.

4. Pamiętaj, $\dot{\mathrm{z}}\mathrm{e}$ pominięcie argumentacji lub istotnych

obliczeń w rozwiązaniu zadania otwartego (26-34) $\mathrm{m}\mathrm{o}\dot{\mathrm{z}}\mathrm{e}$

spowodować, $\dot{\mathrm{z}}\mathrm{e}$ za to rozwiązanie nie otrzymasz pełnej

liczby punktów.

5. Pisz czytelnie i uzywaj tylko długopisu lub pióra

z czatnym tuszem lub atramentem.

6. Nie uzywaj korektora, a błędne zapisy wyra $\acute{\mathrm{z}}\mathrm{n}\mathrm{i}\mathrm{e}$ prze eśl.

7. Pamiętaj, $\dot{\mathrm{z}}\mathrm{e}$ zapisy w brudnopisie nie będą oceniane.

8. $\mathrm{M}\mathrm{o}\dot{\mathrm{z}}$ esz korzystać z zestawu wzorów matematycznych,

cyrkla i linijki oraz kalkulatora prostego.

9. Na tej stronie oraz na karcie odpowiedzi wpisz swój

numer PESEL i przyklej naklejkę z kodem.

10. Nie wpisuj $\dot{\mathrm{z}}$ adnych znaków w części przeznaczonej dla

egzaminatora.

5 MAJA 2020

Godzina rozpoczęcia:

Czas pracy:

170 minut

Liczba punktów

do uzyskania: 50

$\Vert\Vert\Vert\Vert\Vert\Vert\Vert\Vert\Vert\Vert\Vert\Vert\Vert\Vert\Vert\Vert\Vert\Vert\Vert\Vert\Vert\Vert\Vert\Vert|  \mathrm{M}\mathrm{M}\mathrm{A}-\mathrm{P}1_{-}1\mathrm{P}-202$




{\it Egzamin maturalny z matematyki}

{\it Poziom podstawowy}

ZADANIA ZAMKNIETE

$W$ {\it kazdym z zadań od l. do 25. wybierz i zaznacz na karcie odpowiedzipoprawnq odpowied} $\acute{z}.$

Zadanie l. $(1pkt)$

Wartość wyrazenia $x^{2}-6x+9$ dla $x=\sqrt{3}+3$ jest równa

A. l

B. 3

Zadanie 2. $(1pkt)$

Liczba $\displaystyle \frac{2^{50}\cdot 3^{40}}{36^{10}}$ jest równa

A. $6^{70}$

B. $6^{45}$

Zadanie 3. $(1pkt)$

Liczba $\log_{5}\sqrt{125}$ jest równa

A.

-23

B. 2

C. $1+2\sqrt{3}$

D. $1-2\sqrt{3}$

C. $2^{30}\cdot 3^{20}$

D. $2^{10}\cdot 3^{20}$

C. 3

D.

-23

Zadanie 4. (1pkt)

Cenę x pewnego towaru obnizono o 20\% i otrzymano cenę y. Aby przywrócić cenę x, nową

cenę y nalezy podnieść o

A. 25\%

B. 20\%

C. 15\%

D. 12\%

Zadanie 5. $(1pkt)$

Zbiorem wszystkich rozwiązań nierówności 3 $(1-x)>2(3x-1)-12x$ jest przedział

A.

$(-\displaystyle \frac{5}{3},+\infty)$

B.

(-$\infty$, -35)

C.

$(\displaystyle \frac{5}{3},+\infty)$

D.

(-$\infty$'- -35)

Zadanie $\epsilon. (1pkt)$

Suma wszystkich rozwiązań równania $x(x-3)(x+2)=0$ jest równa

A. 0

B. l

C. 2

D. 3

Strona 2 z26

MMA-IP





{\it Egzamin maturalny z matematyki}

{\it Poziom podstawowy}

{\it BRUDNOPIS} ({\it nie podlega ocenie})

MMA-IP

Strona ll z26





{\it Egzamin maturalny z matematyki}

{\it Poziom podstawowy}

Zadanie 23. (1pktJ

Cztery liczby: 2, 3, a, 8, tworzące zestaw danych, są uporządkowane rosnąco. Mediana tego

zestawu czterech danychjest równa medianie zestawu pięciu danych: 5, 3, 6, 8, 2. Zatem

A. $a=7$

B. $a=6$

C. $a=5$

D. $a=4$

Zadanie 24. $(1pkt)$

Dany jest sześcian ABCDEFGH. Sinus kąta $\alpha$ nachylenia przekątnej $HB$ tego sześcianu do

płaszczyzny podstawy ABCD (zobacz rysunek) jest równy

A.

$\displaystyle \frac{\sqrt{3}}{3}$

B.

$\displaystyle \frac{\sqrt{6}}{3}$

C.

-$\sqrt{}$22

D.

-$\sqrt{}$26
\begin{center}
\includegraphics[width=61.572mm,height=58.116mm]{./F1_M_PP_M2020_page11_images/image001.eps}
\end{center}
{\it H  G}

{\it E  F}

{\it D  C}

$\alpha$

{\it A  B}

Zadanie 25. $(1pkt)$

Danyjest stozek o objętości $ 18\pi$, którego przekrojem osiowymjest trójkąt ABC(zobacz rysunek).

Kąt $CBA$ jest kątem nachylenia tworzącej $l$ tego stozka do płaszczyzny jego podstawy.

Tangens kąta $CBA$ jest równy 2.
\begin{center}
\includegraphics[width=64.464mm,height=48.408mm]{./F1_M_PP_M2020_page11_images/image002.eps}
\end{center}
{\it C}

{\it l}

{\it h}

{\it A  B}

Wynika stąd, $\dot{\mathrm{z}}\mathrm{e}$ wysokość $h$ tego stozkajest równa

A. 12

B. 6

C. 4

D. 2

Strona 12 z26

MMA-IP





{\it Egzamin maturalny z matematyki}

{\it Poziom podstawowy}

{\it BRUDNOPIS} ({\it nie podlega ocenie})

MMA-IP

Strona 13 z26





{\it Egzamin maturalny z matematyki}

{\it Poziom podstawowy}

Zadanie 26. $(2pktJ$

Rozwiąz nierówność 2 $(x-1)(x+3)>x-1.$

Odpowied $\acute{\mathrm{z}}$:

Strona 14 $\mathrm{z}26$

MMA-IP





{\it Egzamin maturalny z matematyki}

{\it Poziom podstawowy}

Zadanie 27. $(2pktJ$

Rozwiąz równanie $x^{3}-9x^{2}-4x+36=0.$

Odpowiedzí:
\begin{center}
\includegraphics[width=96.012mm,height=17.784mm]{./F1_M_PP_M2020_page14_images/image001.eps}
\end{center}
WypelnÍa

egzaminator

Nr zadania

Maks. lÍczba kt

2

27.

2

Uzyskana liczba pkt

MMA-IP

Strona 15 z26





{\it Egzamin maturalny z matematyki}

{\it Poziom podstawowy}

Zadanie 28. $(2pktJ$

Wykaz, ze dla kazdych dwóch róznych liczb rzeczywistych $a\mathrm{i}b$ prawdziwajest nierówność

$a(a-2b)+2b^{2}>0.$

Strona 16 z26

MMA-IP





{\it Egzamin maturalny z matematyki}

{\it Poziom podstawowy}

Zadanie 29. $(2pktJ$

Trójkąt ABCjest równoboczny. Punkt $E$ lezy na wysokości $CD$ tego trójkąta oraz $|CE|=\displaystyle \frac{3}{4}|CD|.$

Punkt $F$ lezy na boku $BC$ i odcinek $EF$ jest prostopadły do $BC$ (zobacz rysunek).
\begin{center}
\includegraphics[width=82.140mm,height=70.812mm]{./F1_M_PP_M2020_page16_images/image001.eps}
\end{center}
{\it C}

{\it F}

{\it E}

{\it A  D  B}

Wykaz, $\displaystyle \dot{\mathrm{z}}\mathrm{e}|CF|=\frac{9}{16}|CB|.$
\begin{center}
\includegraphics[width=96.012mm,height=17.784mm]{./F1_M_PP_M2020_page16_images/image002.eps}
\end{center}
Wypelnia

egzaminator

Nr zadania

Maks. liczba kt

28.

2

2

Uzyskana liczba pkt

MMA-IP

Strona 17 z26





{\it Egzamin maturalny z matematyki}

{\it Poziom podstawowy}

Zadanie 30. $(2pktJ$

Rzucamy dwa razy symetryczną sześcienną kostką do gry, która na $\mathrm{k}\mathrm{a}\dot{\mathrm{z}}$ dej ściance ma inną

liczbę oczek-odjednego oczka do sześciu oczek. Oblicz prawdopodobieństwo zdarzenia $A$

polegającego na tym, ze co najmniej jeden raz wypadnie ścianka z pięcioma oczkami.

Odpowiedzí:

Strona 18 z26

MMA-IP





{\it Egzamin maturalny z matematyki}

{\it Poziom podstawowy}

Zadanie 31. $(2pktJ$

Kąt $\alpha$ jest ostry i spełnia warunek $\displaystyle \frac{2\sin\alpha+3\cos\alpha}{\cos\alpha}=4$. Oblicz tangens kąta $\alpha.$

Odpowied $\acute{\mathrm{z}}$:
\begin{center}
\includegraphics[width=96.012mm,height=17.784mm]{./F1_M_PP_M2020_page18_images/image001.eps}
\end{center}
WypelnÍa

egzaminator

Nr zadanÍa

Maks. lÍczba kt

30.

2

31.

2

Uzyskana liczba pkt

MMA-IP

Strona 19 z26





{\it Egzamin maturalny z matematyki}

{\it Poziom podstawowy}

Zadanie 32. $(4pktJ$

Dany jest kwadrat ABCD, w którym $A=(5,-\displaystyle \frac{5}{3})$. Przekątna $BD$ tego kwadratu jest zawarta

w prostej o równaniu $y=\displaystyle \frac{4}{3}x$. Oblicz współrzędne punktu przecięcia przekątnych $AC\mathrm{i}BD$ oraz

pole kwadratu ABCD.

Strona 20 z26

MMA-IP





{\it Egzamin maturalny z matematyki}

{\it Poziom podstawowy}

{\it BRUDNOPIS} ({\it nie podlega ocenie})

MMA-IP

Strona 3 z26





{\it Egzamin maturalny z matematyki}

{\it Poziom podstawowy}

Odpowiedzí:
\begin{center}
\includegraphics[width=82.044mm,height=17.832mm]{./F1_M_PP_M2020_page20_images/image001.eps}
\end{center}
Wypelnia

egzaminator

Nr zadania

Maks. liczba kt

32.

4

Uzyskana liczba pkt

MMA-IP

Strona 21 z 26







\end{document}