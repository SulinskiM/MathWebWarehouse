\documentclass[10pt]{article}
\usepackage[polish]{babel}
\usepackage[utf8]{inputenc}
\usepackage[T1]{fontenc}
\usepackage{amsmath}
\usepackage{amsfonts}
\usepackage{amssymb}
\usepackage[version=4]{mhchem}
\usepackage{stmaryrd}

\title{AKADEMIA GÓRNICZO-HUTNICZA \\
 im. Stanisława Staszica w Krakowie OLIMPIADA „O DIAMENTOWY INDEKS AGH" 2009/10 \\
 MATEMATYKA - ETAP II }

\author{}
\date{}


\begin{document}
\maketitle
\section*{ZADANIA PO 10 PUNKTÓW}
\begin{enumerate}
  \item Pole powierzchni bocznej stożka jest trzy razy większe od pola jego podstawy. Ile razy objętość stożka jest większa od objętości kuli wpisanej w ten stożek?
  \item Dane są funkcje $f(x)=2^{x+1}+5^{x-5}$ i $g(x)=25^{x}+4^{x}$. Rozwiąż równanie $g\left(\frac{x}{2}\right)=f(x+3)$.
  \item Oblicz $\sin 2 \alpha$, jeżeli $\sin \alpha=0,75$ i $\alpha \in\left(\frac{\pi}{2} ; \pi\right)$.
  \item Wyznacz granicę ciagu
\end{enumerate}

$$
\lim _{n \rightarrow+\infty}\left(\sqrt[3]{n^{6}+5 n^{4}}-n^{2}\right)
$$

\section*{ZADANIA PO 20 PUNKTÓW}
\begin{enumerate}
  \setcounter{enumi}{4}
  \item Znajdź równania stycznych do okręgu $x^{2}+y^{2}-8 y+12=0$ przechodzących przez początek układu współrzędnych. Znajdź równania obrazów tego okręgu i jednej z wyznaczonych stycznych w jednokładności o środku w punkcie $S=$ $(1,2)$ i skali $k=-3$.
  \item Funkcja $f$ spełnia dla każdego $x$ należącego do jej dziedziny równanie
\end{enumerate}

$$
1+f(x)+(f(x))^{2}+(f(x))^{3}+\ldots=\frac{x}{2}+1
$$

gdzie lewa strona jest sumą nieskończonego ciagu geometrycznego. Wyznacz dziedzinę i wzór funkcji $f$. Naszkicuj jej wykres.\\
7. Liczby $1,2,3, \ldots, n$, gdzie $n \geq 3$, losowo ustawiamy w ciag. Oblicz prawdopodobieństwa zdarzeń\\
$A$ : liczba $n$ nie będzie ostatnim wyrazem tego ciagu;\\
$B$ : liczby $1,2,3$ wystappią obok siebie w kolejności wzrastania;\\
$C$ : iloczyn każdej pary sąsiednich wyrazów tego ciagu jest liczbą parzystą. Wyniki zapisz w najprostszej postaci.


\end{document}