\documentclass[10pt]{article}
\usepackage[polish]{babel}
\usepackage[utf8]{inputenc}
\usepackage[T1]{fontenc}
\usepackage{amsmath}
\usepackage{amsfonts}
\usepackage{amssymb}
\usepackage[version=4]{mhchem}
\usepackage{stmaryrd}

\title{AKADEMIA GÓRNICZO-HUTNICZA im. Stanisława Staszica w Krakowie OLIMPIADA „O DIAMENTOWY INDEKS AGH" 2018/19 MATEMATYKA - ETAP II }

\author{ZADANIA PO 10 PUNKTÓW}
\date{}


\begin{document}
\maketitle


\begin{enumerate}
  \item Kierowca przejechał połowę drogi autostradą, a drugą połowę lokalnymi drogami z prędkością dwa razy mniejszą. Jaki procent drogi przejechał kierowca po upływie połowy czasu podróży?
  \item Liczba $\alpha \in\left(\frac{\pi}{4} ; \frac{\pi}{2}\right)$ spełnia równanie
\end{enumerate}

$$
\operatorname{tg}^{2} \alpha=5 \operatorname{tg} \alpha-1
$$

Oblicz $\cos 2 \alpha$.\\
3. Na ile sposobów można tak ustawić w ciąg $k$ czarnych kul i $k+1$ białych, by żadne dwie czarne kule nie znalazły się obok siebie? Zakładamy, że kule tego samego koloru są nierozróżnialne.\\
4. Wyznacz dziedzinę funkcji danej wzorem

$$
f(x)=\frac{x^{5}+8 x^{2}}{x^{3}-4 x^{2}-3 x+18} .
$$

W których punktach nienależących do dziedziny można określić wartość funkcji $f$, aby otrzymać funkcję ciągłą w danym punkcie?

\section*{ZADANIA PO 20 PUNKTÓW}
\begin{enumerate}
  \setcounter{enumi}{4}
  \item W prawidłowy ostrosłup czworokątny o krawędzi podstawy długości $a$ i krawędzi bocznej długości $b$ tak wpisany jest walec, że jedna z podstaw walca zawiera się w podstawie ostrosłupa. Wyznacz wymiary walca o możliwie największej objętości.
  \item Dla jakich liczb rzeczywistych $m$ równanie
\end{enumerate}

$$
(m-3) x^{2}+2 m x+m-2=0
$$

ma dwa różne pierwiastki rzeczywiste $x_{1}, x_{2}$, spełniające nierówność

$$
\log _{0,1} x_{1}+\log _{0,1} x_{2} \geqslant 0 ?
$$

\begin{enumerate}
  \setcounter{enumi}{6}
  \item W trójkąt równoboczny o boku długości $a$ tak wpisane są trzy przystające okręgi, że każdy z nich jest styczny do dwóch pozostałych i do dwóch boków trójkąta. Oblicz promień okręgu zewnętrznie stycznego do tych trzech okręgów.
\end{enumerate}

\end{document}