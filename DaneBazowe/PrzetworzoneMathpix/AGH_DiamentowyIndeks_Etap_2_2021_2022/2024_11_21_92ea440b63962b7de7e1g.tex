\documentclass[10pt]{article}
\usepackage[polish]{babel}
\usepackage[utf8]{inputenc}
\usepackage[T1]{fontenc}
\usepackage{amsmath}
\usepackage{amsfonts}
\usepackage{amssymb}
\usepackage[version=4]{mhchem}
\usepackage{stmaryrd}

\title{AKADEMIA GÓRNICZO-HUTNICZA im. Stanisława Staszica w Krakowie OLIMPIADA „O DIAMENTOWY INDEKS AGH" 2021/22 MATEMATYKA - ETAP II }

\author{ZADANIA PO 10 PUNKTÓW}
\date{}


\begin{document}
\maketitle


\begin{enumerate}
  \item Jadąc z prędkością $30 \mathrm{~km} /$ godz. spóźnimy się na spotkanie 10 minut, a jadąc z prędkością $60 \mathrm{~km} /$ godz. będziemy 10 minut za wcześnie. Z jaką prędkością powinniśmy jechać, aby przybyć punktualnie?
  \item Bok kwadratu jest przeciwprostokątną $A B$ trójkąta prostokątnego, którego trzeci wierzchołek $C$ leży na zewnątrz kwadratu. Niech $S$ będzie środkiem kwadratu. Uzasadnij, że kąty $A C S$ i $B C S$ są przystające.
  \item Dane są trzy kolejne liczby całkowite. Udowodnij, że kwadraty dokładnie dwóch z nich dają resztę 1 z dzielenia przez 3.
  \item Liczby $2 \log _{2} x, \log _{2} 2 x, \log _{2}(x-4)$ są trzema początkowymi wyrazami ciągu arytmetycznego. Znajdź setny wyraz tego ciągu.
\end{enumerate}

\section*{ZADANIA PO 20 PUNKTÓW}
\begin{enumerate}
  \setcounter{enumi}{4}
  \item Wyznacz dziedzinę i zbiór wartości funkcji $f$, jeżeli dla każdego $x$ należącego do jej dziedziny spełniona jest równość
\end{enumerate}

$$
f(x)+(f(x))^{2}+(f(x))^{3}+\ldots=-\frac{1}{5}\left(x^{2}+1\right)
$$

\begin{enumerate}
  \setcounter{enumi}{5}
  \item Dane są dodatnie liczby całkowite $n$ oraz $k$, przy czym $k \leqslant n$. Ze zbioru liczb $\{1,2, \ldots, n\}$ losujemy kolejno bez zwracania $k$ liczb, otrzymując w ten sposób ciąg $k$-wyrazowy. Oblicz prawdopodobieństwa zdarzeń\\
$A$ : liczba $k$ nie występuje w tym ciągu,\\
$B: k$ jest ostatnim wyrazem ciagu,\\
$C$ : ciąg jest monotoniczny i $k$ jest jego wyrazem.
  \item Punkty $A=(0,7), B=(1,0), C=(-3,-2)$ są wierzchołkami trójkąta. Znajdź równanie okręgu opisanego na tym trójkącie i równanie jego obrazu w symetrii środkowej względem punktu $A$. Napisz równania wszystkich prostych stycznych jednocześnie do obu tych okręgów.
\end{enumerate}

\end{document}