\documentclass[10pt]{article}
\usepackage[polish]{babel}
\usepackage[utf8]{inputenc}
\usepackage[T1]{fontenc}
\usepackage{amsmath}
\usepackage{amsfonts}
\usepackage{amssymb}
\usepackage[version=4]{mhchem}
\usepackage{stmaryrd}

\title{AKADEMIA GÓRNICZO-HUTNICZA \\
 im. Stanisława Staszica w Krakowie \\
 OLIMPIADA „O DIAMENTOWY INDEKS AGH" 2022/23 \\
 MATEMATYKA - ETAP II }

\author{}
\date{}


\begin{document}
\maketitle
\section*{ZADANIA PO 10 PUNKTÓW}
\begin{enumerate}
  \item Udowodnij, że istnieje tylko jedna trójka liczb pierwszych, które są trzema kolejnymi wyrazami ciągu arytmetycznego o różnicy 2.
  \item Najkrótszy bok trapezu prostokątnego opisanego na okręgu o promieniu $r$ ma długość $\frac{5}{3} r$. Oblicz pole trapezu.
  \item Dwa miasta $A$ i $B$ są odległe od siebie o 960 km . Z tych miast wyjechały naprzeciw siebie dwa pociągi, przy czym pociąg z miasta $B$ wyjechał 2 godziny później i jechał z prędkością o $20 \mathrm{~km} /$ godz. większą niż pociąg z miasta $A$. Pociągi te minęły się dokładnie w połowie drogi. Podaj prędkość pociągu, który wyruszył z miasta $A$.
  \item Spośród wierzchołów $n$-kąta foremnego losujemy trzy. Oblicz prawdopodobieństwo $p_{n}$ wylosowania wierzchołków trójkąta prostokątnego. Zbadaj, czy ciąg $\left(p_{n}\right)$ ma granicę.
\end{enumerate}

\section*{ZADANIA PO 20 PUNKTÓW}
\begin{enumerate}
  \setcounter{enumi}{4}
  \item Oblicz sumę długości wszystkich krawędzi ostrosłupa prawidłowego czworokątnego wpisanego w sferę o promieniu $R$, który ma największą objętość.
  \item Znajdź wszystkie pierwiastki równania
\end{enumerate}

$$
3^{\cos ^{2} 2 x}+3^{\sin ^{2} 2 x}=4
$$

które spełniają nierówność

$$
\log _{x}(x+2)<2
$$

\begin{enumerate}
  \setcounter{enumi}{6}
  \item Wielomian $P$ dany wzorem
\end{enumerate}

$$
P(x)=x^{3}+m x+162
$$

ma pierwiastek wielokrotny. Uzasadnij, że nie jest to jedyny pierwiastek tego wielomianu. Wykaż, że $P(-\sqrt[6]{9})$ jest liczbą całkowitą.


\end{document}