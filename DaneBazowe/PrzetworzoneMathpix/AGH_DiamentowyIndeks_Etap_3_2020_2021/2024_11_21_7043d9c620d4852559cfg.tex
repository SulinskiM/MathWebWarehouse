\documentclass[10pt]{article}
\usepackage[polish]{babel}
\usepackage[utf8]{inputenc}
\usepackage[T1]{fontenc}
\usepackage{amsmath}
\usepackage{amsfonts}
\usepackage{amssymb}
\usepackage[version=4]{mhchem}
\usepackage{stmaryrd}

\title{AKADEMIA GÓRNICZO-HUTNICZA im. Stanisława Staszica w Krakowie OLIMPIADA „O DIAMENTOWY INDEKS AGH" 2020/21 MATEMATYKA - ETAP III }

\author{}
\date{}


\begin{document}
\maketitle
\section*{ZADANIA PO 10 PUNKTÓW}
\begin{enumerate}
  \item W czworokącie $A B C D$ kąt wewnętrzny przy wierzchołku $A$ jest kątem prostym. Długości boków są równe $|A B|=|A D|=15,|B C|=3,|C D|=21$. Oblicz pole tego czworokąta.
  \item Dany jest ciąg $\left(a_{n}\right)$, którego $n$-ty wyraz jest równy
\end{enumerate}

$$
a_{n}=40-6 n
$$

Oblicz sumę tych 20 początkowych wyrazów ciągu, które są ujemne i podzielne przez 8.\\
3. Rozwiąż równanie

$$
2^{\cos ^{2} x}+2^{\sin ^{2} x}=3
$$

\begin{enumerate}
  \setcounter{enumi}{3}
  \item W wycieczce bierze udział $2 n$ osób. Każda z nich ma wśród uczestników wycieczki co najmniej $n-1$ innych osób pochodzących z tego samego miasta co ona. Z ilu miast pochodzą uczestnicy wycieczki? W każdym możliwym przypadku podaj liczby uczestników z poszczególnych miast.
\end{enumerate}

\section*{ZADANIA PO 20 PUNKTÓW}
\begin{enumerate}
  \setcounter{enumi}{4}
  \item W urnie znajdują się 2 kule białe i 10 czerwonych.\\
a) Losujemy ze zwracaniem 2 kule. Oblicz prawdopodobieństwo, że wylosujemy kule o różnych kolorach.\\
b) Losujemy bez zwracania $k$ kul. Wyznacz najmniejszą wartość $k$, dla której prawdopodobieństwo wylosowania co najmniej jednej kuli białej jest większe od 0,5 .
  \item Dla jakich wartości parametru $p$ równanie
\end{enumerate}

$$
\frac{\log _{3}(p x+p)}{\log _{3}(3+x)}=2
$$

ma dokładnie jedno rozwiązanie ?\\
7. W stożku o promieniu podstawy $R$ i wysokości $H$ zawarty jest graniastosłup prawidłowy czworokątny tak, że jego podstawa zawiera się w podstawie stożka. Jaką największą objętość może mieć ten graniastosłup?


\end{document}