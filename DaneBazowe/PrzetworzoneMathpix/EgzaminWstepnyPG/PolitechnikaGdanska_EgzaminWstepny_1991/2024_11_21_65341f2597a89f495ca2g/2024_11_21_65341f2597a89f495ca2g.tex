\documentclass[10pt]{article}
\usepackage[polish]{babel}
\usepackage[utf8]{inputenc}
\usepackage[T1]{fontenc}
\usepackage{amsmath}
\usepackage{amsfonts}
\usepackage{amssymb}
\usepackage[version=4]{mhchem}
\usepackage{stmaryrd}

\title{Tematy I części egzaminu z matematyki }

\author{}
\date{}


\begin{document}
\maketitle
dla kandydatów ubiegających się o przyjęcie na I rok studiów dziennych.\\
Kandydat wybierał 3 dowolne zadania. Rozwiązania wybranych zadan oceniane były w skali 0-10 punktów. Egzamin trwał 120 minut.

\begin{enumerate}
  \item Zbadać przebieg zmienności funkcji
\end{enumerate}

\[
y=\frac{4 x+5}{x^{2}-1}
\]

i na tej podstawie ustalić liczbę pierwiastków równania

\[
\frac{4 x+5}{x^{2}-1}=m
\]

w zależności od parametru \(m\).\\
2. W trójkącie \(A B C\) dany jest wierzchołek \(A(1,3)\) oraz równanie środkowej \(y=7\) i równanie wysokości \(x+4 y-51=0\). Wiedząc, że środkowa i wysokość wychodzą z różnych wierzchołków trójkąta podać równania boków tego trójkąta.\\
3. Dla jakich wartości parametru \(m \in R\) równanie

\[
\log _{2}(x+3)-2 \log _{4} x=m
\]

posiada rozwiązanie należące do przedziału \(\langle 3 ; 4)\) ?\\
4. W urnie znajdują się trzy kule białe o numerach 1,2 i 3 oraz pięć kul czarnych o numerach 1,2 , 3, 4 i 5 . Losujemy bez zwracania dwukrotnie po jednej kuli. Jakie jest prawdopodobieństwo tego, że pierwsza z wylosowanych kul będzie biała, a druga będzie kulą o numerze 1?\\
5. Na trójkącie prostokątnym o kącie ostrym \(x\) opisano okrąg. Okrąg ten i trójkąt obracają się dookoła przeciwprostokątnej. Przy jakim \(x\) stosunek objętości kuli powstałej z obrotu okręgu do objętości bryły powstałej z obrotu trójkąta będzie najmniejszy?

\section*{Tematy II części egzaminu z matematyki}
dla kandydatów ubiegających się o przyjęcie na I rok studiów dziennych.\\
Wszystkie zadania były oceniane w skali 0-2 punkty. Egzamin trwał 120 minut.

\begin{enumerate}
  \item Dana jest funkcja \(f(x)=\sin ^{2} 4 x\). Rozwiązać równanie \(f^{\prime}(x)=-2\).
  \item Rozwiązać nierówność \(\log _{x} 5<1\).
  \item Dany jest trójkąt prostokątny o przyprostokątnych długości 3 i 4. Obliczyć wysokość trójkąta poprowadzoną z wierzchołka kąta prostego.
  \item Rozwiązać nierówność \(\frac{1}{x}>2-x\).
  \item Rozwiązać nierówność \(\operatorname{tg}(2 x) \geqslant 1\).
  \item W płaszczyźnie \(0 x y\) zaznaczyć punkty należące do zbioru
\end{enumerate}

\[
A=\{(x, y):|x|<y\}
\]

\begin{enumerate}
  \setcounter{enumi}{6}
  \item Obliczyć \(\lim _{x \rightarrow 1} \frac{\sin 2(x-1)}{3\left(x^{2}-1\right)}\).
  \item Podać resztę z dzielenia wielomianu \(W(x)=5 x^{4}+2 x^{2}+1\) przez dwumian \(x+1\).
  \item W trójkącie o wierzchołkach \(A(3,1,1), B(2,2,1)\) i \(C(2,1,2)\) wyznaczyć kąt wewnętrzny przy wierzchołku \(A\).
  \item Podać liczby naturalne spełniające nierówność \(\binom{n}{2}-n \leqslant 14\).
  \item Dla jakich wartości parametru \(k\) funkcja \(f(x)=\frac{1}{3} x^{3}+\frac{3}{2} x^{2}+k x+1\) będzie rosnąca w całej swojej dziedzinie?
  \item Obliczyc \(\lim _{n \rightarrow \infty} \frac{\sqrt{n^{2}+1}}{\sqrt[3]{8 n^{3}+2 n+1}}\).
  \item Obliczyć prawdopodobieństwo wyrzucenia w pięciu rzutach kostką co najmniej raz liczby oczek nie większej od 3.
  \item Napisać równanie prostej przechodzącej przez punkt \(P(1,3)\) i prostopadłej do prostej \(y=2 x+5\).
  \item Suma wyrazów nieskończonego ciągu geometrycznego o pierwszym wyrazie \(a_{1}=3\) wynosi 5 . Podać iloraz tego ciągu.
\end{enumerate}

\end{document}