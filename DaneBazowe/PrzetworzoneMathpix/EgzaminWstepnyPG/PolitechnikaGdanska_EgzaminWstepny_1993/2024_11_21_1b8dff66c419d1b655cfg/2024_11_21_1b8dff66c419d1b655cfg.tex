\documentclass[10pt]{article}
\usepackage[polish]{babel}
\usepackage[utf8]{inputenc}
\usepackage[T1]{fontenc}
\usepackage{amsmath}
\usepackage{amsfonts}
\usepackage{amssymb}
\usepackage[version=4]{mhchem}
\usepackage{stmaryrd}

\title{EGZAMIN WSTĘPNY Z MATEMATYKI }

\author{}
\date{}


\begin{document}
\maketitle
Zestaw składa się z 30 zadań. Zadania 1-10 oceniane będą w skali 0-2 punkty, zadania \(11-30\) w skali \(0-4\) punkty. Czas trwania egzaminu - 180 minut.

\section*{Powodzenia!}
\begin{enumerate}
  \item Obliczyć \(\lim _{n \rightarrow \infty} \frac{n \sqrt{1+3+5+\ldots+(2 n-1)}}{2 n^{2}+n+1}\).
  \item Rozwiązać nierówność \(x^{2}-4 x+9 \leqslant \frac{18}{x+2}\).
  \item Rozwiązać nierówność \(\log _{0,3}(x+1)>-1\).
  \item Rozwiązać nierówność \(2-|1-2 x|>1\).
  \item Dla jakich wartości parametru \(\alpha \in(0 ; 2 \pi)\) równanie \(\sin 2 x=2 \cos \alpha\) posiada rozwiązanie?
  \item Obliczyć długość wektora \(\vec{a}\), jeżeli \(\vec{a} \circ \vec{b}=7, \vec{a} \| \vec{b}\) i \(\vec{b}=[3,-2,1]\).
  \item Rozwiązać nierówność \(2^{x^{2}}<5^{x}\).
  \item Wykazać, że funkcja \(f(x)=3 x^{3}+4 x+\cos 2 x\) jest rosnąca w całej swojej dziedzinie.
  \item Wyznaczyc te wartości parametru \(k\), dla których prosta \(y=k x+4\) będzie równoległa do prostej \(\left\{\begin{array}{l}x=1+3 t \\ y=2-t\end{array}\right.\).
  \item Dla jakich \(a\) i \(b\) wielomian \(W(x)=12 x^{4}-17 x^{2}+a x+b\) dzieli się bez reszty przez \(2 x^{2}+x-1\) ?
  \item Dany jest trójkąt o wierzchołkach \(A(1,1), B(-1,3), C(3,7)\) i polu \(S\). Przez wierzchołek \(A\) poprowadzić jedną z prostych, ktora dzieli dany trójkąt na dwa trójkąty o polach \(\frac{1}{4} S\) i \(\frac{3}{4} S\). Podać równanie tej prostej.
  \item Znaleźć ekstrema funkcji \(f(x)=(x+3)^{2}(x+8)^{3}\). Ile pierwiastków ma równanie \(f(x)=108\) ?
  \item Dla jakiej wartości parametru \(a\) funkcja
\end{enumerate}

\[
f(x)=\left\{\begin{array}{ccc}
\frac{x \sin x}{\sqrt{x^{2}+4}-2} & \text { dla } & x \neq 0 \\
a & \text { dla } & x=0
\end{array}\right.
\]

będzie funkcją ciągłą w punkcie \(x=0\) ?\\
14. Który z punktów paraboli \(y=x^{2}\) jest położony najbliżej prostej \(y=2 x-2\) ?\\
15. Wykazać, że pole dowolnego wypukłego czworokąta jest równe połowie iloczynu jego przekątnych pomnożonego przez sinus kąta między nimi, \(S=\frac{1}{2} d_{1} d_{2} \sin \alpha\).\\
16. Dany jest ciąg arytmetyczny (o różnicy różnej od zera), w którym suma \(n\) początkowych wyrazów jest równa połowie sumy następnych \(n\) wyrazów. Wyznaczyć iloraz \(\frac{S_{3 n}}{S_{n}}\), gdzie \(S_{k}\) oznacza sumę \(k\) początkowych wyrazów tego ciągu.\\
17. Wykazać, że dwie styczne do paraboli \(y=x^{2}\) poprowadzone z dowolnego punktu prostej \(y=-\frac{1}{4}\) są do siebie prostopadłe.\\
18. Dany jest trójkąt równoramienny o ramionach \(\overline{A C}\) i \(\overline{B C}\) długości 3 cm i podstawie \(\overrightarrow{A B}\) długości 4 cm . Obliczyć iloczyn skalarny \(\overrightarrow{A B} \circ \overrightarrow{B C}\).\\
19. Miary kątów wewnętrznych trójkąta tworzą ciąg arytmetyczny. Najmniejszy bok jest trzy razy mniejszy od największego boku w tym trójkącie. Obliczyć cosinus najmniejszego kąta.\\
20. Ze zbioru liczb \(\{1,2,3,4,5,6,7,8,9,10\}\) losujemy dwukrotnie po jednej liczbie bez zwracania. Obliczyć prawdopodobieństwo tego, że druga z wylosowanych liczb będzie większa od pierwszej.\\
21. Podać definicję asymptoty pionowej i wyznaczyć asymptoty pionowe funkcji \(f(x)=\) \(\frac{1}{x\left(2^{x}-4\right)}\).\\
22. Wyznaczyć najmniejszą i największą wartość funkcji \(f(x)=\cos \left(\frac{\pi}{2} \cdot x\right)-3 x \mathrm{w}\) przedziale \(\langle 0 ; 1\rangle\).\\
23. Dla jakiej wartości parametru \(m\) okrąg \((x-m)^{2}+(y-1)^{2}=1\) będzie styczny do prostej \(3 x+4 y+1=0\) ?\\
24. Wykazać, że równanie \(x=\frac{1}{2} \sin x+a\), gdzie \(a>0\), ma dokładnie jeden pierwiastek w przedziale \(\langle 0 ; a+1\rangle\).\\
25. Z definicji pochodnej obliczyć \(f^{\prime}(3)\), gdy \(f(x)=\sqrt{2 x+3}\).\\
26. Rozwiązać równanie \(\binom{x+3}{2}+\binom{x+1}{x-1}=31\).\\
27. Długość dłuższej podstawy trapezu równoramiennego jest równa 13 cm , a jego obwód jest równy 28 cm . Wyrazić pole trapezu jako funkcję długości ramienia trapezu. Znaleźć dziedzinę i zbiór wartości tej funkcji.\\
28. Dla jakich wartości parametru \(k\) ciąg \(\left(a_{n}\right)\), gdzie \(a_{n}=\frac{n^{k}}{2+4+\ldots+2 n}\), będzie rozbieżny do \(+\infty\) ?\\
29. Dana jest funkcja \(f(x)=\cos ^{2} 3 x+\frac{3}{2} x-\log 5\). Rozwiązać równanie \(f^{\prime}\left(\frac{1}{3} x\right)=0\).\\
30. Dane są liczby \(A=\frac{5678901234}{6789012345}\) i \(B=\frac{5678901235}{6789012346}\). Która z nich jest większa? Swoją odpowiedź uzasadnić.


\end{document}