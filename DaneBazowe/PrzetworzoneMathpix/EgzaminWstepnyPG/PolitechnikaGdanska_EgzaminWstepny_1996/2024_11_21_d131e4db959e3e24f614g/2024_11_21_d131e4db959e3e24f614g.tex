\documentclass[10pt]{article}
\usepackage[polish]{babel}
\usepackage[utf8]{inputenc}
\usepackage[T1]{fontenc}
\usepackage{amsmath}
\usepackage{amsfonts}
\usepackage{amssymb}
\usepackage[version=4]{mhchem}
\usepackage{stmaryrd}

\title{EGZAMIN WSTĘPNY Z MATEMATYKI }

\author{}
\date{}


\begin{document}
\maketitle
Zestaw składa się z 30 zadań. Zadania 1-10 oceniane będą w skali \(0-2\) punkty, zadania \(11-30\) w skali \(0-4\) punkty. Czas trwania egzaminu - 240 minut.

\section*{Powodzenia!}
\begin{enumerate}
  \item Funkcję kwadratową \(y=(x+3)(1-x)\) przedstawić w postaci kanonicznej. Naszkicować jej wykres.
  \item Rozwiązać równanie \(5^{x} \cdot 5^{x^{2}} \cdot 5^{x^{3}}=\frac{1}{5}\).
  \item Rozwiązać równanie \(\log _{\frac{1}{3}}(|x|-1)>2\).
  \item Dla jakich parametrów \(a \in R\) równanie \(\cos ^{2} x=\frac{2 a}{a-2}\) ma rozwiązanie?
  \item Naszkicować wykres funkcji \(y=x \log _{x^{2}}|x|\).
  \item Wyznaczyć te wartości \(x\), dla których punkty \(A(5,5), B(1,3)\) i \(C(x, 0)\) są współliniowe.
  \item Wskazać większą z liczb \(0,4(9)\) i \(\sin \left(\frac{101}{6} \pi\right)\).
  \item Napisać równanie stycznej do wykresu funkcji \(f(x)=\sqrt{2 x-3} \mathrm{w}\) punkcie o odciętej \(x_{0}=6\).
  \item Dana jest funkcja \(f(x)=\cos ^{2} x\). Narysować wykres funkcji \(y=f^{\prime}(x)\) w przedziale \(\langle 0 ; \pi\rangle\).
  \item Zbadać monotoniczność funkcji \(f(x)=x+\frac{1}{x}\).
  \item Dany jest ciąg \(\left(a_{n}\right)\), gdzie \(a_{n}=\frac{(n!)^{2}}{(2 n)!}\). Obliczyć \(\lim _{n \rightarrow \infty} \frac{a_{n+1}}{a_{n}}\).
  \item Rozwiązać nierówność \(g(f(x)) \geqslant 1\), jeśli \(f(x)=3^{x}\) i \(g(x)=\sin x\).
  \item Wyznaczyć wszystkie wielokąty wypukłe, w których liczba przekątnych jest 3 razy większa od liczby wierzchołków.
  \item Rozwiązać równanie \(|\cos x|=\cos x+2 \sin x\) w przedziale \(\langle 0 ; 2 \pi\rangle\).
  \item Rozwiązać nierówność \(\frac{x^{3}-x+6}{x^{2}} \geqslant 0\).
  \item Rozwiązać równanie \(1-\frac{1}{x}+\frac{1}{x^{2}}-\frac{1}{x^{3}}+\ldots=x-1\).
  \item Dla jakich \(x \in R\) ciąg \(2 \log _{3} x, \log _{3}(x-1),-\log _{3} 4\) jest ciągiem arytmetycznym?
  \item Niech \(g\) będzie granicą ciągu \(\left(a_{n}\right)\), gdzie \(a_{n}=\frac{3 n+1}{n+1}\). Od jakiego \(n\) począwszy wyrazy ciągu \(\left(a_{n}\right)\) spełniają nierówność \(\left|a_{n}-g\right|<0,01\) ?
  \item Dla jakich \(a \in R\) funkcja \(f(x)=\left\{\begin{array}{lll}\cos x+a & \text { dla } & x \geqslant 0 \\ \frac{\sin |2 x|}{x} & \text { dla } & x<0\end{array}\right.\) jest ciągła?
  \item Wielomian \(x^{2}+p x+q\) ma pierwiastki \(x_{1}\) i \(x_{2}\). Wskazać trójmian \(x^{2}+b x+c\), którego pierwiastkami są liczby \(x_{1}+1\) i \(x_{2}+1\).
  \item Ze zbioru \(\{1,2, \ldots, 1000\}\) losujemy jedną liczbę. Obliczyć prawdopodobieństwo tego, że nie będzie to liczba podzielna ani przez 6, ani przez 8.
  \item Obliczyć pole trapezu o podstawach długości \(a\) i \(b\), jeżeli wiadomo, że na tym trapezie można opisać okrąg i można w niego wpisać okrąg.
  \item Znaleźć rzut prostokątny punktu \(A(1,-1)\) na prostą \(\left\{\begin{array}{l}x=4 t \\ y=3 t+2 .\end{array}\right.\)
  \item Dane są zbiory \(A=\left\{(x, y): x, y \in R \quad\right.\) i \(\left.\quad x^{2}+y^{2}-2 y \leqslant 1\right\} \quad\) i \(\quad B=\{(x, y): x, y\) \(\in R\) i \(|x|+y \leqslant 1\}\). Narysować na płaszczyźnie układu współrzędnych zbiór \(A \cap B\) i obliczyć jego pole.
  \item Wyznaczyć asymptoty funkcji \(f(x)=\frac{\sqrt{x^{2}-1}-x}{x}\).
  \item Obliczyć \(|\vec{a}-\vec{b}|\), jeśli \(|\vec{a}+\vec{b}|=5,|\vec{a}|=3 \mathrm{i}|\vec{b}|=2 \sqrt{2}\).
  \item Wyznaczyć zbiór wartości funkcji \(y=x \sqrt{4-x^{2}}\).
  \item Rzucamy symetryczną monetą. Obliczyć prawdopodobieństwo zdarzenia, że w szóstym rzucie otrzymamy trzeciego orła.
  \item Uzasadnić, że równanie \(x^{3}+x+7=0 \mathrm{w}\) zbiorze liczb rzeczywistych posiada dokładnie jedno rozwiązanie. Wraz z uzasadnieniem wskazać przedział o długości co najwyżej \(1 / 2\), do którego należy to rozwiązanie.
  \item Ostrosłup przecięto płaszczyzną równoległą do podstawy i dzielącą wysokość w stosunku \(2: 3\). Obliczyć stosunek objętości powstałych brył.
\end{enumerate}

\end{document}