\documentclass[10pt]{article}
\usepackage[polish]{babel}
\usepackage[utf8]{inputenc}
\usepackage[T1]{fontenc}
\usepackage{amsmath}
\usepackage{amsfonts}
\usepackage{amssymb}
\usepackage[version=4]{mhchem}
\usepackage{stmaryrd}

\title{Politechnika Warszawska }

\author{}
\date{}


\begin{document}
\maketitle
\section*{Egzamin wstępny z matematyki}
w dniu 2 lipca 2013 r.

\begin{enumerate}
  \item Dla jakich wartości parametrów \(a\) i \(b\) wielomian\\
\(P(x)=x^{6}+a x+b\) jest podzielny przez \(x^{2}-4\) ?\\
15 pkt.
  \item W wycinek koła o promieniu \(R\) i kącie ostrym \(\alpha\) wpisano okrąg. Obliczyć jego promień.
\end{enumerate}

\section*{20 pkt.}
\begin{enumerate}
  \setcounter{enumi}{2}
  \item Pan AB wpłacił do banku XY \(20000 \mathrm{zł}\) na dwa lata. Kapitalizacja w tym banku jest miesięczna, a roczne oprocentowanie wynosi \(6 \%\). Ile po 2 latach wynoszą oszczędności pana AB?
\end{enumerate}

\section*{15 pkt.}
\begin{enumerate}
  \setcounter{enumi}{3}
  \item Rozwiązać równanie: \(1+4+7+\ldots+x=117\)
\end{enumerate}

\section*{15 pkt.}
\begin{enumerate}
  \setcounter{enumi}{4}
  \item Dla jakich wartości parametru \(p\) dziedziną funkcji
\end{enumerate}

\[
f(x)=\frac{1}{\sqrt{\left(p^{2}-1\right) x^{2}+2(p-1) x+2}}
\]

jest zbiór wszystkich liczb rzeczywistych?

\section*{20 pkt.}
\begin{enumerate}
  \setcounter{enumi}{5}
  \item Długość boku czworościanu foremnego zwiększono o \(15 \%\).
\end{enumerate}

O ile procent wzrosła objętość tego czworościanu?

\section*{15 pkt.}
Zadania należy rozwiązać na arkuszu egzaminacyjnym w polach oznaczonych odpowiednimi numerami zadań. Treści zadań prosimy nie przepisywać. Jeżeli w określonym polu zabraknie miejsca, zadanie można dokończyć na ostatniej stronie. Kartki brudnopisu nie oddaje się i nie będzie ona oceniana. Czas trwania egzaminu 150 minut.


\end{document}