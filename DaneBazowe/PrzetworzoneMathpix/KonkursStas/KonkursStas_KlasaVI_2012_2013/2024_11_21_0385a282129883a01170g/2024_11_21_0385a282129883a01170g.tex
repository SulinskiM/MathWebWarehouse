\documentclass[10pt]{article}
\usepackage[polish]{babel}
\usepackage[utf8]{inputenc}
\usepackage[T1]{fontenc}
\usepackage{amsmath}
\usepackage{amsfonts}
\usepackage{amssymb}
\usepackage[version=4]{mhchem}
\usepackage{stmaryrd}

\title{XIII Konkurs Matematyczny St@ś }

\author{}
\date{}


\begin{document}
\maketitle
\section*{XIV LO im. Stanisława Staszica}
10 czerwca 2013 roku

\section*{klasa VI}
Na rozwiazanie poniższych zadań masz 90 minut. Kolejność rozwiazywania tych zadań jest dowolna. Wszystkie zadania sa jednakowo punktowane. Maksymalna liczbę punktów może uzyskać jedynie petne rozwiqzanie, z uzasadnieniem i odpowiedzia.\\
Używanie korektora i korzystanie z kalkulatora jest niedozwolone.

\begin{enumerate}
  \item Spośród boków czworokąta wybrano trzy i obliczono sumę ich długości. Taką operację przeprowadzono czterokrotnie, wybierając za każdym razem inne trzy boki. Otrzymano sumy: \(10 \mathrm{~cm}, 13 \mathrm{~cm}, 15 \mathrm{~cm}\) i 16 cm . Oblicz długości boków tego czworokąta.
  \item Wykaż, że liczba \(N=\overline{a a b b}\), w której suma cyfr \(a\) i \(b\) jest równa 11, dzieli się przez 121. Uwaga: Liczba \(\overline{x y z t}\) oznacza zapis liczby w układzie dziesiętnym. Liczba ta ma \(t\) jedności, \(z\) dziesiątek, \(y\) setek i \(x\) tysięcy.
  \item Wiedząc, że
\end{enumerate}

\[
\frac{a}{b}=0,4
\]

oblicz wartośc wyrażenia

\[
\frac{a^{2}+2 b^{2}}{2 a^{2}+b^{2}}
\]

\begin{enumerate}
  \setcounter{enumi}{3}
  \item Pewien ostrosłup ma 4024 krawędzi, ile wierzchołków ma ten ostrosłup?
  \item Jednakowym literom należy przyporządkować jednakowe cyfry, różnym, różne. Wyznacz \(C, O, A, L, W, D\), aby działanie było poprawne. Podaj wszystkie możliwości.\\
\(\begin{array}{r}C O C A \\ +C O L A \\ \hline \overline{W O D A}\end{array}\)
\end{enumerate}

\end{document}