\documentclass[10pt]{article}
\usepackage[polish]{babel}
\usepackage[utf8]{inputenc}
\usepackage[T1]{fontenc}
\usepackage{amsmath}
\usepackage{amsfonts}
\usepackage{amssymb}
\usepackage[version=4]{mhchem}
\usepackage{stmaryrd}

\title{XVI Konkurs Matematyczny St@ś }

\author{}
\date{}


\begin{document}
\maketitle
\section*{XIV LO im. Stanisława Staszica}
30 maja 2016 roku

\section*{klasa VI}
Na rozwiazanie poniższych zadań masz 90 minut. Kolejność rozwiazywania tych zadan jest dowolna. Wszystkie zadania sa jednakowo punktowane. Maksymalna liczbę punktów może uzyskać jedynie petne rozwiqzanie, z uzasadnieniem i odpowiedzia.\\
Używanie korektora i korzystanie z kalkulatora jest niedozwolone.

\begin{enumerate}
  \item Wyznacz cyfry \(a, b, c\) tak aby zachodziła równość
\end{enumerate}

\[
\overline{62 a b c}+\overline{6 a b c 2}+\overline{2 a b c 6}+\overline{a b c 62}=235119
\]

Uwaga: Liczba \(\overline{a b c d e}\) oznacza zapis liczby w układzie dziesiętnym. Liczba ta ma \(e\) jedności, \(d\) dziesiątek, \(c\) setek \(b\) tysięcy i \(a\) dziesiątek tysięcy.\\
2. Dane są dwa kąty przyległe \(\measuredangle A O B\) i \(\measuredangle B O C\). Wiedząc, że miara kąta \(B O C\) jest o \(45^{\circ}\) większa od podwojonego kąta \(A O B\) wyznacz miarę kąta \(B O C\).\\
3. Na prostej dane są punkty \(A, B, C, D\) takie, że punkt \(B\) należy do odcinka \(A C\), punkt \(C\) należy do odcinka \(B D, A C=4\) oraz \(B D=5\). Wiedząc, że punkt \(E\) należy do odcinka \(B C\), oblicz

\[
A E+B E+C E+D E
\]

\begin{enumerate}
  \setcounter{enumi}{3}
  \item Zapas żywności wystarczy dla pewnej liczby osób na 30 dni. Gdyby było o dwie osoby mniej, ten zapas starczyłby na 40 dni. Ile było osób?
  \item Jednakowym literom należy przyporządkować jednakowe cyfry, różnym różne. Wyznacz \(I, K, A, R, S\), aby działanie było poprawne.
\end{enumerate}

\[
\begin{array}{r}
I K A R \\
+I K A R \\
\hline K R A S A
\end{array}
\]


\end{document}