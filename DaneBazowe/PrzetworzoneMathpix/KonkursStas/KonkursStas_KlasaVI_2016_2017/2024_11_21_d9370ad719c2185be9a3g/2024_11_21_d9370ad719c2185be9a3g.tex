\documentclass[10pt]{article}
\usepackage[polish]{babel}
\usepackage[utf8]{inputenc}
\usepackage[T1]{fontenc}
\usepackage{amsmath}
\usepackage{amsfonts}
\usepackage{amssymb}
\usepackage[version=4]{mhchem}
\usepackage{stmaryrd}

\title{XVII Konkurs Matematyczny St@ś \\
 XIV LO im. Stanisława Staszica \\
 29 maja 2017 roku }

\author{}
\date{}


\begin{document}
\maketitle
\section*{klasa VI}
Na rozwiazanie poniższych zadań masz 90 minut. Kolejność rozwiazywania tych zadań jest dowolna. Wszystkie zadania sa jednakowo punktowane. Maksymalna liczbę punktów może uzyskać jedynie petne rozwiqzanie, z uzasadnieniem i odpowiedziq.\\
Używanie korektora i korzystanie z kalkulatora jest niedozwolone.

\begin{enumerate}
  \item Ile zer w zapisie dziesiętnym ma liczba
\end{enumerate}

\[
\frac{10^{2016}-1}{111} ?
\]

\begin{enumerate}
  \setcounter{enumi}{1}
  \item W trapezie \(A B C D(A B>C D)\) wybrano na boku \(A B\) taki punkt \(E\), że odcinek \(D E\) jest równoległy do boku \(B C\). Obwód trójkąta \(A E D\) jest równy 80 . Długość odcinka \(E B\) jest równa 30. Oblicz obwód trapezu \(A B C D\).
  \item Udowodnij tożsamość
\end{enumerate}

\[
\frac{1}{k-1}-\frac{1}{k}=\frac{1}{k(k-1)}
\]

a następnie rozwiąż równanie

\[
\frac{x}{1 \cdot 2}+\frac{x}{2 \cdot 3}+\ldots+\frac{x}{1006 \cdot 2017}=2016
\]

\begin{enumerate}
  \setcounter{enumi}{3}
  \item Pewien graniastosłup ma trzy razy więcej wierzchołków niż pewien ostrosłup. Liczba ścian graniastosłupa jest o 17 większa od liczby ścian ostrosłupa. Ile krawędzi ma ostrosłup?
  \item W czworokącie \(A B C D\) kąty przy wierzchołkach \(B\) i \(D\) są proste, \(A B=B C\), a \(H\) jest takim punktem na boku \(A D\), że odcinek \(B H\) jest prostopadły do odcinka \(A D\). Wiadomo także, że \(B H=1\). Oblicz pole czworokąta \(A B C D\).
\end{enumerate}

\end{document}