\documentclass[10pt]{article}
\usepackage[polish]{babel}
\usepackage[utf8]{inputenc}
\usepackage[T1]{fontenc}
\usepackage{amsmath}
\usepackage{amsfonts}
\usepackage{amssymb}
\usepackage[version=4]{mhchem}
\usepackage{stmaryrd}

\title{VII Konkurs matematyczny St@ś }

\author{}
\date{}


\begin{document}
\maketitle
XIV LO im. Stanisława Staszica\\
28 maja 2007 roku

\section*{klasa V}
Na rozwiazanie poniższych zadań masz 90 minut. Kolejność rozwiazywania tych zadan jest dowolna. Wszystkie zadania sa jednakowo punktowane. Maksymalna liczbę punktów może uzyskać jedynie petne rozwiqzanie, z uzasadnieniem i odpowiedziq.\\
Używanie korektora i korzystanie z kalkulatora jest niedozwolone.

\begin{enumerate}
  \item Książka ma 240 stron, a jej grubość razem z okładką wynosi \(1,18 \mathrm{~cm}\). Grubość kartonu na okładkę jest równa \(0,05 \mathrm{~cm}\). Oblicz grubość jednej kartki
  \item Niektóre cyfry w liczbach poniżej zosały zastąpione gwiazdkami. Czy znaki \(<,=,>\) zostały wpisane prawidłowo? Wpisz TAK, NIE lub NIE WIADOMO.\\
(a) \(3, * 5<4, * *\)\\
(b) \(5,9 *>5, * 9\)\\
(c) \(9,4 *=9,40\)\\
(d) \(1, * 8 *>1,99\)\\
(e) \(8,900>8, *\)\\
(f) \(0,0 *>0,1 *\)
  \item Staś miał trzy patyczki: czerwony, zielony i niebieski. Ich długości to \(2 \mathrm{~cm}, 3 \mathrm{~cm}, 5 \mathrm{~cm}\) (kolejność tych liczb nie odpowiada podanej kolejności kolorów). Za pomocą tych patyczków Staś zmierzył długość krawędzi stołu. Zielony zmieścił się 75 razy, zaś niebieski 50 razy. Czerwony zmieścił się też całkowitą liczbę razy. Jaka to liczba?
  \item Dwie przekątne wychodzące z jednego wierzchołka pięciokąta dzielą go na dwa różne trójkąty równoboczne i jeden trójkąt prostokątny rownoramienny. Podaj miary kątów tego pięciokąta.
  \item Podaj trzy kolejne liczby naturalne, których iloczyn jest 100 razy większy od największej liczby czterocyfrowej.
\end{enumerate}

\end{document}