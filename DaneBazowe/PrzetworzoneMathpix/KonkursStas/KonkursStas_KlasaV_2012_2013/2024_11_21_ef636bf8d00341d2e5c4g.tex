\documentclass[10pt]{article}
\usepackage[polish]{babel}
\usepackage[utf8]{inputenc}
\usepackage[T1]{fontenc}
\usepackage{amsmath}
\usepackage{amsfonts}
\usepackage{amssymb}
\usepackage[version=4]{mhchem}
\usepackage{stmaryrd}

\title{XIII Konkurs Matematyczny St@ś }

\author{}
\date{}


\begin{document}
\maketitle
\section*{XIV LO im. Stanisława Staszica}
10 czerwca 2013 roku

\section*{klasa V}
Na rozwiazanie poniższych zadań masz 90 minut. Kolejność rozwiazywania tych zadań jest dowolna. Wszystkie zadania sa jednakowo punktowane. Maksymalna liczbę punktów może uzyskać jedynie petne rozwiqzanie, z uzasadnieniem i odpowiedzia.\\
Używanie korektora i korzystanie z kalkulatora jest niedozwolone.

\begin{enumerate}
  \item Spośród boków trójkąta wybrano dwa i obliczono sumę ich długości. Taką operację przeprowadzono trzykrotnie, wybierając za każdym razem inne dwa boki. Otrzymano sumy: 11 cm , 13 cm i 14 cm . Oblicz długości boków tego trójkąta.
  \item Wykaż, ze liczba \(N=\overline{a a b b}\), jest podzielna przez 11.
\end{enumerate}

Uwaga: Liczba \(\overline{x y z t}\) oznacza zapis liczby w układzie dziesiętnym. Liczba ta ma \(t\) jedności, \(z\) dziesiątek, \(y\) setek i \(x\) tysięcy.\\
3. Wiedząc, że

\[
\frac{a}{b}=0,4
\]

oblicz wartośc wyrażenia

\[
\frac{a+2 b}{2 a+b}
\]

\begin{enumerate}
  \setcounter{enumi}{3}
  \item Pewien graniastosłup prosty ma 2013 krawędzi. Ile ścian ma ten graniastosłup?
  \item Jednakowym literom należy przyporządkować jednakowe cyfry, różnym, różne. Wyznacz \(T, H, I, S, E, A, Y\), aby działanie było poprawne. Podaj wszystkie możliwości.
\end{enumerate}

\[
\begin{array}{r}
\text { THIS } \\
+\quad+S \\
\hline E A S Y
\end{array}
\]


\end{document}