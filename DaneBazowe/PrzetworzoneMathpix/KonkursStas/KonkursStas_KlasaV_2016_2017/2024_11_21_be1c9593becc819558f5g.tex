\documentclass[10pt]{article}
\usepackage[polish]{babel}
\usepackage[utf8]{inputenc}
\usepackage[T1]{fontenc}
\usepackage{amsmath}
\usepackage{amsfonts}
\usepackage{amssymb}
\usepackage[version=4]{mhchem}
\usepackage{stmaryrd}

\title{XVII Konkurs Matematyczny St@ś \\
 XIV LO im. Stanisława Staszica \\
 29 maja 2017 roku }

\author{}
\date{}


\begin{document}
\maketitle
\section*{klasa V}
Na rozwiqzanie poniższych zadań masz 90 minut. Kolejność rozwiazywania tych zadań jest dowolna. Wszystkie zadania sa jednakowo punktowane. Maksymalna liczbę punktów może uzyskać jedynie petne rozwiazanie, z uzasadnieniem \(\boldsymbol{i}\) odpowiedzia.\\
Używanie korektora i korzystanie z kalkulatora jest niedozwolone.

\begin{enumerate}
  \item Wyznacz cyfry \(a\) i \(b\) tak, aby
\end{enumerate}

\[
\frac{\overline{a b}}{8}=\frac{\overline{b a}}{14}
\]

Podaj wszystkie takie liczby. Uwaga: Liczba \(\overline{x y}\) oznacza zapis liczby w układzie dziesiętnym. Liczba ta ma \(y\) jedności i \(x\) dziesiątek.\\
2. Udowodnij tożsamość

\[
\frac{1}{k}-\frac{1}{k+1}=\frac{1}{k(k+1)}
\]

a następnie oblicz

\[
\frac{1}{1 \cdot 2}+\frac{1}{2 \cdot 3}+\ldots+\frac{1}{2016 \cdot 2017}
\]

\begin{enumerate}
  \setcounter{enumi}{2}
  \item Jednakowym literom należy przyporządkować jednakowe cyfry, różnym różne. Wyznacz \(K, O, T\), aby działanie było poprawne.
\end{enumerate}

\[
\begin{array}{r}
K O T \\
+K T O \\
\hline \overline{O T K}
\end{array}
\]

\begin{enumerate}
  \setcounter{enumi}{3}
  \item Pewien graniastosłup ma dwa razy więcej wierzchołków niż pewien ostrosłup. O ile więcej ścian ma ten graniastosłup niż ostrosłup?
  \item Obwód trójkąta \(A B C\) wynosi 21. Jeden z jego boków zwiększono dwukrotnie. Czy po tej zmianie obwód trójkąta może być równy 33 ?
\end{enumerate}

\end{document}