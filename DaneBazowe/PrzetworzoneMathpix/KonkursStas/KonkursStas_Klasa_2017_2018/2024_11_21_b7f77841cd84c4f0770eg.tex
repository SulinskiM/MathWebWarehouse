\documentclass[10pt]{article}
\usepackage[polish]{babel}
\usepackage[utf8]{inputenc}
\usepackage[T1]{fontenc}
\usepackage{amsmath}
\usepackage{amsfonts}
\usepackage{amssymb}
\usepackage[version=4]{mhchem}
\usepackage{stmaryrd}

\title{XVIII Konkurs Matematyczny St@ś \\
 XIV LO im. Stanisława Staszica \\
 28 maja 2018 roku }

\author{}
\date{}


\begin{document}
\maketitle
Na rozwiazanie poniższych zadań masz 90 minut. Kolejność rozwiazywania tych zadań jest dowolna.\\
Wszystkie zadania sa jednakowo punktowane. Maksymalnq liczbę punktów mȯ̇e uzyskać jedynie petne rozwiazanie, z uzasadnieniem \(\boldsymbol{i}\) odpowiedzia.\\
Używanie korektora \(i\) korzystanie z kalkulatora jest niedozwolone.

\begin{enumerate}
  \item W trójkącie \(A B C\) punkt \(K\) jest środkiem boku \(B C\). Punkt \(L\) leży na boku \(A B\) i prosta \(C L\) dzieli kąt \(A C B\) na dwa równe kąty. Proste \(A K\) i \(C L\) są prostopadłe. Udowodnij, że jeden z boków trójkąta \(A B C\) jest dwa razy dłuższy od jednego z pozostałych boków.
  \item Wyznacz liczbę \(\overline{x y z t}\) tak, aby
\end{enumerate}

\[
\overline{x y z t}+\overline{t z y x}=6555 .
\]

Uwaga: Liczba \(\overline{x y z t}\) oznacza zapis liczby w układzie dziesiętnym. Liczba ta ma \(t\) jedności, \(z\) dziesiątek, \(y\) setek i \(x\) tysięcy.\\
3. Oblicz resztę z dzielenia liczby

\[
A=4988+4989+4990+\ldots+5011+5012+5013
\]

przez 5000.\\
4. Kwadrat \(A B C D\) i trójkąt równoboczny \(C D E\) mają wspólny bok \(C D\) (punkt \(E\) leży na zewnątrz kwadratu). Odcinki \(A E\) i \(C D\) przecinają się w punkcie \(F\). Oblicz miarę kąta \(A F C\).\\
5. Liczby \(x, y, z, t\) spełniają warunek:

\[
\frac{14 x+8 y+14 z+8 t}{7 x-5 y+7 z-5 t}=4
\]

Oblicz wartość wyrażenia:

\[
\frac{t+y}{x+z} .
\]


\end{document}