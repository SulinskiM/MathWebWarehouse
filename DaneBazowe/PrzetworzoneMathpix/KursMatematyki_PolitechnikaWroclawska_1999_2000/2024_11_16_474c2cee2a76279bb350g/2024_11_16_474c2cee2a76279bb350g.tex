\documentclass[10pt]{article}
\usepackage[polish]{babel}
\usepackage[utf8]{inputenc}
\usepackage[T1]{fontenc}
\usepackage{amsmath}
\usepackage{amsfonts}
\usepackage{amssymb}
\usepackage[version=4]{mhchem}
\usepackage{stmaryrd}

\title{KORESPONDENCYJNY KURS PRZYGOTOWAWCZY Z MATEMATYKI }

\author{}
\date{}


\begin{document}
\maketitle
\section*{PRACA KONTROLNA nr 1}
październik 1999 r

\begin{enumerate}
  \item Stop składa się z $40 \%$ srebra próby $0,6,30 \%$ srebra próby 0,7 oraz 1 kg srebra próby 0,8 . Jaka jest waga i jaka jest próba tego stopu?
  \item Rozwiązać równanie
\end{enumerate}

$$
3^{x}+1+3^{-x}+\ldots=4
$$

którego lewa strona jest sumą nieskończonego ciągu geometrycznego.\\
3. W trójkącie $A B C$ znane są wierzchołki $A(0,0)$ oraz $B(4,-1)$. Wiadomo, że w punkcie $H(3,2)$ przecinają się proste zawierające wysokości tego trójkąta. Wyznaczyć współrzędne wierzchołka $C$. Wykonać odpowiedni rysunek.\\
4. Rozwiązać równanie

$$
\cos 4 x=\sin 3 x
$$

\begin{enumerate}
  \setcounter{enumi}{4}
  \item Wykonać staranny wykres funkcji
\end{enumerate}

$$
f(x)=\left|\log _{2}(x-2)^{2}\right|
$$

\begin{enumerate}
  \setcounter{enumi}{5}
  \item Rozwiązać nierówność
\end{enumerate}

$$
\frac{1}{x^{2}} \geqslant \frac{1}{x+6}
$$

\begin{enumerate}
  \setcounter{enumi}{6}
  \item W ostrosłupie prawidłowym sześciokątnym krawędź podstawy ma długość $p$, a krawędź boczna długość $2 p$. Obliczyć cosinus kąta dwuściennego między sąsiednimi ścianami bocznymi tego ostrosłupa.
  \item Wyznaczyć równania wszystkich prostych stycznych do wykresu funkcji $y=\frac{2 x+10}{x+4}$, które są równoległe do prostej stycznej do wykresu funkcji $y=\sqrt{1-x}$ w punkcie $x=0$. Rozwiązanie zilustrować rysunkiem.
\end{enumerate}

\section*{PRACA KONTROLNA nr 2}
\begin{enumerate}
  \item Udowodnić, że dla każdego $n$ naturalnego wielomian $x^{4 n-2}+1$ jest podzielny przez trójmian kwadratowy $x^{2}+1$.
  \item W równoramienny trójkąt prostokątny o polu powierzchni $S=10 \mathrm{~cm}^{2}$ wpisano prostokąty w ten sposób, że jeden z jego boków leży na przeciwprostokątnej, a pozostałe wierzchołki znajdują się na przyprostokątnych. Znaleźć ten z prostokątów, który ma najkrótszą przekątną i obliczyć jej długość.
  \item Rozwiązać nierówność
\end{enumerate}

$$
\log _{125} 3 \cdot \log _{x} 5+\log _{9} 8 \cdot \log _{4} x>1
$$

\begin{enumerate}
  \setcounter{enumi}{3}
  \item Znaleźć wszystkie wartości parametru $p$, dla których wykres funkcji $y=x^{2}+4 x+3$ leży nad prosta $y=p x+1$.
  \item Zbadać liczbę rozwiązań równania
\end{enumerate}

$$
||x+5|-1|=m
$$

w zależności od parametru $m$.\\
6. Rozwiązać układ równań

$$
\left\{\begin{array}{l}
x^{2}+y^{2}=50 \\
(x-2)(y+2)=-9
\end{array}\right.
$$

Podać interpretację geometryczną tego układu i wykonać odpowiedni rysunek.\\
7. Wyznaczyć na osi x-ów punkty A i B, z których okrąg $x^{2}+y^{2}-4 x+2 y=20$ widać pod kątem prostym tzn. styczne do okręgu wychodzące z każdego z tych punktów są do siebie prostopadłe. Obliczyć pole figury ograniczonej stycznymi do okręgu przechodzącymi przez punkty A i B. Wykonać staranny rysunek.\\
8. W przedziale $[0,2 \pi]$ rozwiązać równanie

$$
1-\operatorname{tg}^{2} x+\operatorname{tg}^{4} x-\operatorname{tg}^{6} x+\ldots=\sin ^{2} 3 x
$$

\section*{PRACA KONTROLNA nr 3}
grudzień 1999r

\begin{enumerate}
  \item Nie korzystając z metod rachunku różniczkowego wyznaczyć dziedzinę i zbiór wartości funkcji
\end{enumerate}

$$
y=\sqrt{2+\sqrt{x}-x}
$$

\begin{enumerate}
  \setcounter{enumi}{1}
  \item Jednym z wierzchołków rombu o polu $20 \mathrm{~cm}^{2}$ jest $A(6,3)$, a jedna z przekątnych zawiera się w prostej o równaniu $2 x+y=5$. Wyznaczyć równania prostych, w których zawierają się boki $\overline{A B}$ i $\overline{A D}$.
  \item Stosując zasadę indukcji matematycznej udowodnić prawdziwość wzoru
\end{enumerate}

$$
3\left(1^{5}+2^{5}+\ldots+n^{5}\right)+\left(1^{3}+2^{3}+\ldots+n^{3}\right)=\frac{n^{3}(n+1)^{3}}{2}
$$

\begin{enumerate}
  \setcounter{enumi}{3}
  \item Ostrosłup prawidłowy trójkątny ma pole powierzchni całkowitej $P=12 \sqrt{3} \mathrm{~cm}^{2}$, a kąt nachylenia ściany bocznej do płaszczyzny podstawy $\alpha=60^{\circ}$. Obliczyć objętość tego ostrosłupa.
  \item Wśród trójkątów równoramiennych wpisanych w koło o promieniu $R$ znaleźć ten, który ma największe pole.
  \item Przeprowadzić badanie przebiegu funkcji $y=\frac{1}{2} x^{2} \sqrt{5-2 x}$ i wykonać jej staranny wykres.
  \item W trapezie równoramiennym dane są ramię $r$, kąt ostry przy podstawie $\alpha$ oraz suma długości przekątnej i dłuższej podstawy wynosząca $d$. Obliczyć pole trapezu oraz promień okręgu opisanego na tym trapezie. Ustalić warunki istnienia rozwiązania. Następnie podstawić $\alpha=30^{0}, \quad r=\sqrt{3} \mathrm{~cm}$ i $d=6 \mathrm{~cm}$.
  \item Rozwiązać nierówność
\end{enumerate}

$$
|\cos x+\sqrt{3} \sin x| \leqslant \sqrt{2}, \quad x \in[0,3 \pi]
$$

\section*{PRACA KONTROLNA nr 4}
styczeń 2000r

\begin{enumerate}
  \item Rozwiązać równanie $16+19+22+\cdots+x=2000$, którego lewa strona jest sumą pewnej liczby kolejnych wyrazów ciągu arytmetycznego.
  \item Spośród cyfr $0,1, \cdots, 9$ losujemy bez zwracania pięć cyfr. Obliczyć prawdopodobieństwo tego, że z otrzymanych cyfr można utworzyć liczbę podzielną przez 5.
  \item Zbadać, czy istnieje pochodna funkcji $f(x)=\sqrt{1-\cos x} \mathrm{w}$ punkcie $x=0$. Wynik zilustrować na wykresie funkcji $f(x)$.
  \item Udowodnić, że dwusieczne kątów wewnętrznych równoległoboku tworzą prostokąt, którego przekątna ma długość równą różnicy długości sąsiednich boków równoległoboku.
  \item Rozwiązać układ nierówności
\end{enumerate}

$$
\left\{\begin{array}{l}
x+y \leqslant 3 \\
\log _{y}\left(2^{x+1}+32\right) \leqslant 2 \log _{y}\left(8-2^{x}\right)
\end{array}\right.
$$

i zaznaczyć zbiór jego rozwiązań na płaszczyźnie.\\
6. Wyznaczyć równanie zbioru wszystkich punktów płaszczyzny Oxy będących środkami okręgów stycznych wewnętrznie do okręgu $x^{2}+y^{2}=25$ i równocześnie stycznych zewnętrznie do okręgu $(x+2)^{2}+y^{2}=1$. Jaką linię przedstawia znalezione równanie? Sporządzić staranny rysunek.\\
7. Zbadać iloczyn pierwiastków rzeczywistych równania

$$
m^{2} x^{2}+8 m x+4 m-4=0
$$

jako funkcję parametru m. Sporządzić wykres tej funkcji.\\
8. Podstawą czworościanu ABCD jest trójkąt równoboczny ABC o boku a, ściana boczna BCD jest trójkątem równoramiennym prostopadłym do płaszczyzny podstawy, a kąt płaski ściany bocznej przy wierzchołku A jest równy $\alpha$. Obliczyć pole powierzchni kuli opisanej na tym czworościanie.

\section*{PRACA KONTROLNA nr 5}
\begin{enumerate}
  \item Narysować na płaszczyźnie zbiór $A$ wszystkich punktów $(x, y)$, których współrzędne spełniają warunki
\end{enumerate}

$$
||x|-y| \leqslant 1, \quad-1 \leqslant x \leqslant 2
$$

i znaleźć punkt zbioru $A$ leżący najbliżej punktu $P(0,4)$.\\
2. Obliczyć $\sin ^{3} \alpha+\cos ^{3} \alpha$ wiedząc, że $\sin 2 \alpha=\frac{1}{4}$ oraz $\alpha \in(0,2 \pi)$.\\
3. Rozważmy rodzinę prostych przechodzących przez punkt $P(0,-1)$ i przecinających parabolę $y=\frac{1}{4} x^{2} \mathrm{w}$ dwóch punktach. Wyznaczyć równanie środków powstałych w ten sposób cięciw paraboli. Sporządzić rysunek i opisać otrzymaną krzywą.\\
4. Rozwiązać równanie

$$
\sqrt{x+\sqrt{x^{2}-x+2}}-\sqrt{x-\sqrt{x^{2}-x+2}}=4
$$

\begin{enumerate}
  \setcounter{enumi}{4}
  \item Dwóch strzelców wykonuje strzelanie. Pierwszy trafia do celu z prawdopodobieństwem $\frac{2}{3}$ w każdym strzale i wykonuje 4 strzały, a drugi trafia z prawdpodobieństwem $\frac{1}{3}$ i wykonuje 8 strzałów. Który ze strzelców ma większe prawdopodobieństwo uzyskania co najmniej trzech trafień do celu, jeśli wyniki kolejnych strzałów są wzajemnie niezależne?
  \item Do naczynia w kształcie walca o promieniu podstawy R wrzucono trzy jednakowe kulki o promieniu r, przy czym $R<2 r<2 R$. Okazało się, że płaska pokrywa naczynia jest styczna do kulki znajdującej się najwyżej w naczyniu. Obliczyć wysokość naczynia.
  \item Dla jakich wartości parametru $m$ funkcja
\end{enumerate}

$$
f(x)=\frac{x^{3}}{m x^{2}+6 x+m}
$$

jest określona i rosnąca na całej prostej rzeczywistej.\\
8. Dany jest trójkąt o wierzchołkach $A(-2,1), B(-1,-6), \quad C(2,5)$. Posługując się rachunkiem wektorowym obliczyć cosinus kąta pomiędzy dwusieczną kąta $A$ i środkową boku $\overline{B C}$. Wykonać rysunek.

\section*{PRACA KONTROLNA nr 6}
marzec 2000r

\begin{enumerate}
  \item Rozwiązać równanie
\end{enumerate}

$$
x^{\log _{2}(2 x-1)+\log _{2}(x+2)}=\frac{1}{x^{2}}
$$

\begin{enumerate}
  \setcounter{enumi}{1}
  \item Styczna do okręgu $x^{2}+y^{2}-4 x-2 y=5$ w punkcie $\mathrm{M}(-1,2)$, prosta $l$ o równaniu $24 x+5 y-12=0$ oraz oś Ox tworzą trójkąt. Obliczyć pole tego trójkąta i wykonać rysunek.
  \item Udowodnić prawdziwość tożsamości
\end{enumerate}

$$
\cos \alpha+\cos \beta+\cos \gamma=4 \cos \frac{\alpha+\beta}{2} \cos \frac{\beta+\gamma}{2} \cos \frac{\gamma+\alpha}{2}
$$

gdzie $\alpha, \beta, \gamma$ są kątami ostrymi, których suma wynosi $\frac{\pi}{2}$.\\
4. Długości krawędzi prostopadłościanu o objętości $V=8$ tworzą ciąg geometryczny, a stosunek długości przekątnej prostopadłościanu do najdłuższej z przekątnych ścian tej bryły wynosi $\frac{3}{4} \sqrt{2}$. Obliczyć pole powierzchni całkowitej prostopadłościanu.\\
5. Z urny zawierającej siedem kul czarnych i trzy białe wybrano losowo trzy kule i przełożono do drugiej, pustej urny. Jakie jest prawdopodobieństwo wylosowania kuli białej z drugiej urny?\\
6. Prostokąt obraca się wokół swojej przekątnej. Obliczyć objętość powstałej bryły, jeśli przekątna ma długość $d$, a kąt pomiędzy przekątną, a dłuższym bokiem ma miarę $\alpha$. Wykonać odpowiedni rysunek.\\
7. Wyznaczyć największą i najmniejszą wartość funkcji

$$
f(x)=x^{5 / 2}-10 x^{3 / 2}+40 x^{1 / 2}
$$

w przedziale $[1,5]$.\\
8. Stosunek promienia okręgu wpisanego do promienia okręgu opisanego na trójkącie prostokątnym jest równy $k$. Obliczyć w jakim stosunku środek okręgu wpisanego w ten trójkąt dzieli dwusieczną kąta prostego. Określić dziedzinę dla parametru $k$.

\section*{PRACA KONTROLNA nr 7}
\begin{enumerate}
  \item Rozwiązać nierówność
\end{enumerate}

$$
\left|9^{x}-2\right|<3^{x+1}-2
$$

\begin{enumerate}
  \setcounter{enumi}{1}
  \item Wyznaczyć równanie krzywej będącej obrazem okręgu $(x+1)^{2}+(y-6)^{2}=4$ w powinowactwie prostokątnym o osi Ox i stosunku $k=\frac{1}{2}$. Obliczyć pole figury ograniczonej tą krzywą. Wykonać staranny rysunek.
  \item Pewien zbiór zawiera dokładnie 67 podzbiorów o co najwyżej dwóch elementach. Ile podzbiorów siedmioelementowych zawiera ten zbiór ?
  \item Na kole o promieniu $R$ opisano trapez o kątach przy dłuższej podstawie $15^{0}$ i $45^{0}$. Obliczyć stosunek pola koła do pola tego trapezu.
  \item Rozwiązać układ równań
\end{enumerate}

$$
\left\{\begin{aligned}
m x- & 6 y \\
m x & =3 \\
2 x+(m-7) y & =m-1
\end{aligned}\right.
$$

w zależności od parametru rzeczywistego $m$. Podać wszystkie rozwiązania (i odpowiadające im wartości parametru $m$ ), dla których $x$ jest równe $y$.\\
6. Rozwiązać nierówność

$$
\sin 2 x<\sin x
$$

w przedziale $\left[-\frac{\pi}{2}, \frac{\pi}{2}\right]$. Rozwiązanie zilustrować starannym wykresem.\\
7. Ostrosłup przecięto na trzy części dwiema płaszczyznami równoległymi do jego podstawy. Pierwsza płaszczyzna jest położona w odległości $d_{1}=2 \mathrm{~cm}$, a druga w odległości $d_{2}=3$ cm od podstawy. Pola przekrojów ostrosłupa tymi płaszczyznami równe są odpowiednio $S_{1}=25 \mathrm{~cm}^{2}$ oraz $S_{2}=16 \mathrm{~cm}^{2}$. Obliczyć objętość tego ostrosłupa oraz objętość najmniejszej części.\\
8. Trylogię składającą się z dwóch powieści dwutomowych oraz jednej jednotomowej ustawiono przypadkowo na półce. Jakie jest prawdopodobieństwo tego, że tomy\\
a) obydwu, b) co najmniej jednej z dwutomowych powieści znajdują się obok siebie i przy tym tom I z lewej, a tom II z prawej strony.


\end{document}