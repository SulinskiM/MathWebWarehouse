\documentclass[10pt]{article}
\usepackage[polish]{babel}
\usepackage[utf8]{inputenc}
\usepackage[T1]{fontenc}
\usepackage{amsmath}
\usepackage{amsfonts}
\usepackage{amssymb}
\usepackage[version=4]{mhchem}
\usepackage{stmaryrd}
\usepackage{hyperref}
\hypersetup{colorlinks=true, linkcolor=blue, filecolor=magenta, urlcolor=cyan,}
\urlstyle{same}

\title{PRACA KONTROLNA nr 1 - POZIOM PODSTAWOWY }

\author{}
\date{}


\begin{document}
\maketitle
\begin{enumerate}
  \item Wykaż, że różnica kwadratów dwóch liczb nieparzystych jest podzielna przez 8.
  \item Właściciel hurtowni sprzedał $\frac{1}{3}$ partii bananów po założonej przez siebie cenie. Ponieważ pozostałe owoce zaczęły zbyt szybko dojrzewać, więc obniżył ich cenę o $30 \%$. Dzięki temu sprzedał $60 \%$ aktualnego stanu. Resztę bananów udało mu się sprzedać dopiero, gdy ustalił ich cenę na poziomie $\frac{1}{5}$ ceny początkowej. Ile procent zaplanowanego zysku stanowi kwota uzyskana ze sprzedaży? W jakiej cenie (w porównaniu z założoną) powinien sprzedać pierwszą partię towaru, żeby jednokrotna obniżka ich ceny o $25 \%$ pozwoliła na sprzedanie wszystkich owoców i uzyskanie zaplanowanego początkowo zysku?
  \item Narysuj wykres funkcji
\end{enumerate}

$$
f(x)=\frac{|x-1|+x}{|x+1|}
$$

Następnie rozwiąż nierówność $f(x) \geqslant 1$ i, korzystając z wykresu, podaj jej interpretację graficzną.\\
4. Wykresem funkcji $f(x)=x^{2}+b x+c$ jest parabola o wierzchołku w punkcie $(3,-1)$. Podaj wzór funkcji, której wykres jest obrazem symetrycznym tej paraboli:\\
a) względem prostej $x=1$,\\
b) względem punktu $(1,0)$.

Sporządź staranne wykresy wszystkich funkcji.\\
5. Oblicz

$$
\frac{\sqrt{2 \sin ^{3} \alpha+3 \sin \alpha \cos ^{2} \alpha}}{\sin \alpha \sqrt{\cos \alpha}+\cos \alpha \sqrt{\sin \alpha}}
$$

wiedząc, że $\operatorname{tg} \alpha=\frac{1}{2}$. Wynik podaj bez niewymierności w mianowniku.\\
6. Z miejscowości $A$ i $B$ odległych o 90 kilometrów wyruszyli dwaj rowerzyści. Adam wyjechał z $A$ o godzinę wcześniej niż Bartek z $B$. Od momentu spotkania Adam jechał do $B 90$ minut, a Bartek dotarł do $A$ po 4 godzinach. Z jaką prędkością jechał każdy z rowerzystów?

\section*{PRACA KONTROLNA nr 1 - POZIom RoZsZERzony}
\begin{enumerate}
  \item Wykaż, że różnica czwartych potęg dwóch liczb nieparzystych jest podzielna przez 16.
  \item 31 grudnia Kowalski zaciągnął pożyczkę 4000 złotych oprocentowaną w wysokości $16 \%$ w skali roku. Zobowiązał się spłacić ją w ciągu roku w czterech równych ratach płatnych 31 marca, 30 czerwca, 30 września i 31 grudnia. Oprocentowanie pożyczki liczy się od 1 stycznia, a odsetki od kredytu naliczane są w terminach płatności rat. Oblicz wysokość tych rat w zaokragleniu do pełnych groszy.
  \item Narysuj wykres funkcji
\end{enumerate}

$$
f(x)=\frac{|x+1|+x}{|x-1|}
$$

i wyznacz zbiór jej wartości. Następnie rozwiąż nierówność $f(x-1)<x$ i podaj jej interpretację graficzną.\\
4. Dla jakich wartości parametru rzeczywistego $m$ równanie kwadratowe

$$
2 x^{2}-m x+m+2=0
$$

ma dwa pierwiastki rzeczywiste $x_{1}, x_{2}$, których suma odwrotności jest nieujemna? Sporządź wykres funkcji $f(m)=\frac{1}{x_{1}}+\frac{1}{x_{2}}$.\\
5. Odcinek o końcach $A\left(\frac{5}{2}, \frac{\sqrt{3}}{2}\right), B\left(\frac{5}{2}, \frac{3 \sqrt{3}}{2}\right)$ jest bokiem wielokąta foremnego wpisanego w okrąg styczny do osi $O x$. Wyznacz równanie tego okręgu i współrzędne pozostałych wierzchołków wielokąta. Ile rozwiązań ma to zadanie? Sporządź rysunek.\\
6. Z wierzchołków podstawy $A B$ trójkąta równobocznego o boku $a$ rozpoczęły ruch dwa punkty. Poruszają się one wzdłuż boków $A C$ i $B C$ z prędkościami odpowiednio $v_{1}$ i $v_{2}$. Po jakim czasie odległość między nimi będzie równa wysokości trójkąta?

Rozwiązania (rękopis) zadań z wybranego poziomu prosimy nadsyłać do 28 września 2014r. na adres:

Instytut Matematyki i Informatyki\\
Politechniki Wrocławskiej\\
Wybrzeże Wyspiańskiego 27\\
50-370 WROCEAW.\\
Na kopercie prosimy koniecznie zaznaczyć wybrany poziom! (np. poziom podstawowy lub rozszerzony). Do rozwiązań należy dołączyć zaadresowaną do siebie kopertę zwrotną z naklejonym znaczkiem, odpowiednim do wagi listu. Prace niespełniające podanych warunków nie będą poprawiane ani odsyłane.

Adres internetowy Kursu: \href{http://www.im.pwr.edu.pl/kurs}{http://www.im.pwr.edu.pl/kurs}


\end{document}