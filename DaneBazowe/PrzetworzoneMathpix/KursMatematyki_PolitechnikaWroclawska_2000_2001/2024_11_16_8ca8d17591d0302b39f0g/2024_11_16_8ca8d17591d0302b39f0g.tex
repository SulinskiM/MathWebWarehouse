\documentclass[10pt]{article}
\usepackage[polish]{babel}
\usepackage[utf8]{inputenc}
\usepackage[T1]{fontenc}
\usepackage{amsmath}
\usepackage{amsfonts}
\usepackage{amssymb}
\usepackage[version=4]{mhchem}
\usepackage{stmaryrd}

\title{KORESPONDENCYJNY KURS Z MATEMATYKI }

\author{}
\date{}


\begin{document}
\maketitle
\section*{PRACA KONTROLNA nr 1}
październik 2000r

\begin{enumerate}
  \item Suma wszystkich wyrazów nieskończonego ciągu geometrycznego wynosi 2040. Jeśli pierwszy wyraz tego ciągu zmniejszymy o 172, a jego iloraz zwiększymy 3 -krotnie, to suma wszystkich wyrazów tak otrzymanego ciągu wyniesie 2000. Wyznaczyć iloraz i pierwszy wyraz danego ciągu.
  \item Obliczyć wszystkie te składniki rozwinięcia dwumianu $(\sqrt{3}+\sqrt[3]{2})^{11}$, które są liczbami całkowitymi.
  \item Wykonać staranny wykres funkcji
\end{enumerate}

$$
f(x)=\left|x^{2}-2\right| x|-3|
$$

i na jego podstawie podać ekstrema lokalne oraz przedziały monotoniczności tej funkcji.\\
4. Rozwiązać nierówność

$$
x+1 \geqslant \log _{2}\left(4^{x}-8\right)
$$

\begin{enumerate}
  \setcounter{enumi}{4}
  \item W ostrosłupie prawidłowym trójkątnym krawędź podstawy ma długość $a$, a połowa kąta płaskiego przy wierzchołku jest równa kątowi nachylenia ściany bocznej do podstawy. Obliczyć objętość ostrosłupa. Sporządzić odpowiednie rysunki.
  \item Znaleźć wszystkie wartości parametru $p$, dla których trójkąt KLM o wierzchołkach $\mathrm{K}(1,1), \mathrm{L}(5,0)$ i $\mathrm{M}(\mathrm{p}, \mathrm{p}-1)$ jest prostokątny. Rozwiązanie zilustrować rysunkiem.
  \item Rozwiązać równanie
\end{enumerate}

$$
\frac{\sin 5 x}{\sin 3 x}=\frac{\sin 4 x}{\sin 6 x}
$$

\begin{enumerate}
  \setcounter{enumi}{7}
  \item Przez punkt $P$ leżący wewnątrz trójkąta $A B C$ poprowadzono proste równoległe do wszystkich boków trójkąta. Pola utworzonych w ten sposób trzech mniejszych trójkątów o wspólnym wierzchołku $P$ wynoszą $S_{1}, S_{2}, S_{3}$. Obliczyć pole $S$ trójkąta $A B C$.
\end{enumerate}

\section*{PRACA KONTROLNA nr 2}
listopad 2000 r

\begin{enumerate}
  \item Promień kuli zwiększono tak, że pole jej powierzchni wzrosło o $44 \%$. O ile procent wzrosła jej objętość?
  \item Wyznaczyć równanie krzywej utworzonej przez środki odcinków mających obydwa końce na osiach układu współrzędnych i zawierających punkt $\mathrm{P}(2,1)$. Sporządzić dokładny wykres i podać nazwę otrzymanej krzywej.
  \item Znaleźć wszystkie wartości parametru $m$, dla których równanie
\end{enumerate}

$$
(m-1) 9^{x}-4 \cdot 3^{x}+m+2=0
$$

ma dwa różne rozwiązania.\\
4. Różnica promienia kuli opisanej na czworościanie foremnym i promienia kuli wpisanej w niego jest równa 1 . Obliczyć objętość tego czworościanu.\\
5. Rozwiązać nierówność

$$
\frac{2}{\left|x^{2}-9\right|} \geqslant \frac{1}{x+3} .
$$

\begin{enumerate}
  \setcounter{enumi}{5}
  \item Stosunek długości przyprostokątnych trójkąta prostokątnego wynosi $k$. Obliczyć stosunek długości dwusiecznych kątów ostrych tego trójkąta. Użyć odpowiednich wzorów trygonometrycznych.
  \item Zbadać przebieg zmienności funkcji
\end{enumerate}

$$
f(x)=\frac{x^{2}+4}{(x-2)^{2}}
$$

i wykonać jej staranny wykres.\\
8. Wyznaczyć równania wszystkich prostych stycznych do wykresu funkcji $f(x)=$ $x^{3}-2 x$ i przechodzących przez punkt $A\left(\frac{7}{5},-2\right)$. Wykonać odpowiedni rysunek.

\section*{PRACA KONTROLNA nr 3}
grudzień 2000 r

\begin{enumerate}
  \item Stosując zasadę indukcji matematycznej udowodnić, że dla każdej liczby naturalnej $n$ suma $2^{n+1}+3^{2 n-1}$ jest podzielna przez 7 .
  \item Tworząca stożka ma długość $l$ i widać ją ze środka kuli wpisanej w ten stożek pod kątem $\alpha$. Obliczyć objętość i kąt rozwarcia stożka. Określić dziedzinę dla kąta $\alpha$.
  \item Nie korzystając z metod rachunku różniczkowego wyznaczyć dziedzinę i zbiór wartości funkcji
\end{enumerate}

$$
y=\sqrt{2+\sqrt{x}-x}
$$

\begin{enumerate}
  \setcounter{enumi}{3}
  \item Z talii 24 kart wylosowano (bez zwracania) cztery karty. Jakie jest prawdopodobieństwo, że otrzymano dokładnie trzy karty z jednego koloru (z czterech możliwych)?
  \item Rozwiązać nierówność
\end{enumerate}

$$
\log _{1 / 3}\left(\log _{2} 4 x\right) \geqslant \log _{1 / 3}\left(2-\log _{2 x} 4\right)-1
$$

\begin{enumerate}
  \setcounter{enumi}{5}
  \item Z punktu $C(1,0)$ poprowadzono styczne do okregu $x^{2}+y^{2}=r^{2}, \quad r \in(0,1)$. Punkty styczności oznaczono przez $A$ i $B$. Wyrazić pole trójkąta ABC jako funkcje promienia $r$ i znaleźć największą wartość tego pola.
  \item Rozwiązać układ równań
\end{enumerate}

$$
\begin{cases}x^{2}+y^{2} & =5|x| \\ |4 y-3 x+10| & =10\end{cases}
$$

Podać interpretację geometryczną każdego z równań i wykonać staranny rysunek.\\
8. Rozwiązać w przedziale $[0, \pi]$ równanie

$$
1+\sin 2 x=2 \sin ^{2} x
$$

a następnie nierówność $1+\sin 2 x>2 \sin ^{2} x$.

\section*{PRACA KONTROLNA nr 4}
styczeń 2001 r\\
W celu przybliżenia słuchaczom Kursu, jakie wymagania były stawiane ich starszym kolegom przed ponad dwudziestu laty, niniejszy zestaw zadań jest dokładnym powtórzeniem pracy kontrolnej ze stycznia 1979 r.

\begin{enumerate}
  \item Przez środek boku trójkąta równobocznego przeprowadzono prostą, tworzącą z tym bokiem kąt ostry $\alpha$ i dzielącą ten trójkąt na dwie figury, których stosunek pól jest równy $1: 7$. Obliczyć miarę kąta $\alpha$.
  \item W kulę o promieniu $R$ wpisano graniastosłup trójkątny prawidłowy o krawędzi podstawy równej $R$. Obliczyć wysokość tego graniastosłupa.
  \item Wyznaczyć wartości parametru $a$, dla których funkcja $f(x)=\frac{a x}{1+x^{2}}$ osiąga maksimum równe 2 .
  \item Rozwiązać nierówność
\end{enumerate}

$$
\cos ^{2} x+\cos ^{3} x+\ldots+\cos ^{n+1} x+\ldots<1+\cos x
$$

dla $x \in[0,2 \pi]$.\\
5. Wykazać, że dla każdej liczby naturalnej $n \geqslant 2$ prawdziwa jest równość

$$
1^{2}+2^{2}+\ldots+n^{2}=\binom{n+1}{2}+2\left[\binom{n}{2}+\binom{n-1}{2}+\ldots+\binom{2}{2}\right]
$$

\begin{enumerate}
  \setcounter{enumi}{5}
  \item Wyznaczyć równanie linii będącej zbiorem środków wszystkich okręgów stycznych do prostej $y=0$ i jednocześnie stycznych zewnętrznie do okręgu $(x+2)^{2}+y^{2}=4$. Narysować tę linię.
  \item Wyznaczyć wartości parametru $m$, dla których równanie $9 x^{2}-3 x \log _{3} m+1=0$ ma dwa różne pierwiastki rzeczywiste $x_{1}, x_{2}$ spełniające warunek $x_{1}^{2}+x_{2}^{2}=1$.
  \item Rozwiązać nierówność
\end{enumerate}

$$
\frac{\sqrt{30+x-x^{2}}}{x}<\frac{\sqrt{10}}{5}
$$

\section*{PRACA KONTROLNA nr 5}
luty 2001 r

\begin{enumerate}
  \item Posługując się odpowiednim wykresem wykazać, że równanie
\end{enumerate}

$$
\sqrt{x-3}+x=4
$$

posiada dokładnie jedno rozwiązanie. Następnie wyznaczyć to rozwiązanie analitycznie.\\
2. Wiadomo, że wielomian $w(x)=3 x^{3}-5 x+1$ ma trzy pierwiastki rzeczywiste $x_{1}, x_{2}, x_{3}$. Nie wyznaczając tych pierwiastków obliczyć wartość wyrażenia

$$
\left(1+x_{1}\right)\left(1+x_{2}\right)\left(1+x_{3}\right)
$$

\begin{enumerate}
  \setcounter{enumi}{2}
  \item Rzucamy jeden raz kostką, a następnie monetą tyle razy, ile oczek pokazała kostka. Obliczyć prawdopodobieństwo tego, że rzuty monetą dały co najmniej jednego orła.
  \item Wyznaczyć równania wszystkich okręgów stycznych do obu osi układu współrzędnych oraz do prostej $3 x+4 y=12$.
  \item W ostrosłupie prawidłowym czworokątnym dana jest odległość $d$ środka podstawy od krawędzi bocznej oraz kąt $2 \alpha$ między sąsiednimi ścianami bocznymi. Obliczyć objętość ostrosłupa.
  \item W trapezie równoramiennym o polu $P$ dane są promień okręgu opisanego $r$ oraz suma długości obu podstaw $s$. Obliczyć obwód tego trapezu. Podać warunki rozwiązalności zadania. Wykonać rysunek dla $P=12 \mathrm{~cm}^{2}, r=3 \mathrm{~cm}$ i $s=8 \mathrm{~cm}$.
  \item Rozwiązać układ równań
\end{enumerate}

$$
\left\{\begin{aligned}
p x+y & =3 p^{2}-3 p-2 \\
(p+2) x+p y & =4 p
\end{aligned}\right.
$$

w zależności od parametru rzeczywistego $p$. Podać wszystkie rozwiązania (i odpowiadające im wartości parametru $p$ ), dla których obie niewiadome są liczbami całkowitymi o wartości bezwzględnej mniejszej od 3.\\
8. Odcinek $\overline{A B}$ o końcach $A\left(0, \frac{3}{2}\right)$ i $B(1, y), \quad y \in\left[0, \frac{3}{2}\right]$, obraca się wokół osi Ox. Wyrazić pole powstałej powierzchni jako funkcję $y$ i znaleźć najmniejszą wartość tego pola. Sporządzić rysunek.

\section*{PRACA KONTROLNA nr 6}
marzec 2001 r

\begin{enumerate}
  \item Wykazać, że dla każdego kąta $\alpha$ prawdziwa jest nierówność
\end{enumerate}

$$
\sqrt{3} \sin \alpha+\sqrt{6} \cos \alpha \leqslant 3
$$

\begin{enumerate}
  \setcounter{enumi}{1}
  \item Dane są punkty $A(2,2)$ i $B(-1,4)$. Wyznaczyć długość rzutu prostopadłego odcinka $\overline{A B}$ na prostą o równaniu $12 x+5 y=30$. Sporządzić rysunek.
  \item Niech $f(m)$ będzie sumą odwrotności pierwiatków rzeczywistych równania kwadratowego $\left(2^{m}-7\right) x^{2}-2\left|2^{m}-4\right| x+2^{m}=0$, gdzie $m$ jest parametrem rzeczywistym. Napisać wzór określający $f(m)$ i narysować wykres tej funkcji.
  \item Dwóch strzelców strzela równocześnie do tego samego celu niezależnie od siebie. Pierwszy strzelec trafia za każdym razem z prawdopodobieństwem $\frac{2}{3}$ i oddaje 2 strzały, a drugi trafia z prawdopodobieństwem $\frac{1}{2}$ i oddaje 5 strzałów. Obliczyć prawdopodobieństwo, że cel zostanie trafiony dokładnie 3 razy.
  \item Liczby $a_{1}, a_{2}, \ldots, a_{n}, n \geqslant 3$, tworzą ciąg arytmetyczny. Suma wyrazów tego ciągu wynosi 28 , suma wyrazów o numerach nieparzystych wynosi 16 , a $a_{2} \cdot a_{3}=48$. Wyznaczyć te liczby.
  \item W trójkącie $A B C$, w którym $A B=7$ oraz $A C=9$, a kąt przy wierzchołku $A$ jest dwa razy większy niż kąt przy wierzchołku $B$. Obliczyć stosunek promienia koła wpisanego do promienia koła opisanego na tym trójkącie. Rozwiązanie zilustrować rysunkiem.
  \item Zaznaczyć na płaszczyźnie następujące zbiory punktów:
\end{enumerate}

$$
A=\{(x, y): x+y-2 \geqslant|x-2|\}, \quad B=\left\{(x, y): y \leqslant \sqrt{4 x-x^{2}}\right\}
$$

Następnie znaleźć na brzegu zbioru $A \cap B$ punkt Q, którego odległość od punktu $P\left(\frac{5}{2}, 1\right)$ jest najmniejsza.\\
8. Przeprowadzić badanie przebiegu i sporządzić wykres funkcji

$$
f(x)=\frac{1}{2} x^{2}-4+\sqrt{8-x^{2}}
$$

\section*{PRACA KONTROLNA nr 7}
\begin{enumerate}
  \item Ile elementów ma zbiór $A$, jeśli liczba jego podzbiorów trójelementowych jest większa od liczby podzbiorów dwuelementowych o 48 ?
  \item W sześciokąt foremny o boku 1 wpisano okrąg. W otrzymany okrąg wpisano sześciokąt foremny, w który znów wpisano okrąg, itd. Obliczyć sumę obwodów wszystkich otrzymanych okręgów.
  \item Dana jest rodzina prostych o równaniach $2 x+m y-m-2=0, m \in R$. Które z prostych tej rodziny są:\\
a) prostopadłe do prostej $x+4 y+2=0$,\\
b) równoległe do prostej $3 x+2 y=0$,\\
c) tworzą z prostą $x-\sqrt{3} y-1=0$ kąt $\frac{\pi}{3}$.
  \item Sprawdzić tożsamość: $\operatorname{tg}\left(x-\frac{\pi}{4}\right)-1=\frac{-2}{\operatorname{tgx+1}}$. Korzystając z niej sporządzić wykres funkcji $f(x)=\frac{1}{t g x+1} \mathrm{w}$ przedziale $[0, \pi]$.
  \item Dany jest okrąg K o równaniu $x^{2}+y^{2}-6 y=27$. Wyznaczyć równanie krzywej $\Gamma$ będącej obrazem okręgu K w powinowactwie prostokątnym o osi Ox i skali $k=\frac{1}{3}$. Obliczyć pole figury leżącej poniżej osi odciętych i ograniczonej łukiem okręgu K i krzywą Г. Wykonać rysunek.
  \item Wykorzystując nierówność $2 \sqrt{a b} \leqslant a+b, a, b>0$, wyznaczyć granicę
\end{enumerate}

$$
\lim _{n \rightarrow \infty}\left(\frac{\log _{5} 16}{\log _{2} 3}\right)^{n}
$$

\begin{enumerate}
  \setcounter{enumi}{6}
  \item Trylogię składającą się z dwóch powieści dwutomowych oraz jednej jednotomowej ustawiono przypadkowo na półce. Jakie jest prawdopodobieństwo tego, że tomy a) obydwu, b) co najmniej jednej, z dwutomowych powieści znajdują się obok siebie i przy tym tom I z lewej, a tom II z prawej strony.
  \item W ostrosłupie prawidłowym czworokątnym krawędź boczna jest nachylona do płaszczyzny podstawy pod kątem $\alpha$, a krawędź podstawy ma długość $a$. Obliczyć promień kuli stycznej do wszystkich krawędzi tego ostrosłupa. Wykonać odpowiednie rysunki.
\end{enumerate}

\end{document}