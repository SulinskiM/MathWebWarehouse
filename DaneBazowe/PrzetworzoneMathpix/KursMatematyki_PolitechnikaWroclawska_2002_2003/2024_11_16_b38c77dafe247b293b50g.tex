\documentclass[10pt]{article}
\usepackage[polish]{babel}
\usepackage[utf8]{inputenc}
\usepackage[T1]{fontenc}
\usepackage{amsmath}
\usepackage{amsfonts}
\usepackage{amssymb}
\usepackage[version=4]{mhchem}
\usepackage{stmaryrd}

\title{KORESPONDENCYJNY KURS Z MATEMATYKI }

\author{}
\date{}


\begin{document}
\maketitle
\section*{PRACA KONTROLNA nr 1}
\begin{enumerate}
  \item Narysować wykres funkcji $y=4+2|x|-x^{2}$. Korzystając z tego wykresu określić liczbę rozwiązań równania $4+2|x|-x^{2}=p$ w zależności od parametru rzeczywistego $p$.
  \item Pompa napełniająca pusty basen w pierwszej minucie pracy miała wydajność 0,2 $\mathrm{m}^{3} / \mathrm{s}$, a w każdej kolejnej minucie jej wydajność zwiększano o $0,01 \mathrm{~m}^{3} / \mathrm{s}$. Połowa basenu została napełniona po $2 n$ minutach, a cały basen po kolejnych $n$ minutach, gdzie $n$ jest liczbą naturalną. Wyznaczyć czas napełniania basenu oraz jego pojemność.
  \item Stożek ścięty jest opisany na kuli o promieniu $r=2 \mathrm{~cm}$. Objętość kuli stanowi $25 \%$ objętości stożka. Wyznaczyć średnice podstaw i długość tworzącej tego stożka.
  \item W trójkącie $A B C$ dane są promień okręgu opisanego $R$, kąt $\angle A=\alpha$ oraz $A B=\frac{8}{5} R$. Obliczyć pole tego trójkąta.
  \item Rozwiązać nierówność:
\end{enumerate}

$$
(\sqrt{x})^{\log _{8} x} \geqslant \sqrt[3]{16 x}
$$

\begin{enumerate}
  \setcounter{enumi}{5}
  \item W czworokącie $A B C D$ odcinki $\overline{A B}$ i $\overline{B D}$ są prostopadłe, $A D=2 A B=a$ oraz $\overrightarrow{A C}=\frac{5}{3} \overrightarrow{A B}+\frac{1}{3} \overrightarrow{A D}$. Wyznaczyć cosinus kąta $\angle B C D=\alpha$ oraz obwód czworokąta $A B C D$. Sporządzić rysunek.
  \item Rozwiązać równanie:
\end{enumerate}

$$
\frac{1}{\sin x}+\frac{1}{\cos x}=\sqrt{8}
$$

\begin{enumerate}
  \setcounter{enumi}{7}
  \item Wyznaczyć równanie prostej stycznej do wykresu funkcji $y=\frac{1}{x^{2}}$ w punkcie $P\left(x_{0}, y_{0}\right)$, $x_{0}>0$, takim, że odcinek tej stycznej zawarty w I ćwiartce układu współrzędnych jest najkrótszy. Rozwiązanie zilustrować stosownym wykresem.
\end{enumerate}

\section*{PRACA KONTROLNA nr 2}
\begin{enumerate}
  \item Czy liczby różnych 'słów', jakie można utworzyć zmieniając kolejność liter w 'słowach' TANATAN i AKABARA, są takie same? Uzasadnić odpowiedź. Przez 'słowo' rozumiemy tutaj dowolny ciąg liter.
  \item Reszta z dzielenia wielomianu $x^{3}+p x^{2}-x+q$ przez trójmian $(x+2)^{2}$ wynosi $-x+1$. Wyznaczyć pierwiastki tego wielomianu.
  \item Figura na rysunku poniżej składa się z łuków $B C, C A$ okręgón o promieniu $a$ i środkach odpowiednio w punktach $A, B$, oraz z odcinka $\overline{A B}$ o długości $a$. Obliczyć promień okręgu stycznego do obu łuków oraz do odcinka $\overline{A B}$.
  \item Podstawą pryzmy przedstawionej na rysunku poniżej jest prostokąt $A B C D$, którego bok $\overline{A B}$ ma długość $a$, a bok $\overline{B C}$ długość $b$, gdzie $a>b$. Wszystkie ściany boczne pryzmy są nachylone pod kątem $\alpha$ do płaszczyzny podstawy. Obliczyć objętość tej pryzmy.
  \item Rozwiązać nierówność
\end{enumerate}

$$
\frac{2}{x}<\sqrt{5-x^{2}}
$$

Rozwiązanie zilustrować wykresami funkcji występujących po obu stronach nierówności. Zaznaczyć na rysunku otrzymany zbiór rozwiązań.\\
6. Ciąg $\left(a_{n}\right)$ jest określony warunkami $a_{1}=4, a_{n+1}=1+2 \sqrt{a_{n}}, n \geqslant 1$. Stosując zasadę indukcji matematycznej wykazać, że ciąg $\left(a_{n}\right)$ jest rosnący oraz dla $n \geqslant 1$ spełniona jest nierówność: $4 \leqslant a_{n}<6$.\\
7. Na krzywej o równaniu $y=\sqrt{x}$ znaleźć miejsce, które jest położone najbliżej punktu $P(0,3)$. Sporządzić rysunek.\\
8. Wykazać, że dla każdej wartości parametru $\alpha \in R$ równanie kwadratowe

$$
3 x^{2}+4 x \sin \alpha-\cos 2 \alpha=0
$$

ma dwa różne pierwiastki rzeczywiste. Wyznaczyć te wartości parametru $\alpha$, dla których oba pierwiastki leżą w przedziale $(0,1)$.

\section*{PRACA KONTROLNA nr 3}
grudzień 2002r

\begin{enumerate}
  \item Suma wyrazów nieskończonego ciągu geometrycznego zmniejszy się o $25 \%$, jeśli wykreślimy z niej składniki o numerach parzystych niepodzielnych przez 4. Obliczyć sumę wszystkich wyrazów tego ciągu wiedząc, że jego drugi wyraz wynosi 1.
  \item Z kompletu 28 kości do gry w domino wylosowano dwie kości (bez zwracania). Obliczyć prawdopodobieństwo tego, że kości pasuja do siebie tzn. na jednym z pól obu kości występuje ta sama liczba oczek.
  \item Rozwiązać układ równań
\end{enumerate}

$$
\left\{\begin{aligned}
x+2 y & =3 \\
5 x+m y & =m
\end{aligned}\right.
$$

w zależności od parametru rzeczywistego $m$. Wyznaczyć i narysować zbiór, jaki tworzą rozwiązania $(x(m), y(m))$ tego układu, gdy $m$ przebiega zbiór liczb rzeczywistych.\\
4. W graniastosłupie prawidłowym sześciokątnym krawędź dolnej podstawy $\overline{A B}$ jest widoczna ze środka górnej podstawy $P$ pod kątem $\alpha$. Wyznaczyć cosinus kąta utworzonego przez płaszczyznę podstawy i płaszczyznę zawierającą $\overline{A B}$ oraz przeciwległą do niej krawędź $\overline{D^{\prime} E^{\prime}}$ górnej podstawy. Obliczenia odpowiednio uzasadnić.\\
5. Rozwiązać nierówność

$$
-1 \leqslant \frac{2^{x+1 / 2}}{4^{x}-4} \leqslant 1
$$

\begin{enumerate}
  \setcounter{enumi}{5}
  \item Nie posługując się tablicami wykazać, że $\operatorname{tg} 82^{0} 30^{\prime}-\operatorname{tg} 7^{0} 30^{\prime}=4+2 \sqrt{3}$.
  \item Napisać równanie prostej $k$ stycznej do okręgu $x^{2}+y^{2}-2 x-2 y-3=0 \mathrm{w}$ punkcie $P(2,3)$. Następnie wyznaczyć równania wszystkich prostych stycznych do tego okręgu, które tworzą z prostą $k$ kąt $45^{0}$.
  \item Dobrać parametry $a>0$ i $b \in R$ tak, aby funkcja
\end{enumerate}

$$
f(x)= \begin{cases}(a+1)+a x-x^{2} & \text { dla } x \leqslant a \\ \frac{b}{x^{2}-1} & \text { dla } x>a\end{cases}
$$

była ciągła i miała pochodną w punkcie $a$. Nie przeprowadzając dalszego badania sporządzić wykres funkcji $f(x)$ oraz stycznej do jej wykresu w punkcie $P(a, f(a))$.

\section*{PRACA KONTROLNA nr 4}
styczeń 2003r

\begin{enumerate}
  \item Dla jakich wartości parametru rzeczywistego $t$ równanie
\end{enumerate}

$$
x+3=-(t x+1)^{2}
$$

ma dokładnie jedno rozwiązanie.\\
2. Czworościan foremny o krawędzi $a$ przecięto płaszczyzną równoległą do dwóch przeciwległych krawędzi. Wyrazić pole otrzymanego przekroju jako funkcję długości odcinka wyznaczonego przez ten przekrój na jednej z pozostałych krawędzi. Uzasadnić postępowanie. Przedstawić znalezioną funkcję na wykresie i podać jej największą wartość.\\
3. Zaznaczyć na wykresie zbiór punktów $(x, y)$ płaszczyzny spełniających warunek $\log _{x y}|y| \geqslant 1$.\\
4. Wyznaczyć równanie linii utworzonej przez wszystkie punkty płaszczyzny, których odległość od okręgu $x^{2}+y^{2}=81$ jest o 1 mniejsza niż od punktu $P(8,0)$. Sporządzić rysunek.\\
5. Na dziesiątym piętrze pewnego bloku mieszkają Kowalscy i Nowakowie. Kowalscy mają dwóch synów i dwie córki, a Nowakowie jednego syna i dwie córki. Postanowili oni wybrać młodzieżowego przedstawiciela swojego piętra. W tym celu Kowalscy wybrali losowo jedno ze swoich dzieci, a Nowakowie jedno ze swoich. Następnie spośród tej dwójki wylosowano jedną osobę. Obliczyć prawdopodobieństwo, że przedstawicielem został chłopiec.\\
6. Uzasadnić prawdziwość nierówności $n+\frac{1}{2} \geqslant \sqrt{n(n+1)}, \quad n \geqslant 1$. Korzystając z niej oraz z zasady indukcji matematycznej udowodnić, że dla wszystkich $n \geqslant 1$ jest

$$
\binom{2 n}{n} \geqslant \frac{4^{n}}{2 \sqrt{n}}
$$

\begin{enumerate}
  \setcounter{enumi}{6}
  \item Przeprowadzić badanie przebiegu zmienności funkcji $f(x)=\sqrt{\frac{3 x-3}{5-x}}$ i wykonać jej wykres.
  \item W trójkącie $A B C$ kąt $A$ ma miarę $\alpha$, kąt $B$ miarę $2 \alpha$, a $B C=a$. Oznaczmy kolejno przez $A_{1}$ punkt na boku $\overline{A C}$ taki, że $\overline{B A_{1}}$ jest dwusieczną kąta $B ; B_{1}$ punkt na boku $\overline{B C}$ taki, że $\overline{A_{1} B_{1}}$ jest dwusieczną kąta $A_{1}$, itd. Wyznaczyć długość łamanej nieskończonej $A B A_{1} B_{1} A_{2} \ldots$..
\end{enumerate}

\section*{PRACA KONTROLNA nr 5}
luty 2003r

\begin{enumerate}
  \item Jakiej długości powinien być pas napędowy, aby można go było użyć do połączenia dwóch kół o promieniach 20 cm i 5 cm , jeśli odległość środków tych kół wynosi 30 cm ?
  \item Umowa określa wynagrodzenie na kwotę 4000 zł. Składka na ubezpieczenie społeczne wynosi $18,7 \%$ tej kwoty, a składka na Kasę Chorych $7,75 \%$ kwoty pozostałej po odliczeniu składki na ubezpieczenie społeczne. W celu obliczenia podatku należy od $80 \%$ wyjściowej kwoty umowy odjąć składkę na ubezpieczenie społeczne i wyznaczyć 19\% pozostałej sumy. Podatek jest różnicą tak otrzymanej liczby i kwoty składki na Kasę Chorych. Ile wynosi podatek?.
  \item Przez punkt $P(1,3)$ poprowadzić prostą $l$ tak, aby odcinek tej prostej zawarty pomiędzy dwiema danymi prostymi $x-y+3=0$ i $x+2 y-12=0$ dzielił się w punkcie $P$ na połowy. Wyznaczyć równanie ogólne prostej $l$ i obliczyć pole trójkąta, jaki prosta $l$ tworzy z danymi prostymi.
  \item Podstawą czworościanu jest trójkąt prostokątny $A B C$ o kącie ostrym $\alpha$ i promieniu okręgu wpisanego $r$. Spodek wysokości opuszczonej z wierzchołka $D$ leży w punkcie przecięcia się dwusiecznych trójkąta $A B C$, a ściany boczne wychodzące z wierzchołka kąta prostego podstawy tworzą kąt $\beta$. Obliczyć objętość tego ostrosłupa.
  \item Sporządzić wykres funkcji
\end{enumerate}

$$
f(x)=\log _{4}(2|x|-4)^{2}
$$

Odczytać z wykresu wszystkie ekstrema lokalne tej funkcji.\\
6. Rozwiązać równanie $\cos 2 x+\frac{\operatorname{tg} x}{\sqrt{3}+\operatorname{tg} x}=0$.\\
7. Dla jakich wartości parametru $a \in R$ można określić funkcję $g(x)=f(f(x))$, gdzie $f(x)=\frac{x^{2}}{a x-1}$. Napisać funkcję $g(x) \mathrm{w}$ jawnej postaci. Wyznaczyć asymptoty funkcji $g(x)$ dla największej możliwej całkowitej wartości parametru $a$.\\
8. Odcinek o końcach $A(0,3), B(2, y), y \in[0,3]$, obraca się wokół osi Ox. Wyznaczyć pole powierzchni bocznej powstałej bryły jako funkcję $y$ i znaleźć najmniejszą wartość tego pola. Sporządzić rysunek.

\section*{PRACA KONTROLNA nr 6}
\begin{enumerate}
  \item Dla jakich wartości parametru rzeczywistego $p$ równanie $\sqrt{x+8 p}=\sqrt{x}+2 p$ posiada rozwiązanie?
  \item Obrazem okręgu $K$ w jednokładności o środku $S(0,1)$ i skali $k=-3$ jest okrąg $K_{1}$. Natomiast obrazem $K_{1}$ w symetrii względem prostej o równaniu $2 x+y+3=0$ jest okrąg o tym samym środku co okrąg $K$. Wyznaczyć równanie okręgu $K$, jeśli wiadomo, że okręgi $K$ i $K_{1}$ są styczne zewnętrznie.
  \item W trapezie równoramiennym dane są promień okręgu opisanego $r$, kąt ostry przy podstawie $\alpha$ oraz suma długości obu podstaw $d$. Obliczyć długość ramienia tego trapezu. Zbadać warunki rozwiązalności zadania. Wykonać rysunek dla $\alpha=60^{\circ}$, $d=$ $\frac{5}{2} r$.
  \item W ostrosłupie prawidłowym czworokątnym kąt płaski ściany bocznej przy wierzchołku wynosi $2 \beta$. Przez wierzchołek $A$ podstawy oraz środek przeciwległej krawędzi bocznej poprowadzono płaszczyznę równoległą do przekątnej podstawy wyznaczającą przekrój płaski ostrosłupa. Obliczyć objętość ostrosłupa wiedząc, że pole przekroju wynosi $S$.
  \item Obliczyć granicę
\end{enumerate}

$$
\lim _{n \rightarrow \infty} \frac{n-\sqrt[3]{n^{3}+n^{\alpha}}}{\sqrt[5]{n^{3}}}
$$

gdzie $\alpha$ jest najmniejszym dodatnim pierwiastkiem równania $2 \cos \alpha=-\sqrt{3}$.\\
6. Rozwiązać nierówność

$$
2^{1+2 \log _{2} \cos x}-\frac{3}{4} \geqslant 9^{0.5+\log _{3} \sin x}
$$

\begin{enumerate}
  \setcounter{enumi}{6}
  \item Wybrano losowo 4 liczby czterocyfrowe (cyfra tysięcy nie może być zerem!). Obliczyć prawdopodobieństwo tego, że co najmniej dwie z tych liczb czytane od przodu lub od końca będą podzielne przez 4.
  \item Zaznaczyć na rysunku zbiór punktów $(x, y)$ płaszczyzny określony warunkami $|x-3 y|<2$ oraz $y^{3} \leqslant x$. Obliczyć tangens kąta, pod którym przecinają się linie tworzące brzeg tego zbioru.
\end{enumerate}

\section*{PRACA KONTROLNA nr 7}
\begin{enumerate}
  \item Dwa punkty poruszają się ruchem jednostajnym po okręgu w tym samym kierunku, przy czym jeden z nich wyprzedza drugi co 44 sekund. Jeżeli zmienić kierunek ruchu jednego z tych punktów, to będą się one spotykać co 8 sekund. Obliczyć stosunek prędkości tych punktów.
  \item Dla jakich wartości parametru $p$ nierówność
\end{enumerate}

$$
\frac{2 p x^{2}+2 p x+1}{x^{2}+x+2-p^{2}} \geqslant 2
$$

jest spełniona dla każdej liczby rzeczywistej $x$ ?\\
3. W równoległoboku dane są kąt ostry $\alpha$, dłuższa przekątna $d$ oraz różnica boków $r$. Obliczyć pole równoległoboku.\\
4. Naczynie w kształcie półkuli o promieniu $R$ ma trzy nóźki w kształcie kulek o promieniu $r, 4 r<R$, przymocowanych do naczynia w ten sposób, że ich środki tworzą trójkąt równoboczny, a naczynie postawione na płaskiej powierzchni dotyka ją w jednym punkcie. Obliczyć wzajemną odległość punktów przymocowania kulek. Wykonać odpowiednie rysunki.\\
5. Posługując się rachunkiem różniczkowym określić liczbę rozwiązań równania

$$
2 x^{3}+1=6|x|-6 x^{2} .
$$

\begin{enumerate}
  \setcounter{enumi}{5}
  \item Nie stosując zasady indukcji matematycznej wykazać, że jeżeli $n \geqslant 2$ jest liczbą naturalną, to $\frac{n^{n}-1}{n-1}$ jest nieparzystą liczbą naturalną.
  \item Rozwiązać równanie
\end{enumerate}

$$
\frac{8}{3}\left(\sin ^{2} x+\sin ^{4} x+\ldots\right)=4-2 \cos x+3 \cos ^{2} x-\frac{9}{2} \cos ^{3} x+\ldots
$$

\begin{enumerate}
  \setcounter{enumi}{7}
  \item Rozważmy rodzinę prostych normalnych (tj. prostopadłych do stycznych w punktach styczności) do paraboli o równaniu $2 y=x^{2}$. Znaleźć równanie krzywej utworzonej ze środków odcinków tych normalnych zawartych pomiędzy osią rzędnych i wyznaczającymi je punktami paraboli. Sporządzić rysunek.
\end{enumerate}

\end{document}