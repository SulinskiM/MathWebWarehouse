\documentclass[10pt]{article}
\usepackage[polish]{babel}
\usepackage[utf8]{inputenc}
\usepackage[T1]{fontenc}
\usepackage{amsmath}
\usepackage{amsfonts}
\usepackage{amssymb}
\usepackage[version=4]{mhchem}
\usepackage{stmaryrd}
\usepackage{hyperref}
\hypersetup{colorlinks=true, linkcolor=blue, filecolor=magenta, urlcolor=cyan,}
\urlstyle{same}

\title{L }

\author{}
\date{}


\begin{document}
\maketitle
KORESPONDENCYJNY KURS\\
październik 2020 r.\\
Z MATEMATYKI

\section*{PRACA KONTROLNA nr 2 - POZIOM PODSTAWOWY}
\begin{enumerate}
  \item Niemieckie przepisy drogowe wymagają zachowania bezpiecznego odstępu między poruszajacymi się w tym samym kierunku pojazdami. Zalecane jest przy tym zachowanie zasady „połowa licznika:" jeżeli dwa pojazdy jadą z prędkością $x \mathrm{~km} / \mathrm{h}$, to odstęp między nimi powinien wynosić przynajmniej $x / 2$ metrów. Jaki odstęp czasowy powinien zatem dzielić te dwa pojazdy? Przyjmując, że dla samochodu jadącego z prędkością $v \mathrm{~m} / \mathrm{s}$ droga hamowania wynosi $s_{h}=\frac{v^{2}}{2 a}$ metrów (gdzie $a$ jest stałym współczynnikiem hamowania), sprawdź przy jakiej prędkości $x \mathrm{~km} / \mathrm{h}$ dojdzie do wypadku, jeżeli oba pojazdy jechały z minimalnym zalecanym odstępem, pierwszy zatrzymał się nagle (przyjmij $a=10$ ), a drugi zaczął hamować jedną sekundę później i z siłą taką, że $a=7$.
  \item Jakim kątami mogą być $\alpha$ i $2 \alpha$, jeżeli wiadomo, że $\alpha$ jest kątem ostrym oraz $\sin \alpha+\cos \alpha=$ $\frac{\sqrt{6}}{2}$ ?
  \item Rozważmy funkcję $f(x)=x^{2}-(a+2) x+3(a-1)$. Dla jakich wartości paramertu $a$ :\\
(i) cały wykres $f(x)$ leży ponad prostą $y=-1$ ?\\
(ii) oba miejsca zerowe funkcji $f(x)$ są większe od 2 ?
  \item Rozwiąż nierówność
\end{enumerate}

$$
x \leqslant 1+\sqrt{2+x}
$$

\begin{enumerate}
  \setcounter{enumi}{4}
  \item Narysuj starannie zbiór $A \cap B$, gdzie
\end{enumerate}

$$
\begin{aligned}
& A=\{(x, y): 2|x|+|y| \leqslant 2\} \\
& B=\left\{(x, y): y^{2}-y<2\right\}
\end{aligned}
$$

i oblicz jego pole.\\
6. Jednym z wierzchołków kwadratu jest $A(1,-3)$, a jedna z jego przekątnych zawiera się w prostej $y=-2 x+2$. Wyznaczyć współrzędne pozostałych wierzchołków kwadratu i równanie okręgu wpisanego w ten kwadrat.

\section*{PRACA KONTROLNA nr 2 - POZIOM ROZSZERZONY}
\begin{enumerate}
  \item Wyznacz kąty $\alpha$ i $2 \alpha$ wiedząc, iż $\alpha$ jest kątem rozwartym takim, że $\operatorname{tg} \alpha+\operatorname{ctg} \alpha=-2 \sqrt{2}$.
  \item Rozwiąż równanie
\end{enumerate}

$$
x=\sqrt{5+\sqrt{3+x^{2}}}
$$

Nie używając kalkulatora zbadaj, czy jego rozwiązanie jest liczbą większą niż 3.\\
3. Udowodnij, że jeżeli dwa trójkąty prostokątne mają równe obwody i długości przeciwprostokątnych, to są przystajace.\\
4. Narysuj starannie zbiór $A \cap B$, gdzie

$$
\begin{aligned}
& A=\left\{(x, y): x^{2}-8|x|+y^{2}-8|y|+16 \geqslant 0,|x| \leqslant 4,|y| \leqslant 4\right\} \\
& B=\left\{(x, y): x^{2}+y^{2}>16(3-2 \sqrt{2})\right\}
\end{aligned}
$$

i oblicz jego pole.\\
5. Dla jakich wartości parametrów $p$ i $q$ do zbioru rozwiązań równania

$$
x^{3}-3 p x^{2}+(q+4) x=0
$$

należą zarówno $p$ jak i $q$ ?\\
6. Napisz równanie prostej $k$ stycznej do okręgu $x^{2}-4 x+y^{2}+2 y=0$ w punkcie $P(3,1)$. Następnie wyznacz równania wszystkich prostych stycznych do tego okręgu, które tworzą z prostą $k$ kąt $45^{\circ}$.

Rozwiązania (rękopis) zadań z wybranego poziomu prosimy nadsyłać do 20 października 2020r. na adres:

Wydział Matematyki\\
Politechnika Wrocławska\\
Wybrzeże Wyspiańskiego 27\\
50-370 WROCEAW.\\
Na kopercie prosimy koniecznie zaznaczyć wybrany poziom! (np. poziom podstawowy lub rozszerzony). Do rozwiązań należy dołączyć zaadresowaną do siebie kopertę zwrotną z naklejonym znaczkiem, odpowiednim do formatu listu. Polecamy stosowanie kopert formatu C5 ( $160 \times 230 \mathrm{~mm}$ ) ze znaczkiem o wartości $3,30 \mathrm{zl}$. Na każdą większą kopertę należy nakleić droższy znaczek. Prace niespełniające podanych warunków nie będą poprawiane ani odsyłane.

Uwaga. Wysyłając nam rozwiązania zadań uczestnik Kursu udostępnia Politechnice Wrocławskiej swoje dane osobowe, które przetwarzamy wyłącznie w zakresie niezbędnym do jego prowadzenia (odesłanie zadań, prowadzenie statystyki). Szczegółowe informacje o przetwarzaniu przez nas danych osobowych są dostępne na stronie internetowej Kursu.

Adres internetowy Kursu: \href{http://www.im.pwr.edu.pl/kurs}{http://www.im.pwr.edu.pl/kurs}


\end{document}