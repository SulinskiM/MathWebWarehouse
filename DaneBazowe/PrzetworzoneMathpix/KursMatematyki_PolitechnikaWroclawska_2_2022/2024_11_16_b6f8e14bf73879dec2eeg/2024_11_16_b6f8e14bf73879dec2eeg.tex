\documentclass[10pt]{article}
\usepackage[polish]{babel}
\usepackage[utf8]{inputenc}
\usepackage[T1]{fontenc}
\usepackage{amsmath}
\usepackage{amsfonts}
\usepackage{amssymb}
\usepackage[version=4]{mhchem}
\usepackage{stmaryrd}
\usepackage{hyperref}
\hypersetup{colorlinks=true, linkcolor=blue, filecolor=magenta, urlcolor=cyan,}
\urlstyle{same}

\title{PRACA KONTROLNA nr 2 - POZIOM PODSTAWOWY }

\author{}
\date{}


\begin{document}
\maketitle
\begin{enumerate}
  \item Rozwiąż nierówność
\end{enumerate}

$$
1 \leqslant \log _{\frac{1}{3}} \frac{1}{2 x-1}<2
$$

\begin{enumerate}
  \setcounter{enumi}{1}
  \item Średnia arytmetyczna czwartego, szóstego i dziesiątego wyrazu ciągu arytmetycznego $\left(a_{n}\right)$, gdzie $n \geqslant 1$, wynosi 14 , a ciąg $\left(a_{3}, a_{5}, a_{11}\right)$ jest geometryczny. Uzasadnij, że ciąg $\left(a_{4}, a_{6}, a_{10}\right)$ również jest geometryczny.
  \item W ciągu arytmetycznym $\left(a_{n}\right)$, określonym dla każdej liczby naturalnej $n \geqslant 1$, mamy
\end{enumerate}

$$
a_{3}=0 \quad \text { oraz } \quad a_{6}=7 \sin ^{2} \alpha
$$

gdzie $\operatorname{tg} \alpha=3$. Oblicz sumę 50 początkowych wyrazów tego ciągu, których indeksy są liczbami parzystymi.\\
4. Bank oferuje kredyt, który należy spłacić jednorazowo wraz z odsetkami po roku. Jaki jest całkowity koszt tego kredytu, jeśli co miesiąc bank nalicza odsetki w wysokości $2 \%$ aktualnego zadłużenia, a dodatkowo w chwili przyznania kredytu dolicza prowizję w wysokości $3 \%$ pożyczanej kwoty? Jaką kwotę trzeba będzie spłacić, jeśli pożyczymy 20000 zł? Prowizja naliczana jest jednorazowo i powiększa kwotę, którą należy spłacić.\\
5. Zaznacz na osi liczbowej zbiór wszystkich wartości parametru $t$, dla których funkcja

$$
f(x)=\left(\frac{2-t^{2}}{t-3}\right)^{t-x}+1-t
$$

jest malejąca. Naszkicuj wykres funkcji $f$ dla największej całkowitej wartości $t \mathrm{z}$ wyznaczonego zbioru.\\
6. Niech $c>0$ i $c \neq 1$. Wyznacz najmniejszą liczbę naturalną $m$, dla której suma $m$ początkowych wyrazów ciągu $\left(a_{n}\right), a_{n}=\log _{2} c^{n}$, przekracza liczbę

$$
\log _{2^{m}} c^{m^{2}}+16 \log _{4} c^{2}
$$

\section*{PRACA KONTROLNA nr 2 - POZIOM RoZsZERzony}
\begin{enumerate}
  \item Uzasadnij, że ciąg $\left(a_{n}\right)$, którego $n$-ty wyraz dany jest wzorem
\end{enumerate}

$$
a_{n}=\frac{1}{2^{1}+3^{1}}+\frac{1}{2^{2}+3^{2}}+\frac{1}{2^{3}+3^{3}}+\cdots+\frac{1}{2^{n}+3^{n}},
$$

jest ograniczony.\\
2. Wyznacz dziedzinę $D_{f}$ funkcji

$$
f(x)=\log _{10+3 x-x^{2}}\left(8-\frac{7}{1-x}\right) .
$$

\begin{enumerate}
  \setcounter{enumi}{2}
  \item Niech $c>0$. Zbadaj monotoniczność oraz oblicz sumę wszystkich wyrazów nieskończonego ciągu $\left(a_{n}\right)$, gdzie
\end{enumerate}

$$
a_{n}=\log _{3^{3^{n}}} c \text { dla każdego } n \geqslant 1
$$

Ustal, dla jakiej wartości parametru $c$ suma ta jest nie mniejsza od liczby $\log _{9}\left(c^{2}-2\right)$.\\
4. Rozwiąż nierówność

$$
\sqrt{\log _{\sqrt{x}}(x+2)}>\frac{1}{\log _{\sqrt{x+2}} \sqrt{x}} .
$$

\begin{enumerate}
  \setcounter{enumi}{4}
  \item Określ ilość rozwiązań równania
\end{enumerate}

$$
\left|2^{x-1}-1\right|=m \cdot 2^{x+1}
$$

w zależności od wartości parametru $m$.\\
6. Opisz metodę konstrukcji i starannie narysuj wykres funkcji

$$
f(x)=2+\log _{2} \frac{1}{2-x}
$$

Następnie narysuj obraz tej krzywej w symetrii względem prostej $x=y$. Wyprowadź wzór funkcji, której wykresem jest powstała w ten sposób krzywa.

Rozwiązania (rękopis) zadań z wybranego poziomu prosimy nadsyłać do 20.10.2022r. na adres:

\begin{verbatim}
Wydział Matematyki
Politechnika Wrocławska
Wybrzeże Wyspiańskiego 27
50-370 WROCEAW,
\end{verbatim}

lub elektronicznie, za pośrednictwem portalu \href{http://talent.pwr.edu.pl}{talent.pwr.edu.pl}\\
Na kopercie prosimy koniecznie zaznaczyć wybrany poziom! (np. poziom podstawowy lub rozszerzony). Do rozwiązań należy dołączyć zaadresowaną do siebie kopertę zwrotną z naklejonym znaczkiem, odpowiednim do formatu listu. Prace niespełniające podanych warunków nie będą poprawiane ani odsyłane.

Uwaga. Wysyłając nam rozwiązania zadań uczestnik Kursu udostępnia Politechnice Wrocławskiej swoje dane osobowe, które przetwarzamy wyłącznie w zakresie niezbędnym do jego prowadzenia (odesłanie zadań, prowadzenie statystyki). Szczegółowe informacje o przetwarzaniu przez nas danych osobowych są dostępne na stronie internetowej Kursu.

Adres internetowy Kursu: \href{http://www.im.pwr.edu.pl/kurs}{http://www.im.pwr.edu.pl/kurs}


\end{document}