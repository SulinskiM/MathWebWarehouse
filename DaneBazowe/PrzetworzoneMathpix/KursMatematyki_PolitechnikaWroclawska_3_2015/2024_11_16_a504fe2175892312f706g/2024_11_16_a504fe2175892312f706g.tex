\documentclass[10pt]{article}
\usepackage[polish]{babel}
\usepackage[utf8]{inputenc}
\usepackage[T1]{fontenc}
\usepackage{amsmath}
\usepackage{amsfonts}
\usepackage{amssymb}
\usepackage[version=4]{mhchem}
\usepackage{stmaryrd}
\usepackage{hyperref}
\hypersetup{colorlinks=true, linkcolor=blue, filecolor=magenta, urlcolor=cyan,}
\urlstyle{same}

\title{PRACA KONTROLNA nr 3 - POZIOM PODSTAWOWY }

\author{}
\date{}


\begin{document}
\maketitle
\begin{enumerate}
  \item Rozwiązać równanie $\operatorname{tg} x-\sin x=\frac{1-\cos x}{2 \cos x}$.
  \item Narysować wykres funkcji $f(x)=2 \sin x+|\sin x|$ i rozwiązać nierówność $|f(x)| \leqslant \frac{3 \sqrt{3}}{2}$.
  \item Odcinek $C D$ jest obrazem odcinka o końcach $A(1,1)$ i $B(2,0) \mathrm{w}$ jednokładności o środku $S(1,-1)$ i skali $k=-2$. Obliczyć pole czworokąta $A B C D$. Sporządzić rysunek.
  \item Wielomian $W(x)=x^{3}+a x^{2}+b x+c$ jest podzielny przez dwumian $x+1$, a jego wykres jest symetryczny względem punktu $(0,0)$. Wyznaczyć $a, b, c$ i rozwiązać nierówność
\end{enumerate}

$$
(x-1) W(x+2)-(x-2) W(x+1) \leqslant 0
$$

\begin{enumerate}
  \setcounter{enumi}{4}
  \item Punkty $\mathrm{A}(1,1), \mathrm{B}(0,3)$ są kolejnymi wierzchołkami rombu $A B C D$. Wyznaczyć pozostałe wierzchołki, wiedząc, że jeden z nich leży na prostej $x-y-2=0$. Sporządzić rysunek.
  \item W trójkąt równoramienny wpisano okrąg o promieniu $r$. Wyznaczyć pole trójkąta, jeżeli środek okręgu opisanego na tym trójkącie leży na okręgu wpisanym w ten trójkąt. Ile rozwiązań ma to zadanie? Sporządzić rysunek.
\end{enumerate}

\section*{PRACA KONTROLNA nr 3 - POZIOM ROZSZERZONY}
\begin{enumerate}
  \item Narysować wykres funkcji $f(x)=\cos 2 x-\sin ^{2} x$ i rozwiązać nierówność $f(x) \geqslant \frac{1}{4}$.
  \item Obliczyć pole trójkąta $A B C$ o wierzchołkach $A(3,6), B(1,0)$, wiedząc, że wysokości przecinają się w punkcie $(4,4)$. Sporządzić rysunek.
  \item Dla jakiego kąta ostrego $\alpha$ zachodzi równość
\end{enumerate}

$$
\log _{\sin \alpha}\left(2 \cos ^{2} \alpha+\sin \alpha \cos \alpha-1\right)=2 ?
$$

\begin{enumerate}
  \setcounter{enumi}{3}
  \item Dla jakiego parametru $p$ wielomian $W(x)=x^{3}+p x^{2}+11 x-6$ ma trzy pierwiastki, z których jeden jest średnią arytmetyczną pozostałych? Znaleźć wielomian o powyższej własności, którego wszystkie pierwiastki są wymierne.
  \item Wyznaczyć równania wszystkich prostych stycznych do każdej z parabol $y=(x+1)^{2}$ oraz $y=-(x-3)^{2}-2$. Sporządzić rysunek.
  \item W trójkącie równoramiennym $A B C$ sinus kąta przy wierzchołku $C$ jest równy $3 / 5$. Pod jakim kątem przecinają się środkowe poprowadzone z wierzchołków podstawy $A B$ ?
\end{enumerate}

Rozwiązania (rękopis) zadań z wybranego poziomu prosimy nadsyłać do 18 listopada 2015r. na adres:

\begin{verbatim}
Wydział Matematyki
Politechnika Wrocławska
Wybrzeże Wyspiańskiego 27
50-370 WROCEAW.
\end{verbatim}

Na kopercie prosimy koniecznie zaznaczyć wybrany poziom! (np. poziom podstawowy lub rozszerzony). Do rozwiązań należy dołączyć zaadresowaną do siebie kopertę zwrotną z naklejonym znaczkiem, odpowiednim do wagi listu. Prace niespełniające podanych warunków nie będą poprawiane ani odsyłane.

Adres internetowy Kursu: \href{http://www.im.pwr.edu.pl/kurs}{http://www.im.pwr.edu.pl/kurs}


\end{document}