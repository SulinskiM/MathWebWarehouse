\documentclass[10pt]{article}
\usepackage[polish]{babel}
\usepackage[utf8]{inputenc}
\usepackage[T1]{fontenc}
\usepackage{amsmath}
\usepackage{amsfonts}
\usepackage{amssymb}
\usepackage[version=4]{mhchem}
\usepackage{stmaryrd}
\usepackage{hyperref}
\hypersetup{colorlinks=true, linkcolor=blue, filecolor=magenta, urlcolor=cyan,}
\urlstyle{same}

\title{PRACA KONTROLNA nr 3 - POZIOM PODSTAWOWY }

\author{}
\date{}


\begin{document}
\maketitle
\begin{enumerate}
  \item Dwaj kolarze jeżdżą po torze w kształcie okręgu ze stałymi prędkościami. Jeżeli startują z tego samego punktu i jadą w tę samą stronę, to szybszy z nich pierwszy raz ponownie zrówna się z wolniejszym, wyprzedzając go o jedno okrążenie, po przejechaniu dokładnie 7 okrą̇̇en. Ilu okrążeń potrzebuje szybszy kolarz żeby dogonić kolegę, jeżeli startują z przeciwległych stron toru (tzn. odcinek łączący punkty ich startu jest średnicą koła)?
  \item Liczby o $16 \%$ mniejsza i o $43 \%$ większa od ułamka okresowego $0,(75)$ są pierwiastkami trójmianu kwadratowego o współczynnikach całkowitych względnie pierwszych. Obliczyć resztę z dzielenia tego trójmianu przez dwumian $(x-1)$.
  \item Rozwiązać równanie
\end{enumerate}

$$
\sin x+\cos x=\frac{1}{\sin x}
$$

\begin{enumerate}
  \setcounter{enumi}{3}
  \item Rozwiązać nierówność
\end{enumerate}

$$
\frac{\log _{2}\left(10-x^{2}\right)}{\log _{2}(4-x)}>2
$$

\begin{enumerate}
  \setcounter{enumi}{4}
  \item Dwa okręgi o promieniach $r$ i $R$ styczne zewnętrznie w punkcie $C$, są styczne do prostej $k$ w punktach $A$ i $B$. Wyznaczyć kąt $\angle A C B$ i promień okręgu opisanego na trójkącie $A B C$.
  \item Dane są punkty $A(2,-2)$ i $B(8,1)$. Na paraboli $y=x^{2}-x$ znaleźć taki punkt $C$, żeby pole trójkąta $A B C$ było najmniejsze. Wykonać rysunek.
\end{enumerate}

\section*{PRACA KONTROLNA nr 3 - POZIOM ROZSZERZONY}
\begin{enumerate}
  \item Czy wieża zbudowana z sześciennych klocków o objętościach $1,3,9,27$, zmieści się na półce o wysokości $\frac{15}{2}$ ? Odpowiedź uzasadnić nie stosując obliczeń przybliżonych.
  \item Rozwiązać równanie
\end{enumerate}

$$
\cos 2 x=(\sqrt{3}-1) \sin x(\cos x+\sin x)
$$

\begin{enumerate}
  \setcounter{enumi}{2}
  \item Sporządzić staranny wykres funkcji $f(x)=\left|2^{-|x|+1}-1\right|-\frac{1}{2}$. Opisać sposób postępowania. Rozwiązać nierówność $f(x)>0$.
  \item Rozwiązać nierówność
\end{enumerate}

$$
\log _{2} x+\log _{2}^{3} x+\log _{2}^{5} x+\ldots<\frac{20}{9}
$$

\begin{enumerate}
  \setcounter{enumi}{4}
  \item Pod jakim kątem przecinają się okręgi o równaniach $(x-6)^{2}+y^{2}=9, x^{2}+(y+4)^{2}=25$ (kątem miedzy dwoma okręgami nazywamy kąt między stycznymi w punkcie przecięcia)? Znaleźć równanie okręu, którego środek leży na prostej $2 x-y=0$, i który przecina każdy z danych okręgów pod kątem prostym.
  \item Boisko do gry w football amerykański ma kształt prostokąta o długości a i szerokości $b<a$. Na środku krótszych boków stoją bramki o szerokości $d<b$. Z którego miejsca linii bocznej boiska (czyli dłuższego boku prostokąta) widać bramkę pod największym możliwym kątem? Wyrazić odpowiedź za pomocą wzoru zawierającego symbole $a, b, d$, a następnie wykonać obliczenia dla wartości $a=110 \mathrm{~m}, b=49 \mathrm{~m}, d=5 \mathrm{~m}$.
\end{enumerate}

Rozwiązania (rękopis) zadań z wybranego poziomu prosimy nadsyłać do 18 listopada 2016r. na adres:

Wydział Matematyki\\
Politechnika Wrocławska\\
Wybrzeże Wyspiańskiego 27\\
50-370 WROCEAW.\\
Na kopercie prosimy koniecznie zaznaczyć wybrany poziom! (np. poziom podstawowy lub rozszerzony). Do rozwiązań należy dołączyć zaadresowaną do siebie kopertę zwrotną z naklejonym znaczkiem, odpowiednim do wagi listu. Prace niespełniające podanych warunków nie będą poprawiane ani odsyłane.

Adres internetowy Kursu: \href{http://www.im.pwr.wroc.pl/kurs}{http://www.im.pwr.wroc.pl/kurs}


\end{document}