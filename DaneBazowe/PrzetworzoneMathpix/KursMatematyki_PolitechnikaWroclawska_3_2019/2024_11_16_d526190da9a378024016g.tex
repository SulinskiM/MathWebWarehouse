\documentclass[10pt]{article}
\usepackage[polish]{babel}
\usepackage[utf8]{inputenc}
\usepackage[T1]{fontenc}
\usepackage{amsmath}
\usepackage{amsfonts}
\usepackage{amssymb}
\usepackage[version=4]{mhchem}
\usepackage{stmaryrd}
\usepackage{hyperref}
\hypersetup{colorlinks=true, linkcolor=blue, filecolor=magenta, urlcolor=cyan,}
\urlstyle{same}

\title{PRACA KONTROLNA nr 3 - POZIOM PODSTAWOWY }

\author{}
\date{}


\begin{document}
\maketitle
\begin{enumerate}
  \item Znaleźć największą wartość funkcji
\end{enumerate}

$$
f(x)=\frac{2}{\sqrt{4 x^{2}-12 x+11}}
$$

i rozwiązać nierówność $f(x) \geqslant 1$.\\
2. Rozwiązać równanie

$$
(1+\cos 4 x) \sin 2 x=\cos ^{2} 2 x
$$

\begin{enumerate}
  \setcounter{enumi}{2}
  \item Rozwiązać równanie
\end{enumerate}

$$
\log _{\sqrt{5}}\left(4^{x}-6\right)-\log _{\sqrt{5}}\left(2^{x}-2\right)=2
$$

\begin{enumerate}
  \setcounter{enumi}{3}
  \item Stosunek długości przekątnych rombu jest równy 5:12. Obliczyć stosunek pola rombu do do pola koła wpisanego w ten romb.
  \item Dane są punkty $A(1,1)$ i $B(7,4)$. Na paraboli $y=x^{2}+x+3$ znaleźć taki punkt $C$, żeby pole trójkąta $A B C$ było najmniejsze. Wykonać rysunek.
  \item Ramiona trójkąta równoramiennego zawarte są w prostych o równaniach $8 x-y+17=0$ oraz $4 x+7 y-59=0$, a jego podstawa przechodzi przez punkt $P(0,2)$. Wyznaczyć równanie prostej zawierającej podstawę i obliczyć pole tego trójkąta.
\end{enumerate}

\section*{PRACA KONTROLNA nr 3 - POZIOM ROZSZERZONY}
\begin{enumerate}
  \item Dla jakich wartości parametru $m$ równanie
\end{enumerate}

$$
x^{2}-2(m-4) x+m^{2}+5 m+6=0
$$

ma dwa różne pierwiastki rzeczywiste, których suma odwrotności jest dodatnia?\\
2. Rozwiązać równanie

$$
\frac{1}{\sin ^{2} 2 x}+\operatorname{tg} x-\operatorname{ctg} x=2
$$

\begin{enumerate}
  \setcounter{enumi}{2}
  \item Rozwiązać układ równań
\end{enumerate}

$$
\begin{cases}\frac{\log (x-y)-1}{2 \log 2-\log (x+y)} & =1 \\ \frac{\log x-\log 3}{\log y-\log 7} & =-1\end{cases}
$$

\begin{enumerate}
  \setcounter{enumi}{3}
  \item Dany jest trójkąt $A B C$, w którym $\angle A C B=\frac{2 \pi}{3}$. Dwusieczna kąta $A C B$ przecina prostą przechodzącą przez punkt $A$ i równoległą do boku $B C$ w punkcie $P$, a prostą przechodzącą przez punkt $B$ i równoległą do boku $A C$ w punkcie $Q$. Udowodnić, że $A Q=B P$.
  \item Wyznaczyć stosunek promienia okręgu wpisanego w romb $A B C D$ o kącie ostrym $\alpha=$ $\angle D A B$ do promienia okręgu opisanego na trójkącie $A B D$. Sprawdzić dla jakiego kąta $\alpha$ stosunek ten jest najwięszy.
  \item Wyznaczyć równanie zbioru wszystkich środków tych cięciw paraboli $y=x^{2}$, które zaczynają się w punkcie $A(1,1)$. Rozwiązanie zilustrować rysunkiem.
\end{enumerate}

Rozwiązania (rękopis) zadań z wybranego poziomu prosimy nadsyłać do 18 listopada 2019r. na adres:

Wydział Matematyki\\
Politechnika Wrocławska\\
Wybrzeże Wyspiańskiego 27\\
50-370 WROCEAW.\\
Na kopercie prosimy koniecznie zaznaczyć wybrany poziom! (np. poziom podstawowy lub rozszerzony). Do rozwiązań należy dołączyć zaadresowaną do siebie kopertę zwrotną z naklejonym znaczkiem, odpowiednim do wagi listu. Prace niespełniające podanych warunków nie będą poprawiane ani odsyłane.

Uwaga. Wysyłając nam rozwiązania zadań uczestnik Kursu udostępnia Politechnice Wrocławskiej swoje dane osobowe, które przetwarzamy wyłącznie w zakresie niezbędnym do jego prowadzenia (odesłanie zadań, prowadzenie statystyki). Szczegółowe informacje o przetwarzaniu przez nas danych osobowych są dostępne na stronie internetowej Kursu.\\
Adres internetowy Kursu: \href{http://www.im.pwr.edu.pl/kurs}{http://www.im.pwr.edu.pl/kurs}


\end{document}