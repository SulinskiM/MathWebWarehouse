\documentclass[10pt]{article}
\usepackage[polish]{babel}
\usepackage[utf8]{inputenc}
\usepackage[T1]{fontenc}
\usepackage{amsmath}
\usepackage{amsfonts}
\usepackage{amssymb}
\usepackage[version=4]{mhchem}
\usepackage{stmaryrd}
\usepackage{hyperref}
\hypersetup{colorlinks=true, linkcolor=blue, filecolor=magenta, urlcolor=cyan,}
\urlstyle{same}

\title{PRACA KONTROLNA nr 4 - POZIOM PODSTAWOWY }

\author{}
\date{}


\begin{document}
\maketitle
\begin{enumerate}
  \item Dla jakich kątów $\alpha \in\langle 0,2 \pi\rangle$ równanie $2 x^{2}-2(2 \cos \alpha-1) x+2 \cos ^{2} \alpha-5 \cos \alpha+2=0$ ma dwa różne pierwiastki rzeczywiste?
  \item Dane są punkty $A(-2,0), B(2,4)$ oraz $C(1,5)$. Oblicz pole trapezu $A B C D$, wiedząc, że punkt $D$ jest jednakowo odległy od punktów $A$ i $B$.
  \item W trójkącie równoramiennym kąt przy podstawie ma miarę $30^{\circ}$. Oblicz stosunek długości promienia okręgu opisanego na trójkącie do długości promienia okręgu wpisanego w trójkąt.
  \item Płaszczyzna przechodząca przez środek dolnej podstawy walca jest nachylona do podstawy pod kątem $\alpha$ i przecina górną podstawę walca wzdłuż cięciwy długości $a$. Cięciwa ta odcina łuk, na którym oparty jest kąt środkowy o mierze $120^{\circ}$. Oblicz objętość walca.
  \item Niech $x_{1}$ i $x_{2}$ będą pierwiastkami wielomianu $p(x)=x^{2}-x+a$, a $x_{3}$ i $x_{4}$ - pierwiastkami wielomianu $q(x)=x^{2}-4 x+b$. Dla jakich $a$ i $b$ liczby $x_{1}, x_{2}, x_{3}, x_{4}$ są kolejnymi wyrazami ciągu geometrycznego?
  \item Na dwóch zewnętrznie stycznych kulach opisano stożek tak, że środki tych kul leżą na wysokości stożka. Promień mniejszej kuli jest równy $r$, a stosunek objętości kul wynosi 8. Oblicz pole powierzchni bocznej stożka.
\end{enumerate}

\section*{PRACA KONTROLNA nr 4 - POZIOM RoZsZERZoNY}
\begin{enumerate}
  \item Dane są proste $y=4 x$ i $y=x-2$ oraz punkt $M=(1,2)$. Wyznacz współrzędne punktów $A$ i $B$ leżących odpowiednio na danych prostych takich, że punkty $A, B, M$ są współliniowe oraz $\frac{|A M|}{|B M|}=\frac{2}{3}$.
  \item W równoległoboku o kącie ostrym $60^{\circ}$ stosunek kwadratów długości przekątnych wynosi 1:3. Oblicz stosunek długości dwóch sąsiednich boków.
  \item Niech $a, b, c, d$ będą kolejnymi liczbami naturalnymi. Pokaż, że wielomian $w(x)=a x^{3}-$ $b x^{2}-c x+d$ ma trzy pierwiastki rzeczywiste, wśród których co najmniej jeden jest liczbą całkowitą. Dla jakich parametrów $a, b, c, d$ suma tych pierwiastków jest największa?
  \item Dla jakich kątów $\alpha \in\langle 0,2 \pi\rangle$ spełniona jest nierówność
\end{enumerate}

$$
2^{\sin ^{2} x}+\sqrt[4]{2} \cdot 2^{\cos ^{2} x} \leqslant \sqrt{2}+\sqrt[4]{8} ?
$$

\begin{enumerate}
  \setcounter{enumi}{4}
  \item W ostrosłupie prawidłowym czworokątnym o krawędzi podstawy a stosunek długości krawędzi podstawy do wysokości wynosi 2:3. Ostrosłup przecięto płaszczyzną przechodzącą przez krawędź podstawy i prostopadłą do przeciwległej ściany bocznej. Oblicz pole otrzymanego przekroju.
  \item Wierzchołek stożka jest środkiem kuli a brzeg podstawy stożka zawiera się w powierzchni kuli. Pole powierzchni całkowitej stożka stanowi $\frac{1}{4}$ pola powierzchni kuli. Oblicz stosunek objętości stożka do objętości kuli.
\end{enumerate}

Rozwiązania (rękopis) zadań z wybranego poziomu prosimy nadsyłać do 18 grudnia 2014r. na adres:

Instytut Matematyki i Informatyki\\
Politechniki Wrocławskiej\\
Wybrzeże Wyspiańskiego 27\\
50-370 WROCEAW.\\
Na kopercie prosimy koniecznie zaznaczyć wybrany poziom! (np. poziom podstawowy lub rozszerzony). Do rozwiązań należy dołączyć zaadresowaną do siebie kopertę zwrotną z naklejonym znaczkiem, odpowiednim do wagi listu. Prace niespełniające podanych warunków nie będą poprawiane ani odsyłane.

Adres internetowy Kursu: \href{http://www.im.pwr.wroc.pl/kurs}{http://www.im.pwr.wroc.pl/kurs}


\end{document}