\documentclass[10pt]{article}
\usepackage[polish]{babel}
\usepackage[utf8]{inputenc}
\usepackage[T1]{fontenc}
\usepackage{amsmath}
\usepackage{amsfonts}
\usepackage{amssymb}
\usepackage[version=4]{mhchem}
\usepackage{stmaryrd}
\usepackage{hyperref}
\hypersetup{colorlinks=true, linkcolor=blue, filecolor=magenta, urlcolor=cyan,}
\urlstyle{same}

\title{PRACA KONTROLNA nr 5 - POZIOM PODSTAWOWY }

\author{}
\date{}


\begin{document}
\maketitle
\begin{enumerate}
  \item Rozwiązać równanie $3^{\log _{\sqrt{3}}\left(2^{x}-1\right)}=2^{x+1}+1$.
  \item Jaki zbiór tworzą środki wszystkich cięciw przechodzących przez ustalony punkt zadanego okręgu?
  \item Narysować wykres funkcji $f(x)=\frac{|x+2|-1}{x-1}$. Wyznaczyć zbiór jej wartości oraz najmniejszą i największą wartość na przedziale $[-3,0]$.
  \item Niech $T$ będzie przekształceniem płaszczyzny polegającym na przesunięciu o wektor $[1,2]$, a $S$ - symetrią względem prostej $y=x$. Wyznaczyć (analitycznie) obrazy kwadratu o wierzchołkach $(0,1),(1,1),(1,2)$ i $(0,2)$ w przekształceniach $S \circ T$ i $T \circ S$. Sporządzić staranne rysunki.
  \item Wspólne styczne do stycznych zewnętrznie okręgów o promieniach $r<R$ przecinają się pod kątem $2 \alpha$. Wyznaczyć stosunek pól tych okręgów. Dla jakiego kąta $\alpha$ duże koło ma 9 razy większe pole niż małe?
  \item Pole powierzchni całkowitej ostrosłupa prawidłowego trójkątnego jest 4 razy większe od pola jego podstawy. Obliczyć sinus kąta między ścianami ostrosłupa.
\end{enumerate}

\section*{PRACA KONTROLNA nr 5 - POZIOM ROZSZERZONY}
\begin{enumerate}
  \item W rozwinięciu $(a+b)^{n}=\sum_{k=0}^{n}\binom{n}{k} a^{n-k} b^{k} \quad$ dla $a=\sqrt{x}, \quad b=\frac{1}{2 \sqrt[4]{x}} \quad$ trzy pierwsze współczynniki przy potęgach $x$ tworzą ciąg arytmetyczny. Znaleźć wszystkie składniki rozwinięcia, w którym $x$ występuje w potędze o wykładniku całkowitym.
  \item Punkty $K, L, M$ dzielą boki $A B, B C, C A$ trójkąta $A B C$ (odpowiednio) w tym samym stosunku, tzn.
\end{enumerate}

$$
\frac{|K B|}{|A B|}=\frac{|L C|}{|B C|}=\frac{|M A|}{|C A|}=s
$$

Wykazać, że dla dowolnego punktu $P$ znajdującego się wewnątrz trójkąta zachodzi równość

$$
\overrightarrow{P K}+\overrightarrow{P L}+\overrightarrow{P M}=\overrightarrow{P A}+\overrightarrow{P B}+\overrightarrow{P C}
$$

\begin{enumerate}
  \setcounter{enumi}{2}
  \item Narysować wykres funkcji $f(x)=\frac{(x+1)^{2}-1}{x|x-1|}$. Wyznaczyć styczną do wykresu w punkcie $(-2, f(-2))$ oraz styczną do niej prostopadłą.
  \item Końce odcinka $A B$ o długości $l$ poruszają się po okręgu o promieniu $R(l<2 R)$. Na odcinku obrano punkt $P$ tak, że $\frac{|A P|}{|P B|}=\frac{1}{3}$. Uzasadnić, że poruszający się punkt $P$ zakreśla okrąg o tym samym środku. Dla jakiego $l$ wycięte w ten sposób koło ma pole dwa razy mniejsze od pola dużego koła.
  \item Rozważamy zbiór wszystkich trójkątów o polu 10 , których jednym z wierzchołków jest $A(5,0)$ a pozostałe dwa leżą na osi $O y$. Wyznaczyć zbiór wszystkich punktów płaszczyzny, które są środkami okręgów opisanych na tych trójkątach.
  \item W przeciwległe narożniki sześcianu o boku 1 wpisano dwie kule o takich samych promieniach tak, że każda z nich jest styczna do drugiej i do trzech ścian wychodzących z odpowiedniego wierzchołka. Jaka jest odległość ich środków?
\end{enumerate}

Rozwiązania (rękopis) zadań z wybranego poziomu prosimy nadsyłać do 18 stycznia 2018r. na adres:

\begin{verbatim}
Wydział Matematyki
Politechnika Wrocławska
Wybrzeże Wyspiańskiego 27
50-370 WROCEAW.
\end{verbatim}

Na kopercie prosimy koniecznie zaznaczyć wybrany poziom! (np. poziom podstawowy lub rozszerzony). Do rozwiązań należy dołączyć zaadresowaną do siebie kopertę zwrotną z naklejonym znaczkiem, odpowiednim do wagi listu. Prace niespełniające podanych warunków nie będą poprawiane ani odsyłane.

Adres internetowy Kursu: \href{http://www.im.pwr.wroc.pl/kurs}{http://www.im.pwr.wroc.pl/kurs}


\end{document}