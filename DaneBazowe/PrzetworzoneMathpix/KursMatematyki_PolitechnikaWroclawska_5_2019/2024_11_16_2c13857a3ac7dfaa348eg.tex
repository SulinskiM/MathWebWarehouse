\documentclass[10pt]{article}
\usepackage[polish]{babel}
\usepackage[utf8]{inputenc}
\usepackage[T1]{fontenc}
\usepackage{amsmath}
\usepackage{amsfonts}
\usepackage{amssymb}
\usepackage[version=4]{mhchem}
\usepackage{stmaryrd}
\usepackage{hyperref}
\hypersetup{colorlinks=true, linkcolor=blue, filecolor=magenta, urlcolor=cyan,}
\urlstyle{same}

\title{PRACA KONTROLNA nr 5 - POZIOM PODSTAWOWY }

\author{}
\date{}


\begin{document}
\maketitle
\begin{enumerate}
  \item Znaleźć stuelementowy ciąg arytmetyczny, w którym suma wyrazów o numerach nieparzystych jest dwa razy większa od sumy wyrazów o numerach parzystych i o 50 mniejsza od sumy wszystkich wyrazów.
  \item Rozwiązać układ równań $\left\{\begin{aligned} x^{2}+1 & =2^{y-1}, \\ y-2 & =\log _{2}(x+2) .\end{aligned}\right.$
  \item Narysować wykres funkcji $f(x)=x|x|-4|x|+3$ i określić liczbę rozwiązań równania $f(x)=m$ w zależności od parametru $m$.
  \item W romb $A B C D$ o kącie ostrym $\alpha$ wpisano czworokąt, którego boki są równoległe do przekątnych rombu. Jakie jest możliwie największe pole takiego czworokąta?
  \item Znaleźć równania wspólnych stycznych do wykresów funkcji
\end{enumerate}

$$
f(x)=-x^{2}+2 x \quad \text { i } \quad g(x)=x^{2}+1
$$

\begin{enumerate}
  \setcounter{enumi}{5}
  \item W stożek o promieniu podstawy $R$ wpisano stożek o osiem razy mniejszej objętości. Wysokość małego stożka jest zawarta w wysokości dużego stożka, jego wierzchołek jest w środku podstawy, a okrąg ograniczający podstawę małego stożka jest zawarty w powierzchni bocznej dużego stożka. Obliczyć $\frac{r}{R}$, gdzie $r$ oznacza promień podstawy stożka wpisanego.
\end{enumerate}

\section*{PRACA KONTROLNA nr 5 - POZIOM ROZSZERZONY}
\begin{enumerate}
  \item Rozwiązać układ równań $\left\{\begin{array}{c}x^{\log _{2} y-1}=16, \\ (2 y)^{\log _{2} x-1}=16\end{array}\right.$
  \item Wyznaczyć równania wszystkich stycznych do wykresu funkcji $f(x)=\frac{2 x-1}{x+1}$, które są prostopadłe do prostej $x+3 y+1=0$.
  \item Granicą ciągu o wyrazie ogólnym $a_{n}=n^{2}-\sqrt{n^{4}-a n^{2}+b n}$ jest większy z pierwiastków równania
\end{enumerate}

$$
x^{\log _{2} x}-3=4 x^{\log _{\frac{1}{2}} x} .
$$

Wyznaczyć parametry $a$ i $b$.\\
4. Na boku $B C$ trójkąta równobocznego obrano punkt $D$ tak, że promień okręgu wpisanego w trójkąt $A D C$ jest dwa razy mniejszy niż promień okręgu wpisanego w trójkąt $A B D$. W jakim stosunku punkt $D$ dzieli bok $B C$ ?\\
5. Rozwiązać nierówność

$$
1+\frac{\sin x}{\sqrt{3}+\sin x}+\left(\frac{\sin x}{\sqrt{3}+\sin x}\right)^{2}+\left(\frac{\sin x}{\sqrt{3}+\sin x}\right)^{3}+\cdots \leqslant \cos x
$$

której lewa strona jest sumą wszystkich wyrazów nieskończonego ciągu geometrycznego.\\
6. Jakie wymiary ma walec o możliwie największej objętości wpisany w sześcian o boku $a$ w taki sposób, że jego oś jest zawarta w przekątnej sześcianu, a każda z podstaw jest styczna do trzech ścian wychodzących z jednego wierzchołka.

Rozwiązania (rękopis) zadań z wybranego poziomu prosimy nadsyłać do 18 stycznia 2019r. na adres:

\begin{verbatim}
Wydział Matematyki
Politechnika Wrocławska
Wybrzeże Wyspiańskiego 27
50-370 WROCEAW.
\end{verbatim}

Na kopercie prosimy koniecznie zaznaczyć wybrany poziom! (np. poziom podstawowy lub rozszerzony). Do rozwiązań należy dołączyć zaadresowaną do siebie kopertę zwrotną z naklejonym znaczkiem, odpowiednim do wagi listu. Prace niespełniające podanych warunków nie będą poprawiane ani odsyłane.

Uwaga. Wysyłając nam rozwiązania zadań uczestnik Kursu udostępnia nam swoje dane osobowe, które przetwarzamy wyłącznie w zakresie niezbędnym do jego prowadzenia (odesłanie zadań, prowadzenie statystyki). Szczegółowe informacje o przetwarzaniu przez nas danych osobowych są dosteqpne na stronie internetowej Kursu.\\
Adres internetowy Kursu: \href{http://www.im.pwr.edu.pl/kurs}{http://www.im.pwr.edu.pl/kurs}


\end{document}