\documentclass[10pt]{article}
\usepackage[polish]{babel}
\usepackage[utf8]{inputenc}
\usepackage[T1]{fontenc}
\usepackage{amsmath}
\usepackage{amsfonts}
\usepackage{amssymb}
\usepackage[version=4]{mhchem}
\usepackage{stmaryrd}
\usepackage{hyperref}
\hypersetup{colorlinks=true, linkcolor=blue, filecolor=magenta, urlcolor=cyan,}
\urlstyle{same}

\title{PRACA KONTROLNA nr 6 - POZIOM PODSTAWOWY }

\author{}
\date{}


\begin{document}
\maketitle
\begin{enumerate}
  \item Rozwiązać równanie
\end{enumerate}

$$
\frac{\sin x}{2 \cos ^{2} 2 x-1}=1
$$

\begin{enumerate}
  \setcounter{enumi}{1}
  \item Niech $f(x)=\sqrt{x}$. Podać wzór funkcji:\\
a) $g(x)$, której wykres jest symetrycznym obrazem wykresu $f(x)$ względem prostej $x=1$.\\
b) $h(x)$, której wykres jest symetrycznym obrazem wykresu $f(x)$ względem punktu $(0,-1)$. Narysować wykresy wszystkich funkcji. Uzasadnić, wykonując odpowiednie obliczenia, że znalezione funkcje spełniaja podane warunki.
  \item Wykazać, że dla dowolnego $n \geqslant 2$ liczba $\frac{1}{4} \cdot 100^{n}+4 \cdot 10^{n}+16$ jest kwadratem liczby naturalnej i jest podzielna przez 81.
  \item Narysować wykres funkcji
\end{enumerate}

$$
f(x)=\left\{\begin{array}{lll}
2-x-x^{2} & , \text { gdy }-1 \leqslant x \leqslant 1 \\
\frac{x-1}{x+1} & , \text { gdy } & |x|>1
\end{array}\right.
$$

Posługując się wykresem, podać zbiór wartości funkcji $f$ oraz jej najmniejszą i największą wartość na przedziałach $[-1,2]$ oraz $[0,3]$.\\
5. Znaleźć równanie stycznej $l$ do paraboli $y=x^{2}$ równoległej do prostej $y=2 x-3$.

Wyznaczyć punkt, w którym styczna do tej paraboli jest prostopadła do znalezionej prostej $l$. Sporządzić rysunek.\\
6. Rozwiązać układ równań

$$
\left\{\begin{array}{l}
x^{2}+y^{2}=8 \\
\frac{1}{x}+\frac{1}{y}=1
\end{array}\right.
$$

i podać jego interpretację geometryczną.

\section*{PRACA KONTROLNA nr 5 - POZIOM ROZSZERZONY}
\begin{enumerate}
  \item Niech $f(x)=\frac{x-1}{x+2}$. Podać i uzasadnić wzór funkcji, której wykres jest obrazem symetrycznym wykresu funkcji $f(x)$ względem prostej $x=2$. Sporządzić wykresy obu funkcji w jednym układzie współrzędnych.
  \item Stosując zasadę indukcji matematycznej, udowodnić prawdziwość wzoru
\end{enumerate}

$$
\binom{2}{2}+\binom{4}{2}+\cdots+\binom{2 n}{2}=\frac{n(n+1)(4 n-1)}{6} \quad \text { dla } n \geqslant 1
$$

\begin{enumerate}
  \setcounter{enumi}{2}
  \item Wykorzystując metody rachunku różniczkowego znaleźć zbiór wartości funkcji
\end{enumerate}

$$
f(x)=x^{3}-3 x^{2}-9 x+3
$$

na przedziale $[-1,4]$. Wyznaczyć przedziały o długości 1 , w których znajdują się miejsca zerowe tej funkcji i sporządzić jej wykres.\\
4. Znaleźć równanie stycznej $l$ do wykresu funkcji $f(x)=\frac{2}{x}+x^{2}$ w punkcie przecięcia z prostą $y=x$. Wyznaczyć wszystkie styczne równoległe do znalezionej prostej $l$.\\
5. Narysować wykres funkcji

$$
f(x)=1+\frac{\sin x}{1+\sin x}+\left(\frac{\sin x}{1+\sin x}\right)^{2}+\left(\frac{\sin x}{1+\sin x}\right)^{3}+\left(\frac{\sin x}{1+\sin x}\right)^{4}+\ldots
$$

gdzie prawa strona jest sumą wszystkich wyrazów nieskończonego ciągu geometrycznego. Rozwiązać nierówność

$$
f(x)>\sqrt{3} \cos x
$$

\begin{enumerate}
  \setcounter{enumi}{5}
  \item Wyznaczyć liczbę rozwiązań układu równań
\end{enumerate}

$$
\left\{\begin{array}{l}
x^{2}+y^{2}=2 y \\
y=x^{2}-p
\end{array}\right.
$$

w zależności od parametru $p$. Podać interpretację geometryczną układu.

Rozwiązania prosimy nadsyłać do dnia 18 lutego 2017 na adres:

\begin{verbatim}
Wydział Matematyki
Politechniki Wrocławskiej
Wybrzeże Wyspiańskiego 27
50-370 Wrocław.
\end{verbatim}

Na kopercie prosimy koniecznie zaznaczyć wybrany poziom. Do rozwiązań należy dołączyć zaadresowaną do siebie kopertę zwrotną z naklejonym znaczkiem, odpowiednim do wagi listu. Prace niespełniające podanych warunków nie będą poprawiane ani odsyłane.

Adres internetowy Kursu: \href{http://www.im.pwr.edu.pl/kurs}{http://www.im.pwr.edu.pl/kurs}


\end{document}