\documentclass[10pt]{article}
\usepackage[polish]{babel}
\usepackage[utf8]{inputenc}
\usepackage[T1]{fontenc}
\usepackage{amsmath}
\usepackage{amsfonts}
\usepackage{amssymb}
\usepackage[version=4]{mhchem}
\usepackage{stmaryrd}
\usepackage{hyperref}
\hypersetup{colorlinks=true, linkcolor=blue, filecolor=magenta, urlcolor=cyan,}
\urlstyle{same}

\title{XLIX \\
 KORESPONDENCYJNY KURS \\
 Z MATEMATYKI }

\author{}
\date{}


\begin{document}
\maketitle
luty 2020 r.

\section*{PRACA KONTROLNA nr 6 - POZIOM PODSTAWOWY}
\begin{enumerate}
  \item W szufladzie znajduje się 6 różnych par rękawiczek. Oblicz prawdopodobieństwo, że wśród 5 losowo wybranych rękawic jest co najmniej jedna para.
  \item Wyznacz dziedzinę i zbadaj, dla jakich argumentów funkcja
\end{enumerate}

$$
f(x)=\log _{\sqrt{3}}(x+3)-\log _{3}\left(9-x^{2}\right)
$$

przyjmuje wartości ujemne.\\
3. Wśród prostokątów wpisanych w okrąg o promieniu $R$ bez użycia metod rachunku różniczkowego wskaż ten, którego pole jest największe.\\
4. Rozwiąż nierówność

$$
4^{x^{3}-x+2} \cdot 5^{2 x-3 x^{2}}-2^{4-3 x^{2}} \cdot 25^{x^{3}} \geqslant 0
$$

\begin{enumerate}
  \setcounter{enumi}{4}
  \item Powierzchnia boczna stożka po rozcięciu jest wycinkiem koła o kącie $216^{\circ}$. Obwód podstawy stożka wynosi $6 \pi$. Oblicz objętość kuli wpisanej w ten stożek.
  \item Narysuj wykres funkcji
\end{enumerate}

$$
f(x)=-1+2^{1-|1-|x||}
$$

i precyzyjnie opisz zastosowaną metodę jego konstrukcji. Na podstawie rysunku wskaż przedziały monotoniczności funkcji oraz zbiór jej wartości.

\section*{PRACA KONTROLNA nr 6 - POZIOM ROZSZERZONY}
\begin{enumerate}
  \item Developer chce pomalować każde z 11 pięter nowo wybudowanego wieżowca na jeden z 3 kolorów występujących w jego logo, przy czym każdy kolor ma zostać wykorzystany co najmniej jeden raz. Obliczyć prawdopodobieństwo, że dwaj niezależni graficy, którym zlecono zaprojektowanie kolorystyki budynku, przedstawią ten sam projekt. Przyjąć, że wybór przez nich każdego takiego układu kolorów jest jednakowo prawdopodobny.
  \item Rozwiąż równanie
\end{enumerate}

$$
8 x^{3}=1+6 x
$$

stosując podstawienie $x=\cos \alpha$.\\
3. Określ dziedzinę i zbadaj, dla jakich argumentów funkcja

$$
f(x)=\log _{x^{2}-1}\left(x^{2}-2 x\right)-\log _{x^{2}-1}\left(2-\frac{4}{x}\right)
$$

przyjmuje wartości nieujemne.\\
4. Rozwiąż nierówność

$$
1+\operatorname{tg}^{2} 2 x-\operatorname{tg}^{4} 2 x+\operatorname{tg}^{6} 2 x-\ldots \leqslant 3 \sin 2 x-\sin ^{2} 2 x .
$$

\begin{enumerate}
  \setcounter{enumi}{4}
  \item Wśród prostopadłościanów o podstawie kwadratu wpisanych w kulę o promieniu $R$ wskaż ten, którego objętość jest największa.
  \item Określ dziedzinę, wyznacz przedziały monotoniczności oraz wszystkie lokalne ekstrema funkcji
\end{enumerate}

$$
f(x)=\frac{(x+1)^{2}}{x(x-2)}
$$

Sporządź jej staranny wykres.\\
Rozwiązania (rękopis) zadań z wybranego poziomu prosimy nadsyłać do 18 lutego 2020r. na adres:

\begin{verbatim}
Wydział Matematyki
Politechnika Wrocławska
Wybrzeże Wyspiańskiego 27
50-370 WROCEAW.
\end{verbatim}

Na kopercie prosimy koniecznie zaznaczyć wybrany poziom! (np. poziom podstawowy lub rozszerzony). Do rozwiązań należy dołączyć zaadresowaną do siebie kopertę zwrotną z naklejonym znaczkiem, odpowiednim do formatu listu. Polecamy stosowanie kopert formatu C5 ( $160 \times 230 \mathrm{~mm}$ ) ze znaczkiem o wartości 3,30 zł. Na każdą większą kopertę należy nakleić droższy znaczek. Prace niespełniające podanych warunków nie będą poprawiane ani odsyłane.

Uwaga. Wysyłając nam rozwiązania zadań uczestnik Kursu udostępnia Politechnice Wrocławskiej swoje dane osobowe, które przetwarzamy wyłącznie w zakresie niezbędnym do jego prowadzenia (odesłanie zadań, prowadzenie statystyki). Szczegółowe informacje o przetwarzaniu przez nas danych osobowych są dostępne na stronie internetowej Kursu.

Adres internetowy Kursu: \href{http://www.im.pwr.edu.pl/kurs}{http://www.im.pwr.edu.pl/kurs}


\end{document}