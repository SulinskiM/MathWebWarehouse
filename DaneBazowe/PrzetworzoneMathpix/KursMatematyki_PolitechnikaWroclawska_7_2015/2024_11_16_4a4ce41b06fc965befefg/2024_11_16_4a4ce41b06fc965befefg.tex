\documentclass[10pt]{article}
\usepackage[polish]{babel}
\usepackage[utf8]{inputenc}
\usepackage[T1]{fontenc}
\usepackage{amsmath}
\usepackage{amsfonts}
\usepackage{amssymb}
\usepackage[version=4]{mhchem}
\usepackage{stmaryrd}

\title{PRACA KONTROLNA nr 7 -POZIOM PODSTAWOWY }

\author{}
\date{}


\begin{document}
\maketitle
\begin{enumerate}
  \item Współczynniki $a, b$ trójmianu kwadratowego $x^{2}-2 a x+b$ oraz pierwiastki tego trójmianu, napisane w odpowiedniej kolejności, są czterema początkowymi wyrazami pewnego ciągu arytmetycznego. Dla $a=2$ obliczyć różnicę ciągu, współczynnik $b$ oraz pierwiastki trójmianu.
  \item Kwadrat o boku $a$ zgięto wzdłuż jednej z przekątnych tak, aby odległość pozostałych wierzchołków była równa połowie długości przekątnej kwadratu. W tak powstały czworościan wpisano dwie identyczne, wzajemnie styczne kule. Obliczyć promień tych kul.
  \item Trzy czerwone, trzy żółte i jedną zieloną kredkę włożono w przypadkowy sposób do pudełka. Obliczyć prawdopodobieństwo tego, że żadne dwie kredki tego samego koloru nie będą leżały obok siebie.
  \item Wyznaczyć dziedzinę funkcji $f(x)=\sqrt{\frac{\log _{2} x}{1-\log _{2} x}}$. Uzasadnić, że $f(x)$ jest rosnąca. Korzystajac z tego faktu, określić zbiór wartości funkcji $f(x)$.
  \item W ostrosłup prawidłowy czworokatny wpisano prostopadłościan prosty o podstawie kwadratowej w ten sposób, że wierzchołki jego górnej podstawy leża w środkach ciężkości ścian bocznych ostrosłupa. Pole powierzchni całkowitej prostopadłościanu stanowi trzecią część pola powierzchni całkowitej ostrosłupa. Obliczyć tangens kąta nachylenia krawędzi bocznej ostrostupa do podstawy.
  \item Rozwiązać układ równań
\end{enumerate}

$$
\left\{\begin{array}{l}
x^{2}+y^{2}=2 \\
\frac{1}{x}+\frac{1}{y}=2
\end{array}\right.
$$

Podać interpretację geometryczną tego układu i sporzadzić rysunek.

\section*{PRACA KONTROLNA nr 7 -POZIOM ROZSZERZONY}
\begin{enumerate}
  \item Na każdym z trzech drutów linii elektrycznej wysokiego napięcia siedzi po pięć wróbli. W pewnej chwili odfrunęło przypadkowych sześć wróbli. Obliczyć prawdopodobieństwo tego, że na co najmniej dwóch drutach pozostała taka sama liczba ptaków.
  \item Dolna część namiotu ma ksztalt walca o wysokości $h=2 \mathrm{~m}$, a górna jest stożkiem o tworzacej $l=\sqrt{15} \mathrm{~m}$ i tym samym promieniu, co część dolna. Wyznaczyć pozostałe parametry namiotu tak, aby jego objętość była największa. Sporządzić rysunek.
  \item Z pudełka zawierającego 10 klocków ponumerowanych cyframi od 0 do 9 wylosowano dwa klocki i ustawiono obok siebie w przypadkowej kolejności, tworząc w ten sposób liczbę $k$ (ustawienie 03 rozumiemy jako liczbę 3). Następnie wylosowano trzeci klocek z pozostałych i ustawiono go za tamtymi, gdy suma cyfr liczby $k$ była mniejsza niż 10, lub przed tamtymi, w przeciwnym wypadku. Obliczyć prawdopodobieństwo tego, że otrzymana liczba jest większa od 500.\\
Wsk. Użyć wzoru na prawdopodobieństwo całkowite.
  \item Stosując zasadę indukcji matematycznej, udowodnić tożsamość
\end{enumerate}

$$
\sin ^{2} \alpha+\sin ^{2} 3 \alpha+\ldots+\sin ^{2}(2 n-1) \alpha=\frac{n}{2}-\frac{\sin 4 n \alpha}{4 \sin 2 \alpha}, \quad n \geqslant 1
$$

gdzie $\alpha \neq k \frac{\pi}{2}, k$ całkowite.\\
5. Znaleźć równanie stycznej $l$ do wykresu funkcji $f(x)=\frac{1}{x}+x^{2}$ w punkcie, w którym przecina on oś $O x$. Wyznaczyć wszystkie styczne, które są równoległe do prostej l. Znaleźć punkty wspólne tych stycznych z wykresem funkcji. Rozwiązanie zilustrować odpowiednim rysunkiem.\\
6. Krawędź podstawy graniastosłupa trójkątnego prawidłowego ma długość $a$. Oznaczmy przez $2 \alpha$ kąt między przekątnymi ścian bocznych wychodzącymi z jednego wierzchołka. Graniastosłup przecięto na dwie części płaszczyzną przechodzącą przez krawędź dolnej podstawy i przeciwległy wierzchołek górnej podstawy. Obliczyć tangens kąta $\alpha$, dla którego w większą część graniastosłupa można wpisać kulę. Dla znalezionego kąta $\alpha$, obliczyć promień kuli wpisanej w mniejszą część graniastosłupa.


\end{document}