\documentclass[10pt]{article}
\usepackage[polish]{babel}
\usepackage[utf8]{inputenc}
\usepackage[T1]{fontenc}
\usepackage{amsmath}
\usepackage{amsfonts}
\usepackage{amssymb}
\usepackage[version=4]{mhchem}
\usepackage{stmaryrd}

\title{PRACA KONTROLNA nr 7 - POZIOM PODSTAWOWY }

\author{}
\date{}


\begin{document}
\maketitle
\begin{enumerate}
  \item Pierwszym wyrazem ciagu arytmetycznego jest $a_{1}=2017$, a jego różnica jest rozwiązaniem równania $\sqrt{2-x}-x=10$. Obliczyć sumę wszystkich dodatnich wyrazów tego ciagu.
  \item Spośród dwucyfrowych liczb nieparzystych mniejszych od 50 wylosowano bez zwracania dwie. Obliczyć prawdopodobieństwo tego, że obie wylosowane liczby sa pierwsze oraz prawdopodobieństwo tego, że iloczyn wylosowanych liczb nie jest podzielny przez 15.
  \item Uzasadnić, że ciąg o wyrazach $a_{n}=\frac{2^{n}+2^{n+1}+\ldots+2^{2 n}}{2^{2}+2^{4}+\ldots+2^{2 n}}, n \geq 1$, nie jest rosnący oraz, że jest rosnący, poczynając od $n=2$.
  \item Znaleźć wszystkie wartości parametru rzeczywistego $m$, dla których proste o równaniach $x-m y+2 m=0,2 m x+4 y+1=0, m x-y-3 m-1=0$ są parami różne i przecinaja się w jednym punkcie. Sporządzić odpowiedni rysunek dla najmniejszej ze znalezionych wartości tego parametru.
  \item W ostrosłupie prawidłowym czworokątnym dana jest odległość $d$ środka podstawy od krawędzi bocznej oraz kąt $2 \alpha$ między sąsiednimi ścianami bocznymi. Obliczyć objętość ostrostupa.
  \item Podstawa $A B$ trójkąta równoramiennego $A B C$ jest krótsza od ramion. Wysokości $A D$ i $C E$ dzielą trojkąt na cztery części, z których dwie są trójkątami prostokątnymi o polach równych 9 oraz 2. Obliczyć pola pozostałych części oraz obwód trójkąta.
\end{enumerate}

\section*{PRACA KONTROLNA nr 7 - POZIOM ROZSZERZONY}
\begin{enumerate}
  \item Turysta zabłądził w lesie zajmującym obszar (w km)
\end{enumerate}

$$
D=\left\{(x, y): x^{2}+y^{2} \leq 2 y+3,-2 y \leq x \leq y\right\}
$$

Wskazać mu najkrótszą drogę wyjścia z lasu, jeśli znaduje się w punkcie $P\left(-\frac{1}{4}, \frac{3}{2}\right)$. Ile minut będzie trwała wędrówka, jeśli idzie z prędkością $4 \mathrm{~km} / \mathrm{h}$ ?\\
2. Korzystając z zasady indukcji matematycznej, udowodnić prawdziwość nierówności

$$
1^{5}+2^{5}+\ldots+n^{5}<\frac{n^{3}(n+1)^{3}}{6}, n \geq 1
$$

\begin{enumerate}
  \setcounter{enumi}{2}
  \item Kubuś zaobserwował, że w pewnej chwili w trzypiętrowej kamienicy po drugiej stronie ulicy pali się światło w 10 oknach. Na każdej kondygnacji kamienicy znajdują się 4 okna. Zakładamy, że okna zapalają się i gasną losowo. Obliczyć prawdopodobieństwo tego, że zarówno na drugim jak i na trzecim piętrze kamienicy świecą się co najmniej dwa okna. Wsk. Skorzystać ze wzoru $P(A \cup B)=P(A)+P(B)-P(A \cap B)$.
  \item Podstawa graniastosłupa prostego o wysokości $h=2$ jest trójkat, w którym tangens kąta przy wierzchołku $A$ wynosi $-\sqrt{2}$. Przekạtne $e, f$ sasiednich ścian bocznych, wychodzące z wierzchołka $A$, są do siebie prostopadłe, a liczby $h, e, f$ są kolejnymi wyrazami pewnego ciągu geometrycznego. Obliczyć objętość graniastosłupa.
  \item Znaleźć dziedzinę i zbiór wartości funkcji
\end{enumerate}

$$
f(x)=\sqrt{\log _{2} \frac{1}{\cos x+\sqrt{3} \sin x}} .
$$

\begin{enumerate}
  \setcounter{enumi}{5}
  \item Kąt płaski przy wierzchołku $D$ ostrosłupa prawidłowego trójkątnego o podstawie $A B C$ jest równy $\alpha$. Na krawędzi $B D$ wybrano punkt $E$ tak, że $\triangle A C E$ jest trójkątem równobocznym. Znaleźć stosunek $k(\alpha)$ objętości ostrosłupa $A B C E$ do objętości ostrosłupa $A C E D$ w zależności od kąta $\alpha$. Sporządzić wykres funkcji $k(\alpha)$.
\end{enumerate}

Rozwiązania (rękopis) zadań z wybranego poziomu prosimy nadsyłać do 18 marca 2017 r. na adres:

\section*{Wydział Matematyki \\
 Politechniki Wrocławskiej, \\
 ul. Wybrzeże Wyspiańskiego 27, \\
 50-370 WROCEAW.}
Na kopercie prosimy koniecznie zaznaczyć wybrany poziom! (np. poziom podstawowy lub rozszerzony). Do rozwiązań należy dołączyć zaadresowaną do siebie kopertę zwrotną z naklejonym znaczkiem, odpowiednim do wagi listu (od 1.01.2017 nowe ceny znaczków!). Prace nie spełniające podanych warunków nie będą poprawiane ani odsyłane.


\end{document}