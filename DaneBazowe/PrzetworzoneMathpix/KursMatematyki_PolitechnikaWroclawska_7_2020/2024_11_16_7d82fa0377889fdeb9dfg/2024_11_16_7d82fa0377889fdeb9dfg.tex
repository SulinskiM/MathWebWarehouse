\documentclass[10pt]{article}
\usepackage[polish]{babel}
\usepackage[utf8]{inputenc}
\usepackage[T1]{fontenc}
\usepackage{amsmath}
\usepackage{amsfonts}
\usepackage{amssymb}
\usepackage[version=4]{mhchem}
\usepackage{stmaryrd}
\usepackage{bbold}
\usepackage{hyperref}
\hypersetup{colorlinks=true, linkcolor=blue, filecolor=magenta, urlcolor=cyan,}
\urlstyle{same}

\title{PRACA KONTROLNA nr 7 - POZIOM PODSTAWOWY }

\author{}
\date{}


\begin{document}
\maketitle
\begin{enumerate}
  \item Wyznaczyć $z$ jako funkcję zmiennej $y$, wiedząc, że $x=2^{\frac{1}{1-\log _{2} z}}$ oraz $y=2^{\frac{1}{1-\log _{2} x}}$.
  \item Pokazać, że dla każdej wartości parametru $\alpha \in[0,2 \pi]$, dla której istnieje rozwiązanie równania $x^{2}-2 \cos \alpha \cdot x+\sin ^{2} \alpha=0$ suma kwadratów jego pierwiastków jest równa przynajmniej 1.
  \item W zależności od parametru rzeczywistego $k$ przedyskutować liczbę rozwiązań układu równań $\left\{\begin{array}{l}y=|1-|x-1||, \\ y=k x+k-1\end{array}\right.$\\
Sporządzić ilustrację graficzną układu dla kilku charakterystycznych $k$.
  \item Przekątna $B D$ równoległoboku $A B C D$ jest prostopadła do boku $A D$, a kąt ostry tego równoległoboku jest równy kątowi między jego przekątnymi. Wyznaczyć stosunek długości przekątnych. Sporządzić rysunek.
  \item Wyznaczyć zbiór punktów, z których odcinek o końcach $A(2,0)$ i $B(1, \sqrt{2})$ jest widoczny pod kątem $30^{\circ}$. Sporządzić rysunek.
  \item Podstawą graniastosłupa prostego o wszystkich krawędziach równych $a$, jest romb o kącie ostrym $\alpha$. Graniastosłup przecięto płaszczyzną przechodzącą przechodzącą przez dłuższą przekątną $A C$ podstawy dolnej i przeciwległy wierzchołek podstawy górnej. Wyznaczyć cosinus kąta nachylenia tej płaszczyzny do płaszczyzny podstawy i pole otrzymanego przekroju. Sporządzić rysunek.
\end{enumerate}

\section*{PRACA KONTROLNA nr 3 - POZIOM RoZsZERzony}
\begin{enumerate}
  \item Rozwiązać równanie $\left(\frac{1}{x}\right)^{2-3 \log _{2} x}=\frac{1}{2} x^{1+\log _{x} 2}$.
  \item Dla jakich wartości parametru $m$ równanie $x^{3}+(m-2) x^{2}+\left(2-m-m^{2}\right) x-\left(1-m^{2}\right)=0$ ma trzy różne pierwiastki, których suma kwadratów nie przekracza 5?
  \item Czworokąt wypukły $A B C D$, w którym $A B=1, B C=2, C D=4, D A=3$ jest wpisany w okrąg. Obliczyć promień $R$ tego okręgu. Sprawdzić, czy w czworokąt ten można wpisać okrąg. Jeżeli tak, to obliczyć promień $r$ tego okręgu. Sporządzić rysunek.
  \item Podstawą graniastosłupa prostego o wszystkich krawędziach równych jest romb o kącie ostrym $\frac{\pi}{3}$. Graniastosłup ten przecięto dwiema płaszczyznami: płaszczyzną przechodzącą przez bok $A B$ podstawy dolnej i wierzchołek $C^{\prime}$ oraz płaszczyzną przechodzącą przez bok $A D$ podstawy dolnej i ten sam wierzchołek $C^{\prime}$. Wyznaczyć kąt dwuścienny między tymi płaszczyznami oraz stosunek objętości brył, na jakie został podzielony graniastosłup. Sporządzić rysunek.
  \item W zależności od parametru rzeczywistego $p$ przedyskutować liczbę rozwiązań układu równań
\end{enumerate}

$$
\begin{cases}x^{4}+y^{4}+2 x^{2} y^{2}-4 x^{2} & =0 \\ x^{2}+y^{2}-2 \sqrt{3} y & =p\end{cases}
$$

Sporządzić ilustrację graficzną układu dla kilku charakterystycznych $p$.\\
6. Wykorzystując wzór Newtona i obliczając pochodną wielomianu $w(x)=(1-x)^{n}$, wykazać, że dla dowolnego $n \in \mathbb{N}, n \geqslant 2$ zachodzi równość

$$
\binom{n}{1}-2\binom{n}{2}+3\binom{n}{3}-4\binom{n}{4}+\ldots+(-1)^{n-1} n\binom{n}{n}=0
$$

Wywnioskować stąd, że jeżeli liczby $a_{1}, a_{2}, \ldots, a_{n}, a_{n+1}$ tworzą ciąg arytmetyczny, to dla dowolnego $n \in \mathbb{N}$ zachodzi równość

$$
a_{1}-\binom{n}{1} a_{2}+\binom{n}{2} a_{3}-\binom{n}{3} a_{4}+\ldots+(-1)^{n}\binom{n}{n} a_{n+1}=0
$$

Rozwiązania (rękopis) zadań z wybranego poziomu prosimy nadsyłać do 18 listopada 2019r. na adres:

Wydział Matematyki\\
Politechnika Wrocławska\\
Wybrzeże Wyspiańskiego 27\\
50-370 WROCEAW.\\
Na kopercie prosimy koniecznie zaznaczyć wybrany poziom! (np. poziom podstawowy lub rozszerzony). Do rozwiązań należy dołączyć zaadresowaną do siebie kopertę zwrotną z naklejonym znaczkiem, odpowiednim do formatu listu. Polecamy stosowanie kopert formatu C5 $(160 \times 230 \mathrm{~mm})$ ze znaczkiem o wartości $3,30 \mathrm{zl}$. Na każdą większą kopertę należy nakleić droższy znaczek. Prace niespełniające podanych warunków nie będą poprawiane ani odsyłane.

Uwaga. Wysyłając nam rozwiązania zadań uczestnik Kursu udostępnia Politechnice Wrocławskiej swoje dane osobowe, które przetwarzamy wyłącznie w zakresie niezbędnym do jego prowadzenia (odesłanie zadań, prowadzenie statystyki). Szczegółowe informacje o przetwarzaniu przez nas danych osobowych są dostępne na stronie internetowej Kursu.\\
Adres internetowy Kursu: \href{http://www.im.pwr.edu.pl/kurs}{http://www.im.pwr.edu.pl/kurs}


\end{document}