\documentclass[10pt]{article}
\usepackage[polish]{babel}
\usepackage[utf8]{inputenc}
\usepackage[T1]{fontenc}
\usepackage{amsmath}
\usepackage{amsfonts}
\usepackage{amssymb}
\usepackage[version=4]{mhchem}
\usepackage{stmaryrd}

\title{LIGA MATEMATYCZNA \\
 PAŹDZIERNIK 2010 \\
 GIMNAZJUM }

\author{}
\date{}


\begin{document}
\maketitle
\section*{ZADANIE 1.}
Suma i iloczyn pewnych dziesięciu liczb całkowitych są parzyste. Ile najwięcej może być wśród nich liczb nieparzystych?

\section*{ZADANIE 2.}
W turnieju tenisa stołowego wzięło udział pięćdziesięciu zawodników. Każdy zawodnik rozegrał jeden mecz z każdym innym zawodnikiem. Nie było remisów. Czy możliwe jest, aby każdy z zawodników wygrał taką samą liczbę meczów?

\section*{ZADANIE 3.}
Wykaż, że wartość wyrażenia \(\frac{1}{\sqrt{2}+\sqrt{3}}+\frac{1}{\sqrt{3}+\sqrt{4}}+\frac{1}{\sqrt{4}+\sqrt{5}}+\ldots+\frac{1}{\sqrt{127}+\sqrt{128}}\) jest mniejsza od 10.

\section*{ZADANIE 4.}
Dany jest czworokąt wypukły \(A B C D\). Niech \(E\) będzie środkiem odcinka \(A B\) oraz \(F\) - środkiem odcinka \(C D\). Oblicz pole czworokąta \(E B F D\) wiedząc, że pole czworokąta \(A B C D\) jest równe 77 .

\section*{ZADANIE 5.}
Jan, Henryk, Stanisław i Paweł to znajomi rzemieślnicy. Każdy z nich wykonuje inny zawód i mieszka przy innej ulicy. Na podstawie podanych informacji określ, przy jakiej ulicy każdy z nich mieszka i jaki wykonuje zawód.

\begin{itemize}
  \item Jan nie jest jubilerem.
  \item Zegarmistrz nie mieszka przy ulicy Złotej.
  \item Henryk jest krawcem, ale nie mieszka przy ulicy Głównej.
  \item Jubiler mieszka przy ulicy Mokrej.
  \item Stanisław mieszka przy ulicy Cichej, ale nie jest szewcem.
\end{itemize}

\end{document}