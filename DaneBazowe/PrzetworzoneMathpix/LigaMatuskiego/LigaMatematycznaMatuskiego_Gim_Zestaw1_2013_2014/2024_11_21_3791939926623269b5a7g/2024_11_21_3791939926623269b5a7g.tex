\documentclass[10pt]{article}
\usepackage[polish]{babel}
\usepackage[utf8]{inputenc}
\usepackage[T1]{fontenc}
\usepackage{amsmath}
\usepackage{amsfonts}
\usepackage{amssymb}
\usepackage[version=4]{mhchem}
\usepackage{stmaryrd}

\title{LIGA MATEMATYCZNA \\
 im. Zdzisława Matuskiego \\
 PAŹDZIERNIK 2013 \\
 GIMNAZJUM }

\author{}
\date{}


\begin{document}
\maketitle
\section*{ZADANIE 1.}
Dany jest ułamek \(\frac{a}{b}\). Do licznika tego ułamka dodano liczbę 1. Jaką liczbę należy dodać do mianownika, aby otrzymać ułamek równy danemu?

\section*{ZADANIE 2.}
Na przyjęcie przybyła pewna liczba gości. Każdy z każdym wymienił uścisk dłoni, z wyjątkiem pana Jana, który dwunastu gościom nie chciał podać ręki. W sumie wymieniono 2004 uściski dłoni. Ile osób było na przyjęciu?

\section*{ZADANIE 3.}
W finale Ligi Matematycznej uczestniczyło stu uczniów. Uzasadnij, że wśród nich było piętnastu (lub więcej) uczniów, którzy urodzili się w tym samym dniu tygodnia.

\section*{ZADANIE 4.}
Długości boków kwadratów \(A B C D\) i \(K L M N\) są równe 4 cm . Kwadraty te są tak położone, że wierzchołek \(K\) należy do boku \(A D\), wierzchołek \(L\) - do boku \(A B\), a przekątne kwadratu \(K L M N\) są prostopadłe do odpowiednich boków kwadratu \(A B C D\). Oblicz pole figury będącej częścią wspólną obu kwadratów. Oblicz odległość wierzchołka \(C\) od prostej MN.

\section*{ZADANIE 5.}
Na Międzynarodową Olimpiadę Matematyczną przyjechało 1000 osób. W sprawozdaniu podano, że wśród nich 811 włada językiem angielskim, 752 - językiem rosyjskim, 418 - językiem francuskim, 356 - językiem rosyjskim i francuskim, 570 - jezykkiem rosyjskim i angielskim, 348 - językiem angielskim i francuskim, 297 osób mówi wszystkimi trzema językami. Wykaż, że w sprawozdaniu popełniono błąd.


\end{document}