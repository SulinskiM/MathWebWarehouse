\documentclass[10pt]{article}
\usepackage[polish]{babel}
\usepackage[utf8]{inputenc}
\usepackage[T1]{fontenc}
\usepackage{amsmath}
\usepackage{amsfonts}
\usepackage{amssymb}
\usepackage[version=4]{mhchem}
\usepackage{stmaryrd}

\title{LIGA MATEMATYCZNA im. Zdzisława Matuskiego \\
 PAŹDZIERNIK 2017 \\
 GIMNAZJUM }

\author{}
\date{}


\begin{document}
\maketitle
\section*{ZADANIE 1.}
Wiadomo, że

\[
x-y+2017, y-z+2017, z-t+2017, t-w+2017, w-x+2017
\]

są kolejnymi liczbami całkowitymi. Znajdź je.

\section*{ZADANIE 2.}
Trzy pary małżeńskie: Ania i Adam, Beata i Bartek, Celina i Czarek mają w sumie 137 lat. Każdy z panów jest o 5 lat starszy od swojej żony. Suma liczby lat Bartka i Beaty wynosi 47 lat. Ania jest najstarsza wśród pań i ma o 4 lata więcej niż najmłodsza z kobiet. Ile lat ma każda osoba?

\section*{ZADANIE 3.}
W okrąg o promieniu \(r\) wpisano kwadrat i na tym okręgu opisano trójkąt równoboczny. Suma długości boku kwadratu i boku trójkąta równobocznego jest równa 10. Wyznacz promień okręgu.

\section*{ZADANIE 4.}
Znajdź dwie liczby naturalne, których suma jest równa 432 i których największy wspólny dzielnik to 36.

\section*{ZADANIE 5.}
W liczbie trzycyfrowej \(x\) skreślono cyfrę setek i otrzymano dwucyfrową liczbę \(k\). Gdy w liczbie \(x\) skreślono cyfrę dziesiątek, to otrzymano liczbę dwucyfrową \(l\), a po skreśleniu w liczbie \(x\) cyfry jedności powstała dwucyfrowa liczba \(m\). Okazało się, że suma \(k+l+m\) jest trzykrotnie mniejsza od liczby \(x\). Znajdź \(x\).


\end{document}