\documentclass[10pt]{article}
\usepackage[polish]{babel}
\usepackage[utf8]{inputenc}
\usepackage[T1]{fontenc}
\usepackage{amsmath}
\usepackage{amsfonts}
\usepackage{amssymb}
\usepackage[version=4]{mhchem}
\usepackage{stmaryrd}

\title{LIGA MATEMATYCZNA im. Zdzisława Matuskiego LISTOPAD 2016 GIMNAZJUM }

\author{}
\date{}


\begin{document}
\maketitle
\section*{ZADANIE 1.}
Na przyprostokątnych \(B C\) i \(C A\) trójkąta prostokątnego \(A B C\) zbudowano na zewnątrz kwadraty \(E C B D\) oraz \(C F G A\). Prosta \(A D\) przecina bok \(B C\) w punkcie \(P\), prosta \(B G\) przecina bok \(C A\) w punkcie \(R\). Udowodnij, że odcinki \(C P\) i \(C R\) mają równe długości.

\section*{ZADANIE 2.}
Adam miał pomnożyć dwie liczby naturalne. Jeden z czynników był liczbą dwucyfrową, w której cyfra jedności była dwukrotnie mniejsza od cyfry dziesiątek. Chłopiec pomylił się, przestawił cyfry tej liczby i otrzymał iloczyn o 1539 mniejszy od poprawnego. Podaj poprawny wynik tego mnożenia i liczby, które miał pomnożyć Adam.

\section*{ZADANIE 3.}
Na stole leży 2017 monet. W jednym ruchu Bartek może wziąć dokładnie 3, 36 lub 69 monet. Czy wykonując wiele takich ruchów Bartek może wziąćc wszystkie monety ze stołu?

\section*{ZADANIE 4.}
Wykaż, że liczba \(256^{4}+8^{9}\) jest podzielna przez 11 .

\section*{ZADANIE 5.}
Różnica między czwartymi potęgami pewnych dwóch liczb naturalnych jest równa 34481, a różnica między drugimi potęgami tych liczb wynosi 41 . Wyznacz różnicę tych liczb.


\end{document}