\documentclass[10pt]{article}
\usepackage[polish]{babel}
\usepackage[utf8]{inputenc}
\usepackage[T1]{fontenc}
\usepackage{amsmath}
\usepackage{amsfonts}
\usepackage{amssymb}
\usepackage[version=4]{mhchem}
\usepackage{stmaryrd}

\title{LIGA MATEMATYCZNA \\
 Gimnazjum \\
 Pólfinał \\
 20 lutego 2009 }

\author{}
\date{}


\begin{document}
\maketitle
\section*{ZADANIE 1.}
Pan Jan produkuje reklamowe chusty w kształcie trójkąta prostokątnego równoramiennego. Ponieważ klienci skarżyli się, że są za małe, więc postanowił powiększyć je wydłużając oba krótsze boki po 10 cm . Skutkiem tego powierzchnia chusty wzrosła o \(550 \mathrm{~cm}^{2}\). Ile jest teraz równa powierzchnia chusty?

\section*{ZADANIE 2.}
Mamy 5 kawałków papieru. Niektóre z nich rozcinamy na 5 kawałków. Następnie niektóre kawałki znów dzielimy na 5 kawałków, itd. Czy w ten sposób można otrzymać 1000 kawałków papieru?

\section*{ZADANIE 3.}
Paweł ma 10 kieszeni i 54 monety jednozłotowe. Chce umieścić swoje pieniądze w kieszeniach w taki sposób, aby w każdej kieszeni była inna ilość monet. Czy jest to możliwe?

\section*{ZADANIE 4.}
Czterech przyjaciół wędkarzy, wśród nich Adam i Piotr, wybrało się na ryby. Po zakończonym wędkowaniu okazało się, że trzej z nich - bez Adama - złowili średnio po 14 ryb, a trzej - bez Piotra - średnio po 10 ryb. Kto złowił więcej ryb: Adam czy Piotr i o ile?

\section*{ZADANIE 5.}
Fabryka produkująca cukierki pakuje je do sześciennych pudełek o krawędzi długości 10 cm . Pudełka te mają być pakowane po 12 sztuk w prostopadłościenne paczki. Jak należy ułożyć pudełka, aby pole powierzchni paczki było najmniejsze?


\end{document}