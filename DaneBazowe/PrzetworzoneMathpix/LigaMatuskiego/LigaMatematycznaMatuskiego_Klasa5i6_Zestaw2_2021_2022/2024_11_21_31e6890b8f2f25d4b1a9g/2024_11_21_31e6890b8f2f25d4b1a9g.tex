\documentclass[10pt]{article}
\usepackage[polish]{babel}
\usepackage[utf8]{inputenc}
\usepackage[T1]{fontenc}
\usepackage{amsmath}
\usepackage{amsfonts}
\usepackage{amssymb}
\usepackage[version=4]{mhchem}
\usepackage{stmaryrd}

\title{LIGA MATEMATYCZNA \\
 im. Zdzisława Matuskiego \\
 LISTOPAD 2021 \\
 SZKOŁA PODSTAWOWA \\
 klasy IV - VI }

\author{}
\date{}


\begin{document}
\maketitle
\section*{ZADANIE 1.}
Liczba trzycyfrowa \(n\) ma następujące własności:

\begin{itemize}
  \item suma cyfr jest równa 16 ;
  \item iloczyn cyfr jest różny od zera, ale cyfrą jedności tego iloczynu jest zero;
  \item suma cyfr iloczynu cyfr liczby \(n\) jest równa 3 .
\end{itemize}

Znajdź największą liczbę \(n\) o tych własnościach.

\section*{ZADANIE 2.}
Uzupełnij brakujące mianowniki

\[
\frac{1}{2}+\frac{1}{3}+\frac{1}{\square}+\frac{1}{\square}+\frac{1}{72}+\frac{1}{108}+\frac{1}{216}=1
\]

Wskaż wszystkie rozwiązania.

\section*{ZADANIE 3.}
Boki czworokąta mają długość 8, 6, 5 i 7 (kolejność zapisu tych liczb nie musi być zgodna z długościami kolejnych boków). Przekątna o długości 12 dzieli ten czworokąt na dwa trójkąty. Podaj obwód każdego z nich.

\section*{ZADANIE 4.}
W pewnej kamienicy jest 9 mieszkań. Każde z nich ma dwa lub trzy pokoje. Ile jest mieszkań trzypokojowych, jeżeli łącznie wszystkie mieszkania mają 24 pokoje?

\section*{ZADANIE 5.}
W ośmiu jednakowych skrzynkach znajduje się 140 butelek soku, przy czym w jednej ze skrzynek brakuje kilku butelek, zaś pozostałe skrzynki są pełne. Ile butelek mieści się w dwunastu pełnych skrzynkach?


\end{document}