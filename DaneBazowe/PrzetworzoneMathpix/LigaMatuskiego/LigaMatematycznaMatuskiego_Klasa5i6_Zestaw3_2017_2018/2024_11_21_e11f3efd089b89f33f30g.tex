\documentclass[10pt]{article}
\usepackage[polish]{babel}
\usepackage[utf8]{inputenc}
\usepackage[T1]{fontenc}
\usepackage{amsmath}
\usepackage{amsfonts}
\usepackage{amssymb}
\usepackage[version=4]{mhchem}
\usepackage{stmaryrd}

\title{LIGA MATEMATYCZNA im. Zdzisława Matuskiego GRUDZIEŃ 2017 SZKOŁA PODSTAWOWA }

\author{}
\date{}


\begin{document}
\maketitle
(klasy IV - VI)

\section*{ZADANIE 1.}
Czy wśród liczb

\begin{itemize}
  \item 66, 666, 6666, 66666, ...
  \item 55, 555, 5555, 55555, ...\\
znajduje się kwadrat liczby naturalnej?
\end{itemize}

\section*{ZADANIE 2.}
Drwal Mikołaj ciął drewno na opał do kominka. Wykonując cięcie, rozcinał zawsze jeden kawałek drewna na dwie części. Po wykonaniu 53 cięć Mikołaj miał 72 kawałki drewna. Ile kawałków drewna było na początku?

\section*{ZADANIE 3.}
Na Wigilii spotkało się dwóch kuzynów - matematyków: Adam i Bartek. Adam zapytał kuzyna, ile lat ma trójka jego dzieci. Ten odparł, że iloczyn ich wieku jest równy 72. Dla Adama ta informacja nie była wystarczająca. Wtedy Bartek dodał, że suma ich wieku to 14. Jednak i ta wskazówka nie pozwoliła Adamowi ustalić wieku dzieci. Dopiero, gdy Bartek powiedział, że najmłodsze dziecko ma na imię Ewa, Adam poprawnie ustalił wiek dzieci kuzyna. Ile lat mają dzieci Bartka?

\section*{ZADANIE 4.}
Obwód prostokąta jest równy 67. Dwusieczna jednego z kątów dzieli obwód na dwie części różniące się o 20. Oblicz długości boków prostokąta.

\section*{ZADANIE 5.}
Dany jest trójkąt prostokątny \(A B C\) o kącie prostym przy wierzchołku \(C\). Dwusieczne poprowadzone z wierzchołków \(A\) i \(B\) przecinają się w punkcie \(D\). Znajdź miarę kąta \(A D B\).

Dwusieczna kąta jest to pólprosta o początku w wierzchołku kata dzielqca ten kąt na dwa katy przystajace.


\end{document}