\documentclass[10pt]{article}
\usepackage[polish]{babel}
\usepackage[utf8]{inputenc}
\usepackage[T1]{fontenc}
\usepackage{amsmath}
\usepackage{amsfonts}
\usepackage{amssymb}
\usepackage[version=4]{mhchem}
\usepackage{stmaryrd}

\title{LIGA MATEMATYCZNA \\
 im. Zdzisława Matuskiego GRUDZIEŃ 2021 \\
 SZKOŁA PODSTAWOWA klasy VII - VIII }

\author{}
\date{}


\begin{document}
\maketitle
\section*{ZADANIE 1.}
Suma dwóch liczb jest równa 57460. Jeśli do mniejszej liczby dopiszemy z prawej strony 92, to otrzymamy równe liczby. Znajdź je.

\section*{ZADANIE 2.}
Bartek napisał 15 liczb naturalnych i obliczył ich sumę otrzymując wynik 2022. Basia dopisała znak „minus" przed kilkoma z tych liczb i obliczyła sumę wszystkich swoich liczb. Czy mogła uzyskać wynik 1111?

\section*{ZADANIE 3.}
Pewna liczba ma cztery dzielniki, z których dwa są liczbami pierwszymi. Mikołaj wypisał je w kolejności od najmniejszego do największego. Drugi dzielnik jest o 10 większy od pierwszego, a czwarty o 130 większy od trzeciego. Która liczba ma takie dzielniki?

\section*{ZADANIE 4.}
Ania dzieliła kolejne liczby parzyste (zaczynając od 0 ) przez pewną liczbę naturalną i wypisywała reszty z ich dzielenia. Początek zapisu był następujący: \(0,2,4,6,1,3,5,0,2,4, \ldots\). Bartek też wypisywał reszty z dzielenia przez tę samą liczbę co Ania. Dzielił jednak kolejne liczby nieparzyste zaczynając od 1. Zakończył prace, gdy po raz trzeci uzyskał 0 . Ile liczb wypisał Bartek?

\section*{ZADANIE 5.}
Rozetnij kwadrat na siedem kwadratów. Znajdź stosunek obwodów największego i najmniejszego z otrzymanych kwadratów.


\end{document}