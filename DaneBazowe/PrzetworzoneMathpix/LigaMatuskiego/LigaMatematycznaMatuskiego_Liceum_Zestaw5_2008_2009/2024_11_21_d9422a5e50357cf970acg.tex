\documentclass[10pt]{article}
\usepackage[polish]{babel}
\usepackage[utf8]{inputenc}
\usepackage[T1]{fontenc}
\usepackage{amsmath}
\usepackage{amsfonts}
\usepackage{amssymb}
\usepackage[version=4]{mhchem}
\usepackage{stmaryrd}
\usepackage{bbold}

\title{LIGA MATEMATYCZNA \\
 FINAE \\
 25 kwietnia 2009 \\
 SZKOŁA PONADGIMNAZJALNA }

\author{}
\date{}


\begin{document}
\maketitle
\section*{ZADANIE 1.}
Znajdź wszystkie funkcje \(f: \mathbb{R} \rightarrow \mathbb{R}\) spełniające warunek \(2 f(x)+f(1-x)=x\) dla wszystkich liczb rzeczywistych \(x\).

\section*{ZADANIE 2.}
Wypisujemy kolejne liczby naturalne od 1 do 2009. Każdą z tych liczb zastępujemy sumą jej cyfr i powtarzamy to aż do momentu uzyskania liczb jednocyfrowych. Jakich liczb w tym ciągu jest więcej: jedynek czy ósemek?

\section*{ZADANIE 3.}
Wyznacz wszystkie wartości naturalne \(n\), dla których \(3^{n}-1\) jest liczbą podzielną przez 13. Wykaż, że dla żadnej wartości naturalnej \(n\) liczba \(3^{n}+1\) nie jest podzielna przez 13.

\section*{ZADANIE 4.}
Liczby \(n+2\) oraz \(n-10\) są kwadratami liczb naturalnych. Znajdź \(n\).

\section*{ZADANIE 5.}
W czworokącie wypukłym \(A B C D\) trójkąty \(A B C, B C D, C D A, D A B\) mają równe obwody. Udowodnij, że ten czworokąt jest prostokatem.

\section*{ZADANIE 6.}
Wykaż, że wśród 40 liczb naturalnych można wybrać 4, z których każde dwie dają różnicę podzielną przez 13.

\section*{ZADANIE 7.}
Trapez prostokątny opisano na okręgu. Oblicz długości boków nierównoległych, jeżeli podstawy są równe \(a\) i \(b\).


\end{document}