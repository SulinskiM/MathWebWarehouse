\documentclass[10pt]{article}
\usepackage[polish]{babel}
\usepackage[utf8]{inputenc}
\usepackage[T1]{fontenc}
\usepackage{amsmath}
\usepackage{amsfonts}
\usepackage{amssymb}
\usepackage[version=4]{mhchem}
\usepackage{stmaryrd}

\title{LIGA MATEMATYCZNA \\
 PAŹDZIERNIK 2010 \\
 SZKOŁA PODSTAWOWA }

\author{}
\date{}


\begin{document}
\maketitle
\section*{ZADANIE 1.}
Ile jest liczb naturalnych o sumie cyfr w zapisie dziesiętnym równej 100 i iloczynie tych cyfr równym 5?

\section*{ZADANIE 2.}
Dane są trzy figury: koło, kwadrat i trójkąt, różnej wielkości i w różnych kolorach: czerwonym, zielonym, niebieskim. Koło nie jest małe ani czerwone. Trójkąt nie jest średni ani zielony. Kwadrat nie jest duży ani niebieski. Określ wielkość i kolor każdej figury, jeśli wiadomo, że mała figura jest niebieska.

\section*{ZADANIE 3.}
W kole stanęło 15 dziewcząt i 15 chłopców. Zaczynając od ustalonego miejsca nastąpi odliczanie do dziewięciu zgodnie z ruchem wskazówek zegara. Dziewiąta osoba odpada z gry, a odliczanie będzie odbywać się dalej. Następnie kolejna dziewiąta osoba zostanie wykluczona z koła, i tak dalej aż do momentu, gdy w kole zostanie 15 osób. Jak należy ustawić chłopców i dziewczęta, aby wszystkie dziewczęta zostały w kole?

\section*{ZADANIE 4.}
Z siedmiu patyczków o długościach 3, 4, 6, 7, 9, 10, 11 ułóż prostokąt. Patyczków nie wolno łamać ani nakładać na siebie.

\section*{ZADANIE 5.}
Z miejscowości A, w której mieści się firma kurierska, wyjeżdża kurier do miejscowości \(\mathrm{B}, \mathrm{C}\), D, aby dostarczyć przesyłki. W jakiej kolejności powinien objechać te miejscowości, aby trasa objazdu z A przez pozostałe miejscowości i z powrotem do A była możliwie najkrótsza, jeśli długość drogi od A do B wynosi 50 km , od A do \(\mathrm{C}-70 \mathrm{~km}\), od A do \(\mathrm{D}-70 \mathrm{~km}\), od B do D 80 km , od B do C - 100 km , od C do D - 60 km ?


\end{document}