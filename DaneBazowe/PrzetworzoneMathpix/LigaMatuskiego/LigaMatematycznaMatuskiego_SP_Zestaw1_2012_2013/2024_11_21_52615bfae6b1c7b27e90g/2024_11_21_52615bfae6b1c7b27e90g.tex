\documentclass[10pt]{article}
\usepackage[polish]{babel}
\usepackage[utf8]{inputenc}
\usepackage[T1]{fontenc}
\usepackage{amsmath}
\usepackage{amsfonts}
\usepackage{amssymb}
\usepackage[version=4]{mhchem}
\usepackage{stmaryrd}

\title{LIGA MATEMATYCZNA im. Zdzisława Matuskiego \\
 PAŹDZIERNIK 2012 SZKOŁA PODSTAWOWA }

\author{}
\date{}


\begin{document}
\maketitle
\section*{ZADANIE 1.}
Słoń waży tyle, ile dwa nosorożce, nosorożec tyle, ile dwa niedźwiedzie, niedźwiedź tyle, ile dwa sumy, sum waży tyle, ile dwa tygrysy, tygrys tyle, ile dwa strusie, struś tyle, ile dwie sarny, sarna tyle, ile dwa borsuki, borsuk tyle, ile dwa lisy, lis tyle, ile dwa zające. Słoń waży o \(6,25 \mathrm{~kg}\) więcej niż w sumie nosorożec, niedźwiedź, sum, tygrys, struś, sarna, borsuk, lis i zając. Ile waży słoń?

\section*{ZADANIE 2.}
Kwadrat o boku długości 9 cm rozetnij na trzy prostokąty o obwodach \(20 \mathrm{~cm}, 24 \mathrm{~cm}\) i 28 cm .

\section*{ZADANIE 3.}
Do zapisania pewnej liczby dziesięciocyfrowej użyto jednej jedynki, dwóch dwójek, trzech trójek i czterech czwórek. Rozmieszczenie cyfr jest nieznane. Czy może to być liczba pierwsza?

\section*{ZADANIE 4.}
Pierwszy ślimak potrafi przejść 3 metry w ciągu czterech minut, a drugi - 4 metry w trzy minuty. W tym samym momencie wyszli z tego samego miejsca odległego o 11 metrów od stacji leśnej kolejki. Czy obaj zdążą, jeśli do odjazdu pociągu zostało 13 minut?

\section*{ZADANIE 5.}
Na odcinku \(A B\) zaznaczono punkty \(C, D\). Odległość punktu \(C\) od jednego z końców danego odcinka stanowi \(\frac{5}{6}\) jego długości, a odległość punktu \(D\) od jednego z końców - \(\frac{3}{4}\) długości tego odcinka. Wiedząc, że długość odcinka \(C D\) jest równa 35 cm , oblicz długość odcinka \(A B\). Rozważ wszystkie możliwości.


\end{document}