\documentclass[10pt]{article}
\usepackage[polish]{babel}
\usepackage[utf8]{inputenc}
\usepackage[T1]{fontenc}
\usepackage{amsmath}
\usepackage{amsfonts}
\usepackage{amssymb}
\usepackage[version=4]{mhchem}
\usepackage{stmaryrd}

\title{LIGA MATEMATYCZNA \\
 GRUDZIEŃ 2010 \\
 SZKOŁA PODSTAWOWA }

\author{}
\date{}


\begin{document}
\maketitle
\section*{ZADANIE 1.}
Gwarno będzie w święta u dziadków, gdy zjadą się dzieci z własnym potomstwem. Które dzieci są czyje, jeśli:

\begin{itemize}
  \item Barbara ma więcej dzieci niż brat;
  \item Piotrek i Oleńka mówią do Jerzego - wujku, a Magda do Haliny - ciociu;
  \item Misia nie jest siostrą ani Grzesia, ani Arka, który ma troje rodzeństwa: brata Maćka i dwie siostry;
  \item Halina ma córkę i syna.
\end{itemize}

\section*{ZADANIE 2.}
Zepsuty kalkulator nie wyświetla cyfry 5. Na przykład, jeśli napiszemy liczbę 3535, to pokazuje on liczbę 33 bez żadnych odstępów między cyframi. Michał napisał na tym kalkulatorze pewną liczbę sześciocyfrową i na wyświetlaczu kalkulatora pojawiła się liczba 2010. Dla ilu liczb mogło się tak zdarzyć?

\section*{ZADANIE 3.}
Łączna pojemność butelki i szklanki jest równa pojemności dzbanka. Pojemność butelki jest równa łącznej pojemności szklanki i kufla. Łączna pojemność trzech kufli jest równa łącznej pojemności dwóch dzbanków. Ile szklanek ma łączną pojemność jednego kufla?

\section*{ZADANIE 4.}
Osiem zer i osiem jedynek ustaw w tablicy \(4 \times 4\) tak, aby sumy liczb w każdym wierszu i w każdej kolumnie były nieparzyste.

\section*{ZADANIE 5.}
W pomieszczeniu, które ma kształt prostopadłościanu, osiem pająków mieszka w ośmiu narożach. Jeden z nich postanowił odwiedzić wszystkich swoich kolegów, a następnie wrócić do swojego narożnika wybierając drogę wzdłuż krawędzi. Odległości do najbliższych sąsiadów są równe odpowiednio \(4 \mathrm{~m}, 6 \mathrm{~m}, 8 \mathrm{~m}\). W jakiej kolejności pająk powinien odwiedzić swoich sąsiadów, aby jego spacer był najkrótszy? Jak długi będzie ten spacer?


\end{document}