\documentclass[10pt]{article}
\usepackage[polish]{babel}
\usepackage[utf8]{inputenc}
\usepackage[T1]{fontenc}
\usepackage{amsmath}
\usepackage{amsfonts}
\usepackage{amssymb}
\usepackage[version=4]{mhchem}
\usepackage{stmaryrd}

\title{LIGA MATEMATYCZNA \\
 Szkoła Podstawowa \\
 Pólfinał \\
 20 lutego 2009 }

\author{}
\date{}


\begin{document}
\maketitle
\section*{ZADANIE 1.}
Kasię, Ewę i Anię poczęstowano trzema czekoladkami z orzechami, z rodzynkami i truskawkami. Kasia nie lubi orzechów, Ewa - rodzynek, Ania jest uczulona na truskawki. Na ile sposobów można podzielić te czekoladki między dziewczynki tak, aby były zadowolone?

\section*{ZADANIE 2.}
W trójkącie równoramiennym \(A B C\), w którym \(|A C|=|B C|\), poprowadzono wysokość \(C D\). Oblicz długość tej wysokości, jeżeli obwód trójkąta \(A B C\) jest równy 32 cm , a obwód trójkąta \(A D C\) jest o 6 cm krótszy od obwodu trójkąta \(A B C\).

\section*{ZADANIE 3.}
Ania i Basia ważą łącznie 44 kg , Basia i Celina - 47 kg , Celina i Dorota - 46 kg , Dorota i Ewa - 49 kg , Ewa i Ania - 48 kg . Ile waży Ania?

\section*{ZADANIE 4.}
Przy użyciu cyfr: \(1,2,3,4,5,6\), Tomek napisał dwie liczby całkowite dodatnie takie, że każda z cyfr występowała tylko w jednej z dwóch liczb, i to dokładnie raz. Gdy liczby te dodał, otrzymał 750. Jakie liczby napisał Tomek? Podaj wszystkie pary tych liczb.

\section*{ZADANIE 5.}
Ala pomaga cioci w prowadzeniu sklepu cukierniczego. Po zamknięciu sklepu dziewczynka policzyła, ile tabliczek czekolady pozostało na półkach, ale przez roztargnienie wpisała do zeszytu otrzymaną liczbę bez ostatniej cyfry. Na drugi dzień stwierdziła ze zdziwieniem, że liczba tabliczek czekolady na półkach jest o 89 większa od liczby wpisanej do zeszytu. Jaką liczbę powinna była wpisać Ala?


\end{document}