\documentclass[10pt]{article}
\usepackage[polish]{babel}
\usepackage[utf8]{inputenc}
\usepackage[T1]{fontenc}
\usepackage{amsmath}
\usepackage{amsfonts}
\usepackage{amssymb}
\usepackage[version=4]{mhchem}
\usepackage{stmaryrd}

\title{LIGA MATEMATYCZNA FINAE }

\author{}
\date{}


\begin{document}
\maketitle
\section*{25 kwietnia 2009 \\
 SZKOŁA PODSTAWOWA}
\section*{ZADANIE 1.}
Dziesięć śliwek waży tyle, co trzy jabłka i gruszka. Jabłko i gruszka ważą tyle, co sześć śliwek. Waga ilu śliwek jest równa wadze jednej gruszki?

\section*{ZADANIE 2.}
Na prostej obrano kolejno pięć punktów \(A, B, C, D, E\). Wiadomo, że \(A B=19 \mathrm{~cm}, C E=97 \mathrm{~cm}\), \(A C=B D\). Znajdź długość odcinka \(D E\).

\section*{ZADANIE 3.}
W prostokącie jeden z boków stanowi \(\frac{2}{3}\) drugiego. Z wierzchołka prostokąta do środka dłuższego boku poprowadzono odcinek. Dzieli on prostokąt na dwie figury: trójkąt o obwodzie równym 12 cm i trapez o obwodzie 18 cm . Oblicz obwód prostokąta.

\section*{ZADANIE 4.}
Adam, Bartek i Witek uczą się w tej samej klasie. Jeden z nich dojeżdża do szkoły autobusem, drugi tramwajem, a trzeci rowerem. Pewnego dnia Adam odprowadzał kolegę na przystanek autobusowy. W tym samym czasie obok nich przejechał rowerem trzeci kolega i zawołał: „Bartek, zostawiłeś zeszyt w szkole". Jakim środkiem lokomocji dojeżdża każdy z nich?

\section*{ZADANIE 5.}
Państwo Kowalscy i Wiśniewscy mają po dwóch synów, z których każdy ma mniej niż 9 lat. W każdej rodzinie jeden z synów ma więcej niż 5 lat, a drugi mniej. Andrzej jest o 3 lata młodszy od swojego brata. Wojtek jest najstarszy ze wszystkich chłopców. Krzyś jest o 2 lata młodszy od młodszego syna państwa Kowalskich, a Robert jest 5 lat starszy od młodszego syna państwa Wiśniewskich. Podaj imiona i nazwiska chłopców oraz ich wiek.


\end{document}