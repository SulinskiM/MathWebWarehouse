\documentclass[10pt]{article}
\usepackage[polish]{babel}
\usepackage[utf8]{inputenc}
\usepackage[T1]{fontenc}
\usepackage{amsmath}
\usepackage{amsfonts}
\usepackage{amssymb}
\usepackage[version=4]{mhchem}
\usepackage{stmaryrd}

\title{GIMNAZJUM }

\author{}
\date{}


\begin{document}
\maketitle
\begin{enumerate}
  \item Liczby \(x\) i \(y\) spełniają równanie \((x-y)^{2}+(x+y-4)^{2}=0\). Wyznacz wartość iloczynu \(x \cdot y\).
  \item Turysta przeszedł drogę z miasta A do miasta B i z powrotem w ciągu 3 godzin i 41 minut. Droga z A do B wiodła początkowo pod górę, potem po równym terenie, a następnie z góry. Prędkość turysty pod górę wynosi \(4 \mathrm{~km} / \mathrm{h}\), po równym terenie \(5 \mathrm{~km} / \mathrm{h}\), a z góry \(6 \mathrm{~km} / \mathrm{h}\). Odległość \(\mathrm{z} A\) do \(B\) wynosi 9 km . Na jakiej długości droga z miasta A do B wiedzie po równym terenie?
  \item Dany jest 18 -kąt foremny \(A_{1} A_{2} \ldots A_{18}\). Wykaż, że czworokąt ograniczony prostymi \(A_{2} A_{7}, A_{3} A_{15}, A_{6} A_{12,}, A_{10} A_{17}\) jest prostokątem. Czy ten prostokąt jest kwadratem?
\end{enumerate}

\section*{LICEUM}
\begin{enumerate}
  \item Wykaż, że jeśli \(a, b>0\) i \(a+b=1\), to
\end{enumerate}

\[
\left(1+\frac{1}{a}\right)\left(1+\frac{1}{b}\right) \geq 9
\]

\begin{enumerate}
  \setcounter{enumi}{1}
  \item W turnieju szachowym w grupie A było n zawodników, a w grupie B \(2 n\) zawodników. W każdej grupie każdy grał z każdym. W grupie B rozegrano \(k(k \in N)\) razy więcej meczów niż \(\mathbf{w}\) grupie A. Wyznacz wszystkie możliwe pary ( \(n, k\) ).
  \item Funkcja \(f\), określona w zbiorze liczb rzeczywistych i przyjmująca wartości rzeczywiste, spełnia dla każdego \(x>0\) warunek \(2 f(x)+3 f\left(\frac{2010}{x}\right)=5 x\). Oblicz \(f(6)\).
\end{enumerate}

Rozwiqzania należy oddać do piątku 8 maja do godziny 15.00 koordynatorowi konkursu panu Jarostawowi Szczepaniakowi lub swojemu nauczycielowi matematyki.

Na stronie internetowej szkoły w zakładce Konkursy i olimpiady można znaleźć wyniki dotychczasowych rund i rozwiązania zadań.


\end{document}