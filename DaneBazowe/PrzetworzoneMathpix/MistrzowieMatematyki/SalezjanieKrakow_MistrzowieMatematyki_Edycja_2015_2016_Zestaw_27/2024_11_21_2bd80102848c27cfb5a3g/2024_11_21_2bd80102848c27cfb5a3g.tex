\documentclass[10pt]{article}
\usepackage[polish]{babel}
\usepackage[utf8]{inputenc}
\usepackage[T1]{fontenc}
\usepackage{amsmath}
\usepackage{amsfonts}
\usepackage{amssymb}
\usepackage[version=4]{mhchem}
\usepackage{stmaryrd}
\usepackage{hyperref}
\hypersetup{colorlinks=true, linkcolor=blue, filecolor=magenta, urlcolor=cyan,}
\urlstyle{same}

\title{GIMNAZJUM }

\author{}
\date{}


\newcommand\Varangle{\mathop{{<\!\!\!\!\!\text{\small)}}\:}\nolimits}

%New command to display footnote whose markers will always be hidden
\let\svthefootnote\thefootnote
\newcommand\blfootnotetext[1]{%
  \let\thefootnote\relax\footnote{#1}%
  \addtocounter{footnote}{-1}%
  \let\thefootnote\svthefootnote%
}

%Overriding the \footnotetext command to hide the marker if its value is `0`
\let\svfootnotetext\footnotetext
\renewcommand\footnotetext[2][?]{%
  \if\relax#1\relax%
    \ifnum\value{footnote}=0\blfootnotetext{#2}\else\svfootnotetext{#2}\fi%
  \else%
    \if?#1\ifnum\value{footnote}=0\blfootnotetext{#2}\else\svfootnotetext{#2}\fi%
    \else\svfootnotetext[#1]{#2}\fi%
  \fi
}

\begin{document}
\maketitle
\begin{enumerate}
  \item Dane są takie liczby rzeczywiste \(a, b, c\), że liczby \(a b+b c, b c+c a, c a+a b\) są dodatnie. Udowodnij, że liczby \(a, b, c\) mają jednakowy znak, tzn. wszystkie są dodatnie lub wszystkie są ujemne.
  \item Dany jest trójkąt ostrokątny \(A B C\), przy czym \(\Varangle A C B=60^{\circ}\). Punkty \(D\) i \(E\) są rzutami prostokątnymi odpowiednio punktów \(A\) i \(B\) na proste \(B C\) i \(A C\). Punkt \(M\) jest środkiem boku \(A B\). Wykazać, ze trójkąt \(D E M\) jest równoboczny.
  \item Dany jest trójkąt \(A B C\), w którym \(A C>B C\). Punkt \(P\) jest rzutem prostokątnym punktu \(B\) na dwusieczną kąta \(A C B\). Punkt \(M\) jest środkiem odcinka \(A B\). Wiedząc, że \(B C=a, C A=b, A B=c\), oblicz długość odcinka \(P M\).
\end{enumerate}

\section*{LICEUM}
\begin{enumerate}
  \item Znajdź taką najmniejszą liczbę naturalną \(n\), aby liczby \(n+1\) oraz \(n-50\) były kwadratami liczb naturalnych.
  \item Wykaż, że dla \(a \in R \quad a^{8}+a^{2}+1>a^{5}+a\)
  \item Na tablicy napisano słowo \(a b d c\). W jednym ruchu możemy dopisać lub usunąć (na początku, w środku lub na końcu) palindrom parzystej długości utworzony z liter \(a, b, c, d\). Rozstrzygnąć, czy po skończonej liczbie ruchów możemy uzyskać słowo bacd. (Uwaga: Palindromem nazywamy słowo, które czytane od lewej do prawej jest takie samo jak czytane od prawej do lewej, np. abba, cc, daaaad.)
\end{enumerate}

\footnotetext{Rozwiazania należy oddać do piatku 22 kwietnia do godziny 10.35 koordynatorowi konkursu panu Jarostawowi Szczepaniakowi lub swojemu nauczycielowi matematyki lub przestać na adres \href{mailto:jareksz@interia.pl}{jareksz@interia.pl} do piątku 22 kwietnia do pótnocy.
}
\end{document}