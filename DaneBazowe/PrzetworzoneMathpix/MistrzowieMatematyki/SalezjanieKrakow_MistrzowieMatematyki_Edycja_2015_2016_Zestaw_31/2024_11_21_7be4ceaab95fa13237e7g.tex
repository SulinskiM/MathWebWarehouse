\documentclass[10pt]{article}
\usepackage[polish]{babel}
\usepackage[utf8]{inputenc}
\usepackage[T1]{fontenc}
\usepackage{amsmath}
\usepackage{amsfonts}
\usepackage{amssymb}
\usepackage[version=4]{mhchem}
\usepackage{stmaryrd}
\usepackage{hyperref}
\hypersetup{colorlinks=true, linkcolor=blue, filecolor=magenta, urlcolor=cyan,}
\urlstyle{same}

\title{GIMNAZJUM }

\author{}
\date{}


%New command to display footnote whose markers will always be hidden
\let\svthefootnote\thefootnote
\newcommand\blfootnotetext[1]{%
  \let\thefootnote\relax\footnote{#1}%
  \addtocounter{footnote}{-1}%
  \let\thefootnote\svthefootnote%
}

%Overriding the \footnotetext command to hide the marker if its value is `0`
\let\svfootnotetext\footnotetext
\renewcommand\footnotetext[2][?]{%
  \if\relax#1\relax%
    \ifnum\value{footnote}=0\blfootnotetext{#2}\else\svfootnotetext{#2}\fi%
  \else%
    \if?#1\ifnum\value{footnote}=0\blfootnotetext{#2}\else\svfootnotetext{#2}\fi%
    \else\svfootnotetext[#1]{#2}\fi%
  \fi
}

\begin{document}
\maketitle
\begin{enumerate}
  \item Wyznacz wszystkie takie pary \((a, b)\) dodatnich liczb całkowitych, że liczba \(a+b\) jest liczbą pierwszą oraz liczba \(a^{3}+b^{3}\) jest podzielna przez 3.
  \item Czy istnieje taki wielościan, którego rzuty prostokątne na pewne trzy płaszczyzny są odpowiednio czworokątem, sześciokątem i ośmiokątem? Odpowiedź uzasadnij.
  \item Wskazówki zegara pokrywają się o godzinie 12:00, w tej samej pozycji znajdą się po 12 godzinach. lle razy w międzyczasie (nie licząc pokryć o godzinie 12:00 w południe i o północy) pokryją się?
\end{enumerate}

\section*{LICEUM}
\begin{enumerate}
  \item Ostrosłup prawidłowy sześciokątny przecięto płaszczyzną, która przecina wszystkie jego krawędzie boczne. W przekroju otrzymano sześciokąt wypukły \(A B C D E F\). Wykaż, że proste \(A D, B E\) i \(C F\) przecinają się w jednym punkcie.
  \item W trójkącie \(A B C\) dwusieczna kąta \(A C B\) przecina bok \(A B\) w punkcie \(D\). Długości boków \(B C\) i \(A C\) są równe odpowiednio \(a\) i \(b\), a długość odcinka \(C D\) jest równa \(d\). Wykaż, że
\end{enumerate}

\[
d<\frac{2 a b}{a+b}
\]

\begin{enumerate}
  \setcounter{enumi}{2}
  \item Dodatnie liczby rzeczywiste \(a, b\) mają tę własność, że liczba \(\frac{a-b}{a+b}\) jest wymierna. Udowodnij, że liczba \(\frac{2 a-b}{2 a+b}\) też jest wymierna
\end{enumerate}

\footnotetext{Rozwiqzania należy oddać do środy 25 maja do godziny 15.00 koordynatorowi konkursu panu Jarostawowi Szczepaniakowi lub swojemu nauczycielowi matematyki lub przestać na adres \href{mailto:jareksz@interia.pl}{jareksz@interia.pl} do piatku 27 maja do pótnocy.
}
\end{document}