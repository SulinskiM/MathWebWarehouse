\documentclass[10pt]{article}
\usepackage[polish]{babel}
\usepackage[utf8]{inputenc}
\usepackage[T1]{fontenc}
\usepackage{amsmath}
\usepackage{amsfonts}
\usepackage{amssymb}
\usepackage[version=4]{mhchem}
\usepackage{stmaryrd}

\title{GIMNAZJUM }

\author{}
\date{}


\begin{document}
\maketitle
\begin{enumerate}
  \item W grze w statki, która toczy się na planszy o wymiarach \(9 \times 9\), nasz przeciwnik gdzieś ukrył lotniskowiec, reprezentowany przez prostokąt o wymiarach \(5 \times 1\) lub \(1 \times 5\). Jaka jest minimalna liczba strzałów, które musimy oddać, by choć raz trafić lotniskowiec, niezależnie od jego lokalizacji? Odpowiedź uzasadnij.
  \item Znajdź dziewięciocyfrową liczbę składającą się z cyfr 1, 2, ..., 9 ustawionych w pewnej kolejności, o tej własności, że jej każde dwie kolejne cyfry tworzą liczbę dwucyfrową, którą można przedstawić w postaci iloczynu \(k \cdot l\), gdzie \(k, l \in\{1,2, \ldots, 9\}\)
  \item Mamy 10 worków z monetami. W jednym z nich wszystkie monety są fałszywe, w pozostałych zaś wszystkie są prawdziwe. Prawdziwa moneta waży 10 gramów, a fałszywa 11. Ile ważeń na wadze elektronicznej trzeba wykonać, aby wykryć worek z fałszywymi monetami?
\end{enumerate}

\section*{LICEUM}
\begin{enumerate}
  \item Kierowca ciężarówki zwykle pokonuje dystans między dwoma miastami jadąc autostradą ze stałą prędkością. Niestety podczas ostatniego przejazdu na niektórych odcinkach autostrady trwał remont, w wyniku czego kierowca musiał podczas jazdy na tych odcinkach zredukować prędkość o jedną czwartą. W wyniku tego w czasie, w którym zwykle pokonałby całą trasę, udało mu się przebyć tylko sześć siódmych jej długości. Jaką część czasu poświęconego na przebycie tej drogi kierowca jechał przez remontowane odcinki?
  \item Czy przekrojem czworościanu foremnego może być kwadrat?
  \item Złośliwy czarodziej rzucił urok na jedną z 1000 beczek z winem - po wypiciu choćby kropli każdy zzielenieje w ciągu doby. Codziennie rano dysponujemy dokładnie 10 dzielnymi (i niezielonymi) rycerzami gotowymi ponieść ryzyko. Ile dni trzeba, aby wykryć zaczarowaną beczkę?
\end{enumerate}

\end{document}