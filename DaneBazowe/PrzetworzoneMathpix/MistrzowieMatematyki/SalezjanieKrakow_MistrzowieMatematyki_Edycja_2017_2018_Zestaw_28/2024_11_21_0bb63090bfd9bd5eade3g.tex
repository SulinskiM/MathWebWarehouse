\documentclass[10pt]{article}
\usepackage[polish]{babel}
\usepackage[utf8]{inputenc}
\usepackage[T1]{fontenc}
\usepackage{amsmath}
\usepackage{amsfonts}
\usepackage{amssymb}
\usepackage[version=4]{mhchem}
\usepackage{stmaryrd}

\title{GIMNAZJUM }

\author{}
\date{}


\begin{document}
\maketitle
\begin{enumerate}
  \item Udowodnij (bez użycia kalkulatora lub komputera), że
\end{enumerate}

\[
\frac{1}{2} \cdot \frac{3}{4} \cdot \frac{5}{6} \cdots \cdots \frac{99}{100}<\frac{1}{10}
\]

\begin{enumerate}
  \setcounter{enumi}{1}
  \item W ziemię wbito - w pewnej odległości od siebie - dwa pale: jeden o wysokości 4 m 30 cm , a drugi 2 m 60 cm . Następnie powieszono dwa naprężone sznurki: z wierzchołka pierwszego pala do podstawy drugiego oraz z wierzchołka drugiego pala do podstawy pierwszego. Na jakiej wysokości nad ziemią sznurki się przetną?
  \item Kostka \(3 \times 3 \times 3\) składa się z 27 małych sześcianów. Żuczek matematyczny może przejść z jednego małego sześcianu do drugiego wyłącznie przez sąsiadującą ścianę. Znajdź te wszystkie małe sześciany, z których może wystartować żuczek, jeśli chce w swojej podróży odwiedzić wszystkie małe sześciany - każdy dokładnie raz.
\end{enumerate}

\section*{LICEUM}
\begin{enumerate}
  \item Znajdź wszystkie takie pola na szachownicy \(8 \times 8\), że szachownicę bez tego pola można pokryć kamieniami \(3 \times 1\).
  \item Udowodnij, że 28 jest największą liczbą parzystą, której nie da się rozłożyć na sumę dwóch liczb nieparzystych złożonych.
  \item Niech M i N będą odpowiednio środkami przekątnych AC i BD czworokąta ABCD. Wykaż, że [ABM]+[CDM]= [ABN]+[CDN], gdzie [XYZ] oznacza pole trójkąta XYZ.
\end{enumerate}

\end{document}