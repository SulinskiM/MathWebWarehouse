\documentclass[10pt]{article}
\usepackage[polish]{babel}
\usepackage[utf8]{inputenc}
\usepackage[T1]{fontenc}
\usepackage{amsmath}
\usepackage{amsfonts}
\usepackage{amssymb}
\usepackage[version=4]{mhchem}
\usepackage{stmaryrd}

\title{KLASY PO SZKOLE PODSTAWOWEJ }

\author{}
\date{}


\begin{document}
\maketitle
\begin{enumerate}
  \item W polach tablicy \(4 \times 4\) umieszczono liczbę -1 i piętnaście liczb +1. Można jednocześnie zmienić znaki wszystkich liczb w jednym wierszu lub jednej kolumnie. Czy po pewnej liczbie takich zmian można uzyskać tablicę wypełnioną samymi jedynkami? Odpowiedź uzasadnij.
  \item Mamy do dyspozycji 6 odcinków. Mają one odpowiednio długości: 1, 2, 3, 2019, 2020, 2021. Ile różnych trójkątów możemy z nich ułożyć?
  \item Liczby \(a, b\) spełniają warunek \(2 a+a^{2}=2 b+b^{2}\). Wykaż, że jeżeli liczba \(a\) jest całkowita, to liczba \(b\) także jest całkowita.
\end{enumerate}

\section*{KLASY PO GIMNAZJUM}
\begin{enumerate}
  \item Czy istnieje wielościan, w którym każda ściana ma kąty wewnętrzne nie mniejsze niż \(90^{\circ} \mathrm{i}\) ma on 2021 krawędzi? Odpowiedź uzasadnij.
  \item Dany jest osiemdziesięciokąt foremny. Połowę jego wierzchołków pomalowano na biało, a połowę na czarno. Udowodnij, że można te 80 wierzchołków podzielić na 20 grup po 4 tak, żeby w każdej grupie były dwa wierzchołki białe i dwa czarne, i żeby te cztery wierzchołki były wierzchołkami prostokąta.
  \item Udowodnij, że inwersja względem okręgu o środku O przekształca prostą \(k\) nieprzechodzącą przez punkt O na okrąg przechodzący przez punkt O. O inwersji była mowa w zestawie 7.
\end{enumerate}

\end{document}