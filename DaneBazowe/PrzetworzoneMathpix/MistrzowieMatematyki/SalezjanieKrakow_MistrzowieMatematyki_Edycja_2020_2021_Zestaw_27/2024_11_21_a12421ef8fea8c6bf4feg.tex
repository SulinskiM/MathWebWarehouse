\documentclass[10pt]{article}
\usepackage[polish]{babel}
\usepackage[utf8]{inputenc}
\usepackage[T1]{fontenc}
\usepackage{amsmath}
\usepackage{amsfonts}
\usepackage{amssymb}
\usepackage[version=4]{mhchem}
\usepackage{stmaryrd}

\title{KLASY PO SZKOLE PODSTAWOWEJ }

\author{}
\date{}


\begin{document}
\maketitle
\begin{enumerate}
  \item Rozwiąż układ równań:
\end{enumerate}

\[
\left\{\begin{array}{l}
a^{2}+b^{2}=c d \\
c^{2}+d^{2}=a b
\end{array}\right.
\]

\begin{enumerate}
  \setcounter{enumi}{1}
  \item Spotkała się grupa pięciu osób: Albert, Borys, Cyprian, Daria i Edyta. Każda z nich przywitała się jednym uściskiem dłoni z każdą osobą, którą znała. Albert uścisnął dłoń raz, Borys dwa razy, Cyprian trzy razy, a Daria cztery razy. Ile razy uścisnęła dłoń Edyta?
  \item W banku jest 10 sejfów a w każdym sejfie 10 sztabek złota. Każda sztabka waży 1 kg, lecz w jednym sejfie sztabki są podrobione i ważą o 1 gram mniej czyli 999 gramów. Niestety nie wiesz w którym sejfie są podrobione sztabki, a różnicy w wadze tych sztabek nie da się wyczuć biorąc je do ręki. Masz za to starą wagę elektryczną (czyli taką która pokazuje wagę położonych na niej przedmiotów na liczniku), która jest tak stara, że może wytrzymać tylko jedno ważenie tzn. możesz położyć na niej ile chcesz sztabek na raz, ale tylko raz bo potem się zepsuje (nie możesz dokładać sztabek do tych już położonych na wadze). Jak za pomocą tej wagi dowiedzieć się, które sztabki są podrobione?
\end{enumerate}

\section*{KLASY PO GIMNAZJUM}
\begin{enumerate}
  \item Dany jest zbiór \(\{1,2,3, \ldots, 2 n\}\) gdzie \(n \in N\). Losujemy z tego zbioru ze zwracaniem dwie liczby. Jakie jest prawdopodobieństwo, że iloraz drugiej wylosowanej liczby przez pierwszą należy do przedziału \((1,2)\) ?
  \item Wykaż, że jeśli w czworościanie ABCD wysokości poprowadzone z wierzchołków C i D przecinają się, to krawędzie AB i CD są prostopadłe.
  \item Rozwiąż układ równań
\end{enumerate}

\[
\left\{\begin{array}{l}
a b=a+b+1 \\
b c=b+c+3 \\
c a=c+a+7
\end{array}\right.
\]


\end{document}