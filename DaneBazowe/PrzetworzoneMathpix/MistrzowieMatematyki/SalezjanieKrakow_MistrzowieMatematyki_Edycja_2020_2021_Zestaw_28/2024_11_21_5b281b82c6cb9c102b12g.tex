\documentclass[10pt]{article}
\usepackage[polish]{babel}
\usepackage[utf8]{inputenc}
\usepackage[T1]{fontenc}
\usepackage{amsmath}
\usepackage{amsfonts}
\usepackage{amssymb}
\usepackage[version=4]{mhchem}
\usepackage{stmaryrd}

\title{KLASY PO SZKOLE PODSTAWOWEJ }

\author{}
\date{}


\begin{document}
\maketitle
\begin{enumerate}
  \item Mamy dwie klepsydry: małą i dużą. W małej piasek przesypuje się w ciągu 7 minut, a w dużej w ciągu 11 minut. Jak za pomocą tych klepsydr odmierzyć 15 minut?
  \item W działaniu FOUR + FIVE = NINE te same litery oznaczają te same cyfry a różnym literom odpowiadają różne cyfry (jest to tzw. kryptarytm). Dodatkowo wiemy, że liczba FOUR jest podzielna przez 4, liczba FIVE jest podzielna przez 5, a liczba NINE jest podzielna przez 3. Jaką liczbą jest NINE?
  \item Liczby całkowite \(a\) i \(b\) są dodatnie i spełniają równość \(20 a+21 b=617\). Znajdź wartość \(21 a+20 b\).
\end{enumerate}

\section*{KLASY PO GIMNAZJUM}
\begin{enumerate}
  \item Graniastosłup o wysokości \(H\) ma w podstawie równoległobok o kącie ostrym \(\alpha\). Dłuższa przekątna graniastosłupa jest nachylona do podstawy pod kątem \(\beta\), a krótsza przekątna graniastosłupa jest nachylona do podstawy pod kątem \(\gamma\). Oblicz objętość graniastosłupa.
  \item W czworościanie ABCD punkty K i L są środkami okręgów wpisanych odpowiednio w trójkąty ABC i ABD. Udowodnij, że proste CL i DK przecinają się wtedy i tylko wtedy, gdy \(A D \cdot B C=A C \cdot B D\).
  \item Wyznacz wszystkie liczby pierwsze \(p\), dla których liczba \(2^{p}+p^{2}\) też jest liczbą pierwszą.
\end{enumerate}

\end{document}