\documentclass[10pt]{article}
\usepackage[polish]{babel}
\usepackage[utf8]{inputenc}
\usepackage[T1]{fontenc}
\usepackage{amsmath}
\usepackage{amsfonts}
\usepackage{amssymb}
\usepackage[version=4]{mhchem}
\usepackage{stmaryrd}

\title{KLASY PO SZKOLE PODSTAWOWEJ }

\author{}
\date{}


\begin{document}
\maketitle
\begin{enumerate}
  \item Udowodnij, że jeżeli liczbę naturalną da się przedstawić w postaci sumy kwadratów dwóch liczb całkowitych, to jej dziesięciokrotność również.
  \item W zegarze, kąt pomiędzy wskazówką godzinową a godziną dwunastą wynosi \(133^{\circ}\). Ile minut minęło od ostatniej pełnej godziny?
  \item Panowie A, B, C, usiedli w rzędzie jeden za drugim. Pan A widział wszystkich pozostałych, pan B tylko pana C, a pan C nie widział nikogo. W pudle były 3 kapelusze niebieskie i 2 czerwone, o czym wszyscy panowie wiedzieli. Każdemu z panów nałożono jeden z kapeluszy na głowę w ten sposób, że żaden z nich nie wiedział jaki kapelusz ma na głowie. Zapytano pana A czy wie jaki kapelusz ma na głowie - odpowiedź brzmiała „nie", Zapytano pana B też nie wiedział, lecz okazało się, że Pan C słysząc wszystkie wcześniejsze odpowiedzi, wiedział jaki kapelusz ma na głowie. Jakiego koloru był kapelusz Pana C?
\end{enumerate}

\section*{KLASY PO GIMNAZJUM}
\begin{enumerate}
  \item Oblicz różnicę:
\end{enumerate}

\[
\left(1^{2}+2^{2}+3^{2}+\cdots+2021^{2}\right)-(1 \cdot 3+2 \cdot 4+3 \cdot 5+\cdots+2020 \cdot 2022)
\]

\begin{enumerate}
  \setcounter{enumi}{1}
  \item Funkcja \(f\), określona w zbiorze liczb rzeczywistych i przyjmująca wartości rzeczywiste, spełnia dla każdego \(x>0\) warunek \(2 f(x)+3 f\left(\frac{2021}{x}\right)=5 x\). Oblicz \(f(43)\).
  \item Rozważmy kwadrat magiczny o wymiarach \(3 \times 3\). Na rysunku poniżej widzimy dwie jego liczby. Jaka liczba ukrywa się pod literą \(a\) ?
\end{enumerate}

\begin{center}
\begin{tabular}{|c|l|l|}
\hline
\(a\) &  &  \\
\hline
 &  & 33 \\
\hline
 & 25 &  \\
\hline
\end{tabular}
\end{center}


\end{document}