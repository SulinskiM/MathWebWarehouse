\documentclass[10pt]{article}
\usepackage[polish]{babel}
\usepackage[utf8]{inputenc}
\usepackage[T1]{fontenc}
\usepackage{amsmath}
\usepackage{amsfonts}
\usepackage{amssymb}
\usepackage[version=4]{mhchem}
\usepackage{stmaryrd}

\title{Zestaw 4 }

\author{}
\date{}


\begin{document}
\maketitle
\section*{KLASY PIERWSZE I DRUGIE}
\begin{enumerate}
  \item Ciąg Fibonacciego określony jest następująco:
\end{enumerate}

\[
F_{1}=F_{2}=1
\]

\(F_{n+2}=F_{n+1}+F_{n}\) dla \(n\) całkowitych dodatnich\\
Ustal, czy liczba \(F_{2021}\) jest parzysta.\\
2. Na każdym polu szachownicy \(2021 \times 2021\) mieszka krasnoludek, przy czym żaden z krasnoludków nigdy nie opuszcza pola, na którym mieszka. Okazało się, że 2026 krasnoludków cierpi na nieuleczalną, zaraźliwą chorobę - matemafilię, w tym 9 krasnoludków mieszkających na kwadracie \(3 \times 3\) na samym środku szachownicy. Zdrowy krasnoludek zarazi się matemafilią, jeśli co najmniej dwóch jego sąsiadów jest na nią chorych (sąsiadami są krasnoludki, które zajmują pola o sąsiednim boku). Zarażenie matemafilią następuje zawsze o północy, przy czym zarażony krasnoludek może zarazić innego dopiero po 12 godzinach. Czy jest możliwe, że wszystkie krasnoludki będą chore na matemafilię? Jeśli tak, to po ilu - najpóźniej - dniach się to stanie?\\
3. Turysta idący na stację kolejową przeszedł w ciągu godziny 3,5 km i zorientował się, że idąc nadal z tą samą prędkością, spóźni się na pociąg o godzinę. Przyspieszył więc i pozostałą część trasy przeszedł z prędkością \(5 \mathrm{~km} / \mathrm{h}\), docierając na stację pół godziny przed planowanym odjazdem pociągu. Jaką długą trasę przebył ten turysta?

\section*{KLASY TRZECIE}
\begin{enumerate}
  \item Wyznacz wszystkie pary \((a, b)\) dodatnich liczb całkowitych, które spełniają równanie
\end{enumerate}

\[
a b=(a-b)^{3}
\]

\begin{enumerate}
  \setcounter{enumi}{1}
  \item Rozwiąż w dodatnich liczbach całkowitych nieparzystych równanie
\end{enumerate}

\[
a^{2}-b^{3}=4
\]

\begin{enumerate}
  \setcounter{enumi}{2}
  \item Dany jest czworokąt wypukły \(A B C D\). Punkty K i L są odpowiednio środkami boków \(A B\)\\
CD. Wykaż, że jeżeli pola czworokątów BCLK i DAKL są równe, to czworokąt ABCD jest trapezem.
\end{enumerate}

\end{document}