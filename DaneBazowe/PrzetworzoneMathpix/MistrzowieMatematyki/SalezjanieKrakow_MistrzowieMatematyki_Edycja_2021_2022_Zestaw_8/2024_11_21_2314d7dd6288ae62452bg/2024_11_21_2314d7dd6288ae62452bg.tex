\documentclass[10pt]{article}
\usepackage[polish]{babel}
\usepackage[utf8]{inputenc}
\usepackage[T1]{fontenc}
\usepackage{amsmath}
\usepackage{amsfonts}
\usepackage{amssymb}
\usepackage[version=4]{mhchem}
\usepackage{stmaryrd}

\title{KLASY PIERWSZE I DRUGIE }

\author{}
\date{}


\newcommand\Varangle{\mathop{{<\!\!\!\!\!\text{\small)}}\:}\nolimits}

\begin{document}
\maketitle
\begin{enumerate}
  \item W trójkącie ostrokątnym \(A B C\) dane są wysokości \(A D\) i BE. Udowodnij, że trójkąt CDE jest podobny do trójkąta \(A B C\).
  \item W trójkąt prostokątny \(A B C\) wpisano okrąg. Rzut tego okręgu na przeciwprostokątną \(A B\) jest odcinkiem MN. Wyznacz kąt MCN.
  \item Czworokąt wypukły ABCD jest wpisany w okrąg. Półproste AD i BC przecinają się w punkcie P. Wykazać, że \(\Varangle A P B=\Varangle A D B-\Varangle C A D\)
\end{enumerate}

\section*{KLASY TRZECIE}
\begin{enumerate}
  \item Punkt D leży na boku \(A B\) trójkąta \(A B C\). Okręgi styczne do prostych \(A C\) i \(B C\) odpowiednio w punktach A i B przechodzą przez punkt D i przecinają się po raz drugi w punkcie E. Punkt F jest odbiciem symetrycznym wierzchołka C względem symetralnej boku AB. Wykaż, ze punkty D, E i F są współliniowe.
  \item Dany jest trójkąt równoboczny \(A B C\). Na półprostej \(C A\) wybrano punkty \(A_{1}, A_{2}\) zaś na półprostej \(C B\) wybrano punkty \(B_{1}, B_{2}\). Na zewnątrz kąta \(A C B\) wybrano punkty \(C_{1}, C_{2} \mathrm{w}\) ten sposób, że trójkąty \(A_{1} B_{1} C_{1}\) i \(A_{2} B_{2} C_{2}\) są równoboczne. Wykaż, że punkty \(C, C_{1}, C_{2}\) leżą na jednej prostej.
  \item Udowodnij, że istnieje nieskończenie wiele parami różnych liczb całkowitych \(a, b, c\) i \(d\), że liczby
\end{enumerate}

\[
a^{2}+2 c d+b^{2} \text { oraz } c^{2}+2 a b+d^{2}
\]

są kwadratami.


\end{document}