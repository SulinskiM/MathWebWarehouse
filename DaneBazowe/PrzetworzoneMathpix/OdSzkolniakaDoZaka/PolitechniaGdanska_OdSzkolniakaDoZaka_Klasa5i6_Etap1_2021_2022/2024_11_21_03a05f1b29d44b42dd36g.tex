\documentclass[10pt]{article}
\usepackage[polish]{babel}
\usepackage[utf8]{inputenc}
\usepackage[T1]{fontenc}
\usepackage{amsmath}
\usepackage{amsfonts}
\usepackage{amssymb}
\usepackage[version=4]{mhchem}
\usepackage{stmaryrd}
\usepackage{hyperref}
\hypersetup{colorlinks=true, linkcolor=blue, filecolor=magenta, urlcolor=cyan,}
\urlstyle{same}

\title{OD SZKOLNIAKA DO ŻAKA }

\author{}
\date{}


\begin{document}
\maketitle
\section*{klasy 5 i 6 szkoły podstawowej \\
 rok szkolny 2021/2022 \\
 Zadania - etap I}
Zadanie 1. W dwóch pudełkach było razem 45 czekoladek. Jeśli Antek zjadł 12 czekoladek z jednego pudełka, a z drugiego 15, to w obu pudełkach zostało po tyle samo sztuk. lle czekoladek było w każdym pudełku?

Zadanie 2. W klasie jest 24 uczniów. Ilu jest wśród nich chłopców, a ile dziewczynek, jeżeli połowa liczby wszystkich chłopców jest równa 0,3 liczby dziewczynek?

Zadanie 3. Z prostokątnego kartonu o wymiarach 30 cm i 3,7 dm odcięto w czterech rogach kwadraty o boku 30 mm . Pozostała część kartonu tworzy siatkę otwartego pudełka. Ile litrów wody można zmieścić w tym pudełku?

Zadanie 4. Długość jednego z boków trójkąta wynosi \(\operatorname{NWD}(36,48)\) zaś pole tego trójkąta jest równe \(\operatorname{NWW}(36,48) \mathrm{j}^{2}\). Ile wynosi wysokość tego trójkąta opuszczona na podany bok?

Zadanie 5. Suma dwóch liczb naturalnych dwucyfrowych jest równa największej liczbie dwucyfrowej. Po podzieleniu większej liczby przez mniejszą otrzymamy resztę równą 3. Jakie to pary liczb?

\section*{ZAŁĄCZNIK NR 1 DO KONKURSU „OD SZKOLNIAKA DO ŻAKA"}
\section*{Oświadczenie \\
 Przyjmuję do wiadomości dane od organizatora zawarte w poniższej Klauzuli informacyjnej:}
\section*{Klauzula informacyjna}
Zgodnie z art. 13 ogólnego rozporządzenia o ochronie danych osobowych z dnia 27 kwietnia 2016 r. (Dz. Urz. UE L 119 z 04.05.2016) ( RODO) informujemy, że:

\begin{enumerate}
  \item Administratorem danych osobowych Pani/Pana dziecka jest Politechnika Gdańska z siedzibą przy ul. Narutowicza 11/12, w Gdańsku (kod pocztowy: 80-233);
  \item Administrator wyznaczył Inspektora Ochrony Danych, z którym można się skontaktować za pośrednictwem adresu e-mail: - \href{mailto:iod@pg.edu.pl}{iod@pg.edu.pl}. Do Inspektora Ochrony Danych należy kierować wyłącznie sprawy dotyczące przetwarzania danych Pani/Pana dziecka przez Politechnikę Gdańską, w tym realizacji Pani/Pana dziecka praw;
  \item Dane Pani/Pana dziecka będą przetwarzane w celu organizacji i przeprowadzenia niniejszego konkursu. Przetwarzanie przez Administratora wskazanych danych osobowych jest niezbędne do wykonania zadania realizowanego w interesie publicznym, którym jest umożliwienie Organizatorowi przeprowadzenia Konkursu i umożliwienie uczestnikom Konkursu wzięcia w nim udziału, opublikowanie informacji o finalistach i laureatach oraz archiwizacja dokumentów;
  \item Dane osobowe będą przechowywane do końca roku szkolnego w jakim zostały zebrane za wyjątkiem danych laureatów, które mogą być przechowywane bezterminowo;
  \item Podane dane nie będą podlegały udostępnieniu podmiotom trzecim. Odbiorcami danych będą tylko instytucje upoważnione na mocy prawa (sądy, policja itp.);
  \item Przysługuje Pani/Panu:\\
a. prawo dostępu do treści danych osobowych Pani/Pana dziecka oraz otrzymania ich kopii,\\
b. prawo do sprostowania (poprawiania) danych osobowych Pani/Pana dziecka,\\
c. prawo do usunięcia danych osobowych w sytuacji gdy przetwarzanie danych nie następuje w celu w jakim zostały one zebrane,\\
d. prawo do ograniczenia przetwarzania danych,\\
e. prawo do wniesienia sprzeciwu wobec przetwarzania danych - z przyczyn związanych ze szczególną sytuacją osób, których dane są przetwarzane,\\
f. prawo do wniesienia skargi do Prezesa UODO (na adres Urzędu Ochrony Danych Osobowych, ul. Stawki 2, 00 - 193 Warszawa).
  \item Podanie danych jest dobrowolne, lecz niezbędne do wzięcia udziału Pani/Pana dziecka w niniejszym konkursie;
  \item Dane osobowe dziecka udostępnione przez Panią/Pana nie będą podlegały profilowaniu;
  \item Administrator danych nie zamierza przekazywać danych osobowych do państwa trzeciego lub organizacji międzynarodowej.
\end{enumerate}

Wyrażam zgodę na uczestnictwo mojego dziecka w konkursie matematycznym „Od szkolniaka do żaka" organizowanym przez Politechnikę Gdańską oraz oświadczam, że akceptuję regulamin konkursu „Regulamin konkursu matematycznego w roku akademickim 2021/2022 dla klasy V, VI, VII i VIII szkoły podstawowej „Od szkolniaka do żaka".


\end{document}