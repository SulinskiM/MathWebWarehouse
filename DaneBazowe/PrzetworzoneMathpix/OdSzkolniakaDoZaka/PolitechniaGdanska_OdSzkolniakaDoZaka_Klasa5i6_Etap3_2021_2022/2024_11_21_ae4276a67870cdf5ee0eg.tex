\documentclass[10pt]{article}
\usepackage[polish]{babel}
\usepackage[utf8]{inputenc}
\usepackage[T1]{fontenc}
\usepackage{amsmath}
\usepackage{amsfonts}
\usepackage{amssymb}
\usepackage[version=4]{mhchem}
\usepackage{stmaryrd}

\title{POLITECHNIKA \\
 GDANSKA \\
 \textbackslash author\{ CENTRUM MATEMATYKI }

\author{}
\date{}


\begin{document}
\maketitle
\}

\section*{OD SZKOLNIAKA DO ŻAKA}
\section*{klasy 5 i 6 szkoły podstawowej \\
 rok szkolny 2021/2022 \\
 Zadania - etap III}
Zadanie 1. Kosmici zapakowali 36 milionów mrówek do kartonu w kształcie sześcianu. Jaka była krawędź tego sześcianu jeśli każda mrówka zajmuje 6 mm³?

Zadanie 2. Kwadrat rozcięto na dwa prostokąty o obwodach 12 cm i 18 cm . Oblicz pola tych prostokątów.

Zadanie 3. Dane są liczby \(a=0\), (12) i \(b=0,(21)\). Oblicz \(a+b\) oraz \(\frac{a}{b}\). Wynik zapisz w postaci ułamka zwykłego nieskracalnego.

Zadanie 4. Na trzydniową wycieczkę w góry pojechało 50 uczestników. Ogólny koszt wycieczki wyniósł 9000 zł. Opłata za autokar stanowiła \(20 \%\) ogólnych kosztów wycieczki, zaś noclegi uczestników to \(\frac{3}{10}\) ogólnych kosztów. Wyżywienie stanowiło 60\% pozostałej części pieniędzy. Resztę kwoty przeznaczono na bilety wstępu. Jaki był koszt wyżywienia jednego uczestnika wycieczki? Ile pieniędzy wydano na bilety wstępu?

Zadanie 5. Trójkąt \(A B C\) ma obwód równy \(\frac{0,25 \cdot\left(6^{2}+8^{2}\right)}{\frac{1}{6}+\frac{3}{7}}\). Na boku BC wyznaczono punkt \(D\) tak, że kąty CAD i ACD mają równe miary. Oblicz długość boku AC jeżeli wiadomo, że trójkąt ABD ma obwód równy 29.


\end{document}