\documentclass[10pt]{article}
\usepackage[polish]{babel}
\usepackage[utf8]{inputenc}
\usepackage[T1]{fontenc}
\usepackage{amsmath}
\usepackage{amsfonts}
\usepackage{amssymb}
\usepackage[version=4]{mhchem}
\usepackage{stmaryrd}
\usepackage{hyperref}
\hypersetup{colorlinks=true, linkcolor=blue, filecolor=magenta, urlcolor=cyan,}
\urlstyle{same}

\title{OD SZKOLNIAKA DO ŻAKA }

\author{}
\date{}


\begin{document}
\maketitle
klasy 7 i 8 szkoły podstawowej\\
rok szkolny 2019/2020

\section*{Zadania - etap I}
Zadanie 1. Porównaj liczby: \(a=\sqrt[3]{16}+\sqrt[3]{54}\) i \(b=\sqrt[3]{2}+\sqrt[3]{128}\).

Zadanie 2. Szewczyk Dratewka, po pokonaniu smoka, zgłosił się do króla, który obiecał mu rękę swojej córki. Król jednak nie chciał spełnić obietnicy i powiedział: „Moja jedyna córka zostanie twoją żoną, jeśli podasz mi poprawną wartość liczby: \(a=2009 \frac{11}{13} \cdot 2010 \frac{11}{13}-2008 \frac{11}{13} \cdot 2011 \frac{11}{13}\)." Jeszcze tego samego dnia księżniczka została żoną szewczyka. Jaką wartość a podał królowi szewczyk?

Zadanie 3. Suma czterech liczb jest równa 19. Druga liczba jest 3 razy większa od pierwszej. Trzecia liczba jest o 5 większa od sumy dwóch pierwszych liczb. Czwarta liczba jest średnią arytmetyczną drugiej i trzeciej liczby. Wyznacz te liczby.

Zadanie 4. W trapezie prostokątnym ABCD krótsza przekątna AC dzieli go na trójkąt prostokątny i trójkąt równoboczny. Dłuższa podstawa \(A B\) trapezu ma długość 6 cm . Oblicz obwód i pole tego trapezu.

Zadanie 5. Krawędź podstawy ostrosłupa prawidłowego czworokątnego ma długość 6 cm , a krawędź boczna jest równa 5 cm . Oblicz pole powierzchni i objętość tego ostrosłupa.

CENTRUM NAUCZANIA MATEMATYKI\\
I KSZTALCENIA NA ODLEGLOŚC

\section*{(imię i nazwisko uczestnika)}
\(\qquad\)\\
(szkoła - nazwa i adres)

\section*{ZAŁĄCZNIK NR 1 DO KONKURSU „OD SZKOLNIAKA DO ŻAKA"}
\section*{Oświadczenie}
Zgodnie z art. 6 ust. 1 lit. a ogólnego rozporządzenia o ochronie danych osobowych z dnia 27 kwietnia 2016 r. (Dz. Urz. UE L 119 z 04.05.2016) (RODO) oświadczam, że wyrażam zgodę na przetwarzanie przez Politechnikę Gdańską z siedzibą w Gdańsku, ul. Narutowicza 11/12, 80-233 Gdańsk, danych osobowych mojego dziecka w celu i zakresie niezbędnym do przeprowadzenia konkursu matematycznego „Od szkolniaka do żaka".

\section*{Klauzula informacyjna}
Zgodnie z art. 13 ogólnego rozporządzenia o ochronie danych osobowych z dnia 27 kwietnia 2016 r. (Dz. Urz. UE L 119 z 04.05.2016) (RODO) informujemy, że:

\begin{enumerate}
  \item Administratorem danych osobowych Pani/Pana dziecka jest Politechnika Gdańska z siedzibą przy ul. Narutowicza 11/12, w Gdańsku (kod pocztowy: 80-233);
  \item Administrator wyznaczył Inspektora Ochrony Danych, z którym można się skontaktować za pośrednictwem adresu e-mail: - \href{mailto:iod@pg.edu.pl}{iod@pg.edu.pl}. Do Inspektora Ochrony Danych należy kierować wyłącznie sprawy dotyczące przetwarzania danych Pani/Pana dziecka przez Politechnikę Gdańską, w tym realizacji Pani/Pana dziecka praw;
  \item Dane Pani/Pana dziecka będą przetwarzane w celu przeprowadzenia niniejszego konkursu na podstawie wyrażonej przez Panią/Pana zgody (Art. 6 ust. 1 lit. a RODO);
  \item Dane osobowe będą przechowywane do końca roku szkolnego w jakim zostały zebrane za wyjątkiem danych laureatów, które mogą być przechowywane bezterminowo;
  \item Podane dane nie będą podlegały udostępnieniu podmiotom trzecim. Odbiorcami danych będą tylko instytucje upoważnione na mocy prawa (sądy, policja itp.);
  \item Przysługuje Pani/Panu:\\
a. prawo dostępu do treści danych osobowych Pani/Pana dziecka oraz otrzymania ich kopii,\\
b. prawo do sprostowania (poprawiania) danych osobowych Pani/Pana dziecka,\\
c. prawo do usunięcia danych osobowych w sytuacji gdy przetwarzanie danych nie następuje w celu w jakim zostały one zebrane,\\
d. prawo do ograniczenia przetwarzania danych,\\
e. prawo do cofnięcia zgody w dowolnym momencie bez wpływu na zgodność z prawem przetwarzania, którego dokonano na podstawie zgody przed jej cofnięciem,\\
f. prawo do wniesienia skargi do Prezesa UODO (na adres Urzędu Ochrony Danych Osobowych, ul. Stawki 2, 00 - 193 Warszawa).
  \item Podanie danych jest dobrowolne, lecz niezbędne do wzięcia udziału Pani/Pana dziecka w niniejszym konkursie;
  \item Dane osobowe dziecka udostępnione przez Panią/Pana nie będą podlegały profilowaniu;
  \item Administrator danych nie zamierza przekazywać danych osobowych do państwa trzeciego lub organizacji międzynarodowej.\\
(data)\\
(podpis rodzica lub opiekuna prawnego ucznia)
\end{enumerate}

Wyrażam zgodę na uczestnictwo mojego dziecka w konkursie matematycznym „Od szkolniaka do żaka" organizowanym przez Politechnikę Gdańską oraz oświadczam, że akceptuję regulamin wyżej wymienionego konkursu - „Regulamin konkursu matematycznego w roku akademickim 2019/2020 dla klasy V, VI, VII i VIII szkoły podstawowej „Od szkolniaka do żaka".


\end{document}