\documentclass[10pt]{article}
\usepackage[polish]{babel}
\usepackage[utf8]{inputenc}
\usepackage[T1]{fontenc}
\usepackage{amsmath}
\usepackage{amsfonts}
\usepackage{amssymb}
\usepackage[version=4]{mhchem}
\usepackage{stmaryrd}

\title{Sprawdzian predyspozycji do klas matematycznych }

\author{}
\date{}


\begin{document}
\maketitle
XIV LO im. S. Staszica w Warszawie\\
(2 czerwca 2014 r.)

\begin{enumerate}
  \item Udowodnij, że jeżeli w trójkącie \(A B C\) symetralne boków \(B C\) i \(C A\) oraz dwusieczna kąta \(A C B\) przecinają się w jednym punkcie, to trójkąt \(A B C\) jest równoramienny.
  \item Pewna liczba postaci \(a^{4}-b^{4}\), gdzie \(a\) i \(b\) są dodatnimi liczbami całkowitymi, jest iloczynem dwóch liczb pierwszych \(p\) i \(q\), przy czym \(p>q\). Wykaż, że liczba \(p-g\) jest podwojonym kwadratem liczby naturalnej.
  \item Na bokach \(B C\) i \(C A\) trójkąta \(A B C\) zbudowano, po jego zewnętrznej stronie, takie prostoty \(B C D E\) oraz \(C A G H\), że \(C D=C A\) oraz \(B C=C H\). Punkt \(M\) jest środkiem odcinka \(A B\). Wykaż, że \(C M=\frac{1}{2} D H\).
  \item Oblicz, ile jest liczb naturalnych pięciocyfrowych parzystych, w zapisie których występuje dokładnie jedna cyfra zero i dokładnie jedna cyfra jeden.
  \item Wyznacz wszystkie pary \((a, b)\) dodatnich liczb całkowitych, dla których
\end{enumerate}

\[
2 a+b+3 \sqrt{a b}=3 \sqrt{a}+3 \sqrt{b}
\]

\begin{enumerate}
  \setcounter{enumi}{5}
  \item Dany jest ostrosłup prawidłowy trójkątny, którego podstawą jest trójkąt równoboczny o boku \(a\). Odległość od wierzchołka podstawy do płaszczyzny zawierającej przeciwległą ścianę boczną ostrosłupa równa się \(b\). Oblicz objętość ostrosłupa.
\end{enumerate}

Uwaga. Podanie jedynie prawidłowej odpowiedzi liczbowej nie stanowi rozwiązania zadania. Ocenie podlegał będzie tok rozumowania oraz rachunki prowadzące do uzyskanego wyniku.


\end{document}