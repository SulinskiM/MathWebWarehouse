\documentclass[10pt]{article}
\usepackage[polish]{babel}
\usepackage[utf8]{inputenc}
\usepackage[T1]{fontenc}
\usepackage{amsmath}
\usepackage{amsfonts}
\usepackage{amssymb}
\usepackage[version=4]{mhchem}
\usepackage{stmaryrd}

\title{Sprawdzian predyspozycji do klas matematycznych }

\author{XIV LO im. S. Staszica w Warszawie}
\date{}


\newcommand\Varangle{\mathop{{<\!\!\!\!\!\text{\small)}}\:}\nolimits}

\begin{document}
\maketitle
(28 maja 2018 r.)

\section*{Uwagi}
\begin{itemize}
  \item Poniższe zadania można rozwiązywać w dowolnej kolejności.
  \item Wszystkie zadania są jednakowo punktowane.
  \item Podanie jedynie prawidłowej odpowiedzi liczbowej nie stanowi rozwiązania zadania. Ocenie podlegal będzie tok rozumowania oraz obliczenia prowadzące do uzyskanego wyniku.
\end{itemize}

\begin{enumerate}
  \item Wykaż, że jeżeli dodatnie liczby \(a, b\) są całkowite, to liczba
\end{enumerate}

\[
\frac{a^{4}+b^{4}}{a+b}+\frac{a^{2}+b^{2}}{\frac{1}{a}+\frac{1}{b}}
\]

jest także całkowita.\\
2. Wyznacz najmniejszą nieparzystą liczbę \(n>1\), dla której liczba \(n^{2}-1\) jest podzielna przez 121.\\
3. Dany jest czworokąt wypukły \(A B C D\), w którym \(\Varangle D A B=\Varangle A B C=45^{\circ}\). Udowodnij, że

\[
B C+C D+D A<A B \sqrt{2} .
\]

\begin{enumerate}
  \setcounter{enumi}{3}
  \item Każdy punkt okręgu pokolorowano na jeden z dwóch kolorów. Okazało się, że każda średnica tego okręgu ma końce różnych kolorów. Wykaż, że istnieje trójkąt, którego wierzchołki są tego samego koloru i którego co najmniej jeden z kątów ma miarę \(45^{\circ}\).
  \item Wysokość ostrosłupa prawidłowego trójkątego ma długość równą wysokości jego podstawy. Wiedząc, że pole podstawy ostrosłupa jest równe 1 , wyznacz pole powierzchni bocznej tego ostrosłupa.
\end{enumerate}

\end{document}