\documentclass[10pt]{article}
\usepackage[polish]{babel}
\usepackage[utf8]{inputenc}
\usepackage[T1]{fontenc}
\usepackage{amsmath}
\usepackage{amsfonts}
\usepackage{amssymb}
\usepackage[version=4]{mhchem}
\usepackage{stmaryrd}

\title{Zestaw 1 }

\author{}
\date{}


\begin{document}
\maketitle
Kółko matematyczne dla kandydatów

\begin{enumerate}
  \item Rozwiąż układ równań:
\end{enumerate}

\[
\left\{\begin{array}{l}
(x+y)(x+y+z)=72 \\
(y+z)(x+y+z)=120 \\
(z+x)(x+y+z)=96 .
\end{array}\right.
\]

\begin{enumerate}
  \setcounter{enumi}{1}
  \item Udowodnij, że dla dowolnych liczb rzeczywistych \(a\) i \(b\) zachodzi nierówność:
\end{enumerate}

\[
a^{2}+b^{2}+1 \geq a b+a+b
\]

\begin{enumerate}
  \setcounter{enumi}{2}
  \item Mały majsterkowicz Kazio przygotował na szkolną dyskotekę efekty świetlne własnego pomysłu. Żarówki, których jest 1000 i które są ponumerowane liczbami od 1 do 1000, są włączane i wyłączane specjalnym przełącznikiem. Kolejne \(k\)-te naciśnięcie przełącznika zmienia stan wszystkich żarówek o numerach podzielnych przez \(k\). Na początku dyskoteki wszystkie żarówki były wyłączone. Pierwsze naciśnięcie przełącznika zapala wszystkie żarówki. Drugie naciśnięcie gasi wszystkie żarówki o numerach parzystych. Po trzecim użyciu przełącznika świecą się żarówki o numerach nieparzystych i jednocześnie niepodzielnych przez 3 oraz o numerach parzystych i podzielnych przez 3. Pod koniec dyskoteki okazało się, że Kazio naciskał przełącznik 1000 razy. Które żarówki świeciły się po zakończeniu dyskoteki?
  \item Czy istnieją takie dwie liczby \(x\) i \(y\), aby jednocześnie zachodziły równości:
\end{enumerate}

\[
x(y-x)=3, \quad y(4 y-3 x)=2
\]

Odpowiedź uzasadnij.\\
5. Punkt \(C\) leży wewnątrz odcinka \(A B\). Niech okręgi \(o_{1}, o_{2}\) i \(o\) będą okręgami o średnicach odpowiednio \(A C, B C\) i \(A B\). Prosta \(k\) przechodzi przez punkt \(C\) i przecina okręgi w pięciu punktach \(D, E, C, F, G\), położonych w wymienionej kolejności. Wykaż, że odcinki \(D E\) i \(F G\) mają równe długości.\\
6. Pewna liczba naturalna w układzie dziesiętnym ma postać \(x 0 y z\), gdzie \(x, y, z\) są cyframi, \(x>0\). Liczba ta podzielona przez pewną liczbę naturalną \(n\) daje iloraz, który w układzie dziesiętnym jest postaci \(x y z\). Znaleźć \(x, y, z \mathrm{i} n\).\\
7. Rozstrzygnij czy istnieje wielościan o sześciu ścianach i siedmiu wierzchołkach.


\end{document}