\documentclass[10pt]{article}
\usepackage[polish]{babel}
\usepackage[utf8]{inputenc}
\usepackage[T1]{fontenc}
\usepackage{amsmath}
\usepackage{amsfonts}
\usepackage{amssymb}
\usepackage[version=4]{mhchem}
\usepackage{stmaryrd}

\begin{document}
Kółko matematyczne dla kandydatów

Zestaw 3

\begin{enumerate}
  \item Na okręgu o promieniu 1 opisano trójkąt prostokątny \(A B C\) o kącie prostym przy wierzchołku \(C\). Na przeciwprostokątnej \(A B\) tego trójkąta wybrano takie punkty \(D\) i \(E\), że zachodzą równości \(A D=A C\) i \(B E=B C\). Oblicz długość odcinka \(D E\).
  \item W pudełku znajduje sie 11 kul białych i 11 kul niebieskich. Jaś i Małgosia grają w następującą grę, którą rozpoczyna Małgosia. Wyjmuje ona z tego pudełka wybrane przez siebie dwie kule. Jeżeli wybierze kule jednakowego koloru, to do pudełka dokłada jedną kulę białą; jeżeli wybierze kule róznych kolorów, to dokłada kulę niebieską. Następnie swój ruch, według tych samych zasad, wykonuje Jaś i znów Małgosia, znów Jaś itd., aż w końcu w pudełku zostanie tylko jedna kula. Jeżeli ta kula będzie biała, wygrywa Małgosia. W przeciwnym wypadku wygrywa Jaś. Czy Małgosia może tak prowadzić tę grę, aby wygrać? Odpowiedź uzasadnij.
  \item Rozwiąż układ równań
\end{enumerate}

\[
\left\{\begin{array}{l}
a^{2}+24=9 b+\frac{a+c}{2} \\
b^{2}+25=9 c+\frac{b+a}{2} \\
c^{2}+26=9 a+\frac{c+b}{2}
\end{array} .\right.
\]

\begin{enumerate}
  \setcounter{enumi}{3}
  \item Dany jest sześcian \(A B C D E F G H\). Na krawędziach \(A E, B C\) i \(G H\) tego sześcianu wybrano odpowiednio takie punkty \(M, N\) i \(P\), że \(A M=C N=\) \(H P\). Wykaż, że trójkąt \(M N P\) jest równoboczny.
  \item Powiemy, że liczba całkowita \(n\) jest liczbą słoneczną, jeżeli \(n=a^{2}+5 b^{2}\), gdzie liczby \(a\) i \(b\) są liczbami całkowitymi różnymi od zera. Wykaż, że jeżeli liczba \(n\) jest liczbą słoneczną, to \(n^{4}\) też jest liczbą słoneczną.
  \item Dana jest taka liczba rzeczywista, której rozwinięcie dziesiętne jest nieskończone i składa się wyłącznie z cyfr 1, 2 i 3 . Wykaż, że jeżeli w tym rozwinięciu jest co najwyżej 2010 jedynek i co najwyżej 2010 dwójek, to dana liczba jest wymierna.
  \item Na okręgu napisano \(n\) liczb rzeczywistych w taki sposób, że każda z tych liczb jest równa wartości bezwzględnej różnicy dwóch liczb stojących bezpośrednio za nią (patrząc zgodnie z ruchem zegara).\\
(a) Znajdź te liczby, jeśli \(n=2010\) a ich suma jest równa 1340\\
(b) Znajdź sumę tych liczb, jeśli \(n=1000\)
\end{enumerate}

\end{document}