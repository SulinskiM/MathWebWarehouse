\documentclass[10pt]{article}
\usepackage[polish]{babel}
\usepackage[utf8]{inputenc}
\usepackage[T1]{fontenc}
\usepackage{amsmath}
\usepackage{amsfonts}
\usepackage{amssymb}
\usepackage[version=4]{mhchem}
\usepackage{stmaryrd}

\title{AKADEMIA GÓRNICZO-HUTNICZA \\
 im. Stanisława Staszica w Krakowie OLIMPIADA „O DIAMENTOWY INDEKS AGH" 2019/20 \\
 MATEMATYKA - ETAP I }

\author{}
\date{}


\begin{document}
\maketitle
ZADANIA PO 10 PUNKTÓW

\begin{enumerate}
  \item Udowodnij, że jeżeli czworokąt wypukły ma oś symetrii, to można na nim opisać okrąg lub można weń okrąg wpisać.
  \item Wyznacz dziedzinę funkcji danej wzorem
\end{enumerate}

$$
f(x)=\left(32 x^{2}+28 x^{5}+4 x^{8}-x^{11}\right)^{-\frac{3}{4}}
$$

\begin{enumerate}
  \setcounter{enumi}{2}
  \item W worku znajduje się 50 skarpet czarnych, 40 brazowych, 30 zielonych i 20 niebieskich. Jaka jest najmniejsza liczba skarpet, które musimy wyjać na chybił trafił, aby mieć pewność, że wśród nich znajdziemy jednokolorowe pary skarpet dla 20 osób? Odpowiedź uzasadnij.
  \item Rozwiąż nierówność $\left|1-\log _{x}\left(x-\frac{1}{4}\right)\right| \leqslant 1$.
\end{enumerate}

\section*{ZADANIA PO 20 PUNKTÓW}
\begin{enumerate}
  \setcounter{enumi}{4}
  \item Ile jest par $(a, b)$ liczb rzeczywistych, dla których układ równań
\end{enumerate}

$$
\left\{\begin{array}{c}
a x+b y+1=0 \\
x^{2}+y^{2}=50
\end{array}\right.
$$

ma co najmniej jedno rozwiązanie, przy czym każde jego rozwiązanie jest parą $(x, y)$ liczb całkowitych? Podaj przykład pary $(a, b)$, dla której układ ten ma dwa rozwiązania w liczbach całkowitych oraz przykład pary $(a, b)$, dla której ten układ ma dokładnie jedno rozwiązanie i to rozwiązanie jest parą liczb całkowitych.\\
6. Długości dwóch boków trójkąta wpisanego w okrąg o średnicy $D$ są odpowiednio równe $\frac{3}{4} D$ oraz $\frac{\sqrt{3}}{2} D$. Oblicz długość trzeciego boku.\\
7. Zbiór $S$ jest zbiorem wszystkich dodatnich liczb całkowitych $n$, dla których istnieje permutacja $\left(a_{1}, a_{2}, \ldots, a_{n}\right)$ liczb $1,2, \ldots, n$, taka że $a_{1}+a_{2}+\ldots+a_{k}$ jest wielokrotnością liczby $k$ dla każdego $k=1,2, \ldots, n$. Wykaż, że każda liczba należąca do zbioru $S$ jest nieparzysta. Znajdź dwie liczby tego zbioru. Zbadaj, czy liczba 2019 należy do zbioru $S$.


\end{document}