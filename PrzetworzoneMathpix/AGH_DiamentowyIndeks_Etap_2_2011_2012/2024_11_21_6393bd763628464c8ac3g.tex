\documentclass[10pt]{article}
\usepackage[polish]{babel}
\usepackage[utf8]{inputenc}
\usepackage[T1]{fontenc}
\usepackage{amsmath}
\usepackage{amsfonts}
\usepackage{amssymb}
\usepackage[version=4]{mhchem}
\usepackage{stmaryrd}

\title{AKADEMIA GÓRNICZO-HUTNICZA \\
 im. Stanisława Staszica w Krakowie \\
 OLIMPIADA „O DIAMENTOWY INDEKS AGH" 2011/12 \\
 MATEMATYKA - ETAP II }

\author{}
\date{}


\begin{document}
\maketitle
\section*{ZADANIA PO 10 PUNKTÓW}
\begin{enumerate}
  \item Wykaż, że liczba $a=\sqrt{9-4 \sqrt{5}}-\sqrt{9+4 \sqrt{5}}$ jest całkowita.
  \item Wyznacz dziedzinę funkcji danej wzorem
\end{enumerate}

$$
f(x)=\sqrt{x^{4}+x^{3}-8 x^{2}-12 x}
$$

\begin{enumerate}
  \setcounter{enumi}{2}
  \item Oblicz miarę kąta między wektorami $\vec{a}$ i $\vec{b}$ wiedząc, że wektory $\vec{u}=3 \vec{a}+2 \vec{b} \quad$ i $\quad \vec{v}=-\vec{a}+4 \vec{b} \quad$ sa prostopadłe oraz $\quad|\vec{a}|=|\vec{b}|=1$.
  \item Dwa różne automaty wykonują razem daną pracę w ciągu 6 godzin. Gdyby pierwszy automat pracował sam przez 2 godziny, a następnie drugi pracował sam przez 6 godzin, to wykonałyby połowę całej pracy. W jakim czasie każdy automat może samodzielnie wykonać całą pracę?
\end{enumerate}

\section*{ZADANIA PO 20 PUNKTÓW}
\begin{enumerate}
  \setcounter{enumi}{4}
  \item Ze zbioru $S=\{1,2, \ldots, 2012\}$ losujemy trzy liczby i ustawiamy je w ciag rosnący $(a, b, c)$. Oblicz prawdopodobieństwo zdarzeń $A$ : iloczyn $a b c$ jest liczbą parzysta, $B_{k}: b=k$, gdzie $k$ jest ustaloną liczbą ze zbioru $S$.\\
Dla jakich $k$ prawdopodobieństwo zdarzenia $B_{k}$ jest największe?
  \item Dane są dwa punkty $A=(7,5), B=(1,-1)$ oraz punkt $P=(3,3)$ przecięcia wysokości trójkąta $A B C$. Oblicz pole trójkąta $A B C$ i napisz równanie okręgu opisanego na nim.
  \item Stożek i walec mają równe tworzące, równe objętości i równe pola powierzchni bocznej. Oblicz\\
a) sinus kąta nachylenia tworzącej stożka do jego podstawy,\\
b) stosunek pola przekroju osiowego walca do pola przekroju osiowego stożka.
\end{enumerate}

\end{document}