\documentclass[10pt]{article}
\usepackage[polish]{babel}
\usepackage[utf8]{inputenc}
\usepackage[T1]{fontenc}
\usepackage{amsmath}
\usepackage{amsfonts}
\usepackage{amssymb}
\usepackage[version=4]{mhchem}
\usepackage{stmaryrd}

\title{AKADEMIA GÓRNICZO-HUTNICZA \\
 im. Stanisława Staszica w Krakowie OLIMPIADA „O DIAMENTOWY INDEKS AGH" 2013/14 \\
 MATEMATYKA - ETAP II }

\author{}
\date{}


\begin{document}
\maketitle
\section*{ZADANIA PO 10 PUNKTÓW}
\begin{enumerate}
  \item Urządzenie I wykonuje pewną pracę w ciagu 20 godzin, a urządzenie II w ciągu 30 godzin. W jakim czasie wykonają tę pracę oba urządzenia pracując jednocześnie?
  \item Kotangens kąta rozwartego $\alpha$ jest równy -3 . Oblicz wartości funkcji trygonometrycznych kąta $2 \alpha$.
  \item Rozwiąż nierówność $\left|3 \log _{x} 2-2\right|>1$.
  \item Zbadaj monotoniczność ciągu $\left(a_{n}\right)$, którego $n$-ty wyraz jest równy
\end{enumerate}

$$
a_{n}=\frac{3^{n+2}}{3^{n}+2^{2 n+1}}
$$

Wyznacz granicę ciagu $\left(a_{n}\right)$.

\section*{ZADANIA PO 20 PUNKTÓW}
\begin{enumerate}
  \setcounter{enumi}{4}
  \item Okragg $O$ ma równanie $x^{2}+y^{2}+6 x+4 y-12=0$. Okrag $O^{\prime}$ jest obrazem okręgu $O$ przez translację o wektor $\vec{v}=[7,1]$. Znajdź równania osi symetrii sumy $O \cup O^{\prime}$ tych okręgów. Wyznacz punkty wspólne obu okręgów. Znajdź równania prostych stycznych jednocześnie do $O$ i $O^{\prime}$.
  \item Podstawą ostrostupa o wysokości $H$ jest trójkąt prostokątny $A B C$ o przyprostokątnych $|A B|=a$ i $|A C|=b$. Krawędź boczna wychodząca z wierzchołka $A$ jest prostopadła do podstawy. Ostrosłup ten podzielono płaszczyzną równoległą do podstawy na dwie bryły o równych objętościach. Oblicz pole powierzchni całkowitej tej bryły, która nie jest ostrosłupem.
  \item Do windy na parterze budynku czteropiętrowego wsiada osiem osób. Oblicz prawdopodobieństwa zdarzeń:\\
$A$ : wszyscy wysiądą na tym samym piętrze,\\
$B$ : na czwartym piętrze wysiądą co najmniej dwie osoby,\\
$C$ : na każdym piętrze wysiądą po dwie osoby.
\end{enumerate}

\end{document}