\documentclass[10pt]{article}
\usepackage[polish]{babel}
\usepackage[utf8]{inputenc}
\usepackage[T1]{fontenc}
\usepackage{amsmath}
\usepackage{amsfonts}
\usepackage{amssymb}
\usepackage[version=4]{mhchem}
\usepackage{stmaryrd}

\title{AKADEMIA GÓRNICZO-HUTNICZA \\
 im. Stanisława Staszica w Krakowie OGÓLNOPOLSKA OLIMPIADA „O DIAMENTOWY INDEKS AGH" 2019/20 MATEMATYKA - ETAP II }

\author{ZADANIA PO 10 PUNKTÓW}
\date{}


\begin{document}
\maketitle


\begin{enumerate}
  \item Niech $n$ będzie dowolną nieparzystą liczbą naturalną. Udowodnij, że suma $n$ kolejnych liczb całkowitych jest podzielna przez $n$.
  \item Dla jakich liczb $k$ trójmian kwadratowy
\end{enumerate}

$$
2\left(1-k^{2}\right) x^{2}+k\left(1+k^{2}\right) x+2 k
$$

jest podzielny przez dwumian $x+k$ ?\\
3. Rozwiąż równanie $\cos ^{2} 3 x-\sin ^{2} x=0$.\\
4. Do klasy, w której co czwarty uczeń jest jedynakiem, przyłączono drugą klasę o dwukrotnie mniejszej liczbie uczniów, wśród których jest $40 \%$ jedynaków. Jaki procent uczniów w nowo utworzonej klasie ma rodzeństwo?

\section*{ZADANIA PO 20 PUNKTÓW}
\begin{enumerate}
  \setcounter{enumi}{4}
  \item Ze zbioru $\{1,2, \ldots, 9\}$ losujemy jednocześnie dwie liczby. Czynność tę powtarzamy (zwróciwszy wylosowane liczby) dotąd, aż wylosujmy dwie liczby dające tę samą resztę z dzielenia przez 3. Jakie jest prawdopodobieństwo, że liczba losowań będzie\\
$A$ : mniejsza niż 10,\\
$B$ : równa 6,\\
$C$ : nieparzysta.
  \item Funkcja $f$ dla każdego jej argumentu $x$ spełnia równość
\end{enumerate}

$$
f(x)+(f(x))^{2}+\ldots=x^{3}
$$

której lewa strona jest sumą nieskończonego ciągu geometrycznego. Wyznacz dziedzinę funkcji $f$ oraz jej ekstrema lokalne.\\
7. W równoległobok $A B C D$, w którym kolejność wierzchołków $A B C D$ jest przeciwna do ruchu wskazówek zegara, można wpisać okrąg. Mając dane współrzędne wierzchołków $A=(0,1)$ i $B=(\sqrt{3}, 0)$ oraz miarę $120^{\circ}$ kąta wewnętrznego przy wierzchołku $D$, oblicz pole powierzchni równoległoboku i napisz równanie okręgu weń wpisanego.


\end{document}