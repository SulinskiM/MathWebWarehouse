\documentclass[10pt]{article}
\usepackage[polish]{babel}
\usepackage[utf8]{inputenc}
\usepackage[T1]{fontenc}
\usepackage{amsmath}
\usepackage{amsfonts}
\usepackage{amssymb}
\usepackage[version=4]{mhchem}
\usepackage{stmaryrd}

\title{Tematy I części egzaminu z matematyki }

\author{}
\date{}


\begin{document}
\maketitle
dla kandydatów ubiegających się o przyjęcie na I rok studiów dziennych.\\
Kandydat wybierał 3 dowolne zadania. Rozwiązania wybranych zadan oceniane były w skali \(0-10\) punktów. Egzamin trwał 120 minut.

\begin{enumerate}
  \item Zbadać przebieg zmienności funkcji
\end{enumerate}

\[
y=\frac{x^{2}+x+1}{x^{2}-x+1}
\]

sporządzić jej wykres i na tej podstawie ustalić ile pierwiastków posiada równanie

\[
\frac{x^{2}+x+1}{x^{2}-x+1}=m
\]

w zależności od parametru \(m\).\\
2. Dla jakich wartości parametru \(t\), przy dowolnej wartości parametru \(k\), równanie

\[
x^{2}+x \sqrt{k^{2}+4}-k \log _{\frac{1}{2}}(t+1)=0
\]

posiada dwa różne pierwiastki?\\
3. Rozwiązać nierówność

\[
\lim _{n \rightarrow \infty}\left(\sqrt{n^{2}+(2+\sin 2 x) n+4}-n\right)<1+\frac{1}{2} \cos 2 x
\]

\begin{enumerate}
  \setcounter{enumi}{3}
  \item Dwie kule o promieniach \(R\) i \(x(R>x)\) są styczne zewnętrznie. Przy jakim \(x\) objętość stożka opisanego na tych kulach będzie najmniejsza?
  \item W urnie \(U_{1}\) znajdują się dwie kule czarne i pewna ilość kul białych. W urnie \(U_{2}\) znajduje się 5 kul białych i 3 czarne. Z pierwszej urny losujemy dwie kule i przekładamy je do urny drugiej. Następnie z urny drugiej losujemy jedną kulę. Podać minimalną ilość białych kul znajdujących się w urnie \(U_{1}\), jeśli wiadomo, że prawdopodobieństwo wylosowania kuli białej z urny \(U_{2}\) jest większe od 0,6 .
\end{enumerate}

\section*{Tematy II części egzaminu z matematyki}
dla kandydatów ubiegających się o przyjęcie na I rok studiów dziennych.\\
Wszystkie zadania były oceniane w skali 0-2 punkty. Egzamin trwał 120 minut.

\begin{enumerate}
  \item Naszkicować wykres funkcji \(y=x|x+1|\).
  \item Obliczyć \(\cos ^{2} 105^{\circ}-\sin ^{2} 105^{\circ}\).
  \item Rozwiązać nierówność \(||x|-1|<2\).
  \item Obliczyć granicę \(\lim _{n \rightarrow \infty}\left(1-\frac{1}{2^{1}}+\frac{1}{2^{2}}-\frac{1}{2^{3}}+\ldots+(-1)^{n} \frac{1}{2^{n}}\right)\).
  \item Wektor \(\vec{a}=[3,7]\) przedstawić jako kombinację liniową wektorów \(\vec{e}_{1}=[2,3]\) i \(\vec{e}_{2}=[-1,1]\).
  \item Obliczyć granice \(\lim _{x \rightarrow 0} x \sin \frac{1}{x}\) i \(\lim _{x \rightarrow+\infty} x \sin \frac{1}{x}\).
  \item Dana jest funkcja \(f(x)=\log _{\frac{1}{3}}(x+1)\). Rozwiązać nierówność \(f(f(x))>0\).
  \item Rozwiązać równanie \(2^{2 x}+4^{x}=5^{x}\).
  \item Podać równanie jednej z prostych, na której leży środek okręgu opisanego na trójkącie o wierzchołkach \(A(1,3), B(2,7)\) i \(C(3,10)\).
  \item Dla jakich wartości parametru \(k\) funkcja \(f(x)=x^{3}-x^{2}+k x\) będzie rosnąca w całym zbiorze liczb rzeczywistych?
  \item Dane są zbiory
\end{enumerate}

\[
A=\left\{(x, y):(x-1)^{2}+y^{2} \leqslant 1\right\} \quad \text { oraz } \quad B=\{(x, y): y \geqslant x\}
\]

Naszkicować zbiór \(A \cap B\) i obliczyć jego pole.\\
12. W oparciu o definicję pochodnej obliczyć \(f^{\prime}(1)\) dla funkcji \(f(x)=\sqrt{x^{2}+3}\).\\
13. Zdarzenia losowe \(A\) i \(B\) są rozłączne i \(P(A)=\frac{1}{3}\), a \(P(B)=\frac{1}{2}\). Obliczyć \(P(A \cup B)\) oraz \(P(A-B)\).\\
14. Napisać równanie sycznej do krzywej \(y=x^{3}+x^{2}+x+1\) równoległej do prostej \(y=\frac{2}{3} x\).\\
15. Sformułować twierdzenie odwrotne do twierdzenia Pitagorasa.


\end{document}