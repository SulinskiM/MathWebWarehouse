\documentclass[10pt]{article}
\usepackage[polish]{babel}
\usepackage[utf8]{inputenc}
\usepackage[T1]{fontenc}
\usepackage{amsmath}
\usepackage{amsfonts}
\usepackage{amssymb}
\usepackage[version=4]{mhchem}
\usepackage{stmaryrd}

\begin{document}
Tematy I części egzaminu z matematyki\\
dla kandydatów ubiegających się o przyjęcie na I rok studiów dziennych.\\
Kandydat wybierał 3 dowolne zadania. Rozwiązania wybranych zadań oceniane były w skali 0-10 punktów. Egzamin trwał 120 minut.

\begin{enumerate}
  \item Rozwiązać układ nierówności
\end{enumerate}

\[
\left\{\begin{array}{l}
\sqrt{x+6}>x \\
2+\log _{0,5}(-x)>0
\end{array}\right.
\]

\begin{enumerate}
  \setcounter{enumi}{1}
  \item Dla jakich \(a\) równanie
\end{enumerate}

\[
\cos ^{4} x+(a+2) \sin ^{2} x-(2 a+5)=0
\]

ma rozwiązanie?\\
3. Wykazać, że pole trójkąta ograniczonego osiami układu współrzędnych i dowolną styczną do hiperboli \(y=\frac{a^{2}}{x}\) jest równe \(2 a^{2}\).\\
4. Wysokość stożka jest \(x\) razy większa od promienia jego podstawy. Wyrazić stosunek promieni kul opisanej i wpisanej w ten stożek jako funkcję \(f(x)\) oraz obliczyć granicę \(\lim _{x \rightarrow+\infty} \frac{f(x)}{x}\).\\
5. Dane są zbiory

\[
A=\{1,2,3, \ldots, 222\} \quad \text { i } \quad B=\{1,2,3, \ldots, 444\}
\]

Losowo wybieramy zbiór, a z niego liczbę \(x\). Obliczyć prawdopodobieństwo tego, że liczba \(x^{2}+1\) dzieli się przez 10.

\section*{Tematy II części egzaminu z matematyki}
dla kandydatów ubiegających się o przyjęcie na I rok studiów dziennych.\\
Wszystkie zadania były oceniane w skali 0-2 punkty. Egzamin trwał 120 minut.

\begin{enumerate}
  \item Wyznaczyć dziedzinę funkcji \(f(x)=\sqrt{\frac{5}{x+2}-1}\).
  \item Rozwiązać równanie \(\frac{\cos x}{1-\sin x}=1+\sin x\).
  \item Narysować wykres funkcji \(f(x)=x \sqrt{x^{2}}+\frac{x}{|x|}\).
  \item Na paraboli \(y=48-x^{2}\) znaleźć wszystkie punkty \((x, y)\) takie, że liczby \(3, x\), \(y\) tworzą ciąg geometryczny.
  \item Wyznaczyć dziedzinę funkcji \(f(x)=\log \left(3^{x}-5^{x}\right)\).
  \item Różniczkując tożsamość \(\sin 2 x=2 \sin x \cos x\) wykazać tożsamość \(\cos 2 x=\cos ^{2} x-\) \(\sin ^{2} x\).
  \item Obliczyć \(\lim _{x \rightarrow 0} \frac{x^{2}}{1-\cos 2 x}\).
  \item W trójkącie ostrokątnym \(A B C\) z wierzchołków \(A\) i \(C\) opuszczono wysokości \(A D\) i \(C E\) na boki \(B C\) i \(A B\). Wykazać, że trójkąty \(A B C\) i \(B D E\) są podobne.
  \item Suma pierwiastków trójmianu \(y=a x^{2}+b x+c\) jest równa \(\log _{a^{2}} c \cdot \log _{c^{2}} a\). Znaleźć odciętą wierzchołka paraboli.
  \item Dane są wektory \(\overrightarrow{A B}=[1,2,3]\) i \(\overrightarrow{A C}=[3,2,1]\). Obliczyć pole trójkąta \(A B C\).
  \item Proste \(\ell_{1}, \ell_{2}\) i \(\ell_{3}\) są równoległe i leżą w jednej płaszczyźnie. Na prostej \(\ell_{1}\) wybrano 3 punkty, na \(\ell_{2}\) wybrano 4 punkty, a na \(\ell_{3}\) wybrano 5 punktów. Ile co najwyżej istnieje trójkątów o wierzchołkach w tych punktach?
  \item Obliczyć \(\lim _{n \rightarrow \infty} \frac{1+4+7+\ldots+(3 n-2)}{2 n^{2}+3 n+4}\).
  \item Wykazać, że finkcja \(f(x)=\sqrt{1+x+x^{2}}-\sqrt{1-x+x^{2}}\) jest nieparzysta w swojej dziedzinie.
  \item Dany jest trójkąt o wierzchołkach \(A(1,-1), B(3,3)\) i \(C(-5,1)\). Napisać równanie symetralnej boku \(\overline{B C}\).
  \item Zbadać monotoniczność funkcji \(f(x)=x^{4}-\frac{1}{x}+5\) w przedziale \((0 ;+\infty)\).
\end{enumerate}

\end{document}