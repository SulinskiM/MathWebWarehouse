\documentclass[10pt]{article}
\usepackage[polish]{babel}
\usepackage[utf8]{inputenc}
\usepackage[T1]{fontenc}
\usepackage{amsmath}
\usepackage{amsfonts}
\usepackage{amssymb}
\usepackage[version=4]{mhchem}
\usepackage{stmaryrd}

\title{EGZAMIN WSTĘPNY Z MATEMATYKI }

\author{}
\date{}


\begin{document}
\maketitle
Zestaw składa się z 30 zadań. Zadania 1-10 oceniane będą w skali \(0-2\) punkty, zadania \(11-30\) w skali \(0-4\) punkty. Czas trwania egzaminu - 240 minut.

\section*{Powodzenia!}
\begin{enumerate}
  \item Rozwiązać nierówność \(|4+x-|3 x-2|| \leqslant 0\).
  \item Rozwiązać równanie \(2^{2 x+1}+3 \cdot 4^{x}=10\).
  \item Rozwiązać nierówność \(\frac{1}{x-1} \geqslant \frac{2}{x-2}\).
  \item Obliczyć \(1000^{\frac{1}{3}-\log } \sqrt[3]{3}\).
  \item Na osi \(0 y\) znaleźć punkt \(M\) równo oddalony od punktów \(A(2,-1,5)\) i \(B(-3,2,4)\).
  \item Wielomian \(w(x)=x^{4}+x^{2}+1\) rozłożyć na czynniki.
  \item Wyznaczyć \(n\) z równania \(1+5+9+\ldots+(4 n-3)=120\).
  \item Obliczyć granice \(\lim _{n \rightarrow \infty}\left(\log \left(10 n^{2}+1\right)-2 \log n\right)\).
  \item Obliczyć \(y^{\prime}\left(\frac{\pi}{4}\right)\), jeśli \(y(x)=\sqrt{1+\cos 2 x}\).
  \item Obliczyć stosunek objętości kuli opisanej na walcu do objętości kuli wpisanej w ten walec.
  \item Znaleźć składnik wymierny rozwinięcia dwumianu \((\sqrt[3]{2}+\sqrt[4]{3})^{10}\).
  \item Dla jakich parametrów \(\alpha\) równanie \(x^{2}+4 x \sin \alpha+1=0\) posiada co najmniej jeden pierwiastek rzeczywisty?
  \item Dla jakich wartości \(a\) i \(b\) liczba -1 jest pierwiastkiem podwójnym wielomianu \(w(x)=x^{3}+a x^{2}+b x-3 ?\)
  \item Rozwiązać nierówność \(\log _{2} x+\log _{x} 2 \geqslant 2\).
  \item Wiadomo, że zdarzenia losowe \(A\) i \(B\) są niezależne oraz \(P(A)=p_{1}\) i \(P(B)=p_{2}\). Obliczyć prawdopodobieństwa \(P(A \mid B)\) oraz \(P(A-B)\).
  \item Dane są funkcje \(f(x)=\sqrt{x}\) i \(g(x)=1-x\). Rozwiązać równanie \(f(g(x))\) \(=g(f(x))\).
  \item Dany jest ciąg geometryczny \(\left(a_{n}\right)\). Pokazać, że ciąg \(\left(b_{n}\right)\), gdzie \(b_{n}=a_{n+1}-a_{n}\), też jest ciągiem geometrycznym.
  \item Dwa punkty wyruszają jednocześnie z wierzchołka kąta o mierze \(120^{\circ}\) po jego ramionach z prędkościami odpowiednio \(5 \mathrm{~m} / \mathrm{s}\) i \(3 \mathrm{~m} / \mathrm{s}\). Po jakim czasie odległość między nimi będzie wynosiła 49 m ?
  \item Napisać równanie okręgu stycznego do obu osi układu współrzędnych i przechodzącego przez punkt \(P(2,1)\).
  \item Na podstawie definicji obliczyć pochodną funkcji \(f(x)=\cos 3 x\).
  \item Narysować wykres funkcji \(f(x)=2^{\log _{\frac{1}{2}} x}\).
  \item Wyznaczyć największą i najmniejszą wartość funkcji \(f(x)=x+\operatorname{ctg} x\) w przedziale \(\left\langle\frac{1}{4} \pi ; \frac{3}{4} \pi\right\rangle\).
  \item Z prawdopodobieństwem \(1 / 2 \mathrm{w}\) urnie znajduje się albo kula biała, albo czarna. Do urny dokładamy kulę białą i następnie losujemy jedną kulę. Jakie jest prawdopodobieństwo tego, że wylosujemy kulę białą?
  \item Udowodnić, że wszystkie trójkąty prostokątne, których boki tworzą ciąg arytmetyczny, są podobne.
  \item Wyznaczyć asymptoty funkcji \(y=\frac{\sqrt{x^{2}+x+1}}{x}\).
  \item Obliczyć \(\operatorname{tg} \alpha\), jeśli \(\sin \alpha-\cos \alpha=\frac{\sqrt{2}}{2}\) i \(\alpha \in\left(\frac{\pi}{4} ; \frac{\pi}{2}\right)\).
  \item Narysować na płaszczyźnie zbiór punktów, których współrzędne spełniają nierówność \(y^{2}+x y-2 x^{2}<0\).
  \item Obliczyć długości przekątnych równoległoboku zbudowanego na wektorach \(\vec{a}\) i \(\vec{b}\), jeżeli \(\vec{a}=2 \vec{m}-\vec{n}, \vec{b}=3 \vec{n}-\vec{m}\), gdzie wektory \(\vec{m}\) i \(\vec{n}\) są ortogonalne \(\mathrm{i}|\vec{m}|=|\vec{n}|=1\).
  \item Wykazać, że funkcja \(y=\sqrt{x^{3}-1}\) jest różnowartościowa w swojej dziedzinie. Następnie wyznaczyć funkcją do niej odwrotną.
  \item Wykazać, że jeśli ciąg \(\left(a_{n}\right)\) jest ograniczony i \(\lim _{n \rightarrow \infty} b_{n}=0\), to \(\lim _{n \rightarrow \infty} a_{n} \cdot b_{n}=0\).
\end{enumerate}

\end{document}