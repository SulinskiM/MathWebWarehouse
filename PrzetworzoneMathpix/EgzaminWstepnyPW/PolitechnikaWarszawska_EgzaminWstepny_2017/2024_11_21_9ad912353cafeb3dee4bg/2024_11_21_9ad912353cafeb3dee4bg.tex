\documentclass[10pt]{article}
\usepackage[polish]{babel}
\usepackage[utf8]{inputenc}
\usepackage[T1]{fontenc}
\usepackage{amsmath}
\usepackage{amsfonts}
\usepackage{amssymb}
\usepackage[version=4]{mhchem}
\usepackage{stmaryrd}

\title{Politechnika Warszawska }

\author{}
\date{}


\begin{document}
\maketitle
\section*{Egzamin wstępny z matematyki}
w dniu 29 czerwca 2017 r.

\begin{enumerate}
  \item Na płaszczyźnie dane są zbiory \(A=\left\{(x, y): x^{2}+y \leq 1\right\}\) oraz \(B=\{(x, y): y+|x| \leq 1\}\).\\
Naszkicować zbiory \(A, B, A \backslash B\).\\
15 punktów
  \item Wyznaczyć taką liczbę całkowitą dodatnią \(n\), że
\end{enumerate}

\[
2^{1} \cdot 2^{2} \cdot 2^{3} \cdot \ldots \cdot 2^{n}=(2 \sqrt{2})^{16-2 n} \cdot \sqrt{16^{8-n}}
\]

Odpowiedź uzasadnić.\\
15 punktów\\
3. Dwie sekretarki pracując wspólnie wykonały pewną pracę w ciągu 12 godzin. Gdyby pierwsza sekretarka wykonała sama \(1 / 3\) tej pracy, a następnie druga wykonała \(1 / 2\) tej pracy, to potrzebowałyby na to łącznie (dodając ich czas pracy) 20 godzin. W ciągu ilu godzin każda z nich pracując samodzielnie może wykonać tę pracę?

15 punktów\\
4. Rozwiązać równanie

\[
\log \left(2^{x}-4^{x}\right)-\log 8=\log \left(2^{x-1}-\frac{1}{4}\right)
\]

15 punktów\\
5. Rozwiązać nierówność

\[
\frac{\left|x^{2}-2\right|}{|x-1|-1} \geq 1
\]

\section*{20 punktów}
\begin{enumerate}
  \setcounter{enumi}{5}
  \item Wyznaczyć kąt pomiędzy równymi bokami trójkąta równoramiennego \(A B C\), jeśli wiadomo, że stosunek wysokości tego trójkąta opuszczonej na prostą zawierającą ramię trójkąta do wysokości opuszczonej na jego podstawę jest równy \(\sqrt{3}\).
\end{enumerate}

\section*{20 punktów}
Zadania należy rozwiązać na arkuszu egzaminacyjnym w polach oznaczonych odpowiednimi numerami zadań. Treści zadań prosimy nie przepisywać. Jeżeli w określonym polu zabraknie miejsca, zadanie można dokończyć na ostatniej stronie. Kartki brudnopisu nie oddaje się i nie będzie ona oceniana. Czas trwania egzaminu 120 minut.


\end{document}