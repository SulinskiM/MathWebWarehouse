\documentclass[10pt]{article}
\usepackage[polish]{babel}
\usepackage[utf8]{inputenc}
\usepackage[T1]{fontenc}
\usepackage{amsmath}
\usepackage{amsfonts}
\usepackage{amssymb}
\usepackage[version=4]{mhchem}
\usepackage{stmaryrd}

\title{V Konkurs matematyczny St@ś }

\author{}
\date{}


\begin{document}
\maketitle
XIV LO im. Stanisława Staszica\\
30 maja 2005 roku

\section*{klasa V}
Na rozwiazanie poniższych zadań masz 90 minut. Kolejność rozwiazywania tych zadań jest dowolna. Wszystkie zadania sa jednakowo punktowane. Maksymalna liczbe punktów może uzyskać jedynie petne rozwiazanie, z uzasadnieniem i odpowiedzia.\\
Używanie korektora i korzystanie z kalkulatora jest niedozwolone.\\
Zad. 1 Znajdź wszystkie takie ułamki zwykłe o mianowniku 15, aby były większe od \(\frac{8}{9}\) i mniejsze od 1.

Zad. 2 Dany jest prostokąt \(A B C D\), w którym \(|A B|=|C D|=7 \mathrm{~cm}\), zaś \(|B C|=|A D|=5 \mathrm{~cm}\). Punkty \(E\) oraz \(F\) należą odpowiednio do boków \(A D\) i \(D C\). Wiadomo, że \(|A E|=2 \mathrm{~cm}\) oraz \(|F C|=3 \mathrm{~cm}\). Oblicz pole trójkąta \(E B F\).

Zad. 3 Ułóż ułamki rosnąco:

\[
\frac{12}{13}, \frac{9}{10}, \frac{11}{12}, \frac{10}{11}
\]

Zad. 4 W pewnym trójkącie najmniejszy kąt jest dwa razy mniejszy od kąta średniego, który jest trzy razy mniejszy od kąta największego. Oblicz miarę najmniejszego kąta.

Zad. 5 Janek znalazł drewniany, prostopadłościenny klocek pomalowany na zielono. Następnie pociął go na jednakowe sześcienne kostki, wśród których 8 miało dokładnie jedną ścianę zieloną. Na ile kostek Janek podzielił ten klocek? Podaj wszystkie możliwe rozwiązania.


\end{document}