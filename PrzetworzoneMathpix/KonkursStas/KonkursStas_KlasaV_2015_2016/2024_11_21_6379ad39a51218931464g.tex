\documentclass[10pt]{article}
\usepackage[polish]{babel}
\usepackage[utf8]{inputenc}
\usepackage[T1]{fontenc}
\usepackage{amsmath}
\usepackage{amsfonts}
\usepackage{amssymb}
\usepackage[version=4]{mhchem}
\usepackage{stmaryrd}

\title{XVI Konkurs Matematyczny St@ś }

\author{}
\date{}


\begin{document}
\maketitle
\section*{XIV LO im. Stanisława Staszica}
30 maja 2016 roku

\section*{klasa V}
Na rozwiqzanie poniższych zadań masz 90 minut. Kolejność rozwiazywania tych zadań jest dowolna. Wszystkie zadania sa jednakowo punktowane. Maksymalna liczbę punktów może uzyskać jedynie petne rozwiazanie, z uzasadnieniem \(\boldsymbol{i}\) odpowiedzia.\\
Używanie korektora i korzystanie z kalkulatora jest niedozwolone.

\begin{enumerate}
  \item Wyznacz cyfry \(a\) i \(b\) tak, aby
\end{enumerate}

\[
\overline{a b 5}=5^{a+b}
\]

Uwaga: Liczba \(\overline{x y z}\) oznacza zapis liczby w układzie dziesiętnym. Liczba ta ma \(z\) jedności, \(y\) dziesiątek i \(x\) setek.\\
2. Dane są dwa kąty przyległe: \(\measuredangle A O B\) i \(\measuredangle B O C\). Półprosta \(O E\) jest dwusieczną \(\measuredangle A O B\) oraz \(O D \perp O E\). Sprawdź czy \(O D\) jest dwusieczną \(\measuredangle B O C\).\\
3. Na prostej dane są punkty \(A, B, C, D\) takie, że punkt \(B\) należy do odcinka \(A C\), punkt \(C\) należy do odcinka \(B D, A C=4\) oraz \(B D=5\). Wiedząc, że \(C\) jest środkiem odcinka \(B D\), oblicz długość odcinka \(A D\).\\
4. Staś kupił cztery książki. Trzy książki bez pierwszej kosztowały 45zł, trzy książki bez drugiej kosztowały 43zł, bez trzeciej 41zł a wszystkie bez czwartej 39zł. Ile kosztowała każda książka?\\
5. Jednakowym literom należy przyporządkować jednakowe cyfry, różnym różne. Wyznacz \(W, A, G, O, N, P, R\), aby działanie było poprawne.

WAGON\\
\(\frac{+ \text { WAGON }}{\text { POPRAW }}\)


\end{document}