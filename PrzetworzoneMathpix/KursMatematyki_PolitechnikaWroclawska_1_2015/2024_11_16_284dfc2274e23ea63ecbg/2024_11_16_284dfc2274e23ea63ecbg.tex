\documentclass[10pt]{article}
\usepackage[polish]{babel}
\usepackage[utf8]{inputenc}
\usepackage[T1]{fontenc}
\usepackage{amsmath}
\usepackage{amsfonts}
\usepackage{amssymb}
\usepackage[version=4]{mhchem}
\usepackage{stmaryrd}
\usepackage{hyperref}
\hypersetup{colorlinks=true, linkcolor=blue, filecolor=magenta, urlcolor=cyan,}
\urlstyle{same}

\title{PRACA KONTROLNA nr 1 - POZIOM PODSTAWOWY }

\author{}
\date{}


\begin{document}
\maketitle
\begin{enumerate}
  \item Dla pewnego kąta ostrego $\alpha$ zachodzi równość $\cos \alpha=2 \sin \alpha$. Wyznaczyć wartości wszystkich funkcji trygonometrycznych tego kąta.
  \item Po modernizacji linii kolejowej łączącej Wałbrzych z Wrocławiem średnia prędkość pociągu wzrosła o $14 \mathrm{~km} / \mathrm{h}$, a czas przejazdu 70 km skrócił się o 25 minut. Z jaką średnią prędkością jedzie teraz pociąg na tej linii?
  \item Wyznaczyć dziedzinę oraz najmniejszą wartość funkcji
\end{enumerate}

$$
f(x)=\frac{1}{\sqrt{10+8 x^{2}-x^{4}}} .
$$

\begin{enumerate}
  \setcounter{enumi}{3}
  \item Wyznaczyć wzory tych funkcji kwadratowych $f(x)=a x^{2}+b x+c$, dla których najmniejszą wartością jest $-\frac{9}{2}, f(0)=-4$, a jednym z miejsc zerowych jest $x=4$. Narysować wykresy tych funkcji.
  \item Uprościć wyrażenie (dla tych $a, b$, dla których ma ono sens)
\end{enumerate}

$$
\left(\frac{1}{b}+\frac{2}{\sqrt[6]{a^{2} b^{3}}}+\frac{1}{\sqrt[3]{a^{2}}}\right) \cdot\left(\sqrt[3]{a^{2}}(\sqrt[3]{a}+\sqrt{b})-\frac{a(2 \sqrt{b}+\sqrt[3]{a})}{\sqrt[3]{a}+\sqrt{b}}\right)
$$

Następnie obliczyć jego wartość dla $a=5 \sqrt{5}$ i $b=14-6 \sqrt{5}$.\\
6. Dane sa zbiory $A=\{(x, y): 4|x|-4 \leqslant 2|y| \leqslant|x|+2\}$ oraz $B=\left\{(x, y):|x|+|y| \leqslant \frac{5}{2}\right\}$. Obliczyć pole zbioru $A \cap B$. Wykonać staranny rysunek.

\section*{PRACA KONTROLNA nr 1 - POZIOM ROZSZERZONY}
\begin{enumerate}
  \item Wiedząc, że dla wypukłego kąta $\alpha$ zachodzi równość $\cos \alpha-\sin \alpha=\frac{1}{3}$, wyznaczyć wszystkie funkcje trygonometryczne tego kąta.
  \item Dla jakich wartości parametru $p$ suma kwadratów pierwiastków trójmianu $p x^{2}-2 p x+2$ jest większa od 3 ?
  \item Ciężarówka o długości 16 m jedzie ze stałą prędkością 70km/h. Wyprzedza ją samochód osobowy o długości 4 m jadąc ze stałą prędkością 100km/h. Manewr wyprzedzania rozpoczyna od zjazdu na lewy pas dokładnie 20 m za ciężarówką, a kończy, powracając na prawy pas jezdni dokładnie 20 m przed nią (odstęp między pojazdami wynosi w tych momentach 20m). Z naprzeciwka nadjeżdża inny samochód osobowy z prędkością $105 \mathrm{~km} / \mathrm{h}$. Jaka powinna być odległość między oboma samochodami osobowymi na początku manewru wyprzedzania, żeby zakończył się on bezpiecznie (bez zmiany prędkości obu samochodów)?
  \item Narysować wykres funkcji
\end{enumerate}

$$
f(x)=\left\{\begin{array}{lll}
\left|2^{-x}-2\right| & \text { dla } & x \leqslant 1 \\
\frac{x-4}{x-2} & \text { dla } & x>1
\end{array}\right.
$$

Posługując się nim, podać wzór funkcji $g(m)$ określającej liczbę rozwiązań równania $f(x)=m$, gdzie $m$ jest parametrem rzeczywistym.\\
5. Uprościć wyrażenie (dla tych $a, b$, dla których ma ono sens)

$$
\left(\frac{\sqrt[4]{a}}{\sqrt{b}}-\frac{b}{\sqrt{a}}+\frac{3 \sqrt{b}}{\sqrt[4]{a}}-3\right) \cdot\left(\sqrt[4]{a b^{2}}-b+\frac{2 b \sqrt[4]{a}-\sqrt{b^{3}}}{\sqrt[4]{a}-\sqrt{b}}\right) .
$$

Następnie obliczyć jego wartość dla $a=28-16 \sqrt{3}$ i $b=3$.\\
6. Dane są zbiory $A=\left\{(x, y): x^{2}+y^{2}<16\right\}$ oraz $B=\left\{(x, y): x^{2}+y^{2}<4| | x|-|y||\right\}$. Narysować zbiór $A \backslash B$ oraz obliczyć jego pole.

Rozwiązania (rękopis) zadań z wybranego poziomu prosimy nadsyłać do 28 września 2015r. na adres:

Katedra Matematyki WPPT\\
Politechniki Wrocławskiej\\
Wybrzeże Wyspiańskiego 27\\
50-370 WROCEAW.\\
Na kopercie prosimy koniecznie zaznaczyć wybrany poziom! (np. poziom podstawowy lub rozszerzony). Do rozwiązań należy dołączyć zaadresowaną do siebie kopertę zwrotną z naklejonym znaczkiem, odpowiednim do wagi listu. Prace niespełniające podanych warunków nie będą poprawiane ani odsyłane.

Adres internetowy Kursu: \href{http://www.im.pwr.wroc.pl/kurs}{http://www.im.pwr.wroc.pl/kurs}


\end{document}