\documentclass[10pt]{article}
\usepackage[polish]{babel}
\usepackage[utf8]{inputenc}
\usepackage[T1]{fontenc}
\usepackage{amsmath}
\usepackage{amsfonts}
\usepackage{amssymb}
\usepackage[version=4]{mhchem}
\usepackage{stmaryrd}
\usepackage{bbold}
\usepackage{hyperref}
\hypersetup{colorlinks=true, linkcolor=blue, filecolor=magenta, urlcolor=cyan,}
\urlstyle{same}

\title{PRACA KONTROLNA nr 1 - POZIOM PODSTAWOWY }

\author{}
\date{}


\begin{document}
\maketitle
\begin{enumerate}
  \item Uprościć następujące wyrażenie, określiwszy uprzednio jego dziedzinę:
\end{enumerate}

$$
\frac{1}{\sqrt[6]{a^{3} b^{2}}-\sqrt[6]{b^{5}}}\left(\sqrt[3]{a^{2}}-\frac{b}{\sqrt[3]{a}}\right)+\frac{1}{\sqrt{a}+\sqrt{b}}: \frac{\sqrt[3]{a b}}{a-b}
$$

Obliczyć wartość tego wyrażenia, przyjmując $a=3+2 \sqrt{2}$ i $b=1+\sqrt{2}$.\\
2. Niech $B$ oznacza dziedzinę funkcji $f(x)=\frac{1}{\sqrt{3+2 x-x^{2}}}$, a $A=\left\{x \in \mathbb{R}: \frac{1}{\left|x^{2}-1\right|} \geqslant 4\right\}$. Wyznaczyć i zaznaczyć na osi liczbowej zbiory $A, B, A \cap B, A \cup B$ oraz $(A \backslash B) \cup(B \backslash A)$.\\
3. Podać wzór funkcji kwadratowej, której wykres jest symetrycznym odbiciem wykresu funkcji $f(x)=x^{2}+2 x$ względem: a) prostej $x=1$, b) punktu $(0,0)$, c) punktu $(1,0)$. Odpowiedź uzasadnić, przeprowadzając odpowiednie obliczenia. Sporządzić staranne wykresy wszystkich rozważanych funkcji.\\
4. W pewnym ciągu arytmetycznym różnica piętnastego i drugiego wyrazu jest równa 13. Oblicz $a_{30}-a_{4}$ oraz sumę pierwszych dziesięciu wyrazów o numerach nieparzystych, wiedząc, że suma pierwszych dziesięciu wyrazów o numerach parzystych jest równa 125.\\
5. Przekątne trapezu prostokątnego o podstawach 3 i 4 przecinają się pod kątem prostym. Obliczyć obwód i pole trapezu. Sporządzić rysunek.\\
6. Ostrosłup prawidłowy, którego podstawą jest kwadrat o boku $a$, przecięto płaszczyzną przechodzącą przez wysokość ostrosłupa i przekątną podstawy. Pole otrzymanego przekroju jest równe polu podstawy. Wyznaczyć pole powierzchni całkowitej ostrosłupa oraz cosinus kąta nachylenia ściany bocznej do podstawy.

\section*{PRACA KONTROLNA nr 1 - POZIOM ROZSZERZONY}
\begin{enumerate}
  \item Uprościć następujące wyrażenie, określiwszy uprzednio jego dziedzinę:
\end{enumerate}

$$
\frac{\sqrt[3]{a}-\sqrt[3]{b}}{\sqrt[3]{a^{2}}+\sqrt[3]{a b}+\sqrt[3]{b^{2}}} \cdot \frac{a-b}{\sqrt[3]{a^{2}}-\sqrt[3]{b^{2}}} \cdot\left(1+\frac{\sqrt[3]{b}}{\sqrt[3]{a}-\sqrt[3]{b}}-\frac{1+\sqrt[3]{b}}{\sqrt[3]{b}}\right): \frac{\sqrt[3]{b}(1+\sqrt[3]{b})-\sqrt[3]{a}}{\sqrt[3]{b}}
$$

Obliczyć wartość tego wyrażenia dla $a=7+5 \sqrt{2}$ i $b=7-5 \sqrt{2}$.\\
2. Dla jakiego rzeczywistego parametru $m$ równanie

$$
\frac{m+1}{m x}-\frac{x}{m}=1+\frac{m}{x}
$$

ma dwa pierwiastki będące sinusem i cosinusem kąta z przedziału $\left(\frac{\pi}{2}, \pi\right)$ ?\\
3. Dane są liczby: $m=\frac{\binom{6}{4} \cdot\binom{8}{2}}{\binom{7}{3}}, n=\frac{(\sqrt{2})^{-4}\left(\frac{1}{4}\right)^{-\frac{5}{2}} \sqrt[4]{3}}{(\sqrt[4]{4})^{5} \cdot \sqrt{32} \cdot 27^{-\frac{1}{4}}}$. Wyznaczyć $k$ tak, by liczby $m, k, n$ były odpowiednio: pierwszym, drugim i trzecim wyrazem ciągu geometrycznego, a następnie wyznaczyć sumę wszystkich wyrazów nieskończonego ciągu geometrycznego, którego pierwszymi trzema wyrazami są $m, k, n$. Ile wyrazów tego ciągu należy wziąć, by ich suma przekroczyła $95 \%$ sumy wszystkich wyrazów?\\
4. Podać wzór funkcji homograficznej, której wykres jest symetrycznym odbiciem wykresu funkcji $f(x)=\frac{x-1}{x+1}$ względem: a) prostej $x=1$, b) punktu $(0,0)$, c) punktu $(1,0)$. Odpowiedź uzasadnić, przeprowadzając odpowiednie obliczenia. Sporządzić staranne wykresy wszystkich rozważanych funkcji.\\
5. W czworokącie wypukłym $A B C D$, w którym $A B=1, B C=2, C D=4, D A=3$, cosinus kąta przy wierzchołku $B$ jest równy $-\frac{5}{7}$. Wykazać, że czworokąt ten można wpisać w okrąg i obliczyć promień $R$ tego okręgu. Sprawdzić, czy w rozważany czworokąt można wpisać okrąg. Jeżeli tak, to obliczyć jego promień.\\
6. W ostrosłupie prawidłowym czworokątnym, w którym wszystkie krawędzie są równe, poprowadzono płaszczyznę przechodzącą przez jedną z krawędzi podstawy oraz środki dwu przeciwległych do niej krawędzi bocznych. Obliczyć stosunek pola powierzchni przekroju do pola podstawy oraz stosunek objętości brył, na jakie płaszczyzna podzieliła ostrosłup.

Rozwiązania (rękopis) zadań z wybranego poziomu prosimy nadsyłać do 28 września 2017 r. na adres:

Wydział Matematyki\\
Politechnika Wrocławska\\
Wybrzeże Wyspiańskiego 27\\
50-370 WROCEAW.\\
Na kopercie prosimy koniecznie zaznaczyć wybrany poziom! (np. poziom podstawowy lub rozszerzony). Do rozwiązań należy dołączyć zaadresowaną do siebie kopertę zwrotną z naklejonym znaczkiem, odpowiednim do wagi listu. Prace niespełniające podanych warunków nie będą poprawiane ani odsyłane.

Adres internetowy Kursu: \href{http://www.im.pwr.edu.pl/kurs}{http://www.im.pwr.edu.pl/kurs}


\end{document}