\documentclass[10pt]{article}
\usepackage[polish]{babel}
\usepackage[utf8]{inputenc}
\usepackage[T1]{fontenc}
\usepackage{amsmath}
\usepackage{amsfonts}
\usepackage{amssymb}
\usepackage[version=4]{mhchem}
\usepackage{stmaryrd}
\usepackage{bbold}
\usepackage{hyperref}
\hypersetup{colorlinks=true, linkcolor=blue, filecolor=magenta, urlcolor=cyan,}
\urlstyle{same}

\title{PRACA KONTROLNA nr 1 - POZIOM PODSTAWOWY }

\author{}
\date{}


\begin{document}
\maketitle
\begin{enumerate}
  \item Promień podstawy stożka obrotowego zmniejszono o $20 \%$. O ile procent trzeba zwiększyć wysokość tego stożka, żeby jego objętość nie uległa zmianie?
  \item Dla jakich wartości parametru $m$ nierówność
\end{enumerate}

$$
m x^{2}+(m+1) x+2 m<0
$$

jest spełniona dla wszystkich $x \in \mathbb{R}$ ?\\
3. Określić dziedzinę i uprościć następujące wyrażenie:

$$
\frac{\left(\sqrt[5]{a^{\frac{4}{3}}}\right)^{-\frac{3}{2}}}{\left(\sqrt{a \sqrt[3]{a^{2} b}}\right)^{4}}:\left[\frac{\sqrt[5]{a^{-4}}}{(\sqrt[4]{a \sqrt{b}})^{2}}\right]^{3}
$$

Następnie obliczyć wartość tego wyrażenia dla $a=\sqrt{3}+\sqrt{2}$ i $b=5-2 \sqrt{6}$.\\
4. Niech $f(x)=x^{2}$. Narysować wykres funkcji $g(x)=|f(x+1)-4|$ i określić liczbę rozwiązań równania $g(x)=m$ w zależności od parametru $m$.\\
5. Obliczyć pole koła wpisanego w romb o polu 10 i kącie ostrym $30^{\circ}$.\\
6. Niech $A=\left\{x \in \mathbb{R}: \frac{3}{2 x^{2}+x-6} \geqslant \frac{1}{2 x-3}\right\}$ oraz $B=\left\{x \in \mathbb{R}: \sqrt{x^{2}-4 x+4}<x\right\}$. Wyznaczyć i narysować na osi liczbowej zbiory $A, B$ oraz $A \backslash B, B \backslash A$.

\section*{PRACA KONTROLNA nr 1 - POZIOM ROZSZERZONY}
\begin{enumerate}
  \item Pewna liczba pięciocyfrowa zaczyna się (z lewej strony) cyfrą 8. Jeśli cyfrę tę przestawimy z pierwszej pozycji na ostatnią, to otrzymamy liczbę stanowiąca $16 \%$ liczby pierwotnej. Znaleźć tę liczbę.
  \item Określić dziedzinę i uprościć następujące wyrażenie:
\end{enumerate}

$$
\frac{(\sqrt{a}+\sqrt{b})^{2}-4 b}{(a-b) \cdot\left(\sqrt{\frac{1}{b}}+3 \sqrt{\frac{1}{a}}\right)^{-1}}: \frac{a+9 b+6 \sqrt{a b}}{\frac{1}{\sqrt{b}}+\frac{1}{\sqrt{a}}}
$$

Następnie wyznaczyć jego wartość dla $a=\sqrt{4-2 \sqrt{3}}$ i $b=\sqrt{3}+1$.\\
3. Narysować wykres funkcji $f(x)=\min \left\{\frac{2 x}{x-1}, x^{2}\right\}$. Podać wzór funkcji, której wykres jest symetryczny do wykresu funkcji $f(x)$ względem początku układu współrzędnych. Określić liczbę rozwiązań równania $f(x)=m$ w zależności od parametru $m$.\\
4. Długości boków trójkąta prostokątnego tworzą ciąg arytmetyczny o różnicy $p>0$. Obliczyć stosunek promienia okręgu opisanego na tym trójkącie do promienia okręgu wpisanego w ten trójkąt.\\
5. Dla jakich wartości parametru $m$ suma sześcianów pierwiastków równania

$$
x^{2}+(m-1) x+m=\frac{7}{4}
$$

należy do przedziału $\left[-\frac{1}{2}, 0\right)$ ?\\
6. Dane są zbiory

$$
A=\left\{(x, y) \in \mathbb{R}^{2}: 9-4 \sqrt{2} \leqslant x^{2}+y^{2}<9+4 \sqrt{2}\right\}
$$

oraz

$$
B=\left\{(x, y) \in \mathbb{R}^{2}: x^{2}+y^{2}<4|x|+4|y|-7\right\} .
$$

Narysować starannie zbiór $A \backslash B$ i wyznaczyć jego pole. Zadbać o odpowiednią skalę i czytelność rysunku.

Rozwiązania (rękopis) zadań z wybranego poziomu prosimy nadsyłać do 28 września 2018 r. na adres:

Wydział Matematyki\\
Politechnika Wrocławska\\
Wybrzeże Wyspiańskiego 27\\
50-370 WROCEAW.\\
Na kopercie prosimy koniecznie zaznaczyć wybrany poziom! (np. poziom podstawowy lub rozszerzony). Do rozwiązań należy dołączyć zaadresowaną do siebie kopertę zwrotną z naklejonym znaczkiem, odpowiednim do wagi listu. Prace niespełniające podanych warunków nie będą poprawiane ani odsyłane.\\
Uwaga. Wysyłając nam rozwiązania zadań uczestnik Kursu udostępnia nam swoje dane osobowe, które przetwarzamy wyłącznie w zakresie niezbędnym do jego prowadzenia (odesłanie zadań, prowadzenie statystyki). Szczegółowe informacje o przetwarzaniu przez nas danych osobowych są dostępne na stronie internetowej Kursu.

Adres internetowy Kursu: \href{http://www.im.pwr.edu.pl/kurs}{http://www.im.pwr.edu.pl/kurs}


\end{document}