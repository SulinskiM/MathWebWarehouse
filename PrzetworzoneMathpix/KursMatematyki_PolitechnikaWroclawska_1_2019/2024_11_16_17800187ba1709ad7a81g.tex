\documentclass[10pt]{article}
\usepackage[polish]{babel}
\usepackage[utf8]{inputenc}
\usepackage[T1]{fontenc}
\usepackage{amsmath}
\usepackage{amsfonts}
\usepackage{amssymb}
\usepackage[version=4]{mhchem}
\usepackage{stmaryrd}
\usepackage{bbold}
\usepackage{hyperref}
\hypersetup{colorlinks=true, linkcolor=blue, filecolor=magenta, urlcolor=cyan,}
\urlstyle{same}

\title{PRACA KONTROLNA nr 1 - POZIOM PODSTAWOWY }

\author{}
\date{}


\begin{document}
\maketitle
\begin{enumerate}
  \item Pan Kowalski założył dwie lokaty, wpłacając do banku w sumie 10120 zł. Pierwsza z nich ma oprocentowanie $12 \%$ w skali roku z półroczną kapitalizacją odsetek, a druga daje $18 \%$ zysku, przy czym odsetki są naliczane dopiero po roku. Okazało się, że na obu kontach przybyła mu taka sama kwota. Jakie sumy wpłacił na każdą z lokat i jaki osiągnął zysk? Jaki byłby zysk pana Kowalskiego, gdyby na każdą z lokat wpłacił tę samą sumę 5060 zł.?
  \item Niech $A=\left\{x \in \mathbb{R}: \frac{1}{\sqrt{5-x}} \geqslant \frac{2}{\sqrt{x+1}}\right\}$ oraz $B=\{x \in \mathbb{R}:|x|+|x-1| \geqslant 3\}$. Znaleźć i zaznaczyć na osi liczbowej zbiory $A, B$ oraz $(A \backslash B) \cup(B \backslash A)$.
  \item Uprościć wyrażenie (dla tych $a, b$, dla których ma ono sens)
\end{enumerate}

$$
\left(\frac{1}{b}+\frac{2}{\sqrt[6]{a^{2} b^{3}}}+\frac{1}{\sqrt[3]{a^{2}}}\right): \frac{\sqrt[3]{a}+\sqrt{b}}{b \sqrt[3]{a^{2}}}
$$

Następnie obliczyć jego wartość dla $a=5 \sqrt{5}$ i $b=14-6 \sqrt{5}$.\\
4. Odcinek $A B$ jest średnicą okręgu. Styczna w punkcie $A$ i prosta, na której leży cięciwa $B C$ przecinają się w punkcie $P$ odległym od $A$ o $4 \sqrt{3}$. Wyznaczyć promień okręgu oraz długość cięciwy $B C$, wiedząc, że pole trójkata $A B P$ jest równe $8 \sqrt{3}$.\\
5. Pole trójkąta równobocznego $A B X$ zbudowanego na przeciwprostokątnej $A B$ trójkąta prostokątnego $A B C$ jest dwa razy większe od pola wyjściowego trójkąta. Niech $D$ będzie środkiem boku $A B$. Wykazać, że trójkąty $A B C$ i $A D X$ są przystające.\\
6. Pole powierzchni bocznej stożka jest trzy razy większe niż pole jego podstawy. W stożek wpisano walec, którego dolna podstawa jest zawarta w podstawie stożka, a przekrój płaszczyzną zawierającą oś stożka jest kwadratem. Wyznaczyć stosunek objętości walca do objętości stożka.

\section*{PRACA KONTROLNA nr 1 - POZIOM ROZSZERZONY}
\begin{enumerate}
  \item Określić dziedzinę i uprościć następujące wyrażenie
\end{enumerate}

$$
\left[\frac{y \sqrt[3]{x}}{\sqrt[3]{x}+\sqrt{y}}-\frac{x-y \sqrt{y}}{x+y \sqrt{y}} \cdot \frac{y \sqrt[3]{x^{2}}-y \sqrt{y} \sqrt[3]{x}+y^{2}}{\sqrt[3]{x^{2}}-y}\right]: \frac{y^{2}}{\sqrt[3]{x}+\sqrt{y}}
$$

Następnie wyznaczyć jego wartość dla $x=6 \sqrt{3}-10$ i $y=12-6 \sqrt{3}$.\\
2. Wyznaczyć sinus kąta przy wierzchołku $C$ w trójkącie równoramiennym, w którym środkowe ramion $A C$ i $B C$ przecinają się pod kątem prostym.\\
3. Narysować obszar $D=\left\{(x, y):|y| \leqslant x \leqslant 4-y^{2}\right\}$. Obliczyć pole kwadratu, którego boki są równoległe do osi układu współrzędnych, a wszystkie wierzchołki leżą na krzywej ograniczającej obszar $D$.\\
4. W trójkącie $A B C$ dane są: $|B C|=a,|A B|=c, \angle A B C=\beta$. Okrąg przechodzący przez punkty $B$ i $C$ przecina boki $A B$ i $A C$ w takich punktach $D$ i $E$, że pole czworokąta $B C D E$ stanowi $75 \%$ pola trójkąta $A B C$. Wyznaczyć obwód i pole czworokąta.\\
5. Basen można napełnić, otwierając którykolwiek z trzech zaworów. Otwarcie pierwszych dwu pozwala napełnić basen w czasie o 2 godziny dłuższym niż otwarcie drugiego i trzeciego zaworu, natomiast otwarcie zaworów pierwszego i trzeciego pozwala napełnić basen w czasie dwa razy krótszym niż otwarcie dwu pierwszych. Napełnienie basenu, gdy otwarte są wszystkie trzy zawory, trwa 2 godziny 40 minut. Ile trwa napełnienie basenu, gdy otwarty jest tylko jeden zawór?\\
6. W ostrosłupie prawidłowym czworokątnym przekrój płaszczyzną przechodzącą przez wierzchołek ostrosłupa i środki dwu przeciwległych krawędzi podstawy jest trójkątem równobocznym. Ostrosłup przecięto płaszczyzną przechodzącą przez jedną z krawędzi podstawy prostopadłą do przeciwległej ściany bocznej. Obliczyć stosunek objętości brył, na jakie płaszczyzna ta dzieli ostrosłup.

Rozwiązania (rękopis) zadań z wybranego poziomu prosimy nadsyłać do 28 września 2019r. na adres:

\begin{verbatim}
Wydział Matematyki
Politechnika Wrocławska
Wybrzeże Wyspiańskiego 27
50-370 WROCEAW.
\end{verbatim}

Na kopercie prosimy koniecznie zaznaczyć wybrany poziom! (np. poziom podstawowy lub rozszerzony). Do rozwiązań należy dołączyć zaadresowaną do siebie kopertę zwrotną z naklejonym znaczkiem, odpowiednim do wagi listu. Prace niespełniające podanych warunków nie będą poprawiane ani odsyłane.

Uwaga. Wysyłając nam rozwiązania zadań uczestnik Kursu udostępnia Politechnice Wrocławskiej swoje dane osobowe, które przetwarzamy wyłącznie w zakresie niezbędnym do jego prowadzenia (odesłanie zadań, prowadzenie statystyki). Szczegółowe informacje o przetwarzaniu przez nas danych osobowych są dostępne na stronie internetowej Kursu.\\
Adres internetowy Kursu: \href{http://www.im.pwr.edu.pl/kurs}{http://www.im.pwr.edu.pl/kurs}


\end{document}