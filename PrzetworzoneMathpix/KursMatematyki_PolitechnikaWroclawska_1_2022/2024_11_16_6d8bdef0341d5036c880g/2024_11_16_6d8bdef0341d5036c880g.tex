\documentclass[10pt]{article}
\usepackage[polish]{babel}
\usepackage[utf8]{inputenc}
\usepackage[T1]{fontenc}
\usepackage{amsmath}
\usepackage{amsfonts}
\usepackage{amssymb}
\usepackage[version=4]{mhchem}
\usepackage{stmaryrd}
\usepackage{bbold}
\usepackage{hyperref}
\hypersetup{colorlinks=true, linkcolor=blue, filecolor=magenta, urlcolor=cyan,}
\urlstyle{same}

\title{PRACA KONTROLNA nr 1 - POZIOM PODSTAWOWY }

\author{}
\date{}


\begin{document}
\maketitle
\begin{enumerate}
  \item Uprość wyrażenie
\end{enumerate}

$$
\frac{x^{-1}-a^{-1}}{a^{-1}-b(a x)^{-1}}, \quad \text { jeśli } \quad x=\frac{1}{(a+b)^{-1}}-\left(\frac{a+b}{a^{2}+b^{2}}\right)^{-1}
$$

\begin{enumerate}
  \setcounter{enumi}{1}
  \item W jakim stosunku należy zmieszać dwa roztwory cukru o stężeniach $5 \%$ oraz $23 \%$, aby otrzymać roztwór $17 \%$ ?
  \item Rozwiąż nierówność
\end{enumerate}

$$
x-|5 x-2|<0
$$

\begin{enumerate}
  \setcounter{enumi}{3}
  \item Dla jakich wartości parametru a nierówność
\end{enumerate}

$$
\left(a^{2}-1\right) x^{2}+2(a-1) x+2>0
$$

jest spełniona dla każdego $x \in \mathbb{R}$ ?\\
5. Wiedząc, że 1 i 3 są pierwiastkami równania

$$
x^{3}+m x^{2}+23 x+n=0
$$

oblicz $m, n$ i wyznacz trzeci pierwiastek równania.\\
6. Narysuj wykres funkcji

$$
f(x)=\left\{\begin{array}{lll}
\left|x^{2}-x-2\right|+1 & \text { dla } & |2 x-2| \leqslant 4 \\
5+|x-3| & \text { dla } & |2 x-2|>4
\end{array}\right.
$$

Wykorzystując wykres, wyznacz zbiór wartości funkcji $f(x)$ oraz najmniejszą i największą wartość funkcji w przedziale $[0,4]$.

\section*{PRACA KONTROLNA nr 1 - POZIOM RoZsZERZONY}
\begin{enumerate}
  \item Dla jakich wartości parametru $a$ równanie
\end{enumerate}

$$
2 x^{2}-a x+a+2=0
$$

ma pierwiastki spełniające warunek $\left|x_{2}-x_{1}\right|=1$ ?\\
2. W sali ustawiono krzesła i trzyosobowe ławki, w łącznej liczbie 268. Do sali weszło 480 osób. Po zajęciu wszystkich miejsc siedzących proporcja osób stojących do siedzących okazała się większa niż $\frac{39}{160}$, ale mniejsza niż $\frac{41}{160}$. Ile ławek i ile krzeseł było w sali?\\
3. Rozwiąż nierówność

$$
|||||x|-1|-2|-1|-2| \leqslant 3
$$

\begin{enumerate}
  \setcounter{enumi}{3}
  \item Oblicz
\end{enumerate}

$$
x^{4}+y^{4}+z^{4}, \quad \text { jeśli } \quad x+y+z=0 \quad \text { oraz } \quad x^{2}+y^{2}+z^{2}=3 .
$$

\begin{enumerate}
  \setcounter{enumi}{4}
  \item Rozwiąż układ równań
\end{enumerate}

$$
\left\{\begin{aligned}
x-|y+1| & =1 \\
x^{2}+y & =10
\end{aligned}\right.
$$

Podaj jego interpretację geometryczną (narysuj starannie obie dane powyższymi równaniami krzywe).\\
6. Wyznacz wartości parametru $p$, dla których równanie

$$
(p-1) x^{4}-2(p+4) x^{2}+p=0
$$

ma cztery pierwiastki różne od 0 .

Rozwiązania (rękopis) zadań z wybranego poziomu prosimy nadsyłać do 28.09.2022r. na adres:

Wydział Matematyki\\
Politechnika Wrocławska\\
Wybrzeże Wyspiańskiego 27\\
50-370 WROCEAW.\\
lub elektronicznie, za pośrednictwem portalu \href{http://talent.pwr.edu.pl}{talent.pwr.edu.pl}\\
Na kopercie prosimy koniecznie zaznaczyć wybrany poziom! (np. poziom podstawowy lub rozszerzony). Do rozwiązań należy dołączyć zaadresowaną do siebie kopertę zwrotną z naklejonym znaczkiem, odpowiednim do formatu listu. Prace niespełniające podanych warunków nie będą poprawiane ani odsyłane.

Uwaga. Wysyłając nam rozwiązania zadań uczestnik Kursu udostępnia Politechnice Wrocławskiej swoje dane osobowe, które przetwarzamy wyłącznie w zakresie niezbędnym do jego prowadzenia (odesłanie zadań, prowadzenie statystyki). Szczegółowe informacje o przetwarzaniu przez nas danych osobowych są dostępne na stronie internetowej Kursu.

Adres internetowy Kursu: \href{http://www.im.pwr.edu.pl/kurs}{http://www.im.pwr.edu.pl/kurs}


\end{document}