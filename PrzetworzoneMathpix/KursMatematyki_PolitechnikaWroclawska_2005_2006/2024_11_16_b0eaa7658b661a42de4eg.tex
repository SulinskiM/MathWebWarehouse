\documentclass[10pt]{article}
\usepackage[polish]{babel}
\usepackage[utf8]{inputenc}
\usepackage[T1]{fontenc}
\usepackage{amsmath}
\usepackage{amsfonts}
\usepackage{amssymb}
\usepackage[version=4]{mhchem}
\usepackage{stmaryrd}

\title{XXXV \\
 KORESPONDENCYJNY KURS Z MATEMATYKI }

\author{}
\date{}


\begin{document}
\maketitle
\section*{PRACA KONTROLNA nr 1}
październik 2005r.

\begin{enumerate}
  \item Niech $f(x)=x^{2}+b x+5$. Wyznaczyć wszystkie wartości parametru $b$, dla których: a) wykres funkcji $f$ jest symetryczny względem prostej $x=2$, b) wierzchołek paraboli będącej wykresem funkcji $f$ leży na prostej $x+y+1=0$. Sporządzić staranny rysunek.
  \item Kilkoro dzieci dostało torebkę cukierków do równego podziału. Gdyby liczba dzieci była o 1 mniejsza, to każde z nich dostałoby o 2 cukierki więcej. Gdyby cukierków było dwa razy więcej, a dzieci o dwoje więcej, to każde dostałoby o 5 cukierków więcej. Ile było dzieci a ile cukierków?
  \item Babcia założyła swemu rocznemu wnukowi lokatę w wysokości 1000 zł oprocentowaną w wysokości $6 \%$ w skali roku z półroczną kapitalizacją odsetek i postanowiła co 6 miesięcy wpłacać na to konto 100 zł. Jaką sumę dostanie wnuczek w dniu swoich osiemnastych urodzin?
  \item Dane są wierzchołki $A(-3,2), C(4,2), D(0,4)$ trapezu równoramiennego $A B C D$, w któ$\operatorname{rym} \overline{A B} \| \overline{C D}$. Wyznaczyć współrzędne wierzchołka $B$ oraz pole trapezu. Sporządzić rysunek.
  \item Wyznaczyć stosunek długości przekątnych rombu wiedząc, że stosunek pola koła wpisanego w ten romb do pola rombu wynosi $\frac{\pi}{5}$.
  \item Podstawą prostopadłościanu jest prostokąt o dłuższym boku a. Przekątna prostopadłościanu tworzy z przekątnymi ścian bocznych kąty $\alpha$ oraz $2 \alpha$. Obliczyć objętość tego prostopadłościanu. Dla jakich kątów $\alpha$ zadanie ma rozwiązanie?
  \item Dla jakich wartości parametru $p$ funkcja
\end{enumerate}

$$
f(x)=\frac{x^{3}}{p x^{2}+p x+1}
$$

jest określona i rosnąca na całej prostej rzeczywistej?\\
8. Rozwiązać równanie

$$
\operatorname{ctg} x=2 \sqrt{3} \sin x
$$

\begin{enumerate}
  \setcounter{enumi}{8}
  \item Liczby $a_{1}=(\sqrt{2})^{\log _{\frac{1}{2}} 16}$ oraz $a_{2}=16^{-\log _{\sqrt[3]{2}} \sqrt[4]{2}}$ są odpowiednio pierwszym i drugim wyrazem pewnego ciągu geometrycznego. Rozwiązać nierówność
\end{enumerate}

$$
(\sqrt{x})^{\log ^{2} x-1} \geqslant 2 S
$$

gdzie $S$ oznacza sumę wszystkich wyrazów tego ciągu.

\section*{PRACA KONTROLNA nr 2}
\begin{enumerate}
  \item Stop zawiera $60 \%$ srebra próby 0,6 i $30 \%$ srebra próby 0,7 oraz 20 dkg srebra próby 0,8 . a) Ile srebra i jakiej próby należy dodać, by otrzymać $2,5 \mathrm{~kg}$ srebra próby 0,7 ?\\
b) Obliczyć próbę stopu, jakim należy zastąpić połowę danego stopu, by otrzymać stop o próbie 0,75 ?
  \item Wyznaczyć wszystkie punkty okręgu o środku $(0,0)$ i promieniu 5 , których iloczyn kwadratów współrzędnych jest najmniejszą wspólną wielokrotnością liczb 12 i 14. Obliczyć obwód wielokąta, którego wierzchołkami są znalezione punkty. Bez używania kalkulatora zbadać, czy jest on większy od 30.
  \item Dla jakich wartości $a$ i $b$ wielomian $W(x)=x^{4}-3 x^{3}+b x^{2}+a x+b$ jest podzielny przez trójmian kwadratowy $\left(x^{2}-1\right)$ ? Dla znalezionych wartości współczynników $a$ i $b$ rozwiązać nierówność $W(x) \leqslant 0$.
  \item Wykorzystując tożsamość trygonometryczną $\sin \alpha+\sin \beta=2 \sin \frac{\alpha+\beta}{2} \cos \frac{\alpha-\beta}{2}$ narysować staranny wykres funkcji $f(x)=|\sin x+\cos x|$. Korzystając z tego wykresu, wyznaczyć najmniejszą i największą wartość funkcji $f$ na przedziale $\left[-\frac{\pi}{2}, \pi\right]$. Wyznaczyć rozwiązania równania $f(x)=\frac{1}{\sqrt{2}}$ zawarte w tym przedziale.
  \item Pole powierzchni całkowitej stożka jest dwa razy większe od pola powierzchni kuli wpisanej w ten stożek. Znaleźć cosinus kąta nachylenia tworzącej stożka do podstawy.
  \item W trójkącie równoramiennym suma długości ramienia i promienia okręgu opisanego na tym trójkącie równa jest $m$ a wysokość trójkąta równa jest 2 . Wyznaczyć długość ramienia jako funkcję parametru $m$ oraz wartość $m$, dla której kąt przy wierzchołku trójkąta równy jest $120^{\circ}$ ? Dla jakich wartości $m$ zadanie ma rozwiązanie?
  \item Narysować zbiory $A=\left\{(x, y): x^{2}+2 x+y^{2} \leqslant 0\right\}, B=\left\{(x, y): x^{2}+2 y+y^{2} \leqslant 0\right\}$, $C=\left\{(x, y): x \leqslant 0, y \geqslant 0, x^{2}+y^{2} \leqslant 4\right\}$. Obliczyć pola figur $A \cap B, A \backslash B, C \backslash(A \cup B)$. Podać równania osi symetrii figury $A \cup B$.
  \item Rozwiązać nierówność $\frac{1}{\sqrt{4-x^{2}}} \leqslant \frac{1}{x-1}$.
  \item Wyznaczyć równania wszystkich prostych stycznych do wykresu funkcji $f(x)=\frac{8 x}{x^{2}+3}$, które są prostopadłe do prostej o równaniu $x+y=0$. Obliczyć pole równoległoboku, którego wierzchołkami są punkty wspólne tych stycznych z wykresem funkcji $f(x)$.
\end{enumerate}

\section*{PRACA KONTROLNA nr 3}
\begin{enumerate}
  \item Drogę z miasta $A$ do miasta $B$ rowerzysta pokonuje w ciągu 3 godzin. Po długotrwałych deszczach stan $\frac{3}{5}$ drogi pogorszył się na tyle, że na tym odcinku rowerzysta może jechać z prędkością o $4 \mathrm{~km} / \mathrm{h}$ mniejszą. By czas podróży z $A$ do $B$ nie uległ zmianie, zmuszony jest na pozostałym odcinku zwiększyć prędkość o $12 \mathrm{~km} / \mathrm{h}$. Jaka jest odległość z $A$ do $B$ i z jaką prędkością jeździł rowerzysta przed ulewami?
  \item Niech $f(x)=|4-|x-2||+1$. Sporządzić staranny wykres funkcji $f$ i posługując sę nim: a) wyznaczyć najmniejszą i największą wartość funkcji $f$ w przedziale [0,7], b) podać równanie osi symetrii wykresu funkcji $f$, c) wyznaczyć $a>0$ tak, aby pole figury ograniczonej osiami układu, wykresem funkcji $f$ oraz prostą $x=a$ było równe 32.
  \item Promień światła przechodzi przez punkt $A(1,1)$, odbija się od prostej o równaniu $y=$ $x-2$ (zgodnie z zasadą mówiącą, że kąt padania jest równy kątowi odbicia) i przechodzi przez punkt $B(4,6)$. Wyznaczyć współrzędne punktu odbicia $P$ oraz równania prostych, po których biegnie promień przed i po odbiciu.
  \item Na egzaminie uczeń wybiera losowo 4 pytania z zestawu egzaminacyjnego liczącego 40 pytań. Aby zdać egzamin należy poprawnie odpowiedzieć na co najmniej dwa pytania. Jakie jest prawdopodobieństwo zdania egzaminu przez ucznia znającego odpowiedzi na $40 \%$ pytań z zestawu egzaminacyjnego?
  \item W ciągu arytmetycznym $\left(a_{n}\right)$ mamy $a_{1}+a_{3}=3$ oraz $a_{1} a_{4}=1$. Dla jakich $n$ prawdziwa jest nierówność $a_{1}+a_{2}+a_{3}+\ldots+a_{n} \leqslant 93$ ?
  \item Trójkąt prostokątny o przyprostokątnych $a, b$ obracamy wokół środkowej najdłuższego boku. Obliczyć objętość otrzymanej bryły.
  \item Korzystając z zasady indukcji matematycznej wykazać, że dla każdej liczby naturalnej $n$ liczba $7^{n}-(-3)^{n}$ dzieli się przez 10.
  \item Dla jakich wartości parametru rzeczywistego $m$ równanie
\end{enumerate}

$$
2^{2 x}-2(m-1) 2^{x}+m^{2}-m-2=0
$$

ma dokładnie jeden pierwiastek rzeczywisty?\\
9. Wśród graniastosłupów prawidłowych sześciokątnych o danym polu powierzchni całkowitej $S=27 \sqrt{3} \mathrm{dm}^{2}$ wskazać graniastosłup o największej objętości. Podać objętość tego graniastosłupa z dokładnością do 1 ml .

\section*{PRACA KONTROLNA nr 4}
styczeń 2006r.

\begin{enumerate}
  \item Rozwiązać układ równań
\end{enumerate}

$$
\left\{\begin{array}{l}
x^{2}-y^{2}=2(x-y) \\
x^{3}+y^{3}=6-(x-y)
\end{array}\right.
$$

\begin{enumerate}
  \setcounter{enumi}{1}
  \item Dany jest punkt $P(3,2)$ oraz dwie proste $k$ i $l$ o równaniach odpowiednio: $x+y+4=0$ i $2 x-3 y-9=0$. Znaleźć taki punkt $Q$ na prostej $l$, aby środek odcinka $\overline{P Q}$ leżał na prostej $k$. Rozwiązanie zilustrować odpowiednim rysunkiem.
  \item Dla jakich wartości parametru rzeczywistego $a \neq 0$ pierwiastki wielomianu $w(x)=a^{2} x^{3}-$ $a^{2} x^{2}-\left(a^{2}+1\right) x+a^{2}-1$ są trzema pierwszymi wyrazami pewnego ciągu arytmetycznego? Dla każdego otrzymanego przypadku obliczyć czwarty wyraz ciągu.
  \item Znaleźć liczbę trzycyfrową wiedząc, że iloraz z dzielenia tej liczby przez sumę jej cyfr jest równy 48, a różnica szukanej liczby i liczby napisanej tymi samymi cyframi, ale w odwrotnym porządku wynosi 198.
  \item W okrąg wpisano trapez tak, że jedna z jego podstaw jest średnicą okręgu. Stosunek długości obwodu trapezu do sumy długości jego podstaw jest równy $\frac{3}{2}$. Obliczyć cosinus kąta ostrego w tym trapezie.
  \item Na ostrosłupie prawidłowym trójkątnym opisano stożek, a na tym stożku opisano kulę. Kąt przy wierzchołku przekroju osiowego stożka jest równy $\alpha$. Obliczyć stosunek objętości kuli do objętości ostrosłupa.
  \item Rozwiązać nierówność
\end{enumerate}

$$
3^{x+\frac{1}{2}}-2^{2 x+1}<4^{x}-5 \cdot 3^{x-\frac{1}{2}} .
$$

\begin{enumerate}
  \setcounter{enumi}{7}
  \item Zbadać przebieg zmienności i sporządzić staranny wykres funkcji $f(x)=\frac{4-x^{2}}{x^{2}-1}$. Następnie narysować wykres funkcji $k=g(m)$, gdzie $k$ jest liczbą pierwiastków równania $\left|\frac{4-x^{2}}{x^{2}-1}\right|=m$.
  \item Ze zbioru cyfr $\{0,1,2,3\}$ wylosowano dwie i odrzucono. Z otrzymanego zbioru wylosowano ze zwracaniem pięć cyfr. Jakie jest prawdopodobieństwo, że liczba utworzona z tych cyfr jest podzielna przez 3?
\end{enumerate}

\section*{PRACA KONTROLNA nr 5}
luty 2006r.

\begin{enumerate}
  \item Przyprostokątne trójkąta prostokątnego mają długości 6 i 8 cm . W trójkąt ten wpisano kwadrat tak, że dwa jego wierzchołki leżą na przeciwprostokątnej, a dwa pozostałe na przyprostokątnych. Obliczyć pola figur, na jakie brzeg kwadratu dzieli dany trójkąt.
  \item Niech $A$ będzie zbiorem tych punktów $x$ osi liczbowej, których suma odległości od punktów -1 i 5 jest mniejsza od 12 , a $B=\left\{x \in R: \sqrt{x^{2}-25}-x<1\right\}$. Znaleźć i zaznaczyć na osi liczbowej zbiory $A, B$ oraz $(A \backslash B) \cup(B \backslash A)$.
  \item Wykazać, że liczba $\quad x=\sqrt[3]{2 \sqrt{6}+4}-\sqrt[3]{2 \sqrt{6}-4}$ jest niewymierna.
\end{enumerate}

Wskazówka: obliczyć $x^{3}$.\\
4. Wyznaczyć zbiór wszystkich wartości parametru $m$, dla których równanie

$$
\cos x=\frac{3 m}{m^{2}-4}
$$

ma rozwiązanie w przedziale $\left[-\frac{\pi}{3}, \frac{\pi}{3}\right]$. Obliczyć ctg $x$ dla całkowitych $m$ z tego zbioru.\\
5. W ostrosłupie prawidłowym sześciokątnym przekrój o najmniejszym polu płaszczyzną zawierającą wysokość ostrosłupa jest trójkątem równobocznym o boku 2a. Obliczyć cosinus kąta dwuściennego między ścianami bocznymi tego ostrosłupa.\\
6. Dane jest półkole o średnicy $A B$ i promieniu długości $|A O|=r$. Na promieniu $A O$ jako na średnicy wewnątrz danego półkola zakreślono półokrąg. Na większym półokręgu obrano punkt $P$ i połączono go z punktami $A$ i $B$. Odcinek $A P$ przecina mniejszy półokrąg w punkcie $C$. Obliczyć długość odcinka $A P$, jeżeli wiadomo, że $|C P|+|P B|=1$. Przeprowadzić analizę dla jakich wartości $r$ zadanie ma rozwiązanie.\\
7. Zbadać monotoniczność ciągu $a_{n}=\frac{n-2}{\sqrt{n^{2}+1}}$. Obliczyć granicę tego ciągu, a następnie znaleźć wszystkie jego wyrazy odległe od granicy co najmniej o $\frac{1}{10}$.\\
8. Wykazać, że pole trójkąta ograniczonego styczną do wykresu funkcji $y=\frac{2 x-3}{x-2}$ i jego asymptotami jest stałe. Sporządzić rysunek.\\
9. Rozwiązać układ równań

$$
\left\{\begin{aligned}
\log _{(x-y)}[8(x+y)] & =-2 \\
(x+y)^{\log _{4}(x-y)} & =\frac{1}{2}
\end{aligned}\right.
$$


\end{document}