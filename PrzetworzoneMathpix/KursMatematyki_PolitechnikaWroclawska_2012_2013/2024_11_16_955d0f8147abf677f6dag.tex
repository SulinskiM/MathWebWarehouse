% This LaTeX document needs to be compiled with XeLaTeX.
\documentclass[10pt]{article}
\usepackage[utf8]{inputenc}
\usepackage{amsmath}
\usepackage{amsfonts}
\usepackage{amssymb}
\usepackage[version=4]{mhchem}
\usepackage{stmaryrd}
\usepackage{bbold}
\usepackage[fallback]{xeCJK}
\usepackage{polyglossia}
\usepackage{fontspec}
\setCJKmainfont{Noto Serif CJK TC}

\setmainlanguage{polish}
\setmainfont{CMU Serif}

\title{PRACA KONTROLNA nr 1 - POZIOM PODSTAWOWY }

\author{}
\date{}


\begin{document}
\maketitle
\begin{enumerate}
  \item Niech $A=\left\{x \in \mathbb{R}: \frac{1}{x^{2}+1} \geqslant \frac{1}{7-x}\right\}$ oraz $B=\{x \in \mathbb{R}:|x-2|+|x-7|<7\}$. Znaleźć i zaznaczyć na osi liczbowej zbiory $A, B$ oraz $(A \backslash B) \cup(B \backslash A)$.
  \item Liczba $p=\frac{(\sqrt[3]{54}-2)(9 \sqrt[3]{4}+6 \sqrt[3]{2}+4)-(2-\sqrt{3})^{3}}{\sqrt{3}+(1+\sqrt{3})^{2}}$ jest miejscem zerowym funkcji $f(x)=a x^{2}+b x+c$. Pole trójkąta, którego wierzchołkami są punkty przecięcia wykresu z osiami układu współrzędnych równe jest 20. Wyznaczyć współczynnik b oraz drugie miejsce zerowe tej funkcji wiedząc, że wykres funkcji jest symetryczny względem prostej $x=3$.
  \item Trapez o kątach przy podstawie $30^{\circ}$ oraz $45^{\circ}$ jest opisany na okręgu o promieniu $R$. Obliczyć stosunek pola koła do pola trapezu.
  \item Niech $f(x)=\left\{\begin{array}{ll}\frac{1}{x-1}, & \text { gdy } \\ |x-1| \geqslant 1, \\ x^{2}-x-1, & \text { gdy } \\ |x-1|<1 .\end{array} \quad\right.$ Obliczyć $f\left(\frac{1+\sqrt{3}}{2}\right)$ oraz $f\left(\frac{\pi+1}{\pi-2}\right)$. Narysować wykres funkcji $f$ i na jego podstawie podać zbiór wartości funkcji oraz rozwiązać nierówność $f(x) \geqslant-\frac{1}{2}$.
  \item Tangens kąta ostrego $\alpha$ równy jest $\frac{a}{7 b}$, gdzie
\end{enumerate}

$$
a=(\sqrt{2}+1)^{3}-(\sqrt{2}-1)^{3}, b=(\sqrt{\sqrt{2}+1}-\sqrt{\sqrt{2}-1})^{2} .
$$

Wyznaczyć wartości pozostałych funkcji trygonometrycznych tego kąta oraz kąta $2 \alpha$.\\
6. W trójkąt otrzymany w przekroju ostrosłupa prawidłowego czworokątnego płaszczyzną przechodzącą przez wysokość ostrosłupa i przekątną jego podstawy wpisano kwadrat, którego jeden bok jest zawarty w przekątnej podstawy. Pole kwadratu jest dwa razy mniejsze niż pole podstawy ostrosłupa. Obliczyć stosunek pola powierzchni bocznej ostrosłupa do pola jego podstawy oraz cosinus kąta między ścianami bocznymi.

\section*{PRACA KONTROLNA nr 1 - POZIOM ROZSZERZONY}
\begin{enumerate}
  \item Niech $A=\{(x, y): y \geqslant||x-2|-1|\}, B=\left\{(x, y): y+\sqrt{4 x-x^{2}-3} \leqslant 2\right\}$. Narysować na płaszczyźnie zbiór $A \cap B$ i obliczyć jego pole.
  \item Pole powierzchni bocznej ostrosłupa prawidłowego trójkątnego jest k razy większe niż pole jego podstawy. Obliczyć cosinus kąta nachylenia krawędzi bocznej ostrosłupa do płaszczyzny podstawy.
  \item Dane są liczby: $m=\frac{\binom{6}{4} \cdot\binom{8}{2}}{\binom{7}{3}}, n=\frac{(\sqrt{2})^{-4}\left(\frac{1}{4}\right)^{-\frac{5}{2}} \sqrt[4]{3}}{(\sqrt[4]{16})^{3} \cdot 27^{-\frac{1}{4}}}$. Wyznaczyć $k$ tak, by liczby $m, k, n$ były odpowiednio: pierwszym, drugim i trzecim wyrazem ciągu geometrycznego, a nstępnie wyznaczyć sumę wszystkich wyrazów nieskończonego ciągu geometrycznego, którego pierwszymi trzema wyrazami są $m, k, n$. Ile wyrazów tego ciągu należy wziąć, by ich suma przekroczyła $95 \%$ sumy wszystkich wyrazów?
  \item Narysować wykres funkcji $f(x)=\left\{\begin{array}{lll}\left|3^{x}-1\right| & \text { dla } & x \leqslant 1 \\ \frac{3-x}{x} & \text { dla } & x>1\end{array}\right.$. Posługując się nim podać wzór i narysować wykres funkcji $g(m)$ określającej liczbę rozwiązań równania $f(x)=m$, gdzie $m$ jest parametrem rzeczywistym.
  \item Obliczyć tangens kąta wypukłego $\alpha$ spełniającego warunek $\sin \alpha-\cos \alpha=2 \sqrt{6} \sin \alpha \cos \alpha$.
  \item W trójkącie równoramiennym $A B C$ o podstawie $A B$ ramię ma długość $b$, a kąt przy wierzchołku $C$ - miare $\gamma \cdot D$ jest takim punktem ramienia $B C$, że odcinek $A D$ dzieli pole trójkąta na połowę. Wyznaczyć promienie $\rho_{1}, \rho_{2}$ okręgów wpisanych w trójkąty $A B D$ i $A D C$. Dla jakiego kąta $\gamma$ promienie te są równe, a dla jakiego $\rho_{1}=2 \rho_{2}$ ?
\end{enumerate}

\section*{PRACA KONTROLNA nr 2 - POZIOM PODSTAWOWY}
\begin{enumerate}
  \item Firma budowlana podpisała umowę na modernizację odcinka autostrady o długości 21 km w określonym terminie. Ze względu na zbliżające się mistrzostwa świata w rzucie telefonem komórkowym postanowiono zrealizować zamówienie 10 dni wcześniej, co oznaczało konieczność zwiększenia średniej normy dziennej o 5\%. W jakim czasie firma zamierzała pierwotnie zrealizować to zamówienie?
  \item Pan Kowalski zaciągnął w banku kredyt w wysokości $4000 \mathrm{zł}$ oprocentowany na $16 \%$ w skali roku. Zgodnie z umową będzie go spłacał w czterech ratach co 3 miesiące, spłacając za każdym razem 1000zł oraz 4\% pozostałego zadłużenia. Ile złotych ostatecznie zwróci bankowi pan Kowalski?
  \item Ile jest czterocyfrowych liczb naturalnych:\\
a) podzielnych przez 2,3 lub przez 5 ?\\
b) podzielnych przez dokładnie dwie spośród powyższych liczb?
  \item Na paraboli $y=x^{2}-6 x+11$ znaleźć taki punkt $C$, że pole trójkąta o wierzchołkach $A=(0,3), B=(4,0), C$ jest najmniejsze.
  \item Przy prostoliniowej ulicy (oś Ox ) w punkcie $x=0$ zainstalowano parkomat. W punkcie $x=1$ można korzystać z bankomatu, a w punkcie $x=-2$ jest wejście do galerii handlowej. W którym punkcie $x$ ulicy należy zaparkować samochód, aby droga przebyta od samochodu do parkomatu i z powrotem (bilet parkingowy należy położyć za szybą pojazdu), następnie do bankomatu po pieniądze, stąd do galerii i na końcu z zakupami do samochodu, była najkrótsza? Jaka będzie odpowiedź, gdy wejście do galerii będzie w punkcie $x=2$ ? W obu przypadkach podać wzór i narysować wykres funkcji określającej drogę przebytą przez klienta domu handlowego w zależności od punktu zaparkowania samochodu.
  \item Wykonać działania i zapisać w najprostszej postaci wyrażenie
\end{enumerate}

$$
w(a, b)=\left(\frac{a}{a^{2}-a b+b^{2}}-\frac{a^{2}}{a^{3}+b^{3}}\right):\left(\frac{a^{3}-b^{3}}{a^{3}+b^{3}}-\frac{a^{2}+b^{2}}{a^{2}-b^{2}}\right) .
$$

Wykazać, że dla dowolnych $a<0$ zachodzi nierówność $w\left(-a, a^{-1}\right) \geqslant 1$, a dla dowolnych $a>0$ prawdziwa jest nierówność $w\left(-a, a^{-1}\right) \leqslant 1$.

\section*{PRACA KONTROLNA nr 2 - POZIOM ROZSZERZONY}
\begin{enumerate}
  \item Rozwiązać nierówność $\frac{1}{\sqrt{5+4 x-x^{2}}} \geqslant \frac{1}{|x|-2}$ i zbiór rozwiązań zaznaczyć na osi liczbowej.
  \item Dwaj rowerzyści wyjechali jednocześnie naprzeciw siebie z miast A i B odległych o 30 kilometrów. Minęli się po godzinie i nie zatrzymując się podążyli z tymi samymi prędkościami każdy w swoim kierunku. Rowerzysta, który wyjechał z A dotarł do B półtorej godziny wcześniej niż jego kolega jadący z B dotarł do A. Z jakimi prędkościami jechali rowerzyści?
  \item Pan Kowalski zaciągnął 31 grudnia pożyczkę 4000 złotych oprocentowaną w wysokości $16 \%$ w skali roku. Zobowiązał się spłacić ją w ciągu roku w czterech równych ratach płatnych 30. marca, 30. czerwca, 30. września i 30. grudnia. Oprocentowanie pożyczki liczy się od 1 stycznia, a odsetki od kredytu naliczane są w terminach płatności rat. Obliczyć wysokość tych rat w zaokrągleniu do pełnych groszy.
  \item Dla jakiego parametru $m$ równanie
\end{enumerate}

$$
2 x^{2}-(2 m+1) x+m^{2}-9 m+39=0
$$

ma dwa pierwiastki, z których jeden jest dwa razy większy niż drugi?\\
5. Ile jest liczb pięciocyfrowych podzielnych przez 9, które w rozwinięciu dziesiętnym mają: a) obie cyfry 1,2 i tylko te? b) obie cyfry 1,3 i tylko te? c) wszystkie cyfry $1,2,3$ i tylko te? Odpowiedź uzasadnić. W przypadku b) wypisać otrzymane liczby.\\
6. Z przystani A wyrusza z biegiem rzeki statek do przystani B, odległej od A o 140 km . Po upływie 1 godziny wyrusza za nim łódź motorowa, dopędza statek, po czym wraca do przystani A w tym samym momencie, w którym statek przybija do przystani B. Prędkość łodzi w wodzie stojącej jest półtora raza większa niż prędkość statku w wodzie stojącej. Wyznaczyć te prędkości wiedząc, że rzeka płynie z prędkością $4 \mathrm{~km} /$ godz.

\section*{PRACA KONTROLNA nr 3 - POZIOM PODSTAWOWY}
\begin{enumerate}
  \item Z danych Głównego Urzędu Statystycznego wynika, że wzrost Produktu Krajowego Brutto (PKB) w Polsce w roku 2010 wyniósł $3,7 \%$, a w roku 2011 - 4,3\%. Jaki powinien być wzrost PKB w roku 2012, by średni roczny wzrost PKB w tych trzech latach wyniósł $4 \%$ ? Podać wynik z dokładnością do $0,001 \%$.
  \item Czy liczby $\sqrt{2}, 2,2 \sqrt{2}$ mogą być wyrazami (niekoniecznie kolejnymi) ciągu arytmetycznego? Odpowiedź uzasadnić.
  \item Wielomian $W(x)=x^{5}+a x^{4}+b x^{3}+4 x$ jest podzielny przez $\left(x^{2}-1\right)$. Wyznaczyć współczynniki $a, b$ i rozwiązać nierówność $W(x-1) \leqslant W(x) \leqslant W(x+1)$.
  \item Niech $f(x)=\sqrt{x}, g(x)=x-2, h(x)=|x|$. Narysować wykresy funkcji złożonych: $f \circ h \circ g$, $f \circ g \circ h, g \circ f \circ h, g \circ h \circ f, h \circ f \circ g$ oraz $h \circ g \circ f$.
  \item Przyprostokątną trójkąta prostokątnego $A B C$ jest odcinek $A B$ o końcach $A(-2,2)$ i $B(1,-1)$, a wierzchołek $C$ trójkąta leży na prostej $3 x-y=14$. Wyznaczyć równanie okręgu opisanego na tym trójkącie. Ile rozwiązań ma to zadanie? Sporządzić rysunek.
  \item Na prostej $x+2 y=5$ wyznaczyć punkty, z których okrąg $(x-1)^{2}+(y-1)^{2}=1$ jest widoczny pod kątem $60^{\circ}$. Obliczyć pole obszaru ograniczonego łukiem okręgu i stycznymi do niego poprowadzonymi w znalezionych punktach. Sporządzić rysunek.
\end{enumerate}

\section*{PRACA KONTROLNA nr 3 - POZIOM ROZSZERZONY}
\begin{enumerate}
  \item Pan Kowalski umieścił swoje oszczędności na dwu różnych lokatach. Pieniądze, otrzymane jako honorarium za podręcznik, złożył na lokacie oprocentowanej w wysokości $7 \%$ w skali roku, a wynagrodzenie za cykl wykładów - na lokacie $9 \%$. Po roku jego dochód był o 30 złotych, a po dwu latach - o 70 złotych wyższy od dochodu, który uzyskałby składając całą sumę na lokacie $8 \%$. Ile pieniędzy otrzymał pan Kowalski za podręcznik, a ile za wykłady?
  \item Czy liczby $\sqrt{2}, \sqrt{3}, 2$ mogą być wyrazami (niekoniecznie kolejnymi) ciągu arytmetycznego? Odpowiedź uzasadnić.
  \item Niech $f(x)=2^{x}, g(x)=2-x, h(x)=|x|$. Narysować wykresy funkcji złożonych $f \circ g \circ h$ oraz $g \circ f \circ h$ i rozwiązać nierówność $(f \circ g \circ h)(x)<6+(g \circ f \circ h)(x)$.
  \item Dane są punkty $A(1,2), B(3,1)$. Wyznaczyć równanie zbioru wszystkich punktów $C$ takich, że kąt $B C A$ ma miarę $45^{\circ}$.
  \item Liczby: $a_{1}=\log _{(3-2 \sqrt{2})^{2}}(\sqrt{2}-1), \quad a_{2}=\frac{1}{2} \log _{\frac{1}{3}} \frac{\sqrt{3}}{6}, \quad a_{3}=3^{\log _{\sqrt{3}} \frac{\sqrt{6}}{2}}, \quad a_{4}=\log _{(\sqrt{2}-1)}(\sqrt{2}+1)$, $a_{5}=\left(2^{\sqrt{2}+1}\right)^{\sqrt{2}-1}, \quad a_{6}=\log _{3} 2$ są jedynymi pierwiastkami wielomianu $W(x)$, którego wyraz wolny jest dodatni.\\
a) Które z tych pierwiastków są niewymierne? Odpowiedź uzasadnić.\\
b) Wyznaczyć dziedzinę funkcji $f(x)=\sqrt{W(x)}$, nie wykonując obliczeń przybliżonych.
  \item Niech $f(x)=3(x+2)^{4}+x^{2}+4 x+p$, gdzie $p$ jest parametrem rzeczywistym.\\
a) Uzasadnić, że wykres funkcji $f(x)$ jest symetryczny względem prostej $x=-2$.\\
b) Dla jakiego parametru $p$ najmniejszą wartością funkcji $f(x)$ jest $y=-2$ ? Odpowiedź uzasadnić, nie stosując metod rachunku różniczkowego.\\
c) Określić liczbę rozwiązań równania $f(x)=0 \mathrm{w}$ zależności od parametru $p$.
\end{enumerate}

\section*{PRACA KONTROLNA nr 4 - POZIOM PODSTAWOWY}
\begin{enumerate}
  \item Wyznaczyć wszystkie kąty $\alpha$ z przedziału $[0,2 \pi]$, dla których suma kwadratów pierwiastków rzeczywistych równania $x^{2}+2 x \sin \alpha-\cos ^{2} \alpha=0$ jest równa co najwyżej 3 .
  \item Uzasadnić, że suma średnic okręgu opisanego na trójkącie prostokątnym i okręgu wpisanego w ten trójkąt jest równa sumie długości przyprostokątnych. Znaleźć długości boków trójkąta, jeżeli promienie tych okręgów są równe $R=5$ i $r=2$.
  \item Narysować wykres funkcji $f(x)=\cos ^{2} x+|\sin x| \sin x$ w przedziale $[-2 \pi, 2 \pi]$.\\
a) Podać zbiór wartości i miejsca zerowe.\\
b) Wyznaczyć przedziały monotoniczności.\\
c) Rozwiązać nierówność $|f(x)| \geqslant \frac{1}{2}$.
  \item W kwadracie o boku długości a narysowano cztery półkola, których średnicami są boki kwadratu. Półkola przecinają się parami tworząc czterolistną rozetę. Obliczyć pole i obwód rozety.
  \item Dach wieży kościoła ma kształt ostrosłupa, którego podstawą jest sześciokąt foremny o boku 4 m a największy z przekrojów płaszczyzną zawierającą wysokość jest trójkątem równobocznym. Obliczyć kubaturę dachu wieży kościoła. Ile 2 -litrowych puszek farby antykorozyjnej trzeba kupić do pomalowania blachy, którą pokryty jest dach, jeżeli wiadomo, że 1 litr farby wystarcza do pomalowania $6 \mathrm{~m}^{2}$ blachy i trzeba uwzględnić $8 \%$ farby na ewentualne straty.
  \item Promień kuli opisanej na ostrosłupie prawidłowym czworokątnym wynosi $R$. Prostopadła wyprowadzona ze środka kuli do ściany bocznej ostrosłupa tworzy z wysokością ostrosłupa kąt $\alpha$. Wyznaczyć wysokość ostrosłupa.
\end{enumerate}

\section*{PRACA KONTROLNA nr 4 - POZIOM ROZSZERZONY}
\begin{enumerate}
  \item Dla jakich kątów $\alpha$ z przedziału $\left[0, \frac{\pi}{2}\right]$ równanie $x^{2} \sin \alpha+x+\cos \alpha=0$ ma dwa różne pierwiastki rzeczywiste? Czy iloczyn pierwiastków równania może być równy $\sqrt{3}$ ? Wyznaczyć wszystkie kąty $\alpha$, dla których suma pierwiastków jest większa od -2 .
  \item Przekrój ostrosłupa prawidłowego czworokątnego płaszczyzną przechodzącą przez przekątną podstawy i wierzchołek ostrosłupa jest trójkątem równobocznym. Wyznaczyć stosunek promienia kuli wpisanej w ostrosłup do promienia kuli opisanej na ostrosłupie.
  \item Narysować wykres funkcji $f(x)=\frac{\sin 2 x-|\sin x|}{\sin x}$. W przedziale $[0,2 \pi]$ rozwiązać nierówność $f(x)<2(\sqrt{2}-1) \cos ^{2} x$.
  \item Czworokąt wypukły $A B C D$, w którym $A B=1, B C=2, C D=4, D A=3$ jest wpisany w okrąg. Obliczyć promień $R$ tego okręgu. Sprawdzić, czy w ten czworokąt można wpisać okrąg. Jeżeli tak, to obliczyć jego promien.
  \item W kole $K$ o promieniu 4 cm narysowano 6 kół o promieniu 2 cm przechodzących przez środek koła $K$ i stycznych do niego tak, aby środki tych sześciu kół były wierzchołkami sześciokąta foremnego. Obliczyć pole i obwód figury, która jest sumą tych sześciu kół.
  \item Stosunek pola powierzchni bocznej stożka ściętego do pola powierzchni wpisanej w ten stożek kuli wyrazić jako funkcję kąta nachylenia tworzącej stożka do podstawy.
\end{enumerate}

\section*{PRACA KONTROLNA nr 5 - POZIOM PODSTAWOWY}
\begin{enumerate}
  \item Między każde dwa kolejne wyrazy pięcioelementowego ciągu arytmetycznego wstawiono $m$ liczb, otrzymując ciąg arytmetyczny, którego suma jest 13 razy większa niż suma wyjściowego ciągu. Obliczyć $m$. Jaką jednakową ilość liczb należy wstawić między każde dwa kolejne wyrazy $n$ elementowego ciągu arytmetycznego, aby otrzymać ciąg arytmetyczny o sumie $n$ razy większej niż suma wyjściowego ciągu?
  \item Linie kolejowe malują wagony klasy standard na niebiesko, klasy komfort na różowo, a klasy biznes na szaro. Na ile sposobów można zestawić skład pięciowagonowy, który zawiera co najmniej jeden wagon każdej klasy, a kolejność wagonów jest istotna?
  \item Niech $n$ będzie liczbą naturalną. W przedziale $[0,2 \pi]$ rozwiązać równanie
\end{enumerate}

$$
1+\cos ^{2} x+\cos ^{4} x+\cdots+\cos ^{2 n} x=2-\cos ^{2 n} x
$$

\begin{enumerate}
  \setcounter{enumi}{3}
  \item Zawodnik przebiegł równym tempem pierwsze 10 km biegu maratońskiego ( 42 km ) w czasie 45 minut, a każdy kolejny kilometr pokonywał w czasie o $5 \%$ dłuższym niż poprzedni. Sprawdzić, czy zawodnik zmieścił się w sześciogodzinnym limicie czasowym.
  \item Rozwiązać nierówność
\end{enumerate}

$$
\log _{2}(x+2)-\log _{4}\left(4-x^{2}\right) \geqslant 0
$$

\begin{enumerate}
  \setcounter{enumi}{5}
  \item Niech $A=\{(x, y):|x|+2|y| \leqslant 2\}$. Zbiór $B$ powstaje przez obrót figury $A$ o kąt $\frac{\pi}{2}$ ( w kierunku przeciwnym do ruchu wskazówek zegara) wokół początku układu współrzędnych. Starannie narysować zbiory $A \cup B$ oraz $A \triangle B=(A \backslash B) \cup(B \backslash A)$ i obliczyć ich pola.
\end{enumerate}

\section*{PRACA KONTROLNA nr 5 - POZIOM ROZSZERZONY}
\begin{enumerate}
  \item Zbadać, dla jakich argumentów funkcja $g(x)=2^{x^{3}-5} \cdot 3^{7 x^{2}} \cdot 4^{7 x-1}-2^{7 x^{2}+1} \cdot 3^{x^{3}-2} \cdot 9^{7 x-3}$ przyjmuje wartości ujemne.
  \item Rozwiązać nierówność
\end{enumerate}

$$
2^{-\sin x}+2^{-2 \sin x}+2^{-3 \sin x}+\ldots \leqslant \sqrt{2}+1
$$

której lewa strona jest sumą nieskończonego ciągu geometrycznego.\\
3. Podać dziedzinę i wyznaczyć wszystkie miejsca zerowe funkcji

$$
f(x)=\log _{x+1}(x-1)-\log _{x+1}\left(2 x-\frac{2}{x}\right)+1 .
$$

\begin{enumerate}
  \setcounter{enumi}{3}
  \item Dany jest ciąg liczbowy $\left(a_{n}\right)$, w którym każdy wyraz jest sumą podwojonego wyrazu poprzedniego i 4 , a jego czwarty wyraz wynosi 36 . Podać wzór na $n$-ty wyraz ciągu i udowodnić go, wykorzystując zasadę indukcji matematycznej.
  \item Niech $A=\{(x, y):|x|+2|y| \leqslant 2\}$. Zbiór $B$ otrzymano przez obrót $A$ o kąt $\frac{\pi}{2}$ (w kierunku przeciwnym do ruchu wskazówek zegara) wokół początku układu współrzędnych, a zbiór $C$ - przez obrót zbioru $A \cup B$ o kąt $\frac{\pi}{4}$ wokół początku układu współrzędnych. Wykonać staranny rysunek zbioru $A \cup B \cup C$ oraz obliczyć jego pole.
  \item Boki $\triangle A B C$ zawarte są w prostych $y=2 x+m, y=m x+1$ oraz $2 y=2-x$. Podać wartość rzeczywistego parametru $m \in\left(-\frac{1}{2}, 2\right)$, dla której pole rozważanego trójkąta wynosi $\frac{1}{5}$. Dla wyznaczonego $m$ wykonać staranny rysunek (przyjąć jednostkę równą 3 cm ).
\end{enumerate}

\section*{PRACA KONTROLNA nr 6 - POZIOM PODSTAWOWY}
\begin{enumerate}
  \item Rozwiązać równanie
\end{enumerate}

$$
\sqrt{2^{2 x+1}-5 \cdot 2^{x}+4}=2^{x+2}-5 .
$$

\begin{enumerate}
  \setcounter{enumi}{1}
  \item Spośród cyfr liczby 211521125112 wylosowano trzy (bez zwracania). Obliczyć prawdopodobieństwo tego, że liczba utworzona z wylosowanych cyfr nie jest podzielna przez trzy.
  \item Wyznaczyć dziedzinę funkcji
\end{enumerate}

$$
f(x)=\sqrt{-\log _{2} \frac{3 x}{x^{2}-4}} .
$$

\begin{enumerate}
  \setcounter{enumi}{3}
  \item 20 uczniów posadzono losowo w sali zawierającej 4 rzędy po 5 krzeseł w każdym. Obliczyć prawdopodobieństwo tego, że Bolek będzie siedział przy Lolku, tzn. z przodu, z tyłu, z prawej albo z lewej jego strony.
  \item Uzasadnić, że dla dowolnego $p$ oraz $x>-1$ prawdziwa jest nierówność
\end{enumerate}

$$
p^{2}+(1-p)^{2} x \geqslant \frac{x}{1+x}
$$

Znaleźć i narysować na płaszczyźnie zbiorów wszystkich par $(p, x)$, dla których w powyższej nierówności ma miejsce równość.\\
6. Trapez równoramienny $A B C D$ o polu $P$, ramieniu $c$ i kącie ostrym przy podstawie $\alpha$ zgięto wzdłuż jego osi symetrii $E F$ tak, że obie połowy utworzyły kąt $\alpha$. Obliczyć objętość powstałego w ten sposób wielościanu $A B C D E F$. Obliczyć tangens kąta nachylenia do podstawy tej ściany bocznej, która nie jest prostopadła do podstawy. Sporządzić odpowiednie rysunki. Podać warunki istnienia rozwiązania.

\section*{PRACA KONTROLNA nr 6 - POZIOM RoZsZERzony}
\begin{enumerate}
  \item Rozwiązać równanie
\end{enumerate}

$$
\sqrt{x^{2}-3}+2 \sqrt{5-2 x}=5-x
$$

\begin{enumerate}
  \setcounter{enumi}{1}
  \item Wybrano losowo trzy krawędzie sześcianu. Obliczyć prawdopodobieństwo tego, że żadne dwie nie mają punktów wspólnych.
  \item Gra w pary. W skarbonce znajduje się duża liczba monet o nominałach $1 \mathrm{zł}, 2 \mathrm{zł} \mathrm{i}$ 5 zł. W pierwszym kroku Jaś losuje trzy monety. Jesli wśród nich są dwie jednakowe, to wrzuca je do skarbonki. W kolejnych krokach losuje ze skarbonki każdorazowo tyle monet, ile trzyma w ręce, a następnie pary jednakowych monet wrzuca do skarbonki. Gra kończy się, gdy wrzuci do skarbonki wszystkie monety. Obliczyć prawdopodobieństwo tego, że Jaś skończy grę: a) w drugim kroku; b) w drugim lub trzecim kroku.
  \item Dane są wierzchołki $A(-3,2), C(4,2), D(0,4)$ trapezu równoramiennego $A B C D$, w którym $A B \| C D$. Wyznaczyć współrzędne wierzchołka $B$ oraz równanie okręgu opisanego na trapezie.
  \item Udowodnić, że dla $x>-1$ prawdziwa jest nierówność podwójna
\end{enumerate}

$$
1+\frac{x}{2}-\frac{x^{2}}{2} \leqslant \sqrt{1+x} \leqslant 1+\frac{x}{2}
$$

Zilustrować tę nierówność odpowiednim rysunkiem.\\
6. Z dwóch przeciwległych wierzchołków prostokąta o polu $P$, będącego podstawą prostopadłościanu o wysokości 1 , wystawiono po dwie przekątne sąsiednich ścian bocznych. Wyrazić cosinus kąta pomiędzy płaszczyznami utworzonymi przez te pary przekątnych jako funkcję sinusa kąta między nimi. Sporządzić rysunki.

\section*{PRACA KONTROLNA nr 7 - POZIOM PODSTAWOWY}
\begin{enumerate}
  \item Wyznaczyć rozwiązanie ogólne równania
\end{enumerate}

$$
\sin \left(2 x+\frac{\pi}{3}\right)=\cos \left(x-\frac{\pi}{6}\right)
$$

a następnie podać rozwiązania w przedziale $[-2 \pi, 2 \pi]$.\\
2. Wyrażenie

$$
\left(\frac{a-2 b}{\sqrt[3]{a^{2}}-\sqrt[3]{4 b^{2}}}+\frac{\sqrt[3]{2 a^{2} b}+\sqrt[3]{4 a b^{2}}}{\sqrt[3]{a^{2}}+\sqrt[3]{4 b^{2}}+\sqrt[3]{16 a b}}\right): \frac{a \sqrt[3]{a}+b \sqrt[3]{2 b}+b \sqrt[3]{a}+a \sqrt[3]{2 b}}{a+b}
$$

sprowadzić do najprostszej postaci. Przy jakich założeniach ma ono sens?\\
3. Narysować wykres funkcji $f(x)=2|x|-\sqrt{x^{2}+4 x+4}$ oraz wyznaczyć najmniejszą i największą wartość funkcji $|f(x)|$ w przedziale $[-1,2]$. Dla jakiego $m$ pole figury ograniczonej wykresem funkcji $|f(x)|$ i prostą $y=m$ równe jest 16 ?\\
4. Rozwiązać układ równań

$$
\left\{\begin{array}{l}
x^{2}-4 y^{2}+8 y=4 \\
x^{2}+y^{2}-2 y=4
\end{array}\right.
$$

Podać interpretację geometryczną tego układu i obliczyć pole czworokąta, którego wierzchołkami są cztery punkty będące jego rozwiązaniem.\\
5. W trapezie równoramiennym $A B C D$, w którym $B C \| A D$ dane są $\overrightarrow{A B}=[1,-2]$ oraz $\overrightarrow{A D}=[1,1]$. Obliczyć pole trapezu i wyznaczyć kąt między jego przekątnymi.\\
6. W ostrosłupie prawidłowym trójkątnym cosinus kąta nachylenia ściany bocznej do podstawy równy jest $\frac{1}{9}$. Obliczyć stosunek pola powierzchni całkowitej do pola podstawy. Wykorzystując wzór $\sin 2 \alpha=2 \sin \alpha \cos \alpha$, wyznaczyć sinus kąta między ścianami bocznymi tego ostrosłupa. Sporządzić rysunki.

\section*{PRACA KONTROLNA nr 7 - POZIOM ROZSZERZoNY}
\begin{enumerate}
  \item Rozwiązać równanie
\end{enumerate}

$$
\sin x+\cos x=\frac{\cos 2 x}{\sin 2 x-1}
$$

\begin{enumerate}
  \setcounter{enumi}{1}
  \item Wyrażenie
\end{enumerate}

$$
w(x, y)=\frac{x}{x^{3}+x^{2} y+x y^{2}+y^{3}}+\frac{y}{x^{3}-x^{2} y+x y^{2}-y^{3}}+\frac{1}{x^{2}-y^{2}}-\frac{1}{x^{2}+y^{2}}-\frac{x^{2}+2 y^{2}}{x^{4}-y^{4}}
$$

doprowadzić do najprostszej postaci. Przy jakich założeniach ma ono sens? Obliczyć $w\left(\cos 15^{\circ}, \sin 15^{\circ}\right)$.\\
3. Narysować wykres funkcji

$$
f(x)=\left\{\begin{array}{lll}
2^{-x}-1 & \text { dla } & x \leqslant 1 \\
\log _{2}\left(x-\frac{1}{2}\right) & \text { dla } & x>1
\end{array}\right.
$$

i posługując się nim wyznaczyć zbiór wartości funkcji $|f(x)|$ w przedziale $\left[-\frac{1}{2}, \frac{3}{2}\right]$.\\
4. Rozwiązać układ równań

$$
\left\{\begin{array}{l}
y+x^{2}=4 \\
4 x^{2}-y^{2}+2 y=1
\end{array}\right.
$$

Podać interpretację geometryczną tego układu i wykazać, że cztery punkty, które są jego rozwiązaniem, wyznaczają na płaszczyźnie trapez równoramienny. Znaleźć równanie okręgu opisanego na tym trapezie.\\
5. Odcinek o końcach $A(0,7)$ i $B(5,2)$ jest przeciwprostokątną trójkąta prostokątnego, którego wierzchołek $C$ leży na prostej $x=3$. Posługując się rachunkiem wektorowym obliczyć cosinus kąta między dwusieczną kąta prostego a wysokością opuszczoną z wierzchołka $C$.\\
6. Pole powierzchni całkowitej ostrosłupa prawidłowego trójkątnego jest dziesięć razy większe niż pole jego podstawy. Wyznaczyć cosinus kąta między ścianami bocznymi oraz stosunek objętości ostrosłupa do objętości wpisanej w niego kuli.

\section*{PRACA KONTROLNA nr 8 - POZIOM PODSTAWOWY}
\begin{enumerate}
  \item Cztery kolejne współczynniki wielomianu $f(x)=x^{3}+a x^{2}+b x+c$ tworzą ciąg geometryczny. Wiadomo, że -3 jest pierwiastkiem tego wielomianu. Wyznaczyć współczynniki $a, b, c$ 。
  \item Koło $x^{2}+y^{2}+4 x-2 y-1 \leqslant 0$ zostało przesunięte o wektor $\vec{w}=[3,3]$. Znaleźć równanie osi symetrii figury, która jest sumą koła i jego obrazu oraz obliczyć jej pole.
  \item Podstawą ostrosłupa jest trójkąt o bokach $a, b, c$. Wszystkie kąty płaskie przy wierzchołku ostrosłupa są proste. Obliczyć objętość ostrosłupa.
  \item Dane są punkty $A(0,2), B(4,4), C(3,6)$. Na prostej przechodzącej przez punkt $C$ równoległej do prostej $A B$ znaleźć punkt $D$, który jest równo odległy od punktów $A$ i $B$. Wykazać, że trójkąt $A B D$ jest prostokątny i napisać równanie okręgu opisanego na nim.
  \item Wyznaczyć wartość parametru $m$, dla którego równanie
\end{enumerate}

$$
4 x^{2}-2 x \log _{2} m+1=0
$$

ma dwa różne pierwiastki rzeczywiste $x_{1}, x_{2}$ spełniające warunek $x_{1}^{2}+x_{2}^{2}=1$.\\
6. Dane są funkcje $f(x)=4^{x-2}-7 \cdot 3^{x-3}, g(x)=3^{3 x+2}-5 \cdot 4^{3 x}$. Rozwiązać nierówność

$$
f(x+3)>g\left(\frac{x}{3}\right)
$$

\section*{PRACA KONTROLNA nr 8 - POZIOM ROZSZERZONY}
\begin{enumerate}
  \item Niech $A$ będzie wierzchołkiem kwadratu, a $M$ środkiem przeciwległego boku. Na przekątnej kwadratu wychodzącej z wierzchołka $A$ wybrano punkt $P$ tak, aby $|A P|=|M P|$. Obliczyć, w jakim stosunku punkt $P$ dzieli przekatną kwadratu.
  \item Stosując zasadę indukcji matematycznej udowodnić nierówność
\end{enumerate}

$$
\binom{2 n}{n} \leqslant \frac{4^{n}}{\sqrt{2 n+2}}, \quad n \geqslant 1
$$

\begin{enumerate}
  \setcounter{enumi}{2}
  \item Wyznaczyć równanie okręgu o środku leżącym na prostej $y-x=0$ oraz stycznego do prostej $y-3=0$ i do okręgu $x^{2}+y^{2}-4 x+3=0$. Sporządzić rysunek.
  \item Liczba -2 jest pierwiastkiem dwukrotnym wielomianu $w(x)=\frac{1}{2} x^{3}+a x^{2}+b x+c$, a punkt $S\left(-1, y_{0}\right)$ jest środkiem symetrii wykresu $w(x)$. Wyznaczyć $a, b, c, y_{0}$ oraz trzeci pierwiastek. Sporządzić wykres $w(x)$ w przedziale $\left[-3, \frac{3}{2}\right]$.
  \item Wycinek koła o promieniu $3 R$ i kącie środkowym $\alpha$ zwinięto w powierzchnię boczną stożka $\mathcal{S}_{1}$. Podobnie, wycinek koła o promieniu $R$ i kącie środkowym $3 \alpha$ zwinięto w powierzchnię boczną stożka $\mathcal{S}_{2}$. Następnie obydwa stożki złączono podstawami tak, aby miały wspólną oś obrotu, a ich wierzchołki były skierowane w przeciwnych kierunkach. Obliczyć promień kuli wpisanej w otrzymaną bryłę. Sporządzić rysunek.
  \item Podać interpretację geometryczną równania $\sqrt{2 x+4}=m x+m+1$ z parametrem $m$. Graficznie i analitycznie określić, dla jakich wartości $m$ równanie ma dwa pierwiastki $x_{1}=x_{1}(m), x_{2}=x_{2}(m)$. Nie korzystając z metod rachunku różniczkowego, wykazać, że funkcja $f(m)=x_{1}(m)+x_{2}(m)$ jest malejąca oraz sporządzić jej wykres.
\end{enumerate}

\end{document}