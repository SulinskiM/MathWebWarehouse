\documentclass[10pt]{article}
\usepackage[polish]{babel}
\usepackage[utf8]{inputenc}
\usepackage[T1]{fontenc}
\usepackage{amsmath}
\usepackage{amsfonts}
\usepackage{amssymb}
\usepackage[version=4]{mhchem}
\usepackage{stmaryrd}
\usepackage{hyperref}
\hypersetup{colorlinks=true, linkcolor=blue, filecolor=magenta, urlcolor=cyan,}
\urlstyle{same}

\title{PRACA KONTROLNA nr 2 - POZIOM PODSTAWOWY }

\author{}
\date{}


\begin{document}
\maketitle
\begin{enumerate}
  \item Czy suma długości przekątnych kwadratów o polach 10 i $\frac{21}{2}$ jest większa od długości przekątnej kwadratu o polu $\frac{81}{2}$ ? Odpowiedź uzasadnić nie używając kalkulatora.
  \item Grupa słuchaczy wykładu z algebry liczy 261 osób. Egzamin podstawowy zdała pewna (dodatnia) ilość osób. Po egzaminie poprawkowym liczba osób, które zdały, powiększyła się o 5,6\%. Ile osób zdało egzamin podstawowy (wskazówka: pamiętaj, że ilość osób, które zdały egzamin jest liczbą całkowitą)?
  \item Hasło do pewnego systemu komputerowego ma składać się z dokładnie 2 liter (do wyboru z 26 małych i 26 dużych liter alfabetu) oraz z przynajmniej 2 i co najwyżej 4 cyfr (od 0 do 9). Zarówno litery jak i liczby mogą się powtarzać. Ile jest różnych haseł spełniających te warunki?
  \item Rozwiązać nierówność
\end{enumerate}

$$
x+1 \geqslant \sqrt{5-x}
$$

\begin{enumerate}
  \setcounter{enumi}{4}
  \item Suma 21 pierwszych wyrazów pewnego ciągu arytmetycznego wynosi zero a iloczyn dwunastego i trzynastego wyrazu równy jest 8 . Dla jakich liczb $n$ suma $n$ pierwszych wyrazów tego ciągu jest mniejsza od 9 ?
  \item Marcin stoi nad brzegiem morza i obserwuje odpływający statek.\\
a) Jak daleko będzie statek od (oczu) Marcina w momencie, w którym zniknie on za horyzontem (Marcin przestanie go widzieć)?\\
b) Na jak wysoką wieżę musi on wejść, żeby jeszcze widzieć statek będący w odległości 10 km od niego?
\end{enumerate}

Przyjąć, że Ziemia jest kulą o promieniu 6371 km a oczy Marcina znajdują się na wysokości 170 cm .

\section*{PRACA KONTROLNA nr 2 - POZIOM ROZSZERZONY}
\begin{enumerate}
  \item Ułożono dwie wieże z sześciennych klocków. Pierwszą z trzech klocków o objętości 72, 8 oraz $3 \mathrm{~cm}^{3}$, a drugą z czterech jednakowych klocków o objętości $8 \mathrm{~cm}^{3}$. Która z nich jest wyższa? Odpowiedź uzasadnić nie używając kalkulatora.
  \item Kod do sejfu w willi pana Bogackiego jest pięciocyfrowy. Jego córka, korzystając z chwilowej nieobecności taty, próbuje go otworzyć. Wie jednak tylko, że kod ułożony jest z dokładnie trzech różnych cyfr i nie występują w nim cyfry 1,4 i 9. Ile jest różnych kodów spełniających te warunki?
  \item Rozwiązać nierówność
\end{enumerate}

$$
x-1>\sqrt{4-\frac{6}{x}}
$$

\begin{enumerate}
  \setcounter{enumi}{3}
  \item W jednej szklance znajduje się woda, a w drugiej dokładnie taka sama ilość wina. Z pierwszej szklanki przelano jedną łyżkę wody do szklanki z winem i dokładnie wymieszano. Następnie przelano jedną łyżkę powstałej mieszaniny z powrotem do pierwszej szklanki. Sprawdzić czy po tych zabiegach jest więcej wody w winie czy wina w wodzie.
  \item Trzy liczby tworzą ciąg geometryczny. Ich suma równa jest 13, a suma ich odwrotności wynosi $\frac{13}{9}$. Znaleźć te liczby.
  \item Bocian stoi na słupie o wysokości 5 metrów. Magda, której oczy znajdują się na wysokości 160 cm nad ziemią, stoi 10,2 metra od tego słupa i widzi bociana pod kątem 6 stopni. Jak wysoki jest bocian? Podać wynik z dokładnością do 1 cm . W razie potrzeby odpowiednią funkcję trygonometryczną kąta $6^{\circ}$ przybliżyć za pomocą tablic matematycznych lub kalkulatora.
\end{enumerate}

Rozwiązania (rękopis) zadań z wybranego poziomu prosimy nadsyłać do 19 października 2015r. na adres:

\begin{verbatim}
Wydział Matematyki
Politechnika Wrocławska
Wybrzeże Wyspiańskiego 27
50-370 WROCEAW.
\end{verbatim}

Na kopercie prosimy koniecznie zaznaczyć wybrany poziom! (np. poziom podstawowy lub rozszerzony). Do rozwiązań należy dołączyć zaadresowaną do siebie kopertę zwrotną z naklejonym znaczkiem, odpowiednim do wagi listu. Prace niespełniające podanych warunków nie będą poprawiane ani odsyłane.

Adres internetowy Kursu: \href{http://www.im.pwr.wroc.pl/kurs}{http://www.im.pwr.wroc.pl/kurs}


\end{document}