\documentclass[10pt]{article}
\usepackage[polish]{babel}
\usepackage[utf8]{inputenc}
\usepackage[T1]{fontenc}
\usepackage{amsmath}
\usepackage{amsfonts}
\usepackage{amssymb}
\usepackage[version=4]{mhchem}
\usepackage{stmaryrd}
\usepackage{hyperref}
\hypersetup{colorlinks=true, linkcolor=blue, filecolor=magenta, urlcolor=cyan,}
\urlstyle{same}

\title{PRACA KONTROLNA nr 2 - POZIOM PODSTAWOWY }

\author{}
\date{}


\begin{document}
\maketitle
\begin{enumerate}
  \item Rozwiązać nierówność
\end{enumerate}

$$
2 x-2>\sqrt{7-4 x}
$$

\begin{enumerate}
  \setcounter{enumi}{1}
  \item Dla jakich wartości parametru $m$ pierwiastkiem wielomianu
\end{enumerate}

$$
w(x)=2 x^{3}-x^{2}-\left(m^{2}-2\right) x+m-1
$$

jest $x=2$ ? Dla znalezionych wartości $m$ wyznaczyć pozostałe pierwiastki $w(x)$.\\
3. Narysować staranny wykres funkcji $f(x)=|\sin x| \cos x-\frac{1}{4}$ i rozwiązać nierówność

$$
f(x) \leqslant-\frac{1}{2}
$$

\begin{enumerate}
  \setcounter{enumi}{3}
  \item Rozwiązać równanie
\end{enumerate}

$$
4^{x+\sqrt{x^{2}-2}}-5 \cdot 2^{x-1+\sqrt{x^{2}-2}}=6
$$

\begin{enumerate}
  \setcounter{enumi}{4}
  \item W trójkącie równoramiennym $A B C$ o podstawie $A B$ dane są $A(2,-1)$ oraz $B(-1,3)$. Środkowe poprowadzone z $A$ i z $B$ są prostopadłe. Znaleźć współrzędne punktu $C$ oraz obliczyć pole i obwód tego trójkąta.
  \item W okrąg o promieniu $R$ wpisano trzy jednakowe okręgi wzajemnie styczne w punktach $A, B, C$ i styczne do danego okręgu. Obliczyć pole obszaru ograniczonego mniejszymi łukami $A B, B C$ i $C A$.
\end{enumerate}

\section*{PRACA KONTROLNA nr 2 - POZIOM ROZSZERZONY}
\begin{enumerate}
  \item Rozwiązać nierówność
\end{enumerate}

$$
\sqrt{2 x^{2}-x}<5-4 x
$$

\begin{enumerate}
  \setcounter{enumi}{1}
  \item Rozwiązać układ równań
\end{enumerate}

$$
\left\{\begin{array}{cc}
x y & =400 \\
x^{\log y} & =16
\end{array}\right.
$$

\begin{enumerate}
  \setcounter{enumi}{2}
  \item Narysować staranny wykres funkcji $f(x)=|\sin x|-\cos x$, wyznaczyć jej zbior wartości oraz rozwiązać nierówność
\end{enumerate}

$$
\frac{1}{f(x)} \geqslant 1
$$

\begin{enumerate}
  \setcounter{enumi}{3}
  \item Reszta z dzielenia wielomianu $w(x)=x^{4}+a x^{3}-b x^{2}+b x$ przez trójmian $x^{2}-9$ wynosi $-5 x+45$. Wyznaczyć wartości parametrów $a$ i $b$ oraz rozwiązać nierówność
\end{enumerate}

$$
w(x-1) \geqslant w(x+1)
$$

\begin{enumerate}
  \setcounter{enumi}{4}
  \item Dany jest punkt $A(2,1)$. Wyznaczyć i narysować zbiór tych wszystkich punktów $C$, dla których czworokąt $A B C D$ jest prostokątem takim, że punkty $B$ i $D$ leżą na osiach układu współrzędnych i nie należą do tego samego boku prostokąta. Wykonać rysunek.
  \item Nad sześcianem o krawędzi a stojącym na płaszczyźnie umieszczono punktowe źródło światła na wysokości $b>a$ (rzut prostopadły punktu, w którym jest źródło światła na tę płaszczyznę, zawiera się w podstawie sześcianu). Obliczyć pole obszaru jaki zajmuje cień sześcianu łącznie z jego podstawą na tej płaszczyźnie.
\end{enumerate}

Rozwiązania (rękopis) zadań z wybranego poziomu prosimy nadsyłać do 18 października 2017r. na adres:

\begin{verbatim}
Wydział Matematyki
Politechnika Wrocławska
Wybrzeże Wyspiańskiego 27
50-370 WROCEAW.
\end{verbatim}

Na kopercie prosimy koniecznie zaznaczyć wybrany poziom! (np. poziom podstawowy lub rozszerzony). Do rozwiązań należy dołączyć zaadresowaną do siebie kopertę zwrotną z naklejonym znaczkiem, odpowiednim do wagi listu. Prace niespełniające podanych warunków nie będą poprawiane ani odsyłane.

Adres internetowy Kursu: \href{http://www.im.pwr.wroc.pl/kurs}{http://www.im.pwr.wroc.pl/kurs}


\end{document}