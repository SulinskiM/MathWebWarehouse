\documentclass[10pt]{article}
\usepackage[polish]{babel}
\usepackage[utf8]{inputenc}
\usepackage[T1]{fontenc}
\usepackage{amsmath}
\usepackage{amsfonts}
\usepackage{amssymb}
\usepackage[version=4]{mhchem}
\usepackage{stmaryrd}
\usepackage{hyperref}
\hypersetup{colorlinks=true, linkcolor=blue, filecolor=magenta, urlcolor=cyan,}
\urlstyle{same}

\title{PRACA KONTROLNA nr 2 - POZIOM PODSTAWOWY }

\author{}
\date{}


\begin{document}
\maketitle
\begin{enumerate}
  \item Rozwiązać nierówność $x-1>\sqrt{x^{2}-3}$.
  \item Rozwiązać równanie $\frac{1}{\sin 2 x}+\frac{1}{\sin x}=0$.
  \item Narysować zbiór $\left\{(x, y):-1 \leqslant \log _{\frac{1}{2}}|x|+\log _{2}|y| \leqslant 1,|x|+|y| \leqslant 3\right\}$ i obliczyć jego pole.
  \item Na prostej $l: 2 x-y-1=0$ wyznaczyć punkty, z których odcinek o końcach $A(0,1)$ oraz $B(0,3)$ jest widoczny pod kątem $45^{\circ}$.
  \item W obszar ograniczony wykresem funkcji kwadratowej i osią $O x$ wpisano prostokąt o polu 24, którego jeden z boków zawarty jest w osi $O x$, a dwa wierzchołki leżą na paraboli. Odległość między miejscami zerowymi funkcji wynosi 10. Wyznaczyć wzór tej funkcji, wiedząc, że wierzchołek paraboli leży na osi $O y$ i jeden z boków prostokąta ma długość 6 . Rozwiązanie zilustrować odpowiednim rysunkiem.
  \item W ostrosłupie, którego podstawą jest romb o boku $a$, jedna z krawędzi bocznych również ma długość $a$ i jest prostopadła do podstawy. Wszystkie pozostałe krawędzie boczne są równe. Obliczyć objętość, pole powierzchni całkowitej ostrosłupa oraz sinus kąta nachylenia do podstawy jego pochyłych ścian bocznych.
\end{enumerate}

\section*{PRACA KONTROLNA nr 2 - POZIOM ROZSZERZONY}
\begin{enumerate}
  \item Wyznaczyć dziedzinę funkcji $f(x)=\log _{2}\left(\sqrt{x-1-\sqrt{x^{2}-3 x-4}}-1\right)$.
  \item Rozwiązać równanie $\sin ^{4} x+\cos ^{4} x=\sin x \cos x$.
  \item Narysować zbiór $\left\{(x, y):|x|+|y| \leqslant 6,|y| \leqslant 2^{|x|},|y| \geqslant \log _{2}|x|\right\}$ i napisać równania jego osi symetrii. Podać odpowiednie uzasadnienie.
  \item Niech $f(x)=\frac{2 x-1}{x-2}, g(x)=(\sqrt{2})^{\log _{2}(2 x-1)^{2}+4 \log _{\frac{1}{2}} \sqrt{2-x}}$. Narysować wykres funkcji $h(x)=\max \{f(x), g(x)\}$. Czy można podać wzór funkcji $h(x)$, wykorzystując jedynie $f(x)$ ?
  \item Punkt $A(1,1)$ jest wierzchołkiem rombu o polu 10. Przekątna $A C$ rombu jest równoległa do wektora $\vec{v}=[1,2]$. Wyznaczyć współrzędne pozostałych wierzchołków rombu, wiedząc, że jeden z nich leży na prostej $y=x-2$.
  \item W ostrosłupie, którego podstawą jest romb o boku $a$, jedna z krawędzi bocznych również ma długość $a$ i jest prostopadła do podstawy. Wszystkie pozostałe krawędzie boczne są równe. Wyznaczyć cosinusy kątów płaskich przy wierzchołku każdej ściany bocznej ostrosłupa oraz cosinusy kątów między jego ścianami bocznymi .
\end{enumerate}

Rozwiązania (rękopis) zadań z wybranego poziomu prosimy nadsyłać do 18 października 2018r. na adres:

\begin{verbatim}
Wydział Matematyki
Politechnika Wrocławska
Wybrzeże Wyspiańskiego 27
50-370 WROCEAW.
\end{verbatim}

Na kopercie prosimy koniecznie zaznaczyć wybrany poziom! (np. poziom podstawowy lub rozszerzony). Do rozwiązań należy dołączyć zaadresowaną do siebie kopertę zwrotną z naklejonym znaczkiem, odpowiednim do wagi listu. Prace niespełniające podanych warunków nie będą poprawiane ani odsyłane.

Uwaga. Wysyłając nam rozwiązania zadań uczestnik Kursu udostępnia Politechnice Wrocławskiej swoje dane osobowe, które przetwarzamy wyłącznie w zakresie niezbędnym do jego prowadzenia (odesłanie zadań, prowadzenie statystyki). Szczegółowe informacje o przetwarzaniu przez nas danych osobowych są dostępne na stronie internetowej Kursu.

Adres internetowy Kursu: \href{http://www.im.pwr.edu.pl/kurs}{http://www.im.pwr.edu.pl/kurs}


\end{document}