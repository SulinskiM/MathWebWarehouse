\documentclass[10pt]{article}
\usepackage[polish]{babel}
\usepackage[utf8]{inputenc}
\usepackage[T1]{fontenc}
\usepackage{amsmath}
\usepackage{amsfonts}
\usepackage{amssymb}
\usepackage[version=4]{mhchem}
\usepackage{stmaryrd}
\usepackage{hyperref}
\hypersetup{colorlinks=true, linkcolor=blue, filecolor=magenta, urlcolor=cyan,}
\urlstyle{same}

\title{PRACA KONTROLNA nr 2 - POZIOM PODSTAWOWY }

\author{}
\date{}


\begin{document}
\maketitle
\begin{enumerate}
  \item Niech $\alpha$ będzie kątem ostrym takim, że $\sin \alpha=\sqrt{15} \cos \alpha$. Wyznaczyć wszystkie wartości funkcji trygonometrycznych kątów $\alpha$ oraz $2 \alpha$.
  \item Rozwiązać nierówność
\end{enumerate}

$$
x \geqslant 2+\sqrt{10-3 x}
$$

\begin{enumerate}
  \setcounter{enumi}{2}
  \item Wykres trójmianu kwadratowego $f(x)=a x^{2}+b x+c$ jest symetryczny względem prostej $x=3$, a resztą z jego dzielenia przez wielomian $x-2$ jest -1 . Wiadomo też, że $f(0)=3$. Znaleźć wartości współczynników $a, b, c$ i rozwiązać nierówność
\end{enumerate}

$$
\frac{1}{f(x)} \geqslant \frac{1}{3}
$$

\begin{enumerate}
  \setcounter{enumi}{3}
  \item W ciągu arytmetycznym, w którym trzeci wyraz jest odwrotnością pierwszego, suma pierwszych ośmiu wyrazów wynosi 25 . Obliczyć sumę pierwszych 10 wyrazów o numerach nieparzystych.
  \item Pole trapezu równoramiennego, opisanego na okręgu o promieniu 1, wynosi 5. Obliczyć pole czworokąta, którego wierzchołkami są punkty styczności okręgu i trapezu.
  \item Na szczycie góry, na którą wchodzi Agata po stoku o kącie nachylenia $\beta$, stoi krowa o wysokości 150 cm . Dziewczynka widzi ją pod kątem $\alpha$, przy czym przyjmujemy tutaj dla uproszczenia, że punkt obserwacji znajduje się na poziomie drogi. Na jakiej wysokości nad poziomem morza stoi Agata, jeżeli szczyt jest na wysokości 1520 m n.p.m.? Podać wzór i następnie wykonać obliczenia dla $\beta=43^{\circ}, \alpha=2^{\circ}$.
\end{enumerate}

\section*{PRACA KONTROLNA nr 2 - POZIOM RoZsZERZONY}
\begin{enumerate}
  \item W nieskończonym ciągu geometrycznym, którego suma równa jest 4, trzeci wyraz jest odwrotnością pierwszego. Wyznaczyć pierwszy wyraz i sumę wszystkich wyrazów o numerach parzystych.
  \item Narysować wykres funkcji
\end{enumerate}

$$
f(x)=\frac{\sin x}{\sqrt{1+\operatorname{tg}^{2} x}}
$$

i rozwiązać nierówność $f(x) \geqslant \frac{1}{4}$.\\
3. Rozwiązać nierówność

$$
\sqrt{\frac{4 x-2}{x+4}} \leqslant \frac{2}{x-1}-1
$$

\begin{enumerate}
  \setcounter{enumi}{3}
  \item Resztą z dzielenia wielomianu $w(x)=a x^{5}+b x^{2}+c$ przez $p(x)=x^{3}-x^{2}-2 x$ jest wielomian $r(x)=11 x^{2}+12 x+1$. Wyznaczyć wartości współczynników $a, b, c$ oraz rozwiązać nierówność $w(x) \geqslant(x+1)^{2}$.
  \item Wyznaczyć pole trójkąta równobocznego, którego wierzchołki leżą na trzech różnych prostych równoległych, z których środkowa jest oddalona od skrajnych o $a$ i $b$.
  \item W punktach $A(0,0), B(4,0)$ i $C(0,4)$ ustawione są trzy znaczniki. Sensory robota pozwalają ustalić, że z miejsca, w którym znajduje się on obecnie odcinek $A B$ widać pod kątem $\alpha=90^{\circ}$, a odcinek $A C$ pod kątem $\beta=60^{\circ}$. Ustalić położenie robota w układzie współrzędnych.
\end{enumerate}

Rozwiązania (rękopis) zadań z wybranego poziomu prosimy nadsyłać do 18 października 2019r. na adres:

Wydział Matematyki\\
Politechnika Wrocławska\\
Wybrzeże Wyspiańskiego 27\\
50-370 WROCEAW.\\
Na kopercie prosimy koniecznie zaznaczyć wybrany poziom! (np. poziom podstawowy lub rozszerzony). Do rozwiązań należy dołączyć zaadresowaną do siebie kopertę zwrotną z naklejonym znaczkiem, odpowiednim do wagi listu. Prace niespełniające podanych warunków nie będą poprawiane ani odsyłane.

Uwaga. Wysyłając nam rozwiązania zadań uczestnik Kursu udostępnia Politechnice Wrocławskiej swoje dane osobowe, które przetwarzamy wyłącznie w zakresie niezbędnym do jego prowadzenia (odesłanie zadań, prowadzenie statystyki). Szczegółowe informacje o przetwarzaniu przez nas danych osobowych są dosteqpne na stronie internetowej Kursu.\\
Adres internetowy Kursu: \href{http://www.im.pwr.edu.pl/kurs}{http://www.im.pwr.edu.pl/kurs}


\end{document}