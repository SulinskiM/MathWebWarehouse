\documentclass[10pt]{article}
\usepackage[polish]{babel}
\usepackage[utf8]{inputenc}
\usepackage[T1]{fontenc}
\usepackage{amsmath}
\usepackage{amsfonts}
\usepackage{amssymb}
\usepackage[version=4]{mhchem}
\usepackage{stmaryrd}
\usepackage{hyperref}
\hypersetup{colorlinks=true, linkcolor=blue, filecolor=magenta, urlcolor=cyan,}
\urlstyle{same}

\title{PRACA KONTROLNA nr 3 - POZIOM PODSTAWOWY }

\author{}
\date{}


\begin{document}
\maketitle
\begin{enumerate}
  \item Rozwiąż nierówność
\end{enumerate}

$$
x^{5}+x^{4}-8 x^{2}+16 \geqslant 8 x^{3}-16 x
$$

\begin{enumerate}
  \setcounter{enumi}{1}
  \item W przedziale $[\pi, 2 \pi]$ rozwiąż równanie
\end{enumerate}

$$
\frac{\sin 3 x}{\cos 6 x}=1
$$

\begin{enumerate}
  \setcounter{enumi}{2}
  \item Dane sac trzy wektory $\vec{a}=(1,1), \vec{b}=(2,-1), \vec{c}=(5,2)$. Dobierz takie liczby $p, q$, aby z wektorów $p \vec{a}, q \vec{b}, \vec{c}$ można było zbudować trójkąt.
  \item W przedziale $[0, \pi]$ narysuj wykres funkcji
\end{enumerate}

$$
f(x)=\frac{1}{|\operatorname{tg} x+\operatorname{ctg} x|}+\sin 2 x
$$

i rozwiąż nierówność $f(x)<\frac{3}{4}$.\\
5. Na okręgu $x^{2}-2 x+y^{2}+4 y-4=0$ wyznacz punkt, którego odległość od prostej $x-3 y+6=0$ jest najmniejsza.\\
6. Przekątna rombu o polu 9 zawarta jest w prostej $x-2 y+3=0$, a jednym z jego wierzchołków jest punkt $A(2,-2)$. Wyznacz współrzędne pozostałych wierzchołków tego rombu.

\section*{PRACA KONTROLNA nr 3 - POZIOM RozsZERzoNY}
\begin{enumerate}
  \item Resztą z dzielenia wielomianu $w(x)=x^{4}+p x^{3}-3 x^{2}+q x-14$ przez $x^{2}-x-2$ jest $4 x-28$. Wyznacz współczynniki $p, q$ i rozwiąż nierówność $w(x) \geqslant 0$.
  \item Wyznacz najmniejszą wartość funkcji
\end{enumerate}

$$
f(x)=(\operatorname{tg} x+\operatorname{ctg} x)^{2}
$$

oraz rozwiąż nierówność $f(x) \leqslant f(2 x)$.\\
3. Rozwiąż równanie

$$
\cos x+\cos 2 x+2 \cos 3 x+\cos 4 x+\cos 5 x=0
$$

\begin{enumerate}
  \setcounter{enumi}{3}
  \item Znajdź kąt między wektorami $\vec{a} \mathrm{i} \vec{b}$ wiedząc, że wektor $5 \vec{a}-4 \vec{b}$ jest prostopadły do wektora $2 \vec{a}+4 \vec{b}$, a wektor $\vec{a}-5 \vec{b}$ jest prostopadły do wektora $6 \vec{a}-2 \vec{b}$.
  \item Z wierzchołka $O$ paraboli $y^{2}=2 x$ poprowadzono dwie proste wzajemnie prostopadłe i przecinające parabolę w punktach $P$ i $Q$. Wyznacz zbiór punktów płaszczyzny utworzony przez środki ciężkości trójkątów $O P Q$. Sporządź rysunek.
  \item W trójkącie o wierzchołkach $A(-6,-7), B(8,-9), C(0,10)$ punkt $P$ jest środkiem boku $B C$, a punkt $S$ jest punktem przecięcia środkowej poprowadzonej z wierzchołka $A$ oraz wysokości opuszczonej na bok $A B$. Oblicz pole trójkąta $C S P$ oraz znajdź równanie okregu opisanego na nim.
\end{enumerate}

Rozwiązania (rękopis) zadań z wybranego poziomu prosimy nadsyłać do 18 listopada 2014r. na adres:

Instytut Matematyki i Informatyki\\
Politechniki Wrocławskiej\\
Wybrzeże Wyspiańskiego 27\\
50-370 WROCEAW.\\
Na kopercie prosimy koniecznie zaznaczyć wybrany poziom! (np. poziom podstawowy lub rozszerzony). Do rozwiązań należy dołączyć zaadresowaną do siebie kopertę zwrotną z naklejonym znaczkiem, odpowiednim do wagi listu. Prace niespełniające podanych warunków nie będą poprawiane ani odsyłane.

Adres internetowy Kursu: \href{http://www.im.pwr.wroc.pl/kurs}{http://www.im.pwr.wroc.pl/kurs}


\end{document}