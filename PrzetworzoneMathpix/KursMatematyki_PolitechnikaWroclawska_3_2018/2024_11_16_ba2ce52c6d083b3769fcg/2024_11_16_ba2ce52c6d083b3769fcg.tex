\documentclass[10pt]{article}
\usepackage[polish]{babel}
\usepackage[utf8]{inputenc}
\usepackage[T1]{fontenc}
\usepackage{amsmath}
\usepackage{amsfonts}
\usepackage{amssymb}
\usepackage[version=4]{mhchem}
\usepackage{stmaryrd}
\usepackage{hyperref}
\hypersetup{colorlinks=true, linkcolor=blue, filecolor=magenta, urlcolor=cyan,}
\urlstyle{same}

\title{PRACA KONTROLNA nr 3 - POZIOM PODSTAWOWY }

\author{}
\date{}


\begin{document}
\maketitle
\begin{enumerate}
  \item Narysować wykres funkcji $f(x)=2 \cos x-|\cos x|$ i rozwiązać nierówność $f(x)<-\frac{3}{2}$.
  \item Znaleźć punkt należący do paraboli $y^{2}=4 x$, którego odległość od punktu $A(3,0)$ jest najmniejsza.
  \item Dany jest punkt $A(2,1)$ oraz dwie proste:
\end{enumerate}

$$
p: x+y+2=0, \quad q: x-2 y-4=0 .
$$

Znaleźć taki punkt $B$ na prostej $q$, żeby środek odcinka $A B$ leżał na prostej $p$. Sporządzić rysunek.\\
4. Logarytmy liczb $1,3^{x}-2,3^{x}+4$ tworzą ciąg arytmetyczny (w podanej kolejności). Obliczyć $x$.\\
5. Kolejne środki boków czworokąta wypukłego $A B C D$ połączono odcinkami otrzymując czworokąt $E F G H$. Jaką figurą jest czworokąt $E F G H$ ? Odpowiedź uzasadnić. Obliczyć pole czworokąta $A B C D$, wiedząc, że pole czworokąta $E F G H$ jest równe 5.\\
6. Rozwiązać nierówność

$$
f(x) \leqslant \frac{4}{f(x)}
$$

gdzie $f(x)=-\frac{4}{3} x^{2}+2 x+\frac{4}{3}$.

\section*{PRACA KONTROLNA nr 3 - POZIOM ROZSZERZONY}
\begin{enumerate}
  \item Narysować wykres funkcji $f(x)=2 \cos ^{2} x-\sin \left(2 x-\frac{\pi}{2}\right)$ i rozwiązać nierówność $|f(x)|<2$.
  \item Znaleźć punkt należący do paraboli $y^{2}=2 x$, którego odległość od prostej $x-2 y+6=0$ jest najmniejsza.
  \item Wielomian $w(x)=x^{4}+a x^{3}+b x^{2}+c x+d$ jest podzielny przez trójmian $x^{2}-x-2$, a jego wykres jest symetryczny względem osi $0 y$. Wyznaczyć wartości parametrów $a, b, c, d$ i rozwiązać nierówność $w(x+1) \leqslant w(x-2)$.
  \item Rozwiązać nierówność
\end{enumerate}

$$
\log x+\log ^{3} x+\log ^{5} x+\ldots \leqslant 2 \sqrt{5}
$$

\begin{enumerate}
  \setcounter{enumi}{4}
  \item Punkt $S$ jest środkiem boku $A B$ w trójkącie $A B C$. Ponadto $A C \neq B C$ oraz $\angle B A C+$ $\angle S C B=90^{\circ}$. Niech $D$ będzie punktem przecięcia symetralnej $A B$ z prostą $A C$. Udowodnić, że na czworokącie $S B D C$ można opisać okrąg. Dlaczego musimy założyć, że $A C \neq B C$ ?
  \item Wyznaczyć równanie zbioru wszystkich środków tych cięciw paraboli $y=x^{2}$, które przechodzą przez punkt $A(0,2)$.
\end{enumerate}

Rozwiązania (rękopis) zadań z wybranego poziomu prosimy nadsyłać do 18 listopada 2018r. na adres:

\begin{verbatim}
Wydział Matematyki
Politechnika Wrocławska
Wybrzeże Wyspiańskiego 27
50-370 WROCEAW.
\end{verbatim}

Na kopercie prosimy koniecznie zaznaczyć wybrany poziom! (np. poziom podstawowy lub rozszerzony). Do rozwiązań należy dołączyć zaadresowaną do siebie kopertę zwrotną z naklejonym znaczkiem, odpowiednim do wagi listu. Prace niespełniające podanych warunków nie będą poprawiane ani odsyłane.

Uwaga. Wysyłając nam rozwiązania zadań uczestnik Kursu udostępnia Politechnice Wrocławskiej swoje dane osobowe, które przetwarzamy wyłącznie w zakresie niezbędnym do jego prowadzenia (odesłanie zadań, prowadzenie statystyki). Szczegółowe informacje o przetwarzaniu przez nas danych osobowych są dostępne na stronie internetowej Kursu.

Adres internetowy Kursu: \href{http://www.im.pwr.edu.pl/kurs}{http://www.im.pwr.edu.pl/kurs}


\end{document}