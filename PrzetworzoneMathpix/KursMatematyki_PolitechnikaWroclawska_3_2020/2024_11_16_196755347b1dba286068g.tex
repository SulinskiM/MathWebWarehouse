\documentclass[10pt]{article}
\usepackage[polish]{babel}
\usepackage[utf8]{inputenc}
\usepackage[T1]{fontenc}
\usepackage{amsmath}
\usepackage{amsfonts}
\usepackage{amssymb}
\usepackage[version=4]{mhchem}
\usepackage{stmaryrd}
\usepackage{bbold}
\usepackage{hyperref}
\hypersetup{colorlinks=true, linkcolor=blue, filecolor=magenta, urlcolor=cyan,}
\urlstyle{same}

\title{L }

\author{}
\date{}


\begin{document}
\maketitle
KORESPONDENCYJNY KURS\\
listopad 2020 r.\\
Z MATEMATYKI

\section*{PRACA KONTROLNA nr 3 - POZIOM PODSTAWOWY}
\begin{enumerate}
  \item Punkty $K$ i $L$ są środkami boków $A B$ i $C D$ czworokąta $A B C D$. Wykaż, że
\end{enumerate}

$$
\overrightarrow{K L}=\frac{1}{2}(\overrightarrow{A D}+\overrightarrow{B C})
$$

Wykonaj rysunek.\\
2. W pewnym ciągu geometrycznym każdy (z wyjątkiem pierwszego) wyraz jest różnicą wyrazu następnego i poprzedniego. Znajdź iloraz tego ciągu.\\
3. Rozwiąż nierówność

$$
\left[\log _{0,2}(x-1)\right]^{2}>4
$$

\begin{enumerate}
  \setcounter{enumi}{3}
  \item Rozwiąż równanie
\end{enumerate}

$$
\sin ^{2} x+\frac{1}{2} \sin 2 x=1
$$

\begin{enumerate}
  \setcounter{enumi}{4}
  \item Statek płynie prosto w kierunku klifu. Kąt elewacji (kąt utworzony przez linię poziomą i odcinek łączący obserwatora na statku ze szczytem klifu) wynosi początkowo $\alpha$, ale po przepłynięciu przez statek $d$ metrów wzrasta do $\beta$. Wyznacz wysokość klifu. Wykonaj obliczenia dla wartości $\alpha=10^{\circ}, \beta=15^{\circ}, d=50$.
  \item Obliczyć pole części wspólnej trzech kół o promieniach $r$ i środkach w wierzchołkach trójkąta równobocznego o boku $r \sqrt{2}$.
\end{enumerate}

\section*{PRACA KONTROLNA nr 3 - POZIOM RoZsZERzony}
\begin{enumerate}
  \item Znajdź taki ciąg arytmetyczny, w którym suma pierwszych $n$ wyrazów równa jest $n^{2}$ dla wszystkich $n \in \mathbb{N}$.
  \item W sześciokącie foremnym $A B C D E F$ punkty $M$ i $N$ są środkami boków $C D$ i $D E$. Wyznacz kąt między wektorami $\overrightarrow{A M}$ i $\overrightarrow{B N}$.
  \item Rozwiąż nierówność
\end{enumerate}

$$
\log _{2 x}\left(x^{2}-5 x+6\right)<1
$$

\begin{enumerate}
  \setcounter{enumi}{3}
  \item Rozwiąż równanie
\end{enumerate}

$$
\cos 2 x-3 \cos x=4 \cos ^{2} \frac{x}{2}
$$

\begin{enumerate}
  \setcounter{enumi}{4}
  \item Znajdź najmniejszą wartość ilorazu pola powierzchni bocznej stożka i pola powierzchni kuli wpisanej w ten stożek oraz kąt rozwarcia stożka realizujący tę wartość najmniejszą.
  \item Na dachu budynku stoi antena, której wysokość chcemy wyznaczyć nie wchodząc na górę. Urządzenie pomiarowe ustawione w pewnej odległości od budynku zmierzyło kąty między pionem a odcinkiem łączącym punkt pomiaru ze szczytem anteny oraz między pionem a odcinkiem łączącym punkt pomiaru z podstawą anteny. Otrzymano kąty $\alpha_{1}$ i $\beta_{1}$ odpowiednio. Następnie przesunięto urządzenie o $d$ metrów w kierunku budynku bez zmiany wysokości punktu pomiarowego i ponowiono pomiary, otrzymując kąty $\alpha_{2}$ i $\beta_{2}$. Podaj wzór na wysokość anteny i wykonaj obliczenia dla kątów $\alpha_{1}=53^{\circ}, \beta_{1}=55^{\circ}$, $\alpha_{2}=51^{\circ}, \beta_{2}=53.04^{\circ}$, oraz $d=5 \mathrm{~m}$.
\end{enumerate}

Rozwiązania (rękopis) zadań z wybranego poziomu prosimy nadsyłać do\\
20 listopada 2020r. na adres:

\begin{verbatim}
Wydział Matematyki
Politechnika Wrocławska
Wybrzeże Wyspiańskiego 27
50-370 WROCEAW.
\end{verbatim}

Na kopercie prosimy koniecznie zaznaczyć wybrany poziom! (np. poziom podstawowy lub rozszerzony). Do rozwiązań należy dołączyć zaadresowaną do siebie kopertę zwrotną z naklejonym znaczkiem, odpowiednim do formatu listu. Polecamy stosowanie kopert formatu C5 (160x230mm) ze znaczkiem o wartości 3,30 zł. Na każdą większą kopertę należy nakleić droższy znaczek. Prace niespełniające podanych warunków nie będą poprawiane ani odsyłane.

Uwaga. Wysyłając nam rozwiązania zadań uczestnik Kursu udostępnia Politechnice Wrocławskiej swoje dane osobowe, które przetwarzamy wyłącznie w zakresie niezbędnym do jego prowadzenia (odesłanie zadań, prowadzenie statystyki). Szczegółowe informacje o przetwarzaniu przez nas danych osobowych są dostępne na stronie internetowej Kursu.

Adres internetowy Kursu: \href{http://www.im.pwr.edu.pl/kurs}{http://www.im.pwr.edu.pl/kurs}


\end{document}