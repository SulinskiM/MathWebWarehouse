\documentclass[10pt]{article}
\usepackage[polish]{babel}
\usepackage[utf8]{inputenc}
\usepackage[T1]{fontenc}
\usepackage{amsmath}
\usepackage{amsfonts}
\usepackage{amssymb}
\usepackage[version=4]{mhchem}
\usepackage{stmaryrd}
\usepackage{hyperref}
\hypersetup{colorlinks=true, linkcolor=blue, filecolor=magenta, urlcolor=cyan,}
\urlstyle{same}

\title{PRACA KONTROLNA nr 4 - POZIOM PODSTAWOWY }

\author{}
\date{}


\begin{document}
\maketitle
\begin{enumerate}
  \item Znaleźć miejsca zerowe i naszkicować wykres funkcji $f(x)=x^{2}-x-5|x|+5$. Wyznaczyć najmniejszą i największą wartość tej funkcji na przedziale $[-5,5]$.
  \item Romb o boku $a$ i kącie ostrym $\alpha$ podzielono na trzy części o równych polach odcinkami mającymi wspólny początek w wierzchołku kąta ostrego i końce na bokach rombu. Obliczyć długości tych odcinków. Wykonać odpowiedni rysunek.
  \item Odcinek o końcach $A(-1,-1)$ i $B(3,2)$ jest podstawą trapezu. Druga podstawa jest trzy razy dłuższa i ma środek w punkcie $P(1,5)$. Wyznaczyć współrzędne pozostałych wierzchołków trapezu i obliczyć jego pole.
  \item W okrąg o promieniu 1 wpisujemy trójkąt równoboczny i zakreślamy odcinki koła, które leżą na zewnatrz trójkąta. W otrzymany trójkąt wpisujemy okrąg i powtarzamy procedurę, zaznaczając za każdym razem odcinki kolejnych kół znajdujące się poza kolejnym trójkątem. Obliczyć pole zaznaczonego obszaru po sześciu krokach, czyli po narysowaniu sześciu trójkątów.
  \item Sześcian podzielono na dwie bryły płaszczyzną przechodzącą przez krawędź podstawy. Jedna część ma 5, a druga 6 ścian. Pole powierzchni całkowitej bryły, która ma 5 ścian jest równa połowie pola powierzchni sześcianu. Wyznaczyć tangens kąta nachylenia płaszczyzny dzielącej sześcian do płaszczyzny podstawy.
  \item Rozważamy zbiór liczb całkowitych dodatnich równych co najwyżej 1800 , które nie dzielą się ani przez 5 ani przez 6. Obliczyć sumę liczb z tego zbioru. Ile w tym zbiorze jest liczb parzystych, a ile nieparzystych?
\end{enumerate}

\section*{PRACA KONTROLNA nr 4 - POZIOM RoZsZERzony}
\begin{enumerate}
  \item Punkty $A(2,0)$ i $B(0,2)$ są wierzchołkami podstawy trójkąta równoramiennego. Znaleźć współrzędne wierzchołka $C$, wiedząc, że środkowe $A D$ i $B E$ są prostopadłe.
  \item Trzy pierwiastki wielomianu o współczynnikach całkowitych tworzą ciag arytmetyczny. Suma tych pierwiastków jest równa 21, a iloczyn 315. Pokazać, że wartość wielomianu w dowolnym punkcie, który jest liczbą nieparzystą, jest podzielna przez 48.
  \item W trójkącie równobocznym o boku długości a przeprowadzamy prostą przechodząca przez środek wysokości nachyloną do niej pod kątem $30^{\circ}$. Odcina ona od trójkąta trapez. Obliczyć pole i obwód tego trapezu oraz objętość i pole powierzchni bryły powstałej z jego obrotu dookoła dłuższej podstawy.
  \item W trójkąt równoboczny o boku długości 1 wpisano kwadrat. Następnie w pozostałą część (nad kwadratem) znów wpisano kwadrat, itd. Jaką długość ma bok kwadratu w $n$-tym kroku? Podać wzór ciągu $P_{n}$ określającego sumę pól wpisanych kwadratów po $n$ krokach, a następnie obliczyć jego granicę.
  \item W okrąg o promieniu $r$ wpisano trapez, którego podstawą jest średnica okręgu. Dla jakiego kąta przy podstawie pole trapezu jest największe?
  \item Znaleźć dziedzinę oraz przedziały monotoniczności funkcji
\end{enumerate}

$$
f(x)=1+\frac{2 x}{x^{2}-3}+\left(\frac{2 x}{x^{2}-3}\right)^{2}+\ldots
$$

Naszkicować wykres tej funkcji oraz zbadać liczbę rozwiązań równania $f(x)=m \mathrm{w}$ zależności od parametru $m$.

Rozwiązania (rękopis) zadań z wybranego poziomu prosimy nadsyłać do 18 grudnia 2015r. na adres:

Wydział Matematyki\\
Politechnika Wrocławska\\
Wybrzeże Wyspiańskiego 27\\
50-370 WROCEAW.\\
Na kopercie prosimy koniecznie zaznaczyć wybrany poziom! (np. poziom podstawowy lub rozszerzony). Do rozwiązań należy dołączyć zaadresowaną do siebie kopertę zwrotną z naklejonym znaczkiem, odpowiednim do wagi listu. Prace niespełniające podanych warunków nie będą poprawiane ani odsyłane.

Adres internetowy Kursu: \href{http://www.im.pwr.wroc.pl/kurs}{http://www.im.pwr.wroc.pl/kurs}


\end{document}