\documentclass[10pt]{article}
\usepackage[polish]{babel}
\usepackage[utf8]{inputenc}
\usepackage[T1]{fontenc}
\usepackage{amsmath}
\usepackage{amsfonts}
\usepackage{amssymb}
\usepackage[version=4]{mhchem}
\usepackage{stmaryrd}
\usepackage{hyperref}
\hypersetup{colorlinks=true, linkcolor=blue, filecolor=magenta, urlcolor=cyan,}
\urlstyle{same}

\title{PRACA KONTROLNA nr 4 - POZIOM PODSTAWOWY }

\author{}
\date{}


\begin{document}
\maketitle
\begin{enumerate}
  \item Dwa samochody wyjechały jednocześnie z jednego miejsca i jadą w tym samym kierunku. Pierwszy jedzie z prędkością $50 \mathrm{~km} / \mathrm{h}$, a drugi z prędkością $40 \mathrm{~km} / \mathrm{h}$. Pół godziny później z tego samego miejsca i w tym samym kierunku wyruszył trzeci samochód, który dopędził pierwszy samochód o 1 godzinę i 30 minut później niż drugi. Z jaką prędkością jechał trzeci samochód?
  \item Proste $y=2, y=2 x+10$ oraz $4 x+3 y=0$ wyznaczają trójkąt $A B C$. Otrzymany trójkąt przekształcono używając najpierw jednokładności o środku $O(0,0)$ i skali $k=3$, a następnie symetrii względem osi $O X$. Wyznaczyć współrzędne trójkąta $A B C$ oraz współrzędne obrazów jego wierzchołków. Obliczyć pole trójkąta $A B C$ i jego obrazu w tym przekształceniu.
  \item Rozważmy zbiór wszystkich prostokątów wpisanych w kwadrat o boku długości $a$ w taki sposób, że boki tego prostokąta są parami równoległe do przekątnych danego kwadratu. Obliczyć długości boków tego prostokąta, który ma największe pole.
  \item Podstawą trójkąta równobocznego jest średnica koła o promieniu $r$. Obliczyć stosunek pola powierzchni części trójkąta leżącej na zewnątrz koła do pola powierzchni części trójkąta leżącej wewnątrz koła.
  \item W stożku pole podstawy, pole powierzchni kuli wpisanej w ten stożek i pole powierzchni bocznej stożka, tworzą ciąg arytmetyczny. Znaleźć cosinus kąta nachylenia tworzącej stożka do płaszczyzny jego podstawy.
  \item Okrąg $O_{1}$ o promieniu 1 jest styczny do ramion kąta o mierze $\frac{\pi}{3}$. Mniejszy od niego okrąg $O_{2}$ jest styczny zewnętrznie do niego i obu ramion tego kąta. Procedurę kontynuujemy. Znaleźć sumę obwodów pięciu otrzymanych kolejno w ten sposób okręgów. Dla jakiego $n$ suma obwodów ciągu tych okręgów jest większa od $\frac{299}{100} \pi$ ?
\end{enumerate}

\section*{PRACA KONTROLNA nr 4 - POZIOM RozsZERzony}
\begin{enumerate}
  \item Do punktu $A$ po dwóch prostoliniowych drogach jadą ze stałymi prędkościami samochód i rower. W chwili początkowej samochód, rower i punkt $A$ tworzą trójkąt prostokątny. Gdy samochód przejechał 25 km trójkąt, którego dwa wierzchołki przesunęły się, stał się trójkątem równobocznym. Znaleźć odległość między samochodem a rowerem w chwili początkowej, jeśli w momencie dotarcia samochodu do punktu $A$ rower miał jeszcze do przejechania 12 km .
  \item Na płaszczyźnie dane są punkty $A$ i $B$. Udowodnij, że złożenie symetrii środkowej względem punktu $A$ z przesunięciem o wektor $\overrightarrow{A B}$ jest symetrią środkową względem środka odcinka $\overline{A B}$.
  \item Wyznaczyć największą wartość pola prostokąta, którego dwa wierzchołki leżą na paraboli $y=x^{2}-4 x+4$, a dwa pozostałe na cięciwie paraboli wyznaczonej przez prostą $y=3$.
  \item Suma trzech początkowych wyrazów nieskończonego ciągu geometrycznego wynosi 6 , a suma $S$ wszystkich wyrazów tego ciągu równa się $\frac{16}{3}$. Dla jakich $n$ naturalnych spełniona jest nierówność $\left|S-S_{n}\right|<\frac{1}{96}$ ?
  \item Dwa jednakowe stożki złożono podstawami. Obliczyć objętość powstałej bryły, jeśli promień kuli wpisanej w tę bryłę wynosi $R$, a punkt styczności kuli i stożka dzieli tworzącą stożka w stosunku $m$ do $n$ ?
  \item W czworościan foremny $A B C D$ o krawędzi długości $d$ wpisano kulę. Prowadzimy płaszczyzny równoległe do ścian czworościanu i styczne do wpisanej kuli odcinając w ten sposób cztery przystające czworościany foremne. W każdy z nich wpisujemy kulę i postępujemy analogicznie jak z kulą wpisaną w czworościan $A B C D$. Obliczyć sumę objętości wszystkich kul wpisanych w otrzymane czworościany, jeśli proces ten kontynuujemy nieskończenie wiele razy.
\end{enumerate}

Rozwiązania (rękopis) zadań z wybranego poziomu prosimy nadsyłać do 18 grudnia 2016r. na adres:

Wydział Matematyki\\
Politechnika Wrocławska\\
Wybrzeże Wyspiańskiego 27\\
50-370 WROCEAW.\\
Na kopercie prosimy koniecznie zaznaczyć wybrany poziom! (np. poziom podstawowy lub rozszerzony). Do rozwiązań należy dołączyć zaadresowaną do siebie kopertę zwrotną z naklejonym znaczkiem, odpowiednim do wagi listu. Prace niespełniające podanych warunków nie będą poprawiane ani odsyłane.

Adres internetowy Kursu: \href{http://www.wmat.pwr.wroc.pl/kurs}{http://www.wmat.pwr.wroc.pl/kurs}


\end{document}