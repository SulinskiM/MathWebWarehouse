\documentclass[10pt]{article}
\usepackage[polish]{babel}
\usepackage[utf8]{inputenc}
\usepackage[T1]{fontenc}
\usepackage{amsmath}
\usepackage{amsfonts}
\usepackage{amssymb}
\usepackage[version=4]{mhchem}
\usepackage{stmaryrd}
\usepackage{hyperref}
\hypersetup{colorlinks=true, linkcolor=blue, filecolor=magenta, urlcolor=cyan,}
\urlstyle{same}

\title{PRACA KONTROLNA nr 4 - POZIOM PODSTAWOWY }

\author{}
\date{}


\begin{document}
\maketitle
\begin{enumerate}
  \item Rodzina składa się z pięciorga dzieci i dwojga rodziców. Załóżmy, że dzieci nie mogą wyjść na spacer ani nie mogą zostać w domu bez opieki któregokolwiek z rodziców. W ilu możliwych kombinacjach dzieci mogą wyjść na spacer zakładając, że przynajmniej jedno dziecko idzie na spacer?
  \item Na bokach prostokąta o stałym obwodzie $4 p$ opisano na średnicach półokręgi leżące na zewnątrz prostokąta. Dla jakich wartości boków prostokąta pole figury ograniczonej krzywą złożoną z tych czterech półokręgów jest najmniejsze? Wykonać staranny rysunek.
  \item Punkty $A(1,3), B(5,1), C(4,4)$ są wierzchołkami trójkąta. Obliczyć stosunek pola koła opisanego na tym trójkącie do pola koła wpisanego w ten trójkąt.
  \item Liczby $x_{1}, x_{2}$ są pierwiastkami równania $x^{2}-3 x+A=0$, a liczby $x_{3}, x_{4}$ pierwiastkami równania $x^{2}-12 x+B=0$. Wiadomo, że liczby $x_{1}, x_{2}, x_{3}, x_{4}$ tworzą ciąg geometryczny. Znaleźć ten ciąg oraz liczby $A$ i $B$.
  \item Rozwiązać układ równań:
\end{enumerate}

$$
\left\{\begin{array}{l}
x^{2}+y^{2}-2 x-4 y+1=0 \\
|x-1|-y=0
\end{array}\right.
$$

a następnie obliczyć pole obszaru, który jest rozwiązaniem układu nierówności:

$$
\left\{\begin{array}{l}
x^{2}+y^{2}-2 x-4 y+1 \leqslant 0 \\
|x-1|-y \leqslant 0
\end{array}\right.
$$

Sporządzić staranny rysunek.\\
6. W graniastosłupie prawidłowym czworokątnym okrąg styczny do dwóch boków podstawy i przechodzący przez jej wierzchołek nieleżący na żadnym z tych boków ma promień $r=2$. Płaszczyzna przechodząca przez środki krawędzi wychodzących z jednego wierzchołka graniastosłupa tworzy z płaszczyzną jego podstawy kąt $45^{\circ}$. Obliczyć objętość graniastosłupa.

\section*{PRACA KONTROLNA nr 4 - POZIOM RoZsZERzoNY}
\begin{enumerate}
  \item Na ile sposbów można umieścić 6 osób w pokojach dwuosobowych przy założeniu, że pewne dwie ustalone osoby nie chcą mieszkać razem oraz że a) pokoje są jednakowe, a więc ważne jest kto mieszka z kim, ale nieważne w którym pokoju; b) pokoje są istotnie różne, a więc ważne jest kto mieszka w którym pokoju?
  \item Rozwiązać następującą nierówność
\end{enumerate}

$$
\cos ^{2} x+\cos ^{3} x+\cos ^{4} x+\ldots<\cos x+1
$$

dla $x \in[0,2 \pi]$.\\
3. Pokazać, że dla każdej wartości parametru $m$ wielomian

$$
w(x)=x^{3}+(2 m-1) x^{2}-(3+2 m) x+3
$$

ma pierwiastek całkowity. Dla jakich wartości parametru $m$ pierwiastki tego wielomianu tworzą ciąg arytmetyczny?\\
4. Punkt $A$ należy do obszaru kąta o mierze stopniowej 60. Odległości tego punktu od ramion kąta są równe $a$ i $b$. Wyznaczyć odległość punktu $A$ od wierzchołka kąta. Następnie obliczyć tę odległość dla $a=2$ i $b=\sqrt{3}-1$.\\
5. Z punktu $A(1,1)$ wychodzą dwie półproste prostopadłe przecinające oś $O X$ układu współrzędnych. Niech $F$ będzie obszarem kąta prostego wyznaczonego przez te półproste, $G$ zaś zbiorem wszystkich punktów o obydwóch współrzędnych nieujemnych. Wyznaczyć położenie półprostych, dla których pole figury $F \cap G$ jest najmniejsze.\\
6. Znaleźć najmniejszą możliwą objętość stożka opisanego na walcu, którego przekrojem osiowym jest kwadrat o boku 2.

Rozwiązania (rękopis) zadań z wybranego poziomu prosimy nadsyłać do 18 grudnia 2017 r . na adres:

\begin{verbatim}
Wydział Matematyki
Politechniki Wrocławskiej
Wybrzeże Wyspiańskiego 27
50-370 WROCEAW.
\end{verbatim}

Na kopercie prosimy koniecznie zaznaczyć wybrany poziom! (np. poziom podstawowy lub rozszerzony). Do rozwiązań należy dołączyć zaadresowaną do siebie kopertę zwrotną z naklejonym znaczkiem, odpowiednim do wagi listu. Prace niespełniające podanych warunków nie będą poprawiane ani odsyłane.

Adres internetowy Kursu: \href{http://www.im.pwr.edu.pl/kurs}{http://www.im.pwr.edu.pl/kurs}


\end{document}