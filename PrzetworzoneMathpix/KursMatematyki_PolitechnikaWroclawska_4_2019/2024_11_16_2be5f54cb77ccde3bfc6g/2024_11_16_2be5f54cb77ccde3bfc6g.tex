\documentclass[10pt]{article}
\usepackage[polish]{babel}
\usepackage[utf8]{inputenc}
\usepackage[T1]{fontenc}
\usepackage{amsmath}
\usepackage{amsfonts}
\usepackage{amssymb}
\usepackage[version=4]{mhchem}
\usepackage{stmaryrd}
\usepackage{hyperref}
\hypersetup{colorlinks=true, linkcolor=blue, filecolor=magenta, urlcolor=cyan,}
\urlstyle{same}

\title{PRACA KONTROLNA nr 4 - POZIOM PODSTAWOWY }

\author{}
\date{}


\begin{document}
\maketitle
\begin{enumerate}
  \item Rozwiązać nierówność $\sqrt{2^{x}-1} \leqslant 2^{x}-3$.
  \item Trójkąt prostokątny o przyprostokątnych $a, b$ obracamy wokół każdej z przyprostokątnych. Obliczyć stosunek sumy objętości tych stożków do objętości bryły otrzymanej przez obrót trójkąta wokół przeciwprostokątnej i wyrazić go jako funkcję zmiennej $\frac{a}{b}$.
  \item Punkty $(-1,1),(0,0),(\sqrt{2}, 0)$ są trzema kolejnymi wierzchołkami wielokąta foremnego. Wyznaczyć współrzędne pozostałych wierzchołków wielokąta oraz jego pole. Podać równania okręgów wpisanego i opisanego na tym wielokącie oraz wyznaczyć stosunek ich promieni.
  \item Niech $f(x)=\left\{\begin{array}{rll}\frac{2-|x|}{|x|-1} & \text { gdy } & |x|>\frac{3}{2} . \\ \frac{8}{9} x^{2}-1 & \text { gdy } & |x| \leqslant \frac{3}{2} .\end{array}\right.$\\
a) Narysować wykres funkcji $f$ i na jego podstawie wyznaczyć zbiór wartości funkcji.\\
b) Obliczyć $f(\sqrt{2})$ oraz $f(\sqrt{3})$.\\
c) Rozwiązać nierówność $f(x) \leqslant-\frac{1}{2}$ i zaznaczyć na osi $0 x$ zbiór rozwiązań.
  \item Punkty $A(0,1), B(4,3)$ są dwoma kolejnymi wierzchołkami równoległoboku $A B C D$, a $S(2,3)$ punktem przecięcia przekątnych. Posługując się rachunkiem wektorowym, wyznaczyć pozostałe wierzchołki równoległoboku oraz wierzchołki równoległoboku otrzymanego przez obrót $A B C D$ wokół punktu $A$ o $90^{\circ}$ w kierunku przeciwnym do ruchu wskazówek zegara.
  \item Ostrosłup prawidłowy trójkątny, w którym bok podstawy i wysokość są równe $a$ przecięto płaszczyzną przechodzącą przez jedną z krawędzi podstawy na dwie bryły o tej samej objętości. Wyznaczyć tangens kąta nachylenia tej płaszczyzny do płaszczyzny podstawy. Sporządzić rysunek.
\end{enumerate}

\section*{PRACA KONTROLNA nr 4 - POZIOM ROZSZERZONY}
\begin{enumerate}
  \item Punkty $A(0,1), B(4,3)$ są dwoma kolejnymi wierzchołkami równoległoboku $A B C D$, a $S(2,3)$ punktem przecięcia przekątnych. Posługując się rachunkiem wektorowym, wyznaczyć pozostałe wierzchołki równoległoboku oraz wierzchołki równoległoboku $A^{\prime} B^{\prime} C^{\prime} D^{\prime}$ otrzymanego przez obrót $A B C D$ o kąt $90^{\circ}$ wokół punktu $(0,0)$ w kierunku przeciwnym do ruchu wskazówek zegara. Sprawdzić, że $A^{\prime} B^{\prime} C^{\prime} D^{\prime}$ jest obrazem $A B C D$ w przekształceniu $T_{2} \circ O \circ T_{1}$, gdzie $T_{1}$ jest przesunięciem o wektor $[0,1], O$ - obrotem o kąt $90^{\circ}$ wokół punktu $(0,0)$ w kierunku przeciwnym do ruchu wskazówek zegara, a $T_{2}$ - przesunięciem o wektor $[1,0]$.
  \item Narysować wykres funkcji
\end{enumerate}

$$
f(x)=1-\frac{2^{x}}{3^{x}-2^{x}}+\left(\frac{2^{x}}{3^{x}-2^{x}}\right)^{2}-\ldots
$$

i uzasadnić, że przyjmuje ona wyłącznie wartości większe niż $\frac{1}{2}$.\\
3. Niech $f(x)=\left\{\begin{array}{rll}\left|2^{x}-1\right| & \text { dla } & x \leqslant 1, \\ \log _{\frac{1}{2}}\left(x-\frac{1}{2}\right) & \text { dla } & x>1 .\end{array}\right.$\\
a) Narysować wykres funkcji $f$ i na jego podstawie wyznaczyć zbiór wartości funkcji.\\
b) Obliczyć $f\left(\log _{\frac{1}{2}}\left(\sqrt{2}-\frac{1}{2}\right)\right)$ oraz $f\left(2^{\sqrt{2}}+\frac{1}{2}\right)$.\\
c) Rozwiązać nierówność $f(x) \leqslant \frac{1}{2}$ i zaznaczyć na osi $0 x$ zbiór rozwiązań.\\
4. Punkt $C(0,0)$ jest wierzchołkiem trójkąta równoramiennego, w którym środkowa podstawy $A B$ i wysokość poprowadzona z jednego z wierzchołków $A, B$ przecinają się w punkcie $S(2,1)$. Pole trójkąta $A B S$ jest dwa razy mniejsze niż pole trójkąta $A B C$. Wyznaczyć współrzędne wierzchołków $A, B$ oraz równanie okręgu opisanego na trójkącie $A B C$.\\
5. W ośmiościan foremny wpisano dwa sześciany. Wierzchołki pierwszego z nich leżą na krawędziach ośmiościanu, a wierzchołki drugiego - na wysokościach ścian bocznych. Obliczyć stosunek objętości tych sześcianów.\\
6. Prostokąt o bokach $a$ i $2 a$ obraca się wokół przekątnej. Obliczyć pole powierzchni całkowitej i objętość otrzymanej bryły.

Rozwiązania (rękopis) zadań z wybranego poziomu prosimy nadsyłać do 18 grudnia 2019r. na adres:

Wydział Matematyki\\
Politechnika Wrocławska\\
Wybrzeże Wyspiańskiego 27\\
50-370 WROCEAW.\\
Na kopercie prosimy koniecznie zaznaczyć wybrany poziom! (np. poziom podstawowy lub rozszerzony). Do rozwiązań należy dołączyć zaadresowaną do siebie kopertę zwrotną z naklejonym znaczkiem, odpowiednim do wagi listu. Prace niespełniające podanych warunków nie będą poprawiane ani odsyłane.

Uwaga. Wysyłając nam rozwiązania zadań uczestnik Kursu udostępnia Politechnice Wrocławskiej swoje dane osobowe, które przetwarzamy wyłącznie w zakresie niezbędnym do jego prowadzenia (odesłanie zadań, prowadzenie statystyki). Szczegółowe informacje o przetwarzaniu przez nas danych osobowych są dostepne na stronie internetowej Kursu.\\
Adres internetowy Kursu: \href{http://www.im.pwr.edu.pl/kurs}{http://www.im.pwr.edu.pl/kurs}


\end{document}