% This LaTeX document needs to be compiled with XeLaTeX.
\documentclass[10pt]{article}
\usepackage[utf8]{inputenc}
\usepackage{amsmath}
\usepackage{amsfonts}
\usepackage{amssymb}
\usepackage[version=4]{mhchem}
\usepackage{stmaryrd}
\usepackage{hyperref}
\hypersetup{colorlinks=true, linkcolor=blue, filecolor=magenta, urlcolor=cyan,}
\urlstyle{same}
\usepackage[fallback]{xeCJK}
\usepackage{polyglossia}
\usepackage{fontspec}
\setCJKmainfont{Noto Serif CJK JP}

\setmainlanguage{polish}
\setmainfont{CMU Serif}

\title{PRACA KONTROLNA nr 4 - POZIOM PODSTAWOWY }

\author{}
\date{}


\begin{document}
\maketitle
\begin{enumerate}
  \item Wyznacz miarę kąta ostrego $\alpha$, wiedząc, że $\cos \alpha+\sin \alpha=\frac{1}{\sin \alpha}$.
  \item Dane są wierzchołki $A(-1,-2)$ i $B(6,-1)$ równoległoboku, którego przekątne przecinają się w punkcie $S(4,0)$. Wyznacz współrzędne pozostałych wierzchołków i oblicz pole równoległoboku.
  \item Trójkąt prostokątny o polu 30 jest opisany na okręgu o promieniu 2. Wyznacz długości jego boków.
  \item Cięciwy $A B$ i $C D$ (punkt $C$ leży na łuku $A B$ ) przecinają się pod kątem prostym w punkcie $S$. Pole trójkąta $B S D$ jest równe 4, a pole trójkąta $A S C$ wynosi 9. Oblicz pole czworokąta $A D B C$, jeżeli suma długości tych cięciw jest równa 15 .
  \item Dane są punkty $A(8,2)$ i $B(1,6)$. Punkt $C$ leży na jednej z osi układu i jest wierzchołkiem kąta prostego w trójkącie $A B C$. Wyznacz współrzędne punku $C$.
  \item W ostrosłupie prawidłowym trójkątnym zachodzi równość $\cos \alpha=\sqrt{3} \cos \beta$, gdzie $\alpha$ jest kątem nachylenia krawędzi bocznej, a $\beta$ - kątem nachylenia ściany bocznej do podstawy. Wykaż, że ten ostrosłup jest czworościanem foremnym.
\end{enumerate}

\section*{PRACA KONTROLNA nr 4 - POZIOM ROZSZERZONY}
\begin{enumerate}
  \item Wiedząc, że $\sin 2 x=-\frac{3}{4}$ i $x \in\left(\frac{\pi}{2}, \pi\right)$, oblicz wartość wyrażenia
\end{enumerate}

$$
\frac{\sin \left(3 x+30^{\circ}\right)-\sin \left(x-30^{\circ}\right)}{4 \cos ^{2} x-2} .
$$

\begin{enumerate}
  \setcounter{enumi}{1}
  \item Wektory $\vec{u}, \vec{v}$ mają długość 1 i tworzą kąt $60^{\circ}$. Oblicz długości przekątnych równoległoboku rozpiętego na wektorach $(2 \vec{u}-\vec{v})$ i $(\vec{u}-2 \vec{v})$. Wyznacz jego kąt ostry i sprawdź, czy można w ten równoległobok wpisać okrąg. Jeżeli tak, to oblicz jego promien.
  \item Przekątne trapezu $A B C D$ przecinają się w takim punkcie $P$, że
\end{enumerate}

$$
|A P|^{2}+|B P|^{2}-|A B|^{2}=\frac{2 \sqrt{5}}{3}|A P||B P| .
$$

O ile dłuższy jest promień okręgu opisanego na trójkącie $A B P$ od promienia okręgu opisanego na trójkącie $P C D$, jeżeli $|A B|-|C D|=4$ ?\\
4. Na okręgu $x^{2}+y^{2}-2 x-2 y=0$, opisany jest trapez prostokątny $A B C D$ o polu 12 . Wyznacz współrzędne wierzchołków trapezu, wiedząc, że większa z jego podstaw $A B$ jest zawarta jest w prostej $x+y=0$, a kąt przy wierzchołku $A$ jest prosty.\\
5. W trójkącie równoramiennym $A B C$ kąt przy wierzchołku $C$ ma miarę $20^{\circ}$. Z wierzchołków $A$ i $B$ poprowadzono półproste pod kątami $50^{\circ}$ i $60^{\circ}$ względem podstawy, przecinające ramiona $A C$ i $B C$ w punktach $D$ i $E$ odpowiednio. Wyznacz miarę kąta $B D E$. WSK. Poprowadź półprostą z punktu $A$ przecinającą odcinek $B D$ w punkcie $G$, a bok $B C$ w takim punkcie $F$, że $\angle B A F=60^{\circ}$ i przyjrzyj się czworokątowi $D G E F$.\\
6. W ostrosłupie prawidłowym trójkątnym krawędź boczna jest dwa razy dłuższa niż krawędź podstawy. Wyznacz cosinus kąta między ścianami bocznymi ostrosłupa oraz stosunek promienia kuli opisanej na ostrosłupie do promienia kuli wpisanej w ostrosłup.

Rozwiązania (rękopis) zadań z wybranego poziomu prosimy nadsyłać do 31.12.2022r. na adres:

Wydział Matematyki\\
Politechnika Wrocławska\\
Wybrzeże Wyspiańskiego 27\\
50-370 WROCモAW,\\
lub elektronicznie, za pośrednictwem portalu \href{http://talent.pwr.edu.pl}{talent.pwr.edu.pl}\\
Na kopercie prosimy koniecznie zaznaczyć wybrany poziom! (np. poziom podstawowy lub rozszerzony). Do rozwiązań należy dołączyć zaadresowaną do siebie kopertę zwrotną z naklejonym znaczkiem, odpowiednim do formatu listu. Prace niespełniające podanych warunków nie będą poprawiane ani odsyłane.

Uwaga. Wysyłając nam rozwiązania zadań uczestnik Kursu udostępnia Politechnice Wrocławskiej swoje dane osobowe, które przetwarzamy wyłącznie w zakresie niezbędnym do jego prowadzenia (odesłanie zadań, prowadzenie statystyki). Szczegółowe informacje o przetwarzaniu przez nas danych osobowych są dostępne na stronie internetowej Kursu.\\
Adres internetowy Kursu: \href{http://www.im.pwr.edu.pl/kurs}{http://www.im.pwr.edu.pl/kurs}


\end{document}