\documentclass[10pt]{article}
\usepackage[polish]{babel}
\usepackage[utf8]{inputenc}
\usepackage[T1]{fontenc}
\usepackage{amsmath}
\usepackage{amsfonts}
\usepackage{amssymb}
\usepackage[version=4]{mhchem}
\usepackage{stmaryrd}
\usepackage{hyperref}
\hypersetup{colorlinks=true, linkcolor=blue, filecolor=magenta, urlcolor=cyan,}
\urlstyle{same}

\title{PRACA KONTROLNA nr 5 - POZIOM PODSTAWOWY }

\author{}
\date{}


\begin{document}
\maketitle
\begin{enumerate}
  \item Udowodnić, że różnica kwadratów dwu dowolnych liczb całkowitych niepodzielnych przez 3 jest podzielna przez 3.
  \item Rozwiązać równanie
\end{enumerate}

$$
\sin ^{2}\left(\frac{\pi+x}{2}\right)-\sin \left(\frac{\pi-x}{2}\right)+\sin ^{2}\left(\frac{\pi-x}{2}\right)=1
$$

w przedziale $[0,2 \pi]$.\\
3. Dla jakiego parametru $m$ równanie

$$
\left(\log _{2}^{2} m-1\right) \cdot x^{2}+2\left(\log _{2} m-1\right) \cdot x+2=0
$$

ma tylko jedno rozwiązanie?\\
4. Jedna z krawędzi bocznych ostrosłupa, którego podstawą jest kwadrat o boku $a$, jest prostopadła do podstawy. Najdłuższa krawędź boczna jest nachylona do podstawy pod kątem $60^{\circ}$. Obliczyć pole powierzchni całkowitej oraz sumę długości krawędzi ostrosłupa. Sporządzić rysunek.\\
5. Jaką krzywą tworzą punkty płaszczyzny, z których odcinek o końcach $A(1,0)$ i $B(0,1)$ jest widoczny pod kątem $30^{\circ}$.\\
6. Narysować wykres funkcji $f(x)=\frac{|x+1|-1}{|x-1|}$ i na jego podstawie wyznaczyć przedziały jej monotoniczności oraz najmniejszą wartość w przedziale $\left[-2, \frac{1}{2}\right]$.

\section*{PRACA KONTROLNA nr 4 - POZIOM ROZSZERZONY}
\begin{enumerate}
  \item Udowodnić tożsamość
\end{enumerate}

$$
x^{3}+y^{3}+z^{3}-3 x y z=(x+y+z)\left(x^{2}+y^{2}+z^{2}-x y-x z-y z\right)
$$

i wykorzystując ją, usunąć niewymierność z mianownika ułamka $\frac{1}{1+\sqrt[3]{2}+\sqrt[3]{4}}$.\\
2. Jaką krzywą tworzą środki wszystkich okręgów stycznych równocześnie do osi $O x$ i do okregu $x^{2}+(y-1)^{2}=1$ ?\\
3. Wyznaczyć wszystkie styczne do wykresu funkcji $f(x)=\frac{x-3}{x-2}$ prostopadłe do prostej $4 x+y+1=0$. Sporządzić staranne wykresy wszystkich rozważanych krzywych.\\
4. Narysować wykres funkcji

$$
f(x)=1-\frac{2^{x}}{1-2^{x}}+\left(\frac{2^{x}}{1-2^{x}}\right)^{2}-\left(\frac{2^{x}}{1-2^{x}}\right)^{3}+\left(\frac{2^{x}}{1-2^{x}}\right)^{4}-\ldots
$$

będącej sumą nieskończonego szeregu geometrycznego. Rozwiązać nierówność

$$
f(x)>4^{x}-21 \cdot 2^{x-2}+2 .
$$

\begin{enumerate}
  \setcounter{enumi}{4}
  \item Dla jakiego parametru $m$ równanie
\end{enumerate}

$$
\left(\log _{4} m+1\right) \cdot x^{2}+3 \log _{4} m \cdot x+2 \log _{4} m-1=0
$$

ma dwa pierwiastki $x_{1}, x_{2}$ spełniające warunki: $x_{1}<x_{2}$, oraz $2\left(x_{2}^{2}-x_{1}^{2}\right)>x_{2}^{4}-x_{1}^{4}$ ?\\
6. W ostrosłupie prawidłowym trójkątnym tangens kąta nachylenia ściany bocznej do podstawy jest równy $2 \sqrt{3}$. Obliczyć stosunek objętości kuli wpisanej do objętości kuli opisanej na ostrosłupie.

Rozwiązania (rękopis) zadań z wybranego poziomu prosimy nadsyłać do 18 stycznia 2016 r. na adres:

Wydział Matematyki\\
Politechnika Wrocławska\\
Wybrzeże Wyspiańskiego 27\\
50-370 WROCEAW.\\
Na kopercie prosimy koniecznie zaznaczyć wybrany poziom! (np. poziom podstawowy lub rozszerzony). Do rozwiązań należy dołączyć zaadresowaną do siebie kopertę zwrotną z naklejonym znaczkiem, odpowiednim do wagi listu. Prace niespełniające podanych warunków nie będą poprawiane ani odsyłane.

Adres internetowy Kursu: \href{http://www.im.pwr.wroc.pl/kurs}{http://www.im.pwr.wroc.pl/kurs}


\end{document}