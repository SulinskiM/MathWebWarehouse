\documentclass[10pt]{article}
\usepackage[polish]{babel}
\usepackage[utf8]{inputenc}
\usepackage[T1]{fontenc}
\usepackage{amsmath}
\usepackage{amsfonts}
\usepackage{amssymb}
\usepackage[version=4]{mhchem}
\usepackage{stmaryrd}
\usepackage{hyperref}
\hypersetup{colorlinks=true, linkcolor=blue, filecolor=magenta, urlcolor=cyan,}
\urlstyle{same}

\title{PRACA KONTROLNA nr 6 - POZIOM PODSTAWOWY }

\author{}
\date{}


\begin{document}
\maketitle
\begin{enumerate}
  \item Wyznacz dziedzinę funkcji
\end{enumerate}

$$
f(x)=\log _{4-x^{2}}\left(2^{x}+2^{1-x}-3\right)
$$

\begin{enumerate}
  \setcounter{enumi}{1}
  \item W przedziale $[0,2 \pi]$ rozwiąż nierówność
\end{enumerate}

$$
\cos ^{2} 2 x+\sin ^{2} x \leqslant \frac{1}{2}
$$

\begin{enumerate}
  \setcounter{enumi}{2}
  \item Obwód trójkąta równoramiennego jest równy 8. Jaka powinna być długość boków tego trójkąta, by objętość bryły powstałej z jego obrotu dokoła podstawy była największa?
  \item Rozwiąż równanie
\end{enumerate}

$$
\sqrt{1-2 \cdot 3^{x}+9^{x}}=3^{2 x-1}-7 \cdot 3^{x-1}+2
$$

\begin{enumerate}
  \setcounter{enumi}{4}
  \item Punkt $B(1,1)$ jest wierzchołkiem kąta prostego w trójkącie prostokątnym o polu 2, wpisanym w okrąg $x^{2}+y^{2}+2 x-2 y-2=0$. Znajdź współrzędne pozostałych wierzchołków tego trójkąta. Rozwiązanie zilustruj starannym rysunkiem.
  \item Sporządź staranny wykres funkcji
\end{enumerate}

$$
f(x)=\left\{\begin{array}{lll}
\frac{x}{x-2} & \text { dla } & |2 x-5| \geqslant 3 \\
-x^{2}+6 x-6 & \text { dla } & |2 x-5|<3
\end{array}\right.
$$

i na jego podstawie wyznacz zbiór wartości tej funkcji. Rozwiąż nierówność $f^{2}(x) \leqslant 1$ i zaznacz zbiór jej rozwiązań na osi $0 x$.

\section*{PRACA KONTROLNA nr 6 - POZIOM RozsZERzony}
\begin{enumerate}
  \item Narysuj staranny wykres funkcji
\end{enumerate}

$$
f(x)=\left|2^{|x-1|}-4\right|-2
$$

i opisz dokładnie sposób jego konstrukcji. Korzystając z rysunku, określ ilość rozwiązań równania $f(x)=m$ w zależności od parametru $m$.\\
2. Rozwiąż równanie

$$
2 \cos 2 x+1=\sqrt{2 \cos ^{2} 2 x-6 \sin ^{2} x+5}
$$

\begin{enumerate}
  \setcounter{enumi}{2}
  \item W trójkącie prostokątnym przeciwprostokątna ma długość 3. Jakie powinny być długości przyprostokątnych, aby objętość bryły powstałej z jego obrotu dokoła jednej z nich była największa?
  \item Rozwiąż nierówność
\end{enumerate}

$$
2^{x}\left(1+\frac{\sqrt{3}}{2}\right)^{\frac{1}{x}}-(2-\sqrt{3})^{-x} \geqslant 0
$$

\begin{enumerate}
  \setcounter{enumi}{4}
  \item Znajdź równania prostych stycznych do okręgu $x^{2}+y^{2}=25$ przechodzących przez punkt $S(6,8)$. Wyznacz współrzędne punktów styczności $A, B$ i oblicz pole obszaru ograniczonego odcinkami $A S, B S$ oraz większym łukiem $A B$. Wykonaj staranny rysunek.
  \item Zbadaj przebieg zmienności i narysuj staranny wykres funkcji
\end{enumerate}

$$
f(x)=\frac{3 x-2}{(x-1)^{2}}
$$

Rozwiązania (rękopis) zadań z wybranego poziomu prosimy nadsyłać do 18 lutego 2015r. na adres:

\begin{verbatim}
Katedra Matematyki WPPT
Politechniki Wrocławskiej
Wybrzeże Wyspiańskiego 27
50-370 WROCEAW.
\end{verbatim}

Na kopercie prosimy koniecznie zaznaczyć wybrany poziom! (np. poziom podstawowy lub rozszerzony). Do rozwiązań należy dołączyć zaadresowaną do siebie kopertę zwrotną z naklejonym znaczkiem, odpowiednim do wagi listu. Prace niespełniające podanych warunków nie będą poprawiane ani odsyłane.

Adres internetowy Kursu: \href{http://www.im.pwr.wroc.pl/kurs}{http://www.im.pwr.wroc.pl/kurs}


\end{document}