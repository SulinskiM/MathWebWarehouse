\documentclass[10pt]{article}
\usepackage[polish]{babel}
\usepackage[utf8]{inputenc}
\usepackage[T1]{fontenc}
\usepackage{amsmath}
\usepackage{amsfonts}
\usepackage{amssymb}
\usepackage[version=4]{mhchem}
\usepackage{stmaryrd}
\usepackage{hyperref}
\hypersetup{colorlinks=true, linkcolor=blue, filecolor=magenta, urlcolor=cyan,}
\urlstyle{same}

\title{PRACA KONTROLNA nr 6 - POZIOM PODSTAWOWY }

\author{}
\date{}


\begin{document}
\maketitle
\begin{enumerate}
  \item Rozwiąż nierówność
\end{enumerate}

$$
\frac{3 x-1}{x} \geqslant 1+\frac{\sqrt{1-x}}{x}
$$

\begin{enumerate}
  \setcounter{enumi}{1}
  \item W zagrodzie jest 10 zwierząt, po parze danego gatunku. Oblicz prawdopodobieństwo, że w zagrodzie zostanie choć jedno zwierzę każdego gatunku, jeśli wypuścimy z niej 4 losowo wybrane zwierzęta.
  \item Bez użycia kalkulatora porównaj liczby
\end{enumerate}

$$
a=\sqrt{11-4 \sqrt{7}} \quad \text { oraz } \quad b=\log ^{2} 2 \cdot \log 250+\log ^{2} 5 \cdot \log 40
$$

\begin{enumerate}
  \setcounter{enumi}{3}
  \item Wyznacz wszystkie argumenty $x$, dla których funkcja
\end{enumerate}

$$
f(x)=27^{x^{2}} \cdot 4^{x^{2}(x-3)} \cdot 3^{x}-6 \cdot 3^{x^{3}+2} \cdot 2^{2 x-7}
$$

przyjmuje wartości dodatnie.\\
5. Wyznacz skalę podobieństwa trójkąta równobocznego opisanego na okręgu do trójkąta równobocznego wpisanego w ten okrąg. Jaki jest stosunek pól tych trójkątów, a jaki stosunek objętości stożka o kącie rozwarcia $60^{\circ}$ opisanego na kuli do objętości podobnego stożka wpisanęgo w tę kulę?\\
6. Wśród prostokątów o ustalonej długości przekątnej $p$ wskaż ten o największym polu.

\section*{PRACA KONTROLNA nr 6 - POZIOM ROZSZERZONY}
\begin{enumerate}
  \item Rozwiąż nierówność
\end{enumerate}

$$
x+1+\frac{1}{x-1} \geqslant\left(1+\frac{1}{x-1}\right) \sqrt{2-x}
$$

\begin{enumerate}
  \setcounter{enumi}{1}
  \item Narysuj wykres funkcji $f(x)=\left|1+\log _{2} \frac{1}{|1-|x||}\right|$, opisz słownie metodę jego konstrukcji oraz zbadaj, dla jakich argumentów spełniona jest nierówność $f(x) \leqslant 1$.
  \item Rozwiąż równanie logarytmiczne
\end{enumerate}

$$
\log _{(x+2)^{2}}|x-1|=\log _{|x-1|} \sqrt{x+2}
$$

\begin{enumerate}
  \setcounter{enumi}{3}
  \item Trzech alpinistów atakuje szczyt, wchodząc jednocześnie, niezależnie od siebie, z różnych stron góry. Prawdopodobieństwo zdobycia szczytu szlakiem północnym wynosi $\frac{1}{3}$, szlakiem zachodnim $-\frac{1}{2}$, a południowym $-\frac{3}{7}$. Oblicz prawdopodobieństwo, że atak się powiedzie (tzn. przynajmniej jeden z alpinistów zdobędzie szczyt).
  \item Oblicz tangens kąta rozwarcia stożka, dla którego kula wpisana w ten stożek zajmuje dokładnie połowę jego objętości.
  \item Wyznacz równanie linii będącej zbiorem środków wszystkich okręgów stycznych do prostej $y=0$ i jednocześnie stycznych do okręgu $x^{2}+y^{2}=2$. Wykonaj odpowiedni rysunek.
\end{enumerate}

Rozwiązania prosimy nadsyłać do dnia 18 lutego 2018 na adres:

\begin{verbatim}
Korespondencyjny Kurs z Matematyki
POZIOM... (wpisać właściwy)
Wydział Matematyki
Politechnika Wrocławska
Wybrzeże Wyspiańskiego 27
50-370 Wrocław
\end{verbatim}

Na kopercie prosimy koniecznie zaznaczyć wybrany poziom (podstawowy, rozszerzony lub podstawowy i rozszerzony). Do rozwiązań należy dołączyć zaadresowaną do siebie kopertę zwrotną z naklejonym znaczkiem, odpowiednim do wagi listu i rozmiaru koperty. Prace niespełniające podanych warunków nie będą poprawiane ani odsyłane.

Adres internetowy Kursu: \href{http://www.im.pwr.edu.pl/kurs}{http://www.im.pwr.edu.pl/kurs}


\end{document}