\documentclass[10pt]{article}
\usepackage[polish]{babel}
\usepackage[utf8]{inputenc}
\usepackage[T1]{fontenc}
\usepackage{amsmath}
\usepackage{amsfonts}
\usepackage{amssymb}
\usepackage[version=4]{mhchem}
\usepackage{stmaryrd}
\usepackage{hyperref}
\hypersetup{colorlinks=true, linkcolor=blue, filecolor=magenta, urlcolor=cyan,}
\urlstyle{same}

\title{PRACA KONTROLNA nr 7 - POZIOM PODSTAWOWY }

\author{}
\date{}


\begin{document}
\maketitle
\begin{enumerate}
  \item Wielomian $W(x)=x^{3}-(k+m) x^{2}-(k-m) x+3$ jest podzielny przez dwumian $(x-1)$, a suma jego współczynników przy parzystych potęgach zmiennej $x$ jest równa sumie współczynników przy nieparzystych potęgach zmiennej. Rozwiąż nierówność
\end{enumerate}

$$
W(x) \leqslant x^{2}-1
$$

\begin{enumerate}
  \setcounter{enumi}{1}
  \item Rozwiąż algebraicznie układ równań $\left\{\begin{array}{l}|y|=2-x^{2}, \\ x^{2}+y^{2}=2\end{array}\right.$ a następnie podaj jego interpretację geometryczną.
  \item W przedziale $[0,2 \pi]$ określ liczbę rozwiązań równania
\end{enumerate}

$$
\cos x \cdot \operatorname{ctg} x-\sin x=a \cos 2 x
$$

w zależności od parametru $a$.\\
4. Niech $P(k)$ oznacza pole trójkąta ograniczonego prostą $y=k x$ i wykresem funkcji

$$
f(x)=4-2|x| .
$$

Wyznacz najmniejszą wartość $P(k)$.\\
5. Punkty $A(0,0)$ i $B(4,3)$ są wierzchołkami rombu o kącie ostrym $45^{\circ}$, który zawarty jest w pierwszej ćwiartce układu współrzędnych. Wyznacz współrzędne jego wierzchołków. Podaj równanie okręgu wpisanego w ten romb. Ile jest wszystkich rombów o boku $A B$ i kącie ostrym $45^{\circ}$ ? Oblicz objętość bryły otrzymanej przez obrót rombu wokół jego boku.\\
6. W ostrosłupie prawidłowym czworokątnym środek podstawy jest odległy o $d$ od krawędzi bocznej a kąt między sąsiednimi ścianami bocznymi ostrosłupa jest równy $2 \alpha$. Oblicz objętość ostrosłupa.

\section*{PRACA KONTROLNA nr 7 - POZIOM RoZsZERZONY}
\begin{enumerate}
  \item Dla jakiego parametru $m$ równanie
\end{enumerate}

$$
m x^{3}-(2 m+1) x^{2}+(2-3 m) x+3=0
$$

ma trzy różne pierwiastki, które są kolejnymi wyrazami ciągu arytmetycznego?\\
2. Rozwiąż równanie

$$
\frac{1+\operatorname{tg} x+\operatorname{tg}^{2} x+\operatorname{tg}^{3} x+\ldots+\operatorname{tg}^{n} x+\ldots}{1-\operatorname{tg} x+\operatorname{tg}^{2} x-\operatorname{tg}^{3} x+\ldots+(-1)^{n} \operatorname{tg}^{n} x+\ldots}=1+\sin 2 x .
$$

\begin{enumerate}
  \setcounter{enumi}{2}
  \item Narysuj w prostokątnym układzie współrzędnych zbiór punktów spełniających warunek
\end{enumerate}

$$
\log _{(x-y)}(x+y) \leqslant 1
$$

\begin{enumerate}
  \setcounter{enumi}{3}
  \item Podaj równanie prostej $l$ stycznej do wykresu funkcji $f(x)=\frac{3 x-2}{(x-1)^{2}}$ w punkcie jego przecięcia z osią $O y$ i wyznacz równania wszystkich stycznych do wykresu równoległych do $l$. Oblicz odległość między otrzymanymi prostymi. Sporządź staranny wykres funkcji wraz z otrzymanymi stycznymi.
  \item Ostrosłup prawidłowy czworokątny przecięto płaszczyzną przechodzącą przez przekątną podstawy i środek przeciwległej krawędzi bocznej. Płaszczyzna ta jest nachylona do płaszczyzny podstawy pod kątem $\alpha$. Wyznacz kąt między ścianami bocznymi.
  \item Odcinek o końcach $A(0,0)$ i $B(8,6)$ jest dłuższą podstawą trapezu prostokątnego opisanego na okręgu. Wyznacz współrzędne pozostałych wierzchołków trapezu, wiedząc, że bok $C D$ jest dwa razy krótszy od boku $A B$. Podaj równanie okręgu wpisanego w ten trapez. Oblicz objętość bryły otrzymanej przez obrót trapezu wokół ramienia $B C$.
\end{enumerate}

Rozwiązania (rękopis) zadań z wybranego poziomu prosimy nadsyłać do 20.03.2023r. na adres:

\begin{verbatim}
Wydział Matematyki
Politechnika Wrocławska
Wybrzeże Wyspiańskiego 27
50-370 WROCEAW,
\end{verbatim}

lub elektronicznie, za pośrednictwem portalu \href{http://talent.pwr.edu.pl}{talent.pwr.edu.pl}\\
Na kopercie prosimy koniecznie zaznaczyć wybrany poziom! (np. poziom podstawowy lub rozszerzony). Do rozwiązań należy dołączyć zaadresowaną do siebie kopertę zwrotną z naklejonym znaczkiem, odpowiednim do formatu listu. Prace niespełniające podanych warunków nie będą poprawiane ani odsyłane.

Uwaga. Wysyłając nam rozwiązania zadań uczestnik Kursu udostępnia Politechnice Wrocławskiej swoje dane osobowe, które przetwarzamy wyłącznie w zakresie niezbędnym do jego prowadzenia (odesłanie zadań, prowadzenie statystyki). Szczegółowe informacje o przetwarzaniu przez nas danych osobowych są dostępne na stronie internetowej Kursu.\\
Adres internetowy Kursu: \href{http://www.im.pwr.edu.pl/kurs}{http://www.im.pwr.edu.pl/kurs}


\end{document}