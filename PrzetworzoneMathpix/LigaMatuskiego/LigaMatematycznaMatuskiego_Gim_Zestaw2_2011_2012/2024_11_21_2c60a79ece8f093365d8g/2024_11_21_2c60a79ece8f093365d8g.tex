\documentclass[10pt]{article}
\usepackage[polish]{babel}
\usepackage[utf8]{inputenc}
\usepackage[T1]{fontenc}
\usepackage{amsmath}
\usepackage{amsfonts}
\usepackage{amssymb}
\usepackage[version=4]{mhchem}
\usepackage{stmaryrd}

\title{LIGA MATEMATYCZNA \\
 LISTOPAD 2011 \\
 GIMNAZJUM }

\author{}
\date{}


\begin{document}
\maketitle
\section*{ZADANIE 1.}
Rozwią̇̇ równanie

\[
\begin{gathered}
\left(\frac{x}{\sqrt{144}+\sqrt{145}}-\frac{1}{\sqrt{36}+\sqrt{37}}\right)+\left(\frac{x}{\sqrt{145}+\sqrt{146}}-\frac{1}{\sqrt{37}+\sqrt{38}}\right)+\ldots \\
\ldots+\left(\frac{x}{\sqrt{168}+\sqrt{169}}-\frac{1}{\sqrt{60}+\sqrt{61}}\right)=0
\end{gathered}
\]

\section*{ZADANIE 2.}
Przecinając prostokątny arkusz papieru, otrzymano kwadrat oraz mniejszy prostokąt. Z tego prostokąta również odcięto kwadrat i znów otrzymano mniejszy prostokąt. Sytuacja powtórzyła się jeszcze kilkakrotnie, aż do momentu otrzymania dziewięciu różnych kwadratów i jednego prostokąta o wymiarach \(1 \mathrm{~cm} \times 2 \mathrm{~cm}\). Jakie pole miał arkusz papieru?

\section*{ZADANIE 3.}
Liczby nieparzyste od 1 do 49 wypisano w poniższej tablicy:

\[
\left(\begin{array}{ccccc}
1 & 3 & 5 & 7 & 9 \\
11 & 13 & 15 & 17 & 19 \\
21 & 23 & 25 & 27 & 29 \\
31 & 33 & 35 & 37 & 39 \\
41 & 43 & 45 & 47 & 49
\end{array}\right)
\]

Wybieramy z tej tablicy pięć liczb tak, aby żadne dwie nie leżały ani w jednej kolumnie, ani w jednym wierszu. Wyznacz wszystkie wartości, jakie może przyjmować suma wybranych liczb.

\section*{ZADANIE 4.}
Ania napisała trzy liczby pięciocyfrowe używając do zapisu każdej z tych liczb wszystkich cyfr spośród 1, 2, 3, 4, 5. Czy suma tych liczb jest podzielna przez 3? Czy jest podzielna przez 9?

\section*{ZADANIE 5.}
Dany jest kwadrat o boku długości \(a\). Na jego bokach, na zewnątrz, zbudowano trójkąty równoboczne. Wierzchołki kolejnych trójkątów, nie będące wierzchołkami danego kwadratu, połączono odcinkami. Oblicz pole otrzymanego czworokąta.


\end{document}