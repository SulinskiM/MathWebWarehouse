\documentclass[10pt]{article}
\usepackage[polish]{babel}
\usepackage[utf8]{inputenc}
\usepackage[T1]{fontenc}
\usepackage{amsmath}
\usepackage{amsfonts}
\usepackage{amssymb}
\usepackage[version=4]{mhchem}
\usepackage{stmaryrd}

\title{LIGA MATEMATYCZNA im. Zdzisława Matuskiego \\
 PÓŁFINAE }

\author{}
\date{}


\begin{document}
\maketitle
\section*{11 lutego 2014}
GIMNAZJUM

\section*{ZADANIE 1.}
Wewnątrz kwadratu \(A B C D\) wybrano punkt \(M\) w równej odległości od boku \(C D\) i od wierzchołków \(A\) oraz \(B\). Jaką część pola kwadratu stanowi pole trójkąta \(A B M\) ?

\section*{ZADANIE 2.}
Czy 2014 orzechów można włożyć do 50 woreczków w taki sposób, aby w każdym było więcej niż 20 orzechów, ale w każdym inna ich liczba? Czy można rozłożyć te orzechy tak, aby w każdym woreczku było co najmniej 10 orzechów i w każdym inna ich liczba?

\section*{ZADANIE 3.}
Spośród trzystu uczniów klas drugich i trzecich gimnazjum 100 wzięło udział w olimpiadzie matematycznej, 80 w fizycznej, 60 w informatycznej, w tym 23 w olimpiadzie matematycznej i fizycznej, 16 w olimpiadzie matematycznej i informatycznej, 14 w olimpiadzie fizycznej i informatycznej, 5 we wszystkich trzech olimpiadach. Ilu uczniów wzięło udział tylko w olimpiadzie matematycznej? Ilu uczniów wzięło udział tylko w jednej olimpiadzie, a ilu dokładnie w dwóch? Ilu uczniów nie wzięło udziału w żadnej olimpiadzie?

\section*{ZADANIE 4.}
Srednia arytmetyczna liczb \(a, b, c\) równa się 12 , a średnia arytmetyczna liczb \(2 a+1,2 b, c\) równa się 17. Oblicz średnią arytmetyczną liczb \(a\) i \(b\).

\section*{ZADANIE 5.}
Wykaż, że dla dowolnych liczb rzeczywistych \(a, b, c\) spełniona jest nierówność

\[
a^{2}+b^{2}+c^{2}+3 \geqslant 2(a+b+c) .
\]


\end{document}