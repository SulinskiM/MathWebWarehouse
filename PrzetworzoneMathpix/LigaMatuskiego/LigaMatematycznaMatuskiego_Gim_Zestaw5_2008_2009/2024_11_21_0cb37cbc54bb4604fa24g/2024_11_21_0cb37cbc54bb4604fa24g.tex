\documentclass[10pt]{article}
\usepackage[polish]{babel}
\usepackage[utf8]{inputenc}
\usepackage[T1]{fontenc}
\usepackage{amsmath}
\usepackage{amsfonts}
\usepackage{amssymb}
\usepackage[version=4]{mhchem}
\usepackage{stmaryrd}

\title{LIGA MATEMATYCZNA FINAE \\
 25 kwietnia 2009 \\
 GIMNAZJUM }

\author{}
\date{}


\begin{document}
\maketitle
\section*{ZADANIE 1.}
Długości boków trzech kwadratów, które widzisz na rysunku, są liczbami naturalnymi. Wiadomo, że \(B C=C D\) oraz zamalowana figura ma pole równe \(31 \mathrm{~cm}^{2}\). Oblicz pole największego z tych kwadratów.

\section*{ZADANIE 2.}
Tablicę \(3 x 3\) podzielono na 9 jednakowych kwadratów, w których umieszczono liczby -1, 0,1 . Uzasadnij, że wśród ośmiu sum (liczb z każdego wiersza, każdej kolumny i głównych przekątnych) co najmniej dwie są równe.

\section*{ZADANIE 3.}
Na boku \(B C\) prostokąta \(A B C D\) wybrano punkt \(E\) tak, że stosunek pól trójkąta \(C D E\) i trapezu \(A B E D\) jest równy \(\frac{1}{4}\). Oblicz \(\frac{C E}{E B}\).\\
ZADANIE 4.\\
W pewnej szkole zorganizowano kółko plastyczne, informatyczne i sportowe. W zajęciach plastycznych uczestniczy 73 uczniów, w informatycznych - 128, w sportowych - 103, przy czym w plastycznych i informatycznych - 28, w plastycznych i sportowych - 20, w informatycznych i sportowych -43 oraz w plastycznych, informatycznych i sportowych - 7. Ilu uczniów uczestniczy w zajęciach informatycznych i sportowych, ale nie uczestniczy w plastycznych? Ilu uczniów uczestniczy dokładnie w dwóch rodzajach zajęć? Ilu uczniów bierze udział tylko w zajęciach sportowych?

\section*{ZADANIE 5.}
Janek z Olą zebrali trzy razy więcej grzybów niż Franek, a Ola z Frankiem - pięć razy więcej grzybów niż Janek. Kto uzbierał więcej grzybów: Janek razem z Frankiem czy Ola?


\end{document}