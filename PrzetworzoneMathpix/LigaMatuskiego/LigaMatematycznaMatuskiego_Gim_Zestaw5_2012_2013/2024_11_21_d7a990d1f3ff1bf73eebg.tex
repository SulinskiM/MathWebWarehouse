\documentclass[10pt]{article}
\usepackage[polish]{babel}
\usepackage[utf8]{inputenc}
\usepackage[T1]{fontenc}
\usepackage{amsmath}
\usepackage{amsfonts}
\usepackage{amssymb}
\usepackage[version=4]{mhchem}
\usepackage{stmaryrd}

\title{LIGA MATEMATYCZNA im. Zdzisława Matuskiego \\
 FINAE \\
 10 kwietnia 2013 \\
 GIMNAZJUM }

\author{}
\date{}


\begin{document}
\maketitle
\section*{ZADANIE 1.}
Kostka do gry ma na ściankach liczby oczek \(1,2,3,4,5,6\). W pewnej grze za wyrzucenie parzystej liczby oczek uzyskuje się 25 punktów, a za wyrzucenie nieparzystej liczby oczek traci się 15 punktów. W dziesięciu rzutach tą kostką Kamil uzyskał 90 punktów, Robert 50 punktów. Ile razy parzystą liczbę oczek wyrzucił Kamil, a ile razy nieparzystą liczbę oczek wyrzucił Robert?

\section*{ZADANIE 2.}
Wykaż, że liczba \(2 n^{3}-26 n\) jest podzielna przez 6 dla każdej liczby naturalnej \(n\).

\section*{ZADANIE 3.}
Na boku \(C D\) kwadratu \(A B C D\) wybrano punkt \(E\) taki, że odcinek \(E C\) jest dwa razy dłuższy od \(D E\). Na odcinku \(B E\) wybrano punkt \(F\) taki, że odcinek \(F B\) jest dwa razy dłuższy od \(E F\). Pole trójkąta \(A F E\) jest równe 10. Oblicz pole kwadratu \(A B C D\).

\section*{ZADANIE 4.}
Uzasadnij, że jeśli liczby \(a, b, c\) są dodatnie oraz \(a<b\), to

\[
\frac{a+c}{b+c}>\frac{a}{b} .
\]

\section*{ZADANIE 5.}
Wypisano siedem kolejnych liczb naturalnych, przy czym pierwsza z nich jest parzysta. Suma pięciu pierwszych jest liczbą trzycyfrową, natomiast suma pięciu ostatnich jest liczbą czterocyfrową. Oblicz sumę tych siedmiu liczb.


\end{document}