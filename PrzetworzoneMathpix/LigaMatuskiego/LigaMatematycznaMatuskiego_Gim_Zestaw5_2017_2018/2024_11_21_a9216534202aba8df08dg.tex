\documentclass[10pt]{article}
\usepackage[polish]{babel}
\usepackage[utf8]{inputenc}
\usepackage[T1]{fontenc}
\usepackage{amsmath}
\usepackage{amsfonts}
\usepackage{amssymb}
\usepackage[version=4]{mhchem}
\usepackage{stmaryrd}

\title{LIGA MATEMATYCZNA im. Zdzisława Matuskiego \\
 FINAE \\
 24 kwietnia 2017 \\
 GIMNAZJUM }

\author{}
\date{}


\newcommand\varangle{\mathop{\sphericalangle}}

\begin{document}
\maketitle
\section*{ZADANIE 1.}
Na stole położono po jednym patyczku o długości \(2,4,6,8,9,10,30,40,50\) i 60. Adam zbudował ramkę wybierając trzy patyczki tak, aby obwód trójkąta był jak najmniejszy. Z pozostałych patyków Bartek wybrał trzy, z których można zbudować trójkąt o największym obwodzie. Ostatnimi czterema patykami zainteresował się Czarek, wybrał trzy i zbudował z nich trójkątną ramkę. Który patyk pozostał na stole?

\section*{ZADANIE 2.}
W rombie \(A B C D\) punkty \(M\) i \(N\), różne od punktów \(A, B\) i \(C\), leżą na odcinkach odpowiednio \(A B, B C\) tak, że trójkąt \(D M N\) jest równoboczny oraz \(|A D|=|M D|\). Wyznacz miarę kąta \(\varangle A B C\).

\section*{ZADANIE 3.}
Suma liczby trzycyfrowej i liczby otrzymanej z napisania cyfr poprzedniej liczby w odwrotnej kolejności jest równa 444. Różnicą tych liczb jest 198. Wyznacz liczbę trzycyfrową wiedząc, że suma jej cyfr jest równa 6.

\section*{ZADANIE 4.}
Czy suma 2017 różnych liczb pierwszych może być liczbą parzystą? Czy iloczyn 2017 różnych liczb pierwszych może być liczbą parzystą? Odpowiedź uzasadnij.

\section*{ZADANIE 5.}
Rozważmy liczby trzycyfrowe zaczynające i kończące się tą samą cyfrą. Wykaż, że jeżeli suma pierwszej i drugiej cyfry takiej liczby jest podzielna przez 7, to sama liczba też dzieli się przez 7.


\end{document}