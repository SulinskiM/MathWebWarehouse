\documentclass[10pt]{article}
\usepackage[polish]{babel}
\usepackage[utf8]{inputenc}
\usepackage[T1]{fontenc}
\usepackage{amsmath}
\usepackage{amsfonts}
\usepackage{amssymb}
\usepackage[version=4]{mhchem}
\usepackage{stmaryrd}

\title{LIGA MATEMATYCZNA im. Zdzisława Matuskiego PAŹDZIERNIK 2020 SZKOŁA PODSTAWOWA \\
 klasy IV - VI }

\author{}
\date{}


\begin{document}
\maketitle
\section*{ZADANIE 1.}
Uczniowie klasy IV zjadają worek prażonej kukurydzy w ciągu 6 minut, uczniowie klasy V taki worek zjadają w ciągu 3 minut. W ciągu ilu minut zostanie zjedzony taki worek kukurydzy wspólnie przez uczniów obu klas?

\section*{ZADANIE 2.}
Ania bardzo lubi jabłka, marchewki i ciastka. Każdego dnia zjada albo 9 marchewek, albo 2 jabłka, albo 1 jabłko i 4 marchewki, albo 1 ciastko. Przez 10 kolejnych dni Ania zjadła 30 marchewek i 9 jabłek. Ile ciastek zjadła dziewczynka w czasie tych 10 dni?

\section*{ZADANIE 3.}
Znajdź najmniejszą liczbę naturalną podzielną przez 15, którą zapisano za pomocą samych zer i jedynek.

\section*{ZADANIE 4.}
Spotkało się trzech artystów: Adam Biały - aktor, Bartek Czarny - muzyk i Czarek Rudy malarz.

\begin{itemize}
  \item Zauważcie, że kolor naszych włosów nie pokrywa się z nazwiskiem. - powiedział ten z czarnymi włosami.
  \item Masz rację. - odpowiedział Adam.
\end{itemize}

Jaki kolor włosów miał każdy z nich?

\section*{ZADANIE 5.}
W każdym wierzchołku kwadratu Adam wpisał pewną liczbę, a na każdym boku kwadratu zapisał sumę liczb z obu końców. Liczby wpisane na bokach kwadratu to cztery z następujących: \(15,11,19,21,23\). Która z liczb nie znalazła się na żadnym boku?


\end{document}