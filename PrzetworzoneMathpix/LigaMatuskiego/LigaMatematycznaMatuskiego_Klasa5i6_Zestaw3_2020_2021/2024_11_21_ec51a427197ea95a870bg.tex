\documentclass[10pt]{article}
\usepackage[polish]{babel}
\usepackage[utf8]{inputenc}
\usepackage[T1]{fontenc}
\usepackage{amsmath}
\usepackage{amsfonts}
\usepackage{amssymb}
\usepackage[version=4]{mhchem}
\usepackage{stmaryrd}

\title{LIGA MATEMATYCZNA im. Zdzisława Matuskiego GRUDZIEŃ 2020 SZKOŁA PODSTAWOWA \\
 klasy IV - VI }

\author{}
\date{}


\begin{document}
\maketitle
\section*{ZADANIE 1.}
W sadzie rośnie więcej niż 90, ale mniej niż 100 drzewek owocowych. Trzecią ich część stanowią jabłonie, czwartą część grusze, a resztę wiśnie. Ile drzew jest w sadzie? Podaj liczbę drzew każdego gatunku.

\section*{ZADANIE 2.}
Skacząc z trampoliny na basenie Adam odbija się od niej na wysokość 1 m , następnie spada w dół 5 m , wreszczie - wypływając w górę 2 m - osiąga powierzchnię wody. Na jakiej wysokości nad powierzchnią wody znajduje się trampolina?

\section*{ZADANIE 3.}
Rozważmy liczby naturalne od 1 do 90 . Ile jest wśród nich liczb podzielnych dokładnie przez dwie spośród liczb \(2,3,5\) ?

\section*{ZADANIE 4.}
W szkole uczy się 100 uczniów. Języka angielskiego uczy się 85 uczniów, języka niemieckiego 75, języka francuskiego 48, języka hiszpańskiego 93 . Uzasadnij, że co najmniej jeden uczeń poznaje wszystkie cztery jezyki obce.

\section*{ZADANIE 5.}
Na prostokątnej kartce papieru o wymiarach \(11 \times 24\) Ania nakleiła kwadrat \(A\) i trzy identyczne prostokąty \(B\) tak, że figury nie nachodziły na siebie. Oblicz obwód wyklejonej figury.

\begin{center}
\begin{tabular}{|l|l|l|}
\hline
 & \(B\) &  \\
\hline
\(B\) &  & \(B\) \\
 & \(A\) &  \\
\cline { 3 - 3 }
 &  &  \\
\hline
\end{tabular}
\end{center}


\end{document}