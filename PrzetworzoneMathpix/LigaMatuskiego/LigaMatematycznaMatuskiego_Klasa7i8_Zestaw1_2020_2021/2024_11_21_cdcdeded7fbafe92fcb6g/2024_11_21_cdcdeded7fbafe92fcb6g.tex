\documentclass[10pt]{article}
\usepackage[polish]{babel}
\usepackage[utf8]{inputenc}
\usepackage[T1]{fontenc}
\usepackage{amsmath}
\usepackage{amsfonts}
\usepackage{amssymb}
\usepackage[version=4]{mhchem}
\usepackage{stmaryrd}

\title{LIGA MATEMATYCZNA im. Zdzisława Matuskiego PAŹDZIERNIK 2020 SZKOŁA PODSTAWOWA klasy VII - VIII }

\author{}
\date{}


\begin{document}
\maketitle
\section*{ZADANIE 1.}
Na promenadzie w Helu koloniści kupowali pamiątki: bursztynowe bransoletki, korale z muszelek i pluszowe foczki. Każdy wybrał dwie różne pamiątki. Foczek kupili dwa razy więcej niż bransoletek, a korali trzy razy więcej niż foczek. Uzasadnij, że liczba kolonistów była podzielna przez 9, a liczba kupionych bransoletek była parzysta.

\section*{ZADANIE 2.}
Pole prostokąta \(A B C D\) jest równe 1. Każdy bok tego prostokąta przedłużono o odcinek równy temu bokowi i otrzymano punkty \(P, Q, R, S\) w taki sposób, że punkt \(A\) jest środkiem odcinka \(P B, B\) jest środkiem \(C Q, C\) jest środkiem \(D R, D\) jest środkiem \(A S\). Oblicz pole czworokąta PQRS.

\section*{ZADANIE 3.}
Punkt \(E\) leży wewnątrz kwadratu \(A B C D\) tak, że trójkąt \(A B E\) jest równoboczny. Oblicz miarę kąta \(D C E\).

\section*{ZADANIE 4.}
W kolekcji firmy jubilerskiej są trzy rodzaje naszyjników: z dwiema perłami, z jedną perłą i takie, które nie mają pereł. Naszyjników bez pereł jest dwa razy mniej niż wszystkich pozostałych. W 99 naszyjnikach jest 100 pereł. Ile jest naszyjników z jedną perłą?

\section*{ZADANIE 5.}
Liczba trzycyfrowa ma cyfrę jedności równą 5 . Jeżeli do tej liczby dodamy 1 i otrzymaną sumę podzielimy przez 3, to otrzymamy liczbą trzycyfrową, której cyfrą setek jest 1, a następne jej cyfry są pierwszą i drugą cyfrą liczby wyjściowej. Wyznacz tę liczbę.


\end{document}