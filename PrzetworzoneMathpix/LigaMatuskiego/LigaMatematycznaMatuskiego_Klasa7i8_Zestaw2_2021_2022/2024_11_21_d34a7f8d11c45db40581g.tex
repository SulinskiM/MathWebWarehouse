\documentclass[10pt]{article}
\usepackage[polish]{babel}
\usepackage[utf8]{inputenc}
\usepackage[T1]{fontenc}
\usepackage{amsmath}
\usepackage{amsfonts}
\usepackage{amssymb}
\usepackage[version=4]{mhchem}
\usepackage{stmaryrd}

\title{LIGA MATEMATYCZNA \\
 im. Zdzisława Matuskiego \\
 LISTOPAD 2021 \\
 SZKOŁA PODSTAWOWA \\
 klasy VII - VIII }

\author{}
\date{}


\begin{document}
\maketitle
\section*{ZADANIE 1.}
Znajdź wszystkie liczby czterocyfrowe, które mają takie dwa dzielniki, że ich suma jest równa 110, a różnica 36.

\section*{ZADANIE 2.}
Pole prostokąta \(A B C D\) jest równe 24 . Na boku \(A B\) zaznaczono punkt \(E\) różny od punktów \(A\) i \(B\), na \(D C\) zaznaczono punkt \(F\) różny od punktów \(C\) i \(D\). Pole trójkąta \(A F D\) jest równe 5. Oblicz pole trójkąta \(E C F\).

\section*{ZADANIE 3.}
Adam dodał zerową, pierwszą, drugą i trzecią potęgę pewnej liczby naturalnej i otrzymał 400. Jaka to liczba?

\section*{ZADANIE 4.}
Na okręgu zaznaczono 55 punktów. Trzy z nich oznaczono \(A, B, C\). Ania policzyła punkty od \(A\) do \(C\), przechodząc raz przez \(B\), i otrzymała 31. Gdy liczyła od \(A\) do \(B\) przechodząc raz przez \(C\), to uzyskała 39.

\begin{itemize}
  \item Wyznacz najmniejszą liczbę punktów od \(C\) do \(B\) przy jednokrotnym przejściu przez punkt \(A\).
  \item Wyznacz najmniejszą liczbę punktów od \(B\) do \(A\) przy przejściu przez punkt \(C\).
\end{itemize}

\section*{ZADANIE 5.}
Władca pewnego królestwa nagrodził swoich dwóch dzielnych rycerzy: starszego 110 dukatami, młodszego 100 dukatami. Monety znajdowały się w dwóch rodzajach sakiewek: w małych było po 7 dukatów, w dużych po 17 dukatów. Każdy rycerz otrzymał 10 sakiewek.

\begin{itemize}
  \item Ile dużych sakiewek otrzymał starszy rycerz?
  \item Ile małych sakiewek dostał młodszy rycerz?
\end{itemize}

\end{document}