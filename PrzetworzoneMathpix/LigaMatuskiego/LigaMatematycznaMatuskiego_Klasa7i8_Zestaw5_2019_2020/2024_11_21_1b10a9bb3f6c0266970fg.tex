\documentclass[10pt]{article}
\usepackage[polish]{babel}
\usepackage[utf8]{inputenc}
\usepackage[T1]{fontenc}
\usepackage{amsmath}
\usepackage{amsfonts}
\usepackage{amssymb}
\usepackage[version=4]{mhchem}
\usepackage{stmaryrd}

\title{LIGA MATEMATYCZNA im. Zdzisława Matuskiego PÓŁFINAŁ 28 lutego 2020 SZKOŁA PODSTAWOWA klasy VII - VIII }

\author{}
\date{}


\begin{document}
\maketitle
\section*{ZADANIE 1.}
Dwa tysiące dwadzieścia liczb zapisano jedna za drugą. Druga z nich jest równa 15, a ostatnia 46. Wiadomo, że suma każdych trzech kolejnych liczb jest równa 100. Wyznacz pozostałe 2018 liczb.

\section*{ZADANIE 2.}
Kawałek czworokątnego materiału o obwodzie 3 m przecięto wzdłuż jednej przekątnej i powstały dwie chusty w kształcie trójkątów równoramiennych, pierwszy o obwodzie \(1,8 \mathrm{~m}\), a drugi \(2,8 \mathrm{~m}\). Linia rozcięcia stanowi podstawę pierwszego trójkąta, a dla drugiego trójkąta jest ramieniem. Wyznacz wymiary obu chust.

\section*{ZADANIE 3.}
Dane są liczby rzeczywiste \(x, y\) spełniające równanie

\[
(x-y)^{2}+(x+y-4)^{2}=0
\]

Oblicz iloczyn tych liczb.

\section*{ZADANIE 4.}
Na stole leży 2020 kapsli. W jednym ruchu Bartek może zdjąć dokładnie 3, 24 lub 51 kapsli. Wolno mu wykonać wiele takich ruchów. Czy w pewnej chwili wszystkie kapsle zostaną zdjęte ze stołu?

\section*{ZADANIE 5.}
Trzy liczby naturalne dwucyfrowe ustawione w kolejności malejącej stanowią szyfr do sejfu. Iloczyn pewnych dwóch spośród nich jest równy 888, a iloczyn innych dwóch jest równy 999. Znajdź szyfr do sejfu.


\end{document}