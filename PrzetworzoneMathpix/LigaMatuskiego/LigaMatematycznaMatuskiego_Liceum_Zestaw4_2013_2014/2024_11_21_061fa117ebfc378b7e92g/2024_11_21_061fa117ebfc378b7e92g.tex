\documentclass[10pt]{article}
\usepackage[polish]{babel}
\usepackage[utf8]{inputenc}
\usepackage[T1]{fontenc}
\usepackage{amsmath}
\usepackage{amsfonts}
\usepackage{amssymb}
\usepackage[version=4]{mhchem}
\usepackage{stmaryrd}
\usepackage{bbold}

\title{LIGA MATEMATYCZNA \\
 im. Zdzisława Matuskiego \\
 STYCZEŃ 2014 \\
 SZKOŁA PONADGIMNAZJALNA }

\author{}
\date{}


\begin{document}
\maketitle
\section*{ZADANIE 1.}
Boki trójkąta \(A B C\) są podzielone punktami \(M, N\) i \(P\) tak, że

\[
\frac{A M}{M B}=\frac{B N}{N C}=\frac{C P}{P A}=\frac{1}{4}
\]

Wyznacz stosunek pola trójkąta ograniczonego prostymi \(A N, B P, C M\) do pola trójkąta \(A B C\).

\section*{ZADANIE 2.}
Ciąg liczbowy \(\left(a_{n}\right)_{n \in \mathbb{N}}\) jest określony następująco:

\[
\begin{aligned}
& a_{1}=1 \\
& a_{2}=1 \\
& a_{3}=-1 \\
& a_{n}=a_{n-1} a_{n-3}, \text { gdy } n \geqslant 4
\end{aligned}
\]

Oblicz \(a_{2014}\).

\section*{ZADANIE 3.}
Każdy punkt płaszczyzny pomalowano na jeden z czterech kolorów: żółty, czerwony, zielony oraz niebieski. Każdy kolor został wykorzystany. Wykaż, że istnieje prosta, której punkty są co najmniej trzech kolorów.

\section*{ZADANIE 4.}
Rozwiąż układ równań

\[
\left\{\begin{array}{l}
2 x^{2014}+2 y^{2014}-z^{2014}=4 \\
2 y^{2014}+2 z^{2014}-x^{2014}=22 \\
2 z^{2014}+2 x^{2014}-y^{2014}=16
\end{array}\right.
\]

\section*{ZADANIE 5.}
Porównując wyniki w łowieniu ryb Adam, Bartek i Czarek stwierdzili, że jeden z nich złowił tylko okonie, jeden tylko pstrągi i jeden tylko łososie. Liczba ryb Adama jest o 7 większa od \(\frac{3}{5}\) liczby okoni. Liczba ryb Bartka jest o 3 większa od \(\frac{5}{7}\) liczby łososi. Natomiast liczba wszystkich ryb jest trzycyfrową liczbą pierwszą. Ile ryb złowił każdy z chłopców?


\end{document}