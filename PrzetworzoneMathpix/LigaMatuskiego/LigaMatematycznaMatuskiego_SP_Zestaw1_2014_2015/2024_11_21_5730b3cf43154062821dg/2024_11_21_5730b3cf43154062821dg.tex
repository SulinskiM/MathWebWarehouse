\documentclass[10pt]{article}
\usepackage[polish]{babel}
\usepackage[utf8]{inputenc}
\usepackage[T1]{fontenc}
\usepackage{amsmath}
\usepackage{amsfonts}
\usepackage{amssymb}
\usepackage[version=4]{mhchem}
\usepackage{stmaryrd}

\title{LIGA MATEMATYCZNA im. Zdzisława Matuskiego \\
 PAŹDZIERNIK 2014 SZKOŁA PODSTAWOWA }

\author{}
\date{}


\begin{document}
\maketitle
\section*{ZADANIE 1.}
Podczas kolejnych lotów trzej piloci spotkali się w Paryżu 6 września 2014 roku. Wiadomo, że pierwszy pilot lata do Paryża co 7 dni, drugi co 14 , a trzeci co 5 dni. Wyznacz daté ich kolejnego spotkania w Paryżu.

\section*{ZADANIE 2.}
Piłka nożna i golfowa ważą razem tyle, ile łącznie ważą piłka bejsbolowa i tenisowa. Trzy piłki golfowe ważą tyle, ile piłka bejsbolowa i tenisowa. Piłka bejsbolowa waży tyle, ile osiem piłek tenisowych. Ile piłek tenisowych waży tyle, ile jedna piłka nożna?

\section*{ZADANIE 3.}
Wyznacz cyfry \(a, b\) tak, aby liczba sześciocyfrowa \(a 2479 b\) była podzielna przez 72 .

\section*{ZADANIE 4.}
Na każdej ze ścian sześcianu zapisano jedną z liczb 1, 2, 3, 4, 5, 6 w taki sposób, że suma liczb na każdej parze przeciwległych ścian jest równa 7 . Z każdym wierzchołkiem sześcianu związane są trzy ściany. Liczby zapisane na tych ścianach mnożymy. Oblicz sumę tych iloczynów.

\section*{ZADANIE 5.}
Prostokąt podzielono na dziewięć mniejszych prostokątów. Obwody pięciu z nich podane są na rysunku. Oblicz obwód dużego prostokąta.

\begin{center}
\begin{tabular}{|c|c|c|}
\hline
 & 6 &  \\
\hline
12 & 4 & 6 \\
\hline
 &  &  \\
 & 8 &  \\
 &  &  \\
\hline
\end{tabular}
\end{center}


\end{document}