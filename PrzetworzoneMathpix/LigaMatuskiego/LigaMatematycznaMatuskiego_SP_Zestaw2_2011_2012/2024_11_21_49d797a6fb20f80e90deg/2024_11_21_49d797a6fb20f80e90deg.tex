\documentclass[10pt]{article}
\usepackage[polish]{babel}
\usepackage[utf8]{inputenc}
\usepackage[T1]{fontenc}
\usepackage{amsmath}
\usepackage{amsfonts}
\usepackage{amssymb}
\usepackage[version=4]{mhchem}
\usepackage{stmaryrd}

\title{LIGA MATEMATYCZNA \\
 LISTOPAD 2011 \\
 SZKOŁA PODSTAWOWA }

\author{}
\date{}


\begin{document}
\maketitle
\section*{ZADANIE 1.}
Automat matematyczny działa na następującej zasadzie: do danej liczby dodaje jeden lub ją podwaja. Do automatu wprowadzono liczbę 0 . Ten po wykonaniu pewnej liczby operacji otrzymał liczbę 100. Jaka jest najmniejsza ilość operacji, które musi wykonać automat, aby otrzymać taki wynik?

\section*{ZADANIE 2.}
W smoczej jamie żyły smoki czerwone i smoki zielone. Każdy czerwony smok miał sześć głów, osiem nóg i dwa ogony, natomiast zielony smok miał osiem głów, sześć nóg i cztery ogony. Wszystkich ogonów było 44, a zielonych nóg o 6 mniej niż czerwonych głów. Ile czerwonych smoków żyło w tej jamie?

\section*{ZADANIE 3.}
Jaś pomyślał pewną liczbę naturalną, pomnożył ją przez 13, odrzucił ostatnią cyfrę wyniku, otrzymaną liczbę pomnożył przez 7, znów odrzucił ostatnią cyfrę wyniku i otrzymał 21. Jaką liczbę pomyślał Jasiu?

\section*{ZADANIE 4.}
Ile różnych prostokątów można zbudować z patyczków o długościach \(3 \mathrm{~cm}, 5 \mathrm{~cm}, 8 \mathrm{~cm}, 10 \mathrm{~cm}\), \(11 \mathrm{~cm}, 13 \mathrm{~cm}, 14 \mathrm{~cm}\) ? Patyczki tworzące boki prostokąta nie mogą zachodzić na siebie i nie można ich łamać. Za każdym razem należy wykorzystać wszystkie patyczki.

\section*{ZADANIE 5.}
Do sklepu przywieziono 223 kg cukierków w pojemnikach 10 kg i 17 kg . Ile było pojemników?


\end{document}