\documentclass[10pt]{article}
\usepackage[polish]{babel}
\usepackage[utf8]{inputenc}
\usepackage[T1]{fontenc}
\usepackage{amsmath}
\usepackage{amsfonts}
\usepackage{amssymb}
\usepackage[version=4]{mhchem}
\usepackage{stmaryrd}

\title{LIGA MATEMATYCZNA \\
 im. Zdzisława Matuskiego \\
 LISTOPAD 2015 \\
 SZKOŁA PODSTAWOWA }

\author{}
\date{}


\begin{document}
\maketitle
\section*{ZADANIE 1.}
Liczby \(49,29,9,40,22,15,53,33,13,47\) połaczono w pary tak, że suma liczb w każdej parze jest taka sama. Która z liczb stanowi parę z liczbą 15?

\section*{ZADANIE 2.}
Wykaż, że liczba \(2 \underbrace{555 \ldots 5}_{100 \text { cyfr } 5} 2\) jest podzielna przez 12.

\section*{ZADANIE 3.}
W skarbcu odkrytym przez Ali-Babę było 15 worków z monetami. Wiadomo, że w jednym worku wszystkie monety są fałszywe. Prawdziwa moneta waży 20 gramów, a fałszywa 19 gramów. Ali-Baba ma bardzo dokładną wagę, dzięki której może stwierdzić, ile waży konkretny obiekt. Jak za pomocą jednego ważenia odkryć, w którym worku są fałszywe monety?

\section*{ZADANIE 4.}
Z kwadratu wycinamy w rogach cztery kwadratowe kawałki. Ich boki mają długość odpowiednio \(1 \mathrm{~cm}, 2 \mathrm{~cm}, 3 \mathrm{~cm}, 6 \mathrm{~cm}\). Po ich wycięciu pole figury zmniejszyło się dwukrotnie. Wyznacz obwód powstałej figury.

\section*{ZADANIE 5.}
Za pomocą czterech czwórek, wpisując między nie znaki matematyczne (dozwolone są: dodawanie, odejmowanie, mnożenie, dzielenie, pierwiastkowanie i nawiasy) zapisz liczby od 0 do 10 . Powinno powstać jedenaście różnych zapisów z czterema czwórkami.


\end{document}