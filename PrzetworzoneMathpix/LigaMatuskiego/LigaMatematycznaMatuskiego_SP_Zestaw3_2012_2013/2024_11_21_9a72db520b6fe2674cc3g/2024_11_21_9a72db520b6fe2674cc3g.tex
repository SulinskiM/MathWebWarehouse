\documentclass[10pt]{article}
\usepackage[polish]{babel}
\usepackage[utf8]{inputenc}
\usepackage[T1]{fontenc}

\title{LIGA MATEMATYCZNA im. Zdzisława Matuskiego GRUDZIEŃ 2012 SZKOŁA PODSTAWOWA }

\author{}
\date{}


\begin{document}
\maketitle
\section*{ZADANIE 1.}
Mikołaj złowił karpie. Wypuścił jednego do rzeki, a połowę pozostałych dał Adamowi. Potem znów wypuścił jednego karpia i połowę pozostałych ofiarował Ewie. Zostało mu jeszcze 6 karpi. Ile karpi złowił Mikołaj?

\section*{ZADANIE 2.}
Obwód trójkąta równoramiennego jest równy 56. Środek jednego z ramion połączono z wierzchołkiem przeciwległego kąta. Powstały w ten sposób dwa nowe trójkąty, z których jeden (ten, który zawiera podstawę trójkąta równoramiennego) ma obwód o 10 krótszy niż drugi. Oblicz długości boków trójkąta równoramiennego.

\section*{ZADANIE 3.}
Za siedmioma górami, za siedmioma lasami, za siedmioma morzami rosła czarodziejska wierzba. Z jej grubego pnia wyrastała pewna liczba konarów. Z każdego konara wyrastało tyle gałęzi, ile było wszystkich konarów. Na każdej gałęzi rosło dwa razy więcej magicznych gruszek niż było wszystkich gałęzi na tym drzewie. Magicznych gruszek było 1250. Ile konarów wyrastało z pnia czarodziejskiej wierzby?

\section*{ZADANIE 4.}
W prawej i lewej kieszeni Karol miał łącznie 38 monet. Jeżeli przełoży z prawej do lewej kieszeni tyle monet, ile jest w lewej, a następnie z lewej do prawej tyle monet, ile będzie w prawej kieszeni po pierwszym przełożeniu, to w prawej będzie miał o 2 monety więcej niż w lewej kieszeni. Ile monet miał Karol na początku w każdej kieszeni?

\section*{ZADANIE 5.}
Liczbę naturalną nazywamy palindromiczną, jeżeli jej zapis dziesiętny czytany od lewej strony do prawej jest taki sam, jak czytany od prawej strony do lewej. Podaj wszystkie pary liczb pięciocyfrowych palindromicznych, których różnica jest równa 11.


\end{document}