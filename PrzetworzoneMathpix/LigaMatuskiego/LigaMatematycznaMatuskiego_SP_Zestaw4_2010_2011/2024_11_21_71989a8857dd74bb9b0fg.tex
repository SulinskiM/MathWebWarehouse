% This LaTeX document needs to be compiled with XeLaTeX.
\documentclass[10pt]{article}
\usepackage[utf8]{inputenc}
\usepackage{amsmath}
\usepackage{amsfonts}
\usepackage{amssymb}
\usepackage[version=4]{mhchem}
\usepackage{stmaryrd}
\usepackage[fallback]{xeCJK}
\usepackage{polyglossia}
\usepackage{fontspec}
\setCJKmainfont{Noto Serif CJK JP}

\setmainlanguage{polish}
\setmainfont{CMU Serif}

\title{LIGA MATEMATYCZNA PÓŁFINAモ \\
 18 lutego 2011 \\
 SZKOŁA PODSTAWOWA }

\author{}
\date{}


\begin{document}
\maketitle
\section*{ZADANIE 1.}
W maratonie startowało 2011 zawodników. Które miejsce zajął Michał, jeżeli wiadomo, że liczba uczestników, którzy przybiegli na metę przed nim jest cztery razy mniejsza od liczby uczestników, którzy przybiegli po nim?

\section*{ZADANIE 2.}
Uzupełniamy tablicę wpisując w każde jej pole 0 lub 1 tak, aby sumy liczb w każdym wierszu i w każdej kolumnie były równe 2. Jakie są wartości \(a\) i \(b\) ?

\begin{center}
\begin{tabular}{|c|c|c|c|}
\hline
\(a\) &  &  &  \\
\hline
 & \(b\) &  & 1 \\
\hline
 &  & 0 &  \\
\hline
0 &  & 0 &  \\
\hline
\end{tabular}
\end{center}

\section*{ZADANIE 3.}
Na skraju lasu stoi siedem domków. Każdy domek zamieszkiwany jest przez inną liczbę mieszkańców oraz żaden domek nie jest pusty. Ile osób mieszka w poszczególnych domkach, jeżeli wszystkich mieszkańców jest 29?

\section*{ZADANIE 4.}
Każdą z dwóch identycznych prostokątnych kartek rozcięto na dwie części. Z pierwszej kartki otrzymano dwa prostokąty o obwodach 40 cm każdy, z drugiej - dwa prostokąty o obwodach 50 cm każdy. Oblicz obwód każdej z wyjściowych kartek.

\section*{ZADANIE 5.}
Adaś dostał pod choinkę modele trzech samochodów: mercedesa, opla i fiata. Wszystkie są różnej wielkości i w różnych kolorach: białym, czerwonym i czarnym. Mercedes nie jest biały ani czarny, opel nie jest średni, a fiat nie jest duży ani czarny. Określ wielkość i kolor każdego samochodu, jeśli wiadomo, że mały samochód jest czarny.


\end{document}