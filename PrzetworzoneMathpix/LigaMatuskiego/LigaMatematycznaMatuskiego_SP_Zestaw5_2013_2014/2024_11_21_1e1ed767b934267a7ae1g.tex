% This LaTeX document needs to be compiled with XeLaTeX.
\documentclass[10pt]{article}
\usepackage[utf8]{inputenc}
\usepackage{amsmath}
\usepackage{amsfonts}
\usepackage{amssymb}
\usepackage[version=4]{mhchem}
\usepackage{stmaryrd}
\usepackage[fallback]{xeCJK}
\usepackage{polyglossia}
\usepackage{fontspec}
\setCJKmainfont{Noto Serif CJK JP}

\setmainlanguage{polish}
\setmainfont{CMU Serif}

\title{LIGA MATEMATYCZNA im. Zdzisława Matuskiego \\
 FINAモ \\
 15 kwietnia 2014 SZKOŁA PODSTAWOWA }

\author{}
\date{}


\begin{document}
\maketitle
\section*{ZADANIE 1.}
Znajdź najmniejszą liczbę naturalną większą od 2014 i mającą taką samą sumę cyfr jak 2014.

\section*{ZADANIE 2.}
Sześć dziewczynek ważących \(18 \mathrm{~kg}, 19 \mathrm{~kg}, 20 \mathrm{~kg}, 23 \mathrm{~kg}, 38 \mathrm{~kg}\) oraz 42 kg ma podzielić się na dwie grupy w taki sposób, aby łączna waga dziewczynek w każdej grupie była taka sama. Na ile sposobów można dokonać takiego podziału?

\section*{ZADANIE 3.}
W rodzinie Wojtka są cztery osoby. Suma ich lat jest równa 100. Wojtek jest o cztery lata starszy od Asi, a tata jest o sześć lat starszy od mamy. Asia poprosiła złotą rybkę, aby cofnęła czas o całkowitą liczbę lat do takiego momentu, w którym Asia była sześć razy młodsza od mamy. Złota rybka zastanowiła się i cofnęła czas o pięć lat. Ile lat mają członkowie rodziny po cofnięciu czasu?

\section*{ZADANIE 4.}
Obwód kwadratu jest równy 32 cm . Środki dwóch kolejnych boków tego kwadratu połączono ze sobą i z wierzchołkiem nie należącym do tych boków. Oblicz pole otrzymanego w ten sposób trójkąta.

\section*{ZADANIE 5.}
Adam, Bartek, Czarek i Darek lubią miło spędzać czas wolny. Każdy wybiera swoje ulubione miejsce i dociera tam w inny sposób. Odkryj, kto dokąd wychodzi i jak się tam dostaje, jeżeli:

\begin{itemize}
  \item Bartek nigdy nie chodzi do kina i zawsze jeździ pociągiem;
  \item Adam nie opuszcza żadnego koncertu w filharmonii, ale nie ma roweru;
  \item Czarek jest bywalcem muzeów. Udaje się tam na piechotę;
  \item Jeden z kolegów spędza wieczory w teatrze, a inny wszędzie jeździ samochodem.
\end{itemize}

\end{document}