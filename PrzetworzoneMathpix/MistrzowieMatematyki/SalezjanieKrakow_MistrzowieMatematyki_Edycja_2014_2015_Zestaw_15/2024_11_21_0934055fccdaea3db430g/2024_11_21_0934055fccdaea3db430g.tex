\documentclass[10pt]{article}
\usepackage[polish]{babel}
\usepackage[utf8]{inputenc}
\usepackage[T1]{fontenc}
\usepackage{amsmath}
\usepackage{amsfonts}
\usepackage{amssymb}
\usepackage[version=4]{mhchem}
\usepackage{stmaryrd}

\title{GIMNAZJUM }

\author{}
\date{}


\begin{document}
\maketitle
\begin{enumerate}
  \item Dany jest czworokąt wypukły \(A B C D\) o polu \(S\). Punkt \(A\) jest środkiem odcinka \(D E\), punkt \(B\) jest środkiem odcinka \(A F, C\) jest środkiem odcinka \(B G\), zaś \(D\) jest środkiem odcinka \(C H\). Oblicz pole czworokąta EFGH.
  \item Rozwiąż równanie:
\end{enumerate}

\[
\sqrt{-x^{2}-2 x}+\sqrt{-x^{2}-2 x+8}=4
\]

\begin{enumerate}
  \setcounter{enumi}{2}
  \item Jaś i Małgosia zbierali muchomory w lesie. Okazało się, że kropek na muchomorach Małgosi było 13 razy więcej niż na muchomorach Jasia. Gdyby Małgosia oddała Jasiowi swój muchomor z najmniejszą liczbą kropek, to wtedy u niej byłoby 8 razy więcej kropek niż u Jasia. Oblicz, ile co najwyżej muchomorów mogła zebrać Małgosia?
\end{enumerate}

\section*{LICEUM}
\begin{enumerate}
  \item Dany jest czworokąt wypukły \(A B C D\). Przekątne tego czworokąta przecinają się w punkcie \(P\). Punkty \(K, L, M, N\) są środkami okręgów opisanych odpowiednio na trójkątach \(A B P, B C P\), \(C D P, D A P\). Wykaż, że pole czworokąta KLMN nie jest mniejsze od połowy pola czworokąta \(A B C D\).
  \item Rozwiąż w zbiorze liczb naturalnych równanie:
\end{enumerate}

\[
14^{a}+9^{b}=5^{c}+19^{d}
\]

\begin{enumerate}
  \setcounter{enumi}{2}
  \item Wykazać, ze z dowolnego zbioru 100 dodatnich liczb całkowitych można tak wybrać pewien niepusty podzbiór, by suma liczb z tego podzbioru była podzielna przez 100.
\end{enumerate}

Rozwiazania należy oddać do piątku 15 maja do godziny 12.30 koordynatorowi konkursu panu Jarostawowi Szczepaniakowi lub swojemu nauczycielowi matematyki.

Na stronie internetowej szkoły w zakładce Konkursy i olimpiady można znaleźć wyniki dotychczasowych rund i rozwiązania zadań.


\end{document}