\documentclass[10pt]{article}
\usepackage[polish]{babel}
\usepackage[utf8]{inputenc}
\usepackage[T1]{fontenc}
\usepackage{amsmath}
\usepackage{amsfonts}
\usepackage{amssymb}
\usepackage[version=4]{mhchem}
\usepackage{stmaryrd}

\title{GIMNAZJUM }

\author{}
\date{}


\begin{document}
\maketitle
\begin{enumerate}
  \item Udowodnij, że:
\end{enumerate}

\[
\sqrt{3-\sqrt{8}}+\sqrt{5-\sqrt{24}}+\sqrt{7-\sqrt{48}}=1
\]

\begin{enumerate}
  \setcounter{enumi}{1}
  \item W kole o promieniu 10 wybrano 99 punktów. Dowieść, że wewnątrz koła istnieje punkt odległy od każdego z wybranych punktów o więcej niż 1.
  \item Wiadomo, że prawdziwa moneta waży 10 gramów, a fałszywa 9 gramów. Mamy 5 monet o łącznej wadze 48 gramów i dysponujemy wagą elektroniczną. Wykonując ważenie możemy położyć ma wagę dowolną liczbę wybranych przez nas monet i odczytać ich łączną wagę. Czy wykonując nie więcej niż 3 ważenia możemy zawsze rozpoznać, które z danych monet są fałszywe, a które prawdziwe?
\end{enumerate}

\section*{LICEUM}
\begin{enumerate}
  \item Dany jest sześciokąt wypukły ABCDEF o kątach przy wierzchołkach A, B, C, D równych odpowiednio \(90^{\circ}, 128^{\circ}, 142^{\circ}, 90^{\circ}\). Wykaż, że pole tego sześciokąta jest mniejsze niż \(\frac{1}{2} \cdot|A D|^{2}\).
  \item Dany jest równoległobok \(A B C D\). Punkt \(E\) należy do boku \(A B\), a punkt \(F\) do boku \(A D\). Prosta EF przecina prostą CB w punkcie P, a prostą CD w punkcie Q. Wykaż, że pole trójkąta CEF jest równe polu trójkąta APQ.
  \item W przestrzeni danych jest takich \(n\) punktów ( \(n>4\) ), że żadne cztery nie leżą na jednej płaszczyźnie. Każde dwa z tych punktów połączono odcinkiem niebieskim lub czerwonym. Udowodnij, że można tak wybrać jeden z tych kolorów, aby każde dwa punkty były połączone odcinkiem lub łamaną wybranego koloru.
\end{enumerate}

\end{document}