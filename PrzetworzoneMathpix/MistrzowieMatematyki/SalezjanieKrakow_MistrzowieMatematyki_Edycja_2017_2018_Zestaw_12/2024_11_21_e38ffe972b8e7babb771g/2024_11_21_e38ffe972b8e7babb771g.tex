\documentclass[10pt]{article}
\usepackage[polish]{babel}
\usepackage[utf8]{inputenc}
\usepackage[T1]{fontenc}
\usepackage{amsmath}
\usepackage{amsfonts}
\usepackage{amssymb}
\usepackage[version=4]{mhchem}
\usepackage{stmaryrd}

\title{GIMNAZJUM }

\author{}
\date{}


\begin{document}
\maketitle
\begin{enumerate}
  \item Miary kątów trójkąta ABC wynoszą \(40^{\circ}, 60^{\circ} \mathrm{i} 80^{\circ}\). Punkty KLM są punktami styczności okręgu wpisanego z bokami tego trójkąta. Oblicz kąty trójkąta KLM.
  \item Turysta chce się dostać na wyspę w kształcie kwadratu o boku 100 m. Wyspa otoczona jest rowem z wodą o szerokości 5 m; wyspa wraz z rowem tworzą kwadrat o boku 110 m. Przy brzegu leżą dwie deski o długości 480 cm i szerokości 20 cm . Czy turysta może dostać się na wyspę?
  \item Czy istnieje ostrosłup, którego podstawą jest czworokąt wypukły i którego dwie przeciwległe ściany boczne są prostopadłe zarówno do siebie, jak i do podstawy ostrosłupa?
\end{enumerate}

\section*{LICEUM}
\begin{enumerate}
  \item Wykres funkcji \(y=\frac{1}{x^{2}}\) przecinamy prostą równoległa do osi OX. Oznaczmy punkty przecięcia przez A i B, zaś przez C oznaczmy punkt (2, -4). Udowodnij, że pole trójkąta ABC jest niemniejsze niż 4.
  \item Wykaż, że jeżeli \(x>1\) i \(y>1\) to \(\frac{x}{y+1}+\frac{y}{x+1}>1\).
  \item Dany jest czworokąt, którego dwa przeciwległe kąty są proste. Długości boków przy jednym z kątów prostych wynoszą odpowiednio a i \(b\), długości boków przy drugim kącie prostym są równe (i nie wiadomo, ile wynoszą). Oblicz długość przekątnej łączącej wierzchołki przy kątach prostych.
\end{enumerate}

\end{document}