\documentclass[10pt]{article}
\usepackage[polish]{babel}
\usepackage[utf8]{inputenc}
\usepackage[T1]{fontenc}
\usepackage{amsmath}
\usepackage{amsfonts}
\usepackage{amssymb}
\usepackage[version=4]{mhchem}
\usepackage{stmaryrd}

\title{GIMNAZJUM }

\author{}
\date{}


\begin{document}
\maketitle
\begin{enumerate}
  \item Basen napełniany jest przez trzy krany. Gdy otwarte są pierwszy i drugi, napełnianie trwa 20 godzin, gdy pierwszy i trzeci - 15 godzin, a gdy drugi i trzeci - tylko 12 godzin. W jakim czasie napełni się basen, gdy otwarte będą wszystkie trzy krany?
  \item Punkt P leży wewnątrz kwadratu ABCD. Odległości tego punktu od wierzchołków \(A, B\) i \(C\) wynoszą odpowiednio 2, 7 i 9. Ile wynosi odległość punktu P od wierzchołka D?
  \item Późnym wieczorem, nad rzeką przed kładką wąską i pełną dziur spotkała się czwórka podróżnych. Mieli tylko jedną latarkę. Przejście kładką bez latarki było nazbyt ryzykowne, innej drogi nie było. Jednocześnie na kładce mogły się znajdować nie więcej niż dwie osoby. Oznaczało to, że muszą przechodzić na raty - idą dwie osoby, potem jedna wraca z latarką. Tak się złożyło, że każdy z podróżnych mógł, idąc jak najszybciej, pokonać kładkę w innym czasie. Pierwszy w 10 minut, drugi w 5 minut, trzeci w 2 minuty, a czwarty w minutę. Oczywiście, jeśli idą dwie osoby, to poruszają się tempem tej wolniejszej. W jakim możliwie najkrótszym czasie mogli się wszyscy przeprawić na drugi brzeg?
\end{enumerate}

\section*{LICEUM}
\begin{enumerate}
  \item Do 100 zaadresowanych kopert włożono losowo 100 listów, do każdej koperty po jednym liście (każdy list jest skierowany do innego adresata). Przez \(p_{k}\) oznaczamy prawdopodobieństwo tego, że dokładnie \(k\) listów trafi do właściwych kopert. Oblicz wartość iloczynu
\end{enumerate}

\[
p_{1} \cdot p_{2} \cdot \cdots \cdot p_{100}
\]

\begin{enumerate}
  \setcounter{enumi}{1}
  \item lle pierwiastków ma wielomian
\end{enumerate}

\[
W(x)=x(x-2)(x-4) \cdot \cdots \cdot(x-2018)+(x-1)(x-3) \cdot \cdots \cdot(x-2019) ?
\]

\begin{enumerate}
  \setcounter{enumi}{2}
  \item Wewnątrz trójkąta wybrano punkt P. Przez punkt \(P\) poprowadzono trzy proste równoległe do trzech boków trójkąta. Podzieliły one trójkąt na trzy mniejsze trójkąty i trzy równoległoboki. Pola mniejszych trójkątów wynoszą 1, 4 i 9. Ile wynosi pole wyjściowego trójkąta?
\end{enumerate}

\end{document}