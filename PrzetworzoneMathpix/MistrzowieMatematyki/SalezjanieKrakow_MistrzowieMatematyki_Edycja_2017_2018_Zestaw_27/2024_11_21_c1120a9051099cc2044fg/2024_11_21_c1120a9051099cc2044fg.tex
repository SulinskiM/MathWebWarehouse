\documentclass[10pt]{article}
\usepackage[polish]{babel}
\usepackage[utf8]{inputenc}
\usepackage[T1]{fontenc}
\usepackage{amsmath}
\usepackage{amsfonts}
\usepackage{amssymb}
\usepackage[version=4]{mhchem}
\usepackage{stmaryrd}

\title{Zestaw 27 }

\author{}
\date{}


\begin{document}
\maketitle
\section*{GIMNAZJUM}
\begin{enumerate}
  \item Turysta idący na stację kolejową przeszedł w ciągu godziny \(3,5 \mathrm{~km}\) i zorientował się, że idąc nadal z tą prędkością, spóźni się na pociąg o godzinę. Przyspieszył więc i pozostałą część trasy przeszedł z prędkością \(5 \mathrm{~km} / \mathrm{h}\), docierając na stację pół godziny przed planowanym odjazdem pociągu. Jak długą trasę przebył ten turysta?
  \item W dwóch kubkach znajduje się w sumie 1 litr wody. Jeśli z pierwszego kubka przelalibyśmy do drugiego tyle, aby jego zawartość podwoiła się, a następnie z drugiego przelalibyśmy do pierwszego tyle, aby podwoiła się zawartość pierwszego, to w obu kubkach znalazłoby się tyle samo wody. Ile wody znajduje się w każdym z kubków?
  \item Dany jest trójkąt równoboczny \(A B C\) o boku długości a. Punkt O jest dowolnym punktem tego trójkąta. Przez punkt O poprowadzono proste k, l, m, równoległe odpowiednio do boków BC, CA i AB. Proste te przecinają wysokości opuszczone z wierzchołków A, B i C odpowiednio w punktach X, Y i Z. Oblicz sumę długości odcinków AX+BY+CZ.
\end{enumerate}

\section*{LICEUM}
\begin{enumerate}
  \item Dany jest trójkąt \(A B C\). Punktami wspólnymi z bokami \(A B, B C\) i \(C A\) okręgu wpisanego w ten trójkąt są odpowiednio K, Li M. Okrąg dopisany do trójkąta, leżący po przeciwnej stronie prostej BC, jest styczny do prostych \(\mathrm{AB}, \mathrm{BC}\) i CA odpowiednio w punktach \(\mathrm{P}, \mathrm{Q}\) i R . Wykaż, że \(B K=B L=C Q=C R\).
  \item Dwóch złodziei schroniło się w gospodzie. Mieli rano podzielić łup. W nocy zbudził się pierwszy złodziej i stwierdził, że nie ma cierpliwości czekać do świtu. Zakradł się do skrzyni z łupem i podzielił pieniądze na dwie równe części. Została mu jedna moneta, więc wręczył ją jako zapłatę za milczenie czujnemu gospodarzowi, który go zauważył. Swoją część monet ukrył, a resztę zostawił w skrzyni. Po godzinie zbudził się drugi złodziej. On również zakradł się do skrzyni, a znalezione w niej pieniądze podzielił na dwie równe części; podobnie jak kompan jedną monetę pozostającą wręczył gospodarzowi. Schował swoją część i poszedł spać. Rano obaj złodzieje otwarli skrzynię i znalezione w niej monety rozdzielili po równo między siebie, odkładając wcześniej jedną monetę jako zapłatę dla gospodarza. Ile monet mogło być na początku w skrzyni? Podaj wszystkie możliwości.
  \item Przez punkt wewnątrz trójkąta poprowadzono proste równoległe do boków trójkąta. Proste te dzielą trójkąt na sześć części, z których trzy są trójkątami o polach S, T, U. Oblicz pole danego trójkąta.
\end{enumerate}

\end{document}