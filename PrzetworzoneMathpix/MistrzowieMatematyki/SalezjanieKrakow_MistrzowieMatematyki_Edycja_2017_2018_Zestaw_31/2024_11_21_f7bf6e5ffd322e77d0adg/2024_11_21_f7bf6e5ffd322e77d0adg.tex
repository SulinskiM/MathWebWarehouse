\documentclass[10pt]{article}
\usepackage[polish]{babel}
\usepackage[utf8]{inputenc}
\usepackage[T1]{fontenc}

\title{GIMNAZJUM }

\author{}
\date{}


\begin{document}
\maketitle
\begin{enumerate}
  \item Jak czworokąt ABCD podzielić na dwie części o równych polach prostą przechodzącą przez wierzchołek D?
  \item Kolorowe koperty pakowane są w paczkach po 100 sztuk. Zdjęcie paczki z półki zajmuje sprzedawcy 5 sekund, wyjęcie jednej koperty z paczki - sekundę (niezależnie od tego, czy wyjmuje po jednej kopercie, czy odlicza odpowiednią liczbę i wyjmuje razem). Klient poprosił o 10 kopert zielonych, 10 niebieskich i 60 żółtych. W jakim najkrótszym czasie sprzedawca może mu wręczyć koperty?
  \item Na tablicy napisano liczby od 1 do 2018. Wybieramy dwie z nich, ścieramy i dopisujemy ich różnicę. Postępujemy tak do momentu, gdy zostanie nam jedna liczba. Czy może nią być liczba 34?
\end{enumerate}

\section*{LICEUM}
\begin{enumerate}
  \item Dany jest czworokąt wypukły ABCD. Punkty K i L leżą odpowiednio na odcinkach AB i AD, przy czym czworokąt AKCL jest równoległobokiem. Odcinki KD i BL przecinają się w punkcie M. Wykaż, że pola czworokątów AKML i BCDM są równe.
  \item Rysujemy dziesięciokąt foremny i w każdym wierzchołku kładziemy żeton. Ruch polega na wybraniu dowolnych dwóch żetonów i przełożeniu każdego z nich do dowolnego wierzchołka sąsiadującego z tym, w którym leżał. Czy można doprowadzić do sytuacji, gdy wszystkie żetony leżą w jednym wierzchołku?
  \item Mamy 17 liczb rzeczywistych. Wiadomo, że suma dowolnych dziewięciu spośród tych liczb jest większa od sumy pozostałych ośmiu. Wykaż, że wszystkie te liczby są dodatnie.
\end{enumerate}

\end{document}