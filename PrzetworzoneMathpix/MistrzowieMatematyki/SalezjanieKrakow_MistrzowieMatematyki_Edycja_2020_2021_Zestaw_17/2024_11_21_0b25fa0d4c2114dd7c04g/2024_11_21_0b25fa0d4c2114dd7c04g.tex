\documentclass[10pt]{article}
\usepackage[polish]{babel}
\usepackage[utf8]{inputenc}
\usepackage[T1]{fontenc}
\usepackage{amsmath}
\usepackage{amsfonts}
\usepackage{amssymb}
\usepackage[version=4]{mhchem}
\usepackage{stmaryrd}

\title{KLASY PO SZKOLE PODSTAWOWEJ }

\author{}
\date{}


\begin{document}
\maketitle
\begin{enumerate}
  \item Turysta przeszedł drogę z miasta A do miasta B i z powrotem w ciągu 3 godzin i 41 minut. Droga z A do B wiodła początkowo pod górę, potem po równym terenie, a następnie z góry. Prędkość turysty pod górę wynosi \(4 \mathrm{~km} / \mathrm{h}\), po równym terenie \(5 \mathrm{~km} / \mathrm{h}\), a z góry \(6 \mathrm{~km} / \mathrm{h}\). Odległość z A do B wynosi 9 km. Na jakiej długości droga z miasta A do B wiedzie po równym terenie?
  \item Ciąg Fibonacciego określony jest następująco:
\end{enumerate}

\[
\begin{gathered}
F(1)=F(2)=1 \\
F(n+2)=F(n+1)+F(n) \text { dla } n \text { całkowitych dodatnich }
\end{gathered}
\]

Ustal, czy liczba F(2021) jest parzysta. Odpowiedź uzasadnij.\\
3. Rozstrzygnij, czy szachownicę \(8 x 8\) z której usunięto pola A1 i H8 można pokryć kostkami domina, z których każde pokrywa dwa pola szachownicy i kostki na siebie nie zachodzą.

\section*{KLASY PO GIMNAZJUM}
\begin{enumerate}
  \item Rozważmy wszystkie trójkąty o ustalonej podstawie \(A B\), których wierzchołek \(C\) należy do pewnej prostej \(k\) równoległej do prostej AB i się z nią nie pokrywającej. Udowodnij, że ortocentra wszystkich tych trójkątów tworzą parabolę.
  \item Udowodnij, że dla każdej liczby naturalnej \(n\) prawdziwa jest równość:
\end{enumerate}

\[
\begin{gathered}
n^{2}+\left(n^{2}+1\right)+\left(n^{2}+2\right) \ldots+\left(n^{2}+n\right)=\left(n^{2}+n+1\right)+\left(n^{2}+n+2\right)+\cdots+\left(n^{2}+n+n\right) \\
\text { czyli na przykład: } \\
1+2=3 \\
4+5+6=7+8 \\
9+10+11+12=13+14+15 \text { itd. }
\end{gathered}
\]

\begin{enumerate}
  \setcounter{enumi}{2}
  \item W kasynie toczy się gra w ruletkę. Gdy Piotr opuścił grę z żetonami wartymi 16000 zł, średni majątek przy tym stole zmalał od 1000 zł. Gdy jakiś czas później do stołu dołączyły Ala i Ola, każda z majątkiem wartym 2000 zł, ponownie przeciętny wynik przy stole spadł o 1000 zł. Ilu graczy siedziało przy stole, zanim Piotr opuścił grę?
\end{enumerate}

\end{document}