\documentclass[10pt]{article}
\usepackage[polish]{babel}
\usepackage[utf8]{inputenc}
\usepackage[T1]{fontenc}
\usepackage{amsmath}
\usepackage{amsfonts}
\usepackage{amssymb}
\usepackage[version=4]{mhchem}
\usepackage{stmaryrd}

\title{KLASY PO SZKOLE PODSTAWOWEJ }

\author{}
\date{}


\newcommand\Varangle{\mathop{{<\!\!\!\!\!\text{\small)}}\:}\nolimits}

\begin{document}
\maketitle
\begin{enumerate}
  \item Rozwiąż układ równań:
\end{enumerate}

\[
\begin{gathered}
2 x^{2}+y^{2}=4 \\
2 x y-2 x=-5 .
\end{gathered}
\]

\begin{enumerate}
  \setcounter{enumi}{1}
  \item W pewnym turnieju piłkarskim bierze udział 2021 drużyn. Rozgrywki prowadzone są w kolejne soboty systemem pucharowym: przegrywający odpada (jeśli jest remis decydują rzuty karne). Jeśli w danej rundzie jest nieparzysta liczba drużyn, jedna z drużyn przechodzi do następnej rundy bez rozgrywania meczu (ta drużyna jest losowana). Ile trzeba rozegrać spotkań, żeby wyłonić zwycięzcę?
  \item Czy istnieje wielościan wypukły mający dokładnie 100 ścian, z których przynajmniej jedna jest 99-kątem i taki, że w każdym jego wierzchołku zbiegają się dokładnie trzy krawędzie? Odpowiedź uzasadnij.
\end{enumerate}

\section*{KLASY PO GIMNAZJUM}
\begin{enumerate}
  \item Dany jest trapez \(A B C D\) o podstawach \(A B\) i \(C D\), w którym \(A C=B C\). Punkt \(M\) jest środkiem ramienia AD . Wykaż, że \(\Varangle A C M=\Varangle C B D\).
  \item Czy istnieje wielościan wypukły, w którym każda ściana ma inną ilość krawędzi? Odpowiedź uzasadnij.
  \item Liczba a da się przedstawić w postaci sumy kwadratów dwóch liczb całkowitych. Czy liczba 2020a również da się przedstawić w postaci sumy kwadratów dwóch liczb całkowitych? Odpowiedź uzasadnij.
\end{enumerate}

\end{document}