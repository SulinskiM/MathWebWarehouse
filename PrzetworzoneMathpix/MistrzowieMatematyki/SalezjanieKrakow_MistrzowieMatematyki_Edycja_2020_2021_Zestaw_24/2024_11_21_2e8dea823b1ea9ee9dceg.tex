\documentclass[10pt]{article}
\usepackage[polish]{babel}
\usepackage[utf8]{inputenc}
\usepackage[T1]{fontenc}
\usepackage{amsmath}
\usepackage{amsfonts}
\usepackage{amssymb}
\usepackage[version=4]{mhchem}
\usepackage{stmaryrd}

\title{KLASY PO SZKOLE PODSTAWOWEJ }

\author{}
\date{}


\begin{document}
\maketitle
\begin{enumerate}
  \item Do pewnej restauracji przybył dostojny gość z dwoma przyjaciółmi.
\end{enumerate}

\begin{itemize}
  \item Proszę o trzy wielkie placki ziemniaczane według waszego słynnego przepisu - powiedział.
  \item Dla każdego z nas po jednym. Każdy z nich ma się smażyć dokładnie dwie minuty, po minucie z każdej strony! Bardzo się śpieszymy, za trzy minuty placki mają być na stole!
\end{itemize}

Niestety, w restauracji były tylko dwie patelnie. Na patelni mieścił się dokładnie jeden placek smażony według słynnego przepisu. Czy istnieje sposób, by podać trzy placki zgodnie z zamówieniem?\\
2. Ile wynosi cyfra jedności liczby

\[
2009^{2008^{2007}{ }^{3^{2^{1}}}} ?
\]

\begin{enumerate}
  \setcounter{enumi}{2}
  \item Mamy 10 kolejnych liczb naturalnych. Usuwamy jedną z nich. Suma pozostałych liczb wynosi 2021. Znajdź sumę wszystkich dziesięciu liczb.
\end{enumerate}

\section*{KLASY PO GIMNAZJUM}
\begin{enumerate}
  \item O liczbach \(a, b, c\) wiemy, że należą do przedziału ( 0,2\(\rangle\). Udowodnij, że
\end{enumerate}

\[
a+b+c+2 \geq a b c
\]

\begin{enumerate}
  \setcounter{enumi}{1}
  \item Dany jest trójkąt równoboczny \(A B C\). Punkt \(D\) leży na krótszym łuku \(A B\) okręgu opisanego na tym trójkącie, punkt E jest środkiem odcinka \(A C\), a punkt \(F\) obrazem punktu \(D\) w symetrii względem punktu E. Wykaż, że |DF|=|BF|
  \item Udowodnij, że każda liczba naturalna \(2 n\)-cyfrowa, która ma na początku \(n\) czwórek, potem \(n-1\) ósemek i na końcu dziewiątkę, jest kwadratem liczby naturalnej.
\end{enumerate}

\end{document}