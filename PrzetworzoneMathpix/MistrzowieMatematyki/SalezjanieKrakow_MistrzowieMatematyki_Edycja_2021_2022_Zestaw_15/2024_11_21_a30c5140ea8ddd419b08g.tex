\documentclass[10pt]{article}
\usepackage[polish]{babel}
\usepackage[utf8]{inputenc}
\usepackage[T1]{fontenc}
\usepackage{amsmath}
\usepackage{amsfonts}
\usepackage{amssymb}
\usepackage[version=4]{mhchem}
\usepackage{stmaryrd}

\title{KLASY PIERWSZE I DRUGIE }

\author{}
\date{}


\begin{document}
\maketitle
\begin{enumerate}
  \item Udowodnij, ze dla dowolnych dodatnich liczb rzeczywistych \(a, b, c\) zachodzi nierówność \(\sqrt{a+b}+\sqrt{b+c}+\sqrt{c+a} \geq \sqrt{2 a}+\sqrt{2 b}+\sqrt{2 c}\)
  \item Dany jest prostopadłościan \(A B C D E F G H\) o podstawie \(A B C D\) i krawędziach bocznych \(A E, B F\), CG, DH. Punkt S jest środkiem krawędzi EH. Udowodnij, że z odcinków o długościach \(A G, C H, 2 \cdot A S\) można zbudować trójkąt.
  \item W ośmiokącie wszystkie przekątne mają długość 1 i przecinają się w jednym punkcie. Udowodnij, że obwód tego ośmiokąta jest mniejszy niż 8.
\end{enumerate}

\section*{KLASY TRZECIE}
\begin{enumerate}
  \item Dany jest trójkąt prostokątny o przyprostokątnych długości odpowiednio \(a\) i \(b\). Na pierwszej z tych przyprostokątnych wybrano punkt \(P\), a na drugiej punkt \(Q\). Niech \(K\) i \(H\) będą rzutami prostokątnymi odpowiednio punktów \(P\) i \(Q\) na przeciwprostokątną. Jaka jest najmniejsza możliwa wartość sumy \(|K P|+|P Q|+|Q H|\) ? Odpowiedź uzasadnij.
  \item Mamy 17 liczb rzeczywistych. Wiadomo, że suma dowolnych dziewięciu spośród tych liczb jest większa od sumy pozostałych ośmiu. Wykaż, że wszystkie te liczby są dodatnie.
  \item Wyznacz wszystkie liczby całkowite nieujemne \(n\), dla których liczba \(7^{n}+2 \cdot 4^{n}\) jest liczbą pierwszą.
\end{enumerate}

\end{document}