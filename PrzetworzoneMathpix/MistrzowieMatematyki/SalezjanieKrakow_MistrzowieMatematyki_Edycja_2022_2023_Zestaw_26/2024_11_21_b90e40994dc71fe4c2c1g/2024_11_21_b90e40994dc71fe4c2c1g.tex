\documentclass[10pt]{article}
\usepackage[polish]{babel}
\usepackage[utf8]{inputenc}
\usepackage[T1]{fontenc}
\usepackage{amsmath}
\usepackage{amsfonts}
\usepackage{amssymb}
\usepackage[version=4]{mhchem}
\usepackage{stmaryrd}

\title{KLASY PIERWSZE I DRUGIE }

\author{}
\date{}


\begin{document}
\maketitle
\begin{enumerate}
  \item W pięciokącie wypukłym ABCDE kąty przy wierzchołkach B i D są proste. Wykaż, ze obwód trójkąta ACE jest nie mniejszy od \(2 \cdot B D\).
  \item Na przeciwległych wierzchołkach sześciennego pudła o krawędzi 1 siedzą pająk i mucha. Pająk chce przejść najkrótsza możliwą droga po powierzchni pudła do wierzchołka, w którym znajduje się mucha. Jak długą drogę musi pokonać? Którędy powinien iść? Ile ma do wyboru różnych najkrótszych dróg?
  \item Budowane pomieszczenie w kształcie prostopadłościanu ma mieć wysokość 3 m , podłoga zaś ma mieć wymiary \(3 \mathrm{~m} \times 7,5 \mathrm{~m}\). W pomieszczeniu nie będzie okien, jedynie drzwi na jednej kwadratowej ścianie. Prąd do pomieszczenia ma być doprowadzony nad drzwiami, 25 cm pod sufitem, w odległości 1,5 m od obu sąsiednich ścian. Jedyne gniazdko ma natomiast być umieszczone na przeciwległej ścianie, też w odległości 1,5 m od obu sąsiednich ścian, ale 25 cm nad podłogą. Jak, chcąc zużyć jak najmniej kabla, poprowadzić go od puszki z prądem do kontaktu?
\end{enumerate}

\section*{KLASY TRZECIE I CZWARTE}
\begin{enumerate}
  \item Suma kwadratów dowolnych trzech liczb spośród \(a, b, c, d, e>0\) jest równa sumie sześcianów dwóch pozostałych. Wyznaczyć te liczby.
  \item Liczby rzeczywiste \(x\), y i z spełniają warunki: \(|x| \leq|y-z|,|y| \leq|z-x|\), \(|z| \leq|x-y|\), Dowieść, że jedna z nich jest równa sumie pozostałych.
  \item Wykaż, że funkcja \(f(x)=(x-a)(x-b)+(x-b)(x-c)+(x-c)(x-a)\) ma dla dowolnej trójki liczb rzeczywistych \(a, b, c\), miejsce zerowe.
\end{enumerate}

\end{document}