\documentclass[10pt]{article}
\usepackage[polish]{babel}
\usepackage[utf8]{inputenc}
\usepackage[T1]{fontenc}
\usepackage{graphicx}
\usepackage[export]{adjustbox}
\graphicspath{ {./images/} }
\usepackage{amsmath}
\usepackage{amsfonts}
\usepackage{amssymb}
\usepackage[version=4]{mhchem}
\usepackage{stmaryrd}
\usepackage{hyperref}
\hypersetup{colorlinks=true, linkcolor=blue, filecolor=magenta, urlcolor=cyan,}
\urlstyle{same}
\usepackage{multirow}

\title{KOD }

\author{}
\date{}


\newcommand\Varangle{\mathop{{<\!\!\!\!\!\text{\small)}}\:}\nolimits}

\begin{document}
\maketitle
IMIĘ I NAZWISKO *\\
\includegraphics[max width=\textwidth, center]{2024_11_21_f29375993e8c629c464fg-01}\\
\(\square\)

\begin{itemize}
  \item nieobowiązkowe
\end{itemize}

\section*{PRÓBNY EGZAMIN MATURALNY Z NOWĄ ERĄ MATEMATYKA - POZIOM ROZSZERZONY}
\section*{Instrukcja dla zdającego}
\begin{enumerate}
  \item Sprawdź, czy arkusz egzaminacyjny zawiera 22 strony (zadania 1-16). Ewentualny brak stron zgłoś nauczycielowi nadzorującemu egzamin.
  \item Rozwiązania zadań i odpowiedzi zapisz w miejscu na to przeznaczonym.
  \item Pamiętaj, że pominięcie argumentacji lub istotnych obliczeń w rozwiązaniu zadań otwartych może spowodować, że za to rozwiązanie nie otrzymasz pełnej liczby punktów.
  \item Pisz czytelnie. Używaj długopisu/pióra tylko z czarnym tuszem/atramentem.
  \item Nie używaj korektora, a błędne zapisy wyraźnie przekreśl.
  \item Pamiętaj, że zapisy w brudnopisie nie będą oceniane.
  \item Podczas egzaminu możesz korzystać z zestawu wzorów matematycznych, cyrkla i linijki oraz kalkulatora prostego.
  \item Na tej stronie wpisz swój kod oraz imię i nazwisko.
  \item Nie wpisuj żadnych znaków w części przeznaczonej dla osoby sprawdzającej.\\
\(\square\) dysleksja
\end{enumerate}

STYCZEŃ 2016

Czas pracy:\\
180 minut

Liczba punktów\\
do uzyskania: 50

W zadaniach 1-5 wybierz i zaznacz poprawną odpowiedź.

\section*{Zadanie 1. (0-1)}
Suma wszystkich rozwiązań równania \(||x-4|-2|=3\) jest równa\\
A. 8\\
B. 9\\
C. 12\\
D. 16

Zadanie 2. (0-1)

\[
\left\{\begin{array}{l}
a_{1}=4 \\
a_{n+1}=\frac{2 a_{n}}{\sqrt{2}} \text { dla } n \geqslant 1
\end{array}\right.
\]

Wskaż wzór ogólny ciągu \(\left(b_{n}\right)\), w którym \(b_{n}=a_{n}+a_{n+2}\) dla \(n \geqslant 1\).\\
A. \(b_{n}=(\sqrt{2})^{n+3}\)\\
B. \(b_{n}=2^{n+4}\)\\
C. \(b_{n}=3 \cdot 2^{\frac{n+3}{2}}\)\\
D. \(b_{n}=4 \cdot(\sqrt{2})^{2 n}\)

\section*{Zadanie 3. (0-1)}
Jedyny pierwiastek rzeczywisty wielomianu \(w(x)=2 x^{3}+(c-5) x^{2}+c x-5\) o współczynnikach całkowitych jest liczbą pierwszą. Zatem parametr \(c\) jest równy\\
A. 19\\
B. 4\\
C. -1\\
D. -4

\section*{Zadanie 4. (0-1)}
Liczby \(x, y, z\) są dodatnie i różne od 1 oraz \(\log _{x} \sqrt{y}=\frac{1}{3}\) i \(\log _{y} \sqrt[3]{z}=\frac{1}{4}\). Wskaż wartość wyrażenia \(\log _{z} \sqrt[4]{x}\).\\
A. 18\\
B. 1\\
C. \(\frac{1}{2}\)\\
D. \(\frac{1}{5}\)

\section*{Zadanie 5. (0-1)}
Ile różnych funkcji można utworzyć na zbiorze \(X=\{-2,-1,3,4\}\) o wartościach ze zbioru \(Y=\{5,6,7,8,9,10\}\) ?\\
A. 2880\\
B. 1296\\
C. 360\\
D. 24

\section*{BRUDNOPIS}
Więcej arkuszy znajdziesz na stronie: \href{http://arkusze.pl}{arkusze.pl}\\
\includegraphics[max width=\textwidth, center]{2024_11_21_f29375993e8c629c464fg-03}

\begin{center}
\begin{tabular}{|c|l|c|c|c|c|c|}
\hline
\multirow{2}{*}{\begin{tabular}{c}
Wypełnia \\
sprawdzający \\
\end{tabular}} & Nr zadania & 1 & 2 & 3 & 4 & 5 \\
\cline{2-7} & Maks. liczba pkt & 1 & 1 & 1 & 1 & 1 \\
\cline { 2 - 7 }
 & Uzyskana liczba pkt &  &  &  &  &  \\
\hline
\end{tabular}
\end{center}

W zadaniu 6. zakoduj wynik w kratkach zamieszczonych pod poleceniem.\\
\(W\) zadaniach 7-18 rozwiązania zapisz w wyznaczonych miejscach pod treścią zadania.

\section*{Zadanie 6. (0-2)}
Oblicz wartość parametru \(k\), dla której granica \(\lim _{n \rightarrow \infty} \frac{k n!-(n-2)!}{(k-2) n!+k(n-1)!}\) jest równa 9. W poniższe kratki wpisz kolejno cyfrę jedności i pierwsze dwie cyfry rozwinięcia dziesiętnego otrzymanego wyniku.\\
\(\square\)

Więcej arkuszy znajdziesz na stronie: \href{http://arkusze.pl}{arkusze.pl}\\
\includegraphics[max width=\textwidth, center]{2024_11_21_f29375993e8c629c464fg-04}

\section*{Zadanie 7. (0-2)}
Na rysunku przedstawiono fragment nieskończonego ciągu na przemian czarnych i białych kwadratów o wspólnym wierzchołku \(D\). Kwadrat \(A_{1} B_{1} C_{1} D\) koloru czarnego ma bok długości \(3 \sqrt{13}\). Długość boku każdego następnego kwadratu stanowi \(\frac{2}{3}\) długości boku kwadratu poprzedniego. Oblicz pole obszaru zaznaczonego kolorem czarnym w nieskończonym ciągu tych kwadratów.\\
\includegraphics[max width=\textwidth, center]{2024_11_21_f29375993e8c629c464fg-05}

Więcej arkuszy znajdziesz na stronie: \href{http://arkusze.pl}{arkusze.pl}

\begin{center}
\begin{tabular}{|c|c|c|c|c|c|c|c|c|c|c|c|c|c|c|c|c|c|c|c|c|c|c|c|c|c|c|c|c|c|}
\hline
 &  &  &  &  &  &  &  &  &  &  &  &  &  &  &  &  &  &  &  &  &  &  &  &  &  &  &  &  &  \\
\hline
 &  &  &  &  &  &  &  &  &  &  &  &  &  &  &  &  &  &  &  &  &  &  &  &  &  &  &  &  &  \\
\hline
 &  &  &  &  &  &  &  &  &  &  &  &  &  &  &  &  &  &  &  &  &  &  &  &  &  &  &  &  &  \\
\hline
 &  &  &  &  &  &  &  &  &  &  &  &  &  &  &  &  &  &  &  &  &  &  &  &  &  &  &  &  &  \\
\hline
 &  &  &  &  &  &  &  &  &  &  &  &  &  &  &  &  &  &  &  &  &  &  &  &  &  &  &  &  &  \\
\hline
 &  &  &  &  &  &  &  &  &  &  &  &  &  &  &  &  &  &  &  &  &  &  &  &  &  &  &  &  &  \\
\hline
 &  &  &  &  &  &  &  &  &  &  &  &  &  &  &  &  &  &  &  &  &  &  &  &  &  &  &  &  &  \\
\hline
 &  &  &  &  &  &  &  &  &  &  &  &  &  &  &  &  &  &  &  &  &  &  &  &  &  &  &  &  &  \\
\hline
 &  &  &  &  &  &  &  &  &  &  &  &  &  &  &  &  &  &  &  &  &  &  &  &  &  &  &  &  &  \\
\hline
 &  &  &  &  &  &  &  &  &  &  &  &  &  &  &  &  &  &  &  &  &  &  &  &  &  &  &  &  &  \\
\hline
 &  &  &  &  &  &  &  &  &  &  &  &  &  &  &  &  &  &  &  &  &  &  &  &  &  &  &  &  &  \\
\hline
 &  &  &  &  &  &  &  &  &  &  &  &  &  &  &  &  &  &  &  &  &  &  &  &  &  &  &  &  &  \\
\hline
 &  &  &  &  &  &  &  &  &  &  &  &  &  &  &  &  &  &  &  &  &  &  &  &  &  &  &  &  &  \\
\hline
 &  &  &  &  &  &  &  &  &  &  &  &  &  &  &  &  &  &  &  &  &  &  &  &  &  &  &  &  &  \\
\hline
 &  &  &  &  &  &  &  &  &  &  &  &  &  &  &  &  &  &  &  &  &  &  &  &  &  &  &  &  &  \\
\hline
 &  &  &  &  &  &  &  &  &  &  &  &  &  &  &  &  &  &  &  &  &  &  &  &  &  &  &  &  &  \\
\hline
 &  &  &  &  &  &  &  &  &  &  &  &  &  &  &  &  &  &  &  &  &  &  &  &  &  &  &  &  &  \\
\hline
 &  &  &  &  &  &  &  &  &  &  &  &  &  &  &  &  &  &  &  &  &  &  &  &  &  &  &  &  &  \\
\hline
 &  &  &  &  &  &  &  &  &  &  &  &  &  &  &  &  &  &  &  &  &  &  &  &  &  &  &  &  &  \\
\hline
 &  &  &  &  &  &  &  &  &  &  &  &  &  &  &  &  &  &  &  &  &  &  &  &  &  &  &  &  &  \\
\hline
 &  &  &  &  &  &  &  &  &  &  &  &  &  &  &  &  &  &  &  &  &  &  &  &  &  &  &  &  &  \\
\hline
 &  &  &  &  &  &  &  &  &  &  &  &  &  &  &  &  &  &  &  &  &  &  &  &  &  &  &  &  &  \\
\hline
 &  &  &  &  &  &  &  &  &  &  &  &  &  &  &  &  &  &  &  &  &  &  &  &  &  &  &  &  &  \\
\hline
 &  &  &  &  &  &  &  &  &  &  &  &  &  &  &  &  &  &  &  &  &  &  &  &  &  &  &  &  &  \\
\hline
 &  &  &  &  &  &  &  &  &  &  &  &  &  &  &  &  &  &  &  &  &  &  &  &  &  &  &  &  &  \\
\hline
\end{tabular}
\end{center}

Odpowiedź:

\begin{center}
\begin{tabular}{|l|l|c|c|}
\hline
\multirow{2}{*}{\begin{tabular}{c}
Wypełnia \\
sprawdzający \\
\end{tabular}} & Nr zadania & 6 & 7 \\
\cline { 2 - 4 }
 & Maks. liczba pkt & 2 & 2 \\
\cline { 2 - 4 }
 & Uzyskana liczba pkt &  &  \\
\hline
\end{tabular}
\end{center}

\section*{Zadanie 8. (0-3)}
W trójkącie prostokątnym \(A B C\), w którym bok \(A B\) jest przeciwprostokątną, na boku \(B C\) obrano punkt \(D\) taki, że \(|\Varangle D A B|=2|\Varangle C A D|\). Długość odcinka \(B D\) jest równa \(a\), a kąt \(C A D\) ma miarę \(\alpha\). Wykaż, że \(|A D|=\frac{a\left(1-4 \sin ^{2} \alpha\right)}{2 \sin \alpha}\).

\begin{center}
\begin{tabular}{|c|c|c|c|c|c|c|c|c|c|c|c|c|c|c|c|c|c|c|c|c|c|c|c|c|c|c|c|c|c|}
\hline
 &  &  &  &  &  &  &  &  &  &  &  &  &  &  &  &  &  &  &  &  &  &  &  &  &  &  &  &  &  \\
\hline
 &  &  &  &  &  &  &  &  &  &  &  &  &  &  &  &  &  &  &  &  &  &  &  &  &  &  &  &  &  \\
\hline
 &  &  &  &  &  &  &  &  &  &  &  &  &  &  &  &  &  &  &  &  &  &  &  &  &  &  &  &  &  \\
\hline
 &  &  &  &  &  &  &  &  &  &  &  &  &  &  &  &  &  &  &  &  &  &  &  &  &  &  &  &  &  \\
\hline
 &  &  &  &  &  &  &  &  &  &  &  &  &  &  &  &  &  &  &  &  &  &  &  &  &  &  &  &  &  \\
\hline
 &  &  &  &  &  &  &  &  &  &  &  &  &  &  &  &  &  &  &  &  &  &  &  &  &  &  &  &  &  \\
\hline
 &  &  &  &  &  &  &  &  &  &  &  &  &  &  &  &  &  &  &  &  &  &  &  &  &  &  &  &  &  \\
\hline
 &  &  &  &  &  &  &  &  &  &  &  &  &  &  &  &  &  &  &  &  &  &  &  &  &  &  &  &  &  \\
\hline
 &  &  &  &  &  &  &  &  &  &  &  &  &  &  &  &  &  &  &  &  &  &  &  &  &  &  &  &  &  \\
\hline
 &  &  &  &  &  &  &  &  &  &  &  &  &  &  &  &  &  &  &  &  &  &  &  &  &  &  &  &  &  \\
\hline
 &  &  &  &  &  &  &  &  &  &  &  &  &  &  &  &  &  &  &  &  &  &  &  &  &  &  &  &  &  \\
\hline
 &  &  &  &  &  &  &  &  &  &  &  &  &  &  &  &  &  &  &  &  &  &  &  &  &  &  &  &  &  \\
\hline
 &  &  &  &  &  &  &  &  &  &  &  &  &  &  &  &  &  &  &  &  &  &  &  &  &  &  &  &  &  \\
\hline
 &  &  &  &  &  &  &  &  &  &  &  &  &  &  &  &  &  &  &  &  &  &  &  &  &  &  &  &  &  \\
\hline
 &  &  &  &  &  &  &  &  &  &  &  &  &  &  &  &  &  &  &  &  &  &  &  &  &  &  &  &  &  \\
\hline
 &  &  &  &  &  &  &  &  &  &  &  &  &  &  &  &  &  &  &  &  &  &  &  &  &  &  &  &  &  \\
\hline
 &  &  &  &  &  &  &  &  &  &  &  &  &  &  &  &  &  &  &  &  &  &  &  &  &  &  &  &  &  \\
\hline
 &  &  &  &  &  &  &  &  &  &  &  &  &  &  &  &  &  &  &  &  &  &  &  &  &  &  &  &  &  \\
\hline
 &  &  &  &  &  &  &  &  &  &  &  &  &  &  &  &  &  &  &  &  &  &  &  &  &  &  &  &  &  \\
\hline
 &  &  &  &  &  &  &  &  &  &  &  &  &  &  &  &  &  &  &  &  &  &  &  &  &  &  &  &  &  \\
\hline
 &  &  &  &  &  &  &  &  &  &  &  &  &  &  &  &  &  &  &  &  &  &  &  &  &  &  &  &  &  \\
\hline
 &  &  &  &  &  &  &  &  &  &  &  &  &  &  &  &  &  &  &  &  &  &  &  &  &  &  &  &  &  \\
\hline
 &  &  &  &  &  &  &  &  &  &  &  &  &  &  &  &  &  &  &  &  &  &  &  &  &  &  &  &  &  \\
\hline
 &  &  &  &  &  &  &  &  &  &  &  &  &  &  &  &  &  &  &  &  &  &  &  &  &  &  &  &  &  \\
\hline
 &  &  &  &  &  &  &  &  &  &  &  &  &  &  &  &  &  &  &  &  &  &  &  &  &  &  &  &  &  \\
\hline
 &  &  &  &  &  &  &  &  &  &  &  &  &  &  &  &  &  &  &  &  &  &  &  &  &  &  &  &  &  \\
\hline
\end{tabular}
\end{center}

Więcej arkuszy znajdziesz na stronie: \href{http://arkusze.pl}{arkusze.pl}\\
\(\qquad\)

\begin{center}
\begin{tabular}{|l|l|l|l|l|l|l|l|l|l|l|l|l|l|l|l|l|l|l|l|l|l|}
 \\
\hline
\end{tabular}
\end{center}

\(\qquad\)\\
\(\qquad\)\\
\includegraphics[max width=\textwidth, center]{2024_11_21_f29375993e8c629c464fg-06(1)}\\
\includegraphics[max width=\textwidth, center]{2024_11_21_f29375993e8c629c464fg-06}

\begin{center}
\begin{tabular}{|c|c|c|c|c|c|c|c|c|c|c|c|c|c|c|c|c|c|c|c|c|c|c|c|c|c|c|c|c|c|}
\hline
 &  &  &  &  &  &  &  &  &  &  &  &  &  &  &  &  &  &  &  &  &  &  &  &  &  &  &  &  &  \\
\hline
 &  &  &  &  &  &  &  &  &  &  &  &  &  &  &  &  &  &  &  &  &  &  &  &  &  &  &  &  &  \\
\hline
 &  &  &  &  &  &  &  &  &  &  &  &  &  &  &  &  &  &  &  &  &  &  &  &  &  &  &  &  &  \\
\hline
 &  &  &  &  &  &  &  &  &  &  &  &  &  &  &  &  &  &  &  &  &  &  &  &  &  &  &  &  &  \\
\hline
 &  &  &  &  &  &  &  &  &  &  &  &  &  &  &  &  &  &  &  &  &  &  &  &  &  &  &  &  &  \\
\hline
\end{tabular}
\end{center}

\begin{center}
\includegraphics[max width=\textwidth]{2024_11_21_f29375993e8c629c464fg-07}
\end{center}

\begin{center}
\begin{tabular}{|l|l|c|}
\hline
\multirow{2}{*}{\begin{tabular}{c}
Wypełnia \\
sprawdzający \\
\end{tabular}} & Nr zadania & 8 \\
\cline { 2 - 3 }
 & Maks. liczba pkt & 3 \\
\cline { 2 - 3 }
 & Uzyskana liczba pkt &  \\
\hline
\end{tabular}
\end{center}

Zadanie 9. (0-3)\\
Wykaż, że wielomian \(f(x)=3 x^{10}-5 x^{6}+3\) nie ma pierwiastków rzeczywistych.

Więcej arkuszy znajdziesz na stronie: \href{http://arkusze.pl}{arkusze.pl}

\begin{center}
\begin{tabular}{|l|l|l|l|l|l|l|l|l|l|l|l|l|l|l|l|l|l|l|l|l|l|l|l|l|l|l|l|l|}
\hline
 \\
\hline
\end{tabular}
\end{center}

\includegraphics[max width=\textwidth, center]{2024_11_21_f29375993e8c629c464fg-08}\\
\includegraphics[max width=\textwidth, center]{2024_11_21_f29375993e8c629c464fg-08(1)}

Zadanie 10. (0-4)\\
Rozwiąż równanie \(\sin x \cos 3 x+\operatorname{tg} x \cos ^{2} x=0\).

Więcej arkuszy znajdziesz na stronie: \href{http://arkusze.pl}{arkusze.pl}\\
\includegraphics[max width=\textwidth, center]{2024_11_21_f29375993e8c629c464fg-09}

Odpowiedź:

\begin{center}
\begin{tabular}{|l|l|c|c|}
\hline
\multirow{2}{*}{\begin{tabular}{c}
Wypełnia \\
sprawdzający \\
\end{tabular}} & Nr zadania & 9 & 10 \\
\cline { 2 - 4 }
 & Maks. liczba pkt & 3 & 4 \\
\cline { 2 - 4 }
 & Uzyskana liczba pkt &  &  \\
\hline
\end{tabular}
\end{center}

Zadanie 11. (0-4)\\
Funkcja \(f\) określona jest wzorem \(f(x)=\frac{x+3}{1-x}\) dla każdej liczby rzeczywistej \(x \neq 1\). Wyznacz równania tych stycznych do wykresu funkcji \(f\), które tworzą z osią \(O x\) kąt \(45^{\circ}\).

Więcej arkuszy znajdziesz na stronie: \href{http://arkusze.pl}{arkusze.pl}\\
\includegraphics[max width=\textwidth, center]{2024_11_21_f29375993e8c629c464fg-10}\\
\includegraphics[max width=\textwidth, center]{2024_11_21_f29375993e8c629c464fg-11}

Odpowiedź:

\begin{center}
\begin{tabular}{|l|l|c|}
\hline
\multirow{3}{*}{\begin{tabular}{c}
Wypełnia \\
sprawdzający \\
\end{tabular}} & Nr zadania & 11 \\
\cline { 2 - 3 }
 & Maks. liczba pkt & 4 \\
\cline { 2 - 3 }
 & Uzyskana liczba pkt &  \\
\hline
\end{tabular}
\end{center}

Zadanie 12. (0-4)\\
Doświadczenie losowe polega na tym, że losujemy jednocześnie dwie liczby ze zbioru \(\{1,2,3, \ldots, n\}\), gdzie \(n \geqslant 2\). Wyznacz te wartości \(n\), dla których prawdopodobieństwo wylosowania dwóch liczb różniących się od siebie co najmniej o trzy jest równe \(\frac{7}{12}\).

\begin{center}
\begin{tabular}{|c|c|c|c|c|c|c|c|c|c|c|c|c|c|c|c|c|c|c|c|c|c|c|}
\hline
 &  &  &  &  &  &  &  &  &  &  &  &  &  &  &  &  &  &  &  &  &  &  \\
\hline
 &  &  &  &  &  &  &  &  &  &  &  &  &  &  &  &  &  &  &  &  &  &  \\
\hline
 &  &  &  &  &  &  &  &  &  &  &  &  &  &  &  &  &  &  &  &  &  &  \\
\hline
 &  &  &  &  &  &  &  &  &  &  &  &  &  &  &  &  &  &  &  &  &  &  \\
\hline
 &  &  &  &  &  &  &  &  &  &  &  &  &  &  &  &  &  &  &  &  &  &  \\
\hline
 &  &  &  &  &  &  &  &  &  &  &  &  &  &  &  &  &  &  &  &  &  &  \\
\hline
 &  &  &  &  &  &  &  &  &  &  &  &  &  &  &  &  &  &  &  &  &  &  \\
\hline
 &  &  &  &  &  &  &  &  &  &  &  &  &  &  &  &  &  &  &  &  &  &  \\
\hline
 &  &  &  &  &  &  &  &  &  &  &  &  &  &  &  &  &  &  &  &  &  &  \\
\hline
 &  &  &  &  &  &  &  &  &  &  &  &  &  &  &  &  &  &  &  &  &  &  \\
\hline
 &  &  &  &  &  &  &  &  &  &  &  &  &  &  &  &  &  &  &  &  &  &  \\
\hline
 &  &  &  &  &  &  &  &  &  &  &  &  &  &  &  &  &  &  &  &  &  &  \\
\hline
 &  &  &  &  &  &  &  &  &  &  &  &  &  &  &  &  &  &  &  &  &  &  \\
\hline
 &  &  &  &  &  &  &  &  &  &  &  &  &  &  &  &  &  &  &  &  &  &  \\
\hline
 &  &  &  &  &  &  &  &  &  &  &  &  &  &  &  &  &  &  &  &  &  &  \\
\hline
 &  &  &  &  &  &  &  &  &  &  &  &  &  &  &  &  &  &  &  &  &  &  \\
\hline
 &  &  &  &  &  &  &  &  &  &  &  &  &  &  &  &  &  &  &  &  &  &  \\
\hline
 &  &  &  &  &  &  &  &  &  &  &  &  &  &  &  &  &  &  &  &  &  &  \\
\hline
 &  &  &  &  &  &  &  &  &  &  &  &  &  &  &  &  &  &  &  &  &  &  \\
\hline
 &  &  &  &  &  &  &  &  &  &  &  &  &  &  &  &  &  &  &  &  &  &  \\
\hline
 &  &  &  &  &  &  &  &  &  &  &  &  &  &  &  &  &  &  &  &  &  &  \\
\hline
 &  &  &  &  &  &  &  &  &  &  &  &  &  &  &  &  &  &  &  &  &  &  \\
\hline
 &  &  &  &  &  &  &  &  &  &  &  &  &  &  &  &  &  &  &  &  &  &  \\
\hline
 &  &  &  &  &  &  &  &  &  &  &  &  &  &  &  &  &  &  &  &  &  &  \\
\hline
 &  &  &  &  &  &  &  &  &  &  &  &  &  &  &  &  &  &  &  &  &  &  \\
\hline
 &  &  &  &  &  &  &  &  &  &  &  &  &  &  &  &  &  &  &  &  &  &  \\
\hline
 &  &  &  &  &  &  &  &  &  &  &  &  &  &  &  &  &  &  &  &  &  &  \\
\hline
 &  &  &  &  &  &  &  &  &  &  &  &  &  &  &  &  &  &  &  &  &  &  \\
\hline
 &  &  &  &  &  &  &  &  &  &  &  &  &  &  &  &  &  &  &  &  &  &  \\
\hline
 &  &  &  &  &  &  &  &  &  &  &  &  &  &  &  &  &  &  &  &  &  &  \\
\hline
 &  &  &  &  &  &  &  &  &  &  &  &  &  &  &  &  &  &  &  &  &  &  \\
\hline
 &  &  &  &  &  &  &  &  &  &  &  &  &  &  &  &  &  &  &  &  &  &  \\
\hline
 &  &  &  &  &  &  &  &  &  &  &  &  &  &  &  &  &  &  &  &  &  &  \\
\hline
 &  &  &  &  &  &  &  &  &  &  &  &  &  &  &  &  &  &  &  &  &  &  \\
\hline
 &  &  &  &  &  &  &  &  &  &  &  &  &  &  &  &  &  &  &  &  &  &  \\
\hline
 &  &  &  &  &  &  &  &  &  &  &  &  &  &  &  &  &  &  &  &  &  &  \\
\hline
 &  &  &  &  &  &  &  &  &  &  &  &  &  &  &  &  &  &  &  &  &  &  \\
\hline
 &  &  &  &  &  &  &  &  &  &  &  &  &  &  &  &  &  &  &  &  &  &  \\
\hline
 &  &  &  &  &  &  &  &  &  &  &  &  &  &  &  &  &  &  &  &  &  &  \\
\hline
 &  &  &  &  &  &  &  &  &  &  &  &  &  &  &  &  &  &  &  &  &  &  \\
\hline
 &  &  &  &  &  &  &  &  &  &  &  &  &  &  &  &  &  &  &  &  &  &  \\
\hline
 &  &  &  &  &  &  &  &  &  &  &  &  &  &  &  &  &  &  &  &  &  &  \\
\hline
 &  &  &  &  &  &  &  &  &  &  &  &  &  &  &  &  &  &  &  &  &  &  \\
\hline
\end{tabular}
\end{center}

\begin{center}
\includegraphics[max width=\textwidth]{2024_11_21_f29375993e8c629c464fg-13}
\end{center}

Odpowiedź:

\begin{center}
\begin{tabular}{|l|l|c|}
\hline
\multirow{3}{*}{\begin{tabular}{c}
Wypełnia \\
sprawdzający \\
\end{tabular}} & Nr zadania & 12 \\
\cline { 2 - 3 }
 & Maks. liczba pkt & 4 \\
\cline { 2 - 3 }
 & Uzyskana liczba pkt &  \\
\hline
\end{tabular}
\end{center}

Zadanie 13. (0-5)\\
W trapez równoramienny \(A B C D\), w którym \(A B \| C D\), wpisano okrąg o środku S. Odległość punktu \(S\) od końców dłuższej podstawy \(A B\) jest równa 10 , a cosinus kąta ostrego tego trapezu jest równy \(\frac{3}{5}\). Oblicz pole tego trapezu.

Więcej arkuszy znajdziesz na stronie: \href{http://arkusze.pl}{arkusze.pl}\\
\includegraphics[max width=\textwidth, center]{2024_11_21_f29375993e8c629c464fg-14}\\
\includegraphics[max width=\textwidth, center]{2024_11_21_f29375993e8c629c464fg-15}

Odpowiedź:

\begin{center}
\begin{tabular}{|l|l|c|}
\hline
\multirow{3}{*}{\begin{tabular}{c}
Wypełnia \\
sprawdzający \\
\end{tabular}} & Nr zadania & 13 \\
\cline { 2 - 3 }
 & Maks. liczba pkt & 5 \\
\cline { 2 - 3 }
 & Uzyskana liczba pkt &  \\
\hline
\end{tabular}
\end{center}

\section*{Zadanie 14. (0-5)}
Odcinek \(A^{\prime} B^{\prime}\) jest obrazem odcinka o końcach \(A=(2,6)\) oraz \(B=(-4,4)\) w jednokładności o środku \(O=(0,3)\) i skali \(k \neq 0\). Punkt \(A^{\prime}\), który jest obrazem punktu \(A\) w tej jednokładności, leży na prostej o równaniu \(x+2 y+10=0\). Wyznacz równanie okręgu, którego średnicą jest odcinek \(A^{\prime} B^{\prime}\).

Więcej arkuszy znajdziesz na stronie: \href{http://arkusze.pl}{arkusze.pl}\\
\includegraphics[max width=\textwidth, center]{2024_11_21_f29375993e8c629c464fg-16}\\
\includegraphics[max width=\textwidth, center]{2024_11_21_f29375993e8c629c464fg-17}

Odpowiedź:

\begin{center}
\begin{tabular}{|l|l|c|}
\hline
\multirow{3}{*}{\begin{tabular}{c}
Wypełnia \\
sprawdzający \\
\end{tabular}} & Nr zadania & 14 \\
\cline { 2 - 3 }
 & Maks. liczba pkt & 5 \\
\cline { 2 - 3 }
 & Uzyskana liczba pkt &  \\
\hline
\end{tabular}
\end{center}

\section*{Zadanie 15. (0-6)}
Prosta o równaniu \(y=(a-3) x+a+4\) przecina parabolę o równaniu \(y=\frac{1}{2} x^{2}-2 a x+a+8\) w dwóch punktach o pierwszych współrzędnych \(x_{1}, x_{2}\). Wyznacz wszystkie wartości parametru \(a\), dla których współrzędne \(x_{1}, x_{2}\) spełniają nierówność \(x_{1}^{3}+x_{2}^{3} \leqslant 9 x_{1} x_{2}\).\\
\includegraphics[max width=\textwidth, center]{2024_11_21_f29375993e8c629c464fg-18(4)}

Więcej arkuszy znajdziesz na stronie: \href{http://arkusze.pl}{arkusze.pl}\\
\includegraphics[max width=\textwidth, center]{2024_11_21_f29375993e8c629c464fg-18(5)}\\
\(\qquad\)\\
\includegraphics[max width=\textwidth, center]{2024_11_21_f29375993e8c629c464fg-18(2)}\\
\includegraphics[max width=\textwidth, center]{2024_11_21_f29375993e8c629c464fg-18}

\begin{center}
\begin{tabular}{l|l|l|l|l|l|l|l|l|l|l|l|l|l|l|l|l|l|l|l|l|l|l|l|l|}
 \\
\hline
\end{tabular}
\end{center}

\includegraphics[max width=\textwidth, center]{2024_11_21_f29375993e8c629c464fg-18(3)}\\
\includegraphics[max width=\textwidth, center]{2024_11_21_f29375993e8c629c464fg-18(1)}\\
\includegraphics[max width=\textwidth, center]{2024_11_21_f29375993e8c629c464fg-19}

Odpowiedź:

\begin{center}
\begin{tabular}{|l|l|c|}
\hline
\multirow{3}{*}{\begin{tabular}{c}
Wypełnia \\
sprawdzający \\
\end{tabular}} & Nr zadania & 15 \\
\cline { 2 - 3 }
 & Maks. liczba pkt & 6 \\
\cline { 2 - 3 }
 & Uzyskana liczba pkt &  \\
\hline
\end{tabular}
\end{center}

Zadanie 16. (0-7)\\
Ostrosłup \(A B C S\) o podstawie \(A B C\), będącej trójkątem równobocznym, i wysokości \(S D\), gdzie \(D\) jest środkiem odcinka \(A B\), przecięto płaszczyzną równoległą do płaszczyzny podstawy (zobacz rysunek). Punkty \(E, F, G\) są punktami wspólnymi płaszczyzny przekroju i krawędzi bocznych ostrosłupa. Punkty \(E^{\prime}, F^{\prime}, G^{\prime}\) leżą na podstawie \(A B C\) i odcinki \(E E^{\prime}, F F^{\prime}, G G^{\prime}\) są do niej prostopadłe. Wiedząc, że \(|A B|=12\) i \(|S D|=16\), oblicz, w jakiej odległości od płaszczyzny podstawy należy poprowadzić przekrój ostrosłupa, aby graniastosłup \(E^{\prime} F^{\prime} G^{\prime} E F G\) miał największą objętość.\\
\includegraphics[max width=\textwidth, center]{2024_11_21_f29375993e8c629c464fg-20}\\
\includegraphics[max width=\textwidth, center]{2024_11_21_f29375993e8c629c464fg-20(1)}\\
\includegraphics[max width=\textwidth, center]{2024_11_21_f29375993e8c629c464fg-21}

Odpowiedź:

\begin{center}
\begin{tabular}{|l|l|c|}
\hline
\multirow{3}{*}{\begin{tabular}{c}
Wypełnia \\
sprawdzający \\
\end{tabular}} & Nr zadania & 16 \\
\cline { 2 - 3 }
 & Maks. liczba pkt & 7 \\
\cline { 2 - 3 }
 & Uzyskana liczba pkt &  \\
\hline
\end{tabular}
\end{center}

\section*{BRUDNOPIS}
\begin{center}
\includegraphics[max width=\textwidth]{2024_11_21_f29375993e8c629c464fg-22}
\end{center}


\end{document}