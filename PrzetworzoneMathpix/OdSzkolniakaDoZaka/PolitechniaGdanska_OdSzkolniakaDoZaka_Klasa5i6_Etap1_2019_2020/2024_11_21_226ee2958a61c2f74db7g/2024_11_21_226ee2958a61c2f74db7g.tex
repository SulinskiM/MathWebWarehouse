\documentclass[10pt]{article}
\usepackage[polish]{babel}
\usepackage[utf8]{inputenc}
\usepackage[T1]{fontenc}
\usepackage{amsmath}
\usepackage{amsfonts}
\usepackage{amssymb}
\usepackage[version=4]{mhchem}
\usepackage{stmaryrd}
\usepackage{hyperref}
\hypersetup{colorlinks=true, linkcolor=blue, filecolor=magenta, urlcolor=cyan,}
\urlstyle{same}

\title{OD SZKOLNIAKA DO ŻAKA }

\author{}
\date{}


\begin{document}
\maketitle
\section*{klasy 5 i 6 szkoły podstawowej rok szkolny 2019/2020 Zadania - etap I}
Zadanie 1. Podaj trzy ułamki właściwe \(x\), które spełniają warunek: \(\frac{3}{5}<x<\frac{4}{5}\).

Zadanie 2. Na statku było 31 marynarzy i kapitan. Średnia wieku marynarzy wynosiła 23 lata. Gdyby policzyć średnią wieku wszystkich marynarzy i kapitana, to wyniosłaby ona 24 lata. Ile lat miał kapitan?

Zadanie 3. Baca ma barany, owce i psy - razem 32 zwierzęta. Liczba psów jest trzy razy mniejsza od liczby baranów, a liczba owiec jest o 22 większa od liczby psów. lle owiec i baranów ma baca?

Zadanie 4. W trójkącie prostokątnym o obwodzie 80 cm , w którym c oznacza długość przeciwprostokątnej, zaś \(a, b\) są długościami przyprostokątnych, mamy: \(a+c=\) 50 cm i \(b+c=64 \mathrm{~cm}\). Oblicz pole tego trójkąta.

Zadanie 5. Różnica cyfry dziesiątek i cyfry jedności pewnej liczby dwucyfrowej jest równa 4. Gdy do tej szukanej liczby dodamy liczbę utworzoną z jej cyfr, ale zapisanych w odwrotnej kolejności, to otrzymamy 132. Wyznacz tę liczbę.

Opracował: M.Bednarczyk

CENTRUM NAUCZANIA MATEMATYKI\\
I KSZTALCENIA NA ODLEGLOŚC

\section*{(imię i nazwisko uczestnika)}
\(\qquad\)\\
(szkoła - nazwa i adres)

\section*{ZAŁĄCZNIK NR 1 DO KONKURSU „OD SZKOLNIAKA DO ŻAKA"}
\section*{Oświadczenie}
Zgodnie z art. 6 ust. 1 lit. a ogólnego rozporządzenia o ochronie danych osobowych z dnia 27 kwietnia 2016 r. (Dz. Urz. UE L 119 z 04.05.2016) (RODO) oświadczam, że wyrażam zgodę na przetwarzanie przez Politechnikę Gdańską z siedzibą w Gdańsku, ul. Narutowicza 11/12, 80-233 Gdańsk, danych osobowych mojego dziecka w celu i zakresie niezbędnym do przeprowadzenia konkursu matematycznego „Od szkolniaka do żaka".

\section*{Klauzula informacyjna}
Zgodnie z art. 13 ogólnego rozporządzenia o ochronie danych osobowych z dnia 27 kwietnia 2016 r. (Dz. Urz. UE L 119 z 04.05.2016) ( RODO) informujemy, że:

\begin{enumerate}
  \item Administratorem danych osobowych Pani/Pana dziecka jest Politechnika Gdańska z siedzibą przy ul. Narutowicza 11/12, w Gdańsku (kod pocztowy: 80-233);
  \item Administrator wyznaczył Inspektora Ochrony Danych, z którym można się skontaktować za pośrednictwem adresu e-mail: - \href{mailto:iod@pg.edu.pl}{iod@pg.edu.pl}. Do Inspektora Ochrony Danych należy kierować wyłącznie sprawy dotyczące przetwarzania danych Pani/Pana dziecka przez Politechnikę Gdańską, w tym realizacji Pani/Pana dziecka praw;
  \item Dane Pani/Pana dziecka będą przetwarzane w celu przeprowadzenia niniejszego konkursu na podstawie wyrażonej przez Panią/Pana zgody (Art. 6 ust. 1 lit. a RODO);
  \item Dane osobowe będą przechowywane do końca roku szkolnego w jakim zostały zebrane za wyjątkiem danych laureatów, które mogą być przechowywane bezterminowo;
  \item Podane dane nie będą podlegały udostępnieniu podmiotom trzecim. Odbiorcami danych będą tylko instytucje upoważnione na mocy prawa (sądy, policja itp.);
  \item Przysługuje Pani/Panu:\\
a. prawo dostępu do treści danych osobowych Pani/Pana dziecka oraz otrzymania ich kopii,\\
b. prawo do sprostowania (poprawiania) danych osobowych Pani/Pana dziecka,\\
c. prawo do usunięcia danych osobowych w sytuacji gdy przetwarzanie danych nie następuje w celu w jakim zostały one zebrane,\\
d. prawo do ograniczenia przetwarzania danych,\\
e. prawo do cofnięcia zgody w dowolnym momencie bez wpływu na zgodność z prawem przetwarzania, którego dokonano na podstawie zgody przed jej cofnięciem,\\
f. prawo do wniesienia skargi do Prezesa UODO (na adres Urzędu Ochrony Danych Osobowych, ul. Stawki 2, 00 - 193 Warszawa).
  \item Podanie danych jest dobrowolne, lecz niezbędne do wzięcia udziału Pani/Pana dziecka w niniejszym konkursie;
  \item Dane osobowe dziecka udostępnione przez Panią/Pana nie będą podlegały profilowaniu;
  \item Administrator danych nie zamierza przekazywać danych osobowych do państwa trzeciego lub organizacji międzynarodowej.\\
(data)\\
(podpis rodzica lub opiekuna prawnego ucznia)
\end{enumerate}

Wyrażam zgodę na uczestnictwo mojego dziecka w konkursie matematycznym „Od szkolniaka do żaka" organizowanym przez Politechnikę Gdańską oraz oświadczam, że akceptuję regulamin wyżej wymienionego konkursu - „Regulamin konkursu matematycznego w roku akademickim 2019/2020 dla klasy V, VI, VII i VIII szkoły podstawowej „Od szkolniaka do żaka".


\end{document}