\documentclass[10pt]{article}
\usepackage[polish]{babel}
\usepackage[utf8]{inputenc}
\usepackage[T1]{fontenc}
\usepackage{amsmath}
\usepackage{amsfonts}
\usepackage{amssymb}
\usepackage[version=4]{mhchem}
\usepackage{stmaryrd}
\usepackage{hyperref}
\hypersetup{colorlinks=true, linkcolor=blue, filecolor=magenta, urlcolor=cyan,}
\urlstyle{same}

\title{Zadania - etap I (szkoła podstawowa) }

\author{}
\date{}


\begin{document}
\maketitle
Zadanie 1. Państwo Malinowscy byli ze swoim synkiem na grzybach. Pan Malinowski zebrał \(\frac{2}{9}\) łącznej liczby zebranych grzybów, jego żona \(\frac{7}{10}\) łącznej liczby, a synek znalazł tylko 7 grzybów. Ile grzybów zebrała każda z wymienionych osób?

Zadanie 2. Ojciec i syn mają obecnie razem 50 lat. Pięć lat temu ojciec był 3 razy starszy od syna. Ile lat obecnie ma ojciec, a ile syn?

Zadanie 3. Znajdź liczbę, której \(\frac{2}{7}\) wynosi tyle, co \(\frac{2}{3}\) wartości wyrażenia:

\[
\left(8 \frac{1}{4}-4 \frac{1}{2} \cdot 1 \frac{1}{9}\right): 1 \frac{1}{12}
\]

Zadanie 4. Uzasadnij, że liczba \(123^{80}-57^{80}\) jest podzielna przez 10.\\
Zadanie 5. W chwili obecnej pan Wojciechowski ma 45 lat, a jego trzy córki: 12 lat, 9 lat i 6 lat. Po ilu latach wiek pana Wojciechowskiego będzie równy sumie lat jego córek?\\
imię i nazwisko uczestnika

CENTRUM NAUCZANIA MATEMATYKI\\
I KSZTALCENIA NA ODLEGLOŚ\\
\(\qquad\)

\section*{ZAŁACZNIK DO KARTY UCZESTNIKA KONKURSU „OD SZKOLNIAKA DO ŻAKA"}
\section*{Oświadczenie}
Niniejszym oświadczam, że jako uczestnik konkursu „Od szkolniaka do żaka" zorganizowanego przez Centrum Nauczania Matematyki i Kształcenia na Odległość Politechniki Gdańskiej, wyrażam zgodę na przetwarzanie moich danych osobowych w zakresie niezbędnym dla potrzeb niniejszego konkursu.

Przyjmuję do wiadomości, że moje dane osobowe będą wykorzystane zgodnie z ustawą z dnia 29 sierpnia 1997 r. o ochronie danych osobowych (Dz.U. 1997 nr 133 poz. 883) dla celów przeprowadzenia w/w konkursu.

Jednocześnie oświadczam, że zostałem poinformowany o tym, że:

\begin{itemize}
  \item Administratorem danych osobowych konkursu jest: Politechnika Gdańska z siedzibą przy ul. Gabriela Narutowicza 11/12; 80-233 Gdańsk
  \item Przysługuje mi prawo do wglądu do moich danych i żądania ich poprawienia.
  \item Dane będą przetwarzane dla realizacji konkursu.
  \item Podanie danych jest dobrowolne.
  \item Nie przewiduje się przekazywania danych.
\end{itemize}

Wyrażam zgodę na uczestnictwo mojego dziecka w konkursie „Od szkolniaka do żaka"

Data \(\qquad\) r.\\
(podpis rodzica lub opiekuna prawnego ucznia)

Akceptuję i wyrażam zgodę na postanowienia regulaminu konkursu „Od szkolniaka do żaka" zamieszczonego na stronie internetowej konkursu: \href{http://pg.edu.pl/kursy-z-matematyki/o-konkursie}{http://pg.edu.pl/kursy-z-matematyki/o-konkursie}

Data \(\qquad\) r.


\end{document}