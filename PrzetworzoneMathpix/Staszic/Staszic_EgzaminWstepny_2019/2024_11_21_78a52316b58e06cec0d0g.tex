\documentclass[10pt]{article}
\usepackage[polish]{babel}
\usepackage[utf8]{inputenc}
\usepackage[T1]{fontenc}
\usepackage{amsmath}
\usepackage{amsfonts}
\usepackage{amssymb}
\usepackage[version=4]{mhchem}
\usepackage{stmaryrd}

\title{Sprawdzian predyspozycji do klas matematycznych }

\author{XIV LO im. S. Staszica w Warszawie}
\date{}


\newcommand\Varangle{\mathop{{<\!\!\!\!\!\text{\small)}}\:}\nolimits}

\begin{document}
\maketitle
(3 czerwca 2019 r.)

\section*{Uwagi}
\begin{itemize}
  \item Poniższe zadania można rozwiązywać w dowolnej kolejności.
  \item Wszystkie zadania są jednakowo punktowane.
  \item Podanie jedynie prawidłowej odpowiedzi liczbowej nie stanowi rozwiązania zadania. Ocenie podlegal będzie tok rozumowania oraz obliczenia prowadzące do uzyskanego wyniku.
  \item Czas na rozwiązywanie zadań: \(\mathbf{9 0}\) minut.
\end{itemize}

\begin{enumerate}
  \item Rozstrzygnij, która liczba jest większa: \(111^{333}\) czy \(999^{222}\) ?
  \item Po wykonaniu wszystkich działań liczbę
\end{enumerate}

\[
1^{1}+2^{2}+3^{3}+\ldots+99^{99}+100^{100}
\]

zapisano w systemie dziesiętnym, uzyskując pewną liczbę \(n\)-cyfrową. Wyznacz \(n\).\\
3. Ile jest wszystkich takich par \((i, j)\) liczb naturalnych, dla których \(1 \leqslant i \leqslant 100\), \(1 \leqslant j \leqslant 100\) oraz liczba \(i+j\) jest podzielna przez 3 ? Odpowiedź uzasadnij. Uwaga. (1,1), (2,2), (3,3), itd. są parami.\\
Jeśli \(i \neq j\), to pary \((i, j)\) oraz \((j, i)\) traktujemy jako różne.\\
4. Dany jest kwadrat \(A B C D\). Wewnątrz tego kwadratu wybrano taki punkt \(E\), że trójkąt \(A B E\) jest równoboczny. Na odcinku \(A B\) wybrano taki punkt \(P\), że \(\Varangle A P E=\Varangle B P C=\alpha\). Wyznacz miarę kąta \(\alpha\).\\
5. Dany jest ostrosłup prawidłowy 100-kątny. Jaką największą liczbę krawędzi tego ostrosłupa można przeciąć płaszczyzną (czyli płaskim cięciem), która nie przechodzi przez żaden z wierzchołków ostrosłupa? Odpowiedź uzasadnij.


\end{document}